!\documentclass[problem]{mcs}

\begin{pcomments}
  \pcomment{CP_magic_trick_hide_2}
  \pcomment{from: S09.cp10t}
\end{pcomments}

\pkeywords{
  counting
  counting_rules
  magic_trick
}

%%%%%%%%%%%%%%%%%%%%%%%%%%%%%%%%%%%%%%%%%%%%%%%%%%%%%%%%%%%%%%%%%%%%%
% Problem starts here
%%%%%%%%%%%%%%%%%%%%%%%%%%%%%%%%%%%%%%%%%%%%%%%%%%%%%%%%%%%%%%%%%%%%%

\begin{problem} The Magician can determine the 5th card in a poker hand
  when his Assisant reveals the other 4 cards.  Describe a similar method
  for determining 2 hidden cards in a hand of 9 cards when your Assisant
  reveals the other 7 cards.

\begin{solution}

Since there must be $\ceil{9/4} = 3$ cards with the same suit, our
collaborator chooses to hide two of them and then use the third one as the
first card to be revealed.  So this first revealed card fixes the suit of
the two hidden cards; it will also be used as the origin for the offset
position of the first hidden card.  This first hidden card will in turn be
used as the origin for the offset of the other hidden card.  There are six
cards to code the two offset positions.  These suffice to code two offsets
of size from one to six.  That is, our collaborator can choose one of the
$3! = 6$ orders in which to reveal the first three cards and thereby tell
us the offset position of the first hidden card.  Our collaborator can
then choose the order of the final three cards to describe the offset
position of the second hidden card from the first.  Note that the first
revealed card must be chosen so that both offsets are $\leq 6$; since
the sum of the offsets between successive cards ordered in a cycle from Ace
to King is 13, it is not possible for more than one offset between
successive cards to exceed seven, so this is always possible.

\end{solution}

\end{problem}

\endinput
