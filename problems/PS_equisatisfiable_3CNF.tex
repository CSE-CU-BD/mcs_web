\documentclass[problem]{mcs}

\newcommand{\eqsat}[1]{\mathcal{C}(#1)}

\begin{pcomments}
    \pcomment{PS_equisatisfiable_3CNF}
    \pcomment{similar idea to CP_sat_formulas_vs_circuits}
    \pcomment{by ARM 2/5/11}
\end{pcomments}

\pkeywords{
  satisfiable
  3CNF
  propositional_variable
  conjunctive
}

\begin{problem}
\iffalse
There are small formulas whose equivalent conjunctive forms will be
exponentially larger.  For example, the smallest conjunctive forms
equivalent to disjunctive formulas of the form
\[
(P_1 \QAND Q_1) \QOR (P_2 \QAND Q_2) \QOR \cdots \QOR (P_n \QAND Q_n)
\]
all have more than $2^n$ occurrences of variables.

On the other hand, it's\fi

A 3-conjunctive form (3CF) formula is a conjunctive form formula in
which each \QOR-term is an \QOR\ of at most 3 variables or negations of
variables.  Although it may be hard to tell if a propositional
formula $F$ is satisfiable, it is always easy to construct a
formula $\eqsat{F}$ that is
\begin{itemize}
\item in 3-conjunctive form,
\item has at most 24 times as many occurrences of variables as $F$, and
\item is satisfiable iff $F$ is satisfiable.
\end{itemize}

To construct $\eqsat{F}$, introduce a different new variables for
each operator that occurs in $F$.  For example, if $F$ was
\begin{equation}\label{PXQXROPS}
((P \QXOR Q) \QXOR R) \QOR (\bar{P} \QAND S)
\end{equation}
we might use new variables $X_1$, $X_2$ $O$ and $A$ corresponding to
the operator occurrences as follows:
\[
((P \underbrace{\QXOR}_{X_1} Q) \underbrace{\QXOR}_{X_2} R) \underbrace{\QOR}_{O}
 (\bar{P} \underbrace{\QAND}_{A} S).
\]
Next we write a formula that constrains each new variable to have the
same truth value as the subformula determined by its corresponding
operator.  For the example above, these constraining formulas would be
\begin{align*}
X_1 &\QIFF (P \QXOR Q),\\
X_2 &\QIFF (X_1 \QXOR R),\\
A   &\QIFF (\bar{P} \QAND S),\\
O   & \QIFF (X_2 \QOR A)
\end{align*}

\bparts

\ppart\label{Q4forms} Explain why the \QAND\ of the four constraining
formulas above along with a fifth formula consisting of just the
variable $O$ will be satisfiable iff~\eqref{PXQXROPS} is satisfiable.

\begin{solution}
All the propositions, if true, ensure that the value of the two sides
of the \QIFF\ are the same.  This means that we could substitute any
of the two sides for the other in another proposition and obtain the
same result.  This way, we can use the four constraining formulas to
build~\eqref{PXQXROPS} again, showing that the satisfiability
of~\eqref{PXQXROPS} implies the satisfiability of the four formulas
and $O$ and vice versa.
\end{solution}

\ppart\label{3cf24} Explain why each constraining formula will be
equivalent to a 3CF formula with at most 24 occurrences of variables.

\begin{solution}
Each constraining formula has only three occurrences of variables, so
its CNF will be the \QAND\ of at most 8 \QOR-terms, where each
\QOR-term is an \QOR\ of the 3 variables or their negations.
\end{solution}

\ppart Using the ideas illustrated in the previous parts, explain how
to construct $\eqsat{F}$ for an arbitrary propositional formula $F$.

\begin{solution}
We start by using the technique shown to make a proposition for each
boolean operator.  As in part~\ref{Q4forms}, we can turn any of the
propositions we obtain into 3CF formulas with at most 24 occurrences
of variables. Then, $C(F)$ is the AND of all these 3CF propositions.
So, it is satisfiable whenever the original proposition is (see
part~\ref{Q4forms}, it is also of form 3CF because it is an \QAND\ of
multiple 3CF and it has, by part~\ref{3cf24}, at most 24 times as many
variable occurrences.
\end{solution}

\eparts

\end{problem}

\endinput
