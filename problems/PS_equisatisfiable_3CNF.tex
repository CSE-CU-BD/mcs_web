\documentclass[problem]{mcs}

\newcommand{\eqsat}[1]{\mathcal{C}(#1)}

\begin{pcomments}
    \pcomment{PS_equisatisfiable_3CNF}
    \pcomment{by ARM 2/5/11}
\end{pcomments}

\begin{problem}
\iffalse
There are small formulas whose equivalent conjunctive forms will be
exponentially larger.  For example, the smallest conjunctive forms
equivalent to disjunctive formulas of the form
\[
(P_1 \QAND Q_1) \QOR (P_2 \QAND Q_2) \QOR \cdots \QOR (P_n \QAND Q_n)
\]
all have more than $2^n$ occurrences of variables.

On the other hand, it's\fi

A 3-conjunctive form (3CF) formula is a conjunctive form formula in
which each \QOR-term is a \QOR of at most 3 variables or negations of
variables. Although it may be hard to tell if a propositional formula,
$F$, is satisfiable, it is always easy to construct a formula,
$\eqsat{F}$, that is
\begin{itemize}
\item in 3-conjunctive form,
\item has at most 24 times as many occurrences of variables as $F$, and
\item is satisfiable iff $F$ is satisfiable.
\end{itemize}

To construct $\eqsat{F}$, introduce a different new variables, one for
each operator that occurs in $F$.  For example, if $F$ was
\begin{equation}\label{PXQXROPS}
((P \QXOR Q) \QXOR R) \QOR (\bar{P} \QAND S)
\end{equation}
we might use new variables $X_1$, $X_2$, $O$, and $A$ corresponding to
the the operator occurrences as follows:
\[
((P \underbrace{\QXOR}_{X_1} Q) \underbrace{\QXOR}_{X_2} R) \underbrace{\QOR}_{O}
 (\bar{P} \underbrace{\QAND}_{A} S).
\]
Next we write a formula that contrains each new variable to have the
same truth value as the subformula determined by its corresponding
operator.  For the example above, these contraining formulas would be
\begin{align*}
X_1 &\QIFF (P \QXOR Q),\\
X_2 &\QIFF (X_1 \QXOR R),\\
A   &\QIFF (\bar{P} \QAND S),\\
O   & \QIFF X_2 \QXOR A.
\end{align*}

\bparts

\ppart Explain why the \QAND\ of the above four constraining formulas will
be satisfiable iff~\eqref{PXQXROPS} is satisfiable.

\begin{solution}
TBA
\end{solution}

\ppart Explain why any constraining formula will be equivalent to a
3CF formula with at most 24 occurrences of variables.

\begin{solution}
Each constraining formula has only three occurrences of variables, so
its CNF will be a \QAND\ and of at most 8 \QOR-terms, where each
\QOR-term is a \QOR\ of the 3 variables or their negations.
\end{solution}

\ppart Using the ideas illustrated in the previous parts, explain how
to construct $\eqsat{F}$ for an arbitrary propositional formula, $F$.

\begin{solution}
TBA
\end{solution}

\eparts

\end{problem}

\endinput
