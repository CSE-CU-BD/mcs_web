\documentclass[problem]{mcs}

\begin{pcomments}
  \pcomment{CP_little_oh_strictPO}
  \pcomment{from S07.ps8}
\end{pcomments}

\pkeywords{
  little_oh
  asymptotic
  partial_order
  strict
  incomparable
}

%%%%%%%%%%%%%%%%%%%%%%%%%%%%%%%%%%%%%%%%%%%%%%%%%%%%%%%%%%%%%%%%%%%%%
% Problem starts here
%%%%%%%%%%%%%%%%%%%%%%%%%%%%%%%%%%%%%%%%%%%%%%%%%%%%%%%%%%%%%%%%%%%%%

 
\begin{problem}

\bparts

\ppart Prove that the relation, $R$, on functions such that
$f\mrel{R}g$ iff $f = o(g)$ is a strict partial order.

\begin{solution}
We need only show that $R$ is irreflexive and transitive.

Now for any function, $f$,
\[
\lim_{x \rightarrow \infty} f(x)/f(x) = 1 \neq 0,
\]
so $\QNOT(f\mrel{R}f)$.  This implies $R$ is irreflexive.

To show $R$ is transitive, assume $f\mrel{R}g$ and $g\mrel{R}h$.  This
means that $\lim_{x \rightarrow \infty} f(x)/g(x)=0$ and $\lim_{x
  \rightarrow \infty} g(x)/h(x) = 0$.  Since these limits exist, there
must be an $x_0$ such that the denominators, $g(x)$ and $h(x)$, are
nonzero for all $x \geq x_0$.  Therefore,
\begin{equation}\label{fggh}
\frac{f(x)}{g(x)} \cdot \frac{g(x)}{h(x)} = \frac{f(x)}{h(x)}
\end{equation}
for all $x \geq x_0$.

So, we have
\begin{align*}
0 &= 0 \cdot 0\\
 &= \paren{\lim_{x\to \infty} \frac{f(x)}{g(x)}} \cdot
     \paren{\lim_{x\to \infty} \frac{g(x)}{h(x)}}\\
 &= \lim_{x\to \infty} \frac{f(x)}{g(x)} \cdot \frac{g(x)}{h(x)}
             & \text{(property of limits)}\\
 &= \lim_{x\to \infty} f(x)/h(x) &\text{(by~\eqref{fggh})},
\end{align*}
This means that $f\mrel{R}h$ holds, proving that $R$ is transitive.

\end{solution}

\ppart Describe two functions $f, g$ that are incomparable under big Oh:
\[
f \neq O(g) \QAND g \neq O(f).
\]
Conclude that $R$ is not a path-total order.  How about three such functions?

\begin{solution}
One example is,
\[
f(n) \eqdef \begin{cases}            
n^2 & \text{if $n$ is odd},\\
n   & \text{if $n$ is even}, 
\end{cases}\qquad
g(n) \eqdef \begin{cases}
n & \text{if $n$ is odd},\\
n^2   & \text{if $n$ is even}, 
\end{cases}
\]
which can also be described by the formulas
\[
f(n) \eqdef n+ (n^2-n)\sin\paren{\frac{n\pi}{2}}, \qquad g(n) \eqdef n+ (n^2-n)\cos\paren{\frac{n\pi}{2}}.
\]

Since neither $f$ nor $g$ is big-Oh of the other, they are certainly
not little-oh of each other either.  So they are incomparable under
little-oh, and therefore little-oh is not path-total.

To get three functions that are incomparable under big Oh, define for $i=0,1,2$:
\[
f_i(n) \eqdef \begin{cases}            
n^2 & \text{if $n \equiv i \pmod 3$},\\
n   & \text{otherwise}.
\end{cases}
\]

\end{solution}

\eparts

\end{problem}

%%%%%%%%%%%%%%%%%%%%%%%%%%%%%%%%%%%%%%%%%%%%%%%%%%%%%%%%%%%%%%%%%%%%%
% Problem ends here
%%%%%%%%%%%%%%%%%%%%%%%%%%%%%%%%%%%%%%%%%%%%%%%%%%%%%%%%%%%%%%%%%%%%%

\endinput
