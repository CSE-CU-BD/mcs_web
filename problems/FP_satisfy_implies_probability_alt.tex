\documentclass[problem]{mcs}

\begin{pcomments}
  \pcomment{FP_satisfy_implies_probability_alt}
  \pcomment{perturbation of FP_satisfy_implies_probability}
  \pcomment{ARM 5/19/15}
\end{pcomments}

\pkeywords{
  satisfiability
  propositions
  implies
  probability
}

%%%%%%%%%%%%%%%%%%%%%%%%%%%%%%%%%%%%%%%%%%%%%%%%%%%%%%%%%%%%%%%%%%%%%
% Problem starts here
%%%%%%%%%%%%%%%%%%%%%%%%%%%%%%%%%%%%%%%%%%%%%%%%%%%%%%%%%%%%%%%%%%%%%
\begin{problem}
Truth values for propositional variables $P,Q,R$ are chosen
independently, with
\[
\pr{P=\true} = 1/2,\ \pr{Q=\true} = 1/3,\ \pr{R=\true} = 1/5.
\]
What is the probability that the formula
\insolutions{\begin{equation}\label{pPQR}}
\instatements{\[}
(P \QIMPLIES Q) \QIMPLIES R
\instatements{\]}
\insolutions{\end{equation}}
is true?

\exambox{0.5in}{0.4in}{0in}

\examspace[1.0in]

\begin{solution}
\[
\frac{7}{15}.
\]

Consider the separate cases in which~\eqref{pPQR} is true:

\inductioncase{$R = \true$}: This case happens with probability $1/5$.

\inductioncase{$R = \false \QAND Q = \false \QAND P = \true$}: This case
happens with probability $(4/5)(2/3)(1/2)$.

So the probability~\eqref{pPQR} is true is
\[
\frac{1}{5}+\frac{4}{5} \cdot \frac{2}{3} \cdot \frac{1}{2} = \frac{7}{15}.
\]

Another breakdown into cases is $P =\true \QAND Q=\false$, and
$\QNOT(P =\true \QAND Q=\false) \QAND R = \true$ with probability
\[
\frac{1}{2} \cdot \frac{2}{3} + (1- \frac{1}{2} \cdot \frac{2}{3})\frac{1}{5} = \frac{7}{15}
\]
\end{solution}

\end{problem}

\endinput
