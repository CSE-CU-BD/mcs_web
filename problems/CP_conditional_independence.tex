\documentclass[problem]{mcs}

\begin{pcomments}
  \pcomment{CP_conditional_independence}
  \pcomment{ARM 4/28/14}
\end{pcomments}

\pkeywords{
  independence
  conditional_probability
  conditional_independence
  intersection
  union
}

%%%%%%%%%%%%%%%%%%%%%%%%%%%%%%%%%%%%%%%%%%%%%%%%%%%%%%%%%%%%%%%%%%%%%
% Problem starts here
%%%%%%%%%%%%%%%%%%%%%%%%%%%%%%%%%%%%%%%%%%%%%%%%%%%%%%%%%%%%%%%%%%%%%

\begin{problem}
Let $A,B,C,D$ be events.  For each of the following statements, prove it
or give a counterexample.

\bparts

\ppart If $A$ and $B$ are independent given $C$, and are also
independent given $D$, then $A$ and $B$ are independent given $C
\union D$.

\begin{solution}
\textbf{False}.

Notice that $A$ is certain given $A$, so everything is independent of
$A$ given $A$, and similarly everything is independent of $A$ given
$\bar{A}$.  But an event is independent of $A$ given $A \union
\bar{A}$ iff it simply is independent $A$.  So $C=A$, $D=\bar{A}$, and
$B$ any event that is not independent of $A$ provides a
counterexample.

\iffalse
Let $B$ and $C$ be $A$ and $D$ be $\bar{A}$.  Then
\begin{align*}
prcond{A \intersect B}{C}
  & = \pr{A \intersect A \intersect A}/\pr{A}\\
  & = 1\\
  & = \prcond{A}{A} \cdot \prcond{A}{A}\\
  & = \pr{A}{C} \cdot \pr{A}{B},
\end{align*}
so $A$ and $B$ are independent given $C$.  Also,
\[
prcond{A \intersect B}{D} = 0 = \prcond{B}{D},
\]
so $A$ and $B$ are independent given $D$.

But $A$ and $B$ are independent given $C \union D$ iff $A$ is
independent of $A$, which is false for all nontrivial events $A$.
\fi

\end{solution}

\ppart How about independent of $C \intersect D$?

\begin{solution}
\textbf{False}.

$A$ is certain given $A$, and $B$ is certain given $B$, so $A$ and $B$
are independent given $A$ or given $B$.

Now a simple counterexample is a the space $[1--5]$ with uniform probability.   Let
\[
A \eqdef \set{2,4},\qquad
B \eqdef \set{1,2},\qquad
C \eqdef [1--4], \qquad
D \eqdef \set{1,3}.
\]
So $A$ and $B$ are independent given $C$ since
\begin{align*}
\prcond{A \intersect B}{C}
     & = \frac{\pr{2}}{\pr{[1--4]}\\
     & =  \frac14 = \frac12 \cdot \frac12\\
     & = \prcond{A}{C} \cdot \prcond{B}{C}.
\end{align*}

\end{solution}

\eparts

\end{problem}

%%%%%%%%%%%%%%%%%%%%%%%%%%%%%%%%%%%%%%%%%%%%%%%%%%%%%%%%%%%%%%%%%%%%%
% Problem ends here
%%%%%%%%%%%%%%%%%%%%%%%%%%%%%%%%%%%%%%%%%%%%%%%%%%%%%%%%%%%%%%%%%%%%%

\endinput
