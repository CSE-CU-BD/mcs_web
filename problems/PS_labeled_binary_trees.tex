\documentclass[problem]{mcs}

\begin{pcomments}
  \pcomment{PS_labeled_binary_trees}
  \pcomment{revised from S07.ps3.prob2 by ARM 3/20/11}
  \pcomment{subsumed by simpler FP_binary_tree_induction}
\end{pcomments}

\pkeywords{
  structural_induction
  tree
  leaf
  binary_tree
}

\newcommand{\lbt}{\text{LBT}}

%%%%%%%%%%%%%%%%%%%%%%%%%%%%%%%%%%%%%%%%%%%%%%%%%%%%%%%%%%%%%%%%%%%%%
% Problem starts here
%%%%%%%%%%%%%%%%%%%%%%%%%%%%%%%%%%%%%%%%%%%%%%%%%%%%%%%%%%%%%%%%%%%%%


\begin{problem}
  Let $L$ be some convenient set whose elements will be called
  \emph{labels}.  The labeled binary trees, $\lbt$'s, are defined
  recursively as follows:
\begin{definition*}

\inductioncase{Base case}: if $l$ is a label, then $\ang{l, \STR{leaf}}$ is an $\lbt$, and

\inductioncase{Constructor case}: if $B$ and $C$ are $\lbt$'s, then
$\ang{l,B,C}$ is an $\lbt$.

\end{definition*}
The \emph{leaf-labels} and \emph{internal-labels} of an $\lbt$ are defined
recursively in the obvious way:

\begin{definition*}

  \inductioncase{Base case}: The set of leaf-labels of the $\lbt \ang{l,
    \STR{leaf}}$ is $\set{l}$, and its set of internal-labels is the
  empty set.

  \inductioncase{Constructor case}: The set of leaf labels of the $\lbt$
  $\ang{l, B, C}$ is the union of the leaf-labels of $B$ and of $C$; the
  set of internal-labels is the union of $\set{l}$ and the sets of
  internal-labels of $B$ and of $C$.

\end{definition*}

The set of \emph{labels} of an $\lbt$ is the union of its leaf- and
internal-labels.

The $\lbt$'s with \emph{unique} labels are also defined recursively:
\begin{definition*}

  \inductioncase{Base case}: The $\lbt \ang{l,\STR{leaf}}$ has \emph{unique
    labels}.

  \inductioncase{Constructor case}: If $B$ and $C$ are $\lbt$'s with
  unique labels, no label of $B$ is a label $C$ and vice versa, and $l$ is
  not a label of $B$ or $C$, then $\ang{l, B, C}$ has \emph{unique labels}.
\end{definition*}

If $B$ is an $\lbt$, let $n_B$ be the number of distinct i\textbf{n}ternal-labels
appearing in $B$ and $f_B$ be the number of distinct lea\textbf{f} labels of $B$.
Prove by structural induction that
\begin{equation}\label{fbnb1}
f_B = n_B+1
\end{equation}
for all $\lbt$'s $B$ with unique labels.  This equation can obviously fail if
labels are not unique, so your proof had better use uniqueness of labels
at some point; be sure to indicate where.

\begin{solution}
\inductioncase{Base case}: If $B = \ang{l,\STR{leaf}}$, then $f_B=1$
and $n_B = 0$ so~\eqref{fbnb1} holds.

\inductioncase{Constructor case}: If $A \eqdef \ang{l, B,C}$ has unique labels, then by definition no label appears in both $B$ and $C$, and $l$ and the internal labels of $B$ and $C$ are disjoint, so:

\begin{align*}
f_A & =  \card{\text{leaf-labels}(B) \union \text{leaf-labels}(C)}
         & \text{(by def. of leaf labels)}\\
 & = f_{B} + f_{C} & \text{(no label appears in both $B$ and $C$)}\\
 & = (n_{B}+1) + (n_{C}+1)
          & \text{(by structural induction hypothesis)}\\
 &= (n_{B} + n_{C} +1) +1\\
 & = \card{\set{l} \union \text{internal-labels}(B) \union \text{internal
     labels}(C)} + 1
         &  \text{(uniqueness of labels)}\\
 & = n_A +1 & \text{(by def. of $n_A$)}.
\end{align*}

This proves~\eqref{fbnb1} holds for $A$, completing the proof of the
Constructor case.  It follows by structural induction that ~\eqref{fbnb1}
holds for all $\lbt$'s with unique labels.
\end{solution}

\end{problem}


%%%%%%%%%%%%%%%%%%%%%%%%%%%%%%%%%%%%%%%%%%%%%%%%%%%%%%%%%%%%%%%%%%%%%
% Problem ends here
%%%%%%%%%%%%%%%%%%%%%%%%%%%%%%%%%%%%%%%%%%%%%%%%%%%%%%%%%%%%%%%%%%%%%

\endinput

\iffalse Labeled binary trees (\lbt's) are a familiar data type in
computer science.  An \lbt\ tree has internal nodes that are each
connected by an edge to exactly two nodes below it, and it has leaf
nodes with no nodes below them.  For example,
Figure~\ref{PS_labeled_binary_tree}, pictures an \lbt\ with four
internal nodes labeled STR{root},\texttt{0},\texttt{1},\texttt{10} and five leaf nodes
labeled STR{00}, \texttt{01}, \texttt{100}, \texttt{101}, \texttt{11}.

\begin{figure}[htbp]
\graphic[height=3in]{cp5m-bintree}
\caption{A picture of a labeled binary tree with 4 internal nodes and 5 leaves.}
\label{PS_labeled_binary_tree}
\end{figure}

Formally, we can define a leaf to be a pair like $\ang{\STR{00},
  \STR{leaf}}$ consisting a label like \STR{00} and the tag
\STR{leaf}.  An internal node can be defined as a triple like
$\ang{\texttt{0}, B, C}$ consisting of its label \STR{0} and the triples or
pairs for two nodes $B$ and $C$ below it.  (By this definition, an
internal node is formally the same as the whole tree below it.)

\bparts
\ppart Write out (using angle brackets and labels \STR{bintree},
\STR{leaf}, etc.) the $\lbt$, pictured in
Figure~\ref{PS_labeled_binary_tree}.

\begin{solution}
\begin{equation}
\begin{split}
\langle\STR{bintree}, & \langle\STR{bintree}, \ang{\STR{leaf, win}},\\
  & \hspace{5em} \ang{\STR{bintree}, \ang{\STR{leaf, lose}},
       \ang{\STR{leaf, win}}}\rangle,\\
  & \ang{\STR{leaf, win}}\rangle
\end{split}
\end{equation}

\end{solution}

\fi
