\documentclass[problem]{mcs}

\begin{pcomments}
  \pcomment{CP_foundation_axiom}
  \pcomment{by ARM 2/15/11; staff hints added 8/30/11}
\end{pcomments}

\pkeywords{
 foundation
 axiom
 well_founded
}


%%%%%%%%%%%%%%%%%%%%%%%%%%%%%%%%%%%%%%%%%%%%%%%%%%%%%%%%%%%%%%%%%%%%%
% Problem starts here
%%%%%%%%%%%%%%%%%%%%%%%%%%%%%%%%%%%%%%%%%%%%%%%%%%%%%%%%%%%%%%%%%%%%%

\begin{problem}
Let $R: A \to A$ be a binary relation on a set $A$.  If $a_1 \mrel{R}
a_0$, we'll say that $a_1$ is ``$R$-smaller'' than $a_0$.  $R$ is
called \term{well founded} when there is no infinite
``$R$-decreasing'' sequence:
\begin{equation}\label{Rdecseq}
\cdots \mrel{R}  a_n \mrel{R}  \cdots \mrel{R}  a_1 \mrel{R}  a_0,
\end{equation}
of elements $a_i \in A$.

For example, if $A = \nngint$ and $R$ is the $<$-relation, then $R$ is
well founded because if you keep counting down with nonnegative integers,
you eventually get stuck at zero:
\[
0 < \cdots < n-1 <  n.
\]
But you can keep counting up forever, so the $>$-relation is not well founded:
\[
\cdots >  n >  \cdots >  1 > 0.
\]
Also, the $\le$-relation on $\nngint$ is not well founded because a
\emph{constant} sequence of, say, 2's, gets $\le$-smaller forever:
\[
\cdots \le 2  \le \cdots \le 2  \le 2.
\]

\bparts

\ppart If $B$ is a subset of $A$, an element $b \in B$ is defined to be
\emph{$R$-minimal in $B$} iff there is no $R$-smaller element in
$B$.  Prove that $R:A \to A$ is well founded iff every nonempty subset of
$A$ has an $R$-minimal element.

\begin{solution}
  If there was an infinite $R$-decreasing sequence~\eqref{Rdecseq}, then
  $\set{a_0,a_1,\dots}$ would itself be a nonempty subset of $A$ with no
  minimal element.  This proves the right-to-left direction of the
  `''iff'' (by contrapositive).

  We'll also prove the left-to-right direction by contrapositive.  So
  suppose $B$ is a nonempty subset of $A$ with no $R$-minimal element.  We
  will show how to find an infinite $R$-decreasing sequence of elements of
  $B$:

  Since $B$ is nonempty, there is an element $b_0 \in B$.  Since $b_0$
  cannot be minimal in $B$, there must be an element $b_1 \in B$ that is
  $R$-smaller than $b_0$.  Again, since $b_1$ cannot be minimal in $B$,
  there must be an $R$-smaller $b_2 \in B$.  Continuing in this way, we
  obtain an infinite $R$-decreasing sequence
\[
\cdots \mrel{R} b_n \mrel{R} \cdots \mrel{R} b_1 \mrel{R} b_0.
\]
\end{solution}

\eparts

\medskip

A logic \emph{formula of set theory} has only predicates of the form ``$x
\in y$'' for variables $x,y$ ranging over sets, along with quantifiers and
propositional operations.  For example,
\[
\text{isempty}(x) \eqdef \forall w.\, \QNOT(w \in x)
\]
is a formula of set theory that means that ``$x$ is empty.''

\bparts

\ppart\label{mbrmin} Write a formula $\text{member-minimal}(u, v)$ of set theory
that means that $u$ is $\in$-minimal in $v$.

\begin{solution}
\[
\text{member-minimal}(u, v) \eqdef \ \ u \in v \QAND \forall x \in v.\, x \notin u.
\]
\end{solution}

\ppart The Foundation axiom of set theory says that $\in$ is a well
founded relation on sets.  Express the Foundation axiom as a formula of
set theory.  You may use ``member-minimal'' and ``isempty'' in your
formula as abbreviations for the formulas defined above.

\begin{solution}
\[
\forall x.\  \QNOT(\text{isempty}(x)) \QIMPLIES \exists m.\, \text{member-minimal}(m, x).
\]
\end{solution}

\ppart Explain why the Foundation axiom implies that no set is a member of
itself.

\begin{solution}
If $x \in x$, then
\[
\cdots \in x \in \cdots \in x \in x
\]
is a $\in$-decreasing sequence, violating well foundedness of the
$\in$-relation.  Alternatively, $\set{x}$ would be a nonempty set with no
$\in$-minimal element.
\end{solution}

\begin{staffnotes}
Question:  How about a set being a member of a member of itself?
In other words, sets $S,T$  such that $S \in T \in S$?

Answer:  the would have infinite $\in$-decreasing sequence
\[
\dot S \in T \in S \in T \in S \in T, 
\]
violating Foundation.
\end{staffnotes}

\eparts

\end{problem}

%%%%%%%%%%%%%%%%%%%%%%%%%%%%%%%%%%%%%%%%%%%%%%%%%%%%%%%%%%%%%%%%%%%%%
% Problem ends here
%%%%%%%%%%%%%%%%%%%%%%%%%%%%%%%%%%%%%%%%%%%%%%%%%%%%%%%%%%%%%%%%%%%%%

\endinput
