\documentclass[problem]{mcs}

\begin{pcomments}
  \pcomment{PS_combinatorial_proof}
  \pcomment{from: S09 ps10}
  \pcomment{subsumes almost identical PS_given_combinatorial_proof}
\end{pcomments}

\pkeywords{
 combinatorial_proof
 binomial_coefficient
}

%%%%%%%%%%%%%%%%%%%%%%%%%%%%%%%%%%%%%%%%%%%%%%%%%%%%%%%%%%%%%%%%%%%%%
% Problem starts here
%%%%%%%%%%%%%%%%%%%%%%%%%%%%%%%%%%%%%%%%%%%%%%%%%%%%%%%%%%%%%%%%%%%%%

\begin{problem}

\bparts

\ppart
Find a combinatorial (\emph{not} algebraic) proof that
\[
\sum_{i=0}^n \binom{n}{i} = 2^n.
\]

%\hint Think about the binomial theorem.
\begin{solution}
%Both sides are the sum of the coefficients of $(x + y)^n$.

  There are $2^n$ subsets of $\set{1,2,\dots,n}$.  But the number of such
  subsets is sum of the number of subsets of size $k$, namely,
  $\binom{n}{k}$ for $0 \leq k\leq n$.
\end{solution}

\ppart
Below is a combinatorial proof of an equation.  What is the equation?

\begin{proof}
Stinky Peterson owns $n$ newts, $t$ toads, and $s$ slugs.
Conveniently, he lives in a dorm with $n + t + s$ other students.
(The students are distinguishable, but creatures of the same variety
are not distinguishable.)  Stinky wants to put one creature in each
neighbor's bed.  Let $W$ be the set of all ways in which this can be
done.

On one hand, he could first determine who gets the slugs.  Then, he could
decide who among his remaining neighbors has earned a toad.  Therefore,
$\card{W}$ is equal to the expression on the left.

On the other hand, Stinky could first decide which people deserve newts
and slugs and then, from among those, determine who truly merits a newt.
This shows that $\card{W}$ is equal to the expression on the right.

Since both expressions are equal to $\card{W}$, they must be equal to each
other.
\end{proof}

(Combinatorial proofs are real proofs.  They are not only rigorous,
but also convey an intuitive understanding that a purely algebraic
argument might not reveal.  However, combinatorial proofs are usually
less colorful than this one.)

\begin{solution}
\[
\binom{n + t + s}{s} \cdot \binom{n + t}{t}
    = \binom{n + t + s}{n + s} \cdot \binom{n + s}{n}
\]
\end{solution}

\eparts
\end{problem}

%%%%%%%%%%%%%%%%%%%%%%%%%%%%%%%%%%%%%%%%%%%%%%%%%%%%%%%%%%%%%%%%%%%%%
% Problem ends here
%%%%%%%%%%%%%%%%%%%%%%%%%%%%%%%%%%%%%%%%%%%%%%%%%%%%%%%%%%%%%%%%%%%%%

\endinput
