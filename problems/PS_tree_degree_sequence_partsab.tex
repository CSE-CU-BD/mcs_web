\documentclass[problem]{mcs}

\begin{pcomments}
\pcomment{PS_degree_sequence_connected}
\pcomment{ARM 4/5/17}
\end{pcomments}

\pkeywords{
 simple_graph
 tree
 degree
 handshaking
 induction
}

%%%%%%%%%%%%%%%%%%%%%%%%%%%%%%%%%%%%%%%%%%%%%%%%%%%%%%%%%%%%%%%%%%%%%
% Problem starts here
%%%%%%%%%%%%%%%%%%%%%%%%%%%%%%%%%%%%%%%%%%%%%%%%%%%%%%%%%%%%%%%%%%%%%

\begin{problem}
Suppose $G$ is a simple graph with $\vertices{G} = \set{v_1,\dots,
  v_n}$.  The \emph{degree sequence} $D_G$ of $G$ is the sequence of
vertex degrees, that is,
\[
D_G \eqdef \ang{\degr{v_1},\dots, \degr{v_n}}.
\]

Give an example of two simple graphs $G$ and $H$ with the same
degree sequence such that $G$ is connected and $H$ is not connected.

\begin{staffnotes}
I think the following 5-vertex graphs are the smallest possible based
on a quick (read: ``unreliable'') analysis of simple graphs with at
most 4 vertices.  Would be nice to have a clean proof of this.
\end{staffnotes}

\begin{solution}
\begin{align*}
\vertices{G} & = \vertices{H} \eqdef \Zinitv{1}{5},\\
\edges{G} \eqdef \set{\edge{1}{2}, \edge{2}{3}, \edge{1}{3}, \edge{4}{5}},\\
\edges{H} \eqdef \set{\edge{4}{1}, \edge{1}{2}, \edge{2}{3}, \edge{3}{5}}.
\end{align*}
So $D \eqdef (2,2,2,1,1)$.
\end{solution}

\textbf{Optional}: Argue that your example has the smalles possible
number of vertices.

\end{problem}
%%%%%%%%%%%%%%%%%%%%%%%%%%%%%%%%%%%%%%%%%%%%%%%%%%%%%%%%%%%%%%%%%%%%%
% Problem ends here
%%%%%%%%%%%%%%%%%%%%%%%%%%%%%%%%%%%%%%%%%%%%%%%%%%%%%%%%%%%%%%%%%%%%%

\endinput
 
