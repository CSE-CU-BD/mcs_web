 \documentclass[problem]{mcs}

\begin{pcomments}
\pcomment{PS_tree_degree_sequence_partsab}
\pcomment{PS_tree_degree_sequence w/o part (c), and new part before (a)}
\pcomment{ARM 4/5/17}
\end{pcomments}

\pkeywords{
 simple_graph
 tree
 degree
 handshaking
 induction
}

%%%%%%%%%%%%%%%%%%%%%%%%%%%%%%%%%%%%%%%%%%%%%%%%%%%%%%%%%%%%%%%%%%%%%
% Problem starts here
%%%%%%%%%%%%%%%%%%%%%%%%%%%%%%%%%%%%%%%%%%%%%%%%%%%%%%%%%%%%%%%%%%%%%

\begin{problem}
Suppose $G$ is a simple graph with $\vertices{G} = \set{v_1,\dots,
  v_n}$.  The \emph{degree sequence} $D_G$ of $G$ is the sequence of
vertex degrees, that is,
\[
D_G \eqdef \ang{\degr{1},\dots, \degr{n}}.
\]

\bparts

\ppart Give an example of two simple graphs $G$ and $H$ with the same
degree sequence such that $G$ is connected and $H$ is not connected.

\begin{staffnotes}
I think the following 5-vertex graphs are the smallest possible based
on a quick (read: ``unreliable'') analysis of simple graphs with at
most 4 vertices.  Would be nice to have a clean proof of this.
\end{staffnotes}

\begin{solution}
\begin{align*}
\vertices{G} & = \vertices{H} \eqdef \Zinitv{1}{5},\\
\edges{G} \eqdef \set{\edge{1}{2}, \edge{2}{3}, \edge{1}{3}, \edge{4}{5}},\\
\edges{H} \eqdef \set{\edge{4}{1}, \edge{1}{2}, \edge{2}{3}, \edge{3}{5}}.
\end{align*}
So $D \eqdef (2,2,2,1,1)$.
\end{solution}

\ppart Suppose $G$ is an $n$-vertex \emph{\textbf{tree}} with degree sequence
$\ang{d_1,\dots,d_n}$.  Explain why
\begin{equation}\tag{sum d's}
\sum_{i=1}^n d_i = 2(n-1).
\end{equation}

\begin{solution}
By the Handshaking Lemma\inbook{~\bref{sumedges}}, the sum of the
degrees of the vertices of $G$ is twice the number of edges, and since
$G$ is a tree with $n$ vertices, it has $n-1$ edges.
\end{solution}

\ppart Prove conversely that if $D$ is a sequence of positive integers
satisfying equation~(sum d's), then $D$ is the degree list of some
$n$-vertex tree.  \hint Induction.

\begin{solution}
The proof will be by induction on $n$ with induction hypothesis
\[
P(n) \eqdef \forall D.\, (D\ \text{satisfies~(sum
  d's))\ \QIMPLIES\ \exists\text{ tree } G.\, D = D_G.
 \]

\inductioncase{Base case}: ($n = 1$).  This holds vacuously since
there is no length-one sequence of positive numbers whose sum is zero.

\inductioncase{Induction step}: Suppose $D=
\ang{d_1,d_2,\dots,d_n,d_{n+1}}$ for $n \geq 1$ satisfies~(sum d's)
(with ``$n$'' replaced by ``$n+1$.''

\iffalse
\begin{equation}\tag(sum=2n}
\sum_1^{n+1} d_i = 2n.
\end{equation}\fi

We want to prove that $D$ is a list of the degrees of the vertices of
some tree $T$ with $n+1$ vertices.

irst notice that there must be an $i$ such that $d_i = 1$, otherwise
we would have that $d_{i}\geq 2$ for all $i$ and the
sum in~(sum d's) would be too large.  Similarly, there must be a
$j$ such that $d_j = 1$, otherwise the sum would be $n+1$, 
g%%%I don't follow this last remark.
which is less than $2n$ since $n\geq 2$.  Now we can assume without
loss of generality that $d_{n+1}=1$ and $d_{n} \geq 2$.

Let $D \eqdef (d_1,d_2,\dots, d_{n-1},d_n - 1)$.  Compared to
$D_{n+1}$, the sequence $D$ omits $d_{n+1} = 1$ and replaces $d_{n}$
by $d_{n}-1$.  So the sum of $D$ is two less than the
sum~\eqref{sumdn+1}, namely $2n-2 = 2(n-1)$.  So $D$ satisfies
equation~\eqref{eq:tree_degree_seq}.  Now by induction hypothesis, $D$
is the sequence of degrees of the vertices of some $n$ vertex tree
$T$.

Let $T_{n+1}$ be the $(n+1)$-vertex tree obtained from $T$ by adding a
new leaf to the $n$th vertex.  Then $D_{n+1}$ is a list of the vertex
degrees of $T_{n+1}$, which completes the proof of the inductive step.
\end{solution}

%%%%%%%%%%%%%%%%%%%%%%%%%%%%%%%%%%%%%%%%%%%%%%%%%%%%%%%%%%%%%%%%%%%%%
% Problem ends here
%%%%%%%%%%%%%%%%%%%%%%%%%%%%%%%%%%%%%%%%%%%%%%%%%%%%%%%%%%%%%%%%%%%%%

\endinput
 
