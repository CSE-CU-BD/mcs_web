\documentclass[problem]{mcs}

\begin{pcomments}
  \pcomment{PS_bijective_FLT_S16}
  \pcomment{ARM revise PS_bijective_FLT, 4/15/16}
  \pcomment{S16.ps7, S13.ps10, F11.MQ}
\end{pcomments}

\pkeywords{
  counting
  Fermat
  congruence
  equivalence
}

%%%%%%%%%%%%%%%%%%%%%%%%%%%%%%%%%%%%%%%%%%%%%%%%%%%%%%%%%%%%%%%%%%%%%
% Problem starts here
%%%%%%%%%%%%%%%%%%%%%%%%%%%%%%%%%%%%%%%%%%%%%%%%%%%%%%%%%%%%%%%%%%%%%

\begin{problem}
Fermat's Little Theorem~\bref{fermat_little}\footnote{This Theorem is usually stated as
\[
 a^{p-1}\equiv 1\pmod{p},
\]
for all primes $p$ and integers $a$ not divisible by $p$.  This
follows immediately from~\eqref{apeqvap} by canceling $a$.}
asserts that
\begin{equation}\label{apeqvap}
 a^p\equiv a\pmod{p}
\end{equation}
for all primes $p$ and nonnegative integers $a$.  This is immediate
for $a=0,1$ so we assume that $a \geq 2$.

This problem offers a proof of~\eqref{apeqvap} by counting strings
over a fixed alphabet with $a$ characters.


\bparts

\ppart\label{countak} How many length-$k$ strings are there over an
$a$-character alphabet? \hfill\examrule
\begin{solution}
$a^k$.
\end{solution}
How many of these are strings use more than one character? \hfill\examrule
\begin{solution}
$a^k-a$.

For each of the $a$ characters, there is exactly one length-$k$
string using only that character.
\end{solution}

\eparts

\medskip

Let $z$ be a length $k$ string.  The \emph{length-$n$ rotation} of $z$
is the string $yx$, where $z = xy$ and the length, $\lnth{x}$, of $x$
is $\remainder(n,k)$.

\bparts

\ppart\label{rotateagain} Verify that if $u$ is a length-$n$ rotation
of $z$, and $v$ is a length-$m$ rotation of $u$, then $v$ is a
length-($n+m$) rotation of $z$.

\begin{solution}
Suppose $z = wxy$ where $\lnth{w} = n, \lnth{x} = m$.  Then $u = xyw$,
and so $v = ywx$.  But $\lnth{wx} = n+m$, so $v$ is the length-$n+m$
rotation of $z$.  This argument applies to $n,m,n+m > \lnth{z}$ by
taking remainders on division by $\lnth{z}$.
\begin{staffnotes}
\TBA{Explain about remainders more clearly?}
\end{staffnotes}
\end{solution}

\ppart Let $\approx$ be the ``is a rotation of'' relation on strings.
That is,
\[
v \approx z \quad \QIFF\quad v\ \text{is a length-$n$ rotation of}\ z
\]
for some $n \in \nngint$.  Prove that $\approx$ is an equivalence
relation.

\begin{solution}
\begin{proof}
Reflexivity follows since everything is a length 0 rotation of itself.
Symmetry follows because $v$ is a length $m$ rotation of $z$ iff $z$
is a length $\lnth{z} - \remainder(m,\lnth{z})$ rotation of $v$.
Transitivity follows from part~\eqref{rotateagain}.
\end{proof}
\end{solution}

\ppart\label{xyyxm*} Prove that if $xy=yx$ then $x$ and $y$ each consist of
repetitions of some string $u$.  That is, if $xy=yx$, then $x,y \in
u^*$ for some string $u$.

\hint Let $u$ be the shortest positive length string such that $uy = yu$.

\begin{solution}
\begin{staffnotes}
\TBA{Very wordy, consider reducing it by writing down some of the arguments used multiple times as lemmas}
\end{staffnotes}

Since $yy = yy$ then we must have that $|u| \le |y|$. Therefore, since the string $uy$ starts with $u$ then the string $yu$ must also start with $u$, which implies that $y$ starts with $u$. 

Denote by $s^t$ a string that is composed of the concatenation of $t$ times the string $s$.  Suppose that $y$ does not consist of repetitions of the string $u$.  Let $y = u^kp$, where $k \ge 1$ and $p$ is a string that does not start with $u$. Then $uy = yu$ reduces to $up = pu$. Notice that if $p \ge u$ then, by the same argument as above, we would have that $p$ starts with $u$, which would contradict our assumption. Hence, $|p| < |u|$.  

Now notice that starting from the string $pu^k$, we can sequentially replace the string $pu$ with $up$ and obtain $u^kp$, i.e we have that $pu^k = u^kp$. Adding a string $p$ to both sides of this equality we obtain $pu^kp = u^kpp$ and therefore $py = yp$. This contradicts that $u$ is the shortest positive length string such that $uy = yu$. Hence, $y$ consists of repetitions of the string $u$, i.e $y = u^k$. 

Since $xy = yx$ then we must have $|x| \ge |u|$. Our original equation can now be written as $xu^k = u^kx$ which implies that $x$ starts with $u$. Suppose that $x$ does not consist of repetitions of the string $u$, i.e $x = u^rq$, with $r \ge 1$ and $q$ is a string that does not start with $u$. Then our equations reduces to $qu^k = u^kq$. Since $q$ does not start with $u$, we must have that $|q| < |u|$. However, since $u^k = y$ the last equation is equivalent to $qy = yq$, contradicting the minimality of $u$. Hence, $x$ also consists of repetitions of the string $u$.

\end{solution}

\ppart\label{exactlyp} Conclude that if $p$ is prime and $z$ is a
length-$p$ string containing at least two different characters, then
$z$ is equivalent under $\approx$ to exactly $p$ strings (counting
itself).

\begin{solution}
By part~\eqref{xyyxm*}, $z \in u^*$ for some string $u$, and $z$ is
$\approx$-equivalent to exactly $\lnth{u}$ strings.  In particular,
$\lnth{z}$ is a multiple of $\lnth{u}$.

But $\lnth{z} = p$, so $\lnth{u}$ must be 1 or $p$, and since $z$ has
more at least two different characters, $\lnth{u} \neq 1$.  So $z$ is
equivalent to exactly $p$ strings.
\end{solution}

\ppart Conclude from parts~\eqref{countak} and~\eqref{exactlyp} that
$p \divides (a^p -a$), which proves Fermat's Little
Theorem~\eqref{apeqvap}.

\begin{solution}
From part (a), there are $a^p-a$ length-$p$ strings made from an alphabet with $a$ characters that contain at least 2 different characters. From (e), if $p$ is prime, a length-$p$ string of at least two different characters is equivalent under $\approx$ to exactly $p$ other strings. Thus, if $p$ is prime, then $p | a^p-a$ and therefore $a^p \equiv a \pmod{p}$, as desired.
\end{solution}
\eparts

\end{problem}

%%%%%%%%%%%%%%%%%%%%%%%%%%%%%%%%%%%%%%%%%%%%%%%%%%%%%%%%%%%%%%%%%%%%%
% Problem ends here
%%%%%%%%%%%%%%%%%%%%%%%%%%%%%%%%%%%%%%%%%%%%%%%%%%%%%%%%%%%%%%%%%%%%%

\endinput
