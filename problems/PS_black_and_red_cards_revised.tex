\documentclass[problem]{mcs}

\begin{pcomments}
  \pcomment{PS_black_and_red_cards_revised}
  \pcomment{rewritten from PS_black_and_red_cards by ARM 5/2/12}
\end{pcomments}

\pkeywords{
  probability
  playing cards
}

%%%%%%%%%%%%%%%%%%%%%%%%%%%%%%%%%%%%%%%%%%%%%%%%%%%%%%%%%%%%%%%%%%%%%
% Problem starts here
%%%%%%%%%%%%%%%%%%%%%%%%%%%%%%%%%%%%%%%%%%%%%%%%%%%%%%%%%%%%%%%%%%%%%

\begin{problem}
We play a game with a deck of $52$ regular playing cards, of which
$26$ are red and $26$ are black.  I randomly shuffle the cards and
place the deck face down on a table.  You have the option of
``taking'' or ``skipping'' the top card.  If you skip the top card,
then that card is revealed and we continue playing with the remaining
deck.  If you take the top card, then the game ends; you win if the
card you took was revealed to be black, and you lose if it was red.
If we get to a point where there is only one card left in the deck,
you must take it.

If you've skipped cards to a point where there are $n$ cards left, and
$b$ of them are black, then since each of the $n$ remaining cards is
equally likely to be the top card, the probability that the top card
is black is $b/n$.  So in this situation, the probability that you win
by taking the top card is $b/n$.  Prove that there is no playing
strategy that gives you a higher probability of winning, that is, you
may as well always choose to take the top card.  In particular, no
strategy will give you a better than even chance of winning against
the 52 card deck.

\hint Induction on the size of the deck.

\begin{solution}

Let $p_{b,n}$ be your probability of winning using your best strategy
against a random deck with $n$ cards of which $b$ are black.  We prove
by induction on $n$ that
\begin{equation}\label{pbnleqf}
p_{b,n} \leq \frac{b}{n} \text{  for all } b \in [0, n].
\end{equation}

\inductioncase{base case} ($n=1$):  In this case you win iff the one
card in the deck is black.  So
\begin{align*}
p_{1,1} & = 1 = \frac{1}{1}\\
p_{0,1} & = 0 = \frac{0}{1}, 
\end{align*}
which shows that~\eqref{pbnleqf} holds in this case.

\inductioncase{inductive step}:

Suppose there are $b$ black card in a random deck of $n$ cards.

Then if you take the top card, we know your probability of winning is
$b/(n+1)$.

If you skip the top card, then you are left with a deck of size $n$.
If the top card was black, you are left with $b-1$ black cards in the
deck, which happens with with probability $b/(n+1)$.  Likewise, if the
top card was red, you are still left with $b$ black cards in the deck,
which happens with with probability $1-b/(n+1)$.  So if you play your
best strategy after skipping the top card, your probability of winning
is
\begin{align*}
\lefteqn{p_{b-1,n} \cdot \frac{b}{n+1} + p_{b,n} \cdot \paren{ 1- \frac{b}{n+1}}}
          &\text{(Total Probability)} \\
 & \leq \frac{b-1}{n} \cdot \frac{b}{n+1} + \frac{b}{n} \cdot \paren{1- \frac{b}{n+1}}
          & \text{(induction hypothesis~\eqref{pbnleqf})}\\
 & = \frac{(b-1)b+ b(n+1 -b}{n(n+1)}\\
 & = \frac{bn}{n(n+1)} = \frac{b}{n+1}\ .
\end{align*}

So whether you take or skip the top card, your probability of winning
is at most $b/(n+1)$, which proves
\[
p_{b,n+1} \leq \frac{b}{n+1}\ .
\]
The completes the inductive step, and we conclude that~\eqref{pbnleqf}
holds for all $n \in \integers^+$

\end{solution}

\end{problem}

%%%%%%%%%%%%%%%%%%%%%%%%%%%%%%%%%%%%%%%%%%%%%%%%%%%%%%%%%%%%%%%%%%%%%
% Problem ends here
%%%%%%%%%%%%%%%%%%%%%%%%%%%%%%%%%%%%%%%%%%%%%%%%%%%%%%%%%%%%%%%%%%%%%

\endinput
