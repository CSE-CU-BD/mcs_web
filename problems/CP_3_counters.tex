\documentclass[problem]{mcs}

\begin{pcomments}
  \pcomment{CP_3_counters}
  \pcomment{ARM 2.15.16}
\end{pcomments}

\pkeywords{
 counter
 counter_machine
 concatenation
 exponent
}

%%%%%%%%%%%%%%%%%%%%%%%%%%%%%%%%%%%%%%%%%%%%%%%%%%%%%%%%%%%%%%%%%%%%%
% Problem starts here
%%%%%%%%%%%%%%%%%%%%%%%%%%%%%%%%%%%%%%%%%%%%%%%%%%%%%%%%%%%%%%%%%%%%%


\begin{problem}
Describe how a 2-Counter machine $M_2$ could be designed to simulate
any 3-Counter machine, $M_3$ so that $M_2$ would halt iff $M_3$ would
halt.  In particular, suppose the contents of $M_3$'s counters $X,Y,Z$
are represented by an $M_2$ counter $R$ containing $2^x3^y5^z$.
Explain how the $M_2$ machine could update its representation to
reflect $M_3$ adding one, subtracting one, and branching on zero in
one of its counters.

\begin{solution}
To simulate adding 1 to $Y$ by multiplying the contents of $R$ by 3.
This is easy with two registers: just keep subtracting one $R$ and
adding one 3 times to a temp $T$ until $R$ contains zero..  This
leaves $3r$ in $T$.  Then copy $T$ back into $R$.

Similarly, simulate subtracting 1 by dividing by 3, and simulate an
$M_3$ branch on 0 by an $M_2$ branch on $r$ being 0 or
$\remainder{r}{3}$ not being being 0.
\end{solution}
\end{problem}

\endinput
