\documentclass[problem]{mcs}

\begin{pcomments}
  \pcomment{CP_bag_of_donuts}
  \pcomment{from: S09.cp12m, S07.cp11w, S06.cp11}
\end{pcomments}

\pkeywords{
 generating_function
 convolution
 counting
 partial_fraction
}

%%%%%%%%%%%%%%%%%%%%%%%%%%%%%%%%%%%%%%%%%%%%%%%%%%%%%%%%%%%%%%%%%%%%%
% Problem starts here
%%%%%%%%%%%%%%%%%%%%%%%%%%%%%%%%%%%%%%%%%%%%%%%%%%%%%%%%%%%%%%%%%%%%%

\begin{problem}
We are interested in generating functions for the number of different
ways to compose a bag of $n$ donuts subject to various restrictions.
For each of the restrictions in parts~\eqref{bagp1st}-\eqref{bagplast}
below, find a closed form for the corresponding generating function.

\bparts

\ppart\label{bagp1st} All the donuts are chocolate and there are at least 3.

\begin{solution}
There are no ways to select 0, 1, or 2 donuts, and one way to select
$n$ chocolate donuts for each $n>2$, so the generating function is
\[
x^3+x^4+x^5+\cdots = x^3 \paren{1+ x + x^2 + \cdots} = \frac{x^3}{1-x}
\]
\end{solution}

\ppart All the donuts are glazed and there are at most 2.

\begin{solution}
There is one way to select 0, 1, or 2 glazed donuts, and no ways to
select $n$ donuts for each $n>2$, so so the generating function is
\[
1+x+x^2.
\]
\end{solution}

\ppart All the donuts are coconut and there are exactly 2 or there are none.

\begin{solution}
\[
1+x^2
\]
\end{solution}

\ppart All the donuts are plain and their number is a multiple of 4.

\begin{solution}
The generating function is
\[
1 + x^4 + x^8 + \cdots + x^4n + \cdots = \sum_{i=0}^\infty \paren{x^4}^n =
\frac{1}{1-x^4}
\]
\end{solution}

\ppart\label{bagplast} The donuts must be chocolate, glazed, coconut, or plain with
the numbers of each flavor subject to the constraints above.

\iffalse
\begin{itemize}
\item there must be at least 3 chocolate donuts, and
\item there must be at most 2 glazed, and
\item there must be exactly 0 or 2 coconut, and
\item there must be a multiple of 4 plain.
\end{itemize}
\fi

\begin{solution}
By the Convolution Rule, the generating function for selecting donuts
with these constraints is the product of the preceding generating
functions:

\begin{align*}
\frac{x^3}{1-x}(1+x+x^2)(1+x^2) \frac{1}{1-x^4}
  &  = \frac{x^3(1+x+x^2)(1+x^2)}%
            {(1-x)^2(1+x)(1+x^2)}\\
  & = \frac{x^3(1+x+x^2)}{(1-x)^2(1+x)}
\end{align*}
\end{solution}

\medskip
\ppart Now find a closed form for the number of ways to select $n$
donuts subject to the above constraints.

\begin{solution}
We would like to convert the generating function
\[
\frac{x^3(1+x+x^2)}{(1-x)^2(1+x)}
\]
into partial fraction form.  This requires that the numerator have
lower degree than the denominator.  We could accomplish this by
expressing the ratio as a quotient and remainder, but in this case
another simple approach applies.  Namely, let
\[
G(x) \eqdef \frac{1+x+x^2}{(1-x)^2(1+x)},
\]
so the generating function for donut selections is $x^3G(x)$.  Now we
can express $G(x)$ in partial fraction form and then use the fact that
\[
[x^n]x^3G(x) = [x^{n-3}]G(x)
\]
to obtain the generating function coefficients from the coefficients
of $G(x)$.

Expanding $G(x)$ into partial fractions gives
\begin{equation}\label{Gp}
G(x) =  \frac{A}{1-x} + \frac{B}{(1-x)^2} + \frac{C}{1+x}
\end{equation}
for some constants, $A,B,C$.  We know that the coefficient of $x^n$ in the
series for ${(1-x)^2}$ is, by the Convolution Rule, the number of ways to
select $n$ items of two different kinds, namely, $\binom{n+1}{1}=n+1$, so
we conclude that the $n$th coefficient in the series for $G(x)$ is
\begin{equation}\label{coeff}
A + B(n+1) + C(-1)^n.
\end{equation}

To find $A,B,C$, we multiply both sides of~\eqref{Gp} by the denominator
$(1-x)^2(1+x)$ to obtain
\begin{equation}\label{1xx2}
1+x+x^2 =  A(1-x)(1+x) + B(1+x) + C(1-x)^2.
\end{equation}
Letting $x=1$ in~\eqref{1xx2}, we conclude that $3 = 2B$, so $B=3/2$.
Then, letting $x=-1$, we conclude $(-1)^2=C2^2$, so $C=1/4$.  Finally, letting
$x=0$, we have
\[
1= A+B+C = A+\frac{3}{2}+\frac{1}{4},
\]
so $A=-3/4$.  Then from~\eqref{coeff}, we conclude that
\[
[x^n]G(x) = -\frac{3}{4} + \frac{3(n+1)}{2} + \frac{(-1)^n}{4} = \frac{6n + 3 + (-1)^n}{4}.
\]
So the $n$th coefficient in the series for the generating function,
$x^3G(x)$, for donut selections is zero for $n<3$, and, for $n \ge 3$,
is $[x^{n-3}]G(x)$, namely,
\[
\frac{6(n-3) + 3 + (-1)^{n-3}}{4} = \frac{6n -15 - (-1)^n}{4}.
\]
\end{solution}
\eparts

\end{problem}

\endinput
