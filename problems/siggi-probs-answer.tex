\documentclass[12pt]{article}
\usepackage{amsfonts}
\usepackage{amssymb}
\usepackage{amsmath}
\usepackage{amsthm}
\usepackage{latexsym}
\usepackage{graphicx}
\usepackage{enumerate}
\theoremstyle{definition}
\newtheorem{definition}{Definition}
\newtheorem{claim}{Claim}

\author{Sigtryggur Kjartansson}
\begin{document}
\begin{enumerate}

\item Show that if $p$ is prime then $\binom{p-1}{k} \equiv(-1)^k
  \pmod p$ for $k \in [0,p)$.
\hint $(p-1)(p-2)\cdots(p-k) \equiv (-1)(-2)\cdots (-k) \pmod p$

\begin{proof}

\begin{align*}
\binom{p-1}{k}
 & \eqdef \frac{(p-1)(p-2)\cdots(p-k))}{1\cdot 2 \cdots k}\\
 & \equiv (-1\cdot -2\cdots -k) \cdot \paren{\inv{1}\cdot \inv{2} \cdots \inv{k}}\\
 & \equiv (-1 \cdot \inv{1})(-2\cdot \inv{2})\cdots (-k \cdot \inv{k})
 & \equiv (-1)^k \pmod p
\end{align*}

\end{proof}


\begin{proof}
\textbf{Proof by ordinary induction:}

Induction hypothesis: $$P(k):={p-1 \choose k}\equiv(-1)^k \mod p$$
\textbf{Base case: $k=0$}\\
${p-1 \choose 0}=1$
\\
\textbf{Inductive step}\\
Let's assume that $P(k-1)$ holds true.
If $k<p$
$${p-1 \choose k}={p-1 \choose k-1}\cdot\frac{p-k}{k}\equiv (-1)^{k-1}\cdot\frac{p-k}{k} \mod p$$
Now note that $$-k \equiv p-k \mod p$$
which implies that $\dfrac{p-k}{k} \equiv -1 \mod p$
Hence,
$${p-1 \choose k}\equiv (-1)^k$$
Hence it follows by the principle of mathematical induction that $P(k)$ holds $\forall\;\;  0\leq k \leq p-1$

\textbf{NOTE:} This obviously does not hold for $k\geq p$ because ${n \choose k}=0$ if $k>n$
\end{proof}


\item In 1874 Iceland got its first constitution but what is $3^{1874}
  \mod 6042$?

\begin{proof}
$$6042=2\cdot3\cdot19\cdot53$$
$$\Phi(6042)=(2-1)(3-1)(19-1)(53-1)=1\cdot2\cdot18\cdot52=1872$$
so by Euler's theorem: 
$$3^{1874}=3^{\Phi(6042)+2}=3^{\Phi(6042)}\cdot 3^2 \equiv 3^2\equiv 9 \mod 6042$$
\end{proof}

\item What is the remainder of $2^{100}+3^{100}+4^{100}+5^{100}$ when
  divided by 7?
\begin{proof}[Proof $\#1$]
$$\Phi(7)=6$$
and thus by Euler's theorem $a^6\equiv1 \mod 7$ if gcd($a,7$)$=1$.\\
Also $6|96$.\\
These imply the following
$$2^{100}\equiv 2^4\cdot2^{96}\equiv 16\equiv 2\mod7$$
$$3^{100}\equiv 3^4\cdot3^{96}\equiv 81 \equiv 4 \mod7$$
$$4^{100}\equiv 4^4\cdot4^{96}\equiv 256\equiv 4 \mod7$$
$$5^{100}\equiv 5^4\cdot5^{96}\equiv 625 \equiv 2\mod7$$
Hence,
$$2^{100}+3^{100}+4^{100}+5^{100}\equiv 2+4+4+2\equiv 12 \equiv 5 (mod7)$$
\end{proof}

\begin{proof}[Proof $\#2$]
$$2^3=8\equiv 1 \mod 7 \Rightarrow 2^{100}=2\cdot (2^3)^{33} \equiv 2\cdot 1^{33} \equiv 2 \mod 7$$
$$4^3=64\equiv 1 \mod 7 \Rightarrow 4^{100}=4\cdot (4^3)^{33} \equiv 4\cdot 1^{33} \equiv 4 \mod 7$$
$$3^3=27\equiv -1 \mod 7 \Rightarrow 3^{100}=3\cdot (3^3)^{33} \equiv 3\cdot (-1)^{33} \equiv -3 \mod 7$$
$$5^3=125\equiv -1 \mod 7 \Rightarrow 5^{100}=5\cdot (5^3)^{33} \equiv 5\cdot (-1)^{33} \equiv -5 \mod 7$$
Hence by the addition rule:
$$2^{100}+3^{100}+4^{100}+5^{100}\equiv 2+3-3-5=\equiv-2\equiv 5 \mod 7$$ 
\end{proof}

\item Use CRT (Chinese Remainder Theorem) to find consecutive integers
  $a,b, (a+1=b)$ such that $4|b$ and $9|a$.

\begin{proof}
The problem asks to find $a$ such that
$$a\equiv 0 \mod 9$$
$$a \equiv -1 \mod 4$$

Since gcd($4,9$)=1 we can use the CRT on this system of equations. To use CRT, we need to find $x,y$ such that $x\equiv \bar{4} \mod 9$ and $y\equiv \bar{9} \mod 4$

It's easy to show that $x\equiv 7 \mod 9$ and $y\equiv 1 \mod 4$

Then $$a\equiv \bar{4}\cdot4\cdot0+\bar{9}\cdot9\cdot(-1) \equiv -9
\equiv 27 \mod 36$$ So all candidate solutions are $(27+36k,28+36k)$
$\forall\,k\in\mathbb{Z}$
\end{proof}
\end{enumerate}
\end{document}
