\documentclass[problem]{mcs}

\begin{pcomments}
  \pcomment{FP_countable_sets}
  \pcomment{by drewe 12/03/11, from PS_add_countable_elements and
    TP_union_two_countable}
  \pcomment{overlaps MQ_countable_union}
\end{pcomments}


\pkeywords{
  countable
  union
  list
  bijection
  infinite
}

%%%%%%%%%%%%%%%%%%%%%%%%%%%%%%%%%%%%%%%%%%%%%%%%%%%%%%%%%%%%%%%%%%%%%
% Problem starts here
%%%%%%%%%%%%%%%%%%%%%%%%%%%%%%%%%%%%%%%%%%%%%%%%%%%%%%%%%%%%%%%%%%%%%

\begin{problem}
%\label{countable-union}

\bparts

\ppart Prove that if $A$ and $B$ are countable sets, then so is $A
\union B$.

\begin{solution}

\begin{proof}
Suppose the list of distinct elements of $A$ is $a_0,a_1,\dots$ and the
list of $B$ is $b_0,b_1, \dots$.  Then a list of all the elements in $A
\union B$ is just
\begin{equation}\label{FP_a0b0list}
a_0,b_0,a_1,b_1, \dots a_n,b_n, \dots.
\end{equation}
Of course this list will contain duplicates if $A$ and $B$ have elements
in common, but then deleting all but the first occurrences of each element in
list~\eqref{FP_a0b0list} leaves a list of all the distinct elements of $A$
and $B$.
\end{proof}

\end{solution}

\examspace[3.0in]

\ppart

Prove that if $C$ is a countable set and $D$ is infinite, then
there is a bijection between $D$ and $C \union D$.

\begin{solution}
\TBA{fill in}
\end{solution} 
\eparts

\end{problem}

%%%%%%%%%%%%%%%%%%%%%%%%%%%%%%%%%%%%%%%%%%%%%%%%%%%%%%%%%%%%%%%%%%%%%
% Problem ends here
%%%%%%%%%%%%%%%%%%%%%%%%%%%%%%%%%%%%%%%%%%%%%%%%%%%%%%%%%%%%%%%%%%%%%

\endinput
