\documentclass[problem]{mcs}

\begin{pcomments}
  \pcomment{FP_asymptotics_proof}
  \pcomment{CH, Spring '14; edited ARM 4.1.14 10:15PM}
  \pcomment{Forked from MQ_asymptotics_proof}
\end{pcomments}

\pkeywords{
	asymptotics
        big_Oh
        asymptotic_equality
}

%%%%%%%%%%%%%%%%%%%%%%%%%%%%%%%%%%%%%%%%%%%%%%%%%%%%%%%%%%%%%%%%%%%%%
% Problem starts here
%%%%%%%%%%%%%%%%%%%%%%%%%%%%%%%%%%%%%%%%%%%%%%%%%%%%%%%%%%%%%%%%%%%%%

\begin{problem}

The two parts of this problem can be done in either order.

\bparts

\ppart Describe a very simple example of two functions $f, g : \naturals \to \naturals$ such that
\[
 f = \Theta(g)\ \QAND\  2^f = o(2^g).
\]
Prove it.

\examspace[2in]

\begin{solution}
Let $f(n) = n$ and $g(n) = 2n$.  Then certainly $f = O(g)$.  But $g =
(1/2)f$ so $g = O(f)$.  That is, $f = \Theta(g)$.

Also,
\[
\frac{2^f}{2^g} = \frac{2^n}{2^{2n}} = \frac{1}{2^{n}}
\]
and
\[
\lim_{n\to\infty} \frac{1}{2^{n}} = 0,
\]
so $f = o(g)$.
\end{solution}


\ppart Prove that if $f = O(g)$, then $f^2 = O(g^2)$.

\begin{solution}
We know that $f = O(g)$ iff there exists a constant $c \geq 0$ and an
$n_0$ such that $f(n) \leq c g(n)$ for all $n \geq n_0$.  So
\[
f^2(n) \leq c' g^2(n)
\]
for all $n \geq n_0$, where $c' \eqdef c^2$.
This implies $f^2 = O(g^2)$.
\end{solution}

%\examspace[2in]

\iffalse

\ppart\label{fsimg2} Let $f(x) = x^2 + x$ and $g(x) = x^2$.  Prove that
$f = O(g)$.

\begin{solution}
We have that 
\[
f(x) = x^ + x \leq 2x^2 = 2g(x) .
\]
for all $x \leq 1$. Therefore, $f = O(g)$.

In fact, we can claim a stronger result.  We have that 
 \[
\lim_{x \to \infty} \frac{f(x)}{g(x)} = \lim_{x \to \infty} \frac{x^2
-  + x}{x} = \lim_{x\to\infty} \frac{1}{x} + 1 \to 1,
 \]
which means that $f \sim g$, i.e., $f$ and $g$ are asymptotically
equal. 
\end{solution}

\examspace[2in]
\fi


\eparts
\end{problem}

\endinput
