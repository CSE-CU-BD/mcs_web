\documentclass[problem]{mcs}

\begin{pcomments}
  \pcomment{FP_asymptotics_proof}
  \pcomment{CH, Spring '14; edited ARM 4.1.14 10:15PM}
  \pcomment{Forked from MQ_asymptotics_proof}
\end{pcomments}

\pkeywords{
	asymptotics
        big_Oh
        asymptotic_equality
}

%%%%%%%%%%%%%%%%%%%%%%%%%%%%%%%%%%%%%%%%%%%%%%%%%%%%%%%%%%%%%%%%%%%%%
% Problem starts here
%%%%%%%%%%%%%%%%%%%%%%%%%%%%%%%%%%%%%%%%%%%%%%%%%%%%%%%%%%%%%%%%%%%%%

\begin{problem}

Let $f, g : \mathbb{R} \rightarrow \mathbb{R}$ be nonnegative
  functions. 

\bparts

\ppart Prove that if $f = O(g)$, then $f^2 = O(g^2)$.

\begin{solution}
We know that $f = O(g)$ iff there exists a constant $c \geq 0$ and an
$x_0$ such that for all $x \geq x_0$, $f(x) \leq c g(x)$.  Therefore,
there exists a constant $c' = c^2$ such that $f^2(x) \leq c^2 g^2(x)$
for all $x \geq x_0$.  Therefore, $f^2 = O(g^2)$.
\end{solution}

\examspace[2in]

\ppart\label{fsimg2} Let $f(x) = x^2 + x$ and $g(x) = x^2$.  Show that $f \sim g$.

\begin{solution}
We have that 
\[
\lim_{x \to \infty} \frac{f(x)}{g(x)} = \lim_{x \to \infty} \frac{x^2
  + x}{x} = \lim_{x\to\infty} \frac{1}{x} + 1 \to 1,
\]
which means that $f \sim g$.

\end{solution}

\examspace[2in]

\ppart Give a counter-example to the claim that
\[
 f \sim g\ \QIMPLIES\ 2^f = O(2^g).
\]

\hint Part~\eqref{fsimg2}.

\begin{solution}
Let $f(x) = x^2 + x$ and $g(x) = x^2$.  We know that that $f \sim g$
from part~\eqref{fsimg2}.  \iffalse But $2^f = 2^{x^2 + x}$ and $2^g =
2^{x^2}$.\fi Therefore,
\[
2^{f(x)} = 2^{x^2 + x} = 2^x \cdot 2^{x^2} = 2^x \cdot 2^{g(x)} > c \cdot 2^{g(x)}
\]
for any $c \in \reals^+$ and all $x > \log_2 c$.  Hence $2^f \neq
O(2^g)$.

\iffalse
\[
\lim_{x \to infty} \frac{2^f}{2^f} = \lim_{x\to\infty} \frac{2^{x^2 +
    x}}{2^{x^2}} = \lim_{x\to\infty} 2^x \to \infty, 
\]
which shows that $2^f \sim 2^g$ is false.\fi
\end{solution}

\eparts
\end{problem}

\endinput
