\documentclass[problem]{mcs}

\begin{pcomments}
  \pcomment{FP_graph_isomorphism_multiple}
  \pcomment{from FP_multiple_choice_unhidden_fall13}
  \pcomment{revised from FP_multiple_choice_unhidden by ARM 12/13/13}
  \pcomment{overlaps FP_graphs_short_answer}  
\end{pcomments}

\pkeywords{
  graph
  isomorphism
  connected
  cycle
  vertices
}

%%%%%%%%%%%%%%%%%%%%%%%%%%%%%%%%%%%%%%%%%%%%%%%%%%%%%%%%%%%%%%%%%%%%%
% Problem starts here
%%%%%%%%%%%%%%%%%%%%%%%%%%%%%%%%%%%%%%%%%%%%%%%%%%%%%%%%%%%%%%%%%%%%%

\begin{problem} \mbox{}

Circle the simple graph properties below that are  \textbf{preserved under
isomorphism}.

\bparts


\ppart There is a cycle that includes all the vertices.

\ppart The vertices can be numbered 1 through $2^4$.

\ppart\label{eqlnth} Two edges are of equal length.

%\ppart The graph remains connected if any two edges are removed.

\ppart There exists an edge that is an edge of every spanning tree.

%\ppart The negation of a property that is preserved under isomorphism.

%\ppart There are exacty two spanning trees.

\ppart The $\QOR$ of two properties that are preserved under isomorphism.

%\ppart The graph remains connected if a vertex is removed.

\begin{solution}
\eqref{eqlnth} is a property of drawings, not of graphs.  It doesn't
make sense to say it is preserved by isomorphism.
\end{solution}

\eparts

\end{problem}

\endinput
