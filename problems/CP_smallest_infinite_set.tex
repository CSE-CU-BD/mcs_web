\documentclass[problem]{mcs}

\begin{pcomments}
  \pcomment{CP_smallest_infinite_set}
  \pcomment{from: Ch. ``sets'', notes problem, adapted by ARM 9/20/09}
\end{pcomments}

\pkeywords{
  surjection
  surj
  as_big_as
  axiom_of_choice}

%%%%%%%%%%%%%%%%%%%%%%%%%%%%%%%%%%%%%%%%%%%%%%%%%%%%%%%%%%%%%%%%%%%%%
% Problem starts here
%%%%%%%%%%%%%%%%%%%%%%%%%%%%%%%%%%%%%%%%%%%%%%%%%%%%%%%%%%%%%%%%%%%%%

\begin{problem}


\bparts

\ppart
Several students felt the proof of Lemma~\bref{AUb} was worrisome, if not
circular.  What do you think?

\inhandout{
\textbox{\noindent Lemma~\bref{AUb}.
  Let $A$ be a set and $b \notin A$.  If $A$ is infinite, then there is a
  bijection from $A \union \set{b}$ to $A$.

\begin{proof}
Here's how to define the bijection: since $A$ is infinite, it certainly has
at least one element; call it $a_0$.  But since $A$ is infinite, it has at
least two elements, and one of them must not be equal to $a_0$; call this
new element $a_1$.  But since $A$ is infinite, it has at least three
elements, one of which must not equal $a_0$ or $a_1$; call this new
element $a_2$.  Continuing in the way, we conclude that there is an
infinite sequence $a_0,a_1,a_2,\dots,a_n,\dots$ of different elements of
$A$.  Now we can define a bijection $f: A \union \set{b} \to A$:
\begin{align*}
f(b) & \eqdef a_0,\\
f(a_n) & \eqdef a_{n+1}  &\text{ for } n \in \naturals,\\
f(a) & \eqdef a & \text{ for } a \in A - \set{a_0,a_1,\dots}.
\end{align*}
\end{proof}
}
}

\begin{solution}
There is no ``solution'' for this discussion problem, since it depends
on what seems bothersome.

One issue that puzzles some students (when they are challenged about
it) is why the third clause in the definition of $f$ is needed since
$f$ is already defined on all the $a_n$'s.  The answer is that there
may be elements left over in $A$, and to be a bijection, the value of
$f$ on each ``left-over'' element of $A$ has to be defined somehow.
In fact, if $A$ is uncountable, there are guaranteed to be such
left-over elements.  But even when $A$ is countable, we can't assume
that all the elements in $A$ are included among the $a_i$'s.  All we
know about the $a_i$'s is they define a countably infinite
\emph{subset} of $A$.  This means that it is \emph{possible} that $A$
is equal to the set of all the $a_i$'s, but not necessarily.  For
example, $A$ might be $\naturals$ itself, while the $a_i$'s might be
the set of all nonnegative even integers, or all the prime numbers, or
all the powers of two, \dots.

It may also be bothersome that $f$ is asserted to be a bijection
without spelling out a proof.  But the bijection property really does
follow directly from the definition of $f$, so it shouldn't be much burden
for a bothered reader to fill in such a proof.

Another possibly bothersome point is that the proof assumes that if a set
is infinite, it must have more than $n$ elements, for every nonnegative
integer $n$.  But really that's the definition of infinity: a set is
finite iff it has $n$ elements for some nonnegative integer $n$ and a
set is infinite iff it is \emph{not} finite.

A further possibly worrisome point is how you find an element $a_{n+1}
\in A$ given $a_0,a_1,\dots,a_n$.  But you don't have to \emph{find} a
specific one: there must be an element in $A
-\set{a_0,a_1,\dots,a_n}$---so just pick any one.  Actually, the
justification for this step is the set-theoretic Axiom of Choice
described in \inhandout{the text}\inbook{Section~\bref{ZFC_sec}}, and
some logicians do consider it worrisome.
\end{solution}


\ppart Use the proof of Lemma~\bref{AUb} to show that if $A$ is an
infinite set, then $A \surj \naturals$, that is, every infinite set is
``at least as big as'' the set of nonnegative integers.

\begin{solution}

  By the proof of Lemma~\bref{AUb}, there is an infinite sequence
  $a_0,a_1,a_2,\dots,a_n,\dots$ of different elements of $A$.  Then we can
  define a surjective function $f:A \to \naturals$ by defining
\[
f(a) \eqdef \begin{cases}
               n, & \text{if } a= a_n,\\
               \text{undefined}, & \text{otherwise}.
              \end{cases}
\]
A total surjective function is not required, so it is OK that $f$ is a
partial function.  but if you want a total one, define $f':A \to
\naturals$, by
\[
f'(a) \eqdef \begin{cases}
               n, & \text{if } a= a_n,\\
               0, & \text{otherwise}.
              \end{cases}
\]
\end{solution}

\eparts

\end{problem}

\endinput
