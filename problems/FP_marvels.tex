\documentclass[problem]{mcs}

\begin{pcomments}
  \pcomment{FP_marvels}
  \pcomment{misattribution?: proposed by Justin Zhang, for 2009 fall final}
\end{pcomments}

\pkeywords{
  Fibonacci
  Stirling
  Gamma_function
  prime
}

%%%%%%%%%%%%%%%%%%%%%%%%%%%%%%%%%%%%%%%%%%%%%%%%%%%%%%%%%%%%%%%%%%%%%
% Problem starts here
%%%%%%%%%%%%%%%%%%%%%%%%%%%%%%%%%%%%%%%%%%%%%%%%%%%%%%%%%%%%%%%%%%%%%


\begin{problem}
  Paula is facinated about the following equations: 

\begin{align} 
12345679 \times 9 = 111111111 \\ 
\frac{\pi}{2} = \frac{2}{1} \times \frac{2}{3} \times \frac{4}{3} \times \frac{4}{5}
\times \frac{6}{5} \times \cdots \\
n! \approx (\frac{n}{e})^n \sqrt{2\pi n} \\
\Gamma(1/2) = \sqrt(\pi) \textrm{ where } \Gamma(z) = \int_0^\infty
t^{z-1} e^{-t} dt \\
\textrm{number of prime numbers smaller than x} \approx \frac{x}{\log
  x} \\
\varphi = 1+ \frac{1}{  1+ \frac{1}{1 + \frac{1}{1 + \cdots}}} \textrm
{  where $\varphi$ is the golden ratio} \\
\lim_{n \rightarrow \infty} \frac{F_{n+1}}{F_n} = \varphi \textrm
{  where $F_n$ is the Fibonacci series} \\
e = 1 + \frac{1}{1!} + \frac{1}{2!} + \frac{1}{3!} + \frac{1}{4!} + \cdots
\end{align}

Pick two and explain how you might prove it, and what kind of proving
techniques you are proposing. 

\begin{solution}
(1) prove by trivially evaluating the left side.

(2) ... pi

(3) covered in class.

(4) ... pi

(5) proved in class.

(6) in class

(7) in class, using Fn formula... from generating function

(8) ...? 

\end{solution}

\end{problem}


%%%%%%%%%%%%%%%%%%%%%%%%%%%%%%%%%%%%%%%%%%%%%%%%%%%%%%%%%%%%%%%%%%%%%
% Problem ends here
%%%%%%%%%%%%%%%%%%%%%%%%%%%%%%%%%%%%%%%%%%%%%%%%%%%%%%%%%%%%%%%%%%%%%


\endinput
