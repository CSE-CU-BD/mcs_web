\documentclass[problem]{mcs}

\begin{pcomments}
\pcomment{PS_inclusion-exclusion_paths}
\pcomment{F13.ps10}
\pcomment{perturbed from PS_inclusion-exclusion_paths_afternoon, PS_path_counting}
\pcomment{ARM 11/7/13} 
\end{pcomments}

\pkeywords{
  counting
  inclusion_exclsuion
  paths
  multinomial
}

%%%%%%%%%%%%%%%%%%%%%%%%%%%%%%%%%%%%%%%%%%%%%%%%%%%%%%%%%%%%%%%%%%%%%
% Problem starts here
%%%%%%%%%%%%%%%%%%%%%%%%%%%%%%%%%%%%%%%%%%%%%%%%%%%%%%%%%%%%%%%%%%%%%

\begin{problem}
We want to count step-by-step paths between points with integer
coordinates in three dimensions.  A step may move a unit distance in
the positive $x$, $y$ or $z$ direction.  For example, a step from
point $(2,3,7)$ in the $y$ direction leads to $(2,4,7)$.

For points $\mathbf{p}$ and $\mathbf{q}$ we write $\mathbf{p} \leq
\mathbf{q}$ to mean that $\mathbf{p}$ is coordinatewise less than or
equal to $\mathbf{q}$.  That is, if $\mathbf{p} = (p_x,p_y,p_z)$ and
$\mathbf{q} = (q_x,q_y,q_z)$, then
\[
\mathbf{p} \leq \mathbf{q} \quad \eqdef \quad
    [p_x \leq q_x \QAND\ p_y \leq q_y \QAND\ p_z \leq q_z].
\]
So there is a path from $\mathbf{p}$ to $\mathbf{q}$ iff $\mathbf{p}
\leq \mathbf{q}$.

\bparts 

\ppart Let $P_{\set{\mathbf{p},\mathbf{q}}}$ be the set of paths from
$\mathbf{p}$ to $\mathbf{q}$.  Suppose that $\mathbf{p} \leq
\mathbf{q}$, and let $d_x \eqdef q_x-p_x$, and likewise for $d_y$ and
$d_z$.  Express the number of paths
$\card{P_{\set{\mathbf{p},\mathbf{q}}}}$ as a multinomial coefficient
involving the preceding quantities.

\begin{solution}
There is an obvious bijection between the set
$P_{\set{\mathbf{p},\mathbf{q}}}$ and the strings consisting of $d_x$
$x$'s, $d_y$ $y$'s, and $d_z$ $z$'s.  So
\[
\card{P_{\set{\mathbf{p},\mathbf{q}}}} = \binom{d_x + d_y + d_y}{d_x,d_y,d_z}.
\]
\end{solution}

\eparts
More generally, for any set $S$ of points, let
\[
P_S \eqdef \text{the paths that go through all the points in } S.
\]

\bparts

\ppart Suppose $\mathbf{a} \leq \mathbf{b} \leq \mathbf{c} \leq
\mathbf{d}$.  Express $\card{P_{\set{\mathbf{a}, \mathbf{b},
    \mathbf{c}, \mathbf{d}}}}$ in terms of
$\card{P_{\set{\mathbf{p},\mathbf{q}}}}$ for various
$\mathbf{p},\mathbf{q} \in \set{\mathbf{a}, \mathbf{b}, \mathbf{c},
  \mathbf{d}}$.
\begin{solution}
\[
\card{P_{\set{\mathbf{a}, \mathbf{b}, \mathbf{c}, \mathbf{d}}}} = 
\card{P_{\set{\mathbf{a},\mathbf{b}}}} \cdot
\card{P_{\set{\mathbf{b},\mathbf{c}}}} \cdot
\card{P_{\set{\mathbf{c},\mathbf{d}}}}
\]
\end{solution}

\ppart Let
\begin{align*}
\mathbf{o} & \eqdef (0,0,0),\\
\mathbf{a} & \eqdef (3,7,11),\quad \mathbf{b} \eqdef (11,6,3),\quad \mathbf{c} \eqdef (10,5,40),\\
\mathbf{d} & \eqdef (12,13,14),\quad \mathbf{e} \eqdef (12,6,45),\\
\mathbf{f} & \eqdef (50,50,50).
\end{align*}

Let $N$ be the paths in $P_{\mathbf{o},\mathbf{f}}$ that do
\emph{not} go through any of $\mathbf{a},
\mathbf{b},\mathbf{c},\mathbf{d},\mathbf{e}$.  Express $\card{N}$ as
an arithmetic combination of $\card{P_S}$ for various $S \subseteq
\set{\mathbf{o},\mathbf{a},
  \mathbf{b},\mathbf{c},\mathbf{d},\mathbf{e},\mathbf{f}}$.  Do not
include any terms $\card{P_S}$ that equal zero.

\begin{solution}
$N$ is the union of the points that do \emph{not} go through at least
  one of $\mathbf{a}, \mathbf{b},\mathbf{c},\mathbf{d},\mathbf{e}$.
  So it equals $\card{P_{\mathbf{o},\mathbf{f}}}$ minus the number $k$ of
  paths that go through at least one of $\mathbf{a},
  \mathbf{b},\mathbf{c},\mathbf{d},\mathbf{e}$.

By Inclusion-Exclusion

\begin{align*}
k & =
\card{P_{\mathbf{o},\mathbf{a},\mathbf{f}}} +
\card{P_{\mathbf{o},\mathbf{b},\mathbf{f}}} +
\card{P_{\mathbf{o},\mathbf{c},\mathbf{f}}}\\
&\quad + \card{P_{\mathbf{o},\mathbf{d},\mathbf{f}}} + \card{P_{\mathbf{o},\mathbf{e},\mathbf{f}}}\\
%&\quad - \card{P_{\mathbf{o},\mathbf{a},\mathbf{c},\mathbf{f}}}
- \card{P_{\mathbf{o},\mathbf{a},\mathbf{d},\mathbf{f}}}
- \card{P_{\mathbf{o},\mathbf{b},\mathbf{d},\mathbf{f}}}
- \card{P_{\mathbf{o},\mathbf{b},\mathbf{e},\mathbf{f}}}
- \card{P_{\mathbf{o},\mathbf{c},\mathbf{e},\mathbf{f}}}
\end{align*}
\end{solution}

\eparts

\end{problem}

\endinput
