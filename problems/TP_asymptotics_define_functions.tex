\documentclass[problem]{mcs}

\begin{pcomments}
  \pcomment{TP_asymptotics_define_functions}
  \pcomment{lst part of FP_asymptotics_define_functions}
\end{pcomments}

\pkeywords{
  asymptotics
  little_oh
  big_oh
  Theta
  asymptotically_equal
  partial_order
  equivalence_relation
  implies
}

%%%%%%%%%%%%%%%%%%%%%%%%%%%%%%%%%%%%%%%%%%%%%%%%%%%%%%%%%%%%%%%%%%%%%
% Problem starts here
%%%%%%%%%%%%%%%%%%%%%%%%%%%%%%%%%%%%%%%%%%%%%%%%%%%%%%%%%%%%%%%%%%%%%

\begin{problem} %\textbf{Asymptotics}
Define two functions $f, g$ that are incomparable under big Oh:
\[
f \neq O(g) \QAND g \neq O(f).
\]

\examspace[1.5in]

\begin{solution}
One example is,
\[
f(n) \eqdef \begin{cases}            
n & \text{if $n$ is odd},\\
0   & \text{if $n$ is even}, 
\end{cases}\qquad
g(n) \eqdef \begin{cases}
0 & \text{if $n$ is odd},\\
n   & \text{if $n$ is even}, 
\end{cases}
\]
which can also be described by the formulas
\[
f(n) \eqdef n\, \sin^2\paren{\frac{n\pi}{2}}, \qquad g(n) \eqdef n\, \cos^2\paren{\frac{n\pi}{2}}.
\]

Some students also noticed that a corner case in the book's limit-based definition of big Oh allows $f(x) = g(x) = 0$.  We're going to try to revise the book to avoid that pathology in the future!
\end{solution}

\end{problem}


%%%%%%%%%%%%%%%%%%%%%%%%%%%%%%%%%%%%%%%%%%%%%%%%%%%%%%%%%%%%%%%%%%%%%
% Problem ends here
%%%%%%%%%%%%%%%%%%%%%%%%%%%%%%%%%%%%%%%%%%%%%%%%%%%%%%%%%%%%%%%%%%%%%

\endinput
