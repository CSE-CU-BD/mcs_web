\documentclass[problem]{mcs}

\newcommand{\asciibet}{\text{ASCII}}
\newcommand{\asciistr}{\strings{\asciibet}}

\begin{pcomments}
  \pcomment{PS_infinite_sets_and_number_theory}
  \pcomment{from: F13.ps4}
\end{pcomments}

\pkeywords{
  infinite sets
  number theory
}

%%%%%%%%%%%%%%%%%%%%%%%%%%%%%%%%%%%%%%%%%%%%%%%%%%%%%%%%%%%%%%%%%%%%%
% Problem starts here
%%%%%%%%%%%%%%%%%%%%%%%%%%%%%%%%%%%%%%%%%%%%%%%%%%%%%%%%%%%%%%%%%%%%%

\begin{problem}

\bparts

\ppart\label{a_strict_b_strict_c} Define sets $A$ and $B$ such that $\mathbb{N} \text{ strict } A \text{ strict } B$.

\begin{solution}
$A = \text{pow}(\mathbb{N})$ and $B = \text{pow}(A)$, for example.
\end{solution}

\ppart\label{number-of-programs} Prove or disprove the following claim:  \emph{The set of computer programs is uncountable.}

\hint Refer to the definitions of \emph{$\asciistr$} and \emph{string procedures} from Section~\bref{halting_sec} of the textbook.

\begin{solution}
\emph{False.}

Consider a numeral system where the available digits are exactly the set of $\asciibet$ characters (the numeric value of each character is inconsequential).  Counting in this numeral system equates to enumerating the elements of $\asciistr$, so $\asciistr \text{ bij } \mathbb{N}$.  Let $P$ be the set of string procedures.  For every string procedure $P_s \in P$ there exists at least one string $s \in \asciistr$ which describes it, so $\asciistr \text{ surj } P$, and therefore $\mathbb{N} \text{ surj } P$.  We can define a bijection between $\mathbb{N}$ and $P$ by skipping over repeated string procedures when pairing elements of $P$ with elements of $\mathbb{N}$.  Therefore, $P$ is countable.
\end{solution}

\ppart\label{remainder-arithmetic} Compute $\rem{2498754270^{184638463}9234673466^{844759364}){22}$.

\hint $18^9 \equiv 8 \mod 22$

\begin{solution}
First, we replace the bases of the exponents with their remainders.

$$\text{rem}((12^{184638463})(18^{844759364}), 22)$$

Now, let's examine the remainders of the first few powers of $12$.

\begin{align*}
  \text{rem}(12, 22) &= 12 \\
  \text{rem}(12^2, 22) &= 12 \\
  \text{rem}(12^3, 22) &= 12 \\
  \cdots
\end{align*}

The remainder is always $12$.  So we can now write

$$\text{rem}((12)(18^{844759364}), 22).$$

Now let's look at the remainders of powers of $18$.

\begin{align*}
  \text{rem}(18, 22) &= 18 \\
  \text{rem}(18^2, 22) &= 16 \\
  \text{rem}(18^3, 22) &= 2 \\
  \text{rem}(18^4, 22) &= 14 \\
  \text{rem}(18^5, 22) &= 10 \\
  \text{rem}(18^6, 22) &= 4 \\
  \text{rem}(18^7, 22) &= 6 \\
  \text{rem}(18^8, 22) &= 20 \\
  \text{rem}(18^9, 22) &= 8 \\
  \text{rem}(18^{10}, 22) &= 12 \\
  \text{rem}(18^{11}, 22) &= 18 \\
  \cdots
\end{align*}

So it repeats after $11$ steps.  This means we can keep subtracting $11$ from the exponent until the computation becomes feasable.  In other words, we can replace the exponent with its remainder mod $11$.

\begin{align*}
  \text{rem}((12)(18^9), 22) &= \text{rem}((12)(8), 22) \\
  &= 8
\end{align*}

\end{solution}

\eparts
\end{problem}

%%%%%%%%%%%%%%%%%%%%%%%%%%%%%%%%%%%%%%%%%%%%%%%%%%%%%%%%%%%%%%%%%%%%%
% Problem ends here
%%%%%%%%%%%%%%%%%%%%%%%%%%%%%%%%%%%%%%%%%%%%%%%%%%%%%%%%%%%%%%%%%%%%%

\endinput
