\documentclass[problem]{mcs}

\begin{pcomments}
  \pcomment{PS_infinite_ordinals}
  \pcomment{generalizes CP_finite_ordinals}
  \pcomment{subsumed by PS_infinite_ordinals_full}
  \pcomment{revised ARM 3/13/18 from original 2/20/14}
\end{pcomments}

\pkeywords{
  sets
  set_theory
  subset
  union
  ordinal
  member
}

\newcommand{\nextset}[1]{\text{next}(#1)}
\newcommand{\Ord}{\text{Ord}}

\begin{problem}
For any set $x$, define $\nextset{x}$ to be the set consisting of all
the elements of $x$, along with $x$ itself:
\[
\nextset{x}  \eqdef x \union \set{x}.
\]
So by definition,
\begin{equation}\label{insubset}
x \in \nextset{x}\text{ and }x \subset \nextset{x}.
\end{equation}

Now we give a recursive definition of a collection $\Ord$ of sets
called \term{ordinals} that provide a way to count infinite sets.
Namely,
\begin{definition*}
\begin{align*}
\emptyset & \in \Ord,\\
\text{if } \nu & \in \Ord, \text{ then } \nextset{\nu} \in \Ord,\\
\text{if } S & \subset \Ord, \text{ then } \lgunion_{\nu \in S} \nu \in \Ord.
\end{align*}
\end{definition*}
The recursive definition of ordinals means we can prove things about them using
structural induction.  Namely, let $P(x)$ be some property of sets.  The
\emph{Ordinal Induction Rule} says that to prove that $P(\nu)$ is true for all
ordinals $\nu$, you need only show two things
\begin{itemize}
\item If $P$ holds for all the members of $\nextset{x}$, then it holds
  for $\nextset{x}$, and
\item if $P$ holds for all members of some set $S$, then it holds for
  their union.
\end{itemize}  That is:

\begin{mathrule*}
\textbf{Ordinal Induction}
\Rule{\forall x.\ (\forall y \in \nextset{x}.\, P(y))\ \QIMPLIES\ P(\nextset{x}),\\
\forall S.\ (\forall x \in S.\, P(x))\ \QIMPLIES\ P(\lgunion_{x \in S} x)}
{\forall \nu \in \Ord.\, P(\nu)}
\end{mathrule*}

\begin{definition*}
A set $x$ is \emph{full} if every element of $x$ is also a
subset of $x$, that is
\[
\forall y \in x.\ y \subset x.
\]
\end{definition*}

Prove that every ordinal $\nu$ is full.

\begin{solution}
\begin{proof}
The proof is by Ordinal Induction with hypothesis
\[
P(x) \eqdef x \text{ is full}.
\]

Let $x$ be a set,

\inductioncase{Case: ($\nextset{x}$)}: Suppose every $y \in \nextset{x}$ is
full.  We need to prove that $\nextset{x}$ is full.  To do this, suppose $y \in
\nextset{x}$.  We need to show $y \subset \nextset{x}$.  Since $x \subset
\nextset{x}$ by~\eqref{insubset}, it is enough to show that $y \subseteq x$.

Now there are two subcases: if $y = x$ then $y \subseteq x$ trivially.  On the
other hand, if $y \neq x$, then $y \in x$ by definition of $\nextset{x}$.
Moreover, but $x \in \nextset{x}$, so $x$ is full by hypothesis.  Therefore, $y
\subset x$ in this subcase too.

\inductioncase{Case: ($\lgunion$)}: Let $S$ be a set, $U \eqdef \lgunion_{x \in
  S} x$, and suppose every $x$ in $S$ is full.  We need to prove that $U$ is
full.  To do this, suppose $y \in U$.  Then we need to show that $y \subset U$.

Now $y \in x_0$ for some $x_0 \in S$ by definition of $U$.  Moreover,
$x_0$ will be full by hypothesis, and so $y \subset
x_0$.  But $x_0 \subset U$ by definition of $U$, and hence $y \subset
U$.
\end{proof}

\end{solution}

\begin{editingnotes}
Prove the converse?  Maybe need  $y \in x\ \  \QIFF\ \ y \subset x$.
\end{editingnotes}

\end{problem}

\endinput
