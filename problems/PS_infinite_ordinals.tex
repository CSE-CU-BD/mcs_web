\documentclass[problem]{mcs}

\begin{pcomments}
  \pcomment{PS_infinite_ordinals}
  \pcomment{generalizes CP_finite_ordinals}
  \pcomment{ARM 2/20/14}
  \pcomment{proofs of last parts TBA}
\end{pcomments}

\pkeywords{
  sets
  set_theory
  subset
  union
  ordinal
  member
  contains
  well_order
  foundation
  decreasing
}

\newcommand{\nextset}[1]{\text{next}(#1)}
\newcommand{\Ord}{\text{Ord}}

\begin{problem}
For any set $x$, define $\nextset{x}$ to be the set consisting of all
the elements of $x$, along with $x$ itself:
\[
\nextset{x}  \eqdef x \union \set{x}.
\]
So by definition,
\begin{equation}\label{insubset}
x \in \nextset{x}\text{ and }x \subset \nextset{x}.
\end{equation}

Now we give a recursive definition of a collection, $\Ord$, of sets
called \term{ordinals} that provide a way to count infinite sets.
Namely,
\begin{definition*}
\begin{align*}
\emptyset & \in \Ord,\\
\text{if } \nu & \in \Ord, \text{ then } \nextset{\nu} \in \Ord,\\
\text{if } S & \subset \Ord, \text{ then } \lgunion_{\nu \in S} \nu \in \Ord.
\end{align*}
\end{definition*}
There is a method for proving things about ordinals that follows
directly from the way they are defined.  Namely, let $P(\nu)$ be some
property of ordinals.  The \emph{Ordinal Induction Rule} says that to
prove that $P(\nu)$ is true for \emph{all} ordinals $\nu$, you should
prove:

\inductioncase{Base case}: $P(\emptyset)$.

\inductioncase{Induction steps}:
\begin{enumerate}[(i)]
\item If $\nu$ is an ordinal and $P(\nu)$ is true, then
  $P(\nextset{\nu})$ is true, and
\item if $S \subset \Ord$ and $P(\nu)$ is true for all $\nu \in S$,
  then $P(\lgunion_{\nu \in S} \nu)$.
\end{enumerate}

The intuitive justification for the Ordinal Induction Rule is just like the
intuitive justification for regular induction, and we will accept the
soundness of the Ordinal Induction Rule as a basic axiom.

\begin{editingnotes}
Maybe give part (a) as an example
\end{editingnotes}

\bparts

\ppart A set $c$ is \emph{closed under membership} if for all sets $a,b$,
\[
a \in b \in c\ \QIMPLIES\ a \in c.
\]
Prove that every ordinal $\nu$ is closed under membership.

\begin{solution}
\begin{proof}
The proof is by Ordinal Induction with hypothesis
\[
P(\nu) \eqdef \nu \text{ is closed under membership}.
\]

\inductioncase{Base case}: $P(\emptyset)$ holds vacuously.

\inductioncase{Induction steps}: Let $\nu$ be an ordinal.
\begin{enumerate}[(i)]

\item Suppose $a \in b \in \nextset{\nu}$ for some sets $a,b$.  Then
  $b = \nu$, so $b \subset \nextset{\nu}$ by~\eqref{insubset}, which
  means $a \in \nextset{\nu}$.  That is, $P(\nextset{\nu})$ is
  true.\footnote{Note that we didn't even need to use the induction
    hypothesis $P(\nu)$.}

\item Let $S$ be a set of ordinals and $U \eqdef \lgunion_{\nu \in S}
  \nu$.  Suppose $P(\nu)$ is true for all $\nu \in S$.  We need to
  prove $P(U)$.

  To do this, suppose $a \in b \in U$ for some sets $a,b$.  Then $b
  \in \nu_0$ for some $\nu_0 \in S$ by definition of $U$.  Since
  $P(\nu_0)$ is true, it follows that $a \in \nu_0$, and hence $a \in
  U$ by definition of $U$.  That is, $P(U)$ holds, as claimed.
\end{enumerate}
\end{proof}

\end{solution}

\ppart Prove that if $\mu,\nu$ are different ordinals, then
$\nu \in \mu \QOR\ \mu \in \nu$.

\begin{solution}
By Ordinal Induction with
\[
P(\nu) \eqdef \forall \mu \in \Ord.\ \nu \in \mu \QOR\ \mu \in \nu.
\]
\end{solution}

\ppart A sequence
\begin{equation}\label{indecreasing}
\cdots \in \nu_{n+1} \in \nu_n \in \cdots \in \nu_1 \in \nu_0
\end{equation}
of ordinals $\nu_i$ is called a \emph{member-decreasing} sequence
starting at $\nu_0$.  Use Ordinal Induction to prove that no ordinal
starts an infinite member-decreasing sequence.\footnote{Do not assume
  the Foundation Axiom of ZFC (Section~\bref{ZFC_sec}) which says that
  there isn't \emph{any} set that starts an infinite member-decreasing
  sequence.  Even in versions of set theory in which the Foundation
  Axiom does not hold, there cannot be any infinite member-decreasing
  sequence of ordinals.}

\begin{solution}
\begin{proof}
By Ordinal Induction with
\[
P(\nu) \eqdef \nu \text{ does not start an infinite member-decreasing
  sequence}.
\]
\end{proof}
\end{solution}

\begin{editingnotes}
Maybe add discussion that ordinals can serve as a measure of set size.
Define
\[
\card{S} \eqdef \text{the member-minimal } \nu \in \Ord\text{ such that } S \bij \nu.
\]
\end{editingnotes}

\eparts
\end{problem}

\endinput
