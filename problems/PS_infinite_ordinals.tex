\documentclass[problem]{mcs}

\begin{pcomments}
  \pcomment{PS_infinite_ordinals}
  \pcomment{generalizes CP_finite_ordinals}
  \pcomment{ARM 2/20/14}
  \pcomment{proofs of last parts TBA}
\end{pcomments}

\pkeywords{
  sets
  set_theory
  subset
  union
  ordinal
  member
  contains
  well_order
  foundation
  decreasing
}

\newcommand{\nextset}[1]{\text{next}(#1)}
\newcommand{\Ord}{\text{Ord}}

\begin{problem}
For any set $x$, define $\nextset{x}$ to be the set consisting of all
the elements of $x$, along with $x$ itself:
\[
\nextset{x}  \eqdef x \union \set{x}.
\]
So by definition,
\begin{equation}\label{insubset}
x \in \nextset{x}\text{ and }x \subset \nextset{x}.
\end{equation}

Now we give a recursive definition of a collection $\Ord$ of sets
called \term{ordinals} that provide a way to count infinite sets.
Namely,
\begin{definition*}
\begin{align*}
\emptyset & \in \Ord,\\
\text{if } \nu & \in \Ord, \text{ then } \nextset{\nu} \in \Ord,\\
\text{if } S & \subset \Ord, \text{ then } \lgunion_{\nu \in S} \nu \in \Ord.
\end{align*}
\end{definition*}
There is a method for proving things about ordinals that follows
directly from the way they are defined.  Namely, let $P(x)$ be some
property of sets.  The \emph{Ordinal Induction Rule} says that to
prove that $P(\nu)$ is true for all ordinals $\nu$, you need only show
two things
\begin{itemize}
\item If $P$ holds for all the members of $\nextset{x}$, then it holds
  for $\nextset{x}$, and
\item if $P$ holds for all members of some set $S$, then it holds for
  their union.
\end{itemize}  That is:

\begin{mathrule*}
\textbf{Ordinal Induction}
\Rule{\forall x.\ (\forall y \in \nextset{x}.\, P(y))\ \QIMPLIES\ P(\nextset{x}),\\
\forall S.\ (\forall x \in S.\, P(x))\ \QIMPLIES\ P(\lgunion_{x \in S} x)}
{\forall \nu \in \Ord.\, P(\nu)}
\end{mathrule*}

The intuitive justification for the Ordinal Induction Rule is similar
to the justification for strong induction.  We will accept the
soundness of the Ordinal Induction Rule as a basic axiom.

\begin{editingnotes}
Maybe give part (a) as an example
\end{editingnotes}

\bparts

\ppart A set $x$ is \emph{closed under membership} if every element of $x$ is also a subset of $x$,
that is
\[
\forall y \in x.\ y \subset x.
\]
Prove that every ordinal $\nu$ is closed under membership.

\begin{solution}
\begin{proof}
The proof is by Ordinal Induction with hypothesis
\[
P(x) \eqdef x \text{ is closed under membership}.
\]

Let $x$ be a set,

\inductioncase{Case: ($\nextset{x}$)}: Suppose every $y \in
\nextset{x}$ is closed under membership.  We need to prove that
$\nextset{x}$ is closed under membership.  To do this, suppose $y \in
\nextset{x}$.  We need to show $y \subset \nextset{x}$.  Since $x
\subset \nextset{x}$ by~\eqref{insubset}, it is enough to show that $y
\subseteq x$.

Now there are two subcases: if $y = x$ then $y \subseteq x$ trivially.
On the other hand, if $y \neq x$, then $y \in x$ by definition of
$\nextset{x}$.  Moreover, but $x \in \nextset{x}$, so $x$ is closed
under membership by hypothesis.  Therefore, $y \subset x$ in this
subcase too.

\inductioncase{Case: ($\lgunion$)}: Let $S$ be a set, $U \eqdef
\lgunion_{x \in S} x$, and suppose every $x$ in $S$ is closed under
membership.  We need to prove that $U$ is closed under membership.  To
do this, suppose $y \in U$.  Then we need to show that $y \subset U$.

Now $y \in x_0$ for some $x_0 \in S$ by definition of $U$.  Moreover,
$x_0$ will be closed under membership by hypothesis, and so $y \subset
x_0$.  But $x_0 \subset U$ by definition of $U$, and hence $y \subset
U$.
\end{proof}

\end{solution}

\ppart A sequence
\begin{equation}\label{indecreasing}
\cdots \in \nu_{n+1} \in \nu_n \in \cdots \in \nu_1 \in \nu_0
\end{equation}
of ordinals $\nu_i$ is called a \emph{member-decreasing} sequence
starting at $\nu_0$.  Use Ordinal Induction to prove that no ordinal
starts an infinite member-decreasing sequence.\footnote{Do not assume
  the Foundation Axiom of ZFC (Section~\bref{ZFC_sec}) which says that
  there isn't \emph{any} set that starts an infinite member-decreasing
  sequence.  Even in versions of set theory in which the Foundation
  Axiom does not hold, there cannot be any infinite member-decreasing
  sequence of ordinals.}

\begin{solution}
\begin{proof}
By Ordinal Induction with
\[
P(\nu) \eqdef \nu \text{ does not start an infinite member-decreasing
  sequence}.
\]

\TBA{a little more needed}

\end{proof}
\end{solution}


\begin{editingnotes}
\TBA{How to prove this?}

%\ppart

Prove that if $\mu,\nu$ are different ordinals, then $\nu \in \mu
\QOR\ \mu \in \nu$.


Maybe add discussion that ordinals can serve as a measure of set size.
Define
\[
\card{S} \eqdef \text{the member-minimal } \nu \in \Ord\text{ such that } S \bij \nu.
\]
\end{editingnotes}

\eparts
\end{problem}

\endinput
