\documentclass[problem]{mcs}

\begin{pcomments}
  \pcomment{PS_6_10_15_stamps_by_WOP}
  \pcomment{variation of PS_3_and_5_postage_by_WOP}
  \pcomment{ARM 2/3/14}
\end{pcomments}

\pkeywords{
  WOP
  well_ordering
  postage
  stamps
}

%%%%%%%%%%%%%%%%%%%%%%%%%%%%%%%%%%%%%%%%%%%%%%%%%%%%%%%%%%%%%%%%%%%%%
% Problem starts here
%%%%%%%%%%%%%%%%%%%%%%%%%%%%%%%%%%%%%%%%%%%%%%%%%%%%%%%%%%%%%%%%%%%%%

\begin{problem}
Use the Well Ordering Principle to prove that any integer greater than
or equal to 30 can be represented as the sum of nonnegative integer
multiples of 6, 10, and 15.

\inhandout{ \hint Use the template for WOP proofs to ensure partial
  credit.  Verify that integers in the interval [30,35] are sums of
  nonnegative integer multiples of 6, 10, and 15.  }

\begin{solution}
\begin{claim}
For all $n \geq 30$, it is possible to represent $n$ as a sum of nonnegative integer
multiples of 6, 10 and 15.
\end{claim}

\begin{proof}

  The proof is by the Well Ordering Principle.  Let $P(n)$ be the
  predicate that $n$ is a sum of nonnegative integer multiples of 6,
  10 and 15.

  Let $C \eqdef \set{ n \geq 30 | \QNOT(P(n))}$ be the set of counter
  examples.   Assume for the sake of contradiction that $C$ is not
  empty.  Then by the Well Ordering Principle, $C$ must have some minimum
  element $m\in C$.

  First, observe that $P(n)$ is true for the values of $n \in [30,35]$.
  \begin{itemize}
  \item $n=30$: $3 \cdot 10$.

  \item $n=31$: $6 + 15 + 10$.

  \item $n=32$: $2 \cdot 6 + 2 \cdot 10$.

  \item $n=33$: $3 \cdot 6  + 15$.

  \item $n=34$: $4 \cdot 6  + 10$.

  \item $n=35$: $15 + 2 \cdot 10$.
  \end{itemize}

  We thus have $m \geq 36$.  Since $m$ is the smallest counterexample,
  $m - 6$ is not a counterexample.  Since $m-6 \geq 30$, it follows that
  $P(m-6)$ holds, that is, $m-6$ as the sum of nonnegative integer
  multiples of 6, 10, and 15.  Thus we can represent $m$ by adding 1
  to the coefficient of 6 in our representation of $m-6$.  This shows
  that $m$ is not actually a counterexample, contradicting the
  assumption that $C$ is nonempty.
  \end{proof}
  \end{solution}

\end{problem}

%%%%%%%%%%%%%%%%%%%%%%%%%%%%%%%%%%%%%%%%%%%%%%%%%%%%%%%%%%%%%%%%%%%%%
% Problem ends here
%%%%%%%%%%%%%%%%%%%%%%%%%%%%%%%%%%%%%%%%%%%%%%%%%%%%%%%%%%%%%%%%%%%%%

\endinput
