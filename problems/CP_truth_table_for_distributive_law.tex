\documentclass[problem]{mcs}

\begin{pcomments}
  \pcomment{CP_truth_table_for_distributive_law}
  \pcomment{from: S09.cp2m}
\end{pcomments}

\pkeywords{
  truth_table
  logic
}

%%%%%%%%%%%%%%%%%%%%%%%%%%%%%%%%%%%%%%%%%%%%%%%%%%%%%%%%%%%%%%%%%%%%%
% Problem starts here
%%%%%%%%%%%%%%%%%%%%%%%%%%%%%%%%%%%%%%%%%%%%%%%%%%%%%%%%%%%%%%%%%%%%%

\begin{problem}
Prove by truth table that $\QOR$ distributes over $\QAND$, namely,
\begin{equation}\label{dist}
%\brac{P \; \QOR\ (Q\ \QAND\ R)} \quad\text{is equivalent to}\quad
P \QOR (Q \QAND R) \quad\text{is equivalent to}\quad
%\brac{(P\ \QOR\ Q)\ \QAND\ (P\ \QOR\ R)}
(P \QOR Q) \QAND (P \QOR R)
\end{equation}

\begin{solution}

\[
\begin{array}{|ccccc|}
\hline
P      & \underline{\QOR}  & \text{(}Q    & \QAND & R\text{)}   \\ \hline
\true  &\lgtrue       & \true  & \true      & \true \\ \hline
\true  &\lgtrue       & \true  & \false     & \false\\ \hline
\true  &\lgtrue       & \false & \false     & \true \\ \hline
\true  &\lgtrue       & \false & \false     & \false\\ \hline
\false &\lgtrue       & \true  & \true      & \true \\ \hline
\false &\lgfalse      & \true  & \false     & \false\\ \hline
\false &\lgfalse      & \false & \false     & \true \\ \hline
\false &\lgfalse      & \false & \false     & \false\\ \hline
\end{array}
\]

\[
\begin{array}{|ccccccc|}
\hline
\text{(}P  & \QOR      & Q\text{)} & \underline{\QAND}       & \text{(}P & \QOR      & R\text{)} \\  \hline
     \true  & \true     & \true     & \lgtrue      &   \true   & \true     & \true \\  \hline
     \true  & \true     & \true     & \lgtrue      &   \true   & \true     & \false\\  \hline
     \true  & \true     & \false    & \lgtrue      &   \true   & \true     & \true \\  \hline
     \true  & \true     & \false    & \lgtrue      &   \true   & \true     & \false\\  \hline
     \false & \true     & \true     & \lgtrue      &   \false  & \true     & \true \\  \hline
     \false & \true     & \true     & \lgfalse     &   \false  & \false    & \false\\  \hline
     \false & \false    & \false    & \lgfalse     &   \false  & \true     & \true \\  \hline
     \false & \false    & \false    & \lgfalse     &   \false  & \false    & \false\\  \hline
\end{array}
\]

The column (highlighted) for the principle operator (underlined) of the
first formula is the same as for the principle operator of the second
formula, so the two propositional formulas are equivalent.
\end{solution}


\end{problem}

%%%%%%%%%%%%%%%%%%%%%%%%%%%%%%%%%%%%%%%%%%%%%%%%%%%%%%%%%%%%%%%%%%%%%
% Problem ends here
%%%%%%%%%%%%%%%%%%%%%%%%%%%%%%%%%%%%%%%%%%%%%%%%%%%%%%%%%%%%%%%%%%%%%

\endinput
