\documentclass[problem]{mcs}

\begin{pcomments}
  \pcomment{from: S09.cp2m}
%  \pcomment{}
%  \pcomment{}
\end{pcomments}

\pkeywords{
  truth_table
  logic
}

%%%%%%%%%%%%%%%%%%%%%%%%%%%%%%%%%%%%%%%%%%%%%%%%%%%%%%%%%%%%%%%%%%%%%
% Problem starts here
%%%%%%%%%%%%%%%%%%%%%%%%%%%%%%%%%%%%%%%%%%%%%%%%%%%%%%%%%%%%%%%%%%%%%

\begin{problem}
Prove by truth table that $\QOR$ distributes over $\QAND$:
\begin{equation}\label{dist}
\brac{P \; \QOR\ (Q\ \QAND\ R)} \quad\text{is equivalent to}\quad
\brac{(P\ \QOR\ Q)\ \QAND\ (P\ \QOR\ R)}
\end{equation}

\begin{solution}

\[
\begin{array}{|c|c|ccc|}
\hline
\text{[}P      & \underline{\QOR}  & \text{(}Q    & \QAND & R\text{)]}   \\ \hline
\true  &\true       & \true  & \true      & \true \\ \hline
\true  &\true       & \true  & \false     & \false\\ \hline
\true  &\true       & \false & \false     & \true \\ \hline
\true  &\true       & \false & \false     & \false\\ \hline
\false &\true       & \true  & \true      & \true \\ \hline
\false &\false      & \true  & \false     & \false\\ \hline
\false &\false      & \false & \false     & \true \\ \hline
\false &\false      & \false & \false     & \false\\ \hline
\end{array}
\]

\[
\begin{array}{|ccc|c|ccc|}
\hline
\text{[(}P  & \QOR      & Q\text{)} & \underline{\QAND}       & \text{(}P & \QOR      & R\text{)]} \\  \hline
     \true  & \true     & \true     & \true      &   \true   & \true     & \true \\  \hline
     \true  & \true     & \true     & \true      &   \true   & \true     & \false\\  \hline
     \true  & \true     & \false    & \true      &   \true   & \true     & \true \\  \hline
     \true  & \true     & \false    & \true      &   \true   & \true     & \false\\  \hline
     \false & \true     & \true     & \true      &   \false  & \true     & \true \\  \hline
     \false & \true     & \true     & \false     &   \false  & \false    & \false\\  \hline
     \false & \false    & \false    & \false     &   \false  & \true     & \true \\  \hline
     \false & \false    & \false    & \false     &   \false  & \false    & \false\\  \hline
\end{array}
\]

The two columns for the principle operator (underlined) are the same, and
therefore the corresponding propositional formulas are equivalent.
\end{solution}


\end{problem}

%%%%%%%%%%%%%%%%%%%%%%%%%%%%%%%%%%%%%%%%%%%%%%%%%%%%%%%%%%%%%%%%%%%%%
% Problem ends here
%%%%%%%%%%%%%%%%%%%%%%%%%%%%%%%%%%%%%%%%%%%%%%%%%%%%%%%%%%%%%%%%%%%%%

\endinput
