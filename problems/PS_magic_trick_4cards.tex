\documentclass[problem]{mcs}

\begin{pcomments}
  \pcomment{PS_magic_trick_4cards}
  \pcomment{subsumes PS_magic_trick}
  \pcomment{major edit from: S09 ps9 by ARM 4/5/10}
\end{pcomments}

\pkeywords{
  magic
  degree_constrained
  modulo  card_magic
  bipartite
  matching
  card_magic
}

%%%%%%%%%%%%%%%%%%%%%%%%%%%%%%%%%%%%%%%%%%%%%%%%%%%%%%%%%%%%%%%%%%%%%
% Problem starts here
%%%%%%%%%%%%%%%%%%%%%%%%%%%%%%%%%%%%%%%%%%%%%%%%%%%%%%%%%%%%%%%%%%%%%

\begin{problem}

\newcommand{\pcard}[1]{\,\fbox{\parbox[c][.5in][c]{.3in}{\centering#1}}\,}

Section~\bref{4_card_trick_subsec} explained why it is not possible to
perform a four-card variant of the hidden-card magic trick with one
card hidden.  But the Magician and her Assistant are determined to
find a way to make a trick like this work.  They decide to change the
rules slightly: instead of the Assistant lining up the three unhidden
cards for the Magician to see, he will line up all four cards with one
card face down and the other three visible.  We'll call this the
\term{face-down four-card trick}.

For example, suppose the audience members had selected the cards
$9\heartsuit$, $10\diamondsuit$, $A\clubsuit$, $5\clubsuit$.  Then the
Assistant could choose to arrange the 4 cards in any order so long as
one is face down and the others are visible.  Two possibilities are:
\begin{center}
\pcard{$A\clubsuit$}\pcard{?}\pcard{$10\diamondsuit$}\pcard{$5\clubsuit$}

\pcard{?}\pcard{$5\clubsuit$}\pcard{$9\heartsuit$}\pcard{$10\diamondsuit$}
\end{center}

\bparts

\ppart Explain how to model this the face-down four-card trick as a
matching problem, and show that there must be a bipartite matching
which theoretically will allow the Magician and Assistant to perform
the trick.

\begin{solution}
We proceed as in Section~\bref{4_card_trick_subsec} by modeling the
trick as a bipartite matching problem.  The vertices $L$ are the
possible $\binom{52}{4}$ sets of 4 cards, representing the possible
sets of cards selected by the audience.  The vertices $R$ correspond
to the information available to the Magican, namely, the collection of
all possible sequences of 3 cards along with a ``?'' indicating the
position of the hidden card, as in the figure above.  There are $52
\cdot 51 \cdot 50$ possible sequences of visible cards and 4 positions
for ``?,'' so $R$ has $52 \cdot 51 \cdot 50 \cdot 4$ vertices.

\begin{staffnotes}
It is not necessary to calculate $\card{L}$ or $\card{R}$.  These
numbers were just worked out to help clarify the definitions of $L,R$.
\end{staffnotes}

There is an edge between $l \in L$ and $r \in R$ if the set of visible
cards in $r$ is a subset of $l$.  A matching from $L$ to $R$ would
represent a unique way to encode each set of 4 cards as a sequence of
3 visible cards and a ``?.''.  But for each $l \in L$,
\[
\degr{l} = \binom{4}{3} 3! \cdot 4 = 96
\]
and for each $r \in R$,
\[
\degr{r} = \binom{52-3}{1} = 49.
\]
So the graph is \idx{degree-constrained} and, according to
Theorem~\bref{lem:no_bottleneck_degree_constrained}
satisfies Hall's matching condition and so has a matching.
\end{solution}

\begin{staffnotes}
A good test of understanding is to ask students how big a deck would
still have a matching by this reasonong.  The degree 96 of the $L$
vertices does not depend on deck size.  The degree of the $R$ vertices
is $\text{deck-size} - 3$, so even with deck-size of 99, the graph
remains degree-constrained and hence has a matching.
\end{staffnotes}

\ppart There is actually a simple way to perform the face-down
four-card trick.\footnote{This elegant method was devised in Fall '09
  by student Katie E Everett.}

\textbox{
\textbf{Case 1}. \emph{there are two cards with the same suit}: Say
there are two $\spadesuit$ cards.  The Assistant proceeds as in the
original card trick: he puts one of the $\spadesuit$ cards \emph{face
  up as the first card}.  He will place the second $\spadesuit$ card
\emph{face down}.  He then uses a permutation of the face down card
and the remaining two face up cards to code the offset of the face
down card from the first card.

\textbf{Case 2}. \emph{all four cards have different suits}: Assign
numbers $0,1,2,3$ to the four suits in some agreed upon way.  The
Assistant computes, $s$, the sum modulo 4 of the ranks of the four
cards, and chooses the card with suit $s$ to be placed \emph{face down
  as the first card}.  He then uses a permutation of the remaining
three face-up cards to code the rank of the face down
card.}

Explain how in Case 2.\ the Magician can determine the face down card
from the cards the Assistant shows her.

\begin{solution}
The Magician sees that Case 2 applies because the first card is face
down, so she knows that the hidden card must have the suit that is not
among the visible three suits of the face-up cards.  So the Magician
knows the suit of the face-down card and hence the number, $s$,
assigned to that suit.

Let $t$ be the mod 4 sum of the ranks of the three face-up cards.
Since $s$ was chosen by the Assistant to be the sum mod 4 of the ranks
of all four cards, the Magician knows that the rank of the face-down
card is congruent to $s-t \pmod 4$.  Since at most four numbers from 1
to 13 can be congruent to each other mod 4, the six possible
permutations of the three face-up cards are more than enough to encode
which of the four possible numbers congruent to $s - t \pmod 4$ is the
rank of the hidden card.

\end{solution}

\begin{staffnotes}
For use in class, skip the following part.
\end{staffnotes}

\ppart Explain how any method for performing the face-down four-card
trick can be adapted to perform the regular (5-card hand, show 4
cards) with a 52-card deck consisting of the usual 52 cards along with
a 53rd card call the \emph{joker}.

\begin{solution}
If the 5-card hand does not include the joker, just proceed as in the
regular 5-card trick.

If the 5-card hand does include the joker, just proceed as in the
4-card face-down trick, placing the joker (face up) where the
face-down card would be.
\end{solution}

\eparts
\end{problem}

%%%%%%%%%%%%%%%%%%%%%%%%%%%%%%%%%%%%%%%%%%%%%%%%%%%%%%%%%%%%%%%%%%%%%
% Problem ends here
%%%%%%%%%%%%%%%%%%%%%%%%%%%%%%%%%%%%%%%%%%%%%%%%%%%%%%%%%%%%%%%%%%%%%

\endinput
