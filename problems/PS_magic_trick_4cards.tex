\documentclass[problem]{mcs}

\begin{pcomments}
  \pcomment{PS_magic_trick_4cards}
  \pcomment{major edit from: S09 ps9 by ARM 4/5/10}
\end{pcomments}

\pkeywords{
  card_magic
  bipartite_matching
  degree_constrained
  modulo
}

%%%%%%%%%%%%%%%%%%%%%%%%%%%%%%%%%%%%%%%%%%%%%%%%%%%%%%%%%%%%%%%%%%%%%
% Problem starts here
%%%%%%%%%%%%%%%%%%%%%%%%%%%%%%%%%%%%%%%%%%%%%%%%%%%%%%%%%%%%%%%%%%%%%


\begin{problem}

\newcommand{\pcard}[1]{\,\fbox{\parbox[c][.5in][c]{.3in}{\centering#1}}\,}

Section~\bref{4_card_trick_subsec} explained why it is not possible to
perform a four-card variant of the hidden-card magic trick with one
card hidden.  But the Magician and her Assistant are determined to
find a way to make a trick like this work.  They decide to change the
rules slightly: instead of the Assistant lining up the three unhidden
cards for the Magician to see, he will line up all four cards with one
card face down and the other three visible.  We'll call this the
\term{face-down four-card trick}.

For example, suppose the audience members had selected the cards
$9\heartsuit$, $10\diamondsuit$, $A\clubsuit$, $5\clubsuit$.  Then the
Assistant could choose to arrange the 4 cards in any order so long as
one is face down and the others are visible.  Two possibilities are:
\begin{center}
\pcard{$A\clubsuit$}\pcard{?}\pcard{$10\diamondsuit$}\pcard{$5\clubsuit$}

\pcard{?}\pcard{$5\clubsuit$}\pcard{$9\heartsuit$}\pcard{$10\diamondsuit$}
\end{center}

\bparts

\ppart Explain why there must be a bipartite matching which will in
theory allow the Magician and Assistant to perform the face-down
four-card trick.

\begin{solution}
We proceed by again modeling the trick as a bipartite matching 
problem.  The vertices $L$ are the collection of all possible sets 
of 4 cards, representing the possible sets of cards selected by the 
audience.  The vertices $R$ are the collection of all possible 
length 4 sequences where the first 3 positions contain distinct cards 
representing the ordering of the visible cards and the 4th position 
is an element of $\set{1,2,3,4}$ denoting the position of the hidden 
card.

There is an edge between $l \in L$ and $r \in R$ if the set of visible 
cards in $r$ is a subset of $l$. 

A matching from $L$ to $R$ would represent a unique way to encode each
set of 4 cards as a sequence of 1 hidden and 3 visible cards.  For each
$l \in L$,
\[
\degr{l} = \binom{4}{3} 3! \cdot 4 = 96
\]
and for each $r \in R$,
\[
\degr{r} = \binom{52-3}{1} = 49.
\]

So the graph is \idx{degree constrained} and, according to
Lemma~\bref{degree-constrained_lemma}, therefore satisfies Hall's
matching condition.
\end{solution}


\ppart There is actually a simple way to perform the face-down
four-card trick.

\textbox{
\textbf{Case 1}. \emph{there are two cards with the same suit}: Say
there are two $\spadesuit$ cards.  The Assistant proceeds as in the
original card trick: he puts one of the $\spadesuit$ cards \emph{face
  up as the first card}.  He will place the second $\spadesuit$ card
\emph{face down}.  He then uses a permutation of the face down card
and the remaining two face up cards to code the offset of the face
down card from the first card.

\textbf{Case 2}. \emph{all four cards have different suits}: Assign
numbers $0,1,2,3$ to the four suits in some agreed upon way.  The
Assistant computes, $s$, the sum modulo 4 of the ranks of the four
cards, and chooses the card with suit $s$ to be placed \emph{face down
  as the first card}.  He then uses a permutation of the remaining
three face-up cards to code the rank of the face down
card.\footnote{This elegant method was devised in Fall '09 by student
  Katie E Everett.}  }

Explain how in Case 2.\ the Magician can determine the face down card
from the cards the Assistant shows her.

\begin{solution}
The Magician sees that Case 2 applies because the first card is face
down, so she knows that the hidden card must have the suit that is not
among the visible three suits of the face-up cards.  So the Magician
knows the suit of the face-down card and hence the number, $s$,
assigned to that suit.

Let $t$ be the mod 4 sum of the ranks of the three face-up cards.
Since $s$ was chosen by the Assistant to be the sum mod 4 of the ranks
of all four cards, the Magician knows that the rank of the face-down
card is congruent to $s-t \pmod 4$.  Since at most four numbers from 1
to 13 can be congruent to each other mod 4, the six possible
permutations of the three face-up cards are more than enough to encode
which of the four possible numbers congruent to $s - t \pmod 4$ is the
rank of the hidden card.

\end{solution}

\ppart Explain how any method for performing the face-down four-card
trick can be adapted to perform the regular (5-card hand, show 4
cards) with a 52-card deck consisting of the usual 52 cards along with
a 53rd card call the \emph{joker}.

\begin{solution}
If the 5-card hand does not include the joker, just proceed as in the
regular 5-card trick.

If the 5-card hand does include the joker, just proceed as in the
4-card face-down trick, placing the joker (face up) where the
face-down card would be.
\end{solution}

\eparts
\end{problem}

%%%%%%%%%%%%%%%%%%%%%%%%%%%%%%%%%%%%%%%%%%%%%%%%%%%%%%%%%%%%%%%%%%%%%
% Problem ends here
%%%%%%%%%%%%%%%%%%%%%%%%%%%%%%%%%%%%%%%%%%%%%%%%%%%%%%%%%%%%%%%%%%%%%

\endinput
