\documentclass[problem]{mcs}

\begin{pcomments}
  \pcomment{PS_recitation_scheduling}
  \pcomment{from: S09.ps6, S06.ps4, S04.ps4}
\end{pcomments}

\pkeywords{
  graph_coloring
  scheduling
}

%%%%%%%%%%%%%%%%%%%%%%%%%%%%%%%%%%%%%%%%%%%%%%%%%%%%%%%%%%%%%%%%%%%%%
% Problem starts here
%%%%%%%%%%%%%%%%%%%%%%%%%%%%%%%%%%%%%%%%%%%%%%%%%%%%%%%%%%%%%%%%%%%%%

\begin{problem} Math for Computer Science is often taught using recitations.  Suppose it
happened that 8 recitations were needed, with two or three staff
members running each recitation.  The assignment of staff to
recitation sections is as follows:

\begin{itemize}
\item R1:  Eli, Rong, Eric \\
\item R2:  Eli, Megumi, Albert\\
\item R3:  Rong, Jay\\
\item R4:  Chuck, Megumi, Eric\\
\item R5:  Chuck, William, Albert\\
\item R6:  William, Jay\\
\item R7:  William, Megumi\\
\item R8:  Rong, Jay, Albert
\end{itemize}

Two recitations can not be held in the same 90-minute time slot if some
staff member is assigned to both recitations.  The problem is to determine
the minimum number of time slots required to complete all the recitations.

\bparts

\ppart Recast this problem as a question about coloring the
vertices of a particular graph.  Draw the graph and explain what the
vertices, edges, and colors represent.

\begin{solution}
Each vertex in the graph below represents a recitation
section.  An edge connects two vertices if the corresponding
recitation sections share a staff member and thus can not be scheduled
at the same time.  The color of a vertex indicates the time slot of
the corresponding recitation.

\begin{figure}[h]
\graphic[height=2in]{ps3-schedule}
\end{figure}

\end{solution}

\ppart{Show a coloring of this graph using the fewest possible
colors.  What schedule of recitations does this imply?}

\begin{solution}
Four colors are necessary and sufficient. To see why they
are \emph{sufficient}, consider the coloring:
\begin{figure}[h]
\graphic[height=2in]{ps3-schedule-colored}
\end{figure}
This corresponds to the following assignment of recitations to four
time slots:
\begin{enumerate}

\item R1, R5

\item R2, R3

\item R4, R6

\item R7, R8

\end{enumerate}
Other schedules are also possible.

To see why 4 colors are \emph{necessary}, look at the subgraph defined
by the vertices for R2, R4, R5, and R7.  This is the complete graph on 4
vertices, and it obviously needs 4 colors.
\end{solution}

\eparts

\end{problem}

%%%%%%%%%%%%%%%%%%%%%%%%%%%%%%%%%%%%%%%%%%%%%%%%%%%%%%%%%%%%%%%%%%%%%
% Problem ends here
%%%%%%%%%%%%%%%%%%%%%%%%%%%%%%%%%%%%%%%%%%%%%%%%%%%%%%%%%%%%%%%%%%%%%

\endinput
