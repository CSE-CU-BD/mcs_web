\documentclass[problem]{mcs}

\begin{pcomments}
  \pcomment{FP_rapid_increasing_diag}
  \pcomment{subsumes FP_increasing_sequences_diag}
  \pcomment{ARM 4/13/18}
\end{pcomments}

\pkeywords{
  infinite_sets
  sequences
  increasing
  diagonalization
}

%%%%%%%%%%%%%%%%%%%%%%%%%%%%%%%%%%%%%%%%%%%%%%%%%%%%%%%%%%%%%%%%%%%%%
% Problem starts here
%%%%%%%%%%%%%%%%%%%%%%%%%%%%%%%%%%%%%%%%%%%%%%%%%%%%%%%%%%%%%%%%%%%%%

\newcommand\expinc{\text{Rapid}}

\begin{problem}
A function $f:\nngint \to \nngint$ is defined to \emph{rapidly
  increasing} if
\[
f(n+1) \geq 2^{f(n)}
\]
for all $n \in \nngint$.  Let $\expinc$ be the set of rapidly
increasing functions.

Use a \emph{diagonalization argument} to prove that $\expinc$ is
uncountable.

\begin{solution}
\begin{proof}
Assume to the contrary that $\expinc$ is countable, so $\expinc =
\set{f_0,f_1,\dots,f_n,\dots}$.  Now we define a function $t: \nngint \to
\nngint$ recursively as follows:
\begin{align*}
t(0) & \eqdef 1 + f_0(0),\\
t(n+1) & \eqdef \max(\, 2^{t(n)}, 1+f_{n+1}(n+1)\,).
\end{align*}

We have by definition that
\[
t(n+1) \geq 2^{t(n)},
\]
for all $n \in \nngint$, so $t \in \expinc$.  But we also have
\[
t(n) \geq 1 + f_n(n) > f_n(n),
\]
for all $n \in \nngint$.  In particular, $t \neq f_n$ for all $n \in
\nngint$, contradicting the assumption that $\expinc =
\set{f_n \suchthat n \in \nngint}$.  So the assumption that $\expinc$ is
countable must be false.
\end{proof}
\end{solution}

\end{problem}

%%%%%%%%%%%%%%%%%%%%%%%%%%%%%%%%%%%%%%%%%%%%%%%%%%%%%%%%%%%%%%%%%%%%%
% Problem ends here
%%%%%%%%%%%%%%%%%%%%%%%%%%%%%%%%%%%%%%%%%%%%%%%%%%%%%%%%%%%%%%%%%%%%%

\endinput
