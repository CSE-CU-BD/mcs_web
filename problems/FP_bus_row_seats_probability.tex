\documentclass[problem]{mcs}

\begin{pcomments}
  \pcomment{FP_bus_row_seats_probability}
  \pcomment{ZDz 12/08/15}
  \pcomment{f15.final}
\end{pcomments}

\pkeywords{
  counting
  probability
  expectation
}

%%%%%%%%%%%%%%%%%%%%%%%%%%%%%%%%%%%%%%%%%%%%%%%%%%%%%%%%%%%%%%%%%%%%%
% Problem starts here
%%%%%%%%%%%%%%%%%%%%%%%%%%%%%%%%%%%%%%%%%%%%%%%%%%%%%%%%%%%%%%%%%%%%%

\begin{problem}
You and two friends are getting on a bus. The bus has a total of $3n$
seats: $n$ rows with $3$ seats in each row. $k$ people have entered
the bus already and occupied $k$ seats at random (i.e., the probability
of any subset of $k$ seats being occupied is the same).

\bparts
 
 
\ppart
What is the probability that the first row is empty (all three seats)?

\examspace[2in]

\begin{solution}
\[
\frac{\binom{3n-3}{k}}{\binom{3n}{k}}
\]
There are $\binom{3n}{k}$ possible seating assignments for $k$
people who already entered, out of which $\binom{3n-3}{k}$
result in the first row being empty (by choosing $k$ seats among
the remaining $3n-3$ seats).
\end{solution}


\ppart
What is the expected number of empty rows?
You may express your answer in terms of a number $a$ which
is the answer to the previous part of this problem.
Justify your answer.

\examspace[1in]

\begin{solution}
$n a$.
Let $R_i$ be an indicator random variable that takes value 1 if
row $i$ is empty, and 0 otherwise. Then,
\[
\expect{R_i} = \pr{R_i = 1} = a \, .
\]
By linearity of expectation,
\[
\expect{\sum_{i=1}^{N} R_i} = \sum_{i=1}^{N} \expect{R_i = 1} = n a \, ,
\]
which is the expected number of empty rows.
\end{solution}


\ppart
If you know that there were $n$ people who have already entered the bus
(i.e., $k = n$), what is the probability that none of the rows is empty?

\examspace[2in]

\begin{solution}
\[
\frac{3^n}{\binom{3n}{n}}
\]
The only way that all rows are nonempty is if there is a single person
in each row. A single person can take any of the 3 seats in a row, so,
by product rule, there are $3^n$ ways $n$ people can ``occupy'' all
rows. Therefore, the probability that all rows are nonempty is equal to
$3^n / \binom{3n}{n}$.
\end{solution}


\begin{staffnotes}
The following parts are more suitable for a class or pset problem.
\end{staffnotes}
\ppart

Before entering the bus, someone tells you that the last row is empty.
How does that change the probability that first row is empty?
Prove your answer.

\examspace[3in]

\begin{solution}
Intuitively, that information should decrease the probability that the
first row is empty because all of the people must occupy seats
in the first $n-1$ rows.

That can be proved algebraically in the following way.
Let $R_i$ be an indicator random variable that takes value 1 if
row $i$ is empty, and 0 otherwise. From previous parts, we have
\[
\pr{R_1 = 1} = \frac{\binom{3n-3}{k}}{\binom{3n}{k}} \, .
\]
Also,
\[
\prcond{R_1 = 1}{R_n = 1} = \frac{R_1 = 1 \QAND R_n = 1}{R_n = 1} = \frac{\binom{3n-6}{k}}{\binom{3n-3}{k}} \, ,
\]
which is the ratio of the number of seat assignments in which
the first and the last row are empty and the number of assignments
in which the last row is empty.
These two probabilities can be further written as
\begin{align*}
\pr{R_1 = 1} & = \frac{\binom{3n-3}{k}}{\binom{3n}{k}} \\
	 	    & = \frac{\dfrac{(3n-3)!}{(3n-3-k)! \cdot k!}}{\dfrac{(3n)!}{(3n-k)! \cdot k!}} \\
	 	    & = \frac{(3n-k)(3n-k-1)(3n-k-2)}{(3n)(3n-1)(3n-2)} \\		    
	 	    & = \left(1 - \frac{k}{3n}\right)\left(1-\frac{k}{3n-1}\right)\left(1-\frac{k}{3n-2}\right)		    
\end{align*}
and
\begin{align*}
\prcond{R_1 = 1}{R_n = 1}
		    & = \frac{\binom{3n-6}{k}}{\binom{3n-3}{k}} \\
	 	    & = \frac{\dfrac{(3n-6)!}{(3n-6-k)! \cdot k!}}{\dfrac{(3n-3)!}{(3n-3-k)! \cdot k!}} \\
	 	    & = \frac{(3n-3-k)(3n-4-k)(3n-5-k)}{(3n-3)(3n-4)(3n-5)} \\		    
	 	    & = \left(1 - \frac{k}{3n-3}\right)\left(1-\frac{k}{3n-4}\right)\left(1-\frac{k}{3n-5}\right) \, .		    
\end{align*}
Since $1 - \frac{k}{3n} > 1 - \frac{k}{3n-3}$, $1 - \frac{k}{3n-1} > 1 - \frac{k}{3n-4}$
and $1 - \frac{k}{3n-2} > 1 - \frac{k}{3n-5}$,
$\prcond{R_1 = 1}{R_n = 1} < \pr{R_1 = 1}$ follows.

Here is an alternative solutions. Let $C_i$ be a random variable that is
equal to the number of people sitting in row $i$. Therefore $C_i \in \set{0, 1, 2, 3}$.
Note that $\pr{C_1 = 0}$ is the probability that the first row is empty, while
$\prcond{C_1 = 0}{C_n = 0}$ is the probability that the first row is empty given
that the last row is empty. By the law of total probability,
\begin{align*}
\pr{C_1 = 0} & = \pr{C_n = 0} \prcond{C_1 = 0}{C_n = 0} \\
		    & + \pr{C_n = 1} \prcond{C_1 = 0}{C_n = 1} \\
		    & + \pr{C_n = 2} \prcond{C_1 = 0}{C_n = 2} \\
		    & + \pr{C_n = 3} \prcond{C_1 = 0}{C_n = 3} \\
		    & > \left(\pr{C_n = 0} + \pr{C_n = 1} + \pr{C_n = 2} + \pr{C_n = 3}\right) \prcond{C_1 = 0}{C_n = 0} \\
		    & = \prcond{C_1 = 0}{C_n = 0} \, .
\end{align*}
The inequality above follows from the fact that
\[
\prcond{C_1 = 0}{C_n = 0} < \prcond{C_1 = 0}{C_n = 1} < \prcond{C_1 = 0}{C_n = 2} < \prcond{C_1 = 0}{C_n = 3} \, ,
\]
which is true because increasing the number of people in the first $n-1$ rows decreases
the chances that the first row is empty.
\end{solution}


\ppart
Now, you go into the bus. It is difficult to count the number of people,
which would allow you to use your result from part a.\ to find the exact
probability that any row is empty.
Instead, you look at the first $m$ rows and find that $e$ of them are empty.
You want to conclude that the probability that any row is empty
is around $\frac{e}{m}$
and find a confidence level at which it deviates from this
number by not more than 0.05.
Can you apply the pairwise independent sampling theorem to
do that? Explain.

\examspace[1in]

\begin{solution}
No, the event that row $i$ is empty is not independent of the event that
row $j$ is empty.
\end{solution}

\eparts

\end{problem}

%%%%%%%%%%%%%%%%%%%%%%%%%%%%%%%%%%%%%%%%%%%%%%%%%%%%%%%%%%%%%%%%%%%%%
% Problem ends here
%%%%%%%%%%%%%%%%%%%%%%%%%%%%%%%%%%%%%%%%%%%%%%%%%%%%%%%%%%%%%%%%%%%%%

\endinput
