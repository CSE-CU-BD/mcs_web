\documentclass[problem]{mcs}

\begin{pcomments}
  \pcomment{PS_sum_of_sixth_powers}
  \pcomment{from S07.ps8}
\end{pcomments}

\pkeywords{
  asymptotic
  Theta
  sum
}

%%%%%%%%%%%%%%%%%%%%%%%%%%%%%%%%%%%%%%%%%%%%%%%%%%%%%%%%%%%%%%%%%%%%%
% Problem starts here
%%%%%%%%%%%%%%%%%%%%%%%%%%%%%%%%%%%%%%%%%%%%%%%%%%%%%%%%%%%%%%%%%%%%%


\begin{problem}

Prove that $\sum_{k=1}^n k^6 = \Theta(n^7)$.

\begin{solution}
Let $S_n \eqdef \sum_{k=1}^n k^6$.

One approach is to use the Integral Method:
\[
\frac{n^7}{7} = \int_0^n x^6 \ dx \leq S_n \leq \int_0^n (x+1)^6 \ dx =
\frac{(n+1)^7}{7} - \frac{1}{7}.
\]
So we have $n^7 \leq 7S_n$, and so $n^7= O(S_n)$.  Also $(n+1)^7/7 - 1/7 =
O(n^7)$, and so $S_n = O(n^7)$.  Hence, $S_n = \Theta(n^7)$.

An alternative approach not using the Integral Method goes as follows.
There are $n$ terms in $S_n$ and each term is at most $n^6$, so $S_n \leq
n \cdot n^6 = n^7 = O(n^7)$. So $S_n=O(n^7)$.

On the other hand, at least $(n-1)/2$ of the terms are as large as $[(n-1)/2]^6$, 
so 
\begin{eqnarray*}
S_n &\geq & ((n-1)/2) \cdot [(n-1)/2]^6\\
    & = & [(n-1)/2]^7 \\
    & \geq & (n/3)^7
\end{eqnarray*}
for $n>3$, so $n^7 \leq 3^7 \cdot S_n$.  In other words, $n^7 = O(S_n)$.

\end{solution}

\end{problem}


%%%%%%%%%%%%%%%%%%%%%%%%%%%%%%%%%%%%%%%%%%%%%%%%%%%%%%%%%%%%%%%%%%%%%
% Problem ends here
%%%%%%%%%%%%%%%%%%%%%%%%%%%%%%%%%%%%%%%%%%%%%%%%%%%%%%%%%%%%%%%%%%%%%

\endinput
 
