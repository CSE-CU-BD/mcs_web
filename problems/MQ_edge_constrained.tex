\documentclass[problem]{mcs}

\begin{pcomments}
  \pcomment{from: S09.cp6r}
  \pcomment{from: S07.cp6w (slightly edited/shortened)}
\end{pcomments}

\pkeywords{
  bipartite_matching
  degree-constrained
  Halls_Theorem
}

%%%%%%%%%%%%%%%%%%%%%%%%%%%%%%%%%%%%%%%%%%%%%%%%%%%%%%%%%%%%%%%%%%%%%
% Problem starts here
%%%%%%%%%%%%%%%%%%%%%%%%%%%%%%%%%%%%%%%%%%%%%%%%%%%%%%%%%%%%%%%%%%%%%
\begin{problem}
Because of the incredible popularity of 6.042, Rajeev decides to give up on regular office hours. Instead, students can join any number of study groups that discuss problems and work together.  Each group must choose a delegate to relay unanswered questions to the staff, but in order to share the work, no student is allowed to be the delegate of more than one group.
\bparts

\ppart  Explain how to model the delegate selection problem as a bipartite
matching problem.

\begin{solution}
Define a bipartite graph with the study groups as one set of
vertices and everybody who belongs to some group as the other set of
vertices.  Let a group and a student be adjacent exactly when the student
belongs to the group.  Now a matching of study groups to students will give a
proper selection of delegates: every group will have a delegate, and every
delegate will represent exactly one club.
\end{solution}

\examspace[3.5in]

\ppart The staff's records show that no student is a member of more than 6
groups.  Additionally, in order to receive help, the groups must have at least 10 members.  That's enough to
guarantee there is a proper delegate selection.  Explain. 


\begin{solution}
The degree of every group is at least 10, and the degree of every
student is at most 6, so the graph is \emph{degree-constrained} (see the
Appendix) which implies there will be no bottlenecks to prevent a
matching.  Hall's Theorem then guarantees a matching.
\end{solution}

\eparts
\end{problem}

%%%%%%%%%%%%%%%%%%%%%%%%%%%%%%%%%%%%%%%%%%%%%%%%%%%%%%%%%%%%%%%%%%%%%
% Problem ends here
%%%%%%%%%%%%%%%%%%%%%%%%%%%%%%%%%%%%%%%%%%%%%%%%%%%%%%%%%%%%%%%%%%%%%
