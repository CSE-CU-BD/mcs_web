\documentclass[problem]{mcs}

\begin{pcomments}
  \pcomment{from: S09.cp6r}
  \pcomment{from: S07.cp6w (slightly edited/shortened)}
\end{pcomments}

\pkeywords{
  bipartite_matching
  degree-constrained
  Halls_Theorem
}

%%%%%%%%%%%%%%%%%%%%%%%%%%%%%%%%%%%%%%%%%%%%%%%%%%%%%%%%%%%%%%%%%%%%%
% Problem starts here
%%%%%%%%%%%%%%%%%%%%%%%%%%%%%%%%%%%%%%%%%%%%%%%%%%%%%%%%%%%%%%%%%%%%%

\begin{problem}
Alpha Alpha Alpha is an exclusive sorority with members who are both the thriftiest and the trendiest girls on campus.
Cindi, the social chair, thinks that because the sorority has become so popular over the last year, they may have trouble coordinating outfits.
She tells the president, Wendi, that two girls at their fall formal might be forced to wear identical pairs of shoes, ruining the event. 
Luckily, Wendi took 6.042 and recognizes a matching problem when she sees one.


\bparts

\ppart  Explain how to model the outfit selection problem as a bipartite
matching problem.

\begin{solution}
Define a bipartite graph with the girls as one set of
vertices and every pair of shoes in some girl's wardrobe as the other set of
vertices.  Let a girl and a shoe be adjacent exactly when the girl owns that shoe.  Now a matching of girls to possible shoes will give a
proper selection of outfits : every girl will wear a unique outfit, and every outfit will be worn by exactly one girl.
\end{solution}

\examspace[3.5in]

\ppart
Wendi mentions that the Tri-Alphs take stringent measures to make sure that no sorority is thriftier or trendier:
In order to become a sister, each Tri-Alph pledge must purchase at least 17 pairs of shoes on a budget of only 45 dollars.
Additionally, the Tri-Alph Trendiness Committee strictly enforces the rule that no ten girls in the sorority are allowed to own identical pairs of shoes.
Wendi says that this is enough for her to guarantee there is a proper outfit selection.  Explain. 


\begin{solution}
The degree of every girl is at least 17, and the degree of every
shoe is at most 9, so the graph is \emph{degree-constrained} (see the
Appendix) which implies there will be no bottlenecks to prevent a
matching.  Hall's Theorem then guarantees a matching.
\end{solution}

\eparts
\end{problem}

%%%%%%%%%%%%%%%%%%%%%%%%%%%%%%%%%%%%%%%%%%%%%%%%%%%%%%%%%%%%%%%%%%%%%
% Problem ends here
%%%%%%%%%%%%%%%%%%%%%%%%%%%%%%%%%%%%%%%%%%%%%%%%%%%%%%%%%%%%%%%%%%%%%
