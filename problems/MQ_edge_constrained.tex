\documentclass[problem]{mcs}

\begin{pcomments}
  \pcomment{MQ_edge_constrained}
  \pcomment{from: S09.cp6r}
  \pcomment{from: S07.cp6w, edited/shortened by Tom Brown & ARM, 11/2/09}
\end{pcomments}

\pkeywords{
  bipartite_matching
  degree-constrained
  Halls_Theorem
  matching
  bottleneck
}

%%%%%%%%%%%%%%%%%%%%%%%%%%%%%%%%%%%%%%%%%%%%%%%%%%%%%%%%%%%%%%%%%%%%%
% Problem starts here
%%%%%%%%%%%%%%%%%%%%%%%%%%%%%%%%%%%%%%%%%%%%%%%%%%%%%%%%%%%%%%%%%%%%%
\begin{problem}
  Because of the incredible popularity of Math for Computer Science,
  Rajeev decides to give up on regular office hours.  Instead, each
  student can join some study groups.  Each group must choose a
  representative to talk to the staff, but there is a staff rule that
  a student can only represent one group.  The problem is to find a
  representative from each group while obeying the staff rule.

 \bparts

 \ppart Explain how to model the delegate selection problem as a bipartite
 matching problem.

\begin{solution}
  Define a bipartite graph with the study groups as one set of vertices
  and students in the groups as the other set of vertices.  A group and a
  student are adjacent exactly when the student belongs to the group.  Now
  a matching of study groups to students will give a proper selection of
  delegates: every group will have a delegate, and every delegate will
  represent exactly one club.
\end{solution}

\examspace[3.5in]

\ppart The staff's records show that no student is a member of more than 4
groups, and all the groups must have at least 4 members.  That's enough to
guarantee there is a proper delegate selection.  Explain.

\begin{solution}
  The degree of every group is at least 4, and the degree of every student
  is at most 4, so the graph is \emph{\idx{degree-constrained}}
  (Def.~\bref{degree-constrained_def}) which implies there will be no
  bottlenecks to prevent a matching.  Hall's Theorem then guarantees a
  matching.
\end{solution}

\eparts
\end{problem}

%%%%%%%%%%%%%%%%%%%%%%%%%%%%%%%%%%%%%%%%%%%%%%%%%%%%%%%%%%%%%%%%%%%%%
% Problem ends here
%%%%%%%%%%%%%%%%%%%%%%%%%%%%%%%%%%%%%%%%%%%%%%%%%%%%%%%%%%%%%%%%%%%%%

\endinput
