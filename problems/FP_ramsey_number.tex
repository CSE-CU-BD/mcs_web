\documentclass[problem]{mcs}

\begin{pcomments}
  \pcomment{FP_ramsey_number}
  \pcomment{combinatorial proof of friends-and-strangers theorem}
  \pcomment{probably more suited for CP or PS?}
  \pcomment{CH, Spring '14}
\end{pcomments}

\pkeywords{
  combinatorics
  pigeon hole principle
  ramsey number
  friends and strangers
}

%%%%%%%%%%%%%%%%%%%%%%%%%%%%%%%%%%%%%%%%%%%%%%%%%%%%%%%%%%%%%%%%%%%%%
% Problem starts here
%%%%%%%%%%%%%%%%%%%%%%%%%%%%%%%%%%%%%%%%%%%%%%%%%%%%%%%%%%%%%%%%%%%%%

\begin{problem}

\bparts

Recall that the \emph{complete graph}, $K_n$, is a simple graph with $n$ vertices and an
edge between every pair of vertices. Further, a \emph{triangle} is defined as a simple graph with 3 vertices and 3 edges.

\ppart Show that the number of triangles in $K_6$ equals 20.

\begin{solution}
Denote the vertices of $K_6$ as $V = \{v_1, v_2, \ldots, v_6\}$. There
is a bijection between triangles in $K_6$ and subsets of $V$ of size
3. Therefore, the number of triangles in $K_6$ is
\[
\binom{6}{3} = 20 .
\]
\end{solution}
\examspace[1.5in]

\ppart 
Now suppose that every edge in $K_6$ is either colored red or blue. A
\emph{red-blue sequence} is an ordered triple of distinct vertices $(x,v,y)$
such that $xv$ is red and $vy$ is blue. Show that the number of
red-blue sequences is upper-bounded by 36.
\begin{solution}
Fix the middle vertex $v$. It is connected to $r$ vertices through a red
edge, and $b$ vertices through a blue edge, where $r+b = 5$, the
degree of $v$. The number of choices for red-blue sequences with $v$
as the middle vertex is $rb$, which attains a maximum value of 6 for $(r,b) =
(2, 3)$ or $(3,2)$. Summing over different $v$, the total number of red-blue sequences
is at most $6 \times 6 = 36$.
\end{solution}

\examspace[1.5in]

\ppart Argue that there exists a 2-to-1 mapping between red-blue
sequences and the set of \emph{bi-chromatic triangles}, i.e., the set
of triangles in $K_6$ with both red and blue edges.
\begin{solution}
Consider any bi-chromatic triangle $\{x,y,z\}$. We have two disjoint cases:
\begin{enumerate}
\item Only 1 edge is colored blue. Without loss of
  generality, suppose that $xz$ is blue. Then, $(y,x,z)$ and
  $(y,z,x)$ are the only red-blue sequences.
\item Only 1 edge is colored red. Without loss of
  generality, suppose that $xz$ is red. Then, $(x,z,y)$ and $(z,x,y)$
  are the only red-blue sequences.
\end{enumerate}
Therefore, every bi-chromatic triangle maps to precisely 2 red-blue sequences.
\end{solution}

\examspace[1.5in]

\ppart Use the facts proved above to prove the following statement:
\begin{quote}
Every collection of 6 people always includes a group of 3 mutual
acquaintances, or a group of 3 mutual strangers.
\end{quote}
\begin{solution}
Form the complete graph with 6 vertices and color edge $(i,j)$ \emph{red} if
person $i$ and person $j$ are acquainted, and \emph{blue} if they are
strangers. Then, the statement is equivalent to proving that $K_6$
always includes a monochromatic triangle. 

But we showed above that there are at most 36 red-blue triples and
every bi-chromatic triangle maps to exactly 2 such triples. Therefore,
there are at most 18 bi-chromatic triangles. The total number of
triangles equals 20, and therefore $K_6$ contains at least 2
monochromatic triangles. So in fact, we have proved a (slightly)
stronger statement.

\end{solution}

\eparts

\end{problem}

%%%%%%%%%%%%%%%%%%%%%%%%%%%%%%%%%%%%%%%%%%%%%%%%%%%%%%%%%%%%%%%%%%%%%
% Problem ends here
%%%%%%%%%%%%%%%%%%%%%%%%%%%%%%%%%%%%%%%%%%%%%%%%%%%%%%%%%%%%%%%%%%%%%

\endinput
