\documentclass[problem]{mcs}

\begin{pcomments}
  \pcomment{PS_nim_strategy}
  \pcomment{F14.ps2}
  \pcomment{edited ARM 9/24/14}
\end{pcomments}

\pkeywords{
  Nim
  xor
  binary
  strategy
  state_machines
}

%%%%%%%%%%%%%%%%%%%%%%%%%%%%%%%%%%%%%%%%%%%%%%%%%%%%%%%%%%%%%%%%%%%%%
% Problem starts here
%%%%%%%%%%%%%%%%%%%%%%%%%%%%%%%%%%%%%%%%%%%%%%%%%%%%%%%%%%%%%%%%%%%%%

\begin{problem}
Nim is a game played between two players with three piles of stones.
Players alternate removing stones.  A player picks a pile and removes
any positive number of stones.  The goal is to be the last player to
take a stone.

The winning strategy in Nim requires computing a Nim sum.  A
Nim sum of numbers $r$, $s$ and $t$ is computed by taking the binary
representation of the three numbers and combining these into a single
binary string by applying $\QXOR$ bit-wise.

For example, the Nim sum of $2, 7$ and $9$ is
\[\begin{array}{ccccr}
   &   & 1 & 0 & \text{(binary rep of 2)}\\
   & 1 & 1 & 1 & \text{(binary rep of 7)}\\
 1 & 0 & 0 & 1 & \text{(binary rep of 9)}\\
\hline
 1 & 1 & 0 & 0 & \text{(Nim sum)}
\end{array}\]

\bparts

\ppart Prove that if the Nim sum of the numbers of stones in each pile
is zero, then any move will result in Nim sum that is not zero.

\begin{solution}
When a player removes stones from a pile, some digit of the binary
representation of the number of stones in the pile must change.  This
implies that the corresponding digit of the Nim sum---that is, the Nim
sum digit at the position of the changed digit---must itself change,
and since it was zero, it must become one.  That makes the Nim sum
nonzero.
\end{solution}

\ppart Prove that if the Nim sum is not zero that it is always
possible to make the Nim sum zero with one move.

\begin{solution}
 Let the Nim sum be $s$.  Say the high order bit of $s$ occurs at the
 $n$th bit.  Since the $n$th bit of the Nim sum is one, the number of
 stones in least one of the piles must also have a one in its $n$th
 bit.  Say one such pile has $p$ stones.

 Let $p'$ be integer whose binary representation is the Nim sum of $s$
 and $p$.  Now the binary representation of $p'$ has the same same
 digits as $p$ in positions above the $n$th, and has $n$th bit zero.
 Since $p$ has $n$th bit one, it follows that $p$ is larger than $p'$.
 
 So it's possible to remove stones to reduce the pile of $p$ stones to
 $p'$ stones.  The Nim sum of the piles now becomes the old Nim sum
 with $p$ replaced by the Nim sum of $p$ and $s$.  So the new Nim sum
 is the Nim sum of the old Nim sum and $s$, namley, the Nim sum of $s$
 and $s$, which is zero.
\end{solution}

\ppart Conclude that if the game begins with a nonzero Nim sum, then
the first player has a winning strategy.

\hint Describe a preserved invariant that the first player can
maintain.

\begin{solution}
The first player can remove stones such that the Nim sum is zero.
With this strategy by the first player, the Nim sum being nonzero on
the first player's turn and zero on the second player's turn becomes a
preserved invariant of the game.  Since the total number of stones
decreases at every move, the game must eventually end with no stones,
and this can only happen on the second player's turn.  That is, the
second player must lose.
\end{solution}

\eparts
\end{problem}

\endinput
