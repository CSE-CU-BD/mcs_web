\documentclass[problem]{mcs}

\begin{pcomments}
\pcomment{PS_tournament_matching_problem}
\pcomment{soln by Prekshah Naik & ARM 4/9/17}
\pcomment{Cite Putnam exam?}
\end{pcomments}

\pkeywords{
halls_theorem
tournament
bipartite
matching
}

%%%%%%%%%%%%%%%%%%%%%%%%%%%%%%%%%%%%%%%%%%%%%%%%%%%%%%%%%%%%%%%%%%%%%
% Problem starts here
%%%%%%%%%%%%%%%%%%%%%%%%%%%%%%%%%%%%%%%%%%%%%%%%%%%%%%%%%%%%%%%%%%%%%

\begin{problem}
Suppose $2m$ teams play in a round-robin tournament.  Over a period of
$2m - 1$ days, every team plays every other team exactly once. There
are no ties.  Show that for each day we can select a winning team,
without selecting the same team twice.

\begin{solution}
We make use of Hall's Theorem.  We can model this problem as a
bipartite graph $G$.  Let $\leftbi{G}$ be the set of $2m-1$ days, and
let $\rightbi{G}$ be the set of $2m$ teams.  When on day $d$, team $t$
won its match, there will be an edge from $d$ to $t$.  Selecting a
different winner for each day of the tournament now simply means
finding a bipartite matching that covers $\leftbi{G}$.

For any set of days $D \subseteq \leftbi{G}$, let $\neighbors{D}$ be
the set of teams that won on at least one of those days.  Using Hall's
Theorem, we know the desired matching will exist iff the number of
winners during each of those days is at least as large as the number
of days, that is,
\begin{equation}\tag{DND}
\card{D} \leq \card{\neighbors{D}}.
\end{equation}

Now if every team won on one or more of the days in $D$, then
\[
\card{D} \leq 2m-1 < 2m = \card{\neighbors{D}},
\]
and~(DND) holds.  On the other hand, suppose there is a team $t$ that
did not win on any of the days in $D$.  By the rules of the
tournament, a different team must have won against $t$ on each day in
$D$.  Let $W_t$ be this set of teams.  So we have
\[
\card{D}  = \card{W_t}  \leq \card{\neighbors{D}}
\]
since $W_t \subseteq \neighbors{D}$.  Again~(DND) holds.  

So~(DND) holds in any case, and we conclude by Hall's Theorem that
the desired matching exists.
\end{solution}

\end{problem}

%%%%%%%%%%%%%%%%%%%%%%%%%%%%%%%%%%%%%%%%%%%%%%%%%%%%%%%%%%%%%%%%%%%%%
% Problem ends here
%%%%%%%%%%%%%%%%%%%%%%%%%%%%%%%%%%%%%%%%%%%%%%%%%%%%%%%%%%%%%%%%%%%%%

\endinput
