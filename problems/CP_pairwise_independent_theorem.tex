\documentclass[problem]{mcs}

\begin{pcomments}
  \pcomment{from: S09.cp14t}
  \pcomment{from: F07 rec15t}
\end{pcomments}

\pkeywords{
}

%%%%%%%%%%%%%%%%%%%%%%%%%%%%%%%%%%%%%%%%%%%%%%%%%%%%%%%%%%%%%%%%%%%%%
% Problem starts here
%%%%%%%%%%%%%%%%%%%%%%%%%%%%%%%%%%%%%%%%%%%%%%%%%%%%%%%%%%%%%%%%%%%%%

\begin{problem}
	
The proof of the Pairwise Independent Sampling Theorem you just saw in
lecture was given for a sequence $R_1, R_2, \dots$ of pairwise
independent random variables with the same mean and variance.

We can generalize the Theorem to sequences of pairwise independent random
variables, possibly with \emph{different} distributions, as long as all
their variances are bounded by some constant.

\begin{theorem*}[Generalized Pairwise Independent Sampling]
Let $X_1, X_2, \dots$ be a sequence of pairwise independent random
variables such that $\variance{X_i} \leq b$ for some $b\geq 0$ and all
$i\geq 1$.  Let
\begin{align*}
A_n & \eqdef \frac{X_1 + X_2 + \cdots + X_n}{n},\\
\mu_n & \eqdef \expect{A_n}.
\end{align*}
Then for every $\epsilon > 0$,
\begin{equation}\label{gpis}
\pr{\abs{A_n - \mu_n} > \epsilon} \leq \frac{b}{\epsilon^2} \cdot \frac{1}{n}.
\end{equation}
\end{theorem*}

\bparts
\ppart
Prove the Generalized Pairwise Independent Sampling Theorem. 

\begin{solution}
Essentially identical to the proof in the Appendix, except that
$R$ gets replaced by $X$ and $\variance{X_i}$ by $b$, with the
equality where the $b$ is first used becoming $\leq$.
\end{solution}

\ppart
Conclude that the following holds:
\begin{corollary*}[Generalized Weak Law of Large Numbers]
For every $\epsilon > 0$,
\[
\lim_{n \rightarrow \infty}
        \pr{\abs{A_n - \mu_n}  \leq \epsilon} = 1.
\]
\end{corollary*}

\begin{solution}
\begin{align*}
\pr{\abs{A_n - \mu_n}  \leq \epsilon} = & 1- \pr{\abs{A_n - \mu_n}  >
 \epsilon}\\
   \geq & 1- b/(n\epsilon^2) & \text{(by~\eqref{gpis})},
\end{align*}
and for any fixed $\epsilon$, this last term approaches 1 as $n$
approaches infinity.
\end{solution}

\eparts

\end{problem}

%%%%%%%%%%%%%%%%%%%%%%%%%%%%%%%%%%%%%%%%%%%%%%%%%%%%%%%%%%%%%%%%%%%%%
% Problem ends here
%%%%%%%%%%%%%%%%%%%%%%%%%%%%%%%%%%%%%%%%%%%%%%%%%%%%%%%%%%%%%%%%%%%%%

\endinput
