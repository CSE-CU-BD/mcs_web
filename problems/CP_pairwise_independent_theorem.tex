\documentclass[problem]{mcs}

\begin{pcomments}
  \pcomment{CP_pairwise_independent_theorem}
  \pcomment{from: S09.cp14t, F07.rec15t}
\end{pcomments}

\pkeywords{
  pairwise_independent
  sampling
  variance
  law_of_large_numbers
}

%%%%%%%%%%%%%%%%%%%%%%%%%%%%%%%%%%%%%%%%%%%%%%%%%%%%%%%%%%%%%%%%%%%%%
% Problem starts here
%%%%%%%%%%%%%%%%%%%%%%%%%%%%%%%%%%%%%%%%%%%%%%%%%%%%%%%%%%%%%%%%%%%%%

\begin{problem}
	
The proof of the \idx{Pairwise Independent Sampling}
Theorem~\bref{th:pairwise-sampling} was given for a sequence $R_1,
R_2, \dots$ of pairwise independent random variables with the same
mean and variance.

The theorem generalizes straighforwardly to sequences of pairwise
independent random variables, possibly with \emph{different}
distributions, as long as all their variances are bounded by some
constant.

\begin{theorem*}[Generalized Pairwise Independent Sampling]
Let $X_1, X_2, \dots$ be a sequence of pairwise independent random
variables such that $\variance{X_i} \leq b$ for some $b\geq 0$ and all
$i\geq 1$.  Let
\begin{align*}
A_n & \eqdef \frac{X_1 + X_2 + \cdots + X_n}{n},\\
\mu_n & \eqdef \expect{A_n}.
\end{align*}
Then for every $\epsilon > 0$,
\begin{equation}\label{gpis}
\pr{\abs{A_n - \mu_n} > \epsilon} \leq \frac{b}{\epsilon^2} \cdot \frac{1}{n}.
\end{equation}
\end{theorem*}

\bparts
\ppart
Prove the Generalized Pairwise Independent Sampling Theorem.

\begin{solution}
Essentially identical to the proof of
Theorem~\bref{th:pairwise-sampling} in the text, except that $S_n/n$
gets replaced by $A_n$, the constant $b$ replaces $\variance{G_i}$,
and the equality before the first occurrence of $b$ gets replaced by
an inequality ($\leq$).

\begin{staffnotes}

\begin{proof}
We first observe that
\begin{equation}\label{}
\Variance{A_n} \leq \frac{b}{n}.\label{vanbn}
\end{equation}
because
\begin{align*}
\Variance{A_n}
 & =  \frac{1}{n^2} \Variance{\sum_{i=1}^n X_i} 
          & \text{(def of $A_n$ \& Square Multiple Rule, Theorem~\bref{var.const})}\\
 & =  \frac{1}{n^2} \sum_{i=1}^n \variance{X_i}
        & \text{(pairwise independent additivity)}\\
 & \leq \frac{1}{n^2}\cdot nb
   & (\text{since } \variance{X_i} \leq b)\\
 & =  \frac{b}{n}.
\end{align*}

This is enough to apply \idx{Chebyshev's Theorem} and conclude:
\begin{align*}
\Prob{\abs{A_n - \mu_n} \geq \epsilon}
    & \leq \frac{\Variance{A_n}}{\epsilon^2}.
         & \text{(Chebyshev's bound)}\\
    & \leq \frac{b/n}{\epsilon^2}
         & \text{(by~\eqref{vanbn})}\\
    & = \frac{b}{\epsilon^2} \cdot \frac{1}{n}\ .
\end{align*}

\end{proof}

\end{staffnotes}

\end{solution}

\ppart
Conclude that the following holds:
\begin{corollary*}[Generalized \idx{Weak Law of Large Numbers}]
For every $\epsilon > 0$,
\[
\lim_{n \rightarrow \infty}
        \pr{\abs{A_n - \mu_n}  \leq \epsilon} = 1.
\]
\end{corollary*}

\begin{solution}
\begin{align*}
\pr{\abs{A_n - \mu_n}  \leq \epsilon}
  & =  1 - \pr{\abs{A_n - \mu_n}  > \epsilon}\\
  & \geq 1 - \frac{b}{\epsilon^2} \cdot \frac{1}{n}
    & \text{(by~\eqref{gpis})},
\end{align*}
and for any fixed $\epsilon$, this last term approaches 1 as $n$
approaches infinity.
\end{solution}

\eparts

\end{problem}

%%%%%%%%%%%%%%%%%%%%%%%%%%%%%%%%%%%%%%%%%%%%%%%%%%%%%%%%%%%%%%%%%%%%%
% Problem ends here
%%%%%%%%%%%%%%%%%%%%%%%%%%%%%%%%%%%%%%%%%%%%%%%%%%%%%%%%%%%%%%%%%%%%%


\endinput
