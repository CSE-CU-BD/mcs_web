\documentclass[problem]{mcs}

\begin{pcomments}
  \pcomment{MQ_identify_relations}
  \pcomment{Peter 2/18/12}
\end{pcomments}

\pkeywords{
  relation
  codomain
  inverse
  graph_of_relation
}

%%%%%%%%%%%%%%%%%%%%%%%%%%%%%%%%%%%%%%%%%%%%%%%%%%%%%%%%%%%%%%%%%%%%%
% Problem starts here
%%%%%%%%%%%%%%%%%%%%%%%%%%%%%%%%%%%%%%%%%%%%%%%%%%%%%%%%%%%%%%%%%%%%%

\begin{problem} \mbox{}
  Let $A$ be the set containing the five sets: $\{a}, \{b,c\}, \{b,d\}, \{a,e\}, \{e,f\}$, 
  and let $B$ be the set containing the three sets: $\{a,b\}, \{b,c,d\}, \{e,f\}$.
  the right.  Let $R$ be the ``is subset of'' binary relation from $A$ to $B$ which
  is defined by the rule
\[
S \mrel{R} T \qiff (S \subseteq T).
\]

  \bparts
  \ppart Fill in the arrows so the following figure describes the graph of
  the relation, $R$:

\[\begin{array}{lcr}
A & \hspace{1in} \text{arrows} \hspace{1in} & B\\
\hline
&\\
\{a\}\\
&\\
                                  && \{a, b\}\\
&\\
\{b, c\}\\
&\\
                                  && \{b, c, d\}\\
&\\
\{b, d\}\\
&\\
                                  && \{e, f\}\\
&\\
\{a, e\}\\
&\\
&\\
\{e, f\}\\
&\\
\end{array}\]

\begin{solution}
Four arrows for $R$:
\begin{align*}
\{a\} & \subset & \{a, b\}\\
\{b, c\} & \subset & \{b, c, d\}\\
\{b, d\} & \subset & \{b, c, d\}\\
\{e, f\} & \subset & \{e, f, g\}\\
\end{align*}
\end{solution}

\ppart Circle the properties below possessed by the
 relation $R$:
  \[
  \begin{array}{ccccc}
  \mbox{FUNCTION~~} & 
  \mbox{~~TOTAL~~} &
  \mbox{~~INJECTIVE~~} &
  \mbox{~~SURJECTIVE~~} &
  \mbox{~~BIJECTIVE} 
  \end{array}
  \]

\begin{solution}
From part (a),  the ``is subset of'' relation, $R$, is a surjective function.  
\end{solution}

  \ppart  Circle the properties below possessed by the relation $\inv{R}$:
  \[
  \begin{array}{ccccc}
  \mbox{FUNCTION~~} & 
  \mbox{~~TOTAL~~} &
  \mbox{~~INJECTIVE~~} &
  \mbox{~~SURJECTIVE~~} &
  \mbox{~~BIJECTIVE~}
  \end{array}
  \]

\begin{solution}
From part (b), the inverse relation, $\inv{R}$, is a total injection.
\end{solution}

  \eparts

\end{problem}

%%%%%%%%%%%%%%%%%%%%%%%%%%%%%%%%%%%%%%%%%%%%%%%%%%%%%%%%%%%%%%%%%%%%%
% Problem ends here
%%%%%%%%%%%%%%%%%%%%%%%%%%%%%%%%%%%%%%%%%%%%%%%%%%%%%%%%%%%%%%%%%%%%%

\endinput
