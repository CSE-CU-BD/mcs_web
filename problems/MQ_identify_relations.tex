\documentclass[problem]{mcs}

\begin{pcomments}
  \pcomment{MQ_identify_relations}
  \pcomment{Peter 2/18/12}
\end{pcomments}

\pkeywords{
  relation
  codomain
  inverse
  graph_of_relation
}

%%%%%%%%%%%%%%%%%%%%%%%%%%%%%%%%%%%%%%%%%%%%%%%%%%%%%%%%%%%%%%%%%%%%%
% Problem starts here
%%%%%%%%%%%%%%%%%%%%%%%%%%%%%%%%%%%%%%%%%%%%%%%%%%%%%%%%%%%%%%%%%%%%%

\begin{problem} \mbox{}
  Let $A$ be the set containing the five sets: $\set{a}, \set{b,c},
  \set{b,d}, \set{a,e}, \set{e,f}$, and let $B$ be the set containing the
  three sets: $\set{a,b}, \set{b,c,d}, \set{e,f}$.  Let $R$ be the ``is
  subset of'' binary relation from $A$ to $B$ defined by the
  rule:
\[
X \mrel{R} Y \quad\QIFF\quad X \subseteq Y.
\]

  \bparts
  \ppart Fill in the arrows so the following figure describes the graph of
  the relation, $R$:

\[\begin{array}{lcr}
A & \hspace{1in} \text{arrows} \hspace{1in} & B\\
\hline
&\\
\set{a}\\
&\\
                                  && \set{a, b}\\
&\\
\set{b, c}\\
&\\
                                  && \set{b, c, d}\\
&\\
\set{b, d}\\
&\\
                                  && \set{e, f}\\
&\\
\set{a, e}\\
&\\
&\\
\set{e, f}\\
&\\
\end{array}\]

\begin{solution}
Four arrows for $R$:
\begin{align*}
\set{a} & \subset & \set{a, b}\\
\set{b, c} & \subset & \set{b, c, d}\\
\set{b, d} & \subset & \set{b, c, d}\\
\set{e, f} & \subset & \set{e, f}\\
\end{align*}
\end{solution}

\ppart\label{aabbcbcd} Circle the properties below possessed by the
 relation $R$:
  \[
  \begin{array}{ccccc}
  \mbox{function~~} & 
  \mbox{~~total~~} &
  \mbox{~~injective~~} &
  \mbox{~~surjective~~} &
  \mbox{~~bijective} 
  \end{array}
  \]

\begin{solution}
From part~\eqref{aabbcbcd}, the ``is subset of'' relation $R$ is a
surjective function.
\end{solution}

  \ppart\label{propsinvR}  Circle the properties below possessed by the relation $\inv{R}$:
  \[
  \begin{array}{ccccc}
  \mbox{function~~} & 
  \mbox{~~total~~} &
  \mbox{~~injective~~} &
  \mbox{~~surjective~~} &
  \mbox{~~bijective~}
  \end{array}
  \]

\begin{solution}
From part~\eqref{propsinvR}, the inverse relation $\inv{R}$ is a
total injection.
\end{solution}

  \eparts

\end{problem}

%%%%%%%%%%%%%%%%%%%%%%%%%%%%%%%%%%%%%%%%%%%%%%%%%%%%%%%%%%%%%%%%%%%%%
% Problem ends here
%%%%%%%%%%%%%%%%%%%%%%%%%%%%%%%%%%%%%%%%%%%%%%%%%%%%%%%%%%%%%%%%%%%%%

\endinput
