\documentclass[problem]{mcs}

\begin{pcomments}
  \pcomment{PS_expected_gain}
  \pcomment{S17.final, S01.final}
  \pcomment{added edited ARM 5/20/17}
\end{pcomments}

\pkeywords{
  expectation
  geometric_sum
  bogus
}

%%%%%%%%%%%%%%%%%%%%%%%%%%%%%%%%%%%%%%%%%%%%%%%%%%%%%%%%%%%%%%%%%%%%%
% Problem starts here
%%%%%%%%%%%%%%%%%%%%%%%%%%%%%%%%%%%%%%%%%%%%%%%%%%%%%%%%%%%%%%%%%%%%%


\begin{problem} 

\begin{staffnotes}
\textbf{S17 final 8pts: part(a) 2 pts, parts(b,c) 3 pts each}
\end{staffnotes}


Short-term Capital Management (STCM) wants you to invest in a fund
with the following rules: you invest one million dollars in their
Forward Looking Internet Package (FLIP).  Each year, the money in your
FLIP account will double or halve with equal probability, and each year
STCM will pay you a dividend equal to 10\% of the money in your account.

\bparts

\ppart What is the expected number of dollars in your account at the
end of $k$ years?  Write a simple formula in terms of $k$.

\hint \$1,000,000 is in the account the end of year zero.  Let $X_i$
be $2$ or $1/2$ depending on what happens to your money at the end of
the $i$th year.  So the amount of money in the account at the end of
year one is $X_1 \cdot \$1,000,000$ and the dividend paid is
$(1/10)X_1 \cdot \$1,000,000$.

\begin{center}
\exambox{1.2in}{0.5in}{-0.1in}
\end{center}

\examspace[0.7in]

\begin{solution}
\[
10^6 \paren{\frac{5}{4}}^k \,.
\]

The number of dollars in the account after $k$~years is
\[
10^6 \cdot X_1 \cdot X_2 \cdots X_k,
\]
where $X_i$ equals $2$ or $1/2$ with equal probability.
Since the $X_i$'s~are independent,
\[
\expect{10^6 \cdot X_1 \cdot X_2 \cdots X_k}
   = 10^6 \expect{X_1} \cdot \expect{X_2} \cdots \expect{X_k}.
\]
Also,
\[
\expect{X_i} = 2\frac{1}{2} + \frac{1}{2}\frac{1}{2} = \frac{5}{4}.
\]
Therefore, the expected number of dollars after $k$~years is
\[
10^6 \paren{\frac{5}{4}}^k.
\]
\end{solution}

\ppart Give a closed form numerical expression for the expected total
number of dollars in dividend payments you will receive by the end of
the 10th year.  You do not need to evaluate your expression.

\begin{center}
\exambox{2.0in}{0.5in}{-0.1in}
\end{center}

\examspace[1in]

\begin{solution}
\[
\sum_{k=1}^{10} \frac{1}{10}\cdot 10^6 \cdot \paren{\frac{5}{4}}^k =
10^5 \cdot \sum_{k=1}^{10} \paren{\frac{5}{4}}^k =
10^5 \cdot \frac{(5/4)^{11} - (5/4)}{5/4 - 1} = 5 \cdot 10^5 \paren{\paren{\frac{5}{4}}^{10} - 1}.
\]
\end{solution}

\iffalse
\ppart What is the expected market value of your investment at the end
of $k$~years if your annual payment is taken from your investment
instead of from STCM's pockets?

\begin{solution}
$10^6 \cdot (0.9\cdot5/4)^k = 10^6 \cdot (9/8)^k$.
\end{solution}
\fi

\ppart
Adam Smith does his own analysis of your account.  He lets $Y_i=1$ if
the money doubles at the end of year $i$ and $Y_i= -1$ otherwise.
Then the money in your account after year $k$ is
\[
10^6 2^{Y_1} 2^{Y_2} \cdots 2^{Y_k} = 10^6 2^{Y_1 + Y_2 + \cdots + Y_k}. 
\]
But $\expect{Y_i}=0$, so
\[
2^{\expect{Y_1 + Y_2 + \cdots + Y_k}}
   = 2^{\expect{Y_1} + \expect{Y_2} + \cdots + \expect{Y_k}}
   = 2^{k\cdot 0} = 2^0 = 1.
\]
In other words, the expected amount of money in your account forever
remains the same as your original investment.

What is wrong with Adam Smith's analysis?

\examspace[3.0in]

\begin{solution}
In the equation $2^{\expect{Y_1} + \expect{Y_2} + \cdots +
  \expect{Y_k}} = 2^{k\cdot 0}$, Adam Smith is incorrectly assuming
that
\begin{falseclm*}
\[
\expect{2^{Y_i}} = 2^{\expect{Y_i}}.
\]
\end{falseclm*}
But $2^{Y_i} = X_i$, so
\[
\expect{2^{Y_i}} = \expect{X_i} = \frac{5}{4} \neq 1 = 2^0 = 2^{\expect{Y_i}}.
\]
\end{solution}

\eparts

\end{problem}
