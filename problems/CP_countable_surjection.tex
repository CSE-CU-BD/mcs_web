\documentclass[problem]{mcs}

\begin{pcomments}
  \pcomment{CP_countable_surjection}
  \pcomment{ARM 12/23/11}
\end{pcomments}

\pkeywords{
  countable
  surjection
}

%%%%%%%%%%%%%%%%%%%%%%%%%%%%%%%%%%%%%%%%%%%%%%%%%%%%%%%%%%%%%%%%%%%%%
% Problem starts here
%%%%%%%%%%%%%%%%%%%%%%%%%%%%%%%%%%%%%%%%%%%%%%%%%%%%%%%%%%%%%%%%%%%%%

\begin{problem}
If there is a a surjective function ($[\leq 1 \text{ out}, \geq 1
  \text{ in}]$ mapping) $f:\naturals \to S$, then $S$ is countable.

\hint A Computer Science proof involves filtering for duplicates.

\begin{solution}
Think of $f(0), f(1), f(2), \dots$ as a stream and filter for
duplicates.  Namely, define $g:\naturals \to S$ recursively by the
rules:
\begin{align*}
g(0)   & \eqdef f(0)\\
g(n+1) & \eqdef f(k) & \text{where $k$ is minimum such that}\\
       &             &  f(k) \notin \set{g(0),g(1),\dots,g(n)}.
\end{align*}

If $S$ is infinite, then $g$ is a bijection.  If $S$ is finite, then
it is countable by definition.
\end{solution}

\end{problem}

%%%%%%%%%%%%%%%%%%%%%%%%%%%%%%%%%%%%%%%%%%%%%%%%%%%%%%%%%%%%%%%%%%%%%
% Problem ends here
%%%%%%%%%%%%%%%%%%%%%%%%%%%%%%%%%%%%%%%%%%%%%%%%%%%%%%%%%%%%%%%%%%%%%

\endinput
