\documentclass[problem]{mcs}

\begin{pcomments}
  \pcomment{FP_inbook_connected_induction}
  \pcomment{first part is FP_connected_induction}
  \pcomment{S17.final, S01.final}
  \pcomment{added, edited ARM 5/20/17}
\end{pcomments}

\pkeywords{
  connected
  induction
}

%%%%%%%%%%%%%%%%%%%%%%%%%%%%%%%%%%%%%%%%%%%%%%%%%%%%%%%%%%%%%%%%%%%%%
% Problem starts here
%%%%%%%%%%%%%%%%%%%%%%%%%%%%%%%%%%%%%%%%%%%%%%%%%%%%%%%%%%%%%%%%%%%%%

\begin{problem}
MIT Information Services \& Technology (IS\&T) wants to assemble a
cluster of $n$ computers using wires and hubs.  Each computer must
have exactly one wire attached to it, while each hub can have up to
five wires attached.  There must be a path of wires between every pair
of computers in the cluster.

\bparts
\ppart \label{suffhubs}
Prove by induction that $\ceil{(n-2)/3}$ hubs are sufficient for IS\&T
to assemble the cluster of $n$ computers.

\examspace[4.0in]

\begin{solution}

\begin{proof}
The proof will be by strong induction on $n$ with hypothesis
\[
P(n) \eqdef n \text{ computers can be clustered using } \ceil{\frac{n-2}{3}} \text{ degree-5 hubs}.
\]

\inductioncase{Base Cases}.
($n = 0,1,2$): Zero hubs are needed, and $0 = \ceil{(0-2)/3} = \ceil{(1-2)/3} = \ceil{(2-2)/3}$.

($n = 3$): The three computers can be connected to one hub, and $1 = \ceil{(3-2)/3}$.

\inductioncase{Induction step}.  We must prove $P(n+1)$ assuming $P(k)$ for $0 \leq
k \leq n$, where $n\geq 3$.  In particular, we may assume $P(n-2)$.  That is, there is a
cluster $C$ of $n-2\geq 1$ computers using at most
\[
\ceil{\frac{(n-2)-2}{3}} = \ceil{\frac{(n+1)-2}{3}}-1
\]
hubs.  If we can rearrange the cluster $C$ to include one more hub and
three more computers, we will have a cluster with $n+1$ computers and
$\ceil{((n+1)-2)/3}$ hubs, thereby proving $P(n+1)$.

To rearrange $C$ in this way, we replace a computer $R$ with a new hub
$H$.  This leaves room to attach four more wires to $H$, allowing us
to attach $R$ and three new computers to the ends of these wires.
\end{proof}

\end{solution}

\eparts
\bigskip
Suppose IS\&T uses $m$ hubs.

\bparts 

\ppart\label{ISTMmax} Write a simple formula in terms of $m$ and $n$
that bounds the maximum number of wires that can be attached to the
hubs and computers without leaving dangling ends.

\exambox{1.2in}{0.4in}{-0.2in}

\begin{solution}
\[
\frac{5m+n}{2}\, .
\]

The sum of the hub degrees is at most $5m$ and the sum of the computer
degrees is $n$.  The number of edges is half the degree sum.
\end{solution}

\ppart\label{wire-tree} Write a simple formula in terms of $m$ and $n$
for a lower bound on the number of wires needed to form the cluster,
even using hubs that may have more than five wires attached.

\exambox{1.2in}{0.4in}{-0.2in}

\begin{solution}
\[
m + n -1\, .
\]

This is the number of edges in a tree with the $n$ computers and $m$
hubs as vertices.
\end{solution}

\ppart IS\&T wants to minimize the number of hubs used to assemble the
cluster.  Use the previous results to prove that at $\ceil{(n-2)/3}$
hubs are necessary and sufficient to hook up the cluster.

\examspace[2.5in]

\begin{solution}
By part~\eqref{suffhubs}, $\ceil{(n-2)/3}$ hubs are sufficient.  Also,
parts~\eqref{ISTMmax} and~\eqref{wire-tree} imply that
\begin{equation}\label{constrain-hubs}
m+n-1 \leq (5m+n)/2.
\end{equation}

To prove that $\ceil{(n-2)/3}$ hubs are necessary, we try to minimize
$m$ subject to the constraint~\eqref{constrain-hubs}.  This will
happen when these two terms are equal:
\begin{align*}
m+n-1       & = (5m+n)/2,\\
n - 1 - n/2 & = 5m/2 -m,\\
n/2 - 1     & = 3m/2,\\
n-2         & = 3m,\\
\frac{n-2}{3} & = m.
\end{align*}
Since $m$ must be an integer, we conclude that
\[
m = \ceil{\frac{n-2}{3}}.
\]
\end{solution}

\eparts

\end{problem}

\endinput
