\documentclass[problem]{mcs}

\begin{pcomments}
  \pcomment{FP_recitation_coloring}
  \pcomment{variant of PS_TA_recitation_graph_coloring}
  \pcomment{from: S09.ps6, S06.ps4, S04.ps4}
\end{pcomments}

\pkeywords{
  graph_coloring
  scheduling
}

%%%%%%%%%%%%%%%%%%%%%%%%%%%%%%%%%%%%%%%%%%%%%%%%%%%%%%%%%%%%%%%%%%%%%
% Problem starts here
%%%%%%%%%%%%%%%%%%%%%%%%%%%%%%%%%%%%%%%%%%%%%%%%%%%%%%%%%%%%%%%%%%%%%

\begin{problem}
Next year, Math for Computer Science will be taught using recitations.
Six hour long recitations, each taught by a team of two or three TAs,
will be needed.  The planned assignment of TAs to recitation sections
is as follows:

\begin{itemize}
\item R1:  Percy, Thomas, Molly \\
\item R2:  Thomas, Gordon, Edward \\
\item R3:  Gordon, Emily \\
\item R4:  Emily, Edward \\
\item R5:  Edward, Molly, Emily \\
\item R6:  Molly, Percy, Thomas
\end{itemize}

Two recitations cannot be held in the same one-hour time slot if some
TA is assigned to teach both recitations.  The Registrar must
determine the minimum number of time slots required to complete all
the recitations.

\bparts

\ppart Recast the Registrar's problem of determining the minimum
number of time slots as a question about coloring the vertices of a
particular graph.  Draw the graph and explain what the vertices,
edges, and colors represent.

\examspace[2.5in]

\begin{solution}
Each vertex in the graph below represents a recitation
section.  An edge connects two vertices if the corresponding
recitation sections share a staff member and thus can not be scheduled
at the same time.  The color of a vertex indicates the time slot of
the corresponding recitation.

\begin{figure}[h]
\graphic[height=2in]{rec_sched_Thomas}
\end{figure}

\end{solution}

\ppart Give a coloring of this graph using the fewest possible colors,
and explain why no fewer colors are possible.  Describe a possible
recitation schedule using a minimum number of hour long time slots
based on this coloring.

\begin{solution}
Four colors are necessary and sufficient.   Four
are \emph{sufficient} as shown by the coloring:

\iffalse
\begin{figure}[h]
\graphic[height=2in]{ps3-schedule-colored}
\end{figure}
\fi

\begin{itemize}
\item red: R1, R3

\item white: R2

\item blue: R4, R6

\item green: R5
\end{itemize}
This corresponds having four recitation times, where recitations
assigned the same color run at the same time.  Other schedules are
also possible.

Four colors are \emph{necessary} because each of the four vertices R1,
R2, R5, and R6 are adjacent to other three, so they must all have
different colors.

Another way to say this is that the subgraph with these vertices is
isomorphic to $K_4$, which we know requires four colors.
\end{solution}

\eparts

\end{problem}

%%%%%%%%%%%%%%%%%%%%%%%%%%%%%%%%%%%%%%%%%%%%%%%%%%%%%%%%%%%%%%%%%%%%%
% Problem ends here
%%%%%%%%%%%%%%%%%%%%%%%%%%%%%%%%%%%%%%%%%%%%%%%%%%%%%%%%%%%%%%%%%%%%%

\endinput
