%PS_stable_matching_unlucky.tex

\documentclass[problem]{mcs}

\begin{pcomments}
  \pcomment{by Justin Zhang, for 2009 fall final}
\end{pcomments}

\pkeywords{
  logic
  proposition
  clause
}

%\providecommand{\boys}{\text{Boys}}
%\providecommand{\girls}{\text{Girls}}

%%%%%%%%%%%%%%%%%%%%%%%%%%%%%%%%%%%%%%%%%%%%%%%%%%%%%%%%%%%%%%%%%%%%%
% Problem starts here
n%%%%%%%%%%%%%%%%%%%%%%%%%%%%%%%%%%%%%%%%%%%%%%%%%%%%%%%%%%%%%%%%%%%%%


\begin{problem}
  ``In boolean logic, a formula is in \term{conjunctive normal form}
  (\term{CNF}) if it is a conjunction of clauses, where a clause is a
  disjunction of literals.'' ---Wikipedia (the following examples are
  also taken from Wikipedia) A literal is either a boolean variable or
  the negation of a boolean variable. In other words:
  \[
   CNF = clause \land clause \land ... \land clause
   \]
   \[
    clause = literal \lor iteral \lor ... \lor literal 
   \]
   \[
    literal = variable | \neg variable 
   \]

   For example, the following formulas are in CNF:

  \begin{align}
    (\neg A) \land (B \lor C)  \nonumber \\
    (A \lor B) \land (\neg B \lor C \lor D) \land (D \lor \neg E)
    \nonumber \\
    (A) \land (B)     \nonumber
  \end{align}

  The following formulas are not in CNF:

  \begin{align}
    \neg (B \lor C) \nonumber \\  
    (A \land B) \lor C \nonumber \\
    A \land (B \lor (D \land E))  \nonumber
  \end{align}

  However, they are respectively equivalent to the following formulas
  in CNF:

  \begin{align}
    (\neg B \land \neg C) \nonumber \\
    (A \lor C) \land (B \lor C) \nonumber \\
    (A) \land (B \lor D) \land (B \lor E) \nonumber
  \end{align}
  
\bparts
\ppart
Please tell which of the following are in CNF, where $x_i$ are boolean
variables:

  \begin{align}
    (\neg x_1 \land \neg x_2) \lor x_3 \nonumber \\
    (x_1 \lor x_2) \land (x_3 \lor x_4) \land (x_5 \lor x_6 \lor x_7)
    \nonumber \\
    (x_6) \land (\neg x_6 \lor x_7) \land \neg (\neg x_7 \land \neg x_6) \nonumber
  \end{align}

\begin{solution}
  not CNF;  CNF;  not CNF.
\end{solution}


\ppart
If there are any formulas in part (a) not in CNF, please choose one of
them and convert it into CNF using whatever ways you can (laws, truth
table, etc.)

\begin{solution}
  We use distributive law to convert the first formula into CNF and De
  Morgan's law to convert the third. 
  \begin{align}
    (\neg x_1 \land \neg x_2) \lor x_3  =  (\neg x_1 \lor x_3) \land
    (\neg x_2 \lor x_3)  \nonumber \\
    (x_6) \land (\neg x_6 \lor x_7) \land \neg (\neg x_7 \land \neg x_6) = 
    (x_6) \land (\neg x_6 \lor x_7) \land (x_7 \lor x_6) \nonumber
  \end{align}

\end{solution}

\eparts

\end{problem}


%%%%%%%%%%%%%%%%%%%%%%%%%%%%%%%%%%%%%%%%%%%%%%%%%%%%%%%%%%%%%%%%%%%%%
% Problem ends here
%%%%%%%%%%%%%%%%%%%%%%%%%%%%%%%%%%%%%%%%%%%%%%%%%%%%%%%%%%%%%%%%%%%%%


\endinput
