\documentclass[problem]{mcs}

\begin{pcomments}
  \pcomment{PS_predicate_calculus_power_of_prime}
  \pcomment{clash with PS_predicate_calculus_power_of_two}
  \pcomment{from: S06.ps1}
  \pcomment{edited way down from PS_express_predicates_in_formal_logic_notation}
\end{pcomments}

\pkeywords{
  predicates
  propositional_logic
  prime
  power_of
}

%%%%%%%%%%%%%%%%%%%%%%%%%%%%%%%%%%%%%%%%%%%%%%%%%%%%%%%%%%%%%%%%%%%%%
% Problem starts here
%%%%%%%%%%%%%%%%%%%%%%%%%%%%%%%%%%%%%%%%%%%%%%%%%%%%%%%%%%%%%%%%%%%%%

\newcommand{\isprime}{\text{\sc{is-prime}}}

\begin{problem}
  Express each of the following predicates and propositions in formal
  logic notation.  The domain of discourse is the nonnegative integers,
  $\naturals$.  Moreover, in addition to the propositional operators,
  variables and quantifiers, you may define predicates using addition,
  multiplication, and equality symbols, and nonnegative integer
  \emph{constants} \texttt{0}, \texttt{1},\dots), but no
  \emph{exponentiation} (like $x^y$).  For example, the predicate ``n is
  an even number'' could be defined by either of the following formulas:
\[
\exists m.\; (2m = n), \qquad \exists m.\; (m + m = n).
\]

\bparts

\ppart $m$ is a divisor of $n$.

\begin{solution}
\[
m \divides n 
\eqdef\quad 
\exists k.\; k \cdot m = n
\]
\end{solution}

\problempart 
$n$ is a prime number.

\begin{solution}
\[
\isprime(n)
\eqdef\quad 
(n\neq 1)\ \QAND\ \forall m.\; (m \divides n) \QIMPLIES (m=1 \QOR m=n).
\]
Note that $n\neq 1$ is an abbreviation of the formula $\QNOT(n=1)$.
\end{solution}

\ppart 
$n$ is a power of a prime.

\begin{solution}
  We can say that there is a prime, $p$, such that every divisor of
  $n$ not equal 1 to is itself divisible by $p$:
\[
\exists p.\,[\isprime(p) \QAND \forall m.\; (m \divides n \QAND m \neq 1)
\QIMPLIES p \divides m].
\]

Alternatively, we could say that at most one prime that divides $n$:
\[
\forall p,q.\, (\isprime(p) \QAND \isprime(q) \QAND p \divides n \QAND q
\divides n) \QIMPLIES p = q.
\]
\end{solution}

\eparts
\end{problem}

%%%%%%%%%%%%%%%%%%%%%%%%%%%%%%%%%%%%%%%%%%%%%%%%%%%%%%%%%%%%%%%%%%%%%
% Problem ends here
%%%%%%%%%%%%%%%%%%%%%%%%%%%%%%%%%%%%%%%%%%%%%%%%%%%%%%%%%%%%%%%%%%%%%
