\documentclass[problem]{mcs}

\begin{pcomments}
  \pcomment{PS_proper_subset_partial_order}
  \pcomment{from: S06.ps2}
\end{pcomments}

\pkeywords{
   %I don't know the format for these keywords, how do I look them up?
	proper_subset
	partial_order
}

%%%%%%%%%%%%%%%%%%%%%%%%%%%%%%%%%%%%%%%%%%%%%%%%%%%%%%%%%%%%%%%%%%%%%
% Problem starts here
%%%%%%%%%%%%%%%%%%%%%%%%%%%%%%%%%%%%%%%%%%%%%%%%%%%%%%%%%%%%%%%%%%%%%

\begin{problem}
Consider the proper subset partial order, $\subset$, on the power set
$\power{\set{1,2,\dots 5}}$.

\bparts
\ppart What is the size of a maximal chain in this partial order?
Describe one.
\begin{solution}
Size 6, for example,
\[
\set{\emptyset, \set{1}, \set{1,2}, \set{1,2,3},\set{1,2,3,4},\set{1,2,3,4,5}}.
\]
\end{solution}

\ppart Describe the largest antichain you can find in this partial order.

\begin{solution}
All the size 3 subsets of $\set{1,2,\dots 5}$ form an antichain of size 10.
This is actually the largest, though proving this is a challenge.
\end{solution}

\ppart  What are the maximal and minimal elements?  Are they maximum and
minimum?

\begin{solution}
$\emptyset$ is minimum and $\set{1,2,\dots 5}$ is maximum.
\end{solution}

\ppart Answer the previous part for the $\subset$ partial order on the set
$\power{\set{1,2,\dots 5}} - \emptyset$.

\begin{solution}
Now the five size 1 subsets are minimal and there is no minimum.
$\set{1,2,\dots 5}$ is still maximum.
\end{solution}

\eparts
\end{problem}

%%%%%%%%%%%%%%%%%%%%%%%%%%%%%%%%%%%%%%%%%%%%%%%%%%%%%%%%%%%%%%%%%%%%%
% Problem ends here
%%%%%%%%%%%%%%%%%%%%%%%%%%%%%%%%%%%%%%%%%%%%%%%%%%%%%%%%%%%%%%%%%%%%%

\endinput
