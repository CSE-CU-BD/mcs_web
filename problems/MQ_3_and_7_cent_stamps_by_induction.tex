\documentclass[problem]{mcs}

\begin{pcomments}
  \pcomment{MQ_3_and_7_cent_stamps_by_induction}
  \pcomment{from: F09.mq3}
  \pcomment{similar to CP_10_and_15_cent_stamps_by_WOP but slightly
    different and uses induction}
\end{pcomments}

\pkeywords{
  induction
  strong_induction
}

%%%%%%%%%%%%%%%%%%%%%%%%%%%%%%%%%%%%%%%%%%%%%%%%%%%%%%%%%%%%%%%%%%%%%
% Problem starts here
%%%%%%%%%%%%%%%%%%%%%%%%%%%%%%%%%%%%%%%%%%%%%%%%%%%%%%%%%%%%%%%%%%%%%

\begin{problem}
  Prove \emph{by induction} that every exact amount of postage of $12$
  cents or more can be formed using 3 and 7 cent stamps.
  
\begin{solution}
  
  \begin{proof}
    The following proof is by strong induction on $n$ with induction
    hypothesis
    \[
    S(n) \eqdef \text{exactly $n$ cents postage can be formed using 3 and
      7 cent stamps}.
\]

    \textbf{Base cases:} $S(12)$, $S(13)$ and $S(14)$ are shown to hold
    by explicit calculations:
    \begin{align*}
\iffalse
    3 & = 3 \\
    6 & = 3 + 3 \\
    7 & = 7 \\
    9 & = 3 + 3 + 3 \\
    10 & = 3 + 7 \\
\fi
    12 &= 3+3+3,\\
    13 &= 3+3+7, \\
    14 &= 7+7.
    \end{align*}

    \textbf{Inductive step:} By strong induction, we may assume that
    $S(k)$ holds for $12 \leq k \leq n$ and must then prove that $S(n+1)$
    is true.

    Now if $n+1 \leq 14$, then $S(n+1)$ follows from the base case.  On
    the other hand, if $n+1 > 14$, then $n-2 \geq 12$, so $S(n-2)$ is true
    by induction hypothesis.  So by adding one 3 cent stamp to the stamps
    that form $n-2$ cents postage, we will have postage equal to $n - 2 +
    3 = n + 1$ cents, showing that $S(n+1)$ is true.

    It follows by strong induction that $P(n)$ holds for all $n \geq 14$.
  \end {proof}

\end{solution}

\end{problem}

%%%%%%%%%%%%%%%%%%%%%%%%%%%%%%%%%%%%%%%%%%%%%%%%%%%%%%%%%%%%%%%%%%%%%
% Problem ends here
%%%%%%%%%%%%%%%%%%%%%%%%%%%%%%%%%%%%%%%%%%%%%%%%%%%%%%%%%%%%%%%%%%%%%

\endinput
