\documentclass[problem]{mcs}

\begin{pcomments}
  \pcomment{FP_recstrings}
  \pcomment{Emanuele and ARM 3/12/16}
\end{pcomments}

\pkeywords{
   recursive
   string
   reverse
   concatenation
}

\newcommand{\catOK}{\text{cat-OK}}
%\newcommand{\revstr}{\text{rev}}
\newcommand{\Zerostr}{\text{Zeroes}}


%%%%%%%%%%%%%%%%%%%%%%%%%%%%%%%%%%%%%%%%%%%%%%%%%%%%%%%%%%%%%%%%%%%%%
% Problem starts here
%%%%%%%%%%%%%%%%%%%%%%%%%%%%%%%%%%%%%%%%%%%%%%%%%%%%%%%%%%%%%%%%%%%%%

\begin{problem}
This problem is about binary strings $s \in \finbin$.

Let's call a recursive definition of a set of strings \catOK\ when all
its constructors are defined as concatenations of
strings.\footnote{The concatenation of two strings $x$ and $y$,
  written $xy$, is the string obtained by appending $x$ to the left 
  end of $y$.  For example, the concatenation of $01$ and $101$ is
  $01101$.}

For example, the set, One1, of strings with exactly one \STR{1} has
the \catOK\ definition:

\inductioncase{Base case}: The length-one string \STR{1}\ is in One1.

\inductioncase{Constructor case}: If $s$ is in One1, then so is $\STR{0}s$ and $s\STR{0}$.

\iffalse
\begin{definition}
The set \finbin\ of all binary strings and their lengths has the
following basic \catOK\ definition:

\inductioncase{Base case}: $\emptystring \in \finbin$, and
$\lnth{\emptystring)} \eqdef 0$.

\inductioncase{Constructor case}: If $s \in \finbin$, then so is $sb$
where $b \in \set{0,1}$, and $\lnth{sb} \eqdef \lnth{s}+1$.
\end{definition}

Now we can recursively define the \emph{reversal} function, rev, on strings
\begin{definition}
The function $\rev: \finbin \to \finbin$ is defined recursively as follows:

\inductioncase{Base case}:  $\rev{\emptystring} \eqdef \emptystring.$

\inductioncase{Constructor case}: $\rev{sb} \eqdef b\rev{s}$ for
$s\in \finbin$ and $b\in \set{0,1}$.
\end{definition}
Finally, Here is a \catOK\ definition of the set of binary \emph{palindromes}.
\fi

\bparts

\ppart Give a \catOK\ definition of the set $E$ of even length strings of \texttt{0}'s.

\examspace[1.0in]

\begin{solution}
\begin{definition}
\inductioncase{Base cases}: $\emptystring\in E$.

\inductioncase{Constructor case}: If $s \in E$, then $s00 \in E$.
(NOTE: $\texttt{00}s \in E,\ \text{or}\ \texttt{0}s\texttt{0} \in E$
also work.)
\end{definition}
\end{solution}

\ppart Let $\rev{s}$ be the reversal of the string $s$.  For example,
$\rev{\STR{001}} = \STR{100}$.  A \emph{palindrome} is a
string $s$ such that $s = \rev{s}$.  For example, \STR{11011} and
\STR{010010} are palindromes.

Give a \catOK\ definition of the \emph{palindromes}.

\examspace[1.0in]

\begin{solution}
\begin{definition}
\inductioncase{Base cases}: $\emptystring$, \texttt{0}, and \texttt{1} are palindromes.

\inductioncase{Constructor case}: If $s$ is a palindrome, then so is
$bsb$ where $b \in \set{\texttt{0},\texttt{1}}$.
\end{definition}
\end{solution}

\ppart Give a \catOK\ definition of the set $P$ of strings whose length is a power of two.

\begin{solution}
\begin{definition}
\inductioncase{Base cases}: \texttt{0}, \texttt{1} are in $P$.

\inductioncase{Constructor case}: If $s \in P$, then so is $ss$.
\end{definition}
\end{solution}

\eparts
\end{problem}

\endinput
