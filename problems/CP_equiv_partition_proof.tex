\documentclass[problem]{mcs}

\begin{pcomments}
  \pcomment{CP_equiv_partition_proof}
  \pcomment{ARM 10/17/11}
\end{pcomments}

\pkeywords{
  equivalence_relation
  partition
  block
}

%%%%%%%%%%%%%%%%%%%%%%%%%%%%%%%%%%%%%%%%%%%%%%%%%%%%%%%%%%%%%%%%%%%%%
% Problem starts here
%%%%%%%%%%%%%%%%%%%%%%%%%%%%%%%%%%%%%%%%%%%%%%%%%%%%%%%%%%%%%%%%%%%%%
\begin{problem}
\inbook{Prove Theorem~\bref{equiv-partition_thm}:} The
\idx{equivalence classes} of an \idx{equivalence relation} form a
\idx{partition} of the domain.

Namely, let $R$ be an equivalence relation on a set, $A$, and define
the \idx{equivalence class} of an element $a \in A$ to be
\[
[a]_R \eqdef \set{b \in A \suchthat a\mrel{R}b}.
\]

\bparts

\ppart\label{exhaustA} Prove that every block is nonempty and every element of $A$ is
in some block.

\begin{solution}
\begin{proof}
Since $R$ is reflexive, any element $a \in A$ is a member of the block
$[a]_R$ and also each block $[a]_R$ is nonempty.
\end{proof}
\end{solution}

\ppart\label{nonempt_impaRb} Prove that if $[a]_R \intersect [b]_R
\neq \emptyset$, then $a\mrel{R}b$.  Conclude that the sets $[a]_R$
for $a \in A$ are a partition of $A$.

\begin{solution}
\begin{proof}
Suppose $c \in [a]_R \intersect [b]_R$, that is, $a\mrel{R}c$ and
$b\mrel{R}c$.  Since $b\mrel{R}c$, we have $c\mrel{R}b$ by symmetry.
Since $a\mrel{R}c$, we now conclude by transitivity that $a\mrel{R}b$.
\end{proof}

This and part~\eqref{exhaustA} mean that the sets $[a]_R$ partition
$A$, by the definition of partition.
\end{solution}

\ppart Prove that $a\mrel{R}b$ iff $[a]_R = [b]_R$.

\begin{solution}

\begin{proof}
Now if $x \in [b]_R$, then $b \mrel{R} x$, and since $a\mrel{R}b$ by
part~\eqref{nonempt_impaRb}, we again invoke transitivity to conclude
that $a \mrel{R} x$.  That is, $x \in [a]_R$.  So every element of
$[b]_R$ is in $[a]_R$, which means that $[b]_R \subseteq [a]_R$.

Reversing the roles of $a$ and $b$, it follows that $[a]_R \subseteq
[b]_R$, so in fact $[a]_R = [b]_R$.
\end{proof}
\end{solution}

\eparts

\end{problem}
%%%%%%%%%%%%%%%%%%%%%%%%%%%%%%%%%%%%%%%%%%%%%%%%%%%%%%%%%%%%%%%%%%%%%
% Problem ends here
%%%%%%%%%%%%%%%%%%%%%%%%%%%%%%%%%%%%%%%%%%%%%%%%%%%%%%%%%%%%%%%%%%%%%

\endinput

