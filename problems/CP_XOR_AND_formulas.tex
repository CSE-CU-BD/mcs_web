\documentclass[problem]{mcs}

\begin{pcomments}
  \pcomment{CP_XOR_AND_formulas}
  \pcomment{S16, midterm1}
  \pcomment{ARM 2/24/16}
  \pcomment{was CP_TBA2}
\end{pcomments}

\pkeywords{
  proposition
  XOR
  equivalence
  constant
  well_order
  expression
}

\begin{problem}
Let $P$ be a propositional variable.
\bparts

\ppart\label{PQXT} Show how to express $\QNOT(P)$ using $P$ and a
selection from among the constant $\True$, and the connectives $\QXOR$
and $\QAND$.
\begin{staffnotes}
Rubric: (a) 6pts (b) 6pts (c) 13pts.

for (c) Statements that $P \QXOR P$ equiv to $\False$ and $P \QAND P$
equiv $P$ w/o explanation of how these equivalences explain the
conclusion gets -5pts = 8 of 13.  We're looking for some reference to
WOP as in the soln or some other proper explanation.
\end{staffnotes}

\begin{solution}
\[
\QNOT(P) \quad\equiv\quad P \QXOR \True.
\]
\end{solution}

\examspace[0.6in]

\ppart\label{XATOK} Explain why part~\eqref{PQXT} implies that every
propositional formula is equivalent to one whose only connectives are
$\QXOR$ and $\QAND$, along with the constant $\True$.

\begin{solution}
We know that every propositional formula is equivalent to one using
only $\QNOT$, $\QAND$ and $\QOR$.  Moreover by DeMorgan's Law
(Section~\bref{propositional_equivalences_sec}) $\QOR$ is expressible
in terms of $\QNOT$ and $\QAND$.  So by expressing $\QNOT$ according
to part~\eqref{PQXT}, every formula is equivalent to one that can be
expressed using only $\QXOR$, $\QAND$ and $\True$.
\end{solution}

\examspace[0.75in]

\ppart The constant $\True$ is essential for part~\eqref{XATOK}.  This
follows because every propositional formula using only $P$, the
connectives $\QXOR$ and $\QAND$, and no constants---call this
a ``PXA-formula''---is equivalent to $P$ or to $\False$.  Prove this
claim.

\hint Use WOP and look at the shortest PXA-formula that might not be
equivalent to $P$ or $\False$.

\begin{solution}
Suppose there is a PXA-formula not equivalent to $P$ or to $\False$.
By WOP, there will be a shortest such formula $F$.

Now $F$ cannot consist of just the propositional variable $P$, since
$P$ is equivalent to $P$.  Therefore, $F$ must be of the form ``$G
\QXOR H$'' or ``$G \QAND H$'' for some PXA-formulas $G$ and $H$.  But
since $G$ and $H$ are shorter than $F$, they must each be equivalent
to $P$ or to $\False$.  This leads to the contradiction that $F$ is
equivalent to $P$ or to $\False$, since $X \QAND Y$ and $X \QXOR Y$
are equivalent to $P$ or to $\False$ when $X,Y$ take the values $P$
and/or $\False$.
\end{solution}

\eparts
\end{problem}

\endinput
