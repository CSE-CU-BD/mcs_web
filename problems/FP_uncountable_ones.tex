\documentclass[problem]{mcs}

\begin{pcomments}
  \pcomment{FP_uncountable_ones}
  \pcomment{ARM May 19, 2013}
\end{pcomments}

\pkeywords{
  diagonal
  uncountable
  surjection
  infinite
  string
}

%%%%%%%%%%%%%%%%%%%%%%%%%%%%%%%%%%%%%%%%%%%%%%%%%%%%%%%%%%%%%%%%%%%%%
% Problem starts here
%%%%%%%%%%%%%%%%%%%%%%%%%%%%%%%%%%%%%%%%%%%%%%%%%%%%%%%%%%%%%%%%%%%%%
\begin{problem}
Let $\binw$ be the set of infinite binary strings, and let $B \subset
\binw$ be the set of infinite binary strings containing infinitely
many occurrences of 1's.  Prove that $B$ is uncountable.  (We have
already shown that $\binw$ is uncountable.)

\hint Define a suitable function from $\binw$ to $B$.

\begin{solution}
Since $\binw$ is uncountable, it is sufficient to prove that $B \surj
\binw$, which is equivalent to $\binw \inj B$.  That is, it's sufficient to
define a total injective function $f:\binw \to B$.

An easy way to define such an $f$ is to define, for any infinite
string $b \in \binw$, the value of $f(b)$ to be the string obtained by
inserting a 1 between each of the bits in $b$.

Alternatively, uncountability of $B$ follows from the easily verified
fact (see Problem~\bref{FP_infinite_binary_sequences}) that the set
$\bar{B}$ of infinite binary strings with only \emph{finitely many
  ones} is countable.  So if $B$ was also countable, then $\binw = B
\union \bar{B}$ would be countable, a contradiction.
\end{solution}

\end{problem}

\endinput
