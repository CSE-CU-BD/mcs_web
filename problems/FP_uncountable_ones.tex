\documentclass[problem]{mcs}

\begin{pcomments}
  \pcomment{FP_uncountable_ones}
  \pcomment{ARM May 19, 2013; hint & soln edited, 3/29/17}
\end{pcomments}

\pkeywords{
  uncountable
  surjection
  infinite
  string
}

%%%%%%%%%%%%%%%%%%%%%%%%%%%%%%%%%%%%%%%%%%%%%%%%%%%%%%%%%%%%%%%%%%%%%
% Problem starts here
%%%%%%%%%%%%%%%%%%%%%%%%%%%%%%%%%%%%%%%%%%%%%%%%%%%%%%%%%%%%%%%%%%%%%
\begin{problem}
Let $\binw$ be the set of infinite binary strings, and let $B \subset
\binw$ be the set of infinite binary strings containing infinitely
many occurrences of 1's.  Prove that $B$ is uncountable.  (We have
already shown that $\binw$ is uncountable.)

\hint Start by showing that $\binw \inj B$.

\begin{solution}
We are given that $\binw$ is uncountable (Corollary to Cantor's
Theorem~\bref{powbig}).  By Corollary~\bref{AsurjUA}~\bref{AsurjUA},
to show that $B$ is uncountable, it is sufficient to show that $B
\surj \binw$, which is equivalent to $\binw \inj B$
(Lemma~\bref{surjinjbij_properties}~\bref{surjvsinj}.  That is, it's
sufficient to define a total injective function $f:\binw \to B$.

An easy way is to define, for any $b \in \binw$,
the value of $f(b)$ to be the string obtained by inserting a 1 between
each of the bits in $b$.

Another way is to let $f(b)$ to be $0b$ if $b$ has infinitely many
1's, and otherwise to be $1p01^{\omega}$ where $p$ is the shortest
prefix of $b$ that includes all the 1's in $b$.

\begin{staffnotes}
Alternatively, uncountability of $B$ follows from the easily verified
fact (see Problem~\bref{FP_infinite_binary_sequences_S14}) that the set
$\bar{B}$ of infinite binary strings with only \emph{finitely many
  ones} is countable.  So if $B$ was also countable, then $\binw = B
\union \bar{B}$ would be countable, a contradiction.
\end{staffnotes}

\end{solution}

\end{problem}

\endinput
