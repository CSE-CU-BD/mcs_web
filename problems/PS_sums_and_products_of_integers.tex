!\documentclass[problem]{mcs}

\begin{pcomments}
  \pcomment{PS_sums_and_products_of_integers}
  \pcomment{from: S09.ps4}
  \pcomment{from: F08.ps4}
  \pcomment{from: S04.ps2}
\end{pcomments}

\pkeywords{
  induction
  strong_induction
  ordinary_induction
  inequality
  sum
  product
}

%%%%%%%%%%%%%%%%%%%%%%%%%%%%%%%%%%%%%%%%%%%%%%%%%%%%%%%%%%%%%%%%%%%%%
% Problem starts here
%%%%%%%%%%%%%%%%%%%%%%%%%%%%%%%%%%%%%%%%%%%%%%%%%%%%%%%%%%%%%%%%%%%%%

\begin{problem}
\begin{claim}
If a collection of positive integers (not necessarily distinct) has
sum $n \geq 1$, then the collection has product at most $3^{n/3}$.
\end{claim}

For example, the collection 2, 2, 3, 4, 4, 7 has the sum:

\begin{eqnarray*}
2 + 2 + 3 + 4 + 4 + 7 & = & 22
\end{eqnarray*}

On the other hand, the product is:

\begin{eqnarray*}
2 \cdot 2 \cdot 3 \cdot 4 \cdot 4 \cdot 7
    & = & 1344 \\
    & \leq & 3^{22/3} \\
    & \approx & 3154.2
\end{eqnarray*}

\bparts

\ppart Use strong induction to prove that $n \leq 3^{n/3}$ for every
integer $n \geq 0$.

\begin{solution}
The proof is by strong induction.  Let $P(n)$ be the
proposition that $n \leq 3^{n/3}$.  First, we show that $P(0)$,
$P(1)$, $P(2)$, $P(3)$, and $P(4)$ are true:

\begin{eqnarray*}
0^3 \leq 3^0 & \rightarrow & 0 \leq 3^{0/3} \\
1^3 \leq 3^1 & \rightarrow & 1 \leq 3^{1/3} \\
2^3 \leq 3^2 & \rightarrow & 2 \leq 3^{2/3} \\
3^3 \leq 3^3 & \rightarrow & 3 \leq 3^{3/3} \\
4^3 \leq 3^4 & \rightarrow & 4 \leq 3^{4/3} \\
\end{eqnarray*}

Each implication follows by taking cube roots.  Next, we show that
$P(0), \ldots, P(n)$ imply $P(n + 1)$ for all $n \geq 4$.  Thus, we
assume that $P(0), \ldots, P(n)$ are all true and reason as follows:

\begin{eqnarray*}
3^{(n + 1)/3}
    & = & 3 \cdot 3^{(n-2)/3} \\
    & \geq & 3 \cdot (n - 2) \\
    & \geq & n + 1 \hspace{1in} \text{(for all $n \geq 7/2$)}
\end{eqnarray*}

The first step is algebra.  The second step uses our assumption $P(n -
2)$.  The third step is a linear inequality that holds for all $n \geq
7/2$.  (This forced us to deal individually with the cases $P(3)$ and
$P(4)$, above.) Therefore, $P(n + 1)$ is true, and so $P(n)$ is true
for all $n \geq 0$ by induction.
\end{solution}

\ppart Prove the claim using induction or strong induction.  (You may find
it easier to use induction on the \emph{number of positive integers in the
  collection} rather than induction on the sum $n$.)

\begin{solution}
We use induction on the size of the collection.  Let $P(k)$
be the proposition that every collection of $k$ positive integers with
sum $n$ has product at most $3^{n/3}$.  First, note that $P(1)$ is
true by the preceding problem part.

Next, we must show that $P(k)$ implies $P(k + 1)$ for all $k \geq 1$.
So assume that $P(k)$ is true, and let $x_1, \ldots, x_{k+1}$ be a
collection of positive integers with sum $n$.  Then we can reason as
follows:

\begin{eqnarray*}
x_1 \cdot x_2 \cdots x_k \cdot x_{k+1}
    & \leq & 3^{(n - x_{k+1}) / 3} \cdot x_{k+1} \\
    & \leq & 3^{(n - x_{k+1}) / 3} \cdot 3^{x_{k+1} / 3} \\
    & = & 3^{n / 3}
\end{eqnarray*}

The first step uses the assumption $P(k)$, the second uses the
preceding problem part, and the last step is algebra.  This shows that
$P(k + 1)$ is true, and so the claim holds by induction.

\end{solution}

\eparts

\end{problem}

%%%%%%%%%%%%%%%%%%%%%%%%%%%%%%%%%%%%%%%%%%%%%%%%%%%%%%%%%%%%%%%%%%%%%
% Problem ends here
%%%%%%%%%%%%%%%%%%%%%%%%%%%%%%%%%%%%%%%%%%%%%%%%%%%%%%%%%%%%%%%%%%%%%

\endinput
