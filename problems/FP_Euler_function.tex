\documentclass[problem]{mcs}

\begin{pcomments}
  \pcomment{FP_Euler_function}
  \pcomment{ARM 3/15/15}
\end{pcomments}

\pkeywords{
  Euler_function
  phi
  number_theory
  prime
}

%%%%%%%%%%%%%%%%%%%%%%%%%%%%%%%%%%%%%%%%%%%%%%%%%%%%%%%%%%%%%%%%%%%%%
% Problem starts here
%%%%%%%%%%%%%%%%%%%%%%%%%%%%%%%%%%%%%%%%%%%%%%%%%%%%%%%%%%%%%%%%%%%%%

\begin{problem}
Let $\phi$ be Euler's function.

\bparts

\ppart What is the value of $\phi(2)$? \hfill\examspace[0.5in]

\begin{solution}
$\phi(2) = 2^1 - 2^0 = 1$.
\end{solution}

\ppart What are three nonnegative integers $k>1$ such that $\phi(k) = 2$?  \hfill\examspace[0.75in]

\begin{solution}
$k= 3,4,6$.
\end{solution}

\ppart Prove that $\phi(k)$ is even for $k>2$.

\hint Consider whether $k$ has an odd prime factor or not.
\examspace[2.5in]

\begin{solution}
$\phi(k)$ has $\phi(p^m)=p^m-p^{m-1}$ as a factor for some prime $p$
  and some $m \geq 1$.  But $p-1$ is a factor of $p^m-p^{m-1}$, and
  $p-1$ is even unless $p=2$.

  If $p=2$, then since $k >2$, the exponent $m$ must be greater than
  one, and then $2$ is a factor of $2^m-2^{m-1}$, so again $\phi(k)$ is even.
\end{solution}

\inbook{
\ppart
Briefly explain why $\phi(k) = 2$ for exactly three values of $k$.

\examspace[2.0in]

\begin{solution}
  Assume $k\neq 3,4,6$.  We must show that $\phi(k) \neq 2$.  If $k$
  has a prime factor $p>3$, then $\phi(k)$ has a factor $p-1 >2$ and
  in particular, $\phi(k) \neq 2$.  Otherwise $k = 2^m3^n$, and there
  are just a few easy cases to examine:
\begin{itemize}

\item If $m>2$, then $\phi(k)$ is divisible by $2^m-2^{m-1} =
  2^2(2^{m-2}-2^{m-3})$, so $\phi(k)$ has a factor $4>2$.

\item If $n>1$, then $\phi(k)$ is divisible by $3^m-3^{m-1} =
  3(3^{n-1}-3^{n-2})$, so $\phi(k)$ has a factor $3>2$.  factor.

\item Otherwise, $m=0,1,2$ and $n=0,1$ are the only possibilities, and
  the only one where $k \neq 1,2,3,4,6$ is $k=12$, and $\phi(12) = 4
  \neq 2$.

\end{itemize}

\end{solution}
}

\eparts

\end{problem}


%%%%%%%%%%%%%%%%%%%%%%%%%%%%%%%%%%%%%%%%%%%%%%%%%%%%%%%%%%%%%%%%%%%%%
% Problem ends here
%%%%%%%%%%%%%%%%%%%%%%%%%%%%%%%%%%%%%%%%%%%%%%%%%%%%%%%%%%%%%%%%%%%%%


\endinput
