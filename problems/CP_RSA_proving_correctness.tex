\documentclass[problem]{mcs}

\begin{pcomments}
  \pcomment{CP_RSA_proving_correctness}
  \pcomment{DRAFT: being revised for Zmodn, maybe becomes a TP---ARM 3/27/13}
  \pcomment{from: S09.cp8r}
\end{pcomments}

\pkeywords{
  number_theory
  modular_arithmetic
  primes
  rsa
}

%%%%%%%%%%%%%%%%%%%%%%%%%%%%%%%%%%%%%%%%%%%%%%%%%%%%%%%%%%%%%%%%%%%%%
% problem starts here
%%%%%%%%%%%%%%%%%%%%%%%%%%%%%%%%%%%%%%%%%%%%%%%%%%%%%%%%%%%%%%%%%%%%%

\begin{problem}
  A critical fact about \idx{RSA} is, of course, that decrypting an
  encrypted message, $\hat{m}$, always gives back the original message,
  $m$.  Namely, if $n=pq$ where $p$ and $q$ are distinct primes, $m
  \in [0,pq)$, and
\[
d\cdot e \equiv 1 \pmod{(p-1)(q-1)},
\]
then
\begin{equation}\label{mdepq=m}
\paren{m^d}^e = m \quad \Zmod{n}.
%\rem{\paren{\rem{m^d}{n}}^e}{n} = m.
\end{equation}

\iffalse
We'll now prove this.

\bparts

\iffalse

This will follow from something simpler and more general:

\begin{lemma*}%\label{npdp}
If $n$ is a product of distinct primes and $a \equiv 1
\pmod{\phi(n)}$, then $m^a = m \quad (\Zmod{n})$.
%$m^a \equiv m \pmod {n}$.
\end{lemma*}

\ppart Explain why the Lemma implies that $k$ and $k^5$ have the
same last digit.  For example:
%
\[
\underline{2}^5 = 3\underline{2}

\hspace{1in}
7\underline{9}^5 = 307705639\underline{9}
\]
\hint What is $\phi(10)$?

\begin{solution}
Two nonnegative integers have the same last digit iff they are $\equiv
\pmod {10}$.  Now $\phi(10) = \phi(2)\phi(5) = 4$ and $5 \equiv 1
\pmod {4}$, so by the Lemma,
\[
k^5 \equiv k \pmod {10}.
\]
\end{solution}
\fi

\ppart\label{cong->decode} Verify that
\iffalse for all $d,e \in \naturals$, \fi
if
\begin{equation}\label{mdeqvmpq}
\paren{m^{d}}^e = m \quad (\Zmod{n}),
%\paren{m^{d}}^e \equiv m \pmod {n},
\end{equation}
then~\eqref{mdepq=m} is true.

\iffalse
\begin{staffnotes}
  If a team is stuck, have them first prove that~\eqref{mdeqvmpq}
  implies that~\eqref{mdepq=m} holds as a congruence mod $n$, that
  is~\eqref{mdeqvmpq} implies
\[
\rem{\rem{m^d}{n}^e}{n} \equiv m \pmod {n}.
\]
\end{staffnotes}
\fi

\begin{solution}
If equation~\eqref{mdeqvmpq} is true, then replacing $m^d$ by its
remainder modulo $n$ preserves the congruence,\footnote{See
  Corollary~\bref{aran} of the Remainder Lemma~\bref{lem:conrem}.}
giving
\[
\rem{m^d}{n}^e \equiv m \pmod {n}.
\]
Now replacing the left hand side by its remainder yields
\[
\rem{\rem{m^d}{n}^e}{n} \equiv m \pmod {n}.
\]
Since both the left and right hand sides of the previous congruence are in
$[0,n)$, they must be equal, which proves~\eqref{mdepq=m}.
\end{solution}

\begin{staffnotes}
Someone on your team may notice that~\eqref{mdeqvmpq} follows by by
multiplying both sides of the congruence by $m$ in Euler's theorem:
\[
m^{\phi(n)} \equiv 1 \pmod{n},
\]
so
\[
\paren{m^d}^e = m^{de} = m^{1+ c\cdot \phi(n)} \equiv m \pmod{n},
\]
since $de = 1+ c\cdot \phi(n)$ for some $c \in \naturals$.  However,
\emph{Euler's theorem only works for $m$ relatively prime to $n$.}
All the rest of this problem is about removing this restriction on $m$
and showing that it works for all $m \in [0,n)$.

Note the we're just being pure mathematicians here: we're just trying
to get rid of the relative primality condition, which turns out to be
theoretically unnecessary.  But the whole business of RSA is
predicated on the difficulty of factoring.  If an $m$ that wasn't
relatively prime to $n$ ever came up, we could factor $n$ by computing
$\gcd(m,n)$.  So if we believe in the security of RSA, we must believe
that the probability of a message $m$ that was not relatively prime to
$n$ is negligible, so there is no \emph{practical} reason to worry
about $m$'s that are not relatively prime to $n$.
\end{staffnotes}

\ppart\label{pma} Prove that if $p$ is prime, then $m^a \equiv m \pmod
            {p}$ for all $a \in \naturals$ congruent to 1 mod $p-1$.

\begin{solution}
If $p \divides m$, then the congruence holds since both sides are
$\equiv 0 \pmod {p}$.  So assume $p$ does not divide $m$.  Now if $a
\equiv 1 \pmod{p-1}$, then $a = 1 + (p-1)k$ for some $k$, so
\begin{align*}
m^a & = m^{1+ (p-1)k}\\
    & = m\cdot \paren{m^{p-1}}^k\\
    & \equiv m\cdot (1)^k \pmod {p}
            & \text{(by Fermat's Little Thm.)}\\
    & \equiv m \pmod {p}.
\end{align*}
\end{solution}

\iffalse

\ppart\label{abk} Show that for any positive integers $j,k$,
if $a \equiv b \pmod k$ and $j \divides k$, then $a \equiv b \pmod j$

\begin{solution}

$a \equiv b \pmod k$ iff $k \divides (a-b)$.  But if $k \divides (a-b)$
and $j \divides k$, then also $j \divides (a-b)$, which implies $a \equiv
b \pmod j$.

\end{solution}
\fi

\ppart\label{abp} Prove that if $a \equiv b \pmod {p_i}$ for distinct
primes $p_1,p_2,\dots,p_n$, then $a \equiv b \pmod{p_1p_1\cdots
p_n}$.

\begin{solution}
By definition of congruence, $a \equiv b \pmod {k}$ iff $k \divides
(a-b)$.  So if $a \equiv b \pmod {p_i}$ for each $p_i$, then $p_i
\divides (a-b)$ for each $p_i$.  By the Unique Factorization
Theorem~\bref{thm:unique_factor}, the product of the $p_i$'s must also
divide $(a-b)$, which means that $a \equiv b \pmod {p_1p_1\cdots p_n}$.
\end{solution}

\iffalse

\ppart\label{phip} Verify that for any $n>1$ and any prime divisor, $p$,
of $n$,
\[
\phi(p) \divides \phi(n).
\]

\begin{solution}
Let $p$ be a prime factor of $n$ and factor $n$ as $m\cdot p^k$
where $p$ does not divide $m$.  By the Euler function equations,
$\phi(n)=\phi(m)\phi(p^k)$, so $\phi(p^k) \divides \phi(n)$.  But
$\phi(p) = (p-1)$ which divides $(p-1)p^{k-1} = \phi(p^k)$.
\end{solution}
\fi

\ppart Prove
\begin{lemma*}%\label{npdp}
If $n$ is a product of distinct primes and $a \in \naturals$ is
$\equiv 1\ \pmod{\phi(n)}$, then $m^a = m \quad \Zmod{n}$.
\end{lemma*}

\begin{solution}
Suppose $n$ is a product of distinct primes,
  $p_1p_2\cdots p_k$.  Then from the formulas for the Euler function,
  $\phi$, we have
\[
\phi(n) = (p_1 -1)(p_2 -1)\cdots (p_k-1).
\]

Now suppose $a \equiv 1 \pmod{\phi(n)}$, that is, $a$ is 1 plus a
multiple of $\phi(n)$.   So $a$ is also 1 plus a multiple of $p_i-1$, namely,
\[
a \equiv 1 \pmod {p_i - 1}.
\]
Hence, by part~\eqref{pma},
\[
m^a \equiv m \pmod{p_i}
\]
for all $m$.  Since this holds for all factors, $p_i$, of $n$, we conclude
from part~\eqref{abp} that
\[
m^a \equiv m \pmod{n},
\]
which proves the Lemma.
\end{solution}

\ppart Combine the previous parts to complete the proof
of~\eqref{mdepq=m}.

\begin{solution}
By part~\eqref{cong->decode}, we need only prove~\eqref{mdeqvmpq}.

Now for $n=pq$, we have from Lemma~\bref{phi_pq}---or the more
general Theorem~\bref{th:phi}---that $\phi(n) = (p-1)(q-1)$, so
$a \eqdef d \cdot e \equiv 1 \pmod {\phi(n)}$.  Now the Lemma implies
\[
m^a \equiv m \pmod {n},
\]
and this implies~\eqref{mdeqvmpq} since $m^a = m^{d \cdot e}=
(m^d)^e$.
\end{solution}

\eparts
\fi

\end{problem}



%%%%%%%%%%%%%%%%%%%%%%%%%%%%%%%%%%%%%%%%%%%%%%%%%%%%%%%%%%%%%%%%%%%%%
% Problem ends here
%%%%%%%%%%%%%%%%%%%%%%%%%%%%%%%%%%%%%%%%%%%%%%%%%%%%%%%%%%%%%%%%%%%%%

\endinput
