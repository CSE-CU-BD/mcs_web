\documentclass[problem]{mcs}

\begin{pcomments}
  \pcomment{CP_RSA_proving_correctness}
  \pcomment{from: S09.cp8r}
  \pcomment{revised for Zmodn and way simpler---ARM 3/28/13}
\end{pcomments}

\pkeywords{
  number_theory
  modular_arithmetic
  primes
  RSA
  Euler_theorem
  phi
}

%%%%%%%%%%%%%%%%%%%%%%%%%%%%%%%%%%%%%%%%%%%%%%%%%%%%%%%%%%%%%%%%%%%%%
% problem starts here
%%%%%%%%%%%%%%%%%%%%%%%%%%%%%%%%%%%%%%%%%%%%%%%%%%%%%%%%%%%%%%%%%%%%%

\begin{problem}
  A critical fact about \idx{RSA} is, of course, that decrypting an
  encrypted message always gives back the original message, $m$.
  Namely, if $n=pq$ where $p$ and $q$ are distinct primes, $m \in
  \Zintvco{0}{pq}$, and
    \[
    d\cdot e \equiv 1 \pmod{(p-1)(q-1)},
    \]
    then
    \begin{equation}\label{mdepq=m}
      \widehat{m}^d \eqdef \paren{m^e}^d = m \inzmod{n}.
      %\rem{\paren{\rem{m^d}{n}}^e}{n} = m.
    \end{equation}

We'll now prove this.

\bparts

\ppart\label{relprime:mdepq-m} Explain why~\eqref{mdepq=m} follows
very simply from Euler's theorem when $m$ is \emph{relatively prime
  to} $n$.

\begin{solution}
A key observation is that $(p-1)(q-1) = \phi(n)$, so $de \equiv 1
\pmod{\phi(n)}$ by definition of $d$ and $e$.  Now $\paren{m^e}^d =
m^{ed}$, so equation~\ref{mdepq=m} will follow from the general remark
that if $a \equiv 1 \pmod{\phi(n)}$, then
\begin{equation}\label{mamzn}
m^a = m \inzmod{n}
\end{equation}
for any $n >1$ and $m$ relatively prime to $n$, that is, $m \in \relpr{n}$.  This follows from
that fact, by definition of congruence mod $\phi(n)$, that $a$ must be
of the form $1+k\phi(n)$.  So
\begin{align*}
m^a
 & = m^{1 + k\phi(n)}\\
 & = m \paren{m^{\phi(n)}}^k\\
 & = m \cdot 1^k \inzmod{n}
        & \text{(Euler's Theorem for $m \in \relpr{n}$)}\\
 & = m.
\end{align*}
\end{solution}

\eparts All the rest of this problem is about removing the restriction
that $m$ be relatively prime to $n$.  That is, we aim to prove that
equation~\eqref{mdepq=m} holds for \emph{all} $m \in \Zintvco{0}{n}$.

It is important to realize that, even if it was theoretically
necessary, there would be no practical reason to worry about---or to
bother to check for---this relative primality condition before sending
a message $m$ using RSA.  That's because the whole RSA enterprise is
predicated on the difficulty of factoring.  If an $m$ ever came up
that wasn't relatively prime to $n$, then we could factor $n$ by
computing $\gcd(m,n)$.  So believing in the security of RSA implies
believing that the probability of a message $m$ turning up that was
not relatively prime to $n$ is negligible.

But let's be pure, impractical mathematicians and rid of this
technically unnecessary relative primality side condition, even if it
is harmless.  One gain for doing this is that statements about RSA
will be simpler without the side condition.  More important, the proof
below illustrates a useful general method of proving things about a
number $n$ by proving them separately for the prime factors of $n$.

\bparts

\ppart\label{pma} Prove that if $p$ is prime and $a \equiv 1
\pmod{p-1}$, then
\begin{equation}\label{mamp}
m^a = m \inzmod{p}.
\end{equation}

\begin{solution}
If $p \divides m$, then equation~\eqref{mamp} holds since both sides
equal 0 in $\Zmod{p}$.  On the other hand, if $p$ does not divide
$m$, then~\eqref{mamp} holds by~\eqref{mamzn} since $\phi(p) = p-1$.
\end{solution}

\ppart\label{abp} Give an elementary proof\footnote{There is no need
  to appeal to the Chinese Remainder Theorem.} that if $a \equiv b
\pmod {p_i}$ for distinct primes $p_i$, then $a \equiv b$ modulo the
product of these primes.

%\pmod{p_1p_1\cdots p_n}$.

\begin{solution}
If $a \equiv b \pmod {p_i}$, then by definition, $p_i \divides (a-b)$.
By the Unique Factorization Theorem~\bref{thm:unique_factor}, the
product of the $p_i$'s must therefore also divide $(a-b)$, which means
that $a \equiv b$ modulo the product of all these primes.
\end{solution}

\ppart Note that~\eqref{mdepq=m} is a special case of
\begin{claim*}
If $n$ is a product of distinct primes and $a \equiv 1
\pmod{\phi(n)}$, then
\[
m^a = m \inzmod{n}.
\]
\end{claim*}
Use the previous parts to prove the Claim.

\begin{solution}
Suppose $n$ is a product of distinct primes,
  $p_1p_2\cdots p_k$.  Then by multiplicativity of Euler's functions,
  $\phi$, we have
\[
\phi(n) = \phi(p_1)\phi(p_2)\cdots \phi(p_k).
\]

Now suppose $a \equiv 1 \pmod{\phi(n)}$, that is, $a$ is 1 plus a
multiple of $\phi(n)$.   So $a$ is also 1 plus a multiple of $\phi(p_i)$, namely,
\[
a \equiv 1 \pmod {\phi(p_i)}.
\]
Hence, by equation~\eqref{mamzn},
\[
m^a \equiv m \pmod{p_i}
\]
for all $m$.  Since this holds for all factors, $p_i$, of $n$, we conclude
from part~\eqref{abp} that
\[
m^a \equiv m \pmod{n},
\]
which proves the Claim.
\end{solution}

\eparts
\end{problem}


%%%%%%%%%%%%%%%%%%%%%%%%%%%%%%%%%%%%%%%%%%%%%%%%%%%%%%%%%%%%%%%%%%%%%
% Problem ends here
%%%%%%%%%%%%%%%%%%%%%%%%%%%%%%%%%%%%%%%%%%%%%%%%%%%%%%%%%%%%%%%%%%%%%

\endinput
