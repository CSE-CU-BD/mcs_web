\documentclass[problem]{mcs}

\begin{pcomments}
  \pcomment{PS_immuneset}
  \pcomment{variant of PS_immune}
  \pcomment{ARM 10.9.17}
\end{pcomments}

\pkeywords{
  diagonal_argument
  diagonal
  countable
  function
  immune
  range
}

%%%%%%%%%%%%%%%%%%%%%%%%%%%%%%%%%%%%%%%%%%%%%%%%%%%%%%%%%%%%%%%%%%%%%
% Problem starts here
%%%%%%%%%%%%%%%%%%%%%%%%%%%%%%%%%%%%%%%%%%%%%%%%%%%%%%%%%%%%%%%%%%%%%

\begin{problem}
Let
\[
f_0, f_1, f_2, \dots, f_k,\dots
\]
be an infinite sequence of total functions $f_k:\nngint \to \nngint$.

Define a partial function $\phi: \nngint \to \nngint$ as follows:
\[
\phi(k) \eqdef f_k(\min\set{j \suchthat f_k(j) \geq 2k}).
\]
If all the elements in $\range{f_k}$ are $< 2k$, then $\phi(k)$ is
undefined.

Let $U$ to be the complement of $\range{\phi}$, that is,
\[
U \eqdef \bar{\range{\phi}}.
\]

\bparts

\ppart\label{Uinf} Show that $U$ is infinite.

\begin{solution}
By definition, $\phi(k) \geq 2k$ for all $k \nngint$, so $\range{\phi}$ can
contain at most half the elements in $\Zintv{0}{n}$ for every $n \in
\nngint$.\inhandout{\footnote{$\Zintv{0}{n} \eqdef
    \set{0,1,\dots,n}$.}}  So $U$ must contain \emph{at least} half
the elements in $\Zintv{0}{n}$ for every $n$.
\end{solution}

\ppart Show that for all $k \in \nngint$, if $\range{f_k}$ is infinite, then
\[
\range{f_k} \not\subseteq U.
\]

\begin{solution}
We just need to show that
\[
\range{f_k} \intersect \range{\phi} \neq \emptyset
\]
when $\range{f_k}$ is infinite.

But if $\range{f_k}$ is infinite, then there is a $j \in \nngint$ such
that $f_k(j) \geq 2k$, so $\phi(k)$ will be defined and in
$\range{f_{k}}$.
\end{solution}

\ppart Tweak part~\eqref{Uinf} so
\[
\lim_{n \to \infty} \frac{\card{U \intersect \Zintv{0}{n}}}{n} = 1.
\]

\begin{solution}
Replace $2k$ by $k^2$, or by any other function of $k$ that grows more
than linearly.
\end{solution}

\begin{staffnotes}
Contrast with a majorizing function $h$ for $f_0,f_1,\dots$.  Now like
$U$, $\range{h}$ does not contain any infinite $\range{f_k}$.  But
$\range{h}$ accomplishes this only by being very sparse.
\end{staffnotes}

\eparts

\end{problem}

\endinput
