\documentclass[problem]{mcs}

\begin{pcomments}
  \pcomment{PS_gcd_termination}
  \pcomment{by ARM 2/28/11}
  \pcomment{soln by Alexander J Smith, edited ARM 3/30/17}
\end{pcomments}

\pkeywords{
  Euclidean
  algorithm
  GCD
  Fibonacci
  transition
  golden_ratio
}

%%%%%%%%%%%%%%%%%%%%%%%%%%%%%%%%%%%%%%%%%%%%%%%%%%%%%%%%%%%%%%%%%%%%%
% Problem starts here
%%%%%%%%%%%%%%%%%%%%%%%%%%%%%%%%%%%%%%%%%%%%%%%%%%%%%%%%%%%%%%%%%%%%%

\begin{problem}
The Euclidean state machine is defined by the rule
\begin{equation}\label{euclid_transition_prob}
(x,y) \movesto (y, \rem{x}{y}),
\end{equation}
for $y>0$.

Prove that the smallest positive integers $a \ge b$ for which,
starting in state $(a,b)$, the state machine will make $n$ transitions
are $F(n+1)$ and $F(n)$, where $F(n)$ is the $n$th Fibonacci
number.\footnote{Problem~\bref{CP_fibonacci_by_induction} shows that $F(n) \le
  \varphi^n$ where $\varphi$ is the \idx{golden ratio} $(1 +
  \sqrt{5})/2$.  This implies that the Euclidean algorithm halts after
  at most $\log_\varphi(a)$ transitions, a somewhat smaller bound
  than $2 \log_2 a$ derived from
  equation~(\bref{rxylx2})\inhandout{ in the text}.}

\hint Induction.

\begin{solution}
We prove this by induction.

The induction hypothesis $P(n)$ will be that the algorithm on
$(F(n+1),F(n))$ will terminate in exactly $n$ transitions, and if
either $a<F(n+1)$ or $b<F(n)$, then the algorithm on $(a,b)$ will
terminate in strictly fewer than $n$ transitions.

\inductioncase{Base Case}: ($n=1$).  We have $F(2)=F(1)=1$, so the
Euclidean algorithm yields
\[
(1,1)\movesto(1,0)
\]
for a total of 1 transition.  Note that any pair smaller than $(1,1)$
is already a termination state.

\inductioncase{Inductive Step}: To prove $P(n+1)$, suppose first that
$(a,b)=(F(n+2),F(n+1))$.

Since $\rem{F(n+2)}{F(n+1)} = F(n+2)-F(n+1)$, the Euclidean state
machine transition will be
\[
(a,b)\movesto(b,F(n+2)-F(n+1))=(F(n+1),F(n)).
\]
Now the machine will terminate in exactly $n$ more transitions by
induction hypothesis, for a total of $n+1$ transitions.

Next, if $b<F(n+1)$, the first transition will be to a state whose
first coordinate is $<F(n+1)$.  This state will terminate in at most
$n-1$ more transitions by the inductive hypothesis, so the total
number of transitions will be strictly less than $n+1$.

Finally, if $b\geq F(n+1)$ and $a<F(n+2)$, then $a<2b$, and so
\[
(a,b) \movesto (b,a-b).
\]
Also, $a-b<F(n+2)-F(n+1)=F(n)$.  Now by the inductive hypothesis,
state $(b,a-b)$ will also terminate in at most $n-1$ more transitions,
completing the proof of $P(n+1)$.
\end{solution}
\end{problem}


%%%%%%%%%%%%%%%%%%%%%%%%%%%%%%%%%%%%%%%%%%%%%%%%%%%%%%%%%%%%%%%%%%%%%
% Problem ends here
%%%%%%%%%%%%%%%%%%%%%%%%%%%%%%%%%%%%%%%%%%%%%%%%%%%%%%%%%%%%%%%%%%%%%

\endinput
