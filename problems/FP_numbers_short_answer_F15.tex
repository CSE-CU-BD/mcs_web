\documentclass[problem]{mcs}

\begin{pcomments}
  \pcomment{FP_numbers_short_answer_F15}
  \pcomment{excerpted from FP_multiple_choice_unhidden}
  \pcomment{subsumes FP_numbers_short_answer_fall11}
  \pcomment{revised 12/23/11 ARM}
  \pcomment{revised 12/9/15 tashas, ARM}
\end{pcomments}

\pkeywords{
  gcd
  modulo
  relatively_prime
  inverse
  mod_n
  Fermat
  prime
}

%%%%%%%%%%%%%%%%%%%%%%%%%%%%%%%%%%%%%%%%%%%%%%%%%%%%%%%%%%%%%%%%%%%%%
% Problem starts here
%%%%%%%%%%%%%%%%%%%%%%%%%%%%%%%%%%%%%%%%%%%%%%%%%%%%%%%%%%%%%%%%%%%%%

\begin{problem} \mbox{}

Parts~\eqref{a=bmpapb} to~\eqref{abazb} offer a collection of number
theoretic assertions.  Most of the assertions require some
side-conditions to be correct.  A selection of possible
side-conditions (\textbf{1})--(\textbf{8}) are listed below.

Next to each of assertions, write \True\ if the assertion is correct
as stated, write the number of the \emph{weakest} of the
side-conditions necessary to make the assertion true,\footnote{Writing
  that a true assertion required a side-condition would be a mistake,
  for example.} or write \False\ if none of the side-conditions imply
the assertion is correct.

Assume all variables have appropriate integer values---for example, in
the context of $\Zmod{k}$, numbers are in $\Zintvco{0}{k}$ and $k>1$.

\iffalse
\renewcommand{\theenumi}{\roman{enumi}}
\renewcommand{\labelenumi}{(\theenumi)}

\begin{center}
\textbf{Side-conditions}
\end{center}
 
\begin{enumerate}
\item\label{gcdab1} $\gcd(a, b) = 1$,
\item $a$ does not divide $b$,
\item\label{gcdab>1} $\gcd(a, b) > 1$,
\item $\gcd(a, b) > 2$,
\item $\gcd(a,b)=1=\gcd(a,c)$,
\item $a$ is prime,
\item $b$ is prime,
\item\label{abprmb} $a$ and $b$ are prime.

\item \True,
\item\label{noneedF15} \False.
\end{enumerate}
\fi


\[\begin{array}{llll}
(\mathbf{i}).\ \QNOT(a \text{ divides  } b)
 & (\mathbf{ii}).\ \gcd(a, b) = 1
 & (\mathbf{iii}).\ \gcd(a, b) > 1\\
(\mathbf{iv}).\ \gcd(a, b) > 2
 & (\mathbf{v}).\ \gcd(a,b)=1=\gcd(a,c)\\
(\mathbf{vi}).\ a \text{ is prime}
 & (\mathbf{vii}).\ b \text{ is prime}
 & (\mathbf{viii}).\ a,b \text{ both prime}
\end{array}\]

\bigskip

\bparts

\ppart\label{a=bmpapb} If $a = b \inzmod{n}$, then $p(a) = p(b) \inzmod{n}$,\\
\mbox{} \qquad for any polynomial $p(x)$ with integer coefficients.\hfill \examrule[0.7in]

\begin{solution}
\True
\end{solution}

\ppart If $a \divides b c$, then $a \divides c$.  \hfill\examrule[0.7in]

\begin{solution}
$\gcd(a, b) = 1$.
\end{solution}

\iffalse
\ppart $\gcd(a^m,b^m) = (\gcd(a,b))^m$.\hfill\examrule[0.7in]

\begin{solution}
\True
\end{solution}
\fi

\ppart $\gcd(1 + a, 1 + b) = 1 + \gcd(a, b)$.  \hfill\examrule[0.7in]

\begin{solution}
\False
\end{solution}

\iffalse
\ppart If $\gcd(a, b) \neq 1$ and $\gcd(b, c) \neq 1$, then $\gcd(a, c) \neq 1$. \hfill 

\begin{solution}
\False}  % $a=2\cdot 3, b=3\cdot 5, c=5\cdot 7$
\end{solution}


\ppart Some integer linear combination of $a^2$ and $b^2$ equals
one. \hfill \examrule[0.7in]

\begin{solution}
$\gcd(a, b) = 1$.
\end{solution}
\fi

\ppart No integer linear combination of $a^2$ and $b^2$ equals
one. \hfill \examrule[0.7in]

\begin{solution}
$\gcd(a, b) > 1$.
\end{solution}

\ppart No integer linear combination of $a^2$ and $b^2$ equals two.\hfill\examrule[0.7in]

\begin{solution}
$\gcd(a, b) > 1$.  If gcd is not one, it must be a square.
\end{solution}

\inbook{
\ppart If $m a = n a \inzmod{b}$ then $m = n \inzmod{b}$.\hfill\examrule[0.7in]

\begin{solution}
$\gcd(a,b)=1$
\end{solution}

\ppart  If $b^{-1} = c^{-1} \inzmods{a}$, then $b = c \inzmods{a}$.\hfill\examrule[0.7in]

\begin{solution}
\True, where we take the use of inverse notation to imply the inverses
exist.  The condition that the inverses exist, namely, $\gcd(a,b)=1$
and $\gcd(a,c) = 1$, is also a good answer.
\end{solution}

\iffalse
\ppart If $m = n \inzmod{\phi(a)}$, then $m^b = n^b \inzmod{a}$.\hfill\examrule[0.7in]

\begin{solution}
\False
\end{solution}
\fi

\ppart If $m = n \inzmod{\phi(a)}$, then $b^m = b^n \inzmod{a}$.\hfill\examrule[0.7in]

\begin{solution}
$\gcd(a,b)=1$
\end{solution}
}

\ppart If $m = n \inzmod{ab}$, then $m = n \inzmod{b}$.\hfill\examrule[0.7in]

\begin{solution}
\True
\end{solution}

\iffalse
\ppart $[m = n \inzmod{a} \QAND\ m = n \inzmod{b}]$ iff $[m  = n \inzmod{ab}]$. \hfill\examrule[0.7in]

\begin{solution}
$\gcd(a, b) = 1$: the Chinese Remainder Theorem
  (Problem~\bref{CP_chinese_remainder}).
\end{solution}
\fi

\ppart\label{abazb} $a^{b} = a \inzmod{b}$. \hfill\examrule[0.7in]

\begin{solution}
$b$ is prime.
\end{solution}

\iffalse
\ppart\label{ainvmb}
 $a$ has in inverse in $\Zmod{b}$  iff $b$ has an inverse in $\Zmod{a}$. \hfill\examrule[0.7in]

\begin{solution}
\True:  $a$  has in inverse in $\Zmod{b}$  iff $\gcd(a,b)=1$.
\end{solution}
\fi

\eparts

\end{problem}

\endinput
