\documentclass[problem]{mcs}

\begin{pcomments}
  \pcomment{FP_numbers_short_answer_F15}
  \pcomment{excerpted from FP_multiple_choice_unhidden}
  \pcomment{subsumes FP_numbers_short_answer_fall11}
  \pcomment{revised 12/23/11 ARM}
  \pcomment{revised 12/9/15 tashas, ARM}
\end{pcomments}

\pkeywords{
  gcd
  modulo
  relatively_prime
  inverse
  mod_n
  Fermat
  prime
}

%%%%%%%%%%%%%%%%%%%%%%%%%%%%%%%%%%%%%%%%%%%%%%%%%%%%%%%%%%%%%%%%%%%%%
% Problem starts here
%%%%%%%%%%%%%%%%%%%%%%%%%%%%%%%%%%%%%%%%%%%%%%%%%%%%%%%%%%%%%%%%%%%%%

\begin{problem} \mbox{}

%Circle \textbf{true} or \textbf{false} for the statements below,
%and \emph{provide counterexamples} for those that are \textbf{false}.

Parts~\eqref{a=bmpapb} to~\eqref{ainvmb} offer a collection of number
theoretic assertions involving integer-valued variables $a,b,c,j,k,
m,n$, with
\[
j,k \geq 0\text{  and  } m,n>1.
\]
Most of assertions require some side-conditions to be correct.  A
selection of possible side-conditions are listed in
items~\eqref{gcdab1} to~\eqref{noneedF15}.

For each of assertions in parts~\eqref{a=bmpapb} to~\eqref{ainvmb},
write \True\ if the assertion is correct as stated, write the numeral
of the \emph{weakest} of the side-conditions necessary to make the
assertion true, or write \False\ if none of the side-conditions imply
the assertion is correct.

\renewcommand{\theenumi}{\roman{enumi}}
\renewcommand{\labelenumi}{(\theenumi)}

\begin{center}
\textbf{Side-conditions}
\end{center}

\begin{enumerate}
\item\label{gcdab1} $\gcd(a, b) = 1$,
\item\label{gcdab>1} $\gcd(a, b) > 1$,
\item $\gcd(b, c) = 1$,
\item $\gcd(a, n) = 1$,
\item $n$ is prime,
\item $m$ and $n$ are relatively prime,
\item $m$ and $n$ are prime,
\item $c$ relatively prime to $n$, 
\item $n$ does not divide $c$,
\item $a,b$ have inverses in $\Zmod{n}$,
\item \True,
\item\label{noneedF15} \False.
\end{enumerate}

\bparts

\ppart\label{a=bmpapb} If $a = b \inzmod{n}$, then $p(a) = p(b)
\inzmod{n}$, \hfill \examrule[0.5in]

\noindent for any polynomial $p(x)$ with integer coefficients.
\begin{solution}
\True
\end{solution}

\ppart If $a \divides b c$, then $a \divides c$.  \hfill\examrule[0.5in]

\begin{solution}
$\gcd(a, b) = 1$.
\end{solution}

\ppart $\gcd(m^k,n^k) =  (\gcd(m,n))^k$.\hfill\examrule[0.5in]

\begin{solution}
\True
\end{solution}

\ppart $\gcd(1 + a, 1 + b) = 1 + \gcd(a, b)$.  \hfill\examspace[0.5in]

\begin{solution}
\False
\end{solution}


%\ppart If $\gcd(a, b) \neq 1$ and $\gcd(b, c) \neq 1$, then $\gcd(a, c) \neq 1$. \hfill 

%\begin{solution}
%\textbf{false} $a=2\cdot 3, b=3\cdot 5, c=5\cdot 7$
%\end{solution}

\ppart Some integer linear combination of $a^2$ and $b^2$ equals one.\hfill\examrule[0.5in]

\begin{solution}
$\gcd(a, b) = 1$.
\end{solution}

\ppart No integer linear combination of $a^2$ and $b^2$ equals 2.\hfill\examrule[0.5in]

\begin{solution}
$\gcd{a,b} >2$.
\end{solution}

\ppart If $a c \equiv b c \inzmod{n}$ then $a \equiv b \inzmod{n}$.\hfill\examrule[0.5in]

\begin{solution}
Need $c$ relatively prime to $n$.  
\end{solution}

\ppart  If $a^{-1} = b^{-1} \inzmods{n}$, then $a = b \inzmods{n}$.\hfill\examrule[0.5in]

\begin{solution}
$a,b$ have inverses modulo $n$; \True\ is also correct taking the use
  of inverse notation to imply the inverse exist.
\end{solution}

\ppart If $a c = b c \inzmod{n}$, then $a = b \inzmod{n}$.  \hfill\examrule[0.5in]

\begin{solution}
Need $c$ relatively prime to $n$.  
\end{solution}

\iffalse
\ppart If $a c = b c \inzmods{n}$, then $a = b \inzmods{n}$.  \hfill\examrule[0.5in]

\begin{solution}
\True
\end{solution}
\fi

\ppart If $j = k \inzmod{\phi(n)}$, then $m^j = m^k \inzmod{n}$.\hfill\examrule[0.5in]

\begin{solution}
  Need $m$ relatively prime to $n$.
\end{solution}

\ppart If $a = b \inzmod{nm}$, then $a = b \inzmod{n}$.\hfill\examrule[0.5in]

\begin{solution}
\True
\end{solution}

\ppart $[x = y \inzmod{m} \QAND\ x = y \inzmod{n}]$ iff $[x  = y \inzmod{mn}]$. \hfill\examrule[0.5in]

\begin{solution}
$\gcd(m, n) = 1$: the Chinese Remainder Theorem
  (Problem~\bref{CP_chinese_remainder}).
\end{solution}

\ppart $j^{n} = j \inzmod{n}$. \hfill\examrule[0.5in]

\begin{solution}
$n$ is prime.
\end{solution}

\ppart\label{ainvmb}
 $m \inzmods{n}$  iff  $n \inzmods{m}$. \hfill\examrule[0.5in]

\begin{solution}
\True:  $m\inzmods{n}$ iff $m,n$ relatively prime iff $m\inzmods{n}$.
\end{solution}

\eparts

\end{problem}

\endinput
