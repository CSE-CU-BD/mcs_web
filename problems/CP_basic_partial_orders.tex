%PS_basic_partial_orders

\documentclass[problem]{mcs}

\begin{pcomments}
  \pcomment{from: S09.ps3}
\end{pcomments}

\pkeywords{
  relations
  relational_properties
  partial_orders
}

%%%%%%%%%%%%%%%%%%%%%%%%%%%%%%%%%%%%%%%%%%%%%%%%%%%%%%%%%%%%%%%%%%%%%
% Problem starts here
%%%%%%%%%%%%%%%%%%%%%%%%%%%%%%%%%%%%%%%%%%%%%%%%%%%%%%%%%%%%%%%%%%%%%

\begin{problem}
For each of the binary relations below, state whether it is a strict
partial order, a weak partial order, or neither.  If it is not a partial
order, indicate which of the axioms for partial order it violates.

\iffalse
If it is a partial order, state whether or not it is a total order and
identify its maximal and minimal elements, if any.
\fi

\bparts \ppart The superset relation, $\supseteq$ on the power set
$\power{\set{1, 2, 3, 4, 5}}$.

\begin{solution}
This is a weak partial order, but not a total one.  For example, 
the sets of size 3 form an antichain.
\end{solution}

\ppart The relation between any two nonegative integers, $a$, $b$ that the
remainder of $a$ divided by 8 equals the remainder of $b$ divided by 8.

\begin{solution}
Violates antisymmetry: $8 \mrel{R} 16$ and $16 \mrel{R} 8$ but $8$ does
not equal $16$.  It is transitive, though.

\end{solution}

\ppart The relation between propositional formulas, $G$, $H$, that $G
\QIMPLIES H$ is valid.

\begin{solution}
  Violates antisymmetry: $P$ and $\QNOT(\QNOT(P))$ imply each other but
  are different expressions.  It is transitive, though.  \iffalse
  This does define a p.o. between
  equivalence classes, if we consider the set of all statements that are
  logically equivalent instead of the individual statements.\fi
\end{solution}

\ppart The relation 'beats' on Rock, Paper and Scissor (for those who don't
know the game Rock, Paper, Scissors, Rock beats Scissors, Scissors beats
Paper and Paper beats Rock).

\begin{solution}
Violates transitivity: obviously.
\end{solution}

\ppart The empty (no ``arrows'') relation, on the set of real numbers.
\begin{solution}
  It's vacuously asymmetric and transitive, so it's a strict partial
  order.  It's not reflexive because element don't have self-looping
  arrows.
\end{solution}

\eparts

\end{problem}

%%%%%%%%%%%%%%%%%%%%%%%%%%%%%%%%%%%%%%%%%%%%%%%%%%%%%%%%%%%%%%%%%%%%%
% Problem ends here
%%%%%%%%%%%%%%%%%%%%%%%%%%%%%%%%%%%%%%%%%%%%%%%%%%%%%%%%%%%%%%%%%%%%%

\endinput
