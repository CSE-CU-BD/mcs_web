\documentclass[problem]{mcs}

\begin{pcomments}
  \pcomment{PS_Rabin_cryptosystem} 
  \pcomment{from: S07.ps7}
\end{pcomments}

\pkeywords{
RSA
crypto
Rabin
Chinese_Remainder_Theorem
}

%%%%%%%%%%%%%%%%%%%%%%%%%%%%%%%%%%%%%%%%%%%%%%%%%%%%%%%%%%%%%%%%%%%%%
% Problem starts here
%%%%%%%%%%%%%%%%%%%%%%%%%%%%%%%%%%%%%%%%%%%%%%%%%%%%%%%%%%%%%%%%%%%%%
\begin{problem}
In class and recitation, we will study the RSA cryptosystem.
\iffalse
%which will be used by the Teen Titans in recitation.
\fi
However, we will not be able to give any evidence that it is hard to
break the RSA cryptosystem other than proof by reference to eminent
authority---that is, Rivest, Shamir and Adleman as well as a lot of
other very smart people over the last few decades were not able to
break it. This explains why cryptographers don't sleep very well.
\iffalse
%The Powerpuff girls need to ``one-up'' the
%Teen Titans from Recitation 5.
%To do this, they want to use a cryptosystem that has slightly better
\fi

Here we want to use a cryptosystem, called the Rabin cryptosystem, that
has slightly better security justification than RSA.  That is, if someone
has the ability to break this cryptosystem efficiently, then one also has
the ability to factor numbers that are products of two primes.  Why should
that convince us that it is hard to break the cryptosystem efficiently?
Well, mathematicians have been trying to factor efficiently for centuries,
and they still haven't figured out how to do it.  So, we are again
appealing to a proof by eminent authority, but at least there are more
authorities involved here.

What is the cryptosystem?  Let $N$ be a product of two very large primes
$p,q$ such that $p \equiv q \equiv3 \pmod{4}$.  To send the message $x$,
send $\rem{x^2}{N}$.\footnote{ We will see soon, that there are other
numbers that would be encrypted by $\rem{x^2}{N}$, so we'll have to
disallow those other numbers as possible messages in order to make it
possible to decode this cryptosystem, but let's ignore that for now.}

We need to show that if the person we send the message to knows $p,q$,
then they can decode the message.  On the other hand, if an eavesdropper
who doesn't know $p,q$ listens in, then we must show that they are very
unlikely to figure out this message.

First some definitions.  We know what it means for a number to be a square
over the integers, that is $s$ is a square if there is another integer $x$
such that $s= x^2$.  Over the numbers mod $N$, we say that $s$ is a \emph{
square modulo $N$} if there is an $x$ such that $s \equiv x^2 \pmod{N}$.
If $x$ is such that $0 \leq x <N$ and $s \equiv x^2 \pmod{N}$, then $x$ is
the \emph{square root of $s$}.


\bparts

\ppart What are the squares modulo 5?  For each nonzero square in
the interval $[0,\dots,4]$, how many square roots does it have?

\begin{solution}0,1,4 are the squares.  1 and 4 each have 2 square roots (1,4)
and (2,3) respectively.  2 and 3 have 0 square roots.
\end{solution}

\ppart For each integer in $[1,\dots,14]$ that is relatively prime to 15,
how many square roots (modulo 15) does it have?  Note that all the square
roots are \emph{also} relatively prime to 15.  We won't go through why this
is so here, but keep in mind that this is a general phenomenon!

\begin{solution}1,4 each have 4 square roots. 2,7,8,11,13,14 have no
square roots.
\end{solution}

\ppart Suppose that $p$ is a prime such that $p \equiv 3 \pmod{4}$.  It
turns out that squares modulo $p$ have exactly 2 square roots.  First show
that $(p+1)/4$ is an integer.  Next figure out the two square roots of
1 modulo $p$.  Then show that you can find a ``square root mod a prime
$p$'' of a number by raising the number to the $(p+1)/4$th power.
That is, given $s$, to find $x$ such that $s \equiv x^2 \pmod{p}$, you can
compute $\rem{s^{(p+1)/4}}{p}$.

\begin{solution}If $s \equiv x^2 \pmod{p}$, then
\[
s^{(p+1)/4} \equiv x^{(p+1)/2} \pmod{p} \equiv
x^{(p-1)/2} \cdot x \pmod{p} .
\]
The square roots of $1$ modulo $p$ are just $1$ and $-1$.  Now by Fermat's
theorem, the latter is equivalent modulo $p$ to both $+x$ and $-x$.
\end{solution}

\ppart By something called the Chinese Remainder Theorem, one can
show that if $p,q$ are distinct primes, then $s$ is a square modulo
$N =p \cdot q$ if and only if $s$ is a square modulo $p$ and $s$ is
a square modulo $q$.  In particular, if $s \equiv x^2 \pmod{p} \equiv
(x')^2 \pmod{p}$ and $s\equiv y^2 \pmod{p} \equiv (y')^2 \pmod{p}$
then $s$ has exactly four square roots, namely,
\[
s \equiv (xy)^2 \equiv (x'y)^2  \equiv (xy')^2 \equiv (x'y')^2 \pmod{N}.
\]
So, if you know $p,q$, using the solution to the previous problem
part, you can efficiently find the square roots of $s$! Thus, given
the secret key, decoding is easy.

\iffalse
%Though there is no proof that the RSA cryptosystem is as
%hard to break as factoring (a problem which we believe to
%be hard), the following cryptosystem, which can be thought
%of as using the value of 2 in the exponent in the RSA function,
%does have a proof that breaking it is as hard as factoring
%products of two primes.  We are going to prove a weaker
%version of the statement, although it will contain all of
%the main ideas of the full version.
%
\fi

\emph{But what if you don't know $p,q$?}  Suppose as above that $N= p \cdot
q$, where $p,q$ are two primes equivalent to $3 \pmod{4}$.
%You are not told the factorization of $N$, but instead
Let's assume that the evil message interceptor claims to have a
program that can find all four square roots of any number modulo
$N$.  Show that he can actually use this program to efficiently find
the factorization of $N$.
%Thus, if he can efficiently
%find square roots mod $N$, then he can factor $N$. and
Thus, unless this evil message interceptor is extremely smart and
has figured out something that the rest of the scientific community
has been working on for years, it is very unlikely that this
efficient square root program exists!
%so it is very unlikely that you can find square roots mod $N$!
\hint Pick $r$ arbitrarily from $1,\ldots,N-1$.  If $gcd(N,r)>1$, then you
are done (why?) so you can halt.  Otherwise, use the program to find all
four square roots of $r$, call them $r,-r,r',-r'$.  Note that $r^2 \equiv
r'^2 \pmod{N}$.  How can you use these roots to factor $N$?

\begin{solution}\[
r^2-r'^2
\equiv (r+r')(r-r') \pmod{N} \equiv 0 \pmod{N}.
\]
Therefore one of $(r+r')(r-r')$ divides $N=p \cdot q$, and since $p,q$ are
primes, it must be the case that one of $(r+r'),(r-r')$ divided $p$,
whereas the other divides $q$.  So compute $g = \gcd(N,r-r')$.  Output $g,
N/g$ and halt.

In the actual proof of security of the Rabin cryptosystem, you need
to show that if you have a program that can find \emph{any} square
root of a number modulo $N$ then you can factor $N$.   But the
outline of the proof is essentially the same.
\end{solution}

\ppart If the evil message interceptor knows that the message is the
encoding one of two possible candidate messages (that is, either ``meet at
dome at dusk'' or ``meet at dome at dawn'') and is just trying to figure
out which of the two, then can he break this cryptosystem?

\begin{solution}Yes, he just needs to square both candidate messages, take the
remainder with $N$ (which he knows) and see which one was sent.  If it's
so easy, then why is this a secure cryptosystem?  Well, it's still pretty
good in the case when there are a lot of possible messages.
\end{solution}

\eparts

\end{problem}


%%%%%%%%%%%%%%%%%%%%%%%%%%%%%%%%%%%%%%%%%%%%%%%%%%%%%%%%%%%%%%%%%%%%%
% Problem ends here
%%%%%%%%%%%%%%%%%%%%%%%%%%%%%%%%%%%%%%%%%%%%%%%%%%%%%%%%%%%%%%%%%%%%%

\endinput
