\documentclass[problem]{mcs}

\begin{pcomments}
  \pcomment{PS_Rabin_cryptosystem}
  \pcomment{from: S07.ps7}
  \pcomment{Could make a very nice problem, but needs work before
    using.  It is skectchy and hard to follow.  Also, frustrating
    because so many facts are stipulated w/o proof---ARM 3/3/13.}

\end{pcomments}

\pkeywords{
  RSA
  crypto
  Rabin
  Chinese_Remainder_Theorem
  square_root
  quadratic_residue
  Fermat_theorem
}

%%%%%%%%%%%%%%%%%%%%%%%%%%%%%%%%%%%%%%%%%%%%%%%%%%%%%%%%%%%%%%%%%%%%%
% Problem starts here
%%%%%%%%%%%%%%%%%%%%%%%%%%%%%%%%%%%%%%%%%%%%%%%%%%%%%%%%%%%%%%%%%%%%%
\begin{problem}
Although RSA has successfully withstood cryptographic attacks for a
more than a quarter century, it is not known that breaking RSA would
imply that factoring is easy.

In this problem we will examine the \term{Rabin cryptosystem} that
does have such a security certification.  Namely, if someone has the
ability to break the Rabin cryptosystem efficiently, then they also
have the ability to factor numbers that are products of two primes.

Why should that convince us that it is hard to break the cryptosystem
efficiently?  Well, mathematicians have been trying to factor
efficiently for centuries, and they still haven't figured out how to
do it.

\iffalse
So, we are again appealing to a proof by eminent authority,
but at least there are more authorities involved here.
\fi

What is the Rabin cryptosystem?  The public key will be a number $N$
that is a product of two very large primes $p,q$ such that $p \equiv q
\equiv 3 \pmod{4}$.  To send the message $m$, send
$\rem{m^2}{N}$.\footnote{We will see soon, that there are other
  numbers that would be encrypted by $\rem{m^2}{N}$, so we'll have to
  disallow those other numbers as possible messages in order to make
  it possible to decode this cryptosystem, but let's ignore that for
  now.}

The private key is the factorization of $N$, namely, the primes $p,q$.
We need to show that if the person being sent the message knows $p,q$,
then they can decode the message.  On the other hand, if an
eavesdropper who doesn't know $p,q$ listens in, then we must show that
they are very unlikely to figure out this message.

\begin{editingnotes}
The rest would be clearer written in term s $\Zmod{N}$ instead of
$\equiv \pmod{N}$
\end{editingnotes}

Say that $s$ is a \term{square modulo $N$} if there is an $m \in
[0,N)$ such that $s \equiv m^2 \pmod{N}$.  Such an $m$ is a
  \term{square root of $s$ modulo $N$}.

\bparts

\ppart What are the squares modulo 5?  For each square in the interval
$[0,5)$, how many square roots does it have?

\begin{solution}
The squares are 0,1 \& 4: 0 has one square root, namely itself; 1 has
two square roots, namely, 1 \& 4; 4 also has two square roots, namely
2 \& 3.
\end{solution}

\ppart For each integer in $[1,15)$ that is relatively prime to 15,
how many square roots (modulo 15) does it have?  Note that all the square
roots are \emph{also} relatively prime to 15.  We won't go through why this
is so here, but keep in mind that this is a general phenomenon!

\begin{solution}
1,4 each have 4 square roots. 2,7,8,11,13,14 have no square roots.
\end{solution}

\begin{editingnotes}
Revise this next part and its soln, which are too hard to follow now.
\end{editingnotes}

\ppart\label{peqv34} Suppose that $p$ is a prime such that $p \equiv 3
\pmod{4}$.  It turns out that squares modulo $p$ have exactly 2 square
roots.  First show that $(p+1)/4$ is an integer.  Next figure out the
two square roots of 1 modulo $p$.  Then show that you can find a
``square root mod a prime $p$'' of a number by raising the number to
the $(p+1)/4$th power.  That is, given $s$, to find $m$ such that $s
\equiv m^2 \pmod{p}$, you can compute $\rem{s^{(p+1)/4}}{p}$.

\begin{solution}If $s \equiv m^2 \pmod{p}$, then
\[
s^{(p+1)/4} \equiv m^{(p+1)/2} \pmod{p} \equiv
m^{(p-1)/2} \cdot m \pmod{p} .
\]
The square roots of $1$ modulo $p$ are just $1$ and $-1$.  Now by
Fermat's theorem, the latter is equivalent modulo $p$ to both $+m$ and
$-m$.
\end{solution}

\ppart The Chinese Remainder Theorem
(Problem~\bref{CP_chinese_remainder}) implies that if $p, q$ are
distinct primes, then $s$ is a square modulo $p q$ if and only if $s$
is a square modulo $p$ and $s$ is a square modulo $q$.  In particular,
if $s \equiv x^2 \equiv (x')^2 \pmod{p}$ where $x \neq x'$, and
likewise $s\equiv y^2 \equiv (y')^2 \pmod{q}$ then $s$ has exactly
four square roots modulo $N$, namely,
\[
s \equiv (xy)^2 \equiv (x'y)^2  \equiv (xy')^2 \equiv (x'y')^2 \pmod{pq}.
\]
So, if you know $p,q$, then using the solution to part~\eqref{peqv34},
you can efficiently find the square roots of $s$!  Thus, given the
private key, decoding is easy.

\iffalse
%Though there is no proof that the RSA cryptosystem is as
%hard to break as factoring (a problem which we believe to
%be hard), the following cryptosystem, which can be thought
%of as using the value of 2 in the exponent in the RSA function,
%does have a proof that breaking it is as hard as factoring
%products of two primes.  We are going to prove a weaker
%version of the statement, although it will contain all of
%the main ideas of the full version.
%
\fi

\emph{But what if you don't know $p,q$?}  \iffalse Suppose $N \eqdef p
q$, where $p,q$ are two primes equivalent to $3 \pmod{4}$.  You are
not told the factorization of $N$, but instead\fi

Let's assume that the evil message interceptor claims to have a
program that can find all four square roots of any number modulo
$N$.  Show that he can actually use this program to efficiently find
the factorization of $N$.
%Thus, if he can efficiently
%find square roots mod $N$, then he can factor $N$. and
Thus, unless this evil message interceptor is extremely smart and
has figured out something that the rest of the scientific community
has been working on for years, it is very unlikely that this
efficient square root program exists!
%so it is very unlikely that you can find square roots mod $N$!

\hint Pick $r$ arbitrarily from $[1,N)$.  If $gcd(N,r)>1$, then you
  are done (why?) so you can halt.  Otherwise, use the program to find
  all four square roots of $r$, call them $r,-r,r',-r'$.  Note that
  $r^2 \equiv r'^2 \pmod{N}$.  How can you use these roots to factor
  $N$?

\begin{solution}
\[
r^2-r'^2 \equiv (r+r')(r-r') \equiv 0 \pmod{N}.
\]
Therefore one of $(r+r')$ and $(r-r')$ divides $N=p \cdot q$, and
since $p,q$ are primes, it must be the case that one of $(r+r')$ and
$(r-r')$ divides $p$ and the other divides $q$.  So compute $g =
\gcd(N,r-r')$.  Output $g, N/g$ and halt.

In the actual proof of security of the Rabin cryptosystem, you need
to show that if you have a program that can find \emph{any} square
root of a number modulo $N$ then you can factor $N$.   But the
outline of the proof is essentially the same.
\end{solution}

\ppart If the evil message interceptor knows that the message is the
encoding one of two possible candidate messages (that is, either ``meet at
dome at dusk'' or ``meet at dome at dawn'') and is just trying to figure
out which of the two, then can he break this cryptosystem?

\begin{solution}
Yes, he just needs to square both candidate messages, take the
remainder with $N$ (which he knows) and see which one was sent.  If
it's so easy, then why is this a secure cryptosystem?  Well, it's
still pretty good in the case when there are a lot of possible
messages.
\end{solution}

\eparts

\end{problem}

%%%%%%%%%%%%%%%%%%%%%%%%%%%%%%%%%%%%%%%%%%%%%%%%%%%%%%%%%%%%%%%%%%%%%
% Problem ends here
%%%%%%%%%%%%%%%%%%%%%%%%%%%%%%%%%%%%%%%%%%%%%%%%%%%%%%%%%%%%%%%%%%%%%

\endinput

\iffalse
In class and recitation, we will study the RSA cryptosystem.

%which will be used by the Teen Titans in recitation.

However, we will not be able to give any evidence that it is hard to
break the RSA cryptosystem other than proof by reference to eminent
authority---that is, Rivest, Shamir and Adleman as well as a lot of
other very smart people over the last few decades were not able to
break it.  This explains why cryptographers don't sleep very well.

%The Powerpuff girls need to ``one-up'' the
%Teen Titans from Recitation 5.
%To do this, they want to use a cryptosystem that has slightly better
\fi

