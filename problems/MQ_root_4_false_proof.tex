\documentclass[problem]{mcs}

\begin{pcomments}
  \pcomment{MQ_root_4_false_proof}
  \pcomment{was called DG_root_4_false_proof}
  \pcomment{variant of CP_generalize_root_4_proof}
  \pcomment{from: S09.cp1r}
  \pcomment{Applies same proof to root(4) erroneously}
\end{pcomments}

\pkeywords{
  square_root
  irrational
  rational
  contradiction
  divisible
}

%%%%%%%%%%%%%%%%%%%%%%%%%%%%%%%%%%%%%%%%%%%%%%%%%%%%%%%%%%%%%%%%%%%%%
% Problem starts here
%%%%%%%%%%%%%%%%%%%%%%%%%%%%%%%%%%%%%%%%%%%%%%%%%%%%%%%%%%%%%%%%%%%%%

\begin{problem} %\label{generprob}
Sloppy Sam was trying to generalize the proof that $\sqrt{2}$ is
irrational, but he accidentally proved that $\sqrt{4}$ is also
irrational! Circle the first significant mistake, and write him a
brief note explaining his mistake.

\begin{quote}

\begin{theorem*}
$\sqrt{4}$ is an irrational number.
\end{theorem*}

\begin{proof}
  The proof is by contradiction: assume that $\sqrt{4}$ is rational, that
  is,
  \begin{equation}\label{2nd}
    \sqrt{4} = \frac{n}{d},
  \end{equation}
  where $n$ and $d$ are integers.  Now consider the smallest such positive
  integer denominator $d$.  We will prove in a moment that the numerator,
  $n$, and the denominator $d$ are both divisible by 4.  This implies that
         \[
        \frac{n/4}{d/4}
        \]
  is a fraction equal to $\sqrt{4}$ with a smaller positive integer
  denominator, a contradiction.

  To prove that $n$ and $d$ have 4 as a common factor, we start by
  squaring both sides of~\eqref{2nd} and get $4 = n^2 / d^2$, so
\begin{equation}\label{2d2}
4 d^2 = n^2.
\end{equation}
So 4 is a factor of $n^2$, which is only possible if 4 is in fact a
factor of $n$.

This means that $n=4k$ for some integer $k$ so
\begin{equation}\label{n24}
  n^2 = (4k)^2 = 16 k^2.
\end{equation}
Combining~\eqref{2d2} and~\eqref{n24} gives $4 d^2 = 16 k^2$, so
\begin{equation}\label{n22}
d^2 = 4k^2.
\end{equation}
So 4 is a factor of $d^2$, which again is only possible if 4 is in fact
also a factor of $d$, as claimed.
\end{proof}
\end{quote}


\begin{solution}
The mistake is the claim that if 4 is a factor of $n^2$, then 4 is a factor of $n$.
A simple counterexample to this implication: suppose $n = 6$. Then 4 is a factor of $n^2$, but 4 is not a factor of $n$.
\end{solution}

\end{problem}
%%%%%%%%%%%%%%%%%%%%%%%%%%%%%%%%%%%%%%%%%%%%%%%%%%%%%%%%%%%%%%%%%%%%%
% Problem ends here
%%%%%%%%%%%%%%%%%%%%%%%%%%%%%%%%%%%%%%%%%%%%%%%%%%%%%%%%%%%%%%%%%%%%%

\endinput
