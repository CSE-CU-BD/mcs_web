\documentclass[problem]{mcs}

\begin{pcomments}
    \pcomment{TP_Induction_Rules}
    \pcomment{Converted from prob2.scm by scmtotex and dmj
              on Sun 13 Jun 2010 10:18:32 AM EDT}
    \pcomment{revised by drewe on Tue 19 July 2011}
    \pcomment{edited ARM 8/25/11}
\end{pcomments}

\pkeywords{
induction
strong_induction
well_ordering
inference_rule
quantifier
predicate
}


\begin{problem}
Some fundamental principles for reasoning about nonnegative integers are:

\begin{enumerate}

\item
The Induction Principle,

\item
The Strong Induction Principle,

\item
The Well Ordering Principle.%
\index{Well Ordering Principle}

\end{enumerate}

Identify which, if any, of the above principles is captured by each of the
following inference rules.

\bparts

\ppart
\Rule{P(0), \forall m.\; (\forall k \le m. \; P(k)) \QIMPLIES P(m+1)}
     {\forall n.\; P(n)}  

\begin{solution}
Strong Induction.

The variables are renamed, but otherwise this is the rule for strong
induction in Section~\bref{strong_ind_sec}.

\end{solution}

\ppart

\Rule{P(b), \forall k \ge b.\; P(k) \QIMPLIES P(k+1)}
     {\forall k \ge b.\; P(k)}    

\begin{solution}
Simple Induction.

This is just the rule for simple induction in
Section~\bref{ord_induction_subsec} adjusted to start with base
case~$b$ instead of~0.
\end{solution}

\ppart

\Rule{\exists n.\; P(n)}
     {\exists m.\; [P(m) \QAND (\forall k.\; P(k) \QIMPLIES k \ge m)]}


\begin{solution}
The Well-ordering Principle.

If we let $S$ be the set $\set{k \suchthat P(k) \,}$, then $\exists n.\;
P(n)$ says that $S$ is nonempty, and
$[P(m) \QAND (\forall k. \; P(k) \QIMPLIES k \ge m)]$
says that $m$ is the least number in $S$.
\end{solution}

\ppart

\Rule{P(0), \forall k > 0.\; P(k) \QIMPLIES P(k+1)}
     {\forall n.\; P(n)}

\begin{solution}
None of the above.

This looks like the rule for simple induction, but in the antecedent,
$k$ is strictly greater than~0.  This leaves the possibility that
$P(0)$ does not imply $P(1)$, so the property $P$ may not propagate
from~0 to the rest of the numbers.
\end{solution}

\ppart

\Rule{\forall m. \; (\forall k < m.\; P(k)) \QIMPLIES P(m)}
     {\forall n.\; P(n)}    

\begin{solution}
Strong Induction.

This looks identical to the rule for strong induction, as in Part~1,
but with the base case missing.  Still, the base case is provided:
when $m$ is~0, the assumption of the implication is ``vacuously''
true, and the conclusion in this case is precisely that $P(0)$ is
true.
\end{solution}

\eparts

\end{problem}

\endinput
