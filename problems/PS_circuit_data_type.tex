\documentclass[problem]{mcs}

\begin{pcomments}
  \pcomment{CP_circuit_data-type}
  \pcomment{ARM draft 3/5/16}
\end{pcomments}

\pkeywords{
  circuit
  recursive
  structural_induction
  evaluate
}

\newcommand{\llist}[]{{\mathop{\mathrm{list(#1)}}}}
\newcommand{\lcons}[2]{{\mathop{\mathrm{cons(#1,#2)}}}}

\newcommand{\cwout}{\textbf{O}}
\newcommand{\wirs}{\ensuremath{W}}
\newcommand{\bgat}{\ensuremath{\text{gates}}}
\newcommand{\reccirc}{\text{DigCirc}}
\newcommand{\cwin}[1]{{\mathop{\mathrm{inputs(#1)}}}}
\newcommand{\ctern}[1]{{\mathop{\mathrm{internal(#1)}}}}
\newcommand{\bools}{\set{\true, \false}}

\begin{problem}
We can explain in a simple and precise way how digital circuits work,
and gain the powerful proof method of structural induction to verify
their properties, by defining digital circuits as a recursive data
type, $\reccirc$, consisting of a list of \emph{gate connections}.

Let $\wirs$ be a set $w_0,w_1,\dots$ whose elements are called
\emph{wires}, and $\bgat \eqdef \set{\QAND, \QOR, \QXOR}$ be a set
whose elements are called \emph{2-input gates}.  Let $\cwout \notin
\wirs$ be an object called the \emph{output}.

\TBA{FIGURE showing new gate $G$ connecting to inputs of circuit $C$.}

\begin{definition*}
The set of digital circuit $\reccirc$, and their inputs and internal
wires, are defined recursively as follows:

\inductioncase{Base cases:}
\begin{itemize}
\item If $G \in \bgat, x, y \in \wirs$, then $C \eqdef \llist{(x,y,G)} \in \reccirc$, and
\begin{align*}
\cwin{C} \eqdef \set{x,y},\\
\ctern{C} \eqdef \emptyset.
\end{align*}

\item If $x \in \wirs$, then $C \eqdef \list{(\QNOT,x)} \in \reccirc$, and
\begin{align*}
\cwin{C} \eqdef \set{x},\\
\ctern{C} \eqdef \emptyset.
\end{align*}
\end{itemize}

\inductioncase{Constructor cases:}
If
\begin{align*}
C & \in \reccirc,\\
I &\subseteq \cwin{C}, I \neq \emptyset,\\
x, y &\in \wirs - \tern{C} \union \cwin{C},
\end{align*}
then $D \in \reccirc$, where

\begin{itemize}

\item
$D \eqdef \lcons{(x,y, G, I)}{C}$ and
\begin{align*}
G & \in \bgat,\\
\cwin{D}  & \eqdef \set{x,y} \union \cwin{C} - I,\\
\ctern{D} & \eqdef \ctern{C} \union I.
\end{align*}

\item
$D \eqdef \llist{(x, \QNOT, I)}{C}$ and
\begin{align*}
\cwin{D}  & \eqdef \set{x,y} \union \cwin{C} - I,\\
\ctern{D} & \eqdef \ctern{C} \union I.
\end{align*}
\end{itemize}
\end{definition*}

Define a \emph{wire assignment} of $C$ to be a function
\[
A: \cwin{C} \union \ctern{C) \union {\cwout}} \to \bools
\]
such that \TBA{\dots}

\bparts

\ppart Sanity check: prove that $\cwin{C} \intersect \ctern{C} =
\emptyset$.

\ppart Define an \emph{environment} for $C$ to be a function $e:
\cwin{C} \to \bools$.  Prove that if two wire assignments for $C$
restricted to $\cwin{C}$ equal the same environment, then the wire
assignments are equal.  So any circuit $C$ defines a function from
environments for its inputs to $\bools$.

\ppart Define recursive procedure to find propositional formula $E_C$
with propositional variables $\cwin{C}$ equivalent (same function on
environments) to $C$.

\ppart Give examples where $E_c$ is exponentially larger than $C$.



\eparts

\end{problem}

\endinput
