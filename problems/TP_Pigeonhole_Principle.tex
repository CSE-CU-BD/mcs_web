\documentclass[problem]{mcs}

\begin{pcomments}
    \pcomment{TP_Pigeonhole_Principle}
    \pcomment{Converted from prob1.scm
              by scmtotex and dmj
              on Sun 13 Jun 2010 03:52:26 PM EDT}
    \pcomment{revised by drewe 2 Aug 2011}
\end{pcomments}

\begin{problem}

%% type: short-answer
%% title: Pigeonhole Principle

Below is a list of properties that a group of people might possess.
 
For each property, either give the minimum number of people that must
be in a group to ensure that the property holds, or else indicate that the property need not
hold even for arbitrarily large groups of people.

(Assume that every year has exactly 365 days; ignore leap years.)

\bparts

\ppart 
At least 2 people were born on the same day of the year (ignore year
of birth).

\begin{solution}

366

We let the pigeons be people and the days of the year be the holes
(365 holes).  If we have $365+1$ pigeons, two must be in the same hole
(i.e. the two must be born on the same day).
\end{solution}

\ppart
At least 2 people were born on January 1. 

\begin{solution}
Need not hold.

No matter how many people you have, you cannot force one of them to be
born on a specific day.  For example, everyone might be born on
January~2.
\end{solution}

\ppart 
At least 3 people were born on the same day of the week.

\begin{solution}

15

We let people be pigeons and the days on the week be holes (7
holes). Using the generalized pigeonhole principle, we need $2*7+1$
people to force 3 of them to be in the same hole (i.e. born on the
same day of the week).
\end{solution}

\ppart
At least 4 people were born in the same month. 

\begin{solution}
37

We let the people be the pigeons and the months of the year be the
holes (12 holes).  Using the generalized pigeonhole principle, we need
$3*12+1$ people to force 4 of them to be born in the same month.
\end{solution}

\ppart
At least 2 people were born exactly one week apart. 

\begin{solution}
Need not hold.

Again, you cannot force this.  For example, everyone might be born on
the same day of the year.
\end{solution}

\eparts

\end{problem}

\endinput
