\documentclass[problem]{mcs}

\begin{pcomments}
  \pcomment{CP_N_to_N_diagonal_argument}
  \pcomment{subsumes now deleted PS_N_to_A_diagonal_argument}
  \pcomment{by ARM 8/28/11, N^w notation introduced 8/30/11}
\end{pcomments}

\pkeywords{
  diagonal
  surjection
  powerset
}

%%%%%%%%%%%%%%%%%%%%%%%%%%%%%%%%%%%%%%%%%%%%%%%%%%%%%%%%%%%%%%%%%%%%%
% Problem starts here
%%%%%%%%%%%%%%%%%%%%%%%%%%%%%%%%%%%%%%%%%%%%%%%%%%%%%%%%%%%%%%%%%%%%%

\begin{problem}
  Let $\naturals^\omega$ the set of infinite sequences of nonnegative
  integers.  For example, some sequences of this kind are:
\begin{align*}
&(0, 1, 2, 3, 4, \dots),\\
&(2, 3, 5, 7, 11,\dots),\\
&(3, 1, 4, 5, 9, \dots).
\end{align*}
Prove that this set of sequences is\idx{uncountable}.

\begin{solution}
  
\begin{proof}
One approach is to show that if $\naturals^\omega$ were
countable, then $\power(\naturals)$ would be too, contradicting
Cantor's Theorem~\bref{powbig}.

\begin{staffnotes}
If needed, offer hint: verify that $\naturals^\omega$ is as big as
$\power(\naturals)$.
\end{staffnotes}

Namely, we can define a surjective function from
$f:\naturals^\omega \to \power(\naturals)$ as follows:
\[
f(s) \eqdef \set{n \in \naturals \suchthat s[n] = 0}
\]
where $s[n]$ is the $n$th element of sequence $s$.

Now if there was a surjective function from $g: \naturals \to
\naturals^\omega$, then the composition of $f$ and $g$ would
be a surjective function from $\naturals$ to $\power(\naturals)$
contradicting Cantor's Theorem~\bref{powbig}.
\end{proof}

Alternatively, to show that $\naturals^\omega$ is uncountable, we can
use a basic diagonal argument directly to show that no function from
$\naturals$ to the set of sequences $\naturals^\omega$ is a
surjection.

\begin{proof}
Let $\sigma$ be a function from $\naturals$ to the infinite sequences
of nonnegative integers.  To show that $\sigma$ is not a surjection,
we will describe a sequence, diag, of nonnegative integers that is not
in the range of $\sigma$.

Namely, define a sequence $\text{diag} \in \naturals^\omega$
as follows:
\begin{staffnotes}
If needed, offer this def of diag as a hint.
\end{staffnotes}

\[
\text{diag}[n] \eqdef \sigma(n)[n]+1.
\]
Now by definition,
\[
\text{diag}[n] \neq \sigma(n)[n],
\]
for all $n \in \naturals$, proving that diag is not equal to
$\sigma(n)$ for any $n \in \naturals$.  This means that diag is not in
the range of $\sigma$, as claimed.
\end{proof}
 
\end{solution}

\end{problem}

%%%%%%%%%%%%%%%%%%%%%%%%%%%%%%%%%%%%%%%%%%%%%%%%%%%%%%%%%%%%%%%%%%%%%
% Problem ends here
%%%%%%%%%%%%%%%%%%%%%%%%%%%%%%%%%%%%%%%%%%%%%%%%%%%%%%%%%%%%%%%%%%%%%
\endinput
