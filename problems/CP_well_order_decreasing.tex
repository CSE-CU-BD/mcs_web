\documentclass[problem]{mcs}

\begin{pcomments}
  \pcomment{CP_well_order_decreasing}
  \pcomment{ARM 2/4/12; footnote revised 1/1/13}
\end{pcomments}

\pkeywords{
  well_order
  minimum
  descending
}

\begin{problem}
Prove that a set $R$ of real numbers is well
ordered\inhandout{\footnote{A set of numbers is \emph{well ordered}
    when each of its nonempty subsets has a minimum element.  The Well
    Ordering Principle says, of course, that the set of nonnegative
    integers is well ordered, but so are lots of other sets, for
    example, the set $r\nngint$ of numbers of the form $rn$, where
    $r$ is a positive real number and $n \in \nngint$, see
    section~\bref{well_ordering_sec} of the text.}} iff there is no
infinite decreasing sequence of numbers $R$.  In other words, 
there is no set of numbers $r_i \in R$ such that
\begin{equation}\label{s0s1dots}
r_0 > r_1 > r_2 > \dots .
\end{equation}


\begin{solution}
\begin{proof}
To prove the claim in one direction, suppose there was such an
infinite sequence~\eqref{s0s1dots}.  Then $R$ is not well ordered,
because the set of elements in the sequence would be a subset of $R$
with no minimum element.

Conversely, suppose $R$ is not well ordered.  Then it must have a
nonempty subset $S$ with no minimum element.  Then we can find an
infinite sequence~\eqref{s0s1dots} as follows.  Since $S$ is not
empty, there is an element $r_0 \in S$.  But since $S$ is has no
minimum element, there must be an element $r_1 \in S$ such that $r_0 >
r_1$.  Continuing in this way, we define
\[
r_{n+1} \eqdef \text{ an element of $S$ that is $< r_n$}
\]
for $n \geq 0$.
\end{proof}
\end{solution}

\end{problem}

\endinput

\iffalse
\footnote{The empty set is well ordered because an empty set has no
  nonempty subsets.  So by logical convention (see Section~\bref{}),
  any proposition about all of these nonexistent subsets is true.}
\fi
