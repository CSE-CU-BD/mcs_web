\documentclass[problem]{mcs}

\begin{pcomments}
    \pcomment{TP_divides_n_square_not_n}
    \pcomment{ARM 9/7/13}
\end{pcomments}

\pkeywords{
divisor
square
prime
}

\begin{problem}
Give an example of two distinct positive integers $m,n$ such that $m$
is a divisor of $n^2$, but $m$ is not a divisor of $n$.  How about
having $m$ be less than $n$?

\begin{solution}
The simplest initial example is $m = 2^3$, $n = 2^2$.  In this case
$m$ is not less than $n$, but that's easy to remedy: multiply $n$ by
3.  That is, let $m = 2^3$, $n = 3\cdot 2^2$.
\end{solution}
\end{problem}

\endinput
