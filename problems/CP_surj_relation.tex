\documentclass[problem]{mcs}

\begin{pcomments}
  \pcomment{CP_surj_relation}
  \pcomment{overlaps TP_composition_of_jections, subsumes TP_injR_surjinvR}
  \pcomment{new part(d) added by ARM 2/15/12}
\end{pcomments}

\pkeywords{
  relations
  functions
  injections
  surjections
}

%%%%%%%%%%%%%%%%%%%%%%%%%%%%%%%%%%%%%%%%%%%%%%%%%%%%%%%%%%%%%%%%%%%%%
% Problem starts here
%%%%%%%%%%%%%%%%%%%%%%%%%%%%%%%%%%%%%%%%%%%%%%%%%%%%%%%%%%%%%%%%%%%%%

\begin{problem}
\begin{staffnotes}
Have students look up the definitions of $\surj$ and $\inj$:

\begin{definition*}
  $A \surj B$ iff there is a surjective \textbf{function} ($[\leq
    1\ \text{out}, \geq 1\ \text{in}]$) from $A$ to $B$.

  $A \inj B$ iff there is a total injective \emph{relation} ($[\geq
    1\ \text{out}, \leq 1\ \text{in}]$) from $A$ to $B$.
\end{definition*}

For finite sets, everything below follows trivially from the Mapping
Lemma about sizes of sets.  Congratulate any students who get it that
way, but then challenge them to do it for arbitrary sets.

The proofs below would be clearer using an archery argument.
Encourage students to do their proofs in terms of arrows-in and -out,
but make sure it's sound and clear.
\end{staffnotes}

\bparts

\ppart\label{surjsurj} Prove that if $A \surj B$ and $B \surj C$, then
$A \surj C$.

\begin{solution}
By the $\surj$ relations, there is at least one arrow
that goes into each element of $C$, originating out of an element in $B$. Similarly,
there is at least one arrow that goes into each element of $B$,
originating out of an element in $A$. Therefore, for each element in $C$, there
exists at least one pair of arrows, that go out of an element in $A$, through an
element in $B$, and go into the desired element in $C$. This shows the 
$[\geq 1\ \text{in}]$ relation from $A$ to $C$. Also, there is at most $1$ arrow
coming out from each element in $A$, since $A \surj B$, so the $[\leq 1\ \text{out}]$ 
relation also holds from $A$ to $C$. Thus, $A \surj C$.
\end{solution}

\ppart\label{surjinj} Explain why $A \surj B$ iff $B \inj A$.

\begin{solution}
\begin{proof}
(right to left): Since $B \inj A$, at least one arrow goes out from each element in
$B$, and at most one arrow goes into each element in $A$. However, reversing directions,
this means that at most one arrow goes out of each element in $A$, and at least one
arrow goes into each element in $B$. Thus, $A \surj B$

(left to right): Since $A \surj B$, at most one arrow goes out of each element in $A$,
and at least one arrow goes into each element in $B$. Again, reversing directions means
that at least one arrow goes out from each element in $B$, and at most one arrow goes
into each element in $A$. Thus, $B \inj A$.
\end{proof}
\end{solution}

\ppart Conclude from~\eqref{surjsurj} and~\eqref{surjinj} that if $A \inj
B$ and $B \inj C$, then $A \inj C$.

\begin{solution}
From~\eqref{surjinj} and~\eqref{surjsurj} we have that if $C \inj B$ and
$B \inj A$, then $C \inj A$, so just switch the names $A$ and $C$.
\end{solution}

\ppart Explain why $A \inj B$ iff there is a total injective
\emph{function} ($[= 1\ \text{out}, \leq 1\ \text{in}]$) from $A$ to
$B$. \footnote{The official definition of $\inj$ is with a total
  injective \emph{relation} ($[\geq 1\ \text{out}, \leq 1\ \text{in}]$)}

\begin{solution}
Given a $[\geq 1\ \text{out}, \leq 1\ \text{in}]$ relation, just erase
all but one arrow wherever there is more than one arrow out of the
same domain element to get an $[= 1\ \text{out}, \leq 1\ \text{in}]$
relation.
\end{solution}

\eparts
\end{problem}

%%%%%%%%%%%%%%%%%%%%%%%%%%%%%%%%%%%%%%%%%%%%%%%%%%%%%%%%%%%%%%%%%%%%%
% Problem ends here
%%%%%%%%%%%%%%%%%%%%%%%%%%%%%%%%%%%%%%%%%%%%%%%%%%%%%%%%%%%%%%%%%%%%%

\endinput
