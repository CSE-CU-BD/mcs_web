\documentclass[problem]{mcs}

\begin{pcomments}
  \pcomment{PS_equivalence_relation_operations_part_b}
  \pcomment{slightly modified by Jenn Wong 3/29/13 from
    PS_equivalence_relation_operations}
\end{pcomments}

\pkeywords{
  equivalence_relations
}

%%%%%%%%%%%%%%%%%%%%%%%%%%%%%%%%%%%%%%%%%%%%%%%%%%%%%%%%%%%%%%%%%%%%%
% Problem starts here
%%%%%%%%%%%%%%%%%%%%%%%%%%%%%%%%%%%%%%%%%%%%%%%%%%%%%%%%%%%%%%%%%%%%%

\begin{problem}
Give an example of two equivalence relations $R_1$ and $R_2$ on the
set $\set{1, 2, 3}$ such that $R_1 \cup R_2$ is \textbf{not} an
equivalence relation.  Briefly explain why.

\iffalse
Remember that an equivalence relation must be reflexive, transitive,
and symmetric.
\fi

\begin{solution}
Let $R_1$ and $R_2$ be the relations on $\set{1, 2, 3}$ where
\begin{align*}
\graph{R_1} \eqdef & \set{(1,1) (2,2) (3,3) (1,2) (2,1)},\\
\graph{R_2} \eqdef & \set{(1,1) (2,2) (3,3) (2,3) (3,2)}.
\end{align*}
It's easy to check that $R_1$ and $R_2$ are both equivalence
relations.  But $R_1\cup R_2$ is not transitive, because $(1,2),(2,3)
\in \graph{R_1 \union R_2}$ but $(1,3) \notin \graph{R_1 \union R_2}$.
Therefore $R_1 \union R_2$ is not an equivalence relation.

\iffalse
\vspace{5mm} We can also write a proof instead of providing a
counterexample.  Specifically, we can conclude that transitivity (if
$a\mrel{R}b$ and $b\mrel{R}c$, then $a\mrel{R}c$) can be violated when
$(a, b)$ belongs only to $R_1$ and $(b, c)$ belongs only to
$R_2$.  Then, it is not necessarily true that $a\mrel{R}c$.
\fi

\end{solution}

\end{problem}


%%%%%%%%%%%%%%%%%%%%%%%%%%%%%%%%%%%%%%%%%%%%%%%%%%%%%%%%%%%%%%%%%%%%%
% Problem ends here
%%%%%%%%%%%%%%%%%%%%%%%%%%%%%%%%%%%%%%%%%%%%%%%%%%%%%%%%%%%%%%%%%%%%%

\endinput
