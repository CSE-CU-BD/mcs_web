\documentclass[problem]{mcs}

\begin{pcomments}
  \pcomment{FP_token_state_machine_conflict}
  \pcomment{minor variant of FP_token_state_machine}
  \pcomment{F15. mid2conflict}
  \pcomment{author: Zoran Dzunic, edited ARM 10/15/15}
\end{pcomments}

\pkeywords{
  state_machines
  derived_variable
  induction
  congruences
}

%%%%%%%%%%%%%%%%%%%%%%%%%%%%%%%%%%%%%%%%%%%%%%%%%%%%%%%%%%%%%%%%%%%%%
% Problem starts here
%%%%%%%%%%%%%%%%%%%%%%%%%%%%%%%%%%%%%%%%%%%%%%%%%%%%%%%%%%%%%%%%%%%%%

\begin{problem}
  ``Token replacing'' is a single player game using a set of tokens,
  each colored black or white.  Except for color, the tokens are
  indistinguishable.  %The game starts with one black token.
  In each move, a player can replace one black token with two
  white tokens, or replace one white token with two black tokens.

% Zoran: I somehow prefer to introduce the start state here, as it may
% be hinting too much if introduced in part (b). Also, I came up
% with another part of the problem for which it may be better to
% have the start state fixed as part of rules.

  We can model this game as a state machine whose states are pairs
  $(n_b,n_w)$ where $n_b \geq 0$ equals the number of black tokens,
  and $n_w \geq 0$ equals the number of white tokens.
  
 Let us assume that the game starts with one black token, that is,
 state $(1,0)$.

\bparts

\ppart\label{Tnbnwinvar} Define the predicate $T(n_b,n_w)$ by the rule:
\[
T(n_b,n_w) \eqdef\  [\remainder(n_w - n_b, 3) = 2].
\]
Prove that $T$ is true for all reachable states.

\examspace[2.5in]

\begin{solution}
$T$ is true for the start state since $n_w - n_b = 0-1 = -1$ and
  $\remainder(-1,3) = 2$.\inhandout{\footnote{Remainders are always
      between zero and the divisor minus one.  The quotient of here
      would be -1, so $3\cdot \qcnt{-1}{3} + \rem{-1}{3} = -1$.}}

Next we show that $T$ is a preserved invariant.

So suppose $T(n_b, n_w)$ holds.  The only two possible transitions are
into a state $(n'_b,n'_w)$ equal to $(n_b - 1, n_w + 2)$ or to $(n_b +
2, n_w - 1)$.  But since
\[
n'_w - n'_b = n_w - n_b \pm 3
\]
has the same remainder on division by three as $n_w - n_b$, conclude
that $T(n'_b,n'_w)$, which proves that $T$ is invariant.

Since $T$ is a preserved invariant that is true for the start state,
it is true for all reachable states by the invariant principle.
\end{solution}

\ppart Does the same hold for the predicate 
\[
\rem{n_b - n_w}{3} = 2?
\]
Explain.

\examspace[1.5in]
%\examspace

\begin{solution}
No. This predicate is not true for the start state.
\end{solution}

\ppart
We will now prove the following:
\begin{claim*}
If $T(n_b, n_w)$, then state $(n_b, n_w)$ is reachable.
\end{claim*}

Note that this claim is entirely different from the result in
part~\eqref{Tnbnwinvar} that $T$ is a preserved invariant.  A
preserved invariant satified by the start state tells you what
\emph{can't} be reached, namely, any state that does not satisfy the
invariant.  It does not tell you what \emph{can} be reached.

The proof of the Claim will be by induction in $n$ using induction
hypothesis $P(n) \eqdef$
\[
 \forall (n_b, n_w).\,
 [(n_b + n_w = n) \QAND  T(n_b,n_w)]
     \QIMPLIES (n_b,n_w) \text{ is reachable}.
\]

The base cases will be when $n \leq 2$.  

\begin{itemize}

\item Assuming that the base cases have been verified, complete the
  \textbf{Inductive Step}.

\examspace[3.5in]

\begin{solution}
\begin{proof}
Assume that the induction hypothesis holds for some $n \ge 2$.
Suppose $n_b + n_w = n+1$ and $T(n_b,n_w)$ holds.  We want to show
that $(n_b, n_w)$ is reachable.

Since $n+1 \geq 3$, either $n_b \geq 2$ or $n_w \geq 2$.

In the case that $n_b \geq 2$, we have $n_b-2 \geq 0$, so
$(n_b-2,n_w+1)$ is a state. Also, $T(n_b-2,n_w+1)$ holds because
\[
(n_b-2)-(n_w+1) = n_b-n_w -3,
\]
which has the same remainder on division by three as $n_b-n_w$.

Since
\[
(n_b-2)+(n_w+1) = n_b+n_w - 1 = n,
\]
we conclude by induction hypothesis $P(n)$ that $(n_b-2,n_w+1)$ is
reachable.  But $(n_b-2,n_w+1)$ transitions in one step to
$(n_b,n_w)$, which proves that $(n_b,n_w)$ is reachable.

The same argument applies in the case that $n_w \geq 2$.

We conclude that in any case $(n_b,n_w)$ is reachable, which completes
the induction step.
\end{proof}
\end{solution}

\item Now verify the \textbf{Base Cases}: $P(n)$ for $n \leq 2$.

\examspace[3.5in]

\begin{solution}
There are only six states with $n \leq 2$:
\[
(0,0), (1,0), (0,1), (1,1), (0,2), (2,0).
\]
Of these, only $T(1,0)$ and $T(0,2)$ hold. $(1,0)$ is reachable
since it is the start state, and $(0,2)$ is reachable in one step from
the start state.  So all the states with $n \leq 2$ and satisfying
property $T$ are reachable.
\end{solution}

\end{itemize}

\eparts
\end{problem}

%%%%%%%%%%%%%%%%%%%%%%%%%%%%%%%%%%%%%%%%%%%%%%%%%%%%%%%%%%%%%%%%%%%%%
% Problem ends here
%%%%%%%%%%%%%%%%%%%%%%%%%%%%%%%%%%%%%%%%%%%%%%%%%%%%%%%%%%%%%%%%%%%%%

\endinput
