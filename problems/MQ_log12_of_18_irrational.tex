\documentclass[problem]{mcs}

\begin{pcomments}
  \pcomment{MQ_log12_of_18_irrational}
  \pcomment{variant of PS_log2_of_3_irrational}
  \pcomment{for MQ 9-10-14}
\end{pcomments}

\pkeywords{
  irrational
  power 
  contradiction
  log
}

%%%%%%%%%%%%%%%%%%%%%%%%%%%%%%%%%%%%%%%%%%%%%%%%%%%%%%%%%%%%%%%%%%%%%
% Problem starts here
%%%%%%%%%%%%%%%%%%%%%%%%%%%%%%%%%%%%%%%%%%%%%%%%%%%%%%%%%%%%%%%%%%%%%

\begin{problem} Prove that $\log_{12} 18$ is irrational.
%\hint Proof by contradiction.

\begin{solution}
\begin{proof}
  Suppose to the contrary that
  \[
  \textcolor{red}{\log_{12} 18 = \frac{m}{n}}
  \]
  for some integers $m, n$ where $n>0$.  So we have
\begin{align}
%  \log_9 12 & = m/n, \notag\\
  12^{\log_{12} 18} & = 12^{m/n}  & \text{(raising 12 to equal powers)},\notag\\
  18 & = 12^{m/n} & \text{(def of $\log_{12}$)},\notag\\
  18^n & = 12^m & \text{(raising both sides to the $n$th power)}.\notag\\
  (2\cdot 3^{2})^n & = (2^2 \cdot 3)^m & \text{(factoring 18 \& 12 into primes)}.\notag\\
  2^n\cdot 3^{2n} & = 2^{2m}\cdot 3^m. \label{2n2mfactoring}
\end{align}
By the uniqueness of prime factorization, the powers of primes on the
two sides of equation~\eqref{2n2mfactoring} must be equal.  That is,
\[
n  = 2m \quad \text{and} \quad 2n = m.
\]
This is only possible if $m=n=0$, a contradiction.

\iffalse

Now we have two cases:

\inductioncase{Case 1}: $(n \neq 2m)$.  There are different numbers of
two's on the left and right hand sides of equation~\eqref{factoring},
which contradicts the Unique Factorization Theorem.

\inductioncase{Case 2}: $(n = 2m)$.  There are $4m$ three's on the
left hand side of~\eqref{factoring} and $m$ three's on the right
hand side, which contradicts the Unique Factorization Theorem.

In any case there is a contradiction, which implies that $\log_{12}
18$ must be irrational.

\medskip

An alternative to the argument by cases is to assume (for the sake of
contradiction) that $\textcolor{red}{n = 2m}$.  Then
$\textcolor{red}{\log_{12} 18 = 1/2}$, which is false (since $12^{1/2}
\neq 18$).  This implies that Case 1 is in fact the only possibility.
\fi

\end{proof}
\end{solution}
\end{problem}

%%%%%%%%%%%%%%%%%%%%%%%%%%%%%%%%%%%%%%%%%%%%%%%%%%%%%%%%%%%%%%%%%%%%%
% Problem ends here
%%%%%%%%%%%%%%%%%%%%%%%%%%%%%%%%%%%%%%%%%%%%%%%%%%%%%%%%%%%%%%%%%%%%%

\endinput
