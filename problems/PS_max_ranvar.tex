\documentclass[problem]{mcs}

\begin{pcomments}
  \pcomment{PS_max_ranvar}
  \pcomment{concise PS version of CP_max_ranvar_n}
  \pcomment{generalizes TP_3_random_variables}
  \pcomment{ARM 5/1/12, edited 11/27/13}
\end{pcomments}

\pkeywords{
  random_variables
  mutually_independent
  density
  uniform
}

%%%%%%%%%%%%%%%%%%%%%%%%%%%%%%%%%%%%%%%%%%%%%%%%%%%%%%%%%%%%%%%%%%%%%
% Problem starts here
%%%%%%%%%%%%%%%%%%%%%%%%%%%%%%%%%%%%%%%%%%%%%%%%%%%%%%%%%%%%%%%%%%%%%


\begin{problem}
Suppose $R$, $S$, and $T$ are mutually independent random
variables on the same probability space with uniform distribution on
the range $[1,n]$.
\begin{staffnotes}
That is,
\[
\pr{U=i} = \frac{1}{n} \qquad\text{ for } U=R,S,T \text{ and } i \in [1,n].
\]

\end{staffnotes}

Let $M=\max \set{R,S,T}$.

\bparts

\ppart\label{Mleqkin1n}
Write a simple formula for $\pr{M \leq k}$ where $k \in [1,n]$.

\begin{solution}
\[
\paren{\frac{k}{n}}^3.
\]

To prove this, note that $M \leq k$ iff $R \leq k \QAND S \leq k \QAND
T \leq k$.  Since $R$ is uniform, $\pr{R} \leq k = k/n$, and likewise
for $S$ and $T$, and by mutual independence,
\[
\pr{R \leq k \QAND S \leq k \QAND T \leq k} = \pr{R \leq k} \cdot
\pr{S \leq k} \cdot \pr{T \leq k}.
\]
\end{solution}

\ppart Write a simple formula for $\pdf_M(k)$.
\hint part~\eqref{Mleqkin1n}

\begin{staffnotes}
\hint $M=k$ iff $M\leq k \QAND \QNOT(M \leq k-1)$.
\end{staffnotes}

\begin{solution}
\begin{align}
\pdf_M(1) & = \paren{\frac{1}{n}}^3\notag \\
\pdf_M(k) & = \paren{\frac{k}{n}}^3 - \paren{\frac{k-1}{n}}^3
               &\text{ for } k \in (1,n].\label{pdfmkkn3}
\end{align}

This follows because 
\[
\pr{M \leq k} = \pr{M=k} + \pr{M \leq k-1}
\]
by the Disjoint Sum Rule, and so
\[
\pr{M = k} = \pr{M \leq k} - \pr{M \leq k-1}
\]
which equals~\eqref{pdfmkkn3} by part~\eqref{Mleqkin1n}.
\end{solution}

\eparts

\end{problem}

%%%%%%%%%%%%%%%%%%%%%%%%%%%%%%%%%%%%%%%%%%%%%%%%%%%%%%%%%%%%%%%%%%%%%
% Problem ends here
%%%%%%%%%%%%%%%%%%%%%%%%%%%%%%%%%%%%%%%%%%%%%%%%%%%%%%%%%%%%%%%%%%%%%

\endinput
