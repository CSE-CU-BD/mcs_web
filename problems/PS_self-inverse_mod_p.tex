\documentclass[problem]{mcs}

\begin{pcomments}
  \pcomment{from: S09.ps7}
  \pcomment{from: OCW 05}
\end{pcomments}

\pkeywords{
  number_theory
  modular_arithmetic
  primes
}

%%%%%%%%%%%%%%%%%%%%%%%%%%%%%%%%%%%%%%%%%%%%%%%%%%%%%%%%%%%%%%%%%%%%%
% Problem starts here
%%%%%%%%%%%%%%%%%%%%%%%%%%%%%%%%%%%%%%%%%%%%%%%%%%%%%%%%%%%%%%%%%%%%%

\begin{problem} Suppose that $p$ is a prime and $0<k<p$.

\bparts

\ppart $k$ is \emph{self-inverse} if $k^2 \equiv 1 \pmod {p}$.  Prove that
$k$ is self-inverse iff either $k = 1$ or $k = p - 1$.

\hint $k^2-1 = (k-1)(k+1)$

\begin{solution}
By definition of $\equiv \pmod {p}$, the integer $k$ is
self-inverse iff $p \divides k^2 - 1$.  But $k^2-1 = (k-1)(k+1)$, and
since $p$ is a prime, we conclude that either $p \divides k-1$ or $p
\divides k+1$.  But $0 < k < p$, so $p \divides k-1$ iff $k-1=0$, and $p
\divides k+1$ iff $k+1 = p$, so we must have $k = 1$ or $k = p - 1$.

Conversely, $1 \cdot 1 \equiv 1 \pmod{p}$ and $(p-1)\cdot (p-1) = p^2+2p+1
\equiv 1 \pmod{p}$.
\end{solution}

\ppart Wilson's Theorem asserts
\begin{theorem}[Wilson's Theorem]
If $p$ is a prime, then
\[
(p - 1)!\  \equiv  -1 \pmod{p}
\]
\end{theorem}

The English mathematician Edward Waring said that this theorem would
probably be very difficult to prove because there was no adequate notation
for primes.  Gauss proved it while standing (on one foot, it is rumored).
He suggested that Waring failed for lack of notions, not notations.  Prove
Wilson's Theorem.  \hint While standing on one foot, think about pairing
each term in $(p-1)!$ with its multiplicative inverse.

\begin{solution}
If $p = 2$, then the theorem holds, because $1 \equiv -1
\pmod{2}$.  If $p > 2$, then $p - 1$ and $1$ are distinct terms in the
product $1 \cdot 2 \cdot \cdots (p - 1)$, and these are the only
self-inverses.  Consequently, we can pair each of the remaining terms
with its multiplicative inverse.  Since the product of a number and
its inverse is congruent to 1, all of these remaining terms cancel.
Therefore, we have:

\begin{align*}
(p-1)!  & \equiv  1 \cdot (p - 1) \pmod{p} \\
        & \equiv  -1 \pmod{p}
\end{align*}
	
\end{solution}

\eparts

\end{problem}

%%%%%%%%%%%%%%%%%%%%%%%%%%%%%%%%%%%%%%%%%%%%%%%%%%%%%%%%%%%%%%%%%%%%%
% Problem ends here
%%%%%%%%%%%%%%%%%%%%%%%%%%%%%%%%%%%%%%%%%%%%%%%%%%%%%%%%%%%%%%%%%%%%%
