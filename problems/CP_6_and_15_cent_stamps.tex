\documentclass[problem]{mcs}

\begin{pcomments}
  \pcomment{from: S09.cp4t, S08.cp4m}
\end{pcomments}

\pkeywords{
  well-ordering
  WOP
  postage_stamps
}

%%%%%%%%%%%%%%%%%%%%%%%%%%%%%%%%%%%%%%%%%%%%%%%%%%%%%%%%%%%%%%%%%%%%%
% Problems start here
%%%%%%%%%%%%%%%%%%%%%%%%%%%%%%%%%%%%%%%%%%%%%%%%%%%%%%%%%%%%%%%%%%%%%

\begin{problem}
  The proof below uses the Well Ordering Principle to prove that every
  amount of postage that can be paid exactly using only 6 cent and 15 cent
  stamps, is divisible by 3.  Let the notation ``$j \divides k$'' indicate
  that integer $j$ is a divisor of integer $k$, and let $S(n)$ mean that
  exactly $n$ cents postage can be paid using only 6 and 15 cent
  stamps.  Then the proof shows that
%
\begin{equation}\tag{*}
S(n)\ \QIMPLIES\ 3 \divides n, \quad \text{for all nonnegative integers $n$}.
\end{equation}
Fill in the missing portions (indicated by ``\dots'') of the following
proof of~(*).

\begin{quote}
Let $C$ be the set of \emph{counterexamples} to~(*), namely\footnote{The
  notation ``$\set{n \suchthat \dots}$ means ``the set of elements, $n$,
  such that \dots.''}

\[
C \eqdef \set{n \suchthat \dots}
\]

\begin{solution}
 $n$ is a counterexample to~(*) if $n$ cents postage can be
  made and $n$ is not divisible by 3, so the predicate
\[
S(n)\text{ and } \QNOT(3 \divides n)
\]
defines the set, $C$, of counterexamples.
\end{solution}

Assume for the purpose of obtaining a contradiction that $C$ is
nonempty.  Then by the WOP, there is a smallest number, $m \in C$.
This $m$ must be positive because\dots.

\begin{solution}
\dots $3 \divides 0$, so 0 is not a counterexample.
\end{solution}

But if $S(m)$ holds and $m$ is positive, then $S(m-6)$ or $S(m-15)$
must hold, because\dots.

\begin{solution}
\dots if $m>0$ cents postage is made from 6 and 15 cent
  stamps, at least one stamp must have been used, so removing this
  stamp will leave another amount of postage that can be made.
\end{solution}

So suppose $S(m-6)$ holds.  Then $3 \divides (m-6)$, because\dots
\begin{solution}
\dots if $\QNOT(3 \divides (m-6))$, then $m-6$ would be
  a counterexample smaller than $m$, contradicting the minimality of
  $m$.
\end{solution}

But if $3 \divides (m-6)$, then obviously $3 \divides m$,
contradicting the fact that $m$ is a counterexample.

Next suppose $S(m-15)$ holds.  Then the proof for $m-6$ carries over
directly for $m-15$ to yield a contradiction in this case as well.
Since we get a contradiction in both cases, we conclude that\dots

\begin{solution}
\dots $C$ must be empty.  That is, there are no
  counterexamples to~(*),
\end{solution}

which proves that (*) holds.

\end{quote}
\end{problem}

%%%%%%%%%%%%%%%%%%%%%%%%%%%%%%%%%%%%%%%%%%%%%%%%%%%%%%%%%%%%%%%%%%%%%
% Problems end here
%%%%%%%%%%%%%%%%%%%%%%%%%%%%%%%%%%%%%%%%%%%%%%%%%%%%%%%%%%%%%%%%%%%%%
