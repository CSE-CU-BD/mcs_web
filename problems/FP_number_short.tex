\documentclass[problem]{mcs}

\begin{pcomments}
  \pcomment{FP_number_short}
  \pcomment{Giuliano, May 2012, edited ARM 5/20/12}
\end{pcomments}

\pkeywords{
  number_theory
  gcd
  prime
  relatively_prime
  cancel
}

%%%%%%%%%%%%%%%%%%%%%%%%%%%%%%%%%%%%%%%%%%%%%%%%%%%%%%%%%%%%%%%%%%%%%
% Problem starts here
%%%%%%%%%%%%%%%%%%%%%%%%%%%%%%%%%%%%%%%%%%%%%%%%%%%%%%%%%%%%%%%%%%%%%

\begin{problem}
Give counterexamples for each of the false statements below.

\bparts
\ppart For integers $a$ and $b$ there are integers $x$ and $y$ such that:
$ax + by = 1$

\begin{solution}
FALSE.  $a$ and $b$ must be relatively prime.  $a=b=2$ is a
counterexample.
\end{solution}

\ppart $\gcd(mb+r, b) = \gcd(r,b)$ for all integers $m,r$ and $b$.

\begin{solution}
TRUE.
\end{solution}

\ppart For every prime $p$ and every integer $k$, $k^{p-1} \equiv 1 \pmod p$.

\begin{solution}
FALSE.  $k$ must be relatively prime to $p$.  $k=p=2$ is a counterexample.
\end{solution}

\ppart For primes $p \neq q$, $\phi(pq) = (p-1)(q-1)$, where $\phi$ is
Euler's totient fucntion.

\begin{solution}
TRUE.
\end{solution}

\ppart Suppose $a,b,c,d \in \naturals$ and $a$ and $b$ are relatively prime to $d$.  Then
\[
[ac \equiv bc \bmod d]\quad \QIMP\quad [a \equiv b \bmod d].
\]

\begin{solution}
FALSE.  To cancel $c$, we need that $c$ is relatively prime to $d$.

A counterexample is $a=1, b=2$ and $d=3$, and $c=0$.
\end{solution}

\eparts
\end{problem}

\endinput
