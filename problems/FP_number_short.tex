\documentclass[problem]{mcs}

\begin{pcomments}
  \pcomment{FP_number_short}
  \pcomment{Giuliano, May 2012}
\end{pcomments}

\pkeywords{
  number_theory
  gcd
  prime
}

%%%%%%%%%%%%%%%%%%%%%%%%%%%%%%%%%%%%%%%%%%%%%%%%%%%%%%%%%%%%%%%%%%%%%
% Problem starts here
%%%%%%%%%%%%%%%%%%%%%%%%%%%%%%%%%%%%%%%%%%%%%%%%%%%%%%%%%%%%%%%%%%%%%

\begin{problem}
For the following statements circle T for True or F for False.

(a) For integers $a$ and $b$ there are integers $x$ and $y$ such that:
$ax + by = 1$

(b) $\gcd(mb+r, b) = \gcd(r,b)$ for all integers $m,r$ and $b$.

(c) For every prime $p$ and every integer $k$, $k^{p-1} \equiv 1 \pmod p$

(d) For primes $p \not= q$, $\phi(pq) = (p-1)(q-1)$

(e) Let $a,b c and d$ be integers and $\gcd(a,b) = c$. If $a \equiv b (mod d)$ then $ a/c \equiv b/c (mod d)$.

\begin{solution}

(a) FALSE. $a$ and $b$ must be relative primes.

(b) TRUE.

(c) FALSE. $k$ must be relative prime with $p$.

(d) TRUE.

(e) FALSE. Let $a=4, b=6$ and $d=2$, then $\gcd(a,b) = \gcd(4,6)
=2$. Thus  $a/2 = 2 \not\equiv b/2 = 3 (mod 2)$.

\end{solution}
\end{problem}

\endinput
