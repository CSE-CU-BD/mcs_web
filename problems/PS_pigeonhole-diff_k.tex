\documentclass[problem]{mcs}

\begin{pcomments}
  \pcomment{PS_pigeonhole-diff_k}
  \pcomment{generalizes PS_pigeonhole-diff_2}
  \pcomment{ZDz, 11/24/15, F15.mid4.conflictsecond}
\end{pcomments}

\pkeywords{
  pigeonhole
  difference
}

%%%%%%%%%%%%%%%%%%%%%%%%%%%%%%%%%%%%%%%%%%%%%%%%%%%%%%%%%%%%%%%%%%%%%
% Problem starts here
%%%%%%%%%%%%%%%%%%%%%%%%%%%%%%%%%%%%%%%%%%%%%%%%%%%%%%%%%%%%%%%%%%%%%

\begin{problem}

Suppose $2n+1$ numbers are selected from $\set{1,2,3, \dots,4n}$.
Using the Pigeonhole Principle, show that for any positive integer $j$
that divides $2n$, there must be two selected numbers whose difference
is $j$.  Clearly indicate what are the pigeons, holes, and rules for
assigning a pigeon to a hole.

\begin{solution}
  Partition $\set{1,2,3, \dots,2n}$ into 2-element sets of the form $\set{k,k+j}$.
  You should convince yourself that there is only way one to do this, namely,
\begin{align*}
\set{1, 1+j},\set{2, 2+j}, \dots, \set{j, j+j},\\
\set{2j+1,3j+1},\set{2j+2, 3j+2},\dots, \set{2j+j,3j + j}\\
\set{4j+1,5j+1},\set{4j+2,5j+2},\dots, \set{4j+j,5 + j}\\
\set{5j+1,6j+1},\set{5j+2,6j+2},\dots, \set{5j+j,6j + j}\\
\set{7j+1,8j+1},\set{7j+2,8j+2},\dots, \set{7j+j,8j + j}\\
\qquad \vdots\\
\set{4n-2j+1, 4n-j+1},\set{4n-2j+2, 4n-j+2},\dots \set{4n-2j+j,4n}.
\end{align*}

  The $2n+1$ selected numbers are the pigeons.  The pigeonholes will
  be the $2n$ sets of the form $\set{k,k+j}$ in the partition.  Assign
  a pigeon to the block of the partition that contains it.  Since
  there are more pigeons than holes, two pigeons must be assigned to
  some hole, which means their difference is $k$.
\end{solution}

\end{problem}

\endinput
