\documentclass[problem]{mcs}

\begin{pcomments}
  \pcomment{FP_conditional_prob_inequality}
  \pcomment{resembles MQ_conditional_prob_inequality}
  \pcomment{by Meyer 10/10/09}
  \pcomment{qualitative version of CP_missing_card_probability}
  \pcomment{last part omitted for F11 final}
\end{pcomments}

\pkeywords{
  conditional_probability
  probability
  inequality
  Bayes
}

%%%%%%%%%%%%%%%%%%%%%%%%%%%%%%%%%%%%%%%%%%%%%%%%%%%%%%%%%%%%%%%%%%%%%
% Problem starts here
%%%%%%%%%%%%%%%%%%%%%%%%%%%%%%%%%%%%%%%%%%%%%%%%%%%%%%%%%%%%%%%%%%%%%

\begin{problem}
There are two decks of cards, the red deck and the blue deck.  They differ
slightly in a way that makes drawing the eight of hearts slightly more
likely from the red deck than from the blue deck.

One of the decks is randomly chosen and hidden in a box.  You reach in
the box and randomly pick a card that turns out to be the eight of
hearts.  You believe intuitively that this makes the red deck more
likely to be in the box than the blue deck.

Your intuitive judgment about the red deck can be formalized and
verified using some inequalities between probabilities and conditional
probabilities involving the events
\begin{align*}
  R & \eqdef \text{\textbf{R}ed deck is in the box},\\
  B & \eqdef \text{\textbf{B}lue deck is in the box},\\
  E & \eqdef \text{\textbf{E}ight of hearts is picked from the deck in
    the box}.
\end{align*}

\bparts

\ppart\label{ppEREB} State an inequality between probabilities and/or conditional
probabilities that formalizes the assertion, ``picking the eight of
hearts from the red deck is more likely than from the blue deck.''

\begin{center}
\exambox{1.5in}{0.75in}{-0.2in}
\end{center}
%\examspace[0.2in]

\begin{solution}
\begin{equation}\label{EREB}
\prcond{E}{R} > \prcond{E}{B}.
\end{equation}
\end{solution}

\ppart\label{ppREBE} State a similar inequality that formalizes the assertion
``picking the eight of hearts from the deck in the box makes the red
deck more likely to be in the box than the blue deck.''

\begin{center}
\exambox{1.5in}{0.75in}{-0.2in}
\end{center}
%\examspace[0.2in]

\begin{solution}

\begin{equation}%\label{REBE}
\prcond{R}{E} >  \prcond{B}{E}.
\end{equation}

\end{solution}

\ppart Assuming the each deck is equally likely to be the one in the
box, prove that the inequality of part~\eqref{ppEREB} implies the
inequality of part~\eqref{ppREBE}.

\examspace[3.0in]

\begin{solution}
From~\eqref{EREB} and the definition of conditional probability,
\[
\frac{\pr{E \QAND R}}{\pr{R}} > \frac{\pr{E \QAND B}}{\pr{B}}.
\]
Also, $\pr{R} = \pr{B} = 1/2$ by assumption.  This implies
\[
\pr{E \QAND R} > \pr{E \QAND B}.
\]
Dividing both sides of this inequality by $\pr{E}$ completes the proof:
\[
\prcond{R}{E} \eqdef \frac{\pr{E \QAND R}}{\pr{E}} >
                     \frac{\pr{E \QAND B}}{\pr{E}} = \prcond{B}{E}.
\]
\end{solution}


\ppart Suppose you couldn't be sure that the red deck and blue deck were
equally likely to be in the box.  Could you still conclude that picking
the eight of hearts from the deck in the box makes the red deck more
likely to be in the box than the blue deck?  Briefly explain.

\examspace[2in]

\begin{solution}
No.  In the extreme case, suppose the red deck consisted solely of the
eight of hearts, but an adversary \emph{almost never} chose to put the
red deck in the box.  Also, suppose there was a reasonable
possibility, say 1 in 10, of drawing the eight of hearts from the blue
deck.  Then picking the eight of hearts is ten times more likely from
the red deck than from the blue deck, but the probability that the
blue deck is in the box remains nearly certain.
\end{solution}

\eparts

\end{problem}

%%%%%%%%%%%%%%%%%%%%%%%%%%%%%%%%%%%%%%%%%%%%%%%%%%%%%%%%%%%%%%%%%%%%%
% Problem starts here
%%%%%%%%%%%%%%%%%%%%%%%%%%%%%%%%%%%%%%%%%%%%%%%%%%%%%%%%%%%%%%%%%%%%%

\endinput
