\documentclass[problem]{mcs}

\begin{pcomments}
    \pcomment{MQ_relprime_closed_short}
    \pcomment{part of MQ_relprime_closed}
    \pcomment{Drew 3/6/12, edited ARM}
\end{pcomments}

\pkeywords{
  modular_arithmetic
  relatively_prime
  inverse
  cancellable
}

\begin{problem}
Prove that if $k_1$ and $k_2$ are relatively prime to $n$, then so is
$k_1 \cdot k_2$.  (You may assume any of the results from the text or
class problems.  There are many ways to do this.)

\begin{solution}

\dots using the \emph{Unique Factorization Theorem}.

By Unique Factorization, the prime divisors of $k_1 \cdot k_2$ are the
same as the prime divisors of $k_1$ along with the prime divisors of
$k_2$.  If $k_1$ and $k_2$ are relatively prime to $n$, they have no
prime divisors in common with $n$, so neither does $k_1k_2$, that is, 
$k_1k_2$ is relatively prime to $n$. 

\medskip

\dots using the fact that \emph{$k$ is relatively prime to $n$ iff
$k$ has an inverse modulo $n$.}

If $j_1$ is an inverse of $k_1$ modulo $n$, that is
\[
j_1k_1 \equiv 1 \pmod{n},
\]
and likewise $j_2$ is an inverse of $k_2$, then it follows immediately that
\[
(j_2j_1)(k_1k_2) \equiv 1 \pmod{n}.
\]
That is, $k_1k_2$ also has an inverse.  Since we know that $k_1k_2
\equiv k_1 \cdot_n k_2 \pmod{n}$, any inverse of $k_1k_2$ will also be
an inverse of $k_1 \cdot_n k_2$.

\medskip

\dots using the fact that \emph{$k$ is relatively prime to $n$ iff
$k$ is cancellable modulo $n$.}

If $k_1$ and $k_2$ are cancellable modulo $n$, then you can cancel
$k_1k_2$ by first cancelling $k_1$ and then cancelling $k_2$.
\end{solution}

\end{problem}

\endinput
