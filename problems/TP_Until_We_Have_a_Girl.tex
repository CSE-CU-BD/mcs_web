\documentclass[problem]{mcs}

\begin{pcomments}
    \pcomment{TP_Until_We_Have_a_Girl}
    \pcomment{Converted from geometric-distribution.scm
              by scmtotex and dmj
              on Sun 13 Jun 2010 04:49:03 PM EDT}
\end{pcomments}

\begin{problem}

%% type: short-answer
%% title: Until We Have a Girl

You have just been married and you both want to have children.  Of
course, any child is a blessing, but your spouse prefers girls, so you
decide to keep having children until you have a girl.  In other words,
if your 1st child is a girl, you'll stop there.  If it's a boy, then
you'll have a 2nd child, too. If that one is a girl, you'll stop
there. Otherwise, you'll have a 3rd child, and so on.  Assume that you
will never abandon this ingenious plan and that every child is equally
likely to be a boy or a girl, independently of the number of its
brothers so far.  Let $B$ be the $boys$ that you will eventually have
to put up with to enjoy the company of your beloved daughter.

\bparts

\ppart
For $i = 0,1,2,\dots$, what is the value of $\pdf_{B}(i)$?

\begin{description}

\item[a] $1/2^{i}$

\item[b] $1/2i$

\item[c] $1/2^{i+1}$

\item[d] $1/2^{2i}$

\item[e] can't tell

\end{description}

%% Enter the letter that corresponds to the correct expression:

\begin{solution}

c

By definition, $\pdf_{B}(i) = \Pr{B=i}$, namely the probablility that
you will have to have exactly $i$ boys. This happens exactly when the
first $i$ children are boys and the $i$+1st one is a girl. The
probability of this is
\begin{equation*}
    (1/2)^{i}(1/2) = (1/2)^{i+1}.
\end{equation*}
\end{solution}

\ppart
For $i = 0,1,2,\dots$, what is the value of $\cdf_{B}(i)$?

\begin{description}

\item[a] $( 1 - 1/2^{i} ) / ( 1 - 1/2 )$

\item[b] $2i$

\item[c] $1 - 1/2^{i+1}$

\item[d] $1/2i$

\item[e] can't tell

\end{description}

% Enter the letter that corresponds to the correct expression:

\begin{solution}

c


By definition, $\cdf_{B}(i) = \Pr{B\le i}$, namely the probablility
that you will have to have at most $i$ boys.  To compute this
probablity, we can just add the values of $\pdf_{B}(j)$ for
$j=0,1,2,\dots, i$.  This is
\begin{equation*}
\text{the sum of $1/2^{j+1}$ for $j=0,1,2,\dots, i$.}
\end{equation*}
Doing the algebra, we get $1 - 1/2^{i+1}$.

Alternatively, the probability that we get at most $i$ boys is 1 minus
the probability that we will get $i+1$ boys or more.  But we get $i+1$
or more boys if and only if the first $i+1$ children are boys, which
happens with probability $1/2^{i+1}$.  Subtracting from 1, we get the
same result.
\end{solution}

\ppart
How many boys should you expect to have? (\emph{hint}: no series, just
mean time to failure)

\begin{solution}

1

Since you stop when you have a girl, the number of boys in your family
will be one less than the number of children.  Thinking of having a
girl as a ``failure,'' the expected number of children is the mean
time to failure, namely $1/(1/2) = 2$.
\end{solution}

\eparts

\end{problem}

\endinput
