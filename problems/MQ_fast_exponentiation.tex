\documentclass[problem]{mcs}

\begin{pcomments}
  \pcomment{MQ_fast_exponentiation}
  \pcomment{lighter version of CP_fast_exponentiation}
\end{pcomments}

\pkeywords{
  state_machines
  termination
  partial_correctness
  invariant
  exponentiation
  algorithm
}

%%%%%%%%%%%%%%%%%%%%%%%%%%%%%%%%%%%%%%%%%%%%%%%%%%%%%%%%%%%%%%%%%%%%%
% Problem starts here
%%%%%%%%%%%%%%%%%%%%%%%%%%%%%%%%%%%%%%%%%%%%%%%%%%%%%%%%%%%%%%%%%%%%%

\begin{problem}
The \emph{\idx{Fast Exponentiaiton}} state machine is defined as
follows:

\begin{enumerate}
\item The set of states is $\reals \cross \reals \cross \nngint$,
\item The start state is $(a,1,b)$,
\item the transitions are defined by the rule
\begin{equation*}
(x,y,z) \movesto
\begin{cases}
(x^2, y, \quotient(z,2)) & \text{if $z$ is positive and even},\\
(x^2, xy, \quotient(z,2)) & \text{if $z$ is positive and odd}.
\end{cases}
\end{equation*}
\end{enumerate}

\bparts

\ppart Verify that the predicate $P(x,y,z) \eqdef\ [yx^z = a^b]$ is a
preserved invariant.

\examspace[3.5in]

\begin{solution}
We show that $P$ is preserved, namely, assuming $P(x,y,z)$, that is,
\begin{equation}\label{yxzd}
yx^z = a^b
\end{equation}
holds and $(x,y,z) \movesto (x',y',z')$ is a transition, then
$P(x',y',z')$, that is,
\[
y'x'^{z'} = a^b
\]
holds.

We consider two cases:

If $z > 0$ and is even, then we have that $x' = x^2, y' = y, z' =
\quotient(z,2)$.  Therefore,
\begin{align*}
y'x'^{z'} &  = yx^{2 \cdot \quotient(z,2)}\\
           & = yx^{2 \cdot (z/2)}\\
           & = yx^z\\
          & = a^b & \mbox{(by~\eqref{yxzd})}
\end{align*}

If $z > 0$ and is odd, then we have that $x' = x^2, y' = xy, z' =
\quotient(z,2)$. Therefore,
\begin{align*}
y'x'^{z'} & = xyx^{2 \cdot \quotient(z,2)}\\
& = yx^{1+2 \cdot (z-1)/2}\\
& = yx^{1+(z-1)}\\
& = yx^z\\
& = a^b & \mbox{(by~\eqref{yxzd})}
\end{align*}

So in both cases, $P(x',y',z')$ holds, proving that $P$ is a
preserved invariant.
\end{solution}

\ppart Prove that the algorithm is partially correct: if it stops, it
does so with $y=a^b$.

\begin{solution}
  $P$ holds for the start state $(a,1,b)$ since $1\cdot a^b = a^b$.
  So by the Invariant Principle~\bref{subsec:invariant}, $P$ holds for
  all reachable states.  But a stopped state must have $z = 0$, so if
  any stopped state $(x,y,0)$ is reachable, then $y = yx^0 = a^b$ as
  required.
\end{solution}

\eparts
\end{problem}

%%%%%%%%%%%%%%%%%%%%%%%%%%%%%%%%%%%%%%%%%%%%%%%%%%%%%%%%%%%%%%%%%%%%%
% Problem ends here
%%%%%%%%%%%%%%%%%%%%%%%%%%%%%%%%%%%%%%%%%%%%%%%%%%%%%%%%%%%%%%%%%%%%%

\endinput
