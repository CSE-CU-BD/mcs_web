\documentclass[problem]{mcs}

\begin{pcomments}
  \pcomment{from: S06.cp13w}
%  \pcomment{}
%  \pcomment{}
\end{pcomments}

\pkeywords{
Markov bound
}

%%%%%%%%%%%%%%%%%%%%%%%%%%%%%%%%%%%%%%%%%%%%%%%%%%%%%%%%%%%%%%%%%%%%%
% Problem starts here
%%%%%%%%%%%%%%%%%%%%%%%%%%%%%%%%%%%%%%%%%%%%%%%%%%%%%%%%%%%%%%%%%%%%%


\begin{problem}
A herd of cows is stricken by an outbreak of \emph{cold cow disease}.  The
disease lowers the normal body temperature of a cow, and a cow will die if
its temperature goes below 90 degrees F.  The disease epidemic is so
intense that it lowered the average temperature of the herd to $85$
degrees.  Body temperatures as low as $70$ degrees, \textbf{but no lower},
were actually found in the herd.

\bparts

\ppart Based solely on the information above, use Markov's bound to state
an upper bound on the probability that a randomly chosen cow from the herd
will have a high enough temperature to survive.  Try to make the bound as
small as possible.

\begin{solution}
Let $T$ be the temperature of a random cow.  Apply Markov's Bound to
$T-70$:
\[
\prob{T \geq 90} = \prob{T-70 \geq 20} \leq \expect{T-70}/20 = (85-70)/20 = 3/4.
\]
\end{solution}

%\vspace{3in}

\ppart Suppose there are 400 cows in the herd.  Give an example set of
temperatures for the cows so that the probability that a randomly chosen
cow will have a high enough temperature to survive is as large as
possible, and explain why it cannot be larger.

\begin{solution}Let 100 cows have 70 degrees and 300 have 90 degrees.  So the
probability that a random cow has a high enough temperature to survive is
exactly 3/4.  Also, the mean temperature is
\[
(1/4)70 + (3/4)90 = 85.
\]
So this distribution of temperatures satisfies the conditions under which
the Markov bound implies that the probability of having a high enough
temperature to survive cannot be larger than 3/4.
\end{solution}

%\vspace{2in}

\eparts

\end{problem}

%%%%%%%%%%%%%%%%%%%%%%%%%%%%%%%%%%%%%%%%%%%%%%%%%%%%%%%%%%%%%%%%%%%%%
% Problem ends here
%%%%%%%%%%%%%%%%%%%%%%%%%%%%%%%%%%%%%%%%%%%%%%%%%%%%%%%%%%%%%%%%%%%%%

\endinput
