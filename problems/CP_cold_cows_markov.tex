\documentclass[problem]{mcs}

\begin{pcomments}
  \pcomment{CP_cold_cows_markov}
  \pcomment{revised by ARM 12/5/09, with last part on prob space added
    May 18,2013}
  \pcomment{from: S06.cp13w}
\end{pcomments}

\pkeywords{
  average
  Markov_bound
  deviation
  sample_space
  outcome
  probability_space
}

%%%%%%%%%%%%%%%%%%%%%%%%%%%%%%%%%%%%%%%%%%%%%%%%%%%%%%%%%%%%%%%%%%%%%
% Problem starts here
%%%%%%%%%%%%%%%%%%%%%%%%%%%%%%%%%%%%%%%%%%%%%%%%%%%%%%%%%%%%%%%%%%%%%


\begin{problem}
A herd of cows is stricken by an outbreak of \emph{cold cow disease}.  The
disease lowers the normal body temperature of a cow, and a cow will die if
its temperature goes below 90 degrees F.  The disease epidemic is so
intense that it lowered the average temperature of the herd to $85$
degrees.  Body temperatures as low as $70$ degrees, \textbf{but no lower},
were actually found in the herd.

\bparts

\ppart\label{3/4cows} Use Markov's Bound\inbook{~\bref{markovthm}} to prove
that at most 3/4 of the cows could have survived.

%\hint Let $T$ be the temperature of a random cow.

\examspace[2in]

\begin{solution}
Namely, let $T$ be the temperature of a random cow.

\iffalse

Applying Markov's Bound to $T$:
  \[
  \prob{T \geq 90} \leq \frac{\expect{T}}{90} = \frac{85}{90} =
  \frac{17}{18}\, .
  \]
  But $17/18 > 3/4$, so this bound is not good enough.
\fi

We apply Markov's Bound to $T-70$:
\[
\prob{T \geq 90} = \prob{T-70 \geq 20} \leq \frac{\expect{T-70}}{20} = (85-70)/20 = 3/4.
\]
\end{solution}


\ppart Suppose there are 400 cows in the herd.  Show that the bound of
part~\eqref{3/4cows} is the best possible by giving an example set of
temperatures for the cows so that the average herd temperature is 85
and 3/4 of the cows will have a high enough temperature to survive.

\begin{solution}
  Let 100 cows have temperature 70 degrees and 300 have 90 degrees.
  So 300/400 =3/4 of the cows have a survival temperature.  Also, the
  mean temperature is
\[
(1/4)70 + (3/4)90 = 85.
\]
\end{solution}

\ppart Notice that the results above are purely arithmetic facts about
averages, not about probabilities.  But you verified the claim of
part~\eqref{3/4cows} by applying Markov's bound on the deviation of a
random variable.  Justify this approach by regarding the temperature,
$T$, of a cow as a random variable.  Carefully specify the probability
space on which $T$ is defined: what are the outcomes? what are their
probabilities?  Explain the precise connection between properties of $T$
and average herd temperature that justify application of Markov's Bound.

\begin{solution}
The sample space for $T$ is the set of cows in the herd, that is, each
cow is an outcome.  The probabilities are defined to be
\emph{uniform}---the probability of any cow, $c$, is $1/n$ where $n$
is the size of the herd---and $T(c)$ is the temperature of cow $c$.
Since the probabilities are uniform, it follows that

\begin{itemize}

\item the average temperature of the herd equals $\expect{T}$.

\item the fraction of cows with temperatures $\geq t$ is the
  probability that $T \geq t$.

\end{itemize}

So the fact that $\prob{T \geq 90} \leq 3/4$ implies that at most 3/4
of the herd could have survived.

\end{solution}
\eparts

\end{problem}

%%%%%%%%%%%%%%%%%%%%%%%%%%%%%%%%%%%%%%%%%%%%%%%%%%%%%%%%%%%%%%%%%%%%%
% Problem ends here
%%%%%%%%%%%%%%%%%%%%%%%%%%%%%%%%%%%%%%%%%%%%%%%%%%%%%%%%%%%%%%%%%%%%%

\endinput
