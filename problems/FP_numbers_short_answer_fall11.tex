\documentclass[problem]{mcs}

\begin{pcomments}
  \pcomment{FP_numbers_short_answer_fall11}
  \pcomment{overlaps FP_numbers_short_answer}
  \pcomment{ARM 12/16/11}
\end{pcomments}

\pkeywords{
  gcd
  modulo
  relatively_prime
  inverse
  mod_n
}

%%%%%%%%%%%%%%%%%%%%%%%%%%%%%%%%%%%%%%%%%%%%%%%%%%%%%%%%%%%%%%%%%%%%%
% Problem starts here
%%%%%%%%%%%%%%%%%%%%%%%%%%%%%%%%%%%%%%%%%%%%%%%%%%%%%%%%%%%%%%%%%%%%%

\begin{problem} \mbox{}

\textbf{\large Circle \textbf{true} or \textbf{false} for the statements below,
and \emph{provide counterexamples} for those that are \textbf{false}.}
Variables, $a,b,c,m,n$ range over the integers and $m,n>1$.

\bparts

% ~~~~~~~~~~~~~~~~~~~~~~~~~~~~~~~~~~~~~~~~~~~~~~~~~~~~~~~~~~~~~~~~~~~
% FROM: Spring07 MQ-3/14-5
%
% COMMENTS: Reduced number, DID NOT change statements, now asking for
% counterexample.
%

%NOT USED S11
\ppart If $a \equiv b \pmod{n}$, then $p(a) \equiv p(b) \pmod{n}$ for any
polynomial $p(x)$ with integer coefficients.
 \hfill \textbf{true} \qquad \textbf{false} \examspace[0.4in]
\begin{solution}
\textbf{true}
\end{solution}

\ppart  Assuming $a,b$ have inverses modulo $n$, if $a^{-1} \equiv
b^{-1} \pmod{n}$, then $a \equiv b \pmod{n}$.\hfill \textbf{true} \qquad
\textbf{false} \examspace[0.4in]

\begin{solution}
\textbf{true}
\end{solution}

\iffalse

%S11
\ppart $\gcd(1 + a, 1 + b) = 1 + \gcd(a, b)$.  \hfill 
\textbf{true} \qquad  \textbf{false} \examspace[0.4in]

\begin{solution}
false $a=1,b=2$
\end{solution}

%S11
\ppart If $a \divides b c$ and $\gcd(a, b) = 1$, then $a \divides c$.  \hfill 
\textbf{true} \qquad \textbf{false} \examspace[0.4in]

\begin{solution}
\textbf{true}
\end{solution}
\fi

%S11
\ppart $\gcd(a^n,b^n) =  (\gcd(a,b))^n$  \hfill 
\textbf{true} \qquad \textbf{false} \examspace[0.4in]

\begin{solution}
\textbf{true}.
\end{solution}

%S11
\ppart If $\gcd(a, b) \neq 1$ and $\gcd(b, c) \neq 1$, then $\gcd(a, c) \neq 1$. \hfill 
\textbf{true} \qquad  \textbf{false} \examspace[0.4in]

\begin{solution}
\textbf{false} $a=2\cdot 3, b=3\cdot 5, c=5\cdot 7$
\end{solution}

\iffalse
%S11
\ppart If an integer linear combination of $a$ and $b$ equals 1, then
  so does some integer linear combination of $a^2$ and $b^2$. \hfill
  \textbf{true} \qquad \textbf{false} \examspace[0.4in]

\begin{solution}
\textbf{true}
\end{solution}
\fi

\ppart If no integer linear combination of $a$ and $b$ equals 1, then
  neither does any integer linear combination of $a^2$ and $b^2$.
 \hfill \textbf{true} \qquad \textbf{false} \examspace[0.4in]

\begin{solution}
\textbf{true}  No linear combination of $a,b$ is 1 iff $a,b$ have a
common factor iff $a^2,b^2$ have a common factor iff no linear
combination of $a^2,b^2$ is 1.  So any $a,b$ that not not relatively
prime is a counterexample; $a=b=2$ would be the simplest.

\end{solution}

\iffalse
%%NOT USED S11
%\ppart $\gcd(a, b) = \gcd(b, \rem{a}{b})$. % too easy
%\ppart $\gcd(a,b) \gcd(c,d) =  \gcd(ac,bd)$ % too many vars for counterexample
%\ppart $\gcd(a, b) = \gcd(a+b, a-b)$ % what does it test?

\begin{solution}

\begin{itemize}
\ppart The first one is not valid. counterexample: $a=2$, $b=6$, $c=3$
\ppart The second one is not valid. counterexample: $k=1$, $a=1$, $b=2$
\ppart The third and fourth ones are valid.
\end{itemize}

\end{solution}
\fi

\iffalse  S11
\ppart If $a c \equiv b c \pmod{n}$ and $n$ does not divide $c$, then $a \equiv b \pmod{n}$.
\hfill \textbf{true} \qquad \textbf{false} \examspace[0.4in]

\begin{solution}
\textbf{false}.  Need $c$ relatively prime to $n$.  Counterexample:
$n=2 \cdot 3, a=0, b=2, c=3$
\end{solution}
\fi

%S11
\ppart If $a \equiv b \pmod{\phi(n)}$ for $a, b > 0$, then $c^a \equiv c^b
\pmod{n}$.  \hfill \textbf{true} \qquad \textbf{false} \examspace[0.4in]

\begin{solution}
  \textbf{false}.  Need $c$ relatively prime to $n$.  Counterexample:
  $n=4$, so $\phi(n) = 2$; $a=1, b=3$, so $a \equiv b
  \pmod \phi(n)$, $c = 2$, so $c^a = 2 \not\equiv 0 = c^b \pmod 4$.
\end{solution}

\iffalse %S11
\ppart If $a \equiv b \pmod{nm}$, then $a \equiv b \pmod{n}$.
 \hfill \textbf{true} \qquad \textbf{false} \examspace[0.4in]

\begin{solution}
\textbf{true}
\end{solution}
\fi

%NOT USED S11
%\ppart If $a \equiv b \pmod{n}$ and $b \equiv c \pmod{n}$, then $a \equiv c \pmod{n}$. % too easy
%\ppart If $a \equiv b \pmod{n}$ and $c \equiv d \pmod{n}$, then $a^c \equiv b^d \pmod{n}$. % replaced with phi() one
%\ppart $\rem{a}{n} \equiv a \pmod{n}$. % too easy
%\ppart $\gcd(a \pmod{n}, b \pmod{n}) \equiv \gcd(a,b) \pmod{n}$

\ppart If $\gcd(m,n)=1$, then
$[a \equiv b \pmod{m} \QAND\ a \equiv b \pmod{n}]$ iff $[a \equiv b
  \pmod{mn}]$
\hfill \textbf{true} \qquad \textbf{false} \examspace[0.4in]

\begin{solution}
\textbf{true}
\end{solution}

\ppart If $a,b >1$, then [$a$ has a multiplicative inverse mod $b$ iff $b$ has a
multiplicative inverse mod $a$]. \hfill \textbf{true} \qquad
\textbf{false} \examspace[0.4in]

\begin{solution}
\textbf{true}.  $a$ has a multiplicative inverse mod $b$ iff
$a,b$ relatively prime iff $b$ has a multiplicative inverse mod $a$
\end{solution}

\ppart If $\gcd(a,n)=1$, then $a^{n-1} \equiv 1 \pmod{n}$ \hfill
  \textbf{true} \qquad \textbf{false} \examspace[0.4in]

\begin{solution}
\textbf{false}  Let $a=5$, $n =6$.
\end{solution}

\eparts

\end{problem}

\endinput
