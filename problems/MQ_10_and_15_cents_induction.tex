\documentclass[problem]{mcs}

\begin{pcomments}
  \pcomment{MQ_10_and_15_cents_induction}
  \pcomment{not the same as to CP_10_and_15_cent_stamps_by_WOP}
\end{pcomments}

\pkeywords{
  induction
  strong_induction
  postage
}

%%%%%%%%%%%%%%%%%%%%%%%%%%%%%%%%%%%%%%%%%%%%%%%%%%%%%%%%%%%%%%%%%%%%%
% Problem starts here
%%%%%%%%%%%%%%%%%%%%%%%%%%%%%%%%%%%%%%%%%%%%%%%%%%%%%%%%%%%%%%%%%%%%%

\begin{problem}
  Any amount of ten or more cents postage that is a multiple of five
  can be made using only 10\mcents\ and 15\mcents\ stamps.
  Prove this \emph{by induction} (ordinary or strong, but say which)
  using the induction hypothesis
  \[
  S(n) \eqdef (5n+10)\mcents\ \text{postage can be made using
  only 10\mcents\ and 15\mcents\ stamps}.
  \]

\begin{solution}

  \begin{proof}
    The proof is by strong induction.

    \inductioncase{Base case} ($n=0$): $5\cdot 0+10 =10$\mcents\ postage
    can be made with one 10\mcents\ stamp.

    \inductioncase{Inductive step}: We assume $n\geq 0$ and the
    hypothesis $S(n)$ to prove $S(n+1)$.  The proof is by cases:

    \textbf{case} $n=0$: In this case $5(n+1)+10 = 15$.  So $S(n+1)$
    holds because 15\mcents\ postage can be made using one
    15\mcents\ stamp.

    \textbf{case} $n>0$: Since $n-1 \ge 0$, we know by strong induction that
    $S(n-1)$ holds.  So we can make $5(n-1)+10$\mcents\ postage.
    Adding a 10\mcents\ stamp yields $5(n-1)+10 + 10
    =5(n+1)+10$\mcents\ postage, which proves $S(n+1)$.

    Since $S(n+1)$ holds in any case, the inductive step has been proved.

    It follows by induction that $S(n)$ holds for all $n \in \nngint$,
    that is, starting at 10\mcents, every multiple of
    5\mcents\ postage can be made can be made with 10\mcents\ and
    15\mcents\ stamps.

\end {proof}

\end{solution}

\end{problem}

%%%%%%%%%%%%%%%%%%%%%%%%%%%%%%%%%%%%%%%%%%%%%%%%%%%%%%%%%%%%%%%%%%%%%
% Problem ends here
%%%%%%%%%%%%%%%%%%%%%%%%%%%%%%%%%%%%%%%%%%%%%%%%%%%%%%%%%%%%%%%%%%%%%

\endinput
