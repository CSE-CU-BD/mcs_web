\documentclass[problem]{mcs}

\begin{pcomments}
  \pcomment{CP_proving_basic_gcd_properties}
  \pcomment{S17.cp7m, S09.cp8m, S06.cp6w}
  \pcomment{last part changed by ARM 10/25/09}
  \pcomment{ARM 3/15/16: small edit, gcd(k a, k b) = k * gcd(a, b) restored as part(b)}
  \pcomment{edit soln [gcd(k a, k b) = k gcd(a, b)] -- ARM 3/20/17}
\end{pcomments}

\pkeywords{
  gcd
  linear_combinations
  divides
  number_theory
  common_divisor
}

%%%%%%%%%%%%%%%%%%%%%%%%%%%%%%%%%%%%%%%%%%%%%%%%%%%%%%%%%%%%%%%%%%%%%
% Problem starts here
%%%%%%%%%%%%%%%%%%%%%%%%%%%%%%%%%%%%%%%%%%%%%%%%%%%%%%%%%%%%%%%%%%%%%

\begin{problem}
  For nonzero integers $a$, $b$, prove the following properties of
  divisibility and GCD'S.  You may use
  \inbook{Theorem~\bref{gcd_is_lin_thm}} \inhandout{the fact} that
  $\gcd(a, b)$ is an integer linear combination of $a$ and $b$.  You
  may \emph{not} appeal to uniqueness of prime factorization
  \inbook{Theorem~\bref{thm:unique_factor}}, because some of these
  properties are needed to \emph{prove} unique factorization.)

\bparts

\ppart\label{ecd} Every common divisor of $a$ and $b$ divides $\gcd(a, b)$.

\begin{solution}
Any common divisor of $a$ and $b$ will be a divisor of any linear
combination of $a$ and $b$.  By Theorem~\bref{gcd_is_lin_thm}, $\gcd(a,
b)$ is such a linear combination.
\end{solution}

\ppart $\gcd(k a, k b) = k \cdot \gcd(a, b)$ for all $k \in \nngint$.

\begin{solution}
We prove that the left and right-hand sides of the equality divide
each the other, which implies that one is $\pm$ the other.  Since both
are nonnegative, this implies they are equal.

We know $k\gcd(a,b) =k(sa+tb) = s(ka)+t(kb)$ for some $s,t \in
\integers$.  But any common divisor of $ka$ and $kb$ will divide the linear
combination $s(ka)+t(kb) = k\gcd(a,b)$.  Since $\gcd(ka,kb)$ is
certainly a common divisor of $ka$ and $kb$, we conclude
\[
\gcd(ka,kb) \divides k\gcd(a,b)
\]

Conversely,
\[
\gcd(k a, k b) = s'(ka)+t'(kb) = k\cdot(s'a+t'b)
\]
for some $s',t' \in \integers$.  But $\gcd(a,b) \divides\ \text{[any
    linear combination of $a,b$]}$, so
\[
k\cdot \gcd(a,b) \divides k\cdot(s'a+t'b) = \gcd(k a, k b).
\]
\end{solution}


\ppart\label{adb} If $a \divides b c$ and $\gcd(a, b) = 1$, then $a \divides c$.

\begin{solution}
Since $\gcd(a, b) = 1$, we have $sa+tb=1$ for some $s,t$.  Multiplying by
$c$, we have
\[
sac+tbc=c
\]
but $a$ divides the second term of the sum since $a \divides b c$, and
it appears as a factor of the first term, and therefore it divides the
sum, which equals $c$.

\end{solution}

\ppart If $p \divides bc$ for some prime $p$ then $p \divides b$ or $p
\divides c$.

\begin{solution}
If $p$ does not divide $b$, then since $p$ is prime, $\gcd(p,b)
  =1$.  By part~\eqref{adb}, we conclude that $p \divides c$.
\end{solution}

\ppart Let $m$ be the smallest integer linear combination of $a$ and $b$
that is positive.  Show that $m = \gcd(a,b)$.

\begin{solution}
Since $\gcd(a,b)$ is positive and an integer linear combination of $a$ and $b$,
we have
\[
m \leq \gcd(a,b).
\]

On the other hand, since $m$ is a linear combination of $a$ and $b$, every
common factor of $a$ and $b$ divides $m$.  So in particular, $\gcd(a,b)
\divides m$, which implies
\[
\gcd(a,b) \leq m.
\]

\end{solution}

\begin{staffnotes}
If there is time, challenge students to prove that $m$ is a common
divisor of $a$ and $b$ (and hence $m \leq \gcd(a,b)$) without
appealing to the fact that the gcd is a linear combination of $a$ and
$b$:

It is enough to prove that $m \divides a$.  Suppose not.  Then
dividing $a$ by $m$ leaves a positive remainder.  That is, $a=qm+r$
for some $r \in [1,m)$.  But then $r = a-qm$ is a smaller positive
  linear combination of $a$ and $b$, contradicting the definition of
  $m$.

This now gives a proof that the gcd equals a linear combination,
namely $m$, that does not depend on the pulverizer.
\end{staffnotes}

\eparts

\end{problem}

%%%%%%%%%%%%%%%%%%%%%%%%%%%%%%%%%%%%%%%%%%%%%%%%%%%%%%%%%%%%%%%%%%%%%
% Problem ends here
%%%%%%%%%%%%%%%%%%%%%%%%%%%%%%%%%%%%%%%%%%%%%%%%%%%%%%%%%%%%%%%%%%%%%

\endinput

\iffalse

\ppart $\gcd(a, b) = \gcd(b, \rem{a}{b})$

\begin{solution}
Let $r = \rem{a}{b}$.

%Simpler proof in the Notes already.

Since $r=a-qb$ for some $q$, we have that $r$ is a linear combination of
$a$ and $b$ and is therefore divisible by $\gcd(a,b)$.  So any linear
combination of $r$ and $b$ is divisible by $\gcd(a,b)$.  Hence,
\[
\gcd(a,b) \divides \gcd(b,r)
\]

Conversely, $a = qb+r$ is a linear combination of $b$ and $r$ and is
therefore divisible by $\gcd(b,r)$.  Since $\gcd(b,r)$ divides both $a$
and $b$, we conclude from part~\eqref{ecd} that
\[
\gcd(b,r) \divides \gcd(a,b).
\]

\end{solution}
\fi

\iffalse

\begin{proof} We prove only parts~\ref{gcd3}.\ and~\ref{gcd4}.

\textbf{Proof of~\ref{gcd3}}.  The assumptions together with
Theorem~\ref{gcd_is_lin_thm} imply that there exist integers $s$, $t$, $u$,
and $v$ such that:
\begin{align*}
s a + t b & = 1 \\
u a + v c & = 1
\end{align*}
Multiplying these two equations gives:
\[
(s a + t b)(u a + v c) = 1
\]
The left side can be rewritten as
\[
a \cdot (a s u + b t u + c s v) + (bc) (t v).
\]
This is a linear combination of $a$ and $b c$ that is equal to 1, so
$\gcd(a, bc) = 1$ by Theorem~\ref{gcd_is_lin_thm}.

\textbf{Proof of~\ref{gcd4}}.  Theorem~\ref{gcd_is_lin_thm} says that
$\gcd(ac, bc)$ is equal to a linear combination of $ac$ and $bc$.  Now
$a \divides ac$ trivially and $a \divides bc$ by assumption.
Therefore, $a$ divides \emph{every} linear combination of $ac$ and
$bc$.  In particular, $a$ divides $\gcd(ac, bc) = c \cdot \gcd(a, b) =
c\cdot 1 = c$.  The first equality uses part~\ref{gcd2}.\ of this
lemma, and the second uses the assumption that $\gcd(a, b) = 1$.
\end{proof}

\fi
