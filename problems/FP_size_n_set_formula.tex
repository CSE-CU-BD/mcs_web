\documentclass[problem]{mcs}

\begin{pcomments}
  \pcomment{FP_size_n_set_formula}
  \pcomment{subsumed by PS_size_n_set_formula}
  \pcomment{replaces CP_logical_set_theory}
  \pcomment{ARM 3/19/13}
\end{pcomments}

\pkeywords{
  logic
  predicate
  set_theory
  cardinality
  finite
}

%%%%%%%%%%%%%%%%%%%%%%%%%%%%%%%%%%%%%%%%%%%%%%%%%%%%%%%%%%%%%%%%%%%%%
% Problem starts here
%%%%%%%%%%%%%%%%%%%%%%%%%%%%%%%%%%%%%%%%%%%%%%%%%%%%%%%%%%%%%%%%%%%%%

\begin{problem}

\inhandout{ A \emph{formula of \idx{set theory}} is a predicate
  formula that only uses the predicate ``$x \in y$.''  The domain of
  discourse is the collection of sets, and ``$x \in y$'' is
  interpreted to mean that the set $x$ is a member of the set $y$.

For example, since $x$ and $y$ are the same set iff they have the same
members, here's how we can express equality of $x$ and $y$ with a
formula of set theory:
\begin{equation}\label{x=xAz}
(x = y) \eqdef\ \forall z.\, (z \in x\ \QIFF\ z \in y).
\end{equation}

In writing formulas, it is OK to use abbreviations introduced
earlier (so it is now legal to use ``$=$'' because we just defined
it).
}

\bparts

\ppart Explain how to write a formula,
$\text{Subset}_n(x,y_1,y_2,\dots,y_n)$, of set theory
\inbook{\footnote{See Section~\bref{ZFC_sec}.}} that means $x
\subseteq \set{y_1,y_2,\dots,y_n}$.

\examspace[1in]

\begin{solution}

\begin{align*}
%\lefteqn{\text{Subset}_n(x,y_1,y_2,\dots,y_n)}
\text{Subset}_n(x,y_1,y_2,\dots,y_n)
 \eqdef\ \forall z.\, z \in x \QIMPLIES\ (z = y_1 \QOR\ z = y_2 \QOR\ \dots \QOR\ z = y_n).
\end{align*}

\end{solution}

\problempart Now use the formula $\text{Subset}_n$ to write a formula,
$\text{Atmost}_n(x)$, of set theory that means that $x$ has at most
$n$ elements.

\examspace[1in]

\begin{solution}
\[
\text{Atmost}_n(x) \eqdef\ \exists y_1,y_2,\dots,y_n.\, \text{Subset}_n(x,y_1,y_2,\dots,y_n).
\]
\end{solution}

\problempart Explain how to write a formula, $\text{Exactly}_n$, of
set theory that means that $x$ has exactly $n$ elements.  Your formula
should only be about twice the length of the formula
$\text{Atmost}_n$\inhandout{, but you will get at least half credit
  even if your formula is larger}.

\iffalse

\examspace[1in]

\begin{solution}
\[
\text{Exactly}_n(x) \eqdef\ \text{Atmost}_n(x) \QAND \QNOT(\text{Atmost}_{n-1}(x)).
\]
\end{solution}

\ppart The obvious way to write a formula, $D_n(y_1,\dots,y_n)$, of
set theory that means that $y_1,\dots,y_n$ are distinct elements is to
write an \QAND\ of subformulas ``$y_i \neq y_j$'' for $1 \leq i < j
\leq n$.  Since there are $n(n-1)/2$ such subformulas, this approach
leads to a formula $D_n$ whose length grows proportional to $n^2$.
Describe how to write such a formula $D_n(y_1,\dots,y_n)$ whose length
only grows proportional to $n$.

\hint Use $\text{Subset}_n$ and $\text{Exactly}_n$.

\examspace[1in]

\begin{solution}
\[
\exists x.\, \text{Exactly}_n(x) \QAND\ \text{Subset}_n(x,y_1,y_2,\dots,y_n).
\]
\end{solution}
\fi


\end{problemparts}

\end{problem}

%%%%%%%%%%%%%%%%%%%%%%%%%%%%%%%%%%%%%%%%%%%%%%%%%%%%%%%%%%%%%%%%%%%%%
% Problem ends here
%%%%%%%%%%%%%%%%%%%%%%%%%%%%%%%%%%%%%%%%%%%%%%%%%%%%%%%%%%%%%%%%%%%%%

\endinput
