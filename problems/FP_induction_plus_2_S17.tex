\documentclass[problem]{mcs}

\begin{pcomments}
  \pcomment{FP_induction_plus_2_S17} 
  \pcomment{see also
    FP_induction_plus_2,
    FP_induction_plus_2.spring12,FP_induction_plus_2_S14}
  \pcomment{F05.final, S12, S14} 
  \pcomment{Ask for both T and F examples -- ARM 3/12/17}
\end{pcomments}

%%%%%%%%%%%%%%%%%%%%%%%%%%%%%%%%%%%%%%%%%%%%%%%%%%%%%%%%%%%%%%%%%%%%%
% Problem starts here
%%%%%%%%%%%%%%%%%%%%%%%%%%%%%%%%%%%%%%%%%%%%%%%%%%%%%%%%%%%%%%%%%%%%%
\begin{problem}
Let's say a predicate $P$ on the nonnegative integers is a
\emph{plus2} predicate when the following assertion is true:
\[%\begin{equation}\label{Sk2}
\forall n. \; P(n) \QIMPLIES P(n+2).
\]%\end{equation}

For each of the
assertions~\eqref{p1anp2n1}--\inhandout{\eqref{enem<n}}\inbook{\eqref{enpbqan}}
below, indicate whether the assertion is true for \textbf{A}ll plus2
predicates $P$, true for \textbf{S}ome but not all plus2 predicates
$P$, or \textbf{N}ot true for any plus2 predicate $P$
\inbook{. Explain your reasoning.}\inhandout{by \textbf{circling} the
  correct letter below.  For assertions which are true for
  \textbf{S}ome plus2 predicates $P$ but not others, describe a plus2
  predicate for which the assertion is true and another for which it
  is false.  Do \emph{not} include explanations.}

\begin{problemparts}

%\begin{staffnotes}
%WAS COMMENTED OUT.
%\end{staffnotes}
%
%\problempart  $\forall n \ge 0.\; P(n)$ \inhandout{\hfill \textbf{C\quad A\quad N}}
%
%\begin{solution}
%\textbf{C}.  The assertion means that $P$ is always true.  So $P(k+2)$
%is always true, and therefore $P(k) \QIMPLIES P(k+2)$ is always true.
%So this case is possible.  But~\eqref{Sk2} also holds when $P$ is always
%false, so the asertion does not always hold when \eqref{Sk2} does.
%\end{solution}

%\begin{staffnotes}
%WAS COMMENTED OUT.
%\end{staffnotes}

\problempart\label{p1anp2n1} $P(1) \QIMPLIES \forall n.\, P(2n+1)$
\inhandout{\hfill \textbf{A\quad S\quad N}}

\begin{solution}
  \textbf{A}.  This assertion says that if $P(1)$ holds, then $P(n)$
  holds for all odd $n$.  This is always true for a plus2 predicate..
\end{solution}

\examspace[1.5in]

\problempart $\QNOT(P(0)) \QAND \forall n \ge 1.\; P(n)$
\inhandout{\hfill \textbf{A\quad S\quad N}}

\begin{solution}
  \textbf{S}.

Letting $P(n)\eqdef n\geq 1$ gives a plus2 predicate making the
assertion true, and letting $P(n)$ be the always be true predicate
gives a plus2 predicate that makes the assertion false since $\QNOT(P(0))$ is
false.
\end{solution}

\examspace[1.5in]

\problempart $\exists n.\, \exists m > n.\, [P(2n) \QAND \QNOT(P(2m))]$
\inhandout{\hfill \textbf{A\quad S\quad N}}
\begin{solution}
  \textbf{N}.

 This assertion says that $P$ holds for some even number $2n$, but not
 for some other larger even number $2m$.  However, if $P(2n)$ holds,
 we can invoke the plus2 property $n-m$ times to conclude $P(2m)$ also
 holds.
\end{solution}

\examspace[1.5in]

%\begin{staffnotes}
%WAS COMMENTED OUT.
%\end{staffnotes}
%
%\problempart $\forall n \ge 0.\; \QNOT(P(n))$  \inhandout{\hfill \textbf{A\quad S\quad N}}
%
%\begin{solution}
%\textbf{C.}  Now $P$ is always false.  So $P(k) \QIMPLIES P(k+2)$ is
%always true because $P(k)$ is false.  So this case is possible, but
%again does not always hold.
%\end{solution}

\inbook{
\problempart $(\forall n \le 100.\; P(n)) \QAND (\forall n > 100.\; \QNOT(P(n)))$

%\inhandout{\hfill \textbf{A\quad S\quad N}}

\begin{solution}
\textbf{N}.  In this case, $P$ is true for $n$ up to 100 and false
from 101 on.  So $P(99)$ is true, but $P(101)$ is false.  That means
that $\QNOT[P(k) \QIMPLIES P(k+2)]$ for $k = 99$.  This case is
impossible.
\end{solution}
}

%\begin{staffnotes}
%WAS COMMENTED OUT.
%\end{staffnotes}

\inbook{
\problempart $(\forall n \le 100.\; \QNOT(P(n))) \QAND (\forall n > 100.\; P(n))$
%\inhandout{\hfill \textbf{A\quad S\quad N}}

\begin{solution}
\textbf{S}.  In this case, $P$ is false for $n$ up to 100 and true
from 101 on.  So $P(k) \QIMPLIES P(k+2)$ for $k \le 100$ because
$P(k)$ is false, and $P(k) \QIMPLIES P(k+2)$ for $k \ge 99$ because
$P(k+2)$ is true.  So this assertion can be true, but not always.
\end{solution}
}

%\begin{staffnotes}
%WAS COMMENTED OUT.
%\end{staffnotes}

%\problempart $P(0) \QIMPLIES \forall n.\, P(n+2)$
%\inhandout{\hfill \textbf{A\quad S\quad N}}

%\begin{solution}
%\textbf{C}.  In this case, $P$ is false for $n$ up to 100 and true
%from 101 on.  So $P(k) \QIMPLIES P(k+2)$ for $k \le 100$ because
%$P(k)$ is false, and $P(k) \QIMPLIES P(k+2)$ for $k \ge 99$ because
%$P(k+2)$ is true.  So this case is possible, but again does not always
%hold.
%\end{solution}

%\begin{staffnotes}
%WAS COMMENTED OUT.
%\end{staffnotes}
% \problempart $[\exists n.\, P(2n)] \QIMPLIES \forall n. \ P(2n+2)$
% \inhandout{\hfill \textbf{A\quad S\quad N}}
% \begin{solution}
% \textbf{C}.  We see the case is possible by considering $P(n)$ that is always true.
% A counterexample is $P(n)$ that holds iff $n > 3$, where the $\exists n$
% is satisfied with $n = 2$, but the $\forall n$ fails for $n = 0$.
% \end{solution}

\problempart\label{enem<n} $\exists n.\, \exists m < n.\, [P(2n) \QAND
  \QNOT(P(2m))]$ \inhandout{\hfill \textbf{A\quad S\quad N}}

\begin{solution}
 \textbf{S}.  Letting $P(n) \eqdef n >0$ is a plus2 predicate that
 makes this assertion true since $0<1$ and $P(2\times 1) \QAND
 \QNOT(P(2 \times 0)$ is true.

Letting $P(n)$ be the always be true predicate is a plus2 predicate
that makes this assertion false because $\QNOT(P(2m))$ is false for
every $m$.
\end{solution}


\examspace[1.5in]

\iffalse

\problempart $[\exists n.\, P(2n)] \QIMPLIES \forall n. \ P(2n+2)$
\inhandout{\hfill \textbf{A\quad S\quad N}}
\begin{solution}
 \textbf{C}.  If $P(2n)$ is true only for $n>1$ and false otherwise,
 then~\eqref{Sk2} holds, but this assertion is false because $P(2\cdot
 0+2)$ is false.

  The assertion can also be true, for example if $P(n)$ is always true.
\end{solution}
\fi

\inbook{
\problempart\label{enpbqan} $[\exists n.\, P(n)] \QIMPLIES \forall
n.\, \exists m > n.\, P(m)$
%\inhandout{\hfill \textbf{A\quad S\quad N}}

\begin{solution}
\textbf{A}.  This assertion says that if $P$ holds for some $n$, then
for every number, there is a larger number $m$ for which $P$ also
holds.  Since~\eqref{Sk2} implies that if there is one $n$ for which
$P(n)$ holds, there are an infinite, increasing chain of $k$'s for
which $P(k)$ holds, this case is always true.
\end{solution}}

%\begin{staffnotes}
%WAS COMMENTED OUT.
%\end{staffnotes}
% \problempart $\QNOT(P(0)) \QIMPLIES \forall n. \ \QNOT(P(2n))$
% \inhandout{\hfill \textbf{A\quad S\quad N}}
% \begin{solution}

% \textbf{C}. If for all $n$, $P(n)$ is false, then the statement is true.
% If $P(0)$ is false but $P(2)$ is true, then the statement is false. 
% \end{solution}

\end{problemparts}
\end{problem}

%%%%%%%%%%%%%%%%%%%%%%%%%%%%%%%%%%%%%%%%%%%%%%%%%%%%%%%%%%%%%%%%%%%%%
% Problem ends here
%%%%%%%%%%%%%%%%%%%%%%%%%%%%%%%%%%%%%%%%%%%%%%%%%%%%%%%%%%%%%%%%%%%%%

\endinput
