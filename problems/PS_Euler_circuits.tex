\documentclass[problem]{mcs}

\begin{pcomments}
  \pcomment{PS_Euler_circuits}
  \pcomment{undirected version of PS_directed_Euler_circuits}
  \pcomment{solns could use some editing as in PS_directed_Euler_circuits}
  \pcomment{revised slightly by ARM 4/1/11}
  \pcomment{from: S09.ps5; F06.rec6 (revised by ARM)}
\end{pcomments}

\pkeywords{
  Euler
  Euler_circuit
  Euler_tour
  cycle
  degree
  walk
  closed_walk
}

%%%%%%%%%%%%%%%%%%%%%%%%%%%%%%%%%%%%%%%%%%%%%%%%%%%%%%%%%%%%%%%%%%%%%
% Problem starts here
%%%%%%%%%%%%%%%%%%%%%%%%%%%%%%%%%%%%%%%%%%%%%%%%%%%%%%%%%%%%%%%%%%%%%

\begin{problem}

\iffalse
Can you walk every hallway in the Museum of Fine Arts \emph{exactly
  once}?  If we represent hallways and intersections with edges and
vertices, then this reduces to a question about graphs.

 For example, could you visit every hallway exactly once in a
museum with the floor plan in Figure~\ref{fig:5BC}?

\begin{figure}

\gnote{Add a ``real'' floor-plan?}

\graphic{euler-tour}

\caption{A possible floor plan for a museum. Can you find a walk that
  includes every edge exactly once?}

\label{fig:5BC}

\end{figure}
\fi

The entire field of graph theory began when Euler asked whether their
was a walk through his home city of K\"onigsberg in which all seven of
its famous bridges were each crossed exactly once.  Abstractly, we can
represent the parts of the city separated by rivers as vertices and
the bridges as edges between the vertices.  Then Euler's question asks
whether there is a closed walk through the graph that includes every
edge in a graph exactly once.  In his honor, such a walk is called an
\term*{Euler tour}.

So how do you tell in general whether a graph has an Euler tour?  At
first glance this may seem like a daunting problem.  The similar
sounding problem of finding a cycle that touches every vertex exactly
once is one of those Millenium Prize NP-complete problems known as the
\term{Hamiltonian Cycle Problem}).  But it turns out to be easy to
characterize which graphs have Euler tours.  \iffalse (The same is
true for graphs with Euler walks, but we'll leave that as another
exercise.)\fi

\begin{theorem*}%\label{thm:euler-tour}
A connected graph has an Euler tour if and only if every vertex has
even degree.
\end{theorem*}

\begin{editingnotes}
cut by ARM --seems unnecessary.

Does the graph in the following figure contain an Euler tour?

\begin{figure}
\graphic{example}
\end{figure}

Well, if it did, the edge $(E, F)$ would need to be included.  If the walk
does not start at $F$ then at some point it includes edge $(E,F)$, and
now it is stuck at $F$ since $F$ has no other edges incident to it and an
Euler tour can't include $(E,F)$ twice.  But then the walk could not
be a tour.  On the other hand, if the walk starts at $F$, it must then
go to $E$ along $(E,F)$, but now it cannot return to $F$.  It again cannot
be a tour. This argument generalizes to show that if a graph has a
vertex of degree $1$, it cannot contain an Euler tour.
\end{editingnotes}

\begin{staffnotes}
On the other hand, it is easy to see that any cycle has an Euler
tour.  You can just start at any vertex and walk around back to it.
\end{staffnotes}

\bparts

\ppart Show that if a graph has an Euler tour, then the degree of each
of its vertices is even.

\begin{solution}
Let tour $C \eqdef v_1, v_2, \dots, v_r, v_1$ be an Euler tour.
Consider any vertex $v$.  Then every time $v$ occurs in $C$, there is
a vertex $a$ which comes immediately before $v$ and a vertex $b$ which
comes immediately after $v$.  Note that this holds for $v = v_1$ as
well since $C$ is a tour.  Moreover, $\diredge{a}{v}$ and
$\diredge{v}{b}$ must be distinct edges of $G$ since $C$ is an Euler
tour.  It follows that if $v$ occurs $s$ times in $C$, then it has
degree $2s$ since every edge incident to $v$ occurs in $C$ exactly
once.  Thus, $v$ has even degree.
\end{solution}

\eparts

In the remaining parts, we'll work out the converse: if the degree of
every vertex of a connected finite graph is even, then it has an Euler
tour.  To do this, let's define an Euler \term{walk} to be a walk that
includes each edge \emph{at most} once.

\bparts

\ppart\label{conn-undirected} Suppose that an Euler walk in a connected graph
does not include every edge.  Explain why there must be an unincluded
edge that is incident to a vertex on the walk.

\begin{solution}
If either end of the unincluded edge is on the Euler walk, that
already is the desired edge.  So suppose there's an unincluded edge,
$e$, both of whose endpoints are not on the Euler walk.  Since the
graph is connected, there must be a shortest walk $\walkv{p}$ from
an endpoint of $e$ to a vertex on the Euler walk.  Then none of the
edges on $\walkv{p}$ can be on $\walkv{p}$ or $\walkv{p}$ could be
shortened.  So the last edge on $\walkv{p}$ will be the desired edge.
\end{solution}

\eparts

In the remaining parts, let $\walkv{w}$ be the \emph{longest} Euler
walk in some finite, connected graph.

\bparts

\ppart\label{cycle-circuit-undirected} Show that if $\walkv{w}$ is a closed walk, then
it must be an Euler tour.

\hint part~\eqref{conn-undirected}

\begin{solution}
Suppose an edge was in $\walkv{w}$.  By part~\eqref{conn-undirected},
there must be a vertex on $\walkv{w}$ incident to an edge not in
$\walkv{w}$.  Starting at this vertex, go around $\walkv{w}$ back to
that vertex, and then the follow the edge.  This makes a longer Euler
walk, contradicting the maximality of $\walkv{w}$.  So no edge can be
missing from $\walkv{w}$.
\end{solution}

\ppart\label{already-undirected} Explain why all the edges incident to
the end of $\walkv{w}$ must already be in
$\walkv{w}$.

\begin{solution}
Otherwise we could extend $\walkv{w}$ to a longer Euler walk with any
  edge from the end not already in $\walkv{w}$.
\end{solution}

\ppart\label{odd-undirected} Show that if the end of $\walkv{w}$ was
not equal to the start of $\walkv{w}$, then the degree of the end
would be odd.

\hint part~\eqref{already-undirected}

\begin{solution}
Let $v$ be the end vertex of $\walkv{w}$.  Given that $v$ is not the
start of $\walkv{w}$, it follows that at any occurrence of $v$ in
$\walkv{w}$ other than at the end, $\walkv{w}$ would enter and leave
that occurrence of $v$ with a pair of edges.  Since $\walkv{w}$ is an
Euler walk, all the edges in all these pairs are distinct.  In
addition, the final edge in $\walkv{w}$ as it ends at $v$ is
distinct from all the paired edges.  Altogether, this imples that
there are an odd number of edges in $\walkv{w}$ that are incident to
$v$.  But by part~\eqref{already-undirected}, these are all the edges
incident to $v$, proving that $v$ has odd degree.
\end{solution}

\ppart Conclude that if every vertex of a finite, connected graph has even
degree, then it has an Euler tour.

\begin{solution}
If all vertices in $G$ have even degree, then by part~\eqref{odd-undirected}, the
only possibility is that the end of $\walkv{w}$ equals the start, that
is, $\walkv{w}$ is closed.  So by part~\eqref{cycle-circuit-undirected},
$\walkv{w}$ is an Euler tour.
\end{solution}

\eparts
\end{problem}

%%%%%%%%%%%%%%%%%%%%%%%%%%%%%%%%%%%%%%%%%%%%%%%%%%%%%%%%%%%%%%%%%%%%%
% Problem ends here
%%%%%%%%%%%%%%%%%%%%%%%%%%%%%%%%%%%%%%%%%%%%%%%%%%%%%%%%%%%%%%%%%%%%%

\endinput
