\documentclass[problem]{mcs}

\begin{pcomments}
  \pcomment{PS_Euler_circuits}
  \pcomment{from: S09.ps5}
  \pcomment{from: F06.rec6 (revised by ARM)}
\end{pcomments}

\pkeywords{
  Euler_circuits
  cycles
  degree
}

%%%%%%%%%%%%%%%%%%%%%%%%%%%%%%%%%%%%%%%%%%%%%%%%%%%%%%%%%%%%%%%%%%%%%
% Problem starts here
%%%%%%%%%%%%%%%%%%%%%%%%%%%%%%%%%%%%%%%%%%%%%%%%%%%%%%%%%%%%%%%%%%%%%

\begin{problem}
  In this problem we'll consider some special cycles in graphs called
  \term{Euler circuits}, named after the famous mathematician Leonhard
  Euler.  (Same Euler as for the constant $e\approx 2.718$---he did a
  lot of stuff.)

\begin{definition}
  An Euler circuit of a graph is a cycle which traverses every edge
  exactly once.
\end{definition}

Does the graph in the following figure contain an Euler circuit?

\begin{figure}[h]
\graphic{example}
\end{figure}

Well, if it did, the edge $(E, F)$ would need to be included.  If the path
does not start at $F$ then at some point it traverses edge $(E,F)$, and
now it is stuck at $F$ since $F$ has no other edges incident to it and an
Euler circuit can't traverse $(E,F)$ twice.  But then the path could not
be a circuit.  On the other hand, if the path starts at $F$, it must then
go to $E$ along $(E,F)$, but now it cannot return to $F$.  It again cannot
be a circuit. This argument generalizes to show that if a graph has a
vertex of degree $1$, it cannot contain an Euler circuit.

\begin{staffnotes}
On the other hand, it is easy to see that any cycle contains an Euler
circuit. You can just start at any vertex and walk around back to it.
\end{staffnotes}

So how do you tell in general whether a graph has an Euler circuit?  At
first glance this may seem like a daunting problem (the similar sounding
problem of finding a cycle that touches every vertex exactly once is one
of those million dollar NP-complete problems known as the \term{Traveling
  Salesman Problem})---but it turns out to be easy.

\bparts

\ppart Show that if a graph has an Euler circuit, then the degree of each
of its vertices is even.

  \begin{solution}
Let circuit $C \eqdef v_1, v_2, \dots, v_r, v_1$ be an Euler
    circuit.  Consider any vertex $v$.  Then every time $v$ occurs in
    $C$, there is a vertex $a$ which comes immediately before $v$ and a
    vertex $b$ which comes immediately after $v$.  Note that this holds for
    $v = v_1$ as well since $C$ is a circuit. Moreover, $(a,v)$ and
    $(v,b)$ must be distinct edges of $G$ since $C$ is an Euler circuit.
    It follows that if $v$ occurs $s$ times in $C$, then it has degree
    $2s$ since every edge incident to $v$ occurs in $C$ exactly once.
    Thus, $v$ has even degree.
\end{solution}
\eparts

In the remaining parts, we'll work out the converse: if the degree of
every vertex of a connected finite graph is even, then it has an Euler
circuit.  To do this, let's define an Euler \term{path} to be a path that
traverses each edge \emph{at most} once.

\bparts

\ppart\label{conn} Suppose that an Euler path in a connected graph does
not traverse every edge.  Explain why there must be an untraversed edge
that is incident to a vertex on the path.

\begin{solution}
If either end of the untraversed edge is on the Euler path, that
  already is the desired edge.  So suppose there's an untraversed edge,
  $e$, both of whose endpoints are not on the Euler path.  Since the graph
  is connected, there must be a shortest path, $P$, from an endpoint of
  $e$ to a vertex on the Euler path.  Then all the edges on $P$ must be
  untraversed (or $P$ could be shortened), so the last edge traversed by
  $P$ will be the desired edge.
\end{solution}

\eparts

In the remaining parts, let $W$ be the \emph{longest} Euler path in some
finite, connected graph.

\bparts

\ppart\label{cycle-circuit} Show that if $W$ is a cycle, then it must be
an Euler circuit.

\hint part~\eqref{conn}

\begin{solution}
Suppose an edge was not traversed by $W$.  By part~\eqref{conn},
  there must be a vertex on $W$ incident to an edge not traversed by $W$.
  Starting at this vertex, go around $W$ back to that vertex, and then the
  follow the edge.  This makes a longer Euler path, contradicting the
  maximality of $W$.  So no edge can be missing from $W$.
\end{solution}

\ppart\label{already} Explain why all the edges incident to the end of $W$
must already have been traversed by $W$.

\begin{solution}
Otherwise we could extend $W$ to a longer Euler path with any
  edge from the end not already traversed by $W$.
\end{solution}

\ppart\label{odd} Show that if the end of $W$ was not equal to the start
of $W$, then the degree of the end would be odd.

\hint part~\eqref{already}

\begin{solution}
Let $v$ be the end vertex of $W$.  Given that $v$ is not the
  start of $W$, it follows that at any occurrence of $v$ in $W$ other than
  at the end, $W$ would enter and leave that occurrence of $v$ traversing
  a pair of edges.  Since $W$ is an Euler path, all the edges in all these
  pairs are distinct.  In addition, the final edge traversed by $W$ as it
  ends at $v$ is distinct from all the paired edges.  Altogether, this
  imples that there are an odd number of edges traversed by $W$ and
  incident to $v$.  But by part~\eqref{already}, these are all the edges
  incident to $v$, proving that $v$ has odd degree.
\end{solution}

\ppart Conclude that if every vertex of a finite, connected graph has even
degree, then it has an Euler circuit.

\begin{solution}
If all vertices in $G$ have even degree, then by
  part~\eqref{odd}, the only possibility is that the end of $W$ equals the
  start, that is, $W$ is a cycle.  So by part~\eqref{cycle-circuit}, $W$ is
  an Euler circuit.
\end{solution}

\eparts
\end{problem}

%%%%%%%%%%%%%%%%%%%%%%%%%%%%%%%%%%%%%%%%%%%%%%%%%%%%%%%%%%%%%%%%%%%%%
% Problem ends here
%%%%%%%%%%%%%%%%%%%%%%%%%%%%%%%%%%%%%%%%%%%%%%%%%%%%%%%%%%%%%%%%%%%%%

\endinput
