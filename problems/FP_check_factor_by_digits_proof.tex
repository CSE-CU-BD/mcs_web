\documentclass[problem]{mcs}

\begin{pcomments}
  \pcomment{FP_check_factor_by_digits_proof}
  \pcomment{variant of PS_ & FP_check_factor_by_digits}
  \pcomment{ARM 4/2/16}
\end{pcomments}

\pkeywords{
  modulo
  congruence
  hexadecimal
  decimal
  base_10
  digits
}

%%%%%%%%%%%%%%%%%%%%%%%%%%%%%%%%%%%%%%%%%%%%%%%%%%%%%%%%%%%%%%%%%%%%%
% Problem starts here
%%%%%%%%%%%%%%%%%%%%%%%%%%%%%%%%%%%%%%%%%%%%%%%%%%%%%%%%%%%%%%%%%%%%%

\begin{problem}
The sum of the digits of the base 10 representation of an integer is
congruent modulo 9 to that integer.  For example,
\[
763 \equiv 7+6+3 \pmod 9.
\] 
We can say that ``9 is a \emph{good modulus for base} 10.''

More generally, we'll say ``$k$ is a good modulus for base $b$'' when,
for any nonnegative integer $n$, the sum of the digits of the base $b$
representation of $n$ is congruent to $n$ modulo $k$.  So 2 is
\emph{not} a good modulus for base 10 because
\[
763 \not\equiv 7+6+3 \pmod 2.
\]

\bparts

\ppart What integers $k>1$ are good moduli for base 10?

\begin{center}
\exambox{1.0in}{0.5in}{0.0in}
\end{center}

\examspace[1.0in]
\begin{solution}
3 and 9.
\end{solution}

\ppart Show that if $b \equiv 1 \pmod k$, then $k$ is good for base $b$.

\examspace[1.5in]

\begin{solution}
\begin{align*}
\lefteqn{d_m \cdot b^m + d_{m-1} \cdot b^{m-1} + \cdots + d_1 \cdot b^1 + d_0 \cdot b^0}\\
   & \equiv d_m \cdot 1^m + d_{m-1} \cdot 1^{m-1} + \cdots + d_1 \cdot 1^1 + d_0 \cdot 1^0 \pmod k\\
   & = d_m + d_{m-1} + \cdots + d_1 + d_0.
\end{align*}
\end{solution}

\ppart Prove conversely, that if $b \geq 2$ and $b \equiv 1 \pmod k$,
then $k$ is good for base $b$.

\hint Try $n = b$.

\examspace[2.0in]

\begin{solution}
For $n =b$, the base $b$ represention of $n$ is \texttt{10} so if $k$
is a good  modulus for base $b$ we have
\[
n \equiv 1+0 \pmod k,
\]
that is,
\[
b \equiv 1 \pmod k.
\]
\end{solution}

\ppart Exactly which integers $k>1$ are good moduli for base 106?
\begin{center}
\exambox{4.0in}{0.5in}{0.0in}
\end{center}

\examspace[2.0in]

\begin{solution}
3,5,7,15,21,35,105
\end{solution}
\eparts

\end{problem}

%%%%%%%%%%%%%%%%%%%%%%%%%%%%%%%%%%%%%%%%%%%%%%%%%%%%%%%%%%%%%%%%%%%%%
% Problem ends here
%%%%%%%%%%%%%%%%%%%%%%%%%%%%%%%%%%%%%%%%%%%%%%%%%%%%%%%%%%%%%%%%%%%%%

\endinput

\iffalse
Given any integer $x$ represented in base $b$, for some $k \in [2,b)$,
  we can check whether $x$ is a multiple of $k$ by checking whether
  the sum of the digits of $x$ is a multiple of $k$.  Let $K_b$ be all
  such $k$'s.  For example in decimal representation ($b = 10$), we
  can check whether any integer is a multiple of $3$ by checking
  whether the sum of the its digits is a multiple of $3$.

\bparts

\ppart
Let $x_{d-1}...x_1x_0$ be the representation of $x$ in base $b$,
where $d$ is the total number of digits.
Write an equation to express $x$ in terms of its digits.

\begin{solution}
$x = \sum_{i=0}^{d-1}{x_i}\left(b^i\right)$
\end{solution}

\ppart
Show that if $b \equiv 1 \mod k$,
then $x \equiv \sum_{i=0}^{d-1}{x_i} \mod k$.

\begin{solution}
$x \equiv \sum_{i=0}^{d-1}{x_i}\left(b^i\right)
\equiv \sum_{i=0}^{d-1}{x_i}\left(1^i\right)
\equiv \sum_{i=0}^{d-1}{x_i} \mod k$
\end{solution}

\ppart
Show that if $b \equiv 1 \mod k$,
then $k$ must be a divisor of $b-1$.
\begin{solution}
$b \equiv 1 \mod k$ implies $b-1 \equiv 0 \mod k$,
so $b-1$ is a multiple of $k$
\end{solution}

\ppart
For hexadecimal representation ($b = 16$), find $K_{16}$

\begin{solution}
$\set{3, 5, 15}$
\end{solution}

\eparts

\fi
