\documentclass[problem]{mcs}

\begin{pcomments}
  \pcomment{from: S09.cp3t}
%  \pcomment{}
%  \pcomment{}
\end{pcomments}

\pkeywords{
  bijections
  mapping_lemma
  rational
}

%%%%%%%%%%%%%%%%%%%%%%%%%%%%%%%%%%%%%%%%%%%%%%%%%%%%%%%%%%%%%%%%%%%%%
% Problem starts here
%%%%%%%%%%%%%%%%%%%%%%%%%%%%%%%%%%%%%%%%%%%%%%%%%%%%%%%%%%%%%%%%%%%%%

\begin{problem}
  \bparts \ppart Describe a bijection between the integers, $\integers$,
  and the positive integers, $\integers^+$.

  \begin{solution}
One such bijection is defined by lining up all the integers and
    the positive integers as follows:
\[\begin{array}{ccccccccl}
0 & 1 & -1 & 2 & -2 & 3 & -3 & 4 & \dots\\
1 & 2 & 3 & 4 & 5 & 6 & 7 & 8 & \dots
\end{array}\]
We can also define this bijection, $f:\integers \to \integers^+$, by a specification rule
\[
f(n) = \begin{cases}
       2n & \text{for } n>0,\\
       2\abs{n} +1  & \text{for } n\leq 0.
       \end{cases}
\]

\end{solution}

\ppart Define a bijection between the positive integers, $\integers^+$,
and the set, $\integers^+ \times \integers^+$, of all the ordered pairs of
positive integers:
\[\begin{array}{l}
(1,1),(1,2),(1,3),(1,4),\dots\\
(2,1),(2,2),(2,3),(2,4),\dots\\
(3,1),(3,2),(3,3),(3,4),\dots\\
(4,1),(4,2),(4,3),(4,4),\dots\\
\qquad \vdots
\end{array}\]

\begin{solution}
Line up all the pairs by following successive upper-right to
  lower-left diagonals along the top row.

That is, start with (1,1) which counts as an initial diagonal of length 1.
Then follow the length 2 second diagonal (1,2), (2,1), then the length 3
third diagonal (1,3), (2,2), (3,1), then the length 4 fourth diagonal
(1,4) (2,3) (3,2) (4,1) \dots.  So the line up would be
\[\begin{array}{ccccccccccccl}
(1,1) & (1,2) & (2,1) & (1,3) & (2,2) & (3,1) & (1,4) & (2,3) & (3,2) &
(4,1) & \dots\\ 
   1  & 2     & 3     & 4     & 5     & 6     & 7     & 8     & 9     & 10 &\dots
\end{array}\]

It's interesting that this bijection from $\integers^+ \times \integers^+$
to $\integers^+$ has a simple formula: the pair $(k,m)$ is the $k$th
element on the diagonal consisting of the pairs whose sum is $k+m$.  The
total number of elements in all the preceding diagonals is
\[
1 + 2 + 3 + \cdots + (k+m-2) = (k+m - 1)(k+m-2 )/2
\]
so the pair $(k,m)$ appears as the $(k+m -1)(k+m-2)/2 + k$th element in
the line up.
\end{solution}

\ppart Conclude that $\integers^+$ is the same size as the set,
$\rationals^+$, of all positive rational numbers.

\begin{solution}
One way to line up the positive rationals is to take the list
  of all pairs, $(k,m)$, of integers above and replace each remaining pair
  by the rational number $k/m$,
\begin{quote}
1/1\ \ \ 1/2\ \ \ 2/1\ \ \ 1/2\ \ \ 2/2\ \ \ 3/1\ \ \ 1/3\ \ \ 2/3\ \ \ 3/2\ \ \ 4/1\ \ \ \dots
\end{quote}
and, going from left to right, delete all the occurrences of numbers
that are already in the list:
\begin{quote}
1\ \ \ 1/2\ \ \ 2\ \ \ 3\ \ \ 1/3\ \ \ 2/3\ \ \ 3/2\ \ \ 4 \dots.
\end{quote}

\end{solution}

\eparts
\end{problem}

%%%%%%%%%%%%%%%%%%%%%%%%%%%%%%%%%%%%%%%%%%%%%%%%%%%%%%%%%%%%%%%%%%%%%
% Problem ends here
%%%%%%%%%%%%%%%%%%%%%%%%%%%%%%%%%%%%%%%%%%%%%%%%%%%%%%%%%%%%%%%%%%%%%

\endinput
