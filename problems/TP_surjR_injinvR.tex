\documentclass[problem]{mcs}

\begin{pcomments}
\pcomment{TP_surjR_injinvR}
\pcomment{small part of CP_surj_relation, other direction of TP_injR_surjinvR}
\pcomment{S13, miniq2}
\pcomment{ARM 3/21/13}
\end{pcomments}

\pkeywords{
  relations
  functions
  injections
  surjections
}

%%%%%%%%%%%%%%%%%%%%%%%%%%%%%%%%%%%%%%%%%%%%%%%%%%%%%%%%%%%%%%%%%%%%%
% Problem starts here
%%%%%%%%%%%%%%%%%%%%%%%%%%%%%%%%%%%%%%%%%%%%%%%%%%%%%%%%%%%%%%%%%%%%%

\begin{problem}

Prove that $\paren{A \surj B}$ implies $\paren{B \inj A}$.  It is
fine---even encouraged---to phrase things in terms of ``arrows-in''
and ``arrows-out.''

\begin{solution}

\begin{proof}
By definition of $\surj$, there is a $[\leq 1\ \text{out}, \geq
1\ \text{in}]$ surjective function $R:A \to B$.  Reversing direction
  of the arrows, we have $\inv{R}: B \to A$ is a $[\geq 1\ \text{out},
    \leq 1\ \text{in}]$ total injective relation, so $B \inj A$ by
  definition of $\inj$.
\end{proof}

\end{solution}
\end{problem}
%%%%%%%%%%%%%%%%%%%%%%%%%%%%%%%%%%%%%%%%%%%%%%%%%%%%%%%%%%%%%%%%%%%%%
% Problem ends here
%%%%%%%%%%%%%%%%%%%%%%%%%%%%%%%%%%%%%%%%%%%%%%%%%%%%%%%%%%%%%%%%%%%%%

\endinput


