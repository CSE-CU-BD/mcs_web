\documentclass[problem]{mcs}

\begin{pcomments}
  \pcomment{PS_probability_of_peeta_bread}
  \pcomment{originate by dheins, Apr 2012}
  \pcomment{modified by djcshi}
  \pcomment{problem reworded, solution corrected by yingz, 4/29/2014}
\end{pcomments}

\pkeywords{
  probability
  expectation
  linearity_of_expectation
}

%%%%%%%%%%%%%%%%%%%%%%%%%%%%%%%%%%%%%%%%%%%%%%%%%%%%%%%%%%%%%%%%%%%%%
% Problem starts here
%%%%%%%%%%%%%%%%%%%%%%%%%%%%%%%%%%%%%%%%%%%%%%%%%%%%%%%%%%%%%%%%%%%%%

\begin{problem}
Peeta bakes between $1$ and $2n$ loaves of bread to sell every day.
Each day he rolls a fair, $n$-sided die to get a number from 1 to $n$,
then flips a fair coin.  If the coin is heads, he bakes $m$ loaves of bread , where $m$ is the number on the die that day, and if the coin is
tails, he bakes $2m$ loaves.

\begin{problemparts}

\ppart For any positive integer $k \le 2n$, what is the probability
that Peeta will make $k$ loaves of bread on any given day?
(Hint: you can express your solution by cases.)

\begin{solution}
Each possible outcome on any given day has the form of
$(c,d)$, where $c \in \{H,T\}$ and $d \in [1,n]$.  If $c=H$, every $k<n$
is possible to be the number of loaves of bread, and if $c=T$, every
even $k$ is possible to be the number.  Therefore, there are 4 possible cases:

If $k$ is odd and $k \le n$, then the probability is $\frac{1}{2n}$.

If $k$ is even and $k \le n$, then the probability is $\frac{1}{n}$.

If $k$ is odd and $k > n$, then the probability is $0$.

If $k$ is even and $k > n$, then the probability is $\frac{1}{2n}$.

\end{solution}

\ppart What is the expected number of loaves that Peeta would bake on any
given day?

\begin{solution}
Since each possible dice-coin outcome is equally likely,
we can sum the number of loaves for each outcome
and then divide the sum by the total number of outcomes (i.e., multiply the sum by the probability of each outcome).

If $c=H$, then the sum of the $n$ possible outcomes is
$\sum_{i=1}^{n}{i} = \frac{n(n+1)}{2}$.

If $c=T$, then the sum of the $n$ possible outcomes is doubled,
$\sum_{i=1}^{n}{2i} = n(n+1)$.

Therefore, the expected value is 
$\frac{1}{2n}\left(\frac{n(n+1)}{2} + n(n+1)\right) = \frac{3}{4}(n+1)$.

\end{solution}

\ppart Continuing this process, Peeta bakes bread every day for 30 days.
What is the expected total number of loaves Peeta would have baked?
\begin{solution}
\[
30 \paren{\frac{3}{4}(n+1)} = \frac{45}{2}(n+1)
\]
\end{solution}

\end{problemparts}

\end{problem} 

%%%%%%%%%%%%%%%%%%%%%%%%%%%%%%%%%%%%%%%%%%%%%%%%%%%%%%%%%%%%%%%%%%%%% 
% Problem ends here 
%%%%%%%%%%%%%%%%%%%%%%%%%%%%%%%%%%%%%%%%%%%%%%%%%%%%%%%%%%%%%%%%%%%%%

\endinput
