\documentclass[problem]{mcs}

\begin{pcomments}
  \pcomment{PS_probability_of_peeta.tex}
  \pcomment{originate by dheins}
  \pcomment{modified by djcshi}
\end{pcomments}

\pkeywords{
  probability
  expectation
  linearity_of_expectation
}

%%%%%%%%%%%%%%%%%%%%%%%%%%%%%%%%%%%%%%%%%%%%%%%%%%%%%%%%%%%%%%%%%%%%%
% Problem starts here
%%%%%%%%%%%%%%%%%%%%%%%%%%%%%%%%%%%%%%%%%%%%%%%%%%%%%%%%%%%%%%%%%%%%%

\begin{problem}
Peeta bakes between $1$ and $2n$ loaves of bread to sell every day.
Each day he rolls a fair, $n$-sided die to get a number from 1 to $n$,
then flips a fair coin.  If the coin is heads, he bakes a number of
loaves of bread equal to the value on the die, and if the coin is
tails, he bakes twice that many loaves.

\begin{problemparts}

\ppart For any positive integer $k \le 2n$, What is the probability
that Peeta will make $k$ loaves of bread on any given day?
(You can express your solution by cases.)
\begin{solution}
Since the dice and the coin are independent,
the change of each $2n$ possible outcome is $\frac{1}{2n}$.
Each outcome have the form of $(c,d)$, where $c \in \{H,T\}$ and $0<d\le n$.
If $c=H$, every $k<n$ is possible to be the number of loaves of bread,
and if $c=T$, every even $k$ is possible to be the number.

Therefore, there are four possible cases.
If $k$ is odd and $k \le n$, then the probability is $\frac{1}{2n}$.
If $k$ is even and $k \le n$, then the probability is $\frac{1}{n}$.
If $k$ is odd and $k > n$, then the probability is $0$.
If $k$ is even and $k > n$, then the probability is $\frac{1}{n}$.
\end{solution}

\ppart What is the expected number of loaves Peeta will bake on any given day?
\begin{solution}
Since each possible coin-dice pair outcome is equally likely,
we can sum the number of loaves of bread from each outcome
and then divid by the total number of outcomes.

If $c=H$, then the sum of the $n$ possible outcomes is
$\sum_{i=1}^{n}{i} = \frac{n(n+1)}{2}$.

If $c=T$, then the sum of the $n$ possible outcomes is doubled,
$\sum_{i=1}^{n}{2i} = n(n+1)$.

Therefore, the expected value is 
$\frac{1}{2n}\left(\frac{n(n+1)}{2} + n(n+1)\right) = \frac{3}{4}(n+1)$.

%ANSWER: (3n)/2
\end{solution}

\ppart Continuing this process, Peeta bakes bread every day for 30 days.
What is the expected total number of loaves Peeta will have baked?
\begin{solution}
\[
30 \paren{\frac{3}{4}(n+1)} = \frac{45}{2}(n+1)
\]
\end{solution}

\end{problemparts}

\end{problem} 

%%%%%%%%%%%%%%%%%%%%%%%%%%%%%%%%%%%%%%%%%%%%%%%%%%%%%%%%%%%%%%%%%%%%% 
% Problem ends here 
%%%%%%%%%%%%%%%%%%%%%%%%%%%%%%%%%%%%%%%%%%%%%%%%%%%%%%%%%%%%%%%%%%%%%

\endinput
