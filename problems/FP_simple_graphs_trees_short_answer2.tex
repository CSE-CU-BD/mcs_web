\documentclass[problem]{mcs}

\begin{pcomments}
  \pcomment{FP_simple_graphs_trees_short_answer2}
  \pcomment{variant of FP_simple_graphs_trees_short_answer}
  \pcomment{F15.conflict-final2}
  \pcomment{excerpted from FP_graphs_short_answer, MQ_matching,
      FP_multiple_choice_unhidden, MQ_list_isomorphisms}
  \pcomment{ARM 12/17/15} 
\end{pcomments}

\pkeywords{
  simple_graph
  tree
  isomorphism
  bottle_neck
  chromatic
  complete_graph
}

%%%%%%%%%%%%%%%%%%%%%%%%%%%%%%%%%%%%%%%%%%%%%%%%%%%%%%%%%%%%%%%%%%%%%
% Problem starts here
%%%%%%%%%%%%%%%%%%%%%%%%%%%%%%%%%%%%%%%%%%%%%%%%%%%%%%%%%%%%%%%%%%%%%

\begin{problem} 
Answer the following questions about \textbf{finite simple graphs}.
You may answer with formulas involving exponents, binomial
coefficents, and factorials.

\bparts

\ppart How many edges are there in the \emph{complete graph} $K_{14}$?
\hfill \examrule

\begin{solution}
\[
\binom{14}{2}= 91.
\]
\end{solution}

\ppart How many edges are there in a spanning tree of $K_{14}$? \hfill \examrule

\begin{solution}
\textbf{13}.
\end{solution}

\ppart What is the chromatic number $\chi(K_{14})$? \hfill \examrule
\begin{solution}
14.
\end{solution}

\ppart What is the chromatic number $\chi(C_{14})$, of the cycle of
  length 14? \hfill \examrule

\begin{solution}
\textbf{2}.
\end{solution}

\ppart Let $H$ be the graph in Figure~\ref{self-graph}.
How many distinct isomorphisms are there from $H$ to $H$?
\hfill \examrule

\begin{figure}[h]
\graphic[height=0.75in]{4-chromatic}
\caption{The graph $H$.}
\label{self-graph}
\end{figure}

\begin{solution}
\textbf{8}.

You can flip the entire diagram about a central vertical axis, or flip
the left-hand diamond-shape about its central axis (a $60^o$
line), or flip the right-hand diamond-shape about its central axis (a
$120^o$ line).  Each of these flips can be performed
independently for a total of $2 \cdot 2 \cdot 2$ possible isomorphims.
\end{solution}

\iffalse
\ppart How many cycles can there be in a graph created by adding a
single edge\\
to a tree with 41 vertices? \hfill\examrule

\begin{solution}
Exactly \textbf{1}.
\end{solution}
\fi


\ppart What is the largest possible number of leaves possible in a
tree with 14 vertices? \hfill \examrule
\begin{solution}
\textbf{13}.
\end{solution}


\ppart What is the largest number of degree-two vertices possible in a
tree with 14 vertices? \hfill \examrule
\begin{solution}
\textbf{12}.  All the vertices of the line graph $L_{14}$ are degree-two except for
the necessary two leaves.
\end{solution}


\iffalse
\ppart How many trees are there whose vertices are the integers
$\Zintv{1}{41}$? \hfill \examrule
\begin{solution}
$41^{39}$.  See Problem~(\bref{CP_numbered_trees}).
\end{solution}
\fi

\ppart\label{numl11p} How many length 5-paths are there in $K_{14}$?

\exambox{0.7in}{0.4in}{0in}

\begin{solution}
\[
14 \cdot 13 \cdot 12 \cdot 11 \cdot 10 = \frac{14!}{9!}.
\]
There is a bijection between length-5 paths and length-5 sequences
of distinct vertices.
\end{solution}

\ppart Let $s$ be the number of length-5 paths in $K_{14}$---that is,
$s$ is the correct answer to part~\eqref{numl11p}.  How many length-5
\emph{cycles} are in $K_{14}$?

\exambox{0.7in}{0.4in}{0in}

\hint For vertices $a,b,c,d$, the sequences $abcd$, $bcda$ and
$dcba$ all describe the same length-4 cycle.

\begin{solution}
\[\frac{s}{5 \cdot 2}\]

 A length-5 cycle is determined by the sequence of 5 vertices in it,
 but since a cycle has no ``first'' or ``last'' vertex, and in a
 simple graph it has no direction, 5 different sequences determine the
 same cycle in clockwise order, and another 5---the reversals of the
 first 5---determine it in counterclockwise order.  So the mapping
 from sequences to cycles is $5 \cdot 2$ to one.
\end{solution}

\iffalse
\ppart How many cycles can there be in a graph created by adding two\\
edges to a tree with 41 vertices? \hfill \examrule

\begin{solution}
\textbf{2 or 3}.
\end{solution}
\fi


\eparts

\end{problem}

\endinput
