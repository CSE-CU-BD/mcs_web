\documentclass[problem]{mcs}

\begin{pcomments}
  \pcomment{PS_equivalence_relation_operations_part_a}
  \pcomment{slightly modified by Jenn Wong 3/29/13 from
    PS_equivalence_relation_operations}
\end{pcomments}

\pkeywords{
  equivalence_relations
  reflexive
  transitive
  symmetric
}

%%%%%%%%%%%%%%%%%%%%%%%%%%%%%%%%%%%%%%%%%%%%%%%%%%%%%%%%%%%%%%%%%%%%%
% Problem starts here
%%%%%%%%%%%%%%%%%%%%%%%%%%%%%%%%%%%%%%%%%%%%%%%%%%%%%%%%%%%%%%%%%%%%%

\begin{problem}
Let $R_1$ and $R_2$ be two equivalence relations on a set $A$.  Prove
that $R_1 \intersect R_2$ is also an equivalence relation.
\iffalse
, by showing
that it is reflexive, transitive, and symmetric.
\fi

\begin{solution}
Let $R \eqdef R_1 \intersect R_2$. 

\begin{proof}
We prove that $R$ is an equivalence relation by showing that $R$ is
reflexive, symmetric, and transitive.

\emph{Reflexive:} $R_i$ is reflexive because it is an equivalence
relation, for $i=1,2$.  So $(a,a) \in R_i$ for $i=1,2$ and all $a \in A$.
So, $(a,a) \in (R_1 \intersect R_2) = R$ for all $a \in A$, that is, $R$
is reflexive.

\emph{Transitive:} Suppose $(a,b), (b,c) \in R$.  Since $R = R_1
\intersect R_2$, we have $(a,b), (b,c) \in R_i$ for $i=1,2$.  But $R_i$ is
an equivalence relation, and so is transitive.  Therefore, $(a,c) \in
R_i$, and so $(a,c) \in R_1\intersect R_2 = R$.  This shows that $R$ is
transitive.

\emph{Symmetric:} The proof that $R$ is symmetric follows the same format.
\end{proof}

Other proofs are possible based on the alternative characterizations
of equivalence relations in terms of partitions
(Theorem~\bref{equiv-partition_thm}) or having the same functional
value (Definition~\bref{equiv_f}).
\end{solution}

\end{problem}


%%%%%%%%%%%%%%%%%%%%%%%%%%%%%%%%%%%%%%%%%%%%%%%%%%%%%%%%%%%%%%%%%%%%%
% Problem ends here
%%%%%%%%%%%%%%%%%%%%%%%%%%%%%%%%%%%%%%%%%%%%%%%%%%%%%%%%%%%%%%%%%%%%%

\endinput
