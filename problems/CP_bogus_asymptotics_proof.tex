\documentclass[problem]{mcs}

\begin{pcomments}
  \pcomment{CP_bogus_asymptotics_proof}
  \pcomment{from: S09.cp9t}
\end{pcomments}

\pkeywords{
  asymptotics
  induction
  false_proof
}

%%%%%%%%%%%%%%%%%%%%%%%%%%%%%%%%%%%%%%%%%%%%%%%%%%%%%%%%%%%%%%%%%%%%%
% Problem starts here
%%%%%%%%%%%%%%%%%%%%%%%%%%%%%%%%%%%%%%%%%%%%%%%%%%%%%%%%%%%%%%%%%%%%%

\begin{problem}
\begin{falseclm*}
\begin{equation}\label{2n1}
2^n = O(1).
\end{equation}
\end{falseclm*}

Explain why the claim is false.  Then identify and explain the mistake in
the following bogus proof.

\begin{bogusproof} The proof by induction on $n$ where the induction
hypothesis, $P(n)$, is the assertion~\eqref{2n1}.

\textbf{base case:}  $P(0)$ holds trivially.

\textbf{inductive step:} We may assume $P(n)$, so there is a constant $c >0$
such that $2^n \leq c \cdot 1$.  Therefore,
\[
2^{n+1} = 2 \cdot 2^n \leq (2c) \cdot 1,
\]
which implies that $2^{n+1} = O(1)$.  That is, $P(n+1)$ holds, which
completes the proof of the inductive step.

We conclude by induction that $2^n = O(1)$ for all $n$.  That is, the
exponential function is bounded by a constant.

\end{bogusproof}

\begin{solution}
A function is $O(1)$ iff it is bounded by a constant, and since
the function $2^n$ grows unboundedly with $n$, it is not $O(1)$.

The mistake in the bogus proof is in its misinterpretation of the
expression $2^n$ in assertion~\eqref{2n1}.  The intended interpretation
of~\eqref{2n1} is
\begin{equation}\label{f=exp}
\text{Let $f$ be the function defined by the rule $f(n) \eqdef 2^n$.  Then
$f = O(1)$.}
\end{equation}
But the bogus proof treats~\eqref{2n1} as an assertion, $P(n)$, about $n$.
Namely, it misinterprets~\eqref{2n1} as meaning:
\begin{quote}
  Let $f_n$ be the constant function equal to $2^n$.  That is, $f_n(k)
  \eqdef 2^n$ for all $k \in \naturals$.  Then
\begin{equation}\label{fn=c}
f_n = O(1).
\end{equation}
\end{quote}
Now~\eqref{fn=c} is true since every constant function is $O(1)$, and the
bogus proof is an unnecessarily complicated, but \emph{correct}, proof that
that for each $n$, the constant function $f_n$ is $O(1)$.  But in the
last line, the bogus proof switches from the misinterpretation~\eqref{fn=c}
and claims to have proved~\eqref{f=exp}.

So you could say that the exact place where the proof goes wrong is in its
first line, where it defines $P(n)$ based on
misinterpretation~\eqref{fn=c}.  Alternatively, you could say that the
proof was a correct proof (of the misinterpretation), and its first mistake
was in its last line, when it switches from the misinterpretation to the
proper interpretation~\eqref{f=exp}.
\end{solution}

\end{problem}

%%%%%%%%%%%%%%%%%%%%%%%%%%%%%%%%%%%%%%%%%%%%%%%%%%%%%%%%%%%%%%%%%%%%%
% Problem ends here
%%%%%%%%%%%%%%%%%%%%%%%%%%%%%%%%%%%%%%%%%%%%%%%%%%%%%%%%%%%%%%%%%%%%%
