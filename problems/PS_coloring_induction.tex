\documentclass[problem]{mcs}

\begin{pcomments}
  \pcomment{PS_coloring_induction}
  \pcomment{conflicts with FP_bogus_coloring_proof}
\end{pcomments}

\pkeywords{
graph
coloring
induction
vertex_degree
}

%%%%%%%%%%%%%%%%%%%%%%%%%%%%%%%%%%%%%%%%%%%%%%%%%%%%%%%%%%%%%%%%%%%%%
% Problem starts here
%%%%%%%%%%%%%%%%%%%%%%%%%%%%%%%%%%%%%%%%%%%%%%%%%%%%%%%%%%%%%%%%%%%%%

\begin{problem}
Let $G$ be a simple graph whose vertex degrees are all $\leq k$.
Prove by induction on number of vertices that if every connected
component of $G$ has a vertex of degree strictly less than $k$, then
$G$ is $k$-colorable.

\begin{solution}
Proof by induction on the number $n$ of vertices:

The induction hypothesis, $P(n)$ is:
\begin{quote}
Let $G$ be an $n$-vertex graph whose vertex degrees are all $\leq k$
and such that every connected component of $G$ also has a vertex of
degree strictly less than $k$, then $G$ is $k$-colorable.
\end{quote}

\inductioncase{Base case}: ($n=1$) $G$ has one vertex, the degree of
which must be 0 (since simple graphs have no self-loops).  Since $G$
is 1-colorable, $P(1)$ holds.

\inductioncase{Inductive step}: We may assume that $n\ geq 1$ and
$P(n)$.  To prove $P(n+1)$, let $G_{n+1}$ be a connected graph with
$n+1$ vertices whose vertex degrees are all $k$ or less.  Also,
suppose $G_{n+1}$ has a vertex $v$ of degree strictly less than $k$.
At this point, we only need to prove that $G_{n+1}$ is $k$-colorable.

To do this, first remove the vertex $v$ to produce a graph, $G_n$,
with $n \geq 1$ vertices.  Removing $v$ may split one connected
component of $G_{n+1}$ into several connected components of $G_n$, but
each connected component of $G_n$ that is not also a connected
component of $G_{n+1}$ must contain at least one vertex $u$ adjacent
to $v$.  Moreover, removing $v$ reduces by 1 the degree of each vertex
$u$ adjacent to $v$.

Since no edges were added, the vertex degrees of $G_n$ remain $\leq
k$.  Moreover, every connected component of $G_n$ has a vertex of
degree strictly less than $k$.  So $G_n$ satisfies the conditions of
the induction hypothesis, $P(n)$, and so we conclude that $G_n$ is
$k$-colorable.

Now a $k$-coloring of $G_n$ gives a coloring of all the vertices of
$G_{n+1}$, except for $v$.  Since $v$ has degree less than $k$, there will
be fewer than $k$ colors assigned to the nodes adjacent to $v$.  So among
the $k$ possible colors, there will be a color not used to color these
adjacent nodes, and this color can be assigned to $v$ to form a
$k$-coloring of $G_{n+1}$.

\end{solution}
\end{problem}

\endinput
