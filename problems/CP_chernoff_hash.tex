\documentclass[problem]{mcs}

\begin{pcomments}
  \pcomment{CP_chernoff_hash}
  \pcomment{based on Dec 7, 04 notes}
\end{pcomments}

\pkeywords{
  hash
  chernoff
}

%%%%%%%%%%%%%%%%%%%%%%%%%%%%%%%%%%%%%%%%%%%%%%%%%%%%%%%%%%%%%%%%%%%%%
% Problem starts here
%%%%%%%%%%%%%%%%%%%%%%%%%%%%%%%%%%%%%%%%%%%%%%%%%%%%%%%%%%%%%%%%%%%%%

\begin{problem}

We want to store 2 billion records into a hash table that has 1
billion slots.  Assuming the records are randomly and independently
chosen with uniform probability of being assigned to each slot, two
records are expected to be stored in each slot.  Of course under a
random assignment, some slots may be assigned more than two records.

\bparts

\ppart Show that the probability that a given slot gets assigned more
than 23 records is less than $e^{-36}$.

\hint For $c = 12$, the value of $c \ln c -c +1$ is greater than 18.

\begin{solution}
Let $T$ be the number of records assigned to a particular slot, say
the first one.  So $\expect{T} = 2$.  Then by Chernoff
\[
\prob{T \geq 24} = \prob{T \geq 12 \expect{T}}
                 \leq e^{-\beta(12) \expect{T}}
                 < e^{-18 \cdot 2} = e^{-36}.
\]
\end{solution}

\ppart Show that the probability that there is a slot that gets
assigned more than 23 records is less than $e^{-15}$.  This is less
than $1/3,000,000$.  \hint $\ln 10^9 < 21$.

\begin{solution}
By the Union Bound, the probability that some slot gets assigned more
than 23 records is at most 1 billion times the probability that each
particular slot gets assigned
more than 23 records, and is therefore
\[
\leq 10^9 \cdot e^{-36} < e^{21} \cdot e^{-36} = e^{-15} <
\frac{1}{3,270,000} < \frac{1}{3,000,000}.
\]

\end{solution}

\begin{editingnotes}
\pparts  compare to log search size $\log 10^9 < 21$.
\end{editingnotes}
\eparts


\inhandout{

\textbf{The Chernoff Bound}: Let $T$ be the sum of a finite number of
mutually independent variables whose codomain is the real interval
$[0,1]$.  Then for all $c \geq 1$,
\[
\prob{T \geq c \expect{T}} \leq e^{-\beta(c) \expect{T}}
\]
where $\beta(c) \eqdef c \ln c - c + 1$.
}

\end{problem}
%%%%%%%%%%%%%%%%%%%%%%%%%%%%%%%%%%%%%%%%%%%%%%%%%%%%%%%%%%%%%%%%%%%%%
% Problem ends here
%%%%%%%%%%%%%%%%%%%%%%%%%%%%%%%%%%%%%%%%%%%%%%%%%%%%%%%%%%%%%%%%%%%%%

\endinput
