\documentclass[problem]{mcs}

\begin{pcomments}
  \pcomment{FP_random_sampling}
  \pcomment{from: F09 tutor problem practice13/sampling_conceptual}
  \pcomment{similar to FP_sampling_concepts, TP_sampling_perturbed}
\end{pcomments}

\pkeywords{
  sampling
  confidence
  independence
}

%%%%%%%%%%%%%%%%%%%%%%%%%%%%%%%%%%%%%%%%%%%%%%%%%%%%%%%%%%%%%%%%%%%%%
% Problem starts here
%%%%%%%%%%%%%%%%%%%%%%%%%%%%%%%%%%%%%%%%%%%%%%%%%%%%%%%%%%%%%%%%%%%%%

\begin{problem}

You work for the president and you want to estimate the fraction $p$
of voters in the entire nation that will prefer him in the upcoming
elections.  You do this by \idx{random sampling}.  Specifically, you
select $n$ voters independently and randomly, ask them who they are
going to vote for, and use the fraction $P$ of those that say they
will vote for the President as an estimate for $p$.

\bparts

\ppart[4] Our theorems about \idx{sampling} and distributions allow us
to calculate how confident we can be that the random variable, $P$,
takes a value near the constant, $p$.  This calculation uses some
facts about voters and the way they are chosen.  Which of the
following facts are true?

\begin{enumerate}

\item Given a particular voter, the probability of that voter preferring
  the President is $p$.

\item Given a particular voter, the probability of that voter preferring
  the President is 1 or 0.

\item The probability that some voter is chosen more than once in the
  sequence goes to zero as $n$ increases.

\item All voters are equally likely to be selected as the third in our
  sequence of $n$ choices of voters (assuming $n \geq 3$).

\item The probability that the second voter chosen will favor the
  President, given that the first voter chosen prefers the President, is
  greater than $p$.

\item The probability that the second voter chosen will favor the
  President, given that the second voter chosen is from the same state as
  the first, may not equal $p$.

\end{enumerate}

\begin{solution}

  The preference of any particular voter is a constant: either "the
  President" or "not the President", so (1) is false and (2) is true.  (3)
  is false; in fact, the Birthday "paradox" implies the probability of
  some voter being chosen more than once rapidly approaches 1 as $n$ grows
  beyond 100.  (4) holds by definition of randomly choosing an item from a
  set.  (5) is false because successive voters in the sequence are chosen
  independently.  (6) is true because, for example, the fraction of voters
  who prefer the President in the largest states may all be $< p$.

\end{solution}

\ppart
Suppose that according to your calculations, the following is true about your polling:
\[
\pr{\abs{P-p} \leq 0.04} \geq 0.95.
\]

You do the asking, you count how many said they will vote for the
President, you divide by $n$, and find the fraction is 0.53.  You call the
President, and \dots what do you say?

\begin{enumerate}

\item Mr. President, $p=0.53$!

\item Mr. President, with probability at least 95 percent,
                $p$ is within 0.04 of 0.53.

\item Mr. President, either $p$ is within 0.04 of 0.53
                or something very strange (5-in-100) has happened.

\item Mr. President, we can be 95\% confident that p is within 0.04 of
  0.53.

\end{enumerate}

\begin{solution}

  You cannot say (1): the only way to know the exact value of the constant
  $p$ is to ask all 250,000,000 voters.

  You cannot say (2) either: $p$ is a constant which can either be or not
  be within 0.04 of 0.53.  If it is, then the probability that it is is 1,
  and thus at least 0.95, and therefore (2) will be true.  If it is not,
  then the probability that it is is 0, and thus smaller than 0.95, and
  therefore (2) will be false.

  You can say (3): To see why, start with the statement either $\abs{0.53
    - p} \leq 0.04$ or $\abs{0.53 - p} > 0.04$ which is obviously true.
  Now read it as follows: Either $p$ is within 0.04 of 0.53 or it is not
  and therefore my random variable $P$ took a value from a set that is hit
  only 5 times in 100.  So, clearly, either $p$ is within 0.04 of 0.53 or
  something strange has happened.

  It's probably best to say (4): it means the same thing as (3), but
  is simpler to say and will be easily understandable by anyone with
  basic knowledge of sampling theory.  And for people without such
  knowledge, (4) will make intuitive sense and be less likely to
  cause confusion than the fuller explanation in (3).
\end{solution}

\eparts

\end{problem}

%%%%%%%%%%%%%%%%%%%%%%%%%%%%%%%%%%%%%%%%%%%%%%%%%%%%%%%%%%%%%%%%%%%%%
% Problem ends here
%%%%%%%%%%%%%%%%%%%%%%%%%%%%%%%%%%%%%%%%%%%%%%%%%%%%%%%%%%%%%%%%%%%%%

\endinput
