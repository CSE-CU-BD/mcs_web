\documentclass[problem]{mcs}

\begin{pcomments}
    \pcomment{TP_a_bogus_fibonacci_induction}
    \pcomment{renamed from TP_A_Bogus_Induction}
    \pcomment{variation of TP_another__bogus_fibonacci_induction}
    \pcomment{Converted from prob4.scm by scmtotex and dmj
             on Sun 13 Jun 2010 10:18:32 AM EDT}
    \pcomment{revised by drewe 18 july 2011}
    \pcomment{edited by ARM 8/25/11}
\end{pcomments}

\pkeywords{
induction
strong_induction
Fibonacci
bogus
logical_error
stategic_error
}

\begin{problem}
\inhandout{The Fibonacci numbers $F(0) F(1) F(2), \dots$ are defined as follows:
\[
F(n) \eqdef \begin{cases}
  0               & \mbox{if $n = 0$},\\
  1               & \mbox{if $n = 1$},\\
  F(n-1) + F(n-2) & \mbox{if $n >1$}.
\end{cases}
\]}
\inbook{The Fibonacci numbers $F(n)$ are described in Section~\bref{sec:fib}.}

Which sentences in the bogus proof below contain logical errors?

\begin{falseclm*}
Every Fibonacci number is even.
\end{falseclm*}
      
\begin{falseproof}
	 
\begin{enumerate}
         
\item
We use strong induction.
         
\item
The induction hypothesis is that $F(n)$ is even.
         
\item
We will first show that this hypothesis holds for $n = 0$.
         
\item
This is true, since $F(0) = 0$, which is an even number.
         
\item
Now, suppose $n \ge 2$. We will show that $F(n)$ is even, assuming that
$F(k)$ is even for all $k < n$.
         
\item
By assumption, both $F(n-1)$ and $F(n-2)$ are even.
         
\item
Therefore, $F(n)$ is even, since $F(n) = F(n-1) + F(n-2)$ and the sum
of two even numbers is even.
         
\item\label{last-fib-step}
Thus, the strong induction principle implies that $F(n)$ is even for
all $n > 0$.
        
\end{enumerate}

\end{falseproof}

\begin{solution}
\ref{last-fib-step}. %8.

All steps but the last contain no logical errors.  The fatal flaw is
in step~\ref{last-fib-step}.

Using strong induction, we can conclude that a predicate $P(n)$ holds
for all $n \ge 0$ provided that we show all of the following:

\begin{itemize}
    
\item $P(0)$
    
\item $P(0) \QIMPLIES P(1)$
    
\item $[P(0) \QAND P(1)] \QIMPLIES P(2)$
    
\item $[P(0) \QAND P(1) \QAND P(2)] \QIMPLIES  P(3)$
    
\item \emph{etc.}

\end{itemize}

The first assertion is proved on lines 3 and~4.  The third and
subsequent assertions are proved on lines 5--7.  However, the second
assertion, $P(0) \QIMPLIES P(1)$, is proved nowhere.  Indeed, it can't
be proved because it is actually false.  Therefore, we cannot apply
the strong induction principle in step~8.

If you said that step 5 contains a logical error, you were on the
right track.  The natural place to prove the second assertion would
have been in lines 5--7.  But by saying, ``suppose $n \ge 2$'' instead
of ``suppose $n \ge 1$'', the proof explicitly avoided doing so.

Technically, there is no \emph{logical} error in line~5: it is simply
the beginning of a proof for the case when $n \ge 2$.  On the other
hand, it's reasonable to say that line~5 is the place where the proof
makes a \emph{strategic} error because it skips the $n = 1$ case.
\end{solution}

\end{problem}

\endinput
