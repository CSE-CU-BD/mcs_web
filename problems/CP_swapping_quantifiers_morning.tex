\documentclass[problem]{mcs}

\begin{pcomments}
  \pcomment{problem from S10}
  \pcomment{CP_swapping_quantifiers_morning}
\end{pcomments}

\pkeywords{
  predicate
  quantifier
}

%%%%%%%%%%%%%%%%%%%%%%%%%%%%%%%%%%%%%%%%%%%%%%%%%%%%%%%%%%%%%%%%%%%%%
% Problem starts here
%%%%%%%%%%%%%%%%%%%%%%%%%%%%%%%%%%%%%%%%%%%%%%%%%%%%%%%%%%%%%%%%%%%%%

\begin{problem} \mbox{}
\inbook{Indicate}\inhandout{Circle} the invalid implication, and
provide a counter model for your answer.
\begin{enumerate}
\item $\exists y, \forall x. P(x, y) \implies \forall x, \exists y. P(x, y)$
\item $\forall x, \exists y. P(x, y) \implies \exists y, \forall x. P(x, y)$
\end{enumerate}

\begin{solution}
The second implication, $\forall x, \exists y. P(x, y) \implies
\exists y, \forall x. P(x, y)$, is invalid.

An example counter model is the predicate $P(x, y) = y < x$ where the
domain of discourse is $\mathbb{R}$. For every real number $x$, there
exists a real number $y$ which is strictly less than $x$. So while the
antecedent of the implication is true, the consequence is not since
there is no minimum element for the partial order, the strictly less
than relation, $<$, on $\mathbb{R}$.
\end{solution}

\end{problem}

\end{problem}

%%%%%%%%%%%%%%%%%%%%%%%%%%%%%%%%%%%%%%%%%%%%%%%%%%%%%%%%%%%%%%%%%%%%%
% Problem ends here
%%%%%%%%%%%%%%%%%%%%%%%%%%%%%%%%%%%%%%%%%%%%%%%%%%%%%%%%%%%%%%%%%%%%%
