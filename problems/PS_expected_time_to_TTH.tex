\documentclass[problem]{mcs}

\begin{pcomments}
  \pcomment{CP_expected_time_to_TTH}
  \pcomment{from: SP11}
\end{pcomments}

\pkeywords{
  expectation
  tree_diagram
  subtree
  expected_time
  coin_flip
}

\begin{problem}
A coin will be flipped repeatedly until the sequence tail/tail/head
(\texttt{TTH}) comes up.  Successive flips are independent, and the coin has
probability $p$ of coming up heads.  Let $N_\mathtt{\texttt{TTH}}$ be the
number of coin flips until \texttt{TTH} first appears.  What value of $p$
minimizes $\expect{N_\texttt{TTH}}$?

\begin{solution}
We can describe the event tree, $D$, for the coin flips as follows:
\[
D = \texttt{H}\cdot D + \texttt{T}\cdot (\texttt{H}\cdot D + \texttt{T}\cdot G). 
\]
where
\[
G = \texttt{H} + \texttt{T}\cdot G.
\]
Abusing notation slightly, we can describe $D$ as:
\[
D = H\cdot D + T\cdot H\cdot D + T\cdot T\cdot G.
\]
Now we compute $\expect{N_\mathtt{TTH}}$:
\[
\expect{N_\mathtt{TTH}} = p\paren{1 + \expect{N_\mathtt{TTH}}}  + 
(1-p)p ( 2 + \expect{N_\mathtt{TTH}}) + 
 (1-p)^2 (2 + \expect{N_\mathtt{H}})
\]

We know $\expect{N_\mathtt{H}} = 1/p$, so we can focus on the rest of the tree.
\[ 
\expect{N_\mathtt{TTH}} = (p + (1-p)p) \expect{N_\mathtt{TTH}} + p + 2(1-p)p + (1-p)^2( 2 + 1/p)
\]

\iffalse
Notice how the $2 + 1/p$ term corresponds to the mean time to failure
of the variable given we see \texttt{TT} at the start. Also, if $p =
0$ the $1/p$ term grows infinite, whereas if $p = 1$ the $(1-p)^2$
term grows infinite.
\fi

\begin{align*}
\lefteqn{\expect{N_\mathtt{TTH}}}\\
& = \frac{p + 2(1-p)p + (1-p)^2( 2 + 1/p)}{1 - p - (1-p)p}\\
& = \frac{p(3 - 2p)}{(1-p)^2} + 2 + \frac1p = \frac{1}{p(1-p)^2}
\end{align*}

Now that we have $\expect{N_\mathtt{TTH}}$ in terms of $p$, we can
minimize it using basic calculus.  The derivative of
$\expect{N_\mathtt{TTH}}$ with respect to $p$ is
\[
\frac{1 - 3p }{(p-1)^3p^2}
\]

So the function is minimized at $p = 1/3$, giving an expected time of
$27/4 < 7$ flips to see \texttt{TTH}, in contrast to the expected 8
flips when $p = 1/2$.

\begin{staffnotes}
This generalizes to $\mathtt{T}^n\mathtt{H}}$ with 
\[
\expect{N_\mathtt{T}^n\mathtt{H}} = \frac{1}{p(1-p)^n}
\]
which is minimized when $p = 1/(n+1)$, giving an expected number of flips equal to
\[
\frac{(n+1)^{n+1}}{n^n}.
\]
\end{staffnotes}

\end{solution}
\end{problem}
