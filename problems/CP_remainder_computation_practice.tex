\documentclass[problem]{mcs}

\begin{pcomments}
  \pcomment{CP_remainder_computation_practice}
  \pcomment{by ARM 3/6/12}
  \pcomment{similar to equation \label{ex44427} in number_theory chapter}
\end{pcomments}

\pkeywords{
  number_theory
  modular_arithmetic
  exponent
  remainder
}

%%%%%%%%%%%%%%%%%%%%%%%%%%%%%%%%%%%%%%%%%%%%%%%%%%%%%%%%%%%%%%%%%%%%%
% Problem starts here
%%%%%%%%%%%%%%%%%%%%%%%%%%%%%%%%%%%%%%%%%%%%%%%%%%%%%%%%%%%%%%%%%%%%%

\begin{problem}
Find
\begin{equation}\label{ex3456789}
\remainder\paren{9876^{3456789} \paren{9^{99}}^{5555} - 6789^{3414259},\ 14}.
\end{equation}

\begin{solution}

Its remainder is 7.

Following the General Principle of Remainder Arithmetic from
Section~\bref{remainder_arithmetic_sec}, replace the numbers being
raised to powers by their remainders.  Since $\rem{9876}{14} = 6$ and
$\rem{6789}{14} = 13$, we find that~\eqref{ex3456789} equals the
  remainder on division by 14 of
\begin{equation}\label{ex6345}
6^{3456789} \paren{9^{99}}^{5555} - 13^{3414259}.
\end{equation}
But let's look at the remainders of powers of 6:
\begin{align*}
\rem{6^1}{14} & = 6\\
\rem{6^2}{14} & = 8\\
\rem{6^3}{14} & = 6\\
\rem{6^4}{14} & = 8\\
              & \vdots
\end{align*}
That is, the remainder on division by 14 of 6 raised to any odd power is 6.
In particular
\[
\rem{6^{3456789}}{14} = 6
\]

Similarly,
\begin{align*}
\rem{9^1}{14} & = 9\\
\rem{9^2}{14} & = 11\\
\rem{9^3}{14} & = 1,
\end{align*}
so
\[
\rem{9^{99}}{14} = \rem{\paren{9^{3}}^{33}}{14} = \rem{1^{33}}{14} = 1,
\]
and therefore
\[
\rem{\paren{9^{99}}^{5555}}{14} = \rem{1^{5555}}{14} = 1.
\]
Finally,
\begin{align*}
\rem{13^1}{14} & = 13\\
\rem{13^2}{14} & = 1,
\end{align*}
so
\[
\rem{13^{3456789}}{14} = \rem{13\cdot \paren{13^2}^{34567878/2}}{14} = \rem{13\cdot 1^{34567878/2}}{14} = 13.
\]

Therefore, the number~\eqref{ex6345} has the same remainder on division
by 14 as
\[
6 \cdot 1 - 13 = -7,
\]
which has the same remainder on division by 14 as -7, namely 7.

Notice that \textbf{it would be a disastrous blunder to replace an
  exponent by its remainder}.  The General Principle applies to
numbers that are \emph{operands} of plus and times, whereas the
exponent is a number that controls how many multiplications to
perform.  Watch out for this blunder.
\end{solution}

\end{problem}

%%%%%%%%%%%%%%%%%%%%%%%%%%%%%%%%%%%%%%%%%%%%%%%%%%%%%%%%%%%%%%%%%%%%%
% Problem ends here
%%%%%%%%%%%%%%%%%%%%%%%%%%%%%%%%%%%%%%%%%%%%%%%%%%%%%%%%%%%%%%%%%%%%

\endinput
