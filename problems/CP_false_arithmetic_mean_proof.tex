\documentclass[problem]{mcs}

\begin{pcomments}
  \pcomment{from: S09.cp1t}
%  \pcomment{}
%  \pcomment{}
\end{pcomments}

\pkeywords{
  faulty_logic
}

%%%%%%%%%%%%%%%%%%%%%%%%%%%%%%%%%%%%%%%%%%%%%%%%%%%%%%%%%%%%%%%%%%%%%
% Problem starts here
%%%%%%%%%%%%%%%%%%%%%%%%%%%%%%%%%%%%%%%%%%%%%%%%%%%%%%%%%%%%%%%%%%%%%

\begin{problem} It's a fact that the Arithmetic Mean is at least as
large the Geometric Mean, namely,
\[
\frac{a + b}{2} \geq \sqrt{a b}
\]
for all nonnegative real numbers $a$ and $b$.  But there's something
objectionable about the following proof of this fact.  What's the
objection, and how would you fix it?

\begin{bogusproof}
\begin{align*}
\frac{a + b}{2} & \stackrel{?}{\geq} \sqrt{a b}, & \text{ so} \\
a + b  & \stackrel{?}{\geq} 2 \sqrt{a b}, & \text{ so} \\
a^2 + 2 a b + b^2  & \stackrel{?}{\geq} 4 a b, & \text{ so}\\
a^2 - 2 a b + b^2  & \stackrel{?}{\geq} 0, & \text{ so}\\
(a - b)^2  & \geq 0 & \text{ which we know is true.}
\end{align*}

The last statement is true because $a - b$ is a real number, and the
square of a real number is never negative.  This proves the claim.
\end{bogusproof}

\solution{In this argument, we started with what we wanted to prove and
then reasoned until we reached a statement that is surely true.  The
little question marks presumably are supposed to indicate that we're not
quite certain that the inequalities are valid until we get down to the
last step.  At that step, the inequality checks out, \emph{but that
doesn't prove the claim}.  All we have proved is that \textbf{if} $(a + b)/2
\geq \sqrt{a b}$, \textbf{then} $(a - b)^2 \geq 0$, which is not very
interesting, since we already knew that the square of any nonnegative
number is nonnegative.

To be fair, this bogus proof is pretty good: if it was written in reverse
order -- or if ``is implied by'' was simply inserted after each line -- it
would actually prove the Arithmetic-Geometric Mean Inequality:

\begin{proof}

\begin{align*}
\frac{a + b}{2} & \geq \sqrt{a b} & \text{ is implied by}\\
a + b  & \geq 2 \sqrt{a b},  & \text{ which is implied by}\\
a^2 + 2 a b + b^2  & \geq 4 a b, & \text{ which is implied by}\\
a^2 - 2 a b + b^2  & \geq 0, & \text{ which is implied by}\\
(a - b)^2  & \geq 0.
\end{align*}

The last statement is true because $a - b$ is a real number, and the
square of a real number is never negative.  This proves the claim.
\end{proof}

But the problem with the bogus proof as written is that it reasons
backward, beginning with the proposition in question and reasoning to a
true conclusion.  This kind of backward reasoning can easily ``prove''
false statements.  Here's an example:

\textbf{Bogus Claim}: $0 = 1$.
\begin{bogusproof}
\begin{align*}
0 & \stackrel{?}{=} 1, & \text{ so} \\
1 & \stackrel{?}{=} 0, & \text{ so} \\
0+1 & \stackrel{?}{=} 1+0, & \text{ so} \\
1 & = 1 & \text{ which is trivially true,}
\end{align*}
which proves $0=1$.
\end{bogusproof}

We can also come up with very easy ``proofs'' of true theorems, for
example, here's an easy ``proof'' of the Arithmetic-Geometric Mean
Inequality:

\begin{bogusproof}
\begin{align*}
\frac{a + b}{2} & \stackrel{?}{\geq} \sqrt{a b}, & \text{ so}\\
0 \cdot \frac{a + b}{2} & \stackrel{?}{\geq} 0 \cdot \sqrt{a b}, & \text{ so}\\
0 & \geq 0 & \text{ which is trivially true.}\quad \qedhere
\end{align*}
\end{bogusproof}

So watch out for backward proofs!
}

\end{problem}

%%%%%%%%%%%%%%%%%%%%%%%%%%%%%%%%%%%%%%%%%%%%%%%%%%%%%%%%%%%%%%%%%%%%%
% Problem ends here
%%%%%%%%%%%%%%%%%%%%%%%%%%%%%%%%%%%%%%%%%%%%%%%%%%%%%%%%%%%%%%%%%%%%%

\endinput
