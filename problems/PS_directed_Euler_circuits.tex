\documentclass[problem]{mcs}

\begin{pcomments}
  \pcomment{PS_directed_Euler_circuits}
  \pcomment{digraph version of PS_Euler_circuits}
  \pcomment{adapted to digraphs by ARM 3/7/11}
\end{pcomments}

\pkeywords{
  Euler
  Euler_circuit
  Euler_tours
  cycle
  degree
  closed_walk
  digraph
}

%%%%%%%%%%%%%%%%%%%%%%%%%%%%%%%%%%%%%%%%%%%%%%%%%%%%%%%%%%%%%%%%%%%%%
% Problem starts here
%%%%%%%%%%%%%%%%%%%%%%%%%%%%%%%%%%%%%%%%%%%%%%%%%%%%%%%%%%%%%%%%%%%%%

\begin{problem}
  An \term{Euler tour} of a graph is a closed walk that includes every
  edge exactly once.  Such walks are named after the famous 17th
  century mathematician Leonhard Euler.  (Same Euler as for the
  constant $e\approx 2.718$ and the totient function $\phi$ ---he did
  a lot of stuff.)

\iffalse

Does the graph in the following figure contain an Euler tour?

\begin{figure}
\graphic{example}
\end{figure}

Well, if it did, the edge $(E, F)$ would need to be included.  If the
walk does not start at $F$ then at some point it includes edge
$(E,F)$, and now it is stuck at $F$ since $F$ has no other edges
incident to it and an Euler tour can't include $(E,F)$ twice.  But
then the walk could not be a tour.  On the other hand, if the walk
starts at $F$, it must then go to $E$ along $(E,F)$, but now it cannot
return to $F$.  It again cannot be a tour.  This argument generalizes
to show that if a graph has a vertex of degree $1$, it cannot contain
an Euler tour.

\begin{staffnotes}
On the other hand, it is easy to see that any cycle has an Euler
tour. You can just start at any vertex and walk around back to it.
\end{staffnotes}
\fi

So how do you tell in general whether a graph has an Euler tour?  At
first glance this may seem like a daunting problem (the similar
sounding problem of finding a cycle that touches every vertex exactly
once is one of those million dollar NP-complete problems known as the
\term{Traveling Salesman Problem}) ---but it turns out to be easy.

\bparts

\ppart Show that if a graph has an Euler tour, then the in-degree
of each vertex equals its out-degree.

\begin{solution}
Let
\[
C \eqdef v_1\ \diredge{v_1}{v_2}\ v_2\ \dots\ \diredge{v_r}{v_1}\ v_1
\]
be an Euler tour.  Except for the initial and final occcurrences of
$v_1$, each occurrence of a vertex $v$ in the tour is immediately
preceded by an edge $\diredge{u}{v}$ and immediately followed by an
edge $\diredge{v}{w}$.  It follows that if $v\neq v_1$ occurs $s$
times in $C$, then $\indegr{v} = \outdegr{v} = s$ since every edge incident to $v$
occurs in $C$ exactly once.  For the same reason, if $v_1$ occurs $s$ times on the path, 
then $\indegr{v_1} = \outdegr{v_1} = s-1$.
\end{solution}
\eparts

A digraph is \term{weakly connected} if there is a ``path'' between
any two vertices that may follow edges backwards or
forwards.\footnote{More precisely, a graph $G$ is weakly connected
  iff there is a path from any vertex to any other vertex in the graph
  $H$ with
\begin{align*}
\vertices{H} & = \vertices{G}, \text{and}\\
\edges{H} & = \edges{G} \union \set{\diredge{v}{u} \suchthat \diredge{u}{v} \in \edges{G}}.
\end{align*}
\iffalse
is strongly connected.
\fi
In other words $H = G \union G^{-1}$.}  In the remaining parts, we'll
work out the converse: if a graph is weakly connected, and if the
in-degree of every vertex equals its out-degree, then the graph has an
Euler tour.  To do this, let's define an Euler \term{walk} to be a
walk that includes each edge \emph{at most} once.

\bparts

\ppart\label{conn} Suppose that an Euler walk in a connected graph
does not include every edge.  Explain why there must be an edge not on
the walk that starts or ends at a vertex on the walk.

\begin{solution}
  If an edge is not on the Euler walk starts or ends on it, then that
  already is the desired edge.  So suppose there's an edge, $e$, not
  on the walk that neither starts nor ends at a vertex on the walk.
  Since $G$ is weakly connected, there is a path, $\walkv{p}$, in $G
  \union G^{-1}$ from any vertex, $v$, on the walk to an endpoint of
  $e$.  Then the first edge in $\walkv{p}$ that is not on the walk
  will be the desired edge of $G$.
\end{solution}

\eparts

In the remaining parts, let $\walkv{w}$ be the \emph{longest} Euler
walk in the graph.

\bparts

\ppart\label{cycle-circuit} Show that if $\walkv{w}$ is closed, then it
must be an Euler tour.

\hint part~\eqref{conn}

\begin{solution}
Suppose $\walkv{w}$ is closed and some edge is not on $\walkv{w}$.  By
part~\eqref{conn}, there must be an edge, $e$, not on $\walkv{w}$ that
starts or ends on $\walkv{w}$.  If $e$ starts on $\walkv{w}$,
beginning with the start of $e$, go around $\walkv{w}$ back to vertex
$v$, and then the follow $e$.  This makes a longer Euler walk,
contradicting the maximality of $\walkv{w}$.  Similarly, if $e$ ends
on $\walkv{w}$, then the walk that begins with $e$ and then follows
$\walkv{w}$ from the end of $e$ back around to the end of $e$ is a
longer Euler walk, again contradicting maximality.

So no edge can be missing from $\walkv{w}$.
\end{solution}

\ppart\label{already} Explain why all the edges starting at the end of
$\walkv{w}$ must be on $\walkv{w}$.

\begin{solution}
Otherwise we could extend $\walkv{w}$ to a longer Euler walk with any edge
leaving the end not already in $\walkv{w}$.
\end{solution}

\ppart\label{odd} Show that if the end of $\walkv{w}$ was not closed,
then the in-degree of the end would be one bigger than its out-degree.

\hint part~\eqref{already}

\begin{solution}
Let $v$ be the end vertex of $\walkv{w}$.  Given that $v$ is not the
start of $\walkv{w}$, it follows that at any occurrence of $v$ in
$\walkv{w}$ other than at the end, $\walkv{w}$ includes an edge whose
head is $v$ that appears just before that occurrence of $v$, and
another edge whose tail is $v$ that appears just after that occurrence
of $v$.  Since $\walkv{w}$ is an Euler walk, all the edges in all
these pairs are distinct.  In addition, the final edge in $\walkv{w}$
ends at $v$ and is distinct from all the paired edges.  Altogether,
this implies that there are an equal number of edges into $v$ and out
of $v$ in $\walkv{w}$ except for the last edge, and the last edge adds
one more to the in-degree.  But by part~\eqref{already}, these are all
the edges incident to $v$, proving that $\indegr{v} = 1+ \outdegr{v}$.
\end{solution}

\ppart Conclude that if in a finite, weakly connected digraph, the
in-degree of every vertex equals its out-degree, then the digraph has an
Euler tour.

\begin{solution}
If all vertices in $G$ have even degree, then by part~\eqref{odd}, the
only possibility is that the end of $walkv{w}$ equals the start, that
is, $walkv{w}$ is closed.  So by part~\eqref{cycle-circuit},
$walkv{w}$ is an Euler tour.
\end{solution}

\eparts
\end{problem}

%%%%%%%%%%%%%%%%%%%%%%%%%%%%%%%%%%%%%%%%%%%%%%%%%%%%%%%%%%%%%%%%%%%%%
% Problem ends here
%%%%%%%%%%%%%%%%%%%%%%%%%%%%%%%%%%%%%%%%%%%%%%%%%%%%%%%%%%%%%%%%%%%%%

\endinput
