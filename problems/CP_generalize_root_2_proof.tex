\documentclass[problem]{mcs}

\begin{pcomments}
  \pcomment{CP_generalize_root_2_proof}
  \pcomment{from: S09.cp1r}
%  \pcomment{}
\end{pcomments}

\pkeywords{
  root_2
  irrational
  rational
  contradiction
}

%%%%%%%%%%%%%%%%%%%%%%%%%%%%%%%%%%%%%%%%%%%%%%%%%%%%%%%%%%%%%%%%%%%%%
% Problem starts here
%%%%%%%%%%%%%%%%%%%%%%%%%%%%%%%%%%%%%%%%%%%%%%%%%%%%%%%%%%%%%%%%%%%%%

\begin{problem}\label{generprob}
  Generalize the proof of Theorem~\bref{thm:sqrt2irr_by_contra}
  \inhandout{repeated below} that $\sqrt{2}$ is irrational\inhandout{\footnote{Remember that an
    irrational number is a number that cannot be expressed as a ratio
    of two integers.}}.  For example, how about $\sqrt[3]{2}$?\dots or $\sqrt{3}$? 

\inhandout{
\begin{quote}

\begin{theorem*}
$\sqrt{2}$ is an irrational number.
\end{theorem*}

\begin{proof}
  The proof is by contradiction: assume that $\sqrt{2}$ is rational, that
  is,
  \begin{equation}\label{2nd}
    \sqrt{2} = \frac{n}{d},
  \end{equation}
  where $n$ and $d$ are integers.  Now consider the smallest such positive
  integer denominator, $d$.  We will prove in a moment that the numerator,
  $n$, and the denominator, $d$, are both even.  This implies that
         \[
        \frac{n/2}{d/2}
        \]
  is a fraction equal to $\sqrt{2}$ with a smaller positive integer
  denominator, a contradiction.

\begin{quote}
\emph{Since the assumption that $\sqrt{2}$ is rational leads to this
  contradiction, the assumption must be false.  That is, $\sqrt{2}$ is
  indeed irrational.}  This italicized comment on the implication of
the contradiction normally goes without saying, but since this is an
early example of proof by contradiction, we've said it.
\end{quote}

  To prove that $n$ and $d$ have 2 as a common factor, we start by
  squaring both sides of~\eqref{2nd} and get $2 = n^2 / d^2$, so
\begin{equation}\label{2d2}
2 d^2 = n^2.
\end{equation}
So 2 is a factor of $n^2$, which is only possible if 2 is in fact a
factor of $n$.

This means that $n=2k$ for some integer, $k$, so
\begin{equation}\label{n24}
  n^2 = (2k)^2 = 4 k^2.
\end{equation}
Combining~\eqref{2d2} and~\eqref{n24} gives $2 d^2 = 4 k^2$, so
\begin{equation}\label{n22}
d^2 = 2k^2.
\end{equation}
So 2 is a factor of $d^2$, which again is only possible if 2 is in fact
also a factor of $d$, as claimed.
\end{proof}
\end{quote}
}

\begin{solution}
We prove that for any $n>1$, $\sqrt[n]{2}$ is irrational.

\begin{staffnotes}
A similar but somewhat more interesting generalization to suggest to
students who finish quickly is $\sqrt[n]{3}$ is irrational for $n>1$.
The proof is the same as above with ``2'' replaced by ''3,'' except
that now the needed claim is that if $a^{n}$ is divisible by 3, then
so is $a$, which requires appealing to prime factorization as in the
previous paragraph.  See the comments at the end of the solution.
\end{staffnotes}


\begin{proof}
The proof is by contradiction.

Assume to the contrary that $\sqrt[n]{2}$ is rational.  Under this
assumption, there exist integers $a$ and $b$ with $\sqrt[n]{2} = a/b$,
where $b$ is the smallest such positive integer denominator.  Now we
prove that $a$ and $b$ are both even, so that
         \[
        \frac{a/2}{b/2}
        \]
  is a fraction equal to $\sqrt[n]{2}$ with a smaller positive integer
  denominator, a contradiction.

\begin{align*}
\sqrt[n]{2} & = \frac{a}{b}\\
2           & = \frac{a^{n}}{b^{n}}\\
2b^{n}      & = a^{n}.
\end{align*}
The lefthand side of the last equation is even, so $a^{n}$ is even.  This
implies that $a$ is even as well (see below for justification).

In particular, $a = 2c$ for some integer $c$.  Thus,
\begin{align*}
2b^{n} & = (2c)^n = 2^{n}c^{n},\\
b^{n}  & = 2^{n-1}c^{n}.
\end{align*}
Since $n-1>0$, the righthand side of the last equation is an even number,
so $b^{n}$ is even.  But this implies that $b$ must be even as well,
contradicting the fact that $a/b$ is in lowest terms.
\end{proof}

Now we justify the claim that if $a^n$ is even, so is $a$.

There is a simple proof by contradiction: suppose to the contrary that $a$
is odd.  It's a familiar (and easily verified\footnote{Two odd integers
  can be written as $2x+1$ and $2y+1$ for some integers $x$ and $y$.  Then
  their product is also odd because it equals $2z +1$ where $z=
  2(2xy+x+y)+1$.}) fact that the product of two odd numbers is odd, from
which it follows that the product of \emph{any} finite number of odd
numbers is odd, so $a^n$ would also be odd, contradicting the fact that
$a^n$ is even.

More generally for \emph{any} integers $m,k >0$, if $m^k$ is divisible
by a prime number, $p$, then $m$ must be divisible by $p$.  This
follows from the unique factorization of an integer into primes (see
Section~\bref{fundamental_theorem_sec}): the primes in the
factorization of $m^k$ are precisely the primes in the factorization
of $m$ repeated $k$ times.  Using this fact, the proof above carries
over to prove a broader generalization:
\begin{theorem*}
For all positive integers $n$ and $k$,
\[
\sqrt[k]{n} \text{ must be either an integer or an irrational.}
\]
\end{theorem*}
For an even broader generalization, see
Problem~\bref{CP_roots_of_polynomials}.

\end{solution}

\end{problem}
%%%%%%%%%%%%%%%%%%%%%%%%%%%%%%%%%%%%%%%%%%%%%%%%%%%%%%%%%%%%%%%%%%%%%
% Problem ends here
%%%%%%%%%%%%%%%%%%%%%%%%%%%%%%%%%%%%%%%%%%%%%%%%%%%%%%%%%%%%%%%%%%%%%

\endinput
