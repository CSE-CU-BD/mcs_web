\documentclass[problem]{mcs}

\begin{pcomments}
  \pcomment{from: S09.cp1r}
  \pcomment{Remove inline proof in favor of link to notes version in
            preceding text.}
%  \pcomment{}
\end{pcomments}

\pkeywords{
  root_2
  irrational
  rational
  contradiction
}

%%%%%%%%%%%%%%%%%%%%%%%%%%%%%%%%%%%%%%%%%%%%%%%%%%%%%%%%%%%%%%%%%%%%%
% Problem starts here
%%%%%%%%%%%%%%%%%%%%%%%%%%%%%%%%%%%%%%%%%%%%%%%%%%%%%%%%%%%%%%%%%%%%%

\begin{problem}\label{generprob}
  Generalize the proof from lecture (reproduced below) that $\sqrt{2}$ is
  irrational, for example, how about $\sqrt[3]{2}$?  Remember that an
  irrational number is a number that cannot be expressed as a ratio of two
  integers.

\begin{quote}

\begin{theorem*}
$\sqrt{2}$ is an irrational number.
\end{theorem*}

\begin{proof}
  The proof is by contradiction: assume that $\sqrt{2}$ is rational, that
  is,
  \begin{equation}\label{2nd}
    \sqrt{2} = \frac{n}{d},
  \end{equation}
  where $n$ and $d$ are integers.  Now consider the smallest such positive
  integer denominator, $d$.  We will prove in a moment that the numerator,
  $n$, and the denominator, $d$, are both even.  This implies that
         \[
        \frac{n/2}{d/2}
        \]
  is a fraction equal to $\sqrt{2}$ with a smaller positive integer
  denominator, a contradiction.

\begin{quote}
\emph{Since the assumption that $\sqrt{2}$ is rational leads to this
  contradiction, the assumption must be false.  That is, $\sqrt{2}$ is
  indeed irrational.}  This italicized comment on the implication of the
contradiction normally goes without saying, but since this is the first
6.042 exercise about proof by contradiction, we've said it.
\end{quote}

  To prove that $n$ and $d$ have 2 as a common factor, we start by
  squaring both sides of~\eqref{2nd} and get $2 = n^2 / d^2$, so
\begin{equation}\label{2d2}
2 d^2 = n^2.
\end{equation}
So 2 is a factor of $n^2$, which is only possible if 2 is in fact a
factor of $n$.

This means that $n=2k$ for some integer, $k$, so
\begin{equation}\label{n24}
  n^2 = (2k)^2 = 4 k^2.
\end{equation}
Combining~\eqref{2d2} and~\eqref{n24} gives $2 d^2 = 4 k^2$, so
\begin{equation}\label{n22}
d^2 = 2k^2.
\end{equation}
So 2 is a factor of $d^2$, which again is only possible if 2 is in fact
also a factor of $d$, as claimed.
\end{proof}
\end{quote}

\begin{solution}
\begin{proof}
We prove that for any $n>1$, $\sqrt[n]{2}$ is irrational by
contradiction.

Assume that $\sqrt[n]{2}$ is rational.  Under this assumption, there exist
integers $a$ and $b$ with $\sqrt[n]{2} = a/b$, where $b$ is the smallest
such positive integer denominator.  Now we prove that $a$ and $b$ are both
even, so that
         \[
        \frac{a/2}{b/2}
        \]
  is a fraction equal to $\sqrt[n]{2}$ with a smaller positive integer
  denominator, a contradiction.

\begin{eqnarray*}
\sqrt[n]{2} & = & \frac{a}{b}\\
2 & = & \frac{a^{n}}{b^{n}}\\
2b^{n} & = & a^{n}.
\end{eqnarray*}
The lefthand side of the last equation is even, so $a^{n}$ is even.  This
implies that $a$ is even as well (see below for justification).

In particular, $a = 2c$ for some integer $c$.  Thus, 
\begin{eqnarray*}
2b^{n} & = & (2c)^n = 2^{n}c^{n},\\
b^{n} & = & 2^{n-1}c^{n}.
\end{eqnarray*}
Since $n-1>0$, the righthand side of the last equation is an even number,
so $b^{n}$ is even.  But this implies that $b$ must be even as well,
contradicting the fact that $a/b$ is in lowest terms.
\end{proof}

Now we justify the claim that if $a^n$ is even, so is $a$.

There is a simple proof by contradiction: suppose to the contrary that $a$
is odd.  It's a familiar (and easily verified\footnote{Two odd integers
  can be written as $2x+1$ and $2y+1$ for some integers $x$ and $y$.  Then
  their product is also odd because it equals $2z +1$ where $z=
  2(2xy+x+y)+1$.}) fact that the product of two odd numbers is odd, from
which it follows that the product of \emph{any} finite number of odd
numbers is odd, so $a^n$ would also be odd, contradicting the fact that
$a^n$ is even.

More generally for \emph{any} integers $m,k >0$, if $m^k$ is divisible
by a prime number, $p$, then $m$ must be divisible by $p$.  This follows
from the factorization of an integer into primes (which we'll discuss
further in a coming lecture): the primes in the factorization of $m^k$ are
precisely the primes in the factorization of $m$ repeated $k$ times, so if
there is a $p$ in the factorization of $m^k$ it must be one of $k$ copies
of a $p$ in the factorization of $m$.
\end{solution}
\end{problem}
%%%%%%%%%%%%%%%%%%%%%%%%%%%%%%%%%%%%%%%%%%%%%%%%%%%%%%%%%%%%%%%%%%%%%
% Problem ends here
%%%%%%%%%%%%%%%%%%%%%%%%%%%%%%%%%%%%%%%%%%%%%%%%%%%%%%%%%%%%%%%%%%%%%

\endinput
