\documentclass[problem]{mcs}

\begin{pcomments}
  \pcomment{CP_generalize_root_2_proof}
  \pcomment{from: S09.cp1r}
%  \pcomment{}
\end{pcomments}

\pkeywords{
  square_root
  irrational
  rational
  contradiction
  divisible
}

%%%%%%%%%%%%%%%%%%%%%%%%%%%%%%%%%%%%%%%%%%%%%%%%%%%%%%%%%%%%%%%%%%%%%
% Problem starts here
%%%%%%%%%%%%%%%%%%%%%%%%%%%%%%%%%%%%%%%%%%%%%%%%%%%%%%%%%%%%%%%%%%%%%

\begin{problem}\label{generprob}
  Generalize the proof of Theorem~\bref{thm:sqrt2irr_by_contra}
  \inhandout{repeated below} that $\sqrt{2}$ is irrational\inhandout{\footnote{Remember that an
    irrational number is a number that cannot be expressed as a ratio
    of two integers.}}.  For example, how about $\sqrt{3}$? 

\inhandout{
\begin{quote}

\begin{theorem*}
$\sqrt{2}$ is an irrational number.
\end{theorem*}

\begin{proof}
  The proof is by contradiction: assume that $\sqrt{2}$ is rational, that
  is,
  \begin{equation}\label{2nd}
    \sqrt{2} = \frac{n}{d},
  \end{equation}
  where $n$ and $d$ are integers.  Now consider the smallest such positive
  integer denominator, $d$.  We will prove in a moment that the numerator,
  $n$, and the denominator, $d$, are both even.  This implies that
         \[
        \frac{n/2}{d/2}
        \]
  is a fraction equal to $\sqrt{2}$ with a smaller positive integer
  denominator, a contradiction.

\begin{quote}
\emph{Since the assumption that $\sqrt{2}$ is rational leads to this
  contradiction, the assumption must be false.  That is, $\sqrt{2}$ is
  indeed irrational.}  This italicized comment on the implication of
the contradiction normally goes without saying, but since this is an
early example of proof by contradiction, we've said it.
\end{quote}

  To prove that $n$ and $d$ have 2 as a common factor, we start by
  squaring both sides of~\eqref{2nd} and get $2 = n^2 / d^2$, so
\begin{equation}\label{2d2}
2 d^2 = n^2.
\end{equation}
So 2 is a factor of $n^2$, which is only possible if 2 is in fact a
factor of $n$.

This means that $n=2k$ for some integer, $k$, so
\begin{equation}\label{n24}
  n^2 = (2k)^2 = 4 k^2.
\end{equation}
Combining~\eqref{2d2} and~\eqref{n24} gives $2 d^2 = 4 k^2$, so
\begin{equation}\label{n22}
d^2 = 2k^2.
\end{equation}
So 2 is a factor of $d^2$, which again is only possible if 2 is in fact
also a factor of $d$, as claimed.
\end{proof}
\end{quote}
}

\begin{solution}
The proof that $\sqrt{3}$ is irrational is the same as the $\sqrt{2}$
proof above with
\begin{itemize}
\item ``2'' replaced by ``3,'' 
\item ``is even'' replaced by ``is divisible by 3.''
\end{itemize}
Now the needed claim is that if $a^2$ is divisible by 3, then so is
$a$.
  
More generally for \emph{any} integers $m,k >0$, if $m^k$ is divisible
by a prime number, $p$, then $m$ must be divisible by $p$.  That's
because the prime factorization of $m^k$ consists of $k$ copies of the
prime factorization of $m$.  (This follows from the uniqueness of the
factorization of an integer into primes (see
Problem~\bref{TP_divides_n_square_then_n}).\footnote{The uniquenss of
  prime factorization is proved in
  Section~\bref{fundamental_theorem_sec}.}  Using this fact, the proof
above carries over to prove a broader generalization:
\begin{theorem*}
For all positive integers $n$ and $k$,
\[
\sqrt[k]{n} \text{ must be either an integer or an irrational.}
\]
\end{theorem*}
For an even broader generalization, see
Problem~\bref{CP_roots_of_polynomials}.

\end{solution}

\end{problem}
%%%%%%%%%%%%%%%%%%%%%%%%%%%%%%%%%%%%%%%%%%%%%%%%%%%%%%%%%%%%%%%%%%%%%
% Problem ends here
%%%%%%%%%%%%%%%%%%%%%%%%%%%%%%%%%%%%%%%%%%%%%%%%%%%%%%%%%%%%%%%%%%%%%

\endinput

