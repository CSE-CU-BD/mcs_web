\documentclass[problem]{mcs}

\begin{pcomments}
  \pcomment{PS_strict_partial_order_isomorphic_to_subset}
  \pcomment{from: S09.ps3, F13.ps7}
  \pcomment{revised by ARM 10/4/11}
  \pcomment{good exercise in reasoning from relation axioms and set
    properties, but very abstract with foundational motivation only.}
\end{pcomments}

\pkeywords{
  partial_orders
  isomorphism
  subset
  inverse
}

\newcommand{\srinv}[1]{\text{L}(#1)}

%%%%%%%%%%%%%%%%%%%%%%%%%%%%%%%%%%%%%%%%%%%%%%%%%%%%%%%%%%%%%%%%%%%%%
% Problem starts here
%%%%%%%%%%%%%%%%%%%%%%%%%%%%%%%%%%%%%%%%%%%%%%%%%%%%%%%%%%%%%%%%%%%%%

\begin{problem} Every \idx{partial order} is \idx{isomorphic} to a
  collection of sets under the subset%
\index{set!subset} relation (see
  Section~\bref{poset-as-sets_sec}).  In particular, if $R$ is a
  \emph{strict} partial order on a set $A$ and $a \in A$, define
\begin{equation}\label{srinvadef}
\srinv{a} \eqdef \set{a} \union \set{x \in A \suchthat x \mrel{R} a}.
\end{equation}
Then
\begin{equation}\label{ps3_rgb}
a \mrel{R} b  \qiff  \srinv{a} \subset \srinv{b}
\end{equation}
holds for all $a,b \in A$.

\bparts
\ppart
Carefully prove statement~\eqref{ps3_rgb},
starting from the definitions of strict partial order and the
strict subset relation $\subset$.

\begin{solution}
\iffalse
  \begin{theorem*}
    $a = b \qiff R\set{a} = R\set{b}$.
  \end{theorem*}
  \begin{proof}
    ($\Rightarrow$) Suppose $a = b$.  Then $\set{a} = \set{b}$, and
    hence $R\set{a} = R\set{b}$. 

    ($\Leftarrow$) Suppose $R\set{a} = R\set{b}$, and let $X \eqdef
    R\set{a} = R\set{b}$.  Since $R$ is a weak partial order, it is
    reflexive (i.e., $a \mrel{R} a$ and $b \mrel{R} b$), and thus $a,b \in X$.
    Since $X = \set{ x \suchthat x \mrel{R} a}$, and $b \in X$, it follows that $b \mrel{R}
    a$.  Similarly, $a \mrel{R} b$ by swapping $a$ and $b$ in the previous
    sentence.  Since $R$ is a weak partial order, it must be
    antisymmetric (i.e., $a \mrel{R} b \QIMPLIES (\QNOT(b \mrel{R} a) \QOR (a = b))$).
    Since $a \mrel{R} b$ and $b \mrel{R} a$, we conclude that $a = b$.
  \end{proof}

  \begin{theorem*}
    $a \mrel{R} b \qiff \srinv{a} \subset \srinv{b}$.
  \end{theorem*}
\fi

  \begin{proof}
    ($\Rightarrow$) Suppose $a \mrel{R} b$.  Then
    \begin{equation}\label{ainsrb}
      a \in \srinv{b}
    \end{equation}
    by definition of $\srinv{b}$.  Also, by transitivity of
    partial order, we have
    \begin{equation}\label{xraimparb}
      x \mrel{R} a \QIMPLIES\ x \mrel{R} b,
    \end{equation}
    for all $x \in A$.  According to the definition of $\srinv{a}$,
    equations~\eqref{ainsrb} and~\eqref{xraimparb} together are
    equivalent to the assertion that
    \[
    \srinv{a} \subseteq \srinv{b}.
    \]
    Moreover, $b \notin \srinv{a}$ since $R$ is asymmetric, and $b \in
    \srinv{b}$ by definition, so in fact
    \[
    \srinv{a} \subset \srinv{b}.
    \]

    ($\Leftarrow$) Suppose $\srinv{a} \subset \srinv{b}$.  Since $a
    \in \srinv{a}$ by definition, we have $a \in \srinv{b}$.  Since
    the inclusion is strict, we know $a \neq b$.  So $a \mrel{R} b$ by
    definition of $\srinv{b}$.
  \end{proof}

\end{solution}

\ppart\label{srva=srvb} Prove that if $\srinv{a} = \srinv{b}$ then $a = b$.

\begin{solution}
Suppose $\srinv{a} = \srinv{b}$.  Since $a \in \srinv{a}$, we have $a
\in \srinv{b}$ so
\begin{equation}\label{a=bqorarb}
a=b \QOR\ (a \mrel{R} b)
\end{equation}
by definition of $\srinv{b}$.  Switching $a$ and $b$ in this argument
implies that
\begin{equation}\label{b=aqorarb}
b=a \QOR\ (b \mrel{R} a).
\end{equation}
But $ a \mrel{R} b$ and $b \mrel{R} a$ can't both be true since $R$ is
asymmetric, so~\eqref{a=bqorarb} and~\eqref{b=aqorarb} together imply
$a=b$.
\end{solution}

\ppart Give an example showing that the conclusion of
part~\eqref{srva=srvb} would not hold if the definition of $\srinv{a}$ in
equation~\eqref{srinvadef} had omitted the expression ``$\set{a}
\union$.''

\begin{solution}
The simplest counterexample would be the empty partial order on two
elements $a\neq b$.  Then without the ``$\set{a} \union$'' expression, we
would have $\srinv{a} = \emptyset = \srinv{b}$.
\end{solution}

\eparts

\end{problem}

%%%%%%%%%%%%%%%%%%%%%%%%%%%%%%%%%%%%%%%%%%%%%%%%%%%%%%%%%%%%%%%%%%%%%
% Problem ends here
%%%%%%%%%%%%%%%%%%%%%%%%%%%%%%%%%%%%%%%%%%%%%%%%%%%%%%%%%%%%%%%%%%%%%
 \endinput
