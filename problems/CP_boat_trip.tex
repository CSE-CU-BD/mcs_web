\documentclass[problem]{mcs}

\begin{pcomments}
  \pcomment{CP_boat_trip}
  \pcomment{same as FP_boat_trip}
  \pcomment{from: F11.final, Rajeev F09}
  \pcomment{based on CP_bag_of_donuts}
  \pcomment{revised ARM 12/23/11}
\end{pcomments}

\pkeywords{
  generating_function
  convolution
  partial_fraction
}

%%%%%%%%%%%%%%%%%%%%%%%%%%%%%%%%%%%%%%%%%%%%%%%%%%%%%%%%%%%%%%%%%%%%%
% Problem starts here
%%%%%%%%%%%%%%%%%%%%%%%%%%%%%%%%%%%%%%%%%%%%%%%%%%%%%%%%%%%%%%%%%%%%%

\begin{problem}
T-Pain is planning an epic boat trip and he needs to decide what to
bring with him.

\begin{itemize}

\item He must bring some burgers, but they only come in packs of 6.

\item He and his two friends can't decide whether they want to dress formally or
casually.  He'll either bring 0 pairs of flip flops or 3 pairs.

\item He doesn't have very much room in his suitcase for towels, so he can
  bring at most 2.

\item In order for the boat trip to be truly epic, he has to bring at least 1
nautical-themed pashmina afghan.

\end{itemize}

\bparts

\ppart Let $B(x)$ be the generating function for the number of ways to
bring $n$ burgers, $F(x)$ for the number of ways to bring $n$ pairs of
flip flops, $T(x)$ for towels, and $A(x)$ for Afghans.  Write simple
formulas for each of these.

\iffalse
\begin{center}
$B(x):$\exambox{1.5in}{0.6in}{-0.5in} \qquad\qquad$F(x):$\exambox{1.5in}{0.6in}{-0.5in}
\end{center}
\begin{center}
$T(x):$\exambox{1.5in}{0.6in}{-0.5in} \qquad\qquad$A(x):$\exambox{1.5in}{0.6in}{-0.5in}
\end{center}
\fi
\examspace{1.0in}

\begin{solution}
\begin{align*}
B(x) & = \frac{x^6}{1-x^6},\\
F(x) & = (1+x^3),\\
T(x) & = 1+x+x^2 = \frac{1-x^3}{1-x}\\
A(x) & = \frac{x}{1-x}.
\end{align*}
\end{solution}

\ppart Let $g_n$ be the the number of different ways for T-Pain to
bring $n$ items (burgers, pairs of flip flops, towels, and/or afghans)
on his boat trip.  Let $G(x)$ be the generating function
$\sum_{n=0}^{\infty} g_nx^n$.  Verify that
\[
G(x) = \frac{x^7}{\paren{1-x}^2}.
\]

\examspace[2in]

\begin{solution}
By the \idx{Convolution Rule}~\bref{convolution_rule},
\begin{align*}
G(x) & = B(x)F(x)T(x)A(x)\\
     & = \frac{x^6}{1-x^6}(1+x^3)\frac{1-x^3}{1-x}\frac{x}{1-x}\\
     & = \frac{x^6(1+x^3)(1-x^3)x}{(1-x^6)(1-x)^2}\\
     & = \frac{x^7}{(1-x)^2}
\end{align*}
\end{solution}

\ppart Find a simple formula for $g_n$.
\iffalse

\begin{center}
\exambox{0.75in}{0.6in}{-0.5in}
\end{center}
\fi

\examspace[2in]

\begin{solution}
\begin{equation}\label{gnn-6}
g_n  = \begin{cases} 0 & \text{for $n < 7$}\\
                     n-6 & \text{for $n \geq 7$}.
\end{cases}
\end{equation}

Let
\[
H(x) \eqdef \frac{1}{(1-x)^2},
\]
so $G(x) = x^7H(x)$.  We know that the coefficient, $h_n$, of $x^n$ in
the series for $H(x)$ is, by the Convolution Rule, the number of ways
to select $n$ items of two different kinds, namely, $h_n =
\binom{n+1}{1}=n+1$.  So we conclude that for $n \geq 7$, the $n$th
coefficient in the series for $G(x)$ is $h_{n-7}$
namely~\eqref{gnn-6}.
\end{solution}

\eparts

\end{problem}

%%%%%%%%%%%%%%%%%%%%%%%%%%%%%%%%%%%%%%%%%%%%%%%%%%%%%%%%%%%%%%%%%%%%%
% Problem ends here
%%%%%%%%%%%%%%%%%%%%%%%%%%%%%%%%%%%%%%%%%%%%%%%%%%%%%%%%%%%%%%%%%%%%%

\endinput
