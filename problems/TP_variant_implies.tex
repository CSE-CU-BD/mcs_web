\documentclass[problem]{mcs}

\begin{pcomments}
    \pcomment{TP_variant_implies}
    \pcomment{ARM 2/10/12}
\end{pcomments}

\begin{problem}
Some people are uncomfortable with the idea that from a false
hypothesis you can prove everything, and instead of having $P
\QIMPLIES Q$ be true when $P$ is false, they want $P \QIMPLIES Q$ to
be false when $P$ is false.  This would lead to $\QIMPLIES$ having the
same truth table as what propositional connective?  \inhandout{Does
  this seem reasonable?}
\begin{solution}
This would make $P \QIMPLIES Q$ equivalent to $P \QAND Q$.  But is
$\QIMPLIES$ is not treated the same as $\QAND$ in common or
mathematical usage.
\end{solution}
\end{problem}

\endinput
