\documentclass[problem]{mcs}

\begin{pcomments}
  \pcomment{CP_13th_roots}
  \pcomment{from: S04.ps7, S09.cp8t}
  \pcomment{Commented out in S09 - review before using.}
\end{pcomments}

\pkeywords{
  number_theory
  Eulers_theorem
  modular_arithmetic
}

%%%%%%%%%%%%%%%%%%%%%%%%%%%%%%%%%%%%%%%%%%%%%%%%%%%%%%%%%%%%%%%%%%%%%
% Problem starts here
%%%%%%%%%%%%%%%%%%%%%%%%%%%%%%%%%%%%%%%%%%%%%%%%%%%%%%%%%%%%%%%%%%%%%

\begin{problem}

At one time, the Guinness Book of World Records reported that the
``greatest human calculator'' was a guy who could compute 13th roots of
100-digit numbers that were powers of 13.  What a curious choice of
tasks \dots.

\bparts

\ppart Prove that
\begin{equation}\label{d13}
d^{13} \equiv d \pmod {10}
\end{equation}
for $0 \leq d < 10$.

\begin{solution}
Since $n^{13} = ((n^2)^2)^3 n$, we can verify~\eqref{d13}
exhaustively without too much effort.  It obviously hold for $d=0,1,5$, and

\begin{align*}
2^{13} & =  ((2^2)^2)^3 \cdot 2 \equiv 16^3 \cdot 2 \equiv 6^3
 \cdot 2 \equiv 6 \cdot 2 \equiv 2 \pmod {10}\\
3^{13} & =  ((3^2)^2)^3 \cdot 3 \equiv 81^3 \cdot 3 \equiv 1^3 \cdot 3
           = 3 \pmod {10}\\
4^{13} & =  (2^2)^{13} \equiv (2^{13})^2 \equiv 2^2 = 4 \pmod {10}\\
6^{13} & = 2^{13} 3^{13} \equiv 2 \cdot 3 = 6 \pmod {10}\\
7^{13} & =  ((7^2)^2)^3 \cdot 7 \equiv (9^2)^3 \cdot \equiv 1^3 \cdot 7 =
 7 \pmod {10}\\
8^{13} & =  2^{13} 4^{13} \equiv 2 \cdot 4 = 8 \pmod {10}\\
9^{13} & =  3^{13} 3^{13} \equiv 3 \cdot 3 = 9 \pmod {10}
\end{align*}
\end{solution}

\ppart Now  prove that
\begin{equation}\label{n13}
n^{13} \equiv n \pmod {10}
\end{equation}
for all $n$.

\begin{solution}
We have
\begin{align*}
n^{13}
    & \equiv  (\rem{n}{10})^{13} \pmod{10} & \text{(by Lemma~\bref{lem:conrem})}\\
    & \equiv  \rem{n}{10} \pmod{10} & \text{(by~\eqref{d13} since $0 \leq \rem{n}{10} < 10$)}\\
    & \equiv  n \pmod{10} & \text{(by Lemma~\bref{lem:conrem})}
\end{align*}

\iffalse
An alternative approach uses Euler's Theorem.  Say $n = 2^a 5^b c$, where
$c$ is not divisible by 2 or 5.  Then $c$ is relatively prime to 10.
Since $\phi(10) = 4$, we have $c^{4} \equiv 1 \pmod{10}$ by Euler's
Theorem.  Therefore, $c^{13} \equiv (c^{4})^3 \cdot c \equiv c \pmod{10}$.
Now we can reason as follows:
\begin{align*}
n^{13}
    & = (2^{13})^a (5^{13})^b c^{13}\\
    & \equiv 2^a 5^b c \pmod{10} & \text{(using~\eqref{d13} for $d=2,5$)}\\
    & = n
\end{align*}
\fi
\end{solution}

\iffalse

\ppart Find an integer $c > 1$ such that $n$ and $n^c$ agree in the
last \emph{two digits} whenever $n$ is a positive number relatively
prime to 100.

\begin{solution}
  Finding a $c$ such that for all $k$ relatively prime to 100, $k^c
  \equiv k \pmod{100}$ would satisfy the requirement.  From Euler's
  theorem, we know $c = \Phi(100)+1$ satisfies this equation. Thus $c =
  \Phi(100)+1 = 41$.
\end{solution}
\fi

\eparts

\end{problem}

%%%%%%%%%%%%%%%%%%%%%%%%%%%%%%%%%%%%%%%%%%%%%%%%%%%%%%%%%%%%%%%%%%%%%
% Problem ends here
%%%%%%%%%%%%%%%%%%%%%%%%%%%%%%%%%%%%%%%%%%%%%%%%%%%%%%%%%%%%%%%%%%%%%

\endinput
