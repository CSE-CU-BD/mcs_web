\documentclass[problem]{mcs}

\begin{pcomments}
  \pcomment{PS_super_symmetric_strings}
  \pcomment{S01-ps3-4}
\end{pcomments}

\pkeywords{
  structural_induction
  recursive_data
}

\def\supersym{\mathrm{SuperSymm}}

%%%%%%%%%%%%%%%%%%%%%%%%%%%%%%%%%%%%%%%%%%%%%%%%%%%%%%%%%%%%%%%%%%%%%
% Problem starts here
%%%%%%%%%%%%%%%%%%%%%%%%%%%%%%%%%%%%%%%%%%%%%%%%%%%%%%%%%%%%%%%%%%%%%

\begin{problem}

The set $\supersym$ of ``super-symmetric strings'' is defined recursively as follows:

\textbf{Base Case}: The 26 lower case letters of the Roman alphabet,
\STR{a}, \STR{b},\dots, \STR{z}, are in $\supersym$.

\textbf{Constructor Case}: If $\alpha$ and $\beta$ are strings in
$\supersym$, then the string $\alpha \beta \alpha$ is in $\supersym$.

\bparts
\ppart
Which of the following are super-symmetric strings?  Briefly explain your answers.

\renewcommand{\theenumi}{\roman{enumi}}
\renewcommand{\labelenumi}{(\theenumi)}

\begin{enumerate}

\item \STR{a}
\begin{solution}
Yes, by the Base Case.
\end{solution}

\item  \STR{aaaba}
\begin{solution}
No.  This string is not of the form $\alpha \beta \alpha$.\\
\end{solution}

\item  \STR{abcbacabcba} \label{iii.abc}
\begin{solution}
Yes.  Let $\beta = \mathtt{aca}$, $\alpha=\mathtt{bcb}$.  Then we have a string of the
form $\mathtt{a} \alpha \beta \alpha \mathtt{a}$.
\end{solution}

\item $\emptystring$, the empty string
\begin{solution}
No.  A trivial structural induction implies that all super-symmetric
strings have positive length.
\end{solution}

\item \STR{abaabcbaaba}
\begin{solution}
Yes.  Similar reasoning to case~\eqref{iii.abc} shows that the string
$bcb$ is in the middle of the super-symmetric string, with the string
$a$ wrapped around it, and with the string $aba$ wrapped around that.
\end{solution}

\end{enumerate}

\ppart
Prove by structural induction that in any super-symmetric string,
exactly one letter appears an odd number of times.

\begin{solution}

%Use \alpha, \beta, ... for string vars, not ``e''

\textbf{Proof:} Define $P(e)$: String $e$ has exactly one letter which appears an odd number of times.
Prove: $\forall e \in SSS \ P(e)$.\\
 (Base) $P(x)$: The string $x$, where $x$ is a single letter of the alphabet, has exactly one
  letter which appears an odd number of times - namely $x$.\\ \\
 (Inductive Step) $\forall e, e' \in SSS \ P(e)\wedge P(e') \longrightarrow P(ee'e)$ \\
  Fix $e,e' \in SSS$. Assume $P(e) \wedge P(e')$.  That is, each of $e$ and $e'$ has exactly
  one letter which appears and odd number of times. By the inductive hypothesis, both $e$
  and $e'$ have exactly one letter which appears an odd number of times.  When we form the
  string $ee'e$, the letters in $e$ all get repeated, and thus they all appear an even number
  of times.  By the IH, there is one letter in $e'$ which appears an odd number of times in $e'$.
  Now, even if this letter is also present in $e$, it will still appear an odd number of times in
  $ee'e$ (since the sum of an odd number and an even number is odd).  Therefore, we have shown
  $P(ee'e)$. QED.\\

\end{solution}

\eparts
\end{problem}


%%%%%%%%%%%%%%%%%%%%%%%%%%%%%%%%%%%%%%%%%%%%%%%%%%%%%%%%%%%%%%%%%%%%%
% Problem ends here
%%%%%%%%%%%%%%%%%%%%%%%%%%%%%%%%%%%%%%%%%%%%%%%%%%%%%%%%%%%%%%%%%%%%%

\endinput
