\documentclass[problem]{mcs}

\begin{pcomments}
  \pcomment{PS_gcd_three_integers_hint}
  \pcomment{S14.PS4.2, F04.ps3.03}
\end{pcomments}

\pkeywords{
  gcd
  linear_combinations
  divides
  number_theory
  common_divisor
}

%%%%%%%%%%%%%%%%%%%%%%%%%%%%%%%%%%%%%%%%%%%%%%%%%%%%%%%%%%%%%%%%%%%%%
% Problem starts here
%%%%%%%%%%%%%%%%%%%%%%%%%%%%%%%%%%%%%%%%%%%%%%%%%%%%%%%%%%%%%%%%%%%%%

\begin{problem}
  
  Prove that the greatest common divisor of three integers $a$, $b$, and
  $c$ is equal to their smallest positive linear combination; that is,
  the smallest positive value of $sa + tb + uc$, where $s$, $t$, and
  $u$ are integers.

\hint  Let $m$ be the smallest positive linear combination of $a$, $b$, and
    $c$.  Prove that $\gcd(a, b, c) \leq m$ and $m \leq \gcd(a, b, c)$.

  \begin{solution}

    Let $m$ be the smallest positive linear combination of $a$, $b$, and
    $c$.  We'll prove that $m = \gcd(a, b, c)$ by showing both $\gcd(a, b,
    c) \leq m$ and $m \leq \gcd(a, b, c)$.

    First, we show that $\gcd(a, b, c) \leq m$.  By the definition of
    common divisor, $\gcd(a, b, c)$ divides $a$, $b$, and $c$.  Therefore,
    for every triple of integers $s$, $t$, and $u$:
    %
    \[
    \gcd(a, b, c) \mid s a + t b + u c
    \]
    %
    Thus, in particular, $\gcd(a, b, c)$ divides $m$, and so $\gcd(a, b,
    c) \leq m$.

    Now we show that $m \leq \gcd(a, b, c)$.  We do this by showing
    that $m \divides a$.  Symmetric arguments show that $m \divides b$
    and $m \divides c$, which means that $m$ is a common divisor of
    $a$, $b$, and $c$.  Thus, $m$ must be less than or equal to the
    \emph{greatest} common divisor of $a$, $b$, and $c$.

    All that remains is to show that $m \divides a$.  By the division
    algorithm, there exists a quotient $q$ and remainder $r$ such 
\begin{align*}
    a & = q \cdot m + r & \text{(where $0 \leq r < m$)}
\end{align*}
    Now $m = s a + t b + u c$ for some integers $s$ and $t$.  Subtituting
    in for $m$ and rearranging terms gives:
    \begin{align*}
    a & = q \cdot (s a + t b + u c) + r \\
    r & = (1 - qs) a + (-qt) b + (-qu) c
    \end{align*}
    %
    We've just expressed $r$ as a linear combination of $a$, $b$, and $c$.
    However, $m$ is the \emph{smallest positive} linear combination and $0
    \leq r < m$.  The only possibility is that the remainder $r$ is not
    positive; that is, $r = 0$.  This implies $m \divides a$.
  \end{solution}
  
\end{problem}

%%%%%%%%%%%%%%%%%%%%%%%%%%%%%%%%%%%%%%%%%%%%%%%%%%%%%%%%%%%%%%%%%%%%%
% Problem ends here
%%%%%%%%%%%%%%%%%%%%%%%%%%%%%%%%%%%%%%%%%%%%%%%%%%%%%%%%%%%%%%%%%%%%%

\endinput
