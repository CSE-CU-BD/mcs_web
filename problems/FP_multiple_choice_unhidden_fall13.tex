\documentclass[problem]{mcs}

\begin{pcomments}
  \pcomment{FP_multiple_choice_unhidden_fall13}
  \pcomment{revised from FP_multiple_choice_unhidden by ARM 12/13/13}
  \pcomment{overlaps FP_graphs_short_answer, FP_random_walk_multiple,
    FP_simple_graphs_trees_short_answer_S17}
\end{pcomments}

\pkeywords{
  isomorphism
  partial_order  
  total_order
  linear
  asymptotic
  vertices
  gcd
  modular
  mod_n
  stable_distribution
  countable
}

%%%%%%%%%%%%%%%%%%%%%%%%%%%%%%%%%%%%%%%%%%%%%%%%%%%%%%%%%%%%%%%%%%%%%
% Problem starts here
%%%%%%%%%%%%%%%%%%%%%%%%%%%%%%%%%%%%%%%%%%%%%%%%%%%%%%%%%%%%%%%%%%%%%

\begin{problem} \mbox{}

\textbf{\large Circle \textbf{true} or \textbf{false} for the statements below,
and \emph{provide counterexamples} for those that are \textbf{false}.}

\bparts

\ppart The following statements about the \textbf{greatest common divisor}:

\begin{itemize}

\item If $\gcd(a, b) \neq 1$ and $\gcd(b, c) \neq 1$, then $\gcd(a, c) \neq 1$. \hfill 
\textbf{true} \qquad  \textbf{false} \examspace[0.4in]

\begin{solution}
\textbf{false} $a=2\cdot 3, b=3\cdot 5, c=5\cdot 7$
\end{solution}

\item If $a \divides b c$ and $\gcd(a, b) = 1$, then $a \divides c$.  \hfill 
\textbf{true} \qquad \textbf{false} \examspace[0.4in]

\begin{solution}
\textbf{true}
\end{solution}

\item $\gcd(a^n,b^n) =  (\gcd(a,b))^n$  \hfill 
\textbf{true} \qquad \textbf{false} \examspace[0.4in]

\begin{solution}
\textbf{true}.
\end{solution}

\item $\gcd(ab, ac) = a \gcd(b, c)$.  \hfill 
\textbf{true} \qquad  \textbf{false} \examspace[0.4in]

\begin{solution}
\textbf{true}
\end{solution}

\item $\gcd(1 + a, 1 + b) = 1 + \gcd(a, b)$.  \hfill 
\textbf{true} \qquad  \textbf{false} \examspace[0.4in]

\begin{solution}
false $a=1,b=2$
\end{solution}

\item If an integer linear combination of $a$ and $b$ equals 1, then
  so does some integer linear combination of $a$ and $b^2$. \hfill
  \textbf{true} \qquad \textbf{false} \examspace[0.4in]

\item If no integer linear combination of $a$ and $b$ equals 2, then
  neither does any integer linear combination of $a^2$ and
  $b^2$. \hfill \textbf{true} \qquad \textbf{false} \examspace[0.4in]

\begin{solution}
\textbf{true} No linear combination of $a,b$ is 2 iff $\gcd(a,b) >2$
iff $\gcd(a^2,b^2)>4$.
\end{solution}

\end{itemize}

\ppart The following statements about \textbf{congruence modulo} $n$, where $n > 1$.

\begin{itemize}

\item If $a c \equiv b c \pmod{n}$ and $n$ does not divide $c$, then $a \equiv b \pmod{n}$.
\hfill \textbf{true} \qquad \textbf{false} \examspace[0.4in]

\begin{solution}
\textbf{false}.  Need $c$ relatively prime to $n$.  Counterexample:
$n=2 \cdot 3, a=0, b=2, c=3$
\end{solution}

\item If $a \equiv b \pmod{\phi(n)}$ for $a, b > 0$, then $c^a \equiv c^b
\pmod{n}$.  \hfill \textbf{true} \qquad \textbf{false} \examspace[0.4in]

\begin{solution}
  \textbf{false}.  Need $c$ relatively prime to $n$.  Counterexample:
  $n=4$, so $\phi(n) = 2$; $a=1, b=3$, so $a \equiv b
  \pmod \phi(n)$, $c = 2$, so $c^a = 2 \not\equiv 0 = c^b \pmod 4$.
\end{solution}

\iffalse
\item If $a \equiv b \pmod{n}$, then $P(a) \equiv P(b) \pmod{n}$ for any
polynomial $P(x)$ with integer coefficients.
 \hfill \textbf{true} \qquad \textbf{false} \examspace[0.4in]
\begin{solution}
true
\end{solution}
\fi

\item If $a \equiv b \pmod{nm}$, then $a \equiv b \pmod{n}$, for $m,n > 1$.
 \hfill \textbf{true} \qquad \textbf{false} \examspace[0.4in]

\begin{solution}
\textbf{true}
\end{solution}

\item For relatively prime $m,n >1$,\\
$[a \equiv b \pmod{m} \QAND a \equiv b \pmod{n}] \QIFF [a \equiv b
  \pmod{mn}]$
\hfill \textbf{true} \qquad \textbf{false} \examspace[0.4in]

\begin{solution}
\textbf{true}
\end{solution}

\item Assuming $a,b$ have inverses modulo $n$, if $a^{-1} \equiv
  b^{-1} \pmod{n}$, then $a \equiv b \pmod{n}$. \hfill \textbf{true}
  \qquad \textbf{false} \examspace[0.4in]

\begin{solution}
\textbf{true}
\end{solution}

\item If $a,b >1$, then [$a$ has a multiplicative inverse mod $b$ iff
  $b$ has a multiplicative inverse mod $a$].  \hfill \textbf{true}
\qquad \textbf{false} \examspace[0.4in]

\begin{solution}
\textbf{true}.  $a$ has a multiplicative inverse mod $b$ iff
$a,b$ relatively prime iff $b$ has a multiplicative inverse mod $a$.
\end{solution}

\item If $\gcd(a,n)=1$, then $a^{n-1} \equiv 1 \pmod{n}$. \hfill
  \textbf{true} \qquad \textbf{false} \examspace[0.4in]

\begin{solution}
\textbf{false}  Let $a=5$, $n =6$.
\end{solution}


\end{itemize}

\ppart The following statements about \textbf{trees}:

\begin{itemize}

\item Any connected subgraph is a tree.  \hfill \textbf{true} \qquad
  \textbf{false} \examspace[0.4in]

\begin{solution}
\textbf{true}
\end{solution}

\item Adding an edge between two nonadjacent vertices creates a
  cycle. \hfill \textbf{true} \qquad \textbf{false} \examspace[0.4in]

\begin{solution}
\textbf{true}
\end{solution}

\item The number of vertices is one less than twice the number of
  leaves.  \hfill \textbf{true} \qquad
  \textbf{false} \examspace[0.4in]

\begin{solution}
  \textbf{false}.  This property holds for full binary trees, but not
  in general.  A tree with two vertices is a counterexample.
\end{solution}

\item The number of leaves in a tree is not equal to the number of
  non-leaf vertices.  \hfill \textbf{true} \qquad
  \textbf{false} \examspace[0.4in]

\begin{solution}
  \textbf{false}.  A line graph with 4 vertices has 2 non-leaf
  vertices and 2 leaves.
\end{solution}

\item Any subgraph of a tree is a tree.  \hfill \textbf{true} \qquad
  \textbf{false} \examspace[0.4in]

\begin{solution}
\textbf{false.  Choose a subgraph with 2 vertices and no edges.}
\end{solution}

%S11 %cut from Monday version
\item The number of vertices in a tree is one less than the number of
  edges.  \hfill \textbf{true} \qquad \textbf{false} \examspace[0.4in]

\begin{solution}
  \textbf{false}.  This got ``edges'' and ``vertices'' reversed.
Every tree is a counterexample.
\end{solution}

\end{itemize}

\ppart the following statements about finite \textbf{simple graphs}:

\begin{itemize}
\item  It has a spanning tree.  \hfill \textbf{true} \qquad
  \textbf{false} \examspace[0.4in]

\begin{solution}
\textbf{false.  Any disconnected graph is a counterexample.}
\end{solution}

\item $\card{\vertices{G}} = O(\card{\edges{G}})$ for connected $G$.  \hfill \textbf{true} \qquad
  \textbf{false} \examspace[0.4in]

\begin{solution}
\textbf{true.}
\end{solution}

\item $\card{\edges{G}} = O(\card{\vertices{G}})$.  \hfill \textbf{true} \qquad
  \textbf{false} \examspace[0.4in]

\begin{solution}
\textbf{false.}  $\card{\vertices{K_n}} = n = o(n^2)$, but
  $\card{\edges{K_n}} = \Theta(n^2)$.

\end{solution}

\item The chromatic number $\chi(G) \leq \max\set{\degr{v} \suchthat
  v \in \vertices{G}}$.  \hfill \textbf{true} \qquad
  \textbf{false} \examspace[0.4in]

\begin{solution}
\textbf{false.}  $\chi(K_n) = n > n-1 = \text{ max degree of } v \in \vertices{K_n}$
\end{solution}

\item The chromatic number $\card{\edges{G}} = O(\chi(G))$.  \hfill \textbf{true} \qquad
  \textbf{false} \examspace[0.4in]

\begin{solution}
\textbf{false.}   $\chi(K_n) = n$, but $\card{\edges{K_n}} = \Theta(n^2)$.
\end{solution}

\end{itemize}
\eparts

\examspace[0.2in]
Now answer the following:

\bparts

\ppart List the numbers of the properties below that are preserved under \textbf{graph
isomorphism}. \hfill\examrule[1in]

\begin{enumerate}

\item There is a cycle that includes all the vertices.

\item The vertices can be numbered 1 through 7.

\item Two edges are of equal length.

\item The graph remains connected if any two edges are removed.

\item There exists an edge that is an edge of every spanning tree.

\item The negation of a property that is preserved under isomorphism.

\item There is a cycle that includes all the vertices.

\item There are exacty two spanning trees.

\item The $\QOR$ of two properties that are preserved under isomorphism.

\item The graph remains connected if a vertex is removed.

\end{enumerate}

\begin{solution}
All but the second one are preserved.
\end{solution}

\ppart A \term{sink} in a random walk digraph is a vertex with no
edges leaving it.  A certain random walk graph has exactly two sinks
and they do not have any vertex in common.

Circle whichever of the following assertions are true of \idx{stable
  distributions} of the graph.

\begin{itemize}

\inbook{\item There may not be any.}

\item There is a unique one.

\inbook{\item There are exactly two.}

%\item There are at most a finite number.

\item There are a countable number.

\item There may not be a countable number.

\item There never is a countable number.

\begin{solution}
Only the last is true.  That's because a stable distribution for the
whole graph always consists of a stable distribution of one trap with
total probability $r$ together with a stable distribution of the other
trap with total probability $1-r$, for any $r \in [0,1] \subseteq
\reals$, and there are an uncountable number of real numbers in $r \in
[0,1]$.
\end{solution}


\end{itemize}

\begin{solution}
The first three choices are false, and the last three are true.
That's because a distribution in which one sink has probability $r \in
[0,1] \subseteq \reals$ and the other sink has probability $1-r$ is
stable, and there are an uncountable number of real numbers in $[0,1]$.
\end{solution}

\eparts

\end{problem}

\endinput
