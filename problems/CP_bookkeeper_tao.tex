\documentclass[problem]{mcs}

\begin{pcomments}
  \pcomment{CP_bookkeeper_tao}
  \pcomment{from: S09.cp10r}
\end{pcomments}

\pkeywords{
  counting
  counting_rules
  bookkeeper
  permutations
  combinations
}

%%%%%%%%%%%%%%%%%%%%%%%%%%%%%%%%%%%%%%%%%%%%%%%%%%%%%%%%%%%%%%%%%%%%%
% Problem starts here
%%%%%%%%%%%%%%%%%%%%%%%%%%%%%%%%%%%%%%%%%%%%%%%%%%%%%%%%%%%%%%%%%%%%%


\begin{problem} The Tao of BOOKKEEPER: we seek enlightenment through
contemplation of the word $BOOKKEEPER$.

\bparts

\ppart In how many ways can you arrange the letters in the word
$POKE$?

\solution{There are $4!$ arrangements corresponding to
the $4!$ permutations of the set $\set{P, O, K, E}$.}

\ppart In how many ways can you arrange the letters in the word
$BO_1O_2K$?  Observe that we have subscripted the O's to make them
distinct symbols.

\solution{There are $4!$ arrangements corresponding to
the $4!$ permutations of the set $\set{B, O_1, O_2, K}$.}

\ppart Suppose we map arrangements of the letters in $BO_1O_2K$ to
arrangements of the letters in $BOOK$ by erasing the subscripts.
Indicate with arrows how the arrangements on the left are mapped to
the arrangements on the right.

\[
\begin{array}{lcl}
\begin{array}{l}
O_2BO_1K \\
KO_2BO_1 \\
O_1BO_2K \\
KO_1BO_2 \\
BO_1O_2K \\
BO_2O_1K \\
\ldots
\end{array}
& \hspace{2in} &
\begin{array}{l}
BOOK \\
OBOK \\
KOBO \\
\ldots
\end{array}
\end{array}
\]

\ppart What kind of mapping is this, young grasshopper?

\solution{2-to-1}

\ppart In light of the Division Rule, how many arrangements are there
of $BOOK$?

\solution{$4! / 2$}

\ppart Very good, young master!  How many arrangements are there of
the letters in $KE_1E_2PE_3R$?

\solution{$6!$}

\ppart Suppose we map each arrangement of $KE_1E_2PE_3R$ to an
arrangement of $KEEPER$ by erasing subscripts.  List all the different
arrangements of $KE_1E_2PE_3R$ that are mapped to $REPEEK$ in this way.

\solution{$RE_1PE_2E_3K$, $RE_1PE_3E_2K$,
$RE_2PE_1E_3K$, $RE_2PE_3E_1K$, $RE_3PE_1E_2K$, $RE_3PE_2E_1K$}

\ppart What kind of mapping is this?

\solution{3!-to-1}

\ppart So how many arrangements are there of the letters in $KEEPER$?

\solution{$6! / 3!$}

\ppart {\em Now you are ready to face the BOOKKEEPER!}

How many arrangements of $BO_1O_2K_1K_2E_1E_2PE_3R$ are there?

\solution{$10!$}

\ppart How many arrangements of $BOOK_1K_2E_1E_2PE_3R$ are there?

\solution{$10! / 2!$}

\ppart How many arrangements of $BOOKKE_1E_2PE_3R$ are there?

\solution{$10! / (2! \cdot 2!)$}

\ppart How many arrangements of $BOOKKEEPER$ are there?

\solution{
\[
\binom{10}{1,2,2,3,1} \eqdef
   \frac{10!}{1!\ 2! \ 2! \ 3! \ 1!}
  = \frac{10!}{(2!)^2 \ 3!}
\]
}

\eparts

\begin{center}
\textit{Remember well what you have learned: subscripts on, subscripts off.}

\textit{This is the Tao of Bookkeeper.}
\end{center}

\bparts

\ppart How many arrangements of $VOODOODOLL$ are there?

\solution{
\[
\binom{10}{1,2,5,2} \eqdef \frac{10!}{1!\ 2! \ 5! \ 2!}
\]
}

\ppart How many length $52$ sequences of digits contain exactly 17 two's,
23 fives, and 12 nines?

\solution{
\[
\binom{52}{17,23,12} \eqdef \frac{52!}{17!\ 23!\ 12!}
\]
}

\eparts

\end{problem}

%%%%%%%%%%%%%%%%%%%%%%%%%%%%%%%%%%%%%%%%%%%%%%%%%%%%%%%%%%%%%%%%%%%%%
% Problem ends here
%%%%%%%%%%%%%%%%%%%%%%%%%%%%%%%%%%%%%%%%%%%%%%%%%%%%%%%%%%%%%%%%%%%%%


\endinput
