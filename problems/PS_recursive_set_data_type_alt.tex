\documentclass[problem]{mcs}

\begin{pcomments}
  \pcomment{Part (b) replaced by zabel 10/12/17}
  \pcomment{PS_recursive_set_data_type_alt}
  \pcomment{ARM 3/8/16, revised 3/9/17}
  \pcomment{basically a rephrasing of Fundamental Thm for win-lose
    games in the text}
  
\end{pcomments}

\pkeywords{
  set_theory
  recursive_data
  structural_induction
  game
  strategy
}

\newcommand{\recset}{\text{Recset}}

%%%%%%%%%%%%%%%%%%%%%%%%%%%%%%%%%%%%%%%%%%%%%%%%%%%%%%%%%%%%%%%%%%%%%
% Problem starts here
%%%%%%%%%%%%%%%%%%%%%%%%%%%%%%%%%%%%%%%%%%%%%%%%%%%%%%%%%%%%%%%%%%%%%

\begin{problem}
In this problem, structural induction provides a simple proof about
some utterly infinite objects.

\begin{definition*}
The class \recset\ of \emph{recursive sets} is defined recursively as follows:

\inductioncase{Base case}: The empty set $\emptyset$ is a \recset.

\inductioncase{Constructor step}: If $S$ is a nonempty set of
\recset's, then $S$ is a \recset.
\end{definition*}

\iffalse
\bparts

\ppart Prove that \recset\ satisfies the Foundation Axiom: there is no
infinite sequence of \recset, $r_0,r_1,\dots,r_{n-1}, r_n,\dots$ such
that
\begin{equation}\label{indecr}
\dots r_n \in r_{n-1} \in \dots r_1 \in r_0.
\end{equation}

\hint Structural induction.

\begin{solution}
We prove by structural induction on the definition of \recset, that no
\recset $R$ can be the start $r_0$ of an infinite decreasing
sequence of members shown in~\eqref{indecr}.

\inductioncase{Base case}: ($R = \emptyset$).  $R$ cannot be $r_0$
because there is no $r_1$.

\inductioncase{Constructor step}: ($R$ is a nonempty set of \recset's).

If $R=r_0$ then $r_1 \in R$, and by structural induction hypothesis,
$r_1$ cannot be the start of an infinite decreasing sequence of
members, so neither can $R$.
\end{solution}

\ppart Show that the collection \recset\ is not, itself, a set.

\hint Use the Foundation Axiom.

\begin{solution}
  Assume \recset\ is a set. Then the constructor case applies to
  \recset\ itself, since all of its members certainly belong to
  \recset, so it follows that $\recset\in\recset$. But no set can be a
  member of itself---this is a consequence of the Foundation axiom,
  described in Section~\bref{avoidingrussellsparadox}. So our
  assumption that \recset\ is a set must be false.

  To see why $S\notin S$ follows from the Foundation axiom (as
  formulated in terms of member-minimal elements), observe that if
  $S\in S$, the singleton set $\{S\}$ has no member-minimal
  element. Indeed, the only candidate is $S$ itself, but this is not
  member minimal because $S\in S$.
\end{solution}

% of $S$'s elements are sets, all elements of $S$'s elements are sets, 

% Prove that \emph{every} \emph{pure} set is a \recset.\footnote{A
%   ``pure'' set is empty or is a set whose elements are all pure sets.}

%\hint Use the Foundation axiom.

% \begin{solution}
%   If there were a pure set $S_1$ that was not a \recset, then $S_1$ must be nonempty (since $\emptyset\in\recset$), and $S_1$ must have some element $S_2\in S_1$ that is not a \recset\ (otherwise the \recset\ constructor case would apply to $S_1$). So $S_2$ is a pure set and not a \recset, meaning the same argument may be applied: there exists an element $S_3\in S_2$ that is pure but not a \recset, an element $S_4\in S_3$ with the same properties, and so on for $S_5\ni S_6\ni S_7 \ni \cdots$. This violates the Foundation axiom. 
% %
% % If there was a pure set that was not a \recset, then by Foundation
% % there would be a member-minimal set $S \notin \recset$.  By
% % minimality, every member of $S$ is a member of \recset, and therefore
% % by the constructor step, the set $S$ itself is a \recset, a
% % contradiction.
% \end{solution}

\ppart
\fi

Every \recset\ defines a two-person game in which a player's move
consists of choosing any element of the game.  The two players
alternate moves, and a player loses when it is their turn to move and
there is no move to make.  That is, whoever moves to the empty set is
a winner, because the next player has no move.

So we think of $R \in \recset$ as the initial ``board position'' of a
game.  The player who goes first in $R$ is called the \emph{Next}
player, and the player who moves second in $R$ is called the
\emph{Previous} player.  When the Next player moves to an $S \in R$,
the game continues with the new game $S$ in which the Previous player
moves first.

Prove that for every game in \recset, either the Previous player or
the Next player has a winning strategy.\footnote{\recset\ games are
  called ``uniform'' because the two players have the \emph{same}
  objective: to leave the other player stuck with no move to make.  In
  more general games, the two players have different objectives, for
  example, one wants to maximize the final payoff and the other wants
  to minimize it (Problem~\bref{PS_VG}).}

\begin{solution}
Call a \recset\ a \emph{P-game} if the Previous player has a winning
strategy and an \emph{N-game} if the Next player has a winning
strategy.  We prove that every \recset $R$ is a P-game or and
N-game, by structural induction on the definition of \recset.

\inductioncase{Base case} ($R = \emptyset$): $R$ is an P-game since
the Next player cannot move.

\inductioncase{Constructor step}: ($R$ is a nonempty set of
\recset's): By structural induction, we may assume that every $s \in
R$ is a P-Game or an N-game.  If there is an $s \in R$ that is a
P-Game, then the Next player wins by picking $s$ (since he will be the
Previous player in $s$), and following the P-winning strategy in $s$.
This makes $R$ an N-game.

Otherwise, every $s \in R$ is an N-game, and the Previous player will
have a winning strategy to follow regardless of which game the Next
player selects.  This makes $R$ a P-game.
\end{solution}

%\eparts

\end{problem}

%%%%%%%%%%%%%%%%%%%%%%%%%%%%%%%%%%%%%%%%%%%%%%%%%%%%%%%%%%%%%%%%%%%%%
% Problem ends here
%%%%%%%%%%%%%%%%%%%%%%%%%%%%%%%%%%%%%%%%%%%%%%%%%%%%%%%%%%%%%%%%%%%%%

\endinput
