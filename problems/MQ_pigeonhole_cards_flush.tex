\documentclass[problem]{mcs}

\begin{pcomments}
  \pcomment{MQ_pigeonhole_cards_flush}
\end{pcomments}

\pkeywords{
  counting
  pigeonhole
  cards
  flush
}

%%%%%%%%%%%%%%%%%%%%%%%%%%%%%%%%%%%%%%%%%%%%%%%%%%%%%%%%%%%%%%%%%%%%%
% Problem starts here
%%%%%%%%%%%%%%%%%%%%%%%%%%%%%%%%%%%%%%%%%%%%%%%%%%%%%%%%%%%%%%%%%%%%%


\begin{problem}
A standard 52 card deck has 13 cards of each suit.  Use the Pigeonhole
Principle to determine the smallest $k$ such that every set of $k$
cards from the deck contains five cards of the same suit (called a
\term{flush}).  Clearly indicate what are the pigeons, holes, and
rules for assigning a pigeon to a hole.

\examspace[2in]

\begin{solution}
$17$.

The $k$ cards are the pigeons, the four suits are the holes, and a
card is assigned to its suit.  By the Generalized Pigeonhole
Principle, there must be at least
\[
\ceil{\frac{k}{4}}
\]
pigeons in some hole.  So we want the smallest integer $k$ such that
\[
\frac{k}{4} > 4.
\]
\end{solution}

\end{problem}

%%%%%%%%%%%%%%%%%%%%%%%%%%%%%%%%%%%%%%%%%%%%%%%%%%%%%%%%%%%%%%%%%%%%%
% Problem ends here
%%%%%%%%%%%%%%%%%%%%%%%%%%%%%%%%%%%%%%%%%%%%%%%%%%%%%%%%%%%%%%%%%%%%%

\endinput
