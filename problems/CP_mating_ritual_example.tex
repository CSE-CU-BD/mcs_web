\documentclass[problem]{mcs}

\begin{pcomments}
  \pcomment{CP_mating_ritual_example}
  \pcomment{from: S08 cp5f; S06, cp4f; F03 rec4}
  \pcomment{second part soln upated ARM 9/30/15}
\end{pcomments}

\pkeywords{ stable_matching optimal pessimal Mating_ritual }


%%%%%%%%%%%%%%%%%%%%%%%%%%%%%%%%%%%%%%%%%%%%%%%%%%%%%%%%%%%%%%%%%%%%%
% Problem starts here
% %%%%%%%%%%%%%%%%%%%%%%%%%%%%%%%%%%%%%%%%%%%%%%%%%%%%%%%%%%%%%%%%%%%%

\newcommand{\Jeff}{Sarah} \newcommand{\Tina}{Tasha}
\newcommand{\Jay}{Elizabeth}

\begin{problem} Four Students want separate assignments to four VI-A
Companies.  Here are their preference rankings:
\begin{center}
\begin{tabular}{r|c}
Student & Companies \\ \hline Albert: & HP, Bellcore, AT\&T, Draper
\\ \Jeff: & AT\&T, Bellcore, Draper, HP \\ \Tina: & HP, Draper, AT\&T,
Bellcore \\ \Jay: & Draper, AT\&T, Bellcore, HP
\end{tabular}
\end{center}
\begin{center}
\begin{tabular}{r|c}
Company & Students \\ \hline AT\&T: & \Jay, Albert, \Tina, \Jeff
\\ Bellcore: & \Tina, \Jeff, Albert, \Jay \\ HP: & \Jay, \Tina,
Albert, \Jeff \\ Draper: & \Jeff, \Jay, \Tina, Albert
\end{tabular}
\end{center}

\bparts

\ppart Use the Mating Ritual to find \emph{two} stable assignments of
Students to Companies.

\begin{solution}
Treat Students as Boys and the result is the following assignment:
\begin{center}
\begin{tabular}{r|c|c}
Student & Companies & Rank in the original list \\ \hline Albert: &
Bellcore & 2\\ \Jeff: & AT\&T & 1\\ \Tina: & HP & 1\\ \Jay: & Draper
&1
\end{tabular}
\end{center}

Treat Companies as Boys and the result is the following assignment:
\begin{center}
\begin{tabular}{r|c|c}
Company & Students & Rank in the original list\\ \hline AT\&T: &
Albert & 2\\ Bellcore: & \Jeff & 2\\ HP: & \Tina & 2\\ Draper: & \Jay
& 2
\end{tabular}
\end{center}

\end{solution}

\ppart Describe a simple procedure to determine whether any given
stable marriage problem has a unique solution, that is, only one
possible stable matching.  Briefly explain why it works.

\begin{solution}
See if the Mating Ritual with Boys as suitors yields the same solution
as the algorithm with Girls as suitors.  These two marriage
assignments are respectively boy-optimal and boy-pessimal
(Theorem~\bref{boyopt}).

To see why this implies uniqueness, suppose Alice is both Harry's
optimal wife and also his pessimal wife.  Now in any set of stable
marriages, if Harry had a wife other than Alice, that wife would
either be better than Alice or worse than Alice, which is impossible.
So Alice is Harry's only feasible wife.  So when optimal and pessimal
marriages are the same, every boy has only one feasible wife, which
means only one set of stable marriages is possible.

There is also a simple alternative argument\footnote{Suggested by
Sravya Bhamidipati, September, 2015.}  that does not depend on the
  fact that Boy-optimal is Girl-pessimal.  Namely, suppose a couple is
  married in both the Girl-optimal and Boy-optimal stable marriage
  sets.  Now if they were not married in some other stable marriage
  set, then since each is optimal for the other, they would prefer
  each other to their spouses; that is, they would be a rogue couple,
  contradicting stability of this other marriag set.

These arguments actually yield a stronger result than stated: if a boy
is married to the same girl in both the Boy-optimal and Girl-optimal
marriage sets, then this girl is his only feasible wife.  This holds
even if the Boy-optimal and Girl-optimal matchings are not completely
the same.

\begin{staffnotes}
Students usually can figure out to look for the same solution when
Boys' and Girls' roles are reversed, but they often come up with vague
and unconvincing arguments for uniqueness like ``If a set of marriages
is optimal for boys and also for girls, it must be unique.''  This
leaves unexplained \emph{why} it is unique---for example, why is it
impossible to have another stable set that was suboptimal for Boys and
for Girls?
\end{staffnotes}

\end{solution}

\eparts

\end{problem}

\endinput
