\documentclass[problem]{mcs}

\begin{pcomments}
  \pcomment{CP_mating_ritual_example}
  \pcomment{from: S08 cp5f; S06, cp4f; F03 rec4}
\end{pcomments}

\pkeywords{
 stable_matching
 optimal
 pessimal
 Mating_ritual
}


%%%%%%%%%%%%%%%%%%%%%%%%%%%%%%%%%%%%%%%%%%%%%%%%%%%%%%%%%%%%%%%%%%%%%
% Problem starts here
%%%%%%%%%%%%%%%%%%%%%%%%%%%%%%%%%%%%%%%%%%%%%%%%%%%%%%%%%%%%%%%%%%%%%

\newcommand{\Jeff}{Sarah}
\newcommand{\Tina}{Tasha}
\newcommand{\Jay}{Elizabeth}

\begin{problem} Four Students want separate assignments to four VI-A
Companies.  Here are their preference rankings:
\begin{center}
\begin{tabular}{r|c}
Student & Companies \\ \hline
Albert:  & HP, Bellcore, AT\&T, Draper \\
\Jeff:     & AT\&T, Bellcore, Draper, HP \\
\Tina:   & HP, Draper, AT\&T, Bellcore \\
\Jay:    & Draper, AT\&T, Bellcore, HP
\end{tabular}
\end{center}
\begin{center}
\begin{tabular}{r|c}
Company   & Students \\ \hline
AT\&T:    & \Jay, Albert, \Tina, \Jeff \\
Bellcore: & \Tina, \Jeff, Albert, \Jay \\
HP:   & \Jay, \Tina, Albert, \Jeff \\
Draper:   & \Jeff, \Jay, \Tina, Albert
\end{tabular}
\end{center}

\bparts

\ppart Use the Mating Ritual to find \emph{two} stable assignments of
Students to Companies.

\begin{solution}
Treat Students as Boys and the result is the following
assignment:
\begin{center}
\begin{tabular}{r|c|c}
Student & Companies & Rank in the original list \\ \hline
Albert:  & Bellcore & 2\\
\Jeff:     & AT\&T & 1\\
\Tina:   & HP & 1\\
\Jay:    & Draper &1
\end{tabular}
\end{center}

Treat Companies as Boys and the result is the following assignment:
\begin{center}
\begin{tabular}{r|c|c}
Company & Students & Rank in the original list\\ \hline
AT\&T:    & Albert & 2\\
Bellcore: & \Jeff & 2\\
HP:   & \Tina & 2\\
Draper:   & \Jay & 2
\end{tabular}
\end{center}

\end{solution}

\ppart Describe a simple procedure to determine whether any given stable
marriage problem has a unique solution, that is, only one possible stable
matching.

\begin{solution}
See if the Mating Ritual with Boys as suitors yields the same solution
as the algorithm with Girls as suitors.  These two marriage
assignments are respectively boy-optimal (Theorem~\bref{boyopt}) and
boy-pessimal (Theorem~\bref{girlpess}).  If every boy's optimal and
pessimal choices are the same, then this is the only possible stable
mate for the boy.  So the solution is unique.

\end{solution}

\eparts

\end{problem}

\endinput
