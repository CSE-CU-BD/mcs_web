\documentclass[problem]{mcs}

\begin{pcomments}
  \pcomment{TP_relation_properties_expressions}
  \pcomment{F13.cp7f}
  \pcomment{10/12/13 ARM}
\end{pcomments}

\pkeywords{
  irreflexive
  transitive
  reflexive
  antisymmetric
  asymmetric
  identity_relation
}

%%%%%%%%%%%%%%%%%%%%%%%%%%%%%%%%%%%%%%%%%%%%%%%%%%%%%%%%%%%%%%%%%%%%%
% Problem starts here
%%%%%%%%%%%%%%%%%%%%%%%%%%%%%%%%%%%%%%%%%%%%%%%%%%%%%%%%%%%%%%%%%%%%%

\begin{problem}
Let $R$ be a binary relation on a set $D$.  Each of the following
equalities and containments expresses the fact that $R$ has one of the
basic relational properties: reflexive, irreflexive, symmetric,
asymmetric, antisymmetric, transitive.  Identify which property is
expressed by each of these formulas.

\bparts

\ppart $R \intersect \ident{D} =  \emptyset$
\begin{solution}
\textbf{irreflexive}

Read this expression as saying ``None of the pairs $(d,d) \in
\ident{D}$ are in $R$,'' that is $\QNOT(d \mrel{R} d)$ for all $d \in
D$, which is the definition of irreflexivity.
\end{solution}

\ppart $R \subseteq \inv{R}$
\begin{solution}
\textbf{symmetric}

Read this expression as saying ``If $a \mrel{R} b$, then $a
\mrel{\inv{R}} b$,'' in other words ``If $a \mrel{R} b$, then $b
\mrel{R} a$,'' which is the definition of symmetric.

\end{solution}

\ppart $R = \inv{R}$
\begin{solution}
\textbf{symmetric}

Read this expression as saying ``$a \mrel{R} b\ \QIFF\ a
\mrel{\inv{R}} b$,'' which is equivalent to ``$a \mrel{R} b\ \QIFF\, b
\mrel{R} a$,'' which means that $R$ is symmetric.

\end{solution}

\ppart $\ident{D} \subseteq R$
\begin{solution}
\textbf{reflexive}
\end{solution}

\ppart $R \compose R \subseteq R$
\begin{solution}
\textbf{transitive}

Read this expression as saying ``If there is a path of length two from
$a$ to $b$ in the graph of $R$, then there is an edge from $a$ to $b$.
That is, if $a \mrel{R} c \QAND c \mrel{R} b$ for some $c \in D$, then $a
\mrel{R} b$, which is the definition of $R$ being transitive.
\end{solution}

\ppart $R \intersect \inv{R} = \emptyset$
\begin{solution}
\textbf{asymmetric}
\end{solution}

\ppart $R \intersect \inv{R} \subseteq \ident{D}$
\begin{solution}
\textbf{antisymmetric}
\end{solution}

\eparts

%%%%%%%%%%%%%%%%%%%%%%%%%%%%%%%%%%%%%%%%%%%%%%%%%%%%%%%%%%%%%%%%%%%%%
% Problem ends here
%%%%%%%%%%%%%%%%%%%%%%%%%%%%%%%%%%%%%%%%%%%%%%%%%%%%%%%%%%%%%%%%%%%%%

\endinput

