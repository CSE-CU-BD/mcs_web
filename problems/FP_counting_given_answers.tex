\documentclass[problem]{mcs}

\begin{pcomments}
  \pcomment{FP_counting_given_answers}
  \pcomment{longer version of CP_counting_given_answers}
  \pcomment{from: S08.final,prob7; F03.final,prob5; F01.final}
  \pcomment{Adapted by Steven F09; edited by ARM 12/8/09}
  \pcomment{was duplicated by FP_combinatorics_and_counting}
\end{pcomments}

\pkeywords{
  counting
  binomial_coefficient
  bijection
  distinguishable
}

%%%%%%%%%%%%%%%%%%%%%%%%%%%%%%%%%%%%%%%%%%%%%%%%%%%%%%%%%%%%%%%%%%%%%
% Problem starts here
%%%%%%%%%%%%%%%%%%%%%%%%%%%%%%%%%%%%%%%%%%%%%%%%%%%%%%%%%%%%%%%%%%%%%

\begin{problem}

\newcommand{\apart}[1]{
    \ppart \ \ \parbox[t]{5.0in}{#1}
\hfill\examrule[0.5in]}

Here are the solutions to the short answer questions below, in no
particular order.  Enter the solution number after each question.

\[\begin{array}{rlrlrlrl}
1.& \dfrac{n!}{(n-m)!} & 2. & \dbinom{n+m}{m} & 3. & (n-m)! & 4. & m^n\\
\\
5. & \dbinom{n-1+m}{m} & 6. & \dbinom{n-1+m}{n} & 7. & 2^{mn} & 8. & n^m
\end{array}\]

\begin{staffnotes}
In class, ask that a brief explanation be written for each answer.
\end{staffnotes}

\bparts

\apart{How many solutions over the nonnegative integers are there to the
inequality
\[
x_1 + x_2 + \cdots + x_n \leq m\ ?
\]}

\begin{solution}
\[
\binom{n+m}{m}
\]
This is the same as the number of solutions to the equation the
equality $x_1 + x_2 + \cdots + x_n + y = m$, and which has a bijection
to bit-sequences with $m$ \texttt{0}'s and $n$ \texttt{1}'s.
\end{solution}

\apart{How many $\sqrt{n} \times \sqrt{n}$ matrices are there with
entries drawn from $\set{1, 2, \ldots, m}$?}

\begin{solution}
$m^n$ This follows from the Product Rule, since there are $n^2$ matrix
  entries and $m$ choices for each.
\end{solution}

\apart{How many different subsets of the set $A \times B$ are there,
if $\card{A} = m$ and $\card{B} = n$?}

\begin{solution}
$2^{mn}$, because $\card{A \times B}= mn$.
\end{solution}

\apart{How many functions are there from set $A$ to set $B$, if $\card{A} =
n$ and $\card{B} = m$?}

\begin{solution}
$m^n$
\end{solution}

\apart{How many length $m$ words can be formed from an $n$-letter
alphabet, if no letter is used more than once?}

\begin{solution}
\[
\frac{n!}{(n-m)!}
\]
There are $n$ choices for the first letter, $n-1$ choices for the
second letter, \dots $n - m +1$ choices for the $m$th letter, so by
the Generalized Product rule, the number of words is
\[
n \cdot (n-1) \cdots (n-m +1).
\]

%that is, $P(n,m)$.
\end{solution}

\apart{How many length $m$ words can be formed from an $n$-letter
  alphabet, if letters can be reused?}

\begin{solution}
$n^m$ by the Product Rule.
\end{solution}

\apart{How many binary relations are there from set $A$ to set $B$
  when $\card{A} = m$ and $\card{B} = n$?}

\iffalse
\apart{How many different subsets of the set $A \times B$ are there
  when $\card{A} = m$ and $\card{B} = n$?}
\fi

\begin{solution}
\[
2^{mn}
\]
The graph of a binary relations from $A$ to $B$ is a subset of $A
\times B$.  There are on $2^{mn}$ such subsets because $\card{A \times
  B}= mn$.
\end{solution}

\apart{How many total injective functions are there from set $A$ to
  set $B$, where $\card{A} = m$ and $\card{B} = n \geq m$?}

\begin{solution}
\[
\frac{n!}{(n-m)!}
\]
There is a bijection between the injections and the length $m$
sequences of distinct elements of $B$.  By the Generalized Product
rule, the number of such sequences is
\[
n \cdot (n-1) \cdots (n-m +1).
\]

%that is, $P(n,m)$.
\end{solution}


\apart{How many ways are there to place a total of $m$ distinguishable
  balls into $n$ distinguishable urns, with some urns possibly empty or
  with several balls?}

\begin{solution}
\[
n^m
\]

There is a bijection between a placement of the balls and length $m$
sequence whose $i$th element is the urn where the $i$th ball is
placed.  So the number of placements is the same as the number of
length $m$ sequences of elements from a size-$n$ set.
\end{solution}

\apart{How many ways are there to place a total of $m$ indistinguishable
  balls into $n$ distinguishable urns, with some urns possibly empty or
  with several balls?}

\begin{solution}
\[
\binom{n - 1 + m}{m}
\]
This is the same as the number of selections of $m$ donuts with $n$
possible flavors, which is the number of bit-sequences with $m$
\texttt{0}'s and $n - 1$ \texttt{1}'s.
\end{solution}

\apart{How many ways are there to put a total of $m$ distinguishable
  balls into $n$ distinguishable urns with at most one ball in each
  urn?}

\begin{solution}
\[
\frac{n!}{(n-m)!}
\]
There is a bijection between a placement of balls and a length $m$
sequence whose $i$th element is the urn containing the $i$th ball.  So
the number of ball placements is the same as number of length $m$
sequences of distinct elements from a set of $n$ elements.
%that is, $P(n,m)$.
\end{solution}

\eparts
\end{problem}

%%%%%%%%%%%%%%%%%%%%%%%%%%%%%%%%%%%%%%%%%%%%%%%%%%%%%%%%%%%%%%%%%%%%%
% Problem ends here
%%%%%%%%%%%%%%%%%%%%%%%%%%%%%%%%%%%%%%%%%%%%%%%%%%%%%%%%%%%%%%%%%%%%%

\endinput
