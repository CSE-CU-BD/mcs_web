\documentclass[problem]{mcs}

\begin{pcomments}
  \pcomment{TP_Images}
  \pcomment{Converted from ./00Convert/probs/practice3/prob9.scm
    by scmtotex and drewe
    on Thu 21 Jul 2011 12:06:12 PM EDT}
  \pcomment{format ARM 8/25/11}
  \pcomment{Earlier version converted from mapping-lemma.scm by scmtotex and dmj
              on Sat 12 Jun 2010 09:47:19 PM EDT}
\end{pcomments}

\pkeywords{
  relations
  functions
  injections
  surjections
  mapping_rule
  cardinality
}

\begin{problem}
\iffalse
For any function $f: A \to B$ and subset, $A^\prime \subset A$, we define
\begin{equation*}
   f(A^\prime) \eqdef \set{ f(a) \suchthat a \in A^\prime}
\end{equation*}
For example, if $f(x)$ is the doubling function, $2x$, with domain and
codomain equal to the real numbers, then $f(\integers)$ defines the
set of even integers. \inhandout{($\integers$ stands for the integers.)}

Now \fi

Assume $f: A \rightarrow B$ is total function and $A$ is finite, and replace the $\star$
with one of $\le, =, \ge$ to produce the \emph{strongest} correct
version of the following statements:

\bparts

\ppart
$\card{f(A)} \star \card{B}$.

\begin{solution}
$\le$, since $f(A)  \subset  B$.
\end{solution}

\ppart
If $f$ is a surjection, then $\card{A} \star \card{B}$.

\begin{solution}
$\ge$, by the Mapping Rule.
\end{solution}

\ppart
If $f$ is a surjection, then $\card{f(A)} \star \card{B}$.

\begin{solution}
$=$, since $f(A) = B$.
\end{solution}

\ppart
If $f$ is an injection, then $\card{f(A)} \star \card{A}$.

\begin{solution}
$=$.
\end{solution}

\ppart
If $f$ is a bijection, then $\card{A} \star \card{B}$.

\begin{solution}
$=$, by the Mapping Rule.
\end{solution}

\eparts

\end{problem}

\endinput
