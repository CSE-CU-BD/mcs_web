\documentclass[problem]{mcs}

\begin{pcomments}
  \pcomment{FP_logical_jections}
  \pcomment{Injections and Cardinality}
  \pcomment{ARM may 2013 based on F99.final}
  \pcomment{maybe too tricky for final; use for class prob}
  \pcomment{part (b) added ARM 5/19/17}
\end{pcomments}

\pkeywords{
  injection
  total
  function
  composition
  inverse
  strict
}

\begin{problem}

\bparts

\ppart Let $R:D \to D$ be a binary relation on a set $D$.  Let $x,y$
be variables ranging over $D$.  \inbook{Indicate}\inhandout{Circle}
the expressions below whose meaning is that $R$ is an \emph{injective
  relation} $[\leq 1\ \text{in}]$.  Remember that $R(x) \eqdef \set{y
  \suchthat x\mrel{R}y}$, and $R$ is not necessarily a function or a
total relation.

\begin{enumerate}[(i)]
\inbook{\item $R(x) \intersect R(y) = \emptyset$}
\inbook{\item $R(x) = R(y) \QIMPLIES x = y$}
\inbook{\item $R(x) \intersect R(y) = \emptyset \QIMPLIES x \neq y$}
\item\label{onlyiffuncR} $x \neq y \QIMP R(x) \neq R(y)$
\inbook{\item $R(x) \intersect R(y) \neq \emptyset \QIMPLIES x \neq y$}
\item\label{RxintRyneq} $R(x) \intersect R(y) \neq \emptyset \QIMPLIES x = y$
\inbook{\item $\inv{R}(R(x)) = \set{x}$}
\item\label{invRRsubsx} $\inv{R}(R(x)) \subseteq \set{x}$
\item $\inv{R}(R(x)) \supseteq \set{x}$
\inbook{\item $R(\inv{R}(x)) \subseteq \set{x}$}
\inbook{\item\label{RinvRsupsx} $R(\inv{R}(x)) \supseteq \set{x}$}
\inbook{\item\label{contraRxintRy} $x \neq y \QIMP R(x) \intersect R(y) = \emptyset$}
\end{enumerate}

\begin{solution}
\eqref{RxintRyneq}\inbook{, \eqref{RinvRsupsx}, \eqref{contraRxintRy}} and~\eqref{invRRsubsx}.

Note that~\eqref{onlyiffuncR} works only if $R$ is a function.
\end{solution}

\ppart Give an example of a set $S$ such that there is no total
injective relation from $S$ to the real interval $[0,1]$.

\begin{center}
\exambox{2.0in}{0.6in}{-0.1in}
\end{center}

\begin{solution}
$\power([0,1])$, or any set $S$ such that $S \surj \power([0,1])$.

Note that there is no total injective relation from $A$ to $B$ iff
there is no surjective function from $B$ to $A$ iff $B \strict A$.  By
Cantor's Theorem~\bref{powbig},
\[
\power([0,1]) \strict [0,1].
\]
\end{solution}

\eparts

\end{problem}

\endinput
