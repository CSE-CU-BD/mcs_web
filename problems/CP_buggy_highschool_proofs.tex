\documentclass[problem]{mcs}

\begin{pcomments}
  \pcomment{from: S09.cp1t}
%  \pcomment{}
%  \pcomment{}
\end{pcomments}

\pkeywords{
  faulty_reasoning
}

%%%%%%%%%%%%%%%%%%%%%%%%%%%%%%%%%%%%%%%%%%%%%%%%%%%%%%%%%%%%%%%%%%%%%
% Problem starts here
%%%%%%%%%%%%%%%%%%%%%%%%%%%%%%%%%%%%%%%%%%%%%%%%%%%%%%%%%%%%%%%%%%%%%

\begin{problem}
Identify exactly where the bugs are in each of the following bogus
proofs.\footnote{From Stueben, Michael and Diane Sandford. \emph{Twenty
Years Before the Blackboard}, Mathematical Association of America, \copyright 1998.}

\bparts

\problempart \textbf{Bogus Claim}: $1/8 > 1/4.$
\begin{bogusproof}
\begin{align*}
    3 &> 2 \\
    3 \log_{10} (1/2) &> 2 \log_{10}(1/2) \\
    \log_{10} (1/2)^3 &> \log_{10} (1/2)^2 \\
    (1/2)^3 &> (1/2)^2,
\end{align*}
and the claim now follows by the rules for multiplying fractions.
\end{bogusproof}

\begin{solution}
$\log x < 0$, for $0<x<1$, so since both sides of the inequality
``$3 > 2$'' are being multiplied by the negative quantity
$\log_{10}(1/2)$, the ``$>$'' in the second line should have been
``$<$.''
\end{solution}

\ppart \emph{Bogus proof}: $1 \mbox{\textcent} = \$0.01 = (\$0.1)^2 = (10\mbox{\textcent})^2 =
100\mbox{\textcent} = \$1.\qed$

\begin{solution}
$\$0.01 = \$(0.1)^2 \neq (\$0.1)^2$ because the units $\$^2$ and
$\$$ don't match (just as in physics the difference between $sec^2$ and
$sec$ indicates the difference between acceleration and velocity).
Similarly, $(10\mbox{\textcent})^2 \neq 100$\textcent.

\end{solution}

%FALSE PROOF ARITHMETIC, from Spring94 revised ARM 9/3/01

\ppart \textbf{Bogus Claim}: If $a$ and $b$ are two equal real numbers,
then $a=0$.
\begin{bogusproof}
\begin{eqnarray*}
a&=&b \\
a^2&=&ab \\
a^2-b^2&=&ab-b^2 \\
(a-b)(a+b)&=&(a-b)b\\ % \label{cancel}\\
a+b&=&b\\ % \label{bug}\\
a&=&0.
\end{eqnarray*}
\end{bogusproof}

\begin{solution}
The bug is at the fifth line:
\iffalse~(\ref{bug})\fi
one cannot cancel $(a-b)$ from
both sides of the equation on the fourth line
\iffalse~(\ref{cancel})\fi
because $a-b = 0$.
\end{solution}

\eparts
\end{problem}

%%%%%%%%%%%%%%%%%%%%%%%%%%%%%%%%%%%%%%%%%%%%%%%%%%%%%%%%%%%%%%%%%%%%%
% Problem ends here
%%%%%%%%%%%%%%%%%%%%%%%%%%%%%%%%%%%%%%%%%%%%%%%%%%%%%%%%%%%%%%%%%%%%%

\endinput
