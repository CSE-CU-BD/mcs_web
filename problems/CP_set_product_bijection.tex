\documentclass[problem]{mcs}

\begin{pcomments}
  \pcomment{CP_set_product_bijection}
  \pcomment{written by ARM 9/20/09}
  \pcomment{solution wording modified by CH 2/12/14}
\end{pcomments}

\pkeywords{
 set_product
 Cartesian_product
 bijection
}

%%%%%%%%%%%%%%%%%%%%%%%%%%%%%%%%%%%%%%%%%%%%%%%%%%%%%%%%%%%%%%%%%%%%%
% Problem starts here
%%%%%%%%%%%%%%%%%%%%%%%%%%%%%%%%%%%%%%%%%%%%%%%%%%%%%%%%%%%%%%%%%%%%%
\begin{problem}
Let $A = \set{a_0,a_1,\dots,a_{n-1}}$ be a set of size $n$, and $B =
\set{b_0,b_1,\dots,b_{m-1}}$ a set of size $m$.  Prove that $\card{A
  \times B} = mn$ by defining a simple bijection from $A \times B$ to
the nonnegative integers from $0$ to $mn-1$.

\begin{staffnotes}
Point out the Computer Science connection: this is how a compiler
computes locations in memory of a 2D-array.  It also indicates why
most programming languages require the programmer to specify the array
dimensions in advance.
\end{staffnotes}

\begin{solution}
Let $[0,mn) \eqdef \set{0,1,\dots,mn-1}$. Define a mapping $f:A \times B \to [0,mn)$ by the rule
\[
f(a_j,b_k) \eqdef jm + k.
\]

We will prove that $f$ is a bijection. First, observe that $f$ is a total function
with domain $A \times B$. This is true since $f$ maps every pair
$(a_j,b_k)$ to a nonnegative integer $jm+k$. Therefore, $f$ has the
$[=1~\mathrm{out}]$ property. The range of $f$ is a {subset} of $[0,mn)$ because for $0
\leq j \leq n-1$ and $0 \leq k \leq m-1$,
\[
0 \leq jm + k \leq (n-1)m + m-1 = mn-1.
\]

Further, observe that $f$ is a surjection (i.e., it has the $[\geq 1~\mathrm{in}]$
property.) This follows due to the
fact that every element of $i \in [0,mn)$ is $f(a_j,b_k)$ for at least one pair $(j,k)$, namely, the
  $j$ is the quotient and $k$ is the remainder of $i$ divided by $m$.

In fact, the quotient-remainder pair $(j,k)$ is {\em unique} for every $i$. The proof is by
contradiction. Assume the contrary, i.e., suppose that $i = jm+k =
j'm+k'$, where $ $, $k \neq k'$. Without loss of generality, suppose that
$k > k'$. Then, $j < j'$ (why?) and hence, $j'-j \geq 1$. Therefore, we get $k-k' = m(j'-j)$. The
left hand side is positive and strictly smaller than $m$, while the
right hand side is at least $m$. 

From this uniqueness, it follows $f$ has the $[=1~\mathrm{in}]$
property as well. Therefore, $f$ is a bijection.

\end{solution}

\end{problem}

\endinput

