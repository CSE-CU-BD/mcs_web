\documentclass[problem]{mcs}

\begin{pcomments}
  \pcomment{CP_set_product_bijection}
  \pcomment{written by ARM 9/20/09}
\end{pcomments}

\pkeywords{
 set_product
 Cartesian_product
 bijection
}

%%%%%%%%%%%%%%%%%%%%%%%%%%%%%%%%%%%%%%%%%%%%%%%%%%%%%%%%%%%%%%%%%%%%%
% Problem starts here
%%%%%%%%%%%%%%%%%%%%%%%%%%%%%%%%%%%%%%%%%%%%%%%%%%%%%%%%%%%%%%%%%%%%%
\begin{problem}
Let $A = \set{a_0,a_1,\dots,a_{n-1}}$ be a set of size $n$, and $B =
\set{b_0,b_1,\dots,b_{m-1}}$ a set of size $m$.  Prove that
$\card{A \times B} = mn$ by defining a simple bijection from $A \times B$ to
the nonnegative integers from $0$ to $mn-1$.

\begin{solution}
A bijection $f:A \times B \to \set{0,1,\dots,mn-1}$ can be defined by the rule
nn\[
f(a_j,b_k) \eqdef jm + k.
\]

To verify that $f$ is a bijection, note that $f$ is a total function
with domain $A \times B$ by definition.  

The range of $f$ is a subset of $\set{0,1,\dots,mn-1}$ because for $0
\leq j \leq n-1$ and $0 \leq k \leq m-1$
\[
0 \leq jm + k \leq (m-1)n + m-1 = mn-1.
\]

Finally, $f$ is a surjection and an injection because every element of 
$i \in \set{0,1,\dots,mn-1}$ is $f(a_j,b_k)$ for a unique pair $(j,k)$,
namely, the quotient and remainder of $i$ divided by $m$.
\end{solution}

\end{problem}

\endinput

