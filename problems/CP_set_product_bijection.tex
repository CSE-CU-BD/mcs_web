\documentclass[problem]{mcs}

\begin{pcomments}
  \pcomment{CP_set_product_bijection}
  \pcomment{written by ARM 9/20/09}
  \pcomment{solution modified to use arrows by CH 2/12/14; revised by ARM 10/4/15}
\end{pcomments}

\pkeywords{
 set_product
 Cartesian_product
 bijection
}

%%%%%%%%%%%%%%%%%%%%%%%%%%%%%%%%%%%%%%%%%%%%%%%%%%%%%%%%%%%%%%%%%%%%%
% Problem starts here
%%%%%%%%%%%%%%%%%%%%%%%%%%%%%%%%%%%%%%%%%%%%%%%%%%%%%%%%%%%%%%%%%%%%%
\begin{problem}
Let $A = \set{a_0,a_1,\dots,a_{n-1}}$ be a set of size $n$, and $B =
\set{b_0,b_1,\dots,b_{m-1}}$ a set of size $m$.  Prove that $\card{A
  \times B} = mn$ by defining a simple bijection from $A \times B$ to
the nonnegative integers from $0$ to $mn-1$.

\begin{staffnotes}
Point out the Computer Science connection: this is how a compiler
computes locations in memory of a 2D-array.  It also indicates why
most programming languages require the programmer to specify the array
dimensions in advance.
\end{staffnotes}

\begin{solution}
Let $\Zintvco{0}{mn} \eqdef \set{0,1,\dots,mn-1}$.  Define a mapping
$f:A \times B \to \Zintvco{0}{mn}$ by the rule
\[
f(a_j,b_k) \eqdef jm + k.
\]
Notice that by definition, $f$ is a total function with domain $A
\times B$.  That is,$f$ has the $[=1~\mathrm{out}]$ property.  Also,
\[
f(a_j,b_k) \in \Zintvco{0}{mn}
\]
for $j \in \Zintvco{0}{n}, k \in \Zintvco{0}{m}$ because
\[
0 \leq jm + k \leq (n-1)m + m-1 = mn-1.
\]
Also, $f$ is $[=1~\mathrm{in}]$, since every element $i \in
\Zintvco{0}{mn}$ is $f(a_j,b_k)$ for a unique pair $(j,k)$, namely,
the $j$ is the quotient and $k$ is the remainder of $i$ divided by
$m$.  Therefore, $f$ is a bijection.

\end{solution}

\end{problem}

\endinput

