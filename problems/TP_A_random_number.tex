\documentclass[problem]{mcs}

\begin{pcomments}
  \pcomment{TP_A_random_number}
  \pcomment{Converted from ./00Convert/probs/practice11/pre-random-var.scm
    by scmtotex and drewe
    on Thu 04 Aug 2011 10:31:15 AM EDT}
\end{pcomments}

\pkeywords{
 random
 probability
 conditional_probability
}

\begin{problem}

Here is a process to construct a random number: first, you flip a coin
that comes up heads with probability 3/5.  If you see heads, you roll
a fair die and return the result.  Otherwise, you flip a fair coin 3
times and return twice the number of heads.

Let $N$ be the number that you return, and let $F$ be 1 if the first
coin flip came up heads, and 0 otherwise.  \bparts

\ppart
What is $\prob{N=0}$?

\begin{solution}
1/20

The only way to get 0 is if you first flip tails and then flip 0 heads. 
 Therefore
\begin{align*}
\prob{N=0} & = 
\prob{F=0 \text{ and }0\text{ heads}}\\
& = \prob{F=0}*\prob{0 \text{ heads}}\\
& = (2/5)*(1/8) = 1/20.
\end{align*}

\end{solution}

\ppart
What is $\prob{N=3}$?

\begin{solution}
2/20

The only way to get 3 is if you first flip heads and then roll a 3.  
Therefore
\begin{align*}
\prob{N=3} & =\prob{F=1 \text{ and roll }3}\\
& = \prob{F=1}*\prob{\text{roll }3}\\
& =(3/5)*(1/6) = 2/20.
\end{align*}
\end{solution}

\ppart
What is $\prob{N=6}$?
\begin{solution}
0.15

You can get 6 either by flipping heads and rolling a 6 or by flipping
tails and then flipping 3 heads. Therefore
  
\begin{align*}
\prob{N=6} = \\
  \prob{F=1 \text{ and roll }6} + \prob{F=0\text{ and 3 heads}}  = \\
  (3/5)*(1/6) + (2/5)*(1/8) =\\
  3/20.
\end{align*}

\item 

\end{solution}

\ppart
What is $\prob{N=7}$?

\begin{solution}
0.

There is no way to end up with the number 7.
\end{solution}

\ppart
What is $\prcond{N=6}{F=0}$?

\begin{solution}
0.125

Given that $F=0$, we know that the first coin flip was tails. So, we
are in the case where the fair coin is flipped 3 times.  In that case,
we get 6 only by flipping 3 heads.  The probability of this happening
is 1/8.

\end{solution}

\ppart
What is $\prcond{F=0}{N=6}$?

\begin{solution}
1/3

We use the definition of conditional probability to compute:

\[
\prcond{F=0}{N=6}
 = \frac{\prob{F=0 \QAND N=6}}{\prob{N=6}} = (2/5)*(1/8) / (3/20) = 1/3.
\]
(The denominator comes from the answer to a previous question).  Hence,
if somebody else has run the process and ended up with 6, we know
there is 1 in 3 chance that he had to flip the fair coin.
\end{solution}

\ppart
What is $\prob{N+F=5}$?

\begin{solution}
2/20

If $F=0$, then we are flipping the fair coin and doubling the number of heads, so $N$ is even, 
and therefore  $N+F$ cannot be odd.  Hence, the only way for $N+F$ to be odd is if $F=1$. 
Then $N+F=5$ if and only if $N=4$, which happens exactly when we roll a 4. Overall, 

\begin{align*}
\prob{N+F=5} = 
\prob{F=1 \QAND N=4} = 
\prob{F=1} \prob{\text{roll } 4} = (3/5)*(1/6) = 2/20.
\end{align*}

\end{solution}

\ppart
What is $\prob{N+F=6}$?

\begin{solution}
3/20

This can happen either because $F=1$ and $N=5$ or because $F=0$ and $N=6$.

\[
\prob{N+F=6} =\prob{F=1\QAND N=5} + \prob{F=0 \QAND N=6} = (3/5)*(1/6) + (2/5)*(1/8) = 3/20.
\]

\end{solution}

\eparts


\end{problem}

\endinput
