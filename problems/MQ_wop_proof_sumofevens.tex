\documentclass[problem]{mcs}

\begin{pcomments}
  \pcomment{MQ_wop_proof_sumofevens}
  \pcomment{by Chinmay, 2/14/14}
  \pcomment{edited \& formatted ARM 2/15/14}
\end{pcomments}

\pkeywords{
  wop
  well_ordering
  sum
}

%%%%%%%%%%%%%%%%%%%%%%%%%%%%%%%%%%%%%%%%%%%%%%%%%%%%%%%%%%%%%%%%%%%%%
% Problem starts here
%%%%%%%%%%%%%%%%%%%%%%%%%%%%%%%%%%%%%%%%%%%%%%%%%%%%%%%%%%%%%%%%%%%%%

\begin{problem}
Use the Well Ordering Principle (WOP) to prove that
\begin{equation}\label{}\label{sumofevens}
2+4+\cdots+2n = n(n+1)
\end{equation}
for all $n>0$.

\begin{staffnotes}
Loophole: factor out 2 and get the familiar sum.
\end{staffnotes}

\begin{solution}
The proof is by contradiction.

Suppose to the contrary that~\eqref{sumofevens} failed for some $n >
0$.  Then by the WOP, there is a smallest integer $m>0$
such~\eqref{sumofevens} does not hold for $m$.  In particular,
\begin{equation}\label{not-for-m}
2+4+\cdots+2m \neq m(m+1).
\end{equation}
But~\eqref{sumofevens} clearly holds for $n = 1$, which means that
$m > 1$.  Therefore $m-1>0$, and since it is smaller than $m$,
equation~\eqref{sumofevens} holds for $m-1$, that is,
\begin{equation}\label{sum-to-m-1}
2+4+\cdots+2(m-1) = (m-1)((m-1) +1).
\end{equation}
Adding $2m$ to both sides of equation~\eqref{sum-to-m-1}, the left
hand becomes
\[
2+4+\cdots+2(m-1) + 2m,
\]
while the right hand side becomes
\[
(m-1)((m-1) +1) +2m = (m-1)m +2m = m(m+1).
\]
We conclude that
\[
2+4+\cdots+2(m-1) + 2m = m(m+1),
\]
contradicting~\eqref{not-for-m}.
\end{solution}

\end{problem}


%%%%%%%%%%%%%%%%%%%%%%%%%%%%%%%%%%%%%%%%%%%%%%%%%%%%%%%%%%%%%%%%%%%%%
% Problem ends here
%%%%%%%%%%%%%%%%%%%%%%%%%%%%%%%%%%%%%%%%%%%%%%%%%%%%%%%%%%%%%%%%%%%%%
\endinput
