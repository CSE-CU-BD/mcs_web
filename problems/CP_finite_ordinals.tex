\documentclass[problem]{mcs}

\begin{pcomments}
  \pcomment{CP_finite_ordinals}
  \pcomment{ARM 2/17/13}
\end{pcomments}

\pkeywords{
  sets
  set_theory
  subset
  power_set
  union
  ordinal
  member
  contains
}

\newcommand{\nextset}[1]{\text{grab}(#1)}

\begin{problem}
For any set $x$, define $\nextset{x}$ to be the set consisting of all
the elements of $x$, along with $x$ itself:
\[
\nextset{x}  \eqdef x \union \set{x}
\]
Now we can define a sequence of sets $\nu_0, \nu_1, \nu_2, \dots$
called the \term{finite ordinals} with a simple recursive recipe:
\begin{align*}
\nu_0    & \eqdef \emptyset,\\
\nu_{n+1} & \eqdef \nextset{\nu_n}.
\end{align*}
So we have,
\begin{align*}
\nu_1 & \eqdef \set{\emptyset}\\ %= \set{\nu_0}
\nu_2 & \eqdef \set{\emptyset, \set{\emptyset}}\\  %= \set{\nu_0, \nu_1}
\nu_3 & \eqdef \set{\emptyset, \set{\emptyset}, \set{\emptyset, \set{\emptyset}}\\ %= \set{\nu_0, \nu_1, \nu_2}
\end{align*}
The finte ordinals are kind of weird, but turn out to have some engaging
properties, and more important, they play a siginficant role in set
theory.

\bparts
\ppart How elements are there in $\nu_n$?
\begin{solution}
$n$.
\end{solution}

\ppart Prove that $\nu_{n+1} = \set{\nu_0, \nu_1,\dots, \nu_n}$.  

\ppart Explain why $\nu_m \in \nu_n \QIFF\ m < n$.

\ppart Explain why $\nu_m \subseteq \nu_n \QIFF\ m \leq n$.

\ppart Conclude that if $\mu,\nu, \rho$ are finite ordinals and $\mu
\in \nu \in \rho$, then $mu \in \rho$.  Likewise, if $\mu,\nu$ are
different finite ordinals, then $\nu \in \mu \QOR\ \mu \in \nu$.

\end{problem}

\endinput
