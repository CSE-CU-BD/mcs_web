\documentclass[problem]{mcs}

\begin{pcomments}
  \pcomment{CP_partially_ordered_by_divisibilityb}
  \pcomment{assumes CP_divisibility_partial_order}
  \pcomment{from: F03.ps2}
  \pcomment{edited by ARM 10/17/11}
\end{pcomments}

\pkeywords{
  partial_order
  divisibility
  minimal
  maximal
  chain
  infinite_antichain
  antichain
}

%%%%%%%%%%%%%%%%%%%%%%%%%%%%%%%%%%%%%%%%%%%%%%%%%%%%%%%%%%%%%%%%%%%%%
% Problem starts here
%%%%%%%%%%%%%%%%%%%%%%%%%%%%%%%%%%%%%%%%%%%%%%%%%%%%%%%%%%%%%%%%%%%%%

\begin{problem}
Show that the set of nonnegative integers partially ordered under the divides
relation\dots

\begin{problemparts}

\problempart 
\dots has a minimum element.

\begin{solution}
1 is minimum since ity divides all nonnegative integers,
\end{solution}

\problempart 
\dots has a maximum element.

\begin{solution}
0 is maximum because everything divides 0 (including 0 ---look up the
definition of $ \divides n$).
\end{solution}

\problempart  
\dots has an infinite chain.

\begin{solution}
1 2 4 8 16 \dots is a chain with infinite length.
\end{solution}

\ppart \dots has an infinite antichain.

\begin{staffnotes}

\hint The primes.

\end{staffnotes}

\begin{solution}
The set of prime numbers is infinite.  Since no prime divides another,
any two primes are incomparable.  So the set of prime numbers is an
antichain.
\end{solution}

\ppart  What are the minimal elements of divisibility on the integers
greater than 1?  What are the maximal elements?

\begin{solution}
The primes are the minimal elements.  There are no maximal elements.
\end{solution}

\end{problemparts}

\end{problem}

%%%%%%%%%%%%%%%%%%%%%%%%%%%%%%%%%%%%%%%%%%%%%%%%%%%%%%%%%%%%%%%%%%%%%
% Problem ends here
%%%%%%%%%%%%%%%%%%%%%%%%%%%%%%%%%%%%%%%%%%%%%%%%%%%%%%%%%%%%%%%%%%%%%

 \endinput
