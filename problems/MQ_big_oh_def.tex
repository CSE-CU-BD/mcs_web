\documentclass[problem]{mcs}

\begin{pcomments}
  \pcomment{MQ_big_oh_def}
  \pcomment{by: Kazerani, 4/11}
  \pcomment{based on: CP_big_oh_practice}
  \pcomment{from: S09.cp9t, F02.quiz2}
  \pcomment{has a lot of commented out material}
\end{pcomments}

\pkeywords{
  asymptotics
  big_oh
}

%%%%%%%%%%%%%%%%%%%%%%%%%%%%%%%%%%%%%%%%%%%%%%%%%%%%%%%%%%%%%%%%%%%%%
% Problem starts here
%%%%%%%%%%%%%%%%%%%%%%%%%%%%%%%%%%%%%%%%%%%%%%%%%%%%%%%%%%%%%%%%%%%%%

\begin{problem}
\iffalse
Recall that for functions $f$ and $g$ on $\naturals$, $f = O(g)$ iff
\begin{equation}\label{Oh}
\exists c \in \naturals\, \exists n_0 \in \naturals\,
\forall n \geq n_0\quad c \cdot g(n) \geq \abs{f(n)}.
\end{equation}
\fi

Recall that for functions $f$ and $g$ on $\naturals$, $f = O(g)$ iff 
there exist constants $c$ and $n_0$, \textbf{both nonnegative integers},
such that for all $n \geq n_0$, $\abs{f(n)} \leq c g(n)$.\\\\
For each pair of functions $f$ and $g$ below, indicate the
\textbf{smallest} $c$, and for that smallest $c$, the
\textbf{smallest corresponding} $n_0$, that would establish $f = O(g)$
by the definition given above.

\begin{problemparts}

\problempart $f(n) = \sin{n}, g(n) = n$. \hfill $c =$ \brule{.5in}, $n_0$ = \brule{.5in}

\begin{solution}
$c = 1$ and $n_0=0$.
\end{solution}

\problempart $f(n) = \frac{1}{2} \ln{n^2}, g(n) = n$.  \hfill $c =$ \brule{.5in}, $n_0$ = \brule{.5in}


\begin{solution}
$f(n)=\ln{n}$, and $n$ exceeds $\ln{n}$ for all nonnegative $n$. 
Thus $c = 1$ and $n_0=0$.  ($n_0=1$ should also be acceptable in case
students suppose that logarithms are undefined, rather than infinitely negative,
at zero.)
\end{solution}

\problempart $f(n) = n, g(n) = n\ln{n}$.  \hfill $c =$ \brule{.5in}, $n_0$ = \brule{.5in}


\begin{solution}
Since $\ln{n}$ eventually grows beyond 1, it must be that $n\ln{n}$
eventually grows beyond $n$.  Thus $c=1$.  Now $f(n)=n\leq
cg(n)=g(n)=n\ln{n}$ precisely when $1 \leq \ln{n}$.  That is, when $n
\geq e$.  So $n_0=\lceil e\rceil=3$.
\end{solution}

\problempart $f(n) = 3n^2, g(n) = 2n^2-2n+\frac{3}{2}$. \hfill  $c =$
\brule{.5in}, $n_0$ = \brule{.5in}  

\begin{solution}
For $cg(n)$ to eventually rival $f(n)$, the $x^2$ term's coefficient
in $cg(n)$ must be at least as large as the corresponding coefficient
in $f(n)$.  So $c$ must be at least $\frac{3}{2}$.  The smallest
acceptable nonnegative integer value of $c$ is therefore 2. Now, solve
$cg(n)=f(n)$:
\begin{align*}
4n^2-4n+3 &= 3n^2\\
n^2-4n+3&=0\\
(n-3)(n-1)&=0
\end{align*}
So $n=1$ or $n=3$.  Thus $cg(2)< f(2)$, but $cg(n)\geq f(n)$ for all
$n\geq 3$.

Conclude that $c = 2$ and $n_0=3$.
\end{solution}

\problempart $\displaystyle f(n) = 3\sin{\paren{\frac{\pi
      (n-1)}{100}}}+2, g(n) = 0.2$. \hfill $c =$ \brule{.5in}, $n_0$ =
\brule{.5in}


\begin{solution}
$f(n)$ is periodic.  Its minimum value is $-1$ and its maximum is $5$,
  so the smallest acceptable nonnegative integer value for $c$ is
  $\frac{5}{0.2}=25$.  Now, $cg(n)$ exceeds or equals $f(n)$ for all
  $n\geq 0$, so $n_0=0$.
\end{solution}

\end{problemparts}

\end{problem}

%%%%%%%%%%%%%%%%%%%%%%%%%%%%%%%%%%%%%%%%%%%%%%%%%%%%%%%%%%%%%%%%%%%%%
% Problem ends here
%%%%%%%%%%%%%%%%%%%%%%%%%%%%%%%%%%%%%%%%%%%%%%%%%%%%%%%%%%%%%%%%%%%%%

\endinput
