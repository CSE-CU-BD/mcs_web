\documentclass[problem]{mcs}

\begin{pcomments}
  \pcomment{MQ_big_oh_def}
  \pcomment{by Kazerani, 4/11, one part cut by ARM 5/16/11}
  \pcomment{based on: CP_big_oh_practice}
\end{pcomments}

\pkeywords{
  asymptotics
  big_oh
}

%%%%%%%%%%%%%%%%%%%%%%%%%%%%%%%%%%%%%%%%%%%%%%%%%%%%%%%%%%%%%%%%%%%%%
% Problem starts here
%%%%%%%%%%%%%%%%%%%%%%%%%%%%%%%%%%%%%%%%%%%%%%%%%%%%%%%%%%%%%%%%%%%%%

\begin{problem}

Recall that if $f$ and $g$ are nonnegative real-valued functions on
$\integers^+$, then $f = O(g)$ iff there exist $c, n_0 \in
\integers^+$ such that
\[
\forall n \geq n_0.\, f(n) \leq c g(n).
\]

For each pair of functions $f$ and $g$ below, indicate the
\textbf{smallest} $c \in \integers^+$, and for that smallest $c$, the
\textbf{smallest corresponding} $n_0 \in \integers^+$, that would
establish $f = O(g)$ by the definition given above.  If there is no
such $c$, write $\infty$.

\begin{problemparts}

\problempart $f(n) = \frac{1}{2} \ln{n^2}, g(n) = n$.  \hfill $c =$
\brule{.5in}, $n_0$ = \brule{.5in}

\begin{solution}
$f(n)=\ln{n}$, and $n$ exceeds $\ln{n}$ for all positive $n$. 
Thus $c = 1$ and $n_0=1$.
\end{solution}

\problempart $f(n) = n, g(n) = n\ln{n}$.  \hfill $c =$ \brule{.5in}, $n_0$ = \brule{.5in}

\begin{solution}
Since $\ln{n}$ eventually grows beyond 1, it must be that $n\ln{n}$
eventually grows beyond $n$.  Thus $c=1$.  Now $f(n)=n\leq
cg(n)=g(n)=n\ln{n}$ precisely when $1 \leq \ln{n}$.  That is, when $n
\geq e$.  So $n_0 = \ceil{e} = 3$.
\end{solution}

\iffalse

\problempart $f(n) = 3n^2, g(n) = 2n^2-2n+\frac{3}{2}$. \hfill  $c =$
\brule{.5in}, $n_0$ = \brule{.5in}  

\begin{solution}
For $cg(n)$ to eventually rival $f(n)$, the $x^2$ term's coefficient
in $cg(n)$ must be at least as large as the corresponding coefficient
in $f(n)$.  So $c$ must be at least $\frac{3}{2}$.  The smallest
acceptable nonnegative integer value of $c$ is therefore 2.  Now,
solve $cg(n)=f(n)$:

\begin{align*}
4n^2-4n+3 &= 3n^2\\
n^2-4n+3&=0\\
(n-3)(n-1)&=0
\end{align*}

So $n=1$ or $n=3$.  Thus $cg(2)< f(2)$, but $cg(n)\geq f(n)$ for all
$n\geq 3$.

Conclude that $c = 2$ and $n_0=3$.
\end{solution}
\fi

\problempart $f(n) = 2^n, g(n) = n^4\ln{n}$ \hfill $c =$ \brule{.5in},
$n_0$ = \brule{0.5in}

\begin{solution}
$n^4\ln{n}=o\paren{n^5}$ since $\ln{n}=o(n)$.  Also, any polynomial
is asymptotically smaller than any exponential whose base has magnitude
greater than $1$. So $n^5=o(2^n)$ and hence $n^4\ln{n}=o(2^n)$.  Therefore
$f\neq O(g)$, so there do not exist finite $c, n_0 \in\integers^+$ that
satisfy the required condition.  Thus, here we write $c = \infty$.
\end{solution}

\problempart $\displaystyle f(n) = 3\sin{\paren{\frac{\pi
      (n-1)}{100}}}+2, g(n) = 0.2$. \hfill $c =$ \brule{.5in}, $n_0$ =
\brule{.5in}

\begin{solution}
$f(n)$ is periodic.  Its minimum value is $-1$ and its maximum is $5$,
  so the smallest acceptable positive integer value for $c$ is
  $\frac{5}{0.2}=25$.  Now, $cg(n)$ exceeds or equals $f(n)$ for all
  positive $n$, so $n_0=1$.
\end{solution}

\end{problemparts}

\end{problem}

%%%%%%%%%%%%%%%%%%%%%%%%%%%%%%%%%%%%%%%%%%%%%%%%%%%%%%%%%%%%%%%%%%%%%
% Problem ends here
%%%%%%%%%%%%%%%%%%%%%%%%%%%%%%%%%%%%%%%%%%%%%%%%%%%%%%%%%%%%%%%%%%%%%
\endinput
