\documentclass[problem]{mcs}

\begin{pcomments}
  \pcomment{CH, 4/12/14}
\end{pcomments}

\pkeywords{
  induction
  ordinary induction
  combinatorial identities
}

%%%%%%%%%%%%%%%%%%%%%%%%%%%%%%%%%%%%%%%%%%%%%%%%%%%%%%%%%%%%%%%%%%%%%
% Problem starts here
%%%%%%%%%%%%%%%%%%%%%%%%%%%%%%%%%%%%%%%%%%%%%%%%%%%%%%%%%%%%%%%%%%%%%

\begin{problem}
Use induction to prove that there are $2^n$ subsets of an $n$-element
set (Theorem~\bref{powset_fincard}). 

\begin{solution}
Let $n$ be a positive integer. We define the Induction Hypothesis, $P(n)$, be the statement that every
$n$-element set $A$ has $2^n$ subsets.

\textbf{Base case:} First, we must show that $P(1)$ is true.  This is
immediate: any singleton set $A = \{a\}$ has exactly two
subsets --- $\emptyset$ and $\{a\}$. 

\textbf{Induction step:} Assume that $P(m)$ is true for some positive
integer $m$. Consider a set $A$ with 
$m+1$ elements, and choose an element $a \in A$.  We can write $A$ as $B \union \{a\}$, where $B$
is some $m$-element set. By the Induction Hypothesis, $B$ has $2^m$
subsets. 

Now, let us count the number of subsets of $A$ as follows. Consider
any set $C \subset A$. There are two disjoint cases: (i) $a \not\in C$, in which case $C \subset
B$ and there are $2^m$ possible choices for $C$; (ii)  $a \in C$, in
which case $C = C' \union \{a\}$ where $C' \subset B$ and again,
there are $2^m$ possible choices for $C'$. Therefore, the total number
of choices for $C$ is given by $2^m + 2^m = 2^{m+1}$. This proves $P(m+1)$, completing
the induction step. 

\end{solution}

\end{problem}

%%%%%%%%%%%%%%%%%%%%%%%%%%%%%%%%%%%%%%%%%%%%%%%%%%%%%%%%%%%%%%%%%%%%%
% Problem ends here
%%%%%%%%%%%%%%%%%%%%%%%%%%%%%%%%%%%%%%%%%%%%%%%%%%%%%%%%%%%%%%%%%%%%%

\endinput
