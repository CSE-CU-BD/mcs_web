\documentclass[problem]{mcs}

\begin{pcomments}
  \pcomment{CH, 4/12/14}
\end{pcomments}

\pkeywords{
  induction
  ordinary induction
  combinatorial identities
}

%%%%%%%%%%%%%%%%%%%%%%%%%%%%%%%%%%%%%%%%%%%%%%%%%%%%%%%%%%%%%%%%%%%%%
% Problem starts here
%%%%%%%%%%%%%%%%%%%%%%%%%%%%%%%%%%%%%%%%%%%%%%%%%%%%%%%%%%%%%%%%%%%%%

\begin{problem}
Use induction to prove that there are $2^n$ subsets of an $n$-element set
(Theorem~\bref{powset_fincard}).

\begin{staffnotes}
Another way to formulate this result could have been that
\begin{equation}\label{pA2cA}
\card{\power{A}} = 2^{\card{A}}
\end{equation}
for all finite sets $A$.  Ask students if they think it would then have been fair to phrase the
question as ``Prove~\eqref{pA2cA} by induction.''

Tell them that we think so, and we expect them to be able to identify values for induction in simple
cases: induction on set-size, number of edges, number of vertices, number of factors, absolute value
of integer quantities, sum of integer coordinates, \dots.
\end{staffnotes}

\begin{solution}

\begin{proof}
Let $n$ be a positive integer.  We define the Induction Hypothesis, $P(n)$, be the statement that
every $n$-element set has $2^n$ subsets.

\inductioncase{Base case}: ($n=0$).  The empty set has exactly $2^0 = 1$ subset, namely, itself.

\textbf{Induction step:} Assume that $P(m)$ is true for some $m \in \naturals$, and suppose $A$ is a
set with $m+1$ elements.  Choose an element $a \in A$ and let $B \eqdef A - \set{a}$.  Now $B$ is an
$m$-element set which has $2^m$ subsets by Induction Hypothesis.  That is,
\begin{equation}\label{pB2n}
\power(B) = 2^n
\end{equation}

Define a function, $f$, from the subsets of $A$ to subsets of $B$ by the rule:
\[
f(C) \eqdef C - \set{a}.
\]
This function is 2-to-1 because $f(X)
= Y$ iff $X = Y$ or $X = Y \union \set{a}$.  So by the Division Rule,
\[
\card{\power(A)} = 2 \cdot \card{\power(B)} = 2 \cdot 2^n = 2^{n+1}.
\]
This proves $P(m+1)$, completing the induction step.

Alternative proof:
We can partition the subsets of $A$ according to whether or not they contain $a$.  That is,
\begin{align*}
\card{\power(A)} =
      & \card{\set{C                \suchthat C \subseteq B} \union 
        \set{C \union \set{a} \suchthat C \subseteq B}} \\
    = & \card{\power(B)} +
        \card{\set{C \union \set{a} \suchthat C \in \power(B)}}\\
    = & 2^n + 2^n = 2^{n+1}.
\end{align*}


\iffalse
Now, let us count the number of subsets of $A$ as follows.  Consider
any set $C \subset A$.  There are two disjoint cases: (i) $a \not\in
C$, in which case $C \subset B$ and there are $2^m$ possible choices
for $C$; (ii) $a \in C$, in which case $C = C' \union \set{a}$ where
$C' \subset B$ and again, there are $2^m$ possible choices for $C'$.
Therefore, the total number of choices for $C$ is given by $2^m + 2^m
= 2^{m+1}$.
\fi

\end{proof}

\end{solution}

\end{problem}

%%%%%%%%%%%%%%%%%%%%%%%%%%%%%%%%%%%%%%%%%%%%%%%%%%%%%%%%%%%%%%%%%%%%%
% Problem ends here
%%%%%%%%%%%%%%%%%%%%%%%%%%%%%%%%%%%%%%%%%%%%%%%%%%%%%%%%%%%%%%%%%%%%%

\endinput
