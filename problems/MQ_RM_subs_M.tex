\documentclass[problem]{mcs}

\begin{pcomments}
  \pcomment{MQ_RM_subs_M}
  \pcomment{adapted from PS_M_equal_RM 2/28/11 by Joliat & Kazerani}
  \pcomment{edited by ARM 3/1/11}
\end{pcomments}

\pkeywords{
  string
  matched
  bracket
  structural_induction
  induction
  concatenation
  ambiguous
}

%%%%%%%%%%%%%%%%%%%%%%%%%%%%%%%%%%%%%%%%%%%%%%%%%%%%%%%%%%%%%%%%%%%%%
% Problem starts here
%%%%%%%%%%%%%%%%%%%%%%%%%%%%%%%%%%%%%%%%%%%%%%%%%%%%%%%%%%%%%%%%%%%%%

\begin{problem}

The set, $M$, of strings of brackets is recursively defined as follows:

\textbf{Base case:} $\emptystring \in M$,

\textbf{Constructor cases:} if $s,t \in M$, then
\begin{itemize}

\item the string $\lefbrk s\, \rhtbrk$ is in $M$,

\item the string $s\cdot t$ is in $M$.

\end{itemize}

The set, $\RM$, of strings of matched brackets was defined recursively
in class.  We repeat the definition:

\textbf{Base case:} $\emptystring \in\RM$.

\textbf{Constructor case:} If $s,t \in\RM$, then
$\lefbrk s\, \rhtbrk t \in \RM$.


Fill in the following parts of a proof by structural induction that
\begin{equation}\label{rmsseqm}
\RM \subseteq M.
\end{equation}

\bparts

\ppart State an induction hypothesis suitable for
proving~\eqref{rmsseqm} by structural induction.

\begin{solution}

\[
P(x) \eqdef x \in M
\]

\end{solution}

\ppart State and prove the base case(s).

\begin{solution}

\textbf{Base case} ($x = \emptystring$): By definition of $M$, the empty string is in $M$.  

\end{solution}

\ppart Prove the inductive step.

\begin{solution}
\begin{proof}

\textbf{Constructor case} ($x = \lefbrk s\, \rhtbrk t$ for $s,t \in
\RM$): By structural induction hypothesis, we may assume that $s, t
\in M$.  By the first constructor case of $M$, it follows that $[s]
\in M$.  Then, by the second constructor case of $M$, it follows that
$\lefbrk s\, \rhtbrk t \in M$.

\end{proof}

\end{solution}

\eparts

As a matter of fact, $M = \RM$, though we won't prove this.  An
advantage of the $\RM$ definition is that it is
\emph{\idx{unambiguous}}, while the definition of $M$ is ambiguous.

\bparts

\ppart Give an example demonstrating that $M$ is ambiguous.

\begin{solution}

Consider derivations of the empty
string.  This could be derived directly from the base case
$\lambda$, or by starting with $\lambda$ and then constructing
$\lambda\lambda$ through the second constructor case.

\end{solution}

\ppart Briefly explain what advantage unambiguous recursive
definitions have over ambiguous ones.  (Remember ``ambiguous
definition'' has a technical mathematical meaning which does not imply
that the ambiguous definition is unclear.)

\begin{solution}

If a definition is unambiguous, functions defined recursively on it can always be well-defined. 

\end{solution}

\eparts

\end{problem}

%%%%%%%%%%%%%%%%%%%%%%%%%%%%%%%%%%%%%%%%%%%%%%%%%%%%%%%%%%%%%%%%%%%%%
% Problem ends here
%%%%%%%%%%%%%%%%%%%%%%%%%%%%%%%%%%%%%%%%%%%%%%%%%%%%%%%%%%%%%%%%%%%%%

\endinput
