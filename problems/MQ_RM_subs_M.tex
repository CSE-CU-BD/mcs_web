\documentclass[problem]{mcs}

\begin{pcomments}
  \pcomment{MQ_RM_subs_M}
  \pcomment{subsumed by PS_RM_equal_AM}
  \pcomment{overlaps with MQ_ambiguous_recursive_def}
  \pcomment{MQ.3/2/11}
  \pcomment{adapted from PS_M_equal_RM 2/28/11 by Joliat & Kazerani}
  \pcomment{edited by ARM 3/1/11, 3/10/8}
\end{pcomments}

\pkeywords{
  string
  matched
  bracket
  structural_induction
  induction
  concatenation
  ambiguous
}

%%%%%%%%%%%%%%%%%%%%%%%%%%%%%%%%%%%%%%%%%%%%%%%%%%%%%%%%%%%%%%%%%%%%%
% Problem starts here
%%%%%%%%%%%%%%%%%%%%%%%%%%%%%%%%%%%%%%%%%%%%%%%%%%%%%%%%%%%%%%%%%%%%%

\begin{problem}
\inbook{The set $\RM$ of strings of matched brackets is defined
  recursively in Definition~\bref{RM-def}, and an alternative
  definition as the set $\AM$ is given in~\bref{AM_def}.}

\inhandout{
The set $\RM$ of strings of matched brackets is defined
  recursively by

\inductioncase{Base case}: The empty string $\emptystring \in\RM$.

\inductioncase{Constructor case}: If $s,t \in\RM$, then
\[
\lefbrk s\, \rhtbrk t \in \RM.
\]

An alternative definition is the set $\AM$ defined recursively by

\inductioncase{Base case:} $\emptystring \in \AM$.

\inductioncase{Constructor cases:} If $s,t \in \AM$, then
\begin{itemize}

\item $\lefbrk s\, \rhtbrk \in \AM$, and

\item $s\cdot t \in \AM$.

\end{itemize}
}

Fill in the following parts of a proof by structural induction that
\begin{equation}\tag{*}%\label{rmsseqm}
\RM \subseteq \AM.
\end{equation}
(As a matter of fact, $\AM = \RM$, though we won't prove this
here\inbook{---see Problem~\bref{PS_RM_equal_AM}}.)

\bparts

\ppart State an \textbf{induction hypothesis} suitable for
proving~(*) by structural induction.   % \eqref{rmsseqm}

\begin{solution}
\[
P(x) \eqdef x \in M
\]

Note that
\[
P'(x) \eqdef x \in \RM \QIMPLIES x \in M
\]
would work, but reflects some confusion about structural induction.
In a proof by structural induction, the induction hypothesis $P(x)$ is
only invoked on $x \in \RM$, so the assumption ``$x \in \RM$'' in $P'$ is
redundant.
\end{solution}

\examspace[1in]

\ppart State and prove the \textbf{base case}.

\begin{solution}

\inductioncase{Base case} ($x = \emptystring$): By definition of $\AM$, the
empty string is in $\AM$, so $P(x)$ is true.

\end{solution}

\examspace[2in]

\ppart Prove the \textbf{inductive step}.

\begin{solution}
\begin{proof}

\inductioncase{Constructor case} ($x = \lefbrk s\, \rhtbrk t$ for $s,t \in
\RM$): By structural induction hypothesis, we may assume that $s, t
\in \AM$.  By the first constructor case of $\AM$, it follows that
$\lefbrk s\, \rhtbrk \in \AM$.  Then, by the second constructor case of
$\AM$, it follows that $x = (\lefbrk s\, \rhtbrk)\cdot t \in \AM$, proving
that $P(x)$ is true, as required.

\end{proof}

\end{solution}

\examspace[3in]

\eparts

An advantage of the $\RM$ definition is that it is
\emph{\idx{unambiguous}}, while the definition of $\AM$ is ambiguous.

\bparts

\ppart Give an example demonstrating that a string in $\AM$ is
ambiguously defined.

\examspace[0.7in]

\begin{solution}
The simplest example is the empty string.  The fact that $\emptystring
\in \AM$ could be derived directly from the base case $\emptystring$,
or by starting with $\emptystring$ and then deriving the empty string
again as $\emptystring \cdot \emptystring$ using the second
constructor case.
\end{solution}

\ppart Briefly explain what advantage unambiguous recursive
definitions have over ambiguous ones.  (Remember that ``ambiguous
recursive definition'' has a technical mathematical meaning which does
not imply that the ambiguous definition is unclear.)

\begin{solution}
Ambiguity means that there is more than one way to construct some
element of the recursively defined set.  So if a recursive definition
is ambiguous, functions defined recursively on the set may not be
well-defined.
\end{solution}

%\examspace[2in]

\eparts

\end{problem}

%%%%%%%%%%%%%%%%%%%%%%%%%%%%%%%%%%%%%%%%%%%%%%%%%%%%%%%%%%%%%%%%%%%%%
% Problem ends here
%%%%%%%%%%%%%%%%%%%%%%%%%%%%%%%%%%%%%%%%%%%%%%%%%%%%%%%%%%%%%%%%%%%%%

\endinput











