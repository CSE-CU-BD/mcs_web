\documentclass[problem]{mcs}

\begin{pcomments}
  \pcomment{MQ_Euler_function_of_100}
\end{pcomments}

\pkeywords{
  modular_arithmetic
  gcd
  phi
  Euler
}

%%%%%%%%%%%%%%%%%%%%%%%%%%%%%%%%%%%%%%%%%%%%%%%%%%%%%%%%%%%%%%%%%%%%%
% Problem starts here
%%%%%%%%%%%%%%%%%%%%%%%%%%%%%%%%%%%%%%%%%%%%%%%%%%%%%%%%%%%%%%%%%%%%%

\begin{problem}
% Tests understanding of gcd and modular arithmetic

\begin{problemparts}

\problempart\label{valphi100}
Calculate the value of $\phi(100)$.
\examspace[2in]

\begin{solution}
\[
\phi(100) = \phi(25) \phi(4) = \phi(5^2) \phi(2^2) =  (5^2 - 5)(2^2 - 2) = 40.
\]
\end{solution}


\ppart\label{k121k100} Using your solution to part~\eqref{valphi100}, prove that
\[
k^{121} \equiv k \pmod{100}
\]
when $\gcd(k,100) =1$.
\examspace[2in]
%\hint Use your solution to part (\ref{phi}).

\begin{solution}
\begin{align*}
k^{121} & = k \cdot (k^{40})^3\\
       & = k \cdot (k^{\phi(100)})^3 & \text{(by part~\eqref{valphi100})}\\
       & \equiv k \cdot (1)^3 \pmod{100} & \text{(when $\gcd(k,100) =1$ by Euler's Theorem)}\\
       & = k.
\end{align*}
\end{solution}

\problempart Conclude that the last two digits of the decimal
representations of $k$ and $k^{121}$ are the same for any integer $k >
9$ that is relatively prime to 100.

\begin{solution}
The last two digits of the decimal representations of any pair of positive integers
are the same iff the integers are congruent modulo 100.  Therefore, this
part follows immediately from part~\eqref{k121k100}.
\end{solution}

\end{problemparts}
\end{problem}

\endinput
