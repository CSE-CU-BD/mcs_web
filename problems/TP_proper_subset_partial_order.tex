\documentclass[problem]{mcs}

\begin{pcomments}
  \pcomment{TP_proper_subset_partial_order}
  \pcomment{from: S06.ps2}
\end{pcomments}

\pkeywords{
  partial_order
  proper_subset
  power_set
}

%%%%%%%%%%%%%%%%%%%%%%%%%%%%%%%%%%%%%%%%%%%%%%%%%%%%%%%%%%%%%%%%%%%%%
% Problem starts here
%%%%%%%%%%%%%%%%%%%%%%%%%%%%%%%%%%%%%%%%%%%%%%%%%%%%%%%%%%%%%%%%%%%%%

\begin{problem}
For the proper subset partial order $\subset$ on the power set
$\power(\set{1,2,\dots 5})$:

\bparts
\ppart What is the size of a maximal chain? Describe one.

\begin{solution}
Size 6, for example,
\[
\set{\emptyset, \set{1}, \set{1,2}, \set{1,2,3},\set{1,2,3,4},\set{1,2,3,4,5}}.
\]
\end{solution}

\ppart Describe the unique antichain of maximum size. You don't need to prove that it is the largest.

\begin{solution}
All the size 3 subsets of $\set{1,2,\dots 5}$ form an antichain of size 10.
Proving this is the largest antichain is more challenging.
\end{solution}

\ppart\label{maxminp5}  What are the maximal and minimal elements?  Are they maximum and
minimum?

\begin{solution}
$\emptyset$ is minimum and $\set{1,2,\dots 5}$ is maximum.
\end{solution}

\ppart If we restrict the partial order to the sets that are nonempty,
what are the maximal and minimal elements?  Are they maximum and
minimum?

\begin{solution}
Now the five size-$1$ subsets are minimal and there is no minimum.
$\set{1,2,\dots 5}$ is still maximum.
\end{solution}

\eparts
\end{problem}

%%%%%%%%%%%%%%%%%%%%%%%%%%%%%%%%%%%%%%%%%%%%%%%%%%%%%%%%%%%%%%%%%%%%%
% Problem ends here
%%%%%%%%%%%%%%%%%%%%%%%%%%%%%%%%%%%%%%%%%%%%%%%%%%%%%%%%%%%%%%%%%%%%%

\endinput
