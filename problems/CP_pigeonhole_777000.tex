\documentclass[problem]{mcs}

\begin{pcomments}
  \pcomment{CP_pigeonhole_777000}
  \pcomment{ARM 4/12/12}
  \pcomment{idea taken from Puzzled column, Peter Winkler, CACM Sep
    2011, p.110 DOI.1145.1995376.1995402}
\end{pcomments}

\pkeywords{
  divisibility
  congruence
  pigeon_hole
  base_10
  digits
}

%%%%%%%%%%%%%%%%%%%%%%%%%%%%%%%%%%%%%%%%%%%%%%%%%%%%%%%%%%%%%%%%%%%%%
% Problem starts here
%%%%%%%%%%%%%%%%%%%%%%%%%%%%%%%%%%%%%%%%%%%%%%%%%%%%%%%%%%%%%%%%%%%%%

\begin{problem}
\bparts

\ppart\label{ppart:777000} Prove that every positive integer divides a
number such as 70, 700, 7770, 77000, whose decimal representation
consists of one or more 7's followed by one or more 0's.

\hint $7, 77, 777, 7777, \dots$

\begin{solution}
For any $m>0$, two of the numbers whose representations are
\[
7, 77, 777,\dots, \underbrace{77\dots 7}_{m+1}
\]
must have the same remainder on division by $m$.  Subtracting the
smaller from the larger of these numbers yields a number of the
desired form that is divisible by $m$.
\end{solution}

\ppart Conclude that if a positive number is not divisible by 2 or 5,
then it divides a number whose decimal representation is all 7's.

\begin{solution}
By part~\eqref{ppart:777000}, any $m>0$ divides a number $n$ whose
decimal representation consists of 7's followed by 0's.  So $n$
factors into a power of 10 and a number whose decimal representation
is all 7's.  If $m$ is not divisible by 2 or 5, it must divide the
all 7's factor.
\end{solution}

\eparts

\end{problem}

\endinput
