\documentclass[problem]{mcs}

\begin{pcomments}
  \pcomment{CP_3_and_7_cent_stamps_by_induction}
  \pcomment{from: F09.mq3}
  \pcomment{analysis of postage below 12\mcents\ dropped by ARM, 2/23/12}
\end{pcomments}

\pkeywords{
  induction
  strong_induction
  postage
}

%%%%%%%%%%%%%%%%%%%%%%%%%%%%%%%%%%%%%%%%%%%%%%%%%%%%%%%%%%%%%%%%%%%%%
% Problem starts here
%%%%%%%%%%%%%%%%%%%%%%%%%%%%%%%%%%%%%%%%%%%%%%%%%%%%%%%%%%%%%%%%%%%%%

\begin{problem}
  Any amount of 12 or more cents postage can be made using only
  3\mcents\ and 7\mcents\ stamps.  Prove this \emph{by induction}
  using the induction hypothesis
  \[
  S(n) \eqdef n+12 \text{ cents postage can be made using
  only 3\mcents\ and 7\mcents\ stamps}.
  \]

\begin{staffnotes}
Make sure students follow the basic induction format.  Emphasize the
importance of stating the induction hypothesis explicitly.

The proof we give below uses strong induction.  If need be, tell
students that strong induction counts as being a kind of ``proof by
induction.''

Maybe mention that essentially same proof works for ordinary induction using hypothesis
\[
  S(n) \eqdef \forall m \leq n. m+12 \text{ cents postage can be made using
  only 3\mcents\ and 7\mcents\ stamps}.
\]
\end{staffnotes}

\begin{solution}

  \begin{proof}
    \inductioncase{Base case} ($n=0$):  12\mcents\ postage can be made with four 3\mcents\ stamps.

    \inductioncase{Inductive step}:  We assume the strong hypothesis that
    $S(k)$ for $n \ge k \ge 0$.  Now we must prove $S(n+1)$.  The
    proof is by cases:

\textbf{case} $n=0$: $S(0+1)$ holds because 12+1=13 cents postage can
be made using one 7\mcents\ and two 3\mcents\ stamps.

\textbf{case} $n=1$: $S(1+1)$ holds because $(1+1)+12=
14$\mcents\ postage can be made using two 7\mcents\ stamps.

\textbf{case} $n\ge 2$: Since $n > n-2 \ge 0$, we know by strong
induction that $S(n-2)$ holds.  But including an extra 3\mcents\ stamp
in the collection of 3\mcents\ and 7\mcents\ stamps that made
$(n-2)+12$ cents gives a collection that makes $(n-2)+12+3 = (n+1)+12$
cents, which proves $S(n+1)$.

Since $S(n+1)$ holds in any case, the inductive step has been proved.

It follows by strong induction $S(n)$ holds for all $n \in \nngint$.
That is, every amount of postage of $12$ cents or more can be made
with 3\mcents\ and 7\mcents\ stamps.
\end {proof}

\end{solution}

\end{problem}

%%%%%%%%%%%%%%%%%%%%%%%%%%%%%%%%%%%%%%%%%%%%%%%%%%%%%%%%%%%%%%%%%%%%%
% Problem ends here
%%%%%%%%%%%%%%%%%%%%%%%%%%%%%%%%%%%%%%%%%%%%%%%%%%%%%%%%%%%%%%%%%%%%%

\endinput
