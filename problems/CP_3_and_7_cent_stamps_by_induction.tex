\documentclass[problem]{mcs}

\begin{pcomments}
  \pcomment{CP_3_and_7_cent_stamps_by_induction}
  \pcomment{from: F09.mq3}
  \pcomment{similar to CP_10_and_15_cent_stamps_by_WOP but uses induction}
  \pcomment{analysis of postage below 12\textcent\ dropped by ARM, 2/23/12}
\end{pcomments}

\pkeywords{
  induction
  strong_induction
  postage
}

%%%%%%%%%%%%%%%%%%%%%%%%%%%%%%%%%%%%%%%%%%%%%%%%%%%%%%%%%%%%%%%%%%%%%
% Problem starts here
%%%%%%%%%%%%%%%%%%%%%%%%%%%%%%%%%%%%%%%%%%%%%%%%%%%%%%%%%%%%%%%%%%%%%

\begin{problem}
  Any amount of 12 or more cents postage can be made that using only
  3\textcent\ and 7\textcent\ stamps.  Prove this \emph{by induction} using
  the induction hypothesis
  \[
  S(n) \eqdef n+12\textcent\ \text{postage can be made using
  only 3\textcent\ and 7\textcent\ stamps}.
  \]

\begin{solution}

  \begin{proof}
    \inductioncase{Base case} ($n=0$):  12\textcent\ postage can be made with four 3\textcent\ stamps.

    \inductioncase{Inductive step}:  We assume the strong hypothesis that
    $S(k)$ for $n \ge k \ge 0$.  Now we must prove $S(n+1)$.  The
    proof is by cases:

\textbf{case} $n=0$: $S(0+1)$ holds because 12+1=13 cents postage can
be made using one 7\textcent\ and two 3\textcent\ stamps.

\textbf{case} $n=1$: $S(1+1)$ holds because $(1+1)+12=
14$\textcent\ postage can be made using two 7\textcent\ stamps.

\textbf{case} $n\ge 2$: Since $n > n-2 \ge 0$, we know by strong
induction that $S(n-2)$ holds.  But including an extra 3 cents stamp
in the collection of 3 and 7 cent stamps that paid $(n-2)+12$ cents
gives a collection that pays $(n-2)+12+3 = (n+1)+12$ cents, which
proves $S(n+1)$.

Since $S(n+1)$ holds in any case, the inductive step has been proved.

It follows by strong induction $S(n)$ holds for all $n \in \naturals$,
that is, that every amount of cents postage of $12$ or more cents can
be made with 3 and 7 cent stamps.

\end {proof}

\end{solution}

\end{problem}

%%%%%%%%%%%%%%%%%%%%%%%%%%%%%%%%%%%%%%%%%%%%%%%%%%%%%%%%%%%%%%%%%%%%%
% Problem ends here
%%%%%%%%%%%%%%%%%%%%%%%%%%%%%%%%%%%%%%%%%%%%%%%%%%%%%%%%%%%%%%%%%%%%%

\endinput
