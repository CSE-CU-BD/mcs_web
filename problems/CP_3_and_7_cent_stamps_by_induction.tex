\documentclass[problem]{mcs}

\begin{pcomments}
  \pcomment{CP_3_and_7_cent_stamps_by_induction}
  \pcomment{from: F09.mq3}
  \pcomment{similar to CP_10_and_15_cent_stamps_by_WOP but uses induction}
\end{pcomments}

\pkeywords{
  induction
  strong_induction
  postage
}

%%%%%%%%%%%%%%%%%%%%%%%%%%%%%%%%%%%%%%%%%%%%%%%%%%%%%%%%%%%%%%%%%%%%%
% Problem starts here
%%%%%%%%%%%%%%%%%%%%%%%%%%%%%%%%%%%%%%%%%%%%%%%%%%%%%%%%%%%%%%%%%%%%%

\begin{problem}

  Find all possible amounts of postage that can be paid exactly using
  3 and 7 cent stamps.  Use induction to prove that your answer is
  correct.

\begin{solution}

  \begin{proof}
    We can begin by observing that the following postage amounts can
    be made by 3 and 7 cent stamps:

    \begin{align*}
    0 & \text{no stamps}\\
    3 & = 3 \\
    6 & = 3 + 3 \\
    7 & = 7 \\
    9 & = 3 + 3 + 3 \\
    10 & = 3 + 7,
    \end{align*}
and these are the only amounts $< 12$ cents that can be paid.  Now we
prove that every amount $\ge 12$ can also be paid.  The proof is by
strong induction on $n$ with induction hypothesis
\[
S(n) \eqdef \text{exactly $n+12$ cents postage can be paid with 3 and
  7 cent stamps}.
\]

    \textbf{Base case:} $S(0)$.  12 cents can be paid using four 3 cent stamps.

    \textbf{Inductive step:} We assume the strong hypothesis that $S(k)$ for $n \ge k \ge 0$.  Now we mmust prove $S(n+1)$.  The proof is by cases:

\textbf{case} $n=0$: $S(0+1)$ holds because 13 cents postage can be
paid using two 3 cents and a 7 cents stamps.

\textbf{case} $n=1$: $S(1+1)$ holds because 14 cents postage can be
paid using two 7 cent stamps.

\textbf{case} $n\ge 2$: Since $n \ge n-2 \ge 0$, we know by strong
induction that $S(n-2)$ holds.  But including an extra 3 cents stamp
in the collection of 3 and 7 cent stamps that paid $(n-2)+12$ cents
gives a collection that pays $(n-2)+12+3 = (n+1)+12$ cents, which
proves $S(n+1)$.

Since $S(n+1)$ holds in any case, the inductive step has been proved.

It follows by strong induction that every amount of cents postage $\ge
12$ can be mde with 3 and 7 cent stamps.

\end {proof}

\end{solution}

\end{problem}

%%%%%%%%%%%%%%%%%%%%%%%%%%%%%%%%%%%%%%%%%%%%%%%%%%%%%%%%%%%%%%%%%%%%%
% Problem ends here
%%%%%%%%%%%%%%%%%%%%%%%%%%%%%%%%%%%%%%%%%%%%%%%%%%%%%%%%%%%%%%%%%%%%%

\endinput
