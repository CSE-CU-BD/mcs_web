\documentclass[problem]{mcs}

\begin{pcomments}
  \pcomment{PS_reverse_linear_recurrences}
  \pcomment{ARM 4/10/12, 4/15/12}
  \pcomment{idea taken from Puzzled column, Peter Winkler, CACM Sep
    2011, p.110 DOI.1145.1995376.1995402}
  \pcomment{Pigeonhole commented out and problem moved to number_theory 3/5/13--ARN}
\end{pcomments}

\pkeywords{
  linear_recurrence
  divisibility
  congruence
  pigeon_hole
}

%%%%%%%%%%%%%%%%%%%%%%%%%%%%%%%%%%%%%%%%%%%%%%%%%%%%%%%%%%%%%%%%%%%%%
% Problem starts here
%%%%%%%%%%%%%%%%%%%%%%%%%%%%%%%%%%%%%%%%%%%%%%%%%%%%%%%%%%%%%%%%%%%%%

\begin{problem}
This problem will use \iffalse the Pigeonhole Principle and\fi
elementary properties of congruences to prove that every positive
integer divides infinitely many Fibonacci numbers.

A function $f:\nngint \to \nngint$ that satisifies
\begin{equation}\label{fnc1c2cd}
f(n) = c_1 f(n-1) + c_2 f(n-2) +  \cdots + c_{d} f(n-d)
\end{equation}
for some $c_i \in \nngint$ and all $n \geq d$ is called \emph{degree
  $d$ linear-recursive}.

A function $f:\nngint \to \nngint$ \emph{has a degree $d$ repeat
  modulo $m$ at $n$ and $k$} when it satisfies the following
\emph{repeat congruences}:
\[\begin{array}{rcll}
       f(n) & \equiv &f(k) &\pmod m,\\
     f(n-1) & \equiv &f(k-1) & \pmod m,\\
        & \vdots\\
 f(n-(d-1)) & \equiv & f(k-(d-1)) & \pmod m.

\end{array}\]
for $k > n \geq d-1$.

For the rest of this problem, assume linear-recursive functions and
repeats are degree $d>0$.

\bparts

\ppart\label{ppart:repeatup} Prove that if a linear-recursive function
has a repeat modulo $m$ at $n$ and $k$, then it has one at $n+1$ and
$k+1$.

\begin{solution}
First substitute $n+1$ for $n$ and then $k+1$ for $n$ in
equation~\eqref{fnc1c2cd}.  These two equations and the repeat
congruences above imply that $f(n+1) \equiv f(k+1) \pmod m$.  This,
together with the first $d-1$ repeat congruences, means that $f$ has
a repeat at $n+1$ and $k+1$.
\end{solution}

\ppart\label{ppart:mustrepeat} Prove that for all $m > 1$, every
linear-recursive function repeats modulo $m$ at $n$ and $k$ for some
$n,k \in \Zintvco{d-1}{d+m^d}$.

\begin{solution}
There are only $m^d$ possible sequences of integers in the remainder
interval $(0,m]$, so \iffalse by the Pigeonhole Principle, \fi
some sequence must
  occur more than once among the $m^d +1$ length $d$ sequences of the
  form
\[
(\rem{f(j)}{m}, \rem{f(j-1)}{m}, \dots, \rem{f(j-(d-1))}{m})
\]
for $j \in \Zintvco{d-1}{d+m^d}$.
\end{solution}

\ppart\label{ppart:reverse-repeat} A linear-recursive function is
\emph{reverse-linear} if its $d$th coefficient $c_d = \pm 1$.  Prove
that if a reverse-linear-recursive function repeats modulo $m$ at $n$
and $k$ for some $n \geq d$, then it repeats modulo $m$ at $n-1$ and
$k-1$.

\begin{solution}
  If $c_d \neq 0$, we can solve~\eqref{fnc1c2cd} for $f(n-d)$ and get
\[
f(n-d) = \frac{1}{c_d} \paren{f(n) - \sum_{i=1}^{d-1} c_i f(n-i)}
\]
for all $n \geq d$.  Given $c_d = \pm 1$, we have $1/c_d = c_d$, so that
\begin{equation}\label{fn-dmodm}
f(n-d) = c_d \paren{f(n) - \sum_{i=1}^{d-1} c_i f(n-i)}
\end{equation}
for all $n \geq d$.  Now if the repeat congruences hold,
then~\eqref{fn-dmodm} implies that
\[
f(n-d) \equiv f(k-d) \pmod m,
\]
and hence
\begin{equation}\label{fn-1d-1m}
\rem{f((n-1)-(d-1))}{m} = \rem{f((k-1)-(d-1))}{m}.
\end{equation}
Equation~\eqref{fn-1d-1m}, together with the last $d-1$ repeat
congruences, means that $f$ repeats modulo $m$ at $n-1$ and $k-1$.
\end{solution}

\ppart\label{ppart:repeatatd-1} Conclude that every
reverse-linear-recursive function must repeat modulo $m$ at $d-1$ and
$(d-1)+j$ for some $j>0$.

\begin{solution}
By part~\eqref{ppart:mustrepeat}, the function repeats at some $n$ and
$n+j$ for some $j>0$, and by part~\eqref{ppart:reverse-repeat}, the
least such $n$ must be $d-1$.  In fact, part~\eqref{ppart:mustrepeat}
implies that there will be such a $j$ less than $d+ m^d$.
\end{solution}

\ppart\label{ppart:alldivisors} Conclude that if $f$ is an
reverse-linear-recursive function and $f(k) = 0$ for some $k \in
\Zintvco{0}{d}$, then every positive integer is a divisor of $f(n)$
for infinitely many $n$.

\begin{solution}
By part~\eqref{ppart:repeatatd-1}, $f$ must repeat modulo $m$ at $d-1$
and $(d-1)+j$ for some $j>0$.  Now, from the definition of linear-recursive we get that
\[
f(n) \equiv f(n+j) \pmod m for n \in [0,d)
\]
Then, part~\eqref{ppart:repeatup} implies
that
\[
f(n) \equiv f(n+j) \pmod m
\]
for all $n \in \nngint$.  In particular,
\[
f(k), f(k+j), f(k +2j), \dots
\]
are all $\equiv 0 \pmod m$, that is, the number $m$ is a divisor
of all the numbers in this sequence.
\end{solution}

\ppart Conclude that every positive integer is a divisor of infinitely
many Fibonacci numbers.

\hint Start the Fibonacci sequence with the values 0,1 instead of 1, 1.

\begin{solution}
Let $f$ be the reverse-linear-recursive function such that $f(0) = 0,
f(1) = 1$, and $f(n) = f(n-1) + f(n-2)$.  By
part~\eqref{ppart:alldivisors}, every positive integer is a divisor of
$f(n)$ for infinitely many $n$.  But $\text{Fib}(n) = f(n+1)$.
\end{solution}

\eparts
\end{problem}

\endinput
