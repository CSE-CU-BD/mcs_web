\documentclass[problem]{mcs}

\begin{pcomments}
  \pcomment{CP_variance_properties}
  \pcomment{S02.cp12f}
  \pcomment{from Velleman by Ashish, May 2, 2002}
\end{pcomments}

\pkeywords{
  variance
  expectation
  random_variable
}

%%%%%%%%%%%%%%%%%%%%%%%%%%%%%%%%%%%%%%%%%%%%%%%%%%%%%%%%%%%%%%%%%%%%%
% Problem starts here
%%%%%%%%%%%%%%%%%%%%%%%%%%%%%%%%%%%%%%%%%%%%%%%%%%%%%%%%%%%%%%%%%%%%%

\begin{problem}
Let $R$ be a nonnegative integer valued random variable.
\begin{problemparts}

\iffalse
\inhandout{\problempart
Let $T \eqdef cR$ for some constant~$c$.  Prove that
\[
\variance{T} = c^2\variance{R}\ .
\]

\begin{solution}
\begin{align*}
\variance{T} 
& = \expect{T^2} - \expectsq{T}\\
& = \expect{(cR)^2} - \expectsq{cR}\\
& = \expect{c^2R^2} - \paren{c\expect{R}}^2\\
& = c^2\expect{R^2} - c^2\expectsq{R}\\
& = c^2 \variance{R} 
\end{align*}
\end{solution}
}
\fi

\problempart
If $\expect{R}=1$, how large can $\variance{R}$ be?

\begin{solution}
The variance of $R$ is unbounded.  For example, suppose $R_n$ is a
random variable that equals $n$ with probability $1/n$ and equals 0
otherwise.  So $\expect{R_n} = n/n = 1$ as required.  However,
\[
\variance{R_n} = \expect{R_n^2} - \expectsq{R_n}= \frac{n^2}{n} - 1 = n-1.
\]
So $\variance{R_n}$ grows unboundedly.

In fact, there are nonnegative integer valued random variables with
expectation 1 and infinite variance (see
Problem~\bref{CP_infinite_variance}).
\end{solution}

\problempart
If $R$ is always positive (nonzero), how large can $\expect{1/R}$ be?

\begin{solution}

$\expect{1/R}$ is maximized at 1 when $R = 1$ with probability $1$.
Intuitively, this can be seen by trying to shift some of the
probability mass away from $1$.  No matter how you do it, you will
always wind up with an expectation that is less than $1$.

Formally, we can prove this as follows. Suppose $\prob{R=1} = 1-\delta$
for some $\delta > 0$. Then
\begin{align*}
\expect{R}
& = 1\cdot(1-\delta) + \sum_{i=2}^{\infty}\frac{1}{n}\cdot \prob{R=n}\\
& \leq 1\cdot(1-\delta) + \sum_{i=2}^{\infty}\frac{1}{2}\cdot \prob{R=n}\\
& \leq 1\cdot(1-\delta) + \frac{1}{2}\sum_{i=2}^{\infty}\prob{R=n}\\
& \leq 1\cdot(1-\delta) + \frac{1}{2}\cdot\delta\\
& \leq 1-\delta\cdot\frac{3}{2}.
\end{align*}

So, for any positive $\delta$, $\expect{R}$ is less than 1.
Therefore, the given distribution maximizes $\expect{1/R}$.
\end{solution}

\end{problemparts}
\end{problem}

\endinput
