\documentclass[problem]{mcs}

\begin{pcomments}
  \pcomment{FP_stable_matching_unequal_invariants}
  \pcomment{ARM 4/11/15}
  \pcomment{overlaps TP_Stable_Marriage_Invariants, TP_mating_ritual_invariant}
\end{pcomments}

\pkeywords{
 stable_matching
 Mating_ritual
 invariant
 }

%%%%%%%%%%%%%%%%%%%%%%%%%%%%%%%%%%%%%%%%%%%%%%%%%%%%%%%%%%%%%%%%%%%%%
% Problem starts here
%%%%%%%%%%%%%%%%%%%%%%%%%%%%%%%%%%%%%%%%%%%%%%%%%%%%%%%%%%%%%%%%%%%%%

\begin{problem}
The Mating Ritual\inbook{~\bref{stablemarriagesec}} for finding stable
marriages works without change when there are at least as many, and
possibly more, men than women.  You may assume this.  So the Ritual
ends with all the women married and no rogue couples for these
marriages, where an unmarried man and a married woman who prefers him
to her spouse is also considered to be a ``rogue couple.''

Let Alice be one of the women, and Bob be one of the men.
\inbook{Indicate which of }\inhandout{Circle }the properties below
that are preserved invariants of the Mating
Ritual\inbook{~\bref{stablemarriagesec}} when there are at least as
many men as women.  \inbook{Briefly explain your answers.}

\bparts

\ppart Alice has a suitor (man who is serenading her) whom she prefers
to Bob.

\begin{solution}
Invariant: Alice's suitors can only improve during the Ritual.
\end{solution}

\ppart Alice is the only woman on Bob's list.

\begin{solution}
Not preserved.  It would be invariant if there were at as many women
as men, since Bob must marry Alice in that case.  But if there are
more men, Bob may wind up unmarried.
\end{solution}

\ppart Alice has no suitor.

\begin{solution}
Not preserved; Alice may not have a suitor on the first day---if, for
example, she's not at the top of any man's list---but may get a suitor
after the first round of rejections.
\end{solution}

\ppart Bob prefers Alice to the women he is serenading.

\begin{solution}
Invariant.  Bob works down his list, so if Alice is crossed off,
Bob preferred her to anybody left on his list.
\end{solution}

\ppart Bob is serenading Alice.

\begin{solution}
Not invariant: If Bob serenades Alice and gets rejected by her,
he will stop serenading her.
\end{solution}

\ppart Bob is not serenading Alice.

\begin{solution}
Not invariant: Bob might serenade Alice, get rejected by
her, and then serenade Alice next.
\end{solution}

\ppart Bob's list of women to serenade is empty.

\begin{solution}
Invariant: No woman will ever get added to Bob's list, so once his
list is empty, it stays empty.  (If we have at least as many women as
men, Bob's list will never be empty, since the Ritual guarantees he
will be married in the end.  But this predicate is still an invariant
because it is always false.)
\end{solution}

\ppart Bob prefers Alice to the woman he is serenading.

\begin{solution}
Invariant: Bob's favorite current woman never gets better, so if Alice
is better than Bob's current favorite, she will remain better than all
his later favorites.
\end{solution}

\eparts

\end{problem}

\endinput
