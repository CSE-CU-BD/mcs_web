\documentclass[problem]{mcs}

\begin{pcomments}
  \pcomment{CP_relative_primality_under_remainder}
  \pcomment{soln by ARM 3/17/13}
\end{pcomments}

\pkeywords{
  number_theory
  modular_arithmetic
  Euler_function
  phi_function
  primes
  relatively_prime
}

%%%%%%%%%%%%%%%%%%%%%%%%%%%%%%%%%%%%%%%%%%%%%%%%%%%%%%%%%%%%%%%%%%%%%
% Problem starts here
%%%%%%%%%%%%%%%%%%%%%%%%%%%%%%%%%%%%%%%%%%%%%%%%%%%%%%%%%%%%%%%%%%%%%

\begin{problem}
Show that $a,b \in \relpr{n}$ implies $\rem{ab}{n} \in \relpr{n}$.

\begin{solution}
\begin{proof} (Using the definition of $\relpr{n}$):

By unique factorization, $a$ and $b$ do not share common factor
with $n$ iff $ab$ also does not share a common factor with $n$.  So if
$a,b \in \relpr{n}$, then $ab \in \relpr{n}$, which means
$\gcd(ab,n)=1$.  Using \inhandout{the invariant from Euclid's
  Algorithm}\inbook{Lemma~\bref{lem:gcdrem}}
\[
\gcd(\rem{ab}{n},n)=\gcd(ab,n)=1,
\]
so $\rem{ab}{n} \in \relpr{n}$.
\end{proof}

\begin{proof} 
(Using the fact the cancellability (mod $n$) is equivalent to being
  relatively prime to $n$)

The congruence rules (Lemma~\bref{mod_congruence_lem}) imply that if
$c \equiv d \pmod n$ then $c$ is cancellable iff $d$ is cancellable.
But if $a$ and $b$ are cancellable, then $ab$ is too, since
you can first cancel $a$ and then cancel $b$.
\end{proof}

\begin{proof} 
(Using the fact that having an inverse (mod $n$) is equivalent to being
  relatively prime to $n$)

If $a,b$ have inverses $\inv{a},\inv{b}$, then so does $\inv{ab}$.
Namely, $\inv{b}\inv{a} = \inv{ab}$, since
\[
(ab)(\inv{b}\inv{a}) \equiv a(b \inv{b}) \inv{a} \equiv a\cdot 1 \cdot \inv{a} \equiv a\inv{a} \equiv 1 \pmod n.
\]
\end{proof}

\end{solution}

\end{problem}

%%%%%%%%%%%%%%%%%%%%%%%%%%%%%%%%%%%%%%%%%%%%%%%%%%%%%%%%%%%%%%%%%%%%%
% Problem ends here
%%%%%%%%%%%%%%%%%%%%%%%%%%%%%%%%%%%%%%%%%%%%%%%%%%%%%%%%%%%%%%%%%%%%%

\endinput
