\documentclass[problem]{mcs}

\begin{pcomments}
  \pcomment{CP_totient_for_pq}
  \pcomment{virtually the same as proof of Lemma {phi_pq} in book}
\end{pcomments}

\pkeywords{
  number_theory
  modular_arithmetic
  Euler_function
  phi_function
  primes
  relatively_prime
}

%%%%%%%%%%%%%%%%%%%%%%%%%%%%%%%%%%%%%%%%%%%%%%%%%%%%%%%%%%%%%%%%%%%%%
% Problem starts here
%%%%%%%%%%%%%%%%%%%%%%%%%%%%%%%%%%%%%%%%%%%%%%%%%%%%%%%%%%%%%%%%%%%%%

\begin{problem}
For any integer $n>1$, let $n^*$ be the set of integers in $[0,n)$
that are relative prime to $n$.
Show that $a,b \in n^*$ implies $\rem{ab}{n} \in n^*$

\begin{solution}
By the definition of $n^*$, if $a,b \in n^*$, then $a,b$ are
relative prime to $n$.
By unique factorization, $a,b$ does not share common factor with $n$
implies $ab$ also does not share common factor with $n$,
so $ab$ is also relative prime to $n$,
which is equivalent to $\gcd(ab,n)=1$.
Using the rule from Euler's Algorithm,
$\gcd(\rem{ab}{n},n)=\gcd(ab,n)=1$,
so $\rem{ab}{n}$ is also relative prime to $n$.
By the definition of remainder, $\rem{ab}{n} \in [0,n)$,
thus, $\rem{ab}{n} \in n^*$.

\end{solution}

\end{problem}

%%%%%%%%%%%%%%%%%%%%%%%%%%%%%%%%%%%%%%%%%%%%%%%%%%%%%%%%%%%%%%%%%%%%%
% Problem ends here
%%%%%%%%%%%%%%%%%%%%%%%%%%%%%%%%%%%%%%%%%%%%%%%%%%%%%%%%%%%%%%%%%%%%%


\endinput
