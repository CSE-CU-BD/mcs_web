\documentclass[problem]{mcs}

%\newcommand{\vout}[1]{\text{outof}(#1)}
\newcommand{\vin}[1]{\text{into}(#1)}

\begin{pcomments}
  \pcomment{CP_simple_google_graph}
  \pcomment{from: S08.cp13m}
\end{pcomments}

\pkeywords{
  probability
  random_walk
  stationary_distributions
  simple_graph
  degree
}

%%%%%%%%%%%%%%%%%%%%%%%%%%%%%%%%%%%%%%%%%%%%%%%%%%%%%%%%%%%%%%%%%%%%%
% Problem starts here
%%%%%%%%%%%%%%%%%%%%%%%%%%%%%%%%%%%%%%%%%%%%%%%%%%%%%%%%%%%%%%%%%%%%%

\begin{problem}
A Google-graph is a random-walk graph such that every edge leaving any
given vertex has the same probability.  That is, the probability of
each edge $\diredge{v}{w}$ is $1/\outdegr{v}$.

A digraph is \term{symmetric} if, whenever $\diredge{v}{w}$ is an
edge, so is $\diredge{w}{v}$.  Given any finite, symmetric
Google-graph, let
\[
d(v) \eqdef \frac{\outdegr{v}}{e},
\]
where $e$ is the total number of edges in the graph.

\bparts

\ppart If $d$ was used for webpage ranking, how could you hack this to
give your page a high rank? \dots and explain informally why this
wouldn't work for ``real'' page rank using digraphs?

\begin{solution}
Just fabricate a large number of new pages and link them all to your
page.

This wouldn't work in the directed case because the fabricated pages
would not have ``real'' links pointing to them, and so would have
little if any weight to contribute to the weight of your page.
\end{solution}

\ppart Show that $d$ is a stationary distribution.

\begin{solution}
To show that $d$ is a stationary distribution, we must show that
\begin{equation}\label{rank}
  d(w)= \sum_{v \in \vin{w}} p(v,w) d(v),
\end{equation}
where $\vin{w} \eqdef \set{v \suchthat \diredge{v}{w} \text{ is an edge}}$.

We have
\begin{align*}
  \lefteqn{\sum_{v \in \vin{w}} p(v,w) d(v)} \\
  &= \sum_{v \in \vin{w}}
  \paren{\frac{1}{\outdegr{v}}}\paren{\frac{\outdegr{v}}{e}} & \text{(by def $p$ and $d$)}\\
  \\
  &= \sum_{v \in \vin{w}} \frac{1}{e} = \frac{\card{\vin{w}}}{e} = \frac{\indegr{w}}{e}\\
  &= \frac{\outdegr{w}}{e} & \text{(by symmetry of the graph)}\\
  &= d(w) & \text{(def of $d$)}.
  \end{align*}
\end{solution}

\eparts

\end{problem}

\endinput
