\documentclass[problem]{mcs}

\begin{pcomments}
  \pcomment{from: ???}
\end{pcomments}

\pkeywords{
  well-ordering
  WOP
  series
}

%%%%%%%%%%%%%%%%%%%%%%%%%%%%%%%%%%%%%%%%%%%%%%%%%%%%%%%%%%%%%%%%%%%%%
% Problem starts here
%%%%%%%%%%%%%%%%%%%%%%%%%%%%%%%%%%%%%%%%%%%%%%%%%%%%%%%%%%%%%%%%%%%%%

\begin{problem}
Use the Well Ordering Principle to prove that
\begin{equation}\label{sum-of-sq}
\sum_{k=0}^n k^2 = \frac{n(n+1)(2n+1)}{6}.
\end{equation}
for all nonnegative integers, $n$.


\begin{solution}
The proof is by contradiction.

Suppose to the contrary that equation~\eqref{sum-of-sq} failed for some $n
\geq 0$.  Then by the WOP, there is a \emph{smallest} nonnegative integer,
$m$, such that~\eqref{sum-of-sq} does not hold when $n = m$.

But~\eqref{sum-of-sq} clearly holds when $n = 0$, which means that $m \geq
1$.  So $m-1$ is nonegative, and since it is smaller than $m$,
equation~\eqref{sum-of-sq} must be true for $n = m-1$.  That is,
\begin{equation}\label{sum-to-m-1}
\sum_{k=0}^{m-1} k^2 = \frac{(m-1)((m-1) + 1)(2(m-1)+1)}{6}.
\end{equation}
Now add $m^2$ to both sides of equation~\eqref{sum-to-m-1}.
Then the left hand side equals
\[
\sum_{k=0}^{m} k^2
\]
and the right hand side equals
\[
\frac{(m-1)((m-1) + 1)(2(m-1)+1)}{6} + m^2 
\]
Now a little algebra (given below) shows that the right hand side equals
\[
\frac{m(m+1)(2m+1)}{6}.
\]
That is,
\[
\sum_{k=0}^{m} k^2 = \frac{m(m+1)(2m+1)}{6},
\]
contradicting the fact that equation~\eqref{sum-of-sq} does not hold for
$m$.

It follows that there is no smallest nonnegative integer for which
equation~\eqref{sum-of-sq} fails.  Hence~\eqref{sum-of-sq} must hold for
all nonnegative integers.

Here's the algebra:
\textbox{
\begin{align*}
\frac{(m-1)((m-1) + 1)(2(m-1)+1)}{6} + m^2 
&= \frac{(m-1)m(2m-1)}{6} + m^2\\
 &  = \frac{(m^2-m)(2m-1)}{6} + m^2\\
 & = \frac{(2m^3-3m^2 +m)}{6} + \frac{6m^2}{6}\\
 &  = \frac{(2m^3 +3m^2 +m)}{6}\\
 &  = \frac{m(m+1)(2m+1)}{6}
\end{align*}}

\end{solution}

\end{problem}

%%%%%%%%%%%%%%%%%%%%%%%%%%%%%%%%%%%%%%%%%%%%%%%%%%%%%%%%%%%%%%%%%%%%%
% Problem ends here
%%%%%%%%%%%%%%%%%%%%%%%%%%%%%%%%%%%%%%%%%%%%%%%%%%%%%%%%%%%%%%%%%%%%%
