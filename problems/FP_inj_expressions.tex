\documentclass[problem]{mcs}

\begin{pcomments}
  \pcomment{FP_inj_expressions}
  \pcomment{F99.final, s18.mid2}
  \pcomment{ARM 5/11/17 part(b) cut, reformatted}
\end{pcomments}

\pkeywords{
  cardinality
  infinite
  injection
  bijection
}

\begin{problem}
%\textbf{Injections and Cardinality.}

Let $f:D \to D$ be a total function from some nonempty set $D$
to itself.  In the following propositions, $x$ and $y$ are variables
ranging over $D$, and $g$ is a variable ranging over total functions
from $D$ to $D$.  \inhandout{Circle} \inbook{Indicate} all of the
propositions that are equivalent to the proposition that $f$ is an
\emph{injection}. \inhandout{No explanations are necessary.}

\begin{enumerate}[(i)]
\item $\forall x\forall y.\ x = y \QOR f(x) \neq f(y)$          %i.
\item $\forall x\forall y.\ x = y \QIMPLIES f(x) = f(y)$        %ii.
\item $\forall x\forall y.\ x \neq y \QIMPLIES f(x) \neq f(y)$  %iii
\item $\forall x\forall y.\ f(x) = f(y) \QIMPLIES x = y$        %iv
%\item $\forall x \forall y(f(x) \neq f(y))$
\item $\QNOT[\exists x\exists y.\ x \neq y \QAND f(x)= f(y)]$   %v
\item $\QNOT[\exists y \forall x.\ f(x) \neq y]$                %vi
\item $\exists g \forall x.\ g(f(x))=x$                         %vii
\item $\exists g \forall x. f(g(x))=x$                          %viii
\end{enumerate}

\begin{solution}
Correct choices are:
\begin{itemize}
\item (i), which is equivalent to $\forall x\forall y.\ f(x) = f(y)
  \QIMPLIES x = y$.  That is, ``$f$-arrows to the same place must
  start from the same place,'' which means there is really at most one
  arrow into any one place.
\item (iii), is equivalent to (i) because $P \QIMP Q$ is equivalent to $\QNOT(P) \QOR Q$.
\item (iv), is the contrapositive of (iii)
\item (v), means ``no $f$-arrows from different places go to the same
  place'' which means there is really at most one arrow into any one
  place.  In addition, (v) is equivalent to (i) by DeMorgan's Laws.
\item (vii), means ``given the end of any $f$-arrow, $g$ shows where
  it starts.''  This would be impossible if $f$-arrows that started at
  different places ended at the same place.
\end{itemize}

On the other hand,
\begin{itemize}
\item (ii) is valid,
\item (vi) means ``there is no point $y$ without an $f$-arrow into it,''
  that is, $f$ is a total function.
\item (viii) means ``given any element, $g$ shows the start of an
  arrow into it,'' so there must be an arrow into every element.  That
  is, $f$ is surjective.
\end{itemize}

\end{solution}

\iffalse
\eparts
\medskip

We have proved that the infinite set of real numbers is ``strictly
bigger'' than the infinite set of integers, and the set of integers is
``strictly bigger'' than any finite set.

\bparts

\ppart  State precisely what it means to say a set $A$ is ``strictly
bigger'' than another set $B$.

\begin{solution}
There is no injection from $A$ to $B$.  Alternatively, there is no
bijection between $A$ and $B$, while there is an injection from $B$ to
$A$.
\end{solution}

\examspace{1.5in}

\ppart Give an example of a set that is strictly bigger than the set
of real numbers.

\begin{solution}
$A \eqdef P(\reals)$.
\end{solution}

\eparts
\fi

\end{problem}

%%%%%%%%%%%%%%%%%%%%%%%%%%%%%%%%%%%%%%%%%%%%%%%%%%%%%%%%%%%%%%%%%%%%%
% Problem ends here
%%%%%%%%%%%%%%%%%%%%%%%%%%%%%%%%%%%%%%%%%%%%%%%%%%%%%%%%%%%%%%%%%%%%%

\endinput
