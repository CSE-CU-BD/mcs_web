\documentclass[problem]{mcs}

\begin{pcomments}
  \pcomment{MQ_uncountable_ones}
  \pcomment{ARM May 19, 2013; hint & soln edited, 3/29/17, revised 11/2/17}
  \pcomment{overlaps PS_off_diagonal_arguments, FP_uncountable_sparse1s}
\end{pcomments}

\pkeywords{
  uncountable
  surjection
  infinite
  string
}

%%%%%%%%%%%%%%%%%%%%%%%%%%%%%%%%%%%%%%%%%%%%%%%%%%%%%%%%%%%%%%%%%%%%%
% Problem starts here
%%%%%%%%%%%%%%%%%%%%%%%%%%%%%%%%%%%%%%%%%%%%%%%%%%%%%%%%%%%%%%%%%%%%%
\begin{problem}
Let $\binw$ be the set of infinite binary strings, and let $B \subset
\binw$ be the set of infinite binary strings containing infinitely
many occurrences of 1's.  Prove that $B$ is uncountable.  \iffalse (We
have already shown that $\binw$ is uncountable.)\fi

\hint One approach is by an ``off-diagonal'' argument as in
Problem\bref{PS_off_diagonal_arguments}

 \iffalse Start by showing that $\binw \inj B$.\fi

\begin{solution}
    We'll argue by diagonalization with ``slope -1/2'' as we saw in
    class.  For the sake of contradiction, assume $B$ is countable.
    Since $B$ is certainly infinite, it must be in bijection with
    $\nngint$, so we can write $B = \set{s_0, s_1, s_2, \dots}$.
    Define a new sequence $t\in\binw$ as follows:
    \begin{equation*}
      t \eqdef (\bar{s_0[0]}, 1, \bar{s_1[2]}, 1, \bar{s_2[4]}, 1, \dots).
    \end{equation*}
    In detail, for each $k\in\nngint$, we define $t[2k]$ as the
    opposite of bit $s_k[2k]$, and $t[2k+1]$ is always $0$.  Because
    $1$s can only appear at even indices, $t$ is lonely, so $t\in B$.
    On the other hand, $t\ne s_k$ for $k\in\nngint$, because $t$
    and $s_k$ differ in their $2k$th digit.  Thus,
    $t\notin\set{s_0,s_1,s_2,\dots}$, that is, $t\notin B$.  This is a
    contradiction, so our assumption that $B$ is countable must be
    false, that is, $B$ is uncountable.

An alternative proof using a similar idea does not use diagonalization
at all.  We know $\binw$ is uncountable (Corollary to Cantor's
Theorem~\bref{powbig}).  By Corollary~\bref{AsurjUA}~\bref{AsurjUA},
it is enough to show that $B \surj \binw$, which is equivalent to
$\binw \inj B$ (Lemma~\bref{surjinjbij_properties}~\bref{surjvsinj}.
That is, it's sufficient to define an injective function $f:\binw \to
B$.

An easy way is to define, for any $b \in \binw$, the value of $f(b)$
to be the string obtained by inserting a 1 between each of the bits in
$b$.

Another way is to let $f(b)$ to be $0b$ if $b$ has infinitely many
1's, and otherwise to be $1p01^{\omega}$ where $\widehat{p}$ is the
shortest prefix of $b$ that includes all the 1's in $b$ and $p$ is
obtained from $\widehat{p}$ by replacing each occurrence of 1 by 11,
and each occurrence of 0 by 00.

\begin{staffnotes}
Alternatively, uncountability of $B$ follows from the easily verified
fact (see Problem~\bref{FP_infinite_binary_sequences_S14}) that the set
$\bar{B}$ of infinite binary strings with only \emph{finitely many
  ones} is countable.  So if $B$ was also countable, then $\binw = B
\union \bar{B}$ would be countable, a contradiction.
\end{staffnotes}

\end{solution}

\end{problem}

\endinput
