\documentclass[problem]{mcs}

\newcommand{\discrs}{\ensuremath{\mathcal{D}}}


\begin{pcomments}
  \pcomment{PS_relations_from_predicate_logic_meyer}
  \pcomment{start of ARM edit of PS_relations_from_predicate_logic 2/14/12}
  \pcomment{original by adamc for S12:ps2}
\end{pcomments}


\pkeywords{
  relations
  predicate_logic
}

%%%%%%%%%%%%%%%%%%%%%%%%%%%%%%%%%%%%%%%%%%%%%%%%%%%%%%%%%%%%%%%%%%%%%
% Problem starts here
%%%%%%%%%%%%%%%%%%%%%%%%%%%%%%%%%%%%%%%%%%%%%%%%%%%%%%%%%%%%%%%%%%%%%

\begin{problem}

Many mathematical concepts can be expressed with roughly similar
effort either via sets or predicate formulas.  One might wonder
whether there are concepts that can be expressed only with one of the
approaches and not the other.  It may be easy to believe that any
description via sets can be written as a predicate formula as well,
since, as we will see in Chapter~\bref{set_theory_chap}, the axioms of
ZF set theory show how the concepts of set theory can be defined via
predicate formulas.  Less obviously, we can re-express any predicate
formula via set theory operations.  In this problem, you will invent a
strategy for this translation, dealing with a subset of the usual
predicate formula constructs.

In particular, we will work with the set of predicate formulas defined
recursively as follows:
\begin{itemize}

\item $R(x, y)$ is a predicate formula, for any binary relation symbol
  $R$ and variables $x$ and $y$,

\item $\QNOT(\varphi)$, where $\varphi$ is a predicate formula,

\item $\varphi_1 \QAND \varphi_2$ and $\varphi_1 \QOR \varphi_2$,
  where $\varphi_1$ and $\varphi_2$ are predicate formulas,

\item $\forall x. \, \varphi$ and $\exists x. \, \varphi$, where $x$
  is a variable, and $\varphi$ is a formula that may mention $x$.
\end{itemize}

From now on, we'll assume there is a fixed domain of discourse,
\discrs, that all variables range over and that each relation symbol
$R$ that appears in a formula has an associated binary relation
$R_\discrs$ on the domain \discrs.  For example, we might have
$\discrs$ be the real numbers and $R$ stand for the less-than relation
$\reals$.  Then the formula $R(x,y)$ would be true exactly when the
value of $x$ was a real number less than the value of $y$.

To determine whether or not a formula is true, you need to know the
values of its variables.  An \term{environment} is what computer
scientists call an object that assigns values to variables; logicians
call them \term{variable assignments}.  An environment can be
represented simply as a function from the set of variables to their
values in \discrs.

Now two formulas are equivalent iff the mean the same thing.  More
precisely, they are equivalent iff they are true in the same
environments.  So it's natural to define the \term{meaning},
$\mean{\varphi}$, of a formula $\varphi$ to be the set of environments
in which it is true.

For example, suppose $\varphi_0$ is the formula $R(x,y) \QAND R(y,z)$,
and we're taking the domain to be the real nmnbers $\reals$
and $R$ to be the less-than relation, as above.  Then if $\nu_1$ and
$\nu_2$ are environments with
\[\begin{array}{l}
\nu_1(x) = 0.1,\ \nu_1(y) = 0.2,\ \nu_1(z) = 0.3,\\ 
\nu_2(x) = \nu_2(y) = \nu_1(z) = 0.4,
\end{array}\]
then $\varphi_0$ is true in $\nu_1$ and false in $\nu_2$.  More
generally, $\varphi_0$ is true those environments, $\nu$, where
$\nu(x) < \nu(y) < \nu(z)$.  We take the set of all such environments
  to be the meaning of the formula.  Letting $\Env$ be set of all
  environments, we can express this concisely using set notation:
\[
\mean{\varphi_0} = \set{\nu \in \Env \suchthat \nu(x) < \nu(y) < \nu(z)}.
\]

In this problem, we define the general meaning of predicate formulas.
We can start with the base case $R(x,y)$:
\begin{equation}\label{meanR}
\mean{R(x,y)} \eqdef \set{\nu \in \Env \suchthat R_\discrs(\nu(x), \nu(y))}.
\end{equation}

\bparts

\ppart  Complete a recursive definition of the meaning of
predicate formulas.  The expression for the meaning of an expression
should be expressed by applying set operations to the meanings of its
component expressions.

Two more notational conventions will be useful:
\begin{itemize}
\item  Use the notation $\patch{\nu}{x}{v}$ to indicate the assignment
  that maps $x$ to $v$ and leaves any other variable $y$ with the same
  value $\nu(y)$.
\iffalse
If $f:A \to B$ is a function and $a,b$ are arbitrary elements, define
  \[
\patch{f}{a}{b}
\]
to be the function $g$ such that
\begin{equation*}
g(u) = \begin{cases}
         b & \text{ if } u = a.\\
         f(u) & \text{otherwise.}
       \end{cases}
\end{equation*}
\fi

\item You are probably used to the notation $\sum_{i=1}^n e(i)$ to
  denote the sum of all values of $e(i)$ while variable $i$ ranges
  from 1 to $n$.  Another way to say this is $\sum_{i \in S} e(i)$
  where $S$ is the set of values we want $i$ to take, in this case $S=
  \set{1,\dots n}$.  Similar notation $\bigcup_{x \in S} e(x)$ and
  $\bigcap_{x \in S} e(x)$ will be useful for taking the union or
  intersection, respectively, of all values of $e$ obtained by
  subsituting elements of $S$ for variable $x$.
\end{itemize}

\begin{solution}
  The translation is easily expressed as a \emph{recursive function},
  which you may be familiar with from programming.  We define the
  translation of each formula in terms of the translations of its
  immediate subformulas.
\begin{align*}
  \mean{R(x, y)} & \eqdef  \set{ a \in \Env \mid R(a(x), a(y)) } \\
  \mean{\QNOT{} \varphi} & \eqdef  \Env - \mean{\varphi} \\
  \mean{\varphi_1 \QAND{} \varphi_2} & \eqdef  \mean{\varphi_1} \cap \mean{\varphi_2} \\
  \mean{\varphi_1 \QOR{} \varphi_2} & \eqdef  \mean{\varphi_1} \cup \mean{\varphi_2} \\
  \mean{\forall x. \; \varphi} & \eqdef
     \bigcap_{v \in \discrs} \set{a \in \Env \mid \patch{a}{x}{v} \in \mean{\varphi} } \\
     \mean{\exists x. \; \varphi} & \eqdef
     \bigcup_{v \in \discrs} \set{ a \in \Env \mid \patch{a}{x}{v} \in \mean{\varphi} }
\end{align*}
\end{solution}

\ppart Prove that a predicate formula $\varphi$ is true under some
variable assignment $a$ iff $a$ belongs to the set that your
translation outputs for $\varphi$.

\hint Consider the set of nonnegative integers $n$ such that some
formula $\varphi$ exists where (a) $\varphi$ is $n$ characters long
and (b) $\varphi$ is a counterexample to the claim above.  Derive a
contradiction via the Well Ordering Principle.

\begin{solution}
  Let $C$ be the set of nonnegative integers described in the hint.
  Assume for a contradiction that $C$ is nonempty.  By the WOP, there
  exists a minimal $n \in C$.

  Since $n \in C$, there exist $\varphi$ and $a$ such that $\varphi$
  is $n$ characters long and where we do not have that $a$ satisfies
  $\varphi$ iff $a \in \mean{\varphi}$.  We proceed with a proof by cases
  over the ways $\varphi$ might be constructed:

  \begin{itemize}

  \item Case $\varphi = R(x, y)$:

\begin{align*}
\lefteqn{a \textrm{ satisfies } \varphi}\\
&\QIFF R(a(x), a(y))
       & \textrm{(definition of relation formulas)}\\
&\QIFF a \in \set{ a \in \Env \suchthat R(a(x), a(y)) }
       & \textrm{(definition of set builder notation)}\\
&\QIFF a \in \mean{R(x, y)}
       & \textrm{(definition of $\mean{}$)}
\end{align*}

  \item Case $\varphi = \QNOT(\mean{\varphi'})$:

    Since $\varphi'$ is a subformula of $\varphi$, and since $\varphi$
    contains the character $\QNOT{}$ and $\varphi'$ doesn't, we know
    that $\varphi'$ has a lower character length $n'$ than $\varphi$
    does.  This implies that $n' \not\in C$, since we chose $n$ to be
    the minimum element of $C$.  Therefore, the required equivalence
    holds for $\varphi'$ and any $a$, which we can use to derive the
    equivalence for $\varphi$:
\begin{align*}
\lefteqn{a \textrm{ satisfies } \varphi}\\
&\QIFF \QNOT(a \textrm{ satisfies } \varphi')
      & \textrm{(definition of $\QNOT$)} \\
&\QIFF \QNOT(a \in \mean{\varphi'})
      & \textrm{(equivalence holds for $\varphi'$)} \\
&\QIFF a \in \mathcal \Env - \mean{\varphi'}
      & \textrm{(definition of $-$)} \\
&\QIFF a \in \mean{\QNOT(\varphi')}
      & \textrm{(definition of $\mean{}$)}
\end{align*}

  \item Case $\varphi = \varphi_1 \QAND \varphi_2$: \\

    By similar reasoning to above, both $\varphi_1$ and $\varphi_2$
    are shorter than $\varphi$, so the equivalence holds for them.
\begin{align*}
\lefteqn{a \textrm{ satisfies } \varphi}\\
&\QIFF (a \textrm{ satisfies } \varphi_1) \QAND (a \textrm{ satisfies } \varphi_2)
        & \textrm{(definition of $\QAND{}$)} \\
&\QIFF (a \in \mean{\varphi_1}) \QAND (a \in \mean{\varphi_2})
        & \textrm{(equivalence holds for $\varphi_1$ and $\varphi_2$)} \\
&\QIFF a \in \mean{\varphi_1} \cap \mean{\varphi_2}
        & \textrm{(definition of $\cap$)} \\
&\QIFF a \in \mean{\varphi_1 \QAND{} \varphi_2}
        & \textrm{(definition of $\mean{}$)}
\end{align*}

  \item Case $\varphi = \varphi_1 \QOR{} \varphi_2$: \\

    By similar reasoning to above, both $\varphi_1$ and $\varphi_2$
    are shorter than $\varphi$, so the equivalence holds for them.
\begin{align*}
\lefteqn{a \textrm{ satisfies } \varphi}\\
&\QIFF (a \textrm{ satisfies } \varphi_1) \QOR{} (a \textrm{ satisfies } \varphi_2)
        & \textrm{(definition of $\QOR{}$)} \\
&\QIFF (a \in \mean{\varphi_1}) \QOR (a \in \mean{\varphi_2})
        & \textrm{(equivalence holds for $\varphi_1$ and $\varphi_2$)} \\
&\QIFF a \in \mean{\varphi_1} \cup \mean{\varphi_2}
        & \textrm{(definition of $\cup$)} \\
&\QIFF a \in \mean{\varphi_1 \QOR{} \varphi_2}
        & \textrm{(definition of $\mean{}$)}
\end{align*}

  \item Case $\varphi = \forall x. \; \varphi'$: \\
    By similar reasoning to above, $\varphi'$ is shorter than $\varphi$, so the equivalence holds for it.
\begin{align*}
\lefteqn{a \textrm{ satisfies } \varphi}\\
 &\QIFF \forall v \in \mathcal D. \; \patch{a}{x}{v} \textrm{ satisfies } \varphi'
     & \textrm{(definition of $\forall$)} \\
&\QIFF \forall v \in \mathcal D. \; \patch{a}{x}{v} \in
\mean{\varphi'}
     & \textrm{(equivalence holds for $\varphi'$)} \\
&\QIFF a \in \bigcap_{v \in \mathcal D} \set{a' \in \Env \mid \patch{a'}{x}{v} \in \mean{\varphi'}}
     & \textrm{(definitions of $\bigcap$ and set builder notation)} \\
&\QIFF a \in \mean{\forall x. \; \varphi'}
     & \textrm{(definition of $\mean{}$)}
\end{align*}

  \item Case $\varphi = \exists x. \; \varphi'$: \\
    By similar reasoning to above, $\varphi'$ is shorter than $\varphi$, so the equivalence holds for it.
\begin{align*}
\lefteqn{a \textrm{ satisfies } \varphi}\\
 &\QIFF \exists v \in \mathcal D. \; \patch{a}{x}{v} \textrm{ satisfies } \varphi'
      & \textrm{(definition of $\exists$)} \\
 &\QIFF \exists v \in \mathcal D. \; \patch{a}{x}{v} \in \mean{\varphi'}
      & \textrm{(equivalence holds for $\varphi'$)}\\
 &\QIFF a \in \bigcup_{v \in \mathcal D}
              \set{a' \in \Env \mid \patch{a'}{x}{v} \in \mean{\varphi'}}
      & \textrm{(definitions of $\bigcup$ and set builder notation)} \\
 &\QIFF a \in \mean{\exists x. \; \varphi'}
      & \textrm{(definition of $\mean{}$)}
\end{align*}

  \end{itemize}

  Since every case shows that the desired equivalence holds for $\varphi$, we have contradicted our assumption that $C$ is nonempty.  Therefore, $C$ is empty, and every formula $\varphi$ of every character length satisfies the equivalence.
\end{solution}

\eparts

\end{problem}

\endinput
