\documentclass[problem]{mcs}

\begin{pcomments}
  \pcomment{from: S09.cp6t}
\end{pcomments}

\pkeywords{
  connectivity
  graphs
  complete_graphs
}

%%%%%%%%%%%%%%%%%%%%%%%%%%%%%%%%%%%%%%%%%%%%%%%%%%%%%%%%%%%%%%%%%%%%%
% Problem starts here
%%%%%%%%%%%%%%%%%%%%%%%%%%%%%%%%%%%%%%%%%%%%%%%%%%%%%%%%%%%%%%%%%%%%%

\begin{problem}
Prove that $K_n$ is $(n-1)$-edge connected for $n > 1$.

\solution{
\begin{proof}
  Consider any two distinct vertices $u$ and $v$ in $K_n$.  For each of the
  other $n-2$ vertices in $K_n$, there is a length 2 path from $u$ to $v$
  via that vertex.  This gives us a total of $n-1$ edge disjoint paths
  between $u$ and $v$, namely the the $n-2$ length 2 paths above and the
  length 1 path directly from $u$ to $v$.  This implies it takes at least
  $n-1$ edge deletions to disconnect any 2 vertices, meaning that $K_n$ is
  $(n-1)$-edge connected.
\end{proof}}

\ppart Let $M_n$ be a graph defined as follows: begin by taking $n$ graphs
with non-overlapping sets of vertices, where each of the $n$ graphs is
$(n-1)$-edge connected (they could be disjoint copies of $K_n$, for
example).  These will be subgraphs of $M_n$.  Then pick $n$ vertices, one
from each subgraph, and enough edges between pairs of picked vertices that
the subgraph of the $n$ picked vertices and the edges between them is also
$(n-1)$-edge connected.

Explain why $M_n$ is $(n-1)$-edge connected.

\solution{Remove $(n-2)$ edges from $M_n$ to obtain a subgraph, $G$, of
$M_n$ with the same vertices.

Since each of the $n$ subgraphs are $(n-1)$-edge connected, the vertices
within each copy will remain connected in $G$.  Hence, if all the picked
vertices still are connected in $G$, then $G$ will be connected.  But the
subgraph of picked vertices is also $(n-1)$-edge connected, so will still
be connected after removal of $(n-2)$ edges.  So $M_n$ remains connected
after removal of $n-2$ edges, and hence is indeed $(n-1)$-edge connected.}

\eparts

\end{problem}

%%%%%%%%%%%%%%%%%%%%%%%%%%%%%%%%%%%%%%%%%%%%%%%%%%%%%%%%%%%%%%%%%%%%%
% Problem ends here
%%%%%%%%%%%%%%%%%%%%%%%%%%%%%%%%%%%%%%%%%%%%%%%%%%%%%%%%%%%%%%%%%%%%%
