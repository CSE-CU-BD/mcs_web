\documentclass[problem]{mcs}

\begin{pcomments}
  \pcomment{TP_sort_cyclic_shift_three}
  \pcomment{related to CP_fifteen_puzzle}
  \pcomment{ARM and CH, S14}
\end{pcomments}

\pkeywords{
  state_machines
  sorting
  fifteen_puzzle
  parity
}

%%%%%%%%%%%%%%%%%%%%%%%%%%%%%%%%%%%%%%%%%%%%%%%%%%%%%%%%%%%%%%%%%%%%%
% Problem starts here
%%%%%%%%%%%%%%%%%%%%%%%%%%%%%%%%%%%%%%%%%%%%%%%%%%%%%%%%%%%%%%%%%%%%%

\begin{problem}
The following problem is a twist on the {Fifteen-Puzzle} problem that
we did in class.

Let $A$ be a sequence consisting of the numbers $1,\dots,n$ in some
order.  A pair of integers in $A$ is called an \emph{out-of-order
  pair} when the first element of the pair both comes \emph{earlier}
in the sequence, and \emph{is larger}, than the second element of the
pair.  For example, the sequence $(1,2,4,5,3)$ has two out-of-order
pairs: $(4,3)$ and $(5,3)$.  We let $t(A)$ equal the number of
out-of-order pairs in $A$.  For example, $t((1,2,4,5,3)) = 2$.

The elements in $A$ can rearranged using the \emph{Rotate-Triple}
operation, in which three consecutive elements of $A$ are rotated to
move the smallest of them to be first.

For example, in the sequence $(2,4,1,5,3)$, the \emph{Rotate-Triple}
operation could rotate the consecutive numbers $4,1,5$, into $1, 5, 4$
so that
\[
(2,4,1,5,3) \movesto (2, 1, 5, 4, 3).
\]

The \emph{Rotate-Triple} could also rotate the consecutive numbers
$2,4,1$ into $1,2,4$ so that
\[
(2,4,1,5,3) \movesto (1, 2, 4, 5, 3).
\]

We can think of a sequence $A$ as a state of a state machine whose
transitions correspond to possible applications of the
\emph{Rotate-Triple} operation.

\begin{problemparts}
\iffalse

\problempart Write out the set of transitions that describe the
actions of $\emph{Rotate-Triple}$ at any time step.

\hint You will need to consider six different cases.

\begin{solution}

Consider any consecutive triple in $A$ containing the elements $\{ a,b,c
\}$ in some order. Without loss of generality, assume that $a < b < c$. The
six different cases correspond to the different permutations of
$\{a,b,c\}$. Explicitly, the set of transitions can be described as follows:
\begin{align*}
\emph{Rotate-Triple}(a,b,c) &\movesto (a,b,c) \\
\emph{Rotate-Triple}(a,c,b) &\movesto (a,c,b) \\
\emph{Rotate-Triple}(b,c,a) &\movesto (a,b,c) \\
\emph{Rotate-Triple}(b,a,c) &\movesto (a,c,b) \\
\emph{Rotate-Triple}(c,b,a) &\movesto (a,c,b) \\
\emph{Rotate-Triple}(c,a,b) &\movesto (a,b,c).
\end{align*}

\end{solution}

\examspace[1in]
\fi

\problempart
Argue that for the state machine model described above, the derived
variable $t$ is \emph{weakly decreasing}.

\begin{solution}
Suppose $a,b,c$ are three consecutive elements in $A$.

If $b$ is the smallest element, then these elements get rearranged
into $b,c,a$.  But $a,b,c \movesto b,c,a$ removes one out-of-order
pair $(a,b)$.  Moreover, if $a < c$, this move introduces the
out-of-order pair $c,a$ for a net change of zero in $t(A)$, and if $c
< a$, this move removes the out-of-order pair $a,c$ for a net decrease
of two in $t(A)$.

If $c$ is the smallest, they get rearranged into $c,a,b$ and a similar
argument shows that $t(A)$ remains the same or decreases by two.

The transitions $(b,c,a) \movesto (a,b,c)$ and $(c,a,b) \movesto (a,b,c)$
both \emph{decrease} the number of out-of-order pairs from 2 to 0. 

So In each of the case, $t$ is either constant or decreases, showing
that $t$ is weakly decreasing.

\end{solution}

\examspace[2.0in]

\problempart Define the \emph{parity} of a list $L$ to be 0 or 1,
depending on whether the number of out-of-order pairs in $L$ is even
or odd.   For example, the parity of the list $L = (1,2,4,5,3)$ is 0
(since $t(L)=2$).

Show that the parity of a state is a preserved invariant.  

\begin{solution}
This part follows directly from the previous argument showing that $t$
is weakly decreasing, namely, $t$ changes by 0 or $-2$, so its parity
remains unchanged.

\end{solution}

\examspace[1.0in]

\problempart Starting with $S \eqdef (n,n-1,n-2,\dots, 2 ,1)$ there is
only one sequence at which the procedure could terminate, that is,
from whcih no further transition is possible.  Describe this sequence.

\begin{solution}
$(1, 2, \dots, n-2, n, n-1)$.

The reason is that the smallest element $k<n-1$ that is not in the
$k$th position will be the smallest element in a consecutive triple in
which this element appears second or third, so a rotation to
move it to first place will be possible.  Hence every element $k<n-1$
must be in place, and so the terminal sequence must be $(1, 2, \dots,
n-2, n, n-1)$ or $(1, 2, \dots, n-2, n-1, n)$.  But $t(S) = n(n-1)/2$
is odd, while $t((1, 2, \dots, n-2, n-1, n)) = 0$ and $t((1, 2, \dots,
n-2, n, n-1)) = 1$.  Since parity is preserved, only $(1, 2, \dots,
n-2, n, n-1)$ could be reachable from $S$.  (It is reachable, but this
problem did not ask that this be proved.)

\end{solution}

\examspace[1.0in]

\end{problemparts}

\end{problem}

%%%%%%%%%%%%%%%%%%%%%%%%%%%%%%%%%%%%%%%%%%%%%%%%%%%%%%%%%%%%%%%%%%%%%
% Problem ends here
%%%%%%%%%%%%%%%%%%%%%%%%%%%%%%%%%%%%%%%%%%%%%%%%%%%%%%%%%%%%%%%%%%%%%

\endinput
