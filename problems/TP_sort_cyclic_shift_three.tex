\documentclass[problem]{mcs}

\begin{pcomments}
  \pcomment{TP_sort_cyclic_shift_three}
  \pcomment{related to CP_fifteen_puzzle}
  \pcomment{ARM and CH, S14}
\end{pcomments}

\pkeywords{
  state_machines
  sorting
  fifteen_puzzle
  parity
}

%%%%%%%%%%%%%%%%%%%%%%%%%%%%%%%%%%%%%%%%%%%%%%%%%%%%%%%%%%%%%%%%%%%%%
% Problem starts here
%%%%%%%%%%%%%%%%%%%%%%%%%%%%%%%%%%%%%%%%%%%%%%%%%%%%%%%%%%%%%%%%%%%%%

\begin{problem}
The following problem is a twist on the {Fifteen-Puzzle}
problem that we did in class. Let $A$ be a given array containing the numbers $1,\dots,n$ in some order.  A pair of
integers in $A$ is called an \emph{out-of-order pair} when the first element of
the pair both comes \emph{earlier} in the array, and \emph{is larger}, than
the second element of the pair. For example, the array $A = (1,2,4,5,3)$ has two out-of-order pairs:
$(4,3)$ and $(5,3)$. 

We are allowed to swap elements in $A$ using a rule called
\textsc{Rotate-Triple}. In each time step,
\textsc{Rotate-Triple} chooses any sub-array of three {\em consecutive}
elements, and rotates them such that the smallest of the three is
placed in the first position. For example, if $A = (1,2,4,5,3)$ and
\textsc{Rotate-Triple} chooses the sub-array
$(4,5,3)$, then we obtain the (new) array $A' = (1, 2, 3, 4, 5)$. On the
other hand, if \textsc{Rotate-Triple} chooses the sub-array $(1,2,4)$,
then the smallest of the three is already in the first position and the new array $A'$ is the same as $A$.

A state machine model of this problem can be described as follows: the
state at any time instant is a list of numbers in the array in order.
The start state is a list of the numbers in the initial (given) array $A$.

\begin{problemparts}

\problempart Write out the set of transitions that describe the
actions of $\textsc{Rotate-Triple}$ at any time step.

\hint You will need to consider six different cases.

\begin{solution}

Consider any consecutive triple in $A$ containing the elements $\{ a,b,c
\}$ in some order. Without loss of generality, assume that $a < b < c$. The
six different cases correspond to the different permutations of
$\{a,b,c\}$. Explicitly, the set of transitions can be described as follows:
\begin{align*}
\textsc{Rotate-Triple}(a,b,c) &\movesto (a,b,c) \\
\textsc{Rotate-Triple}(a,c,b) &\movesto (a,c,b) \\
\textsc{Rotate-Triple}(b,c,a) &\movesto (a,b,c) \\
\textsc{Rotate-Triple}(b,a,c) &\movesto (a,c,b) \\
\textsc{Rotate-Triple}(c,b,a) &\movesto (a,c,b) \\
\textsc{Rotate-Triple}(c,a,b) &\movesto (a,b,c).
\end{align*}

\end{solution}

\examspace[1in]

\problempart 
Denote the \emph{number} of out-of-order pairs in $A$ by the function
$f(A)$. For example, for the array $A = (1,2,4,5,3)$, the value of $f(A)$ is 2.

Argue that for the state machine model described above, the derived variable $f$ is \emph{weakly decreasing}.

\begin{solution}

Consider some triple containing elements $a < b < c$. At each
time-step, there are six possible transitions. Two of the transitions
--- $(a,b,c) \movesto (a,b,c)$ and $(a,c,b) \movesto (a,c,b)$ ---
leave the state unchanged and therefore $f$ remains the same. 

The transition $(b,a,c) \movesto (a,c,b)$ removes one
out-of-order pair $(b,a)$, but introduces a {\em new} out-of-order
pair $(c,b)$.  therefore resulting in no change in $f$. 

The transitions $(b,c,a) \movesto (a,b,c)$ and $(c,a,b) \movesto (a,b,c)$
both \emph{decrease} the number of out-of-order pairs from 2 to 0. 

Finally, the transition $(c,b,a) \movesto (a,c,b)$ decreases the
number of out-of-order pairs from 3 to 1.

In each of the cases, $f$ is either constant or reduces in value. In other words, $f$ is weakly decreasing.

\end{solution}

\examspace[1in]

\problempart  Define the \emph{parity} of a list $L$ to be 0 or 1, depending on whether the number of
out-of-order pairs in $L$ is even or odd. For example, the parity of the list $L = (1,2,4,5,3)$ is 0
(since $f(L)=2$). 

Show that the parity of a state is preserved under transitions.  

\begin{solution}

This part follows directly from the argument that $f$ is weakly
decreasing. We have shown that in each transition, $f$ changes  by a
value of 0 or 2. Therefore, the parity of $f$ remains unchanged.

\end{solution}

\examspace[1.5in]

\problempart Argue that the procedure ensures that all
of the numbers will eventually end up in their ``correct'' place, {except possibly
  the last two numbers $n-1$ and $n$}.

\begin{solution}
\TBA{\textbf{FILL IN.}}
\end{solution}

\end{problemparts}

\end{problem}

%%%%%%%%%%%%%%%%%%%%%%%%%%%%%%%%%%%%%%%%%%%%%%%%%%%%%%%%%%%%%%%%%%%%%
% Problem ends here
%%%%%%%%%%%%%%%%%%%%%%%%%%%%%%%%%%%%%%%%%%%%%%%%%%%%%%%%%%%%%%%%%%%%%

\endinput
