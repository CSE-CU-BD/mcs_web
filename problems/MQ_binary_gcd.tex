\documentclass[problem]{mcs}

\begin{pcomments}
  \pcomment{MQ_binary_gcd}
  \pcomment{lighter version of PS_binary_gcd}
\end{pcomments}

\pkeywords{
  preserved_invariant
  GCD
  binary_GCD
  state_machine
}

%%%%%%%%%%%%%%%%%%%%%%%%%%%%%%%%%%%%%%%%%%%%%%%%%%%%%%%%%%%%%%%%%%%%%
% Problem starts here
%%%%%%%%%%%%%%%%%%%%%%%%%%%%%%%%%%%%%%%%%%%%%%%%%%%%%%%%%%%%%%%%%%%%%

\begin{problem}
  The following binary-GCD state machine computes the GCD of integers $a>b>0$:

\begin{align*}
\text{states} & \eqdef \naturals^3\\
\text{start state} & \eqdef (a, b, 1)\\
\text{transitions} & \eqdef \text{ if } \min(x,y) > 0, \text{ then } (x,  y, e) \movesto\\
     &\qquad \text{the first possible state according to the rules:}\\
     &\qquad
       \begin{cases}
       (1, 0, ex)     & \text{(if $x = y$)}\\
       (1, 0, e)      & \text{(if $y = 1$)},\\
       (x/2, y/2, 2e) & \text{(if $2 \divides x$ and $2 \divides y$)},\\
       (x/2, y, e)    & \text{(if $2 \divides x$)}\\
       (x, y/2, e)    & \text{(if $2 \divides y$)}\\
       (y, x, e)      & \text{(if $y>x$)}\\
       (x-y,y,e)      & \text{(otherwise)}.
       \end{cases}
\end{align*}

The predicate
\[
\gcd(a,b) = e\gcd(x,y)
\]
is claimed to be a preserved invariant of this state machine.

\bparts 

\ppart Verify that this predicate is a preserved invariant for the
3rd: $(x/2,y/2,e)$, 4th: $(x/2,y,e)$ and last: $(x-y,y,e)$ of the
above transition rules.

\begin{solution}
To verify preserved invariance, we assume the
invariant holds for state $(x,y,e)$ and show that if $(x,y,e) \movesto
(x',y',e')$, then $\gcd(a,b) = e'\gcd(x',y')$.

The proof is by cases according to which kind of transition occurs.

\inductioncase{Case}: ($2 \divides x$ and $2 \divides y$). 
So $(x',y',e')= (x/2, y/2, 2e)$.

We use the easily verified fact that $\gcd(au,av) = a\gcd(u,v)$.
In this case, we have $\gcd(x,y) = 2\gcd(x/2,y/2)$.  Since the
invariant holds for $(x,y,e)$, we conclude that
\begin{equation}\label{gcdabe2x2}
\gcd(a,b) = e 2\gcd(x/2,y/2).
\end{equation}
\begin{align*}
e'\gcd(x',y')
 & = 2e\gcd(x/2,y/2)\\
& = \gcd(a,b) & \text{(by~\eqref{gcdabe2x2})},
\end{align*}
which shows that the invariant holds for $(x',y',e')$.

\inductioncase{Case}: $2 \divides x$ and 2 does not divide $y$.
So $(x',y',e') = (x/2,y, e)$.

In this case, we have
\begin{equation}\label{gcdxyx2}
\gcd(x,y) = \gcd(x/2,y),
\end{equation}
So
\begin{align*}
\gcd(a,b)
  & = e\gcd(x,y)
      & \text{(invariant for $(x,y,e)$)}\\
  & = e\gcd(x/2,y)
      & \text{(by~\eqref{gcdxyx2})}\\
  & = e'\gcd(x',y'),
\end{align*}
proving that the invariant holds for $(x',y',e')$.

\inductioncase{Case}: (otherwise clause).
So $(x',y',e') = (x-y,y,e)$,

We use the easily verified fact that
\begin{equation}\label{gcdu-v}
\gcd(u-v,v) = \gcd(u,v).
\end{equation}
This gives,
\begin{align*}
\gcd(a,b)
  & = e\gcd(x,y) & \text{(invariant for $(x,y,e)$)}\\
  & = e\gcd(x-y,y) & \text{(by~\eqref{gcdu-v})}\\
  & = e'\gcd(x',y'),
\end{align*}
proving that the invariant holds for $(x',y',e')$.
\end{solution}

\ppart The definition of the transition rules implies that there are
only two ways to reach a stopped state:
\begin{align}
(x,x,e) & \movesto (1, 0, ex) \label{xxe10ex}\\
(x,1,e) & \movesto (1, 0, e)  \label{x1e10e}.
\end{align}
You may assume this fact.

Use the Invariant Principle to conclude that if this machine reaches a
stopped state $(1,0,e')$, then $e' = \gcd(a,b)$.

\begin{solution}
We first observe that the preserved invariant holds trivially in the
start state $(a,b,1)$ because $\gcd(a,b) = 1\cdot\gcd(a,b)$.

By the Invariant Principle, we conclude that the preserved invariant
holds in every reachable state.  So if a stopped state is reached as
in case~\eqref{xxe10ex}, we have
\begin{align*}
\gcd(a,b)
   & = e\gcd(x,x)
       & \text{(invariant for $(x,x,e)$)}\\
   & = e x \\
   & = e',
\end{align*}
as required.

In case~\eqref{x1e10e}, we have
\begin{align*}
\gcd(a,b)
   & = e\gcd(x,1)
       & \text{(invariant for $(x,1,e)$)}\\
   & = e\cdot 1\\
   & = e = e',
\end{align*}
as required.
\end{solution}

\eparts

\end{problem}

%%%%%%%%%%%%%%%%%%%%%%%%%%%%%%%%%%%%%%%%%%%%%%%%%%%%%%%%%%%%%%%%%%%%%
% Problem ends here
%%%%%%%%%%%%%%%%%%%%%%%%%%%%%%%%%%%%%%%%%%%%%%%%%%%%%%%%%%%%%%%%%%%%%

\endinput
