\documentclass[problem]{mcs}

\begin{pcomments}
  \pcomment{MQ_stirlings_asymptotics}
  \pcomment{from miniquiz-4-20}
\end{pcomments}

\pkeywords{
   Stirlings
   Asymptotics
}

%%%%%%%%%%%%%%%%%%%%%%%%%%%%%%%%%%%%%%%%%%%%%%%%%%%%%%%%%%%%%%%%%%%%%
% Problem starts here
%%%%%%%%%%%%%%%%%%%%%%%%%%%%%%%%%%%%%%%%%%%%%%%%%%%%%%%%%%%%%%%%%%%%%
\begin{problem}

Show that 
\[
\ln (n^2!) = \Theta(n^2 \ln n)
\]


\begin{solution}By Stirling's formula:
\[
n^2! \sim \sqrt{2 \pi n^2} \left(\frac{n^2}{e}\right)^{n^2}
\]
Taking logarithms gives:
%
\begin{align*}
\ln(n^2!)
    & \sim \ln(\sqrt{2 \pi n^2} \left(\frac{n^2}{e}\right)^{n^2}) \\
    & = \ln(\sqrt{2 \pi n^2}) + \ln\left(\frac{n^2}{e}\right)^{n^2} \\
    & = \frac{1}{2}\ln 2\pi + \ln n + n^2 \ln (\frac{n^2}{e}) \\
    & = \frac{1}{2}\ln 2\pi + \ln n + n^2(2 \ln n - 1)
\end{align*}
%
It is then easy to see that this expression and $n^2 \ln n$ are big-O of each other, so we conclude that $\ln (n^2!) = \Theta(n^2 \ln n)$.
\end{solution}

\end{problem}

\endinput
