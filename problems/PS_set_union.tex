\documentclass[problem]{mcs}

\begin{pcomments}
  \pcomment{PS_set_union}
  \pcomment{from: F09.ps2, S06.ps1}
\end{pcomments}

\pkeywords{
  set_theory
  set_equality
  proof
  iff
  cases
}

%%%%%%%%%%%%%%%%%%%%%%%%%%%%%%%%%%%%%%%%%%%%%%%%%%%%%%%%%%%%%%%%%%%%%
% Problem starts here
%%%%%%%%%%%%%%%%%%%%%%%%%%%%%%%%%%%%%%%%%%%%%%%%%%%%%%%%%%%%%%%%%%%%%

\begin{problem}

Let $A$, $B$ and $C$ be sets.
Prove that:
\begin{equation}\label{ABCA-B}
A \union B \union C = (A - B) \union (B - C) \union (C - A) \union (A \intersect B \intersect C).
\end{equation}
%
\iffalse
You are welcome to use a diagram to aid your own reasoning, but your
proof must be text.
\fi


\hint $P \QOR Q \QOR R$ is equivalent to
\[
(P \QAND \bar{Q}) \QOR (Q \QAND \bar{R}) \QOR (R \QAND \bar{P}) \QOR (P \QAND Q \QAND R).
\]

\begin{editingnotes}
Refine hint to ask for IFF proof as in slides
\end{editingnotes}

\begin{solution}

\begin{proof}
We prove that an element $x$ is a member of the left-hand side
of~\eqref{ABCA-B} iff it is a member of the right-hand side.

\begin{align*}
\lefteqn{x \in A \union B \union C}\\
 & \qiff (x \in A) \QOR (x \in B) \QOR (x \in C)
           & \text{(by def of $\union$)}\\
& \qiff ((x \in A) \QAND \bar{(x \in B)})\ \QOR\\
& \qquad ((x \in B) \QAND \bar{(x \in C)})\ \QOR\\
& \qquad ((x \in C) \QAND \bar{(x \in A)})\ \QOR \\
& \qquad ((x \in A) \QAND (x \in B) \QAND (x \in C)) & \text{(by the equivalence in the Hint)}\\
& \qiff (x \in A - B) \QOR (x \in B - C) \QOR\\
& \qquad\qquad (x \in C - A)\ \QOR (x \in A \intersect B \intersect C)
          & \text{(by def of $-$, $\intersect$)}\\
& \qiff x \in (A - B) \union (B - C) \union (C - A) \union\\
& \qquad\qquad (A \intersect B \intersect C)
          & \text{(by def of $\union$)}
\end{align*}
\end{proof}

\textbf{Alternative solution by cases:}

  We prove that the left-hand side is contained in the right-hand
  side, and that the right-hand side is contained in the left-hand
  side.

First, we show that the left-hand side is contained in the right-hand
on the right.

Next, we show that the right-hand side is contained in the left-hand.
This is easier.  Let $x$ belong to the right side. Then it belongs to
one of $A - B$, $B - C$, $C - A$, or $A\intersect B\intersect C$.  In
the first case, we clearly know $x\in A$. In the second case, $x\in
B$. In the third case, $x\in C$. In the last case, $x\in A$ again. So,
in all cases, $x$ belongs to one of $A$, $B$ or $C$. So $x$ belongs to
the left-hand side. Therefore, the set on the right is contained in
the set on the left.

Since each set is contained in the other, they are equal.

\end{solution}

\end{problem}

%%%%%%%%%%%%%%%%%%%%%%%%%%%%%%%%%%%%%%%%%%%%%%%%%%%%%%%%%%%%%%%%%%%%%
% Problem ends here
%%%%%%%%%%%%%%%%%%%%%%%%%%%%%%%%%%%%%%%%%%%%%%%%%%%%%%%%%%%%%%%%%%%%%

\endinput
