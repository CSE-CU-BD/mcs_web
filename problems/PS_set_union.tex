\documentclass[problem]{mcs}

\begin{pcomments}
  \pcomment{PS_set_union}
  \pcomment{from: S17.ps2, F09.ps2, S06.ps1}
  \pcomment{edited to refer to method from text, ARM 2/11/17}
\end{pcomments}

\pkeywords{
  set_theory
  set_equality
  proofn
  iff
  cases
}

%%%%%%%%%%%%%%%%%%%%%%%%%%%%%%%%%%%%%%%%%%%%%%%%%%%%%%%%%%%%%%%%%%%%%
% Problem starts here
%%%%%%%%%%%%%%%%%%%%%%%%%%%%%%%%%%%%%%%%%%%%%%%%%%%%%%%%%%%%%%%%%%%%%

\begin{problem}
Let $A$, $B$ and $C$ be sets. Prove that 
\begin{equation}\label{ABCA-B}
A \union B \union C = (A - B) \union (B - C) \union (C - A) \union (A \intersect B \intersect C)
\end{equation}
using a chain of \QIFF's \inbook{as
  Section~\ref{set_equality_sec}.}\inhandout{to reduce the
  set-theoretic equality to a propositional equivalence, as in the
  text.}

%
\iffalse
You are welcome to use a diagram to aid your own reasoning, but a diagram is not a
proof.
\fi

\begin{solution}

\begin{proof}
We prove that an element $x$ is a member of the set described on the
left hand side of equality~\eqref{ABCA-B} iff it is a member of the
set described on the right hand side.

The key step in the proof uses the fact that the following two
propositional formulas are equivalent.
\begin{align}P \QOR  Q & \QOR R,\label{PQQQR}\\
\text{and}\notag\\
(P \QAND & \bar{Q}) \QOR (Q \QAND \bar{R}) \QOR (R \QAND \bar{P})
      \QOR (P \QAND Q \QAND R).\label{PbQQQbR}
\end{align}
There are multiple ways to verify this equivalence, and we will take
it for granted.

\begin{align*}
\lefteqn{x \in A \union B \union C}\\
 & \QIFF (x \in A) \QOR (x \in B) \QOR (x \in C)
           & \text{(by def of $\union$)}\\
& \QIFF  ((x \in A) \QAND \bar{x \in B})\ \QOR\\
& \qquad ((x \in B) \QAND \bar{x \in C})\ \QOR\\
& \qquad ((x \in C) \QAND \bar{x \in A})\ \QOR \\
& \qquad ((x \in A) \QAND (x \in B) \QAND (x \in C))
            & \text{(equivalence of~\eqref{PQQQR} and~\eqref{PbQQQbR})}\\
& \QIFF (x \in A - B) \QOR (x \in B-C) \QOR (x \in C - A)\\
& \qquad\qquad  \QOR\ (x \in A \intersect B \intersect C)
          & \text{(by def of $-$, $\intersect$)}\\
& \QIFF x \in (A - B) \union (B - C) \union (C - A)\\
& \qquad\qquad  \union (A \intersect B \intersect C)
          & \text{(by def of $\union$)}
\end{align*}
\end{proof}

\iffalse
\textbf{Alternative solution by cases:}

  We prove that the set described on the left hand side of
  equality\eqref{ABCA-B} is contained in the set on the right hand
  side.  Then we show conversley, that the set described on right hand
  side is contained in the left hand side.

First, we show that the set on the left hand side is contained in the
right hand side. \TBA{left in right}

Next, we show that the right-hand side is contained in the left-hand.
This is easier.  Let $x$ belong to the right side.  Then it belongs to
one of $A - B$, $B - C$, $C - A$ or $A\intersect B\intersect C$.  In
the first case, we clearly know $x\in A$. In the second case, $x\in
B$.  In the third case, $x\in C$. In the last case, $x\in A$ again.
So, in all cases, $x$ belongs to one of $A$, $B$ or $C$.  So $x$
belongs to the left-hand side.  Therefore, the set on the right is
contained in the set on the left.

Since each set is contained in the other, they are equal.
\fi

\end{solution}

\end{problem}

%%%%%%%%%%%%%%%%%%%%%%%%%%%%%%%%%%%%%%%%%%%%%%%%%%%%%%%%%%%%%%%%%%%%%
% Problem ends here
%%%%%%%%%%%%%%%%%%%%%%%%%%%%%%%%%%%%%%%%%%%%%%%%%%%%%%%%%%%%%%%%%%%%%

\endinput
