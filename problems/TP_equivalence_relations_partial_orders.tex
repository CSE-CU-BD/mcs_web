\documentclass[problem]{mcs}

\begin{pcomments}
  \pcomment{TP_equivalence_relations_partial_orders}
  \pcomment{renamed from TP_Equivalence_relations__Partial_Orders}
  \pcomment{Converted from ./00Convert/probs/practice3/prob6.scm
              by scmtotex and drewe
              on Tue 19 Jul 2011 04:11:29 PM EDT}
\end{pcomments}

\begin{problem}

%% type: short-answer
%% title: Equivalence relations & Partial Orders

For each of the following relations, indicate
whether it is an equivalence relation, a partial but
\emph{not} a total order, a total order, or none of the
above.
\bparts


\ppart
$\{(p,q) | p\text{ and }q\text{ are people
   of the same age}\}$

\begin{solution}

Equivalence relation.

\end{solution}

\ppart
$\{(a,b) | a\text{ is the age of someone who is not younger than anyone of age }b\}$

\begin{solution}

Total order.

\end{solution}

\ppart
$\{(p,q) | p\text{ is a person whose age is an
integer multiple of person }q\text{'s age}\}$

\begin{solution}

None of the above.


Two different people can be the same age, so
the relation is not antisymmetric, ruling out partial and total order.  It
is not symmetric, since a 4-year-old is related to a 2-year-old, but not
conversely, ruling out equivalence.  Note that as a relation on their
\emph{ages}, this would be the same as the divisibility relation on
nonnegative integers, which \emph{is} a partial order.
\end{solution}

\eparts


\end{problem}

\endinput
