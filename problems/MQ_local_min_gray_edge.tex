\documentclass[problem]{mcs}

\begin{pcomments}
  \pcomment{MQ_local_min_gray_edge}
  \pcomment{subsumes FP_MST_min_out, and _S16}
  \pcomment{ARM 11/2/17}
\end{pcomments}

\pkeywords{
  tree
  spanning_tree
  MST
  weight
  minimum
  cycle
}

%%%%%%%%%%%%%%%%%%%%%%%%%%%%%%%%%%%%%%%%%%%%%%%%%%%%%%%%%%%%%%%%%%%%%
% Problem starts here
%%%%%%%%%%%%%%%%%%%%%%%%%%%%%%%%%%%%%%%%%%%%%%%%%%%%%%%%%%%%%%%%%%%%%

\begin{problem}
Let $G$ be a connected weighted simple graph and let $v$ be a vertex
of $G$.  Suppose $e$ is an edge incident to $v$, that is, $g=
\edge{v}{w}$ for some vertex $w \neq v$, and $e$ is strictly smaller
than the weight of every other edge incident to $v$.  

\bparts

\ppart Suppose $T$ is a minimum weight spanning tree of $G$.  Explain
why the Gray Edge Lemma\inbook{~\bref{lem:necessary-gray}} immediately implies
that $e$ is an edge of $T$.

\examspace[0.7in]

\ppart Impulsive Ira says, ``This is great.  Assuming as usual that
all edge-weights are different, we can forget all this gray edge stuff
and just form the min-weight spanning tree by taking the min-weight
edge out of each vertex.  Since every vertex contributes an edge, we
will wind up with the min-weight spanning tree---a lot simpler than
Prim/Kruskal/\dots.''

Give a simple counter-example to Ira's claim.  Explain to him his
mistake in one sentence.

\examspace[1.0in]

\begin{solution}
Let $\vertices{G} \eqdef \set{a,b,c,d}$, $\edges{G} \eqdef
\set{\edge{a}{b}, \edge{c}{d}, \edge{b}{c}} $, with $\edge{a}{b}$ and
$\edge{c}{d}$ of weight one, and $\edge{b}{c}$ of weight two.  So
selecting the min-weight edges out of each vertex just gives the two
edges of weight one.  ``Ira, the problem is that even though each
vertex `contributes' an edge to the tree, two vertices may share the
same min-weight edge, so you may only get half as many edges as needed
for a spanning tree.''
\end{solution}

\begin{staffnotes}
This last part for in-class only:

\ppart Give a direct proof that $e$ is an edge of $T$ without appeal
to ``gray edges.''

\begin{solution}
Suppose to the contrary that $e$ is not in $T$.  There is a
path $\vec{p}$ in $T$ from $v$ to the other endpoint $w$ of $e$.  This
path must start with some edge $f \eqdef \edge{v}{u}$.

Now $\vec{p} + e$ is a cycle, so when we remove $f$, the endpoints of
$f$ remain connected by the rest of the cycle.  This means that $T - f
+ e$ remains connected.  Therefore $T-f+e$ is also a spanning tree of
$G$ since it has the same number of edges as $T$.

Further, since the weight of $e$ is less than than the weight of $f$,
the weight of $T - f + e$ is less than the weight of $T$,
contradicting the minimality of $T$.
\end{solution}
\end{staffnotes}

\eparts

\end{problem}

%%%%%%%%%%%%%%%%%%%%%%%%%%%%%%%%%%%%%%%%%%%%%%%%%%%%%%%%%%%%%%%%%%%%%
% Problem ends here
%%%%%%%%%%%%%%%%%%%%%%%%%%%%%%%%%%%%%%%%%%%%%%%%%%%%%%%%%%%%%%%%%%%%%
\endinput
