\documentclass[problem]{mcs}

\begin{pcomments}
  \pcomment{PS_VG}
  \pcomment{formerly CP_VG}
  \pcomment{revision of PS_50_point_games, PS_value_games by ARM 3/4/16, revised 3/8/17}
  \pcomment{S16.ps4, S17.cp5f}
\end{pcomments}

\pkeywords{
  recursive_data
  structural_induction
  games
  perfect_information
  max-value
  min-value
}

%%%%%%%%%%%%%%%%%%%%%%%%%%%%%%%%%%%%%%%%%%%%%%%%%%%%%%%%%%%%%%%%%%%%%
% Problem starts here
%%%%%%%%%%%%%%%%%%%%%%%%%%%%%%%%%%%%%%%%%%%%%%%%%%%%%%%%%%%%%%%%%%%%%

\newcommand{\VG}{\ensuremath{\text{VG}}}

\begin{problem}
We're going to characterize a large category of games as a recursive
data type and then prove, by structural induction, a fundamental
theorem about game strategies.  We are interested in two person games
of perfect information that end with a numerical score.  Chess and
Checkers would count as value games using the values $1,-1, 0$ for a
win, loss or draw for the first player.  The game of Go really does
end with a \emph{score} based on the number of white and black stones
that remain at the end.

Here's the formal definition:
\begin{definition*}
Let $V$ be a nonempty set of real numbers.  The class \VG\ of
\emph{$V$-valued two-person deterministic games of perfect
  information} is defined recursively as follows:

\inductioncase{Base case}:
A value $v \in V$ is a \VG\ known as a \emph{payoff}.

\inductioncase{Constructor case}: If $G$ is a nonempty set of \VG's,
then $G$ is a $\VG$.  Each game $M \in G$ is called a possible
\emph{first move} of $G$.
\end{definition*}

\begin{staffnotes}
In all the games like this that we're familiar with, there are only a
finite number of possible first moves and a bound on the possible
length of play.  It's worth noting that the definition of \VG\ does
not require this.  An infinite \VG\ might have plays of every finite
length, but it can't have an infinite play.  Since finiteness is not
needed to prove any of the results below, it would potentially even be
misleading to assume it.  Later, we'll suggest how games with an
infinite number of possible first moves might come up.

Infinite games do have their uses in the study of set theory and logic.
\end{staffnotes}

A \emph{strategy} for a player is a rule that tells the player which
move to make whenever it is their turn.  That is, a strategy is a
function $s$ from games to games with the property that $s(G) \in G$
for all games $G$.  Given which player has the first move, a pair of
strategies for the two players determines exactly which moves the
players will choose.  So the strategies determine a unique play of the
game and a unique payoff.\footnote{We take for granted the fact that
  no \VG\ has an infinite play.  The proof of this by structural
  induction is essentially the same as that for win-lose games given
  \inbook{in Lemma~\bref{goutcom}.} \inhandout{in the text.}}

The \emph{max-player} wants a strategy that guarantees as high a
payoff as possible, and the \emph{min-player} wants a strategy that
guarantees as low a payoff as possible.

The Fundamental Theorem for deterministic games of perfect information
says that in any game, each player has an optimal strategy, and these
strategies lead to the same payoff.  More precisely,

\begin{theorem*}[Fundamental Theorem for \VG's]
Let $V$ be a finite set of real numbers and $G$ be a $V$-valued \VG.
Then there is a value $v \in V$, called a \term{max-value}
$\text{max}_G$ for $G$, such that if the \emph{max-player moves
  first},
\begin{itemize}

\item the max-player has a strategy that will finish with a payoff of
  \emph{at least} $\text{max}_G$, no matter what strategy the min-player
  uses, and

\item the min-player has a strategy that will finish with a payoff of
  \emph{at most} $\text{max}_G$, no matter what strategy the max-player
  uses.
\end{itemize}
\end{theorem*}

It's worth a moment for the reader to observe that the definition of
$\text{max}_G$ implies that if there is one for $G$, it is unique.  So
if the max-player has the first move, the Fundamental Theorem means
that there's no point in playing the game: the min-player may just as
well pay the max-value to the max-player.

\bparts

\ppart Prove the Fundamental Theorem for \VG's.

\hint \VG's are a recursively defined data type, so the basic method
for proving that all \VG's have some property is structural induction
on the definition of \VG.  Since the min-player moves first in
whichever game the max-player picks for their first move, the
induction hypothesis will need to cover that case as well.

\begin{solution}
The proof is by structural induction on the definition of a game $G
\in \VG$.  The induction hypothesis $P(G)$ will be that
\[
P(G) \eqdef G\ \text{has both a max-value $\text{max}_G$ and a min-value $\text{min}_G$},
\]
where a min-value for $G$ is defined in the same way as a max-value
except that max and min are switched.  To be precise, a
\emph{min-value} for $G$ is a value $\text{min}_G \in V$ such that if the
\emph{min-player moves first},
\begin{itemize}
\item the min-player has a strategy that will finish with a payoff of
  \emph{at most} $\text{min}_G$, no matter what strategy the max-player
  uses, and

\item the max-player has a strategy that will finish with a payoff of
  \emph{at least} $\text{min}_G$, no matter what strategy the min-player
  uses.
\end{itemize}

In general, the max-value and the min-value for $G$ are different,
since playing $G$ with the max-player moving first is quite different
from playing $G$ with the min-player moving first.

\begin{staffnotes}
Add problem: Prove that the max-value of $G$ is always $\geq$ the
min-value.
\end{staffnotes}
\begin{proof}

\inductioncase{Base case}: ($G = v \in V$).  The max-value and
min-value will be $v$.  The strategy for the players---regardless of
who moves first---is ``Suck it up, the game is over and you have to
take $v$ as your payoff.''  This proves $P(G)$.

\inductioncase{Constructor case}: ($G$ is a nonempty set of \VG's).
By structural induction we may assume that each $M \in G$ has a a
min-value $\text{min}_M$. Define
\[
v_\text{max} \eqdef \max\set{\text{min}_M \suchthat M \in G}.
\]
This $\text{max}$ will exist because $V$ is finite.

We claim that $\text{max}_G = v_\text{max}$.  To prove this, we begin
by describing a strategy with the max-player moving first that
guarantees a payoff of at least $v_\text{max}$.  Namely, the
max-player's first move should be to some $M \in G$ such that
$\text{min}_M = v_\text{max}$.  After this first move, the max-player
has a strategy that, by definition of $\text{min}_M$, guarantees a
payoff off of at least $\text{min}_M = v_\text{max}$ in the game $M$
where the min-player moves first.  The max-player should then use this
strategy in the game $M$.

Moreover, with the max-player moving first, the min-player has a
strategy for $G$ that guarantees a payoff of at most $v_\text{max}$.
Namely, if the first player's first move is to $M \in G$, the
min-player has a strategy that, by definition of $\text{min}_M$,
guarantees a payoff of at most $\text{min}_M$ in the game $M$ where
the min-player moves first.  The min-player should then use this
strategy in the game $M$.  Since by definition $v_\text{max} \geq
\text{min}_M$, this strategy guarantees the min-player a payoff in $G$
of at most $v_\text{max}$.

This proves that $v_\text{max}$ has the properties required of
$\text{max}_G$, and since $\text{max}_G$ is by definition unique, it
follows that $\text{max}_G = v_\text{max}$.

Reasoning as above with max and min switched similarly proves that $G$
has a min-value.  This proves $P(G)$, and we conclude by structural
induction that $P(G)$ is true for every $\VG$ $G$ with a finite number
of payoff values.  In particular every $G$ has a max-value, as
claimed.
\end{proof}
\end{solution}


\ppart \textbf{(OPTIONAL)}.  State some reasonable generalization of
the Fundamental Theorem to games with an infinite set $V$ of possible
payoffs.

\begin{solution}
A straighforward generalization is to redefine the max-value to be the
least upper bound of the values for which the max-player has a
guaranteed strategy.  With an infinite number of possible payoffs,
some games may now have an infinite max-player value, and even in the
case that the max-value is finite, it might not be guaranteed exactly
by player strategies.  Rather, for any $v < \text{max}_G$, when the
max-player moves first, there will be a strategy for the max-player
that guarantees a payoff of at least $v$, and there will be a strategy
for the min-player that guarantees a payoff of at most $v +
(\text{max}_G - v)/2$.
\end{solution}

\eparts

\end{problem}

\endinput
