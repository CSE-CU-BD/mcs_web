\documentclass[problem]{mcs}

\begin{pcomments}
  \pcomment{PS_VG}
  \pcomment{formerly CP_VG}
  \pcomment{revision of PS_50_point_games, PS_value_games by ARM 3/4/16, revised 3/8/17}
  \pcomment{S16.ps4, S17.cp5f}
\end{pcomments}

\pkeywords{
  recursive_data
  structural_induction
  games
  perfect_information
  max-value
  min-value
}

%%%%%%%%%%%%%%%%%%%%%%%%%%%%%%%%%%%%%%%%%%%%%%%%%%%%%%%%%%%%%%%%%%%%%
% Problem starts here
%%%%%%%%%%%%%%%%%%%%%%%%%%%%%%%%%%%%%%%%%%%%%%%%%%%%%%%%%%%%%%%%%%%%%

\newcommand{\VG}{\ensuremath{\text{VG}}}

\begin{problem}
We're going to characterize a large category of games as a recursive
data type and then prove, by structural induction, a fundamental
theorem about game strategies.  We are interested two person games of
perfect information that end with a numerical score.  Chess and
Checkers would count as value games using the values $1,-1, 0$ for a
win, loss or draw for the first player.  The game of Go really does
end with a \emph{score} based on the number of white and black stones
that remain at the end.

Here's the formal definition:
\begin{definition*}
Let $V$ be a nonempty set of real numbers.  The class \VG\ of
\emph{$V$-valued two-person deterministic games of perfect
  information} is defined recursively as follows:

\inductioncase{Base case}:
A value $v \in V$ is a \VG\ known as a \emph{payoff}.

\inductioncase{Constructor case}: If $G$ is a nonempty set of \VG's,
then $G$ is a $\VG$.  Each game $M \in G$ is called a possible
\emph{first move} of $G$.
\end{definition*}

\begin{staffnotes}
In all the games like this that we're familiar with, there are only a
finite number of possible first moves and a bound on the possible
length of play.  It's worth noting that the definition of \VG\ does
not require this.  An infinite \VG\ might have plays of every finite
length, but it can't have an infinite play.  Since finiteness is not
needed to prove any of the results below, it would potentially even be
misleading to assume it.  Later, we'll suggest how games with an
infinite number of possible first moves might come up.

Infinite games do have their uses in the study of set theory and logic.
\end{staffnotes}

A \emph{strategy} for a player is a rule that tells the player which
move to make whenever it is their turn.  That is, a strategy is a
function $s$ from games to games with the property that $s(G) \in G$
for all games $G$.

The \emph{max-player} wants a strategy that leads to as high a payoff
as possible, and the \emph{min-player} wants a strategy that leads to
as low a payoff as possible.  A pair of strategies for the two players
determines exactly which moves the players choose and so determines a
unique play of the game, depending on who moves first.

\begin{staffnotes}
Pedantic.  Maybe useful as exercise alone.

\begin{definition}
Suppose $s_\text{min}$ and $s_\text{max}$ are strategies for the
players in a game $G \in \VG$.

The play $P_{\mathbf{m}}(G, s_\mathbf{m}, s_{\mathbf{\bar{m}}})$ they
determine when player $\mathbf{m}$ moves first is defined recursively
on the definition of $\VG$:

\inductioncase{Base case}: If $G = v$, then
$P_\mathbf{m}(G, s_\mathbf{m}, s_{\bar{\mathbf{m}}})$ is the play
$\ang{v}$ of $G$.

\inductioncase{Constructor case}:  The play
$P_\mathbf{m}(G, s_\mathbf{m}, s_{\mathbf{\bar{m}}})$ is defined to be $G$
followed by the play determined by the two strategies in the game
$s_{\mathbf{m}}(G)$, except that now player $\bar{\mathbf{m}}$ moves
first.  More precisely,
\[
P_{\mathbf{m}}(G,s_{\mathbf{m}}, s_{\mathbf{\bar{m}}})
\eqdef
G \cdot P_{\bar{\mathbf{m}}}(s_{\mathbf{m}}(G), s_{\mathbf{\bar{m}}}, s_\mathbf{m}).
\]
\end{definition}
\end{staffnotes}

The Fundamental Theorem for deterministic games of perfect information
says that in any game, each player has an optimal strategy, and these
strategies lead to the same payoff.

\begin{theorem*}[Fundamental Theorem for \VG's]
Let $V$ be a finite set of real numbers and $G$ be a $V$-valued \VG.
Then there is a value $v \in V$ called \emph{the value of $G$} such
that
\begin{itemize}
\item the max-player has a strategy that, matched with \emph{any}
  min-player strategy, will define a play of $G$ that finishes with a
  payoff of at least $v$,

\item the min-player has a strategy that, matched with \emph{any}
  max-player strategy, will define a play of $G$ that finishes with a
  payoff of at most $v$.
\end{itemize}
\end{theorem*}

\bparts

\ppart Prove the Fundamental Theorem for \VG's.

\hint Assume by induction that each first move $M \in G$ has a value
$v_M$.

\begin{solution}
The proof is by structural induction on the definition of a $G \in
\VG$.  The induction hypothesis is that $G$ has a value.

\inductioncase{Base case}: ($G = v \in V$).  Then all strategies
finish with the value $v$, so $v$ is the value of $G$.

\inductioncase{Constructor case}: ($G =\mathcal{M}$).  By structural
induction we may assume that each $M \in \mathcal{M}$ has a value
$v_M$.

\inductioncase{Induction step} 
Suppose the $\mathbf{m}$-player moves first, where $\mathbf{m}$ is
\text{max} or \text{min}.  Let $\bar{\mathbf{m}}$ be whichever
of \text{max} or \text{max} differs from $\mathbf{m}$.

Define
\[
v \eqdef \mathbf{m}\set{v_M \suchthat M \in \mathcal{M}}.
\]
This \textbf{max} or \textbf{min} will exist because $V$ is finite.

Now a strategy for the $\mathbf{m}$-player when they move first that
finishes with a payoff of at most $v$ if $\mathbf{m} = \textbf{min}$
and at least $v$ if $\mathbf{m} = \textbf{max}$ is:

\begin{quote}
Choose a first move, $M$ such that $v = v_M$.  Now by structural
induction hypothesis, there is an $\mathbf{m}$-strategy for $M$ that
guarantees a finish with a value at most $v$ if $\mathbf{m} =
\textbf{min}$ and at least $v$ if $\mathbf{m} = \textbf{max}$.  Follow
that strategy.
\end{quote}
\end{solution}
\eparts

\begin{center}
\textbf{Supplemental Part (optional)}
\end{center}

\bparts

\ppart State some reasonable generalization of the Fundamental Theorem
to games with an infinite set $V$ of possible payoffs.  Prove your
generalization.

\begin{solution}
The obvious generalization would redefine the max-value as the lub of
the ensured values, and the min-value as the glb of the limits to
payoffs.  The result is that some games may now have a value $v$ that
is positive or negative infinity, and that $v$ can't exactly be
ensured or limted to, but rather that for any $\epsilon >0$ there will
be a strategy that ensures a value of at least $v - \epsilon$ and a
strategy that limits payoff to at most $v + \epsilon$.
\end{solution}

\eparts

\end{problem}

\endinput
