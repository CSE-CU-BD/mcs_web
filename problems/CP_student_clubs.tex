\documentclass[problem]{mcs}

\begin{pcomments}
  \pcomment{from: S09.cp6r}
  \pcomment{from: S07.cp6w (slightly edited/shortened)}
\end{pcomments}

\pkeywords{
  bipartite_matching
  degree-constrained
  Halls_Theorem
}

%%%%%%%%%%%%%%%%%%%%%%%%%%%%%%%%%%%%%%%%%%%%%%%%%%%%%%%%%%%%%%%%%%%%%
% Problem starts here
%%%%%%%%%%%%%%%%%%%%%%%%%%%%%%%%%%%%%%%%%%%%%%%%%%%%%%%%%%%%%%%%%%%%%

\begin{problem}
MIT has a lot of student clubs loosely overseen by the MIT Student
Association.  Each eligible club would like to delegate one of its members
to appeal to the Dean for funding, but the Dean will not allow a student to
be the delegate of more than one club.  Fortunately, the Association VP
took 6.042 and recognizes a matching problem when she sees one.

\bparts

\ppart  Explain how to model the delegate selection problem as a bipartite
matching problem.

\begin{solution}
Define a bipartite graph with the student clubs as one set of
vertices and everybody who belongs to some club as the other set of
vertices.  Let a club and a student be adjacent exactly when the student
belongs to the club.  Now a matching of clubs to students will give a
proper selection of delegates: every club will have a delegate, and every
delegate will represent exactly one club.
\end{solution}

\ppart The VP's records show that no student is a member of more than 9
clubs.  The VP also knows that to be eligible for support from the Dean's
office, a club must have at least 13 members.  That's enough for her to
guarantee there is a proper delegate selection.  Explain.  (If only the VP
had taken 6.046, \emph{Algorithms}, she could even have found a delegate
selection without much effort.)

\begin{solution}
The degree of every club is at least 13, and the degree of every
student is at most 9, so the graph is \emph{degree-constrained} (see the
Appendix) which implies there will be no bottlenecks to prevent a
matching.  Hall's Theorem then guarantees a matching.
\end{solution}

\eparts
\end{problem}

%%%%%%%%%%%%%%%%%%%%%%%%%%%%%%%%%%%%%%%%%%%%%%%%%%%%%%%%%%%%%%%%%%%%%
% Problem ends here
%%%%%%%%%%%%%%%%%%%%%%%%%%%%%%%%%%%%%%%%%%%%%%%%%%%%%%%%%%%%%%%%%%%%%
