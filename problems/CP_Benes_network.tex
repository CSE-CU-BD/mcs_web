\documentclass[problem]{mcs}

\begin{pcomments}
  \pcomment{CP_Benes_network}
  \pcomment{from: S09.cp7t}
\end{pcomments}

\pkeywords{
  networks
  Benes
  congestion
  routing
  coloring
  collision
}

%%%%%%%%%%%%%%%%%%%%%%%%%%%%%%%%%%%%%%%%%%%%%%%%%%%%%%%%%%%%%%%%%%%%%
% Problem starts here
%%%%%%%%%%%%%%%%%%%%%%%%%%%%%%%%%%%%%%%%%%%%%%%%%%%%%%%%%%%%%%%%%%%%%

\begin{problem}
The Bene\v{s} network has a max congestion of one---every permutation
can be routed in such a way that a single packet passes through each
switch.  Let's work through an example.  A diagram of the Bene\v{s}
network $B_3$ of size $N = 8$ \inhandout{is attached}\inbook{appears
  in Figure~\bref{fig:B_3}}.  The two subnetworks of size $N = 4$ are
marked.  We'll refer to these as the \emph{upper} and \emph{lower}
subnetworks.

\bparts

\ppart Now consider the following permutation routing problem:
%
\begin{align*}
\pi(0) & = 3 & \pi(4) & = 2 \\
\pi(1) & = 1 & \pi(5) & = 0 \\
\pi(2) & = 6 & \pi(6) & = 7 \\
\pi(3) & = 5 & \pi(7) & = 4
\end{align*}
Each packet must be routed through either the upper subnetwork or the
lower subnetwork.  Construct a graph with vertices numbered by
integers 0 to 7 and draw a \emph{dashed} edge between each pair of
packets that cannot go through the same subnetwork because a collision
would occur in the second column of switches.

\begin{solution}
See Figure~\ref{bhere}.
\iffalse
\begin{figure}[h]
\graphic[height=1in]{rec-const1}
\end{figure}
\fi
\end{solution}

\ppart Add a \emph{solid} edge in your graph between each pair of
packets that cannot go through the same subnetwork because a collision
would occur in the next-to-last column of switches.

\begin{solution}
See Figure~\ref{bhere}.

\begin{figure}[h]
\graphic[height=1.25in]{rec-const2}
\caption{Conflict Graph for $B_3$}
\label{bhere}
\end{figure}
\end{solution}

\ppart Assign colors red and blue to the vertices of your graph so
that vertices that are adjacent by either a dashed or a solid edge get
different colors.  Why must this be possible, regardless of the
permutation $\pi$?

\begin{solution}
This must be possible, because edges in a cycle are alternately dashed
and solid.  Thus, every cycle has even length, which implies that the
graph is 2-colorable.  A coloring is given in Figure~\ref{redblue}.

\begin{figure}[h]
\graphic[height=1.5in]{rec-const3}
\caption{Red-Blue coloring of the Conflict Graph}
\label{redblue}
\end{figure}

\end{solution}

\ppart Suppose that red vertices correspond to packets routed through
the upper subnetwork and blue vertices correspond to packets routed
through the lower subnetwork.  \inbook{Referring to the Bene\v{s}
network shown in Figure~\bref{fig:benes-recursive}, indicate}
\inhandout{On the attached copy of the Bene\v{s}
network, highlight} the first and last edge traversed by each packet.

\begin{solution}
See Figure~\ref{firstlast}.

\begin{figure}[h]
\graphic[width=3.5in]{rec-benes1}
\caption{Packet First and Last Edges}
\label{firstlast}
\end{figure}
\end{solution}

\ppart All that remains is to route packets through the upper and
lower subnetworks.  One way to do this is by applying the procedure
described above recursively on each subnetwork.  However, since the
remaining problems are small, see if you can complete all the paths
on your own.

\begin{solution}
See Figure~\ref{bthere}.

\begin{figure}[h]
\graphic[width=3.5in]{rec-benes2}
\caption{Congestion Free Routing}
\label{bthere}
\end{figure}

\end{solution}

\eparts
\end{problem}

\iffalse
\examspace
\begin{figure}
\graphic{benes-decomp}
\end{figure}

\begin{figure}[h]
\graphic{benes}
\end{figure}
\fi

%%%%%%%%%%%%%%%%%%%%%%%%%%%%%%%%%%%%%%%%%%%%%%%%%%%%%%%%%%%%%%%%%%%%%
% Problem ends here
%%%%%%%%%%%%%%%%%%%%%%%%%%%%%%%%%%%%%%%%%%%%%%%%%%%%%%%%%%%%%%%%%%%%%

\endinput
