\documentclass[problem]{mcs}

\begin{pcomments}
  \pcomment{PS_infinite_repeat_expectation}
  \pcomment{solution by Alex Arkhipov, MIT Math 5/4/11 in response to
    ARM query}
   \pcomment{DO NOT USE.  Replace by FTL writeup in terms of independent variables}
\end{pcomments}

\pkeywords{
  expectation
  infinite
}

%%%%%%%%%%%%%%%%%%%%%%%%%%%%%%%%%%%%%%%%%%%%%%%%%%%%%%%%%%%%%%%%%%%%%
% Problem starts here
%%%%%%%%%%%%%%%%%%%%%%%%%%%%%%%%%%%%%%%%%%%%%%%%%%%%%%%%%%%%%%%%%%%%%

\begin{problem}
Let $T_0, T_1, \dots$ be a sequence of mutually independent random
variables with the same distribution.  Let
\[
R \eqdef \min\set{k > 0 \suchthat T_k > T_0}.
\]

\bparts

\ppart Suppose the range of the $T_0$ is the set $\set{t_0 < t_1 < t_2
  < \cdots}$.  Explain why the following Theorem implies that
$\expect{R} = \infty$.

\begin{theorem*}%\label{sumpsumpp}
If $p_0 + p_{1}+p_{2}+\cdots=1$ and all $p_{i}\geq 0$, then the sum
\[
\Omega \eqdef \sum_{k \in \nngint} \frac{p_{k}}{p_{k+1}+p_{k+2}+\cdots}.
\]
diverges.
\end{theorem*}

\begin{solution}
Let $p_i \eqdef \pr{T_0 = t_i}$.  ...COMPLETE THIS
\end{solution}

\ppart Let
\[
S_{k} \eqdef p_{k}+p_{k+1}+\dots,
\]
and
\[
a_{k} \eqdef \frac{S_{k}}{S_{k+1}} - 1.
\]
Prove that
\begin{equation}\label{omegasum}
\Omega = \sum_{k \in \nngint} a_k.
\end{equation}

\begin{solution}
Note that $S_{k} - S_{k+1} = p_{k}$.  Then, the sum $\Omega$ is:
\begin{align*}
\Omega = \sum_{k \in \nngint} \frac{S_{k}-S_{k+1}}{S_{k+1}}
       = \sum_{k \in \nngint} \paren{\frac{S_{k}}{S_{k+1}} - 1}
       = \sum_{k \in \nngint} a_{k}.
\end{align*}
\end{solution}

\ppart\label{prodknak1}  Prove that
\[
\prod_{k \leq n} (a_{k}+1) =  \frac{1}{S_{n+1}}.
\]

\begin{solution}
\[
\prod_{k \leq n} (a_{k}+1) =   \frac{S_{0}}{S_{1}} \cdot
\frac{S_{1}}{S_{2}} \cdot \frac{S_{2}}{S_{3}} \cdots
\frac{S_{n-1}}{S_{n}} \cdot \frac{S_{n}}{S_{n+1}} =
\frac{S_{0}}{S_{n+1}}
\]
Also, $S_0 \eqdef 1$.
\end{solution}

\ppart Conclude from part~\eqref{prodknak1} that
\begin{equation}\label{prodakinfty}
\prod_{k \in \nngint} (a_{k}+1) = \infty.
\end{equation}

\begin{solution}
\begin{align}
\prod_{k \in \nngint} (a_{k}+1)
  & \eqdef \lim_{k \to \infty} \prod_{k \leq n} (a_{k}+1)\notag\\
  & = \lim_{k \to \infty} \frac{1}{S_{n+1}}  & \text{(by part~\eqref{prodknak1})}\notag\\
  & = \frac{1}{\lim_{k \to \infty} S_{n+1}}.\label{f1limksn1}
\end{align}
But $1 = \sum_{k \in \nngint} p_i = \paren{\sum_{k = 0}^n p_i} + S_{n+1}$
which implies that $\lim_{n \to \infty} S_{n+1} = 0$, and so the
limit~\eqref{f1limksn1} is infinite.
\end{solution}

\ppart Conclude that $e^{\Omega} = \infty$ and hence $\Omega =
\infty$. 

\begin{solution}
\begin{align*}
e^{\Omega} & =  \exp\left[\sum_{k} a_{k}\right] & \text{(by~\eqref{omegasum})}\\
          & = \prod_k e^{a_{k}} \geq \prod_k (a_{k}+1) & \text{($e^x > 1+x$
for $x \neq 0$)}\\
        & = \infty & \text{(by~\eqref{prodakinfty})},
\end{align*}
so $\Omega = \infty$. 
\end{solution}

\eparts 
\end{problem}

%%%%%%%%%%%%%%%%%%%%%%%%%%%%%%%%%%%%%%%%%%%%%%%%%%%%%%%%%%%%%%%%%%%%%
% Problem ends here
%%%%%%%%%%%%%%%%%%%%%%%%%%%%%%%%%%%%%%%%%%%%%%%%%%%%%%%%%%%%%%%%%%%%%

\endinput
