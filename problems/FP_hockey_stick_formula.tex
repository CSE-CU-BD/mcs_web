\documentclass[problem]{mcs}

\begin{pcomments}
  \pcomment{FP_hockey_stick_formula}
  \pcomment{CH, Spring '14}
\end{pcomments}

\pkeywords{
	binomial_coefficient
        induction
        Pascal
        triangle
}

%%%%%%%%%%%%%%%%%%%%%%%%%%%%%%%%%%%%%%%%%%%%%%%%%%%%%%%%%%%%%%%%%%%%%
% Problem starts here
%%%%%%%%%%%%%%%%%%%%%%%%%%%%%%%%%%%%%%%%%%%%%%%%%%%%%%%%%%%%%%%%%%%%%

\begin{problem}
\iffalse
In this problem, we prove a couple of identities involving binomial
coefficients.
\fi
\qquad
\bparts 

\ppart\label{kteamn} There are $n$ children at an amusement park, but for safety
reasons only a team of $k$ children can be chosen to ride the
roller-coaster.  Little Blaise is one of the $n$ children.  How many
different teams are possible that (i) \emph{include} Little Blaise?
(ii) \emph{do not include} Little Blaise?

\begin{solution}
(i) If Little Blaise is included then there are $n-1$ children and
  $k-1$ spots remaining.  Therefore, the number of choices is
  $\binom{n-1}{k-1}. $

ii) If Little Blaise is \emph{not} included then there are $n-1$
children and $k$ spots. Therefore, the number of choices is
$\binom{n}{k-1}$.
\end{solution}

\examspace[1.0in]

\ppart\label{pascalnk-1} Use part~\eqref{kteamn} to conclude \emph{Pascal's Triangle
  Identity}:
\begin{equation*}
\binom{n}{k} = \binom{n-1}{k} + \binom{n-1}{k-1},
%\insolutions{\label{eq:pascalid}}\instatements{\notag}
\end{equation*}
for all positive integers $k < n$.

\begin{solution}
Each team either contains Little Blaise (there are $\binom{n-1}{k-1}$
such teams) or does not (there are $\binom{n-1}{k}$ such teams).  By
the sum rule, the total number of possible teams is the sum of these
numbers.  But we also know that the number of ways of choosing $k$ out
of $n$ children is $\binom{n}{k}$.
\end{solution}

\examspace[1.5in]

\ppart Prove \textbf{by induction} the \emph{Hockey-Stick Identity}:
\begin{equation}
\binom{2}{2} + \binom{3}{2} + \cdots + \binom{n}{2} = \binom{n+1}{3}.\tag{H-S}
%\label{eq:hockeystick}
\end{equation}

\hint Use part~\eqref{pascalnk-1}.

\begin{solution}
\begin{proof}

Let the induction hypothesis $P(n)$ be equation (H-S).
%\eqref{eq:hockeystick}.

\inductioncase{Base case}: ($n=2$).  The left hand side of equation (H-S)
%\eqref{eq:hockeystick}
is $\binom{2}{2} = 1$ and the right hand is $\binom{3}{3} = 1$.

\inductioncase{Induction step}: Suppose that $P(m)$ is true for some
positive integer $m \geq 2$, that is,
\[
\binom{2}{2} + \binom{3}{2} + \cdots + \binom{m}{2} = \binom{m+1}{3}.
\]
Adding $\binom{m+1}{2}$ to both sides, we get:
\[
\binom{2}{2} + \binom{3}{2} + \cdots + \binom{m}{2} + \binom{m+1}{2} =
\binom{m+1}{3} + \binom{m+1}{2}.
\]
But if we substitute $k=3$ and $n=m+2$ in Pascals' Triangle Identity,
%\eqref{eq:pascalid},
we have:
\[
\binom{m+1}{3} + \binom{m+1}{2} = \binom{(m+2}{3}.
\]
Combining, we get
\[
\binom{2}{2} + \binom{3}{2} + \cdots + \binom{m}{2} + \binom{m+1}{2} =
\binom{m+2}{3},
\]
and therefore $P(m+1)$ is true. 
\end{proof}

\end{solution}

\eparts

\end{problem}

%%%%%%%%%%%%%%%%%%%%%%%%%%%%%%%%%%%%%%%%%%%%%%%%%%%%%%%%%%%%%%%%%%%%%
% Problem ends here
%%%%%%%%%%%%%%%%%%%%%%%%%%%%%%%%%%%%%%%%%%%%%%%%%%%%%%%%%%%%%%%%%%%%%

\endinput
