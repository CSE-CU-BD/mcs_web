\documentclass[problem]{mcs}

\begin{pcomments}
  \pcomment{FP_hockey_stick_formula}
  \pcomment{CH, Spring '14}
\end{pcomments}

\pkeywords{
	binomial coefficients
        Pascal's triangle
        algebra
        induction
}

%%%%%%%%%%%%%%%%%%%%%%%%%%%%%%%%%%%%%%%%%%%%%%%%%%%%%%%%%%%%%%%%%%%%%
% Problem starts here
%%%%%%%%%%%%%%%%%%%%%%%%%%%%%%%%%%%%%%%%%%%%%%%%%%%%%%%%%%%%%%%%%%%%%

\begin{problem}

In this problem, we prove a couple of interesting identities involving
binomial coefficients.

\bparts 

\ppart There are $n$ children at an amusement park, but for
safety reasons only a team of $k$ children can be chosen to ride the roller-coaster. Little Blaise is one
of the $n$ children. How many different teams are possible that
(i) \emph{include} Little Blaise? (ii) \emph{do not include} Little Blaise?

\begin{solution}
(i) If Little Blaise is included then there are $n-1$ children and
$k-1$ spots remaining. Therefore, the number of choices is
$\binom{n-1}{k-1}. $ 
 (ii) If Little Blaise is \emph{not} included then there are $n-1$
 children and $k$ spots. Therefore, the number of choices is $\binom{n}{k-1}$. 

\end{solution}

\examspace[1.5in]

\ppart Conclude the following statement (known as \emph{Pascal's Triangle
  Identity}): for all positive integers $n , k$ such
that $k \leq n$,
\begin{equation}
\binom{n-1}{k} + \binom{n-1}{k-1} = \binom{n}{k} .
\label{eq:pascalid}
\end{equation}

\begin{solution}
Each team either contains Little Blaise (there are $\binom{n-1}{k-1}$
such teams) or does not (there are
$\binom{n-1}{k}$ such teams). Therefore, by the sum rule, the total
number of possible teams equals $\binom{n-1}{k-1} + \binom{n-1}{k}.$
But we also know that the number of ways of choosing $k$ out of $n$
children is $\binom{n}{k}.$ Therefore, \eqref{eq:pascalid} follows.
\end{solution}

\examspace[1.5in]

\ppart \textbf{Using induction}, prove the following statement (known as the \emph{Hockey-Stick
  Identity}): for all positive integers $n \geq 2$, 
\begin{equation}
\binom{2}{2} + \binom{3}{2} + \ldots + \binom{n}{2} = \binom{n+1}{3}.
\label{eq:hockeystick}
\end{equation}

\hint Invoke Equation \eqref{eq:pascalid} .

\begin{solution}

Let $P(n)$ be the statement \eqref{eq:hockeystick}. 

\inductioncase{Base case}: ($n=2$). It is easy to see that
$\binom{2}{2} = 1 = \binom{3}{3}$. Therefore, $P(2)$ is true.

\inductioncase{Induction step}: Suppose that $P(n)$ is true for some
positive integer $m \geq 2$, i.e.,
\[
\binom{2}{2} + \binom{3}{2} + \ldots + \binom{m}{2} = \binom{m+1}{3}.
\]
Adding $\binom{m+1}{2}$ to both sides, we get:
\[
\binom{2}{2} + \binom{3}{2} + \ldots + \binom{m}{2} + \binom{m+1}{2} =
\binom{m+1}{3} + \binom{m+1}{2}.
\]
But if we substitute $k=3$ and $n=m+2$ in \eqref{eq:pascalid}, we have:
\[
\binom{m+1}{3} + \binom{m+1}{2} = \binom{m+2}{3}.
\]
Combining, we get
\[
\binom{2}{2} + \binom{3}{2} + \ldots + \binom{m}{2} + \binom{m+1}{2} =
\binom{m+2}{3},
\]
and therefore $P(m+1)$ is true. By induction, $P(n)$ is true for all
$n \geq 2$.

\end{solution}

\eparts

\end{problem}

%%%%%%%%%%%%%%%%%%%%%%%%%%%%%%%%%%%%%%%%%%%%%%%%%%%%%%%%%%%%%%%%%%%%%
% Problem ends here
%%%%%%%%%%%%%%%%%%%%%%%%%%%%%%%%%%%%%%%%%%%%%%%%%%%%%%%%%%%%%%%%%%%%%

\endinput
