\documentclass[problem]{mcs}

\begin{pcomments}
  \pcomment{FP_partial_order_short_answer_asymptotics}
  \pcomment{by ARM 5/20/12}
  \pcomment{based on FP_partial_order_short_answer and
    FP_asymptotics_define_functions}
\end{pcomments}

\pkeywords{
  partial_orders
  weak_partial_order  
  strict_partial_order
  transitive
  asymmetric
  path_total
  asymptotic bounds
}

%%%%%%%%%%%%%%%%%%%%%%%%%%%%%%%%%%%%%%%%%%%%%%%%%%%%%%%%%%%%%%%%%%%%%
% Problem starts here
%%%%%%%%%%%%%%%%%%%%%%%%%%%%%%%%%%%%%%%%%%%%%%%%%%%%%%%%%%%%%%%%%%%%%

\begin{problem}
Indicate which of the following relations below are equivalence
relations, (\textbf{E}), strict partial orders (\textbf{S}), weak
partial orders (\textbf{W}), or \emph{none} of the above (\textbf{N}).

For the partial orders, also indicate whether it is \emph{linear} (\textbf{T}).

\bparts

\iffalse

\ppart The relation $a = b + 1$ between integers, $a$, $b$,
\hfill \examrule

\begin{solution}
\textbf{N}
\end{solution}
\fi

\ppart The superset relation, $\supseteq$ on the power set of the integers.
\hfill \examrule

\begin{solution}
\textbf{W}
\end{solution}

\ppart The relation $\expect{R} < \expect{S}$ between real-valued
  random variables $R,S$.
\hfill \examrule

\begin{solution}
\textbf{N}
\end{solution}

\iffalse
\ppart The \idx{empty relation} on the set of rationals.
\hfill \examrule

\begin{solution}
\textbf{SPO}
\end{solution}
\fi

\ppart The divides relation on the positive powers of 4.  \hfill \examrule

\begin{solution}
\textbf{W, T}
\end{solution}
\eparts

\vspace{0.1in}

For the next parts, let $f,g$ be nonnegative functions from the
integers to the real numbers.

\bparts

\iffalse

\ppart $f \sim g$, the ``asymptotically Equal'' relation. \hfill \examrule
\begin{solution}
\textbf{E}
\end{solution}
\fi

\ppart $f=o(g)$,  the ``little Oh'' relation. \hfill \examrule

\begin{solution}
\textbf{S}
\end{solution}

\ppart $f=O(g)$,  the ``big Oh'' relation. \hfill \examrule

\begin{solution}
\textbf{N} because it is not antisymmetric,
\end{solution}

\ppart $f=\Theta(g)$, the ``Theta'' relation. \hfill \examrule

\begin{solution}
\textbf{E}
\end{solution}

\ppart $f=O(g) \QAND \QNOT(g=O(f))$. \hfill \examrule

\begin{solution}
\textbf{S}.
\end{solution}

\eparts

\end{problem}

%%%%%%%%%%%%%%%%%%%%%%%%%%%%%%%%%%%%%%%%%%%%%%%%%%%%%%%%%%%%%%%%%%%%%
% Problem ends here
%%%%%%%%%%%%%%%%%%%%%%%%%%%%%%%%%%%%%%%%%%%%%%%%%%%%%%%%%%%%%%%%%%%%%

\endinput
