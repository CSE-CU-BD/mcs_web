\documentclass[problem]{mcs}

\begin{pcomments}
  \pcomment{PS_translate_sentence_into_predicate_formula}
  \pcomment{small perturbation of PS_emailed_exactly_2_others}
  \pcomment{from: S06.ps1}
  \pcomment{edited ARM 2/13/12}
\end{pcomments}

\pkeywords{
  predicate
  formula
  translate
  sentence
  first-order logic
}

%%%%%%%%%%%%%%%%%%%%%%%%%%%%%%%%%%%%%%%%%%%%%%%%%%%%%%%%%%%%%%%%%%%%%
% Problem starts here
%%%%%%%%%%%%%%%%%%%%%%%%%%%%%%%%%%%%%%%%%%%%%%%%%%%%%%%%%%%%%%%%%%%%%

\begin{problem}
Translate the following sentence into a predicate formula:
\begin{quote}
There is a student who has e-mailed at most two other people in the class,
besides possibly himself.
\end{quote}

The domain of discourse should be the set of students in the class; in
addition, the only predicates that you may use are 
\begin{itemize}
\item equality, and
\item $E(x,y)$, meaning that ``$x$ has sent e-mail to $y$.''
\end{itemize}


\begin{solution}
A good way to begin tackling this problem is by trying to translate parts
of the sentence. First of all, our formula must be of the form
\[
\exists x. P(x)
\]
where $P(x)$ should be a formula that says that ``student $x$ has
e-mailed at most two other people in the class, besides possibly
himself''.

One way to write $P(x)$ is to write that ``whenever we meet a student
that has been e-mailed by $x$, this student is either $x$ himself or
$y$ or $z$, where $y$ and $z$ are two students in the class.''

To write the part ``whenever we meet a student that has been e-mailed
by $x$, this student is either $x$ himself or $y$ or $z$'' we write
\[
\forall s\, \paren{ E(x,s) \QIMPLIES (s=x \QOR s=y \QOR s=z)}.
\]
The part ``where $y$ and $z$ are two students in the class'' simply
means that there exist two such students; so by adding the appropriate
existential quantifiers, we get
\[
P(x) \eqdef \quad\exists y. \exists z.\; 
\forall s\, \paren{ E(x,s) \QIMPLIES (s=x \QOR s=y \QOR s=z)}
\]
At this point you may be thinking that $P(x)$ says that ``$x$ has
e-mailed \emph{exactly} two students besides possibly
himself.''  However we did not require that $y$ and $z$ be
distinct, or that they be different from $x$.  So our formula 
describes all possibilities: 
\begin{itemize}
\item $x$ and exactly 2 other students: $x\neq y$, $x\neq z$, $y\neq z$.
\item $x$ and exactly 1 other student: $x\neq y$ $x\neq z$, $y=z$.
\item $x$ and no other student: $x=y=z$.
\end{itemize}
Overall the full formula is:
\[
\exists x.\, \exists y. \, \exists z.\, \forall s. 
\paren{E(x,s) \QIMPLIES (s=x \QOR s=y \QOR s=z)}.
\]
\end{solution}

\end{problem}

%%%%%%%%%%%%%%%%%%%%%%%%%%%%%%%%%%%%%%%%%%%%%%%%%%%%%%%%%%%%%%%%%%%%%
% Problem ends here
%%%%%%%%%%%%%%%%%%%%%%%%%%%%%%%%%%%%%%%%%%%%%%%%%%%%%%%%%%%%%%%%%%%%%

\endinput
