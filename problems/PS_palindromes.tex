\documentclass[problem]{mcs}

\begin{pcomments}
  \pcomment{PS_palindromes}
  \pcomment{Emanuele and ARM 3/12/16}
\end{pcomments}

\pkeywords{
   recursive
   string
   reverse
   concatenation
   palindrome
}

\newcommand{\recpal}{\text{RecPal}}
%\newcommand{\revstr}{\text{rev}}
%\newcommand{\Zerostr}{\text{Zeroes}}


%%%%%%%%%%%%%%%%%%%%%%%%%%%%%%%%%%%%%%%%%%%%%%%%%%%%%%%%%%%%%%%%%%%%%
% Problem starts here
%%%%%%%%%%%%%%%%%%%%%%%%%%%%%%%%%%%%%%%%%%%%%%%%%%%%%%%%%%%%%%%%%%%%%

\begin{problem}
The string \emph{reversal} function, $\text{rev}: \strings{A} \to
\strings{A}$ has a simple recursive definition.

\inductioncase{Base case}: $\rev{\emptystring} \eqdef \emptystring$.

\inductioncase{Constructor case}: $\rev{as} \eqdef \rev{s}a$ for $s\in
\strings{A}$ and $a\in A$.

% Notice that $\rev{a} = a$ for $a \in A$ by the constructor case with
% $s = \emptystring$.

% HACK: the below inductioncase is a hacky visual formatting workaround
\inductioncase{} \hspace{-2mm} A string $s$ is a \emph{palindrome} when $\rev{s} = s$.  The
\emph{palindromes} also have a simple recursive definition as the set
\recpal.

\inductioncase{Base cases}: $\emptystring \in \recpal$ and $a \in \recpal$ for $a \in A$.

\inductioncase{Constructor case}: If $s \in \recpal$, then $asa \in \recpal$
for $a \in A$.

% HACK: the below inductioncase is a hacky visual formatting workaround
\inductioncase{} \hspace{-2mm} Verifying that the two definitions agree offers a nice exercise in
structural induction and also induction on length of strings.  The
verification rests on three basic properties of concatenation and
reversal proved in separate
problems\inbook{~\bref{CP_string_associativity}
  and~\bref{CP_string_reversal}}.

\begin{fact*}
\begin{align}
(rs = uv \QAND \lnth{r}=\lnth{u}) & \QIFF (r=u \QAND\ s = v)\label{parsefact} \\
r\cdot (s \cdot t) & = (r\cdot s) \cdot t \label{assocfact}\\
\rev{st}&  = \rev{t}\rev{s} \label{revfact}
\end{align}
\end{fact*}

\bparts

\ppart Prove that $s = \rev{s}$ for all $s \in \recpal$.

\examspace[3.0in]

\begin{solution}
\begin{proof}
By structural induction on the definition \recpal,
with induction hypothesis
\[
P(s) \eqdef\ s = \rev{s}.
\]

\inductioncase{Base case}: ($s = \emptystring$ or $s \in A$).
$P(\emptystring)$ follows immediately from the $\emptystring$ case of
the definition of reversal.  We already observed that $\rev{a} = a$
for $a \in A$ and so $P(a)$ holds

\inductioncase{Constructor case}: ($s = ara$)
\begin{align*}
\rev{s}
  & = \rev{a(ra)}
       & \text{(by~\eqref{assocfact})}\\
  & =  \rev{ra} \cdot \rev{a}
       & \text{(def of rev)}\\
  & =  \rev{ra} a
       & \text{($\rev{a} = a$ by def of rev)}\\
  & =  (\rev{a}\rev{r})a
       & \text{by~\eqref{revfact}}\\
  & =  (a \rev{r}) a
       & \text{($\rev{a} = a$)}\\ 
  & =  a\rev{r}a
          & \text{(by~\eqref{assocfact})}\\
  & =  ara
          & \text{(ind hyp $P(r)$)}\\
  & = s.
\end{align*}\end{proof}
\end{solution}

\ppart Prove conversely that if $s = \rev{s}$, then $s \in \recpal$.

\hint By induction on $n = \lnth{s}$.

\begin{solution}
Let
\[
P(n) \eqdef \forall s.\, (\lnth{s} = n \QAND s = \rev{s}) \QIMPLIES s \in \recpal.
\]

\inductioncase{Base cases}: ($n \leq 1$).  The only length 0 string is
$\emptystring$ and $\emptystring \in \recpal$ by the base case of the
definition of \recpal.  The only length 1 strings are elements $a \in
A$, and $a \in \recpal$ also by definition of \recpal.

\inductioncase{Inductive step}: We must show that if $s \in \strings{A}$ and
$\lnth{s} = n+1$ for some $n>0$, and $s = \rev{s}$, then $s \in \recpal$.

Since $\lnth{s} \geq 2$, we may assume $s = arb$ for some $a,b \in A$
and $r \in \strings{A}$.  So $\lnth{r} = n-1$.  No
\begin{align*}
arb & = s = \rev{s}\\
    & = \rev{a(rb)}
        & \text{(by associativity~\eqref{assocfact})}\\
  & = \rev{rb}a
     & \text{(def of rev)}\\
  & = b\rev{r}a
      & (by~\eqref{revfact}
\end{align*}
But by~\eqref{parsefact}, $arb = b\rev{r}a$ implies that $a=b$ and $r
= rev{r}$.  Now the induction hypothesis $P(r)$ implies $r \in
\recpal$, and since $s= ara$, it follows that $s \in \recpal$ by the
constructor case of the definition of \recpal.

\end{solution}

\eparts
\end{problem}

\endinput
