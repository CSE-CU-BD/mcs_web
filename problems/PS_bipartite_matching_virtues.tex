\documentclass[problem]{mcs}

\begin{pcomments}
  \pcomment{PS_bipartite_matching_virtues}
  \pcomment{from: S06.ps4, commented out}
  \pcomment{minor edits by ARM, 10/16/09}
\end{pcomments}

\pkeywords{
  graphs
  bipartite
  matching
  Halls_theorem  
}

%%%%%%%%%%%%%%%%%%%%%%%%%%%%%%%%%%%%%%%%%%%%%%%%%%%%%%%%%%%%%%%%%%%%%
% Problem starts here
%%%%%%%%%%%%%%%%%%%%%%%%%%%%%%%%%%%%%%%%%%%%%%%%%%%%%%%%%%%%%%%%%%%%%

\begin{problem}
  Scholars through the ages have identified \emph{twenty} fundamental
  human virtues: honesty, generosity, loyalty, prudence, completing the
  weekly 6.042 reading-response email, etc.  At the beginning of the term,
  every student in 6.042 possessed exactly \emph{eight} of these virtues.
  Furthermore, every student was unique; that is, no two students
  possessed exactly the same set of virtues.  The 6.042 course staff must
  select \emph{one} additional virtue to impart to each student by the end
  of the term.  Prove that there is a way to select an additional virtue
  for each student so that every student is unique at the end of the term
  as well.

Suggestion: Use \idx{Hall's theorem}.  Try various interpretations for
the vertices on the left and right sides of your \idx{bipartite graph}.

\begin{solution}
Construct a bipartite graph $G$ as follows.  The vertices on
on the left are all students and the virtues on the right are all
subset of nine virtues.  There is an edge between a student and a set
of 9 virtues if the student already has 8 of those virtues.

Each vertex on the left has degree 12, since each student can learn one of
12 additional virtues.  The vertices on the right have degree at most 9,
since each set of 9 virtues has only 9 subsets of size 8.  So this
bipartite graph is \idx{degree-constrained}, and therefore, by
Lemma~\bref{degree-constrained_lemma}, there is a matching for the
students.  Thus, if each student is taught the additional virtue in the
set of 9 virtues with whom he or she is matched, then every student is
unique at the end of the term.
\end{solution}

\end{problem}


%%%%%%%%%%%%%%%%%%%%%%%%%%%%%%%%%%%%%%%%%%%%%%%%%%%%%%%%%%%%%%%%%%%%%
% Problem ends here
%%%%%%%%%%%%%%%%%%%%%%%%%%%%%%%%%%%%%%%%%%%%%%%%%%%%%%%%%%%%%%%%%%%%%

\endinput
