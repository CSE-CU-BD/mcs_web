\documentclass[problem]{mcs}

\begin{pcomments}
  \pcomment{FP_satisfy_implies_probability}
  \pcomment{perturbation of FP_satisfy_implies_probability_alt}
  \pcomment{ARM 5/19/15}
\end{pcomments}

\pkeywords{
  satisfiability
  propositions
  implies
  probability
}

%%%%%%%%%%%%%%%%%%%%%%%%%%%%%%%%%%%%%%%%%%%%%%%%%%%%%%%%%%%%%%%%%%%%%
% Problem starts here
%%%%%%%%%%%%%%%%%%%%%%%%%%%%%%%%%%%%%%%%%%%%%%%%%%%%%%%%%%%%%%%%%%%%%
\begin{problem}
Truth values for propositional variables $P,Q,R$ are chosen
independently, with $\pr{P=\true} = 1/2, \pr{Q=\true} = 1/3,
\pr{R=\true} = 1/5$.  What is the probability that the formula
\begin{equation}\label{PpQR}
P \QIMPLIES (Q \QIMPLIES R)
\end{equation}
is true?\hfill \examrule{0.7in}

\examspace[3in]

\begin{solution}
The simplest approach uses the clever observation that~\eqref{PpQR} is
false iff $P$ and $Q$ are true and $R$ is false, so the probability is
\[
1 - \frac{1}{2}\cdot \frac{1}{3} \cdot \frac{4}{5} = \frac{13}{15}
\]

A more direct approach is by separate cases in which~\eqref{PpQR} is
true.

\inductioncase{$P = \false$} This case happens with probability $1/2$.

\inductioncase{$P = \true \QAND R = \false$} This case happens with probability $(1/2)(2/3)$.

\inductioncase{$P = \true \QAND R = \true \QAND Q = \true$} This case
happens with probability $(1/2)(1/3))(1/5)$.

So the probability~\eqref{PpQR} is true is
\[
\frac{1}{2}+ \frac{1}{2}\frac{2}{3}+ \frac{1}{2}\frac{1}{3}\frac{1}{5} = \frac{13}{15}.
\]
\end{solution}
\end{problem}

\endinput
