\documentclass[problem]{mcs}

\begin{pcomments}
  \pcomment{FP_increasing_sequences_diag}
  \pcomment{variant of PS_increasing_sequences_bijection}
  \pcomment{zabel for s18 midterm3}
\end{pcomments}

\pkeywords{
  infinite sets
  sequences
  increasing
  diagonalization
}

%%%%%%%%%%%%%%%%%%%%%%%%%%%%%%%%%%%%%%%%%%%%%%%%%%%%%%%%%%%%%%%%%%%%%
% Problem starts here
%%%%%%%%%%%%%%%%%%%%%%%%%%%%%%%%%%%%%%%%%%%%%%%%%%%%%%%%%%%%%%%%%%%%%

\newcommand\Inc{\text{Inc}}

\begin{problem}
  As we did in PSET 4 Problem \#2, let $\nngint^\omega$ be the set of
  infinite sequences of natural numbers, and say that a sequence
  $(a_0,a_1,a_2,\dots)\in\nngint^\omega$ is \emph{strictly increasing}
  if $a_0 < a_1 < a_2 < \cdots$.  Define the subset
  $\Inc \subset \nngint^\omega$ to be the set of strictly increasing
  sequences.

  Carefully prove that $\Inc$ is uncountable by a direct \textbf{diagonalization} argument over $\Inc$. Your proof should be self-contained, i.e., written in a way that doesn't assume the reader is intimately familiar with similar diagonalization proofs.

  \medspace
  \noindent\emph{Note}: PSET 4 Problem \#2 provides a different way to prove that $\Inc$ is uncountable, but please don't cite or duplicate that problem here, as it isn't helpful in constructing a proof by diagonalization.

  \begin{staffnotes}
    PSET 4 Problem \#2 refers to \verb+PS_increasing_sequences_bijection+ in spring18. Remove or replace this reference if using this problem in later semesters and/or in the book.
  \end{staffnotes}
  
\begin{solution}
  The proof is by contradiction, so assume that $\Inc$ is countable. $\Inc$ is not finite (for example, there are infinitely many arithmetic sequences in $\Inc$ of the form $(n,n+1,n+2,\ldots)$ for $n\in\nngint$), so $\Inc$ must instead be countably infinite. Therefore there must exist a function $f:\nngint\to\Inc$ that is a \emph{bijection}.

  Define a new sequence $s\in\nngint^\omega$ as follows: choose $s[0] \eqdef 1 + f(0)[0]$, and thereafter, for each integer $k \ge 1$, define
  \begin{equation*}
    s[k] \eqdef 1 + \max(s[k-1], f(k)[k]).
  \end{equation*}
  This defines a sequence $s\in\nngint^\omega$.

  If $s = f(n)$ for some integer $n \ge 0$, then these sequences must have the same $n$th entry: $s[n] = f(n)[n]$. But by the definition of $s$, we have
  \begin{equation*}
    s[n] \ge 1 + f(n)[n] > f(n)[n]
  \end{equation*}
  (even in the $n = 0$ case), so $s[n] = f(n)[n]$ cannot happen. This means $s \ne f(n)$ for \emph{every} $n\in\nngint$, i.e., $s$ is not in the image of $f$.

  We also claim that $s\in\Inc$. The definition of $s$ shows that
  \begin{equation*}
    s[k] \ge 1 + s[k-1] > s[k-1]
  \end{equation*}
  for each $k\ge 1$, so $s$ is indeed \emph{strictly increasing}, i.e., $s\in\Inc$.

  Because we have found a sequence $s\in\Inc$ that is not in the image of $f$, it follows that $f$ is not surjective and therefore cannot be a bijection. This is a contradiction, as desired.
\end{problem}


%%%%%%%%%%%%%%%%%%%%%%%%%%%%%%%%%%%%%%%%%%%%%%%%%%%%%%%%%%%%%%%%%%%%%
% Problem ends here
%%%%%%%%%%%%%%%%%%%%%%%%%%%%%%%%%%%%%%%%%%%%%%%%%%%%%%%%%%%%%%%%%%%%%

\endinput
