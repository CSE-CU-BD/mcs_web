\documentclass[problem]{mcs}

\begin{pcomments}
  \pcomment{TP_indicator_independence}
  \pcomment{added by ARM 4/23/11}
  \pcomment{edited by YZ, CH April 2014}
\end{pcomments}

\pkeywords{
 indicator
 independence
 event
 random_variable
 }

%%%%%%%%%%%%%%%%%%%%%%%%%%%%%%%%%%%%%%%%%%%%%%%%%%%%%%%%%%%%%%%%%%%%%
% Problem starts here
%%%%%%%%%%%%%%%%%%%%%%%%%%%%%%%%%%%%%%%%%%%%%%%%%%%%%%%%%%%%%%%%%%%%%

\begin{problem}

%\bparts

\begin{comment}  % For Spring14 only - this problem is used for cp12f, but %this part is also in Problem 3 of cp12w

\ppart\label{AindBindbarB} Prove that if $A$ and $B$ are independent events, then so are
$A$ and $\setcomp{B}$.
\begin{solution}
\begin{proof}
\begin{align*}
\pr{A \intersect \setcomp{B}}
  &  = \pr{A} - \pr{A \intersect B}
       & \text{(difference rule)}\\
  & =  \pr{A} - \pr{A} \cdot \pr{B}
       & \text{(independence of $A$ and $B$)}\\
  & =  \pr{A} (1 - \pr{B})\\
  & =  \pr{A} \cdot \pr{\setcomp{B}}. & \text{(complement rule)}
\end{align*}
\end{proof}
\end{solution}

\end{comment}

%\ppart
Let $I_A$ and $I_B$ be the indicator variables for events $A$ and $B$.
Prove that $I_A$ and $I_B$ are independent iff $A$ and $B$ are
independent.  

\hint For any event, $E$, let $E^1 \eqdef E$ and $E^0 \eqdef \setcomp{E}$.
So the event $[I_E = a]$ is the same as $E^a$.

\begin{solution}

\begin{staffnotes}
The first part was proved in an earlier class problem. 
\end{staffnotes}

First, we prove that if $A$ and $B$ are independent events, then so are
$A$ and $\setcomp{B}$. This follows since:
\begin{align*}
\pr{A \intersect \setcomp{B}}
  &  = \pr{A} - \pr{A \intersect B}
       & \text{(difference rule)}\\
  & =  \pr{A} - \pr{A} \cdot \pr{B}
       & \text{(independence of $A$ and $B$)}\\
  & =  \pr{A} (1 - \pr{B})\\
  & =  \pr{A} \cdot \pr{\setcomp{B}}. & \text{(complement rule)}
\end{align*}
Similarly, we can show that $\setcomp{A}$ and $B$ are independent,
and $\setcomp{A}$ and $\setcomp{B}$ are independent. 

Let us define some notation. For any event, $E$, let $E^1 \eqdef E$ and $E^0 \eqdef \setcomp{E}$. 
So the event $[I_E = a]$ is the same as $E^a$. Therefore, our above proof implies that the following propositions are equivalent:
\begin{itemize}
\item $A$ and $B$ are independent,
%\item $\exists a,b \in \set{0,1}$. \ [$A^a$ and $B^b$ are independent],
\item $\forall a,b \in \set{0,1}$. \ [$A^a$ and $B^b$ are independent].
\end{itemize}

However, by the definition of independence for random variables, the following propositions are equivalent as well:
\begin{itemize}
%\item $I_A$ and $I_B$ are independent,
\item $\forall a,b \in \set{0,1}$. [$A^a$ and $B^b$ are independent],
\item  $\forall a,b \in \set{0,1}$. [$[I_A = a]$ and $[I_A = b]$ are
  independent events].
\end{itemize}

Therefore, combining the two equivalences, we conclude that the
indicator random variables $I_A$ and $I_B$ are independent iff the
events $A$ and $B$ are independent.

\end{solution}

%\eparts

\end{problem}

%%%%%%%%%%%%%%%%%%%%%%%%%%%%%%%%%%%%%%%%%%%%%%%%%%%%%%%%%%%%%%%%%%%%%
% Problem ends here
%%%%%%%%%%%%%%%%%%%%%%%%%%%%%%%%%%%%%%%%%%%%%%%%%%%%%%%%%%%%%%%%%%%%%

\endinput
 
