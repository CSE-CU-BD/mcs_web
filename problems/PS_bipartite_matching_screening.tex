\documentclass[problem]{mcs}

\begin{pcomments}
  \pcomment{PS_bipartite_matching_screening}
  \pcomment{expansion of and new context for FP_bipartite_matching_recitations}
  \pcomment{zabel for spring18 ps3, original from S090.MQmar19}
\end{pcomments}

\pkeywords{
  graphs
  bipartite
  matching
  Halls_theorem
}

%%%%%%%%%%%%%%%%%%%%%%%%%%%%%%%%%%%%%%%%%%%%%%%%%%%%%%%%%%%%%%%%%%%%% 
% Problem starts here
%%%%%%%%%%%%%%%%%%%%%%%%%%%%%%%%%%%%%%%%%%%%%%%%%%%%%%%%%%%%%%%%%%%%% 

\begin{problem}

  Marvel is staging $4$ test screenings of \emph{Avengers: $\infty$
    War} exclusively for a random selection of MIT
  students!\footnote{Sadly this isn't actually happening, as far as we
    know.} For scheduling purposes, each of the selected students will
  specify which of the four screenings don't conflict with their
  schedule--every student is available for at least two out of the
  four screenings.  However, each screening has only $20$ available
  seats, not all of which need to be filled each time.
  % 
  Marvel is thus faced with a difficult scheduling problem: how do
  they make sure each of the chosen students is able to find a seat at
  a screening?  They've recruited you to help solve this dilemma.

  \bparts

  \ppart Describe how to model this situation as a matching problem.
  Be sure to specify what the vertices/edges should be and briefly
  describe how a matching would determine seat assignments for each
  student in a screening for which they are available.  (This is a \emph{modeling problem}; we aren't looking for a description
  of an algorithm to solve the problem.)
  
  \begin{solution}
    There will be one vertex on the left for each student, and 20
    vertices on the right \emph{for each screening} (corresponding to
    the 20 seats), making $80$ right vertices in all.  There is an
    edge between a student and all seats in screenings for which they
    are available.  A
    matching for the students assigns a student to a seat in a
    screening that they can attend and assigns at most 20 students to
    any screening.

    It is possible to assign the students to screenings iff a matching
    exists.
  \end{solution}

  \ppart Suppose 41 students have been selected.  Can you guarantee
  that a matching exists, or are there some situations where not all
  of the 41 students can be accommodated?  Briefly explain.

  \begin{solution}
    In many situations a matching will exist, but there isn't
    \emph{always} a matching.  For example, if all 41 students can
    only attend the first two screenings, then the number of students
    exceeds the number of seats to which they can be assigned.  Thus,
    the set of all $41$ students would form a bottleneck.
  \end{solution}

  \ppart

  If instead only 40 students are chosen, prove that there is always a matching.

  \hint Use Hall's Theorem or something similar.  Is your graph degree
  constrained?

  \begin{solution}
    Each student (left node) has degree at least $40$ because they are
    available for at least two showings.  On the other hand, since
    there are $40$ left nodes in total, each right node can have
    degree at most $40$.  Thus, the graph is degree constrained with
    $d=40$.  As shown in the book, this is enough to guarantee the
    existence of a matching.

    Alternatively, we can argue directly from Hall's theorem by
    showing that there can't be a subset of students forming a
    bottleneck.  As before, each student (left node) has degree $\ge
    40$.  Thus, any nonempty subset $S\subseteq\text{Students}$ will
    connect to at least $40$ seats, and as there are at most $|S|\le
    40$ students in this subset, $S$ is not a bottleneck.  So by
    Hall's Theorem, a matching must exist.
  \end{solution}

  \eparts

\end{problem}


%%%%%%%%%%%%%%%%%%%%%%%%%%%%%%%%%%%%%%%%%%%%%%%%%%%%%%%%%%%%%%%%%%%%% 
% Problem ends here
%%%%%%%%%%%%%%%%%%%%%%%%%%%%%%%%%%%%%%%%%%%%%%%%%%%%%%%%%%%%%%%%%%%%% 

\endinput
