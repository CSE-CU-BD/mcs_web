\documentclass[problem]{mcs}

\begin{pcomments}
  \pcomment{PS_bipartite_matching_screening}
  \pcomment{expansion of and new context for FP_bipartite_matching_recitations}
  \pcomment{zabel for S18.ps5, original from S090.MQmar19}
\end{pcomments}

\pkeywords{
  graphs
  bipartite
  matching
  Halls_theorem
}

%%%%%%%%%%%%%%%%%%%%%%%%%%%%%%%%%%%%%%%%%%%%%%%%%%%%%%%%%%%%%%%%%%%%% 
% Problem starts here
%%%%%%%%%%%%%%%%%%%%%%%%%%%%%%%%%%%%%%%%%%%%%%%%%%%%%%%%%%%%%%%%%%%%% 

\begin{problem}
Marvel is staging four test screenings of \emph{Avengers: $\infty$
  War} exclusively for a random selection of MIT
students!\footnote{Sadly this isn't actually happening, as far as I
  know.} For scheduling purposes, each of the selected students will
specify which of the four screenings don't conflict with their
schedule--every student is available for at least two out of the four
screenings. However, each screening has only twenty available seats,
not all of which need to be filled each time.  Marvel is thus faced
with a difficult scheduling problem: how do they make sure each of the
chosen students is able to find a seat at a screening?  They've
recruited you to help solve this dilemma.

\bparts

\ppart Describe how to model this situation as a matching problem.  Be
sure to specify what the vertices/edges should be and briefly describe
how a matching would determine seat assignments for each student in a
screening for which they are available. (This is a \emph{modeling
  problem}; we aren't looking for a description of an algorithm to
solve the problem.)
  
  \begin{solution}
    There will be one vertex on the left for each student, and 20
    vertices on the right \emph{for each screening} (corresponding to
    the 20 seats), making $80$ right vertices in all.  There is an
    edge between a student and all seats in screenings for which they
    are available. A
    matching for the students assigns a student to a seat in a
    screening that they can attend and assigns at most 20 students to
    any screening.

    It is possible to assign the students to screenings iff a matching
    exists.
  \end{solution}

\ppart Suppose 41 students have been selected.  Can you guarantee that
a matching exists, or are there some situations where not all of the
41 students can be accommodated?  Briefly explain.

  \begin{solution}
    In many situations a matching will exist, but there isn't
    \emph{always} a matching. For example, if
    all 41 students can only attend the first two screenings, then the
    number of students exceeds the number of seats to which they can
    be assigned. Thus, the set of all $41$ students would form a bottleneck.
  \end{solution}

\ppart
If instead only 40 students are chosen, prove that there is always a matching.

\hint Your graph probably isn't degree constrained.

  \begin{solution}
    The graph is not guaranteed to be degree constrained.  For
    example, it is possible for some seats to have degree $80$ and
    others to have degree $0$.  So we must rely directly on Hall's
    Theorem: we'll show that there can't be a subset of students
    forming a bottleneck.

    Each student individually is available for at least two screenings
    and thus has degree $\ge 40$.  Thus, any nonempty subset
    $S\subseteq\text{Students}$ will connect to at least $40$ seats,
    and as there are at most $|S|\le 40$ students in this subset, $S$
    is not a bottleneck. So by Hall's Theorem, a matching must exist.
  \end{solution}

  \eparts

\end{problem}


%%%%%%%%%%%%%%%%%%%%%%%%%%%%%%%%%%%%%%%%%%%%%%%%%%%%%%%%%%%%%%%%%%%%% 
% Problem ends here
%%%%%%%%%%%%%%%%%%%%%%%%%%%%%%%%%%%%%%%%%%%%%%%%%%%%%%%%%%%%%%%%%%%%% 

\endinput
