\documentclass[problem]{mcs}

\begin{pcomments}
  \pcomment{PS_MIT_Harvard_degree_value}
  \pcomment{from: S07.ps7}
  \pcomment{commented out in S09.ps8}
\end{pcomments}

\pkeywords{
  geometric_sum
  series
  summation
  interest
}

%%%%%%%%%%%%%%%%%%%%%%%%%%%%%%%%%%%%%%%%%%%%%%%%%%%%%%%%%%%%%%%%%%%%%
% Problem starts here
%%%%%%%%%%%%%%%%%%%%%%%%%%%%%%%%%%%%%%%%%%%%%%%%%%%%%%%%%%%%%%%%%%%%%

\begin{problem}
  Is a Harvard degree really worth more than an MIT degree?  Let us
  say that a person with a Harvard degree starts with \$40,000 and
  gets a \$20,000 raise every year after graduation, whereas a person
  with an MIT degree starts with \$30,000, but gets a 20\% raise every
  year.  Assume inflation is a fixed $8\%$ every year.  That is,
  \$1.08 a year from now is worth \$1.00 today.

\begin{problemparts}

\problempart How much is a Harvard degree worth today if the holder will work
for $n$ years following graduation?

\problempart How much is an MIT degree worth in this case?

\problempart If you plan to retire after twenty years, which degree would
be worth more?

\begin{solution}
One dollar after year $i$ is worth $r^i$ in today's currency, where
\[
r= \frac{1}{1.08} = 0.925\, 925\, 925 \dots
\]
So
\begin{align*}
\mbox{Hvd}_n
    & = \sum_{i=0}^{n} (40000+ 20000i)r^i\\
    & = 40000\sum_{i=0}^{n} r^i + 20000\sum_{i=0}^{n} i r^i,\\
\mbox{MIT}_n
    & = 30000\sum_{i=0}^{n} {1.2}^i r^i\\
    & = 30000\sum_{i=0}^{n} (1.2r)^i
\end{align*}

But
\[
\sum_{i=0}^{n} ir^i = \frac{{r - (n+1) r^{n+1} + n r^{n+2}}}{(1-r)^2},
\]
so
\begin{align*}
\mbox{Hvd}_n
    & =  40000 \frac{(1- r^{n+1})}{1- r}  +
                   20000 \frac{(r - (n+1) r^{n+1} + n r^{n+2})}{(1-r)^2}\\
    & =   \frac{20000(2(1- r^{n+1} -r +r^{n+2}) + r - (n+1) r^{n+1} + n r^{n+2})}{(1-r)^2}\\
    & =   \frac{{20000(2 - r - (n+3)r^{n+1} + (n + 2)r^{n+2})}}{(1-r)^2}\\
\mbox{MIT}_n
    & =   \frac{{30000(1 - (1.2r)^{n+1})}}{1 - 1.2r}
\end{align*}
and for $n=20$,
\begin{align*}
\mbox{Hvd}_{20} & = \frac{{20000(2 - r - 23r^{21} + 22r^{22})}}{(1-r)^2} = 2,010,885\\
\mbox{MIT}_{20} & = \frac{{30000(1 - (1.2r)^{21})}}{1 - 1.2r} = 2,197,579.
\end{align*}
so the MIT degree is more valuable! (But we knew that already.)
\end{solution}

\end{problemparts}
\end{problem}

%%%%%%%%%%%%%%%%%%%%%%%%%%%%%%%%%%%%%%%%%%%%%%%%%%%%%%%%%%%%%%%%%%%%%
% Problem ends here
%%%%%%%%%%%%%%%%%%%%%%%%%%%%%%%%%%%%%%%%%%%%%%%%%%%%%%%%%%%%%%%%%%%%%
