\documentclass[problem]{mcs}

\begin{pcomments}
  \pcomment{from: S06.ps1}
\end{pcomments}

\pkeywords{
  predicates
  propositional_logic
}

%%%%%%%%%%%%%%%%%%%%%%%%%%%%%%%%%%%%%%%%%%%%%%%%%%%%%%%%%%%%%%%%%%%%%
% Problem starts here
%%%%%%%%%%%%%%%%%%%%%%%%%%%%%%%%%%%%%%%%%%%%%%%%%%%%%%%%%%%%%%%%%%%%%

\begin{problem}
Express each of the following predicates and propositions in formal
logic notation.  The domain of discourse is the nonnegative integers,
$\naturals$. Moreover, in addition to the propositional operators,
variables and quantifiers, you may define predicates using addition,
multiplication, and equality symbols, but no \emph{constants} (like 0,
1,\dots) and no \emph{exponentiation} (like $x^y$).  For example, the
proposition ``n is an even number'' could be written
\[
\exists m.\; (m + m = n).
\]

\begin{problemparts}
\problempart 
$n$ is the sum of two cubes (a cube is a number equal to $k^3$ for some
integer $k$).

\solution{
\[
\exists x \exists y.\; (x \cdot x \cdot x + y \cdot y \cdot y = n)
\]
}
\end{problemparts}

Since the constant 0 is not allowed to appear explicitly, the predicate
``$x = 0$'' can't be written directly, but note that it could be expressed
in a simple way as:
\[
x + x = x.
\]
Then the predicate $x > y$ could be expressed
\[
\exists w.\; (y + w = x) \land (w \neq 0).
\]
Note that we've used ``$w \neq 0$'' in this formula, even though it's
technically not allowed.  But since ``$w \neq 0$'' is equivalent to the
allowed formula ``$\neg(w+w= w)$,'' we can use ``$w \neq 0$'' with the
understanding that it abbreviates the real thing.  And now that we've shown
how to express ``$x>y$,'' it's ok to use it too.

\begin{problemparts}
\problempart $x = 1$.

\solution{
One formula is $\forall y.\; xy=y$.  Another is $(x \cdot x = x) \conj (x
\neq 0)$.
}

\ppart $m$ is a divisor of $n$ (notation: $m \divides n$)

\solution{\[
m \divides n 
\eqdef\quad 
\exists k.\; k \cdot m = n
\]}

\problempart 
$n$ is a prime number (hint: use the predicates from the previous parts)

\solution{
\[
\isprime(n)
\eqdef\quad 
(n\neq 1) 
\;\;\conj\;\;
\forall m.\; (m \divides n) \implies (m=1 \disj m=n).
\]
Note that the requirement $n\neq 1$ is necessary, or else our
predicate would be satisfied by 1, which is not considered to be a
prime number. Also note that $n\neq 1$ is given here as an
abbreviation of the formula $\neg(n=1)$; and thus of $\neg\forall y.yn=n$.

If we don't want to use the divisor predicate, we can write:
\[
\isprime(n) 
\eqdef\quad 
(n\neq 1) 
\;\;\conj\;\;
\neg\bigl(
\exists x 
\exists y.\; 
(x > 1) \conj 
(y > 1) \conj 
(x \cdot y = n)
\bigr).
\]
Do you agree this formula is also correct? If so, note that the
formulas $x>1$ and $y>1$ are not allowed. Can you express them in
terms of allowed formulas?  }

\problempart $n$ is a power of 2.

\solution{We can simply say that the only divisors of $n$ are 1 and 2:
\[
\forall m.\; m \divides n \implies m =1 \disj m=2
\]
Still, $m=2$ is not allowed, so we have to express it in terms of
allowed formulas. Here is one way to do this:
\[
m=2
\eqdef\quad
\exists z.
z=1
\;\conj\;
m= z + z
\]
}
\end{problemparts}
\end{problem}

%%%%%%%%%%%%%%%%%%%%%%%%%%%%%%%%%%%%%%%%%%%%%%%%%%%%%%%%%%%%%%%%%%%%%
% Problem ends here
%%%%%%%%%%%%%%%%%%%%%%%%%%%%%%%%%%%%%%%%%%%%%%%%%%%%%%%%%%%%%%%%%%%%%