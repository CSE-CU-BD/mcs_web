\documentclass[problem]{mcs}

\begin{pcomments}
  \pcomment{CP_directed_walks_and_cycles}
  \pcomment{from: S09.cp7r}
  \pcomment{part (a) PS_shortest_directed_closed_walk}
  \pcomment{part(b) subsumed by text Theorem~\ref{shortestwalk_thm}}}
\end{pcomments}

\pkeywords{
  digraphs
  cycle
  walk
}

%%%%%%%%%%%%%%%%%%%%%%%%%%%%%%%%%%%%%%%%%%%%%%%%%%%%%%%%%%%%%%%%%%%%%
% Problem starts here
%%%%%%%%%%%%%%%%%%%%%%%%%%%%%%%%%%%%%%%%%%%%%%%%%%%%%%%%%%%%%%%%%%%%%

\begin{problem}
\bparts

\ppart Give an example showing that two vertices in a digraph may be
on the same closed walk, but \emph{not} necessarily on the same cycle.

\hint There is an example with 4 vertices.

\begin{solution}
Let the vertices be $a,b,c$ and edges be $(a,b), (b,a), (b,c), (c,b)$.
Now $a$ and $c$ are on the closed walk $a,b,c,b,a$, but every closed
walk from $a$ to $c$ must go through $b$ at least twice, and so will
not be a cycle.
\end{solution}

\ppart Prove that if two vertices in a digraph are connected, then they are
connected by a path.

\begin{solution}
Consider a shortest path from $a$ to $b \neq a$:
\[
a=a_0,a_1,\dots,a_i,\dots, a_j, \dots ,a_k=b,
\]
and suppose this path is not simple.  That is, suppose $a_i=a_j$ for some
$i,j$.  Then
\[
a=a_0,a_1,\dots,a_i, a_{j+1}, \dots ,a_k=b.
\]
is a shorter path from $a$ to $b$, a contradiction.

\end{solution}

\eparts


\end{problem}

%%%%%%%%%%%%%%%%%%%%%%%%%%%%%%%%%%%%%%%%%%%%%%%%%%%%%%%%%%%%%%%%%%%%%
% Problem ends here
%%%%%%%%%%%%%%%%%%%%%%%%%%%%%%%%%%%%%%%%%%%%%%%%%%%%%%%%%%%%%%%%%%%%%

\endinput
