\documentclass[problem]{mcs}

\begin{pcomments}
  \pcomment{CP_multinomial_theorem}
  \pcomment{could be used as FP}
  \pcomment{essentially the same as PS_multinomial_theorem, but does
    not mention combinatorial proof}
  \pcomment{from: S09.cp11r}
\end{pcomments}

\pkeywords{
  multinomial
}

%%%%%%%%%%%%%%%%%%%%%%%%%%%%%%%%%%%%%%%%%%%%%%%%%%%%%%%%%%%%%%%%%%%%%
% Problem starts here
%%%%%%%%%%%%%%%%%%%%%%%%%%%%%%%%%%%%%%%%%%%%%%%%%%%%%%%%%%%%%%%%%%%%%

\begin{problem}
According to the Multinomial theorem, $(w+x+y+z)^n$ can be expressed as a
sum of terms of the form
\[
\binom{n}{r_1,r_2,r_3,r_4}w^{r_1}x^{r_2}y^{r_3}z^{r_4}.
\]

\bparts

\ppart How many terms are there in the sum?

\begin{solution}
The sum is over all 4-tuples of nonnegative integers
$(r_1,r_2,r_3,r_4)$ such that
\[
r_1+r_2+r_3+r_4 = n.
\]
We know this is the same as the number of binary words with $n$ zeroes and
3 ones, namely
\[
\binom{n+3}{3}.
\]
\end{solution}

\ppart The sum of these multinomial coefficients has an easily expressed
value.  What is it?
\begin{equation}\label{multi-coeff-sum}
\sum_{\substack{r_1+r_2+r_3+r_4 = n,\\
      r_i \in \nngint}}         \binom{n}{r_1,\, r_2,\, r_3,\, r_4} = ?
\end{equation}


\hint How many terms are there when $(w+x+y+z)^n$ is expressed as a sum of
monomials in $w,x,y,z$ \emph{before} terms with like powers of these
variables are collected together under a single coefficient?

\begin{solution}
  $4^n$.

  This is a nice example of a combinatorial proof: there are $4^n$ ways to
  choose one of the four variables from each of the $n$ expressions
  $(w+x+y+z)$, and each choice corresponds to a degree $n$ product of
  these variables.  Each of the $\binom{n+3}{3}$ coefficients in the
  sum~\eqref{multi-coeff-sum} is a count of the number of these monomial
  products with a given number of occurrences of the variables, so the sum
  of these coefficients is simply the total number of products.
\end{solution}

\eparts
\end{problem}


%%%%%%%%%%%%%%%%%%%%%%%%%%%%%%%%%%%%%%%%%%%%%%%%%%%%%%%%%%%%%%%%%%%%%
% Problem ends here
%%%%%%%%%%%%%%%%%%%%%%%%%%%%%%%%%%%%%%%%%%%%%%%%%%%%%%%%%%%%%%%%%%%%%
\endinput
