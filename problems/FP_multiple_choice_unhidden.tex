\documentclass[problem]{mcs}

\begin{pcomments}
  \pcomment{FP_multiple_choice_unhidden}
  \pcomment{revised and extended from FP_multiple_choice by ARM
    5/15/11, 2/2/12}
  \pcomment{stable distribution commented out 12/15/11}
  \pcomment{overlaps FP_graphs_short_answer}  
\end{pcomments}

\pkeywords{
  isomorphism
  partial_order  
  total_order
  linear
  asymptotic
  vertices
  gcd
  modular
  mod_n
  stable_distribution
  countable
}

%%%%%%%%%%%%%%%%%%%%%%%%%%%%%%%%%%%%%%%%%%%%%%%%%%%%%%%%%%%%%%%%%%%%%
% Problem starts here
%%%%%%%%%%%%%%%%%%%%%%%%%%%%%%%%%%%%%%%%%%%%%%%%%%%%%%%%%%%%%%%%%%%%%

\begin{problem} \mbox{}

% FROM: Spring07 MQ-4/6-2a
%
% COMMENTS: Reduced number, changed statements, now asking for
% counterexample. Solution incomplete. Should we replace "Lemma 3.4"
% with an \eqref?
%
% Need to make boxes (make sure they know that counterexamples req'd): 
%     true        false__________________ <-- room for counterexample
%

\textbf{\large Circle \textbf{true} or \textbf{false} for the statements below,
and \emph{provide counterexamples} for those that are \textbf{false}.}

\bparts

\ppart The following statements about the greatest common divisor:

\begin{itemize}

\item $\gcd(ab, ac) = a \gcd(b, c)$.  \hfill 
\textbf{true} \qquad  \textbf{false} \examspace[0.4in]

\begin{solution}
\textbf{true}
\end{solution}

\item If $a \divides b c$ and $\gcd(a, b) = 1$, then $a \divides c$.  \hfill 
\textbf{true} \qquad \textbf{false} \examspace[0.4in]

\begin{solution}
\textbf{true}
\end{solution}


\item $\gcd(a^n,b^n) =  (\gcd(a,b))^n$  \hfill 
\textbf{true} \qquad \textbf{false} \examspace[0.4in]

\begin{solution}
\textbf{true}.
\end{solution}

\item If $\gcd(a, b) \neq 1$ and $\gcd(b, c) \neq 1$, then $\gcd(a, c) \neq 1$. \hfill 
\textbf{true} \qquad  \textbf{false} \examspace[0.4in]

\begin{solution}
\textbf{false} $a=2\cdot 3, b=3\cdot 5, c=5\cdot 7$
\end{solution}

%%Removed
%\item $\gcd(a, b) = \gcd(b, \rem{a}{b})$. % too easy
%\item $\gcd(a,b) \gcd(c,d) =  \gcd(ac,bd)$ % too many vars for counterexample
%\item $\gcd(a, b) = \gcd(a+b, a-b)$ % what does it test?
\end{itemize}

% ~~~~~~~~~~~~~~~~~~~~~~~~~~~~~~~~~~~~~~~~~~~~~~~~~~~~~~~~~~~~~~~~~~~
% FROM: Spring07 MQ-4/6-2b
%
% COMMENTS: Reduced number, changed statements, now asking for
% counterexample. Solution needs update. Should we replace "Lemma 7.3"
% "Corollary 7.2" and "Corollary 10.3" with \eqref's?  Is the third
% one too hard?  Do we need m>=0?
%
% Need to make boxes (make sure they know that counterexamples req'd): 
%     true        false__________________ <-- room for counterexample
%

\ppart The following statements about congruence modulo $n$, where $n > 1$.

\begin{itemize}

\item If $a c \equiv b c \pmod{n}$ and $n$ does not divide $c$, then $a \equiv b \pmod{n}$.
\hfill \textbf{true} \qquad \textbf{false} \examspace[0.4in]

\begin{solution}
\textbf{false}.  Need $c$ relatively prime to $n$.  Counterexample:
$n=2 \cdot 3, a=0, b=2, c=3$
\end{solution}

\item If $a \equiv b \pmod{\phi(n)}$ for $a, b > 0$, then $c^a \equiv c^b
\pmod{n}$.  \hfill \textbf{true} \qquad \textbf{false} \examspace[0.4in]

\begin{solution}
  \textbf{false}.  Need $c$ relatively prime to $n$.  Counterexample:
  $n=4$, so $\phi(n) = 2$; $a=1, b=3$, so $a \equiv b
  \pmod \phi(n)$, $c = 2$, so $c^a = 2 \not\equiv 0 = c^b \pmod 4$.
\end{solution}

\iffalse
\item If $a \equiv b \pmod{n}$, then $P(a) \equiv P(b) \pmod{n}$ for any
polynomial $P(x)$ with integer coefficients.
 \hfill \textbf{true} \qquad \textbf{false} \examspace[0.4in]
\begin{solution}
true
\end{solution}
\fi

\item If $a \equiv b \pmod{nm}$, then $a \equiv b \pmod{n}$, for $m,mn > 1$.
 \hfill \textbf{true} \qquad \textbf{false} \examspace[0.4in]

\begin{solution}
\textbf{true}
\end{solution}

%%Removed
%\item If $a \equiv b \pmod{n}$ and $b \equiv c \pmod{n}$, then $a \equiv c \pmod{n}$. % too easy
%\item If $a \equiv b \pmod{n}$ and $c \equiv d \pmod{n}$, then $a^c \equiv b^d \pmod{n}$. % replaced with phi() one
%\item $\rem{a}{n} \equiv a \pmod{n}$. % too easy
%\item $\gcd(a \pmod{n}, b \pmod{n}) \equiv \gcd(a,b) \pmod{n}$

\item For relatively prime $m,n >1$,\\

$[a \equiv b \pmod{m} \QAND a \equiv b \pmod{n}] \qiff [a \equiv b
  \pmod{mn}]$
\hfill \textbf{true} \qquad \textbf{false} \examspace[0.4in]

\begin{solution}
\textbf{true}
\end{solution}

\end{itemize}

\iffalse

\begin{solution}

\begin{itemize}
\item The first one is not valid. counterexample: $a=1$, $b=2$, $c=2$, $n=2$.
\item The second one is not valid. counterexample: $n=4, a=1, b=3, c=2$.
\item The third one is valid. A polynomial is constructed with only addition and multiplication.
\item The fourth one is valid. An equivalent statment is ``If $mn | a$, then $m | a$.''
\end{itemize}
% (Corollary 7.2 and Lemma 7.3 in lecture notes 7). The
%first statement does not hold in general if $c$ and $n$ are not
%relatively prime (Corollary 10.3 in lecture notes 7). 
%The third holds.
%The third does not hold, for instance, if $n=3$, $a=2$, $b=5$, $c=0$ and $d=3$.
\end{solution}
\fi


\ppart The following statements about trees:

\begin{itemize}

\item Any connected subgraph is a tree.  \hfill \textbf{true} \qquad \textbf{false} \examspace[0.4in]

\begin{solution}
\textbf{true}
\end{solution}

\item Adding an edge between two nonadjacent vertices creates a cycle. \hfill 
\textbf{true} \qquad \textbf{false}  \examspace[0.4in]

\begin{solution}
\textbf{true}
\end{solution}

\item The number of vertices is one less than twice the number of
  leaves.  \hfill 
\textbf{true} \qquad \textbf{false}  \examspace[0.4in]

\begin{solution}
  \textbf{false}.  This property holds for full binary trees, but not in
  general.  A tree with two vertices is a counterexample.
\end{solution}

\end{itemize}

\eparts

\examspace[0.2in]
Now answer the following:

\bparts

\ppart Which of the properties below are preserved under graph
isomorphism?  Write their numbers: \hfill\examrule[1in]

\begin{enumerate}

\item There is a cycle that includes all the vertices.

\item Two edges are of equal length.

\item The graph remains connected if any two edges are removed.

\item There exists an edge that is an edge of every spanning tree.

\item The negation of a property that is preserved under isomorphism.

\end{enumerate}
%% Removed
%\item There are two degree 8 vertices. % too easy
%\item The area enclosed by a graph equals some constant, $A$.  % true for trees?
%\item There are two cycles that do not share any vertices. % too easy
%\item There are two connected components. % too easy
%\item The graph can be drawn such that all the edges have the same length. % huh?
%\item The graph is 4-colorable. % too easy
%\item Adding an edge between any two vertices creates a cycle. % spanning tree one better

\begin{solution}
All but the second one are preserved.
\end{solution}

\ppart What is the minimum number of vertices possible in a nonplanar
graph?  \hfill \examrule[0.5in]

\begin{solution}
5.  $K_5$ is the smallest nonplanar graph.
\end{solution}

\ppart What is the minimum number of edges possible in a nonplanar
graph that is 2-colorable?  \hfill \examrule[0.5in]

\begin{solution}
9.  $K_{3,3}$ is the smallest 2-colorable nonplanar graph.
\end{solution}

\iffalse

\ppart A \term{sink} in a digraph is a vertex with no edges leaving
it.  Circle whichever of the following assertions are true of
\idx{stable distributions} on finite digraphs with exactly two sinks:

\begin{itemize}

\item there may not be any

\item there may be a unique one

\item there are exacty two

\item there may be a countably infinite number

\item there may be a uncountable number

\item there always is an uncountable number

\end{itemize}

\begin{solution}
The first three choices are false, and the last three are true.
That's because a distribution in which one sink has probability $r \in
[0,1] \subseteq \reals$ and the other sink has probability $1-r$ is
stable, and there are an is uncountable number of real numbers in $[0,1]$.
\end{solution}

\ppart \TBA{pulverizer, inverse mod $n$}

\ppart \TBA{sampling, confidence}

\ppart \TBA{Chernoff}
\fi

\eparts

\end{problem}

\endinput
