\documentclass[problem]{mcs}

\begin{pcomments}
\pcomment{MQ_more_counting_practice}
\pcomment{extracted from CP_more_counting}
\pcomment{from: S09.cp10m commented out; S07 cp10f? maybe use S08.10f instead?}
\pcomment{last part extended based on FP_more_counting.(c).(d) by ARM
  4/7/10}
\pcomment{solns for $\le 100$ changed to positive from nonnegative by ARM 4/12/11}
\end{pcomments}

\pkeywords{
  counting
  counting_rules
  bijection
}

%%%%%%%%%%%%%%%%%%%%%%%%%%%%%%%%%%%%%%%%%%%%%%%%%%%%%%%%%%%%%%%%%%%%%
% Problem starts here
%%%%%%%%%%%%%%%%%%%%%%%%%%%%%%%%%%%%%%%%%%%%%%%%%%%%%%%%%%%%%%%%%%%%%


\begin{problem}
\iffalse
Solve the following counting problems by defining an appropriate
mapping (bijective or $k$-to-1) between a set whose size you know and
the set in question.
\fi

\bparts


\ppart How many solutions over the \emph{positive} integers are there to the
inequality:

\begin{equation*}
x_1 + x_2 + \ldots + x_{10} \leq 100
\end{equation*}

\begin{solution}
\[
\binom{90+10}{10} \,.
\]

  There is a bijection between solutions and bit-strings
  $0^{x_1-1}10^{x_2-1}1\dots 0^{x_9-1}10^{x_{10}-1}10^k$ with $x_i >0$ and
  $k+\sum_1^{10} x_i =100$.  So the number of solutions is the same as
  the number of bit-strings with ten 1's and number of 0's equal to
\[
k+\sum_1^{10} (x_i-1)  =   \paren{k+\sum_1^{10} x_i}-10 = 100-10 = 90.
\]
\end{solution}

\ppart In how many ways can Mr. and Mrs. Grumperson distribute 13
identical pieces of coal to their three children for Christmas so that
each child gets at least one piece?

\begin{solution}
\[
\binom{12}{2}\, .
\]

There is an obvious bijection between distributions of coal to
children and bit strings $0^{a+1}10^{b+1}10^{c+1}$ where
$(a+1)+(b+1)+(c+1) = 13$, namely such a string corresponds to
distributing $a+1$ coals to the first child, $b+1$ coals to the
second, and $c+1$ coals to the third.  There is also an obvious
bijection between such bit strings and bitstrings of the form
$0^{a}10^{b}10^{c}$ where
$a+b+c = 10$, that is, bit-strings with ten 0's and two 1's.
\end{solution}


\eparts
\end{problem}

\endinput
