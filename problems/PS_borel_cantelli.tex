\documentclass[problem]{mcs}

\begin{pcomments}
  \pcomment{PS_borel_cantelli}
  \pcomment{from F02.cp15m}
\end{pcomments}

\pkeywords{
  murphys_law
  murphy
  borel  
  cantelli
  mutually_independent
}

%%%%%%%%%%%%%%%%%%%%%%%%%%%%%%%%%%%%%%%%%%%%%%%%%%%%%%%%%%%%%%%%%%%%%
% Problem starts here
%%%%%%%%%%%%%%%%%%%%%%%%%%%%%%%%%%%%%%%%%%%%%%%%%%%%%%%%%%%%%%%%%%%%%

\begin{problem}

\begin{editingnotes}
label{BC} The first Borel-Cantelli lemma.
\end{editingnotes}

An infinite version of Murphy's Law is that if an infinite number of
mutually independent events are expected to happen, then the
probability that only finitely many happen is 0.  This is known as the
first\term{Borel-Cantelli lemma}.

\bparts

\ppart\label{no-event-pr0} 

Let $A_0,A_1,\dots$ be any infinite sequence of mutually independent
events such that
\begin{equation}\label{sai}
\sum_{n \in \naturals} \pr{A_n} = \infty.
\end{equation}
Prove that $\prob{\text{no $A_n$ occurs}} = 0$.

\hint $B_k$ the event that no $A_n$ with $n \leq k$ occurs.  So the
event that no $A_n$ occurs is 
\[
B \eqdef \lgintersect_{k \in \naturals} B_k.
\]
Apply Murphy's Law, Theorem~\bref{th:murphy}, to $B_k$.

\begin{solution}
By Murphy's Law
\[
\prob{B_k} \leq e^{-\sum_{n=0}^k \pr{A_n}}.
\]
Since
\[
\lim_{k \to \infty} \sum_{n=0}^k \pr{A_n} = \infty,
\]
we conclude that
\[
\lim_{k \to \infty} \prob{B_k} \leq \lim_{k \to \infty}  e^{-\sum_{n=0}^k \pr{A_n}} = 0.
\]
But the event $B \subset B_k$, so $\prob{B} \leq \prob{B_k}$, which implies
\[
\pr{B} = 0.
\]
\end{solution}

\ppart Conclude that $\prob{\text{only finitely many $A_n$'s occur}} =
0$.

\hint Let $C_k$ be the event that no $A_n$ with $n \geq k$ occurs.  So
the event that only finitely many $A_n$'s occur is
\[
C \eqdef \lgunion_{k \in \naturals} C_k.
\]
Apply part~\eqref{no-event-pr0} to $C_k$.

\begin{solution}
Since $\sum_{n \in \naturals} \pr{A_n}$ is infinite, so is $\sum_{n
  \geq k} \pr{A_n}$.  So part~\eqref{no-event-pr0} applies to the
sequence $A_k,A_{k+1},A_{k+2},\dots$ and implies that the probability
that no $A_n$ occurs for $n \geq k$ is 0.  That is, $\prob{C_k} =0$.
Therefore
\[
\prob{C} = \prob{\lgunion_{k \in \naturals} C_k}
         \leq \sum_{k \in \naturals} \prob{C_k} = \sum_{k \in
           \naturals} 0 = 0.
\]

\end{solution}

\eparts

\end{problem}

%%%%%%%%%%%%%%%%%%%%%%%%%%%%%%%%%%%%%%%%%%%%%%%%%%%%%%%%%%%%%%%%%%%%%
% Problem ends here
%%%%%%%%%%%%%%%%%%%%%%%%%%%%%%%%%%%%%%%%%%%%%%%%%%%%%%%%%%%%%%%%%%%%%

\endinput
 
