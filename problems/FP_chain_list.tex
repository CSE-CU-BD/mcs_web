\documentclass[problem]{mcs}

\begin{pcomments}
  \pcomment{FP_chain_list}
  \pcomment{ARM 4/1/16}
\end{pcomments}

\pkeywords{
  partial_order
  maximum
  linear
  chain
  comparable
}

%%%%%%%%%%%%%%%%%%%%%%%%%%%%%%%%%%%%%%%%%%%%%%%%%%%%%%%%%%%%%%%%%%%%%
% Problem starts here
%%%%%%%%%%%%%%%%%%%%%%%%%%%%%%%%%%%%%%%%%%%%%%%%%%%%%%%%%%%%%%%%%%%%%

\begin{problem} 
Let $R$ be a weak partial order on a set, $A$.  Suppose $C$ is a
finite chain.\footnote{A set $C$ is a \emph{chain} when it is
  nonempty, and all elements $c,d \in C$ are comparable.  Elements $c$
  and $d$ are \emph{comparable} iff $[c \mrel{R} d\ \QOR\ d \mrel{R}
    c]$.  \iffalse A partial order for which every two different
  elements are comparable is called a \emph{linear order}\fi}

\bparts

\ppart\label{hasmax} Prove that $C$ has a maximum element.  \hint Induction on the
size of $C$
\examspace[3.0in]

\begin{solution}
As hinted, we give a proof by induction on the size of $C$.  Let the
inductive hypothesis be the following:

\begin{quote}
$P(n) :=$ If $C$ is a finite chain of size $n$, then $C$ has a maximum
  element.
\end{quote}

Since $C$ is nonempty, the base case is given by $n = 1$. In this
case, since there is only one element in the chain, that element must
be the maximum.

Consider now the inductive step.  We want to prove $P(n+1)$.  Consider
any chain of size $n+1$ (call it $C_{n+1}$), and remove an element $x$
from it.  The remaining set is a chain of size $n$ (call it $C_{n}$),
so by $P(n)$ it must be the case that it has a maximum element.  Call
this maximum element $m$. For every other element $e$ in $C_{n}$, we
have that $e \mrel{R} m$.  Compare now $x$ and $m$. If $x \mrel{R} m$,
then for every other element $e$ in $C_{n+1}$, we have that $e
\mrel{R} m$ so $m$ is a maximum element of $C_{n+1}$.  Otherwise, if
$m \mrel{R} x$, it must follow by transitivity of $R$ that for every
other element $e$ in $C_{n+1}$, we have that $e \mrel{R} x$ so $x$ is
a maximum element of $C_{n+1}$.  This shows that in all cases
$C_{n+1}$ has a maximum element, which proves $P(n+1)$.  By induction,
$C$ always has a maximum element.

\end{solution}

\ppart Conclude that there is a unique sequence of all the elements of
$C$ that is strictly increasing.

\examspace[3.0in]

\begin{solution}
We can use part~\eqref{hasmax} to construct a strictly increasing
sequence.  Given $C$, remove the maximum element $m_0$.  This gives a
chain $C'$, whose maximum element is $m_1$. Remove $m_1$; the
resulting chain $C''$ has a maximum element $m_2$. Continuing this
process until there is only one element $m_n$ in the set yields a
sequence of elements $m_0,m_1, \dots, m_n$ such that $m_i \mrel{R}
m_{j}$ for all $j < i$.  By antisymmetry of $R$, it follows that $m_n,
m_{n-1},\dots,m_1, m_0$ is a strictly increasing sequence.

As a last note, since $m_i \mrel{R} m_{j}$ for all $j < i$, any other
sequence is not strictly increasing.  To see this, consider any
sequence different than the one given.  Since the sequence is
different, there must be at least one pair of elements $(m_i, m_j)$
such that $m_i$ occurs after $m_j$ but $m_i \mrel{R} m_{j}$. Clearly,
the sequence cannot be strictly increasing.
\end{solution}

\eparts

\end{problem} 


%%%%%%%%%%%%%%%%%%%%%%%%%%%%%%%%%%%%%%%%%%%%%%%%%%%%%%%%%%%%%%%%%%%%%
% Problem ends here
%%%%%%%%%%%%%%%%%%%%%%%%%%%%%%%%%%%%%%%%%%%%%%%%%%%%%%%%%%%%%%%%%%%%%

\endinput


\endinput
