\documentclass[problem]{mcs}

\begin{pcomments}
  \pcomment{MQ_fibonacci_by_induction}
  \pcomment{taken from book/recursive_data.tex staff notes.}
\end{pcomments}

\pkeywords{
  induction
  fibonacci
  recurrence
}

%%%%%%%%%%%%%%%%%%%%%%%%%%%%%%%%%%%%%%%%%%%%%%%%%%%%%%%%%%%%%%%%%%%%%
% Problem starts here
%%%%%%%%%%%%%%%%%%%%%%%%%%%%%%%%%%%%%%%%%%%%%%%%%%%%%%%%%%%%%%%%%%%%%

\begin{problem} 
\inhandout{The Fibonacci numbers $F(0) F(1) F(2), \dots$ are defined as follows:
\[
F(n) \eqdef \begin{cases}
  0               & \mbox{if $n = 0$},\\
  1               & \mbox{if $n = 1$},\\
  F(n-1) + F(n-2) & \mbox{if $n >1$}.
\end{cases}
\]}
\inbook{The Fibonacci numbers $F(n)$ are described in
  Section~\bref{sec:fib}.}

These numbers satisfy many unexpected identities, such as
\begin{equation}\label{f02f12fnfn1}
F(0)^2 + F(1)^2 + \cdots + F(n)^2 = F(n)F(n+1).
\end{equation}
Equation~\eqref{f02f12fnfn1} can be proved to hold for all $n \in
\nngint$ by induction, using the equation itself as the induction
hypothesis $P(n)$.

\bparts

\ppart  Prove the \inductioncase{base case} $(n=0)$.

\begin{solution}
  \[
\sum_{i=0}^0 F(i)^2 = F(0)^2 = 0 = 0\cdot 1 = F(0)F(1).
\]
Therefore, $P(0)$ is true.
\end{solution}

\examspace[2in]

\ppart Now prove the \inductioncase{inductive step}.

\begin{solution}
We need to prove that $P(n)$:
\[
\sum_{i=0}^{n} F(i)^2 = F(n)F(n+1)
\]
implies $P(n+1)$:
\[
\sum_{i=0}^{n+1} F(i)^2 = F(n+1) F(n+2).
\]

\begin{proof}
  \begin{align*}
    \sum_{i=0}^{n+1} F(i)^2 & = \sum_{i=0}^n F(i)^2 + F(n+1)^2 \\
    & = F(n) F(n+1) + F(n+1)^2 & \mbox{(by $P(n)$)}\\
    & = F(n+1)\paren{F(n)+F(n+1)} \\
    & = F(n+1)F(n+2)
 & \mbox{(def of Fibonacci sequence).}
  \end{align*}
\end{proof}

\end{solution}

\eparts

\end{problem}

%%%%%%%%%%%%%%%%%%%%%%%%%%%%%%%%%%%%%%%%%%%%%%%%%%%%%%%%%%%%%%%%%%%%%
% Problem ends here
%%%%%%%%%%%%%%%%%%%%%%%%%%%%%%%%%%%%%%%%%%%%%%%%%%%%%%%%%%%%%%%%%%%%%

\endinput
