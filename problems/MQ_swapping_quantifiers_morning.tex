\documentclass[problem]{mcs}

\begin{pcomments}
  \pcomment{MQ_swapping_quantifiers_morning}
\pcomment{grading was hard because we didnt make it clear how exactly to mark whether something was a countermodel
or not, we recommend you constrain this further than we did.}
\end{pcomments}

\pkeywords{
  quantifiers
  predicate_logic
}

%%%%%%%%%%%%%%%%%%%%%%%%%%%%%%%%%%%%%%%%%%%%%%%%%%%%%%%%%%%%%%%%%%%%%
% Problem starts here
%%%%%%%%%%%%%%%%%%%%%%%%%%%%%%%%%%%%%%%%%%%%%%%%%%%%%%%%%%%%%%%%%%%%%

\begin{problem} \mbox{}

The following predicate logic formula is invalid:

\[\forall x, \exists y. P(x, y) \implies \exists y, \forall x. P(x, y)\]

Which of the following are counter models for the implication above?

\begin{enumerate}
\examspace[0.3in]
\item \examrule[0.5in] The predicate $P(x, y) = \mbox{`}yx = 1\mbox{'}$ where the domain of discourse is $\mathbb{Q}$.
\examspace[0.3in]
\item \examrule[0.5in] The predicate $P(x, y) = \mbox{`}y < x\mbox{'}$ where the domain of discourse is $\mathbb{R}$. 
\examspace[0.3in]
\item \examrule[0.5in] The predicate $P(x, y) = \mbox{`}yx = 2\mbox{'}$ where the domain of discourse is $\mathbb{R}$ without 0.
\examspace[0.3in]
\item \examrule[0.5in] The predicate $P(x, y) = \mbox{`}yxy = x\mbox{'}$ where the domain of discourse is the set of all binary strings, including the empty string.
\end{enumerate}

\begin{solution}
\begin{enumerate}
\item In the rationals, 0 has no inverse. Hence the hypothesis is false, since not all rationals
have inverses. An implication with a false hypothesis is automatically true, so this is not a countermodel.
\item COUNTERMODEL. For every real number $x$, there exists a real number $y$ which is 
strictly less than $x$.  So while the antecedent of the implication is true, 
the consequence is not since there is no minimum element for the partial order,
 the strictly less than relation, $<$,  on $\mathbb{R}$. 
\item COUNTERMODEL. in this case the hypothesis is true, but the conclusion is not: its not possible to find a single number that will do this.
\item In the set of binary strings, both sides of the implication are true if we let $y = \lambda$, the empty string.
\end{enumerate}
\end{solution}

\end{problem}

%%%%%%%%%%%%%%%%%%%%%%%%%%%%%%%%%%%%%%%%%%%%%%%%%%%%%%%%%%%%%%%%%%%%%
% Problem ends here
%%%%%%%%%%%%%%%%%%%%%%%%%%%%%%%%%%%%%%%%%%%%%%%%%%%%%%%%%%%%%%%%%%%%%
