\documentclass[problem]{mcs}

\begin{pcomments}
  \pcomment{MQ_swapping_quantifiers_morning}
  \pcomment{wordsmithed ARM 9/15/17}
\end{pcomments}

\pkeywords{
  quantifiers
  predicate_logic
  valid
  counter_model
}

%%%%%%%%%%%%%%%%%%%%%%%%%%%%%%%%%%%%%%%%%%%%%%%%%%%%%%%%%%%%%%%%%%%%%
% Problem starts here
%%%%%%%%%%%%%%%%%%%%%%%%%%%%%%%%%%%%%%%%%%%%%%%%%%%%%%%%%%%%%%%%%%%%%

\begin{problem} \mbox{}

The following predicate logic formula is invalid:
\begin{equation}\label{axeypxyimp}
[\forall x, \exists y. P(x, y) \implies \exists y, \forall x. P(x, y)
\end{equation}

\inbook{Indicate which of the following are counter models for~\eqref{axeypxyimp}, and briefly explain.}
\inhandout{Circle the items describing counter models for~\eqref{axeypxyimp}, and briefly explain.}

\begin{enumerate}
\examspace[0.3in]

\item \examrule[0.5in] The predicate $P(x, y) = \mbox{`}y\cdot x =
  1\mbox{'}$ where the domain of discourse is $\mathbb{Q}$.

\examspace[0.3in]
\item \examrule[0.5in] The predicate $P(x, y) = \mbox{`}y < x\mbox{'}$
  where the domain of discourse is $\mathbb{R}$.

\examspace[0.3in]

\item \examrule[0.5in] The predicate $P(x, y) = \mbox{`}y\cdot x =
  2\mbox{'}$ where the domain of discourse is $\mathbb{R}$ without 0.
\examspace[0.3in]

\item \examrule[0.5in] The predicate $P(x, y) = \mbox{`}yxy =
  x\mbox{'}$ where the domain of discourse is the set of all binary
  strings, including the empty string.
\end{enumerate}

\begin{solution}
\begin{enumerate}

\item In the rationals, 0 has no inverse. Hence the hypothesis is
  false, since not all rationals have inverses. An implication with a
  false hypothesis is automatically true, so this is not a
  countermodel.

\item COUNTERMODEL. For every real number $x$, there exists a real
  number $y$ which is strictly less than $x$.  So while the antecedent
  of the implication is true, the consequence is not since there is no
  minimum element for the partial order, the strictly less than
  relation $<$ on $\mathbb{R}$.

\item COUNTERMODEL. In this case the hypothesis is true, but the
  conclusion is not: its not possible to find a single number that
  will do this.

\item In the set of binary strings, both sides of the implication are
  true if we let $y = \emptystring$, the empty string.
\end{enumerate}
\end{solution}

\end{problem}

%%%%%%%%%%%%%%%%%%%%%%%%%%%%%%%%%%%%%%%%%%%%%%%%%%%%%%%%%%%%%%%%%%%%%
% Problem ends here
%%%%%%%%%%%%%%%%%%%%%%%%%%%%%%%%%%%%%%%%%%%%%%%%%%%%%%%%%%%%%%%%%%%%%
