\documentclass[problem]{mcs}

\begin{pcomments}
  \pcomment{CP_multinomial_fermat}
  \pcomment{old name: CP_multinomial_euler}
  \pcomment{old name: CP_multinomial_theorem_2}
  \pcomment{from: S09.cp12m}
\end{pcomments}

\pkeywords{
  multinomial
  Fermat
  mod
  counting
}

%%%%%%%%%%%%%%%%%%%%%%%%%%%%%%%%%%%%%%%%%%%%%%%%%%%%%%%%%%%%%%%%%%%%%
% Problem starts here
%%%%%%%%%%%%%%%%%%%%%%%%%%%%%%%%%%%%%%%%%%%%%%%%%%%%%%%%%%%%%%%%%%%%%

\begin{problem}

\bparts

\ppart Use the Multinomial Theorem~\bref{multinom-thm} to prove that
\begin{equation}\label{xp}
(x_1 + x_2 + \cdots +x_n)^p \equiv x_1^p + x_2^p + \cdots +x_n^p \pmod p
\end{equation}
for all primes $p$.  (Do not prove it using Fermat's ``little''
Theorem.  The point of this problem is to offer an independent proof
of Fermat's theorem.)

\hint Explain why $\binom{p}{k_1, k_2, \dots, k_n}$ is divisible by $p$ if
all the $k_i$'s are positive integers less than $p$.

\begin{solution}
  By the Multinomial Theorem~\bref{multinom-thm}, $(x_1 + x_2 + \cdots
  +x_n)^p$ is a sum of monomials in $x_1,\dots,x_n$ whose coefficients are
\[
\binom{p}{k_1, k_2, \dots, k_n}
\]
where the sum of the $k_i$'s is $p$.  But if all the $k_i$'s are less than
$p$, then none of the denominator terms divides the numerator, $p$, and
so the multinomial coefficient is divisible by $p$.  So the only
coefficients not divisible by $p$ are the coefficients of the terms
$x_i^p$, and all the other terms are $\equiv 0 \pmod p$.
\end{solution}

\ppart Explain how~\eqref{xp} immediately proves Fermat's Little
Theorem~\bref{fermat_little}: $n^{p-1} \equiv 1 \pmod p$ when $n$ is not a
multiple of $p$.

\begin{solution}
  Let $x_1=x_2=\cdots x_n = 1$.  Then~\eqref{xp} implies $n^p \equiv
  n\cdot 1^p =n \pmod p$.  If $n$ is not a multiple of $p$, then we can
  then cancel $n$ to get $n^{p-1} \equiv 1 \pmod p$.
\end{solution}

\eparts
\end{problem}


%%%%%%%%%%%%%%%%%%%%%%%%%%%%%%%%%%%%%%%%%%%%%%%%%%%%%%%%%%%%%%%%%%%%%
% Problem ends here
%%%%%%%%%%%%%%%%%%%%%%%%%%%%%%%%%%%%%%%%%%%%%%%%%%%%%%%%%%%%%%%%%%%%%
\endinput
