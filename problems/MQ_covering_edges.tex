\documentclass[problem]{mcs}

\begin{pcomments}
  \pcomment{MQ_covering_edges}
  \pcomment{from CP_covering_edges, edited by Rich & ARM}
  \pcomment{from: S10.mq3}
\end{pcomments}

\pkeywords{
  relations
  digraphs
  covering_edges
  transitive_closure
  DAGs
  positive_path_relation
}

%%%%%%%%%%%%%%%%%%%%%%%%%%%%%%%%%%%%%%%%%%%%%%%%%%%%%%%%%%%%%%%%%%%%%
% Problem starts here
%%%%%%%%%%%%%%%%%%%%%%%%%%%%%%%%%%%%%%%%%%%%%%%%%%%%%%%%%%%%%%%%%%%%%
\newcommand{\covering}[1]{\text{covering}\paren{#1}}

\begin{problem}
  
   \emph{Covering edges} were introduced in class problem: if $a$ and
   $b$ are distinct vertices of a digraph, then $a$ is said to
   \term{cover} $b$ if there is an edge from $a$ to $b$ and every path
   from $a$ to $b$ traverses this edge.  If $a$ covers $b$, the edge
   from $a$ to $b$ is called a \term{covering edge}.

   Let $D$ be a finite directed acyclic graph (DAG).
   
  \bparts

  \ppart\label{isalongpath} If there is a path in $D$ from a vertex,
  $u$, to vertex, $v$, explain why there must be a \emph{longest} path
  from $u$ to $v$.

  \examspace[1in]

\begin{solution}
   If $D$ has $m$ vertices, then no path can be longer $m-1$
   ---otherwise some vertex must repeat on the path, which means there
   would be a cycle, contradicting the fact that $D$ is a DAG.  So
   there must be a \emph{longest} path from $u$ to $v$.  (Technically,
   this follows from the Well Ordering Principle applied to the set
   $\set{v - n \in \naturals \suchthat \text{there is a path of length
       $n$ from $u$ to $v$}}$.)
\end{solution}

  \ppart Give a proof of the following claim from the class problem:
    
   \begin{claim*}
     If there is a path in $D$ from a vertex, $u$, to vertex, $v$, then
     there is a path from $u$ to $v$ that only traverses covering
     edges.
   \end{claim*}

  \examspace[3in]
  
  \begin{solution}
   By part~\eqref{isalongpath}, there is a longest path from $u$ to
   $v$.  If some edge on this path was not a covering edge, then by
   definition there is a path of length 2 or more between its
   endpoints, and replacing this edge by the path would yield a longer
   path from $u$ to $v$, a contradiction.  Hence all edges must be
   covering edges.
  \end{solution}
  
  \ppart Show that the Claim fails for the finite digraph, $F$, with
  three vertices and edges from every vertex to every other vertex.
  \hint What are the covering edges of $F$?

  \begin{solution}
    There are no covering edges in $F$, since for each edge
    $\diredge{u}{v}$ there is a length 2 path $uwv$ through the
    remaining vertex, $w$, that does not traverse this edge.  So there
    is no path of covering edges from any vertex to any other vertex.
  \end{solution}
  
\eparts

\end{problem}
