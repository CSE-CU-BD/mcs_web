\documentclass[problem]{mcs}

\begin{pcomments}
  \pcomment{MQ_covering_edges}
  \pcomment{verbatim from CP_covering_edges part b, by Rich}
  \pcomment{from: S10.mq3}
\end{pcomments}

\pkeywords{
  relations
  digraphs
  covering_edges
  transitive_closure
  DAGs
  positive_path_relation
}

%%%%%%%%%%%%%%%%%%%%%%%%%%%%%%%%%%%%%%%%%%%%%%%%%%%%%%%%%%%%%%%%%%%%%
% Problem starts here
%%%%%%%%%%%%%%%%%%%%%%%%%%%%%%%%%%%%%%%%%%%%%%%%%%%%%%%%%%%%%%%%%%%%%

\begin{problem}
  
  If $a$ and $b$ are distinct nodes of a digraph, then $a$ is said to
  \term{cover} $b$ if there is an edge from $a$ to $b$ and every path from
  $a$ to $b$ traverses this edge.  If $a$ covers $b$, the edge from $a$ to
  $b$ is called a \term{covering edge}.

  \newcommand{\covering}[1]{\text{covering}\paren{#1}}

  \label{cover-ok} Let $\covering{D}$ be the subgraph of $D$
  consisting of only the covering edges.  Suppose $D$ is a finite \idx{DAG}.
  Explain why $\covering{D}$ has the same positive path relation as $D$.

  \hint Consider \emph{longest} paths between a pair of vertices.

  \begin{solution}
    What we need to show is that if there is a path in $D$ between
    vertices $a \neq b$, then there is a path consisting only of covering
    edges from $a$ to $b$.  But since $D$ is a finite DAG, there must be a
    \emph{longest} path from $a$ to $b$.  Now every edge on this path must be a
    covering edge or it could be replaced by a path of length 2 or more,
    yielding a longer path from $a$ to $b$.
  \end{solution}

\end{problem}
