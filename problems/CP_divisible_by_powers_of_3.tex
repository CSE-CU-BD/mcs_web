\documentclass[problem]{mcs}

\begin{pcomments}
  \pcomment{CP_divisible_by_powers_of_3}
  \pcomment{from: F08 Rec4 P4}
\end{pcomments}

\pkeywords{
  divides
  induction
}

%%%%%%%%%%%%%%%%%%%%%%%%%%%%%%%%%%%%%%%%%%%%%%%%%%%%%%%%%%%%%%%%%%%%%
% Problem starts here
%%%%%%%%%%%%%%%%%%%%%%%%%%%%%%%%%%%%%%%%%%%%%%%%%%%%%%%%%%%%%%%%%%%%%


\begin{problem}
Let $N$ be a number whose arabic (decimal) numeral consists of $3^n$
identical digits, for example, $N = 4, 555, 777777777,\dots$.  Show by
induction that $3^n \divides N$.  For example:
\[
3^2 \text{ divides } \underbrace{777777777}_{\text{$3^2 = 9$ digits}}
\]

\hint You may assume the fact that $3$ divides a number iff it divides
the sum of its digits.

\begin{solution}
We proceed by induction on $n$.  The induction hypothesis is:
\[
P(n) \eqdef 3^n \text{ divides } N,
\]
where the arabic numeral of $N$ consists of $3^n$ identical digits.

\inductioncase{Base case}: $(n=0)$.  $P(0)$ is true because $3^0 = 1$
divides every number.

\inductioncase{Inductive step}.  Now we show that, for all $n\geq
0$, $P(n)$ implies $P(n+1)$.  Fix any $n\geq 0$ and assume $P(n)$ is
true. Consider a number whose decimal expansion consists of $3^{n+1}$
copies of the digit $a$:
\begin{align*}
\underbrace{aaaaaa \dots aaaaaa}_{\text{$3^{n+1}$ digits}}
    & = \underbrace{aaa \dots aaa}_{\text{$3^n$ digits}}
        \underbrace{aaa \dots aaa}_{\text{$3^n$ digits}}
        \underbrace{aaa \dots aaa}_{\text{$3^n$ digits}} \\
    & = \underbrace{aaa \dots aaa}_{\text{$3^n$ digits}}
        \quad \cdot \quad
        1
        \underbrace{000 \dots 001}_{\text{$3^n$ digits}}
        \underbrace{000 \dots 001}_{\text{$3^n$ digits}} \\
\end{align*}
Now $3^n$ divides the first term by the assumption $P(n)$, and 3
divides the second term since the digits sum to 3.  Therefore, the
whole expression is divisible by $3^{n+1}$.  This proves $P(n+1)$.

By the principle of induction $P(n)$ is true for all $n \geq 0$.

\end{solution}
\end{problem}


%%%%%%%%%%%%%%%%%%%%%%%%%%%%%%%%%%%%%%%%%%%%%%%%%%%%%%%%%%%%%%%%%%%%%
% Problem ends here
%%%%%%%%%%%%%%%%%%%%%%%%%%%%%%%%%%%%%%%%%%%%%%%%%%%%%%%%%%%%%%%%%%%%%

\endinput

