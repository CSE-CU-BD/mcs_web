\documentclass[problem]{mcs}

\begin{pcomments}
  \pcomment{CP_birthday_deviation}
  \pcomment{new S10 revised from book; edited by ARM 4/31/11}
\end{pcomments}

\pkeywords{
  deviation
  pairwise_independent
  birthday
  Chebyshev
}

%%%%%%%%%%%%%%%%%%%%%%%%%%%%%%%%%%%%%%%%%%%%%%%%%%%%%%%%%%%%%%%%%%%%%
% Problem starts here
%%%%%%%%%%%%%%%%%%%%%%%%%%%%%%%%%%%%%%%%%%%%%%%%%%%%%%%%%%%%%%%%%%%%%
\begin{problem}

\begin{staffnotes}
This problem is a direct application of Chebyshev's bound using
pairwise independent additivity of variance (not sampling).  It comes
pretty directly from Section~\bref{bday_deviation_subsec}.  It's OK
for students to look that up.
\end{staffnotes}

Let $B_1,B_2,\dots,B_n$ be mutually independent random variables with
a uniform distribution on the integer interval $[1,d]$.  Let $D$ equal
to the number of events $[B_i = B_j]$ that happen where $i \neq j$.
It was observed in Section~\bref{birthday_principle_sec} (and proved
in Problem~\bref{PS_equal_birthdays}) that $\pr{B_i = B_j} = 1/d$ for
$i \neq j$ and that the events $[B_i = B_j]$ are pairwise independent.

Let $E_{i,j}$ be the indicator variable for the event $[B_i = B_j]$.

\bparts

\ppart\label{varEij} What are $\expect{E_{i,j}}$ and
$\variance{E_{i,j}}$ for $i \neq j$?

\begin{solution}
For $i \neq j$,
\begin{align}
\expect{E_{i,j}}   & = \frac{1}{d}\label{expeij}\\
\variance{E_{i,j}} & = \frac{1}{d}\paren{1- \frac{1}{d}}.\label{vareij}
\end{align}

Equation~\ref{expeij} follows from the fact that $\pr{B_i = B_j} =
1/d$ and that $\expect{E_{i,j}} = \pr{B_i = B_j}$ by
Lemma~\bref{expindic}.

Equation~\ref{vareij} follows from~\eqref{expeij} and the formula for
the variance of an indicator variable given in
Lemma~\bref{bernoulli-variance}.

\iffalse
\[
\variance{E_{i,j}} = \pr{B_i = B_j}(1- \pr{B_i = B_j})
   = \frac{1}{d}\paren{1- \frac{1}{d}}.
\]
\fi

\end{solution}

\ppart What are $\expect{D}$ and $\variance{D}$?\label{expvarD}

\begin{staffnotes}
\hint $D \eqdef \sum_{1\le i < j \le n} E_{i,j}$
\end{staffnotes}

\begin{solution}
By definition
\[
D = \sum_{1\le i < j \le n} E_{i,j},
\]
so by linearity of expectation
\[
\expect{D} = \Expect{\sum_{1\le i < j \le n} E_{i,j}} = \sum_{1\le i <
  j \le n} \expect{E_{i,j}} = \binom{n}{2}\cdot \frac{1}{d}.
\]

\begin{align*}
\variance{D} & = \Variance{\sum_{1\le i < j \le n} E_{i,j}}\\ & =
\sum_{1\le i < j \le n} \variance{E_{i,j}} & \text{(pairwise
  independent additivity)}\\ & = \binom{n}{2} \cdot
\frac{1}{d}\paren{1-\frac{1}{d}} & \text{(by part~\eqref{varEij})}.
\end{align*}

\end{solution}

\ppart In a 6.01 class of 500 students, the youngest student was born
15 years ago and the oldest 35 years ago.  Let $D$ be the number of
students in the class who were born on exactly the same date.  What is
the probability that $4 \leq D \leq 32$?  (For simplicity, assume that
the distribution of birthdays is uniform over the 7305 days in the two
decade interval from 35 years ago to 15 years ago.)

\begin{solution}
For a class of $n= 500$ students with $d=7305$ possible days, we have
from part~\eqref{expvarD}
\[
\expect{D} = 500\cdot 499/2 \cdot \frac{1}{7305} \approx 17.1
\]
and
\[
\variance{D} = 500\cdot 499/2 \cdot \frac{1}{7305} \cdot
\paren{1-\frac{1}{7305}} < 17.1\, .
\]
So by Chebyshev's Theorem
\[
\pr{\abs{S - 17.1} \geq 14} < \frac{17.1}{14^2} < 0.09.
\]
We conclude that there is a better than 91\% chance that there will be
between 4 and 32 pairs of students born on the same date.
\end{solution}

\eparts

\end{problem}

%%%%%%%%%%%%%%%%%%%%%%%%%%%%%%%%%%%%%%%%%%%%%%%%%%%%%%%%%%%%%%%%%%%%%
% Problem ends here
%%%%%%%%%%%%%%%%%%%%%%%%%%%%%%%%%%%%%%%%%%%%%%%%%%%%%%%%%%%%%%%%%%%%%

\endinput
