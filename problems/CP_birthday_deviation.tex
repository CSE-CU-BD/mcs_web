\documentclass[problem]{mcs}

\begin{pcomments}
  \pcomment{CP_birthday_deviation}
  \pcomment{new S10 revised from book; edited by ARM 4/31/11}
  \pcomment{revised 5/13/14}
\end{pcomments}

\pkeywords{
  deviation
  pairwise_independent
  birthday
  Chebyshev
}

%%%%%%%%%%%%%%%%%%%%%%%%%%%%%%%%%%%%%%%%%%%%%%%%%%%%%%%%%%%%%%%%%%%%%
% Problem starts here
%%%%%%%%%%%%%%%%%%%%%%%%%%%%%%%%%%%%%%%%%%%%%%%%%%%%%%%%%%%%%%%%%%%%%
\begin{problem}

\begin{staffnotes}
This problem is a direct application of Chebyshev's bound using
pairwise independent additivity of variance.  It comes pretty directly
from Section~\bref{bday_deviation_subsec}.  It's OK for students to
look that up.
\end{staffnotes}

Let $B_1,B_2,\dots,B_n$ be mutually independent random variables with
a uniform distribution on the integer interval $\Zintv{1}{d}$.  Let
$E_{i,j}$ be the indicator variable for the event $[B_i = B_j]$.

Let $M$ equal the number of events $[B_i = B_j]$ that are true, where
$1 \leq i < j\leq n$.  So
\[
M = \sum_{1\le i < j \le n} E_{i,j}.
\]

It was observed in Section~\bref{birthday_principle_sec} (and proved
in Problem~\bref{PS_equal_birthdays}) that $\pr{B_i = B_j} = 1/d$ for
$i \neq j$ and that the random variables $E_{i,j}$, where $1\le i < j
\le n$, are pairwise independent.

\bparts

\ppart\label{varEij} What are $\expect{E_{i,j}}$ and
$\variance{E_{i,j}}$ for $i \neq j$?

\begin{solution}
For $i \neq j$,
\begin{align}
\expect{E_{i,j}}   & = \frac{1}{d}\label{expeij},\\
\variance{E_{i,j}} & = \frac{1}{d} \cdot \paren{1- \frac{1}{d}}.\label{vareij}
\end{align}

Equation~\eqref{expeij} follows from the fact that $\pr{B_i = B_j} =
1/d$ and that $\expect{E_{i,j}} = \pr{B_i = B_j}$ by
Lemma~\bref{expindic}.

Equation~\ref{vareij} follows from~\eqref{expeij} and the formula for
the variance of an indicator variable given in
Lemma~\bref{bernoulli-variance}.

\iffalse
\[
\variance{E_{i,j}} = \pr{B_i = B_j}(1- \pr{B_i = B_j})
   = \frac{1}{d}\paren{1- \frac{1}{d}}.
\]
\fi

\end{solution}

\ppart What are $\expect{M}$ and $\variance{M}$?\label{expvarD}

\begin{solution}
By definition,
\[
D = \sum_{1\le i < j \le n} E_{i,j},
\]
so by linearity of expectation,
\[
\expect{D} = \Expect{\sum_{1\le i < j \le n} E_{i,j}} = \sum_{1\le i <
  j \le n} \expect{E_{i,j}} = \binom{n}{2}\cdot \frac{1}{d}.
\]

\begin{align*}
\variance{D} & = \Variance{\sum_{1\le i < j \le n} E_{i,j}}\\
  & = \sum_{1\le i < j \le n} \variance{E_{i,j}}
       & \text{(pairwise independent additivity)}\\
  & = \binom{n}{2} \cdot\frac{1}{d} \cdot \paren{1-\frac{1}{d}}. & \text{(by part~\eqref{varEij})}
\end{align*}

\end{solution}

\ppart In a 6.01 class of 500 students, the youngest student was born
15 years ago and the oldest 35 years ago.  Show that more than half
the time, there will be will be between 12 and 23 pairs of students
who have the same birth date.  (For simplicity, assume that the
distribution of birthdays is uniform over the 7305 days in the two
decade interval from 35 years ago to 15 years ago.)

\hint Let $D$ be the number of pairs of students in the class who have
the same birth date.  Note that $\abs{D - \expect{D}} < 6 \QIFF D \in
\Zintv{12}{23}$.

\begin{solution}
For a class of $n= 500$ students with $d=7305$ possible days, we have
from part~\eqref{expvarD}
\[
\expect{D} = \frac{500 \cdot 499}{2} \cdot \frac{1}{7305} \approx 17.1
\]
and
\[
\variance{D} = \frac{500 \cdot 499}{2} \cdot \frac{1}{7305} \cdot
\paren{1-\frac{1}{7305}} < 17.1\, .
\]
Now we will use Chebyshev's bound to get an upper bound on the
probability that $D$ is between 12 and 23.  Since $D$ is an integer,
\[
\abs{D - \expect{D}} < 6\ \QIFF\ 11.1 \lessapprox D \lessapprox 23.1\ \QIFF\ D \in [12, 23].
\]
Chebyshev's bound gives
\[
\pr{\abs{D - \expect{D}} \geq 6} \leq \frac{\variance{D}}{6^2} < \frac{17.1}{6^2} < \frac{1}{2}.
\]
So there is a less than even chance that $D$ will be outside the
interval $\Zintv{12}{23}$.  That means there there is a better than
even chance that there will be between 12 and 23 pairs of students
with the same birth date.
\end{solution}

\eparts

\end{problem}

%%%%%%%%%%%%%%%%%%%%%%%%%%%%%%%%%%%%%%%%%%%%%%%%%%%%%%%%%%%%%%%%%%%%%
% Problem ends here
%%%%%%%%%%%%%%%%%%%%%%%%%%%%%%%%%%%%%%%%%%%%%%%%%%%%%%%%%%%%%%%%%%%%%

\endinput
   
