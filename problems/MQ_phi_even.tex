\documentclass[problem]{mcs}

\begin{pcomments}
  \pcomment{MQ_phi_even}
  \pcomment{ARM 4/5/14}
\end{pcomments}

\pkeywords{
  number_theory
  primes
  Eulers_function
  phi
  factoring
}

%%%%%%%%%%%%%%%%%%%%%%%%%%%%%%%%%%%%%%%%%%%%%%%%%%%%%%%%%%%%%%%%%%%%%
% Problem starts here
%%%%%%%%%%%%%%%%%%%%%%%%%%%%%%%%%%%%%%%%%%%%%%%%%%%%%%%%%%%%%%%%%%%%%

\begin{problem}
\bparts

\ppart\label{pdivnp-1} Show that if $p \divides n$ for some prime $p$ and integer $n
> 0$, then $(p-1) \divides \phi(n)$.

\begin{solution}
If $p \divides n$ for some prime $p$ and integer $n \geq 2$, then $n = m p^k$ where
$m$ is relatively prime to $p$ and $k > 1$.  By
Theorem~\bref{th:phi}),
\[
\phi(n) = \phi(m p^k)= \phi(m)\phi(p^k) = \phi(m) p^{k-1}(p-1).
\]
So $(p-1) \divides \phi(n)$ for all $n \geq 2$.
\end{solution}

\ppart Conclude that $\phi(n)$ is even for all $n>2$.

\begin{solution}
Any $n>2$ must have a prime factor $p > 2$, and since $ p$ must be odd, $p-1$ will be
even.  So $\phi(n)$ will also be even by part~\eqref{pdivnp-1}.
\end{solution}

\eparts
\end{problem}


%%%%%%%%%%%%%%%%%%%%%%%%%%%%%%%%%%%%%%%%%%%%%%%%%%%%%%%%%%%%%%%%%%%%%
% Problem ends here
%%%%%%%%%%%%%%%%%%%%%%%%%%%%%%%%%%%%%%%%%%%%%%%%%%%%%%%%%%%%%%%%%%%%%

\endinput
