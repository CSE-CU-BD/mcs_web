\documentclass[problem]{mcs}

\begin{pcomments}
  \pcomment{CP_binary_relations_on_01_mq}
  \pcomment{should be renamed to MQ_partial_order_on_123}
  \pcomment{from: S09.cp4m}
\end{pcomments}

\pkeywords{
  partial_orders
  chain
  antichain
  minimal
  maximal
}

%%%%%%%%%%%%%%%%%%%%%%%%%%%%%%%%%%%%%%%%%%%%%%%%%%%%%%%%%%%%%%%%%%%%%
% Problem starts here
%%%%%%%%%%%%%%%%%%%%%%%%%%%%%%%%%%%%%%%%%%%%%%%%%%%%%%%%%%%%%%%%%%%%%

\begin{problem} 

\mbox{}

\bparts

\ppart For each row in the following table, indicate whether the binary
relation, $R$, on the set, $A$, is a weak partial order or a linear order
by filling in the appropriate entries with either Y = YES or N = NO.  In
addition, list the minimal and maximal elements for each relation.

\renewcommand\arraystretch{2}
\begin{center}
  \begin{tabular*}{\textwidth}{@{\extracolsep{\fill}} | c | c | c | c | c | c |}
    \hline
    $\mathbf{A}$ & $\mathbf{a\mrel{R}b}$ & \textbf{weak p. o.} &
    \textbf{linear order} & \textbf{minimal(s)} &
    \textbf{maximal(s)} \\ \hline
    $\reals - \reals^+$ & $a \divides b$ & & & & \\ \hline
    $\power(\set{1, 2, 3})$ & $a \subseteq b$ & & & &  \\ \hline
    $\naturals \union \set{i}$ & $a> b$ & & & &  \\
    \hline
  \end{tabular*}
\end{center}

\begin{solution}
TBA
\end{solution}

\examspace[0.5in]

\ppart What is the longest \emph{chain} on the subset relation,
$\subseteq$, on $\power(\set{1, 2, 3})$?  (If there is more than one,
provide \emph{one} of them.)

\examspace[1.5in]

\begin{solution}
  One maximum length chain is $\emptyset, \set{1}, \set{1,2},
  \set{1,2,3}$.  There are five others, which can be gotten by permuting
  the elements $1, 2, 3$.
\end{solution}

\ppart What is the longest \emph{antichain} on the subset relation,
$\subseteq$, on $\power(\set{1, 2, 3})$?  (If there is more than one, provide
\emph{one} of them.)

\begin{solution}
  One antichain consists of the three one-element subsets, $\set{1},
  \set{2}, \set{3}$.  Another antichain consists of the three two-element
  subsets, $\set{1,2}, \set{1,3}, \set{2,3}$.  These are the only two
  maximum-size antichains.
\end{solution}

\eparts

\end{problem}

%%%%%%%%%%%%%%%%%%%%%%%%%%%%%%%%%%%%%%%%%%%%%%%%%%%%%%%%%%%%%%%%%%%%%
% Problem ends here
%%%%%%%%%%%%%%%%%%%%%%%%%%%%%%%%%%%%%%%%%%%%%%%%%%%%%%%%%%%%%%%%%%%%%

\endinput
