\documentclass[problem]{mcs}

\begin{pcomments}
  \pcomment{TP_pigeonhole_mod}
\end{pcomments}

\pkeywords{
  counting
  pigeonhole
  modulo
  congruent
}

%%%%%%%%%%%%%%%%%%%%%%%%%%%%%%%%%%%%%%%%%%%%%%%%%%%%%%%%%%%%%%%%%%%%%
% Problem starts here
%%%%%%%%%%%%%%%%%%%%%%%%%%%%%%%%%%%%%%%%%%%%%%%%%%%%%%%%%%%%%%%%%%%%%


\begin{problem}
Use the Pigeonhole Principle to determine the smallest nonnegative
integer $n$ such that every set of $n$ integers is guaranteed to
contain three integers that are congruent mod 211.  Clearly indicate
what are the pigeons, holes, and rules for assigning a pigeon to a
hole, and give the value of $n$.

\examspace[2in]

\begin{solution}
\[
n = 423.
\]

The $n$ integers are the pigeons, $\Zmod{211}$ is the set of pigeon holes, and a
pigeon is assigned to its remainder on division by 211.  By the Generalized Pigeonhole
Principle, there must be at least
\[
\ceil{\frac{n}{211}}
\]
pigeons in some hole.  So we want the smallest integer $n$ such that
\[
\frac{n}{211} > 2.
\]
\end{solution}

\end{problem}

%%%%%%%%%%%%%%%%%%%%%%%%%%%%%%%%%%%%%%%%%%%%%%%%%%%%%%%%%%%%%%%%%%%%%
% Problem ends here
%%%%%%%%%%%%%%%%%%%%%%%%%%%%%%%%%%%%%%%%%%%%%%%%%%%%%%%%%%%%%%%%%%%%%

\endinput
