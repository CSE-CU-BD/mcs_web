\documentclass[problem]{mcs}

\begin{pcomments}
  \pcomment{from: F09.mq3}
  \pcomment{similar to CP_10_and_15_cent_stamps_by_WOP but slightly different and uses induction}
\end{pcomments}

\pkeywords{
  induction
  strong_induction
}

%%%%%%%%%%%%%%%%%%%%%%%%%%%%%%%%%%%%%%%%%%%%%%%%%%%%%%%%%%%%%%%%%%%%%
% Problem starts here
%%%%%%%%%%%%%%%%%%%%%%%%%%%%%%%%%%%%%%%%%%%%%%%%%%%%%%%%%%%%%%%%%%%%%

\begin{problem}

  Find all possible (nonzero) amounts of postage that can be paid 
  exactly using 7 and 3 cent stamps. Use induction to prove that
  your answer is correct.
  
  \hint Let $S(n)$ mean that exactly $n$ cents of postage can be
  paid using only 7 and 3 cent stamps. Prove that the following 
  proposition is true as part of your solution.
  \[ 
    \forall n.\ ( n \geq 12 ) \QIMPLIES\ S(n).
  \]
    
% \ppart
% 
%   Let $S(n)$ mean that exactly $n$ cents of postage can be paid using 
%   only 5 and 3 cent stamps. 
%   
%   Find the smallest $k$ such that the following 
%   proposition is true.
%   \[ 
%   \forall n.\ ( n \geq k ) \QIMPLIES\ S(n).
%   \]
%   
% \vspace{1in}
  % Use induction to prove that your answer to the previous problem 
  % part is correct. 
  
\begin{solution}
  
  \begin{proof}
    The following proof is by strong induction on $n$.
    
    We can begin by observing that the following postage amounts can 
    be made by 7 and 3 cent stamps:

    \begin{align*}
    3 & = 3 \\
    6 & = 3 + 3 \\
    7 & = 7 \\
    9 & = 3 + 3 + 3 \\
    10 & = 3 + 7 \\
    12 &= 3+3+3 \\
    13 &= 3+3+7 \\
    14 &= 7+7 \\
    \end{align*}

    The 3 consecutive postage values $12, 13, 14$ will be our base cases 
    for the induction proof.

    \textbf{Base cases:} $S(12)$, $S(13)$ and $S(14)$ are shown to hold
    by explicit calculations.

    \textbf{Inductive step:} For all $n \geq 14$, we assume that $P(12)$,
    $\dots$, $P(n)$ are true in order to prove that $P(n+1)$ is true.

    By the assumption that $P(n-2)$ is true, we know that the postage value
    $n - 2$ can be paid with 3 and 7 cent stamps. By adding one 3 cent
    stamp to that postage, we will be able to pay for a postage of 
    $n - 2 + 3 = n + 1$ cents, showing that $P(n+1)$ is true. It follows
    by strong induction that $P(n)$ holds for all $n \geq 14$.
    
    We have therefore shown that all postage values $ \geq 12 $
    can be paid by 3 and 7 cent stamps.
  \end {proof}
  
  Therefore, we have shown that the postage amounts 3, 7, and any 
  $k \geq 12$ can be paid by 7 and 3 cent stamps.
  
  Note that we have actually seen a proof for this using WOP in the 
  \href{http://courses.csail.mit.edu/6.042/fall09/slides2m.pdf}{Well Ordering Principle}
  lecture.

\end{solution}

\end{problem}

%%%%%%%%%%%%%%%%%%%%%%%%%%%%%%%%%%%%%%%%%%%%%%%%%%%%%%%%%%%%%%%%%%%%%
% Problem ends here
%%%%%%%%%%%%%%%%%%%%%%%%%%%%%%%%%%%%%%%%%%%%%%%%%%%%%%%%%%%%%%%%%%%%%
