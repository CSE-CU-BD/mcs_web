\documentclass[problem]{mcs}

\begin{pcomments}
  \pcomment{MQ_graph_state_machine_f15}
  \pcomment{variant of MQ_graph_state_machine}
  \pcomment{Proposed by ZDz for Miderm 3 in F15}
\end{pcomments}

\pkeywords{
graphs,
state_machines,
connectivity,
well_order,
termination,
partial_correctness
}

%%%%%%%%%%%%%%%%%%%%%%%%%%%%%%%%%%%%%%%%%%%%%%%%%%%%%%%%%%%%%%%%%%%%%
% Problem starts here
%%%%%%%%%%%%%%%%%%%%%%%%%%%%%%%%%%%%%%%%%%%%%%%%%%%%%%%%%%%%%%%%%%%%%

\begin{problem}
 
We apply the following operation to a \textbf{simple graph} $G$:
pick two vertices $u \neq v$ such that either
\begin{enumerate}
\item there is an edge of $G$ between $u$ and $v$, and there is also a path
from $u$ to $v$ which does \emph{not} include this edge; in this case,
delete the edge $\edge{u}{v}$.

\item there is no path from $u$ to $v$; in this case, add the edge
  $\edge{u}{v}$.
\end{enumerate}

Keep repeating these operations until it is no longer possible to find two
vertices $u \neq v$ to which an operation applies.

Let $n \geq 2$ be the number of nodes of $G$.

%Assume the vertices of $G$ are the integers $1,2,\dots,n$ for some $n \geq
%2$.  This procedure can be modelled as a state machine whose states are
%all possible simple graphs with vertices $1,2,\dots,n$.  $G$ is the start
%state, and the final states are the graphs on which no operation is
%possible.

\bparts

\ppart
What is the maximum possible number of times operation 2.\ will be applied? Explain.

\examspace[1.5in]

\begin{solution}
Operation 2.\ is applied when there is no path between two vertices, which is the
case if they belong to different connected components. By adding an edge between
them, these two components are connected and become a single component.
Therefore, every time operation 2.\ is applied, the number of connected components
is reduced by 1. Since there can be at most $n$ different connected components
to begin with (if there are no edges in $G$), operation 2.\ will be applied
at most $n-1$ times.
\end{solution}

\ppart
What is the maximum possible number of times operation 1.\ will be applied? Explain.
(Finding an upper bound is fine.)
%You will receive partial credit if you find an upper bound that you can justify, instead
%of the exact number.

\examspace[1.5in]

\begin{solution}
When operation 1.\ is applied, the number of edges is decreased by 1. Since
there can be at most ${n \choose 2}$ edges to begin with and at most
$n-1$ edges can be added by operation 2., operation 1.\ will be invoked
no more than ${n \choose 2} + n-1$ times.
\end{solution}

\begin{staffnotes}
Proving an exact max is not easy. It would actually be easier to first conclude that
the procedure terminate with a tree (using upper bound as in previous solution,
and then utilize that to have a shorter proof for this one.
\begin{solution}
First, note that edges added by operation 2.\ are never deleted because they are only added
between two nodes that are not connected. If nodes $u$ and $v$ are not connected and
an edge is added between them, this edge would be removed only if there was
another path between them that does not contain edge $\edge{u}{v}$. However,
such path could not exist before adding edge $\edge{u}{v}$, and could not be
created after adding this edge. To show the latter, let us assume to the contrary
that a path between $u$ and $v$ that does not contain edge $\edge{u}{v}$ is
created after adding this edge. Let $u - w_1 - w_2 - \lodts - w_k - v$ be that path,
and let $\edge{w_i}{w_{i+1}}$ be the last edge on it that was added.
However, before adding this edge, $w_i$ was connected to $u$ and $w_{i+1}$ was
connected to $v$ and the edge $\edge{u}{v}$ existed, implying that
there was a path from $w_i$ to $w_{i+1}$. Therefore, edge $\edge{w_i}{w_{i+1}}$
could not have been added by operation 2., and hence a contradiction.
We conclude that edges that are added are never deleted.

When operation 1.\ is applied, the number of edges is decreased by 1, and it
cannot be applied if there is no cycle in the graph, which is when the graph is a forest.
The maximum number of deletions is achieved when $G$ has the most edges to start with,
i.e., if it is a complete graph. In that case, $G$ contains ${n \choose 2}$ edges and
edges will be removed until it remains a tree, i.e., when it contains $n-1$ edges.
Therefore, operation 1.\ can be applied at most
${n \choose 2} - (n-1) = n(n-1)/2 - (n-1) = (n-1)(n-2)/2 = {n-1\choose 2}$ times.

To show that this is indeed maximum, we argue by cases. If $G$ is connected
but not complete, the same number of edges will remain in the end, but the starting
number will be smaller, and thus fewer edges would be deleted.
If $G$ is not connected, then edges will be added between connected components
until $G$ becomes connected. These edges will not be removed, and thus we know
that in the end there will be $n-1$ edges that were never removed, which reduces
to the previous case.
\end{solution}
\end{staffnotes}

\ppart
Conclude that starting from any graph, $G$, the procedure above terminates.
(Note that you can argue about this even if you did not solve previous parts,
by utilizing what their statements imply.)

\examspace[1in]

\begin{solution}
In the previous two parts, we showed that both operations can be applied
only a finite number of times, and thus the procedure must terminate in
finite number of steps.
\end{solution}

\ppart Prove that the procedure always terminates with a graph that is a tree.

\examspace[2in]

\begin{solution}
We use the characterization of a tree as a cycle-free,
connected, simple graph.

A final graph must be connected, because otherwise there would be two
vertices with no path between them, and then a transition adding the edge
between them would be possible.

A final graph can't have a cycle, because deleting any edge on the
cycle would be a possible operation.
\end{solution}

%\ppart
%
%Let $G$ be the graph with vertices $\set{1,2,3,4}$ and edges
%\[
%\set{\set{1,2},\set{3,4}}
%\]
%How many possible final states are reachable from start state
%$G$? \hfill \examrule{1in}
%
%\begin{solution}It's not possible to delete any edge.  The procedure can only
%add an edge connecting exactly one of vertices 1 or 2 to exactly one of
%vertices 3 or 4, and then terminate.  So there are four possible final
%states.
%\end{solution}
%
%\ppart
%\label{derived} On the line next to each of the derived
%state variables below, indicate the \emph{strongest} property from the
%list below that the variable is guaranteed to satisfy, no matter what the
%starting graph $G$ is.  The properties are:
%\begin{center}
%\begin{tabular}{ccc}
%\emph{constant} & \emph{increasing} & \emph{decreasing}\\
%\emph{nonincreasing} & \emph{nondecreasing} & \emph{none of these}
%\end{tabular}
%\end{center}
%
%For any state, let $e$ be the number of edges in it, and let $c$ be the
%number of \textbf{connected components} it has.  Since $e$ may increase or
%decrease in a transition, it does not have any of the first four
%properties.  The derived variables are:
%\begin{enumerate}
%
%\item[0)] $e$ \hfill \underline{\emph{none of these}}
%
%
%\item[i)] $c$ \hfill \examrule{1.0in}
%
%\begin{solution}
%nonincreasing
%\end{solution}
%
%\item[ii)]  $c+e$ \hfill \examrule{1.0in}
%
%\begin{solution}
%nonincreasing
%\end{solution}
%
%\item[iii)] $2c+e$ \hfill \examrule{1.0in}
%
%\begin{solution}
%decreasing
%\end{solution}
%
%\begin{editingnotes}
%USABLE ONLY IF LEXICOGRAPHIC ORDER IS BACK IN TEXT
%\item[iv)] $(c,e)$, under the \textbf{lexicographic order} on $\naturals \cross \naturals$.
%\hfill \examrule{1.0in}
%\end{editingnotes}
%
%\item[iv)] $c+\frac{e}{e+1}$
%\hfill \examrule{1.0in}
%
%\begin{solution}
%decreasing
%\end{solution}
%
%\end{enumerate}
%
%\ppart
%Explain why, starting from any state, $G$, the procedure
%terminates.  If your explanation depends on answers you gave to part~(b),
%you must justify those answers.
%
%\begin{solution}
%The derived variable iii) is a natural-number valued variable and is
%decreasing, so by Theorem~\bref{th:decr}, the procedure must terminate.
%
%\emph{Alternative answer}: derived variable iv) is decreasing, and its
%values are all in the set $\naturals + \twdone$ shown in
%Section~\bref{to1_well-order} to be a well ordered set of real
%numbers.
%
%\begin{editingnotes}
%under the lexicographic order on $\naturals \cross \naturals$
%which we know from Problem Set 6 is a well ordered set.
%\end{editingnotes}
%
%To show that the variable iii) decreases, note that the rule for
%deleting an edge ensures that the connectedness relation does not change,
%so neither does the number of connected components; therefore the variable
%decreases by one when an edge is deleted.  The rule for adding an edge
%ensures that the number of connected components decreases by one, so the
%variable changes by $-2$ for the reduced number of components and $+1$ for
%the added edge, for a net decrease of 1.
%
%Variable iv) decreases when an edge is deleted and the number of
%components stays the same.   The rule for adding an edge
%ensures that the number of connected components decreases by one, so the
%variable changes by $-1$ for the reduced number of components and
%increases by an amount $(e+1)/(e+2) - e/(e+1) < 1$ for
%the added edge, for a net decrease.
%\end{solution}
%
%\ppart Prove that any final state must be an \textbf{unordered tree}
%on the set of vertices, that is, a spanning tree.
%
%\begin{solution}
%We use the characterization of a tree as a cycle-free,
%connected, simple graph.
%
%A final state must be connected, because otherwise there would be two
%vertices with no path between them, and then a transition adding the edge
%between them would be possible, contradicting finality of the state.
%
%A final state can't have a cycle, because deleting any edge on the
%cycle would be a possible transition.
%\end{solution}

\eparts
\end{problem}


%%%%%%%%%%%%%%%%%%%%%%%%%%%%%%%%%%%%%%%%%%%%%%%%%%%%%%%%%%%%%%%%%%%%%
% Problem ends here
%%%%%%%%%%%%%%%%%%%%%%%%%%%%%%%%%%%%%%%%%%%%%%%%%%%%%%%%%%%%%%%%%%%%%

\endinput
