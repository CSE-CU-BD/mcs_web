\documentclass[problem]{mcs}

\begin{pcomments}
\pcomment{PS_more_numbered_trees}
\pcomment{from S08, ps9; f07.ps9;prob4 edited by ARM 11/14/09}
\pcomment{solution can and should be explained more simply --ARM 11/14/09}}
\end{pcomments}

\pkeywords{
 combinatorial_proof
 binomial_coefficient
 numbered_trees
 bijection
 degree_of_a_vertex
}

%%%%%%%%%%%%%%%%%%%%%%%%%%%%%%%%%%%%%%%%%%%%%%%%%%%%%%%%%%%%%%%%%%%%%
% Problem starts here
%%%%%%%%%%%%%%%%%%%%%%%%%%%%%%%%%%%%%%%%%%%%%%%%%%%%%%%%%%%%%%%%%%%%%

\begin{problem}
  Continuing Problem~\bref{CP_numbered_trees} on the bijection between
  $n$-vertex \idx{numbered trees} and the length $n-2$ sequences of integers
  between $1$ and $n$.

\bparts

\ppart How is the degree of a vertex in a numbered tree related to the
number of times it appears in the code of the tree?

\begin{solution}
  During the course of creating the code, every vertex is reduced from
  it's orginal degree, to degree at most $1$.  (The vertex may be
  subsequently be deleted, or may be one of the two vertices that are not
  deleted.)  Every time the degree is reduced, it is because the vertex is
  the father of some current leaf, which is then deleted.  But this means
  the vertex appears in the code.  When the vertex has been reduced to
  degree $1$, it can no longer appear in the code.  Thus every vertex, $v$,
  appears $\deg(v)-1$ times in the code.
\end{solution}

\ppart The \term{degree sequence} of a graph, is the weakly decreasing
sequence of degrees of it vertices.  How many trees 9-vertex numbered
trees have degree sequence $4,3,2,2,1,1,1,1,1$?

\begin{solution}
\[
\binom{9}{5,\,2,\,1,\, 1}\cdot \binom{7}{3,\,2,\,1,\,1}.
\]

By the product rule, the number of trees is the number of different ways
to assign one these degrees to vertices, times the number of different
ways to attach the vertices, given their assigned degrees.




\iffalse
To help define this value, we first note: If there are $n_1$ letters of
type $1$,  $n_k$ letters of type $k$, the ``bookeeper rule'' gives us
the number of ways of rearranging these letters into a word.  The formal
name for this number is the {\em multinomial coefficient}: ${n_1 + n_2 +
  \cdots + n_k \choose n_1,n_2, \ldots, n_k} := (n_1 + n_2 + \cdots +
n_k)!/(n_1!n_2!\cdots n_k!)$.
\fi

Now, the number of different ways to assign which vertex has which degree
corresponds to the number of permutations on the sequencea
$4,3,2,2,1,1,1,1,1$, which is $\binom{9}{5,\,2,\,1,\,1}$.

The number of different ways to attach the vertices, given their assigned
degrees, can be defined by the previous section as a bijection to the
number of different code sequences, given the degree sequence.  For degree
sequence $4,3,2,2,1,1,1,1,1$, the number of times the corresponding
vertices appear in the code is $3,2,1,1,0,0,0,0,0$, and a code will
therefore be a permutation of the sequence $a,a,a,b,b,c,d$. The number of
permutations is $\binom{7}{3,\,2,\,1,\,1}$.
\end{solution}

\eparts
\end{problem} 

%%%%%%%%%%%%%%%%%%%%%%%%%%%%%%%%%%%%%%%%%%%%%%%%%%%%%%%%%%%%%%%%%%%%%
% Problem ends here
%%%%%%%%%%%%%%%%%%%%%%%%%%%%%%%%%%%%%%%%%%%%%%%%%%%%%%%%%%%%%%%%%%%%%

\endinput
