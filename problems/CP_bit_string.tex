\documentclass[problem]{mcs}

\begin{pcomments}
  \pcomment{CP_bit_string}
  \pcomment{from: S09.cp11r}
\end{pcomments}

\pkeywords{
  binomial
  coefficients
  algebraic_proof
  combinatorial
}

%%%%%%%%%%%%%%%%%%%%%%%%%%%%%%%%%%%%%%%%%%%%%%%%%%%%%%%%%%%%%%%%%%%%%
% Problem starts here
%%%%%%%%%%%%%%%%%%%%%%%%%%%%%%%%%%%%%%%%%%%%%%%%%%%%%%%%%%%%%%%%%%%%%

\begin{problem}
What do the following expressions equal?
Give both algebraic and combinatorial proofs for your answers.

\bparts
\ppart
\[
\sum_{i=0}^n \binom{n}{i}
\]

\begin{solution}$2^n$.

\emph{Algebraic proof:} This is the Binomial theorem with $x=y=1$.

\emph{Combinatorial proof:} There are $2^n$ length $n$ bit strings.  The
number of such sequences is also equal to the number of length $n$ bit
strings with 0 ones, plus the number with 1 one, plus the number with 2
ones, etc., which is precisely $\sum_{i=0}^n \binom{n}{i}$.
\end{solution}

\ppart
\[
\sum_{i=0}^{n} \binom{n}{i}(-1)^i
\]
\hint Consider the bit strings with an even number of ones and an odd
number of ones.

\begin{solution}$0$.

\textbf{Algebraic proof}: This is just the Binomial theorem with $x=1$ and $y=-1$.

\textbf{Combinatorial proof}: Consider the $n-$bit sequences, and divide
them into two sets, those with an even number of ones (even $i$ terms) and
those with an odd number of ones (odd $i$ terms).  The sum is then equal
to the number of strings with an even number of ones, minus the number of
strings with an odd number of ones.

Next, we note that the number of strings with an even number of ones is
equal to the number with an odd number of ones.  This can be seen by
establishing a bijection between the two sets: any string in one set can
be made into a string in the other set by complementing the first bit in
the string. Since the number of strings with an even number of ones is
equal to the number with an odd number, the entire expression must be
equal to 0.
\end{solution}

\eparts
\end{problem}



%%%%%%%%%%%%%%%%%%%%%%%%%%%%%%%%%%%%%%%%%%%%%%%%%%%%%%%%%%%%%%%%%%%%%
% Problem ends here
%%%%%%%%%%%%%%%%%%%%%%%%%%%%%%%%%%%%%%%%%%%%%%%%%%%%%%%%%%%%%%%%%%%%%

\endinput
