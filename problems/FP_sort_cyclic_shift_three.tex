\documentclass[problem]{mcs}

\begin{pcomments}
  \pcomment{TP_sort_cyclic_shift_three}
  \pcomment{related to CP_fifteen_puzzle}
  \pcomment{ARM and CH, S14}
\end{pcomments}

\pkeywords{
  state_machines
  sorting
  fifteen_puzzle
  parity
}

%%%%%%%%%%%%%%%%%%%%%%%%%%%%%%%%%%%%%%%%%%%%%%%%%%%%%%%%%%%%%%%%%%%%%
% Problem starts here
%%%%%%%%%%%%%%%%%%%%%%%%%%%%%%%%%%%%%%%%%%%%%%%%%%%%%%%%%%%%%%%%%%%%%

\begin{problem}
The following problem is a twist on the Fifteen-Puzzle analyzed in
Problem~\bref{CP_fifteen_puzzle}.

Let $A$ be a sequence consisting of the numbers $1,\dots,n$ in some
order.  A pair of integers in $A$ is called an \emph{out-of-order
  pair} when the first element of the pair both comes \emph{earlier}
in the sequence, and \emph{is larger}, than the second element of the
pair.  For example, the sequence $(1,2,4,5,3)$ has two out-of-order
pairs: $(4,3)$ and $(5,3)$.  We let $t(A)$ equal the number of
out-of-order pairs in $A$.  For example, $t((1,2,4,5,3)) = 2$.

The elements in $A$ can be rearranged using the \emph{Rotate-Triple}
operation, in which three consecutive elements of $A$ are rotated to
move the smallest of them to be first.

For example, in the sequence $(2,4,1,5,3)$, the \emph{Rotate-Triple}
operation could rotate the consecutive numbers $4,1,5$, into $1, 5, 4$
so that
\[
(2,4,1,5,3) \movesto (2, 1, 5, 4, 3).
\]

The \emph{Rotate-Triple} could also rotate the consecutive numbers
$2,4,1$ into $1,2,4$ so that
\[
(2,4,1,5,3) \movesto (1, 2, 4, 5, 3).
\]

We can think of a sequence $A$ as a state of a state machine whose
transitions correspond to possible applications of the
\emph{Rotate-Triple} operation.

\begin{problemparts}
\iffalse

\problempart Write out the set of transitions that describe the
actions of $\emph{Rotate-Triple}$ at any time step.

\hint You will need to consider six different cases.

\begin{solution}

Consider any consecutive triple in $A$ containing the elements $\{ a,b,c
\}$ in some order. Without loss of generality, assume that $a < b < c$. The
six different cases correspond to the different permutations of
$\{a,b,c\}$. Explicitly, the set of transitions can be described as follows:
\begin{align*}
\emph{Rotate-Triple}(a,b,c) &\movesto (a,b,c) \\
\emph{Rotate-Triple}(a,c,b) &\movesto (a,c,b) \\
\emph{Rotate-Triple}(b,c,a) &\movesto (a,b,c) \\
\emph{Rotate-Triple}(b,a,c) &\movesto (a,c,b) \\
\emph{Rotate-Triple}(c,b,a) &\movesto (a,c,b) \\
\emph{Rotate-Triple}(c,a,b) &\movesto (a,b,c).
\end{align*}

\end{solution}

\examspace[1in]
\fi

\problempart Argue that the derived variable $t$ is \emph{weakly
  decreasing}.

\begin{solution}
Suppose the \emph{Rotate-Triple} operation is applied to three
consecutive elements $a,b,c$ in $A$.  This has no effect on the
out-of-order pairs involving at most one of $a,b$ and $c$.

To analyze pairs where both elements are one of $a,b$ and $c$, there
are two cases.

If $b$ is the smallest element, then $a,b,c$ get rearranged into
$b,c,a$.  This has the effect of reversing the two pairs $(a,b),(a,c)$
into $(b,a),(c,a)$.  If $a < c$, this causes a net change of zero in
$t(A)$, while if $c < a$, this causes a net decrease of two in $t(A)$.

If $c$ is the smallest, then $a,b,c$ get rearranged into $c,a,b$.
This has the effect of reversing the two pairs $(a,c),(b,c)$ into
$(c,a),(c,b)$, which similarly leaves $t(A)$ unchanged or decreased by
two.

So in each case, $t$ is either constant or decreases, showing that $t$
is weakly decreasing.
\end{solution}

\examspace[2.0in]

\problempart Prove that having an even number of out-of-order pairs is
a preserved invariant of this machine.

\begin{solution}
This part follows directly from the argument in the previous part
showing that $t$ changes by 0 or $-2$.  So if the number of
out-of-order pairs is even, then it stays even.
\end{solution}

\examspace[1.5in]

\problempart Starting with
\[
S \eqdef (2014, 2013, 2012,\ldots, 2,1),
\]
explain why it is impossible to reach
\[
T \eqdef (1, 2, \dots, 2012, 2013, 2014).
\]

\begin{solution}
Since $T$ has no out-of-order pairs, $t(T) = 0$.  On the other hand,
all the pairs in $S$ are out-of-order, so $t(S) = (2014 \cdot 2013)/2$
which is odd.  Since parity is preserved, $T$ cannot be reachable from
$S$.
\end{solution}

\examspace[1.0in]

\end{problemparts}

\end{problem}

%%%%%%%%%%%%%%%%%%%%%%%%%%%%%%%%%%%%%%%%%%%%%%%%%%%%%%%%%%%%%%%%%%%%%
% Problem ends here
%%%%%%%%%%%%%%%%%%%%%%%%%%%%%%%%%%%%%%%%%%%%%%%%%%%%%%%%%%%%%%%%%%%%%

\endinput
