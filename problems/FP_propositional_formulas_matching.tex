\documentclass[problem]{mcs}

\begin{pcomments}
  \pcomment{FP_propositional_formulas_matching}
  \pcomment{forked from MQ_implies_relation_on_propositional_formulas_modified}
  \pcomment{Original: ARM 9/20/11}
  \pcomment{CH, Spring '14}
\end{pcomments}

\pkeywords{
  propositional formula
  implies
  valid
  relation
  codomain
  image
  inverse
  graph_of_relation
  subset
  matching
  bottleneck
}

%%%%%%%%%%%%%%%%%%%%%%%%%%%%%%%%%%%%%%%%%%%%%%%%%%%%%%%%%%%%%%%%%%%%%
% Problem starts here
%%%%%%%%%%%%%%%%%%%%%%%%%%%%%%%%%%%%%%%%%%%%%%%%%%%%%%%%%%%%%%%%%%%%%

\begin{problem}\label{ppart:fig} \mbox{}
 Let $R$ be the ``implies'' binary relation defined by the rule
\begin{equation}
F \mrel{R} G \qiff [\text{the formula } (F \QIMPLIES G) \text{ is valid}].
\label{eq:rule_implies}
\end{equation}
For example, $(P \QAND Q) \mrel{R} P$, because the formula $(P \QAND
Q)$ does imply $P$.  Also, it is not true that $(P \QOR Q) \mrel{R} P$
since $(P \QOR Q)$ does not imply $P$.

\bparts

\ppart Fill in the arrows so that the following figure describes the graph of
  the relation $R$ from $A$ to $B$.  

\[\begin{array}{lcr}
A & \hspace{1in} \text{arrows} \hspace{1in} & B\\
\hline
% &\\
% M\\
% &\\
                                  && Q\\
&\\
P \QXOR Q\\
&\\
                                  && \bar{P} \QOR \bar{Q}\\
&\\
P \QAND Q\\
&\\
                                  && \bar{P} \QOR \bar{Q} \QOR
                                  (\bar{P} \QAND \bar{Q})\\
&\\
\QNOT(P \QAND Q)\\
&\\
                                  && P\\
\end{array}\]

\begin{solution}
Arrows for $R$:
\begin{align*}
P \QXOR Q & \qimplies & \bar{P} \QOR \bar{Q}\\
P \QXOR Q & \qimplies &  \bar{P} \QOR \bar{Q} \QOR (\bar{P} \QAND
                                  \bar{Q})\\
P \QAND Q & \qimplies & Q\\
P \QAND Q & \qimplies & P\\
\QNOT(P \QAND Q) & \qiff & \bar{P} \QOR \bar{Q} \\
\QNOT(P \QAND Q) & \qiff &  \bar{P} \QOR \bar{Q} \QOR (\bar{P} \QAND
                                  \bar{Q})
\end{align*}
\end{solution}

\ppart Let $R'$ be the binary relation on the set $A \union B$ defined
by the rule~\eqref{eq:rule_implies}.  Explain why $R'$ is not a
partial order.

\examspace[1in]

\begin{solution}
$R'$ is not a partial order since the condition of antisymmetry is violated; for
example, $\QNOT(P \QAND Q) \qiff \bar{P} \QOR \bar{Q}$ but the two
formulas are not equal. 
\end{solution}

\ppart We can think of the figure in part~\eqref{ppart:fig} as a
bipartite graph $G$ with the formulas in $A$ constituting the left
vertices, the formulas in $B$ constituting the right vertices, and
the arrows denote the edges.  Argue that there exists a matching from
$A$ to $B$. 

\begin{solution}
The degree of each vertex in the left is 2, while the degree on each
vertex on the right is at most 2.  Therefore, the graph $G$ is degree
constrained and there exists a matching.
\end{solution}

\eparts

\end{problem}

%%%%%%%%%%%%%%%%%%%%%%%%%%%%%%%%%%%%%%%%%%%%%%%%%%%%%%%%%%%%%%%%%%%%%
% Problem ends here
%%%%%%%%%%%%%%%%%%%%%%%%%%%%%%%%%%%%%%%%%%%%%%%%%%%%%%%%%%%%%%%%%%%%%

\endinput
