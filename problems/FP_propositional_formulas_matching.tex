\documentclass[problem]{mcs}

\begin{pcomments}
  \pcomment{FP_propositional_formulas_matching}
  \pcomment{forked from MQ_implies_relation_on_propositional_formulas_modified}
  \pcomment{Original: ARM 9/20/11}
  \pcomment{CH, Spring '14}
\end{pcomments}

\pkeywords{
  propositional formula
  implies
  valid
  relation
  codomain
  image
  inverse
  graph_of_relation
  subset
  matching
  bottleneck
}

%%%%%%%%%%%%%%%%%%%%%%%%%%%%%%%%%%%%%%%%%%%%%%%%%%%%%%%%%%%%%%%%%%%%%
% Problem starts here
%%%%%%%%%%%%%%%%%%%%%%%%%%%%%%%%%%%%%%%%%%%%%%%%%%%%%%%%%%%%%%%%%%%%%

\begin{problem}
Let $\widehat{R}$ be the ``implies'' binary relation on propositional
formulas defined by the rule that
\begin{equation}\label{eq:rule_implies}
F \mrel{\widehat{R}} G\ \qiff\  [(F \QIMPLIES G) \text{ is a valid formula}].
\end{equation}
For example, $(P \QAND Q) \mrel{\widehat{R}} P$, because the formula
$(P \QAND Q) \QIMPLIES P$ is valid.  Also, it is not true that $(P
\QOR Q) \mrel{\widehat{R}} P$ since $(P \QOR Q) \QIMPLIES P$ is not
valid.

\bparts

\ppart Let $A$ and $B$ be the sets of formulas listed below.  Explain
why $\widehat{R}$ is not a weak partial order on the set $A \union B$.

\examspace[1.0in]

\begin{solution}
$\widehat{R}$ is not a weak partial order since the condition of
  antisymmetry is violated; for example, $\QNOT(P \QAND Q)$ implies
  and is implied by $\bar{P} \QOR \bar{Q}$, but the two formulas are
  not equal.
\end{solution}

\ppart\label{ppart:fig}
Fill in the $\widehat{R}$ arrows from $A$ to $B$.

\[\begin{array}{lcr}
A & \hspace{0.6in} \text{arrows} \hspace{0.6in} & B\\
\hline
% &\\
% M\\
% &\\
                                  && Q\\
&\\
P \QXOR Q\\
&\\
                                  && \bar{P} \QOR \bar{Q}\\
&\\
P \QAND Q\\
&\\
                                  && \bar{P} \QOR \bar{Q} \QOR
                                  (\bar{P} \QAND \bar{Q})\\
&\\
\QNOT(P \QAND Q)\\
&\\
                                  && P\\
\end{array}\]

\begin{solution}
Arrows for $\widehat{R}$:
\begin{align*}
P \QXOR Q & \qimplies & \bar{P} \QOR \bar{Q}\\
P \QXOR Q & \qimplies &  \bar{P} \QOR \bar{Q} \QOR (\bar{P} \QAND
                                  \bar{Q})\\
P \QAND Q & \qimplies & Q\\
P \QAND Q & \qimplies & P\\
\QNOT(P \QAND Q) & \qiff & \bar{P} \QOR \bar{Q} \\
\QNOT(P \QAND Q) & \qiff &  \bar{P} \QOR \bar{Q} \QOR (\bar{P} \QAND
                                  \bar{Q})
\end{align*}
\end{solution}


% \ppart We can think of the figure in part~\eqref{ppart:fig} as a
% bipartite graph $G$ with the formulas in $A$ constituting the left
% vertices, the formulas in $B$ constituting the right vertices, and
% the arrows denote the edges.  Argue that there exists a matching from
% $A$ to $B$. 
% \begin{solution}
% The degree of each vertex in the left is 2, while the degree on each
% vertex on the right is at most 2.  Therefore, the graph $G$ is degree
% constrained and there exists a matching.
% \end{solution}

\ppart The diagram in part~\eqref{ppart:fig} defines a bipartite graph
$G$ with $\leftbi{G}=A$, $\rightbi{G} = B$ and an edge between $F$ and
$G$ iff $F \mrel{\widehat{R}} G$.  Exhibit a subset $S$ of $A$ such
that both $S$ and $A-S$ are nonempty, and the set $\neighbors{S}$ of neighbors
of $S$ is the same size as $S$, that is, $\abs{\neighbors{S}} = \abs{S}$.

\examspace[1.0in]
\begin{solution}
$S \eqdef \set{P \QXOR Q, \QNOT(P \QAND Q)}$.

Now $\neighbors{S} = \set{\bar{P} \QOR \bar{Q}, \bar{P} \QOR \bar{Q} \QOR
  (\bar{P} \QAND \bar{Q})}$, so $\abs{S} = \abs{\neighbors{S}} = 2$.
\end{solution}

\ppart Let $G$ be an arbitrary, finite, bipartite graph.  For any
subset $S \subseteq \leftbi{G}$, let $\bar{S} \eqdef \leftbi{G} - S$,
and likewise for any $M \subseteq \rightbi{G}$, let $\bar{M} \eqdef
\rightbi{G} - M$.  Suppose $S$ is a subset of $\leftbi{G}$ such that
$\abs{\neighbors{S}} = \abs{S}$, and both $S$ and $\bar{S}$ are nonempty.
\textbf{Circle the formula} that correctly completes the following
statement:

There is a matching from $\leftbi{G}$ to $\rightbi{G}$ if and only if
there is both a matching from $S$ to its neighbors,
$\neighbors{S}$,\inhandout{\footnote{\emph{Reminder}: $\neighbors{S}$
    is the set of vertices that are adjacent to at least one vertex in
    $S$.}} and also a matching from $\bar{S}$ to
\[
\neighbors{\bar{S}} \qquad \bar{\neighbors{S}} \qquad
N^{-1}(\neighbors{S}) \qquad N^{-1}(\neighbors{\bar{S}})\qquad
\neighbors{\bar{S}} - \bar{\neighbors{S}} \qquad \neighbors{S} - \neighbors{\bar{S}}
\]

\hint The proof of Hall's Bottleneck Theorem.

\begin{solution}
$\bar{\neighbors{S}}$
\end{solution}

\eparts

\end{problem}

%%%%%%%%%%%%%%%%%%%%%%%%%%%%%%%%%%%%%%%%%%%%%%%%%%%%%%%%%%%%%%%%%%%%%
% Problem ends here
%%%%%%%%%%%%%%%%%%%%%%%%%%%%%%%%%%%%%%%%%%%%%%%%%%%%%%%%%%%%%%%%%%%%%

\endinput
