!\documentclass[problem]{mcs}

\begin{pcomments}
\pcomment{CP_six_shooter_probability}
\end{pcomments}

\pkeywords{
conditional_probability
}

%%%%%%%%%%%%%%%%%%%%%%%%%%%%%%%%%%%%%%%%%%%%%%%%%%%%%%%%%%%%%%%%%%%%%
% Problem starts here
%%%%%%%%%%%%%%%%%%%%%%%%%%%%%%%%%%%%%%%%%%%%%%%%%%%%%%%%%%%%%%%%%%%%%

\begin{problem} Dirty Harry places two bullets in the six-shell cylinder of his
revolver.  He gives the cylinder a random spin and says ``Feeling
lucky?'' as he holds the gun against your heart.

\bparts
\ppart What is the probability that you will get shot if he pulls the trigger?

\begin{solution}

$1/3$.  You will get shot when one of the slots containing a bullet is
  under the hammer (lined up with the gun barrel).  Each of the six
  slots in the cylinder is equally likely to be under the hammer, and
  two of these slots contain bullets, so your probability of getting
  shot is $2/6$,

\end{solution}

\ppart Suppose he pulls the trigger and you don't get shot.  What is
the probability that you will get shot if he pulls the trigger a
second time?

\begin{solution}
$2/5$.  There are five remaining slots that might be under the hammer
  next, and two of these contain bullets.
\end{solution}

\begin{staffnotes}
Add a tree diagram for 4-step method.
\end{staffnotes}

\ppart Suppose you noticed that he placed the two shells next to each
other in the cylinder.  How does this change the answers to the
previous two questions?
\begin{solution}
First answer remains $1/3$.  Second answer becomes $1/4$.

\begin{staffnotes}
Use the 4-step method to explain this.
\end{staffnotes}

\end{solution}
\eparts


\end{problem}


%%%%%%%%%%%%%%%%%%%%%%%%%%%%%%%%%%%%%%%%%%%%%%%%%%%%%%%%%%%%%%%%%%%%%
% Problem ends here
%%%%%%%%%%%%%%%%%%%%%%%%%%%%%%%%%%%%%%%%%%%%%%%%%%%%%%%%%%%%%%%%%%%%%

\endinput
