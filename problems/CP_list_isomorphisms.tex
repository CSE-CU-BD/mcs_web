\documentclass[problem]{mcs}

\begin{pcomments}
\pcomment{CP_list_isomorphisms}
\pcomment{MQ_list_isomorphisms is same but gives number of isomorphims}
\pcomment{subsumes CP_isomorphic_or_not_morning}
\pcomment{contributed by Rajeev Nayak, 10/18/09}
\end{pcomments}

\pkeywords{
  simple_graphs
  isomorphism
  preserved
}

%%%%%%%%%%%%%%%%%%%%%%%%%%%%%%%%%%%%%%%%%%%%%%%%%%%%%%%%%%%%%%%%%%%%%
% Problem starts here
%%%%%%%%%%%%%%%%%%%%%%%%%%%%%%%%%%%%%%%%%%%%%%%%%%%%%%%%%%%%%%%%%%%%%

\begin{problem}
List all the isomorphisms between the two graphs give in
Figure~\ref{iso_graphs_S12_cp7w}.  Explain why there are no others.

\begin{figure}[h]
\graphic{iso_graphs_morning}
\caption{Graphs with several isomorphisms}
\label{iso_graphs_S12_cp7w}
\end{figure}

\begin{solution}
These are the vertex correspondences for the four isomorphisms:
\begin{align*}
1A, 2B, 3C, 4D, 5E, 6F \\
1A, 2B, 3D, 4C, 5F, 6E \\
1B, 2A, 3C, 4D, 5E, 6F \\
1B, 2A, 3D, 4C, 5F, 6E
\end{align*}

Some simple reasoning leads us to this answer.  The first graph in
this problem has exactly two nodes with degree 3 (nodes 3 and 4), as
does the second graph ($c$ and $d$).  Recall that nodes related by an
isomorphism must have equal degrees.  Thus, the isomorphism must map 3
to either $c$ or $d$, and it must map 4 to the other.  Independently
of this choice, nodes 1, 2, $a$, and $b$ are the only nodes connected
to exactly the degree-3 nodes, so the isomorphism must divide 1 and 2
between $a$ and $b$.  Node 5 is the only unassigned node connected to
node 3, so 5 must be mapped to either $e$ or $f$, depending on which
is adjacent to the counterpart of node 3; and now node 6 maps to
whichever node is left.
\end{solution}

\end{problem}

%%%%%%%%%%%%%%%%%%%%%%%%%%%%%%%%%%%%%%%%%%%%%%%%%%%%%%%%%%%%%%%%%%%%%
% Problem ends here
%%%%%%%%%%%%%%%%%%%%%%%%%%%%%%%%%%%%%%%%%%%%%%%%%%%%%%%%%%%%%%%%%%%%%

\endinput
