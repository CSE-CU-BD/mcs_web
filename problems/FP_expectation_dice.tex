\documentclass[problem]{mcs}

\begin{pcomments}
  \pcomment{Source(s):F01, PS11}
\end{pcomments}

\pkeywords{
  conditional-probability
  probability
}

%%%%%%%%%%%%%%%%%%%%%%%%%%%%%%%%%%%%%%%%%%%%%%%%%%%%%%%%%%%%%%%%%%%%%
% Problem starts here
%%%%%%%%%%%%%%%%%%%%%%%%%%%%%%%%%%%%%%%%%%%%%%%%%%%%%%%%%%%%%%%%%%%%%

\begin{problem}
Suppose that $R$ is a random variable.  Let $C$ be an event.  The
\emph{conditional expectation, $\expect{R \mid C}$, of $R$ given
event $C$} is defined by:
\[
\expcond{R}{C} \eqdef \sum_{x \in \range{R}} x \cdot \prcond{R = x}{C}.
\]

\bparts

\ppart Compute the expected value of the number rolled on a
fair, six-sided die, given that the outcome is even.

\examspace[1.0in]
\exambox{0.5in}{0.5in}{0in}

\begin{solution}
Let the random variable $R$ be the number rolled, and let
$C$ be the event that the number rolled is even.
\begin{eqnarray*}
\expcond{R}{C}       & = &   \sum_{x=1}^{6} x \cdot \prob{R = x \mid C} \\
                & = &   1 \cdot 0 + 2 \cdot \frac{1}{3} + 
                        3 \cdot 0 + 4 \cdot \frac{1}{3} + 
                        5 \cdot 0 + 6 \cdot \frac{1}{3} \\
                & = &   4
\end{eqnarray*}
\end{solution}

\begin{staffnotes}
needs a set of possible justifications and reformatting
\end{staffnotes}

\ppart Consider the following identity (we assume that all
expectations exist).
\begin{eqnarray*}
\expect{R}      & = &   \prob{C} \cdot \expcond{R}{C} +
                        \prob{\overline{C}} \cdot \expcond{R}{\overline{C}}
\end{eqnarray*}

Below is a proof of this identity.  Give a justification for each step
in the given proof, appealing to either basic algebra or some rule
of probability covered in the course.

We transform the right side into the left side.  All
summations are over $x \in \text{Range}(R)$.
\begin{eqnarray*}
\lefteqn{\prob{C} \cdot \expect{R \mid C} +
        \prob{\overline{C}} \cdot \expect{R \mid \overline{C}}} \\
        & = &   \prob{C} \cdot ( \sum_{x}
                        x \cdot \prob{R = x \mid C} )
                + \prob{\overline{C}} \cdot  ( \sum_{x}
                        x \cdot \prob{R = x \mid \overline{C}} ) \\
        & = &   \prob{C} \cdot ( \sum_{x}
                        x \cdot \frac{\prob{[R = x] \cap C}}{\prob{C}} )
                + \prob{\overline{C}} \cdot  ( \sum_{x}
                        x \cdot \frac{\prob{[R = x] \cap \overline{C}}}
                        {\prob{\overline{C}}} ) \\
        & = &   \sum_{x} x \cdot (
                        \prob{[R = x] \cap C} + \prob{[R = x] \cap \overline{C}}
                        ) \\
        & = &   \sum_{x} x \cdot (
                        \prob{([R = x] \cap C) \cup ([R = x] \cap \overline{C})}
                        ) \\
        & = &   \sum_{x} x \cdot \prob{R = x} \\
        & = &   \expect{R}
\end{eqnarray*}
\begin{solution}
\end{solution}

\eparts
\end{problem}


%%%%%%%%%%%%%%%%%%%%%%%%%%%%%%%%%%%%%%%%%%%%%%%%%%%%%%%%%%%%%%%%%%%%%
% Problem ends here
%%%%%%%%%%%%%%%%%%%%%%%%%%%%%%%%%%%%%%%%%%%%%%%%%%%%%%%%%%%%%%%%%%%%%
