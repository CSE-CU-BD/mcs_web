\documentclass[problem]{mcs}

\begin{pcomments}
  \pcomment{FP_logic_relations}
  \pcomment{from: F08.final}
\end{pcomments}

\pkeywords{
  asymptotic bounds
  partial order
}

%%%%%%%%%%%%%%%%%%%%%%%%%%%%%%%%%%%%%%%%%%%%%%%%%%%%%%%%%%%%%%%%%%%%%
% Problem starts here
%%%%%%%%%%%%%%%%%%%%%%%%%%%%%%%%%%%%%%%%%%%%%%%%%%%%%%%%%%%%%%%%%%%%%

\begin{problem}
We define a relation $R$ on pairs of the Boolean values $\true$ and $\false$ 
by
\[
(p_1,p_2)\mrel{R}(q_1,q_2) \qiff \left[
\begin{array}{ll}
(p_1\QOR\ p_2) \QIMPLIES (q_1\QOR\ q_2) & \mbox{ and } \\
(p_1\QOR\ p_2) \QIMPLIES (q_1 \QOR\ q_2) & 
\end{array} \right].
\]
For example, since $(\false \QOR\ \true) \QIMPLIES\ (\true \QOR\ \false)$
and $(\false \QAND\ \true) \QIMPLIES\ (\true \QAND\ \false)$,
\[
(\false,\true)\mrel{R}(\true,\false).
\]

\bparts

\ppart Give another couple of pairs of Boolean values $(p_1,p_2)$ and
$(q_1,q_2)$ for which $(p_1,p_2)\mrel{R}(q_1,q_2)$.

\begin{solution}[\vspace{1.5cm}]
For example, $(\false,\false)\mrel{R}(\true,\true)$.
\end{solution}

\ppart Give a couple of pairs of Boolean values $(p_1,p_2)$ and
$(q_1,q_2)$ for which $(p_1,p_2)\mrel{R}(q_1,q_2)$ does not hold.

\begin{solution}[\vspace{1.5cm}]
For example, $(\true,\true)\mrel{R}(\false,\false)$.
\end{solution}

For the following parts provide a brief explanation (e.g., by giving a counter example):

\ppart Is $R$ reflexive? 

\begin{solution}[\vspace{3cm}]
Yes, $(p_1\QOR\ p_2) \QIMPLIES\ (p_1\QOR\ p_2)$ and $(p_1\QAND\ p_2)
\QIMPLIES\ (p_1\QAND\ p_2)$.
\end{solution}

\ppart Is $R$ symmetric? 

\begin{solution}[\vspace{3cm}]
No, $(\true, \false)\mrel{R}(\true, \true)$ but not $(\true,
\true)\mrel{R}(\true, \false)$.
\end{solution}

\ppart Is $R$ antisymmetric? 

\begin{solution}
No, $(\true, \false)\mrel{R}(\false,\true)$,
$(\false,\true)\mrel{R}(\true,\false)$, and $(\true,\false)\neq
(\false,\true)$.
\end{solution}

\ppart Is $R$ transitive?   \TBA{OMITTED F08}

\begin{solution}
Yes, this follows directly from the transitivity of $\Rightarrow$. If
$(p_1\QOR\ p_2) \QIMPLIES\ (q_1\QOR\ q_2)$ and $(q_1\QOR\ q_2)
\QIMPLIES\ (z_1\QOR\ z_2)$, then $(p_1\QOR\ p_2)
\QIMPLIES\ (z_1\QOR\ ! z_2)$. If $(p_1\QAND\ p_2)
\QIMPLIES\ (q_1\QAND\ q_2)$ and $(q_1\QAND\ q_2)
\QIMPLIES\ (z_1\QAND\ z_2)$, then $(p_1\QAND\ p_2)
\QIMPLIES\ (z_1\QAND\ z_2)$. So, $(p_1,p_2)\mrel{R}(q_1,q_2)$ and
$(q_1,q_2)\mrel{R}(z_1,z_2)$ implies $(p_1,p_2)\mrel{R}(z_1,z_2)$
\end{solution}

\eparts

\end{problem}

\endinput
