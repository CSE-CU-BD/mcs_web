\documentclass[problem]{mcs}

\begin{pcomments}
  \pcomment{CP_bogus_st_peterburg}
  \pcomment{added by ARM 4/15/11}
\end{pcomments}

\pkeywords{
  st_peterburg
  expectation
  linearity
}

%%%%%%%%%%%%%%%%%%%%%%%%%%%%%%%%%%%%%%%%%%%%%%%%%%%%%%%%%%%%%%%%%%%%%
% Problem starts here
%%%%%%%%%%%%%%%%%%%%%%%%%%%%%%%%%%%%%%%%%%%%%%%%%%%%%%%%%%%%%%%%%%%%%

\begin{problem}
One of the simplest casino bets is on ``red'' or ``black'' at the roulette
table.  In each play at roulette, a small ball is set spinning around a
roulette wheel until it lands in a red, black, or green colored slot.
The payoff for a bet on red or black matches the bet; for example, if you bet
$\$10$ on red and the ball lands in a red slot, you get back your original
$\$10$ bet plus another matching $\$10$.

In the US, a roulette wheel has 2 green slots among 18 black and 18 red
slots, so the probability of red is $p::= 18/38 \approx 0.473$.  In
Europe, where roulette wheels have only 1 green slot, the odds for red
are a little better ---that is, $p = 18/37 \approx 0.486$ ---but still less
than even.

There is a notorious gambling strategy known as the
\term{St.\ Petersburg strategy} to profit at roulette: bet $\$10$ on
red, and keep doubling the bet until a red comes up.  Since we
shouldn't expect any strategy to work against a biased game, let's
analyze how badly the St.\ Petersburg strategy will fare.

To model the gambling process, let's first make the sample space
explicit: a sample point wil be a sequence $B^nR$ representing a run
of $n \geq 0$ black spins terminated by a red spin.  Let $N$ be the
number of spins before a red first comes up, that is, $N$ is the
random variable whose value at $B^nR$ is $n$.  So
\[
\pr{N = n} = \paren{\frac{20}{38}}^{n}\frac{18}{38},
\]
and
\[
\pr{N \geq n] = \pr{\text{first $n$ spins are not red}} = \paren{\frac{20}{38}}^{n}.
\]

\iffalse
\sum_{i\in\naturals} \pr{N = n + i}
             = \sum_{i\in\naturals} \paren{\frac{20}{38}}^{n+i}\frac{18}{38}
             = \paren{\frac{20}{38}}^{n}\frac{18}{38}
                  \sum_{i\in\naturals} \paren{\frac{20}{38}}^i
             = \paren{\frac{20}{38}}^{n}\frac{18}{38} \frac{1}{1-(20/38)}
             = \paren{\frac{20}{38}}^{n}\frac{18}{38} \frac{38}{18}
             = \paren{\frac{20}{38}}^{n}\fi

Let $C_n$ be the number of dollars won on the $n$th spin.  So $C_n =
10 \cdot 2^{n-1}$ when red comes up for the first time on the $n$th
spin, that is, when $N= n-1$.
Similarly, $C_n = -10 \cdot 2^{n-1}$ when the first red spin comes up
after the $n$th spin, namely, when $N \geq n$.  We will define $C_n$
to be 0 when $N < n$.  So
\[
\expect{C_n} = \paren{10 \cdot 2^{n-1}}\cdot \pr{N=n} +
               \paren{-10  \cdot  2^{n-1}}\cdot \pr{N \geq n+1}
             =  \paren{10 \cdot 2^{n-1}}\cdot
                 \paren{\paren{\frac{20}{38}}^{n}\frac{18}{38} -
                        \paren{\frac{20}{38}}^{n+1}}
             = \paren{10 \cdot 2^{n-1}}\cdot \paren{\frac{20}{38}}^{n}\paren{-\frac{2}{38}}
             = \feac{10}{2}\paren{2 \frac{20}{38}}^{n}\paren{-\frac{2}{38}}
             = -\frac{5}{19}\paren{\frac{20}{19}}^{n}.
\]
Now using the St. Petersburg strategy, you will win $W$ dollars, where
\[
W = \sum_{n \in \naturals} C_n,
\]
so
\[
expect{W} = \expect{\sum_{n \in \naturals} C_n},
\]
which by linearity of expectation is
\[
\sum_{n \in \naturals} \expect{C_n}
    = -\frac{5}{19}\sum_{n \in \naturals} \paren{\frac{20}{19}}^{n} =
    - \infty.
\]

\begin{editingnotes}
Editing stopped here.  The bias complicates things and makes the sum
diverge, so better to revise to be a fair roulette wheel as wasd
originally done below.
\end{editingnotes}

The dollar amount won in any gambling session is the value of the sum
$\sum_{i = 1}^\infty C_i$.  At any sample point $B^nR$, the value of this
sum is
\[
10 \cdot -(1 + 2 + 2^2 + \dots + 2^{n - 1}) + 10 \cdot 2^n  = 10,
\]
which trivially implies that its expectation is 10 as well.  That is, the
amount we are \emph{certain} to leave the casino with, as well as
expectation of the amount we win, is $\$10$.

Moreover, our reasoning that $\expect{C_i} = 0$ is sound, so
\[
\sum_{i = 1}^\infty \expect{C_i} = \sum_{i = 1}^\infty 0 = 0.
\]

This strategy
implies that a player will leave the game as a net winner of $\$10$ as
soon as the red first appears.  Of course the player may need an
awfully large bankroll to avoid going bankrupt before red shows up
---but we know that the mean time until a red occurs is $1/p$, so it
seems possible that a moderate bankroll might actually work out.  (In
this setting, a ``win'' on red corresponds to a ``failure'' in a
mean-time-to-failure situation.)

But wait a minute.  As long as there is a fixed, positive probability of
red appearing on each spin of the wheel ---even if the wheel is
unfair ---it's \emph{certain} that red will eventually come up.  So with
probability one, we leave the casino having won $\$10$, and our expected
dollar win is obviously $\$10$, not zero!

Something's wrong here.  What?

\begin{solution}

The expected amount won is indeed $\$10$.

The argument claiming the expectation is zero is flawed by an invalid use
of linearity of expectation for an infinite sum.  To pinpoint this flaw,
let's first make the sample space explicit: a sample point is a sequence
$B^nR$ representing a run of $n \geq 0$ black spins terminated by a red
spin.  Since the wheel is fair, the probability of $B^nR$ is $2^{-(n+1)}$.

Let $C_i$ be the number of dollars won on the $i$th spin.  So
$C_i = 10 \cdot 2^{i-1}$
when red comes up for the first time on the $i$th spin, that is, at
precisely one sample point, namely $B^{i-1}R$.  Similarly,
$C_i = -10  \cdot  2^{i-1}$
when the first red spin comes up after the $i$th spin, namely, at the
sample points $B^nR$ for $n \geq i$.  Finally, we will define $C_i$ by
convention to be zero at sample points in which the session ends before the
$i$th spin, that is, at points $B^nR$ for $n < i-1$.

The dollar amount won in any gambling session is the value of the sum
$\sum_{i = 1}^\infty C_i$.  At any sample point $B^nR$, the value of this
sum is
\[
10 \cdot -(1 + 2 + 2^2 + \dots + 2^{n - 1}) + 10 \cdot 2^n  = 10,
\]
which trivially implies that its expectation is 10 as well.  That is, the
amount we are \emph{certain} to leave the casino with, as well as
expectation of the amount we win, is $\$10$.

Moreover, our reasoning that $\expect{C_i} = 0$ is sound, so
\[
\sum_{i = 1}^\infty \expect{C_i} = \sum_{i = 1}^\infty 0 = 0.
\]

The flaw in our argument is the claim that, since the expectation at each
spin was zero, therefore the final expectation would also be zero.
Formally, this corresponds to concluding that
\[
\expect{\mbox{amount won}}  =  \expect{\sum_{i = 1}^\infty C_i}
  =  \sum_{i = 1}^\infty \expect{C_i} = 0.
\]
The flaw lies exactly in the second equality.  This is a case where
linearity of expectation fails to hold---even though both $\sum_{i =
1}^\infty \expect{C_i}$ and $\expect{\sum_{i = 1}^\infty C_i}$ are
finite---because the convergence hypothesis needed for linearity is false.
Namely, the sum
\[
\sum_{i = 1}^\infty \expect{\abs{C_i}}
\]
does not converge.  In fact, the expected value of $\abs{C_i}$ is $10$
because $\abs{C_i} =  10 \cdot 2^{i}$  with probability $2^{-i}$ and
otherwise is zero, so this sum rapidly approaches infinity.

Probability theory truly leads to this apparently paradoxical conclusion: a
game allowing an unbounded---even though always finite---number of ``fair''
moves may not be fair in the end.  In fact, our reasoning leads to an even
more startling conclusion: even against an \emph{unfair} wheel, as long as
there is some fixed positive probability of red on each spin, we are
certain to win $\$10$!

This is clearly a case where naive intuition is unreliable: we don't
expect to beat a fair game, and we do expect to lose when the odds are
against us.  Nevertheless, the ``paradox'' that in fact we always win by
bet-doubling cannot be denied.

But remember that from the start we chose to assume that no one goes
bankrupt while executing our bet-doubling strategy.  This assumption is
crucial, because the expected loss while waiting for the strategy to
produce its ten dollar profit is actually infinite!  So it's not
surprising, after all, that we arrived at an apparently paradoxical
conclusion from an unrealistic assumption.

This example also serves a warning that in making use of infinite
linearity of expectation, the convergence hypothesis which justifies it
had better be checked.

\end{solution}
\end{problem}

%%%%%%%%%%%%%%%%%%%%%%%%%%%%%%%%%%%%%%%%%%%%%%%%%%%%%%%%%%%%%%%%%%%%%
% Problem ends here
%%%%%%%%%%%%%%%%%%%%%%%%%%%%%%%%%%%%%%%%%%%%%%%%%%%%%%%%%%%%%%%%%%%%%

\endinput
 
