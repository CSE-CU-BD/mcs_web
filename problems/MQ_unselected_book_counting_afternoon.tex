\documentclass[problem]{mcs}

\begin{pcomments}
  \pcomment{MQ_unselected_book_counting_afternoon}
  \pcomment{from: S09.mq5} edited/shortened by Tom Brown 11/14/09}
\end{pcomments}

\pkeywords{
  counting
  bijection
}

%%%%%%%%%%%%%%%%%%%%%%%%%%%%%%%%%%%%%%%%%%%%%%%%%%%%%%%%%%%%%%%%%%%%%
% Problem starts here
%%%%%%%%%%%%%%%%%%%%%%%%%%%%%%%%%%%%%%%%%%%%%%%%%%%%%%%%%%%%%%%%%%%%%

\begin{problem}

Suppose you want to select $k$ out of $n$ books on a shelf so that
there are always at least 4 unselected books between selected books.
Describe a bijection between bit strings and book selection.
(Assume $n$ is large enough for this to be possible.)

\end{problem}

\begin{solution}
\[
\binom{n-4(k-1)}{k}
\]
for $n \geq 5k-4$.

A selection of $k$ among $n$ books on a shelf corresponds in
  an obvious way to an $n$-bit string with exactly $k$ \texttt{1}'s.

  So the problem reduces to finding a bijection between $n$-bit strings
  with $k$ \texttt{1}'s that are at least 4 apart, and $n-4(k-1)$-bit
  strings with $k$ \texttt{1}'s.

  But in a string, $s$, with $k$ \texttt{1}'s that are at least 4 apart,
  all but the last \texttt{1} must have a \texttt{0000} to its right.  So we
  can map $s$ to a string with $k$ \texttt{1}'s and $4(k-1)$
  fewer \texttt{0}'s by erasing the \texttt{0000}'s immediately to the right
  of each of the first $k-1$ \texttt{1}'s.

  This map is a bijection because given any $n-4(k-1)$-bit string with
  $k$ \texttt{1}'s, there is a unique $n$-bit string with \texttt{1}'s that
  are at least 4 apart that maps to it, namely, the string obtained by
  replacing each of the first $k-1$ \texttt{1}'s in the $n-4(k-1)$-bit
  string by a \texttt{10000}.
\end{solution}

%%%%%%%%%%%%%%%%%%%%%%%%%%%%%%%%%%%%%%%%%%%%%%%%%%%%%%%%%%%%%%%%%%%%%
% Problem ends here
%%%%%%%%%%%%%%%%%%%%%%%%%%%%%%%%%%%%%%%%%%%%%%%%%%%%%%%%%%%%%%%%%%%%%

\endinput
