\documentclass[problem]{mcs}

\begin{pcomments}
  \pcomment{FP_simple_graphs_asymptotics_conflict}
  \pcomment{variation of FP_simple_graphs_asymptotics}
  \pcomment{ARM 11/5/15}
\end{pcomments}

\pkeywords{
  vertices
  edge
  coloring
  chromatic_number
  big_Oh
  connected
}

%%%%%%%%%%%%%%%%%%%%%%%%%%%%%%%%%%%%%%%%%%%%%%%%%%%%%%%%%%%%%%%%%%%%%
% Problem starts here
%%%%%%%%%%%%%%%%%%%%%%%%%%%%%%%%%%%%%%%%%%%%%%%%%%%%%%%%%%%%%%%%%%%%%

\begin{problem} 
Let $f,g$ be positive real-valued functions on finite,
\emph{connected}, simple graphs.  We will extend the $O()$ notation to
such graph functions as follows: $f = O(g)$ iff
there is a constant $c>0$ such that
\[
f(G) \leq c \cdot g(G) \text{ for all connected simple graphs } G.
\]
For each of the following assertions, state whether it is \True\ or
\False\, and briefly explain your answer.  You are \textbf{not}
expected to offer a careful proof or detailed counterexample.

\emph{Reminder}: $\vertices{G}$ is the set of vertices and $\edges{G}$
is the set of edges of $G$, and $G$ is connected.

\bparts

\ppart (The \term{diameter} of a simple graph is the length of its
longest path.)

\[
\card{\vertices{G}} = O(\text{diameter}(G)).
\]

\begin{solution}
\False.

The $n$-vertex star graph \inbook{illustrated in Figure~\bref{fig:5T}}
\inhandout{illustrated in Figure~\ref{fig:star7}} for $n=7$, has $n-1$
edges and diameter 2.

\begin{figure}

\graphic{star-graph}

\caption{A 7-vertex star graph.}

\label{fig:star7}

\end{figure}

\end{solution}

\examspace[1in]

\ppart 
\[
\card{\vertices{G}} = O(\card{\text{spanning-trees}(G)})
\]

\begin{solution}
\False.

If $G$ is a line graph, it has only one spanning tree.
\end{solution}

\ppart
\[
\card{\vertices{G}} = O(\card{\edges{G}}).
\]

\begin{solution}
\True.

Since $G$ is connected, the number of edges, $\card{\edges{G}}$, is at
most one less than the number of vertices.  That is,
\[
\card{\vertices{G}} \leq \card{\edges{G}} + 1 = O(\card{\edges{G}}).
\]

\end{solution}


\ppart
\[
\card{\text{spanning-trees}(G)} = O(\card{\edges{G}}).
\]

\begin{solution}
\False.

If $G$ is a complete graph, every path of length
$\card{\vertices{G}}-1$ is a spanning tree, and there are $n!$ such
paths.
\end{solution}

\eparts

\end{problem}

\endinput
