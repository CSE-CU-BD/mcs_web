\documentclass[problem]{mcs}

\begin{pcomments}
  \pcomment{FP_graphs_short_answer}
  \pcomment{excerpted from FP_multiple_choice_unhidden}
\end{pcomments}

\pkeywords{
  isomorphism
  partial_order  
  total_order
  linear
  asymptotic
  vertices
  stationary_distribution
  countable
}

%%%%%%%%%%%%%%%%%%%%%%%%%%%%%%%%%%%%%%%%%%%%%%%%%%%%%%%%%%%%%%%%%%%%%
% Problem starts here
%%%%%%%%%%%%%%%%%%%%%%%%%%%%%%%%%%%%%%%%%%%%%%%%%%%%%%%%%%%%%%%%%%%%%

\begin{problem} \mbox{}

\bparts

\ppart Circle all the properties below that are preserved under graph
isomorphism.

\begin{itemize}

\item There is a cycle that includes all the vertices.

\item Two edges are of equal length.

\item The graph remains connected if any two edges are removed.

\item There exists an edge that is an edge of every spanning tree.

\item The negation of a property that is preserved under isomorphism.

%% Removed
%\item There are two degree 8 vertices. % too easy
%\item The area enclosed by a graph equals some constant, $A$.  % true for trees?
%\item There are two simple cycles that do not share any vertices. % too easy
%\item There are two connected components. % too easy
%\item The graph can be drawn such that all the edges have the same length. % huh?
%\item The graph is 4-colorable. % too easy
%\item Adding an edge between any two vertices creates a cycle. % spanning tree one better
\end{itemize}

\begin{solution}
All but the second one are preserved.
\end{solution}

% FROM: Spring07 MQ-4/6-2a
%
% COMMENTS: Reduced number, changed statements, now asking for
% counterexample. Solution incomplete. Should we replace "Lemma 3.4"
% with an \eqref?
%
% Need to make boxes (make sure they know that counterexamples req'd): 
%     true        false__________________ <-- room for counterexample
%


% ~~~~~~~~~~~~~~~~~~~~~~~~~~~~~~~~~~~~~~~~~~~~~~~~~~~~~~~~~~~~~~~~~~~
% FROM: Spring07 MQ-3/14-5
%
% COMMENTS: Reduced number, DID NOT change statements, now asking for
% counterexample.
%

\ppart For the following statements about \textbf{finite trees}, circle \textbf{true}
or \textbf{false}, and \emph{provide counterexamples} for those that
are \textbf{false}.

\begin{itemize}

\item Any connected subgraph is a tree.  \hfill \textbf{true} \qquad
  \textbf{false} \examspace[0.8in]

\begin{solution}
\textbf{true.}
\end{solution}

\item Adding an edge between two nonadjacent vertices creates a
  cycle. \hfill \textbf{true} \qquad \textbf{false} \examspace[0.8in]

\begin{solution}
\textbf{true.}
\end{solution}

\item The number of vertices is one less than twice the number of
  leaves.  \hfill 
\textbf{true} \qquad \textbf{false}  \examspace[0.8in]

\begin{solution}
  \textbf{false}.  This property holds for full binary trees, but not in
  general.  A tree with two vertices is a counterexample.
\end{solution}

\item The number of vertices is one less than the number of
  edges.  \hfill 
\textbf{true} \qquad \textbf{false}  \examspace[0.8in]

\begin{solution}
  \textbf{false}.  This got ``edges'' and ``vertices'' reversed.
Every tree is a counterexample.
\end{solution}

\item For every finite graph (not necessarily a tree), there is one (a
  finite tree) that spans it.
\hfill \textbf{true} \qquad \textbf{false} \examspace[0.8in]

\begin{solution}
\textbf{false}.  Any disconnected graph is a counterexample.
\end{solution}

\end{itemize}

\iffalse

\item For every, possibly infinite, connected graph, there is one (a tree) that spans it.
\hfill \textbf{true} \qquad \textbf{false} \examspace[0.8in]

\begin{solution}
false.  Any dense order, for example, $\rationals$ with
$\diredge{r}{s}$ iff $r < s$
\end{solution}

\ppart What is the minimum number of \textbf{vertices} possible in a
nonplanar graph?  \hfill \examrule[0.5in]

\begin{solution}
5.  $K_5$ is the smallest nonplanar graph.
\end{solution}

\ppart What is the minimum number of \textbf{edges} possible in a
nonplanar graph that is 2-colorable?  \hfill \examrule[0.5in]

\begin{solution}
9.  $K_{3,3}$ is the smallest 2-colorable nonplanar graph.
\end{solution}

\ppart A \term{sink} in a digraph is a vertex with no edges leaving
it.  Circle whichever of the following assertions are true of
\idx{stationary distributions} on finite digraphs with exactly two sinks:

\begin{itemize}

\item there may not be any

\item there may be a unique one

\item there are exacty two

\item there may be a countably infinite number

\item there may be a uncountable number

\item there always is an uncountable number

\end{itemize}

\begin{solution}
The first four choices are false, and the last two are true.
That's because a distribution in which one sink has probability $r \in
[0,1] \subseteq \reals$ and the other sink has probability $1-r$ is
stable, and there are an uncountable number of real numbers in $[0,1]$.
\end{solution}
\fi

\eparts

\end{problem}

\endinput
