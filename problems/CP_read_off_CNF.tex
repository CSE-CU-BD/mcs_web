\documentclass[problem]{mcs}

\begin{pcomments}
    \pcomment{CP_read_off_CNF}
    \pcomment{by ARM 2/3/16}
\end{pcomments}

\pkeywords{
  proposition
  formula
  dnf
  cnf
  equivalence
  truth_table}

\begin{problem}
Let $P$ be the proposition with truth table given below.  Write out
both a disjunctive and a conjunctive normal form for $P$.

\[\begin{array}{| l | c | c | c | r |}
  \hline
  A & B & C & D & P \\ \hline
  \true & \true & \true & \true & \true  \\ \hline
  \true & \true & \true & \false & \false  \\ \hline
  \true & \true & \false & \true & \true  \\ \hline
  \true & \true & \false & \false & \false  \\ \hline
  \true & \false & \true & \true & \true  \\ \hline
  \true & \false & \true & \false & \true  \\ \hline
  \true & \false & \false & \true & \true  \\ \hline
  \true & \false & \false & \false & \true  \\ \hline
 \false & \true & \true & \true & \true \\ \hline
 \false & \true & \true & \false & \false  \\ \hline
 \false & \true & \false & \true & \true  \\ \hline
 \false & \true & \false & \false & \false  \\ \hline
 \false & \false & \true & \true & \false  \\ \hline
 \false & \false & \true & \false & \false  \\ \hline
 \false & \false & \false & \true & \true  \\ \hline
 \false & \false & \false & \false & \true  \\ \hline

\end{array}\]


\begin{solution}
The proposition that is \true\ when $ABCD$ have the respective values
$\true\true\false\true$ and is \false\ in all other cases can be expressed
as the product
\[
A_{\true\true\false\true} \eqdef A \QAND\, B \QAND\, \bar{C} \QAND\, D.
\]

So if we let $\vec{r}, \vec{s}, \dots, \vec{t}$ be the rows where $P$
is \true, then
\[
A_{\vec{r}}\ \QOR\ A_{\vec{s}}\ \QOR \dots \QOR\ A_{\vec{t}}
\]
will be a disjunctive normal form that is \true\ in exactly the same
rows as $P$.  That is, it will be a disjunctive normal form for $P$.
This reasoning leads to the following disjunctive normal form for $P$:
\[\begin{array}{lr}
(A \QAND\, B \QAND\, C \QAND\, D)           &       \QOR\\
(A \QAND\, B \QAND\, \bar{C} \QAND\, D)     &       \QOR\\
(A \QAND\, \bar{B} \QAND\, C \QAND\, D)     &       \QOR\\
(A \QAND\, \bar{B} \QAND\, C \QAND\, \bar{D}) &     \QOR\\
(A \QAND\, \bar{B} \QAND\, \bar{C} \QAND\, D) &     \QOR\\
(A \QAND\, \bar{B} \QAND\, \bar{C} \QAND\, \bar{D}) & \QOR\\
(\bar{A} \QAND\, B \QAND\, C \QAND\, D)     &       \QOR\\
(\bar{A} \QAND\, B \QAND\, \bar{C} \QAND\, D) &     \QOR\\
(\bar{A} \QAND\, \bar{B} \QAND\, \bar{C} \QAND\, D) & \QOR\\
(\bar{A} \QAND\, \bar{B} \QAND\, \bar{C} \QAND\, \bar{D})

\end{array}\]

We can find a conjunctive normal form by similar reasoning.  Namely,
a row at which $P$ is \false\ is when $ABCD$ have the values
$\true\true\true\false$.  The proposition that is \false\ for this row
alone, and \true\ for all other rows can be expressed as a conjunct:
\[
C_{\true\true\true\false} \eqdef \bar{A} \QOR\ \bar{B} \QOR\  \bar{C} \QOR\ D.
\]
This means that for any proposition, $Q$, the proposition
\[
C_{\true\true\true\false} \QAND Q
\]
will be \false\ in row $\true\true\true\false$ and will agree with $Q$
in all other rows.

So if we let $\vec{r}, \vec{s}, \dots, \vec{t}$ be the rows where $P$
is \false, then 
\[
C_{\vec{r}} \QAND C_{\vec{s}} \QAND \dots \QAND\ C_{\vec{t}}
\]
will be a conjunctive normal form that is \false\ in exactly the same
rows as $P$.  That is, it will be a conjunctive normal form for $P$.

This reasoning leads to the following conjunctive normal form for $P$:
\[\begin{array}{lr}
(\bar{A} \QOR\, \bar{B} \QOR\, \bar{C} \QOR\, D) &
   \QAND\\
(\bar{A} \QOR\, \bar{B} \QOR\, C \QOR\, D)       &
   \QAND\\
(A \QOR\, \bar{B} \QOR\, \bar{C} \QOR\, D)       &
   \QAND\\
(A \QOR\, \bar{B} \QOR\, C \QOR\, D)             &
   \QAND\\                                            
(A \QOR\, B \QOR\, \bar{C} \QOR\, \bar{D})       &
   \QAND\\                                            
(A \QOR\, B \QOR\, \bar{C} \QOR\, D)             
\end{array}\]

\end{solution}

\inhandout{\hint See Section~\bref{normal_form_sec}.}

\end{problem}

\endinput
