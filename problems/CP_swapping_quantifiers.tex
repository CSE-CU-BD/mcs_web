\documentclass[problem]{mcs}

\begin{pcomments}
  \pcomment{CP_swapping quantifiers}
  \pcomment{revised ARM 2/8/11}
\end{pcomments}

\pkeywords{
  quantifier
  predicate
  counter_model
  valid
}

%%%%%%%%%%%%%%%%%%%%%%%%%%%%%%%%%%%%%%%%%%%%%%%%%%%%%%%%%%%%%%%%%%%%%
% Problem starts here
%%%%%%%%%%%%%%%%%%%%%%%%%%%%%%%%%%%%%%%%%%%%%%%%%%%%%%%%%%%%%%%%%%%%%

\begin{problem}
Provide a counter-model for the implication that is not valid.
Informally explain why the other one is valid.
\begin{enumerate}
\item $\forall x.\, \exists y. P(x, y) \QIMPLIES \exists y.\, \forall x.\, P(x, y)$
\item $\exists y.\, \forall x. P(x, y) \QIMPLIES \forall x.\, \exists y.\, P(x, y)$
\end{enumerate}

\begin{solution}
  The first implication, $\forall x.\, \exists y.\, P(x, y) \QIMPLIES \exists
  y.\, \forall x.\, P(x, y)$, is invalid.

  One counter-model is the predicate $P(x, y) \eqdef y < x$ where the
  domain of discourse is the real numbers $\reals$.  For every real
  number $x$, there exists a real number $y$ which is strictly less than
  $x$, so the antecedent of the implication is true.  But there is no
  minimum real number, so the consequent is false.

  The second implication is valid.  Let's say that ``$x$ is good for $y$''
  when $P(x,y)$ is true.  The hypothesis says that there is some
  element, call it $g$, that is good for everything.  The conclusion is
  that every element has something that is good for it, which of course is
  true since $g$ will be good for it.

\begin{staffnotes}
It's not clear students will be able to articulate the validity
explanation.  If they get stuck, offer them the ``$x$ is good for $y$''
phrase as helpful.  If it doesn't help, then explain the answer.
\end{staffnotes}

\end{solution}


\end{problem}

%%%%%%%%%%%%%%%%%%%%%%%%%%%%%%%%%%%%%%%%%%%%%%%%%%%%%%%%%%%%%%%%%%%%%
% Problem ends here
%%%%%%%%%%%%%%%%%%%%%%%%%%%%%%%%%%%%%%%%%%%%%%%%%%%%%%%%%%%%%%%%%%%%%

\endinput
