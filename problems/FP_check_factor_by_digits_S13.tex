\documentclass[problem]{mcs}

\begin{pcomments}
  \pcomment{FP_check_factor_by_digits_S13}
  \pcomment{perturbation of FP_check_factor_by_digits}
  \pcomment{ARM 5/22/13}
\end{pcomments}

\pkeywords{
  number theory
  modular
}

%%%%%%%%%%%%%%%%%%%%%%%%%%%%%%%%%%%%%%%%%%%%%%%%%%%%%%%%%%%%%%%%%%%%%
% Problem starts here
%%%%%%%%%%%%%%%%%%%%%%%%%%%%%%%%%%%%%%%%%%%%%%%%%%%%%%%%%%%%%%%%%%%%%

\begin{problem}
The sum of the digits of the base 10 representation of an integer is
congruent modulo 9 to that integer.  For example,
\[
763 \equiv 7+6+3 \pmod 9.
\] 
This is not always true for the base 11 representation,
however.  For example,
\[
(763)_{11} = 7\cdot 11^2 + 6\cdot 11 + 3 \equiv 3 \not\equiv 5 \equiv 7 + 6 + 3 \pmod {11}.
\]
For exactly what integers $k \in (1,10]$ is it true that the sum of
  the digits of the base 11 representation of every nonnegative
  integer is congruent modulo $k$ to that integer?
  \inhandout{(No
    explanation is required, but no part credit without an
    explanation.)}

\begin{center}
\exambox{1.0in}{0.5in}{0.0in}
\end{center}

\examspace[3in]

\begin{solution}
\[
2,5,10.
\]

Summing the digits mod $k$ clearly works when $11 \equiv 1 \pmod k$.
This is equivalent to $k \divides 11-1 = 10$.  So the three factors of
10 are $k$'s that work.

To see why only these $k$'s work, just look at hex representation of
11, namely, the string \STR{10}.  The digit-sum requirement means
$11 \equiv 1 + 0 = 1 \pmod k$.
\end{solution}

\end{problem}

%%%%%%%%%%%%%%%%%%%%%%%%%%%%%%%%%%%%%%%%%%%%%%%%%%%%%%%%%%%%%%%%%%%%%
% Problem ends here
%%%%%%%%%%%%%%%%%%%%%%%%%%%%%%%%%%%%%%%%%%%%%%%%%%%%%%%%%%%%%%%%%%%%%

\endinput

