\documentclass[problem]{mcs}

\begin{pcomments}
  \pcomment{PS_unique_MST}
  \pcomment{by ARM 3/14/11}
\end{pcomments}

\pkeywords{
spanning_tree
MST
}

%%%%%%%%%%%%%%%%%%%%%%%%%%%%%%%%%%%%%%%%%%%%%%%%%%%%%%%%%%%%%%%%%%%%%
% Problem starts here
%%%%%%%%%%%%%%%%%%%%%%%%%%%%%%%%%%%%%%%%%%%%%%%%%%%%%%%%%%%%%%%%%%%%%

\begin{problem}

\iffalse
\bparts

\ppart Show that every minimum spanning tree of a graph can be the result of
Algorithm~\bref{alg:MST1}.

\begin{solution}
\TBA{soln needed}
\end{solution}

\ppart

\fi

Prove Corollary~\bref{cor:uniqMST}: If all edges in a finite weighted
graph have distinct weights, then the graph has a \emph{unique} MST.

\hint Suppose $M$ and $N$ were different MST's of the same graph.  Let
$e$ be the smallest edge in one and not the other, say $e \in M-N$,
and observe that $N+e$ must have a cycle.

\iffalse

\begin{solution}
  Algorithm~\bref{alg:MST1} is deterministic without ties, so the
  unique tree it produces must be all the trees there are.
\end{solution}
\fi

\begin{solution}
Assume for the sake of contradiction that $M$ and $N$ were different
MST's of the same graph.  Let $e$ be a minimum weight edge as in the
hint.

Since $N$ is a spanning tree, we know that $N+e$ is connected, and it
has too many edges to be a tree, so $N+e$ has a cycle.  Since $M$ has
no cycles, the cycle in $N+e$ cannot consist solely of edges from $M$.
So there must be an edge $g$ on the cycle that is larger than $e$.
Removing $g$ from $N+e$ leaves a connected graph with the same number
of nodes and edges as $N$, so $N+e-g$ must be a spanning tree.  But
$N+e-g$ weighs $w(g)-w(e)$ less than $N$, contradicting the minimality
of $N$.

\end{solution}

%\eparts

\end{problem}



%%%%%%%%%%%%%%%%%%%%%%%%%%%%%%%%%%%%%%%%%%%%%%%%%%%%%%%%%%%%%%%%%%%%%
% Problem ends here
%%%%%%%%%%%%%%%%%%%%%%%%%%%%%%%%%%%%%%%%%%%%%%%%%%%%%%%%%%%%%%%%%%%%%
