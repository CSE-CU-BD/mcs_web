\documentclass[problem]{mcs}

\begin{pcomments}
  \pcomment{PS_unique_MST}
  \pcomment{by ARM 3/14/11}
\end{pcomments}

\pkeywords{
spanning_tree
MST
}

%%%%%%%%%%%%%%%%%%%%%%%%%%%%%%%%%%%%%%%%%%%%%%%%%%%%%%%%%%%%%%%%%%%%%
% Problem starts here
%%%%%%%%%%%%%%%%%%%%%%%%%%%%%%%%%%%%%%%%%%%%%%%%%%%%%%%%%%%%%%%%%%%%%

\begin{problem}
\bparts
\ppart Show that every minimum spanning tree of a graph can be the result of
Algorithm~\ref{alg:MST1}.

\ppart Conclude \inbook{Corollary~\bref{cor:uniqMST}} that if all
edges in a weighted graph have distinct weights, then the graph has a
\emph{unique} MST.

\begin{solution}
  Algorithm~\ref{alg:MST1} is deterministic without ties, so the
  unique tree its produces must be all the trees there are.
\end{solution}

\begin{editingnotes} 
Another approach is by contradiction: assume $M \neq N$ are MST's.
Choose the smallest edge in one and not the other, say $e \in M-N$.
Then $N+e$ has a cycle with a larger edge $g$, so $N+e-g$ will be a
smaller weight MST than $N$ by the same reasoning as proof of
Lemma~\ref{lem:edgeextends}.
\end{editingnotes}

\end{problem}

%%%%%%%%%%%%%%%%%%%%%%%%%%%%%%%%%%%%%%%%%%%%%%%%%%%%%%%%%%%%%%%%%%%%%
% Problem ends here
%%%%%%%%%%%%%%%%%%%%%%%%%%%%%%%%%%%%%%%%%%%%%%%%%%%%%%%%%%%%%%%%%%%%%
