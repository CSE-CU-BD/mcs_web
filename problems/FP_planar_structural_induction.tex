\documentclass[problem]{mcs}

\begin{pcomments}
  \pcomment{FP_structural_induction}
  \pcomment{revised by ARM 11/2/11 from 10/12/09 \& 10/11/09}
  \pcomment{could be proved from v-e+f=2 w/o needing induction ---ARM 10/11/09}
  \pcomment{Might be interesting if shared faces are represented, then get
    v-e+f-c=1 ---ARM 10/11/09}
  \pcomment{from: Megumi F09}
\end{pcomments}

\pkeywords{
  planar embedding
  structural induction
  Euler_formula
  faces
}

%%%%%%%%%%%%%%%%%%%%%%%%%%%%%%%%%%%%%%%%%%%%%%%%%%%%%%%%%%%%%%%%%%%%%
% Problem starts here
%%%%%%%%%%%%%%%%%%%%%%%%%%%%%%%%%%%%%%%%%%%%%%%%%%%%%%%%%%%%%%%%%%%%%

\begin{problem}  %\textbf{Planar Embeddings}

Definition~\bref{def:embedding} of planar graph embeddings \iffalse
\inhandout{(repeated in the Appendix)}\fi applied only to connected
planar graphs.  The definition can be extended to planar graphs that
are not necessarily connected by adding the following additional
constructor case to the definition:

%\fbox{\
%\begin{minipage}[t]{6.5in}
%\vspace{.1in}
\begin{itemize}

\item \textbf{Constructor Case:} (collect disjoint graphs) Suppose
  $\embed{E}_1$ and $\embed{E}_2$ are planar embeddings with no vertices in common.
  Then $\embed{E}_1 \union \embed{E}_2$ is a planar embedding.
\end{itemize}
%\vspace{.1in}

%\end{minipage}}

%\examspace[0.1in]

Euler's Planar Graph Theorem now generalizes to unconnected graphs as
follows: if a planar embedding, $\embed{E}$, has $v$ vertices, $e$
edges, $f$ faces, and $c$ connected components, then
\begin{equation}\label{vef2c}
v-e+f-2c = 0.
\end{equation}
This can be proved by structural induction on the definition of planar
embedding.
\examspace

\bparts
\ppart[3] State and prove the base case of the structural induction.

\begin{solution}
\begin{proof}
The base case is for a single vertex, where the induction
hypothesis~\eqref{vef2c} holds because
\[
v-e+f-2c = 1-0+1-2\cdot1 = 0.
\]
\end{proof}
\end{solution}

\examspace[1.0in]

\ppart[3]\label{v=v1+v2efc} Let $v_i,e_i,f_i,$ and $c_i$ be the number of vertices, edges,
faces, and connected components in embedding $\embed{E}_i$ and let $v,
e, f, c$ be the numbers for the embedding from the (collect disjoint
graphs) constructor case.  Express $v,e,f,c$ in terms of
$v_i,e_i,f_i,c_i$.

\examspace[1.0in]

\begin{solution}
$v = v_1+v_2, e = e_1+e_2, f = f_1+f_2, c = c_1+c_2$
\end{solution}

\ppart[4] Prove the (collect disjoint graphs) case of the structural
induction.

\examspace[2.0in]
\begin{solution}
We may assume the structural induction hypothesis that
\begin{equation}\label{vi-ei}
v_i-e_i+f_i-2c_i=0
\end{equation}
for $i=1,2$.  Then 
\begin{align*}
v-e+f-2c  & = (v_1+v_2) -(e_1+e_2) +(f_1+f+2) -2(c_1+c_2)
                 & \text{(by part~\eqref{v=v1+v2efc})}\\
          &  = (v_1-e_1+f_1-2c_1) + (v_2-e_2+f_2-2c_2)\\
          &  = 0 + 0 = 0 & \text{(by~\eqref{vi-ei})}.
\end{align*}
This proves that the structural induction hypothesis~\eqref{vef2c} holds
for the (collect disjoint graphs) case.
\end{solution}

\eparts
\end{problem}


%%%%%%%%%%%%%%%%%%%%%%%%%%%%%%%%%%%%%%%%%%%%%%%%%%%%%%%%%%%%%%%%%%%%%
% Problem ends here
%%%%%%%%%%%%%%%%%%%%%%%%%%%%%%%%%%%%%%%%%%%%%%%%%%%%%%%%%%%%%%%%%%%%%

\endinput
