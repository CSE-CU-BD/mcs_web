\documentclass[problem]{mcs}

\begin{pcomments}
  \pcomment{FP_structural_induction}
  \pcomment{revised by ARM 10/12/10 from 10/11/09 vesion}
  \pcomment{could be proved from v-e+f=2 w/o needing induction --ARM 10/11/09}
  \pcomment{Might be interesting if shared faces are represented, then get
    v-e+f-c=1. --ARM 10/11/09}
  \pcomment{from: Megumi F09}
\end{pcomments}

\pkeywords{
  planar embedding
  structural induction
  Euler_formula
  faces
}

%%%%%%%%%%%%%%%%%%%%%%%%%%%%%%%%%%%%%%%%%%%%%%%%%%%%%%%%%%%%%%%%%%%%%
% Problem starts here
%%%%%%%%%%%%%%%%%%%%%%%%%%%%%%%%%%%%%%%%%%%%%%%%%%%%%%%%%%%%%%%%%%%%%

\begin{problem}  \textbf{Planar Embeddings}

The planar graph embeddings in class (repeated in the Appendix) were
only defined for connected, planar graphs.  The definition can be
extended to planar graphs that are not necessarily connected by adding
the following additional constructor case to the definition:

\fbox{\
\begin{minipage}[t]{6.5in}
\vspace{.1in}
\begin{itemize}

\item \textbf{Constructor Case:} (collect disjoint graphs) Suppose $G$
  and $H$ are disjoint graphs with planar embeddings.  Then thesn
  union of their embeddings is an embedding of the union graph, $U$,
  consisting of all the connected components of $G$ and
  $H$.\footnote{Formally, the set, $V_U$, of vertices of $U$ is simply
    $V_G \union V_H$, and the set, $E_U$, of edges of $U$ is $E_G
    \union E_H$.}

\end{itemize}
\vspace{.1in}

\end{minipage}}

%\examspace[0.1in]

Euler's Planar Graph Theorem now generalizes to unconnected graphs as
follows: if a planar embedding, $\mathcal{E}$, has $v$ vertices, $e$
edges, $f$ faces, and $c$ connected components, then
\begin{equation}\label{vef2c}
v-e+f-2c = 0.
\end{equation}
This can be proved by structural induction on the definition of planar
embedding.
%\inhandout{\instatements{\newpage}}

\bparts
\ppart State and prove the base case of the structural induction.

\begin{solution}
TBA
\end{solution}

\examspace[1in]

\ppart State precisely (but do not prove) what must be proved for each
of the constructor cases of the structural induction.

\examspace{2in}

\begin{solution}
\begin{proof}
TBA
%  Assuming $f,g$ are rational functions of $x$ for which $P(f)$ and $P(g)$
%  both hold, we must prove $P(h)$ where

%\textbf{Case $h= f + g$}:  In this case,
%\[
%h^{\prime} = f^{\prime} + g^{\prime},
%\]
%and since $f^{\prime}$ and $g^{\prime}$ are rational functions of $x$ by
%hypothesis, so is their sum by the constructor rules, which proves $P(h)$.

%\textbf{Case $h= f \cdot g$}:

%The Product Rule of derivatives states that:
%\begin{equation}\label{fgderiv}
%h^{\prime} =  f^{\prime} \cdot g + f \cdot g^{\prime},
%\end{equation}
%and since $f, f^{\prime}, g, g^{\prime}$ are rational functions of $x$ by
%hypothesis, so is the right hand side of~\eqref{fgderiv} by the
%constructor rules, which proves $P(h)$.

%\textbf{Case $h= \dfrac{1}{f}$}:

%The Chain Rule gives: 
%\begin{equation}\label{1/fderiv}
%h^{\prime} = \frac{-1}{f^2} \cdot f^{\prime},
%\end{equation}
%and since $f$ and $f^{\prime}$ are rational by hypothesis, so is the right
%hand side of~\eqref{1/fderiv} by the constructor rules, which proves
%$P(h)$.

%We have shown that the induction hypothesis holds in all Constructor cases.
%This completes the proof by structural induction.
\end{proof}
\end{solution}

\ppart Prove the contructor case (collect disjoint graphs) of the
structural induction.

\examspace{3in}
\begin{solution}


\end{solution}
\eparts
\end{problem}


%%%%%%%%%%%%%%%%%%%%%%%%%%%%%%%%%%%%%%%%%%%%%%%%%%%%%%%%%%%%%%%%%%%%%
% Problem ends here
%%%%%%%%%%%%%%%%%%%%%%%%%%%%%%%%%%%%%%%%%%%%%%%%%%%%%%%%%%%%%%%%%%%%%

\endinput
