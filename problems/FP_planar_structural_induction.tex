\documentclass[problem]{mcs}

\begin{pcomments}
  \pcomment{FP_structural_induction}
  \pcomment{revised by ARM 10/12/09 from 10/11/09 version}
  \pcomment{could be proved from v-e+f=2 w/o needing induction---ARM 10/11/09}
  \pcomment{Might be interesting if shared faces are represented, then get
    v-e+f-c=1.---ARM 10/11/09}
  \pcomment{from: Megumi F09}
\end{pcomments}

\pkeywords{
  planar embedding
  structural induction
  Euler_formula
  faces
}

%%%%%%%%%%%%%%%%%%%%%%%%%%%%%%%%%%%%%%%%%%%%%%%%%%%%%%%%%%%%%%%%%%%%%
% Problem starts here
%%%%%%%%%%%%%%%%%%%%%%%%%%%%%%%%%%%%%%%%%%%%%%%%%%%%%%%%%%%%%%%%%%%%%

\begin{problem}  %\textbf{Planar Embeddings}

The planar graph embeddings \inhandout{(repeated in the Appendix)} were
only defined for connected planar graphs.  The definition can be
extended to planar graphs that are not necessarily connected by adding
the following additional constructor case to the definition:

%\fbox{\
%\begin{minipage}[t]{6.5in}
%\vspace{.1in}
\begin{itemize}

\item \textbf{Constructor Case:} (collect disjoint graphs) Suppose
  $\mathcal{E}$ and $\mathcal{F}$ are planar embeddings with no vertices in common.
  Then $\mathcal{E} \union \mathcal{F}$ is a planar embedding.
\end{itemize}
%\vspace{.1in}

%\end{minipage}}

%\examspace[0.1in]

Euler's Planar Graph Theorem now generalizes to unconnected graphs as
follows: if a planar embedding, $\mathcal{E}$, has $v$ vertices, $e$
edges, $f$ faces, and $c$ connected components, then
\begin{equation}\label{vef2c}
v-e+f-2c = 0.
\end{equation}
This can be proved by structural induction on the definition of planar
embedding.
%\inhandout{\instatements{\newpage}}

\bparts
\ppart[4] State and prove the base case of the structural induction.

\begin{solution}
TBA
\end{solution}

\examspace[1.5in]
\iffalse

\ppart State precisely (but do not prove) what must be proved for each
of the constructor cases of the structural induction.

\examspace[2in]

\begin{solution}

\end{solution}
\fi

\ppart[2] Carefully state what must be proved in the new constructor
case (collect disjoint graphs) of the structural induction.

\examspace[1in]
\begin{solution}


\end{solution}


\ppart[4] Prove the (collect disjoint graphs) case of the structural
induction.

\examspace[2.5in]
\begin{solution}

\end{solution}

\eparts
\end{problem}


%%%%%%%%%%%%%%%%%%%%%%%%%%%%%%%%%%%%%%%%%%%%%%%%%%%%%%%%%%%%%%%%%%%%%
% Problem ends here
%%%%%%%%%%%%%%%%%%%%%%%%%%%%%%%%%%%%%%%%%%%%%%%%%%%%%%%%%%%%%%%%%%%%%

\endinput
