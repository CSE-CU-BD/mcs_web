\documentclass[problem]{mcs}

\begin{pcomments}
  \pcomment{CP_express_goldbach_twinprime_in_logic}
  \pcomment{ARM 9/18/17}
  \pcomment{subsumes PS_rewrite_assertions_prime_goldbach_bertrand_fermat}
  \pcomment{subsumes PS_translate_to_predicate_logic}
\end{pcomments}

\pkeywords{
  predicate
  quantifier
  Goldbach
  Bertrand
  Fermat
}

%%%%%%%%%%%%%%%%%%%%%%%%%%%%%%%%%%%%%%%%%%%%%%%%%%%%%%%%%%%%%%%%%%%%%
% Problem starts here
%%%%%%%%%%%%%%%%%%%%%%%%%%%%%%%%%%%%%%%%%%%%%%%%%%%%%%%%%%%%%%%%%%%%%

\begin{problem}
In this problem we'll examine predicate logic formulas where the domain
of discourse is $\nngint$.  In addition to the logical symbols, the
formulas may contain ternary predicate symbols $A$ and $M$, where
\begin{align*}
A(k,m,n) &\ \text{means}\ k = m + n,\\
M(k,m,n) &\ \text{means}\ k = m \cdot n.
\end{align*}
For example, a formula ``$\text{Zero}(n)$'' meaning that $n$ is zero
could be defined as
\[
\text{Zero}(n) \eqdef A(n,n,n).
\]
Having defined ``Zero,'' it is now OK to use it in subsequent
formulas.  So a formula ``$\text{Greater}(m,n)$'' meaning $m > n$
could be defined as
\[
\text{Greater}(m,n) \eqdef \exists k. \QNOT(\text{Zero}(k)) \QAND A(m,n,k).
\]
This makes it OK to use ``Greater'' in subsequent formulas.

Write predicate logic formulas using only the allowed predicates $A,M$ that
define the following predicates:

\begin{problemparts}

\ppart $\text{Equal}(m,n)$ meaning that $m=n$.
\begin{solution}
You could say that neither $m$ nor $n$ is greater than the other:
\[
\text{Equal}(m,n) \eqdef \QNOT(\text{Greater}(m,n) \QOR \text{Greater}(n,m)).
\]
It's even simpler to say that $m = n+0$:
\[
\text{Equal}(m,n) \eqdef \exists z. \text{Zero}(z) \QAND A(m,n,z).
\]
\end{solution}

\ppart\label{defone} $\text{One}(n)$ meaning that $n=1$.

\begin{solution}
\[
\text{One}(n) \eqdef M(n,n,n) \QAND \QNOT(\text{Zero}(n))
\]
clearly works.  So does
\[
\text{One}(n) \eqdef \forall m.\, M(m,m,n).
\]
Another approach uses Greater, namely, one is the number that is only
greater than zero:
\[
\text{One}(n) \eqdef \forall m.\,  \text{Greater}(n,m) \QIFF \text{Zero}(m).
\]
\end{solution}

\ppart\label{avoidterms} $n = i(m \cdot j + k^2)$

\begin{solution}
The idea is to introduce variables for the values of the arithmetic subformulas:
\[
\exists a, b, c.\,  M(a,m,j) \QAND M(b,k,k) \QAND A(c,a,b) \QAND M(n,i,c).
\]

\end{solution}

\problempart $\text{Prime}(p)$ meaning $p$ is a prime number.

\begin{solution}
One way to define a prime number is:
\begin{quote}
``A prime is an integer greater than one that cannot be expressed as the
  product of two numbers that are also greater than one.''
\end{quote}
This definition then translates directly into the predicate formula:
\begin{align*}
\text{Prime}(p)
  \eqdef & \exists w. \text{One}(w) \QAND \text{Greater}(p,w)\ \QAND\\
         & \qquad \QNOT(\exists m.\, \exists n.\, 
          \text{Greater}(m,w) \QAND \text{Greater}(n,w) \QAND M(p,m,n)).
\end{align*}

Another, equivalent, definition is:
\begin{quote}
``A prime is an integer other than one whose only factors are itself and one.''
\end{quote}
This would translate into
\begin{align*}
\text{Prime}(p) & \eqdef  \QNOT(\text{Zero}(p))\ \QAND\\
      & \qquad \forall m, n.\, 
 M(p,m,n) \QIMPLIES [\text{Equal}(m,p) \QOR \text{One}(m)].
\end{align*}

A clever third characterization of primes leads to a more concise formula:
\begin{quote}
``A prime is an integer that can only be factored into two
  numbers when exactly one of those numbers equals one.''
\end{quote}
\[
\text{Prime}(p)
    \eqdef \forall m, n.\,  M(p,m,n)
       \QIMPLIES (\text{One}(m) \QXOR \text{One}(n)).
\]
\end{solution}

\ppart\label{deftwo} $\text{Two}(n)$ meaning that $n=2$.

\begin{solution}
The usual definition would be that two is one plus one:
\[
\text{Two}(n) \eqdef \exists w. \text{One}(w) \QAND A(n, w, w).
\]

Another definition is that two is an integer that is only greater than
either zero or one:

\[
\text{Two}(n) \eqdef \forall m.\, \text{Greater}(n,m) \QIMP (\text{Zero}(m) \QOR \text{One}(m)).
\]
\end{solution}

\eparts

The results of part~\eqref{deftwo} will entend to formulas
$\text{Three}(n), \text{Four}(n), \text{Five}(n), \dots$ which are
allowed from now on.

\bparts

\iffalse
\problempart There is no largest prime number.

\begin{solution}
\[
\QNOT(\exists p.\, (\text{Prime}(p) \QAND (\forall q.\, (\text{Prime}(q) \QIMPLIES p \geq q)))).
\]
\end{solution}
\fi

\ppart $\text{Even}(n)$ meaning $n$ is even.

\begin{solution}
A standard definition would be that an even number is a multiple of two:
\[
\text{Even}(n)\eqdef \exists t.\, (\text{Two}(t) \QAND  \exists k.\, M(n,t,k)).
\]
A simpler alternative is that an even number is the sum of some number with itself:
\[
\text{Even}(n)\eqdef \exists k.\, A(n,k,k).
\]
\end{solution}

\problempart (\idx{Goldbach Conjecture}) Every even
integer $n \geq 4$ can be expressed as the sum of two primes.

\begin{solution}
\begin{align*}
\exists r. & \text{Three}(r)\ \QAND\\
           & \forall n.\, (\text{Greater}(n,r) \QAND \text{Even}(n)) \QIMPLIES\\
           & \qquad \exists p, q.\,
                 \text{Prime}(p) \QAND \text{Prime}(q) \QAND A(n, p, q).
\end{align*}
\end{solution}

\iffalse
\problempart (\idx{Bertrand's Postulate}) If $n > 1$, then there is always
at least one prime $p$ such that $n < p < 2n$.

\begin{solution}
\[
\forall n.\,
\( (n > 1) \QIMPLIES (\exists p ( \text{Prime}(p)  \QAND (n < p) \QAND (p < 2n))) 
)
\]
\end{solution}
\fi

\inbook{
\problempart (\idx{Fermat's Last Theorem}) Now suppose we also have
\[
X(k,m,n) \quad \text{means}\ k = m^n.
\]
Express the assertion that there are no positive integer solutions to
the equation:
\begin{equation*}
    x^n + y^n = z^n
\end{equation*}
when $n > 2$.

\begin{solution}
We express $x^n + y^n = z^n$ with
\[
F(x,y,z,n) \eqdef \exists a,b,c.\, X(a,x,n) \QAND X(b,y,n) \QAND A(c,a,b) \QAND X(c,z,n).
\]
Then a formula for Fermat's Last Theorem is:
\begin{align*}
\forall x, y, z, n.\, & F(x,y,x,n) \QIMPLIES
    &  [\text{Zero}(x) \QOR \text{Zero}(y) \QOR \text{Zero}(z) \QOR
   \text{Zero}(n) \QOR \text{One}(n) \QOR \text{Two}(n)].
\end{align*}
\end{solution}
}

\ppart (Twin Prime Conjecture) There are infinitely many primes that differ by two.

\begin{solution}
\begin{align*}
\forall n \exists p \exists q.
       & \text{Prime}(p) \QAND \text{Prime}(q) \QAND\\
       & \text{Greater}(p,n) \QAND \exists t.\, (\text{Two}(t) \QAND A(q,p,t)).
\end{align*}
\end{solution}

\eparts

\end{problem}

%%%%%%%%%%%%%%%%%%%%%%%%%%%%%%%%%%%%%%%%%%%%%%%%%%%%%%%%%%%%%%%%%%%%%
% Problem ends here
%%%%%%%%%%%%%%%%%%%%%%%%%%%%%%%%%%%%%%%%%%%%%%%%%%%%%%%%%%%%%%%%%%%%%
