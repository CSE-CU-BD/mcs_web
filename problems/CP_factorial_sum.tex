\documentclass[problem]{mcs}

\newcommand{\mnf}[1]{\mopt{mnp}{(#1)}}

\begin{pcomments}
  \pcomment{CP_factorial_sum}
  \pcomment{by ARM 10/3/12}
\end{pcomments}

\pkeywords{
  combinatorial_proof
  permutation
  factorial
}

%%%%%%%%%%%%%%%%%%%%%%%%%%%%%%%%%%%%%%%%%%%%%%%%%%%%%%%%%%%%%%%%%%%%%
% Problem starts here
%%%%%%%%%%%%%%%%%%%%%%%%%%%%%%%%%%%%%%%%%%%%%%%%%%%%%%%%%%%%%%%%%%%%%

\begin{problem}
When an integer $k$ occurs as the $k$th element of a sequence, we'll
say it is ``in place'' in the sequence.  For example, in the sequence
\[
12453678
\]
precisely the integers $1, 2, 6, 7$ and $8$ occur in place.  We're
going to classify the sequences of distinct integers from 1 to $n$,
that is the permutations of $\Zintv{1}{n}$, according to which integers do
not occur ``in place.''  Then we'll use this classification to prove
the combinatorial identity\footnote{This problem is based on ``Use of
  everywhere divergent generating function,''
  \texttt{math}\textbf{overflow}, response 8,147 by Aaron Meyerowitz,
  Nov. 12, 2010.}
\begin{equation}\label{n!sum}
n! = 1 + \sum_{k=1}^{n} (k-1) \cdot (k-1)!\, .
\end{equation}

If $\pi$ is a permutation of $\Zintv{1}{n}$, let $\mnf{\pi}$ be the
\emph{maximum} integer in $\Zintv{1}{n}$ that does not occur in place in
$\pi$.  For example, for $n=8$,
\begin{align}
\mnf{12345687}=8, \notag\\
\mnf{21345678}=2, \notag\\
\mnf{23145678}=3. \notag
\end{align}

\bparts

\ppart For how many permutations of $\Zintv{1}{n}$ is every element in place?
\begin{solution}
Just one, namely the identity permutation $123\dots n$.
\end{solution}

\ppart How many permutations $\pi$ of $\Zintv{1}{n}$ have $\mnf{\pi} = 1$?
\begin{solution}
None.  If 1 is not in place, then it is in some position $k \neq 1$,
and $k$ will be a larger number that is also not in place.
\end{solution}

\ppart\label{mnfpi=k} How many permutations of $\Zintv{1}{n}$ have $\mnf{\pi}
= k$?

\begin{solution}
\[
(k-1)\cdot (k-1)!\, .
\]

The numbers greater than $k$ must all be in place.  One of the remaining
integers 1 to $k-1$, must appear in the $k$th position, and the
remaining $k-2$ integers along with $k$ may appear in any order in
positions 1 to $k-1$.
\end{solution}

\ppart Conclude the equation~\eqref{n!sum}.

\begin{solution}
There are $n!$ permutations of $\Zintv{1}{n}$.  One of these is the identity
permutation, and the others can be divided into dijoint classes
according to their maximum not-in-place element $k \in \Zintv{1}{n}$.  The
number of elements in the $k$th class is given by
part~\eqref{mnfpi=k}, so there are
\[
\sum_{k=1}^{n} (k-1) \cdot (k-1)!
\]
non-identity permutations, and equation~\eqref{n!sum} now follows immediately.

\end{solution}
\eparts
\end{problem}


%%%%%%%%%%%%%%%%%%%%%%%%%%%%%%%%%%%%%%%%%%%%%%%%%%%%%%%%%%%%%%%%%%%%%
% Problem ends here
%%%%%%%%%%%%%%%%%%%%%%%%%%%%%%%%%%%%%%%%%%%%%%%%%%%%%%%%%%%%%%%%%%%%%
\endinput
