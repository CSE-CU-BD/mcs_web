\documentclass[problem]{mcs}

\begin{pcomments}
    \pcomment{TP_Polynomial_for_the_Primes}
    \pcomment{Converted from 41-polynomial.scm by scmtotex and dmj
              on Sat 12 Jun 2010 09:47:19 PM EDT}
    \pcomment{revised by ARM 1/2/11}
\end{pcomments}

\begin{problem}

%% type: short-answer
%% title: Polynomial for the Primes

Let $p(n) = n^2 + n + 41$.   
\begin{problemparts}

\ppart\label{prle41} What are the primes $\le 41$?
\begin{solution}
A composite number $\le 48$ would have to be divisible by 2, 3, or 5
(why?).  So
\[
2, 3, 5, 7, 11, 13, 17, 19, 23, 29, 31, 37, 41
\]
are primes since none is divisible by 2, 3 or 5.  These are the only
primes $\le 41$, since it's easy to check that all the other integers
between 2 and 41 are divisible by 2, 3, or 5.

\end{solution}

\ppart What are the factors of $p(41)$?

\begin{solution}
$p(41) = 41^1 + 41 + 41 = 41 \cdot (41 + 1 + 1) = 41 \cdot 43$.  But
43 is not divisible by 2, 3, or 5, and so is also prime.
\end{solution}

\ppart Verify that $p(39)$ is prime.

\begin{solution}
  $p(39) = 39^2 + 39 + 41 = 1601$.  Now if 1601 was a product of two or
  more primes, at least one prime factor would have to be $\le
  \ceil{\sqrt{1601}} = 41$ (why?).  But it is easy to check that none of
  the primes from part~\eqref{prle41} is a divisor of 1601.
\end{solution}


\end{problemparts}

\iffalse
What is the smallest nonnegative integer $n$ such that $p(n)$ is not
prime?

\begin{solution}
40
\end{solution}
\fi

\end{problem}

\endinput
