\documentclass[problem]{mcs}

\begin{pcomments}
\pcomment{FP_prove_strong_induction}
\pcomment{revised ARM 5/16/15}
\end{pcomments}

\pkeywords{
  strong_induction
  induction
  proof
  rule
  anteceedent
}

%%%%%%%%%%%%%%%%%%%%%%%%%%%%%%%%%%%%%%%%%%%%%%%%%%%%%%%%%%%%%%%%%%%%%
% Problem starts here
%%%%%%%%%%%%%%%%%%%%%%%%%%%%%%%%%%%%%%%%%%%%%%%%%%%%%%%%%%%%%%%%%%%%%

\begin{problem}
The purpose of this question is to use ordinary induction to prove the
validity of the \emph{Strong Induction Rule}:
\begin{rul*} 
\Rule{P(0), \quad \forall n. \paren{\forall k \leq n.\, P(k)} \QIMPLIES P(n+1)}
{\forall m.\, P(m)}
\end{rul*}
where all variables range over the nonnegative integers $\nngint$.

\hint Let
\[
Q(n) \eqdef \forall k \leq n.\, P(k).
\]
Assuming the antecedents (assertions above the line) of the Strong
Induction Rule, prove by ordinary induction that $\forall m.\, Q(m)$.

\examspace[8in]

\begin{solution}

\begin{proof}
Suppose we are given the antecedents
\begin{itemize}
\item $P(0)$ and

\item
\begin{equation}\label{fknpkq}
Q(n) \QIMPLIES P(n+1)/
\end{equation}
\end{itemize}
From this, we want to conclude that $\forall m.\, P(m)$.  Noticing
that by definition, $Q(m)$ implies $P(m)$, we need only prove that
\begin{equation}\label{fmqm}
\forall m.\, Q(m).
\end{equation}
Now we will prove~\eqref{fmqm} by ordinary induction, using $Q(n)$ as
the induction hypothesis.

\inductioncase{Base Case} ($n=0$): We want to prove $Q(0)$.  But
saying that something holds for all $k \leq 0$ is the same as saying
it holds for 0.  So $Q(0)$ is equivalent to $P(0)$, and since we are
given the antecedent $P(0)$, we conclude that $Q(0)$ is true.

\inductioncase{Inductive Step}:
We assume $Q(n)$ and then must prove $Q(n+1)$.

Now we are given the antecedent $Q(n) \QIMPLIES P(n+1)$, so the
assumption $Q(n)$ implies that $P(n+1)$ is true.  That is, we have
that $Q(n) \QAND P(n+1)$ is true.  But $Q(n+1)$ by definition is
equivalent to $Q(n) \QAND P(n+1)$, and we conclude that $Q(n+1)$ is
true, which completes the inductive step.

So we conclude by ordinary induction that $Q(m)$ holds for all $m$, as required.
\end{proof}

\end{solution}

\end{problem}

%%%%%%%%%%%%%%%%%%%%%%%%%%%%%%%%%%%%%%%%%%%%%%%%%%%%%%%%%%%%%%%%%%%%%
% Problem ends here
%%%%%%%%%%%%%%%%%%%%%%%%%%%%%%%%%%%%%%%%%%%%%%%%%%%%%%%%%%%%%%%%%%%%%

\endinput
 
