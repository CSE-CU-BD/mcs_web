\documentclass[problem]{mcs}

\begin{pcomments}
  \pcomment{FP_relation_properties_expressions_final_f13}
  \pcomment{scramble of TP_relation_properties_expressions}
  \pcomment{10/12/13 ARM}
\end{pcomments}

\pkeywords{
  irreflexive
  transitive
  reflexive
  antisymmetric
  asymmetric
  identity_relation
}

%%%%%%%%%%%%%%%%%%%%%%%%%%%%%%%%%%%%%%%%%%%%%%%%%%%%%%%%%%%%%%%%%%%%%
% Problem starts here
%%%%%%%%%%%%%%%%%%%%%%%%%%%%%%%%%%%%%%%%%%%%%%%%%%%%%%%%%%%%%%%%%%%%%

\begin{problem}
Let $R$ be a binary relation on a set $D$.  Each of the following
predicate formulas expresses the fact that $R$ has a familiar
relational property such as reflexivity, asymmetry, transitivity.  The
relational equalities and containments listed after these formula are
equivalent to one or more of the predicate formulas.  Write the
number(s) of the equivalent predicate formulas next to each of the
relational equalities and containments.  You can get paet credit for
writing the name of the relational property expressed.  For example,
part~\eqref{irrpart} expresses \emph{irreflexivity}.

\begin{romanlist}
\item\label{reflexform} $\forall d.\quad d\mrel{R}d$
\item\label{irreflexform} $\forall d.\quad \QNOT(d\mrel{R}d)$
\item\label{symmiffform} $\forall c,d.\quad c\mrel{R}d \QIFF d\mrel{R}c$
\item\label{symmimpform} $\forall c,d.\quad c\mrel{R}d \QIMPLIES d\mrel{R}c$
\item\label{asymmform} $\forall c, d.\quad c\mrel{R}d \QIMPLIES \QNOT(d\mrel{R}c)$
\item\label{antisymmform} $\forall c \neq d.\quad c\mrel{R}d \QIMPLIES \QNOT(d\mrel{R}c)$
\item\label{tournamentform} $\forall c \neq d.\quad c\mrel{R}d \QIFF \QNOT(d\mrel{R}c)$
\item\label{transform} $\forall b,c,d.\quad (b\mrel{R}c \QAND c\mrel{R}d) \QIMPLIES b\mrel{R}d$
\item\label{transexform} $\forall b,d.\quad [\exists c.\ (b\mrel{R}c \QAND c\mrel{R}d)] \QIMPLIES b\mrel{R}d$
\item\label{denseform} $\forall b,d.\quad  b\mrel{R}d\ \QIMPLIES [\exists c.\ (b\mrel{R}c \QAND c\mrel{R}d)]$
\end{romanlist}

\bparts

\ppart\label{irrpart} $R \intersect \ident{D} =  \emptyset$ \hfill \examrule[0.5in]
\begin{solution}
\eqref{irreflexform}: \textbf{irreflexive}
\end{solution}

\ppart $R \subseteq \inv{R}$ \hfill \examrule[0.5in]
\begin{solution}
\eqref{symmimpform}, \eqref{symmiffform}: \textbf{symmetric}
\end{solution}

\ppart $R = \inv{R}$ \hfill \examrule[0.5in]
\begin{solution}
\eqref{symmimpform}, \eqref{symmiffform}: \textbf{symmetric}
\end{solution}

\ppart $\ident{D} \subseteq R$ \hfill \examrule[0.5in]
\begin{solution}
\eqref{reflexform}: \textbf{reflexive}
\end{solution}

\ppart $R \compose R \subseteq R$ \hfill \examrule[0.5in]
\begin{solution}
\eqref{transform}, \eqref{transexform}: \textbf{transitive}
\end{solution}

\ppart $R \subseteq R \compose R$ \hfill \examrule[0.5in]
\begin{solution}
\eqref{denseform}: \textbf{dense}
\end{solution}

\ppart $R \intersect \inv{R} \subseteq \ident{D}$ \hfill \examrule[0.5in]
\begin{solution}
\eqref{antisymmform}: \textbf{antisymmetric}
\end{solution}

\ppart $\bar{R} \subseteq \inv{R} $ \hfill \examrule[0.5in]
\begin{solution}
\eqref{asymmform}: \textbf{asymmetric}
\end{solution}

\ppart $\bar{R} \intersect \ident{R} = \inv{R} \intersect \ident{R}$ \hfill \examrule[0.5in]
\begin{solution}
\eqref{tournamentform}: \textbf{tournament graph}
\end{solution}

\ppart $R \intersect \inv{R} = \emptyset$ \hfill \examrule[0.5in]
\begin{solution}
\eqref{asymmform}: \textbf{asymmetric}
\end{solution}


\eparts
\end{problem}
%%%%%%%%%%%%%%%%%%%%%%%%%%%%%%%%%%%%%%%%%%%%%%%%%%%%%%%%%%%%%%%%%%%%%
% Problem ends here
%%%%%%%%%%%%%%%%%%%%%%%%%%%%%%%%%%%%%%%%%%%%%%%%%%%%%%%%%%%%%%%%%%%%%

\endinput

