\documentclass[problem]{mcs}

\begin{pcomments}
  \pcomment{MQ_task_parallel_scheduling_v3}
  \pcomment{written for Spring 2013 Miniquiz 5 (morning)}
  \pcomment{F15.final}
\end{pcomments}

\pkeywords{
  DAG
  scheduling
  chains_and_antichains
}

%%%%%%%%%%%%%%%%%%%%%%%%%%%%%%%%%%%%%%%%%%%%%%%%%%%%%%%%%%%%%%%%%%%%%
% Problem starts here
%%%%%%%%%%%%%%%%%%%%%%%%%%%%%%%%%%%%%%%%%%%%%%%%%%%%%%%%%%%%%%%%%%%%%

\begin{problem}
The following DAG describes the prerequisites among tasks $\set{A, \dots, H}$.

\begin{figure}[h]
\graphic[height=1.5in]{miniquiz5-p2}
\end{figure}

\bparts

\iffalse

\ppart
What is the largest chain in this DAG?
\begin{solution}
$A, D, E, G$
\end{solution}
\examspace[2cm]


\ppart
What are the two maximum size antichains?
\begin{solution}
$A, B, C$ or $D, B, C$
\end{solution}
\examspace[2cm]

\ppart If each task takes 1 minute to complete, what is the
minimum amount of time needed to complete all the tasks with a single
processor?
\begin{solution}
8 minutes
\end{solution}
\examspace[3cm]
\fi

\ppart If each task takes unit time to complete, what is the minimum
parallel time to complete all the tasks?

\begin{center}
\exambox{0.7in}{0.4in}{0in}
\end{center}

\begin{solution}
\textbf{4}.  This is the size of the maximum chain.
\end{solution}

%\examspace[2cm]

\ppart What is the minimum parallel time if no more than two tasks can
be completed in parallel?

\begin{solution}
\textbf{4}, still: schedule $A,B$ then $C,D$ then $E,F$ then $G,H$.
\end{solution}

\begin{center}
\exambox{0.7in}{0.4in}{0in}
\end{center}

%\examspace[2cm]

\eparts

\end{problem}

%%%%%%%%%%%%%%%%%%%%%%%%%%%%%%%%%%%%%%%%%%%%%%%%%%%%%%%%%%%%%%%%%%%%%
% Problem ends here
%%%%%%%%%%%%%%%%%%%%%%%%%%%%%%%%%%%%%%%%%%%%%%%%%%%%%%%%%%%%%%%%%%%%%

\endinput
