\documentclass[problem]{mcs}

\begin{pcomments}
  \pcomment{PS_RSA_collision_probability_200}
  \pcomment{adapted from MQ_RSA_collision_probability_200 in Fall 2013}
\end{pcomments}

\pkeywords{
  RSA
  gcd
  modulus
  factor
  prime
}

%%%%%%%%%%%%%%%%%%%%%%%%%%%%%%%%%%%%%%%%%%%%%%%%%%%%%%%%%%%%%%%%%%%%%
% Problem starts here
%%%%%%%%%%%%%%%%%%%%%%%%%%%%%%%%%%%%%%%%%%%%%%%%%%%%%%%%%%%%%%%%%%%%%

\begin{problem}
Suppose the RSA modulus $n=pq$ is the product of distinct $200$ digit
primes $p$ and $q$.  A message $m\in \Zintvco{0}{n}$ is called
\emph{dangerous} if $\gcd(m,n)=p$ or $\gcd(m,n)=q$, because such an
$m$ can be used to factor $n$ and so crack RSA.

Estimate the fraction of messages in $\Zintvco{0}{n}$ that are
dangerous to the nearest order of magnitude.

\begin{solution}
Either $10^{-200}$ or $10^{-199}$ is acceptable.

For $m \in \Zintvco{0}{pq}$,
\[
\gcd(m,pq)=p \qiff p \divides m.
\]

The fraction of numbers divisible by $p$ in an interval is $1/p$, and
since $10^{199} \le p < 10^{200}$, $1/p$ is on the order of
$1/10^{200}$. The same logic holds for $q$.

Because $n = pq$, none of the messages $m\in \Zintvco{0}{n}$ can be
divisible by both $p$ and $q$, so the total fraction of dangerous
messages is approximately $1/p + 1/q$, which can be on the order of
$10^{-200}$ or $10^{-199}$ depending on $p$ and $q$.

\end{solution}

\end{problem}

%%%%%%%%%%%%%%%%%%%%%%%%%%%%%%%%%%%%%%%%%%%%%%%%%%%%%%%%%%%%%%%%%%%%%
% Problem ends here
%%%%%%%%%%%%%%%%%%%%%%%%%%%%%%%%%%%%%%%%%%%%%%%%%%%%%%%%%%%%%%%%%%%%%

\endinput
