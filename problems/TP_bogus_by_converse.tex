\documentclass[problem]{mcs}

\begin{pcomments}
  \pcomment{TP_bogus_by_converse}
  \pcomment{ARM 02/20/17 based on bibitem Beam2017}
\end{pcomments}

\pkeywords{
  bogus_proof
  converse
  valid
  quantifier
}

%%%%%%%%%%%%%%%%%%%%%%%%%%%%%%%%%%%%%%%%%%%%%%%%%%%%%%%%%%%%%%%%%%%%%
% Problem starts here
%%%%%%%%%%%%%%%%%%%%%%%%%%%%%%%%%%%%%%%%%%%%%%%%%%%%%%%%%%%%%%%%%%%%%

\begin{problem}

\bparts

\ppart\label{PQQP} Verify that the propositional formula
\[
(P \QIMP Q) \QOR (Q \QIMP P)
\]
is valid.

\begin{solution}
By truth table, or immediately by cases: $P$ is \true\ or \false\ (by
symmetry, cases on $Q$ work the same).
\end{solution}

\ppart The valid formula of part~\eqref{PQQP} leads to sound proof
method: to prove that an implication is true, just prove that its
converse is false.\footnote{This problem was stimulated by the
  discussion of the fallacy in~\cite{Beam2017}.}  For example,
from elementary calculus we know that the assertion
\begin{quote}
  If a function is continuous, then it is differentiable
\end{quote}
is false.  This allows us to reach at the correct conclusion that its
converse,
\begin{quote}
  If a function is differentiable, then it is continuous
\end{quote}
is true, as indeed it is.

But wait a minute!  The implication
\begin{quote}
  If a function is differentiable, then it is not continuous
\end{quote}
is completely false.  So we could conclude that its converse 
\begin{quote}
  If a function is not continuous, then it is differentiable,
\end{quote}
should be true, but in fact the converse is also completely false.

So something has gone wrong here.  Explain what.
\begin{staffnotes}
The calculus properties in the above example may be a distraction
here, so the following much simpler example may help crystallize the
issue.

The implication
\begin{quote}
If a color is black, then it is white,
\end{quote}
is patently false.  But its converse
\begin{quote}
If a color is white, then it is black,
\end{quote}
is patently false as well.
\end{staffnotes}

\begin{solution}
The mistake is in trying to treat the hypothesis and conclusions of
the implications as though they were propositions with definite truth
values.  But the hypothesis refers to``a function,'' and its truth
depends on what the function is.  For example, if we replace
``function'' by ``step function,'' we get
\begin{quote}
If a step function is continuous, then it is differentiable.
\end{quote}
Now this special case of the so-called patently false implication is
actually true, because its hypothesis is false: step functions are not
continuous.

Another way to understand the fallacy is to recognize that the
assertions above about continuous functions may look like
implications, but they are really universally quantified predicate
formulas.  The first assertion would translate into the predicate
formula:
\[
\forall\, \text{functions}\, f.\ [f\ \text{is continuous}\ \QIMP
  f\ \text{is differentiable}].
\]

\begin{staffnotes}
The black-white example would translate to:
\[
\forall\, \text{colors}\, c.\ [c\ \text{is black}\ \QIMP
  c\ \text{is white}].
\]
\end{staffnotes}
The rule from part~\eqref{PQQP} applies to propositions with definite
truth values and can't be applied inside the $\forall$ to predicates
whose truth value depends on the unknown value of the variable $f$.
\end{solution}

\eparts
\end{problem}

%%%%%%%%%%%%%%%%%%%%%%%%%%%%%%%%%%%%%%%%%%%%%%%%%%%%%%%%%%%%%%%%%%%%%
% Problem ends here
%%%%%%%%%%%%%%%%%%%%%%%%%%%%%%%%%%%%%%%%%%%%%%%%%%%%%%%%%%%%%%%%%%%%%

\endinput


