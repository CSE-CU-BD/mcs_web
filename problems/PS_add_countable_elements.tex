\documentclass[problem]{mcs}

\begin{pcomments}
  \pcomment{PS_add_countable_elements}
  \pcomment{add by ARM 4/5/11, solns by ARM, Adamc}
  \pcomment{simplified and generalized by ARM 3/15/17}
\end{pcomments}

\pkeywords{
  countable
  bijection
  union
}

%%%%%%%%%%%%%%%%%%%%%%%%%%%%%%%%%%%%%%%%%%%%%%%%%%%%%%%%%%%%%%%%%%%%%
% Problem starts here
%%%%%%%%%%%%%%%%%%%%%%%%%%%%%%%%%%%%%%%%%%%%%%%%%%%%%%%%%%%%%%%%%%%%%

\begin{problem}
Prove that if $A$ is an infinite set and $B$ is a countable, then
\[
A \bij (A \union B).
\]

\hint Since $A$ is infinite, we can find an infinite sequence $a_0,
a_1, a_2, \dots$ of distinct elements of $A$ as \inhandout{shown in
  the text}\inbook{in the proof of Lemma~\bref{AUb}}.

\begin{solution}
The sequence $a_0, a_1, a_2, \dots$ may include all the elements of
$A$---possible when $A$ is countable---or not.

Since $B$ is countable, it may likewise be enumerated as a sequence
of distinct elements $b_0, b_1, b_2, \dots$.  If $B$ is finite, the
sequence will end after $\card{B}$ elements.  (If $\card{B} = 0$, then
the proof is trivial.)

If $B$ is infinite, we can define a bijection $g:A \to (A \union B)$ as follows:
\[
  g(a) \eqdef
 \begin{cases}
   a_i  & \text{if } a = a_{2i}\\
   b_i  & \text{if } a = a_{2i+1} \QAND\ b_i \notin A,\\
     a  & \text{otherwise}.
 \end{cases}
\]

If $B$ is finite, we can define a bijection $f:A \to (A \union B)$ as follows:
\[
  f(a) \eqdef \begin{cases}
    b_i           & \text{if } a = a_i, 0 \leq i < \card{B} \text{ and } a_i \notin B,\\
    a_{i - \card{B}}  & \text{if } a = a_i \text{ and } i \geq \card{B},\\
    a             & \text{otherwise}.
  \end{cases}
\]

The functions $f$ and $g$ are by definition total functions from $A$
to $A \union B$ and they are injections because the cases in each of
their definitions are mutually exclusive.  They are surjections
because in each of their definitions, every element of $A \union B$
falls under one of the cases.
\end{solution}

\end{problem}

%%%%%%%%%%%%%%%%%%%%%%%%%%%%%%%%%%%%%%%%%%%%%%%%%%%%%%%%%%%%%%%%%%%%%
% Problem ends here
%%%%%%%%%%%%%%%%%%%%%%%%%%%%%%%%%%%%%%%%%%%%%%%%%%%%%%%%%%%%%%%%%%%%%

\endinput
