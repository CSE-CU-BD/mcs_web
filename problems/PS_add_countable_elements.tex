\documentclass[problem]{mcs}

\begin{pcomments}
  \pcomment{PS_add_countable_elements}
  \pcomment{add by ARM 4/5/11, solns by ARM, Adamc}
\end{pcomments}

\pkeywords{
  countable
  bijection
  union
}

%%%%%%%%%%%%%%%%%%%%%%%%%%%%%%%%%%%%%%%%%%%%%%%%%%%%%%%%%%%%%%%%%%%%%
% Problem starts here
%%%%%%%%%%%%%%%%%%%%%%%%%%%%%%%%%%%%%%%%%%%%%%%%%%%%%%%%%%%%%%%%%%%%%

\begin{problem}
Prove that if $A$ is an infinite set and $B$ is a countably infinite
set that has no elements in common with $A$, then
\[
A \bij (A \union B).
\]
\emph{Reminder}: You may assume any of the results from class or text
as long as you state them explicitly.

\examspace[4.0in]

\begin{staffnotes}
\hint See Problem~\bref{CP_smallest_infinite_set}.
\end{staffnotes}

\begin{solution}
Since $A$ is infinite, we can find an infinite sequence $a_0, a_1,
a_2, \dots$ of distinct elements of $A$.  This sequence may be all
of $A$ (if $A$ is countable) or not.  Since $B$ is countably infinite,
it may be enumerated similarly as $b_0, b_1, b_2, \dots$.  Now it's
easy to define the bijection we need.
\begin{eqnarray*}
  f(a_{2i}) &=& a_i \textrm{ (for all $i \in \nngint$)} \\
  f(a_{2i+1}) &=& b_i \textrm{ (for all $i \in \nngint$)} \\
  f(a) &=& a \textrm{ (otherwise)}
% for $a \in A$ such that $a \neq a_i$ for all $i \in \nngint$
\end{eqnarray*}

We check that $f$ is really a bijection.

First, $f$ is a \textbf{total function} $[= 1\text{ out}]$, by definition.
\iffalse
$a_i$ are distinct, then no two different $a_{2i}$ or $a_{2i+1}$ can
be equal, because no two $2i$ or $2i+1$ can be equal, by basic
properties of arithmetic.  Also, the final clause doesn't overlap the
first two, pretty much by definition.\fi


\iffalse Second, $f$ is \textbf{total} $[\\geq 1\text{ out}]$, since
every $n \in \mathbb N$ can be written as either $2i$ or $2i+1$ for an
appropriate $i$, and the last clause covers those $a$ values that
don't match some $a_i$.  Third\fi

Second, $f$ is an \textbf{injection} $[\leq 1\text{ in}]$, since all
the right-hand sides of the defining equations for $f$ are distinct: we
assumed that $B$ does not overlap $A$, as well as that the sequences
of $a_i$ and $b_i$ contain no duplicates.

\iffalse Also, by definition, the last clause applies only for an $a$
that matches no $a_i$.\fi

Finally, $f$ is a \textbf{surjection}$[\geq 1\text{ in}]$.  Any
element of $A \cup B$ is in either $A$ or $B$.  For $a \in A$, if $a =
a_i$ for some $i$, then we have $f(a_{2i}) = a$.  Otherwise, the last
clause gives us $f(a) = a$.  For $b \in B$, we have $b = b_i$ for some
$i$, so $f(a_{2i+1}) = b$.

\end{solution}

\end{problem}

%%%%%%%%%%%%%%%%%%%%%%%%%%%%%%%%%%%%%%%%%%%%%%%%%%%%%%%%%%%%%%%%%%%%%
% Problem ends here
%%%%%%%%%%%%%%%%%%%%%%%%%%%%%%%%%%%%%%%%%%%%%%%%%%%%%%%%%%%%%%%%%%%%%

\endinput
