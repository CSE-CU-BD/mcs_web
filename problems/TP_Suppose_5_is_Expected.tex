\documentclass[problem]{mcs}

\begin{pcomments}
    \pcomment{TP_Suppose_5_is_Expected}
    \pcomment{Converted from expectation-implications.scm
              by scmtotex and dmj
              on Sun 13 Jun 2010 04:49:03 PM EDT}
\end{pcomments}

\begin{problem}
Suppose $X$ is a nonnegative integer random variable for which you
know that $\expect{X}=5$.  However, you know nothing else about it.

In the following two parts, determine whether each completion of the
given sentence is true, false, or could be either (cannot be
determined from the information given).  Justify your answers briefly.

\bparts

\ppart
At least one of the possible values of $X$ is
\begin{itemize}

\item equal to $0$.
\item at most $2.5 = (0 + 5)/2$.
\item less than $5$.
\item at most $5$.
\item at least $10 = 2\cdot 5$.

\end{itemize}

\begin{solution}
\begin{itemize}

\item Equal to $0$: Can't say. The random variable $X$ that is $0$
  with probability $1/2$ and $10$ with probability $1/2$ has
  expectation $5$, as does the constant random variable $X$ that is
  always equal to $5$.

\item At most $2.5$: Can't say.  Same examples as the previous part
  demonstrate this.

\item Less than $5$: Can't say.  Same two examples as first part.

\item At most $5$: True.  If all values of $X$ were greater than $5$,
  then we would have $\expect{X} > 5$.  So, at least one of the values
  must be at most $5$.

\item At least $10$: Can't say. Same two examples as first part.

\end{itemize}

\end{solution}

\ppart 
The expectation $\expect{X^{2}}$
\begin{itemize}

\item is $25=\expect{X}^{2}$.
\item is not $25$.
\item is at most $25$.
\item could be $100$.
\item is at most $\expect{X}$.

\end{itemize}

\begin{solution}
We can answer all of these by considering two possible distributions
of the value of $X$: the constant random variable that is always equal
to $5$ and the random variable that is $20$ with probability $1/4$ and
$0$ with probability $3/4$.  In both cases $\expect{X}=5$ but
$\expect{X^{2}}=25$ in the first case and $\expect{X^{2}}=(3/4) \cdot
0^{2} + (1/4) \cdot 20^{2}=100$ in the second. We cannot tell which,
if either, of these is the true distribution of $X$.
\begin{itemize}
\item Is $25$: Can't say.
\item Is not $25$: Can't say.
\item Is at most $25$: False. It could be $100$, which is greater than $25$.
\item Could be $100$: True.  We have an example of this.
\item Is at most $\expect{X}$: False. $\expect{X}=5$, $\expect{X}$ could be 100, $100>5$. 
\end{itemize}

\end{solution}

\ppart
Give the tightest possible upper bound on the probability $\prob{X \ge  1000}$. 

\begin{solution}
at most 1/200.

A direct application of Markov's Theorem.  Since $X$ is a nonnegative
variable, we know that
\begin{equation*}
    \prob{X \ge a} \le  \expect{X}/a
\end{equation*}
for any~$a$.  In this case, we have
\begin{equation*}
    \prob{X \ge  1000} \le 5/1000 = 1/200 
\end{equation*}
On the other hand, $X$ could be 1000 with probability 1/200 and 0
otherwise, in which case the smaller bounds would not hold.
\end{solution}

\eparts

\end{problem}

\endinput
