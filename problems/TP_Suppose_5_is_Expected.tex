\documentclass[problem]{mcs}

\begin{pcomments}
    \pcomment{TP_Suppose_5_is_Expected}
    \pcomment{Converted from expectation-implications.scm
              by scmtotex and dmj
              on Sun 13 Jun 2010 04:49:03 PM EDT}
\end{pcomments}

\begin{problem}

%% type: short-answer
%% title: Suppose 5 is Expected

Suppose $X$ is a nonnegative integer random variable for which you
know that $\expect{X}=5$.  However, you know nothing else about it.

Complete the following statements so that they are guaranteed to be
true.

\bparts

\ppart
At least one of the possible values of $X$ is
\begin{itemize}

\item equal to $0$.
\item at most $2.5 = (0 + 5)/2$.
\item less than $5$.
\item at most $5$.
\item at least $10 = 2\cdot 5$.

\end{itemize}

\begin{solution}
At most $5$.

If all values of $X$ were greater than $5$, then we would have $\Ex[X] > 5$.
So, at least one of the values must be at most $5$.

The rest of the options are disproved by the following counterexample:
let $X$ be the constant random variable equal to $5$.  Then $\Ex[X]=5$,
and the other statements are obviously false.
\end{solution}

\ppart 
The expectation $\expect{X^{2}}$
\begin{itemize}

\item is $25$.
\item is not $25$.
\item is at most $25$.
\item could be $100$.
\item is at most $\expect{X}$.

\end{itemize}

\begin{solution}
could be 100.

$X$ could be $20$ with probability $1/4$ and $0$ with probability $3/4$.  Then
its expectation would be $5$, but the expectation of its square would be
$100 = (3/4) \cdot 0^{2} + (1/4) \cdot 20^{2}$.

On the other hand:
\begin{itemize}

\item
$X$ could be 0 with probability $1/2$ and 10 with probability $1/2$.  Then
  the expectation of $X^{2}$ would be $50$.  So the first, the third, and
  the last answers are wrong.

\item
$X$ could be 5 with probability $1$.  Then the expectation of $X^{2}$
  would be 25. So, the 2nd answer is wrong, too.

\end{itemize}
\end{solution}

\ppart
The probability $\prob{X \ge  1000}$ is 

\begin{solution}
at most 1/200.

A direct application of Markov's Theorem. Since $X$ is a nonnegative
variable, we know that
\begin{equation*}
    \prob{X \ge a} \le  \expect{X}/a
\end{equation*}
for any~$a$.  In this case, we have
\begin{equation*}
    \prob{X \ge  1000} \le 5/1000 = 1/200 
\end{equation*}
On the other hand, $X$ could be 1000 with probability 1/200 and 0
otherwise, in which case the smaller bounds would not hold.
\end{solution}

\eparts

\end{problem}

\endinput
