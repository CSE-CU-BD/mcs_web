\documentclass[problem]{mcs}

\begin{pcomments}
\pcomment{PS_starfree_language}
\pcomment{ARM 2/10/15}
\end{pcomments}

\pkeywords{
  sequence
  word
  complement
  union
  concatenation
}

%%%%%%%%%%%%%%%%%%%%%%%%%%%%%%%%%%%%%%%%%%%%%%%%%%%%%%%%%%%%%%%%%%%%%
% Problem starts here
%%%%%%%%%%%%%%%%%%%%%%%%%%%%%%%%%%%%%%%%%%%%%%%%%%%%%%%%%%%%%%%%%%%%%

\begin{problem}
A \emph{binary word} is a finite sequence of \texttt{0}'s and
\texttt{1}'s.  For example, $(\texttt{1},\texttt{1},\texttt{0})$ and
$(\texttt{1})$ are words of length three and one, respectively.  We
usually omit the parentheses and commas in the descriptios of words,
so the preceding binary words would just be written as $\texttt{110}$
and $\texttt{1}$.

The basic operation of making one word immediately follow another is
called \term{concatentation}.  For example, the concatentation of
$\texttt{110}$ and $\texttt{1}$ is $\texttt{1101}$, and the
concatentation of $\texttt{110}$ with itself is $\texttt{110110}$.

If $R$ and $S$ are \emph{sets} of words, then $R \cdot S$, is the set
of all words you can get by concatenating a word from $R$ with a word
from $S$.  That is,
\[
R \cdot S \eqdef \set{rs \suchthat r \in R \QAND s \in S}.
\]
For example $T \eqdef \set{\texttt{1}, \texttt{0}} \cdot
\set{\texttt{1}, \texttt{0}}$ is the set
$\set{\texttt{00},\texttt{01},\texttt{10},\texttt{11}}$ of length two
binary words, and, $T \cdot T \cdot T$, which we abbreviate as $T^3$,
is the set of length six binary words.

If $S$ is a set of words, the set of all words you can get by
concatenating copies of any number of words in $S$ is called $S^*$.
By convention, the empty word, $\emptystring$, always included in
$S^*$.  For example, for $T$ as above, the set $T^*$ would be all the
even length binary words.  So the set, $B$, of all binary words is
$\set{\texttt{0},\texttt{1}}^*$.

A set of binary words is called \emph{concatenation-definable}
(\emph{c-d}) if it can be constructed by starting from finite sets of
words and then repeatedly applying the operations of concatenation,
union, and complement (relative to $B$) to these sets.  For example,
we can show that the set $B$ of all binary words is c-d as follows: if
$u$ and $v$ are any two different binary words, then \emph{every} word
is either not equal $u$ or not equal to $v$ (right?).  So
\[
B = \bar{\set{u}} \union \bar{\set{u}}.
\]
It follows that the empty set of words, $\emptyset$, is c-d since
\[
\emptyset = \bar{B}.
\]

Now a more interesting example of a c-d set is the set of all binary
words that include three consecutive \texttt{1}'s:
\[
B\cdot \texttt{111} \cdot B.
\]

\begin{problemparts}
\ppart Show that $\set{\texttt{0}}^*$ is c-d over the binary alphabet.

\begin{solution}
This is simply the set of binary words that do not contain a \texttt{1}
\[
\set{0}^* = \bar{B \cdot \texttt{1} \cdot B}.
\]
\end{solution}

\ppart Show that the set of binary words that start with \texttt{0}
and end with \texttt{1} is c-d.

\begin{solution}
$0B \intersect B1$.
\end{solution}

\ppart Show that $\set{\texttt{01}}^*$ is c-d.

\begin{solution}
This is the set of words that do not include \texttt{00} or
\texttt{11} and start with \texttt{0} and end with \texttt{1}.
\[
\set{\texttt{01}}^* = \bar{B\set{\texttt{00},\texttt{11}}B} \intersect 0B \intersect B1.
\]
\end{solution}

\eparts

Let's say A set of binary words is \emph{boring} on
\texttt{0}'s---boring, for short---if it includes only a finite number
of all \texttt{0} words, or its complement contains only a finite
number of all \texttt{0} words.

\iffalse
Maybe it will help to rephrase this
with formulas: a set of words $S$ is boring iff either $S \intersect
\set{\texttt{0}}^*$ or $\bar{S} \intersect \set{\texttt{0}}^*$ is a
finite set.
\fi

\bparts

\ppart Explain why $\set{\texttt{00}}^*$ is not boring.

\begin{solution}
The set $\set{\texttt{00}}^*$ is the infinite set of all even length
words of \texttt{0}'s, and its complement contains the infinite set of
all the odd length words of \texttt{0}'s.
\end{solution}

\ppart\label{uc0simp} Verify that if $R$ and $S$ are boring, then so
is $R \union S$.

\begin{solution}
There are two cases:

\inductioncase{Case 1}: (Both $R$ and $S$ contain only a finite number
of all-\texttt{0} words.)

Since the union of finite sets is finite, $R \union S$ must also
contain only a finite number of all-\texttt{0} words, so $R \union S$ is boring.

\inductioncase{Case 2}: (At least one $\bar{R}$ and $\bar{S}$ contains
only a finite number of all-\texttt{0} words.)

We can safely assume that it is $\bar{R}$ that has only a finite
number of all \texttt{0} words.  But $\bar{R \union S} \subseteq
\bar{R}$, so $\bar{R \union S}$ also can only have a finite number of
all-\texttt{0} words.  So $R \union S$ is boring in this case as well.
\end{solution}

\ppart Verify that if $R$ and $S$ are boring, then so is $R \cdot S$.

\begin{staffnotes}
A hint indicating the main cases might be worthwhile, but I don't
have the evergy to draft one now.
\end{staffnotes}

\begin{solution}
There are four cases.

\inductioncase{Case 1}: (Both $R$ and $S$ contain only a finite number
of all-\texttt{0} words.)

Since the concatenation of two finite sets of words is finite, $R
\cdot S$ contains only a finite number of all-\texttt{0} words and
therefore is boring in this case.

\inductioncase{Case 2}: (Either $R$ or $S$ contains no all-\texttt{0} words.)

In this case, $R \cdot S$ won't contain any all-\texttt{0} words
either, so it certainly will be boring.

\inductioncase{Case 3}: ($\bar{R}$ contains only a finite number of
all-\texttt{0} words.)

Let's say the length of the longest all-\texttt{0} word in $\bar{R}$
is $n$.  Then $R$ must contain every all-\texttt{0} word longer than
$n$.  We can assume that $S$ includes some all-\texttt{0} word, say of
length $k$, since otherwise Case 2.\ would apply.  But this means that
that $R \cdot S$ contains every all-\texttt{0} word longer than $n+k$.
So every all-\texttt{0} word in $\bar{R \cdot S}$ has length at most
$n+k$, and there are only a finite number of such words.  Hence $R
\cdot S$ is boring in this case.

\inductioncase{Case 4}: ($\bar{S}$ contains only a finite number of
all-\texttt{0} words.)

The proof here is essentially the same as for Case 3.
\end{solution}


\ppart\label{cdimp0s} Explain why all c-d sets are boring.

\begin{solution}
The starting c-d sets include only a single word, and all finite sets
are clearly boring.  Also, $S$ is boring iff $\bar{S}$ is boring
because $S = \bar{\bar{S}}$.  By part~\eqref{uc0simp}, we conclude
that all the operations for constructing c-d sets preserve boredome.
Hence all c-d sets are boring.
\end{solution}

So we have proved that $\set{\texttt{00}}^*$ is not a c-d set.

\end{problemparts}

\end{problem}

%%%%%%%%%%%%%%%%%%%%%%%%%%%%%%%%%%%%%%%%%%%%%%%%%%%%%%%%%%%%%%%%%%%%%
% Problem ends here
%%%%%%%%%%%%%%%%%%%%%%%%%%%%%%%%%%%%%%%%%%%%%%%%%%%%%%%%%%%%%%%%%%%%%

\endinput


\iffalse

\eparts

Let $w$ be a binary word and $n$ a nonnegative integer.  A set $S$ of
words \emph{$n$-repeats $w$}, a word with $n$ consective occurrences
of $w$ is in $S$ iff that word with an extra occurrence of $w$ is also
in $S$.  That is,
\[
xw^ny \in S \QIFF xw^nwy \in S.
\]
We say a set of words is \emph{$n$-repeating} if it $n$-repeats every
word $w$.  For example, $\texttt{0}^*$ is 1-repeating and $B$ is actually 0-repeating.

\begin{problemparts}
\ppart Verify that $\set{\texttt{01}}^*$ is 2-repeating.

\begin{solution}
TBA
\end{solution}

\ppart Explain why $\set{\texttt{00}}^*$ is not $n$-repeating for any $n$.

\begin{solution}
$\set{\texttt{00}}^*$ does not $n$-repeat \texttt{0} for any $n$,
  because \texttt{0}^n is in $\set{\texttt{00}}^*$ but $\texttt{0}^n
  \texttt{0}$ is not.
\end{solution}
\fi

