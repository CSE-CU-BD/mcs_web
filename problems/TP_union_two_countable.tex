\documentclass[problem]{mcs}

\begin{pcomments}
  \pcomment{TP_union_two_countable}
  \pcomment{by ARM 2/11/11}
\end{pcomments}


\pkeywords{
  countable
  union
  duplicates
  list
}

%%%%%%%%%%%%%%%%%%%%%%%%%%%%%%%%%%%%%%%%%%%%%%%%%%%%%%%%%%%%%%%%%%%%%
% Problem starts here
%%%%%%%%%%%%%%%%%%%%%%%%%%%%%%%%%%%%%%%%%%%%%%%%%%%%%%%%%%%%%%%%%%%%%

\begin{problem}
%\begin{lemma}\label{countable-union}

Prove that if $A$ and $B$ are countable sets, then so is $A \union B$.

\begin{solution}

\begin{proof}
Suppose the list of distinct elements of $A$ is $a_0,a_1,\dots$ and the
list of $B$ is $b_0,b_1, \dots$.  Then a list of all the elements in $A
\union B$ is just
\begin{equation}\label{a0b0list}
a_0,b_0,a_1,b_1, \dots a_n,b_n, \dots.
\end{equation}
Of course this list will contain duplicates if $A$ and $B$ have elements
in common, but then deleting all but the first occurrences of each element in
list~\eqref{a0b0list} leaves a list of all the distinct elements of $A$
and $B$.
\end{proof}

\end{solution}

\begin{staffnotes}
  If students get stuck, give them the hint that it's just like the
  bijection between $\naturals$ and $\integers$ given in the
  notes~(\bref{intlist}).
\end{staffnotes}
\end{problem}

%%%%%%%%%%%%%%%%%%%%%%%%%%%%%%%%%%%%%%%%%%%%%%%%%%%%%%%%%%%%%%%%%%%%%
% Problem ends here
%%%%%%%%%%%%%%%%%%%%%%%%%%%%%%%%%%%%%%%%%%%%%%%%%%%%%%%%%%%%%%%%%%%%%

\endinput
