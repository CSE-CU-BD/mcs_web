\documentclass[problem]{mcs}

\begin{pcomments}
  \pcomment{FP_random_walk_multiple}
  \pcomment{from FP_multiple_choice_unhidden_fall13}
  \pcomment{revised from FP_multiple_choice_unhidden by ARM 12/13/13}
  \pcomment{overlaps FP_graphs_short_answer}  
\end{pcomments}

\pkeywords{
  stable_distribution
  countable
}

%%%%%%%%%%%%%%%%%%%%%%%%%%%%%%%%%%%%%%%%%%%%%%%%%%%%%%%%%%%%%%%%%%%%%
% Problem starts here
%%%%%%%%%%%%%%%%%%%%%%%%%%%%%%%%%%%%%%%%%%%%%%%%%%%%%%%%%%%%%%%%%%%%%

\begin{problem} \mbox{}
A \term{trap} in a random walk digraph is a subgraph all of whose
edges end within the trap.  Circle whichever of the following
assertions are true of \idx{stable distributions} on random walk
graphs with exactly two traps which do not share any vertices:

\bparts

%\ppart there may not be any

\ppart there may be a unique one

\ppart there may be exactly two

\ppart there are at most a finite number

%\ppart there are at most a countably infinite number

\ppart there may be a uncountable number, but not necessarily

\ppart there always is an uncountable number

\begin{solution}
Only the last is true.  That's because a stable distribution for
the whole graph always consists of a stable distribution of one trap
weighted with total probability $r$ together with a stable
distribution of the other trap weighted with total probability $1-r$,
for any $r \in [0,1] \subseteq \reals$, and there are an is
uncountable number of real numbers in $[0,1]$.
\end{solution}

\eparts

\end{problem}

\endinput
