\documentclass[problem]{mcs}

\begin{pcomments}
  \pcomment{FP_random_walk_multiple}
  \pcomment{from FP_multiple_choice_unhidden_fall13}
  \pcomment{revised from FP_multiple_choice_unhidden by ARM 12/13/13}
  \pcomment{overlaps FP_graphs_short_answer}  
\end{pcomments}

\pkeywords{
  stable_distribution
  countable
}

%%%%%%%%%%%%%%%%%%%%%%%%%%%%%%%%%%%%%%%%%%%%%%%%%%%%%%%%%%%%%%%%%%%%%
% Problem starts here
%%%%%%%%%%%%%%%%%%%%%%%%%%%%%%%%%%%%%%%%%%%%%%%%%%%%%%%%%%%%%%%%%%%%%

\begin{problem} \mbox{}
A \term{trap} in a random walk digraph is a subgraph all of whose
edges end within the trap.  Suppose a graph has exactly two traps and
they do not share any vertices.  Circle whichever of the following
assertions are \True\ of \idx{stable distributions} the graph.

\bparts

\inbook{\ppart There may not be any.}

\ppart There is a unique one.

\inbook{\ppart There are exactly two.}

%\ppart There are at most a finite number.

\ppart There are a countable number.

\ppart There may not be a countable number.

\ppart There never is ancountable number.

\begin{solution}
Only the last is true.  That's because a stable distribution for the
whole graph always consists of a stable distribution of one trap with
total probability $r$ together with a stable distribution of the other
trap with total probability $1-r$, for any $r \in [0,1] \subseteq
\reals$, and there are an is uncountable number of real numbers in
$[0,1]$.
\end{solution}

\eparts

\end{problem}

\endinput
