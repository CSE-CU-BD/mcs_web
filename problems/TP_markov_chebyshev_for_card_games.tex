\documentclass[problem]{mcs}

\begin{pcomments}
  \pcomment{TP_markov_chebyshev_for_card_games}
  \pcomment{same TP_markov_chebyshev_chernoff_for_card_games w/o last part on Chernoff}
  \pcomment{from: S09.cp14t}
  \pcomment{same as FP_gambling_man but numbers doubled there}
  \pcomment{edited 5/9/14}
\end{pcomments}

\pkeywords{
  probability
  expectation
  variance
  Markov
  Chebyshev
}

\newcommand{\Albert}{\text{Albert}}

%%%%%%%%%%%%%%%%%%%%%%%%%%%%%%%%%%%%%%%%%%%%%%%%%%%%%%%%%%%%%%%%%%%%%
% Problem starts here
%%%%%%%%%%%%%%%%%%%%%%%%%%%%%%%%%%%%%%%%%%%%%%%%%%%%%%%%%%%%%%%%%%%%%

\begin{problem}
  \Albert\ has a gambling problem.  He plays 240 hands of draw poker, 120 hands
  of black jack, and 40 hands of stud poker per day.  He wins a hand of
  draw poker with probability 1/6, a hand of black jack with probability
  1/2, and a hand of stud poker with probability 1/5.  Let $W$ be the
  the number of hands that \Albert\ wins in a day.

\iffalse
  In your answers for this problem, you may finish with unevaluated numerical
  expressions if you like, assigning names to such quantities so that you
  may reference them later.
\fi

\bparts

\ppart What is $\expect{W}$?

\exambox{0.6in}{0.5in}{0.0in}

\begin{solution}
\textbf{108}.

The outcomes of the sample space are the $240+120+40= 400$ hands
played on a given day, with uniform probability.  For each hand $h$,
let $W_h$ be the indicator variable for \Albert\ winning hand $h$.  So
the number of hands won is
\[
W = \sum_h W_h.
\]

By linearity, the expectation of $W$ is the sum over the hands $h$
of the expectation of $W_h$.  But
\[
\expect{W_h} = \pr{\Albert\text{ won }h}.
\]
So
\[
\expect{W} = \sum_h \expect{h} = 240(1/6)+120(1/2)+40(1/5)=108.
\]
\end{solution}

\ppart What would the Markov bound be on the probability that
\Albert\ will win at least 216 hands on a given day?

\exambox{0.6in}{0.5in}{0.0in}

\begin{solution}
$\mathbf{\dfrac{1}{2}}$.

The expected number of games won is 108, so
by Markov,
\[
\pr{W \geq 216} = \pr{W \geq 2\expect{W}} \leq  1/2.
\]
\end{solution}

\ppart Assume that the outcomes of the card games are pairwise
independent.  Write a fraction equal to the variance of the number of
hands won per day.

\exambox{0.7in}{0.6in}{0.0in}

\begin{solution}
$\mathbf{\dfrac{1046}{15}}$.

Pairwise independence is sufficient to ensure additivity of variance.
For an individual hand, the variance is $p(1-p)$, where $p$ is the
probability of winning.  Therefore, the variance of the total number
of hands won in a day is
\[
\variance{W} = 240 \cdot \frac{1}{6} \cdot \frac{5}{6} + 
120 \cdot \frac{1}{2} \cdot \frac{1}{2} +
40 \cdot \frac{1}{5} \cdot \frac{4}{5}
= \frac{1046}{15},
\]
which is $69\ \tfrac{11}{15}$.

\end{solution}

\ppart What would the Chebyshev bound be on the probability that
\Albert\ will win at least 216 hands on a given day?  Express your
answer as a simple arithmetic expression.

\exambox{0.9in}{0.8in}{0.0in}

\begin{solution}
$\mathbf{\dfrac{1046}{15 \cdot (108)^2}}$.

\[
\pr{W \geq 216} = \pr{W - 108 \geq 108} \leq \pr{\abs{W -108} \geq 108} \leq
\frac{\variance{W}}{108^2} = \frac{1046}{15 \cdot (108)^2}, 
\]
which is approximately $0.006$.  So it's very unlikely that
\Albert\ will win 216 hands.
\end{solution}

\eparts
\end{problem}

\endinput
