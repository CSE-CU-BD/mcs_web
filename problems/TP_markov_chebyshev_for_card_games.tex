\documentclass[problem]{mcs}

\begin{pcomments}
  \pcomment{TP_markov_chebyshev__for_card_games}
  \pcomment{same TP_markov_chebyshev_chernoff_for_card_games w/o last part on Chernoff}
  \pcomment{from: S09.cp14t}
  \pcomment{same as FP_gambling_man but numbers doubled there}
  \pcomment{edited 5/9/14}
\end{pcomments}

\pkeywords{
  probability
  expectation
  variance
  Markov
  Chebyshev
}

\newcommand{\Albert}{\text{Albert}}

%%%%%%%%%%%%%%%%%%%%%%%%%%%%%%%%%%%%%%%%%%%%%%%%%%%%%%%%%%%%%%%%%%%%%
% Problem starts here
%%%%%%%%%%%%%%%%%%%%%%%%%%%%%%%%%%%%%%%%%%%%%%%%%%%%%%%%%%%%%%%%%%%%%

\begin{problem}
  \Albert\ has a gambling problem.  He plays 240 hands of draw poker, 120 hands
  of black jack, and 40 hands of stud poker per day.  He wins a hand of
  draw poker with probability 1/6, a hand of black jack with probability
  1/2, and a hand of stud poker with probability 1/5.  Let $W$ be the
  the expected number of hands that \Albert\ wins in a day

  In your answers for this problem, you may finish with unevaluated numerical
  expressions if you like, assigning names to such quantities so that you
  may reference them later.

\bparts

\ppart What is $\expect{W}$?

\begin{center}
\exambox{1.0in}{0.4in}{0.0in}
\end{center}
\examspace[0.5in]

\begin{solution}
The expectation is the sum of the expectations of all the individual hands, namely
\[
120(1/6)+60(1/2)+20(1/5)=54.
\]
\end{solution}

\ppart What would the Markov bound be on the probability that
\Albert\ will win at least 108 hands on a given day?

\begin{center}
\exambox{0.5in}{0.4in}{0.0in}
\end{center}
\examspace[0.5in]

\begin{solution}
The expected number of games won is 54, so
by Markov, $\pr{R \geq 108} \leq 54/108 = 1/2$.
\end{solution}

\ppart Assume that the outcomes of the card games are \emph{pairwise}, but
possibly \emph{not} mutually, independent.  What is the variance of
the number of hands won per day?

\begin{center}
\exambox{2.0in}{0.4in}{0.0in}
\end{center}
\examspace[0.5in]


\begin{solution}
Pairwise independence is sufficient to ensure additivity of variance.
For an individual hand, the variance is $p(1-p)$, where $p$ is the
probability of winning.  Therefore, the variance of the total number
of hands won in a day is
%\[
%120(1/6)(5/6) + 60(1/2)(1/2) + 20(1/5)(4/5) = 523/15 = 34\ \tfrac{13}{15}.
%\]

\[
120 \cdot \frac{1}{6} \cdot \frac{5}{6} + 60 \cdot \frac{1}{2} \cdot \frac{1}{2} + 20 \cdot \frac{1}{5} \cdot \frac{4}{5},
\]

which evaluates to $34\ \tfrac{13}{15}$.

\end{solution}

\ppart\label{chernoff108hands} What would the Chebyshev bound be on
the probability that \Albert\ will win at least 108 hands on a given
day?

\begin{center}
\exambox{2.0in}{0.4in}{0.0in}
\end{center}
\examspace[0.5in]

\begin{solution}
%\[
%\pr{R \geq 108} = \pr{R - 54 \geq 54} \leq \pr{\abs{R - 54} \geq 54} \leq
%\frac{\variance{R}}{54^2} = \frac{523}{15(54)^2} \approx 0.01196.
%\]

\[
\pr{R \geq 108} = \pr{R - 54 \geq 54} \leq \pr{\abs{R - 54} \geq 54} \leq
\frac{\variance{R}}{54^2} = \frac{523}{15 \cdot (54)^2}, 
\]

which is approximately $0.01196$.
\end{solution}

\iffalse

\ppart Assuming outcomes of the card games are \emph{mutually}
independent, show that the probability that \Albert\ will win at
least 108 hands on a given day is much smaller than the bound in
part~\eqref{chernoff108hands}.  \hint $e^{-( 2 \ln 2 + 1)} \leq 0.7$

\begin{solution}
Use the Chernoff Bound
\begin{align*}
\prob{R \geq 108}
  &  = \prob{R \geq 2\expect{R}}\\
  & \leq e^{-\beta(2) \expect{T}}\\
  & = \paren{e^{-\beta(2)}}^{54}
\end{align*}
where $\beta(c) \eqdef c \ln c - c + 1$.  Using the hint, we have
\[
\prob{R \geq 108} \leq \paren{e^{-(2\ln2 -2 +1)}}^{54} \leq (0.7)^{54}.
\]
This last term is much smaller than the bound of 0.01 from
part~\eqref{chernoff108hands}.  In fact, it is less than 1 in 90
billion.

%$9\cdot %10^{-10}$
\end{solution}
\fi

\eparts
\end{problem}

\endinput
