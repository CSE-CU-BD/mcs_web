\documentclass[problem]{mcs}

\begin{pcomments}
  \pcomment{TP_markov_chebyshev_chernoff_for_card_games}
  \pcomment{same TP_markov_chebyshev_for_card_games plus last part on Chernoff}
  \pcomment{from: S09.cp14t}
  \pcomment{same as FP_gambling_man but numbers doubled there}
\end{pcomments}

\pkeywords{
  probability
  expectation
  variance
  Markov
  Chebyshev
}

%%%%%%%%%%%%%%%%%%%%%%%%%%%%%%%%%%%%%%%%%%%%%%%%%%%%%%%%%%%%%%%%%%%%%
% Problem starts here
%%%%%%%%%%%%%%%%%%%%%%%%%%%%%%%%%%%%%%%%%%%%%%%%%%%%%%%%%%%%%%%%%%%%%

\begin{problem}
  A gambler plays 120 hands of draw poker, 60 hands of black jack, and 20
  hands of stud poker per day.  He wins a hand of draw poker with
  probability 1/6, a hand of black jack with probability 1/2, and a hand
  of stud poker with probability 1/5.  %He hopes to win 108 hands per day.

\bparts

\ppart What is the expected number of hands the gambler wins in a day?

\begin{center}
\exambox{1.0in}{0.4in}{0.0in}
\end{center}
\examspace[0.5in]

\begin{solution}
$120(1/6)+60(1/2)+20(1/5)=54$.
\end{solution}

\ppart What would the Markov bound be on the probability that the
gambler will win at least 108 hands on a given day?

\begin{center}
\exambox{0.5in}{0.4in}{0.0in}
\end{center}
\examspace[0.5in]

\begin{solution}
The expected number of games won is 54, so
by Markov, $\pr{R \geq 108} \leq 54/108 = 1/2$.
\end{solution}

\ppart Assume the outcomes of the card games are \emph{pairwise}, but
possibly \emph{not} mutually, independent.  What is the variance in
the number of hands won per day?  You may answer with a
numerical expression that is not completely evaluated.

\begin{center}
\exambox{2.0in}{0.4in}{0.0in}
\end{center}
\examspace[0.5in]


\begin{solution}
Pairwise independence is sufficient to ensure additivity of variance.
For an individual hand the variance is $p(1-p)$ where $p$ is the
probability of winning.  Therefore the variance is
\[
120(1/6)(5/6) + 60(1/2)(1/2) + 20(1/5)(4/5) = 523/15 = 34\ \tfrac{13}{15}.
\]

\iffalse
It is important to note that in this problem part and the previous
one, all hands must be treated individually; we cannot lump the hands
of the same game together because we are only given that the hands are
pairwise independent, not mutually independent.  If they were, we
would be able to use the formulas for binomial distribution with mean
$np$ and variance $np(1-p)$, where the values of $n$ and $p$ depend on
which game you are analyzing.  The individual means and variances for
each game type could then be summed together respectively, which would
give us the same result.  However, this is NOT a valid way to approach
this problem, again, because we are dealing with pairwise
independence, not mutual.
\fi

\end{solution}

\ppart\label{chernoff108hands} What would the Chebyshev bound be on the probability that the
gambler will win at least 108 hands on a given day?  You may answer with a
numerical expression that is not completely evaluated.

\begin{center}
\exambox{2.0in}{0.4in}{0.0in}
\end{center}
\examspace[0.5in]

\begin{solution}
\[
\pr{R \geq 108} = \pr{R - 54 \geq 54} \leq \pr{\abs{R - 54} \geq 54} \leq
\frac{\variance{R}}{54^2} = \frac{523}{15(54)^2} \approx 0.01196.
\]

\end{solution}


\ppart Assuming outcomes of the card games are \emph{mutually}
independent, show that the probability that the gambler will win at
least 108 hands on a given day is much smaller than the bound in
part~\eqeref{chernoff108hands}.  \hint $e^{-( 2 \ln 2 + 1)} \leq 0.7$

\begin{solution}
Use the Chernoff Bound
\begin{align*}
\prob{R \geq 108}
  &  = \prob{R \geq 2\expect{R}}\\
  & \leq e^{-\beta(2) \expect{T}}\\
  & = \paren{e^{-\beta(2)}}^{54}
\end{align*}
where $\beta(c) \eqdef c \ln c - c + 1$.  Using the hint, we have
\[
\prob{R \geq 108} \leq \paren{e^{-(2\ln2 -2 +1)}}^{54} \leq (0.7)^{54}.
\]
This last term is much smaller than the bound of 0.01 from
part~\eqref{chernoff108hands}.  In fact, it is less than 1 in 90
billion.

%$9\cdot %10^{-10}$
\end{solution}

\eparts
\end{problem}

\endinput
