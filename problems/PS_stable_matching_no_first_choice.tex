%PS_stable_matching_no_first_choice

\documentclass[problem]{mcs}

\begin{pcomments}

  \pcomment{from: S08.ps4, F05.ps6}
\end{pcomments}

\pkeywords{
 stable_matching
 first_choice
}


%%%%%%%%%%%%%%%%%%%%%%%%%%%%%%%%%%%%%%%%%%%%%%%%%%%%%%%%%%%%%%%%%%%%%
% Problem starts here
%%%%%%%%%%%%%%%%%%%%%%%%%%%%%%%%%%%%%%%%%%%%%%%%%%%%%%%%%%%%%%%%%%%%%


\begin{problem}

Give an example of a stable matching between 3 boys and 3 girls
where no person gets their first choice.  Briefly explain why your matching
is stable.

\begin{solution}

Call the boys $1, 2, 3$ and the girls $a, b, c$.  Consider
the following preference list:

\begin{center}
\begin{tabular}{llll||llll} \hline
choice& 1st& 2nd& 3rd& choice& 1st& 2nd& 3rd \\ \hline
1& a& b& c& a& 2& 3& 1 \\
2& b& c& a& b& 3& 1& 2 \\
3& c& a& b& c& 1& 2& 3 \\ \hline
\end{tabular}
\end{center}
   
The matching $(1, b), (2, c), (3, a)$ is stable even though no person
gets their first choice.  

To see the intuition behind this solution, notice first that the first
choice of any boy has that boy as her last choice and vice versa.  
Second, notice that everyone ends up with their second choice.  

Since we show a pairing where everyone has their second choice, this 
is stable because the only way to have a rogue pair is for a boy
or girl to want their first choice, but their first choice always 
likes them least so will never want to leave their current partner.  

\end{solution}

\end{problem}

%%%%%%%%%%%%%%%%%%%%%%%%%%%%%%%%%%%%%%%%%%%%%%%%%%%%%%%%%%%%%%%%%%%%%
% Problem ends here
%%%%%%%%%%%%%%%%%%%%%%%%%%%%%%%%%%%%%%%%%%%%%%%%%%%%%%%%%%%%%%%%%%%%%

\endinput
