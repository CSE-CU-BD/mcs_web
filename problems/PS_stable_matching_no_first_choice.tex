\documentclass[problem]{mcs}

\begin{pcomments}
  \pcomment{PS_stable_matching_no_first_choice}
  \pcomment{from: S08.ps4, F05.ps6}
\end{pcomments}

\pkeywords{
 stable_matching
 mating_ritual
}


%%%%%%%%%%%%%%%%%%%%%%%%%%%%%%%%%%%%%%%%%%%%%%%%%%%%%%%%%%%%%%%%%%%%%
% Problem starts here
%%%%%%%%%%%%%%%%%%%%%%%%%%%%%%%%%%%%%%%%%%%%%%%%%%%%%%%%%%%%%%%%%%%%%


\begin{problem}

Give an example of a stable matching between 3 boys and 3 girls where
no person gets their first choice.  Briefly explain why your matching
is stable.  Can your matching be obtained from the Mating Ritual or
the Ritual with boys and girls reversed?

\begin{editingnotes}
Is there a general result here: if all 1st choices differ, then a
stable match where no one gets their first choice is neither boy- nor
girl-optimal?
\end{editingnotes}

\begin{solution}

The idea is to let all the boys have different first choices and all the
girls have different first choices, and each person's first choice
ranks them last.  Moreover, let all the boys also have
different second choices.
                                   
Now having each boy marry his second choice will be stable: a boy,
Tom, can only be in a rogue couple with his first choice girl, Nicole.
But since Tom is Nicole's last choice, she will be married to someone
she prefers to Tom, and so won't have a rogue relationship with Tom.

For example with boys $1, 2, 3$ and the girls $a, b, c$,  the
preferences could be:
\begin{center}
\begin{tabular}{llll||llll} \hline
choice& 1st& 2nd& 3rd& choice& 1st& 2nd& 3rd \\ \hline
1& a& b& c& a& 2& 3& 1 \\
2& b& c& a& b& 3& 1& 2 \\
3& c& a& b& c& 1& 2& 3 \\ \hline
\end{tabular}
\end{center}

So the second-choice matching $(1, b), (2, c), (3, a)$ will be stable
even though no person gets their first choice.  Moreover, this
matching won't be one that comes from the Mating Ritual since it is
neither boy optimal nor girl optimal: giving boys their first choice
would obviously be boy optimal, and giving girls their first choice
would be girl optimal.

\end{solution}

\end{problem}

%%%%%%%%%%%%%%%%%%%%%%%%%%%%%%%%%%%%%%%%%%%%%%%%%%%%%%%%%%%%%%%%%%%%%
% Problem ends here
%%%%%%%%%%%%%%%%%%%%%%%%%%%%%%%%%%%%%%%%%%%%%%%%%%%%%%%%%%%%%%%%%%%%%

\endinput
