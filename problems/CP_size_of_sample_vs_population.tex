\documentclass[problem]{mcs}

\begin{pcomments}
  \pcomment{CP_size_of_sample_vs_population}
  \pcomment{from: S06.cp12f}
  \pcomment{revised by ARM 10/6/09}
\end{pcomments}

\pkeywords{
  sample
  confidence
  estimation
  probability
}

%%%%%%%%%%%%%%%%%%%%%%%%%%%%%%%%%%%%%%%%%%%%%%%%%%%%%%%%%%%%%%%%%%%%%
% Problem starts here
%%%%%%%%%%%%%%%%%%%%%%%%%%%%%%%%%%%%%%%%%%%%%%%%%%%%%%%%%%%%%%%%%%%%%

\begin{problem}
  A defendent in traffic court is trying to beat a speeding ticket on
  the grounds that ---since virtually everybody speeds on the turnpike
  ---the police have unconstitutional discretion in giving tickets to
  anyone they choose.  (By the way, we don't recommend this defense
  :-) )

  To support his argument, the defendent arranged to get a random sample
  of trips by 3,125 cars on the turnpike and found that 94\% of them broke
  the speed limit at some point during their trip.  He says that as a
  consequence of sampling theory (in particular, the Pairwise Independent
  Sampling Theorem), the court can be 95\% confident that the actual
  percentage of all cars that were speeding is $94\pm4\%$.

  The judge observes that the actual number of car trips on the turnpike
  was never considered in making this estimate.  He is skeptical that,
  whether there were a thousand, a million, or 100,000,000 car trips
  on the turnpike, sampling only 3,125 is sufficient to be so confident.

  Suppose you were were the defendent.  How would you explain to the judge
  why the number of randomly selected cars that have to be checked for
  speeding \emph{does not depend on the number of recorded trips}?
  Remember that judges are not trained to understand formulas, so you have
  to provide an intuitive, nonquantitative explanation.

\begin{solution}
This was intended to be a thought-provoking, conceptual question. 
In past terms, although most of the class could follow the
derivations and crank through the formulas to calculate sample size and
confidence levels, many students couldn't articulate, and indeed didn't
really believe that the derived sample sizes were actually adequate to
produce reliable estimates.

Here's a way to explain why we model sampling cars as independent coin
tosses that might work, though we aren't sure about this.

\begin{quote}
  Of the approximately 36,000,000 recorded turnpike trips by cars in 2009,
  there were some \emph{unknown} number, say 35,000,000, that broke the
  speed limit at some point during their trip.  So in this case, the
  \emph{fraction} of speeders is 35,000,000/36,000,000 which is a little
  over 0.97.

  To estimate this unknown fraction, we randomly select some trip from the
  36,000,000 recorded in such a way that \emph{every trip has an equal
    chance of being picked}.  Picking a trip to check for speeding this
  way amounts to rolling a pair dice and checking that double
  sixes were not rolled ---this has exactly the same probability as picking
  a speeding car.

  After we have picked a car trip and checked if it ever broke the speed
  limit, make another pick, again making sure that every recorded trip is
  equally likely to be picked the second time, and so on, for picking a
  bunch of trips.  Now each pick is like rolling the dice and checking
  against double sixes.

  Now everyone understands that if we keep rolling dice looking for double
  sixes, then the longer we roll, the closer the fraction of rolls that
  are double sixes will be to 1/36, since only 1 out of the 36 possible
  dice ouotcomes is double six.  Mathematical theory lets us
  calculate us how many times to roll the dice to make the fraction of
  double sixes very likely close to 1/36, but we needn't go into the
  details of the calculation.

  Now suppose we had a different number of recorded trips, but the same
  fraction were speeding.  Then we could simply use the same dice in the
  same way to estimate the speeding fraction from this different set of
  trip records.

  So the number of rolls needed does not depend on how many trips were
  recorded, it just depends on the fraction of recorded speeders.
\end{quote}

\end{solution}

\end{problem}

%%%%%%%%%%%%%%%%%%%%%%%%%%%%%%%%%%%%%%%%%%%%%%%%%%%%%%%%%%%%%%%%%%%%%
% Problem ends here
%%%%%%%%%%%%%%%%%%%%%%%%%%%%%%%%%%%%%%%%%%%%%%%%%%%%%%%%%%%%%%%%%%%%%

\endinput
