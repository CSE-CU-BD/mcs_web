\documentclass[problem]{mcs}

\begin{pcomments}
  \pcomment{MQ_a_baseball_series_revised}
  \pcomment{MQ version of CP_a_baseball_series}
  \pcomment{CH, Spring '14}
\end{pcomments}

\pkeywords{
  probability
  tree_diagram
}

%%%%%%%%%%%%%%%%%%%%%%%%%%%%%%%%%%%%%%%%%%%%%%%%%%%%%%%%%%%%%%%%%%%%%
% Problem starts here
%%%%%%%%%%%%%%%%%%%%%%%%%%%%%%%%%%%%%%%%%%%%%%%%%%%%%%%%%%%%%%%%%%%%%

\begin{problem}

The Yankees and the Red Sox are playing a two-out-of-three series; in
other words, they play until one team has won two games.  Assume that
the Red Sox win each game with probability $3/5$, regardless of the
outcomes of previous games.

Answer the questions below using the four-step method. Exhibit the
tree diagram, assign probabilities to each outcome, and calculate the required probabilities. Use
the same tree diagram for both problems. 

\bparts

\ppart What is the probability that the series goes three games? 

\begin{solution}
\begin{figure}[h]
\graphic{series}
\end{figure}

From the tree diagram, we get: 
\[
\pr{\text{3 games played}}
     = \frac{12}{125} + \frac{18}{125} + \frac{12}{125} + \frac{18}{125}
      = \frac{12}{25} .
\]
Alternatively: consider the event when the series ends in 2. Then,
either the Yankees win both games, which occurs with probability 
$(2/5)^2$ or the Red Sox win both games, which occurs with probability $(3/5)^2$.
Summing these yields $(4/25)+(9/25) = 13/25 .$ Therefore, the
probability that the series goes 3 games = $1 - 13/25 = 12/25 .$
\end{solution}

\examspace[3.5in]

\iffalse
\ppart What is the probability that the winner of the series loses the
first game?

\begin{solution}
From the tree diagram, we get:
\begin{align*}
\pr{\text{Winner Lost First Game}}
    & = \frac{18}{125} + \frac{12}{125} = \frac{6}{25} .
\end{align*}

\end{solution}

\examspace[0.75in]
\fi

\ppart What is the probability that the Red Sox win the series?

\begin{solution}
From the tree diagram, we get:
\begin{align*}
\pr{\text{Sox Win}}
    & = \frac{18}{125} + \frac{18}{125} + \frac{9}{25} = \frac{81}{125}.
\end{align*}
\end{solution}

%\examspace[1in]

\eparts

\end{problem}

%%%%%%%%%%%%%%%%%%%%%%%%%%%%%%%%%%%%%%%%%%%%%%%%%%%%%%%%%%%%%%%%%%%%%
% Problem ends here
%%%%%%%%%%%%%%%%%%%%%%%%%%%%%%%%%%%%%%%%%%%%%%%%%%%%%%%%%%%%%%%%%%%%%

\endinput
