\documentclass[problem]{mcs}

\begin{pcomments}
  \pcomment{CP_geometric_series_induction}
  \pcomment{Eka Palamadai, added solution on 3/7/2013}
\end{pcomments}

\pkeywords{
  geometric_series
  induction
  ordinary_induction
  series
}

%%%%%%%%%%%%%%%%%%%%%%%%%%%%%%%%%%%%%%%%%%%%%%%%%%%%%%%%%%%%%%%%%%%%%
% Problem starts here
%%%%%%%%%%%%%%%%%%%%%%%%%%%%%%%%%%%%%%%%%%%%%%%%%%%%%%%%%%%%%%%%%%%%%

\begin{problem}
Prove by induction on $n$ that
\begin{equation} \label{geometric_series}
1+r+r^2+\cdots+r^n = \frac{r^{n+1}-1}{r-1}
\end{equation}
for all $n \in \nngint$ and numbers $r\neq 1$.

\begin{solution}
The induction hypothesis $P(n)$ will be equation~\eqref{geometric_series}.

\inductioncase{Base case}: $P(0)$ is true, because both sides of
equation~\eqref{geometric_series} equal one when $n=0$.

\inductioncase{Inductive step}: Assume that $P(n)$ is true, for any
$n \in \nngint$.  Then
\begin{align*}
1+r+r^2+\cdots+r^n + r^{n+1} 
	& = \frac{r^{n+1}-1}{r-1} + r^{n+1} & \text{(by induction hypothesis)}\\
	& = \frac{r^{n+2}-1}{r-1}   & \text{(by simple algebra)}
\end{align*}
which proves $P(n+1)$.

So it follows by induction that $P(n)$ is true for all $n \in \nngint$.
\end{solution}

\end{problem}


%%%%%%%%%%%%%%%%%%%%%%%%%%%%%%%%%%%%%%%%%%%%%%%%%%%%%%%%%%%%%%%%%%%%%
% Problem ends here
%%%%%%%%%%%%%%%%%%%%%%%%%%%%%%%%%%%%%%%%%%%%%%%%%%%%%%%%%%%%%%%%%%%%%

\endinput

