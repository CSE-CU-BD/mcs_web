\documentclass[problem]{mcs}

\begin{pcomments}
  \pcomment{Question obtained from:}
  \pcomment{PS_counting_graphs}
  \pcomment{from: F08 PS9 -> S09 PS9; S09 CP7R}
  \pcomment{part(c) rewritten 4/8/11 by ARM}
  \pcomment{edited 11/3/11 by ARM }
\end{pcomments}

\pkeywords{
  counting
  graph
  asymmetric
  linear
  partial_order
  permutation
  simple_graph
  digraph
}

%%%%%%%%%%%%%%%%%%%%%%%%%%%%%%%%%%%%%%%%%%%%%%%%%%%%%%%%%%%%%%%%%%%%%
% Problem starts here
%%%%%%%%%%%%%%%%%%%%%%%%%%%%%%%%%%%%%%%%%%%%%%%%%%%%%%%%%%%%%%%%%%%%%

\begin{problem}
This problem is about binary relations on the set of integers in the
interval $[1,n]$, and digraphs and simple graphs whose vertex set is
$[1,n]$.

\bparts

\ppart How many digraphs are there?

\begin{solution}
There are $n^2$ potential edges, each of which may or
may not appear in a given graph.  Therefore, the number of graphs is:
\[
2^{n^2}
\]
\end{solution}

\ppart How many asymmetric binary relations are there?

\begin{solution}
There are no self-loops in an asymmetric relation and 
for each of the $\binom{n}{2}$ pairs of distinct elements $a$ and $b$,
either
\begin{enumerate}
\item $a \mrel{R} b$, or
\item $b \mrel{R} a$, or
\item neither,
\end{enumerate}
but not both.  Therefore, the number of asymmetric binary relations is
\[
3^{\binom{n}{2}}.
\]
\end{solution}

\eparts
\end{problem}

%%%%%%%%%%%%%%%%%%%%%%%%%%%%%%%%%%%%%%%%%%%%%%%%%%%%%%%%%%%%%%%%%%%%%
% Problem ends here
%%%%%%%%%%%%%%%%%%%%%%%%%%%%%%%%%%%%%%%%%%%%%%%%%%%%%%%%%%%%%%%%%%%%%

\endinput
