\documentclass[problem]{mcs}

\begin{pcomments}
  \pcomment{PS_Catalan_numbers}
  \pcomment{overlaps PS_Catalan_numbers_meyer_version}
  \pcomment{from: S08.ps10}
\end{pcomments}

\pkeywords{
  generating_functions
  Catalan
  Taylor_expansion
  Taylor_theorem
  recursive
  parentheses
}

%%%%%%%%%%%%%%%%%%%%%%%%%%%%%%%%%%%%%%%%%%%%%%%%%%%%%%%%%%%%%%%%%%%%%
% Problem starts here
%%%%%%%%%%%%%%%%%%%%%%%%%%%%%%%%%%%%%%%%%%%%%%%%%%%%%%%%%%%%%%%%%%%%%

\begin{problem} The method of partial fractions can only be applied to
  rational functions---that is, quotients of polynomials---and there
  are many interesting generating functions that are not of this form.
  Catalan numbers come up frequently in counting the sizes of various
  recursively defined sets---there are in fact hundreds of
  interpretations of these
  numbers!\footnote{\href{http://www-math.mit.edu/~rstan/ec/catadd.pdf}
    {http://www-math.mit.edu/~rstan/ec/catadd.pdf}}

  We are only going to look at one that is relevant to a familiar
  recursively defined set $\RM$ of strings of matched brackets given in
  Definition~\bref{RM-def}.

The $n$th Catalan number, $c_n$, is equal to the number of strings in 
$\RM$ having exactly $n$ left brackets.

Note that $c_0 = 1$ since $\lambda$ is the only string in $\RM$ with $0$
left brackets.

\bparts

\ppart Find a recursive definition for $c_n$ in terms of 
$c_0, c_1, \ldots c_{n-1}$ for $n \geq 1$. (\emph{Hint: assuming
$s$ and $t$ are in $\RM$, what are the possible pairs of integers 
describing the numbers of left brackets in each such that 
$\lefbrk\rhtbrk$ has exactly $n$ left brackets in total?})

\begin{solution}Define $\card{s}$ to be the number of left brackets in
  the string $s \in \RM$.  Answering the question posed in the hint, if
  $|\lefbrk\rhtbrk| = n$ then $|s|+|t|=n-1$ and so the
  possibilities for the pair $(|s|,|t|)$ are $(0,n-1),(1,n-2),
  (n-2,1),(n-1,0)$.

By structural induction we already know the numbers of matched 
brackets strings with less than $n$ left brackets.  Summing
over all the possible pairs,
\begin{equation}\label{Cn_recur}
c_n = \sum_{i=0}^{n-1} c_{i} \cdot c_{n-i-1}
\end{equation}
\end{solution}

\ppart Show that the generating function $C(x)$ corresponding to 
$\ang{c_0, c_1, c_2, \ldots}$ satisfies the following equation
\begin{equation}\label{Cx_poly}
C(x) = 1 + xC(x)^2.
\end{equation}

\begin{solution}By the product rule for generating functions, $c_n$ in
$\eqref{Cn_recur}$ is the coefficient of $(xC(x)) \cdot C(x)$ for
  $n>0$.  But the coefficient $[x^0]xC(x)^2 = 0 $, whereas the
  coefficient $[x^0]C(x) = c_0 = 1$.  Therefore,
\[
C(x)-xC(x)^2 = c_0.
\]
\end{solution}

Solving for $C(x)$ in $\eqref{Cx_poly}$ shows that one of the two 
possible generating functions corresponding to the Catalan numbers
is
\[
C(x)=\frac{1-\sqrt{1-4x}}{2x}
\]

\ppart Use this generating function to show that
\[
c_n = \frac{1}{n+1} \binom{2n}{n} \text{ for $n \geq 0$.}
\] 

\hint It may be easier to find a closed form for the coefficient
$[x^n]2 x C(x)$ first.

\begin{solution}$C(x)$ is not a rational function and therefore cannot 
be simplified by the method of partial fractions.

Let $D(x) = 2 x C(x) = 1 - (1-4x)^{\frac{1}{2}}$.  By Taylor's 
theorem, the $n$th coefficient of $D(x)$ is given by 
$\frac{D^{(n)}(0)}{n!}$.

\begin{align*}
D'(x)      & = 2 \cdot (1-4x)^{-\frac{1}{2}} \\
D''(x)     & = 1 \cdot 2^2 \cdot (1-4x)^{-\frac{3}{2}} \\
D'''(x)    & = 3 \cdot 1 \cdot 2^3 \cdot (1-4x)^{-\frac{5}{2}} \\
D^{(4)}(x) & = 5 \cdot 3 \cdot 1 \cdot 2^4 \cdot (1-4x)^{-\frac{7}{2}} \\
D^{(5)}(x) & = 7 \cdot 5 \cdot 3 \cdot 1 \cdot 2^5 \cdot (1-4x)^{-\frac{9}{2}} \\
D^{(n)}(x) & = (2n-3) \cdot (2n-5) \cdots 5 \cdot 3 \cdot 1 \cdot 2^n \cdot (1-4x)^{\frac{1-2n}{2}}
\end{align*}

\[
d_n = D^{(n)}(0)/n! = \frac{(2n-3) \cdot (2n-5) \cdots 5 \cdot 3 \cdot 1 \cdot 2^n}{n!}
\]

Since $D(x)=2 x C(x)$, 
\begin{align*}
c_n   & = \frac{1}{2} d_{n+1}\\
      & = (2n-1) \cdot (2n-3) \cdots 5 \cdot 3 \cdot 1 \cdot \frac{2^n}{(n+1)!}\\
      & = \frac{(n) \cdot (2n-1) \cdot (n-1) \cdot (2n-3) \cdot (n-2) \cdots 3 \cdot 1 \cdot 1}
               {(n) \cdot (n-1) \cdot (n-2) \cdots 1} \cdot
          \frac{2^n}{(n+1)!}\\
      & = \frac{2(n) \cdot (2n-1) \cdot 2(n-1) \cdot (2n-3) \cdot 2(n-2) \cdots 3 \cdot 2(1) \cdot 1}
               {n!} \cdot
          \frac{1}{(n+1)!}\\
      & = \frac{2n!}{n!(n+1)!}\\
      & = \frac{1}{n+1} \binom{2n}{n}
\end{align*}
\end{solution}

\eparts
\end{problem}

%%%%%%%%%%%%%%%%%%%%%%%%%%%%%%%%%%%%%%%%%%%%%%%%%%%%%%%%%%%%%%%%%%%%%
% Problem ends here
%%%%%%%%%%%%%%%%%%%%%%%%%%%%%%%%%%%%%%%%%%%%%%%%%%%%%%%%%%%%%%%%%%%%%

\endinput
