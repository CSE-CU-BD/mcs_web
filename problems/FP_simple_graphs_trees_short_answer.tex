\documentclass[problem]{mcs}

\begin{pcomments}
  \pcomment{FP_simple_graphs_trees_short_answer}
  \pcomment{excerpted from FP_graphs_short_answer, MQ_matching,
      FP_multiple_choice_unhidden, MQ_list_isomorphisms}
  \pcomment{ARM 12/14/15} 

\end{pcomments}

\pkeywords{
  simple_graph
  tree
  isomorphism
  bottle_neck
  chromatic
  complete_graph
}

%%%%%%%%%%%%%%%%%%%%%%%%%%%%%%%%%%%%%%%%%%%%%%%%%%%%%%%%%%%%%%%%%%%%%
% Problem starts here
%%%%%%%%%%%%%%%%%%%%%%%%%%%%%%%%%%%%%%%%%%%%%%%%%%%%%%%%%%%%%%%%%%%%%

\begin{problem} 
Answer the following questions about \textbf{finite simple graphs}.
You may answer with formulas involving exponents, binomial
coefficents, and factorials.

\bparts

\ppart How many edges are there in the \emph{complete graph} $K_{41}$?
\hfill \examrule

\begin{solution}
\[
\binom{41}{2} = 820.
\]
\end{solution}

\ppart How many edges are there in a spanning tree of $K_{41}$? \hfill \examrule

\begin{solution}
\textbf{40}.
\end{solution}

\ppart What is the chromatic number $\chi(K_{41})$? \hfill \examrule
\begin{solution}
41.
\end{solution}

\ppart What is the chromatic number $\chi(C_{41})$, of the cycle of
  length 41? \hfill \examrule

\begin{solution}
\textbf{3}.
\end{solution}

\ppart Let $G$ be the graph in Figure~\ref{graphs_for_isomorphism_a}.
How many distinct isomorphisms are there from $G$ to $G$?
\hfill \examrule

\begin{figure}[h]
\graphic[height=0.75in]{isomorphism_a}
\caption{The graph $G$}
\label{graphs_for_isomorphism_a}
\end{figure}

\begin{solution}
\textbf{4}.
\end{solution}

\ppart How many cycles can there be in a graph created by adding a
single edge\\
to a tree with 41 vertices? \hfill\examrule

\begin{solution}
Exactly \textbf{1}.
\end{solution}

\ppart What is the smallest number of leaves possible in a tree with
41 vertices? \hfill \examrule
\begin{solution}
\textbf{2}.
\end{solution}

\ppart What is the largest possible number of leaves possible in a
tree with 41 vertices? \hfill \examrule
\begin{solution}
\textbf{40}.
\end{solution}

\iffalse
\ppart How many trees are there whose vertices are the integers
$\Zintv{1}{41}$? \hfill \examrule
\begin{solution}
$41^{39}$.  See Problem~(\bref{CP_numbered_trees}).
\end{solution}
\fi

\ppart\label{numl11p} How many length 11-paths are there in $K_{41}$?

\begin{center}
\exambox{0.7in}{0.4in}{0in}
\end{center}

\begin{solution}
\[
\frac{41!}{30!}
\]
There is a bijection between length-11 paths and length-11 sequences
of vertices.
\end{solution}

\ppart Let $s$ be the number of length-11 paths in $K_{41}$---that is,
answer to part~\eqref{numl11p}.  How many length-11 \emph{cycles} are
in $K_{41}$?

\begin{center}
\exambox{0.7in}{0.4in}{0in}
\end{center}

\hint For vertices $a,b,c,d,e$, the sequences $abcde$, $bcdea$, and
$edcba$ all describe the same length-5 cycle.

\begin{solution}
\[\frac{s}{11! \cdot 2}\]

 A cycle is determined by the sequence of 11 vertices in it, but since
 a cycle has no ``first'' or ``last'' vertex, 11 different sequences
 determine the same cycle in clockwise order, and another 11---the
 reversals of the first 11---determine it in counterclockwise order.
 So the number of cycles equals the number of sequences, namely
 $s=41!/30!$, divided by $11 \cdot 2$.
\end{solution}

\iffalse
\ppart How many cycles can there be in a graph created by adding two\\
edges to a tree with 41 vertices? \hfill \examrule

\begin{solution}
\textbf{2 or 3}.
\end{solution}
\fi


\eparts

\end{problem}

\endinput
