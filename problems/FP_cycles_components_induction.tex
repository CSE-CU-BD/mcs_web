\documentclass[problem]{mcs}

\begin{pcomments}
  \pcomment{FP_cycles_components_induction}
  \pcomment{F10.midterm; soln on OCW}
  \pcomment{edited ARM 4/14/15}
\end{pcomments}

\pkeywords{
  connected
  component
  cycle
  induction
}

\begin{problem}
If a simple graph has $e$ edges, $v$ vertices, and $k$ connected
components, then it has at least $e - v + k$ cycles.

Prove this by induction on the number of edges $e$.

\begin{solution}
The proof is by induction on $e$ with hypothesis 
\begin{quote}
$P(e) \eqdef \forall v,c,k \in \naturals.$ if simple a graph has $v$
  vertices, $e$ edges, $c$ cycles, and $k$ components, then
\begin{equation}\label{cge-v+k}
c \geq e - v + k.
\end{equation}
\end{quote}

\inductioncase{Base Case} ($e=0$): If a graph has 0 edges and $v$
vertices, then it has no cycles and each vertex is its own connected
component.  Hence there are $v$ connected components.  So
\[
c = 0 = 0 - v + v = e - v + k,
\]
as required.

\inductioncase{Inductive Step} Assume that $P(e)$ holds for some $e
\geq 0$, and let $G$ be any graph with $e+1$ edges, $v$ vertices, $k$
components, and $c$ cycles.  Now remove an arbitrary edge $\alpha$
from $G$ to obtain $G'$.

\textit{Case 1:} $\alpha$ appears on a cycle in $G$.  Then removing
$\alpha$ decreased the number of cycles by at least one, but because
it was on a cycle, the number of connected components remains the
same.  So $G'$ has $e$ edges, $v$ vertices, and $k$ components and at
most $c-1$ cycles.  By induction, we conclude that
\[
c-1 \geq \#\text{cycles in }G' \geq e - v + k,
\]
so
\[
c \geq (e+1)  - v + k,
\]
which proves that $P(e+1)$ holds for $G$ in this case.

\textit{Case 2:} $\alpha$ is not part of any cycle in $G$.  Then
removing $\alpha$ increases the number of connected components by one,
leaving the number of cycles and vertices unchanged.

So $G'$ has $e$ edges, $v$ vertices, and $k+1$ components and at
most $c$ cycles.  By induction, we conclude that
\[
c \geq \#\text{cycles in }G' \geq e - v + (k+1),
\]
so
\[
c \geq (e+1) - v + k,
\]
which proves that $P(e+1)$ also holds for $G$ in this case.

So $P(e+1)$ holds in any case, completing the proof of the inductive
step.
\end{solution}

\end{problem}

\endinput
