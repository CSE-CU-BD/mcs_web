\documentclass[problem]{mcs}

\begin{pcomments}
  \pcomment{TP_cardinality_examples}
  \pcomment{subsumes TP_uncountable_example}
  \pcomment{much overlap with TP_cardinality_class}
  \pcomment{prepared for midterm 2, Fall15}
  \pcomment{Zoran Dzunic 10/10/15, some format \& solns ARM 10/16/15}
\end{pcomments}

%%%%%%%%%%%%%%%%%%%%%%%%%%%%%%%%%%%%%%%%%%%%%%%%%%%%%%%%%%%%%%%%%%%%%
% Problem starts here
%%%%%%%%%%%%%%%%%%%%%%%%%%%%%%%%%%%%%%%%%%%%%%%%%%%%%%%%%%%%%%%%%%%%%
\begin{problem}

\bparts

\ppart
For each of the following sets, indicate whether it is
finite\inhandout{ (\textbf{F})},
countably infinite\inhandout{ (\textbf{C})},
or uncountable\inhandout{ (\textbf{U})}.

\renewcommand{\theenumi}{\roman{enumi}}
\renewcommand{\labelenumi}{(\theenumi)}

\begin{enumerate}
\item %\label{itm:ints}
The set of even integers greater than $10^{100}$.\hfill\examrule[0.4in]

\begin{solution}
\textbf{C}.

The function $f(n) \eqdef (n-10^{100})/2$ defines a bijection with $\nngint$.
\end{solution}

\item %\label{itm:complex}
The set of ``pure'' complex numbers of the form $ri$ for nonzero real
numbers~$r$.\hfill\examrule[0.4in]

%\item \label{itm:char} The set of words in the English language no more
%  than 20 characters long. \instatements{\hfill\examrule[0.4in]}

\item %\label{itm:self_bij}
The powerset of the integer interval $\Zintvcc{10}{10^{10}}$.\hfill\examrule[0.4in]

\begin{solution}
\textbf{F}.

The powerset of a finite set is exponentially larger, but still finite.
\end{solution}


\item %\label{itm:roots}
 The complex numbers $c$ such that $\exists m,n \in \integers.\,
 (m+nc)c = 0$. \hfill\examrule[0.4in]

\medskip Let $\mathcal{U}$ be an uncountable set, $\mathcal{C}$ be a
countably infinite subset of $\mathcal{U}$, and $\mathcal{D}$ be a
countably infinite set.

\item %\label{itm:unionuc}
 $\mathcal{U} \union \mathcal{D}$. \hfill\examrule[0.4in]

\begin{solution}
\textbf{U}.

Adding a countable number of elements to an infinite set $A$ yields a
set of the ``same size,'' that is, a set with a bijection to $A$,
\inhandout{as remarked in the text}\inbook{(see
  Section~\bref{countable_subsec} and
  Problem~\ref{PS_add_countable_elements})}.  So since $U$ is
uncountable, $\mathcal{U} \union \mathcal{D}$ is also uncountable.
\end{solution}


\item %\label{itm:interuc}
$\mathcal{U} \intersect \mathcal{C}$\hfill\examrule[0.4in]

\begin{solution}
\textbf{C}.

Since $\mathcal{C} \subseteq \mathcal{U}$,
\[
\mathcal{U} \intersect \mathcal{C} = \mathcal{C}
\]
\end{solution}

\item %\label{itm:diffuc}
$\mathcal{U} - \mathcal{D}$ \hfill\examrule[0.4in]

\begin{solution}
Adding a countable number of elements to an infinite set $A$ yields a
set with a bijection to $A$, as noted above.  Since we can add the
elements of $D$ back into $\mathcal{U} - \mathcal{D}$ to get the
uncountable set $U$, the set $\mathcal{U} - \mathcal{D}$ must have
been uncountable in the first place.
\end{solution}

\end{enumerate}

\iffalse
\begin{center}
\begin{tabular}{c |c | c}
Finite & Countably infinite & Uncountable \\ \hline
\ref{itm:roots}  &  \ref{itm:ints}  & \ref{itm:complex}\\
\ref{itm:self_bij} & \ref{itm:total_surj} &
\end{tabular}
\end{center}
\fi

\end{solution}

\examspace[0.1in]

\ppart
Given examples of sets $A$ and $B$ such that
\[
\reals \strict A \strict B.
\]
Recall that $A \strict B$ means that $A$ is not ``as big as'' $B$.

\begin{solution}
Let $A$ be $\power(\reals)$ and $B$ be $\power(A)$.
\end{solution}

%\examspace[2.0in]

\eparts

\end{problem}

%%%%%%%%%%%%%%%%%%%%%%%%%%%%%%%%%%%%%%%%%%%%%%%%%%%%%%%%%%%%%%%%%%%%%
% Problem ends here
%%%%%%%%%%%%%%%%%%%%%%%%%%%%%%%%%%%%%%%%%%%%%%%%%%%%%%%%%%%%%%%%%%%%%

\endinput

