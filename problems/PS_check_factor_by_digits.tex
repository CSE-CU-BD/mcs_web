\documentclass[problem]{mcs}

\begin{pcomments}
  \pcomment{PS_check_factor_by_digitsb}
  \pcomment{same as FP_check_factor_by_digits but asks for explanation}
  \pcomment{rewritten ARM 12/15/11}
\end{pcomments}

\pkeywords{
  number theory
  modular
}

%%%%%%%%%%%%%%%%%%%%%%%%%%%%%%%%%%%%%%%%%%%%%%%%%%%%%%%%%%%%%%%%%%%%%
% Problem starts here
%%%%%%%%%%%%%%%%%%%%%%%%%%%%%%%%%%%%%%%%%%%%%%%%%%%%%%%%%%%%%%%%%%%%%

\begin{problem}
\iffalse
The sum of the digits of the base 10 representation of an integer is
congruent modulo 9 to that integer.  For example
\[
763 \equiv 7+6+3 \pmod 9.
\] 
However, this is not always true for the hexadecimal (base 16) representation.  For example,
\[
(763)_{16} = 7\cdot 16^2 + 7\cdot 16 + 3 \equiv 1 \not\equiv 7 \equiv 7 + 6 + 3 \pmod 9.
\]
For exactly what integers $k>1$ is it true that the sum of the digits
of the base 16 representation of an integer is congruent modulo $k$ to
that integer?  %Explain your answer.
\begin{center}
\exambox{1.0in}{0.5in}{0.0in}
\end{center}

\examspace[3in]

\begin{solution}
\[
3,5,15.
\]

Summing the digits mod $k$ works iff $16 \equiv 1 \pmod k$.  This is
equivalent to $k \divides 16-1 = 15$.  So the three factors of 15 are
exactly the $k$s that work.

To see why only these $k$s work, just look at two-digit hex numbers
$16a +b$ where $a,b \in [0,16)$.  In this case the digit-sum requirement
means that for all such $a,b$, 
\begin{align*}
16a+b & \equiv a+b \pmod k,\\
16 \equiv 1 \pmod{k} & (letting $a = 1, b=0$).
\end{align*}
\end{solution}
\fi

%\iffalse
Given any integer $x$ represented in base $b$, for some $k \in [2,b)$,
  we can check whether $x$ is a multiple of $k$ by checking whether
  the sum of the digits of $x$ is a multiple of $k$.  Let $K_b$ be all
  such $k$'s.  For example in decimal representation ($b = 10$), we
  can check whether any integer is a multiple of $3$ by checking
  whether the sum of its digits is a multiple of $3$.

\bparts

\ppart
Let $x_{d-1}...x_1x_0$ be the representation of $x$ in base $b$,
where $d$ is the total number of digits.
Write an equation to express $x$ in terms of its digits.

\begin{solution}
$x = \sum_{i=0}^{d-1}{x_i}\left(b^i\right)$
\end{solution}

\ppart
Show that if $b \equiv 1 \mod k$,
then $x \equiv \sum_{i=0}^{d-1}{x_i} \mod k$.

\begin{solution}
$x \equiv \sum_{i=0}^{d-1}{x_i}\left(b^i\right)
\equiv \sum_{i=0}^{d-1}{x_i}\left(1^i\right)
\equiv \sum_{i=0}^{d-1}{x_i} \mod k$
\end{solution}

\ppart
Show that if $b \equiv 1 \mod k$,
then $k$ must be a divisor of $b-1$.
\begin{solution}
$b \equiv 1 \mod k$ implies $b-1 \equiv 0 \mod k$,
so $b-1$ is a multiple of $k$
\end{solution}

\ppart
For hexadecimal representation ($b = 16$), find $K_{16}$

\begin{solution}
$\set{3, 5, 15}$
\end{solution}

\eparts

%\fi

\end{problem}

%%%%%%%%%%%%%%%%%%%%%%%%%%%%%%%%%%%%%%%%%%%%%%%%%%%%%%%%%%%%%%%%%%%%%
% Problem ends here
%%%%%%%%%%%%%%%%%%%%%%%%%%%%%%%%%%%%%%%%%%%%%%%%%%%%%%%%%%%%%%%%%%%%%

\endinput
