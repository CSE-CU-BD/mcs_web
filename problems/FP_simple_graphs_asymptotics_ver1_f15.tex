\documentclass[problem]{mcs}

\begin{pcomments}
  \pcomment{FP_simple_graphs_asymptotics}
  \pcomment{adapted into an explanation question from FP_simple_graphs_TF} 
  \pcomment{ARM/CH, 5/16/14, def if O() revised ARM 11/1/15}
\end{pcomments}

\pkeywords{
  vertices
  edge
  coloring
  chromatic_number
  big_Oh
}

%%%%%%%%%%%%%%%%%%%%%%%%%%%%%%%%%%%%%%%%%%%%%%%%%%%%%%%%%%%%%%%%%%%%%
% Problem starts here
%%%%%%%%%%%%%%%%%%%%%%%%%%%%%%%%%%%%%%%%%%%%%%%%%%%%%%%%%%%%%%%%%%%%%

\begin{problem} 
Let $f,g$ be positive real-valued functions on finite trees
with at least two vertices.
We will extend the $O()$ notation to
such tree functions as follows: $f = O(g)$ iff
there is a constant $c>0$ such that
\[
f(T) \leq c \cdot g(T) \text{ for all trees } T.
\]
For each of the following assertions, state whether it is \True\ or
\False\, and briefly explain your answer.  You are \textbf{not}
expected to offer a careful proof or detailed counterexample.

\emph{Reminder}: $\vertices{T}$ is the set of vertices and $\edges{T}$
is the set of edges of $T$. %, and $G$ is connected.

\bparts

\ppart $\card{\vertices{T}} = O(\card{\edges{T}})$.

\begin{solution}
\True.

$\card{\vertices{T}} = \card{\edges{T}} + 1\leq 2 \cdot \card{\edges{T}}$.

\end{solution}

\examspace[1in]

\ppart $\card{\edges{T}} = O(\card{\vertices{T}})$.

\begin{solution}
\True.

$\card{\edges{T}} = \card{\vertices{T}} - 1 \leq \card{\vertices{T}}$.

\end{solution}

\examspace[1in]

\ppart $\card{\vertices{T}} = O(\chi(T))$, where $\chi(T)$ is the
chromatic number of $T$.

\begin{solution}
\False.

For example, every tree with more than one vertex has a chromatic
number 2.
\end{solution}

\examspace[1in]

\ppart $\chi(T) = O(\card{\vertices{T}})$.

\begin{solution}
\True.

Assigning a different color to each vertex gives a valid coloring, so
$\chi(T) \leq \card{\vertices{T}}$.
\end{solution}

\eparts

\end{problem}

\endinput
