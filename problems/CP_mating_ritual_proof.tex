%CP_mating_ritual_proof

\documentclass[problem]{mcs}

\begin{pcomments}
  \pcomment{from: S08 cp5f(?)}
\end{pcomments}

\pkeywords{
 stable_matching
 Mating_ritual
 invariant
}


%%%%%%%%%%%%%%%%%%%%%%%%%%%%%%%%%%%%%%%%%%%%%%%%%%%%%%%%%%%%%%%%%%%%%
% Problem starts here
%%%%%%%%%%%%%%%%%%%%%%%%%%%%%%%%%%%%%%%%%%%%%%%%%%%%%%%%%%%%%%%%%%%%%

\begin{problem}
A preserved invariant of the Mating ritual is:
\begin{quote}
For every girl, $G$, and every boy, $B$, if $G$ is crossed off $B$'s list,
then $G$ has a favorite suitor and she prefers him over $B$.
\end{quote}

Use the invariant to prove that the Mating Algorithm produces stable
marriages.  (Don't look up the proof in the Notes or slides.)

\begin{solution}

\begin{proof}
Let Brad be some boy and Jen be any girl that he is \emph{not} married to
on the last day of the Mating Ritual.  We claim that Brad and Jen are not
a rogue couple.  Since Brad is an arbitrary boy, it follows that no boy is
part of a rogue couple.  Hence the marriages on the last day are stable.

To prove the claim, we consider two cases:

\emph{Case} 1.  Jen is not on Brad's list.  Then by invariant $P$, we know
that Jen prefers her husband to Brad.  So she's not going to run off with
Brad: the claim holds in this case.

\emph{Case} 2.  Otherwise, Jen is on Brad's list.  But since Brad is not
married to Jen, he must be choosing to serenade his wife instead of Jen,
so he must prefer his wife.  So he's not going to run off with Jen: the
claim also holds in this case.
\end{proof}

\end{solution}

\end{problem}

\endinput
