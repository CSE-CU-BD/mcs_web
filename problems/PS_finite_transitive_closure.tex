\documentclass[problem]{mcs}

\begin{pcomments}
  \pcomment{PS_finite_transitive_closure}
  \pcomment{from: digraph notes problem}
\end{pcomments}

\pkeywords{
  transitive_closure
  identity_relation
  composition
}

%%%%%%%%%%%%%%%%%%%%%%%%%%%%%%%%%%%%%%%%%%%%%%%%%%%%%%%%%%%%%%%%%%%%%
% Problem starts here
%%%%%%%%%%%%%%%%%%%%%%%%%%%%%%%%%%%%%%%%%%%%%%%%%%%%%%%%%%%%%%%%%%%%%

\begin{problem}
If $R$ is a binary relation on a set, $A$, then $R^k$ denotes the
relational composition of $R$ with itself $k$ times.

\bparts

\ppart\label{pathleqn} 
Prove that if $R$ is a relation on a finite set,
$A$, then
\[
a  \mrel{(R \union I_A)^n} b \qiff \mbox{ there is a path in $R$ of length
   length $\leq n$ from $a$ to $b$}.
\]

\ppart Conclude that if $A$ is a finite set, then
\begin{equation}\label{R*RUIn}
R^* = (R \union I_A)^{\card{A}-1}.
\end{equation}

\begin{solution}
  Since a path cannot be longer than $\card{A}-1$, it follows that
  there a path from one vertex to another iff there is a path of length at
  most $\card{A}-1$.  By part~\eqref{pathleqn}, that's exactly what
  equation~\eqref{R*RUIn} means.
\end{solution}

\eparts

\end{problem}

%%%%%%%%%%%%%%%%%%%%%%%%%%%%%%%%%%%%%%%%%%%%%%%%%%%%%%%%%%%%%%%%%%%%%
% Problem ends here
%%%%%%%%%%%%%%%%%%%%%%%%%%%%%%%%%%%%%%%%%%%%%%%%%%%%%%%%%%%%%%%%%%%%%

\endinput
