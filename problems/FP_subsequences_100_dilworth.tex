\documentclass[problem]{mcs}

\begin{pcomments}
  \pcomment{TP_subsequences_100_dilworth}
  \pcomment{CH, spring 2014}
  \pcomment{forked from TP_subsequence_of_101}
\end{pcomments}

\pkeywords{chain
           anti-chain
           increasing
           decreasing
           partial order
}

%%%%%%%%%%%%%%%%%%%%%%%%%%%%%%%%%%%%%%%%%%%%%%%%%%%%%%%%%%%%%%%%%%%%%
% Problem starts here
%%%%%%%%%%%%%%%%%%%%%%%%%%%%%%%%%%%%%%%%%%%%%%%%%%%%%%%%%%%%%%%%%%%%%

\begin{problem}

Let $S = (s_1, s_2, \ldots  s_{100})$ be a sequence consisting of the integers from 1 to 100 listed in some order. 

\bparts

\ppart Suppose we construct a directed graph $G$ with 100 vertices as
follows: there is a (directed) edge from $i$ to $j$ iff $s_i <
s_j$. What is the minimum number of edges in $G$ over all
possible choices of $S$? What is the maximum possible number of edges?

\begin{solution}
The minimum number of edges in $G$ occurs when the numbers $1, \ldots, 100$ are listed
in decreasing order; for this case, there are no edges and the graph
is empty.

The maximum possible number of edges in $G$ occurs when the
numbers $1, \ldots, 100$ are listed in increasing order. The number in
this case is $100*99/2 = 4950$.

\end{solution}

\ppart Argue that the following statement is false: ``There exists a
sequence $S$ containing all the numbers from 1 to 100, such that
neither the longest increasing subsequence nor the longest decreasing
subsequence are greater than 9.''

\begin{solution}
The directed graph from the first part induces a partial order on the
elements of $S$. Every increasing subsequence corresponds a chain, while every decreasing
subsequence corresponds to an antichain. By Dilworth's Lemma, either a
chain or an antichain must have length at least $\sqrt{100} = 10$,
contradicting the statement.
\end{solution}

\ppart Give an example of $S$ such that the longest
increasing subsequence and the longest decreasing
subsequence are \emph{both} of length 10.

\begin{solution}
Break the sequence of integers from 1 to 100 into 10 contiguous
blocks of 10 integers:
\[
1,2,\dots,10\quad 11, 12,\dots,20 \quad \dots \quad 91,92,\dots, 100.
\]
Now, list the 100 numbers by concatenating these 10 contiguous
subsequences, each listed in reverse order. There
\[
91,92,\dots,10000,\quad \dots, \quad
11,12,\dots,20,\quad 1,2,\dots, 10.
\]

The longest increasing subsequence in
has length 10.  That is because for any given number, $n$, in the
list, the only numbers to the right that are also greater than $n$ are
those in the same contiguous subsequence as $n$.  So the only way to
form an increasing subsequence of the whole list is to use all the
numbers in a contiguous subsequence.

The longest decreasing subsequence in the whole list also has length
10.  That is because no decreasing subsequence of the whole list can
use more than one number from each of the 10 contiguous subsequences.
Moreover, any sequence numbers containing exactly one one number from each
of the 100 contiguous subsequences will be a decreasing subsequence of
length 100.
\end{solution}

\eparts

\end{problem}


%%%%%%%%%%%%%%%%%%%%%%%%%%%%%%%%%%%%%%%%%%%%%%%%%%%%%%%%%%%%%%%%%%%%%
% Problem ends here
%%%%%%%%%%%%%%%%%%%%%%%%%%%%%%%%%%%%%%%%%%%%%%%%%%%%%%%%%%%%%%%%%%%%%

\endinput
