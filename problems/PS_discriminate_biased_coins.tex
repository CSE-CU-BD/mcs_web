\documentclass[problem]{mcs}

\begin{pcomments}
  \pcomment{from: S06 PS10 -> S09 PS11}
\end{pcomments}

\pkeywords{
}

%%%%%%%%%%%%%%%%%%%%%%%%%%%%%%%%%%%%%%%%%%%%%%%%%%%%%%%%%%%%%%%%%%%%%
% Problem starts here
%%%%%%%%%%%%%%%%%%%%%%%%%%%%%%%%%%%%%%%%%%%%%%%%%%%%%%%%%%%%%%%%%%%%%

\begin{problem}
We have two coins: one is a fair coin, but the other
produces heads with probability $\frac{3}{4}$.  One of the two coins is 
picked, and this coin is tossed $n$ times.  

\begin{problemparts}

\ppart  How large must $n$ be for you to be able to infer, with 95\% confidence,
which of the two coins had been chosen?  (Get close to the minimum
value of $n$ required without considering any details of the relevant distribution
functions, apart from mean and variance.)  
\hint Use Chebyshev's Theorem.

\begin{solution}
To guess which coin was picked, set a threshold $t$ between $\frac{1}{2}$ 
and $\frac{3}{4}$.  If the proportion of heads is less than the threshold,
guess that the fair coin had been picked; otherwise, guess the biased coin.
Let the random variable $F$ be the number of heads that would appear in 
the first $n$ flips of the fair coin, and let $B$ denote the number of heads
that would appear in the first $n$ flips of the biased coin.  We must flip 
the coin sufficiently many times to ensure that 
\begin{equation}\label{confreqF}
\pr{\frac{F}{n} \geq t} \leq 0.05
\end{equation}
and
\begin{equation}\label{confreqB}
\pr{\frac{B}{n} < t} \leq 0.05
\end{equation}
A natural threshold to choose is
$t=\frac{5}{8}$, exactly in the middle of $\frac{1}{2}$ and $\frac{3}{4}$.

Now, $F$ has an $\paren{n,\frac{1}{2}}$-binomial distribution, so its 
expectation and
variance are $n\paren{\frac{1}{2}}=\frac{n}{2}$ and 
$n\paren{\frac{1}{2}}\paren{1-\frac{1}{2}}=\frac{n}{4}$, respectively.  Using Chebyshev's
inequality for the fair coin,
\begin{align*}
\pr{\frac{F}{n} \geq \frac{5}{8}} 
& = \pr{\frac{F}{n} - \frac{1}{2} \geq \frac{5}{8} - \frac{1}{2}}
  = \pr{F - \frac{n}{2} \geq \frac{n}{8} } \\
& = \pr{F - \expect{F} \geq \frac{n}{8} } 
  \leq \pr{ \abs{F - \expect{F}} \geq \frac{n}{8} } \\
& \leq \frac{\variance{F}}{\paren{\frac{n}{8}}^2} 
  = \frac{\frac{n}{4}}{\frac{n^2}{64}}
  = \frac{16}{n}
\end{align*}

$B$, on the other hand, has an $\paren{n,\frac{3}{4}}$-binomial 
distribution, 
so its expectation and variance are $n\paren{\frac{3}{4}}=\frac{3n}{4}$ and
$n\paren{\frac{3}{4}}\paren{1-\frac{3}{4}}=\frac{3n}{16}$, respectively.  Using Chebyshev's
inequality for the biased coin,
\begin{align*}
\pr{\frac{B}{n} < \frac{5}{8}} 
& = \pr{\frac{3}{4} - \frac{B}{n} > \frac{3}{4} - \frac{5}{8}}
  = \pr{\frac{3n}{4} - B > \frac{n}{8} } \\
& = \pr{\expect{B} - B > \frac{n}{8} } 
  \leq \pr{ \abs{B - \expect{B}} \geq \frac{n}{8} } \\
& \leq \frac{\variance{B}}{\paren{\frac{n}{8}}^2} 
  = \frac{\frac{3n}{16}}{\frac{n^2}{64}}
  = \frac{12}{n}
\end{align*}

So, for the required confidence level, demand that $\frac{16}{n} \leq 0.05$ and $\frac{12}{n} \leq 0.05$.
These hold iff $\frac{16}{n} \leq 0.05$, which is true iff $n \geq 320$.  So knowing the results of at
least $320$ flips of the chosen coin will allow us to guess its identity with 95\% confidence. 

(Because the variance of the biased coin is less than that of the fair coin, we
can do slightly better if we increase our threshold a bit to about
$0.634$, which gives 95\% confidence with 279 coin flips.)
\end{solution}

\ppart Suppose you had access to a computer program that would accept any $n \geq 0$ and $p\in[0,1]$ and 
generate, in the form of a plot or table, the full binomial probability 
density and cumulative distribution functions corresponding to 
those 
parameters.  How would you find the minimum number of coin flips needed to infer the identity of the
chosen coin with 95\% confidence?  (You do not need to determine the 
numerical value of this minimum $n$,
but we'd be interested to know if you did.)

\begin{solution}
Again, we seek to determine the values of $n$ that satisfy both~\eqref{confreqF} 
and~\eqref{confreqB}.  Using the same threshold as before, $t=\frac{5}{8}$, it is obvious 
that~\eqref{confreqF} is equivalent to
\begin{equation}\label{cdfF}
\cdf_F\paren{\frac{5}{8}n} \geq 0.95
\end{equation}
while~\eqref{confreqB} is equivalent to
\begin{equation}\label{cdfB}
\cdf_B\paren{\frac{5}{8}n} \leq 0.05
\end{equation}

Knowing that $F$ is $\paren{n,\frac{1}{2}}$-binomially distributed and 
$B$ is
$\paren{n,\frac{3}{4}}$-binomially distributed, we can use the computer 
program to
find the smallest $n$ that satisfies both~\eqref{cdfF} and~\eqref{cdfB}.
\end{solution}
\end{problemparts}
\end{problem}

%%%%%%%%%%%%%%%%%%%%%%%%%%%%%%%%%%%%%%%%%%%%%%%%%%%%%%%%%%%%%%%%%%%%%
% Problem ends here
%%%%%%%%%%%%%%%%%%%%%%%%%%%%%%%%%%%%%%%%%%%%%%%%%%%%%%%%%%%%%%%%%%%%%

\endinput

