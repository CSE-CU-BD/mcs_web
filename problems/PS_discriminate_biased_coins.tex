\documentclass[problem]{mcs}

\begin{pcomments}
  \pcomment{PS_discriminate_biased_coins}
  \pcomment{DO NOT USE: subsumed by
    PS_discriminate_biased_coins_chebyshev &
    PS_discriminate_biased_coins_chernoff}
  \pcomment{from: S06 PS10 -> S09 PS11}
\end{pcomments}

\pkeywords{
  Chernoff
  Chebyshev
  confidence
  sampling
  variance
  binomial_distribution
}

%%%%%%%%%%%%%%%%%%%%%%%%%%%%%%%%%%%%%%%%%%%%%%%%%%%%%%%%%%%%%%%%%%%%%
% Problem starts here
%%%%%%%%%%%%%%%%%%%%%%%%%%%%%%%%%%%%%%%%%%%%%%%%%%%%%%%%%%%%%%%%%%%%%

\begin{problem}
There is a fair coin and a biased coin that flips heads with
probability $3/4$.  You are given one of the coins, but you don't know
which.  To determine which coin was picked, your strategy will be to
choose a number $n$ and flip the picked coin $n$ times.  If the number
of heads flipped is closer to $\frac{3}{4} n$ than to $\frac{1}{2} n$,
you will guess that the biased coin had been picked and otherwise you
will guess that the fair coin had been picked.

\begin{problemparts}

  \ppart\label{cheb95discrim} Use the Chebyshev Bound to find a value
  $n$ so that with probability 0.95 your strategy makes the correct
  guess, no matter which coin was picked.

\begin{solution}
Let the random variable $F$ be the number of heads that would appear
in the first $n$ flips of the fair coin, and let $B$ denote the number
of heads that would appear in the first $n$ flips of the biased coin.
We must flip the coin sufficiently many times to ensure that if the
coin is biased, then with probability 0.95, the number of heads will
be closer to $\frac{3}{4} n$ than to $\frac{1}{2} n$, that is,
\begin{equation}\label{confreqB}
\Prob{B > \frac{5}{8}n} \ge 0.95\, .
\end{equation}
Similarly, if the coin is fair, then with probability 0.95, the number of heads will
not be closer to $\frac{3}{4} n$ than to $\frac{1}{2} n$, that is,
\begin{equation}\label{confreqF}
\Prob{F < \frac{5}{8}n} \ge 0.95\, .
\end{equation}

The variable $B$ has an $(n,3/4)$-binomial distribution, so its
expectation is $(3/4)n$ and its variance is $(3/4)(1-3/4)n=(3/16)n$.
Using Chebyshev's inequality for the biased coin,

\TBA{NEED TO FIX:}

\begin{align*}
\Prob{B > \frac{5}{8} n} 
& = \Prob{B - \frac{3}{4} n > -\frac{n}{8} } \\
& = \Prob{B - \expect{B} > - \frac{n}{8} }\\
&  \geq \Prob{ \abs{B - \expect{B}} > \frac{n}{8} } \\
& \leq \frac{\variance{B}}{(n/8)^2}  & \text{(by Chebyshev)}\\
  = \frac{(3/16)n}{n^2/64}
  = \frac{12}{n}\ .
\end{align*}

Now, $F$ has an $(n,1/2)$-binomial distribution, so its expectation is
$n/2$ and its variance is $n/4$.  Using Chebyshev's inequality for the
fair coin,
\begin{align*}
\Prob{F < \frac{5}{8} n} 
& = \Prob{F - \frac{n}{2} < \frac{n}{8} } \\
& = \Prob{F - \expect{F} < \frac{n}{8} }\\
& = 1-\Prob{F - \expect{F} \geq \frac{n}{8} }\\
& \geq 1-\Prob{\abs{F - \expect{F}} \geq \frac{n}{8} }\\
& \geq 1- \frac{\variance{F}}{(n/8)^2} & \text{(by Chebyshev)}\\
& = 1- \frac{n/4}{n^2/64}\\
& = 1 - \frac{16}{n}\, .
\end{align*}

So, for the required confidence level, demand that
\begin{align*}
\frac{12}{n} & \leq 0.05,
1- \frac{16}{n} & \geq 0.95.\\
\end{align*}
These hold iff $16/n \leq 0.05$, which is true iff $n \geq
320$.  So knowing the results of at least $320$ flips of the chosen
coin will allow us to guess its identity with 95\% confidence.

Because the variance of the biased coin is less than that of the fair
coin, we can do slightly better if we increase our threshold from $(5/8)n$ to
about $0.634$, which gives probability 0.95 with 279 coin flips.
\end{solution}


\ppart Now use the Chernoff Bound to estimate how large must $n$.

\begin{solution}
\TBA{REVISE AS PREVIOUS PART}

Using the same approach as part~\eqref{cheb95discrim} but with Chernoff's
Bound instead of Chebyshev's, we have
\begin{align*}
\Prob{\frac{F}{n} \geq \frac{5}{8}} 
& = \Prob{F \geq \frac{5}{4} \cdot \expect{F}}\\
& \leq e^{-\beta(5/4)\cdot (n/2)}\\
& \leq (0.986)^n
\end{align*}

The variable $B$ has an $(n,3/4)$-binomial distribution, so $C \eqdef n -
B$ has an $(n,1/4)$-binomial distribution.  Now $B \leq k$ iff $C \geq n -
k$. and $\expect{C} = n/4$.  Using Chernoff's Bound for $C$, we have
\begin{align*}
\Prob{\frac{B}{n} \leq \frac{5}{8}}
& = \Prob{B \leq \frac{5}{8}n}\\
& = \Prob{C \geq \frac{3}{8}n}\\
& = \Prob{C \geq \frac{3}{2}\expect{C}}\\
& \leq e^{-\beta(3/2)\cdot (n/4)}\\
& \leq (0.974)^n.
\end{align*}

So, for the required confidence level, we require that
\[
(0.986)^n \leq 0.05,
\]
namely, $n \geq 220$, which is a modest improvement over the estimate of
$279$ flips obtained using the Chebyshev Bound in
part~\eqref{cheb95discrim}.

\end{solution}


\ppart Suppose you had access to a computer program that would
generate, in the form of a plot or table, the full binomial-$(n,p)$
probability density and cumulative distribution functions.  How would
you find the minimum number of coin flips needed to infer the identity
of the chosen coin with probability 0.95?  (You do not need to
determine the numerical value of this minimum $n$, but we'd be
interested to know if you did.)

\begin{solution}
Again, we seek to determine the values of $n$ that satisfy
both~\eqref{confreqF} and~\eqref{confreqB}.  Using the same threshold
as before, $t=5/8$, it is obvious that~\eqref{confreqF} is equivalent
to
\begin{equation}\label{cdfF}
\cdf_F((5/8)n) \geq 0.95
\end{equation}
while~\eqref{confreqB} is equivalent to
\begin{equation}\label{cdfB}
\cdf_B((5/8)n) \leq 0.05
\end{equation}

Knowing that $F$ is $(n,1/2)$-binomially distributed and $B$ is
$(n,3/4)$-binomially distributed, we can use the computer program to
  find the smallest $n$ that satisfies both~\eqref{cdfF}
  and~\eqref{cdfB}.
\end{solution}

\end{problemparts}

\end{problem}

%%%%%%%%%%%%%%%%%%%%%%%%%%%%%%%%%%%%%%%%%%%%%%%%%%%%%%%%%%%%%%%%%%%%%
% Problem ends here
%%%%%%%%%%%%%%%%%%%%%%%%%%%%%%%%%%%%%%%%%%%%%%%%%%%%%%%%%%%%%%%%%%%%%

\endinput

