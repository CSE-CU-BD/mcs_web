\documentclass[problem]{mcs}

\begin{pcomments}
  \pcomment{PS_discriminate_biased_coins}
  \pcomment{from: S06 PS10 -> S09 PS11}
\end{pcomments}

\pkeywords{
  Chebyshev
  confidence
  sampling
  variance
  binomial_distribution
}

%%%%%%%%%%%%%%%%%%%%%%%%%%%%%%%%%%%%%%%%%%%%%%%%%%%%%%%%%%%%%%%%%%%%%
% Problem starts here
%%%%%%%%%%%%%%%%%%%%%%%%%%%%%%%%%%%%%%%%%%%%%%%%%%%%%%%%%%%%%%%%%%%%%

\begin{problem}
We have two coins: one is a fair coin, but the other
produces heads with probability $3/4$.  One of the two coins is 
picked, and this coin is tossed $n$ times.  

\begin{problemparts}

\ppart How large must $n$ be for you to be able to infer, with 95\%
confidence, which of the two coins had been chosen?  (Get close to the
minimum value of $n$ required without considering any details of the
relevant distribution functions, apart from mean and variance.)

\begin{editingnotes}
Reword to ask for calculation using Chebyshev ---ignoring other info
about binomial distribution besides mean \& variance.  Say that much
better estimates are available for using tail estimates for binomial
distributions.
\end{editingnotes}

\hint Use Chebyshev's Theorem.

\begin{solution}
To guess which coin was picked, set a threshold $t$ between $1/2$ 
and $3/4$.  If the proportion of heads is less than the threshold,
guess that the fair coin had been picked; otherwise, guess the biased coin.
Let the random variable $F$ be the number of heads that would appear in 
the first $n$ flips of the fair coin, and let $B$ denote the number of heads
that would appear in the first $n$ flips of the biased coin.  We must flip 
the coin sufficiently many times to ensure that 
\begin{equation}\label{confreqF}
\Prob{\frac{F}{n} \geq t} \leq 0.05
\end{equation}
and
\begin{equation}\label{confreqB}
\Prob{\frac{B}{n} < t} \leq 0.05
\end{equation}
A natural threshold the midpoint between $1/2$ and $3/4$, namely,
$t=5/8$.

Now, $F$ has an $(n,1/2)$-binomial distribution, so its expectation
and variance are $n/2$ and $n/4$, respectively.  Using Chebyshev's
inequality for the fair coin,
\begin{align*}
\Prob{\frac{F}{n} \geq \frac{5}{8}} 
& = \Prob{\frac{F}{n} - \frac{1}{2} \geq \frac{5}{8} - \frac{1}{2}}
  = \Prob{F - \frac{n}{2} \geq \frac{n}{8} } \\
& = \Prob{F - \expect{F} \geq \frac{n}{8} } 
  \leq \Prob{ \abs{F - \expect{F}} \geq \frac{n}{8} } \\
& \leq \frac{\variance{F}}{(n/8)^2}
  = \frac{n/4}{n^2/64}
  = \frac{16}{n}
\end{align*}

The variable $B$, on the other hand, has an $(n,3/4)$-binomial
distribution, so its expectation and variance are $n(3/4)$ and
$n(3/4)(1-3/4)=3n/16$, respectively.  Using Chebyshev's inequality for
the biased coin,
\begin{align*}
\Prob{\frac{B}{n} < \frac{5}{8}} 
& = \Prob{\frac{3}{4} - \frac{B}{n} > \frac{3}{4} - \frac{5}{8}}
  = \Prob{\frac{3n}{4} - B > \frac{n}{8} } \\
& = \Prob{\expect{B} - B > \frac{n}{8} } 
  \leq \Prob{ \abs{B - \expect{B}} \geq \frac{n}{8} } \\
& \leq \frac{\variance{B}}{(n/8)^2}
  = \frac{3n/16}{n^2/64}
  = \frac{12}{n}\ .
\end{align*}

So, for the required confidence level, demand that
\begin{align*}
\frac{16}{n} & \leq 0.05,\\
\frac{12}{n} & \leq 0.05.
\end{align*}
These hold iff $16/n \leq 0.05$, which is true iff $n \geq
320$.  So knowing the results of at least $320$ flips of the chosen
coin will allow us to guess its identity with 95\% confidence.

Because the variance of the biased coin is less than that of the fair
coin, we can do slightly better if we increase our threshold a bit to
about $0.634$, which gives 95\% confidence with 279 coin flips.
\end{solution}

\ppart Suppose you had access to a computer program that would
generate, in the form of a plot or table, the full binomial-$(n,p)$
probability density and cumulative distribution functions.  How would
you find the minimum number of coin flips needed to infer the identity
of the chosen coin with 95\% confidence?  (You do not need to
determine the numerical value of this minimum $n$, but we'd be
interested to know if you did.)

\begin{solution}
Again, we seek to determine the values of $n$ that satisfy
both~\eqref{confreqF} and~\eqref{confreqB}.  Using the same threshold
as before, $t=5/8$, it is obvious that~\eqref{confreqF} is equivalent
to
\begin{equation}\label{cdfF}
\cdf_F((5/8)n) \geq 0.95
\end{equation}
while~\eqref{confreqB} is equivalent to
\begin{equation}\label{cdfB}
\cdf_B((5/8)n) \leq 0.05
\end{equation}

Knowing that $F$ is $(n,1/2)$-binomially distributed and $B$ is
$(n,3/4)$-binomially distributed, we can use the computer program to
  find the smallest $n$ that satisfies both~\eqref{cdfF}
  and~\eqref{cdfB}.
\end{solution}
\end{problemparts}
\end{problem}

%%%%%%%%%%%%%%%%%%%%%%%%%%%%%%%%%%%%%%%%%%%%%%%%%%%%%%%%%%%%%%%%%%%%%
% Problem ends here
%%%%%%%%%%%%%%%%%%%%%%%%%%%%%%%%%%%%%%%%%%%%%%%%%%%%%%%%%%%%%%%%%%%%%

\endinput

