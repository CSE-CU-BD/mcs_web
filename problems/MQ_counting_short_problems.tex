\documentclass[problem]{mcs}

\begin{pcomments}
  \pcomment{Short questions on counting using Pigeonhole, Inclusion-exclusion,
			Combinatorial proof, Generating Functions}
\end{pcomments}

\pkeywords{
}

%%%%%%%%%%%%%%%%%%%%%%%%%%%%%%%%%%%%%%%%%%%%%%%%%%%%%%%%%%%%%%%%%%%%%
% Problem starts here
%%%%%%%%%%%%%%%%%%%%%%%%%%%%%%%%%%%%%%%%%%%%%%%%%%%%%%%%%%%%%%%%%%%%%

\begin{problem}
\begin{problemparts}
\problempart
Use the Generalized Pigeon Hole Rule to find the smallest number $n$ such that 
a set of $n$ cards must have $3$ cards of the same suit.

\examspace[1in]
\begin{solution}
TBA
\end{solution}

\problempart
Give a combinatorial proof that \\
\begin{equation*}
\binom{n}{k} = \binom{n}{n-k}
\end{equation*}

\examspace
\begin{solution}
TBA
\end{solution}

\problempart
What is the coefficient of $x^n$ in the generating function defined by the 
following expression? 
\begin{equation*}
\frac{1}{1-3x}
\end{equation*}
\examspace[1in]
\begin{solution}
TBA
\end{solution}

\problempart
Express $[x^n] 3G(x)$ in terms of $[x^n]G(x)$.
\examspace[1in]
\begin{solution}
TBA
\end{solution}
\end{problemparts}
\end{problem}

\endinput
