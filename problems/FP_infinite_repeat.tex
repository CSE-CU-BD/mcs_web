\documentclass[problem]{mcs}

\begin{pcomments}
  \pcomment{FP_infinite_repeat}
  \pcomment{compare to TP_infinite_repeat}
  \pcomment{by ARM 5/13/11}
\end{pcomments}

\pkeywords{
  infinite
  expectation
}

%%%%%%%%%%%%%%%%%%%%%%%%%%%%%%%%%%%%%%%%%%%%%%%%%%%%%%%%%%%%%%%%%%%%%
% Problem starts here
%%%%%%%%%%%%%%%%%%%%%%%%%%%%%%%%%%%%%%%%%%%%%%%%%%%%%%%%%%%%%%%%%%%%%

\begin{problem}

You have a process for generating a positive integer, $K$, such that
$\prob{K = k} = \Theta(k^{-4})$.  The integer your process generates
is (mutually) independent of all the other integers it may generate.

You use your process to generate a random integer, and then use your
procedure repeatedly until you generate an integer as big as your
first one. 

\bparts

\ppart Show that $\prob{K \geq k} = \Theta(k^{-3})$.

\examspace[4in]

\begin{solution}
Since $\prob{K = k} = \Theta(k^{-4})$, we have constants $a,b >0$ such
that
\[
ak^{-4} \leq \prob{K = k} \leq bk^{-4}
\]
for all $k \in \integers^+$.
So
\[
which proves $\prob{K \geq k} = \Theta(k^{-3})$.

\end{solution}
\eparts

Let $R$ be the number of additional integers you have to
generate.

\bparts

\ppart State and briefly explain a simple formula for
$\expcond{R}{K=k}$ in terms of $\prob{K \geq k}$.

\examspace[1in]

\begin{solution}
\[
\expcond{R}{K=k} = \frac{1}{\prob{K \geq k}}
\]

If $K=k$, then we can think of generating a number $\geq k$ as a
failure, so the expected number of repeats is the mean time to
failure, which is the reciprocal of probability of failure on a given
try.

\end{solution}

\ppart  Show that $\expect{R}$ is infinite.

\examspace[3in]

\begin{solution}
By Total Expectation,
\begin{align*}
\expect{R}
   & = \sum_{k \in \integers^+} \expcond{R}{K=k} \cdot \prob{K=k}\\
   & = \sum_{k \in \integers^+} \frac{1}{\prob{K \geq k} \cdot \prob{K=k}\\
   & = \sum_{k \in \integers^+} \Theta(k^{3}) \cdot \Theta(k^{-4})\\
   & = \sum_{k \in \integers^+} \Theta(k^{-1}) = \lim_{k \to \infty} H_k = \infty.
\end{align*}
\end{solution}

\eparts

\end{problem}

%%%%%%%%%%%%%%%%%%%%%%%%%%%%%%%%%%%%%%%%%%%%%%%%%%%%%%%%%%%%%%%%%%%%%
% Problem ends here
%%%%%%%%%%%%%%%%%%%%%%%%%%%%%%%%%%%%%%%%%%%%%%%%%%%%%%%%%%%%%%%%%%%%%

\endinput
 
