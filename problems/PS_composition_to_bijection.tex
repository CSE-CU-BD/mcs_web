\documentclass[problem]{mcs}

\begin{pcomments}
  \pcomment{PS_composition_to_bijection}
  \pcomment{by estick from: S02.ps3}
  \pcomment{total function f added by ARM 2/22/12}
\end{pcomments}

\pkeywords{
  composition
  surjection
  injection
  bijection
}

%%%%%%%%%%%%%%%%%%%%%%%%%%%%%%%%%%%%%%%%%%%%%%%%%%%%%%%%%%%%%%%%%%%%%
% Problem starts here
%%%%%%%%%%%%%%%%%%%%%%%%%%%%%%%%%%%%%%%%%%%%%%%%%%%%%%%%%%%%%%%%%%%%%

\begin{problem}

Let $f: A \to B$ and $g: B \to C$ be functions.

\iffalse
The composed function $g \circ f$ has domain $A$, range $C$, and is
defined by $(g \circ f)(a) = g(f(a))$.
\fi

\bparts
  
\ppart Prove that if the composition $g \compose f$ is a bijection,
then $f$ is a total injection and $g$ is a surjection.

\begin{solution}

\begin{proof}

\begin{staffnotes}
An explanation in terms arrows would be clearer.  Tell students it's
OK to do an ``archery'' proof, but make sure their argument is sound
and clear.
\end{staffnotes}

Suppose that $g \compose f$ is a bijection.

Then $f$ must be total, since if $f(a)$ is undefined for some $a \in
A$, then so is $g(f(a)) = (g \compose f)(a)$, contradicting the fact
that $g \compose f$ is a bijection.

Now assume for the purpose of contradiction that $f$ is not an
injection.  Then there exist \emph{distinct} elements $a_1, a_2 \in
A$, such that $f(a_1) = f(a_2)$.  This implies that $g(f(a_1)) =
g(f(a_2))$.  Therefore, $g \compose f$ is not an injection and thus
not a bijection.  This is a contradiction; therefore, $f$ must be an
injection.

Now assume for the purpose of contradiction that $g$ is not a
surjection.  Then there exists an element $c \in C$ such that for all
$b \in B$, $g(b) \neq c$.  Therefore, for all $a \in A$, $g(f(a)) \neq
c$.  This implies that $g \compose f$ is not a surjection and thus not a
bijection.  This is again a contradiction; therefore, $g$ must be a
surjection.
\end{proof}
\end{solution}

\problempart Show there is a total injection, $f$, and a bijection,
$g$, such that $g \compose f$ is not a bijection.

\begin{solution}
Here is an example:
\begin{align*}
A & = \set{ 1 } \\
B & = \set{ 1, 2 } \\
C & = \set{ 1, 2 } \\[\smallskipamount]
f(x) & \eqdef x \\
g(x) & \eqdef x
\end{align*}
\end{solution}

\eparts

\end{problem}


%%%%%%%%%%%%%%%%%%%%%%%%%%%%%%%%%%%%%%%%%%%%%%%%%%%%%%%%%%%%%%%%%%%%%
% Problem ends here
%%%%%%%%%%%%%%%%%%%%%%%%%%%%%%%%%%%%%%%%%%%%%%%%%%%%%%%%%%%%%%%%%%%%%

\endinput
