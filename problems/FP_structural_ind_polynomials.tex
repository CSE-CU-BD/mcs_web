\documentclass[problem]{mcs}

\begin{pcomments}
  \pcomment{FP_structural_ind_polynomials}
  \pcomment{overlaps FP_structural_induction_polynomial}
  \pcomment{part of CP_polynomials_produce_multiples}
  \pcomment{ARM 5/20/12}
\end{pcomments}

\pkeywords{
  structural_induction
  polynomial
  congruence 
}

%%%%%%%%%%%%%%%%%%%%%%%%%%%%%%%%%%%%%%%%%%%%%%%%%%%%%%%%%%%%%%%%%%%%%
% Problem starts here
%%%%%%%%%%%%%%%%%%%%%%%%%%%%%%%%%%%%%%%%%%%%%%%%%%%%%%%%%%%%%%%%%%%%%

\begin{problem}
The set $\aexp$ of Arithmetic Expressions in the variable $x$ was
defined recursively: expressions consisting solely of the variable $x$
or an arabic numeral $\mtt{k}$ were the base cases, and the
contructors were forming the sum $\lefbrk e_1 \sumsym e_2 \rhtbrk$,
product $\lefbrk e_1 \prodsym e_2 \rhtbrk$ or minus
$\minussym\lefbrk e_1 \rhtbrk$ of $\aexp$'s $e_1,e_2$.  Then the value
$\meval{e}{n}$ of an $\aexp$, $e$ when the variable $x$ is equal to the
integer $n$ has an immediate recursive definition based on the
definition of $\aexp$'s.

Prove by structural induction that for all $\aexp$'s $e$, and integer
$m,n,d$ with $d>1$,,
\begin{equation}\label{evalmodindhyp}
[m \equiv n \pmod d]\quad \QIMP\quad [\meval{e}{m} \equiv \meval{e}{n} \pmod d].
\end{equation}

\hint Be sure to consider \textbf{both base cases}.  The proofs for
the three constructors are very similar, so \textbf{just write out the
  case for the sum constructor}.

\begin{solution}
The proof is by structural induction on the definition of $e \in \aexp$.
The hypothesis $P(e)$ is given by~\eqref{evalmodindhyp}.

\inductioncase{Base case} ($e$ is $x$).  Then $\meval{e}{m}
  \eqdef m, \meval{e}{n} \eqdef n,$ so~\eqref{evalmodindhyp} reduces
  to the tautology
\[
[m \equiv n \pmod d]\quad \QIMP\quad [m \equiv n  \pmod d].
\]

\inductioncase{Base case} ($e$ is $\mtt{k}$).  Then letting $k
  \in \integers$ be the number represented by the numeral $\mtt{k}$,
\[
\meval{e}{m} = \meval{e}{n} \eqdef k
\]
and again so~\eqref{evalmodindhyp} reduces to the trivial assertion
\[
k \equiv k  \pmod d.
\]

\inductioncase{Constructor case} ($g \eqdef \lefbrk e \sumsym f
\rhtbrk$): To prove $P(g)$, suppose $m \equiv n \pmod d$.  We may
assume by induction that $P(e)$ and $P(f)$ hold, which implies that
\begin{align}
\meval{e}{m} & \equiv \meval{e}{n} \pmod d,\label{emeend}\\
\meval{f}{m} & \equiv \meval{f}{n} \pmod d.\label{fmefnd}\\
\end{align}
Now
\begin{align*}
\meval{g}{m}
 & = \meval{\lefbrk e \sumsym f \rhtbrk}{m}\\
 & = \meval{e}{m} + \meval{f}{m}
     & \text{(def of $\meval{\cdot}{m}$)}\\
 & \equiv \meval{e}{n} + \meval{f}{n} \pmod d
     & \text{(by~\eqref{emeend},~\eqref{fmefnd}
       since sums preserve $\equiv \bmod d$)}\\
 & = \meval{\lefbrk e \sumsym f \rhtbrk}{n}
     & (\text{def of $\meval{\cdot}{n}$)}\\
 & = \meval{g}{n}.
\end{align*}
That is, $P(g)$ holds, and we have completed the proof of this
constructor case.
\end{solution}

\end{problem}

%%%%%%%%%%%%%%%%%%%%%%%%%%%%%%%%%%%%%%%%%%%%%%%%%%%%%%%%%%%%%%%%%%%%%
% Problem ends here
%%%%%%%%%%%%%%%%%%%%%%%%%%%%%%%%%%%%%%%%%%%%%%%%%%%%%%%%%%%%%%%%%%%%%

\endinput
