%TP_mating_ritual_invariant

\documentclass[problem]{mcs}

\begin{pcomments}

  \pcomment{from F07.miniquiz-oct12 edited by ARM 10/8/09}
\end{pcomments}

\pkeywords{
 stable_matching
 Mating_ritual
 invariant
 }


%%%%%%%%%%%%%%%%%%%%%%%%%%%%%%%%%%%%%%%%%%%%%%%%%%%%%%%%%%%%%%%%%%%%%
% Problem starts here
%%%%%%%%%%%%%%%%%%%%%%%%%%%%%%%%%%%%%%%%%%%%%%%%%%%%%%%%%%%%%%%%%%%%%

\begin{problem} Suppose that Harry is one of the boys and Alice is one of
  the girls in the \emph{Mating Ritual}.  Which of the properties below
  are preserved invariants?  Why?

\begin{itemize}

\item[a.] Alice is the only girl on Harry's list.

\item[b.] There is a girl who does not have any boys serenading her.

\item[c.] If Alice is not on Harry's list, then Alice has a suitor that
  she prefers to Harry.

%\item[d.] All the boys have the same number of girls left uncrossed in their list. 

\item[d.] Alice is crossed off Harry's list and Harry prefers Alice to anyone
  he is serenading.

\item[e.] % obscure & probably too tricky

If Alice is on Harry's list, then she prefers to Harry to any suitor
  she has.

\end{itemize}

\begin{solution}

The 1st, 3rd, and 4th are preserved invariants.

\begin{enumerate}

\item[a.] Invariant; no girl will be added to Harry's list. If Alice got
crossed off, there would be no one for Harry to marry.  So she must
remain as the sole girl on his list.

\item[b.] Not invariant; a girl may not have a suitor on the first day, ---if,
for example, she's not at the top of any boy's list ---but every girl is
guaranteed to have one at the end, namely, her husband.

\item[c.] Invariant; this is the basic invariant used to verify
the Ritual.

\item[d.] Invariant; Harry crosses off the girls in his order of
preference, so if Alice is crossed off, Harry likes her better than
anybody that's left.

\item[e.] Not invariant.  Suppose the preference among two couples are

\begin{tabular}{llll}
Harry:    & Alice,   & Elvira,   & \dots \\
Billy:    & Elvira,  & Alice,    & \dots \\
Wilfred:  & Elvira,  & \dots             \\
Alice:    & Billy,   & Harry,    & \dots \\
Elvira:   & Wilfred, & Billy,    & \dots
\end{tabular}

The alleged invariant is true on the first day since Harry is Alice's only
suitor.  But Elvira rejects Billy in favor of Wilfred on the first
afternoon, so on the second day, Billy and Harry are serenading Alice, and
since Alice prefers Billy to Harry, the alleged invariant is no longer true.

\end{enumerate}

\end{solution}

\end{problem}

\endinput
