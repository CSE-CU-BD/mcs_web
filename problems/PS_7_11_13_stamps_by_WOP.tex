\documentclass[problem]{mcs}

\begin{pcomments}
  \pcomment{PS_7_11_13_stamps_by_WOP}
  \pcomment{variation of PS_6_10_15_postage_by_WOP}
  \pcomment{ARM 2/3/17}
\end{pcomments}

\pkeywords{
  WOP
  well_ordering
  postage
  stamps
}

%%%%%%%%%%%%%%%%%%%%%%%%%%%%%%%%%%%%%%%%%%%%%%%%%%%%%%%%%%%%%%%%%%%%%
% Problem starts here
%%%%%%%%%%%%%%%%%%%%%%%%%%%%%%%%%%%%%%%%%%%%%%%%%%%%%%%%%%%%%%%%%%%%%

\begin{problem}
Use the Well Ordering Principle to prove that any integer greater than
or equal to 50 can be represented as the sum of nonnegative integer
multiples of 7, 11, and 13.

\inhandout{ \hint Use the template for WOP proofs to ensure partial
  credit.  Verify that integers in the interval $\Zintv{50}{55}$ are
  sums of nonnegative integer multiples of 7, 11, and 13.  }

\begin{solution}
\begin{claim*}
For all $n \geq 50$, it is possible to represent $n$ as a sum of nonnegative integer
multiples of 7, 11 and 13.
\end{claim*}

\begin{proof}

  The proof is by the Well Ordering Principle.  Let $P(n)$ be the
  predicate that $n$ is a sum of nonnegative integer multiples of 7,
  11 and 13.

  Let $C \eqdef \set{ n \geq 30 | \QNOT(P(n))}$ be the set of counter
  examples.   Assume for the sake of contradiction that $C$ is not
  empty.  Then by the Well Ordering Principle, $C$ must have some minimum
  element $m\in C$.

  First, observe that $P(n)$ is true for the values of $n \in \Zintv{50}{55}$.
  \begin{itemize}
  \item $n=50$: $11 + 3 \cdot 13$.

  \item $n=51$: $2 \cdot 7 + 11 + 2 \cdot 13$.

  \item $n=52$: $4 \cdot 7 + 11 + 13$.

  \item $n=53$: $6 \cdot 7  + 11$.

  \item $n=54$: $3 \cdot 7  + 3 \cdot 11$.

  \item $n=55$: $5 \cdot 11$.

  \item $n=56$: $8 \cdot 7$.
  \end{itemize}

  We thus have $m \geq 57$.  Since $m$ is the smallest counterexample,
  $m - 7$ is not a counterexample.  Since $m-7 \geq 50$, it follows that
  $P(m-7)$ holds, that is, $m-7$ as the sum of nonnegative integer
  multiples of 7, 11, and 13.  Thus we can represent $m$ by adding 1
  to the coefficient of 7 in our representation of $m-7$.  This shows
  that $m$ is not actually a counterexample, contradicting the
  assumption that $C$ is nonempty.
  \end{proof}
  \end{solution}

\end{problem}

%%%%%%%%%%%%%%%%%%%%%%%%%%%%%%%%%%%%%%%%%%%%%%%%%%%%%%%%%%%%%%%%%%%%%
% Problem ends here
%%%%%%%%%%%%%%%%%%%%%%%%%%%%%%%%%%%%%%%%%%%%%%%%%%%%%%%%%%%%%%%%%%%%%

\endinput
