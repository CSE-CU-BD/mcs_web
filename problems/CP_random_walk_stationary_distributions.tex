\documentclass[problem]{mcs}

\begin{pcomments}
  \pcomment{CP_random_walk_stationary_distributions}
  \pcomment{from: S08.cp13m}
\end{pcomments}

\pkeywords{
  random_walk
  stationary_distributions
}

%%%%%%%%%%%%%%%%%%%%%%%%%%%%%%%%%%%%%%%%%%%%%%%%%%%%%%%%%%%%%%%%%%%%%
% Problem starts here
%%%%%%%%%%%%%%%%%%%%%%%%%%%%%%%%%%%%%%%%%%%%%%%%%%%%%%%%%%%%%%%%%%%%%

\begin{problem}
Consider the following random-walk graph:

\mfigure{!}{1in}{figures/randomWalkFigs/bipart} 

\bparts
\ppart Find a stationary distribution.
\begin{solution}
$d(x) = d(y) = 1/2$
\end{solution}

\ppart If you start at node $x$ and take a (long) random walk, does the
distribution over nodes ever get close to the stationary distribution?
Explain.

\begin{solution}
No! you just alternate between nodes $x$ and $y$.
\end{solution}

\eparts

Consider the following random-walk graph:

\mfigure{!}{1in}{figures/randomWalkFigs/stable}

\bparts
\ppart Find a stationary distribution.

\begin{solution}
$d(w) = 9/19$, $d(z) = 10/19$.  You can derive this  by
  setting $d(w) = (9/10)d(z)$, $d(z) = d(w) + (1/10)d(z)$, and $d(w) +
  d(z) = 1$.  There is a unique solution.
\end{solution}

\ppart If you start at node $w$ and take a (long) random walk, does the
distribution over nodes ever get close to the stationary distribution?  We
don't want you to prove anything here, just write out a few steps and see
what's happening.
\begin{solution}
Yes, it does.
\end{solution}

\eparts

Consider the following random-walk graph:

\mfigure{!}{1in}{figures/randomWalkFigs/sinky}

\bparts
\ppart Is there more than one stationary distribution? How many are
there?

\begin{solution}
Yes, there are infinitely many.  Just set $d(b)=d(c)=0$, $d(a)
  = p$ for any $p$, and $d(d) = 1-p$.
\end{solution}

\ppart If you start at node $b$ and take a long random walk, the
probability you are at node $d$ will be close to what fraction?
You needn't prove anything here, just do a few steps to explain what's happening.

\begin{solution}
1/3
\end{solution}

\eparts
\end{problem}

\endinput
