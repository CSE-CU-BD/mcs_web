\documentclass[problem]{mcs}

\begin{pcomments}
\pcomment{TP_injR_surjinvR}
\pcomment{small part of CP_surj_relation}
\pcomment{S13, miniqFeb25-afternoon}
\pcomment{ARM 3/21/13}
\end{pcomments}

\pkeywords{
  relations
  functions
  injections
  surjections
}

%%%%%%%%%%%%%%%%%%%%%%%%%%%%%%%%%%%%%%%%%%%%%%%%%%%%%%%%%%%%%%%%%%%%%
% Problem starts here
%%%%%%%%%%%%%%%%%%%%%%%%%%%%%%%%%%%%%%%%%%%%%%%%%%%%%%%%%%%%%%%%%%%%%

\begin{problem}
Prove that $\paren{A \inj B}$ implies $\paren{B \surj A}$.  It is
fine---even encouraged---to phrase things in terms of ``arrows-in'' and
``arrows-out.''
\end{problem}
\begin{solution}

\begin{proof}
By definition of $\inj$, there is a total injective $[\geq
  1\ \text{out}, \leq 1\ \text{in}]$ relation $R:A \to B$.  Reversing
direction of the arrows, we have $\inv{R}: B \to A$ is a $[\leq
  1\ \text{out}, \geq 1\ \text{in}]$ surjective function, so $B \surj
  A$ by definition of $\surj$.
\end{proof}

\end{solution}


%%%%%%%%%%%%%%%%%%%%%%%%%%%%%%%%%%%%%%%%%%%%%%%%%%%%%%%%%%%%%%%%%%%%%
% Problem ends here
%%%%%%%%%%%%%%%%%%%%%%%%%%%%%%%%%%%%%%%%%%%%%%%%%%%%%%%%%%%%%%%%%%%%%

\endinput


