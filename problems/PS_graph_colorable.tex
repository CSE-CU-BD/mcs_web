\documentclass[problem]{mcs}

\begin{pcomments}
  \pcomment{PS_graph_colorable}
  \pcomment{from: S07, pset5-1}
\end{pcomments}

\pkeywords{
 graph theory
 colorable
 width
 handshaking
 average_degree 
}

\begin{problem}
  This problem generalizes the result proved Theorem~\bref{k+1-colorable}
  that any graph with maximum degree at most $w$ is $(w+1)$-colorable.

A simple graph, $G$, is said to have \hyperdef{psfive}{width}{\term{width}},
$w$, iff its vertices can be arranged in a sequence such that each vertex
is adjacent to at most $w$ vertices that precede it in the sequence.  If
the degree of every vertex is at most $w$, then the graph obviously has
width at most $w$ ---just list the vertices in any order.

\bparts

\ppart Describe an example of a graph with 100 vertices, width 3, but
\emph{average} degree more than 5.  \hint Don't get stuck on this; if you
don't see it after five minutes, ask for a hint.

\begin{solution} The hint is to line up the 100 vertices and have each vertex be
adjacent to the 3 immediately preceding vertices, if any.  By definition
of width, this graph has width 3.  All vertices other than the first three
are now adjacent to three preceding vertices, and all vertices except the
last three are also adjacent to the three following vertices.  So vertices
4 through 97 all have degree 6; this alone ensures that the average degree
is at least $6\cdot 94/100 = 5.76$.
\end{solution}

\ppart Prove that every graph with width at most $w$ is $(w +
1)$-colorable.

\begin{solution} We use induction on $n$, the number of vertices.  Let $P(n)$ be
the proposition that for all $w$, every $n$-vertex graph with width $w$ is
$(w+1)$-colorable.

\textbf{Base case:} ($n=1$) Every graph with $1$ vertex has width 0 and is
$0 + 1 = 1$ colorable.  Therefore, $P(1)$ is true.

\textbf{Inductive step:} Now we assume $P(n)$ in order to prove $P(n+1)$.
Let $G$ be an $(n+1)$-vertex graph with width at most $w$.  This means
that the $n+1$ vertices can be arranged in a sequence, $S$, such that each
vertex is connected to at most $w$ preceding vertices.  Removing the last
vertex, $v$, and all edges incident to it gives a subgraph $G'$ with $n$
vertices.  The subgraph $G'$ also has width at most $w$, since the
sequence $S$ with its last vertex removed is a sequence of all the
vertices of $G'$ with each vertex adjacent to exactly the same previous
vertices.  So by Induction Hypothesis, $G'$ is $(w+1)$-colorable.  But any
$(w+1)$-coloring of $G'$ can be extended to a $(w+1)$-coloring of $G$ by
assigning a color to $v$ that differs from the colors of its adjacent
vertices.  Since there are at most $w$ colors among the $w$ vertices
adjacent to $v$, there will always be a different one of the $w+1$ colors
to assign to $v$.  So $G$ is $(w+1)$-colorable, which proves $P(n+1)$.
This completes the proof of the Induction step.

The result now follows for all $G$ by the Principle of Induction.
\end{solution}

\ppart Prove that the average degree of a graph of width $w$ is at most
$2w$.

\begin{solution} 
If we line up the vertices, we can define the \emph{backdegree} of a
vertex to be the number of preceding vertices it is adjacent to.  The sum
of the back degrees equals the number, $e$, of edges.  Since
there is a sequence in which all the back degrees are at most $w$, the
total number of edges is at most $w$ times the number, $n$, of vertices.
But by the Handshaking Lemma, the sum of all the degrees is $2e$,
so the average degree is $2e/n \leq 2wn/n = 2w$.
\end{solution}

\eparts

\end{problem}
\endinput
