\documentclass[problem]{mcs}

\begin{pcomments}
  \pcomment{CP_random_walk_stationary_distributions_F13}
  \pcomment{subsumes CP_random_walk_stationary_distributions}
  \pcomment{from: S08.cp13m}
\end{pcomments}

\pkeywords{
  random_walk
  stationary_distributions
}

%%%%%%%%%%%%%%%%%%%%%%%%%%%%%%%%%%%%%%%%%%%%%%%%%%%%%%%%%%%%%%%%%%%%%
% Problem starts here
%%%%%%%%%%%%%%%%%%%%%%%%%%%%%%%%%%%%%%%%%%%%%%%%%%%%%%%%%%%%%%%%%%%%%

\begin{problem}

\bparts

\ppart
Find a stationary distribution for the random walk graph in Figure~\ref{fig:bipart}.

\begin{solution}
$d(x) = d(y) = 1/2$
\end{solution}

\begin{figure}[h]
\graphic[height=1in]{randomWalkFigs/bipart}
\caption{}
\label{fig:bipart}
\end{figure}

\ppart Explain why a long random walk starting at node $x$ in
Figure~\ref{fig:bipart} will not converge to a stationary
distribution.  Characterize which starting distributions will converge
to the stationary one.

\begin{solution}
A starting distribution $(p, 1-p)$ will oscillate between itself and
$(p, 1-p)$, so it will converge to a single distribution iff $p =
1/2$.  That is, the stationary distribution is the \emph{only} initial
distribution that converges to a stationary distribution.
\end{solution}

\ppart Find a stationary distribution for the random walk graph in
Figure~\ref{fig:stable}.

\begin{figure}[h]
\graphic[height=1in]{randomWalkFigs/stable}
\caption{}
\label{fig:stable}
\end{figure}

\begin{solution}
$d(w) = 9/19$, $d(z) = 10/19$.  You can derive this by setting $d(w) =
  (9/10)d(z)$, $d(z) = d(w) + (1/10)d(z)$, and $d(w) + d(z) = 1$.
  There is a unique solution.
\end{solution}

\ppart If you start at node $w$ Figure~\ref{fig:stable} and take a
(long) random walk, does the distribution over nodes ever get close to
the stationary distribution?  You needn't prove anything
here, just write out a few steps and see what's happening.

\begin{solution}
Yes, it does.  The graph in Figure~\ref{fig:stable} is strongly
connected, and as mentioned in Section~\bref{stationary_sec} and
proved in Problem~\bref{PS_random_walk_strongly_connected}, strongly
connected graphs have \emph{unique} stationary distributions.
\end{solution}

\ppart Explain why the random walk graph in Figure~\ref{fig:sinky} has
an uncountable number of stationary distributions.

\begin{figure}[h]
\graphic[height=1in]{randomWalkFigs/sinky}
\caption{}
\label{fig:sinky}
\end{figure}

\begin{solution}
Any distribution with $d(b)=d(c)=0$, and $d(a) = p$ and $d(d) = 1-p$
is stable, and their are uncountable many $p$ in the unit interval
$[0,1]$.
\end{solution}

\ppart If you start at node $b$ in Figure~\ref{fig:sinky} and take a
long random walk, the probability you are at node $d$ will be close to
what fraction?  Explain.

\begin{solution}
\textbf{1/3}.

If you ever arrive at $d$ you will stay there, so consider all walks
from $b$ to $d$.  They are $bcd$, $bcbcd= (bc)^2d, (bc)^3d$, \dots,
with probabilities, $1/4, (1/4)^2, (1/4)^3,\dots$.  So the probability
of these walks 1/4 times the geometric sequence $\sum_0^\infty
(1/4)^n$, namely, (1/4)[1/(1-(1/4))] = 1/3.

An easy, alternative argument uses the symmetry of the graph.  Let $x$
be the probability that starting at node $b$ you wind up stuck in $d$.
The only way to get to $d$ from $b$ is through $c$, so the probability
of getting stuck at $d$ starting from $b$ is the probability of moving
to $c$, namely $1/2$, times the probability of getting stuck at $d$
starting at $c$.  Now by symmetry, the probability that starting at
node $c$ you wind up stuck in $a$ is also $x$.  Since the probability
of never getting stuck at either $a$ or $d$ is obviously 0, the
probability that starting at node $c$ you wind up stuck at $d$ must be
$1-x$.  Therefore,
\[
x = \frac{1}{2}(1-x),
\]
and so $x = 1/3$.
\end{solution}

\ppart

\eparts
\end{problem}

\endinput
