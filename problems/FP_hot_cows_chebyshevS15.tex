\documentclass[problem]{mcs}

\begin{pcomments}
  \pcomment{FP_hot_cows_chebyshevS15}
  \pcomment{subsumes FP_hot_cows_chebyshev}
  \pcomment{variant of FP_hot_cows_markov}
  \pcomment{S17.final, S14.final, commented out of S15 final}
  \pcomment{ARM 5/18/14}
\end{pcomments}

\pkeywords{
  average
  probability
  Chebyshev
  deviation
  variance
  sample_space
  outcome
  probability_space
}

%%%%%%%%%%%%%%%%%%%%%%%%%%%%%%%%%%%%%%%%%%%%%%%%%%%%%%%%%%%%%%%%%%%%%
% Problem starts here
%%%%%%%%%%%%%%%%%%%%%%%%%%%%%%%%%%%%%%%%%%%%%%%%%%%%%%%%%%%%%%%%%%%%%

\begin{problem}

\begin{staffnotes}
\textbf{S17 final 10pts: part(a) 3pts, parts(b,c) 2pts each, part(d) 3pts}
\end{staffnotes}


There is a herd of cows whose average body temperature turns out to be
$100$ degrees.  Our thermometer produces such sensitive readings that
no two cows have exactly the same body temperature.  The herd is
stricken by an outbreak of \emph{wacky cow disease}, which will
eventually kill any cow whose body temperature differs from the
average by $10$ degrees or more.

It turns out that the \emph{collection-variance} of all the body
temperatures is $20$, where the \emph{collection-variance}
$\text{CVar}(A)$ of set $A$ of numbers is
\[
\text{CVar}(A) \eqdef \frac{\sum_{a \in A} (a - \mu)^2}{\card{A}},
\]
where $\mu$ is the average value of the numbers in
$A$.\footnote{$\text{CVar}(A)$ is called $A$'s \emph{mean square
    deviation}.}

\bparts

\ppart\label{2/10cows}
Apply the Chebyshev bound to the temperature $T$ of a random cow to
show that at most 20\% of the cows will be killed by this disease
outbreak.

\examspace[2in]

\begin{solution}
Let $A$ be the set of body temperatures of the herd.  Let $T$ be the
temperature of a random cow.  Then
\begin{align*}
\prob{ \abs{T - 100} \geq 10}
  & \leq \frac{\variance{T}}{10^2} &\text{(Chebyshev's bound)} \\
  & = \frac{20}{100} .
\end{align*}

So at most 20\% of the herd can have a temperature that differs from
the average by as much as 10 degrees.  The justification for this
claim is given more carefully in the next parts.
\end{solution}
\eparts

\medskip
The conclusion of part~\eqref{2/10cows} about a certain fraction of
the herd was derived by bounding the deviation of a random variable.
We can justify this approach by explaining how to define a suitable
probability space in which, the temperature $T$ of a cow is a random
variable.

\bparts
\ppart Carefully specify the probability space on which $T$ is
defined: what are the outcomes?  what are their probabilities?

\examspace[1in]

\begin{solution}
The outcomes are the cows in the herd, with uniform distribution.

Alternatively, the outcomes could be chosen to be the temperatures,
since our sensitive thermometer ensures there is a bijection between
cows and temperatures.
\end{solution}

\ppart Explain why for this probability space, the fraction of cows
with any given cow property $P$ is the same as $\pr{P}$.

\examspace[2.0in]

\begin{solution}
\[
\pr{P} \eqdef \sum_{\omega \in P} \pr{\omega} =
\frac{\card{P}}{\card{\sspace}} = \text{fraction of herd with propert
  $P$}
\]
\end{solution}

\instatements{
\begin{center}
\textbf{(CONTINUED ON NEXT PAGE)}
\end{center}
}

\examspace

\iffalse
\examspace[1in]

\ppart Show that $\expect{T}$ equals the average temperature of the herd.
\fi

\ppart Show that $\variance{T}$ equals the collection variance of the
temperatures in the herd.

\examspace[3in]

\begin{solution}
Since the probabilities are uniform, $\expect{T}$ is the average
temperature $\mu$.

Then
\begin{align*}
\variance{T} & \eqdef \expectsq{T-\mu}\\
 & = \sum_{\text{cows} \in \text{herd}} \pr{\text{cow}}(T(\text{cow}) - \mu)^2\\
 & = \frac{\sum_{\text{cows} \in \text{herd}} (T(\text{cow}) - \mu)^2}{\card{\text{herd}}}
      & (\text{uniform distribution on herd})\\
 & = \frac{\sum_{t \in  \text{cow temperatures}} (t - \mu)^2}{\card{\text{cow temperatures}}}
      & \text{(each cow has unique temperature)}\\
 & =::\, \text{CVar}(\text{cow temperatures}).
\end{align*}
\end{solution}


\eparts

\end{problem}

%%%%%%%%%%%%%%%%%%%%%%%%%%%%%%%%%%%%%%%%%%%%%%%%%%%%%%%%%%%%%%%%%%%%%
% Problem ends here
%%%%%%%%%%%%%%%%%%%%%%%%%%%%%%%%%%%%%%%%%%%%%%%%%%%%%%%%%%%%%%%%%%%%%

\endinput
