\documentclass[problem]{mcs}

\begin{pcomments}
  \pcomment{FP_hot_cows_chebyshevS15}
  \pcomment{variant of FP_hot_cows_chebyshev, FP_hot_cows_markov}
  \pcomment{ARM 5/18/14}
\end{pcomments}

\pkeywords{
  average
  probability
  Chebyshev
  deviation
  variance
  sample_space
  outcome
}

%%%%%%%%%%%%%%%%%%%%%%%%%%%%%%%%%%%%%%%%%%%%%%%%%%%%%%%%%%%%%%%%%%%%%
% Problem starts here
%%%%%%%%%%%%%%%%%%%%%%%%%%%%%%%%%%%%%%%%%%%%%%%%%%%%%%%%%%%%%%%%%%%%%

\begin{problem}
There is a herd of cows whose average body temperature turns out to be
$100$ degrees.  Our thermometer produces such sensitive readings that
no two cows have exactly the same body temperature.  The herd is
stricken by an outbreak of \emph{wacky cow disease}, which will
eventually kill any cow whose body temperature differs from the
average by $10$ degrees or more.

It turns out that the \emph{collection-variance} of all the body
temperatures is $20$, where the \emph{collection-variance}
$\text{CVar}(A)$ of set $A$ of numbers is
\[
\text{CVar}(A) \eqdef \frac{\sum_{a \in A} (a - \mu)^2}{\card{A}},
\]
where $\mu$ is the average value of the numbers in $A$.  (In other
words, $\text{CVar}(A)$ is $A$'s average square deviation from its
mean.)

\bparts

\ppart\label{2/10cows}
Apply the Chebyshev bound to the temperature $T$ of a random cow to
show that at most 20\% of the cows will be killed by this disease
outbreak.

\examspace[2in]

\begin{solution}
Let $A$ be the set of body temperatures of the herd.  Let $T$ be the
temperature of a random cow.  Then
\begin{align*}
\prob{ \abs{T - 100} \geq 10}
  & \leq \frac{\variance{T}}{10^2} &\text{(Chebyshev's bound)} \\
  & = \frac{20}{100} .
\end{align*}

So at most 20\% of the herd can have a temperature that differs from
the average by as much as 10 degrees.  The justification for this
claim is given more carefully in the next parts.
\end{solution}
\eparts

\mdeskip
The conclusion of part~\eqref{2/10cows} is a bound on a certain
fraction of the herd and was derived by bounding the deviation of a
random variable.  We can justify this approach by explaining how to
define a probability space in which, the temperature, $T$, of a cow is
a random variable.

\ppart Carefully specify the probability space on which $T$ is
defined: what are the outcomes?  what are their probabilities?

\examspace[1in]

\ppart Explain why for this probability space, the fraction of cows
with any given cow property, $P$, is the same as $\pr{P}$.

\examspace[1in]

\ppart Show that $\expect{T}$ equals the average temperature of the herd.

\examspace[1in]

\ppart Show that $\variance{T}$ equals the collection variance of the herd.

\examspace[2in]

\begin{solution}
The sample space for $T$ is the set of cows in the herd, that is, each
cow is an outcome.  (Alternatively, the outcomes could be chosen to be
the temperatures, since our sensitive thermometer ensures there is a
bijection between cows and temperatures.)  The probabilities are
defined to be \emph{uniform}---the probability of any outcome, $c$, is
$1/n$ where $n$ is the size of the herd---and $T(c)$ is the
temperature of cow $c$. 

So the fact that $\prob{\abs{T - 100} \geq 10} \leq 0.2$ implies that
at most 20\% of the herd will die from the outbreak.
\end{solution}
\eparts

\end{problem}

%%%%%%%%%%%%%%%%%%%%%%%%%%%%%%%%%%%%%%%%%%%%%%%%%%%%%%%%%%%%%%%%%%%%%
% Problem ends here
%%%%%%%%%%%%%%%%%%%%%%%%%%%%%%%%%%%%%%%%%%%%%%%%%%%%%%%%%%%%%%%%%%%%%

\endinput
