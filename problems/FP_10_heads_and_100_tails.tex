\documentclass[problem]{mcs}

\begin{pcomments}
  \pcomment{FP_10_heads_and_100_tails}
  \pcomment{original version of MQ_red_blue_machine}
\end{pcomments}

\pkeywords{
  state_machines
  increasing_decreasing_variables
}

%%%%%%%%%%%%%%%%%%%%%%%%%%%%%%%%%%%%%%%%%%%%%%%%%%%%%%%%%%%%%%%%%%%%%
% Problem starts here
%%%%%%%%%%%%%%%%%%%%%%%%%%%%%%%%%%%%%%%%%%%%%%%%%%%%%%%%%%%%%%%%%%%%%

\begin{problem}
Start with 110 coins on a table, 10 showing heads and 100 showing tails.

There are two ways to change the coins:

\begin{enumerate}
\item[(i)] Remove $20$ coins from the table, $10$ of which must be 
  \emph{heads} and the other $10$ must be \emph{tails}, or
\item[(ii)] Let $n$ be the number of heads showing. If there are 
  \emph{more tails than heads} on the table, place $n$ additional coins,
  all showing heads, on the table.
\end{enumerate}

\bparts

\ppart
Model this situation as a state machine, carefully defining the set of
states, the start state and the possible state transitions.
\hint Be sure to state the conditions of the state transitions.

\vspace{1.5in}

\ppart
For each of the derived variables below, indicate the \emph{strongest} 
of the properties that it satisfies. 
(Let $H \eqdef \text{the number of heads}$ and $T \eqdef \text{the number of heads}$).

The choices for properties are: \emph{constant}, \emph{strictly
increasing}, \emph{strictly decreasing}, \emph{weakly increasing},
\emph{weakly decreasing}, \emph{none of these}.

% \renewcommand{\labelenumi}{(\roman{enumi})}
\begin{enumerate}

% \item $H$
% \vspace{0.5in}

\item $T$
\vspace{0.5in}

\item $H+T$
\vspace{0.5in}

\item $T-H$
\vspace{0.5in}

\item $2T-H$
\vspace{0.5in}

\end{enumerate}
% 
% \ppart
% Prove that any long enough sequence of transitions will arrive at a 
% state in which no transition is possible: by providing a strictly decreasing
% derived variable and proving that it has a minimum value.
% 
% 
% \begin{solution}
% TBA
% \end{solution}
% 
\eparts
\end{problem}

%%%%%%%%%%%%%%%%%%%%%%%%%%%%%%%%%%%%%%%%%%%%%%%%%%%%%%%%%%%%%%%%%%%%%
% Problem ends here
%%%%%%%%%%%%%%%%%%%%%%%%%%%%%%%%%%%%%%%%%%%%%%%%%%%%%%%%%%%%%%%%%%%%%

\endinput
