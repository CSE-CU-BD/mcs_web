\documentclass[problem]{mcs}

\begin{pcomments}
  \pcomment{CP_consecutive_coin_flips}
  \pcomment{from: S09.cp14m}
\end{pcomments}

\pkeywords{
  expectation
  total_expectation
}

%%%%%%%%%%%%%%%%%%%%%%%%%%%%%%%%%%%%%%%%%%%%%%%%%%%%%%%%%%%%%%%%%%%%%
% Problem starts here
%%%%%%%%%%%%%%%%%%%%%%%%%%%%%%%%%%%%%%%%%%%%%%%%%%%%%%%%%%%%%%%%%%%%%

\begin{problem}

\bparts

\ppart Suppose we flip a fair coin and let $N_\mathtt{TT}$ be the
number of flips until the first time two consecutive Tails appear.  What
is $\expect{N_\mathtt{TT}}$?

\hint Let $D$ be the tree diagram for this process.  Explain why
$D$ can be described by the tree in Figure~\ref{tree_D}.
\iffalse
\[
D = H\cdot D + T\cdot(H\cdot D + T).
\]
\fi
Use the \textbf{Law of Total Expectation}%
\inbook{~\bref{total_expect}.}%
\inhandout{: Let $R$~be a random variable and $A_1$, $A_2$, \dots, be
  a partition of the sample space.  Then
\[
    \expect{R} = \sum_i \expcond{R}{A_i} \prob{A_i}.
\]
}

\begin{figure}
\graphic{tree_D}
  \caption{Sample space tree for coin toss until two consecutive heads.}
  \label{tree_D}
\end{figure}


\begin{staffnotes}
Ask what's wrong with the following argument: \emph{the probability of
  flipping two heads in a row in 1/4, so by the mean time to failure
  rule, $\expect{N_\mathtt{TT}} = 1/(1/4) = 4$, contradicting the
  value 6 derived above.}

The answer is that MTF applies only if the probability of failure at
each step is the same \emph{independent} of the previous flips, but
that is obviously not true here, since the probability is twice as
large if the previous flip was $\mathtt{T}$ than if it was
$\mathtt{H}$.
\end{staffnotes}

\begin{solution}
%\[
%\expect{N_\mathtt{TT}}=6.
%\]

Let $H$ be the event that a Head appears on the first flip, $TH$ the
event that the first flips are Tail then Head, and likewise $TT$.
From $D$ and the Law of Total Expectation:

\begin{align*}
\expect{N_\mathtt{TT}}
    & = \expcond{N_\mathtt{TT}}{H}    \cdot \pr{H}
       + \expcond{N_\mathtt{TT}}{TH}  \cdot \pr{TH}
       + \expcond{N_\mathtt{TT}}{TT}  \cdot \pr{TT}\\ 
  & = \paren{1 + \expect{N_\mathtt{TT}}}\cdot \frac{1}{2} +
      \paren{2 + \expect{N_\mathtt{TT}}}\cdot \frac{1}{4} +
       2 \cdot \frac{1}{4}\\
  & = \frac{1}{2} + \frac{\expect{N_\mathtt{TT}}}{2}
       + \frac{1}{2} + \frac{\expect{N_\mathtt{TT}}}{4}
       + \frac{1}{2}\\
  & = \frac{3}{2} + \frac{3\expect{N_\mathtt{TT}}}{4}
\end{align*}
So
\[
\expect{N_\mathtt{TT}} = \frac{3}{2} \cdot 4 = 6.
\]
\end{solution}


\ppart Let $N_\mathtt{TH}$ be the number of flips until a Tail
immediately followed by a Head comes up.  What is
$\expect{N_\mathtt{TH}}$?

\begin{solution}
\[
\expect{N_\mathtt{TH}}=4.
\]

\begin{editingnotes}
Replace the formula below with a tree figure as in fig~\ref{tree_D}.
\end{editingnotes}

This time the tree diagram is $C \eqdef H\cdot C +T\cdot B$, where the subtree
$B \eqdef H + T\cdot B$.

So
\[
\expect{N_\mathtt{TH}} =  \paren{1+ \expect{N_\mathtt{TH}}} \cdot \frac{1}{2}
+  (1+ \expect{N_B}) \cdot \frac{1}{2}
\]
where $N_B$ is the expected number of flips in the $B$ subtree.  But
\[
\expect{N_B} = 1 \cdot \frac{1}{2} + (1+\expect{N_B}) \cdot  \frac{1}{2}.
\]
That is, $\expect{N_B} = 2$.  Hence,
\[
\expect{N_\mathtt{TH}} = \frac{1}{2} +\frac{\expect{N_\mathtt{TH}}}{2}
+ \frac{1}{2} + \frac{2}{2}
\]
which implies $\expect{N_\mathtt{TH}} = 4$.
\end{solution}

\ppart Suppose we now play a game: flip a fair coin until either
\texttt{TT} or \texttt{TH} occurs.  You win if \texttt{TT} comes up
first, and lose if \texttt{TH} comes up first.  Since \texttt{TT}
takes 50\% longer on average to turn up, your opponent agrees that he
has the advantage.  So you tell him you're willing to play if you pay
him \$5 when he wins, and he pays you with a mere 20\% premium---\$6,
that is---when you win.

If you do this, you're sneakily taking advantage of your opponent's
untrained intuition, since you've gotten him to agree to unfair odds.
What is your expected profit per game?

\begin{staffnotes}
After the problem is solved, start a discussion of the apparent
paradox: \texttt{TT} and \texttt{TH} are equally likely to show up
first, but \texttt{TT} takes longer to show up on average.
\end{staffnotes}

\begin{solution}
It's easy to see that both \texttt{TT} and \texttt{TH} are equally
likely to show up first: every game play consists of a sequence of
\texttt{H}'s followed by a \texttt{T}, after which the game ends with
a \texttt{T} or an \texttt{H}, with equal probability.  So your
expected profit is
\[
\frac{1}{2}\cdot 6 + \frac{1}{2}\cdot (-5)
\]
dollars, that is 50 cents per game.
%You leap to play.

It may seem paradoxical that \texttt{TT} and \texttt{TH} are equally
likely to show up first, but \texttt{TT} takes longer to show up on
average.  The explanation is that \texttt{TT} takes longer to show up
after \texttt{TH} has appeared than \texttt{TH} takes to show up after
\texttt{TT} has appeared, because when \texttt{TT} appears, we're
already one step along the way to having \texttt{TH} appear afterward,
yet when \texttt{TH} shows up first, we have to start waiting for
\texttt{TT} to appear without a similar head start.

\end{solution}

\eparts

\end{problem}


%%%%%%%%%%%%%%%%%%%%%%%%%%%%%%%%%%%%%%%%%%%%%%%%%%%%%%%%%%%%%%%%%%%%%
% Problem ends here
%%%%%%%%%%%%%%%%%%%%%%%%%%%%%%%%%%%%%%%%%%%%%%%%%%%%%%%%%%%%%%%%%%%%%
