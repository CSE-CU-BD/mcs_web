\documentclass[problem]{mcs}

\begin{pcomments}
  \pcomment{FP_probability}
  \pcomment{from: Steven F09}
  \pcomment{Adapted from S01 Final problem 8}
\end{pcomments}

\pkeywords{
  probability
}

%%%%%%%%%%%%%%%%%%%%%%%%%%%%%%%%%%%%%%%%%%%%%%%%%%%%%%%%%%%%%%%%%%%%%
% Problem starts here
%%%%%%%%%%%%%%%%%%%%%%%%%%%%%%%%%%%%%%%%%%%%%%%%%%%%%%%%%%%%%%%%%%%%%


\begin{problem}
\ppart\pts{4} What's the probability that $0$ doesn't appear
among $k$~digits chosen independently and uniformly at random?

\begin{solution}
$9^k/10^k$
\end{solution}

\vspace{1in}

\ppart 
A box contains $90$ good and $10$ defective
screws. What's the probability that if we pick $10$~screws from the
box, none will be defective?

\begin{solution}
$$\frac{\binom{90}{10}}{\binom{100}{10}} = \frac{(90!)^2}{100!80!}$$
\end{solution}

\vspace{1in}

\ppart\pts{3} First one digit is chosen uniformly at random from
$\{1,2,3,4,5\}$ and is removed from the set; then a second digit is
chosen uniformly at random from the remaining digits. What is the
probability that an odd digit is picked the second time?

\begin{solution}
Condition on the event that the first digit is odd, get
  $\frac{2}{4} \cdot \frac{3}{5}+\frac{3}{4} \cdot \frac{2}{5} =
  \frac{12}{20} = \frac{3}{5}$.
\end{solution}

\vspace{1in}

\ppart\pts{3} Suppose that you \emph{randomly} permute the
digits $1, 2,\cdots,n$, i.e., you select a permutation uniformly at
random. What is the probability the digit $k$ ends up in the
$i\th$ position after the permutation?

\begin{solution}$(n-1)!/ n! = 1/n$. This makes perfect sense, since it's a random permutation, so any digit ends up in position i with probability $\frac{1}{n}$.
\end{solution}

\vspace{1in}

%From http://www.cs.cmu.edu/afs/cs.cmu.edu/academic/class/15251/Site/
%Final exam review 1998

\ppart\pts{3} A fair coin is flipped $n$ times.  What's the
probability that all the heads occur at the end of the sequence?
\emph{Clarification:} If no heads occur, then ``all the heads are at
the end of the sequence'' (the statement is vacuously true).

\begin{solution}
$(n+1) {1/2}^{n}$.
\end{solution}

%%%%%%%%%%%%%%%%%%%%%%%%%%%%%%%%%%%%%%%%%%%%%%%%%%%%%%%%%%%%%%%%%%%%%
% Problem ends here
%%%%%%%%%%%%%%%%%%%%%%%%%%%%%%%%%%%%%%%%%%%%%%%%%%%%%%%%%%%%%%%%%%%%%