\documentclass[problem]{mcs}

\begin{pcomments}
  \pcomment{FP_monochromatic_rectangles}
  \pcomment{Simpler version of PS_monochromatic rectangles}
  \pcomment{CH, Spring '14}
\end{pcomments}

\pkeywords{
  counting
  Pigeonhole Principle
  coloring
}

%%%%%%%%%%%%%%%%%%%%%%%%%%%%%%%%%%%%%%%%%%%%%%%%%%%%%%%%%%%%%%%%%%%%%
% Problem starts here
%%%%%%%%%%%%%%%%%%%%%%%%%%%%%%%%%%%%%%%%%%%%%%%%%%%%%%%%%%%%%%%%%%%%%

\begin{problem}

\bparts

Let $R$ be a rectangular chessboard, with 3 rows and 9 columns, in which each
square is colored either black or white. 

\ppart In general, how many different colorings of $R$ are possible?
\begin{solution}
There are 27 squares in $R$ and each square can be colored either
black or white. Therefore, by the Product Rule, the total number of
colorings of $R$ is $2^{27}$. 
\end{solution}

\examspace[1.5in]

\ppart Suppose that some row of $R$ is constrained to be monochromatic (i.e. all squares
are of the same color). How many such chessboards $R$ are possible?
\begin{solution}
We can choose the row index of the monochromatic in 3 ways, and the row color in 2
ways. The rest of the 18 squares can be colored in $2^{18}$ different
ways. Therefore, by the Product Rule, the total number of choices is $3 \times 2 \times
2^{18} = 3 \times 2^{19}$. 
\end{solution}

\examspace[1.5in]

\ppart For a general coloring, explain why some pair of columns of $R$ must be colored the same.
\begin{solution} 
There are $2^3 = 8$ ways to color each column of $R$, while there are
$9$ columns. Therefore, by the Pigeonhole Principle, some pair of columns
must be colored the same. 
\end{solution}

\examspace[1.5in]

\ppart Argue that $R$ always contains a rectangle whose corner squares are
either all-black or all-white.
\begin{solution}
Consider a pair of columns of $R$ with the same coloring (we know from the previous
part that such a pair always exists). These columns have 3 rows
and therefore some color (say, white) must be repeated. The
first two white squares in each of these two columns form the desired
rectangle. 
\end{solution}

\eparts
\end{problem}

%%%%%%%%%%%%%%%%%%%%%%%%%%%%%%%%%%%%%%%%%%%%%%%%%%%%%%%%%%%%%%%%%%%%%
% Problem ends here
%%%%%%%%%%%%%%%%%%%%%%%%%%%%%%%%%%%%%%%%%%%%%%%%%%%%%%%%%%%%%%%%%%%%%

\endinput
