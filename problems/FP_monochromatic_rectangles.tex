\documentclass[problem]{mcs}

\begin{pcomments}
  \pcomment{FP_monochromatic_rectangles}
  \pcomment{Simpler version of PS_monochromatic rectangle}
  \pcomment{CH, Spring '14}
\end{pcomments}

\pkeywords{
  counting
  inclusion-exclusion
  pigeonhole
}

%%%%%%%%%%%%%%%%%%%%%%%%%%%%%%%%%%%%%%%%%%%%%%%%%%%%%%%%%%%%%%%%%%%%%
% Problem starts here
%%%%%%%%%%%%%%%%%%%%%%%%%%%%%%%%%%%%%%%%%%%%%%%%%%%%%%%%%%%%%%%%%%%%%

\begin{problem}
Let $R$ be a rectangular chessboard with 3 rows and 9 columns.  Each
square of $R$ may be colored either black or white.

\bparts

\ppart How many different colorings of $R$ are possible?

\begin{solution}
There are 27 squares in $R$ and each square can be colored either
black or white. Therefore, by the Product Rule, the total number of
colorings of $R$ is $2^{27}$. 
\end{solution}

\examspace[0.75in]

\ppart Write a simple arithmetic formula for the number of possible
colorings of $R$ in which at least one row is \emph{monochromatic},
that is, all squares are the same color.

\begin{solution}
For $I \subset [1,3]$, let $M_I$ be the set of chessboards with
monochromatic rows numbered by $I$.  For example, if $R$ is in
$M_{1,3}$, then is first and third rows are monochromatic.  So $M_1
\union M_2 \union M_3$ is the set of monochromatic triangles.  By
inclusion-exclusion
\[
\card{M_1 \union M_2 \union M_3}} =
         \card{M_1} + \card{M_2} + \card{M_3}
       - \card{M_{12}} - \card{M_{13}} - \card{M_{23}}
       + \card{M_{123}}.
\]

But $\card{M_i} = 2 \cdot 2^{18}$ since there are two ways to color
row $i$ and $2^{18}$ ways to color the remaining squares.  Likewise
$\card{M_{ij}} = 2 \cdot 2 \cdot 2^{9}$, and $\card{M_{123} = 2 \cdot
  2 \cdot 2$.  So
\[
\card{M_1 \union M_2 \union M_3}} = 3 \cdot  2^19 + 3 \cdot  2^11 + 2^3.
\]

\iffalse We can choose the row index of the monochromatic in 3 ways,
and the row color in 2 ways. The rest of the 18 squares can be colored
in $2^{18}$ different ways. Therefore, by the Product Rule, the total
number of choices is $3 \times 2 \times 2^{18} = 3 \times 2^{19}$.
\fi

\end{solution}

\examspace[1.0in]

\ppart Explain why every coloring will include two columns with the
same coloring.

\begin{solution} 
There are $2^3 = 8$ ways to color each column of $R$, while there are
$9$ columns.  Therefore, by the Pigeonhole Principle, some pair of columns
must be colored the same. 
\end{solution}

\examspace[1.0in]

\ppart Conclude every coloring includes four squares of the same color
that form the corners of a rectangle.

\begin{solution}
There must be two columns with the same coloring.  These columns have
3 rows and therefore some color must be repeated.  The first two
squares of this repeated color in each of the columns form the corners
of the desired rectangle.
\end{solution}

\eparts
\end{problem}

%%%%%%%%%%%%%%%%%%%%%%%%%%%%%%%%%%%%%%%%%%%%%%%%%%%%%%%%%%%%%%%%%%%%%
% Problem ends here
%%%%%%%%%%%%%%%%%%%%%%%%%%%%%%%%%%%%%%%%%%%%%%%%%%%%%%%%%%%%%%%%%%%%%

\endinput
