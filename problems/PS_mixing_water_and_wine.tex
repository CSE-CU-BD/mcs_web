\documentclass[problem]{mcs}

\begin{pcomments}
  \pcomment{PS_mixing_water_and_wine}
  \pcomment{from: S09.ps8}
\end{pcomments}

\pkeywords{
  geometric_sum
  asymptotics
  recurrences
  closed_form
}

%%%%%%%%%%%%%%%%%%%%%%%%%%%%%%%%%%%%%%%%%%%%%%%%%%%%%%%%%%%%%%%%%%%%%
% Problem starts here
%%%%%%%%%%%%%%%%%%%%%%%%%%%%%%%%%%%%%%%%%%%%%%%%%%%%%%%%%%%%%%%%%%%%%

\begin{problem}
We begin with two large glasses.  The first glass contains a pint of
water, and the second contains a pint of wine.  We pour 1/3 of a pint
from the first glass into the second, stir up the wine/water mixture in
the second glass, and then pour 1/3 of a pint of the mix back into the
first glass and repeat this pouring back-and-forth process a total of
$n$ times.
\begin{problemparts}
\problempart
Describe a closed-form formula for the amount of wine in the first glass
after $n$ back-and-forth pourings.

\begin{staffnotes}
If students get caught up in complicated formulas involving the
different quantities of wine and water in each glass, point out that
the complete state is determined by the number $w$ of pints in the
first glass.
\end{staffnotes}


\begin{solution}
The state of the system of glasses/wine/water at the beginning
of a round of pouring and pouring back is determined by the total amount
of wine in the first glass.  Suppose at the beginning of some round, the
first glass contains $w$ pints of wine, $0\leq w \leq 1$ and $1-w$ pints
of water.  The second glass contains the rest of the wine and water.

Pouring 1/3 pint from the first glass to the second leaves 2/3 pints of liquid and
$(2/3)w$ wine in the first glass, and 4/3 pints of liquid and 
$1 - (2/3)w$ wine 
in the second glass.  Pouring 1/3 pint back from the second into the first
transfers a proportion of (1/3)/(4/3) of the liquid in the second glass into
the first.  So the round completes with both glasses containing a pint of
liquid, and the first glass containing
\[
(2/3)w + (1/4)(1- (2/3)w) = 1/4 + w/2
\]
pints of wine.  After one more round, the first glass contains
\[1/4 + (1/4 + w/2)/2 = 1/4 + 1/8 + w/2^2\]
pints of wine, and after $n$ more rounds
\[
\begin{array}{lll}
w/2^n + \Sigma_{i=1}^n (1/2)^{i+1} &=  w/2^n + (1/2)\Sigma_{i=1}^n
 (1/2)^{i}\\
& = w/2^n + (1/2)(-1 + \Sigma_{i=0}^n (1/2)^{i})\\
& = w/2^n + (1/2)(-1 + (1 - (1/2)^{n+1})/(1 - 1/2))\\
& = w/2^n - 1/2 + 1 - (1/2)^{n+1}\\
& = w/2^n + 1/2 - (1/2)^{n+1}.
\end{array}
\]
Since $w=0$ initially, the pints of wine in the first glass
after $n$ rounds is
\[
1/2 - (1/2)^{n+1}.
\]

\end{solution}

\problempart What is the limit of the amount of wine in each glass as $n$
approaches infinity?

\begin{solution}
The limiting amount of wine in the first glass approaches 1/2 from below
as $n$ approaches infinity.  In fact, it approaches 1/2 no matter how the
wine was initially distributed.  This of course is what you would expect:
after a thorough mixing the glasses should contain essentially the same
amount of wine.

\end{solution}

\end{problemparts}

\end{problem}

%%%%%%%%%%%%%%%%%%%%%%%%%%%%%%%%%%%%%%%%%%%%%%%%%%%%%%%%%%%%%%%%%%%%%
% Problem ends here
%%%%%%%%%%%%%%%%%%%%%%%%%%%%%%%%%%%%%%%%%%%%%%%%%%%%%%%%%%%%%%%%%%%%%
