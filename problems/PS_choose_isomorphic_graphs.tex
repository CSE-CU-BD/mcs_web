\documentclass[problem]{mcs}

\begin{pcomments}
  \pcomment{PS_choose_isomorphic_graphs}
  \pcomment{from: S09.ps5, S06.ps4, F07.ps5, F06, S06.ps4, S15.ps7}
  \pcomment{in S06 essentially taken from the Course Notes for the
            University of Waterloo class C&O 230; add citation}
\end{pcomments}

\pkeywords{
  graphs
  isomorphisms
  bijections
}

%%%%%%%%%%%%%%%%%%%%%%%%%%%%%%%%%%%%%%%%%%%%%%%%%%%%%%%%%%%%%%%%%%%%%
% Problem starts here
%%%%%%%%%%%%%%%%%%%%%%%%%%%%%%%%%%%%%%%%%%%%%%%%%%%%%%%%%%%%%%%%%%%%%

\begin{problem}
Determine which among the four graphs pictured in
Figure~\ref{fig:isog} are isomorphic.  For each pair of isomorphic
graphs, describe an isomorphism between them.  For each pair of graphs
that are not isomorphic, give a property that is preserved under
isomorphism such that one graph has the property, but the other does
not.  For at least one of the properties you choose, \emph{prove} that
it is indeed preserved under isomorphism (you only need prove one of
them).

\begin{figure}%[h] %[htbp]
\begin{center}
\mbox{  \subfloat[$G_1$]{\graphic{G1}}
        \hspace{17mm}
        \subfloat[$G_2$]{\graphic{G4}} }
\mbox{  \subfloat[$G_3$]{\graphic{G2}}
        \hspace{17mm}
        \subfloat[$G_4$]{\graphic{G3}}
        }
\end{center}
\caption{Which graphs are isomorphic?}
\label{fig:isog}
\end{figure}

\iffalse

\begin{figure}[h] %[htbp]
\graphic[width=1.5in,clip]{G1}
\caption{$G_1$}
\label{fig:G1}
\end{figure}


\begin{figure}[h] %[htbp]
\graphic[width=1.5in,clip]{G2}
\caption{$G_2$}
\label{fig:G2}
\end{figure}


\begin{figure}[h] %[htbp]
\graphic[width=1.5in,clip]{G3}
\caption{$G_3$}
\label{fig:G3}
\end{figure}

\begin{figure}[h] %[htbp]
\graphic[width=1.5in,clip]{G4}
\caption{$G_4$}
\label{fig:G4}
\end{figure}
\fi

\begin{solution}
$G_1$ and $G_4$ are isomorphic.  In particular, the function
  $f:\vertices{G_1} \to \vertices{G_4}$ is an isomorphism, where
\begin{align*}
&f(1)=1 \quad&& f(2)=2 \quad&& f(3)=3 \quad&& f(4)=8 \quad&& f(5)=9 \\
&f(6)=10 \quad&& f(7)=4 \quad&& f(8)=5 \quad&& f(9)=6 \quad&& f(10)=7
\end{align*}

$G_1$ and $G_2$ are not isomorphic to $G_3$: $G_3$ has a vertex of degree
four and neither $G_1$ nor $G_2$ has one.

$G_1$ and $G_2$ are not isomorphic: $G_2$ has a cycle of length
four and $G_1$ does not.

There are many examples of properties preserved under graph
isomorphism noted in Chapter~\bref{simple_graphs_chap}; for
example, number of vertices and edges, vertex degrees, size of cycles
and connectedness.  See Problem~\bref{PS_neighbors_under_isomorphisms}
for a formal proof that isomorphisms preserve vertex degrees.

\iffalse

For example, we will prove that the degree of each
vertex is preserved under isomorphism.

Let $G$ and $H$ be isomorphic graphs. Since they are isomorphic, there
is an edge-preserving bijection $f: \vertices{G} \to \vertices{H}$:
\[
\edge{u}{v} \in \edges{G} \iff \edge{f(u)}{f(v)} \in \edges{H}
\]
for all $u,v \in \vertices{G}$.

We let the set of vertices adjacent to $u$ be $N(u)$.  Because $f$ is
an edge-preserving bijection, there is an edge from $f(u)$ to a vertex
$f(k)$ iff $k \in N(u)$.  Thus $\card{N(f(u))} = \card{N(u)}$ and the
degree of each vertex is preserved under isomorphism.
\fi


\end{solution}

\end{problem}

%%%%%%%%%%%%%%%%%%%%%%%%%%%%%%%%%%%%%%%%%%%%%%%%%%%%%%%%%%%%%%%%%%%%%
% Problem ends here
%%%%%%%%%%%%%%%%%%%%%%%%%%%%%%%%%%%%%%%%%%%%%%%%%%%%%%%%%%%%%%%%%%%%%

\endinput
