\documentclass[problem]{mcs}

\begin{pcomments}
  \pcomment{FP_MST_min_out_S16}
  \pcomment{subsumes FP_MST_min_out}
  \pcomment{ARM 4/24/16}
\end{pcomments}

\pkeywords{
  tree
  spanning_tree
  MST
  weight
  minimum
  cycle
}

%%%%%%%%%%%%%%%%%%%%%%%%%%%%%%%%%%%%%%%%%%%%%%%%%%%%%%%%%%%%%%%%%%%%%
% Problem starts here
%%%%%%%%%%%%%%%%%%%%%%%%%%%%%%%%%%%%%%%%%%%%%%%%%%%%%%%%%%%%%%%%%%%%%

\begin{problem}
Let $G$ be a connected weighted simple graph and be $T$ be a minimum
weight spanning tree of $G$.  Let $e$ be an edge of $G$ that is
incident to some vertex $v$ such that the weight of $e$ is strictly
smaller than the weight of every other edge incident to $v$.  Prove
that $e$ is an edge of $T$.  \hint Suppose $e$ is not an edge of $T$.

\examspace[4in]

\begin{solution}
Suppose to the contrary that $e$ is not in $T$.  There is a unique
path, $P$, in $T$ from $v$ to the other endpoint of $e$.  This path
must start with some edge $f$.

Now $T - f + e$ is still connected since $v$ is connected to the
other endpoint of $f$ by the path $P-f+e$.  Therefore $T-f+e$ is also
a spanning tree of $G$ since it has the same number of edges as $T$.

But the weight of $T - f + e$ is smaller than the weight than $T$,
since the weight of $e$ is smaller thatn the weight of $f$,
contradicting that fact that $T$ is a minimum weight spanning tree..
\end{solution}

\end{problem}

%%%%%%%%%%%%%%%%%%%%%%%%%%%%%%%%%%%%%%%%%%%%%%%%%%%%%%%%%%%%%%%%%%%%%
% Problem ends here
%%%%%%%%%%%%%%%%%%%%%%%%%%%%%%%%%%%%%%%%%%%%%%%%%%%%%%%%%%%%%%%%%%%%%
\endinput
