\documentclass[problem]{mcs}

\begin{pcomments}
    \pcomment{CP_CNF_from_DNF}
    \pcomment{by ARM 2/5/11}
\end{pcomments}

\begin{problem}
Explain how to find a \idx{conjunctive form} for a propositional formula
directly from a \idx{disjunctive form} for its complement.

\begin{solution}

We want to find a negation to the given formula $F$, which is an $\QOR$ of $\QAND$s. We do so by negating the entire formula, which ends up being the negation of a disjunctive normal form. Notice that, by DeMorgan's law, $\bar{A \QOR B} \Longleftrightarrow \bar{A} \QAND \bar{B}$. This means we can apply DeMorgan's law to the entire formula, obtaining the $\QAND$ of many negated $\QAND$ terms. We apply DeMorgan's laws on each of these terms, turning them into $\QOR$ terms. This yields the desired conjunctive form.

For example, if we are given $\bar{(A \QAND B) \QOR (B \QAND C)}$, the first step yields $\bar{(A \QAND B)} \QAND \bar{(B \QAND C)}$ and then the second step yields $(\bar{A} \QOR \bar{B}) \QAND (\bar{B} \QOR \bar{C})$.

If $D$ a disjunctive form equivalent to $\QNOT(F)$, then by double
negation, $\QNOT(D)$ is equivalent to $F$.  So applying DeMorgan's Law
for \QNOT\ over \QOR~\bref{DeMOR} to $\QNOT(D)$, and then again for
\QNOT\ over \QAND\ for each conjunct, will yield a
conjunctive form equivalent to $F$.
\end{solution}

\end{problem}

\endinput
