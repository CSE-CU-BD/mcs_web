\documentclass[problem]{mcs}

\begin{pcomments}
    \pcomment{CP_CNF_from_DNF}
    \pcomment{by ARM 2/5/11}
\end{pcomments}

\pkeywords{
  proposition
  formula
  dnf
  cnf
  equivalence
  conjunt
  disjunct}

\begin{problem}
Explain how to read off a \idx{conjunctive form} for a propositional
formula directly from a \idx{disjunctive form} for its complement.

\begin{solution}
We want to find a conjunctive form for a formula $F$, given a
disjunctive form---a sum of products for $\bar{F}$.   We do so by
negating the disjunctive form for $\bar{F}$.  This negated formula is equivalent to $F$.

Now according to DeMorgan's law for distributing \QNOT over
\QOR or \QAND, we have
\begin{align}
\QNOT(A \QOR B \QOR \cdots \QOR Z) & \qiff \bar{A} \QAND \bar{B} \QAND \cdots \QAND \bar{Z})
\label{moreDeMnotor} \\
\QNOT(A \QAND B \QAND \cdots \QAND Z) & \qiff \bar{A} \QOR \bar{B} \QOR \cdots \QOR \bar{Z})
\label{moreDeMnotand}
\end{align}

So we can apply~\eqref{moreDeMnotor} to $\bar{F}$ to obtain the
$\QAND$ of the negations of each of the product terms in $\bar{F}$.
For example, if the disjunctive form of $\bar{F}$ was
\[
(A \QAND B) \QOR (B \QAND \bar{C}),
\]
then negating this formula and applying~\eqref{moreDeMnotor} yields
\begin{equation}\label{prodnotprods}
\QNOT(A \QAND B) \QAND \QNOT((B \QAND \bar{C})
\end{equation}

Now we can turn each of the negated products into a sum by
applying~\eqref{moreDeMnotand}.  For example,~\eqref{prodnotprods}
would turn into
\begin{equation}\label{almostCF}
(\bar{A} \QOR \bar{B}) \QAND (\bar{B} \QOR \bar{\bar{C}})
\end{equation}

Then removing double-negations yields the desired conjunctive form.
For example, removing double-negations from~\eqref{almostCF} yields
\[
(\bar{A} \QOR \bar{B}) \QAND (\bar{B} \QOR C),
\]
as a conjunctive formula for $F$.
\end{solution}

\end{problem}

\endinput
