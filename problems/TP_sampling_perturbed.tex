\documentclass[problem]{mcs}

\begin{pcomments}
  \pcomment{TP_sampling_perturbed}
  \pcomment{variant of FP_sampling_concepts, similar to FP_random_sampling}
\end{pcomments}

\pkeywords{
  random_variable
  independence
  sampling
  confidence
}

%%%%%%%%%%%%%%%%%%%%%%%%%%%%%%%%%%%%%%%%%%%%%%%%%%%%%%%%%%%%%%%%%%%%%
% Problem starts here
%%%%%%%%%%%%%%%%%%%%%%%%%%%%%%%%%%%%%%%%%%%%%%%%%%%%%%%%%%%%%%%%%%%%%

\begin{problem}

  Yesterday, the bakers at a local cake factory baked a huge number of
  cakes.  To estimate the fraction, $b$, of cakes in this program that
  are improperly prepared, the cake-testers will take a small \idx{sample}
  of cakes chosen randomly and independently (so it is possible,
  though unlikely, that the same cake might be chosen more than once).
  For each cake chosen, they perform a variety of non-destructive
  tests to determine if the cake is improperly prepared, after which
  they will use the fraction of bad cakes in their sample as their
  estimate of the fraction $b$.

  The factory statistician can use estimates of a binomial distribution to
  calculate a value, $s$, for a number of cakes to sample which
  ensures that with 97\% confidence, the fraction of bad cakes in the
  sample will be within 0.006 of the actual fraction, $b$, of bad cakes
  in the back.

  Mathematically, the \emph{batch} is an actual outcome that already
  happened.  The \emph{sample} is a random variable defined by the
  process for randomly choosing $s$ cakes from the batch.  The
  justification for the statistician's \idx{confidence} depends on
  some properties of the batch and how the sample of $s$ cakes from
  the batch are chosen.  These properties are described in some of the
  statements below.  Mark each of these statements as \textbf{T}
  (true) or \textbf{F} (false), and then briefly explain your answer.

\begin{enumerate}

\item The probability that the ninth cake in the
\emph{batch} is bad is $b$.\hfill\examrule

\examspace[0.7in]

\begin{solution}
False.

The batch has already been baked, so there's nothing probabilistic
about the badness of the ninth (or any other) cake:
either it is or it isn't bad, though we don't know which.  You could
argue that this means it is bad with probability zero or one, but in
any case, it certainly isn't $b$.
\end{solution}

\item All cakes in the batch are equally likely to be the
third cake chosen in the \emph{sample}.\hfill\examrule

\examspace[0.7in]

\begin{solution}
True.

 The meaning of ``random choices of cakes from the batch'' is
  precisely that at each of the $s$ choices in the sample, in particular
  at the third choice, each cake in the batch is equally likely to be
  chosen.
\end{solution}

\item The probability that the ninth cake chosen for the
  \emph{sample} is bad, is $b$.\hfill\examrule

\examspace[0.7in]

\begin{solution} True.

The ninth cake sampled is equally likely to be any cake in the
batch, so the probability it is bad is the same as the fraction, $b$,
of bad cakes in the program.
\end{solution}

\item Given that the first cake chosen for the \emph{sample} is bad,
the probability that the second cake chosen will also be bad is greater
than~$b$.

\hfill\examrule

\examspace[0.7in]

\begin{solution}
False.

  The meaning of ``\emph{independent} random choices of cakes from
  the batch'' is precisely that at each of the $s$ choices in the
  sample, in particular at the second choice, each cake in the batch is
  equally likely to be chosen, independent of what the first or any other
  choice happened to be.
\end{solution}

\item  Given that the last cake in the \emph{batch} is bad, the
  probability that the next-to-last cake in the batch will also be
  bad is greater than~$b$.

\hfill\examrule

\examspace[0.7in]

\begin{solution}
False.

  As noted above, it's zero or one.
\end{solution}

\item Given that the first two cakes selected in the \emph{sample} are
  the same kind of cake ---they might both be chocoloate, or both be
  angel food cakes,\dots ---the probability that the first cake is bad
  may be greater than $b$.

\hfill\examrule

\examspace[0.7in]

\begin{solution}
True.

  We don't know how prone to badness different kinds of cakes may be.
  It could be for example, that chocolate cakes are more prone to
  being prepared improperly than other kinds of cakes, and that there
  are more chocolate cakes than any other kind of cake in the batch.
  Then given that two randomly chosen cakes in the sample are the same
  kind, they are more likely to be chocolate, which makes them more
  prone to badness.  That is, the conditional probability that they
  will be bad would be greater than $b$.
\end{solution}

\item  The expectation of the indicator variable for the last cake in
  the \emph{sample} being \text{bad} is~$b$.\hfill\examrule

\examspace[0.7in]

\begin{solution}
True.

  The expectation of the indicator variable is the same as the probability
  that it is 1, namely, it is the probability that the $s$th cake chosen
  is bad, which is $b$, by the reasoning above.
\end{solution}

\item There is zero probability that all the cakes in the
\emph{sample} will be different.\hfill\examrule

\examspace[0.7in]

\begin{solution}
False.

  We know the size, $r$, of the batch is larger than the ``small''
  sample size, $s$, in which case the probability that all the cakes in
  the sample are different is
  \[
  \frac{r}{r}\cdot \frac{r-1}{r}\cdot \frac{r-2}{r} \cdots \frac{r-(s-1)}{r}
  = \frac{r!}{(r-s)!\, r^s} > 0.
  \]

  Of course it would be true by the Pigeonhole Principle if $s>r$.
\end{solution}


\end{enumerate}
\end{problem}


%%%%%%%%%%%%%%%%%%%%%%%%%%%%%%%%%%%%%%%%%%%%%%%%%%%%%%%%%%%%%%%%%%%%%
% Problem ends here
%%%%%%%%%%%%%%%%%%%%%%%%%%%%%%%%%%%%%%%%%%%%%%%%%%%%%%%%%%%%%%%%%%%%%

\endinput
