\documentclass[problem]{mcs}

\begin{pcomments}
  \pcomment{PS_asymptotics_table}
  \pcomment{from: S09.ps8, S08.ps8}
\end{pcomments}

\pkeywords{
  asymptotics
  Stirlings_formula
}

%%%%%%%%%%%%%%%%%%%%%%%%%%%%%%%%%%%%%%%%%%%%%%%%%%%%%%%%%%%%%%%%%%%%%
% Problem starts here
%%%%%%%%%%%%%%%%%%%%%%%%%%%%%%%%%%%%%%%%%%%%%%%%%%%%%%%%%%%%%%%%%%%%%

\begin{problem}
  Indicate which of the following holds for each pair of functions $(f(n),
  g(n))$ in the table below.  Assume $k\geq 1$, $\epsilon > 0$, and $c >
  1$ are constants.  \inbook{Pick the four table entries you consider to be the
  most challenging or interesting and justify your answers to these.}

\begin{equation*}
\begin{array}{|cc|c|c|c|c|c|c|}
\hline
f(n) & g(n) &
f = O(g) & f=o(g) &
g = O(f) & g=o(f) &
f = \Theta(g) & f\sim g
\\\hline\hline
2^n & 2^{n/2} & & & & & & \\\hline
\sqrt{n} & n^{\sin (n\pi/2)} & & & & & & \\\hline
\log(n!) & \log(n^n) & & & & & & \\\hline
n^k & c^n & & & & & & \\\hline
\log^k n & n^\epsilon & & & & & & \\\hline
\end{array}
\end{equation*}

%\item $f=O(g)$
%\item $f=o(g)$
%\item $g=O(f)$
%\item $g=o(f)$
%\item $f = \Theta(g)$
%\item $f \sim g$
%\item none of the above.
%
%\begin{equation*}
%\begin{array}{|cc|}
%\hline
%f(n) & g(n) \\\hline\hline
%\log^k n & n^\epsilon \\\hline
%n^k & c^n \\\hline
%\sqrt{n} & n^{\sin (n\pi/2)} \\\hline
%2^n & 2^{n/2} \\\hline
%\log(n!) & \log(n^n) \\\hline
%\end{array}
%\end{equation*}

\begin{staffnotes}
Have students justify their answers to a few table entries that they
consider most challenging or interesting.
\end{staffnotes}

\begin{solution}
\newcommand\y{\text{yes}}
\newcommand\n{\text{no}}
\[
\begin{array}{|cc|c|c|c|c|c|c|}
\hline
f(n) & g(n) &
f = O(g) & f=o(g) &
g = O(f) & g=o(f) &
f = \Theta(g) & f\sim g
\\\hline\hline
2^n & 2^{n/2} &
\n & \n & \y & \y & \n & \n
\\\hline
\sqrt{n} & n^{\sin (n\pi/2)} &
\n & \n & \n & \n & \n & \n
\\\hline
\log(n!) & \log(n^n) &
\y & \n & \y & \n & \y &  \y
\\\hline
n^k & c^n &
\y & \y & \n & \n & \n & \n
\\\hline
\log^k n & n^\epsilon &
\y & \y & \n & \n & \n & \n
\\\hline
\end{array}
\]

Following are some hints on deriving the table above:

\begin{itemize}

\item[(a)]
$\dfrac{2^n}{2^{n/2}}=2^{n/2}$ grows without bound as $n$
grows---it is not bounded by a constant.

\item[(b)]
When $n$ is even, then $n^{\sin (n\pi/2)}=1$.  So, no constant times
$n^{\sin (n\pi/2)}$ will be an upper bound on $\sqrt n$ as $n$ ranges
over even numbers.  When $n \equiv 1 \bmod 4$, then $n^{\sin (n\pi/2)}=
n^1 =n$. So, no constant times $\sqrt n$ will be an upper bound on
$n^{\sin (n\pi/2)}$ as $n$ ranges over numbers $\equiv 1 \bmod 4$.

\item[(c)]
\begin{align}
\log(n!)
    & = \log \sqrt{2\pi n}\left(\frac{n}{e}\right)^n \pm c_n\label{logS}\\
    & = \log n + n(\log n - 1) \pm d_n \label{logprod}\\
    & \sim n \log n\label{sim}\\
    & = \log n^n.\notag
\end{align}
where $a \leq c_n,d_n \leq b$ for some constants $a,b \in \reals$ and all
$n$.  Here equation~(\ref{logS}) follows by taking logs of Stirling's
formula,~(\ref{logprod}) follows from the fact that the log of a product
is the sum of the logs, and~(\ref{sim}) follows because any constant,
$\log n$, and $n$ are all $o(n \log n)$ and hence so is their sum.

\iffalse

\dots follows\textbf{from Problem~\ref{o+o}, part~\ref{f+o} because any
constant, $\log n$, and $n$ are all $o(n \log n)$.}
\fi

\item[(d)]
\emph{Polynomial} growth versus \emph{exponential} growth.

\item[(e)]
\emph{Polylogarithmic} growth versus \emph{polynomial} growth.

\end{itemize}


\end{solution}
\end{problem}

%%%%%%%%%%%%%%%%%%%%%%%%%%%%%%%%%%%%%%%%%%%%%%%%%%%%%%%%%%%%%%%%%%%%%
% Problem ends here
%%%%%%%%%%%%%%%%%%%%%%%%%%%%%%%%%%%%%%%%%%%%%%%%%%%%%%%%%%%%%%%%%%%%%

\endinput
