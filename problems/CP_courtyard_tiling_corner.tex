\documentclass[problem]{mcs}

\begin{pcomments}
  \pcomment{CP_courtyard_tiling_corner}
  \pcomment{from: S17.ps2}
  \pcomment{adapted from: S09.cp4r which was subsumed by Notes}
\end{pcomments}

\pkeywords{
  induction
  ordinary_induction
  recursive_procedure
}

%%%%%%%%%%%%%%%%%%%%%%%%%%%%%%%%%%%%%%%%%%%%%%%%%%%%%%%%%%%%%%%%%%%%%
% Problem starts here
%%%%%%%%%%%%%%%%%%%%%%%%%%%%%%%%%%%%%%%%%%%%%%%%%%%%%%%%%%%%%%%%%%%%%

\begin{problem}
\bparts

\ppart\label{cornertile_induction} Prove by induction that a $2^n \times
2^n$ courtyard with a $1 \times 1$ statue of Bill in \emph{a corner} can be
covered with L-shaped tiles.  (Do not assume or reprove the (stronger)
result of Theorem~\bref{bill} that Bill can be placed anywhere.  The point
of this problem is to show a different induction hypothesis that works.)

\begin{solution}
Let $P(n)$ be the proposition Bill can be placed in a corner of a
$2^n~\times~2^n$ courtyard with a proper tiling of the remainder with
L-shaped tiles.

\inductioncase{Base case:} $P(0)$ is true because Bill fills the whole courtyard.

\inductioncase{Inductive step:} Assume that $P(n)$ is true for some
$n \geq 0$; that is, there exists a tiling of the $2^n \times 2^n$
courtyard leaving Bill in a corner.

To prove $P(n+1)$, divide the $2^{n+1} \times 2^{n+1}$ courtyard into
four quadrants, each $2^n \times 2^n$.  One quadrant will contain the
corner designated for Bill.  By induction hypothesis, we can get Bill
into some corner of the quadrant, which means we can actually get him
into \emph{any} desired corner of the quadrant by rotating the tiling
of the quadrant.  So place Bill in the designated corner of the
quandrant, and tile the rest of the quadrant.

Now tile the remaining three quadrants, leaving a tile space open in
the quadrant corners that are in the middle of the whole $2^{n+1}
\times 2^{n+1}$ courtyard (as in the diagram in the proof of
Theorem~\bref{bill}).  These three spaces form an L-shape that that
can be filled with a single L-shaped tile, completing the full
courtyard tiling.  This proves $P(n+1)$, completing the proof by
induction that a square courtyard with side length any power of 2 can
be tiled with Bill in a corner.

\end{solution}

\ppart Use the result of part~\eqref{cornertile_induction} to prove the
original claim that there is a tiling with Bill in the middle.

\begin{solution}
To put Bill in the middle, tile each of the four quadrants, leaving
the empty corner of the quadrant in the middle of the full courtyard.
This leaves the four central squares of the full courtyard empty, so
fill three of these squares with an L-shaped tile.  This leaves a
single central square untiled for Bill.
\end{solution}

\eparts
\end{problem}

%%%%%%%%%%%%%%%%%%%%%%%%%%%%%%%%%%%%%%%%%%%%%%%%%%%%%%%%%%%%%%%%%%%%%
% Problem ends here
%%%%%%%%%%%%%%%%%%%%%%%%%%%%%%%%%%%%%%%%%%%%%%%%%%%%%%%%%%%%%%%%%%%%%

\endinput
