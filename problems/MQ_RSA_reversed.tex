\documentclass[problem]{mcs}

\begin{pcomments}
    \pcomment{RSA_reversed}
\end{pcomments}

\begin{problem}

%% type: short-answer
%% title: RSA reversed

\bparts

For communication between two users $A$ and $B$, if $A$ generates the keys, 
RSA encryption scheme allows $B$ to send $A$ a message 
encrypted by the public key that cannot be eavesdropped, 
but since the public key is public, a third party $C$ can pretend to be $B$ 
and send $A$ a message encrypted by the public key.
In this problem we will see RSA can also prevent forged messages 
when the keys are used in reverse.
Using the same keys $A$ had generated, consider when $A$ sends $B$ a message 
encrypted using the private key.

\ppart
User $B$ can decrypt the message using the public key.
Explain why this works.

\begin{solution}
The normal RSA scheme decrypt messages using the property
$\rem{\rem{m^{e}}{n}^d}{n} = \rem{m^{ed}}{n} = m$.
With the keys reversed, the message can still be decrypted, because
$\rem{\rem{m^{d}}{n}^e}{n} = \rem{m^{de}}{n} = m$.
\end{solution}

\ppart
%n = p * q = 29 * 37 = 1073 
%phi(n) = 28 * 26 = 1008
%e,d = 605,5 
Given the public key (5, 1073), and the encrypted message 33, 
what is the original message?

\begin{solution}
The original message is $\rem{33^5}{1073} = 937$
\end{solution}

\ppart
Explain in words why $B$ can be confident that this message in not forged?

\begin{solution}
Because only $A$ knows the private key.
\end{solution}

\eparts

\end{problem}

\endinput
