\documentclass[problem]{mcs}

\begin{pcomments}
  \pcomment{CP_binomial_distribution_max}
  \pcomment{from: S02 ps9 revised by ARM 5/1/10}
\end{pcomments}

\pkeywords{
  distribution
  density
  binomial_coefficient
  Stirling
}

%%%%%%%%%%%%%%%%%%%%%%%%%%%%%%%%%%%%%%%%%%%%%%%%%%%%%%%%%%%%%%%%%%%%%
% Problem starts here
%%%%%%%%%%%%%%%%%%%%%%%%%%%%%%%%%%%%%%%%%%%%%%%%%%%%%%%%%%%%%%%%%%%%%

\begin{problem}
Suppose you have a biased coin that has probability $p$ of flipping
heads.  Let $J$ be the number of heads in $n$ independent coin flips.
So $J$ has the general binomial distribution:
\[
\pdf_J(k) = \binom{n}{k} p^k q^{n-k}
\]
where $q \eqdef 1-p$.

\bparts

\ppart\label{pdjjk<} Show that
\begin{align*}
\pdf_J(k-1) & < \pdf_J(k) & \text{ for } k<np+p,\\
\pdf_J(k-1) & > \pdf_J(k) & \text{ for } k>np+p.
\end{align*}

\begin{solution}
Consider the ratio of the probability of $k$ heads over the
probability of $k-1$ heads.

\begin{align*}
\frac{\pdf_J(k)}{\pdf_J(k-1)}
 &= \frac{ \tbinom{n}{k} p^k q^{n-k}}{\tbinom{n}{k-1} p^{k-1}q^{n-k+1}}\\
 &= \frac{\frac{n!}{k!\, (n-k)!}}{\frac{n!}{(k-1)!\, (n-k+1)!}}\frac{p}{q} \\
 &= \frac{(n-k+1) p}{k q}
\end{align*}
This fraction is greater than 1 precisely when $(n-k+1) p> kq = k
(1-p)$, that is when $k < np+p$.  So for $k<np+p$, the probability of
$k$ heads increases as $k$ increases, and for $k>np+p$, the
probability decreases as $k$ increases.
\end{solution}

\ppart Conclude that the maximum value of $\pdf_J$ is
asymptotically equal to
\[
\frac{1}{\sqrt{2 \pi n p q }}.
\]


\hint For the asymptotic estimate, it's ok to assume that $np$ is an
integer, so by part~\eqref{pdjjk<}, the maximum value is $\pdf_J(np)$.
Use \idx{Stirling's formula}~(\bref{nfacsim})\inbook{.}\inhandout{:
\[
n! \sim  \stirling{n}\,.
\]
}
\begin{solution}

\begin{align*}
\pdf_J(np) & \eqdef  \binom{n}{np} p^{np} q^{n-np}\\
          & = \frac{n!}{(np)!\, (nq)!}p^{np} q^{nq}\\
          & \sim \frac{\paren{\frac{n}{e}}^n \sqrt{2 \pi n}}
                  {\paren{\paren{\frac{np}{e}}^{np} \sqrt{2 \pi np}} \paren{\paren{\frac{nq}{e}}^{nq} \sqrt{2 \pi nq}}}p^{np} q^{nq}\\
          & = \frac{\frac{n^n}{e^n} \sqrt{2 \pi n}}
                   {\paren{\frac{n^{np}p^{np}}{e^{np}} \sqrt{2 \pi np}}
                    \paren{\frac{n^{nq}q^{nq}}{e^{nq}} \sqrt{2 \pi nq}}} p^{np} q^{nq}\\
          & = \frac{\frac{n^n}{e^n} \sqrt{2 \pi n}}{\frac{n^{np+nq}p^{np}q^{nq}}{e^{np+nq}} \sqrt{2 \pi np}\sqrt{2 \pi nq}} p^{np} q^{nq} p^{np} q^{nq}\\
          & = \frac{\frac{n^n}{e^n} \sqrt{2 \pi n}}{\frac{n^{n}}{e^{n}} \sqrt{2 \pi np}\sqrt{2 \pi nq}}\\
%          & = \frac{\sqrt{2 \pi n}}{\sqrt{2 \pi np}\sqrt{2 \pi nq}}\\
          & = \frac{1}{\sqrt{2 \pi npq}}.
\end{align*}

\end{solution}


\eparts

\end{problem}

%%%%%%%%%%%%%%%%%%%%%%%%%%%%%%%%%%%%%%%%%%%%%%%%%%%%%%%%%%%%%%%%%%%%%
% Problem ends here
%%%%%%%%%%%%%%%%%%%%%%%%%%%%%%%%%%%%%%%%%%%%%%%%%%%%%%%%%%%%%%%%%%%%%

\endinput


