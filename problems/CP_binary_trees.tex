\documentclass[problem]{mcs}

\begin{pcomments}
  \pcomment{CP_binary_trees}
  \pcomment{from: F09,ps4, S09.cp5m}
\end{pcomments}

\pkeywords{
  recursive_data
  trees
  binary_trees
  induction
  structural_induction
}

%%%%%%%%%%%%%%%%%%%%%%%%%%%%%%%%%%%%%%%%%%%%%%%%%%%%%%%%%%%%%%%%%%%%%
% Problem starts here
%%%%%%%%%%%%%%%%%%%%%%%%%%%%%%%%%%%%%%%%%%%%%%%%%%%%%%%%%%%%%%%%%%%%%

\begin{problem}

\begin{definition*}
 The recursive data type $\btg$ of \term{binary trees} with
 leaf labels $L$ is defined recursively as follows:

\begin{itemize}

\item \textbf{Base case:}
$\ang{\texttt{leaf},l} \in \btg$, for all labels $l\in L$.

\item \textbf{Constructor case:} If $G_1,G_2 \in \btg$, then
\[
\ang{\texttt{bintree},G_1,G_2} \in \btg.
\]
\end{itemize}


The \emph{size} $\card{G}$ of $G \in \btg$ is defined recursively on
this definition by:

\begin{itemize}
\item \textbf{Base case:}
\[
\card{\ang{\texttt{leaf}, l}} \eqdef 1, \quad \text{ for all } l \in L.
\]

\item \textbf{Constructor case:}
\[
\card{\ang{\texttt{bintree}, G_1, G_2}} \eqdef \card{G_1}+ \card{G_2} + 1.
\]
\end{itemize}

\end{definition*}

For example, the size of the $\btg$ $G$ pictured in
Figure~\ref{CP_binary_trees:small-tree}, is 7.

\begin{figure}[htbp]
\graphic[height=3in]{binary-game-tree}
\caption{A picture of a binary tree $G$.}
\label{CP_binary_trees:small-tree}
\end{figure}

\bparts
\ppart Write out (using angle brackets and labels \texttt{bintree},
\texttt{leaf}, etc.) the $\btg$ $G$ pictured in
Figure~\ref{CP_binary_trees:small-tree}.

\begin{solution}
\begin{equation}
\begin{split}
\langle\texttt{bintree}, & \langle\texttt{bintree}, \ang{\texttt{leaf, win}},\\
  & \hspace{5em} \ang{\texttt{bintree}, \ang{\texttt{leaf, lose}},
       \ang{\texttt{leaf, win}}}\rangle,\\
  & \ang{\texttt{leaf, win}}\rangle
\end{split}
\end{equation}

\end{solution}

\eparts

The value of $\text{flatten}(G)$ for $G \in \btg$ is the sequence of
labels in $L$ of the leaves of $G$.  For example, for the $\btg$ $G$
pictured in Figure~\ref{CP_binary_trees:small-tree},
\[
\text{flatten}(G) = (\texttt{win}, \texttt{lose}, \texttt{win}, \texttt{win}).
\]  %Need updating for general labels

\bparts

\ppart
Give a recursive definition of flatten.  (You may use the operation of
\emph{concatenation} (append) of two sequences.)

\begin{solution}
Define flatten recursively on the definition of $\btg$.

\begin{itemize}

\item \textbf{Base case:}
  \begin{align*}
    \text{flatten}(\ang{\texttt{leaf}, \texttt{win}}) & \eqdef (\texttt{win}),\\
    \text{flatten}(\ang{\texttt{leaf}, \texttt{lose}})& \eqdef (\texttt{lose}).
  \end{align*}

\item \textbf{Constructor case:}
  \[
  \text{flatten}(\ang{\texttt{bintree}, G_1, G_2}) \eqdef
  \text{flatten}(G_1)\text{flatten}(G_2)
  \]
  where $\text{flatten}(G_1)\text{flatten}(G_2)$ is the concatenation of
  the two sequences of leaf labels, that is the sequence of labels in
  $\text{flatten}(G_1)$ followed by the labels in $\text{flatten}(G_2)$.
\end{itemize}

\end{solution}

\ppart Prove by structural induction on the definitions of flatten and
size that
\begin{equation}\label{CP_binary_trees:2gg1}
  2\cdot \text{length}(\text{flatten}(G)) = \card{G}+1.
\end{equation}

\begin{solution}
  The proof is by structural induction on the given function
  definitions.  The induction hypothesis is the equation~\eqref{CP_binary_trees:2gg1}.

\begin{itemize}
\item \textbf{Base cases:}
  \[
  2 \cdot \text{length}(\text{flatten}(\ang{\texttt{leaf},
    \text{label}})) = 2\cdot \text{length}((\text{label})) = 2 \cdot 1 =
  2 = 1 + 1 = \card{\ang{\texttt{leaf}, \text{label}}} + 1.
  \]
  So~\eqref{CP_binary_trees:2gg1} holds in the base cases.

\item \textbf{Constructor case:}
  Say $G = \ang{\texttt{bintree}, G_1, G_2}$ where we assume the Structural
  Induction hypothesis that $G_1$ and $G_2$ satisfy~\eqref{CP_binary_trees:2gg1}.  Now,
  \begin{align*}
    \lefteqn{2\cdot \text{length}(\text{flatten}(G))}\\
    & = 2 \cdot \text{length}(\text{flatten}(\ang{\texttt{bintree}, G_1, G_2}))\\
    & = 2\cdot \text{length}(\text{flatten}(G_1)\text{flatten}(G_2))
    & \text{(def of flatten)}\\
    & = 2\cdot \text{length}(\text{flatten}(G_1))+
    2 \cdot \text{length}(\text{flatten}(G_2))
    & \text{(length of a string)}\\
    & = (\card{G_1}+1) + \card{G_2}+1 & \text{(Structural Induction hyp.)}\\
    & = (\card{G_1} + \card{G_2}+1) +1\\
    & = \card{\ang{\texttt{bintree}, G_1, G_2}} + 1
    & \text{(def. of $\card{\ang{\text{bintree},\dots}}$)}\\
    & = \card{G} +1,
  \end{align*}
  proving that~\eqref{CP_binary_trees:2gg1} holds for $G$.
  This completes the proof for the Constructor cases.
\end{itemize}

We conclude by Structural Induction that equation~\eqref{CP_binary_trees:2gg1}
holds for all $G \in \btg$.
\end{solution}

\eparts

\end{problem}

%%%%%%%%%%%%%%%%%%%%%%%%%%%%%%%%%%%%%%%%%%%%%%%%%%%%%%%%%%%%%%%%%%%%%
% Problem ends here
%%%%%%%%%%%%%%%%%%%%%%%%%%%%%%%%%%%%%%%%%%%%%%%%%%%%%%%%%%%%%%%%%%%%%

\endinput
