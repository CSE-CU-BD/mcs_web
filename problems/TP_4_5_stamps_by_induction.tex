\documentclass[problem]{mcs}

\begin{pcomments}
    \pcomment{TP_Postage_by_Induction}
    \pcomment{Converted from prob2-postage.scm by scmtotex and dmj
              on Sun 13 Jun 2010 10:18:32 AM EDT}
    \pcomment{edited by drewe 19 july 2011}
    \pcomment{format ARM 8/25/11}
\end{pcomments}

\pkeywords{
induction
strong_induction
well_ordering
postage
}

\begin{problem}
Prove that every amount of postage of 12 cents or more can be formed
using just 4-cent and 5-cent stamps.

\begin{solution}
We can do this by Simple Induction, Strong Induction, or Well-Ordering,
among other ways.

By \textbf{Strong Induction}:

Consider the Induction hypothesis:
\begin{equation*}
    P(n) \eqdef [\text{4-cent and 5-cent stamps can form $n$-cent postage}]
\end{equation*}
with base cases $n$ = 12, 13, 14, and 15.  Each of the base cases is
easy to construct (three 4s, two 4s and a 5, two 5s and a 4, three 5s,
respectively) and since any integer greater than 12 can be produced by
adding a multiple of 4 to one of these base cases, we can conclude
that indeed every postage amount greater than or equal to 12 cents is
constructible from 4-cent and 5-cent stamps; in fact, at most three
5-cent stamps are needed.
 
By \textbf{Well-Ordering}:  

Begin with the set of all
\emph{counterexamples}, namely,
\[
\set{\, n \ge 12 \suchthat \text{$n$-cent postage can\emph{not} be
    formed with 4-cent and 5-cent stamps} \,}
\]

Assuming that this set is not empty, Well ordering implies it has a
minimum element.  Using this, a contradiction can be proved using the same
base cases as the Strong Induction proof and exactly the same reasoning
as in the Inductive Step of the Strong Induction proof.  This implies the
set of counterexamples must be empty, which implies the claim about
postage.

Finally, \textbf{Simple Induction} would work using the same proof as Strong
Induction, but with an induction hypothesis $Q(n)$ decorated with
an extra~$\forall$:
\[
Q(n) \eqdef \forall k.\; 12 \le k \le n \QIMPLIES [\text{4-cent and
    5-cent stamps can form $k$-cent postage}]
\]

Having proved $\forall n.\; Q(n)$ by Simple Induction, the desired
assertion about postage: $\forall n \ge 12. \; P(n)$ would then be an
immediate corollary.
\end{solution}

\end{problem}

\endinput
