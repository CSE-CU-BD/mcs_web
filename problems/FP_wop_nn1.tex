\documentclass[problem]{mcs}

\begin{pcomments}
\pcomment{FP_wop_nn1}
\pcomment{rewrite of FP_induction_nn1}
\pcomment{from F04rec2}
\pcomment{edited by ARM 9/20/15, 2/17/17}
\end{pcomments}

\pkeywords{
  induction
  series
  sum
}

%%%%%%%%%%%%%%%%%%%%%%%%%%%%%%%%%%%%%%%%%%%%%%%%%%%%%%%%%%%%%%%%%%%%%
% Problem starts here
%%%%%%%%%%%%%%%%%%%%%%%%%%%%%%%%%%%%%%%%%%%%%%%%%%%%%%%%%%%%%%%%%%%%%

\begin{problem}
Use the Well Ordering Principle to prove that
\begin{equation}\tag{*}
1 \cdot 2 + 2 \cdot 3 + 3 \cdot 4 + \dots + n (n+1)
    = \frac{n (n+1) (n+2)}{3}
\end{equation}
for all integers $n\geq 1$.

\begin{solution}

\begin{proof}
Suppose~(*) fails for some integer $n \geq 1$.  By the Well Ordering
Principle there is a minimum integer $m \geq 1$ that is a
counter-example to~(*)---that is,~(*) is not true when we let $n = m$.
\iffalse that is,
\begin{equation}\label{mm1m22}
1 \cdot 2 + 2 \cdot 3 + 3 \cdot 4 + \dots + m (m+1)
    \neq \frac{m (m+1) (m+2)}{3},
\end{equation}
but~\eqref{nn1n2} holds for all $n$ such that $1 \leq n < m$.
\fi

But when $n=1$, the left hand side of~(*) is $1 \cdot 2 = 2$, and the
right hand side is $(1\cdot 2 \cdot 3)/ 3 = 2$.  So~(*) is true for
$n=1$, which means that the counter-example $m$ must be greater than
one.  Now since $1 \leq m-1 <m$, and $m$ is a minimum counter-example, it
must be that~(*) is true when we let $n = m-1$, namely
\[
1 \cdot 2 + 2 \cdot 3 + 3 \cdot 4 + \dots + (m-1) ((m-1)+1) =
     \frac{(m-1) ((m-1)+1) ((m-1)+2)}{3}.
\]
Simplifying, we have
\begin{equation}\tag{**}
1 \cdot 2 + 2 \cdot 3 + 3 \cdot 4 + \dots + (m-1) m =
     \frac{(m-1) m (m+1)}{3}
\end{equation}
Adding $m(m+1)$ to both sides of~(**), we obtain
\begin{align*}
\lefteqn{[1 \cdot 2 + 2 \cdot 3 + \dots + (m-1)m] + m (m+1)}\\
    & = \frac{(m-1) m (m+1)}{3} + m(m+1) & \text{by~(**)}\\
    & = \frac{(m-1) m (m+1)}{3} + \frac{3m(m+1)}{3}\\
    & = \frac{((m-1)+3) [m (m+1)]}{3}\\
    & = \frac{m(m+1)(m+2)}{3}.
\end{align*}
This shows that~(*) is actually true when we let $n=m$, contradicting
the hypothesis that $m$ was a counter-example.\footnote{We spelled out
  the last three algebraic simplification steps here.  In general,
  such routine algebra steps would, and should, be skipped.}

Since the assumption that~(*) failed for some $n\geq 1$ led to a
contradiction, we conclude that~(*) is true for all $n\geq 1$.
\end{proof}

\inbook{We have convincingly shown that equation~(*) holds for all
  $n\geq 1$, but the proof gives no hint about how someone came up
  with the right hand formula in~(*).  %Ref to generating function chapter.}
\end{solution}

\end{problem}

%%%%%%%%%%%%%%%%%%%%%%%%%%%%%%%%%%%%%%%%%%%%%%%%%%%%%%%%%%%%%%%%%%%%%
% Problem ends here
%%%%%%%%%%%%%%%%%%%%%%%%%%%%%%%%%%%%%%%%%%%%%%%%%%%%%%%%%%%%%%%%%%%%%

\endinput
