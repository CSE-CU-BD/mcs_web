\documentclass[problem]{mcs}

\begin{pcomments}
  \pcomment{FP_string_counting}
  \pcomment{by Rich 12/9/09}
  \pcomment{renamed from FP_dice_rolling}
\end{pcomments}

\pkeywords{
  combinatorics
  counting
}

%%%%%%%%%%%%%%%%%%%%%%%%%%%%%%%%%%%%%%%%%%%%%%%%%%%%%%%%%%%%%%%%%%%%%
% Problem starts here
%%%%%%%%%%%%%%%%%%%%%%%%%%%%%%%%%%%%%%%%%%%%%%%%%%%%%%%%%%%%%%%%%%%%%

\begin{problem}

%\textbf{Combinatorics and Counting}

\iffalse

\ppart
Suppose that we are flipping a fair coin $n$ times. What is 
the probability that there are exactly $k$ heads, where the heads
must be separated by at least 2 tails?

\begin{solution}
  There are $\binom{n-2k+2}{k}$ number of ways to choose where to 
  place the heads (consider a bijection with having $k$ 1s 
  in an $n-2k+2$ bit binary string). Note that it is $n-2(k-1)$ 
  because the last head does not need to be followed by 2 tails.
  
  The total number of results is $2^n$, so the probability is:
  
  \[ \binom{n-2k+2}{k} \over {2^n} \]
\end{solution}

\ppart
\fi

An evil TA was trying to make a difficult probability question for the
Finals. The question goes as follows:

``Suppose that we are flipping a fair coin $n \geq 5$ times.  What is the
probability that there are at least $5$ consecutive heads?  For example,
\textit{THTTHHHHHT} and \textit{HHHHHHHHH} are such strings, but
\textit{HHHHTHHHHT} is not.''

The TA then gave a solution that turned out to be wrong:

\begin{falseproof}
  Consider the possible results as a length $n$ binary string.  The total
  of all possible results is therefore $2^n$.

We can then count the total number of strings that have $5$ 
consecutive ones: there are $n - 4$ places to begin the
length $5$ substring and $2^{n-5}$ ways to fill in the 
remaining $n-5$ bits.  There are, therefore, a total of 
$(n-4) \cdot 2^{n-5}$ possible strings.

The probability is the number of results with $5$ consecutive ones 
divided by the number of all possible results:
\begin{equation}\label{n416}
(n-4) \cdot \frac{2^{n-5}}{2^n} = \frac{n-4}{32}
\end{equation}
\end{falseproof}

\bparts

\ppart Give a counter-example to~\eqref{n416}.

\examspace[1.5in]
\begin{solution}
For $n = 6$, there are only 3 strings with 5 consecutive zeros:
$1^6, 01^5, 1^5 0$, so the probability of getting 5 consecutive ones is
$3/64 \neq 1/8$.
\end{solution}

\ppart Explain what the mistake is in the false proof.  You do
\textit{not} need to correct the proof or solve for the actual answer.

\examspace[2in]
\begin{solution}
  The mistake is in over-counting strings with more than 5
  consecutive ones.  For instance, for $n = 6$, the string
  \textit{111111} will be counted in the case where the substring 
  starts at the first 1 and again in the case where the substring
  starts at the second 1.
\end{solution}

\eparts
\end{problem}
