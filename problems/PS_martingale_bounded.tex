\documentclass[problem]{mcs}

\begin{pcomments}
  \pcomment{INCOMPLETE DRAFT}
  \pcomment{formerly PS_st_petersburg_bounded}
  \pcomment{related to CP_st_petersburg, CP_fair_st_petersburg}
  \pcomment{ARM 5/28/12}
\end{pcomments}

\pkeywords{
  martingale
  paradox
  expectation
}

%%%%%%%%%%%%%%%%%%%%%%%%%%%%%%%%%%%%%%%%%%%%%%%%%%%%%%%%%%%%%%%%%%%%%
% Problem starts here
%%%%%%%%%%%%%%%%%%%%%%%%%%%%%%%%%%%%%%%%%%%%%%%%%%%%%%%%%%%%%%%%%%%%%

\iffalse
A gambler bets \$10 on ``red'' at a roulette table (the odds of red are
18/38, slightly less than even) to win \$10.  If he wins, he gets
back twice the amount of his bet and he quits.  Otherwise, he doubles his
previous bet and continues.
\fi

\begin{problem}
A gambler bets on the toss of a fair coin: if the toss is Heads, the
gambler wins the amount of his bet.  If he loses, he doubles his bet
on the next toss, and continues in this way until a Head finally gets
tossed.  

\iffalse
\item
For example, suppose his first bet is \$10.  If the first toss is
Heads, the gambler gets his original bet plus \$10 back and walks away
a \$10 winner.  If the first toss is Tails, he loses his \$10 bet, but
then doubles his bet to \$20.  Now if the second toss is Heads, he
gets his \$20 bet plus \$20 back and again walks away with a net win
of $20-10 = \$10$.  Continuing in this way, he will win \$10 whenever
a Head finally gets tossed.  \fi

\bparts

\ppart Verify that if there was a fixed number, $n$, of bets, the
gamblers expected win is zero.


The first thing that's wrong is the argument claiming that the
expectation is 0.  It would be 0 if the number of bets had a fixed
bound.  If you could only make $n$ bets, then your expectation in the
fair game would be the sum of your expected wins on each of the bets,
namely, $n \cdot 0 = 0$.  But there is no such fixed bound, and that
changes things.

To explain this carefully, let $C_i$ be the number of dollars won on
the $i$th spin.  So $C_i = 2^{i-1}$ when red comes up for the first time
on the $i$th spin, and $C_i = -2^{i-1}$, when the first red spin comes up
after the $i$th spin.  We can define $C_i$ to be 0 if the first red
comes up before the $i$th spin.  This means
\[
\expect{C_i} = 0.
\]
Also, the total of your winnings is 
\[
C \eqdef \sum_{i \in \integers^+} C_i.
\]
The conclusion that $\expect{C} = 10$ follows from Total
Expectation, conditioning on the number of spins till a red first
occurs.  Namely, if the first red occurs on the $i$th spin, the amount
won is
\[
-10 \cdot (1 + 2 + 2^2 + \dots + 2^{i - 2}) + 10 \cdot 2^{i-1}   = 10.
\]
Then by Total Expectation,
\begin{align*}
\expect{C}
  & = \sum_{i  \in \integers^+}
      \expcond{C}{\text{first red on $i$th spin}}
      \cdot \prob{\text{first red on $i$th spin}}\\
  & = \sum_{i  \in \integers^+} 10 \cdot 2^{-i}
    = 10 \cdot \sum_{i  \in \integers^+} 2^{-i} = 10 \cdot 1 = 10.
\end{align*}

So sure enough,
\begin{equation}\label{expCsum10}
\expect{C} \eqdef \Expect{\sum_{i \in \integers^+} C_i} = 10.
\end{equation}

But since $\expect{C_i} = 0$,
\begin{equation}\label{sumCi0}
\sum_{i \in \integers^+} \expect{C_i} = \sum_{i \in \integers^+} 0 = 0.
\end{equation}
It seems that~\eqref{sumCi0} and~\eqref{expCsum10} contradict each
other, but they don't.  The apparent contradiction comes from applying
infinite linearity to conclude
\begin{falseclm*}
\[
\Expect{\sum_{i \in \integers^+} C_i} = \sum_{i \in \integers^+} \expect{C_i}.
\]
\end{falseclm*}
But this is a case where the convergence conditions required for
infinite linearity don't hold.  Even though the left hand sum
\idx{converges} (to 10) and the right hand sum converges (to 0), the
infinite linearity Theorem~\eqref{linexp} requires that the sum of
expectations of \emph{\idx{absolute value}s} converges.  That is,
infinite linearity would follow if the sum
\begin{equation}\label{sumabsC}
\sum_{i \in \integers^+} \expect{\, \abs{C_i}\, }
\end{equation}
converged.  But
\begin{align*}
\expect{\, \abs{C_i}\, }
  & = (\abs{10 \cdot 2^{i-1}}) \cdot \prob{\text{1st red in $i$th spin}}\\
  &\quad  + (\abs{-10 \cdot 2^{i-1}}) \cdot \prob{\text{1st red after $i$th spin}}\\
  &\quad  + 0 \cdot \prob{\text{1st red before the $i$th spin}}\\
  & = (10 \cdot 2^{i-1}) \cdot 2^{-(i)}
       + (10 \cdot 2^{i-1}) \cdot 2^{-(i)}
       + 0 = 10,
\end{align*}
so the sum~\eqref{sumabsC} diverges---rapidly.

Probability theory truly leads to this absurd conclusion: a game
allowing an unbounded number of fair bets may not be fair in the end.
In fact, even against an \emph{unfair} wheel, as long as there is some
fixed positive probability of red on each spin, you are certain to win
$\$10$ playing the Martingale strategy!

\end{problem}

%%%%%%%%%%%%%%%%%%%%%%%%%%%%%%%%%%%%%%%%%%%%%%%%%%%%%%%%%%%%%%%%%%%%%
% Problem ends here
%%%%%%%%%%%%%%%%%%%%%%%%%%%%%%%%%%%%%%%%%%%%%%%%%%%%%%%%%%%%%%%%%%%%%

\endinput
