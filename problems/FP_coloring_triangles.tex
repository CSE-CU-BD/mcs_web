\documentclass[problem]{mcs}

\begin{pcomments}
  \pcomment{FP_coloring_triangles}
  \pcomment{from F03.final}
\end{pcomments}

\pkeywords{
 coloring
 probability
 expectation
 variance
}

%%%%%%%%%%%%%%%%%%%%%%%%%%%%%%%%%%%%%%%%%%%%%%%%%%%%%%%%%%%%%%%%%%%%%
% Problem starts here
%%%%%%%%%%%%%%%%%%%%%%%%%%%%%%%%%%%%%%%%%%%%%%%%%%%%%%%%%%%%%%%%%%%%%

\begin{problem}
Let $T_n$ be the graph consisting of $n$ consecutive triangles arranged as
follows:
\begin{verbatim}
                 __________________________ ...__________
                /\      /\      /\      /       /\      /
               /  \ 2  /  \    /  \    /   ... /  \  n /
              / 1  \  / 3  \  /    \  /            \  /
             /______\/______\/______\/___ ... ______\/

\end{verbatim}

\bparts

\ppart Each edge in $T_n$ is colored red with probability $r$ and blue
with probability $b \eqdef 1 - r$ mutually independently.  A triangle
is \emph{\idx{monochromatic}} if its edges are all blue or all red.  What is
the probability, $m$, that a particular triangle is monochromatic?

\begin{center}
\exambox{1.5in}{0.5in}{-0.2in}
\end{center}
\examspace[0.5in]

\begin{solution}
The triangle is all red with probability $r^3$
and all blue with probability $b^3$, and so it is monochromatic with
probability $m = r^3 + b^3$.
\end{solution}

\ppart Let $I_T$ be an indicator random variable for the event that a
given triangle, $T$, is monochromatic.

\begin{center}
What is $\expect{I_T}$? \exambox{0.7in}{0.5in}{-0.2in} \qquad
What is $\variance{I_T}$?  \exambox{0.7in}{0.5in}{-0.2in}
\end{center}
You may state your answer in terms of the probability, $m$, from the
previous problem part.

\begin{solution}

\begin{align*}
\expect{I_T} = & \pr{I_T = 1} = m & \text{since $I_T$ is 0-1 valued.}\\
\variance{I_T} = & \expect{I_T^2} - \expect{I_T}^2 \\
               = & m - m^2.
\end{align*}

\end{solution}

\ppart Let $T$ and $T'$ be any two different triangles.  If the
triangles don't share an edge, then the random variables $I_T$ and
$I_{T'}$ are obviously independent.  Suppose that $r = 1/2$ and $T$
and $T'$ do share an edge.  Show that $I_T$ and $I_{T'}$ remain
independent in this case.

\examspace[2.0in]

\begin{solution}
It is enough to show that
\begin{eqnarray*}
\pr{I_T = 1 \intersect I_{T'} = 1} & = & \pr{I_T = 1} \cdot \pr{I_{T'} = 1}.
\end{eqnarray*}
The probability on the left is the probability that all five edges in $T
\union T'$ are the same color, namely, $r^5 + b^5 = 1/16$, and the
probability on the right is $(r^3 + b^3)^2 = 1/16$.
\end{solution}

\iffalse

\ppart[3] For each triangle in the graph, we can consider the
event that the triangle is monochromatic.  Are all such
events mutually independent when $r = 1/2$?  Justify your answer.

\solution[\vspace{3.5in}]{Yes.}
\fi


\ppart Let $M$ be the random variable equal to the total number of
monochromatic triangles in the graph.  If $r = 1/2$, what is
$\variance{M}$?

\begin{center}
\exambox{0.75in}{0.5in}{-0.2in}
\end{center}
\examspace[0.5in]


\begin{solution}
$3n/16$.

This follows because in this case $m = r^3+b^3 = 1/4$, and
\begin{eqnarray*}
M & = & I_{T_1} + I_{T_2} + \ldots + I_{T_n},
\end{eqnarray*}
and since the indicator random variables are pairwise independent, we have:
\begin{eqnarray*}
\variance{M} & = & \variance{I_{T_1}} +
                   \variance{I_{T_2}} + \ldots +
                   \variance{I_{T_n}} \\
             & = & n\variance{T_1}\\
             &=  & n(m-m^2)= n(3/16).
\end{eqnarray*}

\end{solution}

%\examspace

\ppart Prove that
\[
\lim_{n\to\infty} \pr{\abs{M - \expect{M}} \geq n/1000} = 0.
\]

\examspace[3.5in]
\begin{solution}

\begin{align*}
\pr{\abs{M - \expect{M}} \geq n/1000} & \leq  \variance{M}/(n/1000)^2
           & \text{by Chebyshev}\\
       & = (3n/16)10^6/n^2\\
       & = 3\cdot 10^6/16n\\
       & \to 0  & \text{as } n \to \infty.
\end{align*}

\end{solution}

\eparts

\end{problem}

%%%%%%%%%%%%%%%%%%%%%%%%%%%%%%%%%%%%%%%%%%%%%%%%%%%%%%%%%%%%%%%%%%%%%
% Problem ends here
%%%%%%%%%%%%%%%%%%%%%%%%%%%%%%%%%%%%%%%%%%%%%%%%%%%%%%%%%%%%%%%%%%%%%

\endinput
