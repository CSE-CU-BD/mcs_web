\documentclass[problem]{mcs}

\begin{pcomments}
  \pcomment{PS_symmetric_relations}
  \pcomment{from: F09.ps3, S02.ps3}
\end{pcomments}

\pkeywords{
  symmetric
  relations
  relational_properties
  inverse
}

%%%%%%%%%%%%%%%%%%%%%%%%%%%%%%%%%%%%%%%%%%%%%%%%%%%%%%%%%%%%%%%%%%%%%
% Problem starts here
%%%%%%%%%%%%%%%%%%%%%%%%%%%%%%%%%%%%%%%%%%%%%%%%%%%%%%%%%%%%%%%%%%%%%

\begin{problem}

Suppose $R$ and $S$ are binary relations on a set $A$, and
that both $R$ and $S$ are symmetric.  Must each of the following new
relations be symmetric?  For each part, justify your answer with a
brief argument if the new relation is symmetric and a counterexample
if it is not.

\bparts
\ppart $R^{-1}$

\begin{solution}
$R^{-1}$: \textbf{Yes.} Suppose $(x, y) \in R^{-1}$.  By
the definition of inverse, this means that $yRx$.  But since $R$ is
symmetric, this means that $xRy$.  By the definition of inverse again,
this means that $(y, x) \in R^{-1}$.  So $R^{-1}$ is symmetric.
\end{solution}

\ppart $R \intersect S$

\begin{solution}
\textbf{Yes.} If $(x, y) \in R \intersect S$, we must prove
that $(y, x) \in R \intersect S$.  We have that $xRy$ (in which case
$yRx$ as well) and $xSy$ (in which case $ySx$ as well).  Since $yRx
\land ySx$, we have $(y, x) \in R \intersect S$.
\end{solution}

\ppart $R \composition S$

\begin{solution}
\textbf{No.}  Suppose $A = \set{a,b,c}$, $R = \set{(b, c), (c, b)}$ ,
and $S = \set{(a, b), (b, a)}$ .  Then $R \composition S = \set{(a,
  c)}$, which is not symmetric.
\end{solution}

\ppart $R \composition R$

\begin{solution}
\textbf{Yes.} $R^2$ is symmetric.  For every pair
$(a,b)$ that is a member of $R$, $(b,a)$ must also be in $R$.  Then in
the composition, $R^2$, $(a,a)$ is certainly symmetric.  What about
the case where $(a,b)$ is in $R$, and so is $(b,c)$?  $(a,c)$ will be
in $R^2$, but so will $(c,a)$ since $(c,b)$ and $(b,a)$ are both in
$R$ because of symmetry. In fact, $R^n$ is symmetric if $R$ is
symmetric.
\end{solution}

\eparts

\end{problem}

%%%%%%%%%%%%%%%%%%%%%%%%%%%%%%%%%%%%%%%%%%%%%%%%%%%%%%%%%%%%%%%%%%%%%
% Problem ends here
%%%%%%%%%%%%%%%%%%%%%%%%%%%%%%%%%%%%%%%%%%%%%%%%%%%%%%%%%%%%%%%%%%%%%

\endinput
