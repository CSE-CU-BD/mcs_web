\documentclass[problem]{mcs}

\begin{pcomments}
  \pcomment{PS_3_and_5_postage_by_WOP}
  \pcomment{from: S09.ps3}
\end{pcomments}

\pkeywords{
  WOP
  well_ordering
  postage
  stamps
}

%%%%%%%%%%%%%%%%%%%%%%%%%%%%%%%%%%%%%%%%%%%%%%%%%%%%%%%%%%%%%%%%%%%%%
% Problem starts here
%%%%%%%%%%%%%%%%%%%%%%%%%%%%%%%%%%%%%%%%%%%%%%%%%%%%%%%%%%%%%%%%%%%%%

\begin{problem}
Use the Well Ordering Principle to prove that any integer greater than or
equal to 8 can be represented as the sum of integer multiples of 3 and 5.

\begin{solution}
\begin{claim}
For all $n \geq 8$, it is possible to represent $n$ as a sum of integer
multiples of 3 and 5.
\end{claim}

\begin{proof}
  The proof is by the Well Ordering Principle.  Let $P(n)$ be the
  predicate that it is possible to produce $n$ as a sum of integer multiples
  of 3 and 5. 

  Let $C = \set{ n \geq 8 | \neg P(n)}$ be the set of counter
  examples.  Assume for the sake of contradiction that $C$ is not
  empty.  Then by the Well Ordering Principle, $C$ must have some minimum
  element $m\in C$.  

  First, observe that $P(n)$ is true for the following small values of $n$.
  \begin{itemize}
  \item $n=8$: $8 = 3 + 5$.
    
  \item $n=9$: $9 = 3 + 3 + 3$.

  \item $n=10$: $10 = 5 + 5$.
  \end{itemize}

  We thus have $m \geq 11$.  For $m \geq 11$, we have $m-3 \geq 8$.  Since
  $m$ is the smallest counterexample, we can represent $m - 3$ as the sum
  of integer multiples of 3 and 5 Thus we can represent $m$ by adding 1 to
  the coefficient of 3 in our representation of $m-3$.  Therefore, $m$ is
  not a counterexample, contradicting the assumption that $C$ is nonempty.
\end{proof}
\end{solution}
\end{problem}

%%%%%%%%%%%%%%%%%%%%%%%%%%%%%%%%%%%%%%%%%%%%%%%%%%%%%%%%%%%%%%%%%%%%%
% Problem ends here
%%%%%%%%%%%%%%%%%%%%%%%%%%%%%%%%%%%%%%%%%%%%%%%%%%%%%%%%%%%%%%%%%%%%%

\endinput
