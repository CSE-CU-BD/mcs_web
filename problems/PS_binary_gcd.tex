\documentclass[problem]{mcs}

\begin{pcomments}
  \pcomment{PS_binary_gcd}
  \pcomment{DRAFT based on Shoup}
\end{pcomments}

\pkeywords{
GCD
binary_GCD
state_machine
}

%%%%%%%%%%%%%%%%%%%%%%%%%%%%%%%%%%%%%%%%%%%%%%%%%%%%%%%%%%%%%%%%%%%%%
% Problem starts here
%%%%%%%%%%%%%%%%%%%%%%%%%%%%%%%%%%%%%%%%%%%%%%%%%%%%%%%%%%%%%%%%%%%%%

\begin{problem}
  The binary-GCD state machine computes the GCD of $a$ and $b$ using only
  division by 2 and subtraction, which makes it run very efficiently on
  hardware that uses binary representation of numbers.  In practice, it
  runs more quickly that the Euclidean algorithm state
  machine~(\bref{euclid_transition}).

\begin{align*}
\text{states} & \eqdef \naturals^3\\
\text{start state} & \eqdef (a, b, 1)
             & \text{(where $a \ge b >0$)}\\
\text{transitions} & \eqdef (x,  y, e) \movesto\\
     &\qquad \begin{cases}
       (x/2, y/2, 2e) & \text{(if $2 \divides x$ and $2 \divides y$)},\\
       (x/2, y, e)    & \text{(if $2 \divides x$ and $2 \not\divides y$)}\\
       (x, y/2, e)    & \text{(if $2 \not\divides x$ and $2 \divides y$)}\\
       (x-y,y,e)      & \text{(if $2 \not\divides x$ and
                        $2 \not \divides y$ and $x-y > y$)}\\
       (y,x-y,e)      & \text{(if $2 \not\divides x$ and
                        $2 \not \divides y$ and $x-y < y$)}\\
       (x, y, ey) & \text{(if $2 \not\divides x$ and
                        $2 \not \divides x$ and $x-y = y$)}
       \end{cases}
\end{align*}
Prove that if this machine reaches a ``final'' state in which no
transition is possible, then $y = \gcd(a,b)$.

\end{problem}

%%%%%%%%%%%%%%%%%%%%%%%%%%%%%%%%%%%%%%%%%%%%%%%%%%%%%%%%%%%%%%%%%%%%%
% Problem ends here
%%%%%%%%%%%%%%%%%%%%%%%%%%%%%%%%%%%%%%%%%%%%%%%%%%%%%%%%%%%%%%%%%%%%%

\endinput

