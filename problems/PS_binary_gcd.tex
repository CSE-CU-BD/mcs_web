\documentclass[problem]{mcs}

\begin{pcomments}
  \pcomment{PS_binary_gcd}
  \pcomment{extended version in FP_binary_gcd}
  \pcomment{By ARM 4/24/11 based on Shoup}
  \pcomment{revised ARM 3/17/14; 5/10/14}
\end{pcomments}

\pkeywords{
  preserved_invariant
  GCD
  binary_GCD
  state_machine
}

%%%%%%%%%%%%%%%%%%%%%%%%%%%%%%%%%%%%%%%%%%%%%%%%%%%%%%%%%%%%%%%%%%%%%
% Problem starts here
%%%%%%%%%%%%%%%%%%%%%%%%%%%%%%%%%%%%%%%%%%%%%%%%%%%%%%%%%%%%%%%%%%%%%

\begin{problem}
  The \term{Binary GCD} state machine computes the GCD of integers $a,
  b > 0$ using only division by 2 and subtraction, which makes it run
  very efficiently on hardware that uses binary representation of
  numbers.  In practice, it runs more quickly than the more famous
  Euclidean algorithm described in Section~\bref{sec: Euclid}.

\begin{align}
\text{states}  \eqdef     &\nngint^3\notag\\
\text{start state} \eqdef & (a, b, 1)\notag\\
\text{transitions} \eqdef & \text{ if } \min(x,y) > 0, \text{ then } (x,  y, e) \movesto\notag\\
     &  (x/2, y/2, 2e) & \text{(if $2 \divides x$ and $2 \divides y$)}\tag{i1}\\
     &  (x/2, y, e)    & \text{(else if $2 \divides x$)}\tag{i2}\\
     &  (x, y/2, e)    & \text{(else if $2 \divides y$)}\tag{i3}\\
     &  (x-y, y, e)    & \text{(else if $x > y$)}\tag{i4}\\
     &  (y-x, x, e)    & \text{(else if $y > x$)}\tag{i5}\\
     &  (1,   0, ex)   & \text{(otherwise ($x=y$))}.\tag{i6}
\end{align}

\bparts

\ppart Use the Invariant Principle \iffalse
(Section~\bref{subsec:invariant})\fi to prove that if this machine
stops, that is, reaches a state $(x,y,e)$ in which no transition is
possible, then $e = \gcd(a,b)$.

\begin{solution}
We claim that a preserved invariant of this machine is
\[
\gcd(a,b) = e\gcd(x,y).
\]
To show this, we assume the invariant holds for state $(x,y,e)$ and
show that if $(x,y,e) \movesto (x',y',e')$, then $\gcd(a,b) =
e'\gcd(x',y')$.

The proof is by cases according to which kind of transition occurs.

\inductioncase{Case}~(i1): ($2 \divides x$ and $2 \divides y$).  
In this case, $(x',y',e') = (x/2, y/2, 2e)$.

We use the the gcd property from Lemma~\bref{lem:gcd-hold}
\begin{equation}\tag{agcd}
\gcd(au,av) = a\gcd(u,v).
\end{equation}

Now
\begin{align*}
\gcd(a,b)
  & =  e\gcd(x,y)
        & \text{(by the invariant for $(x,y,e)$)}\\
  & = e 2\gcd(x/2,y/2)
        & \text{(by~(agcd))}\\
  & = e'\gcd(x',y'),
\end{align*}
which shows that the invariant holds for $(x',y',e')$.

\inductioncase{Case}~(i2): ($2 \divides x$ and 2 does not divide $y$).
In this case, $(x',y',e') = (x/2,y, e)$.

We use the easily verified fact
\begin{equation}\tag{acancel}
\gcd(au,v) = \gcd(u,v)
\end{equation}
for $a$ relatively prime to $v$.

Now
\begin{align*}
\gcd(a,b)
  & = e\gcd(x,y)
      & \text{(invariant for $(x,y,e)$)}\\
  & = e\gcd(x/2,y)
      & \text{(by~(acancel))}\\
  & = e'\gcd(x',y'),
\end{align*}
which shows that the invariant holds for $(x',y',e')$.

\inductioncase{Case}~(i4): (neither $x$ nor $y$ is even).

In this case $(x',y',e') = (x-y,y,e)$.

We use the easily verified fact that
\begin{equation}\tag{gcdu-v}
\gcd(u-v,v) = \gcd(u,v).
\end{equation}

Now,
\begin{align*}
\gcd(a,b)
  & = e\gcd(x,y) & \text{(invariant for $(x,y,e)$)}\\
  & = e\gcd(x-y,y) & \text{(by~(gcdu-v))}\\
  & = e'\gcd(x',y'),
\end{align*}
proving that the invariant holds for $(x',y',e')$.

\inductioncase{Case}~(i6): ($x=y$).  In this case, $(x',y',e') = (1,0,ex)$.
Now we have
\begin{align*}
\gcd(a,b) 
    & = e\gcd(x,x) 
          & \text{(by the invariant for $(x,x,e)$)}\\
    & = ex\\
    & = ex\gcd(1,0)
          & \text{(since $\gcd(1,0) = 1$)}\\
    & =  e'\gcd(x',y'),
\end{align*}
which shows that the invariant holds for $(x',y',e')$.

Verification of the remaining cases follows similarly.

To apply the Invariant Principle, we now first observe that the
preserved invariant holds trivially in the start state $(a,b,1)$
because $\gcd(a,b) = 1\cdot\gcd(a,b)$.  We conclude that the preserved
invariant holds in every reachable state.

We claim that only rule~(i6) can lead to a stopped state,
which must be of the form $(1,0,e')$.  If such a state is reachable,
then the invariant implies
\[
\gcd(a,b) = e'\gcd(1,0) = e',
\]
as required.

To see why this is the only stopped state, note that a transition is
always possible from the start state, and
transitions~(i1)--(i5) do not lead to stopped
states since the minimum of the new $x$ and $y$ values remains
positive.
\end{solution}

\ppart Prove that rule~(i1)
\[
(x, y, e) \to (x/2, y/2, 2e)
\]
is never executed after any of the other rules is executed.

\begin{staffnotes}
\hint An invariant about the parity of $x$ and $y$.
\end{staffnotes}

\begin{solution}
We claim that another preserved invariant is
\begin{equation}\tag{noteven}
\QNOT(2 \divides x\ \QAND\ 2 \divides y).
\end{equation}

To verify this, suppose a state $(x,y,e)$ satisfies~(noteven).  Then
rule~(i1) will not be executed in that state.

Suppose the second rule~(i2) gets executed leading to state
$(x/2, y ,e)$.  Then $x$ must be even and $y$ must be odd, and so this
state satisfies~(noteven) since $y$ is odd.  A symmetric
argument applies to the third rule~(i3).

If rule~(i4) gets executed leading to state $(x-y,y,e)$.
Then if $y$ is odd this state satisfies~(noteven), and if $y$ is
even, then $x$ must be odd, so this state satisfies~(noteven)
because $x-y$ must be odd.  A symmetric argument applies to
rule~(i5).

Finally, rule~(i6) leading to state  $(1,0,ex)$
trivially satisfies~(noteven) since 1 is odd.

Now if rule~(i1) is not executed in some state $(x,y,e)$,
then~(noteven) must hold, and since~(noteven) is
preserved, we can conclude that rule~(i1) will never be
executed in any subsequent state.
\end{solution}

\ppart Prove that the machine reaches a final state in at most
$1+2(\log a +\log b)$ transitions.  (This is a coarse bound; you may
be able to get a better one.)

%\hint Strong induction on $\max(a,b)$.

\begin{solution}
Either $x$ or $y$ gets halved after at most three transitions---the
worst case is when $x$ and $y$ are both odd and $y > x > 1$.  In that
case, the first transition switches $x$ and $y$, the next subtracts
one from the other yielding an even value of $x$, and the third
transition will halve $x$.  So after at most $3(\log a + \log b)$
transitions, one of $x$ and $y$ must have been reduced to 1, after
which there can be at most one more transition.
\end{solution}


\eparts

\end{problem}

%%%%%%%%%%%%%%%%%%%%%%%%%%%%%%%%%%%%%%%%%%%%%%%%%%%%%%%%%%%%%%%%%%%%%
% Problem ends here
%%%%%%%%%%%%%%%%%%%%%%%%%%%%%%%%%%%%%%%%%%%%%%%%%%%%%%%%%%%%%%%%%%%%%

\endinput

