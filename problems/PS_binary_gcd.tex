\documentclass[problem]{mcs}

\begin{pcomments}
  \pcomment{PS_binary_gcd}
  \pcomment{By ARM 4/24/11 based on Shoup}
\end{pcomments}

\pkeywords{
  strong_induction
  GCD
  binary_GCD
  state_machine
}

%%%%%%%%%%%%%%%%%%%%%%%%%%%%%%%%%%%%%%%%%%%%%%%%%%%%%%%%%%%%%%%%%%%%%
% Problem starts here
%%%%%%%%%%%%%%%%%%%%%%%%%%%%%%%%%%%%%%%%%%%%%%%%%%%%%%%%%%%%%%%%%%%%%

\begin{problem}
  The binary-GCD state machine computes the GCD of $a$ and $b$ using
  only division by 2 and subtraction, which makes it run very
  efficiently on hardware that uses binary representation of numbers.
  In practice, it runs more quickly than the more famous Euclidean
  algorithm described in Section~\bref{sec: Euclid}.

\begin{align*}
\text{states} & \eqdef \naturals^3\\
\text{start state} & \eqdef (a, b, 1)
             & \text{(where $a > b >0$)}\\
\text{transitions} & \eqdef \text{ if } \min(x,y) > 0, \text{ then } (x,  y, e) \movesto\\
     &\qquad \text{the first possible state according to the rules:}\\
     &\qquad
       \begin{cases}
       (1, 0, ex)     & \text{(if $x = y$)}\\
       (1, 0, e)      & \text{(if $y=1$)},\\
       (x/2, y/2, 2e) & \text{(if $2 \divides x$ and $2 \divides y$)},\\
       (x/2, y, e)    & \text{(if $2 \divides x$)}\\
       (x, y/2, e)    & \text{(if $2 \divides y$)}\\
       (y, x, e)      & \text{(if $y>x$)}\\
       (x-y,y,e)      & \text{(otherwise)}.
       \end{cases}
\end{align*}

\bparts 

\ppart Prove that if this machine stops, that is, reaches a state
$(x,y,e)$ in which no transition is possible, then $e = \gcd(a,b)$.

\begin{solution}
Invariant is $\gcd(a,b) = e\gcd(x,y)$.  To show this, we assume the
invariant holds for state $(x,y,e)$ and show that if $(x,y,e) \movesto
(x',y',e')$, then $\gcd(a,b) = e'\gcd(x',y')$.

The proof is by cases according to which kind of transition occurs.

\inductioncase{Case}: ($x=y$).  in this case we have $\gcd(x,y) = x$
so by the invariant, $\gcd(a,b) = ex$.  So $e'\gcd(x',y') =
ex\gcd(1,0) = ex = \gcd(a,b)$, which shows that the invariant holds
for $(x',y',e')$.

\inductioncase{Case}: $2 \divides x$ and $2 \divides y$.  We use the
fact that $\gcd(ax,ay) = a\gcd(x,y)$. 

\inductioncase{Case}: $2 \divides x$ and 2 does not divide $y$.  Now
\begin{equation}\label{gcdxyx2}
\gcd(x,y) = \gcd(x/2,y),
\end{equation}.  So
\begin{align*}
\gcd(a,b)
  & = e\gcd(x,y)
      & \text{(invariant for $(x,y,e)$)}\\
  & = e\gcd(x/2,y)
      & \text{(by~\eqref{gcdxyx2})}\\
  & = e'\gcd(x',y')
      & \text{(since $(x',y',e') = (x/2,y,e)$)},
\end{align*}
proving that $(x',y',e')$ satisfies the invariant.

We omit the remaining cases which are just as easy.
\end{solution}

\ppart Prove that the machine reaches a final state in at most
$1+3(\log a +\log b)$ transitions.  (This is a coarse bound; you may
be able to get a better one.)

%\hint Strong induction on $\max(a,b)$.

\begin{solution}
Either $x$ or $y$ gets halved after at most three transitions---the
worst case is when $x$ and $y$ are both odd and $y > x > 1$.  In that
case, the first transition switches $x$ and $y$, the next subtracts
one from the other yielding an even value of $x$, and the third
transition will halve $x$.  So after at most $3(\log a + \log b)$
transitions, one of $x$ and $y$ must have been reduced to 1, after
which there can be at most one more transition.
\end{solution}

\eparts

\end{problem}

%%%%%%%%%%%%%%%%%%%%%%%%%%%%%%%%%%%%%%%%%%%%%%%%%%%%%%%%%%%%%%%%%%%%%
% Problem ends here
%%%%%%%%%%%%%%%%%%%%%%%%%%%%%%%%%%%%%%%%%%%%%%%%%%%%%%%%%%%%%%%%%%%%%

\endinput

