\documentclass[problem]{mcs}

\begin{pcomments}
  \pcomment{PS_pigeonhole_power_of_2}
  \pcomment{subsumes MQ_power_of_pigeon, PS_pigeonhole-power_of_3}
  \pcomment{ZDz, 11/19/15, F15.mid4}
\end{pcomments}

\pkeywords{
  Pigeonhole
  power_of_2
}

%%%%%%%%%%%%%%%%%%%%%%%%%%%%%%%%%%%%%%%%%%%%%%%%%%%%%%%%%%%%%%%%%%%%%
% Problem starts here
%%%%%%%%%%%%%%%%%%%%%%%%%%%%%%%%%%%%%%%%%%%%%%%%%%%%%%%%%%%%%%%%%%%%%

\begin{problem}

Suppose $n+1$ numbers are selected from $\set{1,2,3, \dots,2n}$.
Using the Pigeonhole Principle, show that there must be two selected
numbers whose quotient is a power of two.  Clearly indicate what are
the pigeons, holes, and rules for assigning a pigeon to a hole.
  
\hint{Factor each number into the product of an odd number and a power
  of 2.}

\begin{solution}
  The $n+1$ selected numbers are the pigeons.  There are $n$ odd
  numbers between 1 and $2n$, namely, the numbers of the form $2k+1$ for
  $0 \leq k < n$.  The pigeonholes will be the sets consisting of
  power-of-two multiples of each of these odd numbers.  For example, the
  third odd number is 5, so the third pigeonhole will be the set of
  multiples of 5 by a power of two: $\set{5,10,20,40,\dots}$.  Two pigeons
  must be assigned to some hole, and these are the required two selected
  numbers, since the quotient of the larger of these two by the smaller
  will be a power of two by definition.
\end{solution}

\end{problem}

\endinput
