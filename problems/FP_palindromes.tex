\documentclass[problem]{mcs}

\begin{pcomments}
  \pcomment{FP_palindromes}
  \pcomment{Emanuele and ARM 3/12/16}
\end{pcomments}

\pkeywords{
   recursive
   string
   reverse
   concatenation
   palindrome
}

%\newcommand{\catOK}{\text{cat-OK}}
%\newcommand{\revstr}{\text{rev}}
%\newcommand{\Zerostr}{\text{Zeroes}}


%%%%%%%%%%%%%%%%%%%%%%%%%%%%%%%%%%%%%%%%%%%%%%%%%%%%%%%%%%%%%%%%%%%%%
% Problem starts here
%%%%%%%%%%%%%%%%%%%%%%%%%%%%%%%%%%%%%%%%%%%%%%%%%%%%%%%%%%%%%%%%%%%%%

\begin{problem}
\begin{definition}
The set \finbin\ of all binary strings and their lengths has the
following basic definition:

\inductioncase{Base case}: $\emptystring \in \finbin$, and
$\lnth{\emptystring} \eqdef 0$.

\inductioncase{Constructor case}: If $s \in \finbin$, then so is $sb$
where $b \in \set{0,1}$, and $\lnth{sb} \eqdef \lnth{s}+1$.
\end{definition}

Based on this definition, we can now recursively define
\begin{definition}
The \emph{reversal} function, $\text{rev}: \finbin \to \finbin$ is defined follows:

\inductioncase{Base case}:  $\rev{\emptystring} \eqdef \emptystring$.

\inductioncase{Constructor case}: $\rev{sb} \eqdef b\rev{s}$ for
$s\in \finbin$ and $b\in \set{0,1}$.
\end{definition}

\begin{definition}
The binary \emph{palindromes} are defined recursively by:

\inductioncase{Base case}: $\emptystring$ is a palindrome.

\inductioncase{Constructor case}: If $s$ is a palindrome, then so is $bsb$
where $b \in \set{0,1}$.
\end{definition}

\bparts

\ppart Prove that $s = \rev{s}$ for all palindromes $s$.

\examspace[3.0in]

\begin{solution}
\TBA{needs to be checked!}
\end{solution}

\ppart Prove conversely that if $s = \rev{s}$, then $s$ is a
palindrome.

\hint By induction on $\lnth{s}$.

\begin{solution}
\TBA{needs to be checked!}
\end{solution}

\eparts
\end{problem}

\endinput
