\documentclass[problem]{mcs}

\begin{pcomments}
    \pcomment{PS_sat_formulas_vs_circuits}
    \pcomment{subsumed by CP_sat_formulas_vs_circuits}
    \pcomment{by ARM 2/17/13}
\end{pcomments}

\pkeywords{
  satisfiable
  circuit
  digital_circuit
  3CNF
  propositional_variable
  Boolean_function
}

\begin{problem}
It doesn't matter whether we formulate the SAT problem
(Section~\bref{SAT_sec} in terms of propositional formulas or
digital circuits.  Here's why:

Let $f$ be a Boolean function of $k$ variables.  That is,
\[
f:\set{\true,\false}^k \to \set{\true,\false}.
\]

When $P$ is a propositional formula that has, among its variables,
propositional variables labelled $X_1, \dots, X_k$.  For any truth
values $b_1,\dots,b_k \in \set{\true, \false}$, we let let
$P(b_1,\dots,b_k)$ be the result of substituting $b_i$ for all
occurrences of $X_i$ in $P$, for $1 \leq i \leq k$.

If $P_f$ is a formula such that $P_f(b_1,\dots,b_k)$ is satisfiable
exactly when $f(b_1,\dots,b_k) = \true$, we'll say that $P_f$
\emph{SAT-represents} $f$.

Suppose there is a \idx{digital circuit} using two-input, one-output
binary gates (like the circuits for binary addition in
Problems~\bref{CP_binary_adder_logic}
and~\bref{PS_faster_adder_logic}) that has $n$ wires and computes the
function $f$.  Explain how to construct a formula $P_f$ of size $cn$
that SAT-represents $f$ for some small constant $c$.  (Letting $c =6$
will work).

Conclude that the SAT problem for digital circuits---that is,
determining if there is some set of input values that will lead a
circuit to give output 1---is no more difficult than the SAT problem
for propositional formulas.

\hint Introduce a new variable for each wire.  The idea is similar to
the one used in Problem~\bref{PS_equisatisfiable_3CNF} to show that
satisfiablity of 3CNF propositional formmulas is just as hard as for
arbitrary formulas.

\begin{solution}
....
\end{solution}

\end{problem}

\endinput
