\documentclass[problem]{mcs}

\begin{pcomments}
  \pcomment{PS_Reasoner_net}
  \pcomment{from: S09.ps7; S08,ps6; F07.ps6}
  \pcomment{slightly muddled solution from latency ambiguity (do we include input edges?)}
\end{pcomments}

\pkeywords{
  networks
  congestion
  routing
  latency
}

%%%%%%%%%%%%%%%%%%%%%%%%%%%%%%%%%%%%%%%%%%%%%%%%%%%%%%%%%%%%%%%%%%%%%
% Problem starts here
%%%%%%%%%%%%%%%%%%%%%%%%%%%%%%%%%%%%%%%%%%%%%%%%%%%%%%%%%%%%%%%%%%%%%

\begin{problem}
Louis Reasoner figures that, wonderful as the Bene\v{s} network may be, the
butterfly network has a few advantages, namely: fewer switches, smaller
diameter, and an easy way to route packets through it.  So Louis designs an
$N$-input/output network he modestly calls a \emph{Reasoner-net}%
\index{communication net!Reasoner net} 
with the aim of combining the best features of both the butterfly and Bene\v{s} nets:
\begin{quote}
The $i$th input switch in a Reasoner-net connects to two switches, $a_i$
and $b_i$, and likewise, the $j$th output switch has two switches, $y_j$
and $z_j$, connected to it.  Then the Reasoner-net has an $N$-input
Bene\v{s} network connected using the $a_i$ switches as input switches and
the $y_j$ switches as its output switches.  The Reasoner-net also has an
$N$-input butterfly net connected using the $b_i$ switches as inputs and<
the $z_j$ switches as outputs.
\end{quote}

In the Reasoner-net a minimum latency routing does not have minimum
congestion.  The \emph{latency for min-congestion}%
\index{communication net!latency!for min-congestion} (\emph{LMC}) of a
net is the best bound on latency achievable using routings that
minimize congestion.  Likewise, the \emph{congestion for min-latency}%
\index{communication net!congestion!for min-latency} (\emph{CML}) is
the best bound on congestion achievable using routings that minimize
latency.

Fill in the following chart for the Reasoner-net and briefly explain your
answers.

\[
\begin{array}{|c|c|c|c|c|c|}
\hline
\textbf{diameter} &
\textbf{switch size(s)} &
\textbf{\# switches} &
\textbf{congestion} &
\textbf{LMC} &
\textbf{CML}\\
\hline
&&&&&\\
\hline
\end{array}
\]

\begin{solution}
\iffalse

\[
\begin{array}{|c|c|c|c|c|c|}
\textbf{diameter} &
\textbf{switch size(s)} &
\textbf{\# switches} &
\textbf{congestion} &
\textbf{LMC}&
\textbf{CML}\\
\hline \log N + 4 & 2 \times 2 & 3N (\log N + 1) & 1 & 2 \log N +
3 &
\sqrt{N}\\
\hline
\end{array}
\]

or
\fi

\[
\begin{array}{|c|c|c|c|c|c|}
\textbf{diameter} & \textbf{switch size(s)} & \textbf{\# switches}
& \textbf{congestion} & \textbf{LMC}&
\textbf{CML}\\
\hline \log N + 2 & 2 \times 2 & 3N (\log N + 1) & 1 & 2 \log N +1
&
\sqrt{N}\\
\hline
\end{array}
\]

%REVISE wrt to current notes---ARM
\iffalse

There are two possible answers for the diameter, because in lecture the
special ``square'' input/output switches were not used (they mess up the
recursive definitions).  To match the Notes, these switches have to be
added after the recursion, increasing the diameter by 2.  So the first
answer corresponds to the lecture notes, and the second one corresponds to
the slides.
\fi

The diameter of a Reasoner-net is 2 plus the smaller of the
diameters of its Bene\v{s} component, $2\log
N-1$, and its butterfly component, $\log N$.

The number of switches is the number of input and output switches in the
Reasoner-net, namely $2N$ plus the number of switches in its butterfly
component, $N (\log N + 1)$, and its Bene\v{s} component, $2N \log N$.

The congestion is the congestion of the better of the two component nets.

The LMC for the butterfly net equals its diameter, and likewise for the LMC
of the Bene\v{s} net.  So the LMC of the Reasoner-net is 2 plus the LMC of
the routing through the component with minimum congestion, namely, 2 plus
the diameter of the Bene\v{s} net.

The CML equals the congestion of the routing through the component
with minimum latency, namely, the congestion of butterfly net.
\end{solution}
\end{problem}

%%%%%%%%%%%%%%%%%%%%%%%%%%%%%%%%%%%%%%%%%%%%%%%%%%%%%%%%%%%%%%%%%%%%%
% Problem ends here
%%%%%%%%%%%%%%%%%%%%%%%%%%%%%%%%%%%%%%%%%%%%%%%%%%%%%%%%%%%%%%%%%%%%%
