\documentclass[problem]{mcs}

\providecommand{\RAF}{\text{RAF}}

\begin{pcomments}
  \pcomment{FP_rational_structural_induction}
  \pcomment{by kazerani on 12/5/11 verbatim from S09 Final P2}
\end{pcomments}


\pkeywords{
  rational
  structural induction
  polynomial algebra
  differentiation
  proof
}

%%%%%%%%%%%%%%%%%%%%%%%%%%%%%%%%%%%%%%%%%%%%%%%%%%%%%%%%%%%%%%%%%%%%%
% Problem starts here
%%%%%%%%%%%%%%%%%%%%%%%%%%%%%%%%%%%%%%%%%%%%%%%%%%%%%%%%%%%%%%%%%%%%%

\begin{problem}

\begin{definition*}
The \term{rational functions}, \RAF, of a single variable $x$ are defined
recursively as follows:

\vspace{.1in}
\textbf{Base cases:}
\begin{itemize}
\item The identity function, $\ide(x) \eqdef x$, and
\item any constant function
\end{itemize}
are in \RAF. % rational functions of $x$.

\vspace{.1in}
\textbf{Constructor cases:}

If $f,g$ are in \RAF, then so are $f + g, f
\cdot g, \text{ and } 1/f$.
\end{definition*}

\examspace[0.1in]

Prove by structural induction that \RAF\ is %the rational functions of $x$ are
closed under taking derivatives.  That is, using the induction
hypothesis,
\[
P(h) \eqdef [h^{\prime} \in \RAF], % \text{ is a rational function}],
\]
where $h^{\prime} \eqdef dh/dx$, prove that $P(h)$ holds for all
functions $h \in RAF$.

\begin{problemparts}

\problempart Prove the base cases of the structural induction.

\examspace[1.5in]

\begin{solution}
\begin{proof}
We must show $P(\ide(x))$ and $P(\text{constant-function})$.  But
$\ide^{\prime}$ is the constant function 1, and the derivative of a
constant function is the constant function 0, and these are in \RAF by
definition.

This proves that the induction hypothesis holds in the Base cases.
\end{proof}
\end{solution}

\problempart Prove the constructor cases of the structural induction.

\examspace[3.5in]

\begin{solution}
\begin{proof}
  Assuming $f,g \in \RAF$, $P(f)$ and $P(g)$, we must prove $P(h)$
  where

\inductioncase{Case}: ($h = f + g$).  In this case,
\[
h^{\prime} = f^{\prime} + g^{\prime},
\]
and since $f^{\prime}$ and $g^{\prime}$ are in \RAF\ by hypothesis, so
is their sum by the constructor rules, which proves $P(h)$.

\inductioncase{Case}: ($h= f \cdot g$).

The Product Rule of derivatives states that:
\begin{equation}\label{fgderiv}
h^{\prime} =  f^{\prime} \cdot g + f \cdot g^{\prime},
\end{equation}
and since $f, f^{\prime}, g, g^{\prime} \in \RAF$ by
hypothesis, so is the right-hand side of~\eqref{fgderiv} by the
constructor rules, which proves $P(h)$.

\inductioncase{Case}: ($h= 1/f$).

The Chain Rule gives:
\begin{equation}\label{1/fderiv}
h^{\prime} = \frac{-1}{f^2} \cdot f^{\prime},
\end{equation}
and since $f$ and $f^{\prime}$ are in \RAF\ by hypothesis, so is the
right-hand side of~\eqref{1/fderiv} by the constructor rules, which
proves $P(h)$.

We have shown that the induction hypothesis holds in all Constructor cases.
This completes the proof by structural induction.
\end{proof}
\end{solution}

\end{problemparts}

\end{problem}

%%%%%%%%%%%%%%%%%%%%%%%%%%%%%%%%%%%%%%%%%%%%%%%%%%%%%%%%%%%%%%%%%%%%%
% Problem ends here
%%%%%%%%%%%%%%%%%%%%%%%%%%%%%%%%%%%%%%%%%%%%%%%%%%%%%%%%%%%%%%%%%%%%%

\endinput
