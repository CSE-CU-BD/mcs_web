\documentclass[problem]{mcs}

\begin{pcomments}
  \pcomment{from: S09.cp1t}
%  \pcomment{}
%  \pcomment{}
\end{pcomments}

\pkeywords{
  faulty_reasoning
}

%%%%%%%%%%%%%%%%%%%%%%%%%%%%%%%%%%%%%%%%%%%%%%%%%%%%%%%%%%%%%%%%%%%%%
% Problem starts here
%%%%%%%%%%%%%%%%%%%%%%%%%%%%%%%%%%%%%%%%%%%%%%%%%%%%%%%%%%%%%%%%%%%%%

\begin{problem} Albert announces that he plans a surprise 6.042
quiz next week.  His students wonder if the quiz could be next Friday.
The students realize that it obviously cannot, because if it hadn't been
given before Friday, everyone would know that there was only Friday left
on which to give it, so it wouldn't be a surprise any more.

So the students ask whether Albert could give the surprise quiz Thursday?
They observe that if the quiz wasn't given \emph{before} Thursday, it
would have to be given \emph{on} the Thursday, since they already know it
can't be given on Friday.  But having figured that out, it wouldn't be a
surprise if the quiz was on Thursday either.  Similarly, the students
reason that the quiz can't be on Wednesday, Tuesday, or Monday.  Namely,
it's impossible for Albert to give a surprise quiz next week.  All the
students now relax, having concluded that Albert must have been bluffing.

And since no one expects the quiz, that's why, when Albert gives it on
Tuesday next week, it really is a surprise!

What do you think is wrong with the students' reasoning?

\begin{solution}
The basic problem is that ``surprise'' is not a mathematical
concept, nor is there any generally accepted way to give it a mathematical
definition.  The ``proof'' above assumes some plausible axioms about
surprise, without defining it.  The paradox is that these axioms are
inconsistent.  But that's no surprise \texttt{:-)}, since ---mathematically
speaking ---we don't know what we're talking about.

Mathematicians and philosophers have had a lot more to say about what might
be wrong with the students' reasoning, (see Chow, Timothy Y.
\href{http://courses.csail.mit.edu/6.042/spring02/handouts/surprise-paradox.pdf}
{\emph{The surprise examination or unexpected hanging paradox}}, American
Mathematical Monthly (January 1998), pp.41--51.)
\end{solution}

\end{problem}

%%%%%%%%%%%%%%%%%%%%%%%%%%%%%%%%%%%%%%%%%%%%%%%%%%%%%%%%%%%%%%%%%%%%%
% Problem ends here
%%%%%%%%%%%%%%%%%%%%%%%%%%%%%%%%%%%%%%%%%%%%%%%%%%%%%%%%%%%%%%%%%%%%%

\endinput
