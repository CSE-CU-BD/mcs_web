\documentclass[problem]{mcs}

\begin{pcomments}
  \pcomment{CP_proving_probability_rules}
  \pcomment{from: S09.cp12r}
\end{pcomments}

\pkeywords{
  probability
}

%%%%%%%%%%%%%%%%%%%%%%%%%%%%%%%%%%%%%%%%%%%%%%%%%%%%%%%%%%%%%%%%%%%%%
% Problem starts here
%%%%%%%%%%%%%%%%%%%%%%%%%%%%%%%%%%%%%%%%%%%%%%%%%%%%%%%%%%%%%%%%%%%%%

% F09, S09

\begin{problem}
Here are some handy rules for reasoning about probabilities that all
follow directly from the Disjoint Sum Rule in the Appendix.  Prove them.

\begin{equation}
\pr{A-B}=\pr{A}-\pr{A \intersect B}  \tag{Difference Rule}
\end{equation}

\begin{solution}
Any set $A$ is the disjoint union of $A-B$ and $A \intersect B$, so
\[
\pr{A}= \pr{A-B}+\pr{A \intersect B}
\]
by the Disjoint Sum Rule.
\end{solution}

\begin{equation}
\pr{\bar{A}} = 1 - \pr{A} \tag{Complement Rule}
\end{equation}

\begin{solution}
$\bar{A} \eqdef \sspace - A$, so by the Difference Rule
\[
\pr{\bar{A}} = \pr{\sspace} - \pr{A} =  1 - \pr{A}.
\]
\end{solution}

\begin{equation}
\pr{A \union B} = \pr{A}+\pr{B} - \pr{A \intersect B} \tag{Inclusion-Exclusion}
\end{equation}

\begin{solution}
$A \union B$ is the disjoint union of $A$ and $B-A$ so
\begin{align*}
\pr{A \union B} & = \pr{A} + \pr{B-A}  & \text{(Disjoint Sum Rule)}\\
     & = \pr{A} + (\pr{B} - \pr{A \intersect B})    & \text{(Difference Rule)}
\end{align*}
\end{solution}

\begin{equation}
\pr{A \union B} \leq \pr{A}+\pr{B}. \tag{2-event Union Bound}
\end{equation}

\begin{solution}
This follows immediately from Inclusion-Exclusion and the fact that
$\pr{A \intersect B} \geq 0$.
\end{solution}

\begin{equation}
\text{If } A \subseteq B, \text{ then } \pr{A} \leq \pr{B}. \tag{Monotonicity}
\end{equation}

\begin{solution}
\begin{align*}
\pr{A} & = \pr{B} - (\pr{B} - \pr{A})\\
       & = \pr{B} - (\pr{B} - \pr{A \intersect B})
                & (\text{since } A = A \intersect B)\\
       & = \pr{B} - \pr{B - A}   & \text{(difference rule)}\\
       & \leq \pr{B} & (\text{since } \pr{B-A} \geq 0).
\end{align*}
\end{solution}

\end{problem}

%%%%%%%%%%%%%%%%%%%%%%%%%%%%%%%%%%%%%%%%%%%%%%%%%%%%%%%%%%%%%%%%%%%%%
% Problem ends here
%%%%%%%%%%%%%%%%%%%%%%%%%%%%%%%%%%%%%%%%%%%%%%%%%%%%%%%%%%%%%%%%%%%%%

\endinput
