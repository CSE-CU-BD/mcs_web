\documentclass[problem]{mcs}

\begin{pcomments}
  \pcomment{CP_financial_crisis}
  \pcomment{from F10.rec23}
\end{pcomments}

\pkeywords{
  Chernoff
  loan
  tranche
  mutual_independence
  Markov_bound
}

%%%%%%%%%%%%%%%%%%%%%%%%%%%%%%%%%%%%%%%%%%%%%%%%%%%%%%%%%%%%%%%%%%%%%
% Problem starts here
%%%%%%%%%%%%%%%%%%%%%%%%%%%%%%%%%%%%%%%%%%%%%%%%%%%%%%%%%%%%%%%%%%%%%

\begin{problem}
Reasoning based on the Chernoff bound goes a long way in explaining
the recent subprime mortgage collapse.  A bit of standard vocabulary
about the mortgage market is needed:

\begin{itemize}

\item A \textbf{loan} is money lent to a borrower.  If the borrower
  does not pay on the loan, the loan is said to be in
  \textbf{default}, and collateral is seized.  In the case of mortgage
  loans, the borrower's home is used as collateral.
	
\item A \textbf{bond} is a collection of loans, packaged into one
  entity.  A bond can be divided into \textbf{tranches}, in some
  ordering, which tell us how to assign losses from defaults.  Suppose
  a bond contains 1000 loans, and is divided into 10 tranches of 100
  bonds each.  Then, all the defaults must fill up the lowest tranche
  before the affect others.  For example, suppose 150 defaults
  happened.  Then, the first 100 defaults would occur in tranche 1,
  and the next 50 defaults would happen in tranche 2.

\item The lowest tranche of a bond is called the \textbf{mezzanine
  tranche}.
	
\item We can make a ``super bond'' of tranches called a
  \textbf{\idx{collateralized debt obligation} (\idx{CDO})} by
  collecting mezzanine tranches from different bonds.  This super bond
  can then be itself separated into tranches, which are again ordered
  to indicate how to assign losses.
	
\end{itemize}

\bparts

\ppart Suppose that 1000 loans make up a bond, and the fail rate is
$5\%$ in a year.  Assuming mutual independence, give an upper bound
for the probability that there are one or more failures in the
second-worst tranche.  What is the probability that there are failures
in the best Tranche?
	
\begin{solution}
If we assume mutual independence, we can use Chernoff's Theorem to
give an upper bound to this.  Because the fail rate is $5\%$, $\expect{X} =
50$.  From Chernoff, we have that
\[
\prob{X \geq 2 \cdot 50} \leq e^{-\beta(2) \cdot 50}
\]
which evaluates to $1.759 \times 10^{-6}$.
		
In order for failure to reach the best Tranche, there must have been
at least 900 failures.  This corresponds to $c=18$ in the Chernoff
bound, so the bound is
\[
\prob{X \geq 18 \cdot 50} \leq e^{-\beta(18) \cdot 50}
\]
		
This evaluates to $e^{-2551.33}$, which evaluates to less than
$10^{-1000}$.
\end{solution}	
	
	
\ppart\label{pMarkov} Now, do not assume that the loans are independent.  Give an
upper bound for the probability that there are one or more failures in
the second tranche.  What is an upper bound for the probability that
the entire bond defaults?  Show that it is a tight bound.  \hint Use
Markov's theorem.

\begin{solution}
From the Markov bound, we have that
\[
\prob{X \geq 100} \leq \frac{50}{100} = 1/2
\]
		
Applying Markov's theorem again in the case of failures occuring in
every bond, we have
\[
\prob{X \geq 1000} \leq \frac{50}{1000} = 1/20
\]
		
We can see that these bounds are tight by constructing situations
where they occur.  In the first case, consider if the bond consisted
of 100 bonds that always defaulted half the time, and the other bonds
never defaulted.  In the second case, assume that all the bonds are
completely correlated (they all default or all do not default), and
default $5\%$ of the time.
\end{solution}
	
\ppart Given this setup (and assuming mutual independence between the
loans), what is the expected failure rate in the mezzanine tranche?

\begin{solution}
The expected number of failures per bond is $50$.  Because all losses
are sent to the mezzanine, the expected failure rate is $50\%$ (mighty
high!).
	
\end{solution}
	
\ppart We take the mezzanine tranches from 100 bonds and create a CDO.
What is the expected number of underlying failures to hit the CDO?

\begin{solution}
From the previous part, each mezzanine tranche has a $50\%$ rate of
failure.  The CDO contains $10000$ loans, at a $50\%$ rate of failure,
and so we expect $5000$ failures in the CDO.
\end{solution}
	
\ppart We divide this CDO into $10$ tranches of $1000$ bonds each.
Assuming mutual independence, give an upper bound on the probability
of one or more failures in the best tranche.  The third tranche?

\begin{solution}
We use a $50\%$ failure rate to model the loans in this CDO.  If we
assume mutual independence, we use the Chernoff bound to bound this.
One or more failures in the best tranche corresponds to over $9000$
failures in the CDO.  Using the Chernoff bound, we find that
\[
\prob{X \geq 9/5*5000} \leq e^{-\beta(9/5) \cdot 5000}
\] 
which is approximately zero.
		
For the third tranche, we make a similar calculation, finding the
probability that there are over $7000$ failures in the CDO.

\[
\prob{X \geq 7/5*5000} \leq e^{-\beta(7/5) \cdot 5000}
\] 
This is about $10^{-155}$ ---that is, still vanishingly small.
\end{solution}
	
\ppart  Repeat the previous question without the assumption of mutual independence.

\begin{solution}
Without a mutual independence assumption, the best we can do is bounds
from Markov's theorem, as we did in part~\eqref{pMarkov}.  Thus, the
upper bound on failures in the top tranche, or over $9000$ failures,
is:
\[
\prob{X \geq 9000} \leq \frac{5000}{9000} = 5/9
\]

For failures in the third trance, Markov's theorem gives us:
\[
\prob{X \geq 7000} \leq \frac{5000}{7000} = 5/7
\]
To put it generously, these probabilities are not very low.
	
\end{solution}

\eparts

\end{problem}

%%%%%%%%%%%%%%%%%%%%%%%%%%%%%%%%%%%%%%%%%%%%%%%%%%%%%%%%%%%%%%%%%%%%%
% Problem ends here
%%%%%%%%%%%%%%%%%%%%%%%%%%%%%%%%%%%%%%%%%%%%%%%%%%%%%%%%%%%%%%%%%%%%%

\endinput
