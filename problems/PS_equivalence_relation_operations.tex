\documentclass[problem]{mcs}

\begin{pcomments}
  \pcomment{PS_equivalence_relation_operations}
  \pcomment{renamed from PS_equivalence_relations}
  \pcomment{from: F03.ps2}
\end{pcomments}

\pkeywords{
  equivalence_relations
  intersection
  union
}

%%%%%%%%%%%%%%%%%%%%%%%%%%%%%%%%%%%%%%%%%%%%%%%%%%%%%%%%%%%%%%%%%%%%%
% Problem starts here
%%%%%%%%%%%%%%%%%%%%%%%%%%%%%%%%%%%%%%%%%%%%%%%%%%%%%%%%%%%%%%%%%%%%%

\begin{problem}

Let $R_1$ and $R_2$ be two equivalence relations on a set $A$.  Prove
or give a counterexample to the claims that the following are also
equivalence relations:

\begin{problemparts}

\problempart $R_1 \intersect R_2$.

\begin{solution}
Let $R \eqdef R_1 \intersect R_2$.  We give two proofs that $R$ is an
equivalence relation using different characterizations of equivalence
relations.

\begin{proof}
We first prove that $R$ is an equivalence relations by showing that $R$ is
reflexive, symmetric, and transitive.

\emph{Reflexive:} $R_i$ is reflexive because it is an equivalence
relation, for $i=1,2$ (i.e. $R_1$ and $R_2$ are both reflexive because 
they are each equivalence relations).  So $(a,a) \in R_i$ for $i=1,2$ and all $a \in A$.
So, $(a,a) \in (R_1 \intersect R_2) = R$ for all $a \in A$, that is, $R$
is reflexive.

\emph{Transitive:} Suppose $(a,b), (b,c) \in R$.  Since $R = R_1
\intersect R_2$, we have $(a,b), (b,c) \in R_i$ for $i=1,2$.  But $R_i$ is
an equivalence relation, and so is transitive.  Therefore, $(a,c) \in
R_i$, and so $(a,c) \in R_1\intersect R_2 = R$.  This shows that $R$ is
transitive.

\emph{Symmetric:} The proof that $R$ is symmetric follows the same format.

\end{proof}

\begin{proof}
This second proof that $R$ is an equivalence relation uses the fact
that $R$ is an equivalence relation on $A$ iff $R = \equiv_f$ for some
total function $f$ with domain $A$\inbook{ according to
  Definition~\bref{equiv_f}}.

Since $R_i$ is an equivalence relation for $i=1,2$, there is a total
function, $f_i$, with domain $A$ such that
\[
aR_i b \quad\text{ iff }\quad f_i(a) = f_i(b).
\]
Define the function $f$ with domain $A$ by
\[
f(a) \eqdef (f_1(a),f_2(a)).
\]
Clearly $f$ is total, since $f_i$ is total for $i=1,2$.  Now we have,
\begin{align*}
aRb &  \QIFF\  a(R_1 \intersect R_2)b & \text{def.\ of $R$}\\
    &  \QIFF\  aR_ib\text{ for }i=1,2 & \text{def.\ of $\intersect$}\\
    &  \QIFF\  f_i(a)=f_i(b) \text{ for }i=1,2 & \text{def.\ of $f_i$}\\
    &  \QIFF\  (f_1(a),f_2(a)) = (f_1(b),f_2(b)) & \text{def.\ of $=$ pairs}\\
    &  \QIFF\  f(a) = f(b) & \text{def.\ of $f$}.
\end{align*}
That is
\[
aRb\ \QIFF\ f(a) = f(b),
\]
which proves that $R$ is the equivalence relation $\equiv_f$.
\end{proof}
\end{solution}

\problempart $R_1 \union R_2$.

\begin{solution}
We give a counterexample showing that $R_1 \union R_2$ may not be
an equivalence relation.  Let $R_1$ and $R_2$ be the relations on
$\set{1, 2, 3}$ where
\begin{align*}
R_1 \eqdef & \set{(1,1) (2,2) (3,3) (1,2) (2,1)},\\
R_2 \eqdef & \set{(1,1) (2,2) (3,3) (2,3) (3,2)}.
\end{align*}
It's easy to check that $R_1$ and $R_2$ are both equivalence relations.
But $R_1\union R_2$ is not transitive, because $(1,2),(2,3) \in R_1\union R_2$
and $(1,3) \notin R_1\union R_2$.  Therefore $R_1\union R_2$ is not an
equivalence relation.
\end{solution}

\end{problemparts}

\end{problem}


%%%%%%%%%%%%%%%%%%%%%%%%%%%%%%%%%%%%%%%%%%%%%%%%%%%%%%%%%%%%%%%%%%%%%
% Problem ends here
%%%%%%%%%%%%%%%%%%%%%%%%%%%%%%%%%%%%%%%%%%%%%%%%%%%%%%%%%%%%%%%%%%%%%

\endinput
