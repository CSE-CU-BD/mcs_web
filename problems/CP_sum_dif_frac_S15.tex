\documentclass[problem]{mcs}

\begin{pcomments}
  \pcomment{CP_sum_dif_frac_S15}
  \pcomment{excerpted from CP_sum_dif_frac by AC, 4/7/15}
  \pcomment{S15.ps9}
\end{pcomments}

\pkeywords{
  sums
  summation
  telescoping
  polynomial
  cubes
}

%%%%%%%%%%%%%%%%%%%%%%%%%%%%%%%%%%%%%%%%%%%%%%%%%%%%%%%%%%%%%%%%%%%%%
% Problem starts here
%%%%%%%%%%%%%%%%%%%%%%%%%%%%%%%%%%%%%%%%%%%%%%%%%%%%%%%%%%%%%%%%%%%%%

\begin{problem}
\iffalse
Find a closed form for each of the following sums:
\bparts

\ppart 
\[
\sum_{i=1}^{n} \paren{\frac{1}{i+2012} - \frac{1}{i+2013}}\, .
\]

\begin{solution}
\begin{align*}
\lefteqn{\sum_{i=1}^{n} \paren{\frac{1}{i+2012} - \frac{1}{i+2013}}}\\
 & = \paren{\frac{1}{2013} - \frac{1}{2014}} + \paren{\frac{1}{2014} - \frac{1}{2015}} +\\
 & \qquad  \cdots + \paren{\frac{1}{n+2012}- \frac{1}{n+2013}}\\
 & = \frac{1}{2013} + \paren{- \frac{1}{2014} + \frac{1}{2014}} + \paren{-\frac{1}{2015} + \frac{1}{2015}} +\\
 & \qquad \cdots + \paren{-\frac{1}{n+2012} + \frac{1}{n+2012}} - \frac{1}{n+2013}\\
 & = \frac{1}{2013} - \frac{1}{n+2013}.
\end{align*}
\end{solution}

\ppart
\fi

Assuming the following sum equals a polynomial in $n$, find the
polynomial.  Optionally, you might want to use induction to prove that the sum
equals the polynomial you find, but no such proof is required for full credit.
\[
\sum_{i=1}^{n} i^3
\]

\begin{solution}
As in Section~\bref{sec:sum_powers}, a sensible guess is that the sum
of the first $n$ cubes will result in a fourth-degree polynomial in
$n$:
\[
\sum_{i=1}^{n} i^3 = an^4 + bn^3 + cn^2 + dn + e.
\]

We need to determine the coefficients.

$n=0$ implies $0 = e$.

$n=1$ implies $1 = a + b + c + d + e$.

$n=2$ implies $9 = 16a + 8b + 4c + 2d + e$.

$n=3$ implies $36 = 81a + 27b + 9c + 3d+ e$.

$n=4$ implies $100 = 256a + 64b + 16c + 4d + e$.

Solving this equation gives:
\[
a = \frac{1}{4},\ b = \frac{1}{2},\ c = \frac{1}{4},\ d = 0,\ e = 0,
\]
which would imply that
\begin{equation}\label{cubesumpoly}
\sum_{i=1}^{n} i^3 = \frac{n^4 + 2n^3 + n^2}{4}.
\end{equation}

%\sum_{i=1}^{n} i^3 = \frac{(n^2 + n)^2}{4}.

We now verify~\eqref{cubesumpoly} by induction on $n$ with induction
hypothesis $P(n)$ given by~\eqref{cubesumpoly}.

\inductioncase{Base case}: ($n=0$).  The left hand side
of~\eqref{cubesumpoly} is an empty sum, which equals 0 by convention.
The right hand side is also 0.

\inductioncase{Inductive step}:
\begin{align*}
\sum_{i=1}^{n+1} i^3
   & = \paren{\sum_{i=1}^{n} i^3} + (n+1)^3\\ 
%   & = \frac{(n^2 + n)^2}{4}  + (n+1)^3
   & = \frac{n^4 + 2n^3 + n^2}{4}  + (n+1)^3\\
       & \text{\hspace{0.7in}(induction hypothesis~\eqref{cubesumpoly})}\\
%   & = \frac{(n^2 + n)^2  + 4(n+1)^3}{4}\\
   & = \frac{n^4 + 2n^3 + n^2 + 4(n^3 + 3n^2 + 3n +1)}{4}\\
%   & = \frac{n^4 + 6n^3 + 13n^2 + 12n +4}{4}\\
   & = \frac{\paren{n^4 + 4n^3 +6n^2 + 4n + 1} + \paren{2n^3 + 6n^2 +6n +2} + \paren{n^2 + 2n + 1}}{4}\\
   & = \frac{(n+1)^4 + 2(n+1)^3 + (n+1)^2}{4}.
%   & = \frac{(n^2 + 3n + 2)^2}{4}\\
%   & = \frac{\paren{(n+1)^2 + (n+1)}^2}{4}.
\end{align*}
This proves $P(n+1)$, completing the induction step.
\end{solution}

%\eparts

\end{problem}

%%%%%%%%%%%%%%%%%%%%%%%%%%%%%%%%%%%%%%%%%%%%%%%%%%%%%%%%%%%%%%%%%%%%%
% Problem ends here
%%%%%%%%%%%%%%%%%%%%%%%%%%%%%%%%%%%%%%%%%%%%%%%%%%%%%%%%%%%%%%%%%%%%%

\endinput
