\documentclass[problem]{mcs}

\begin{pcomments}
  \pcomment{FP_santa_state_machine}
  \pcomment{from: F09.cp5f, spruced up xmas style by Tom Brown 12/09}
  \pcomment{from: S09.cp5r, F03.ps4}
  \pcomment{edited S09 by ARM}
\end{pcomments}

\pkeywords{
  state_machines
  unreachable_states
  increasing_decreasing_variables
}

%%%%%%%%%%%%%%%%%%%%%%%%%%%%%%%%%%%%%%%%%%%%%%%%%%%%%%%%%%%%%%%%%%%%%
% Problem starts here
%%%%%%%%%%%%%%%%%%%%%%%%%%%%%%%%%%%%%%%%%%%%%%%%%%%%%%%%%%%%%%%%%%%%%

\begin{problem}
  Santa's elves have decided to primp his ride.  They add spinners,
  hydraulics and a huge array of lights to the side of Santa's sleigh.
  The lights can switch between Off, Green and Red, and are controlled by
  two big buttons:
\begin{enumerate}
\item[(i)] Switch the color of any ten lights, chosen by winking at them.

\item[(ii)] Let $n$ be the number of green lights showing.  Turn on $n+1$
  additional red lights.
\end{enumerate}

For example, starting with 98 green and 4 red lights on, you might begin
by flipping nine green lights and one red, yielding 90 green and 12 red
lights, then add 91 red lights, yielding 90 green and 103 red lights.

\bparts

\ppart[2]
Model this situation as a state machine, carefully defining the set of
states, the start state and the possible state transitions.

\examspace[1.5in]

\begin{solution}
This can be modeled by a state machine.  The state of the
  machine is the number of green and red lights.  The start state is $(98,4)$,
  and the transitions are:
\[
(g,r) \to
\begin{cases}
(g-a+(10-a),r+a-(10-a)) &\mbox{for } 10 \leq g+r \&
                         0\leq a\leq \min{10,g}.\\
(g,r+g+1).
\end{cases}
\]

%\vskip0.5in
%{\bf Comment:} Most students forgot to specify the range of $a$ precisely
%for the first transition.

\end{solution}

\iffalse

\ppart Explain how to reach a state with exactly one red light showing.

\examspace[1.5in]

\begin{solution}
One way is to:
\begin{enumerate}

\item Do operation 2 three times, yielding $(98, 4+3\cdot99)= (98,301)$.

\item Repeat 30 times: Do operation 1 to flip 10 red lights to green. This
will result in the state $(398,1)$, which is the desired state.
\end{enumerate}

\end{solution}
\fi

\ppart[2]\label{derivedvars_part}
Define the following derived variables:
\[\begin{array}{|ll|l|}\hline
 G & \eqdef  \mbox{the number of green lights},&  \rem{G}{2},\\
 R & \eqdef  \mbox{the number of red lights},  &  \rem{R}{2},\\
 R+G, &                                        &  \rem{R+G}{2}.\\\hline
\end{array}\]

Which of these variables is
\begin{enumerate}\renewcommand{\itemsep}{0pt}

\item strictly increasing \hfill \examrule[0.75in]

\begin{solution}
NONE
\end{solution}

\item weakly increasing  \hfill \examrule[0.75in]

\begin{solution}
$R+G$, $\rem{G}{2}$

{\bf Comment:} Notice that a constant variable like $G_2$ is
also weakly decreasing and weakly increasing, by definition.
\end{solution}

\item strictly decreasing  \hfill \examrule[0.75in]

\begin{solution}
NONE
\end{solution}

\item weakly decreasing  \hfill \examrule[0.75in]

\begin{solution}
$G_2$
\end{solution}

\item constant  \hfill \examrule[0.75in]

\begin{solution}
$\rem{G}{2}$
\end{solution}
\end{enumerate}

\examspace

\ppart[3] Prove that it is not possible to reach a state in which there is
exactly one green light showing.  (If you appeal to one of your answers to
part~\eqref{derivedvars_part}, you must prove it.) 

\examspace[3in]

\begin{solution}

We claimed above that $\rem{G}{2}$ is an invariant value, that is, it does not
change under state transitions.  To prove this, let $(g,r)$ be a state
with $g$ even.  For the next state, we have two cases to consider:
\begin{enumerate}
\item The first operation is executed:
$(g,r)\rightarrow(g-2a+10,r+2a-10)$.  Since $-2a+10$ is even, $\rem{(g,r)}{2}=
\rem{(g-2a+10, r+2a-10)}{2}$.

\item The second operation is executed: $(g,r)\rightarrow(g,r+g+1)$. The
  number of green lights does not change in this case, so $\rem{(g,r)}{2}$
  does not change.
\end{enumerate}
Since the initial number of green lights, 98, is even, that is,
$\rem{(98,4)}{2})=0$, the Invariant Method now implies that the number of
green lights in a reachable state is always even.  But since 1 is odd, it
is not possible to reach a state in which there is exactly 1 green light
showing.

\end{solution}

\eparts
\end{problem}

%%%%%%%%%%%%%%%%%%%%%%%%%%%%%%%%%%%%%%%%%%%%%%%%%%%%%%%%%%%%%%%%%%%%%
% Problem ends here
%%%%%%%%%%%%%%%%%%%%%%%%%%%%%%%%%%%%%%%%%%%%%%%%%%%%%%%%%%%%%%%%%%%%%

\endinput
