\documentclass[problem]{mcs}

\begin{pcomments}
  \pcomment{CP_short_pigeonhole}
  \pcomment{renamed from CP_pigeon_hole}
  \pcomment{from: S08 cp9r}
\end{pcomments}

\pkeywords{
  pigeonhole
}

%%%%%%%%%%%%%%%%%%%%%%%%%%%%%%%%%%%%%%%%%%%%%%%%%%%%%%%%%%%%%%%%%%%%%
% Problem starts here
%%%%%%%%%%%%%%%%%%%%%%%%%%%%%%%%%%%%%%%%%%%%%%%%%%%%%%%%%%%%%%%%%%%%%

\begin{problem}
Solve the following problems using the pigeonhole principle.  For each
problem, try to identify the \emph{pigeons}, the \emph{pigeonholes}, and
a \emph{rule} assigning each pigeon to a pigeonhole.

\begin{staffnotes}
Students tend to omit the rule assigning pigeons to holes, which can
cause a problem in part~\eqref{unitsq}, so make sure they state their
rule explicitly.
\end{staffnotes}

\bparts

\iffalse

\ppart In a room of 500 people, there exist two who share a birthday.

\begin{solution}The pigeons are the 500 people.  The
pigeonholes are 366 possible birthdays.  Map each person to his or her
own birthday.  Since there 500 people and 366 birthdays, some two
people must have the same birthday by the Pigeonhole Principle.
\end{solution}

\fi

\ppart In a certain Institute of Technology, every ID number starts
with a 9.  Suppose that each of the 75 students in a class sums the
nine digits of their ID number.  Explain why two people must arrive at
the same sum.

\begin{solution}
The students are the pigeons, the possible sums are the pigeonholes,
and we map each student to the sum of the digits in his or her MIT ID
number.  Every sum is in the range from $9 + 8 \cdot 0 = 9$ to $9 + 8
\cdot 9 = 81$, which means that there are 73 pigeonholes.  Since there
are more pigeons than pigeonholes, there must be two pigeons in the
same pigeonhole; in other words, there must be two students with the
same sum.
\end{solution}


\ppart In every set of 100 integers, there exist two whose difference
is a multiple of 37.

\begin{solution}
The pigeons are the 100 integers.  The pigeonholes are the numbers in
the division by 37 remainder interval $[0,37)$.  Map integer $k$ to
  $\rem{k}{37}$.  Since there are 100 pigeons and only 37 pigeonholes,
  two pigeons must go in the same pigeonhole.  This means
  $\rem{k_1}{37} = \rem{k_2}{37}$, which implies that $k_1 - k_2$ is a
  multiple of 37.
\end{solution}


\ppart\label{unitsq} For any five points inside a unit square (not on
the boundary), there are two points at distance \emph{less than}
$1/\sqrt{2}$.

\begin{solution}
  The pigeons are the points.  The pigeonholes are the four
  subsquares of the unit square, each of side length 1/2.

  Pigeons are assigned to the subsquare that contains them, except that if
  the pigeon is on a boundary, it gets assigned to the leftmost and then
  lowest possible subsquare that includes it (so the point at $(1/2,1/2)$
  is assigned to the lower left subsquare).

  There are five pigeons and four pigeonholes, so more than one point must
  be in the same subsquare.  The diagonal of a subsquare is $1/\sqrt{2}$,
  so two pigeons in the same hole are at most this distance.  But pigeons
  must be inside the unit square, so two pigeons cannot be at the
  opposite ends of the same subsquare diagonal.  So at least one of them
  must be inside the subsquare, so their distance is less than the length
  of the diagonal.
\end{solution}

\iffalse

\ppart For any five points in an equilateral triangle of side length
2, there are two points at distance less than $1$.

\begin{solution}The pigeons are the points.  The pigeonholes are the four
  sub-equilateral triangles of side length 1.  There are five pigeons
  and four pigeonholes, so more than one point must be in the same
  sub-equilateral triangle.  Points in the same sub-equilateral
  triangle are at distance at most $1$.
\end{solution}

\fi

\ppart Show that if $n+1$ numbers are selected from $\set{1,2,3,
  \dots,2n}$, two must be consecutive, that is, equal to $k$ and $k+1$ for
some $k$.

\begin{solution}
  The pigeonholes will be the $n$ sets $\set{1,2}, \set{3,4},
  \set{5,6},\dots, \set{2n-1,2n}$.  The pigeons will be the $n+1$ selected
  numbers.  A pigeon is assigned to the unique pigeon hole of which it is a
  member.  By the Pigeonhole Principle, two pigeons must assigned to some hole, and
  these are the two consecutive numbers required.  Notice that we've
  actually shown a bit more: there will be two consecutive numbers with
  the smaller being odd.
\end{solution}

\eparts

\end{problem}

%%%%%%%%%%%%%%%%%%%%%%%%%%%%%%%%%%%%%%%%%%%%%%%%%%%%%%%%%%%%%%%%%%%%%
% Problem ends here
%%%%%%%%%%%%%%%%%%%%%%%%%%%%%%%%%%%%%%%%%%%%%%%%%%%%%%%%%%%%%%%%%%%%%
