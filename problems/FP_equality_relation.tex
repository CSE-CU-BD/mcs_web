\documentclass[problem]{mcs}

\begin{pcomments}
  \pcomment{FP_equality_relation}
  \pcomment{ARM 3/30/15}
  \pcomment{S15.mid3}
\end{pcomments}

\pkeywords{
  partial_orders
  weak_partial_order  
  strict_partial_order
  transitive
  asymmetric
}

%%%%%%%%%%%%%%%%%%%%%%%%%%%%%%%%%%%%%%%%%%%%%%%%%%%%%%%%%%%%%%%%%%%%%
% Problem starts here
%%%%%%%%%%%%%%%%%%%%%%%%%%%%%%%%%%%%%%%%%%%%%%%%%%%%%%%%%%%%%%%%%%%%%

\begin{problem}
Let $A$ be a nonempty set.
\begin{problemparts}

\ppart\label{idwpoA} Describe a single relation on $A$ that is
\emph{both} an equivalence relation and a weak partial order on $A$.

\examspace[2cm]
%\examrule[0.4in]

\begin{solution}
The equality relation $\ident{A}$ on $A$ is certainly an equivalence.
It is also \emph{vacuously} antisymmetric, since there are no $a, b
\in A$ such that
\[
a \neq b\ \QAND\ a \mrel{\ident{A}} b.
\]
\end{solution}

\ppart Prove that the relation of part~\eqref{idwpoA} is the only
relation on $A$ with these properties.

\begin{solution}

\begin{proof}
Suppose $R$ is a relation on $A$ that is an equivalence and a wpo.
Since $R$ is an equivalence, we know that $R$ is reflexive.  So to
show that $R = \ident{A}$, we need only show that
\[
a \mrel{R} b \QIMPLIES a = b.
\]

So suppose $a \mrel{R} b$.  Since $R$ is symmetric, we can conclude
that $b \mrel{R} a$.  Now since $R$ is antisymmetric, we conclude that
$a = b$.
\end{proof}

\end{solution}

\end{problemparts}
\end{problem}

%%%%%%%%%%%%%%%%%%%%%%%%%%%%%%%%%%%%%%%%%%%%%%%%%%%%%%%%%%%%%%%%%%%%%
% Problem ends here
%%%%%%%%%%%%%%%%%%%%%%%%%%%%%%%%%%%%%%%%%%%%%%%%%%%%%%%%%%%%%%%%%%%%%

\endinput
