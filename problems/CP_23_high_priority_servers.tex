\documentclass[problem]{mcs}

\begin{pcomments}
  \pcomment{from: S09.cp6t}
\end{pcomments}

\pkeywords{
  graphs
  connectivity
}

%%%%%%%%%%%%%%%%%%%%%%%%%%%%%%%%%%%%%%%%%%%%%%%%%%%%%%%%%%%%%%%%%%%%%
% Problem starts here
%%%%%%%%%%%%%%%%%%%%%%%%%%%%%%%%%%%%%%%%%%%%%%%%%%%%%%%%%%%%%%%%%%%%%

\begin{problem}
There is a network of web-servers connected by optical cables that go
between servers.  Even when there is no direct cable between two servers,
it is possible to transmit data from one to the other indirectly along
cables through other servers.

The network company regularly has to disconnect cables for inspection and
repair, but there are 23 high priority servers which must always be able to
transmit data to each other.

In the best of networks, there is a smallest number of cables that
the company would have to keep connected in order for the 23 servers to
communicate.  What is this number?  Explain why a much bigger number might
be necessary in some networks.

\begin{solution}
22 cables is the absolute minimum.

The network can be modeled as a graph whose vertices are the servers, with
an edge between two servers when there is a direct cable connection between
them.  The fact that data transmission is possible between any two servers
means the graph is connected.

The minimum edge subgraph needed to maintain connections among 23 vertices
will be a tree which includes these 23 vertices, so it must have at least
22 edges.  But many more edges may be needed, for example, if the only
connection between two of the servers was through long sequence of cables,
then all the cables in the sequence would have to remain connected.
\end{solution}

\end{problem}

%%%%%%%%%%%%%%%%%%%%%%%%%%%%%%%%%%%%%%%%%%%%%%%%%%%%%%%%%%%%%%%%%%%%%
% Problem ends here
%%%%%%%%%%%%%%%%%%%%%%%%%%%%%%%%%%%%%%%%%%%%%%%%%%%%%%%%%%%%%%%%%%%%%
