\documentclass[problem]{mcs}

\begin{pcomments}
  \pcomment{FP_outdegree_induction}
  \pcomment{ARM 3/14/12}
\end{pcomments}

\pkeywords{
  handshaking
  degree
  graphs
}

%%%%%%%%%%%%%%%%%%%%%%%%%%%%%%%%%%%%%%%%%%%%%%%%%%%%%%%%%%%%%%%%%%%%%
% Problem starts here
%%%%%%%%%%%%%%%%%%%%%%%%%%%%%%%%%%%%%%%%%%%%%%%%%%%%%%%%%%%%%%%%%%%%%

\begin{problem}
The proof of the Handshaking Lemma~\bref{digraph-handshake} invoked
the ``obvious'' fact that in any finite digraph, the sum of the
in-degrees of the vertices equals the number of arrows in the graph.
That is,
\begin{claim*}
For any finite digraph $G$
\begin{equation}\label{indegree-count-arrows}
\sum_{v \in \vertices{G}} \indegr{v} = \card{\graph{G}},
\end{equation}
\end{claim*}
But this Claim might not be obvious to everyone.  So prove it by
induction on the number, $\card{\graph{G}}$, of arrows.

\begin{solution}
\TBA{needed}
\end{solution}

\end{problem}

%%%%%%%%%%%%%%%%%%%%%%%%%%%%%%%%%%%%%%%%%%%%%%%%%%%%%%%%%%%%%%%%%%%%%
% Problem ends here
%%%%%%%%%%%%%%%%%%%%%%%%%%%%%%%%%%%%%%%%%%%%%%%%%%%%%%%%%%%%%%%%%%%%%

\endinput
