\documentclass[problem]{mcs}

\begin{pcomments}
  \pcomment{TP_stable_wedding}
  \pcomment{excerpted from FP_stable_matching_unlucky}
\end{pcomments}

\pkeywords{
  stable_matching
  state_machines
  invariant
}

\providecommand{\boys}{\text{Boys}}
\providecommand{\girls}{\text{Girls}}

%%%%%%%%%%%%%%%%%%%%%%%%%%%%%%%%%%%%%%%%%%%%%%%%%%%%%%%%%%%%%%%%%%%%%
% Problem starts here
%%%%%%%%%%%%%%%%%%%%%%%%%%%%%%%%%%%%%%%%%%%%%%%%%%%%%%%%%%%%%%%%%%%%%

\begin{problem}
  In the Mating Ritual for stable marriages between an equal number of
  boys and girls, explain why there must be a girl to whom no boy
  proposes (serenades) until the last day.

\begin{solution}
Since there are an equal number of boys and girls, the wedding day is
when every girl is being serenaded by some (necessarily only one) boy.

Since ``being serenaded'' is a preserved invariant, any girl being
serenaded on some day continues being serenaded from then on.  So if
every girl has been serenaded on some day before the wedding day, then
all the girls are being serenaded on some day before the wedding day,
a contradiction.
\end{solution}

\end{problem}


%%%%%%%%%%%%%%%%%%%%%%%%%%%%%%%%%%%%%%%%%%%%%%%%%%%%%%%%%%%%%%%%%%%%%
% Problem ends here
%%%%%%%%%%%%%%%%%%%%%%%%%%%%%%%%%%%%%%%%%%%%%%%%%%%%%%%%%%%%%%%%%%%%%

\endinput

