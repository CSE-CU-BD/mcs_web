\documentclass[problem]{mcs}

\begin{pcomments}
  \pcomment{TP_independence_size}
  \pcomment{ARM 10/23/15}
\end{pcomments}

\pkeywords{
  probability
  conditional_probability
  sample_space
}

%%%%%%%%%%%%%%%%%%%%%%%%%%%%%%%%%%%%%%%%%%%%%%%%%%%%%%%%%%%%%%%%%%%%%
% Problem starts here
%%%%%%%%%%%%%%%%%%%%%%%%%%%%%%%%%%%%%%%%%%%%%%%%%%%%%%%%%%%%%%%%%%%%%

\begin{problem}
What is the smallest size sample space in which there are two
independent events, neither of which has probability zero or
probability one?  Explain.

\begin{solution}
The smallest such sample has four elements.  If
$\sspace=\set{1,2,3,4}$ has uniform probabilities, then $\set{1,2},
\set{1,3}$ are examples of independent events, each with probability
$1/2$.

To rule out sizes less than four:
\begin{itemize}

\item in a space of size one, there are no events with non-zero non-one probability,

\item in a space of size two, only a one-element event and its
  complement have non-zero non-one probability, and these are
  obviously not independent.

\item in a space of size three, the only non-zero, non-one events have
  probability equal to $1/3$ or $2/3$.  Since no product of a pair of
  these numbers equals either of them, no two events with these
  probabilities can be independent.
\end{itemize}

\end{solution}
\end{problem}

%%%%%%%%%%%%%%%%%%%%%%%%%%%%%%%%%%%%%%%%%%%%%%%%%%%%%%%%%%%%%%%%%%%%%
% Problem ends here
%%%%%%%%%%%%%%%%%%%%%%%%%%%%%%%%%%%%%%%%%%%%%%%%%%%%%%%%%%%%%%%%%%%%%

\endinput
