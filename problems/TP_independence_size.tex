\documentclass[problem]{mcs}

\begin{pcomments}
  \pcomment{TP_independence_size}
  \pcomment{ARM 10/23/15}
\end{pcomments}

\pkeywords{
  probability
  conditional_probability
  sample_space
}

%%%%%%%%%%%%%%%%%%%%%%%%%%%%%%%%%%%%%%%%%%%%%%%%%%%%%%%%%%%%%%%%%%%%%
% Problem starts here
%%%%%%%%%%%%%%%%%%%%%%%%%%%%%%%%%%%%%%%%%%%%%%%%%%%%%%%%%%%%%%%%%%%%%

\begin{problem}
What is the smallest size sample space in which there are two
independent events, neither of which has probability zero or
probability one?  Explain.

\begin{solution}
The smallest such sample has four elements.  If
$\sspace=\set{1,2,3,4}$ has uniform probabilities, then $\set{1,2},
\set{1,3}$ are examples of independent events, each with probability
$1/2$.

To rule out sizes less than four:
\begin{itemize}

\item in a space of size one, there are no events with non-zero non-one probability,

\item in a space of size two, only a one-element event and its
  complement have non-zero non-one probability, and these are
  obviously not independent.

\item
Let the three outcomes probabilities be $a, b, c > 0$, where
$a+b+c=0$.  Now 1-outcome events are disjoint, so are not pairwise
independent.  The remaining non-zero, non-one events are the three
2-outcome events.  These are not independent of any of the 1-outcome
events since they contain, or are disjoint from, the 1-outcome events.

The 2-outcome events have probabilities $a+b, a+c, b+c$.  For two of
these to be independent, we would need the probability of the
intersection---a 1-outcome event---to be the product of their
probabilities.  For example, we would need
\begin{equation}\label{abxaca}
(a+b)(a+c) = a
\end{equation}
But from~\eqref{abxaca} we get
\begin{align*}
(1 - c) (a +c) & = a\\
a - ac + c(1-c) & = a\\
c(1-c) & = ac\\
(1-c) & = a)
b &= 0,
\end{align*}
contradicting the fact that $b>0$.

\end{solution}
\end{problem}

%%%%%%%%%%%%%%%%%%%%%%%%%%%%%%%%%%%%%%%%%%%%%%%%%%%%%%%%%%%%%%%%%%%%%
% Problem ends here
%%%%%%%%%%%%%%%%%%%%%%%%%%%%%%%%%%%%%%%%%%%%%%%%%%%%%%%%%%%%%%%%%%%%%

\endinput
