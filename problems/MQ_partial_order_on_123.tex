\documentclass[problem]{mcs}

\begin{pcomments}
  \pcomment{MQ_partial_order_on_123}
  \pcomment{renamed from CP_binary_relations_on_01_mq}
  \pcomment{from: S09.cp4m}
  \pcomment{revised ARM 11/3/13}
\end{pcomments}

\pkeywords{
  partial_orders
  chain
  antichain
  parallel_time
  processors
}

%%%%%%%%%%%%%%%%%%%%%%%%%%%%%%%%%%%%%%%%%%%%%%%%%%%%%%%%%%%%%%%%%%%%%
% Problem starts here
%%%%%%%%%%%%%%%%%%%%%%%%%%%%%%%%%%%%%%%%%%%%%%%%%%%%%%%%%%%%%%%%%%%%%

\begin{problem} 

Suppose the precedence partial order on a set of unit time tasks was
isomorphic to the powerset $\power([1,5])$ of the integers from 1 to 5
under the strict subset relation $\subset$.

\bparts

\ppart What is the length of the minimum parallel time schedule to
complete the tasks?
\begin{solution}
\textbf{6}.

A maximum length chain is $\emptyset, \set{1}, \set{1,2}, \set{1,2,3},
\set{1,2,3,4}, \set{1,2,3,4,5}$.  Overall, there are 120 such chains
determined by the permutations of $[1,5]$.
\end{solution}

\ppart Describe a maximum size antichain is this partial order.

\begin{solution}
  The size 2 subsets of $[1,5]$ are a maximum antichain, and there are 10
  such subsets.

 The size 3 subsets are another maximum size antichain.
\end{solution}

\ppart What is the minimum number of processors required to complete the
tasks in minimum parallel time? 

\begin{solution}
\textbf{10}.

The size of the maximum antichain is always sufficient.

No smaller number is possible in this case since all the size 2 subsets
must be scheduled at time 3 in a minimum time schedule because each of
them is the minimum element of a chain of size 4, and completing such a
chain requires 3 more time units to complete.
\end{solution}

\eparts

\end{problem}

%%%%%%%%%%%%%%%%%%%%%%%%%%%%%%%%%%%%%%%%%%%%%%%%%%%%%%%%%%%%%%%%%%%%%
% Problem ends here
%%%%%%%%%%%%%%%%%%%%%%%%%%%%%%%%%%%%%%%%%%%%%%%%%%%%%%%%%%%%%%%%%%%%%

\endinput
