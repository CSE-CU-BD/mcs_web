\documentclass[problem]{mcs}

\begin{pcomments}
    \pcomment{Converted from prob2-postage.scm by scmtotex and dmj
              on Sun 13 Jun 2010 10:18:32 AM EDT}
\end{pcomments}

\begin{problem}

%% type: short-answer
%% title: Postage by Induction

\bparts

\ppart Choose the best of the following comments about approaches to
proving that every amount of postage of 12~cents or more can be formed
using just 4-cent and 5-cent stamps.

\begin{solution}
Any of the above, but Strong induction or Well ordering are easier in
this case

The three principles, Simple/Strong Induction and Well Ordering, are
equivalent: a proof of a theorem using one of them can be transformed
into a proof using either of the others without need for additional
insight.  But Simple induction requires including an extra quantifier
in the induction hypothesis, which makes it a little more awkward.

In particular, Strong Induction is a good choice for this problem,
using the straightforward induction hypothesis
\begin{equation*}
    P(n) ::= [\text{4-cent and 5-cent stamps can form $n$-cent postage}]
\end{equation*}
with base cases $n$ = 12, 13, 14, and 15.
 
Well ordering works fine too.  A proof would start with the set of all
\emph{counterexamples}, namely,

\begin{equation*}
    \{\, n \ge 12 \mid
        \text{n-cent postage can\emph{not} be formed with 4-cent and
          5-cent stamps}
    \,\}
\end{equation*}

Assuming that this set is not empty, Well ordering implies it has a
minimum element.  Using this, a contradiction can be proved using the same
base cases as the Strong Induction proof and exactly the same reasoning
as in the Inductive Step of the Strong Induction proof.  This implies the
set of counterexamples must be empty, which implies the claim about
postage.

Finally, Simple Induction would work using the same proof as Strong
Induction, but with an induction hypothesis, $Q(n)$, cluttered up with
an extra~$\forall$:
\begin{equation*}
    Q(n) ::= \forall k.\;
    12 \le k \le n \implies [\text{4-cent and 5-cent
stamps can form $k$-cent postage}]
\end{equation*}

Having proved $\forall n.\; Q(n)$ by Simple Induction, the desired
assertion about postage: $\forall n \ge 12. \; P(n)$ would then be an
immediate corollary.
\end{solution}

\eparts

% "The Well-ordering Principle"

\end{problem}

\endinput
