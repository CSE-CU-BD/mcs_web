\documentclass[problem]{mcs}

\begin{pcomments}
    \pcomment{Converted from prob4.scm by scmtotex and dmj
              on Sun 13 Jun 2010 10:18:32 AM EDT}
\end{pcomments}

\begin{problem}

%% type: short-answer
%% title: A Bogus Induction

The Fibonacci numbers
\begin{equation*}
0 1 1 2 3 5 8 13 \dots 
\end{equation*}
are defined as follows.  Let $F(n)$ be the $n$th Fibonacci number.
Then
\begin{itemize}

\item $F(0) ::= 0$,

\item $F(1) ::= 1$,


\item $F(n) ::= F(n-1) + F(n-2)$, for $n \ge 2$.

\end{itemize}

Now consider the following:
\begin{falseclm}
Every Fibonacci number is even.
\end{falseclm}
      
\begin{falseproof}
	 
\begin{enumerate}
         
\item
We use strong induction.
         
\item
The induction hypothesis is that $F(n)$ is even.
         
\item
We will first show that this hypothesis holds for $n = 0$.
         
\item
This is true, since $F(0) = 0$, which is an even number.
         
\item
Now, suppose $n \ge 2$. We will show that $F(n)$ is even, assuming that
$F(k)$ is even for all $k < n$.
         
\item
By assumption, both $F(n-1)$ and $F(n-2)$ are even.
         
\item
Therefore, $F(n)$ is even, since $F(n) = F(n-1) + F(n-2)$ and the sum
of two even numbers is even.
         
\item
Thus, the strong induction principle implies that $F(n)$ is even for
all $n > 0$.
        
\end{enumerate}

\end{falseproof}

Which sentences in the proof contain logical errors?

\begin{solution}
8,
5,
8,

Steps 1 through~7 contain no logical errors.  The fatal flaw is in
step~8.  Using strong induction, we can conclude that a predicate
$P(n)$ holds for all $n \ge 0$ provided that we show all of the
following:

\begin{itemize}
    
\item $P(0)$
    
\item $P(0) \implies P(1)$
    
\item $[P(0) \land P(1)] \implies P(2)$
    
\item $[P(0) \land P(1) \land P(2)] \implies  P(3)$
    
\item \emph{etc.}

\end{itemize}

The first assertion is proved on lines 3 and~4.  The third and
subsequent assertions are proved on lines 5--7.  However, the second
assertion, $P(0) \implies P(1)$, is proved nowhere (and is actually
false).  Therefore, we cannot apply the strong induction principle in
step~8.

If you said that step 5 contains a logical error, you were on the
right track.  The natural place to prove the second assertion would
have been in lines 5--7.  But by saying, ``suppose $n \ge 2$'' instead
of ``suppose $n \ge 1$'', the proof explicitly avoided doing so.


Technically, there is no \emph{logical} error in line~5: it is simply
the beginning of a proof for the case when $n \ge 2$.  On the other
hand, it's reasonable to say that line~5 is the place where the proof
makes a \emph{strategic} error because it skips the $n = 1$ case.  So
while (5 8) is really \emph{not} a correct answer, we decided to be
generous, since this issue hasn't come up earlier.

\end{solution}

%% Give your answer as a sequence of integers separated by some 
%% spaces, for example, 
%% \begin{equation*}
%% 4 2 3
%% \end{equation*}
%%  Don't use commas or periods.

\end{problem}

\endinput
