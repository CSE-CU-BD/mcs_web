\documentclass[problem]{mcs}

\begin{pcomments}
    \pcomment{Converted from gcd-1.scm by scmtotex and dmj
              on Sat 12 Jun 2010 09:14:23 PM EDT}
\end{pcomments}

\begin{problem}

%% type: short-answer
%% title: GCD's I

Consider the two integers:

\begin{align*}
x &= 21212121,\\
y &= 12121212.
\end{align*}

\bparts

\ppart
What is the GCD of $x$ and $y$? (Hint: Looks scary, but it's not.)

\begin{solution}

3030303

We run the algorithm: 
\begin{align*}
\GCD(21212121, 12121212)  
& = \GCD(12121212, 9090909) \\
& = \GCD(9090909, 3030303)  \\
& = \GCD(3030303, 0).
\end{align*}
\end{solution}

\ppart
How many iterations of the Euclidean algorithm are needed to compute
this GCD?
 
(An iteration of the Euclidean algorithm is defined as an application
of the equation

\begin{equation*}
\GCD(a, b) = \GCD(b, \rem{a}{b}).  
\end{equation*}

The algorithm begins with $\GCD(x, y)$ and ends with $\GCD(d, 0)$
for some $d$.)

\begin{solution}
3

In the execution of the algorithm in Part 1 we needed 3 applications
of the equation.
\end{solution}

\eparts

\end{problem}

\endinput
