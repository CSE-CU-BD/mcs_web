\documentclass[problem]{mcs}

\begin{pcomments}
    \pcomment{Converted from bogus-WOP.scm by scmtotex and dmj
              on Sat 12 Jun 2010 09:47:19 PM EDT}
\end{pcomments}

\begin{problem}

%% type: short-answer
%% title: A Bogus Well Ordering Principle proof

The Fibonacci numbers
\begin{equation*}
0, 1, 1, 2, 3, 5, 8, 13, \dots
\end{equation*}
are defined as follows.  Let $F(n)$ be the
$n$th Fibonacci number.  Then

\begin{description}

\item $F(0) ::= 0$,

\item $F(1) ::= 1$,

\item $F(n) ::= F(n-1) + F(n-2),$    for $n \ge  2$. ($*$)

\end{description}

Now consider the following:

\begin{falseclm}
Every Fibonacci number is even.
\end{falseclm}
      
\begin{falseproof}
	 
\begin{enumerate}
         
\item The proof is by the WOP.

\item Let $\Even(n)$ mean that $F(n)$ is even.

\item
Let $C$ be the set of counterexamples to the assertion that
$\Even(n)$ holds for all $n \ge 0$. That is
\begin{equation*}
C ::= \{n \ge  0  \mid  \mathop{\mathrm{NOT}} \Even(n) \}.
\end{equation*}
We prove by contradiction that $C$ is empty.

\item Assume that $C$ is not empty.


\item By WOP, there is a least nonnegative integer, $m \in C$,


\item Then $m > 0$, since $F(0) = 0$ is an even number.

\item
Now, suppose $m \ge 2$ so the definition ($*$) of $F(m)$ applies.
         
\item
In this case, both $F(m-1)$ and $F(m-2)$ are both even, since $m$ is
the minimum counterexample such that $F(m)$ is not even.

         
\item
But by ($*$) in the case that $n = m$, we see that $F(m)$ is the sum
of two even numbers, and so it is also even, that is $\Even(m)$ is
true.
         
\item
This contradicts the condition in the definition of $m$ that NOT
$\Even(m)$ is true.

\item

This contradition implies that $C$ must be empty.  Hence, $F(n)$ is
even for all $n \ge 0$.

\end{enumerate}

\end{falseproof}

Which sentences in the proof contain logical errors?

\begin{solution}
11, 7, 7, 11

Steps 1 through 10 contain no logical errors.  The fatal flaw is in
step that final step~11.  The proof only shows that a minimum $m \in
C$ is not 0, and the assumption that $m \ge 2$ leads to a
contradiction.

However, this leaves unexamined that case that $m = 1$, and in fact,
$1 \in C$.  (The supposition that ``$m \ge 2$ so the definition
($*$) of $F(m)$ applies'' is no excuse for ignoring the case $m = 1$.)

If you said that step~7 contains a logical error, you were on the
right track.  The natural place to handle the case $F(1)$ would have
been right after line~6.  But the the proof explicitly avoided the
case $m = 1$, by saying, ``suppose $m \ge 2$.''

Technically, there is no \emph{logical} error in line~7: it is simply
the beginning of a proof for the case when $m \ge 2$.  On the other
hand, it's reasonable to say that line~7 is the place where the proof
makes an \emph{organizational}, or perhaps \emph{strategic}, error
because it skips the $m = 1$ case.  So while 7 is really \emph{not} a
correct answer, we decided to be generous, since this distinction
between logical and organizational/strategic errors hasn't come up
before.
\end{solution}

%% Give your answer as a sequence of integers separated by some
%%      spaces ($e.g.$,  ``4 2 3 ``).  Don't use commas or
%%    parentheses.

\end{problem}

\endinput
