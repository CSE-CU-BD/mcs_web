\documentclass[problem]{mcs}

\begin{pcomments}
    \pcomment{Converted from prob3.scm by scmtotex and dmj
              on Sat 12 Jun 2010 08:32:06 PM EDT}
\end{pcomments}

\begin{problem}

%% type: short-answer
%% title: Generating function of a recurrence

Let $b$, $c$, $a_{0}$, $a_{1}$, $a_{2}$,\dots be real numbers such
that

\begin{equation*}
a_{n} = b(a_{n-1}) + c
\end{equation*}
for $n \ge 1$.

Let $G(x)$ be the generating function for this sequence.  

\bparts

\ppart
The
coefficient of $x^{n}$ in the series expansion of $G(x)$ is

\begin{solution}
$a_n$
\end{solution}

\ppart
The coefficient of $x^{n}$ for $n \ge 1$ in
    the series expansion of $bxG(x)$ is

\begin{solution}
$b(a_(n-1))$
\end{solution}

\ppart
The coefficient of $x^{n}$ for $n \ge 1$ in the series expansion of
$cx/(1-x)$ is

\begin{solution}
$c$
\end{solution}

\ppart
Therefore, $G(x) - bxG(x) - cx/(1-x) = {}$

\begin{solution}
$a_0$
\end{solution}

\ppart 
Using the method of partial fractions, we can find real numbers $d$
and $e$ such that
\begin{equation*}
G(x) = d/L(x) + e/M(x)
\end{equation*}
where $L(x)$ and $M(x)$ are

\begin{solution}
$(1-x)$ and $(1-bx)$

We have
\begin{equation*}
G(x) - bxG(x) - cx/(1-x) = a_0.
\end{equation*}
Solving for $G(x)$, we have

\begin{equation*}
    G(x) = cx/(1-x)(1-bx) + a_0/(1-bx).
\end{equation*}
which by the method of partial fractions  can be expressed in the form
\begin{equation*}
d/(1-x) + e/(1-bx).
\end{equation*}

\end{solution}

\eparts

\end{problem}

\endinput
