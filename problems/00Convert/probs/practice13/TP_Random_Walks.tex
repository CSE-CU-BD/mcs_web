\documentclass[problem]{mcs}

\begin{pcomments}
    \pcomment{TP_Random_Walks}
    \pcomment{Converted from random-walks.scm
              by scmtotex and dmj
              on Sun 13 Jun 2010 05:25:50 PM EDT}
\end{pcomments}

\begin{problem}

%% type: short-answer
%% title: Random Walks

Consider the following random-walk graphs:
\begin{center}
\includegraphics{random_walks}
\end{center}

\bparts

\ppart
Find $d(x)$ for a stationary distribution for graph 1.

\begin{solution}
1/2
\end{solution}

\ppart
Find $d(y)$ for a stationary distribution for graph 1.

\begin{solution}
1/2
\end{solution}

\ppart

If you start at node $x$ in graph 1 and take a (long) random walk,
does the distribution over nodes ever get close to the stationary
distribution?

\begin{solution}
No

No, you will just alternate between nodes $x$ and $y$.
\end{solution}

\ppart
Find $d(w)$ for a stationary distribution for graph 2.

\begin{solution}
9/19

Found by solving $d(w)=.9d(z)$, $d(z)=d(w)+.1d(z)$, and $d(w)+d(z)=1$
simultaneously.
\end{solution}

\ppart

Find $d(z)$ for a stationary distribution for graph 2.

\begin{solution}
10/19
\end{solution}

\ppart

If you start at node $w$ in graph 2 and take a (long) random walk,
does the distribution over nodes ever get close to the stationary
distribution?  (\hint try a few steps and watch what is happening.)

\begin{solution}
Yes
\end{solution}

\ppart

How many stationary distributions are there for graph~3?

\begin{solution}
infinitely many
\end{solution}

\ppart

If you start at node $b$ in graph~3 and take a (long) random walk, the
probabililty you are at node~$d$ will be close to:

\begin{solution}
1/3
\end{solution}

\eparts

\end{problem}

\endinput
