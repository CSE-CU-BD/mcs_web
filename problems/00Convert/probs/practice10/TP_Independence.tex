\documentclass[problem]{mcs}

\begin{pcomments}
    \pcomment{TP_Independence}
    \pcomment{Converted from prob5.scm
              by scmtotex and dmj
              on Sun 13 Jun 2010 04:23:49 PM EDT}
\end{pcomments}

\begin{problem}

%% type: multi-part
%% title: Independence

A length 3 bit string, $x$, is chosen at random.  Consider the
following events:

\begin{align*}
\mathbf{E} & ::= \text{$x$ contains an odd number of 1s.} \\
\mathbf{F} & ::= \text{$x$ starts with a 1.} \\
\mathbf{G} & ::= \text{$x$ starts with a 0.} \\
\mathbf{H} & ::= \text{$x$ ends with a 1.}
\end{align*}

\bparts

\ppart
%% type: short-answer
%% title: Independence of two events

Which pairs of the above events are independent?

%           Answer with a sequence of parenthesized pairs separated by
%         spaces such as
% 
%   ``(E F)  (G H) ``.
%         The pairs may be in any order, for example,
%             ``(G H)  (E F) `` is ok, but the letters in the pair
%             must be in alphabetical order, for example,
%             ``(F E)  (G H) `` is not ok.
% 

\begin{solution}

(E F) (E G) (E H) (F H) (G H)

(\textbf{E F}).  To check independence,
we want
\begin{equation*}
\pr{\mathbf{E} \intersect \mathbf{F}} =
    \pr{\mathbf{E}} \cdot \pr{\mathbf{F}}.
\end{equation*}

But $\pr{\mathbf{E} \intersect \mathbf{F}} = 2/8 = 1/4$ by enumerating
strings, and $\pr{\mathbf{E}} = \pr{\mathbf{F}} = 1/2$.  So, the
events are independent.

(\textbf{E G}). To check independence, we want
\begin{equation*}
\pr{\mathbf{E} \intersect \mathbf{G}}
    = \pr{\mathbf{E}} \cdot \pr{\mathbf{G}}.
\end{equation*}

But $\pr{\mathbf{E} \intersect \mathbf{G}} = 2/8 = 1/4$ by enumerating
strings, and $\pr{\mathbf{E}} = \pr{\mathbf{G}} = 1/2$.  So, the
events are independent.

(\textbf{E H}). To check independence, we want
\begin{equation*}
\pr{\mathbf{E} \intersect \textbf{H}}
    = \pr{\mathbf{E}} \cdot \pr{\textbf{H}}
        \cdot \pr{\mathbf{E} \intersect \textbf{H}} = 2/8 = 1/4
\end{equation*}
by enumerating strings, and $\pr{\mathbf{E}} = \pr{\mathbf{H}} =
1/2$.  So, the events are independent.

(\textbf{F G}). To check independence, we want
\begin{equation*}
\pr{\mathbf{F} \intersect \mathbf{G}}
    = \pr{\mathbf{G}} \cdot \pr{\mathbf{F}}.
\end{equation*}

But $\pr{\mathbf{G} \intersect \mathbf{F}} = 0/8 = 0$ by enumerating
strings, and $\pr{\mathbf{G}} = \pr{\mathbf{F}} = 1/2$.  So, the
events are \emph{not} independent.  In fact, the events are disjoint
so they cannot be independent!

(\textbf{F H}).
To check independence, we want
\begin{equation*}
\pr{\mathbf{H} \intersect \mathbf{F}}
    = \pr{\mathbf{H}} \cdot \pr{\mathbf{F}}.
\end{equation*}
But $\pr{\mathbf{H} \intersect \mathbf{F}} = 2/8 = 1/4$ by enumerating
strings, and $\pr{\mathbf{H}} = \pr{\mathbf{F}} = 1/2$.  So, the
events are independent.

(\textbf{G H}). These are also independent, similarly.
\end{solution}

\ppart
%% type: short-answer
%% title: Pairwise and Mutual Independence

Are the events $\mathbf{E}$, $\mathbf{F}$, and $\mathbf{H}$
\emph{pairwise} independent?

\begin{solution}
yes

To check whether the three events are \emph{pairwise} independent, we
need to check whether the pairs of events ($\mathbf{E}$,
$\mathbf{F}$), ($\mathbf{E}$, $\mathbf{H}$), and ($\mathbf{F}$,
$\mathbf{H}$) are independent.  But we already know they are, from
Part~1.
\end{solution}

\ppart
Are the events $\mathbf{E}$, $\mathbf{F}$, and $\mathbf{H}$
\emph{mutually} independent?

\begin{solution}
yes

To check whether the three events are \emph{mutually} independent, we
need to check every subset of the them.  We already know that subsets
of size~2 are ok, since by the previous question the three events are
\emph{pairwise} independent.  So, we just need to check the one subset
of all three, that is, {$\mathbf{E}$, $\mathbf{F}$,
$\mathbf{H}$}. Specifically, we want to know if
\begin{equation*}
\pr{\mathbf{E} \intersect \mathbf{F} \intersect \mathbf{H}}
    \stackrel{?}{=} \pr{\mathbf{E}} \cdot \pr{\mathbf{F}} \cdot \pr{\mathbf{H}}.
\end{equation*}
The only string satisfying all three events is ``111'', so
\begin{align*}
\pr{\mathbf{E} \intersect \mathbf{F} \intersect \mathbf{H}}
    & = 1/8 \\
    & = 1/2 \cdot 1/2 \cdot 1/2 \\
    & = \pr{\mathbf{E}} \cdot \pr{\mathbf{F}} \cdot \pr{\mathbf{H}}.
\end{align*}
Hence, this subset is also ok.  Therefore, the three events are also
\emph{mutually} independent.
\end{solution}

\ppart
%% type: short-answer
%% title: More

What about the events $\mathbf{F}$, $\mathbf{G}$, and $\mathbf{H}$?
They are\dots

\begin{solution}
not pairwise independent

From Part~1, we know the events $\mathbf{F}$ and $\mathbf{G}$ are not
independent.  Therefore, the events $\mathbf{F}$, $\mathbf{G}$, and
$\mathbf{H}$ are not \emph{pairwise} independent.  This immediately
implies that they are not \emph{mutually} independent, either.
\end{solution}

\ppart 
What about the events $\mathbf{E}$, $\mathbf{G}$, and $\mathbf{H}$?
They are\dots

\begin{solution}
mutually independent

From Part~1, we know all subsets of~2 events are ok. Hence,
$\mathbf{E}$, $\mathbf{G}$, and $\mathbf{H}$ are \emph{pairwise}
independent.  For mutual independence, we need to also check the
subset of all three events, that is, $\{\mathbf{E}, \mathbf{G},
\mathbf{H}\}$. We want to know if
\begin{equation*}
\pr{\mathbf{E} \intersect \mathbf{G} \intersect \mathbf{H}}
\stackrel{?}{=} \pr{\mathbf{E}} \cdot \pr{\mathbf{G}} \cdot
\pr{\mathbf{H}}.
\end{equation*}
The only string satisfying all three events is ``001'', so
\begin{equation*}
\pr{\mathbf{E} \intersect \mathbf{G} \intersect \mathbf{H}}
    = 1/8 = 1/2 \cdot 1/2 \cdot 1/2
    = \pr{\mathbf{E}} \cdot \pr{\mathbf{G}} \cdot \pr{\mathbf{H}}.
\end{equation*}
Hence, this subset is also ok.  Therefore, the three events are indeed
$mutually$ independent.
\end{solution}

\ppart
What about the events $\mathbf{E}$, $\mathbf{F}$, and $\mathbf{G}$?
They are\dots

\begin{solution}
not pairwise independent

As in the first question, the events $\mathbf{F}$ and $\mathbf{G}$
spoil it.  So, $\mathbf{E}$, $\mathbf{F}$, and $\mathbf{G}$ are not
$pairwise$ independent, and therefore not $mutually$ independent,
either.
\end{solution}

\eparts

\end{problem}

\endinput
