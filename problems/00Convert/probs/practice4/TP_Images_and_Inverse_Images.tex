\documentclass[problem]{mcs}

\begin{pcomments}
  \pcomment{TP_Images_and_Inverse_Images}
  \pcomment{needs solutions}
  \pcomment{Converted from images.scm by scmtotex and
    dmj on Sun 13 Jun 2010 10:52:29 AM EDT}
  \pcomment{edited for book by ARM 6/15/10}
\end{pcomments}

\begin{problem}

\bparts

\ppart
Let $V$ be the relation with the integers from 7 to~15, and codomain
the integers from 2 to~30, such that $m\mathcal{V}n$ is true iff $m$
is a divisor of~$n$.

\begin{enumerate}

\item
List the elements of $\mathcal{V}(\set{10,14})$, the \emph{image} of the set
$\set{10,14}$ under $\mathcal{V}$.

%%  Answer with a sequence of integers in any order separated by spaces.
%%  For example, 9 7 11

\begin{solution}
10,
20,
30,
14,
28
\end{solution}

\item
List the elements of $\mathcal{V^{-1}}{10,14}$, the \emph{inverse image} of
$\set{10,14}$ under $\mathcal{V}$.

\begin{solution}
7,
10,
14
\end{solution}

\end{enumerate}

\eparts

Here are some assertions about images under a binary relation $R:A \to B$:
\begin{align}
R(A) & = B \label{RA=B}\\ %total false\ surj true See Section~4.6 of the notes
R(B) & = A \label{RB=A}\\ %total/surj  false
\inv{R}(A) & =B \label{iRA=B}\\  %total/surj false 
\inv{R}(B) & = A \label{iRB=A}  %total true  See Section~4.6 of the  notes, surj false

\end{align}
\bparts

\ppart

Which of these assertions mean that $R$ is \emph{total}?
Describe counterexamples for those assertions that don't do the job.
That is, describe an $R$ for which the equivalence fails.

\begin{solution}
$\inv{R}(B) & = A \QIFF R \text{is total}$, as explained in Section~\ref{surj_sec}.

Counterexamples\dots
\end{solution}
\ppart

Same question for $R$ being \emph{surjective}.
\begin{solution}
$R(A) & = B \QIFF R \text{is surjective}$, as explained in Section~\ref{surj_sec}.

Counterexamples\dots
\end{solution}
\eparts

\end{problem}

\endinput
