\documentclass[problem]{mcs}

\begin{pcomments}
    \pcomment{TP_Graph_Coloring_I}
    \pcomment{Converted from prob5.scm
              by scmtotex and dmj
              on Sun 13 Jun 2010 10:52:29 AM EDT}
\end{pcomments}

\begin{problem}

%% type: short-answer
%% title: Graph Coloring I


\includegraphic{prob5}

What is the chromatic number of the above graph?

\begin{solution}
3

The chromatic number of this graph is~3. 

First, 3 colors are \emph{sufficient}: Two colors are enough for the
vertices of the outer rim ($a$, $b$, $c$, $d$, $e$, and~$f$), if we
alternate them along the rim.  Then a third color can be used for the
vertex, $g$, in the center.

Second, 3 colors are \emph{necessary}: Because the graph contains a
triangle, for example, the one formed by $a$, $b$, and $g$, and every
triangle needs 3 colors.
\end{solution}

\end{problem}

\endinput
