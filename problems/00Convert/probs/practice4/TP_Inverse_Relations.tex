\documentclass[problem]{mcs}

\begin{pcomments}
    \pcomment{Converted from inverse-relations.scm
              by scmtotex and dmj
              on Sun 13 Jun 2010 10:52:29 AM EDT}
\end{pcomments}

\begin{problem}

%% type: short-answer
%% title: Inverse Relations

The inverse, $\mathcal{R}^{-1}$, of a binary relation, $\mathcal{R}:
A\implies B$ is the relation from $B\implies A$ defined by
\begin{equation*}
    b \mathcal{R}^{-1} a \leftrightarrow  a\mathcal{R}b.
\end{equation*}

In other words, you get the diagram for $\mathcal{R}^{-1}$ from
$\mathcal{R}$ by ``reversing the arrows'' in the diagram describing
$\mathcal{R}$. Now many of the relational properties of
$\mathcal{R}^{-1}$ correspond to different properties of
$\mathcal{R}$. For example, $\mathcal{R}$ is total iff
$\mathcal{R}^{-1}$ is a surjection.

\bparts

\ppart $\mathcal{R}$ is a function iff $\mathcal{R}^{-1}$ is

\begin{solution}
an injection
\end{solution}

\ppart $\mathcal{R}$ is a surjection iff $\mathcal{R}^{-1}$ is

\begin{solution}
total
\end{solution}

\ppart
$\mathcal{R}$ is an injection iff $\mathcal{R}^{-1}$ is

\begin{solution}
a function
\end{solution}

\ppart
$\mathcal{R}$ is a bijection iff $\mathcal{R}^{-1}$ is

\begin{solution}
a bijection
\end{solution}

\eparts


\end{problem}

\endinput
