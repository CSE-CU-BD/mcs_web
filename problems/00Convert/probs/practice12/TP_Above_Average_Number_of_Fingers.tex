\documentclass[problem]{mcs}

\begin{pcomments}
    \pcomment{Converted from above-average-fingers.scm
              by scmtotex and dmj
              on Sun 13 Jun 2010 05:11:14 PM EDT}
\end{pcomments}

\begin{problem}

%% type: short-answer
%% title: Above Average Number of Fingers

There's a common wisecrack that 90\% of drivers consider themselves
above average.  (A recent mention of this appears in NY Times writer
\href{http://pogue.blogs.nytimes.com/2010/04/29/text-blocking-apps-only-work-if-you-use-them/}{David
  Pogue's blog, April 29, 2010}.)  This may sound absurd, but maybe
not: after all, the vast majority of people have an above average
number of fingers.  More than one of the following assertions explain
this fact; which ones?

\begin{enumerate}
    
\item
Most people have a super secret extra bonus finger of which they are unaware.
    
\item
A pedantic minority don't count their thumbs as fingers, while the
majority of people do.
    
\item
Polydactylism is rarer than amputation.
    
\item
When you add up the total number of fingers among the world's
population and then divide by the size of the population, you get a
number less than ten.

% <!--
% \item
% Some people have webbed fingers which are considered a  ``finger unit. ``
% -->
    
\item
This follows from Markov's Theorem, since no one has a negative number
of fingers.
    
\item
Missing fingers are much more common than extra ones.
    
\item
Missing fingers are at least slightly more common than extra ones.

\end{enumerate}

\begin{solution}

3

6

7

6

7


1: We don't think so.

2: The eccentric opinions of pedants don't matter, since they won't
affect the way we count fingers.

3: This is a polysyllabic \verb+:-)+ way of saying that there are more
people with fewer than ten fingers than there are with greater than
ten fingers, which ought to pull the average number of figures at
least a little below ten.  Since the vast majority of people have ten
fingers, they would all have more than the average number of fingers,
explaining the claim.

However, read literally, this statement is not quite strong enough, so
we also counted it as correct to omit it.  The reason is that even with
slightly more \emph{people} with missing fingers than \emph{people}
with extra fingers, the average number of fingers could be more than ten
if amputees were usually missing only one finger (which might be true),
while people with extra fingers usually had at least two extra
fingers (which is true).

What's needed is the stronger statement that the total number of extra
fingers in the world population is fewer than the total number of
missing fingers.  One reason this holds is that no one ever has more
than four extra fingers, and people with extra fingers are rarer than
people missing a hand.

4: This statement means that the average number of fingers is less
than ten, which is just a restatement of the claim using the obvious
fact that the vast majority of people have the usual ten fingers.  But
restating the claim does not explain why it is true.

% <!--
% : 5 could be part of an explanation for this claim if you assume that
% this phenomenon plus amputations outnumbers extra fingers from birth.
% -->

5: Markov's Theorem has no apparent relevance to the claim.  Markov's
Theorem simply states that the fraction of people who make up the
majority, multiplied by the above average number of fingers, namely
10, cannot be greater than the average.  So if 99 percent of the world
has 10 fingers, Markov's Theorem implies that the average number of
fingers is at least 9.9, which says nothing about how the average
compares to 10.

6 and 7: As long as the total number of missing fingers in the
population is larger than the total number of extra ones---it it
doesn't matter $how much$ larger---the average number of fingers will
be less than ten.
\end{solution}

%% 
%% Answer with a sequence of integers separated by
%% some spaces, for example, 
%% \begin{equation*}
%% 4 3 2 
%% \end{equation*}
%%  Don't use commas or
%% periods.

\end{problem}

\endinput
