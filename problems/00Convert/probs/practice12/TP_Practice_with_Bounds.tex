\documentclass[problem]{mcs}

\begin{pcomments}
    \pcomment{TP_Practice_with_Bounds}
    \pcomment{Converted from markov.scm
              by scmtotex and dmj
              on Sun 13 Jun 2010 05:11:14 PM EDT}
\end{pcomments}

\begin{problem}

%% type: short-answer
%% title: Practice with Bounds

Suppose 120 students take the 6.042 final exam and the mean of their
grades is 90.  However, you have no other information about the
students and the exam, \emph{e.g.}\ you should not assume that the
final is worth 100 points.

\bparts

\ppart

State the best possible upper bound on the number of students who
scored at least 180.

\begin{solution}

60

Let $R$ be the score of a student chosen at random.  According to
Markov's Theorem:

\begin{equation*}
    \pr{R \ge 180} \le E[R]/180 = 90/180 = 1/2.
\end{equation*}
So at most $(1/2) \cdot 120 = 60$ students scored greater than or
equal to~180.
\end{solution}

\ppart

Now suppose somebody tells you that the lowest score on the exam
is~30.  Compute the new best possible upper bound on the number of
students who scored at least~180.

\begin{solution}

48

Let $R$ be as in Question~1. We can apply Markov's Theorem to the
variable $R-30$:

\begin{equation*}
    \pr{R \ge 180} = \pr{R-30 \ge 150} \le \Ex[R-30]/150 = 60/150 = 2/5. 
\end{equation*}
So at most $(2/5) \cdot 120 = 48$ students scored greater than or
equal to~180.
\end{solution}

\eparts

\end{problem}

\endinput
