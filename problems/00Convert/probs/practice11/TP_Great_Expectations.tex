\documentclass[problem]{mcs}

\begin{pcomments}
    \pcomment{TP_Great_Expectations}
    \pcomment{Converted from great-expectations.scm
              by scmtotex and dmj
              on Sun 13 Jun 2010 04:49:03 PM EDT}
\end{pcomments}

\begin{problem}

%% type: short-answer
%% title: Great Expectations

\bparts

\ppart
What is the expected sum of the numbers that come up when you roll
a 6-sided and a 12-sided dice?

\begin{solution}

10


The expectation of a sum is the sum of the individual expectations, so

\begin{equation*}
(1+2+\dots +6)/6 + (1+2+\dots +12)/12 = 7/2 + 13/2=10.
\end{equation*}
\end{solution}

\ppart

Suppose you have two computers: Computer~1 generates a random number
in the set $\{1,2,\dots ,99\}$ with all numbers equally likely.
Similarly, Computer~2 generates a random number in ~$\{1,2,\dots,
999\}$ with all numbers equally likely.

You roll a fair die, and if a 5 comes up, you generate a random number
using Computer~1, otherwise you generate a random number using
Computer~2.  What is the expected value of the number you generate?

\begin{solution}

$2550/6$

By the law of Total Expectation,
\begin{align*}
\Ex[\text{generated number}]
    & = \pr{\text{roll $5$}} \cdot \Ex[\text{Computer 1 number}]
        + \pr{\neg (\text{roll $5$})} \cdot \Ex[\text{Computer 2 number}]\\
    & = (1/6)50+(5/6)500 \\
    & = 2550/6.
\end{align*}
\end{solution}

\ppart
Assuming that Computers 1 and~2 act independently, what is the
expected value of the product of the numbers they generate?

\begin{solution}

25000

The Product Rule can be used because the two random variables are
independent, so
\begin{align*}
\Ex[C_{1} \cdot C_{2}]
    & = \Ex[C_{1}] \cdot \Ex[C_{2}] \\
    & = 50 \cdot 500 \\
    & = 25000.
\end{align*}
\end{solution}

\eparts

\end{problem}

\endinput
