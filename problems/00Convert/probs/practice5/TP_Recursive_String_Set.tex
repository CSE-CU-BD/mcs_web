\documentclass[problem]{mcs}

\begin{pcomments}
    \pcomment{TP_Recursive_String_Set}
    \pcomment{Converted from recursive-stringset.scm
              by scmtotex and dmj
              on Sun 13 Jun 2010 11:22:25 AM EDT}
\end{pcomments}

\begin{problem}

%% type: multi-part
%% title: Recursive String Set

Let $\mathbf{W}$ be a set of strings.  The set $\mathbf{W}^{+}$ is
another set of strings defined recursively as follows:

\textbf{Base case:}
If $w \in \mathbf{W}$, then $w \in \textbf{W}^{+}$.

\textbf{Constructor case:}  

If $w \in \textbf{W}$ and $x \in \textbf{W}^{+}$, then $xw \in
\textbf{W}^{+}$ (here $xw$ is the \emph{concatenation} of the words
$x$ and~$w$).

%%  Write your answer as a sequence of digits in any order, such as 
%%  \begin{equation*}
%%  \texttt{1 4 3} 
%%  \end{equation*}

\bparts

\ppart
%% type: short-answer
%% title: 

Suppose $\mathbf{W} = \{\mathrm{ab}, \mathrm{abba}, \mathrm{a}\}$.

Which of the following strings are in $\textbf{W}^{+}$?

\begin{enumerate}
  
\item abababba
  
\item abbab
  
\item babba
  
\item aaaaaab

\end{enumerate}

\begin{solution}

1,
4

\begin{enumerate}
  
\item ((ab)ab)abba
  
\item
every possible start (''a'', ''ab'', ''abba'') is followed by ''b''
  
\item
cannot start with ''b''
  
\item
(((((a)a)a)a)a)ab

\end{enumerate}
\end{solution}

\ppart
%% type: short-answer
%% title: 

Suppose $\mathbf{W}$ is as in Part~1. 

Which of the predicates $P(x)$ below satisfy BOTH of the following
conditions:
\begin{enumerate}

\item[(a)]
$P(x)$ holds for all $x \in \mathbf{W}^{+}$

\item[(b)]
$P(x)$ leads to an \emph{easy, direct proof} when used as a
  \emph{structural induction} hypothesis in a proof of condition~(a).

\end{enumerate}
For example, one of the predicates below, call it $Q$, has another of
the predicates, call it $R$, as its base case.  $R$ itself has an easy
induction proof, but $Q$ is not considered to have such a proof since
it requires a subproof of $R$.

(Note: A string $y$ is a \emph{prefix} of string $x$ iff $yz$ = $x$
for some string $z$.  Because $z$ may be the empty string, every
string is considered to be a prefix of itself.)

\begin{enumerate}
  
\item
 $x$ has at least as many a's as b's
  
\item
 every prefix of $x$ has at least as many a's as b's
  
\item
 every prefix of $x$ has at most one more b than a's
  
\item
 $ax \in \mathbf{W}^{+}$
  
\item
 $xy \in \mathbf{W}^{+}$ for all $y \in \mathbf{W}^{+}$
  
\item
 $wx \in \mathbf{W}^{+}$ for all $w \in \mathbf{W}$

\end{enumerate}

\begin{solution}

1,
4,
6

Intuitively, a string is in $\mathbf{W}^{+}$ iff it can be written as
the concatenation of $\ge 1$ strings of~$\mathbf{W}$.  So, it is easy
to see that (1), (3), (4), (5), and (6) are all true for all $x \in
\mathbf{W}^{+}$.  Only (2) is false: a counterexample is
$\mathrm{abba} \in \mathbf{W}^{+}$, which has the prefix~abb with more
b's than a's.


Predicates (1), (4), and (6) lead to easy, direct proofs by structural
induction.  There isn't room to include these proofs here, but we
expected you to have thought through how these simple proofs would go
in order to figure out these answers.  (If you didn't do this already,
it's a valuable exercise to do it now.)

(3) does not lead to an easy direct proof.  The base case goes through
smoothly, but in the constructor case we also need to know that (1) is
true, and the induction hypothesis is not any help in proving this.
(By the way, a strengthening of (3) does have to an easy proof:
\begin{quote}
$x$ has at least as many a's as b's AND
 every prefix of $x$ has at most one more b than
a's.)
\end{quote}

(5) also does not lead to an easy direct proof because the \emph{base case}
of~(5) is precisely statement~(6) which requires its own inductive
proof.  That is, (5) is the $Q$ and (6) is the $R$ indicated in the
problem description above.

(On the other hand, once (6) is proved (by a separate inductive
proof), then (5) \emph{does} have an easy, direct proof by structural
induction, since the constructor step, namely, proving that
\begin{quote}
$(xw)y \in \mathbf{W}^{+}$ for all $y \in \mathbf{W}^{+}$, assuming
  that $xz \in \mathbf{W}^{+}$ for all $z \in \mathbf{W}^{+}$
\end{quote}
is easy, using (6) again.)
\end{solution}

\eparts

\end{problem}

\endinput
