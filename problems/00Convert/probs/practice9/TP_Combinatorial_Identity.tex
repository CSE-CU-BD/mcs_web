\documentclass[problem]{mcs}

\begin{pcomments}
    \pcomment{Converted from comb-identity.scm
              by scmtotex and dmj
              on Sun 13 Jun 2010 04:04:28 PM EDT}
\end{pcomments}

\begin{problem}

%% type: short-answer
%% title: Combinatorial Identity

From a set of $n$ balls, your task is to pick a group of $k$ balls and
a separate group of $(r - k)$ balls.
      
You could do it by picking a group of $r$ balls and then partitioning
it into sets of $k$ and $(r - k)$.  On the other hand, you could first
pick $k$ balls and then pick $(r - k)$ balls from the remaining $(n -
k)$ balls.
      
The equivalence of these two picking procedures directly translates
into one of the following identities.  Which one?
      
\begin{enumerate}
      
\item
${n \choose k} {k \choose {r-k}} = {n \choose k}{k \choose r}$

\item
${n \choose r}{r \choose k} = {n \choose k}{n-k \choose r-k}$

\item
${n \choose r}{r \choose k} = {n \choose r}{r \choose r-k}$

\item
${n \choose k}{n \choose r-k} = {n \choose r}{n-r \choose r-k}$

\item
${k \choose n}{n \choose r-k} = {r \choose n}{n-r \choose r-k}$

\item
${n \choose r}{r \choose k} = {n \choose k}{n \choose r-k}$
      
\end{enumerate}

\begin{solution}

2

\end{solution}

\end{problem}

\endinput
