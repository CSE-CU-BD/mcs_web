\documentclass[problem]{mcs}

\begin{pcomments}
  \pcomment{FP_odd_length_walk_simple}
  \pcomment{simple graph version of FP_odd_length_walk, variation of PS_odd_length_walk}
  \pcomment{ARM 11/9/15}
\end{pcomments}

\pkeywords{
  walk
  cycle
  closed_walk
  simple_graph
  odd_length
}

%%%%%%%%%%%%%%%%%%%%%%%%%%%%%%%%%%%%%%%%%%%%%%%%%%%%%%%%%%%%%%%%%%%%%
% Problem starts here
%%%%%%%%%%%%%%%%%%%%%%%%%%%%%%%%%%%%%%%%%%%%%%%%%%%%%%%%%%%%%%%%%%%%%

\begin{problem}
Since you can go back and forth on an edge in a simple graph, every
vertex is on an even length closed walk.  So even length closed walks
don't tell you much about even length cycles.  The situation with
odd-length closed walks is more interesting.

\bparts

\ppart Give an example of a simple graph in which every vertex is on a
unique odd-length cycle and a unique even-length cycle.

\hint Four vertices.

\examspace[0.75in]

\begin{solution}
A square with one diagonal edge.
\end{solution}

\ppart\label{oddcycleonly} Give an example of a simple graph in which
every vertex is on a unique odd-length cycle and no vertex is on an
even-length cycle.

\examspace[0.75in]

\begin{solution}
A graph consisting of an odd-length cycle is an example.  Then the
\emph{only} cycle in the graph is the whole graph.

The smallest example is a length three cycle.  That is, the graph with
three vertices $a,b,c$ and edges $\edge{a}{b},
\edge{b}{c},\edge{c}{a}$.
\end{solution}

\ppart Prove that in a digraph, a smallest size odd-length closed walk
must be a cycle.  Note that there will always be lots of even-length
closed walks that are shorter than the smallest odd-length one.

\hint Let $\walkv{e}$ be an odd-length closed walk of minimum size,
and suppose it begins and ends at vertex $a$.  If it is not a cycle,
then it must include a repeated vertex $b \neq a$.  That is,
$\walkv{e}$ starts with a walk $\walkv{f}$ from $a$ to $b$, followed
by a walk $\walkv{g}$ from $b$ to $b$, followed by a walk $\walkv{h}$
from $b$ to $a$.\footnote{In the notation of the text
\[
\walkv{e} = a\, \catv{\catv{\walkv{f}}{b}{\walkv{g}}}{b}{\walkv{h}}\, a.
\]}

\examspace[3in]

\begin{solution}
\begin{proof}
Since $\walkv{g}$ is a closed walk that is shorter than $\walkv{e}$, it
must be even-length.  But then
\[
\walkv{i} \eqdef a\, \catv{\walkv{f}}{b}{\walkv{h}}\,a
\]
is a closed walk that is shorter than $\walkv{e}$ by the length of
$\walkv{g}$, which is an even number.  So $\walkv{i}$ is an odd-length
closed walk shorter than $\walkv{e}$, a contradiction.
\end{proof}

\end{solution}

\eparts

\end{problem}

%%%%%%%%%%%%%%%%%%%%%%%%%%%%%%%%%%%%%%%%%%%%%%%%%%%%%%%%%%%%%%%%%%%%%
% Problem ends here
%%%%%%%%%%%%%%%%%%%%%%%%%%%%%%%%%%%%%%%%%%%%%%%%%%%%%%%%%%%%%%%%%%%%%

\endinput

\iffalse
\begin{figure}
\graphic{Fig_walkpath}
\caption{A positive even-length $v$ to $v$ walk, but no even-length $v$ to $v$ cycle.}
\label{fig:walkpath}
\end{figure}
}
As in Figure~\ref{fig:walkpath}, let
\begin{align*}
V & \eqdef \set{u,v,w,x},\\
E & \eqdef \set{\diredge{u}{w}, \diredge{w}{x}, \diredge{x}{u}, \diredge{v}{w}, \diredge{x}{v}}.
\end{align*}
There is a positive even-length closed walk from $v$ to $v$:
\[
v \diredge{v}{w} w \diredge{w}{x} x \diredge{x}{u} u  \diredge{u}{w} w \diredge{w}{x} x \diredge{x}{v}
\]
It is easy to check that the \emph{only} cycle including $v$ is the odd-length cycle
\[
v \diredge{v}{w} w \diredge{w}{x} x \diredge{x}{v}.
\]
There is no even-length cycle from $v$ to $v$ from~$u$ to~$u$
\emph{that contains~$v$}.  The reason is that the sole edge out of $u$
goes to $w$, and the sole edge out of~$v$ likewise goes to $w$, so any
walk from~$u$ to~$u$ that goes through~$v$ must go through $w$ at
least twice and therefore won't be a cycle.  \fi
