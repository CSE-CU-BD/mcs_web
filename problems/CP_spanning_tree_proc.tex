\documentclass[problem]{mcs}

\begin{pcomments}
  \pcomment{CP_spanning_tree_proc}
  \pcomment{subsumes CP_Mark_a_spanning_tree}
  \pcomment{from: S09.cp6t; revised ARM 3/2/10}
  \pcomment{commented out in S09 so check/revise before using}
  \pcomment{edited ARM 10/23/11}
\end{pcomments}

\pkeywords{
  graphs
  spanning_trees
  state_machines
  derived_variables
  increasing
  decreasing
  termination
  trees
}

%%%%%%%%%%%%%%%%%%%%%%%%%%%%%%%%%%%%%%%%%%%%%%%%%%%%%%%%%%%%%%%%%%%%%
% Problem starts here
%%%%%%%%%%%%%%%%%%%%%%%%%%%%%%%%%%%%%%%%%%%%%%%%%%%%%%%%%%%%%%%%%%%%%

\begin{problem}%\mbox{} %latex artifact

A procedure for connecting up a (possibly disconnected) simple graph
and creating a spanning tree can be modelled as a state machine whose
states are finite simple graphs.  A state is \emph{final} when no
further transitions are possible.  The transitions are determined by the
following rules:

\textbox{
%\begin{center}
\textboxheader{Procedure \textbf{create-spanning-tree}}
%\end{center}

\begin{enumerate}
\item If there is an edge $\edge{u}{v}$ on a cycle, then
  delete $\edge{u}{v}$.

\iffalse
there is an edge of $G$ between $u$ and $v$ and there is
also a path from $u$ to $v$ which does \emph{not} include this edge;
in this case, delete the edge $\edge{u}{v}$.
\fi

\item If vertices $u$ and $v$ are not connected, then add the edge
  $\edge{u}{v}$.
\end{enumerate}
}

\bparts

\ppart Draw all the possible final states reachable starting with the
graph with vertices $\set{1,2,3,4}$ and edges
\[
\set{\edge{1}{2},\edge{3}{4}}.
\]

\begin{solution}
It's not possible to delete any edge.  The procedure can only
add an edge connecting exactly one of vertices 1 or 2 to exactly one of
vertices 3 or 4, and then terminate.  So there are four possible final
states.
\end{solution}

\ppart Prove that if the machine reaches a final state, then the final
state will be a tree on the vertices graph on which it started.

\begin{solution}
We use the characterization of a tree as an acyclic connected
graph.

A final state must be connected, because otherwise there would be two
unconnected vertices, and then a transition adding the edge between
them would be possible, contradicting finality of the state.

A final state can't have a cycle, because deleting any edge on the
cycle would be a possible transition.
\end{solution}

\ppart \label{derived-ecs} For any graph, $G'$, let $e$ be the number of
edges in $G'$, $c$ be the number of connected components it has, and
$s$ be the number of cycles.  For each of the quantities \iffalse
derived variables\fi below, indicate the \emph{strongest} of the
properties that it is guaranteed to satisfy, no matter what the
starting graph is\inhandout{ and be prepared to briefly explain your
  answer}.

The choices for properties are: \emph{constant}, \emph{strictly
increasing}, \emph{strictly decreasing}, \emph{weakly increasing},
\emph{weakly decreasing}, \emph{none of these}.  %The quantiderived variables are

\begin{enumerate}

\item[(i)] $e$ \begin{solution}
none of these
\end{solution}

\item[(ii)] $c$ \begin{solution}
weakly decreasing
\end{solution}

\item[(iii)] $s$ \begin{solution}
weakly decreasing
\end{solution}

\item[(iv)] $e-s$  \begin{solution}
weakly increasing
\end{solution}

\item[(v)] $c+e$  \begin{solution}
weakly decreasing
\end{solution}

\item[(vi)] $3c + 2e$  \begin{solution}
strictly decreasing
\end{solution}

\item[(vii)] $c+s$ \begin{solution}
strictly decreasing
\end{solution}

\iffalse  %uncomment when well-founded partial order topic is included.
\item[(viii)] $(c,e)$, partially ordered coordinatewise (the
  \emph{product} partial order~\bref{productrel}).
\begin{solution}
    none of these
\end{solution}

\item[(ix)] $(c,e)$, ordered lexicographically  \begin{solution}
strictly decreasing
\end{solution}
\fi

\end{enumerate}

\ppart Prove that one of the quantities from part~\eqref{derived-ecs}
strictly decreases at each transition.  Conclude that for every
starting state, the machine will reach a final state.

\iffalse
by proving that one of the
derived variables above is strictly decreasing under some well founded
partial order.
\fi

\begin{solution}
  If a value associated with states, known as a \term{derived
    variable}, is nonnegative integer-valued and decreases at each
  transition, then the machine must reach a final state after at most
  as many transitions as the initial value of the variable.  So we
  need only identify such a derived variable.  There are two in the
  list above, namely (vi) and (vii).

  To show that the variable (vi) strictly decreases, note that
  the rule for deleting an edge ensures that the connectedness relation
  does not change, so neither does the number of connected components $c$.
  Meanwhile the number of edges $e$ decreases by one when an edge is
  deleted.  Therefore the variable $3c+2e$ decreases by $2$.  The rule for
  adding an edge ensures that the number of connected components $c$
  decreases by one and the number of edges $e$ increases by one.
  Therefore the variable $3c+2e$ decreases by $1$.

  To show that the variable (vii) strictly decreases, note that the
  rule for deleting an edge ensures that the number of connected
  components $c$ does not change and the number of cycles $s$
  decreases by $n$, where $n \geq 1$. Therefore the variable $c+s$
  decreases by $n$.  The rule for adding an edge ensures that the
  number of connected components $c$ decreases by one and the number
  of cycles $s$ does not change.  Therefore the variable $c+s$
  decreases by one.

\iffalse
  The variable (viii) is neither increasing nor decreasing since under
  the rule for adding an edge, the number of connected components $c$
  decreases by one and the number of edges $e$ increases by one.
\fi

\iffalse
To show that the lexicographically ordered $(c,e)$ strictly decreases,
note that the rule for deleting an edge ensures that the number of
connected components $c$ does not change and the number of edges $e$
decreases by one.  The rule for adding an edge ensures that the number of
connected components $c$ decreases by one.
\fi

\end{solution}

\eparts
\end{problem}

%%%%%%%%%%%%%%%%%%%%%%%%%%%%%%%%%%%%%%%%%%%%%%%%%%%%%%%%%%%%%%%%%%%%%
% Problem ends here
%%%%%%%%%%%%%%%%%%%%%%%%%%%%%%%%%%%%%%%%%%%%%%%%%%%%%%%%%%%%%%%%%%%%%

\endinput
