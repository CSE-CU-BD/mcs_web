\documentclass[problem]{mcs}

\begin{pcomments}
  \pcomment{MQ_state_machine_buckets_S15}
\end{pcomments}

\pkeywords{
  state_machine
}

%%%%%%%%%%%%%%%%%%%%%%%%%%%%%%%%%%%%%%%%%%%%%%%%%%%%%%%%%%%%%%%%%%%%%
% Problem starts here
%%%%%%%%%%%%%%%%%%%%%%%%%%%%%%%%%%%%%%%%%%%%%%%%%%%%%%%%%%%%%%%%%%%%%

\begin{problem}
% Tests understanding of state machines

The Stata Center's delicate balance depends on 2 buckets of water hidden in a secret room. The big bucket has a volume of \textbf{25 gallons}, and the little bucket has a volume of \textbf{10 gallons}. If at any time a bucket contains exactly \textbf{13 gallons}, the Stata Center will collapse. There is an interactive display where tourists can remotely fill and empty the buckets according to certain rules. We represent the buckets as a state machine.\newline
The state of the machine is a pair $(b, l)$, where $b=$ volume of water in big bucket and $l=$ volume of water in little bucket.

\examspace[3in]

\begin{problemparts}

\problempart
We have described all the legal operations tourists can perform below. Represent each of the following operations as a transition of the state machine. The first is done for you as an example.
\begin{enumerate}
\item Fill the big bucket. $(b,l) \rightarrow (25, l)$
\item Fill the little bucket.
\item Empty the big bucket.
\item Empty the little bucket.
\item Pour the big bucket into the little bucket.  You should have two cases: if all the water from the big bucket fits in the little bucket, then pour all the water. If it doesn't, pour until the little jar is full, then stop so the big jar will still have some water remaining.
\item Pour the little bucket into the big bucket (two cases again).
\end{enumerate}

\examspace[3in]

\begin{solution}

\end{solution}


\problempart
When the buckets were first created, both were empty to ensure the Stata Center would not collapse: the start state of the machine was $(0, 0)$. Use induction to prove that the Stata Center will never collapse: we will never reach the state $(13, x)$.\newline
\hint Show that $5$ always divides $b$ and $l$.

\begin{solution}

\end{solution}

\end{problemparts}
\end{problem}

\endinput
