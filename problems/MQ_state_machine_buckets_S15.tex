\documentclass[problem]{mcs}

\begin{pcomments}
  \pcomment{MQ_state_machine_buckets_S15}
  \pcomment{proposed by Lisa}
\end{pcomments}

\pkeywords{
  state_machine
  gcd
  linear_combination
  invariant
}

%%%%%%%%%%%%%%%%%%%%%%%%%%%%%%%%%%%%%%%%%%%%%%%%%%%%%%%%%%%%%%%%%%%%%
% Problem starts here
%%%%%%%%%%%%%%%%%%%%%%%%%%%%%%%%%%%%%%%%%%%%%%%%%%%%%%%%%%%%%%%%%%%%%

\begin{problem}
% Tests understanding of state machines

The Stata Center's delicate balance depends on two buckets of water
hidden in a secret room.  The big bucket has a volume of 25 gallons,
and the little bucket has a volume of 10 gallons.  If at any time a
bucket contains exactly \textcolor{red}{13 gallons}, the Stata Center
will collapse.  There is an interactive display where tourists can
remotely fill and empty the buckets according to certain rules.  We
represent the buckets as a state machine.

The state of the machine is a pair $(b, l)$, where $b$ is the volume
of water in big bucket, and $l$ is the volume of water in little
bucket.

%\examspace[3in]

\begin{problemparts}

\problempart\label{1025moves} We informally describe some of the legal
operations tourists can perform below.  Represent each of the
following operations as a transition of the state machine.  The first
is done for you as an example.
\begin{enumerate}
\item Fill the big bucket.
\[
(b,l) \movesto (25, l).
\]
%\item Fill the little bucket.
%\item Empty the big bucket.

\item Empty the little bucket.

\examspace[0.4in]

\item Pour the big bucket into the little bucket.  You should have two
  cases defined in terms of the state $(b,l)$: if all the water from
  the big bucket fits in the little bucket, then pour all the water.
  If it doesn't, pour until the little jar is full, leaving some water
  remaining in the big jar.

%\item Pour the little bucket into the big bucket (two cases again).
\end{enumerate}

\examspace[0.8in]

\begin{solution}

All three cases together:

\begin{enumerate}

\item \[
(b,l) \movesto (25, l).
\]

\item \[
(b,l) \movesto (b, 0).
\]

\item When $b + l \leq 10$: \[
(b,l) \movesto (0, b + l).
\]

Otherwise: \[
(b, l) \movesto (b - (10 - l), 10)
\]

\end{enumerate}

\end{solution}

\problempart Use the Invariant Principle to show that, starting with
empty buckets, the Stata Center will never collapse.  That is, the
state $(13, x)$ in unreachable.  (In verifying your claim that the
invariant is preserved, you may restrict to the representative
transitions of part~\eqref{1025moves}.)

\begin{staffnotes}
A simple invariant is that $5$ divides $b$ and $l$.
\end{staffnotes}

\begin{solution}
  We apply the invariant principle with the invariant $I(b, l) ::= 5 \divides b \QAND{} 5 \divides l$.  We must show that the initial state satisfies $I$ and that every move preserves $I$.  Clearly the initial state $(0, 0)$ satisfies $I$, as $0 = 0 \times 5$.

  Now we assume $I(b, l)$ and prove $I(b', l')$ for every possible move $(b, l) \movesto (b', l')$.  It turns out that we don't even need to consider the side condition in the 3rd rule; we'll just check both possibilities.  So, looking at the right-hand side of each $\movesto$ step above, we see that each result is built out of $b$, $l$, constants that are divisible by 5, and addition and subtraction.  By what we know about properties of arithmetic mod 5, since $b$ and $l$ are assumed divisible by 5, the \emph{results} of all these expressions must also be divisible by 5, so $I(b', l')$.

  Therefore, by the invariant principle, $I$ holds of every reachable state.  But $I$ \emph{does not} hold of any state where either bucket is 13, so the Stata Center is safe.
\end{solution}

\end{problemparts}
\end{problem}

\endinput
