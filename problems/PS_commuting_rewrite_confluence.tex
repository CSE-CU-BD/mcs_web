\documentclass[problem]{mcs}

\begin{pcomments}
  \pcomment{PS_commuting_rewrite_confluence}
  \pcomment{ARM 1/13/17}
\end{pcomments}

\pkeywords{
  mating_ritual
  transition
  commuting
  confluence
}

%%%%%%%%%%%%%%%%%%%%%%%%%%%%%%%%%%%%%%%%%%%%%%%%%%%%%%%%%%%%%%%%%%%%%
% Problem starts here
%%%%%%%%%%%%%%%%%%%%%%%%%%%%%%%%%%%%%%%%%%%%%%%%%%%%%%%%%%%%%%%%%%%%%

\begin{problem}
A state machine has \emph{commuting transitions} if for any states
$p,q,r$
\[
(p \movesto q \ \QAND\ q \movesto r) \QIMPLIES \exists t.\, q \movesto t \ \QAND\ r \movesto t.
\]
The state machine is \emph{confluent} if
\[
(p \movesto^* q \ \QAND\ q \movesto^* r) \QIMPLIES^* \exists t.\, q \movesto^* t \ \QAND\ r \movesto^* t
\]

\bparts
\ppart Prove that if a state machine has commuting transitions, then it is confluent.

\hint By induction on the number of moves from $p$ to $q$ plus the number from $p$ to $r$.

\begin{solution}
\TBA{needed}
\end{solution}

\ppart A \emph{final state} of a state machine is one from which no
transition is possible.  Explain why, if a state machine is confluent,
then at most one final state is reachable from the start state.

\begin{solution}
If $q$ is a final state from which no transition is possible, then $q
\movesto t$ is only possible if $t=q$.  Now suppose $p \movesto^* q
\ \QAND\ p \movesto^* r$ where both $q$ and $r$ are final states.  So
if $q \movesto t$ and $r \movesto t$, it must be that $q=t=r$.  Hence
confluence would imply that any final states $q,r$ that are reachable
from the start state $p$ must be equal, that is, there is at most one
final state.
\end{solution}

\eparts

\end{problem}

%%%%%%%%%%%%%%%%%%%%%%%%%%%%%%%%%%%%%%%%%%%%%%%%%%%%%%%%%%%%%%%%%%%%%
% Problem ends here
%%%%%%%%%%%%%%%%%%%%%%%%%%%%%%%%%%%%%%%%%%%%%%%%%%%%%%%%%%%%%%%%%%%%%

\endinput
