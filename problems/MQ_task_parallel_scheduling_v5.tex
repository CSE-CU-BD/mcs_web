\documentclass[problem]{mcs}

\begin{pcomments}
  \pcomment{MQ_task_parallel_scheduling_v5}
  \pcomment{variant of _v3}
\end{pcomments}

\pkeywords{
  DAG
  scheduling
  chains_and_antichains
}

%%%%%%%%%%%%%%%%%%%%%%%%%%%%%%%%%%%%%%%%%%%%%%%%%%%%%%%%%%%%%%%%%%%%%
% Problem starts here
%%%%%%%%%%%%%%%%%%%%%%%%%%%%%%%%%%%%%%%%%%%%%%%%%%%%%%%%%%%%%%%%%%%%%

\begin{problem}
The following DAG describes the prerequisites among tasks $\set{1, \dots, 9}$.

\begin{figure}[h]
\graphic[height=3.0in]{sequence-poset}
\end{figure}

\bparts

\ppart If each task takes unit time to complete, what is the minimum
time to complete all the tasks?  Briefly explain.

\exambox{0.7in}{0.4in}{0in}

\begin{solution}
\textbf{4}.  This is the size of a maximum chain, for example $1238$.
\end{solution}

\examspace[0.75in]

%\examspace[2cm]

\ppart What is the minimum time if no more than two tasks can be
completed in parallel?  Briefly explain.

\begin{solution}
\textbf{5}.  There are 9 tasks and two processors, so time at least
$\ceil{9/2} = 5$ is required.  A schedule that achieves this is
$14,26,57,39,8$.
\end{solution}

\exambox{0.7in}{0.4in}{0in}
\examspace[0.75in]

\eparts

\end{problem}

%%%%%%%%%%%%%%%%%%%%%%%%%%%%%%%%%%%%%%%%%%%%%%%%%%%%%%%%%%%%%%%%%%%%%
% Problem ends here
%%%%%%%%%%%%%%%%%%%%%%%%%%%%%%%%%%%%%%%%%%%%%%%%%%%%%%%%%%%%%%%%%%%%%

\endinput
