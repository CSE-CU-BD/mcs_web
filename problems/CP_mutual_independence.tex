\documentclass[problem]{mcs}

\begin{pcomments}
  \pcomment{CP_mutual_independence}
  \pcomment{Christos from Fall 01 Repository}
  \pcomment{minor edits by ARM 4/23/11}
\end{pcomments}

\pkeywords{
  independence
  mutual_independence
  intersection
  union
}

%%%%%%%%%%%%%%%%%%%%%%%%%%%%%%%%%%%%%%%%%%%%%%%%%%%%%%%%%%%%%%%%%%%%%
% Problem starts here
%%%%%%%%%%%%%%%%%%%%%%%%%%%%%%%%%%%%%%%%%%%%%%%%%%%%%%%%%%%%%%%%%%%%%

\begin{problem}
Let $A,B,C$ be events.  For each of the following statements, prove it
or give a counterexample.

\bparts

\iffalse

\ppart If $A$ is independent of $B$, then $A$ is also independent of
$\bar{B}$.

\begin{solution}
 
This is true. To prove it, suppose $A$ is independent of $B$; that
is, $\prob{A\intersect B}=\prob{A}\prob{B}$. Then
\begin{align}
\label{eq:pr-rule-1}
\prob{A\intersect \overline{B}}
&=
\prob{A}\prcond{\bar{B}}{A}
\\
\label{eq:co-rule-1}
&=
\prob{A}(1-\prcond{B}{A})
\\
\nonumber
&=
\prob{A}-\prob{A}\prcond{B}{A}
\\
\label{eq:pr-rule-2}
&=
\prob{A}-\prob{A\intersect B}
\\
\label{eq:independence}
&=
\prob{A}-\prob{A}\prob{B}
\\
\nonumber
&=
\prob{A}(1-\prob{B})
\\
\label{eq:co-rule-2}
&=
\prob{A}\prob{\overline{B}}
\end{align}
Equations~\eqref{eq:pr-rule-1} and~\eqref{eq:pr-rule-2} make use of 
the Product Rule. The Complement Rule is used in~\eqref{eq:co-rule-1}
and~\eqref{eq:co-rule-2}. We use independence of $A$ and $B$
in~\eqref{eq:independence}. 

\end{solution}
\fi

\ppart\label{AindBindCthenABC}
If $A$ is independent of $B$, and $A$ is independent of $C$, 
then $A$ is independent of $B\intersect  C$.

\begin{solution}

This is false.  To see why, consider the usual random experiment of
rolling a die and let $A$, $B$, $C$ be the events that the die rolls
less than 3, rolls an even number, or rolls a prime number,
respectively, namely,
\[
A = \set{1,2} \qquad
B = \set{2,4,6} \qquad
C = \set{2,3,5}.
\]
Then, 
\[
\prob{A}=\tfrac{1}{3}, 
\qquad
\prob{B}=\prob{C}=\tfrac{1}{2},
\]
and
\[
\prob{A\intersect B}=
\prob{A\intersect C}=
\prob{B\intersect C}=
\prob{A\intersect B\intersect C}=
\prob{\set{2}}=
\tfrac{1}{6}.
\]
Easily, 
\[
\prcond{A}{B}=
\prcond{A}{C}=
\tfrac{1}{6}/\tfrac{1}{2}=
\tfrac{1}{3}=
\prob{A},
\]
which proves $A$ is independent of $B$ and also independent of
$C$. However, 
\[
\prcond{A}{B\intersect C}=
\prob{A\intersect B\intersect C}/\prob{B\intersect C}=
\tfrac{1}{6}/\tfrac{1}{6}=
1\neq
\prob{A},
\]
which implies $A$ is not independent of $B\intersect C$.

\end{solution}

\ppart 
If $A$ is independent of $B$, and $A$ is independent of $C$, 
then $A$ is independent of $B\union C$.

\begin{solution}

This is false using the same example as for
part~\eqref{AindBindCthenABC}, where
\[
\prob{B\union C}=\tfrac{5}{6}, \qquad
\prob{A\intersect(B\union C)}=\tfrac{1}{6}.
\]
Hence,
\[
\prcond{A}{B\union C}=
\tfrac{1}{6}/\tfrac{5}{6}=
\tfrac{1}{5}\neq
\prob{A}
\]
which again implies $A$ is not independent of $B\union C$.

\end{solution}

\ppart 
If $A$ is independent of $B$, and $A$ is independent of $C$, and $A$
is independent of $B\intersect C$, 
then $A$ is independent of $B\union C$.

\begin{solution}
This is true.  To prove it, suppose $A$ is independent of each of
$B$, $C$, and $B\intersect C$, so
\begin{align}
\prob{A\intersect B} &= \prob{A}\prob{B} \label{AintBAB}\\
\prob{A\intersect C} &= \prob{A}\prob{C}  \label{AintCAC}\\
\prob{A\intersect (B\intersect C)} &= \prob{A}\prob{B\intersect C}. \label{AintBintCABC}\\
\end{align}
Then, 
\begin{align*}
\prob{A\intersect(B\union C)}
  & = \prob{(A\intersect B) \union (A\intersect C)}\\
  & = \prob{A\intersect B} + \prob{A\intersect C}
          - \prob{(A\intersect B)\intersect(A\intersect C)}
            & \text{(inclusion-exclusion)}\\
  & = \prob{A\intersect B} + \prob{A\intersect C}
          - \prob{A\intersect B\intersect C}\\
  & = \prob{A}\prob{B} + \prob{A}\prob{C}
          - \prob{A}\prob{B\intersect C}
            & \text{(by \eqref{AintBAB}, \eqref{AintCAC}, \eqref{AintBintCABC})}\\
  & = \prob{A}(\prob{B} + \prob{C} - \prob{B\intersect C})\\
  & = \prob{A}\prob{B \union C}.
       & \text{(inclusion-exclusion)}
\end{align*}

\end{solution}

\eparts


\end{problem}

%%%%%%%%%%%%%%%%%%%%%%%%%%%%%%%%%%%%%%%%%%%%%%%%%%%%%%%%%%%%%%%%%%%%%
% Problem ends here
%%%%%%%%%%%%%%%%%%%%%%%%%%%%%%%%%%%%%%%%%%%%%%%%%%%%%%%%%%%%%%%%%%%%%

\endinput
