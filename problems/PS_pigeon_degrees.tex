\documentclass[problem]{mcs}

\begin{pcomments}
  \pcomment{PS_pigeon_degreee}
  \pcomment{part (b) of PS_pigeon_hunting}
\end{pcomments}

\pkeywords{
  Pigeon_Hole
  simple_graph
  degree
  vertex
}

%%%%%%%%%%%%%%%%%%%%%%%%%%%%%%%%%%%%%%%%%%%%%%%%%%%%%%%%%%%%%%%%%%%%%
% Problem starts here
%%%%%%%%%%%%%%%%%%%%%%%%%%%%%%%%%%%%%%%%%%%%%%%%%%%%%%%%%%%%%%%%%%%%%

\begin{problem}

\iffalse
\bparts

\ppart Show that any odd integer $x$ in the range $10^9 < x < 2 \cdot 10^9$
containing all ten digits $0,1,\dots,9$ must have consecutive even digits.
\hint What can you conclude about the parities of the first and last digit?

\begin{solution}
In any such number, the first digit must be 1, the last must be odd
and each digit from 0 to 9 must appear exactly once.  The only way for
there to be no consecutive even digits is if every even digit has an
odd digit on either side of it.  Treating the 4 positions between a
sequence of the 5 odd digits beginning with 1 as the pigeonholes and
the 5 even digits as pigeons, the pigeonhole principle implies that
there must exist consecutive even digits.
\end{solution}

\ppart\fi

Show that there are 2 vertices of equal degree in any finite
undirected graph with $n \geq 2$ vertices. \hint Cases conditioned upon
the existence of a degree zero vertex.

\begin{solution}
In the case where there are no degree 0 vertices, the domain for the
degrees is the integer interval $[1,n-1]$ since each vertex can be
adjacent to at most all of the remaining $n-1$ vertices.  If there is
a degree zero vertex then it is not possible for any vertex to have
degree $n-1$ and so the domain is $[0,n-2]$.  In either
case we treat the $n$ vertices as the pigeons and the $n-1$ possible
degrees as the pigeonholes, the Pigeonhole Principle implies that
there must exist a pair of vertices with the same degree.
\end{solution}

%\eparts
\end{problem}

%%%%%%%%%%%%%%%%%%%%%%%%%%%%%%%%%%%%%%%%%%%%%%%%%%%%%%%%%%%%%%%%%%%%%
% Problem ends here
%%%%%%%%%%%%%%%%%%%%%%%%%%%%%%%%%%%%%%%%%%%%%%%%%%%%%%%%%%%%%%%%%%%%%

\endinput
