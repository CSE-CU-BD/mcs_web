\documentclass[problem]{mcs}

\begin{pcomments}
  \pcomment{CP_counter_cancat}
  \pcomment{ARM 2.15.16}
\end{pcomments}

\pkeywords{
 counter
 counter_machine
 concatenation
 binary
 exponentiate
}

\renewcommand{\lnth}[1]{\text{lnth}(#1)}

%%%%%%%%%%%%%%%%%%%%%%%%%%%%%%%%%%%%%%%%%%%%%%%%%%%%%%%%%%%%%%%%%%%%%
% Problem starts here
%%%%%%%%%%%%%%%%%%%%%%%%%%%%%%%%%%%%%%%%%%%%%%%%%%%%%%%%%%%%%%%%%%%%%


\begin{problem}
\begin{staffnotes}
If time allows, have students work out the counter machine code (using
subroutines for the given operations) for one of the problem parts.

\end{staffnotes}

This is an elementary programming exercise for Counter Machines.  You
are asked to describe how a Counter Machine could calculate certain
quantities, assuming registers $X,Y,R,Q,\dots$ initially contain
integers $x,y,r,q,\dots \in \naturals$ or 0, and all temporary
registers initally contain zero.

You may assume the programs in lecture for setting to zero, copying,
adding, and multiplying are available as subroutines.  For most parts,
you do not actually have to write the Counter Machine code.

\bparts

\ppart Describe how a Counter Machine with counters $X$ and $Y$ could
and halt leaving in counters $Q$ and $R$ containing $\quotient(x,y)$
and $\remainder(x.y)$. How many counters would your machine use?

\begin{solution}
Keep subtracting the contents $y$ of $Y$ from $X$, restoring $Y$ to
contain $y$ after each subtraction.  Use a counter $Q$ to count how
many times $x$ can successfully be subtracted from $y$ without trying
to subtract 1 from $Y$ when it contains 0.  When $y$ can no longer
be subtracted from the contents of $X$, copy the contents of $Y$ to
$R$ and halt.

I think this can be done with just one temp counter in additions to
$X,Y,Q,R$.
\end{solution}

\ppart Describe how a Counter Machine with counter $X$ could halt
leaving the length, $\lnth{x}$, of the binary representation of $x$ in
counter $L$.

\begin{solution}
Keep computing the quotient of the contents $X$ by 2, using counter
$L$ to count how many quotients by $2$ are needed until $X$ contains
zero.
\end{solution}


\ppart Describe how a Counter Machine with counters $X$ and $Y$ could
halt leaving in register $S$ the numerical result of a $\lnth{y}$ left
shift of the binary representation of $x$.

\begin{staffnotes}
\hint $2^y x$
\end{staffnotes}

\ppart Describe how a Counter Machine with counters $X$ and $Y$ could
halt leaving in register $C$ the numerical value of concatenation of
the binary representation of $x$ and the binary representation of $y$.

\begin{solution}
Compute $2^{\lnth{y}}x + y$ using the subroutine for addition along
with the ones described above.
\end{solution}

\eparts

\end{problem}
\endinput
