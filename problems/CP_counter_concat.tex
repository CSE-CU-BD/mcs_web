\documentclass[problem]{mcs}

\begin{pcomments}
  \pcomment{CP_counter_cancat}
  \pcomment{ARM 2.15.16}
\end{pcomments}

\pkeywords{
 counter
 counter_machine
 concatenation
 binary
 exponentiate
}

\newcommand{\lnthb}[1]{\text{lnth}(#1)}
\newcommand{\contrr}[1]{\widehat{#1}}

%%%%%%%%%%%%%%%%%%%%%%%%%%%%%%%%%%%%%%%%%%%%%%%%%%%%%%%%%%%%%%%%%%%%%
% Problem starts here
%%%%%%%%%%%%%%%%%%%%%%%%%%%%%%%%%%%%%%%%%%%%%%%%%%%%%%%%%%%%%%%%%%%%%


\begin{problem}
\begin{staffnotes}
If time allows, have students work out the counter machine code (using
subroutines for the given operations) for one of the problem parts.

\end{staffnotes}

This is an elementary programming exercise for Counter Machines.  You
are asked to describe how a Counter Machine could do calculations on
the contents of its counters.  For any register $R$, let $\contrr{R}
\in \nngint$ be the contents of $R$.

You may assume the programs in lecture for setting to zero, copying,
adding, and multiplying are available as subroutines.  For most parts,
you do not actually have to write the Counter Machine code.

\bparts

\ppart Describe how a Counter Machine with distinct counters $N,Q,R$
could execute the assignment command
\[
Q := \quotient(N,2).
\]
Likewise
\[
R := \remainder(N,2).
\]
How many counters would your machine use?

\begin{solution}
Keep subtracting 2 from $N$, using $Q$ to count the number of
subtractions performed until $\contrr{N}$ is reduced to 1 or 0.  Then
execute an assignment command $R := N$.

This requires no other counters besides $N,Q,R$ if the contents of $N$
don't have to be preserves.  Otherwise, two extra counters would be
needed to save the initial contents of $N$ and restore it at the end.
\end{solution}

\iffalse
\ppart For $n \in \nngint$, let $\lnthb{n}$ be the length of the
binary representation of $n$. Describe how a Counter Machine could
execute the assignment command
\[
L := \lnthb{N}.
\]
where 

\begin{solution}
Keep computing the quotient of the contents $N$ by 2, using counter
$L$ to count how many quotients by $2$ are needed until $X$ contains
zero.
\end{solution}


\ppart For $m,n \in \nngint$, let $\text{left-shift}(n,m)$ be the
number whose binary representation is

Describe how a Counter Machine with counters $X$ and $Y$ could
halt leaving in register $S$ the number whose binary representation is
the $\lnthb{y}$ left shift of the binary representation of $x$.

\begin{staffnotes}
\hint $2^y x$
\end{staffnotes}

\ppart Describe how a Counter Machine with counters $X$ and $Y$ could
halt leaving in register $C$ the numerical value of concatenation of
the binary representation of $x$ and the binary representation of $y$.

\begin{solution}
Compute $2^{\lnthb{y}}x + y$ using the subroutine for addition along
with the ones described above.
\end{solution}
\fi

\eparts

Suppose $R_0,R_1,R_2,R_3,\dots,R_n$ are $n+1$ distinct counters.  We
can represent the contents of all these counters with a binary string
of the form
\begin{equation}\label{bincode}
10^{\contrr{R_n}}10^{\contrr{R_{n-1}}}1 \dots 10^{\contrr{R_2}}10^{\contrr{R_1}}10^{\contrr{R_0}}1.
\end{equation}
In other words, the 1's divide the string into blocks of 0's, and the
length of block $i$ is $\contrr{R_i}$.  Now this binary string is the binary
representation of some nonnegative integer $s \in \nngint$, and so
the single number $s$ encodes all the numbers $\contrr{R_0},\contrr{R_1},\dots,\contrr{R_n}$.

With a counter $S$ containing the code number $s$, a single counter
machine with a fixed number of counters can simulate the operations of
an $n+1$ counter machine.  The rest of this problem illustrate how
this works.

\bparts

\ppart Describe how a Counter Machine could execute the decrement $R_0$
command
\[
R_0 := R_0 - 1.
\]

\begin{solution}
Simulate the successive commands
\begin{align*}
S & := \quotient(S,2);\\
  & \qquad \text{\%delete the right hand 1 in the binary rep}\\
\textbf{ if } & \remainder(S,2) = 0 \textbf{  then } S:= \quotient(S,2);\\
  & \qquad  \text{\%delete a 0, if any, from the right hand block of 0's}\\
S & := 2*S+1.\\
  & \qquad  \text{\%put the 1 back on the right end of the string}\\
\end{align*}
\end{solution}

\ppart Let $T$ be another register and assume $n\geq 1$.  Describe how
a Counter Machine could execute the ``test $R_1$ for zero'' command:
\begin{equation}\label{ifzT}
\textbf{if } R_1 = 0 \textbf{ then } T := 1 \textbf{ else } T := 0.
\end{equation}

\begin{solution}
Save the contents of $S$.  Do the assignment command
\[
S :=\quotient(S,2)
\]
which removes the 1 on the right hand end of the code string~\eqref{bincode}.
Then repeat
\[
S := \quotient(S,2)
\]
until $\remainder(S,2) = 1$.  This removes all the 0's in block 0 as
well as the 1 at the end of the block.  So $S$ now contains a number
whose binary representation ends with $0^{\contrr{R_1}}$.  This means
$\contrr{R_1}=0$ iff the binary representation of $\contrr{S}$ ends
with the 1 at the left end of block 1.  That is $\contrr{R_1}=0$ iff
$\remainder(\contrr{S},2) = 1$.  So the simulation of the
command~\eqref{ifzT} can be completed by executing
\[
T := \remainder(S,2),
\]
and then restoring the original contents of $S$.
\end{solution}

\ppart Let $x \eqdef \contrr{X}$ and suppose $x \leq n$.  Describe how
a Counter Machine could execute the command ``indirect $X$ + 1'' command
\[
X^* := X^*+1
\]
which means
\[
R_{x} : = R_{x} + 1.
\]

\begin{solution}
Let $H,L$ be extra counters.  Execute the commands
\[
H := 10^{\contrr{R_n}}10^{\contrr{R_{}n-1}}1 \dots 10^{\contrr{R_x}};\quad
L := 10^{\contrr{R_{}x-1}}10^{\contrr{R_{}x-21}}1 \dots 10^{\contrr{R_0}}1.
\]
This can be done by taking quotients and remainders by 2 of $S$ to
simulate reading the successive bits of the binary representation of
$S$ until $x+1$ 1-bits have been read.  Now do
\[
H := 2*H; S:= \text{concat}(H,L),
\]
where $\text{concat}(H,L)$ is the number whose binary representation consists
of the concatenation of the representations of $\contrr{H}$ and $\contrr{L}$.
\end{solution}

\ppart (optional) Discuss how to handle letting $n$ vary.  In
particular, suppose $n$ was kept in some counter $N$.  How would you
update $S$ to represent $R_0,R_1,\dots,R_{n+1}$?
\begin{solution}
Like the previous part.  Details Left to the reader.
\end{solution}

\eparts

\end{problem}
\endinput
