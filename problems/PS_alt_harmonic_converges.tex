\documentclass[problem]{mcs}

\begin{pcomments}
  \pcomment{PS_alt_harmonic_converges}
  \pcomment{F01.ps7}
  \pcomment{ARM 5/11/17}
  \pcomment{Special case of Leibniz's Rule}
\end{pcomments}

\pkeywords{
  sum
  series
  alternating
  converge
  Leibniz
}

\begin{problem}
Prove that the Alternating Harmonic Series
\[
1 - 1/2 + 1/3 - 1/4 + \cdots \pm 
\]
converges.

\begin{solution}
\begin{proof}
\[
\lefteqn{1 - \frac12 + \frac13 - \frac14 + \cdots \pm \frac1n} = \sum_{i=1^n} \frac{(-1)^{n+1}}{i}
\]
For $n = 2m$,
\begin{align*}
& = \sum_{i=1^n} \frac{(-1)^{i+1}}{i}\\
& = \sum_{k=1}^{m} \paren{\frac1{2k-1} -\frac1{2k}}\\
& = \sum_{k=1}^{m} \frac1{(2k-1)2k}\\
& \leq \sum_{k=1}^{m} \frac1{(2k)^2}\\
& = 4 \sum_{k=1}^{m} \frac1{k^2}\\
& \leq 4\int_1^{m} \frac1{x^2}\, dx
& = 4\paren{-\frac{1}{m} - (- \frac11)})\\
& \leq 4.
\end{align*}
So $\sum_{k=1}^{m} \paren{\frac1{2k-1} -\frac1{2k}}$ is an increasing
sequence bounded above by 4, and therefore it converges to some $c>0$.
Similarly, when $n = 2m+1$, the sum $\sum_{k=1}^{m} \paren{-\frac1{2k}
  +\frac1{2k+1}}$ is a decreasing sequence bounded below, and
therefore it converges to some $d<0$.  Hence, $\sum_{i=1^n}
\frac{(-1)^{i+1}}{i}$ has a limsup of $c$ and and a liminf of $1-d$.
Since the absolute value of the terms $\frac{(-1)^{i+1}}{i}$
approaches zero, the limsup and liminf must be equal.  Therefore,
\[
\lim_{n \to \infty} \sum_{i=1^n} \frac{(-1)^{i+1}}{i} = c.
\]
\end{proof}

More generally, Leibniz's Rule, \cite{Apostol67}, p.404, states that
any sum of alternating positive and negative terms whose values
approach zero will converge.
\end{solution}

\end{problem}

\endinput
