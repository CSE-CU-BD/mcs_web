\documentclass[problem]{mcs}

\begin{pcomments}
  \pcomment{FP_doublefib}
  \pcomment{ARM 5/18/13}
\end{pcomments}

\pkeywords{
  induction
  fibonacci
  recurrence
  gcd
  relatively_prime
}

%%%%%%%%%%%%%%%%%%%%%%%%%%%%%%%%%%%%%%%%%%%%%%%%%%%%%%%%%%%%%%%%%%%%%
% Problem starts here
%%%%%%%%%%%%%%%%%%%%%%%%%%%%%%%%%%%%%%%%%%%%%%%%%%%%%%%%%%%%%%%%%%%%%

\begin{staffnotes}
(a) 4 pts, (b) 5, (c) 3
\end{staffnotes}


\begin{problem}
Define the \emph{Double Fibonacci} numbers $D_0,D_1,\dots$ recursively
by the rules
\begin{align}
D_0 & = D_1 \eqdef 1,\notag\\
D_n & \eqdef 2D_{n-1} + D_{n-2} & \text{(for $n \geq 2$)}.\label{dn2dn1}
\end{align}

\bparts

\ppart\label{oddDF} Prove that all Double Fibonacci numbers are odd.

\examspace[2.5in]

\begin{solution}
\begin{proof}
We use strong induction on $n$ with induction hypothesis
\[
P(n) \eqdef\ [D_n \text{ is odd}].
\]

\inductioncase{Base case:} $(n=0,1)$.  $P(0)$ and $P(1)$ are true
since $D_0=D_1=1$ are both odd.

\inductioncase{Inductive step:} For $n \geq 2$, $D_n$ is defined
by~\eqref{dn2dn1}.  By strong induction, we may assume that $D_{n-2}$
is odd, and hence $D_n$ is the sum of an even number and an odd
number, and therefore is odd.

\end{proof}
\end{solution}

\ppart Prove that every two consecutive Double Fibonacci numbers are
relatively prime.

\examspace[3.5in]

\begin{solution}

\begin{proof}
By induction on $n$ with induction hypothesis
\[
Q(n) \eqdef\  [\gcd(D_{n},D_{n-1}) = 1].
\]

\inductioncase{Base case:} $(n= 1)$.  $P(1)$ holds since 1 is
relatively prime to 1.

\inductioncase{Inductive step:} To prove that $D_{n+1}$ and $D_n$ are
relatively prime, we use that fact that for $n \geq 1$,
\begin{align*}
\gcd(D_{n+1}, D_{n})
& = \gcd(D_{n+1}, 2D_{n})  & \text{(since $D_{n+1}$ is odd by part~\eqref{oddDF})}\\
& = \gcd(2D_{n}, D_{n+1} - 2D_{n})  & (\gcd(a,b) = \gcd(b, a-b))\\
& = \gcd(2D_{n}, D_{n-1})  & \text{(def of $D_{n+1}$)}\\
& = \gcd(D_{n}, D_{n-1})   & \text{(since $D_{n-1}$ is odd by part~\eqref{oddDF})}\\
& = 1    & \text{(by induction hypothesis $Q(n)$)}.
\end{align*}
That is, $D_{n+1}$ and $D_{n}$ are relatively prime.

\end{proof}
\end{solution}

\examspace
\ppart Express the generating function $D(x)$ for the Double Fibonacci
as a quotient of polynomials.  (You do \emph{not} have to find a
formula for $[x^n]D(x)$.)

\begin{solution}
\[
D(x)(1-2x-x^2) = D_0 + D_1x -2D_0x = 1-x
\]
so
\[
D(x) = \frac{1-x}{1-2x-x^2}
\]
\end{solution}

\eparts

\end{problem}


%%%%%%%%%%%%%%%%%%%%%%%%%%%%%%%%%%%%%%%%%%%%%%%%%%%%%%%%%%%%%%%%%%%%%
% Problem ends here
%%%%%%%%%%%%%%%%%%%%%%%%%%%%%%%%%%%%%%%%%%%%%%%%%%%%%%%%%%%%%%%%%%%%%

\endinput
