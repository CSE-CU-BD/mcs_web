\documentclass[problem]{mcs}

\begin{pcomments}
  \pcomment{FP_degree_sequences}
  \pcomment{CH Spring '14; edited ARM 4/1/14 9:50PM}
\end{pcomments}

\pkeywords{
   degree
   simple_graph
   handshaking
}


%%%%%%%%%%%%%%%%%%%%%%%%%%%%%%%%%%%%%%%%%%%%%%%%%%%%%%%%%%%%%%%%%%%%%
% Problem starts here
%%%%%%%%%%%%%%%%%%%%%%%%%%%%%%%%%%%%%%%%%%%%%%%%%%%%%%%%%%%%%%%%%%%%%

\begin{problem}

The \emph{degree sequence} of a simple graph $G$ with $n$ vertices is
the length-$n$ sequence of the degrees of the vertices listed in
weakly increasing order.  For example, if $G$ is a 4-vertex tree, then
its degree sequence is either $\ang{1,1,1,3}$ or $\ang{1,1,2,2}$.

Briefly explain why each of the following sequences is \textbf{not} a
degree sequence of any \emph{connected} simple graph.

\bparts

\ppart $\ang{1, 2, 3, 4, 5, 6, 7}$

\begin{solution}
There are only 7 vertices, so the degree of any vertex is at most 6.
\end{solution}

\examspace[1in]

\ppart $\ang{0, 2, 2, 2, 2}$
\begin{solution}
There is a vertex with degree 0, and there is more than one vertex, so
the graph is not connected.
\end{solution}

\examspace[1in]

\ppart $\ang{1, 3, 3, 4, 4, 4}$
\begin{solution}
By the Handshaking Lemma, the sum of degrees in any simple graph must
be even, which is not true in this case since $1 + 3 + 3+ 4 + 4 + 4 =
19$.
\end{solution}

\examspace[1in]

\ppart A sequence of $n$ integers whose sum
is less than $2n - 2$.

\begin{solution}
There are too few edges.  The sum of the degrees is twice the number
of edges by the Handshaking Lemma~\bref{sumedges}.  However, the
number of edges in any $n$-vertex connected graph is at least $n-1$
(Corollary~\bref{cor:n-1}).  Therefore, the sum of the degrees must be
at least $2n-2$.
\end{solution}

\examspace[1in]

\ppart $\ang{1, 2, 3, 4, 4}$

\begin{solution}
There are five vertices, two of which have degree 4.  So both degree-4
vertices have to be connected to \emph{all} the other vertices.  That
means that the degree of every vertex is greater than one, which is
violated in this case.

\end{solution}

\examspace[0.8in]

\eparts

\end{problem}

%%%%%%%%%%%%%%%%%%%%%%%%%%%%%%%%%%%%%%%%%%%%%%%%%%%%%%%%%%%%%%%%%%%%%
% Problem ends here
%%%%%%%%%%%%%%%%%%%%%%%%%%%%%%%%%%%%%%%%%%%%%%%%%%%%%%%%%%%%%%%%%%%%%

\endinput
