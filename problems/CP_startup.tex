\documentclass[problem]{mcs}

\begin{pcomments}
  \pcomment{CP_startup}
  \pcomment{from: S09.cp11r}
\end{pcomments}

\pkeywords{
  binomial
  combinatorial_proof
  algebraic_proof
  bijection
}

%%%%%%%%%%%%%%%%%%%%%%%%%%%%%%%%%%%%%%%%%%%%%%%%%%%%%%%%%%%%%%%%%%%%%
% Problem starts here
%%%%%%%%%%%%%%%%%%%%%%%%%%%%%%%%%%%%%%%%%%%%%%%%%%%%%%%%%%%%%%%%%%%%%

\begin{problem}
You want to choose a team of $m$ people from a pool of $n$ people for
your startup company, and from these $m$ people you want to choose $k$
to be the team managers.  You took 6.042, so you know you can do this
in
\[
\binom{n}{m}\binom{m}{k}
\]
ways.  But your CFO, who went to Harvard Business School, comes up with
the formula
\[
\binom{n}{k}\binom{n-k}{m-k}.
\]
Before doing the reasonable thing ---dump on your CFO or Harvard
Business School ---you decide to check his answer against yours.

\bparts
\ppart
Start by giving an \emph{algebraic proof} that your CFO's formula
agrees with yours. 

\begin{solution}
\begin{eqnarray*}
\binom{n}{m}\binom{m}{k}
& = & \frac{n!}{m! (n-m)!} \frac{m!}{k! (m-k)!} \\
& = & \frac{n!}{(n-m)! k! (m-k)!} \\
& = & \frac{n! (n-k)!}{(n-m)! k! (m-k)! (n-k)!} \\
& = & \frac{n!}{k! (n-k)!} \frac{(n-k)!}{(n-m)!(m-k)!} \\
& = & \frac{n!}{k! (n-k)!} \frac{(n-k)!}{((n-k) - (m-k))! (m-k)!} \\
& = & \binom{n}{k}\binom{n-k}{m - k}.
\end{eqnarray*} 
\end{solution}

\ppart Now give a \emph{combinatorial argument} proving this same fact.

\begin{solution}
Instead of choosing first $m$ from $n$ and
then $k$ from $m$, you could alternately choose the $k$ managers from
the $n$ people and then choose $m-k$ people to fill out the team from
the remaining $n-k$ people.  This gives you $\dbinom{n}{k}
\dbinom{n-k}{m-k}$ ways of picking your team.  Since you must have the
same number of options regardless of the order in which you choose to
pick team members and managers,
\[
\binom{n}{m} \binom{m}{k} = \binom{n}{k} \binom{n-k}{m-k}.
\]

Formally, in the first method we count the number of pairs $(A,B)$, where
$A$ is a size $m$ subset of the pool of $n$ people, and $B$ is size $k$
subset of $A$.  By the Generalized Product Rule, there are
\[
\binom{n}{m}\cdot \binom{m}{k}
\]
such pairs.

In the second method, we count pairs $(C,D)$, where $C$ is a size $k$
subset of the pool and $D$ is a size $(m-k)$ subset of the pool that is
disjoint from $C$.  By the Generalized Product Rule, there are
\[
\binom{n}{k}\cdot \binom{n-k}{m-k}
\]
such pairs.

These two expressions are equal because there is an obvious bijection
between the two kinds of pairs, namely map $(A,B)$ to $(A-B,B)$.
\end{solution}

\eparts
\end{problem}


%%%%%%%%%%%%%%%%%%%%%%%%%%%%%%%%%%%%%%%%%%%%%%%%%%%%%%%%%%%%%%%%%%%%%
% Problem ends here
%%%%%%%%%%%%%%%%%%%%%%%%%%%%%%%%%%%%%%%%%%%%%%%%%%%%%%%%%%%%%%%%%%%%%
\endinput
