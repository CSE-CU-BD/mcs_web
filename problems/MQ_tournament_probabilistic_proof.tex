\documentclass[problem]{mcs}

\begin{pcomments}
\pcomment{MQ_tournament_probabilistic_proof}
\pcomment{renamed from CP_}
\pcomment{F95.ps11}
\pcomment{edited by ARM 5/9/12}
\end{pcomments}

\pkeywords{
 probability
 expectation
 probabilistic_proof
 probabilistic_method
 tournament
}


%%%%%%%%%%%%%%%%%%%%%%%%%%%%%%%%%%%%%%%%%%%%%%%%%%%%%%%%%%%%%%%%%%%%%
% Problem starts here
%%%%%%%%%%%%%%%%%%%%%%%%%%%%%%%%%%%%%%%%%%%%%%%%%%%%%%%%%%%%%%%%%%%%%

\begin{problem}
A record of who beat whom in a round-robin tournament can be described
with a \term{tournament digraph}, where the vertices correspond to
players and there is an edge $\diredge{x}{y}$ iff $x$ beat $y$ in
their game.  A \term*{ranking} of the players is a path that includes
all the players.  A tournament digraph may in general have one or more
rankings.\footnote{It has a unique ranking iff it is a DAG, see
  Problem~\bref{CP_tournament_chain}.}

Suppose we construct a random tournament digraph by letting each of the
players in a match be equally likely to win and having results of all
the matches be mutually independent.  Find a formula for the expected
number of rankings in a random 10-player tournament.  Conclude that
there is a 10-vertex tournament digraph with more than $7000$ rankings.

This problem is an instance of the \emph{probabilistic method}.  It
uses probability to prove the existence of an object without
constructing it.

\begin{solution}
For any sequence $s$ of the 10 players, let the indicator random
variable $X_s$ be $1$ if $s$ is a ranking and $0$ otherwise.  The
probability that $s$ is a ranking is $2^{-9}$.  So, the expected
number of rankings is
\[
\expect{\sum_s X_s} = \sum_s \expect{X_s} = 10! \cdot 2^{-9} = 7087.5.
\]
Since there has to be a tournament with at least as many rankings as
the expected number, it follows that there must be a tournament with
at least 7088 rankings.
\end{solution}
\end{problem}

%%%%%%%%%%%%%%%%%%%%%%%%%%%%%%%%%%%%%%%%%%%%%%%%%%%%%%%%%%%%%%%%%%%%%
% Problem ends here
%%%%%%%%%%%%%%%%%%%%%%%%%%%%%%%%%%%%%%%%%%%%%%%%%%%%%%%%%%%%%%%%%%%%%

\endinput
