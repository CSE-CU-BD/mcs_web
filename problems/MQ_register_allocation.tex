\documentclass[problem]{mcs}

\begin{pcomments}
  \pcomment{similar to CP_register_allocation.tex}
\end{pcomments}

\pkeywords{
  graph_coloring
}

%%%%%%%%%%%%%%%%%%%%%%%%%%%%%%%%%%%%%%%%%%%%%%%%%%%%%%%%%%%%%%%%%%%%%
% Problem starts here
%%%%%%%%%%%%%%%%%%%%%%%%%%%%%%%%%%%%%%%%%%%%%%%%%%%%%%%%%%%%%%%%%%%%%

\begin{problem}
A portion of a computer program consists of a sequence of calculations
where the results are stored in variables, like this:
\[
\begin{array}{rrrcl}
&& \text{Inputs:} &  & u, v \\
\text{Step } 1. & \hspace{0.5in} & w & = & u + v \\
2. && x & = & u - v \\
3. && y & = & w + x \\
4. && z & = & w - x \\
&& \text{Outputs:} & & y, z
\end{array}
\]

A computer can perform such calculations most quickly if the value of
each variable is stored in a \emph{register}, a chunk of very fast
memory inside the microprocessor.  Programming language compilers face
the problem of assigning each variable in a program to a register.
Computers usually have few registers, however, so they must be used
wisely and reused often.  This is called the \term{register
  allocation} problem.
\iftrue (Assume that the computer carries out each step in the order
listed and that each step is completed before the next is begun.)  \fi

\bparts

\ppart Recast the register allocation problem as a question
about graph coloring.  What do the vertices correspond to?  Under what
conditions should there be an edge between two vertices?  Construct
the graph corresponding to the example above.

\ppart Color your graph using as few colors as you can.  Call
the computer's registers $R1$, $R2$, etc.  Describe the assignment of
variables to registers implied by your coloring.  How many registers
do you need?

\eparts
\end{problem}

%%%%%%%%%%%%%%%%%%%%%%%%%%%%%%%%%%%%%%%%%%%%%%%%%%%%%%%%%%%%%%%%%%%%%
% Problem ends here
%%%%%%%%%%%%%%%%%%%%%%%%%%%%%%%%%%%%%%%%%%%%%%%%%%%%%%%%%%%%%%%%%%%%%
