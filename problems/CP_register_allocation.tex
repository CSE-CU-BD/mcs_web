\documentclass[problem]{mcs}

\begin{pcomments}
  \pcomment{CP_register_allocation}
  \pcomment{overlaps MQ_}
  \pcomment{from: f01.ps4-4 revised from Rosen 7.8.21, S07.cp6m; S09.cp6r}
\end{pcomments}

\pkeywords{
  simple_graph
  coloring
  scheduling
  conflict
}

%%%%%%%%%%%%%%%%%%%%%%%%%%%%%%%%%%%%%%%%%%%%%%%%%%%%%%%%%%%%%%%%%%%%%
% Problem starts here
%%%%%%%%%%%%%%%%%%%%%%%%%%%%%%%%%%%%%%%%%%%%%%%%%%%%%%%%%%%%%%%%%%%%%

\begin{problem}
A portion of a computer program consists of a sequence of calculations
where the results are stored in variables, like this:
\[
\begin{array}{rrrcl}
&& \text{Inputs:} &  & a, b \\
\text{Step } 1. & \hspace{0.5in} & c & = & a + b \\
2. && d & = & a * c \\
3. && e & = & c + 3 \\
4. && f & = & c - e \\
5. && g & = & a + f \\
6. && h & = & f + 1 \\
&& \text{Outputs:} & & d, g, h
\end{array}
\]
A computer can perform such calculations most quickly if the value of
each variable is stored in a \emph{register}, a chunk of very fast
memory inside the microprocessor.  Programming language compilers face
the problem of assigning each variable in a program to a register.
Computers usually have few registers, however, so they must be used
wisely and reused often.  This is called the \term{register
  allocation} problem.

In the example above, variables $a$ and $b$ must be assigned different
registers, because they hold distinct input values.  Furthermore, $c$
and $d$ must be assigned different registers; if they used the same
one, then the value of $c$ would be overwritten in the second step and
we'd get the wrong answer in the third step.  On the other hand,
variables $b$ and $d$ may use the same register; after the first step,
we no longer need $b$ and can overwrite the register that holds its
value.  Also, $f$ and $h$ may use the same register; once $f + 1$ is
evaluated in the last step, the register holding the value of $f$ can
be overwritten.

\iffalse (Assume that the computer carries out each step in the order
listed and that each step is completed before the next is begun.)  \fi


\bparts

\ppart Recast the register allocation problem as a question
about graph coloring.  What do the vertices correspond to?  Under what
conditions should there be an edge between two vertices?  Construct
the graph corresponding to the example above.

\begin{solution}
There is one vertex for each variable.  An edge between two vertices
indicates that the values of the variables must be stored in different
registers.  We can tell when two variables must be stored in different
registers as follows: classify each appearance of a variable in the
program as either an \emph{assignment} or a \emph{use}.  An appearance
is an \term{assignment} when the variable is on the left side of an
equation or on the ``Inputs'' line.  An appearance of a variable is a
\index{variable use}\emph{use} if the variable is on the right side of
an equation or on the ``Outputs'' line.  The \index{variable
  lifetime}\emph{lifetime} of a variable is the segment of code
extending from the initial assignment of the variable until the last
use.  There is an edge between two variables iff their lifetimes
overlap.\footnote{This specification of edges is for the case that
  each variable is assigned at most once (see
  part~\eqref{assignedmore}).

  We are also assuming that all variables are relevant to the Outputs,
  where a variable is \index{relevant variable}\emph{relevant} iff it
  is an Output or is used in an assignment to a relevant variable.
  This is a recursive---not a circular---definition of relevant
  variable!

  Likewise, we assume that all variables are \index{dependent
    variable}\emph{dependent} on the Inputs, where a variable is
  dependent on the Inputs iff it is an Input or appears in the left
  hand side of an assignment whose right-hand side contains a
  dependent variable.}

This rule generates the following graph:

\begin{figure}[h]
\graphic[width=5in]{register1}
\end{figure}
\end{solution}

\ppart Color your graph using as few colors as you can.  Call
the computer's registers $R1$, $R2$ etc.  Describe the assignment of
variables to registers implied by your coloring.  How many registers
do you need?

\begin{solution}
Four registers are needed.

One possible assignment of variables to registers is indicated in the
figure above.  In general, coloring a graph using the minimum number
of colors is quite difficult; no efficient procedure is known.
However, the register allocation problem always leads to an
\term{interval graph}, and optimal colorings for interval graphs are
always easy to find.  This makes it easy for compilers to allocate a
minimum number of registers.
\end{solution}

\ppart\label{assignedmore} Suppose that a variable is assigned a value more than
once, as in the code snippet below:
\begin{align*} 
  & \ldots \\
t & =  r + s \\
u & =  t * 3 \\
t & =  m - k \\
v & =  t + u \\  
& \ldots
\end{align*}

How might you cope with this complication?

\begin{solution}
Each time a variable is reassigned, we could regard it as a completely
new variable.  Then we would regard the example as equivalent to the
following:
\begin{align*}
  & \ldots  \\
t & =  r + s \\
u & =  t * 3 \\
t' & =  m - k \\
v & =  t' + u \\
  & \ldots 
\end{align*}

We can now proceed with graph construction and coloring as before.
\end{solution}

\eparts
\end{problem}

%%%%%%%%%%%%%%%%%%%%%%%%%%%%%%%%%%%%%%%%%%%%%%%%%%%%%%%%%%%%%%%%%%%%%
% Problem ends here
%%%%%%%%%%%%%%%%%%%%%%%%%%%%%%%%%%%%%%%%%%%%%%%%%%%%%%%%%%%%%%%%%%%%%
