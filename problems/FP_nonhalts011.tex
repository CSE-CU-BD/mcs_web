\documentclass[problem]{mcs}

\begin{pcomments}
  \pcomment{FP_nonhalts011}
  \pcomment{ARM 3/12/16}
\end{pcomments}

\pkeywords{
  halting_problem
  recognizable
  unrecognizable
}

\newcommand{\nohaltohoh}{\small{\text{NoHalt00-11}}}

%%%%%%%%%%%%%%%%%%%%%%%%%%%%%%%%%%%%%%%%%%%%%%%%%%%%%%%%%%%%%%%%%%%%%
% Problem starts here
%%%%%%%%%%%%%%%%%%%%%%%%%%%%%%%%%%%%%%%%%%%%%%%%%%%%%%%%%%%%%%%%%%%%%

\begin{problem}
If applying a one-argument procedure $P$ to string a $s$
results in a computation that eventually halts, we say that $P$
\emph{recognizes} $s$.  The \emph{language} recognized by $P$ is:
\[
\rcg{P} \eqdef \set{s \in \asciistr \suchthat P\ \text{recognizes}\ s}.
\]
Let
\[
\nohaltohoh \eqdef \set{s \in \asciistr \suchthat P_s\ \text{recognizes neither 00 nor 11}}.
\]
Prove that \nohaltohoh is not recognizable.

\begin{solution}
We show this by making a "reduction argument": that is, we show that
if there was one recognizer for \nohaltohoh\ then we could use such
recognizer to define another recognizer for \nohalt, which we know is
not recognizable.

Give a declaration $s \in \asciistr$ of one-argument procedure, modify
the declaration so it specifies a new procedure that does the
following: when applied to any argument, ignore the argument, and
simulate $P_s$ on $s$.  Let $s' \in \asciistr$ be this modified
declaration.

This means that $P_{s'}$ behaves the same no matter what it is applied
to.  So if $P_s$ recognizes $s$, then $P_{s'}$ recognizes all strings,
and in particular recognizes both 00 and 11.  And if $P_s$ does not
recognize $s$, then $P_{s'}$ does not recognize anything, and in
particular recognize neither 00 nor 11.  Therefore,
\[
s \in \nohalts
    \QIFF P_s\ \text{does not recognize}\ s
     \QIFF P_{s'} \text{recognizes neither}\ 00\ \text{nor}\ 11
     \QIFF s' \in \nohaltohoh.
\]
This means that if we had a recognizer for \nohaltohoh, we could use
it to define a recognizer for \nohalt, on input $s$, construct $s'$
and apply the \nohaltohoh\ recognizer to $s'$.

But we know that \nohalts\ is unrecognizable, so we conclude that
\nohaltohoh\ must also be unrecognizable.
\end{solution}

\end{problem}

\endinput
