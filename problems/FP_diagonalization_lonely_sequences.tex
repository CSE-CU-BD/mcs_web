\documentclass[problem]{mcs}

\begin{pcomments}
  \pcomment{FP_diagonalization_lonely_sequences}
  \pcomment{overlaps PS_off_diagonal_arguments,
    FP_uncountable_sparse1s, FP_diagonalization_lonely_subsets}
  \pcomment{zabel, 10/28/17 for f17 midterm3conflict}
\end{pcomments}

\pkeywords{
  countable
  union
  sequences
  uncountable
  diagonalization
}

%%%%%%%%%%%%%%%%%%%%%%%%%%%%%%%%%%%%%%%%%%%%%%%%%%%%%%%%%%%%%%%%%%%%%
% Problem starts here
%%%%%%%%%%%%%%%%%%%%%%%%%%%%%%%%%%%%%%%%%%%%%%%%%%%%%%%%%%%%%%%%%%%%%

\begin{problem}
  Let $\binw$ be the set of infinite binary sequences.  Call a sequence
  in $\binw$ \emph{lonely} if it never has two~1s in a row.  For
  example, the repeating sequence $$(0,1,0,1,0,1,0,1,0,\dots)$$ is
  lonely, but the sequence $$(0,0,1,1,0,1,0,1,0,0,0,1,\dots)$$ is not
  lonely because it has two $1$s next to each other.  Let $F$ be the set
  of lonely sequences.  Show that $F$ is uncountable.

  \hint Consider an ``off-diagonal'' argument as in
  Problem~\bref{PS_off_diagonal_arguments}.

\begin{solution}
 
  \begin{proof}
    We'll argue by diagonalization with ``slope -1/2'' as in
    Problem~\bref{PS_off_diagonal_arguments}.  For the sake of
    contradiction, assume $F$ is countable.  Since $F$ is certainly
    infinite, it must be in bijection with $\nngint$, so we can write
    $F = \set{s_0, s_1, s_2, \dots}$.  Define a new sequence
    $t\in\binw$ as follows:

    \begin{equation*}
      t \eqdef (\bar{s_0[0]}, 0, \bar{s_1[2]}, 0, \bar{s_2[4]}, 0, \dots).
    \end{equation*}

    So $t[2k]$ is defined to be the
    opposite of bit $s_k[2k]$, and $t[2k+1]$ is always $0$.  Because
    $1$s can only appear at even indices, $t$ is lonely, so $t\in F$.
    On the other hand, $t\ne s_k$ for $k\in\nngint$, because $t$
    and $s_k$ differ in their $2k$th digit.  Thus,
    $t\notin\set{s_0,s_1,s_2,\dots}$, that is, $t\notin F$.  This is a
    contradiction, so our assumption that $F$ is countable must be
    false, that is, $F$ is uncountable.
  \end{proof}

  A alternate proof based on a similar idea does not rely on
  diagonalization at all.

  \begin{proof}
    We need only show that $\binw\inj F$ because we already know that
    $\binw$ is uncountable.  Define the function $g: \binw\to F$ as
    follows: for a sequence $s\in\binw$, $g(s)$ is the sequence
    \[
    g(s) \eqdef (s[0], 0, s[1], 0, s[2], 0, \dots).
    \]
    The sequence $g(s)$ contains no adjacent $1$s, so $g(s)\in F$.  We
    can see that $g$ is total by definition.  Furthermore, $g$ is
    injective, because if $s,t\subset\binw$ are different sequences,
    then $s[k]\ne t[k]$ for some $k\in\nngint$, and then $g(s)$ and
    $g(t)$ will differ in the $2k$th digit.  So $g$ is a total
    injective relation, meaning $\binw \inj F$, as required.

  \end{proof}
\end{solution}

\end{problem}

%%%%%%%%%%%%%%%%%%%%%%%%%%%%%%%%%%%%%%%%%%%%%%%%%%%%%%%%%%%%%%%%%%%%%
% Problem ends here
%%%%%%%%%%%%%%%%%%%%%%%%%%%%%%%%%%%%%%%%%%%%%%%%%%%%%%%%%%%%%%%%%%%%%

\endinput
