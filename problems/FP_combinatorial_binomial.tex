\documentclass[problem]{mcs}

\begin{pcomments}
  \pcomment{FP_combinatorial_binomial}
  \pcomment{edited from: F07 Final P8}
\end{pcomments}

\pkeywords{
  combinatorics
  counting
}

%%%%%%%%%%%%%%%%%%%%%%%%%%%%%%%%%%%%%%%%%%%%%%%%%%%%%%%%%%%%%%%%%%%%%
% Problem starts here
%%%%%%%%%%%%%%%%%%%%%%%%%%%%%%%%%%%%%%%%%%%%%%%%%%%%%%%%%%%%%%%%%%%%%

\begin{problem}

\bparts\mbox{}

\ppart[2] \label{divide-S} Let $S$ be a set with $i$ elements.  How
many ways are there to divide $S$ into two subsets?


\begin{center}
\exambox{1.5in}{0.5in}{0.3in}
\end{center}
\examspace[0.5in]
%\examspace[1.0in]


\begin{solution}
  There are $2^i$ ways to choose the first subset, and 
  then the second subset must be the rest of $S$.
\end{solution}

\ppart[4]

Here is a combinatorial proof of an equation giving a closed form for a
certain summation $\sum_{i=0}^n$:
\begin{quote}
There are $n$ marbles, each of which is to be painted red, green,
blue, or yellow.  One way to assign colors is to choose red, green, blue, or yellow successively for each marble.

An alternative way to assign colors to the marbles is to
\begin{itemize}

\item choose a number, $i$, between 0 and $n$,

\item choose a set, $S$, of $i$ marbles,

\item divide $S$ into two subsets; paint the first subset red and the
  other subset green.

\item divide the set of all the marbles not in $S$ into two subsets;
  paint the first subset blue and the other subset yellow.

\end{itemize}
\end{quote}

What is the equation?

\examspace[1.5in]

\begin{solution}
  There are 4 choices of color for each successive marble, so there are
  $4^n$ ways to color the marbles.

  In the second coloring process, there are $\binom{n}{i}$ ways to choose a
  set, $S$, of size $i$; by part~\eqref{divide-S}, there are $2^i$ ways to
  select the subset of $S$ be painted red, leaving the rest of $S$ green;
  similarly, there are $2^{n-i}$ ways to select the subset of the marbles not
  in $S$ to be painted blue, leaving the rest yellow. So there are
  $\binom{n}{i} 2^i 2^{n-i}$ ways to paint the marbles with a total of $i$ of
  them painted red or green, and $n-i$ of them painted blue or yellow. But $i$
  can be any number from $0$ to $n$. This shows that the total number of ways
  to paint the marbles is given by the above sum. \end{solution}

\ppart[4]
Now use the binomial theorem to prove the same equation.

\examspace[2in]

\begin{solution}
\[
\sum_{i=0}^n \binom{n}{i} 2^i 2^{n-i}
= (2+2)^n = 4^n.
\]
\end{solution}

\eparts
\end{problem}

%%%%%%%%%%%%%%%%%%%%%%%%%%%%%%%%%%%%%%%%%%%%%%%%%%%%%%%%%%%%%%%%%%%%%
% Problem ends here
%%%%%%%%%%%%%%%%%%%%%%%%%%%%%%%%%%%%%%%%%%%%%%%%%%%%%%%%%%%%%%%%%%%%%

\endinput
