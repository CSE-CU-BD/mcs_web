\documentclass[problem]{mcs}

\begin{pcomments}
  \pcomment{FP_combinatorial_binomial}
  \pcomment{verbatim from: F07 Final P8 with minor syntax edits}
\end{pcomments}

\pkeywords{
  combinatorics
  counting
}

%%%%%%%%%%%%%%%%%%%%%%%%%%%%%%%%%%%%%%%%%%%%%%%%%%%%%%%%%%%%%%%%%%%%%
% Problem starts here
%%%%%%%%%%%%%%%%%%%%%%%%%%%%%%%%%%%%%%%%%%%%%%%%%%%%%%%%%%%%%%%%%%%%%

\begin{problem}

\bparts

\ppart \label{divide-S}
Let $S$ be a set with $i$ elements.  How many
ways are there to divide $S$ into two subsets?

\examspace{2in}

\begin{solution}
  There are $2^i$ ways to choose the first subset, and 
  then the second subset must be the rest of $S$.
\end{solution}

\ppart

Here is a combinatorial proof of an equation giving a closed form for a
certain summation $\sum_{i=0}^n$:
\begin{quote}
There are $n$ fire hydrants, each of which is to be painted red, green or
blue.  One way to assign colors is to choose red, green or blue
successively for each hydrant.

An alternative way to assign colors to the hydrants is to
\begin{itemize}

\item choose a number, $i$, between 0 and $n$,

\item choose a set, $S$, of $i$ hydrants,

\item divide $S$ into two subsets; paint the first subset red and the other
  subset green.

\item paint all the hydrants not in $S$ blue.

\end{itemize}
\end{quote}

What is the equation?

\examspace{3in}

\begin{solution}
  There are 3 choices of color for each successive hydrant, so there are
  $3^n$ ways to color the hydrants.

  In the second coloring process, there are $\binom{n}{i}$ ways to choose a
  set, $S$, of size $i$; by part~\eqref{divide-S}, there are $2^i$ ways to
  select the subset of $S$ be painted red, leaving the rest of $S$ green, and
  the hydrants not in $S$ blue.  So there are $\binom{n}{i}2^i$ ways to paint
  the hydrants with a total of $i$ of them painted red or green.  But $i$ can
  be any number from $0$ to $n$.  This shows that the total number of ways to
  paint the hydrants is given by the above sum.
\end{solution}

\ppart
Now use the binomial theorem to prove the same equation.

\examspace{2in}

\begin{solution}
\[
\sum_{i=0}^n \binom{n}{i} 2^i = \sum_{i=0}^n \binom{n}{i} 2^i 1^{n-i}
   = (2+1)^n = 3^n.
\]
\end{solution}

\eparts
\end{problem}

%%%%%%%%%%%%%%%%%%%%%%%%%%%%%%%%%%%%%%%%%%%%%%%%%%%%%%%%%%%%%%%%%%%%%
% Problem ends here
%%%%%%%%%%%%%%%%%%%%%%%%%%%%%%%%%%%%%%%%%%%%%%%%%%%%%%%%%%%%%%%%%%%%%

\endinput
