\documentclass[problem]{mcs}

\begin{pcomments}
  \pcomment{TP_majorizing}
  \pcomment{ARM 10.12.17}
\end{pcomments}

\pkeywords{
  diagonal_argument
  diagonal
  countable
  function
}

%%%%%%%%%%%%%%%%%%%%%%%%%%%%%%%%%%%%%%%%%%%%%%%%%%%%%%%%%%%%%%%%%%%%%
% Problem starts here
%%%%%%%%%%%%%%%%%%%%%%%%%%%%%%%%%%%%%%%%%%%%%%%%%%%%%%%%%%%%%%%%%%%%%

\begin{problem}
Let
\[
f_0, f_1, f_2, \dots, f_k,\dots
\]
be an infinite sequence of positive real-valued total functions
$f_k:\nngint \to \reals^{+}$.

\bparts

\ppart\label{majorize} A function from $\nngint$ to $\reals$ that
eventually gets bigger that every one of the $f_k$ is said to
\emph{majorize} the set of $f_k$'s.  So finding a majorizing function
is a different way than diagonalization to find a function that is not
equal to any of the $f_k$'s.

Given an explicit formula for a majorizing function $g: \nngint \to
\reals$ for the $f_k$'s, and indicate how big $n$ should be to ensure
that $g(n) > f_k(n)$.

\begin{solution}
One of the simplest definitions is
\[
g(n) \eqdef 1+ \max\set{f_k(n) \suchthat 0 \leq k \leq n}.
\]
So for every $k$,
\[
g(n) > f_k(n)
\]
for all $n \geq k$.
\end{solution}

\ppart Modify your answer to part~\eqref{majorize}, if necessary, to
define a majorizing function $h$ that \emph{grows} more rapidly than
$f_k$ for every $k \in \nngint$, namely,
\[
\lim_{n \to \infty}\frac{f_k(n)}{h(n)} = 0.
\]
Explain why.

\begin{solution}
\[
h(n) \eqdef n g(n).
\]

So for all $n \geq k \geq 0$,
\[
\frac{f_k(n)}{h(n)} = \frac{1}{n} \cdot
\frac{f_k(n)}{g(n)} \leq \frac{1}{n} \cdot 1 = \frac{1}{n}.
\]
So
\[
\lim_{n \to \infty} \frac{f_k(n)}{h(n)} < \frac{1}{n} = 0.
\]
for each $k \in \nngint$.
\end{solution}

\iffalse
\ppart Show that the range of any majorizing function $g$ must be
\emph{sparser} than the range of any weakly increasing $f_k$, namely,
\[
\lim_{n \to \infty} \frac{\card{\range{g} \intersect
    \Zintv{0}{n}}}{\card{\range{f_k} \intersect \Zintv{0}{n}}} < 1.


\begin{solution}

\end{solution}
\fi

\eparts

\end{problem}
\endinput
