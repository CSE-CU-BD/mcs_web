\documentclass[problem]{mcs}

\begin{pcomments}
  \pcomment{TP_handshake_average}
  \pcomment{renamed from MQ_}
  \pcomment{ARM 10/21/13}
\end{pcomments}

\pkeywords{
  handshaking
  degree
  simple_graph
  average
}

%%%%%%%%%%%%%%%%%%%%%%%%%%%%%%%%%%%%%%%%%%%%%%%%%%%%%%%%%%%%%%%%%%%%%
% Problem starts here
%%%%%%%%%%%%%%%%%%%%%%%%%%%%%%%%%%%%%%%%%%%%%%%%%%%%%%%%%%%%%%%%%%%%%

\begin{problem}
The average degree of the vertices in an $n$-vertex graph is twice the
average number of edges of per vertex.  Explain why.

\examspace[3in]

\begin{solution}
Let $a_E$ be the average number of edges per vertex.  So by definition
\[
a_E \eqdef \frac{\card{E}}{n}.
\]
The average degree, $d_V$, of a vertex is by definition the sum of the
vertex degrees divided by $n$.  That is,
\begin{align*}
d_V & \eqdef \frac{\sum_{v \in V} \degr{v}}{n}\\
   & = \frac{2\card{E}}{n} & \text{(Handshake Lemma\inbook{~\bref{sumedges}})}\\
   & = 2\frac{\card{E}}{n} = 2 a_E.
\end{align*}
\end{solution}
\end{problem}

%%%%%%%%%%%%%%%%%%%%%%%%%%%%%%%%%%%%%%%%%%%%%%%%%%%%%%%%%%%%%%%%%%%%%
% Problem ends here
%%%%%%%%%%%%%%%%%%%%%%%%%%%%%%%%%%%%%%%%%%%%%%%%%%%%%%%%%%%%%%%%%%%%%

\endinput
