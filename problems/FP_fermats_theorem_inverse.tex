\documentclass[problem]{mcs}

\begin{pcomments}
  \pcomment{Taken from PS_calculating_inverses_fermat}
  \pcomment{this is just part b of PS_calculating_inverses}
  \pcomment{from: S09.ps7}
  \pcomment{from: F08, revised by ARM 10/29/09}
\end{pcomments}

\pkeywords{
  number_theory
  modular_arithmetic
  inverses
  Fermat_theorem
  remainder
}

%%%%%%%%%%%%%%%%%%%%%%%%%%%%%%%%%%%%%%%%%%%%%%%%%%%%%%%%%%%%%%%%%%%%%
% Problem starts here
%%%%%%%%%%%%%%%%%%%%%%%%%%%%%%%%%%%%%%%%%%%%%%%%%%%%%%%%%%%%%%%%%%%%%

\begin{problem}
Use Fermat's theorem to find the inverse $i$ of 13 modulo 23 
with $1 \le i < 23$.

\begin{solution}
Since 23 is prime, Fermat's theorem implies
$13^{23-2} \cdot 13 \equiv 1 \pmod{23}$ and so $\rem{13^{23-2}}{23}$ is
the inverse of 13 in the range $\set{1,\dots,22}$.  Now using the method
of repeated squaring,
%MAYBE REVISE TO USE FAST EXPONENTIATION
we have the following congruences modulo 23:
\[
\begin{array}{lcl}
13^{2}  & =      & 169\\
	& \equiv & \rem{169}{23} = 8\\
&&\\
13^{4}  & \equiv & 8^2\\
	& =      & 64\\
	& \equiv & \rem{64}{23} = 18\\
&&\\
13^{8}  & \equiv & 18^2\\
	& =      & 324\\
	& \equiv & \rem{324}{23} = 2\\
&&\\
13^{16} & \equiv & 2^2\\
	& =      & 4\\
&&\\
13^{21} & =      & 13^{16} \cdot 13^{4} \cdot 13\\
	& \equiv & 4 \cdot 18 \cdot 13\\
	& =      & (4 \cdot 6) \cdot (3 \cdot 13)\\
	& =      & 24 \cdot 39\\
	& \equiv & 1 \cdot 39\\
	& \equiv & \rem{39}{23} = \fbox{$16$}.
\end{array}
\]

\end{solution}

\end{problem}

%%%%%%%%%%%%%%%%%%%%%%%%%%%%%%%%%%%%%%%%%%%%%%%%%%%%%%%%%%%%%%%%%%%%%
% Problem ends here
%%%%%%%%%%%%%%%%%%%%%%%%%%%%%%%%%%%%%%%%%%%%%%%%%%%%%%%%%%%%%%%%%%%%%

\endinput
