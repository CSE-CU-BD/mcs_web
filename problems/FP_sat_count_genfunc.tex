\documentclass[problem]{mcs}

\begin{pcomments}
  \pcomment{FP_sat_count_induction_genfunc}
  \pcomment{incompatible overlap with FP_sat_count_induction}
  \pcomment{edited ARM 12/10/15}
\end{pcomments}

\pkeywords{
  induction
  satisfiability
  predicate
  proposition
}

%%%%%%%%%%%%%%%%%%%%%%%%%%%%%%%%%%%%%%%%%%%%%%%%%%%%%%%%%%%%%%%%%%%%%
% Problem starts here
%%%%%%%%%%%%%%%%%%%%%%%%%%%%%%%%%%%%%%%%%%%%%%%%%%%%%%%%%%%%%%%%%%%%%

\begin{problem}
Consider the following sequence of predicates:
%
\[\begin{array}{ll}
Q_1(x_1)& \eqdef\quad   x_1 \\
Q_2(x_1, x_2)& \eqdef\quad   x_1 \QIMP x_2 \\
Q_3(x_1, x_2, x_3)& \eqdef\quad   (x_1 \QIMP x_2) \QIMP x_3 \\
Q_4(x_1, x_2, x_3, x_4)& \eqdef\quad   ((x_1 \QIMP x_2) \QIMP x_3) \QIMP x_4 \\
Q_5(x_1, x_2, x_3, x_4, x_5)& \eqdef\quad   (((x_1 \QIMP x_2) \QIMP x_3) \QIMP x_4) \QIMP x_5 \\
\qquad \vdots & \qquad \qquad \vdots
\end{array}\]
%
Let $T_n$ be the number of different true/false settings of the
variables $x_1, x_2, \dots, x_n$ for which $Q_n(x_1, x_2, \dots,
x_n)$ is true.  For example, $T_2 = 3$ since $Q_2(x_1, x_2)$ is true
for 3 different settings of the variables $x_1$ and $x_2$:
%
\[
\begin{array}{cc|c}
x_1 & x_2 & Q_2(x_1, x_2) \\ \hline
T & T & T \\
T & F & F \\
F & T & T \\
F & F & T
\end{array}
\]
We let $T_0 =1$ by convention.

\bparts

\ppart Express $T_{n+1}$ in terms of $T_n$ and $n$, assuming $n \geq 0$.

\examspace[0.7in]

\begin{solution}
We have:
%
\[
Q_{n+1}(x_1, x_2, \dots, x_{n+1})
    = Q_n(x_1, x_2, \dots, x_n) \QIMP x_{n+1}
\]
%
If $x_{n+1}$ is true, then $Q_{n+1}$ is true for all $2^n$ settings of
the variables $x_1, x_2, \dots, x_n$.  If $x_{n+1}$ is false, then
$Q_{n+1}$ is true for all $2^n$ settings of $x_1, x_2, \dots, x_n$
\textit{except} for the $T_n$ settings that make $Q_n$ true.  Thus,
altogether we have:
\begin{equation}\label{Tn+12n2n}
T_{n+1} = 2^n + (2^n - T_n) = 2^{n+1} - T_n
\end{equation}

\end{solution}

 \ppart Prove that 
\[
T_n = \frac{2^{n+1} + (-1)^n}{3}
\]
for $n \geq 1$.

\begin{solution}
\begin{proof}
Using a generating function approach, we have
\begin{align}
T(x)       & \eqdef  T_0 + &T_1 x + &T_2 x^2 +  &T_3 x^3 + \cdots\\
xT(x)      & =             &T_0x  + &T_1x^2  +  &T_2x^3 + \cdots\\
-1/(1-2x)  & =        -1 - &2x    - &(2x)^2  - &(2x)x^3 + \cdots\\
\hline\\
          & =         0  + &0    + &0 +       &0       +  \cdots
\end{align}

So
\[
T(x) = \frac{1}{(1-2x)(1+x)}.
\]
Expanding into partial fractions, we have
\[
T(x) = \frac{2/3}{1-2x} + \frac{3}{1+x}.
\]
That means
\[
[x^n]T(x) = (2/3) (2^n) + 3 (-1)^n = \frac{2^{n+1} + (-1)^n}{3},
\]
as claimed.

An alternative is proof by induction.  Let $P(n)$ be the proposition
that
\[
T_n = \frac{2^{n+1} + (-1)^n}{3}.
\]

\inductioncase{Base case:} $(n=1)$.  There is a single setting of
$x_1$ that makes $Q_1(x_1) = x_1$ true, and $T_1 = (2^{1+1} + (-1)^1)
/ 3 = 1$.  Therefore, $P(1)$ is true.

\inductioncase{Inductive step:} For $n \geq 1$, we assume $P(n)$ and
reason as follows:
\begin{align*}
T_{n+1}
    & = 2^{n+1} - T_n
         & \text{(by~\eqref{Tn+12n2n})}\\
    & = 2^{n+1} - \frac{2^{n+1} + (-1)^n}{3}
         & \text{(by ind. hyp.)}\\
    & = \frac{3\cdot 2^{n+1} - 2^{n+1} + (-1)(-1)^{n}}{3}\\
    & = \frac{2^{n+2} + (-1)^{n+1}}{3}.
\end{align*}
That is, $P(n+1)$ holds.

\end{proof}
\end{solution}

\eparts

\end{problem}

%%%%%%%%%%%%%%%%%%%%%%%%%%%%%%%%%%%%%%%%%%%%%%%%%%%%%%%%%%%%%%%%%%%%%
% Problem ends here
%%%%%%%%%%%%%%%%%%%%%%%%%%%%%%%%%%%%%%%%%%%%%%%%%%%%%%%%%%%%%%%%%%%%%

\endinput
