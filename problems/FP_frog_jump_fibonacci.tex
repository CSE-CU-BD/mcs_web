\documentclass[problem]{mcs}

\begin{pcomments}
  \pcomment{FP_frog_jump_fibonacci}
  \pcomment{CH, Spring '14}
\end{pcomments}

\pkeywords{
	linear recurrences
        counting
        generating functions
        Fibonacci numbers
}

%%%%%%%%%%%%%%%%%%%%%%%%%%%%%%%%%%%%%%%%%%%%%%%%%%%%%%%%%%%%%%%%%%%%%
% Problem starts here
%%%%%%%%%%%%%%%%%%%%%%%%%%%%%%%%%%%%%%%%%%%%%%%%%%%%%%%%%%%%%%%%%%%%%

\begin{problem}

Kermit the Frog wants to climb up a staircase.  His movements are
limited to \emph{hopping} up one step or \emph{skipping} up two
steps.  We're interested in counting the number of ways that Kermit
can climb a staircase of any given size.  For example, if the
staircase has three stairs, then there are three ways that Kermit
could climb to the top: 3 hops; a hop and a skip; a skip and a hop.

Let $w_n$ be the number ways Kermit can climb a size $n$ staircase.
W have just observed that $w_3 = 3$.

\bparts 

\ppart What is the value of $w_4$?\hfill\examrule

\begin{solution}
5.

This could be done by enumerating the possible ways, but it's simpler
to anticipate part~\eqref{frog-recur} and observe that $w_4 = w_3 +
w_2 = 3 + 2$.
\end{solution}

\examspace[0.75in]

\ppart\label{frog-recur} Explain how to count the number of ways
Kermit can climb a size $n \geq 3$ staircase in terms of the number of
ways he can climb size $n-1$ and size $n-2$ staircases.  Derive from
this a simple linear recurrence for $w_n$ that holds for all $n \geq
3$.

%\hint Related to a famous sequence of numbers.

\begin{solution}
\[
w_n = w_{n-1} + w_{n-2} .
\]
The recurrence is the same as the one used to define the Fibonacci
sequence.  However the base cases here are different.

The argument is as follows.  Every way of climbing $n \geq 3$ stairs
must end with a hop---from the $n-1$st stair, or a skip---from the
$n-2$nd stair.  So there are $w_{n-1}$ ways to end with a hop and
$w_{n-2}$ ways to end with a skip.  The recurrence follows.

\end{solution}

\examspace[2.5in]

\ppart Let $W(x)$ be the generating function
\[
W(x) = w_3 x^3 + w_4 x^4 + \cdots + w_n x^n + \cdots .
\]
Derive a simple algebraic expression for $W(x)$.

\begin{solution}
As with the derivation for Fibonacci sequences, 
\[\begin{array}{rcrcrcrcrl}
     W(x) &=   &w_3 x^3  &+ &w_4  x^4 &+& w_5 x^5  &+ \cdots  + &  w_n x^n &+ \cdots \\
-x   W(x) &=   &        &- &w_3  x^4 &-& w_4 x^5  &- \cdots  - &w_{n-1} x^n &- \cdots \\
-x^2 W(x) &=   &        &  &         &-& w_3 x^5  &- \cdots  - &w_{n-2} x^n &- \cdots .
\end{array}\]
Adding, we get:
\begin{align*}
W(x) -x W(x) - x^2W(x) &= w_3 x^3 + (w_4 - w_3) x^4 + 0 x^5  + \cdots + 0 x^n +
\cdots \\
&= 3x^3 + 2x^4.
\end{align*}

Therefore,
\[
W(x) = \frac{3x^3 + 2x^4}{1 - x - x^2} .  
\]

\end{solution}

\eparts
\end{problem}

%%%%%%%%%%%%%%%%%%%%%%%%%%%%%%%%%%%%%%%%%%%%%%%%%%%%%%%%%%%%%%%%%%%%%
% Problem ends here
%%%%%%%%%%%%%%%%%%%%%%%%%%%%%%%%%%%%%%%%%%%%%%%%%%%%%%%%%%%%%%%%%%%%%

\endinput
