\documentclass[problem]{mcs}

\begin{pcomments}
  \pcomment{TP_another_bogus_fibonacci_induction}
  \pcomment{renamed from PS_bogus_fibonacci_induction}
  \pcomment{variation of TP_a_bogus_fibonacci_induction}
  \pcomment{Adapted from TP_bogus_well_ordering_fibonacci_proof by
    ARM 2/20/11}
  \pcomment{variation of TP_a_bogus_fibonacci_induction}
  \pcomment{revised/simplified by ARM 1/15/12}
\end{pcomments}

\pkeywords{
  bogus
  induction
  strong_induction
  false_theorem
  Fibonacci
  recursive
}

\begin{problem}
\inhandout{The Fibonacci numbers $F(0) F(1) F(2), \dots$ are defined as follows:
\[
F(n) \eqdef \begin{cases}
  0               & \mbox{if $n = 0$},\\
  1               & \mbox{if $n = 1$},\\
  F(n-1) + F(n-2) & \mbox{if $n >1$}.
\end{cases}
\]}
\inbook{The Fibonacci numbers $F(n)$ are described in Section~\bref{sec:fib}.}

Indicate exactly which sentence(s) in the following bogus proof contain
logical errors?  Explain.

\begin{falseclm*}
Every Fibonacci number is even.
\end{falseclm*}

\begin{bogusproof}
  Let all the variables $n,m,k$ mentioned below be nonnegative integer
  valued.  Let $\Even(n)$ mean that $F(n)$ is even.  The proof is by
  strong induction with induction hypothesis $\Even(n)$.

  \inductioncase{base case}: $F(0) = 0$ is an even number, so $\Even(0)$ is true.

  \inductioncase{inductive step}: We assume may assume the strong induction
  hypothesis
\[
\Even(k) \text{ for } 0 \le k \le n,
\]
and we must prove $\Even(n+1)$.

Then by strong induction hypothesis, $\Even(n)$ and $\Even(n-1)$ are
true, that is, $F(n)$ and $F(n-1)$ are both even.  But by the
definition, $F(n+1)$ equals the sum $F(n) + F(n-1)$ of two even
numbers, and so it is also even.  This proves $\Even(n+1)$ as
required.

Hence, $F(m)$ is even for all $m \in \nngint$ by the Strong Induction Principle.

\end{bogusproof}

\begin{solution}
  The error in proof is the assertion that in proving $\Even(n+1)$, it
  is ok to assume the induction hypothesis that $\Even(n)$ and
  $\Even(n-1)$ are both true.  This assumption is only allowed when
  both $n$ and $n-1$ are nonnegative.  In particular, for $n=0$, we
  can't assume $\Even(-1)$, and indeed, this doesn't even make sense
  since $F(-1)$ is not defined.  \iffalse Moreover, the equation
  $F(n+1) = F(n) + F(n-1)$ only holds for $n > 0$.\fi

  So the reasoning of the inductive step does not cover $\Even(n+1)$
  in the case that $n=0$.  That is, the proof does not show that
  $\Even(1)$ holds, and of course $\Even(1)$ is actually false since
  $F(1) \eqdef 1$.

In short, the proof in the inductive step that $\Even(n) \QIMPLIES
\Even(n+1)$ doesn't work for $n=0$, invalidating the whole proof.

\iffalse
Incidentally, saying that the inductive step made a logical error when
it assumed $n+1 \ge 2$ is on the right track, but not quite right.
After the base case $F(0)$, the orderly thing to have done would have
been to consider $F(1)$ next, and the proof didn't do that.  But
technically, starting with the $n+1 \ge 2$ case is an
\emph{organizational} flaw, or perhaps a \emph{strategic} error,
because it skips the $n+1 = 1$ case.  But this is not a \emph{logical}
error, because the proof was correct for the $n+1 \ge 2$ case and
could ultimately have been correct if it had gone on to cover the
$n+1=1$ case.  The logical error was in skipping the $n+1 = 1$ case
altogether.  This distinction between logical errors where there is a
mistake in reasoning, and strategic errors which will cause a later
difficulty, is a useful one to recognize.
\fi

\end{solution}

\end{problem}

\endinput
