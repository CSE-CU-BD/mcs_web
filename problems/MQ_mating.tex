\documentclass[problem]{mcs}

\begin{pcomments}
  \pcomment{MQ_mating}
\end{pcomments}

\pkeywords{
}

%%%%%%%%%%%%%%%%%%%%%%%%%%%%%%%%%%%%%%%%%%%%%%%%%%%%%%%%%%%%%%%%%%%%%
% Problem starts here
%%%%%%%%%%%%%%%%%%%%%%%%%%%%%%%%%%%%%%%%%%%%%%%%%%%%%%%%%%%%%%%%%%%%%

\begin{problem}

In the Mating Ritual, suppose Tiger is one of the boys and Elin is one of the girls.
Which of the following are preserved invariants in general?
\begin{enumerate}
\item Tiger is Elin's only suitor.
\item Tiger is Elin's favorite.
\item Elin is the only girl on Tiger's list.
\item Tiger's optimal wife is the most preferred girl on his list.
\item If Elin is not on Tiger's list, then Elin has a suitor that she prefers to Tiger.
\item Elin does not have any boys serenading her.
\item Elin has been crossed off Tiger's list, and Tiger prefers her to whomever he is serenading.
\end{enumerate}
\examspace[2in]
\begin{solution}
The statements that are preserved invariants in general appear in boldface below:
\begin{enumerate}
\item Tiger is Elin's only suitor.
\item Tiger is Elin's favorite.
\item \textbf{Elin is the only girl on Tiger's list.}
\item \textbf{Tiger's optimal wife is the most preferred girl on his list.}
\item \textbf{If Elin is not on Tiger's list, then Elin has a suitor that she prefers to Tiger.}
\item Elin does not have any boys serenading her.
\item \textbf{Elin has been crossed off Tiger's list, and Tiger prefers her to whomever he is serenading.}
\end{enumerate}
\end{solution}

\end{problem}

\endinput
