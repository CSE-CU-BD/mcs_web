\documentclass[problem]{mcs}

\begin{pcomments}
  \pcomment{CP_subset_take_away}
  \pcomment{from: S09.cp2r}
\end{pcomments}

\pkeywords{
  set_theory
  fun_game
  Gale
  subset_take_away
}

%%%%%%%%%%%%%%%%%%%%%%%%%%%%%%%%%%%%%%%%%%%%%%%%%%%%%%%%%%%%%%%%%%%%%
% Problem starts here
%%%%%%%%%%%%%%%%%%%%%%%%%%%%%%%%%%%%%%%%%%%%%%%%%%%%%%%%%%%%%%%%%%%%%

\begin{problem}
%
Subset take-away\footnote{From Christenson \& Tilford, \emph{David Gale's
Subset Takeaway Game, American Mathematical Monthly, Oct. 1997}} is a two
player game involving a fixed finite set, $A$.  Players alternately choose
nonempty subsets of $A$ with the conditions that a player may not choose
\begin{itemize}
\item the whole set $A$, or
\item any set containing a set that was named earlier.
\end{itemize}
The first player who is unable to move loses the game.

For example, if $A$ is $\set{1}$, then there are no legal moves and the
second player wins.  If $A$ is $\set{1,2}$, then the only legal moves are
$\set{1}$ and $\set{2}$.  Each is a good reply to the other, and so once
again the second player wins.

The first interesting case is when $A$ has three elements.  This time, if
the first player picks a subset with one element, the second player picks
the subset with the other two elements.  If the first player picks a
subset with two elements, the second player picks the subset whose sole
member is the third element.  \iffalse In short, in response to any first
move, the second player may choose the complementary set.\fi Both cases
produce positions equivalent to the starting position when $A$ has two
elements, and thus leads to a win for the second player.

Verify that when $A$ has four elements, the second player still has a
winning strategy.\footnote{David Gale worked out some of the properties of
this game and conjectured that the second player wins the game for
any set $A$.  This remains an open problem.}

\begin{solution}
There are way too many cases to work out by hand if we tried to
list all possible games.  But the elements of $A$ all behave the same, so
we can cut to a small number of cases using the fact that permuting around
the elements of $A$ in any game yields another possible game.  We can do
this by not mentioning specific elements of $A$, but instead using the
\emph{variables} $a,b,c,d$ whose values will be the four elements of $A$.

We consider two cases for the move of the Player 1 when the game starts:

\begin{enumerate}
\item Player 1 chooses a one element or a three element subset.  Then
Player 2 should choose the complement of Player one's choice.  The game
then becomes the same as playing the $n=3$ game on the three element set
chosen in this first round, where we know Player 2 has a winning
strategy.

\item Player 1 chooses a subset of 2 elements.  Let $a,b$ be these
elements, that is, the first move is $\set{a,b}$.  Player 2 should choose
the complement, $\set{c,d}$, of Player 1's choice.  We then have the
following subcases:
\begin{enumerate}

\item Player 1's second move is a one element subset, $\set{a}$.  Player 2
should choose $\set{b}$.  The game is then reduced to the two element game
on $\set{c,d}$ where Player 2 has a winning strategy.

\item
Player 1's second move is a two element subset, $\set{a,c}$.  Player 2
should choose its complement, $\set{b,d}$.  This leads to two subsubcases:
\begin{enumerate}

\item Player 1's third move is one of the remaining sets of size two,
$\set{a,d}$.  Player 2 should choose its complement, $\set{b,c}$.  The
remaining possible moves are the four sets of size 1, where the Player 2
clearly wins after two more rounds.

\item Player 1's third move is a one element set, $\set{a}$.  Player 2
should choose $\set{b}$.  The game is then reduced to the case two element
game on $\set{c,d}$ where Player 2 has a winning strategy.
\end{enumerate}
\end{enumerate}
\end{enumerate}
So in all cases, Player 2 has a winning strategy in the Gale game for
$n=4$.
\end{solution}

\end{problem}

%%%%%%%%%%%%%%%%%%%%%%%%%%%%%%%%%%%%%%%%%%%%%%%%%%%%%%%%%%%%%%%%%%%%%
% Problem ends here
%%%%%%%%%%%%%%%%%%%%%%%%%%%%%%%%%%%%%%%%%%%%%%%%%%%%%%%%%%%%%%%%%%%%%

\endinput
