\documentclass[problem]{mcs}

\begin{pcomments}
  \pcomment{CP_nth_derivative_of_A}
  \pcomment{from: S09.cp12m}
\end{pcomments}

\pkeywords{
  generating functions
  counting
}

%%%%%%%%%%%%%%%%%%%%%%%%%%%%%%%%%%%%%%%%%%%%%%%%%%%%%%%%%%%%%%%%%%%%%
% Problem starts here
%%%%%%%%%%%%%%%%%%%%%%%%%%%%%%%%%%%%%%%%%%%%%%%%%%%%%%%%%%%%%%%%%%%%%

%S06 cp11m 
%revised from S04 cp13w

\begin{problem}\label{tay} Let $A(x) =
\sum_{n=0}^\infty a_nx^n$.   Then it's easy to check that
\[
a_n = \frac{A^{(n)}(0)}{n!},
\]
where $A^{(n)}$ is the $n$th derivative of $A$.  Use this fact (which
you may assume) instead of the \idx{Convolution Counting Principle},
to prove that
\[
\frac{1}{\paren{1-x}^k} = \sum_{n=0}^\infty \binom{n+k-1}{k-1}x^n.
\]

\begin{solution}
\begin{align*}
\frac{d\paren{1-x}^{-k}}{dx} & = k\paren{1-x}^{-(k+1)}.\\
\frac{d^2\paren{1-x}^{-k}}{(dx)^2} & = \frac{d\, k\paren{1-x}^{-(k+1)}}{dx} = (k+1)k\paren{1-x}^{-(k+2)}\\
\frac{d^3\paren{1-x}^{-k}}{(dx)^3} & = \frac{d\,
  (k+1)k\paren{1-x}^{-(k+2)}}{dx} = (k+2)(k+1)k\paren{1-x}^{-(k+3)}\\
& \vdots\\
\frac{d^n \paren{1-x}^{-k}}{(dx)^n} & = (k+n-1)\cdots(k+2)(k+1)k\paren{1-x}^{-(k+n)}.
\end{align*}

Now suppose $(1-x)^{-k} = A(x)$.  Then we have
\begin{align*}
a_n & = \frac{A^{(n)}(0)}{n!}\\
    & = \frac{(k+n-1)\cdots(k+2)(k+1)k\paren{1-0}^{-(k+n)}}{n!}\\
    & = \frac{\dfrac{(n+k-1)!}{(k-1)!}\cdot 1}{n!}\\
    & = \frac{(n+k-1)!}{(k-1)!n!}\\
    & = \binom{n+k-1}{k-1}
\end{align*}
\end{solution}

So if we didn't already know the \idx{Bookkeeper Rule}, we could have proved
it from this calculation and the Convolution Rule for generating
functions.
\end{problem}

\endinput
