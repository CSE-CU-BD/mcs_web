\documentclass[problem]{mcs}

\begin{pcomments}
  \pcomment{FP_binary_tree_induction}
  \pcomment{revised from S07.ps3.prob2 by ARM 3/20/11}
  \pcomment{subsumes PS_labeled_binary_trees}
  \pcomment{revised 12/15/15 ARM}
\end{pcomments}

\pkeywords{
  structural_induction
  tree
  leaf
  binary_tree
}

\newcommand{\OBT}{\text{OBT}}
\newcommand{\bleaf}{\textbf{leaf}}
\newcommand{\bnode}{\textbf{node}}

%%%%%%%%%%%%%%%%%%%%%%%%%%%%%%%%%%%%%%%%%%%%%%%%%%%%%%%%%%%%%%%%%%%%%
% Problem starts here
%%%%%%%%%%%%%%%%%%%%%%%%%%%%%%%%%%%%%%%%%%%%%%%%%%%%%%%%%%%%%%%%%%%%%

\begin{problem}
  The set $\OBT$ of \emph{Ordered Binary Trees} is defined recursively
  as follows:

\inductioncase{Base case}: $\ang{\bleaf}$ is an $\OBT$, and

\inductioncase{Constructor case}: if $R$ and $S$ are $\OBT$'s, then
$\ang{\bnode,R,S}$ is an $\OBT$.

\bigskip
If $T$ is an $\OBT$, let $n_T$ be the number of \bnode\ labels in $T$
and $l_T$ be the number of \bleaf\ labels in $T$.

Prove by structural induction that for all $T \in \OBT$,
\instatements{\[}\insolutions{\begin{equation}\label{lTnT1}}
l_T = n_T+1.
\instatements{\]}\insolutions{\end{equation}}

\begin{solution}
The induction hypothesis will be~\eqref{lTnT1}.

\inductioncase{Base case} ($T = \ang{\bleaf}$): Here $l_T = 1 = 0+1 =
n_T+1$, so~\eqref{lTnT1} holds in this case.

\inductioncase{Constructor case} ($T \eqdef \ang{\bnode, R,S}$):
\begin{align*}
n_T + 1
  & = (n_R + n_S + 1) + 1
    & \text{(def of $n_T$)}\\
  & = (l_R - 1) + (l_S - 1) + 1 + 1
    & \text{(induction hyp for $R,S$)}\\
  & = l_R + l_S\\
  & = l_T.
    & \text{(def of $l_T$)}
\end{align*}

This proves~\eqref{lTnT1} holds for $T$, completing the proof of the
Constructor case.  It follows by structural induction
that~\eqref{lTnT1} holds for all $T \in \OBT$.
\end{solution}

\end{problem}


%%%%%%%%%%%%%%%%%%%%%%%%%%%%%%%%%%%%%%%%%%%%%%%%%%%%%%%%%%%%%%%%%%%%%
% Problem ends here
%%%%%%%%%%%%%%%%%%%%%%%%%%%%%%%%%%%%%%%%%%%%%%%%%%%%%%%%%%%%%%%%%%%%%

\endinput
