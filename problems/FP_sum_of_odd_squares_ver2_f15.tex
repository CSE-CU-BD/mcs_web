\documentclass[problem]{mcs}

\begin{pcomments}
  \pcomment{FP_sum_of_odd_squares}
  \pcomment{ARM 10/29/15}
  \pcomment{f15.mid3cnflict}
\end{pcomments}

\pkeywords{
  induction
  sum
  square
  closed_form
}

\begin{problem}
We are interesed in finding a closed form formula for the sum
\begin{equation}\label{sumoddsq}
\sum_{i=1}^n (2i-1)(i+1).
\end{equation}
Display such a closed form formula for~\eqref{sumoddsq}.
\textbf{Alternatively}, you can get \emph{full credit by giving a
  clear description} of a procedure for deriving such a closed formula
without fully executing your procedure to derive an explicit closed
formula.

\begin{solution}
As in Section~\bref{sec:sum_powers}, we can (correctly) guess that a
formula for~\eqref{sumoddsq} will be a cubic polynomial
\[
an^3 + bn^2 + cn + d.
\]
Then we can determine the parameters $a$, $b$, $c$ and $d$ by
plugging in a few values for $n$ until we get enough equations in
$a,b,c,d$ to solve for their values.  Applying this method to our
example gives:
\begin{align*}
n = 1 & \qimplies  2 = a + b + c + d \\
n = 2 & \qimplies  11 = 8a + 4b + 2c + d \\
n = 3 & \qimplies  31 = 27a + 9b + 3c + d \\
n = 4 & \qimplies  66 = 64a + 16b + 4c + d.
\end{align*}
Solving this system gives the solution $a = 2/3, b = 3/2, c = -1/6, d = 0$
So the closed form formula for~\eqref{sumoddsq} would be
\begin{equation}\label{4n312n2}
\frac{4n^3 + 9n^2 - n}{6}.
\end{equation}
Of course this remains a guess until we have verified it somehow, for
example, by induction.

Another approach is to rewrite $(2i-1)(i+1)$ as $2i^2+i -1$ and then use
the known formula for the sum of squares\inbook{~(\bref{eqn:G27})}:
\[
\sum_{i=0}^n i^2 = \frac{(2n+1) (n+1) n}{6}\, .
\]
So,
\begin{align*}
\sum_{i=1}^n (2i-1)(i+1)
   & = \sum_{i=1}^n (2i^2 + i - 1)\\
   & = 2 \sum_{i=1}^n i^2 + \sum_{i=1}^n i  - \sum_{i=1}^n 1\\
   & = 2 \frac{(2n+1) (n+1) n}{6} + \frac{n(n+1)}{2} - n\\
   & = \frac{2}{6}(2n^3+ 3n^2 +n) + \frac{1}{6}(3n^2 + 3n - 6n)\\
   & = \frac{1}{6}{(4n^3+ 9n^2 - n)},
\end{align*}
which agrees with~\eqref{4n312n2}.

A third approach involves using generating functions as described in
Chapter~\bref{generating_function_chap}, but let's not get ahead of
ourselves.

\end{solution}
\end{problem}

\endinput
