\documentclass[problem]{mcs}

\begin{pcomments}
  \pcomment{FP_graph_100_dilworth}
  \pcomment{CH, spring 2014}
  \pcomment{forked from TP_subsequence_of_101}
\end{pcomments}

\pkeywords{chain
           anti-chain
           increasing
           decreasing
           partial order
}

%%%%%%%%%%%%%%%%%%%%%%%%%%%%%%%%%%%%%%%%%%%%%%%%%%%%%%%%%%%%%%%%%%%%%
% Problem starts here
%%%%%%%%%%%%%%%%%%%%%%%%%%%%%%%%%%%%%%%%%%%%%%%%%%%%%%%%%%%%%%%%%%%%%

\begin{problem}

Let $G$ be a directed acyclic graph with $|V| = 100$ vertices. No other
\emph{a priori} information is available regarding its edges.

\bparts

\ppart What is the \emph{maximum} possible size of the largest chain in $G$?
Justify your reasoning.

\begin{solution}
The maximum-size chain has a size of 100, occurs when the vertices are arranged in a
linear order. 
\end{solution}

\ppart What is the \emph{minimum} possible size of the largest chain
in $G$? Justify your reasoning.

\begin{solution}
 The empty graph contains no edges and the vertices are fully
unordered. Therefore, the minimum-size chain has a length of 1.
\end{solution}

\ppart Argue that for any such $G$, either the maximum-size chain or
the maximum-size antichain has a size of at least 10.

\begin{solution}
Dilworth's Lemma states that any DAG with $n$ vertices must have either a chain of
size at least $t$, or an antichain of size at least
$n/t$. Substitute $n=100$ and $t=10$ to obtain the result.
\end{solution}

\ppart Give an example of $G$ where the maximum-size chain and the
maximum-size antichain \emph{both} have size 10. Justify your reasoning.

\begin{solution}

Partition the vertices $V$ into 10 blocks of 10 vertices
each. Linearly order vertices within each block, and leave different
blocks disconnected. 

Clearly, the length of the maximum-size chain is 10. We can construct
a set containing all vertices from any one block, forming a chain of
10 vertices. Any vertex from another block is incomparable with the
vertices we have chosen,

On the other hand, the length of the maximum-size
antichain is also 10. We can construct a set consisting of exactly one vertex from each of
the 10 blocks, forming an antichain of 10 vertices. Adding any other
vertex in this set would result in a pair of comparable vertices.

\end{solution}

\eparts

\end{problem}


%%%%%%%%%%%%%%%%%%%%%%%%%%%%%%%%%%%%%%%%%%%%%%%%%%%%%%%%%%%%%%%%%%%%%
% Problem ends here
%%%%%%%%%%%%%%%%%%%%%%%%%%%%%%%%%%%%%%%%%%%%%%%%%%%%%%%%%%%%%%%%%%%%%

\endinput
