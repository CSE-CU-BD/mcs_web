\documentclass[problem]{mcs}

\begin{pcomments}
  \pcomment{FP_graph_100_dilworth}
  \pcomment{CH, spring 2014}
  \pcomment{forked from TP_subsequence_of_101}
\end{pcomments}

\pkeywords{chain
           anti-chain
           increasing
           decreasing
           partial order
}

%%%%%%%%%%%%%%%%%%%%%%%%%%%%%%%%%%%%%%%%%%%%%%%%%%%%%%%%%%%%%%%%%%%%%
% Problem starts here
%%%%%%%%%%%%%%%%%%%%%%%%%%%%%%%%%%%%%%%%%%%%%%%%%%%%%%%%%%%%%%%%%%%%%

\begin{problem}

Let $G$ be an arbitrary directed acyclic graph with $100$ vertices. 
%No other\emph{a priori} information is available regarding its edges.

\bparts

\ppart What is the size of the largest chain that $G$ could possibly
have?  Justify your reasoning.

\begin{solution}
100.

Let $G$ be a linear order.  This gives a chain of size 100.
\end{solution}

\examspace[1in]

\ppart What is the size of the largest antichain that $G$ could
possibly have?  Justify your reasoning.

\begin{solution}
100.

Let $G$ be the empty graph with 100 vertices (and no edges).  This
gives an antichain of size 100.
\end{solution}

\examspace[1in]

\ppart Argue that for any such $G$, either there is a chain or there
is an antichain containing at least 10 vertices.

\begin{solution}
Dilworth's Lemma states that any DAG with $n$ vertices must have
either a chain of size at least $t$, or an antichain of size at least
$n/t$.  Substitute $n=100$ and $t=10$ to obtain the result.
\end{solution}

\examspace[1in]

\ppart Give an example of $G$ where the maximum-size chain and the
maximum-size antichain \emph{both} have size 10.   %Justify your reasoning.

\begin{solution}
Partition the vertices $V$ into 10 blocks of 10 vertices each.
Linearly order vertices within each block, and leave different blocks
disconnected.

The set of all vertices from any one block forms a chain of 10
vertices.  Any vertex from another block is incomparable with the
vertices we have chosen, and therefore no chain of bigger size is
possible.

On the other hand, the length of the maximum-size antichain is also
10.  We can construct a set consisting of exactly one vertex from each
of the 10 blocks, forming an antichain of 10 vertices.  Any set of 11
vertices would have to contain 2 vertices from the same chain, and so
would not be an antichain, so no antichain of size bigger than 10 is
possible.
\end{solution}

\eparts

\end{problem}


%%%%%%%%%%%%%%%%%%%%%%%%%%%%%%%%%%%%%%%%%%%%%%%%%%%%%%%%%%%%%%%%%%%%%
% Problem ends here
%%%%%%%%%%%%%%%%%%%%%%%%%%%%%%%%%%%%%%%%%%%%%%%%%%%%%%%%%%%%%%%%%%%%%

\endinput
