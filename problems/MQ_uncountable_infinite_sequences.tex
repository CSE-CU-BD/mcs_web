\documentclass[problem]{mcs}

\begin{pcomments}
  \pcomment{MQ_uncountable_infinite_sequences}
  \pcomment{from: S10 by wing, revised 3/19/10, 8/28/11 by ARM}
  \pcomment{strengthening of CP_N_to_N_diagonal_argument}
  \pcomment{special case of PS_A_to_B_diagonal_argument}
\end{pcomments}

\pkeywords{
  diagonal
  uncountable
  surjection
  power_set
}

%%%%%%%%%%%%%%%%%%%%%%%%%%%%%%%%%%%%%%%%%%%%%%%%%%%%%%%%%%%%%%%%%%%%%
% Problem starts here
%%%%%%%%%%%%%%%%%%%%%%%%%%%%%%%%%%%%%%%%%%%%%%%%%%%%%%%%%%%%%%%%%%%%%

\begin{problem}
  Let $\infstrings{\set{1,2,3}}$ be the set of infinite sequences
  containing only the numbers 1, 2, and 3.  For example, some sequences of
  this kind are:
\begin{align*}
(1, 1, 1, 1 ...),\\
(2, 2, 2, 2 ...),\\
(3, 2, 1, 3 ...).
\end{align*}
  Prove that $\infstrings{\set{1,2,3}}$ is \idx{uncountable}.

  \hint One approach is to define a surjective function from
  $\infstrings{\set{1,2,3}}$ to the power set $\power(\nngint)$.

\begin{solution}
  
  \begin{proof}
    We can define a surjective function from
    $f: \infstrings{\set{1,2,3}} \to \power(\nngint)$ as follows:
   \[
    f(s) \eqdef \set{n \in \nngint \suchthat s[n] = 1}
   \]
   where $s[n]$ is the $n$th element of sequence $s$.

    Now if there was a surjective function from $g: \nngint \to
    \infstrings{\set{1,2,3}}$, then the composition of $f$ and $g$
    would be a surjective function from $\nngint$ to
    $\power(\nngint)$ contradicting Cantor's Theorem~\bref{powbig}.
    \end{proof}

Alternatively, to show that $\infstrings{\set{1,2,3}}$ is uncountable,
we can directly use a basic diagonal argument to show that no
function, $\sigma: \nngint \to \infstrings{\set{1,2,3}}$ is a surjection.

\begin{proof}
Let $\sigma$ be a function from $\nngint$ 
to the infinite sequences of 1's, 2's, and 3's, that is,
\[
\sigma: \nngint \to \{1,2,3\}^\omega.
\]
To show that $\sigma$ is not a surjection, we will describe a
sequence, diag, of 1's, 2's, and 3's that is not in the range of
$\sigma$.

Let $r:\set{1,2,3}\to \set{1,2,3}$ be defined by
\begin{align*}
r(1) & \eqdef 2,\\
r(2) & \eqdef 3,\\
r(3) & \eqdef 1.\\
\end{align*}
In particular $r(i) \neq i$ for $i=1,2,3$.  Define a sequence
$\text{diag} \in \infstrings{\set{1,2,3}}$ as follows:
\[
\text{diag}[n] \eqdef r(\sigma(n)[n]).
\]
Now by definition,
\[
\text{diag}[n] \neq \sigma(n)[n],
\]
for all $n \in \nngint$, proving that diag is not equal to
$\sigma(n)$ for any $n \in \nngint$.  That is, diag is not in the
range of $\sigma$ as claimed.
\end{proof}
 
\end{solution}

\end{problem}

\endinput
