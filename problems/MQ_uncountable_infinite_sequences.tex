\documentclass[problem]{mcs}

\begin{pcomments}
  \pcomment{MQ_uncountable_infinite_sequences}
  \pcomment{from: S10 by wing, revised 3/19/10 by ARM}
  \pcomment{special case of PS_N_to_A_diagonal_argument}
\end{pcomments}

\pkeywords{
  Russells_paradox
  uncountable
}

%%%%%%%%%%%%%%%%%%%%%%%%%%%%%%%%%%%%%%%%%%%%%%%%%%%%%%%%%%%%%%%%%%%%%
% Problem starts here
%%%%%%%%%%%%%%%%%%%%%%%%%%%%%%%%%%%%%%%%%%%%%%%%%%%%%%%%%%%%%%%%%%%%%

\begin{problem}
  Let $[\naturals\to \set{1,2,3}]$ be the set of infinite sequences
  containing only the numbers 1, 2, and 3.  For example, some sequences of
  this kind are:
\begin{align*}
(1, 1, 1, 1 ...),\\
(2, 2, 2, 2 ...),\\
(3, 2, 1, 3 ...).
\end{align*}
  Prove that $[\naturals \to \set{1,2,3}]$ is \idx{uncountable}.

  \hint One approach is to define a surjective function from $[\naturals
  \to \set{1,2,3}]$ to the power set $\power(\naturals)$.

\begin{solution}
  
  \begin{proof}
    We can define a surjective function from
    $f:[\naturals\to \set{1,2,3}] \to \power(\naturals)$ as follows:
   \[
    f(s) \eqdef \set{n \in \naturals \suchthat s[n] = 1}
   \]
   where $s[n]$ is the $n$th element of sequence $s$.

    Now if there was a surjective function from $g: \naturals \to
    [\naturals\to \set{1,2,3}]$, then the composition of $f$ and $g$ would
    be a surjective function from $\naturals$ to $\power(\naturals)$
    contradicting Theorem~\bref{pow-big} in the text.
    \end{proof}

  \begin{proof}
    Alternatively, to show that $[\naturals \to \set{1,2,3}]$ is
    uncountable, we show that no function, $\sigma: \naturals \to
    [\naturals \to \set{1,2,3}]$ is a surjection.  In particular, we will
    describe a sequence $\text{diag} \in [\naturals \to \set{1,2,3}]$ such
    that $\text{diag} \notin \range{\sigma}$.

    Let
    \[
     \sigma_0, \sigma_1, \dots
    \]
    be the sequences in the range of $\sigma$.  Then we can
    define $\text{diag}$ as follows:
  \[
  \text{diag} \eqdef r(\sigma_0[0]), r(\sigma_1[1]),  r(\sigma_2[2]), \dots,
  \]
where $r:\set{1,2,3}\to \set{1,2,3}$ is some function such that $r(i)
\neq i$ for $i=1,2,3$.

  Now by definition,
  \[
  \text{diag}[n] \neq \sigma_n[n],
  \]
  forall $n \in \naturals$, proving that diag is not in the range of
  $\sigma$, as claimed.
  \end{proof}
 
\end{solution}

\end{problem}

\endinput
