\documentclass[problem]{mcs}

\begin{pcomments}
  \pcomment{FP_boat_trip}
  \pcomment{from: Rajeev F09}
  \pcomment{based on CP_bag_of_donuts}
\end{pcomments}

\pkeywords{
  generating_function
 convolution
}

%%%%%%%%%%%%%%%%%%%%%%%%%%%%%%%%%%%%%%%%%%%%%%%%%%%%%%%%%%%%%%%%%%%%%
% Problem starts here
%%%%%%%%%%%%%%%%%%%%%%%%%%%%%%%%%%%%%%%%%%%%%%%%%%%%%%%%%%%%%%%%%%%%%

\begin{problem}

T-Pain is planning an epic boat trip and he needs to decide what to bring with him.

\begin{itemize}

\item He definitely wants to bring burgers, but they only come in packs of 6.

\item He and his two friends can't decide whether they want to dress formally or
casually. He'll either bring 0 pairs of flip flops or 3 pairs.

\item He doesn't have very much room in his suitcase for towels, so he can
  bring at most 2.

\item In order for the boat trip to be truly epic, he has to bring at least 1
nautical-themed pashmina afghan.

\end{itemize}

\bparts

\ppart[3] Let $g_n$ be the the number of different ways for T-Pain to bring
$n$ items (burgers, pairs of flip flops, towels, and/or afghans) on his
boat trip.  Express the generating function $G(x) \eqdef
\sum_{n=0}^{\infty} g_nx^n$ as a quotient of polynomials.

\examspace[2in]

\begin{solution}
\begin{align*}
\frac{x^6}{1-x^6}(1+x^3)(1+x+x^2) \frac{x}{1-x}
  &  = \frac{(1+x^3)(1+x+x^2)x^7}%
            {(1-x^3)(1+x^3)(1-x)}\\
  &  = \frac{(1+x+x^2)x^7}%
            {(1-x)(1+x+x^2)(1-x)}\\
  & = \frac{x^7}{(1-x)^2}
\end{align*}
\end{solution}

\ppart[4] Find a closed formula in $n$ for the number of ways T-Pain can
bring exactly $n$ items with him.

\examspace[3in]

\begin{solution}
Let
\[
G(x) \eqdef \frac{1}{(1-x)^2},
\]
so the generating function for T-Pain is $x^7G(x)$.  We know that the
coefficient of $x^n$ in the series for ${(1-x)^2}$ is, by the Convolution
Rule, the number of ways to select $n$ items of two different kinds,
namely, $\binom{n+1}{1}=n+1$, so we conclude that the $n$th coefficient in
the series for $G(x)$ is $n+1$.

So the $n$th coefficient in the series for the generating function
$x^7G(x)$ is zero for $n \leq 6$, and, for $n \ge 7$, is
the $(n-7)$th coefficient of $G$, namely,
\[
(n+1)-7 = n-6.
\]
\end{solution}

\eparts

\end{problem}

%%%%%%%%%%%%%%%%%%%%%%%%%%%%%%%%%%%%%%%%%%%%%%%%%%%%%%%%%%%%%%%%%%%%%
% Problem ends here
%%%%%%%%%%%%%%%%%%%%%%%%%%%%%%%%%%%%%%%%%%%%%%%%%%%%%%%%%%%%%%%%%%%%%

\endinput
