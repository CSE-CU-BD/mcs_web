\documentclass[problem]{mcs}

\begin{pcomments}
  \pcomment{FP_expected_adjacent}
  \pcomment{from: F04.final}
\end{pcomments}

\pkeywords{
  expectation
  linearity
}

%%%%%%%%%%%%%%%%%%%%%%%%%%%%%%%%%%%%%%%%%%%%%%%%%%%%%%%%%%%%%%%%%%%%%
% Problem starts here
%%%%%%%%%%%%%%%%%%%%%%%%%%%%%%%%%%%%%%%%%%%%%%%%%%%%%%%%%%%%%%%%%%%%%

\begin{problem}
Six pairs of cards with ranks 1--6 are shuffled and laid out in a row,
for example,

\[
\fbox{1}\ \fbox{2}\ \fbox{3}\ \fbox{3}\ \fbox{4}\ \fbox{6}\ 
\fbox{1}\ \fbox{4}\ \fbox{5}\ \fbox{5}\ \fbox{2}\ \fbox{6}
\]
In this case, there are two adjacent pairs with the same value, the
two 3's and the two 5's.  What is the expected number of adjacent
pairs with the same value? \hfill \examrule

\begin{solution}
Consider an adjacent pair.  The left card matches only one
of the other 11 cards, which is equally likely to be in any of the 11
other positions.  Therefore, the probability that an adjacent pair
matches is $1/11$.  Since there are 11 adjacent pairs, the expected
number of matches is $11 \cdot 1/11 = 1$ by linearity of expectation.
\end{solution}

\end{problem}

%%%%%%%%%%%%%%%%%%%%%%%%%%%%%%%%%%%%%%%%%%%%%%%%%%%%%%%%%%%%%%%%%%%%%
% Problem ends here
%%%%%%%%%%%%%%%%%%%%%%%%%%%%%%%%%%%%%%%%%%%%%%%%%%%%%%%%%%%%%%%%%%%%%

\endinput
