\documentclass[problem]{mcs}

\begin{pcomments}
  \pcomment{FP_expected_adjacent2}
  \pcomment{variant of FP_expected_adjacent}
  \pcomment{F15.final-conflict2}
\end{pcomments}

\pkeywords{
  expectation
  linearity
}

%%%%%%%%%%%%%%%%%%%%%%%%%%%%%%%%%%%%%%%%%%%%%%%%%%%%%%%%%%%%%%%%%%%%%
% Problem starts here
%%%%%%%%%%%%%%%%%%%%%%%%%%%%%%%%%%%%%%%%%%%%%%%%%%%%%%%%%%%%%%%%%%%%%

\begin{problem}
Six pairs of cards with ranks 1--6 are shuffled and laid out in a row,
for example,
\[
\fbox{1}\ \fbox{2}\ \fbox{3}\ \fbox{3}\ \fbox{4}\ \fbox{6}\ 
\fbox{1}\ \fbox{4}\ \fbox{5}\ \fbox{5}\ \fbox{2}\ \fbox{6}
\]
In this case, there are two adjacent pairs with the same value, the
two 3's and the two 5's.  What is the expected number of adjacent
pairs with the same value?

\begin{center}
\exambox{0.5in}{0.4in}{0in}
\end{center}

\begin{solution}
\textbf{1} is the expected number of matches.

In an adjacent pair of positions, there is only one card which can
appear on the right that will match whatever card appears in the left.
Since the other eleven cards are equally likely to appear on the
right, the probability that the cards in any given pair of adjacent
positions match is $1/11$.  This implies that the expected number of
matches in any given pair of adjacent positions is 1/11.  Since there
are 11 adjacent pairs, the expected number of matches is $11 \cdot
1/11 = 1$ by linearity of expectation.
\end{solution}

\end{problem}

%%%%%%%%%%%%%%%%%%%%%%%%%%%%%%%%%%%%%%%%%%%%%%%%%%%%%%%%%%%%%%%%%%%%%
% Problem ends here
%%%%%%%%%%%%%%%%%%%%%%%%%%%%%%%%%%%%%%%%%%%%%%%%%%%%%%%%%%%%%%%%%%%%%

\endinput
