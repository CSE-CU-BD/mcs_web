\documentclass[problem]{mcs}

\begin{pcomments}
  \pcomment{FP_numbers_short_answer}
  \pcomment{excerpted from FP_multiple_choice_unhidden}
  \pcomment{subsumes FP_numbers_short_answer_fall11}
  \pcomment{revised 12/23/11 ARM}
  \pcomment{revised 12/9/15 tashas}
\end{pcomments}

\pkeywords{
  gcd
  modulo
  relatively_prime
  inverse
  mod_n
  Fermat
  prime
}

%%%%%%%%%%%%%%%%%%%%%%%%%%%%%%%%%%%%%%%%%%%%%%%%%%%%%%%%%%%%%%%%%%%%%
% Problem starts here
%%%%%%%%%%%%%%%%%%%%%%%%%%%%%%%%%%%%%%%%%%%%%%%%%%%%%%%%%%%%%%%%%%%%%

\begin{problem} \mbox{}

%Circle \textbf{true} or \textbf{false} for the statements below,
%and \emph{provide counterexamples} for those that are \textbf{false}.

Choose the weakest of the following necessary to make the statement true: $\gcd(a, b) = 1$, $c$ relatively prime to $n$, 
$a,b$ have inverses modulo $n$, $n$ does not divide $c$, $\gcd(b, c) = 1$, $\gcd(a, n) = 1$, $a$ and $b$ are prime, Nothing Needed.
Variables, $a,b,c,m,n$ range over the integers and $m,n>1$.

\bparts

%\ppart $\gcd(1 + a, 1 + b) = 1 + \gcd(a, b)$.  \hfill 
%\textbf{true} \qquad  \textbf{false} \examspace[0.4in]

%\begin{solution}
%false $a=1,b=2$
%\end{solution}

\ppart If $a \equiv b \pmod{n}$, then $p(a) \equiv p(b) \pmod{n}$

\noindent for any polynomial $p(x)$ with integer coefficients.
\begin{solution}
Nothing needed
\end{solution}

\ppart If $a \divides b c$, then $a \divides c$.  

\begin{solution}
$\gcd(a, b) = 1$
\end{solution}

\ppart $\gcd(a^n,b^n) =  (\gcd(a,b))^n$  

\begin{solution}
Nothing needed
\end{solution}

%\ppart If $\gcd(a, b) \neq 1$ and $\gcd(b, c) \neq 1$, then $\gcd(a, c) \neq 1$. \hfill 
%\textbf{true} \qquad  \textbf{false} \examspace[0.4in]

%\begin{solution}
%\textbf{false} $a=2\cdot 3, b=3\cdot 5, c=5\cdot 7$
%\end{solution}

\ppart Some integer linear combination of $a^2$ and
$b^2$ equals $1$. 

\begin{solution}
$\gcd(a, b) = 1$
\end{solution}

%\ppart If no integer linear combination of $a$ and $b$ equals 2,
%\noindent then neither does any integer linear combination of $a^2$ and $b^2$.

%\begin{solution}
%\textbf{true} No linear combination of $a,b$ is 2 iff $\gcd(a,b) >2$ iff $\gcd(a^2,b^2)>4$.
%\end{solution}

\ppart If $a c \equiv b c \pmod{n}$ then $a \equiv b \pmod{n}$.

\begin{solution}
Need $c$ relatively prime to $n$.  
\end{solution}

\ppart  If $a^{-1} \equiv b^{-1} \pmod{n}$, then $a \equiv b
\pmod{n}$.

\begin{solution}
$a,b$ have inverses modulo $n$
\end{solution}

\ppart If $a c \equiv b c \pmod{n}$, then $a \equiv b \pmod{n}$.  

\begin{solution}
Need $c$ relatively prime to $n$.  
\end{solution}

\ppart If $a \equiv b \pmod{\phi(n)}$ for $a, b > 0$, 
\noindent then $c^a \equiv c^b \pmod{n}$.  

\begin{solution}
  Need $c$ relatively prime to $n$.  
  Example:  $n=4$, so $\phi(n) = 2$; $a=1, b=3$, so $a \equiv b
  \pmod \phi(n)$, $c = 2$, so $c^a = 2 \not\equiv 0 = c^b \pmod 4$.
\end{solution}

\ppart If $a \equiv b \pmod{nm}$, then $a \equiv b \pmod{n}$.

\begin{solution}
Nothing needed.
\end{solution}

\ppart $[x \equiv y \pmod{a} \QAND\ x \equiv y \pmod{b}]$ iff $[x
  \equiv y \pmod{ab}]$ 

\begin{solution}
$\gcd(a, b) = 1$ The Chinese Remainder Theorem (Problem~\bref{CP_chinese_remainder}).
\end{solution}

\ppart If $n$ is prime, then $a^{n-1} \equiv 1 \pmod{n}$ \hfill
  \textbf{true} \qquad \textbf{false} \examspace[0.4in]

\begin{solution}
$\gcd(a,n)=1$
Fermat's Little Theorem~\bref{fermat_little} says this holds if $n$ is prime and $a$ is not divisible by $n$.
\end{solution}

\ppart If $a,b >1$, then

\noindent
 [$a$ has a inverse mod $b$  iff $b$ has an inverse mod $a$].
  \hfill \textbf{true} \qquad \textbf{false} \examspace[0.4in]

\begin{solution}
Nothing needed.
$a$ has a multiplicative inverse mod $b$ iff
$a,b$ relatively prime iff $b$ has a multiplicative inverse mod $a$.
\end{solution}


\eparts

\end{problem}

\endinput
