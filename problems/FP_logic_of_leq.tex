\documentclass[problem]{mcs}

\begin{pcomments}
  \pcomment{FP_logic_of_leq}
  \pcomment{ARM 5/21/12 from ARM draft F09}
  \pcomment{hint for y=x+1 added by ARM 9/24/15}
  \pcomment{F15.midterm1conflictsecond}
\end{pcomments}

\pkeywords{
 predicate_calculus
 domain_of_discourse
 domain
 less_than
 integer 
}

%%%%%%%%%%%%%%%%%%%%%%%%%%%%%%%%%%%%%%%%%%%%%%%%%%%%%%%%%%%%%%%%%%%%%
% Problem starts here
%%%%%%%%%%%%%%%%%%%%%%%%%%%%%%%%%%%%%%%%%%%%%%%%%%%%%%%%%%%%%%%%%%%%%

\begin{problem}
We want to find predicate formulas about the nonnegative integers
$\nngint$ in which $\leq$ is the only predicate that appears, and
no constants appear.

For example, there is such a formula defining the equality predicate:
\[
[x = y] \eqdef\ [x \leq y\ \QAND\ y \leq x].
\]
Once predicate is shown to be expressible solely in terms of $\leq$, it
may then be used in subsequent translations.  For example,
\[
[x > 0] \eqdef\ \exists y.\ \QNOT(x = y) \QAND y \leq x.
\]

\bparts

\ppart $[x = 0]$.

\examspace[0.8in]
\begin{solution}
\[
\QNOT(x > 0).
\]
Alternatively,
\[
\forall y.\ x \leq y.
\]
\end{solution}

\ppart $[x = y+1]$.

\hint If an integer is bigger than $y$, then it must be $\geq x$.

\examspace[0.8in]
\begin{solution}
  \[ % y \leq x \QAND
    \forall z.\ [z \neq y \QAND y \leq z] \QIMPLIES x \leq z.
  \]
\end{solution}

\ppart $x = 3$.

\examspace[0.8in]

\begin{solution}
\[
\exists w,y,z.\ w = 0 \QAND y = w+1 \QAND z = y+1 \QAND x = x+1.
\]

\end{solution}
\eparts
\end{problem}

\endinput
