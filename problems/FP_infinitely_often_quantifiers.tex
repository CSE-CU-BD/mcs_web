\documentclass[problem]{mcs}

\begin{pcomments}
  \pcomment{FP_infinitely_often_quantifiers}
  \pcomment{ARM 5/21/17}
\end{pcomments}

\pkeywords{
 quantifier
 limit
 infinitely
 predicate
}

%%%%%%%%%%%%%%%%%%%%%%%%%%%%%%%%%%%%%%%%%%%%%%%%%%%%%%%%%%%%%%%%%%%%%
% Problem starts here
%%%%%%%%%%%%%%%%%%%%%%%%%%%%%%%%%%%%%%%%%%%%%%%%%%%%%%%%%%%%%%%%%%%%%


\begin{problem}
\begin{staffnotes}
\textbf{S17 final 4pts: 1 pt each}
\end{staffnotes}


\bparts

\ppart A predicate $R$ on the nonnegative integers is true
\emph{infinitely often (i.o.)} when $R(n)$ is true for infinitely
many $n \in \nngint$.

We can express the fact that $R$ is true i.o.\ with a formula of the
form:
\[
\mathbf{Q}_1\ \ \mathbf{Q}_2.\ R(n),
\]
where $\mathbf{Q}_1, \mathbf{Q}_2$ are quantifiers from among
\[\begin{array}{cccc}
\forall n, & \exists n, & \forall n \geq n_0, & \exists n \geq n_0,\\
\forall n_0, & \exists n_0, & \forall n_0 \geq n, & \exists n_0 \geq n,
\end{array}\]
and $n, n_0$ range over nonnegative integers.

Identify the proper quantifers:
\begin{center}
$\mathbf{Q}_1$\hspace{0.4in} \examrule[0.7in]
\end{center}

\begin{center}
$\mathbf{Q}_2$\hspace{0.4in} \examrule[0.7in]
\end{center}

\begin{solution}
\[
\forall n_0\ \ \exists  n \geq n_0.\ R(n)\, .
\]
\end{solution}

\ppart A predicate $S$ on the nonnegative integers is true
\emph{almost everywhere (a.e.)} when $S(n)$ is false for only finitely
many $n \in \nngint$.

We can express the fact that $S$ is true a.e.\ with a formula of the
form
\[
\mathbf{Q}_3\ \ \mathbf{Q}_4.\ S(n),
\]
where $\mathbf{Q}_3, \mathbf{Q}_4$ are quantifiers from those above:
\[\begin{array}{cccc}
\forall n, & \exists n, & \forall n \geq n_0, & \exists n \geq n_0,\\
\forall n_0, & \exists n_0, & \forall n_0 \geq n, & \exists n_0 \geq n.
\end{array}\]

Identify the proper quantifers:
\begin{center}
$\mathbf{Q}_3$\hspace{0.4in} \examrule[0.7in]
\end{center}

\begin{center}
$\mathbf{Q}_4$\hspace{0.4in} \examrule[0.7in]
\end{center}

\begin{solution}
\[
\exists n_0 \forall n \geq n_0.\ S(n)\, .
\]
\end{solution}

\eparts

\end{problem}

%%%%%%%%%%%%%%%%%%%%%%%%%%%%%%%%%%%%%%%%%%%%%%%%%%%%%%%%%%%%%%%%%%%%%
% Problem ends here
%%%%%%%%%%%%%%%%%%%%%%%%%%%%%%%%%%%%%%%%%%%%%%%%%%%%%%%%%%%%%%%%%%%%%

\endinput
