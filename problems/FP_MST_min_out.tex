\documentclass[problem]{mcs}

\begin{pcomments}
  \pcomment{FP_MST_min_out}
  \pcomment{F15.conflictmid3}
  \pcomment{by ARM 11/3/15}
\end{pcomments}

\pkeywords{
  tree
  spanning_tree
  MST
  weight
  minimum
  cycle
  gray_edge
}

%%%%%%%%%%%%%%%%%%%%%%%%%%%%%%%%%%%%%%%%%%%%%%%%%%%%%%%%%%%%%%%%%%%%%
% Problem starts here
%%%%%%%%%%%%%%%%%%%%%%%%%%%%%%%%%%%%%%%%%%%%%%%%%%%%%%%%%%%%%%%%%%%%%

\begin{problem}
\bparts

\ppart Give an example of a connected, weighted simple graph with four
vertices and distinct edge weights, that contains a cycle, and in which
the edge with the largest weight belongs to the minumum weight
spanning tree (MST) of the graph.

\examspace[1.25in]

\begin{solution}
$V = \set{a,b,c,d}$, $E = \set{\edge{a}{b}, \edge{b}{c}, \edge{b}{d},
    \edge{c}{d}}$, $W = \set{4,3,2,1}$.
\end{solution}

\ppart
Can you construct such an example with three vertices?  Explain.

 %Why is it not possible to construct such an example with three
 %vertices?

\examspace[1.5in]

\begin{solution}
No.

A cycle must contain at least three vertices. Since there are only
three vertices, they must be all connected in order to form a
cycle.  It follows that the edge with the largest weight must also be
on that cycle.  But then it is not possible for the largest edge to be
in MST, since taking the other two edges would result in an MST with
smaller weight.
\end{solution}

\ppart Let $G$ be a connected, weighted simple graph.  Let $v \in
\vertices{G}$ be an arbitrary vertex whose incident edges have
distinct weights.  Suppose that edge $\edge{v}{w} \in \edges{G}$ has
minimum weight among the edges incident to $v$.  Prove that
$\edge{v}{w}$ must be an edge of some\footnote{Actually, the minimum
  weight edge incident to $v$ must be in \emph{every} MST, but you
  need not prove this.} MST of $G$.

\hint Appeal to the gray edge construction \inhandout{in the text}
\inbook{of Lemma~\bref{lem:enough-gray}}.  Alternatively, consider a
path in an MST from between $v$ and $w$.

\examspace[3in]

\begin{solution}
Using the gray edge Lemma, color vertex $v$ black and all other
vertices white.  Then the only gray edges are the edges incident to
$v$, so $\edge{v}{w}$ will be the minimum weight gray edge, and
therefore will be in the MST built starting with this coloring.

Alternatively, if $\edge{v}{w}$ is not in some MST $M$ then there is
a path in $M$ from $v$ to $w$, and this path must start with some edge
$g$ incident to $v$ that is heavier than $\edge{v}{w}$.  Then
$M-g+\edge{v}{w}$ would be an MST with smaller weight than $M$, a
contradiction.  So $\edge{v}{w}$ must be a member of every MST for $G$.
\end{solution}

\eparts

\end{problem}

%%%%%%%%%%%%%%%%%%%%%%%%%%%%%%%%%%%%%%%%%%%%%%%%%%%%%%%%%%%%%%%%%%%%%
% Problem ends here
%%%%%%%%%%%%%%%%%%%%%%%%%%%%%%%%%%%%%%%%%%%%%%%%%%%%%%%%%%%%%%%%%%%%%
\endinput
