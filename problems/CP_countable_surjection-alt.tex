\documentclass[problem]{mcs}

\begin{pcomments}
  \pcomment{CP_countable_surjection-alt}
  \pcomment{part of CP_countable_from_surj}
  \pcomment{ARM 12/23/11}
  \pcomment{Mild rephrasing by zabel, 3/12/18}
\end{pcomments}

\pkeywords{
  countable
  surjection
}

%%%%%%%%%%%%%%%%%%%%%%%%%%%%%%%%%%%%%%%%%%%%%%%%%%%%%%%%%%%%%%%%%%%%%
% Problem starts here
%%%%%%%%%%%%%%%%%%%%%%%%%%%%%%%%%%%%%%%%%%%%%%%%%%%%%%%%%%%%%%%%%%%%%

\begin{problem}
If set $S$ is infinite and there is a surjective function ($[\leq 1
  \text{ out}, \geq 1 \text{ in}]$ mapping) $f:\nngint \to S$, prove
that $S$ is countable by explicitly constructing a bijection
$g:\nngint \to S$.  (This proves the more challenging part of
Lemma~\bref{NsurjC}, so please only rely on
Definition~\bref{def_countable}.)

\hint The ``sequence'' $f(0), f(1), f(2), \ldots$ may have undefined
and/or repeated values.  How can you filter these out?

\begin{solution}
The function $f$ is not required to be total, but it is also safe to
assume that $f$ is total because we could always extend $f$ to be
total by defining it to be an arbitrary element of $S$ wherever it
might have been undefined.

Now we can think of $f(0), f(1), f(2), \dots$ as a stream of elements
in $S$.  Since $f$ is a surjection, all the elements of $S$ appear in
this sequence, but there may be lots of duplicate occurrences in this
sequence.  If we could find a sequence of the same elements without
duplicates, then we would have the bijection we need.  But it is easy
to filter out the duplicates and get the desired bijection.  Namely,
let $g(n)$ be the $n$th distinct element in the list.  More precisely,
we can define $g:\nngint \to S$ recursively by the rules:
\begin{align*}
g(0)   & \eqdef f(0),\\
g(n+1) & \eqdef f(k) & \text{where $k$ is minimum such that}\\
       &             &  f(k) \notin \set{g(0),g(1),\dots,g(n)}.
\end{align*}
Because $S$ is infinite, then there always will be an $f(k)$ available to
be the value of $g(n+1)$.  Since all the elements in the sequence
$f(0), f(1), \dots$ still appear in the sequence $g(0), g(1), \dots$,
the function $g$ is a surjection.  Also, all the elements in the
sequence $g(0), g(1), \dots$ are by definition different, so the
function $g$ is also an injection.  This means that $g$ is a bijection
from $\nngint$ to $S$, as required.
\end{solution}

\end{problem}

%%%%%%%%%%%%%%%%%%%%%%%%%%%%%%%%%%%%%%%%%%%%%%%%%%%%%%%%%%%%%%%%%%%%%
% Problem ends here
%%%%%%%%%%%%%%%%%%%%%%%%%%%%%%%%%%%%%%%%%%%%%%%%%%%%%%%%%%%%%%%%%%%%%

\endinput
