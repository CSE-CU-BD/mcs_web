\documentclass[problem]{mcs}

\begin{pcomments}
  \pcomment{MQ_state_machine_multiply}
  \pcomment{S07.miniquiz-3-07}
  \pcomment{MQ version of CP_state_machine_multiply}
  \pcomment{renamed from PS_state_machine_multiply 2/21/11}
\end{pcomments}

\pkeywords{
  state_machines
  termination
  partial_correctness
  invariant
  algorithm
  multiply
}

%%%%%%%%%%%%%%%%%%%%%%%%%%%%%%%%%%%%%%%%%%%%%%%%%%%%%%%%%%%%%%%%%%%%%
% Problem starts here
%%%%%%%%%%%%%%%%%%%%%%%%%%%%%%%%%%%%%%%%%%%%%%%%%%%%%%%%%%%%%%%%%%%%%

\begin{problem}
\begin{editingnotes}
Tests state machine understanding, invariant method, termination arguments.
IMPORTANT: If reusing this problem, the second part should be split into 2 parts:
 1. Prove the invariant., 2. Apply invariant principle...
\end{editingnotes}

The following state machine describes a procedure that terminates with the
product of two nonnegative integers $x$ and $y$ in register $a$.  Its
states are triples of nonnegative integers $(r,s,a)$.  The initial state
is $(x,y,0)$.  The transitions are given by the rule that for $s>0$:
\[
(r,s,a)\rightarrow\begin{cases}
        (2r,s/2,a) &\text{if $s$ is even},\\
        (2r,(s-1)/2,a+r) &\text{otherwise}.
       \end{cases}
\]

%\iffalse
%\solution{
%\medskip
%\textbf{Solution:}
%\begin{description}
%\item $Q=\{(r,s,a,x, n) | r,s,a,x, n\in \mathbb{N}\}$
%\item $Q_0=\{(x, n,0,x, n)\}$
%\item $\delta: (r,s,a,x, n)\rightarrow (3r,s',a',x, n)$, where \\
%If $s \equiv 0 \pmod{3}$: $s'=s/3$, and $a'=a$ \\
%If $s \equiv 1 \pmod{3}$: $s'=(s-1)/3$, and $a'=a+r$\\
%If $s \equiv 2 \pmod{3}$: $s'=(s-2)/3$, and $a'=a+2r$
%\end{description}
%}
%\fi

\bparts

\ppart\label{rsa} Circle the predicates below that are invariant for
this state machine:
\begin{itemize}
\item $P((r,s,a)) \eqdef\quad [r+a = xy]$
\item $P((r,s,a)) \eqdef\quad [s+a = xy]$
\item $P((r,s,a)) \eqdef\quad [rs+a = xy]$
\item $P((r,s,a)) \eqdef \quad [r=r+1]$

\end{itemize}

\begin{solution}
$[rs+a = xy]$ and $[r=r+1]$ are invariants.  The second of these
  is vacuously invariant because it is always false.
\end{solution}

\iffalse

\ppart Prove that the predicate you picked in \emph{Part a} holds for
the base case and is invariant under all possible transitions.
\begin{solution}
Let
\[
P((r,s,a)) \eqdef\quad [rs+a = xy].
\]

Clearly, $P$ holds for the start state because
\[
P((x,y,0)) \qiff [xy+0 = xy].
\]

Now, we show that $P$ is indeed invariant, namely, assuming $P((r,s,a))$,
\begin{equation}\label{inv}
rs+a = xy,
\end{equation}
holds and $(r,s,a) \to (r',s',a')$ is a transition, then $P((a',b',p'))$,
\begin{equation}\label{inv'}
r's'+a' = xy,
\end{equation}
holds.

We consider 2 cases:

If $2 \divides s$, then we have that $r' = 2r, s' = s/2, a'=a$.
Therefore,
\begin{align*}
  r's' + a' = & 2r \cdot \frac{s}{2} + a\\
            = & rs+a\\
            = & xy & \text{(by~\eqref{inv})}.
\end{align*}

Otherwise, we have $r' = 2r, s' = (s-1)/2,a = a+r$.  So:
\begin{align*}
  r's' + a'  = & 2r \cdot \frac{s-1}{2} + a+r\\
   = & r\cdot(s-1) + a + r\\
   = & rs+a\\
   = & xy & \text{(by~\eqref{inv})}.
\end{align*}
So in both cases,~\eqref{inv'} holds, proving that $P$ is indeed an
invariant.
\end{solution}
\fi

\ppart Use an invariant from part~\eqref{rsa} to prove that the
algorithm is partially correct---that is, if the machine terminates (that
is, reachs a state from which no transition is possible), then $a = xy$.
(You do \textbf{not} have to prove that the invariant you select from
part~\eqref{rsa} actually \emph{is} an invariant.)

\begin{solution}
Since the procedure's only termination
  condition is that $s=0$, partial correctness will follow if we can show
  that if $s=0$, then $a=xy$.  But this follows immediately from the
  invariant $[rs+a = xy]$.
\end{solution}

%\instatements{\newpage}

\ppart Briefly explain why this state machine will in fact terminate
for all $x,y \in \nngint$.

\begin{solution}
We'll actually prove something stronger, namely,
  that the algorithm terminates after at most $1+\log_2 y$ executions of
  \texttt{do} statement.  We first notice that $s \in \nngint$ is an
  invariant.  Also, each transition corresponds to an execution of the
  \texttt{do} statement body, and at each transition, $s$ is reduced by a
  factor of at most $1/2$.  Hence, after at most $1+ \log_2 y$ executions
  of the body, the final value of $s$ is at most $1/2^{1+ \log_2 y} =
  1/2y$ times its initial value $y$.  This means the value of $s$ will be
  less than 1, and since the value is a nonnegative integer, $s$ must be 0
  at this point if it wasn't 0 earlier.  But with $s=0$, the procedure
  terminates.
\end{solution}

\eparts
\end{problem}

\endinput
