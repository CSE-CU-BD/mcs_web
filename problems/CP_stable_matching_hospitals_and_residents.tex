\documentclass[problem]{mcs}

\begin{pcomments}
  \pcomment{CP_stable_matching_hospitals_and_residents}
  \pcomment{Added by Oshani on Prof. Meyer's instructions 03/16/11}
\end{pcomments}

\pkeywords{
 stable_matching
 stable_marriage
 residents
 hospital
 }


%%%%%%%%%%%%%%%%%%%%%%%%%%%%%%%%%%%%%%%%%%%%%%%%%%%%%%%%%%%%%%%%%%%%%
% Problem starts here
%%%%%%%%%%%%%%%%%%%%%%%%%%%%%%%%%%%%%%%%%%%%%%%%%%%%%%%%%%%%%%%%%%%%%

\begin{problem}

The most famous application of stable matching was in assigning
graduating medical students to hospital residencies.  Each hospital has a
preference ranking of students and each student has a preference order of
hospitals, but unlike the setup in the notes where there are an equal
number of boys and girls and monogamous marriages, hospitals generally have
differing numbers of available residencies, and the total number of
residencies may not equal the number of graduating students.  

What would be a `rogue couple' in this situation?

Modify the definition of stable matching so it applies for hospitals and residents. 


\begin{solution}

A `rogue couple' will be a pair consisting hospital $H$ that wants resident $S$ from another hospital $H'$ better than its own residents $R_1, R_2,..., R_n$ , and the resident $S$ preferring hospital $H$ better than her own hospital $H'$.

A matching is an assignment of medical students to residencies in each of the hospitals (an injection, $A:
\text{students} \to \text{residencies}$) such that every student has a
residency ($A$ is total), or every residency has an assigned student ($A$
is a surjection).  A stable assignment is one with no \emph{rogue couples},
where a rogue couple is a hospital student pair $(H,S)$ such that $S$ is
not assigned to one of the residencies at $H$, which she prefers over
her current assignment, and
\begin{itemize}
\item $H$ has some students assigned to some of its residencies and
prefers $S$ to at least one of its assigned students, or

\item $H$ has none of its residencies assigned.

\end{itemize}

\end{solution}


\end{problem}

\endinput
