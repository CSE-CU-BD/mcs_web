\documentclass[problem]{mcs}

\begin{pcomments}
  \pcomment{PS_stable_marriage_better_worse}
  \pcomment{by ARM 4/11/15}
\end{pcomments}

\pkeywords{
 stable_matching
 optimal_spouse
 pessimal
 invariant
}


%%%%%%%%%%%%%%%%%%%%%%%%%%%%%%%%%%%%%%%%%%%%%%%%%%%%%%%%%%%%%%%%%%%%%
% Problem starts here
%%%%%%%%%%%%%%%%%%%%%%%%%%%%%%%%%%%%%%%%%%%%%%%%%%%%%%%%%%%%%%%%%%%%%

\begin{problem}
Suppose there are two stable sets of marriages, a first set and a
second set.  So each man has a first wife and a second wife (they may
be the same), and likewise each woman has a first husband and a second
husband.  We will show that

\begin{lemma*}%\label{lem:better-worse}
A man does better in his second marriage iff his first wife does worse
in her second marriage.
\end{lemma*}

Let's call someone \emph{active} when their first spouse is different
from their second spouse.  This means that a spouse of an active
person must also be active.  We'll say a marriage is active when the
married couple are active.

\bparts

\ppart\label{same-active} Explain why the total number of active marriages is
the same as the total number of active people.

\begin{solution}
One simple way to see this is to observe that the number of active
first marriages is the same as the number of active men, and the
number of active second marriages is the same as the number of active
women.
\end{solution}

\eparts
Now let's count the number of times active people do better in one of
their marriages.  Since every active person by definition does better
in exactly one of their two marriages, this total count simply equals
the number of active people.  But instead of counting by active
people, we can count by active marriages.  Hypothetically, each active
marriage could contribute a count of zero, one, or two to the total
number of times people do better.

When an active marriage contributes a count of one, it means that one
member of the married couple did better in their other marriage, while
their spouse did worse in their other marriage.  The Lemma above can
now be rephrased as saying that \emph{every} active marriage
contributes a count of one.

So all that's needed is to prove that no active marriage can
contribute a count of zero or two.

\bparts

\ppart\label{no2better} Prove that no active marriage can contribute a
score of two.

\hint If an active first marriage contributed a score of two, then
both spouses do better in their second marriages.

\begin{solution}
If both spouses in an active marriage---say it is a first marriage--do
better than in their second marriages, they would be a rogue couple
for the second set of marriages, contradicting stabillity.
\end{solution}

\ppart Complete the proof of the Lemma by arguing that if some active
marriage contributed a score of zero, then some other active marriage
would have to contribute a score of two.

\hint Part~\ref{same-active}.

\begin{solution}
By part~\ref{same-active}, the total number of times active people do
better in one of their marriages equals the number of active
marriages.  So if one active marriage contributed a count of zero,
some other active marriage would have to contribute of score of two in
order for the total count to equal the number of active marriages.
\end{solution}

\ppart Conclude:

\begin{corollary}\label{lem:opt-pess}
In a stable set of marriages, a man is married to his optimal wife iff
he is her pessimal husband.
\end{corollary}

\begin{solution}
Suppose a man is married to his optimal wife in a stable set of
marriages.  If he was not her pessimal husband, then there would be a
second stable set of marriages in which his wife had a worse husband,
and by Lemma~\ref{lem:better-worse}, this would mean he had a better
second wife, contradicting optimality of his first wife.
\end{solution}

\eparts
\end{problem}

\endinput
