\documentclass[problem]{mcs}

\begin{pcomments}
  \pcomment{PS_stable_marriage_better_worse}
  \pcomment{by ARM 5/2/15}
  \pcomment{based on suggestions of Leonid Levin}
\end{pcomments}

\pkeywords{
 stable_matching
 optimal_spouse
 pessimal
 invariant
}


%%%%%%%%%%%%%%%%%%%%%%%%%%%%%%%%%%%%%%%%%%%%%%%%%%%%%%%%%%%%%%%%%%%%%
% Problem starts here
%%%%%%%%%%%%%%%%%%%%%%%%%%%%%%%%%%%%%%%%%%%%%%%%%%%%%%%%%%%%%%%%%%%%%

\begin{problem}
Suppose there are two stable sets of marriages\iffalse , a first set
and a second set\fi.  So each man has a first wife and a second wife
(they may be the same), and likewise each woman has a first husband
and a second husband.

Someone in a given marriage is a \emph{winner} when they prefer their
current spouse to their other spouse, and they are a \emph{loser}
when they prefer their other spouse to their current spouse.  (If
someone has the same spouse in both of their marriages, then they will
be neither a winner nor a loser.)

We will show that
\begin{lemma*}
\begin{equation}\label{WL}
\text{In every marriage, someone is a winner iff their spouse is a loser.}
\end{equation}
\end{lemma*}

\bparts

\ppart\label{2l2w} One direction of~\eqref{WL} follows directly from
the definition of a rogue couple: explain why a winner must have a
spouse who is a loser.

\begin{solution}
If both members of a marriage were winners, then they would, by
definition, be a rogue couple for the other set marriages,
contradicting stability.
\end{solution}

\ppart\label{same-active} Part~\eqref{2l2w} implies that the set of
men who are first marriage winners is at least as large as the set of
women who are first marriage losers.  Prove that these two sets are
the same size.

\hint Symmetry between men/women, first/second marriages,
winners/losers.

\begin{solution}
By symmetry, Part~\eqref{2l2w} holds if we switch men and women and
also first and second sets of marriages.  That is, Part~\eqref{2l2w}
implies that:
\begin{quote}
The set women who are second marriage winners is at least as large
as the set of men who are second marriage losers.
\end{quote}
But someone is a winner in one marriage iff they are a loser in their
other marriage.   So the statement above is the same as
\begin{quote}
The set of women who are first marriage losers is at least as large as
the set of men who are first marriage winners.
\end{quote}
So the set of women who are first marriage losers, and the set of men
who are first marriage winners, are each at least as large as the
other, which means they are the same size.
\end{solution}

\ppart Conclude Lemma~\eqref{WL}.

\begin{solution}
By parts~\eqref{2l2w} and~\eqref{same-active}, the set of wives of men
who are first marriage winners is a same-size subset of the women who
are first marriage losers.  So these sets are equal, which means that
every woman who is a first marriage loser has a first husband who is a
first marriage winner.  We conclude that a man is a first marriage
winner iff his first wife is a first marriage loser.  By symmetry
between men and women, and between first and second marriages, we
conclude Lemma~(WL).
\end{solution}

\iffalse
\begin{corollary*}%\label{lem:opt-pess}
In a stable set of marriages, a man is married to his optimal wife iff
he is her pessimal husband.
\end{corollary*}

\begin{solution}
Suppose a man is married to his optimal wife in a stable set of
marriages.  If he was not her pessimal husband, then there would be a
second stable set of marriages in which his wife had a worse husband,
and by the Lemma, this would mean he had a better second wife,
contradicting optimality of his first wife.
\end{solution}
\fi


\eparts
\end{problem}

\endinput
