!\documentclass[problem]{mcs}

\begin{pcomments}
  \pcomment{from: S09.cp14m}
\end{pcomments}

\pkeywords{
  expectation
}

%%%%%%%%%%%%%%%%%%%%%%%%%%%%%%%%%%%%%%%%%%%%%%%%%%%%%%%%%%%%%%%%%%%%%
% Problem starts here
%%%%%%%%%%%%%%%%%%%%%%%%%%%%%%%%%%%%%%%%%%%%%%%%%%%%%%%%%%%%%%%%%%%%%

\begin{problem} A classroom has sixteen desks arranged as shown below.
%
\begin{center}
\begin{picture}(270,198)
% \put(0,0){\dashbox(270,198){}} % bounding box
\multiput(0,0)(72,0){4}{\framebox(54,36){}}
\multiput(0,54)(72,0){4}{\framebox(54,36){}}
\multiput(0,108)(72,0){4}{\framebox(54,36){}}
\multiput(0,162)(72,0){4}{\framebox(54,36){}}
\end{picture}
\end{center}
%
If there is a girl in front, behind, to the left, or to the right of a
boy, then the two of them \emph{flirt}.  One student may be in
multiple flirting couples; for example, a student in a corner of the
classroom can flirt with up to two others, while a student in the
center can flirt with as many as four others.  Suppose that desks are
occupied by boys and girls with equal probability and mutually
independently.  What is the expected number of flirting couples?
\hint Linearity.

\begin{solution}
First, let's count the number of pairs of adjacent desks.
There are three in each row and three in each column.  Since there are
four rows and four columns, there are $3 \cdot 4 + 3 \cdot 4 = 24$
pairs of adjacent desks.

Number these pairs of adjacent desks from 1 to 24.  Let $F_i$ be an
indicator for the event that occupants of the desks in the $i$-th pair
are flirting.  The probability we want is then:
%
\begin{align*}
\expect{\sum_{i=1}^{24} F_i}
	& = \sum_{i=1}^{24} \expect{F_i} & \text{(linearity of $\expect{\cdot}$)}\\
	& = \sum_{i=1}^{24} \pr{F_i = 1} & \text{($F_i$ is an indicator)}
\end{align*}

The occupants of adjacent desks are flirting iff they are of opposite
sexes, which happens with probability 1/2, that is, $\pr{F_i = 1} = 1/2$.
Plugging this into the previous expression gives:
%
\[
\expect{\sum_{i=1}^{24} F_i}
	 = \sum_{i=1}^{24} \pr{F_i = 1}
	 = 24 \cdot \frac{1}{2}
	 = 12
\]
\end{solution}
\end{problem}

%%%%%%%%%%%%%%%%%%%%%%%%%%%%%%%%%%%%%%%%%%%%%%%%%%%%%%%%%%%%%%%%%%%%%
% Problem ends here
%%%%%%%%%%%%%%%%%%%%%%%%%%%%%%%%%%%%%%%%%%%%%%%%%%%%%%%%%%%%%%%%%%%%%
