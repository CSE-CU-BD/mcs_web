\documentclass[problem]{mcs}

\begin{pcomments}
  \pcomment{CP_sixteen_desks}
  \pcomment{from: S09.cp14m}
\end{pcomments}

\pkeywords{
  expectation
  linearity
}

%%%%%%%%%%%%%%%%%%%%%%%%%%%%%%%%%%%%%%%%%%%%%%%%%%%%%%%%%%%%%%%%%%%%%
% Problem starts here
%%%%%%%%%%%%%%%%%%%%%%%%%%%%%%%%%%%%%%%%%%%%%%%%%%%%%%%%%%%%%%%%%%%%%

\begin{problem} A classroom has sixteen desks in a $4 \times 4$ arrangement as shown below.
%
\begin{center}
\begin{picture}(300,160)
% \put(0,0){\dashbox(270,198){}} % bounding box
\multiput(0,  0)(72,0){4}{\framebox(40,25){}}
\multiput(0, 40)(72,0){4}{\framebox(40,25){}}
\multiput(0, 80)(72,0){4}{\framebox(40,25){}}
\multiput(0,120)(72,0){4}{\framebox(40,25){}}
\end{picture}
\end{center}
%
If there is a girl in front, behind, to the left, or to the right of a
boy, then the two \emph{flirt}.  One student may be in
multiple flirting couples; for example, a student in a corner of the
classroom can flirt with up to two others, while a student in the
center can flirt with as many as four others.  Suppose that desks are
occupied mutually independently by boys and girls with equal probability.  What is the expected number of flirting couples?
\hint Linearity.

\begin{solution}
A natural first approach to this problem is to calculate the expected
number of flirtations that each desk is involved in, add the
expectations for each desk, and then divide by two (since each
flirtation involves two desks).  This approach works fine, but it requires finding the expectations for three different kinds of desks: corner, side, and middle.

A more elegant approach is to note that the expected number of
flirtations between adjacent desks is 1/2, and the number of pairs of
adjacent desks is 24---there are 12 pairs adjacent horizontally (3 in
each of 4 rows) and likewise 12 pairs adjacent vertically.  So by
linearity of expectation, the expected number of flirtations is
$(1/2)\cdot 24 = 12$.

To be more explicit about this application of linearity, let's
arbitrarily number the pairs of adjacent desks from 1 to 24 and let
$F_i$ be an indicator random variable for the event that occupants of the desks in the $i$-th pair are flirting.  The occupants of adjacent desks are
flirting iff they are of opposite sexes, which happen with
probability 1/2, so $\pr{F_i = 1} = 1/2$.  The
expectation of an indicator variable is the same as the probability that it
equals 1\inbook{(Lemma~\bref{expindic})}, so
\begin{equation}\label{fi=12}
\expect{F_i} = \frac{1}{2}.
\end{equation}
Now, if $F$ is the number of flirting couples, then $F =
\sum_{i=1}^{24} F_i$, so the expectation we want is
\begin{align*}
\expect{F}
        & = \Expect{\sum_{i=1}^{24} F_i}\\
	& = \sum_{i=1}^{24} \expect{F_i} 
			& \text{(linearity of $\expect{\cdot}$)}\\
	& = \sum_{i=1}^{24} \frac{1}{2} 
			& \text{(equation~\eqref{fi=12})}\\
        & = 24 \cdot \frac{1}{2} = 12.
\end{align*}

\end{solution}
\end{problem}

%%%%%%%%%%%%%%%%%%%%%%%%%%%%%%%%%%%%%%%%%%%%%%%%%%%%%%%%%%%%%%%%%%%%%
% Problem ends here
%%%%%%%%%%%%%%%%%%%%%%%%%%%%%%%%%%%%%%%%%%%%%%%%%%%%%%%%%%%%%%%%%%%%%

\endinput
