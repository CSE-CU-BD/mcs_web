\documentclass[problem]{mcs}

\begin{pcomments}
  \pcomment{PS_gen_fcns_pennies_nickels_etc}
  \pcomment{from: S08.ps8}
  \pcomment{edited ARM 1/12/13}
\end{pcomments}

\pkeywords{
  generating functions
  symbolic_calculation
  convolution
}

%%%%%%%%%%%%%%%%%%%%%%%%%%%%%%%%%%%%%%%%%%%%%%%%%%%%%%%%%%%%%%%%%%%%%
% Problem starts here
%%%%%%%%%%%%%%%%%%%%%%%%%%%%%%%%%%%%%%%%%%%%%%%%%%%%%%%%%%%%%%%%%%%%%

\begin{problem}
  We will use generating functions to determine how many ways there
  are to use pennies, nickels, dimes, quarters, and half-dollars to
  give $n$ cents change.

  \bparts \ppart Write the generating function $P(x)$ for for the
  number of ways to use only pennies to make $n$ cents.

  \begin{solution} Since there is only one way to change any given amount
    with only pennies, so the generating function for the sequence is
  \[
  P(x) \eqdef 1 + x + x^2 + x^3 + \cdots = \frac{1}{1-x}.
  \] 
\end{solution}

  \ppart Write the generating function $N(x)$ for the number of ways
  to use only nickels to make $n$ cents.

  \begin{solution} There is no way to change amounts that are not multiples
    of five with only nickels.  So the generating function for the sequence is
\begin{align*}
\lefteqn{N(x)}\\
     & \eqdef 1 + 0x + 0x^2 + 0x^3 + 0x^4 + 1x^5 + 0x^6 + 0x^7 +0x^8 +0x^9 + x^{10} + 0x^{11} + \cdots\\
     & = 1 + x^5 + x^{10} + x^{15} + \cdots\\
     & = \frac{1}{1-x^5}.
\end{align*}
\end{solution}

  \ppart Write the generating function for the number of ways to
  use only nickels and pennies to change $n$ cents.

  \begin{solution} Since $P(x)$ and $N(x)$ are generating functions for the number of ways
    to choose pennies and nickels separately, by the Convolution
    Rule Counting Rule, the generating function for the number
    of ways to choose pennies and nickels together, is the their product
    \[
    N(x) \cdot P(x) = \frac{1}{(1-x)(1-x^5)}.
    \]
\end{solution}

  \ppart Write the generating function for the number of ways to use
  pennies, nickels, dimes, quarters, and half-dollars to give $n$
  cents change.

  \begin{solution} Generalizing our method gives;
    \[
    C(x) \eqdef \frac{1}{(1-x)(1-x^5)(1-x^{10})(1-x^{25})(1-x^{50})}.
    \]

%=(1-x) (1-x^5)^2 (1+x^5) (1-x^{25})^2 (1+x^{25})
\end{solution}

  \ppart Explain how to use this function to find out how many ways are
  there to change 50 cents; you do \emph{not} have to provide the answer
  or actually carry out the process.

  \begin{solution}
    The answer is the coefficient to $x^{50}$ of the power series for
    $C(x)$, which happens to be 50.  This coefficient could be
    extracted by taking the 50th derivative of $C(x)$ or using the
    partial fraction method to obtain a system of 50+25+10+1 linear
    equations in as many variables and then solving for the variables.
    Neither of these approaches would be appropriate for hand
    calculation, but the answer would be easy to get using a
    symbolic mathematics program such as \term{Mathematica}.
  \end{solution}

  \eparts
\end{problem}
%%%%%%%%%%%%%%%%%%%%%%%%%%%%%%%%%%%%%%%%%%%%%%%%%%%%%%%%%%%%%%%%%%%%%
% Problem ends here
%%%%%%%%%%%%%%%%%%%%%%%%%%%%%%%%%%%%%%%%%%%%%%%%%%%%%%%%%%%%%%%%%%%%%

\endinput
