\documentclass[problem]{mcs}

\begin{pcomments}
  \pcomment{PS_pigeonhole-power_of_3}
  \pcomment{commented out of PS_pigeon_hunting from S08.ps9b}
\end{pcomments}

\pkeywords{
  Pigeonhole
  power_of_3  
}

%%%%%%%%%%%%%%%%%%%%%%%%%%%%%%%%%%%%%%%%%%%%%%%%%%%%%%%%%%%%%%%%%%%%%
% Problem starts here
%%%%%%%%%%%%%%%%%%%%%%%%%%%%%%%%%%%%%%%%%%%%%%%%%%%%%%%%%%%%%%%%%%%%%

\begin{problem}

  Show that for any set of 201 positive integers less than 300, there
  must be two whose quotient is a power of three (with no remainder).

\begin{solution}
We use the Pigeonhole Principle.

The pigeons are the possible integers 1,\dots, 299.  The pigeonholes are
the 200 numbers among these that are \emph{not} divisible by 3.  If a
number factors into $m$ times a nonnegative power of 3, where $m$ is not
divisible by 3, then assign it to pigeonhole $m$.  Then there are two
numbers placed in the same pigeonhole, and the larger divided by
the smaller will be a nonnegative power of 3
\end{solution}

\iffalse

\ppart Suppose $n+1$ numbers are selected from $\set{1,2,3, \dots,2n}$.
Show that there must be two selected numbers whose quotient is a power of
two.

\begin{solution}
  The $n+1$ selected numbers are the pigeons.  There are $n$ odd
  numbers between 1 and $2n$, namely, the numbers of the form $2k+1$ for
  $0 \leq k <2n$.  The pigeonholes will be the sets consisting of
  power-of-two multiples of each of these odd numbers.  For example, the
  third odd number is 5, so the third pigeonhole will be the set of
  multiples of 5 by a power of two: $\set{5,10,20,40,\dots}$.  Two pigeons
  must assigned to some hole, and these are the required two selected
  numbers, since the quotient of the larger of these two by the smaller
  will be a power of two by definition.
\end{solution}
\fi

\end{problem}

\endinput
