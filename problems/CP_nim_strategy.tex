\documentclass[problem]{mcs}

\begin{pcomments}
  \pcomment{CP_nim_strategy}
  \pcomment{F14.ps2}
  \pcomment{edited ARM 9/23/14}
\end{pcomments}

\pkeywords{
  Nim
  xor
  binary
  strategy
  state_machines
}

%%%%%%%%%%%%%%%%%%%%%%%%%%%%%%%%%%%%%%%%%%%%%%%%%%%%%%%%%%%%%%%%%%%%%
% Problem starts here
%%%%%%%%%%%%%%%%%%%%%%%%%%%%%%%%%%%%%%%%%%%%%%%%%%%%%%%%%%%%%%%%%%%%%

\begin{problem}
Nim is a game played between two players with three piles of stones.
Players alternate removing stones.  A player picks a pile and removes
any positive number of stones.  The goal is to be the last player to
take a stone.

The winning strategy in Nim requires computing a Nim sum.  A
Nim sum of numbers $r$, $s$ and $t$ is computed by taking the binary
representation of the three numbers and combining these into a single
binary string by applying $\QXOR$ bit-wise.

For example, the Nim sum of $2, 7$ and $9$ is
\[\begin{array}{ccccr}
   &   & 1 & 0 & \text{(binary rep of 2)}\\
   & 1 & 1 & 1 & \text{(binary rep of 7)}\\
 1 & 0 & 0 & 1 & \text{(binary rep of 9)}\\
\hline
 1 & 1 & 0 & 0 & \text{(Nim sum)}
\end{array}\]

\bparts

\ppart Prove that if the Nim sum of the numbers of stones in each pile
is zero, then any move will result in Nim sum that is not zero.

\begin{solution}
When a player removes stones from a pile, the binary representation must change
in at least one digit, otherwise the number of stones would be the same.  At that
digit, the xor can no longer be zero since the bit has flipped.  The Nim sum is
then non-zero.
\end{solution}

\ppart Prove that if the Nim sum is not zero that it is always
possible to make the Nim sum zero with one move.

\begin{solution}
 Let the Nim sum be $t$.  Let $d$ be the position of the
  most significant bit of the Nim sum.  One of the piles must have
  this bit non-zero; pick one of them.  It has $s$ stones remaining.
  Remove stones so that it has $s \oplus t$ stones remaining, where
  $\oplus$ is the bit-wise $\QXOR$ of the binary representations of
  $s$ and $t$.  Such move is always possible because bit $d$ will be
  zero in $s \oplus t$, so it will be less than $s$.  This results in
  a Nim sum of zero because the Nim sum of the other two piles stays
  the same, $s \oplus t$ and the third pile has $s \oplus t$ stones
  left, so the Nim sum is zero.
\end{solution}

\ppart Conclude that if the game begins with a non-zero Nim sum, then
the first player has a winning strategy.

\begin{solution}
The first player can remove stones such that the Nim sum is zero.
Either this results in no stones left or the second player must make a
move that results in a non-zero Nim sum.  There must be at least one
stone left because the Nim sum is non-zero, so the second player has
not won.  This is the same situation as the start except there are
less stones.  The result follows from induction on the number of
stones.
\end{solution}

\eparts
\end{problem}

\endinput
