\documentclass[problem]{mcs}

\begin{pcomments}
  \pcomment{PS_walk_relation_composition}
  \pcomment{revised for walks by ARM, 3/10/11}
  \pcomment{from: digraph notes}
\end{pcomments}

\pkeywords{ binary_relation composition path_relation }

%%%%%%%%%%%%%%%%%%%%%%%%%%%%%%%%%%%%%%%%%%%%%%%%%%%%%%%%%%%%%%%%%%%%%
% Problem starts here
% %%%%%%%%%%%%%%%%%%%%%%%%%%%%%%%%%%%%%%%%%%%%%%%%%%%%%%%%%%%%%%%%%%%%

\begin{problem}
%Prove Lemma~\bref{lem:Rn-paths}, Namely,

Let $R$ be a binary relation on a set $A$.  Then $R^{(n)}$ denotes the
composition of $R$ with itself $n$ times.\inhandout{\footnote{ For
    binary relations $R: B\to C$ and $S: A \to B$, the
    \idx{composition} of $R$ with $S$ is the binary relation $(R
    \compose S): A\to C$ defined by the rule
\begin{equation}\label{rel_compose_def}
a \mrel{(R \compose S)} c \eqdef\ \exists b \in B.\, (a \mrel{S} b)
\QAND (b \mrel{R} c).
\end{equation}}}
Regarding $R$ as a digraph, let $R^n$ denote the length $n$ walk
relation $R$, that is,
\[
a \mrel{R^n} b \eqdef \mbox{there is a length $n$ walk from $a$ to
    $b$ in $R$}.
\]
Prove that
\begin{equation}\label{RnRpnp}
R^n = R^{(n)}
\end{equation}
for all $n \in \naturals$.

\begin{solution}

\begin{proof}
By induction on $n$ with equation~\ref{RnRpnp} as induction
hypothesis.

\textbf{Base case} $n=0$: $R^{(0)} = \ident{A}$ by convention, and
$R^0$ is the length-0 walk relation, so we have $a \mrel{R^0} b$ iff
$a = b$ iff $a \mrel{\ident{A}} b$ iff $a R^{(0)}b$.  So $R^0 =
R^{(0)}$.

\textbf{Inductive step:} Suppose~\eqref{RnRpnp} holds for some $n\ge 0$.
We want to prove it holds with ``$n$'' replaced by $n+1$."

First consider a length $n+1$ walk
\[
a=a_0\ \diredge{a_0}{a_1}\ a_1\ \diredge{a_1}{a_2} \dots \diredge{a_n}{a_{n+1}}\ a_{n+1} =b
\]
in $R$.  By induction hypothesis, we can assume that $a\mrel{R^{(n)}}
a_n$.  Also, we have by definition of walk in $R$ that $a_n \mrel{R}
b$.  Therefore,
\[
a \mrel{(R^{(n)} \compose R)} b
\]
by the definition~(\bref{rel_compose_def}) of composition.  So $a
\mrel{R^{(n+1)}} b$.  This shows that $R^{(n+1)} \subseteq R^{n+1}$.

Conversely, $a \mrel{R^{n+1}} b$, that is, there is a length-$(n+1)$
walk from $a$ to $b$.  This implies there is a length-$n$ walk from
$a$ to some vertex $a_n$, and an edge from $a_n$ to $b$.  That is, $a
\mrel{R^n} a_n$ and $a_n \mrel{R} b$.  Now by induction hypothesis, $a
\mrel{R^{(n)}} a_n$ so by definition of relational composition,
\[
a \mrel{R^{(n)} \compose R} b,
\]
so $a \mrel{R^{n+1}} b$ by the definition of $R^{n+1}$.  This shows
that $R^{n+1} \subseteq R^{(n+1)}$.

Hence, $R^{n+1} = R^{(n+1)}$, which is the exactly~\eqref{RnRpnp} with
``$n$'' replaced by $n+1$," completing the proof by induction.
\end{proof}

\end{solution}
\end{problem}

%%%%%%%%%%%%%%%%%%%%%%%%%%%%%%%%%%%%%%%%%%%%%%%%%%%%%%%%%%%%%%%%%%%%%
% Problem ends here
%%%%%%%%%%%%%%%%%%%%%%%%%%%%%%%%%%%%%%%%%%%%%%%%%%%%%%%%%%%%%%%%%%%%%

\endinput
