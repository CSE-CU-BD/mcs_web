\documentclass[problem]{mcs}

\begin{pcomments}
  \pcomment{MQ_tennis_match_partial_order}
  \pcomment{from S16.mid3, F11.ps6}
  \pcomment{don't use with MQ_reindeer_games_partial_order}
\end{pcomments}

\pkeywords{
  incomparable
  comparable
  maximal
  maximum
  minimal
  minimum
  chain
  linear
  topological_sort
  partial_order
  antichain
  maximal
  irreflexive
  reflexive
}

%%%%%%%%%%%%%%%%%%%%%%%%%%%%%%%%%%%%%%%%%%%%%%%%%%%%%%%%%%%%%%%%%%%%%
% Problem starts here
%%%%%%%%%%%%%%%%%%%%%%%%%%%%%%%%%%%%%%%%%%%%%%%%%%%%%%%%%%%%%%%%%%%%%

\begin{problem}
A tennis tournament among a set of players consists of a series of
two-player matches.  Usually the objective is to determine a single
best player.  The organizers of the Math for Computer Science
tournament want to do more: they want to find a linear ranking of all
the players.  To avoid controversy, they want to avoid the awkward
situation of having a sequence of players each of whom beats the next
player in the sequence and then having last player beat the first.  So
the organizers will keep a running record of who beat whom during the
tournament, and they never allow simultaneous matches whose outcomes
could lead to an awkward situation.

Knowledge of binary relations can help the organizers in arranging the
tournament.  Namely, at any stage of the tournament, the organizers
have a record of who lost to whom.  Mathematically, we can say that
there is a binary relation, $L$, on players where $p\mrel{L}q$ means
that player $p$ lost a match to player $q$.  No awkward situations
means that the positive length walk relation, $L^+$, is a strict
partial order.  Indicate which of the following partial order concepts
correspond to the properties~\eqref{unbeat}--\eqref{min_matches} of the
partial order $L^+$.

\begin{center}
Partial Order Concepts
\end{center}
\begin{quote}
comparable \quad incomparable \quad  maximum \quad  maximal \quad minimum\quad minimal\\
a chain \quad an antichain \quad reflexive\quad irreflexive\quad asymmetric\\
 a topological sort \quad a linear order
\end{quote}
%\item a linear order %*
%\item an empty partial order
%\item a maximal antichain 

\begin{staffnotes}
-2pts for each wrong answer.
\end{staffnotes}

\bparts
\ppart\label{unbeat} An unbeaten player so far is a \instatements{\brule{1in}}\insolutions{\underline{maximal}} element.

\iffalse
\begin{solution}
maximal
\end{solution}
\fi

\ppart\label{lostall} A player who has lost every match he was in is a
\instatements{\brule{1in}}\insolutions{\underline{minimal}} element.
\iffalse

\begin{solution}
minimal
\end{solution}
\fi

\ppart A player who is sure to rank first at the end of the tournament is
a \instatements{\brule{1in}}\insolutions{\underline{maximum}} element.

\iffalse
\begin{solution}
maximum
\end{solution}
\fi

\ppart A set of players whose rankings relative to each other are
unique is \instatements{\brule{1in}}\insolutions{\underline{a chain}}.

\iffalse
\begin{solution}
\insolutions{\underline{a chain}}
\end{solution}
\fi

\ppart Two players can be matched in the next stage of the
tournament only if they are \instatements{\brule{1in}}\insolutions{\underline{incomparable}} elements.

\iffalse
\begin{solution}
\insolutions{\underline{incomparable}}
\end{solution}
\fi

\ppart\label{finalrank} The final ranking at the end of the tournament will be
\instatements{\brule{1in}}\insolutions{\underline{a topological sort}}.

\iffalse
\begin{solution}
\insolutions{\underline{a topological sort}}
\end{solution}
\fi

\ppart No more matches are possible if and only if $L^+$ is
\instatements{\brule{3in}}\insolutions{\underline{a linear order}}.

\iffalse
\begin{solution}
\insolutions{\underline{a linear order}}
\end{solution}
\fi

\ppart A set of players any two of whom could be paired up to play
the next match is \instatements{\brule{1in}}\insolutions{\underline{an antichain}}.

\iffalse
\begin{solution}
\insolutions{\underline{an antichain}}
\end{solution}
\fi

\ppart\label{noloseself} The fact that no player loses to himself corresponds to $L^+$ being
\instatements{\brule{1in}}\insolutions{\underline{irreflexive}}.  

\iffalse
\begin{solution}
\insolutions{\underline{irreflexive}}
\end{solution}
\fi

%\eparts
%
%For the following parts, assume there are 256 players in the tournament.
%
%\bparts
%

\ppart\label{min_matches} If there are 256 players,
what is the smallest number of matches that could possibly have been
played in a completed tournament? \examrule
\begin{solution}
\textbf{255}.
\end{solution}

%\ppart Assuming each match takes an hour and matches are scheduled to be
%played simultaneously on the hour, what is the smallest number of hours the
%tournament could take?  \examrule
%\begin{solution}
%8
%\TBA{explanation}
%\end{solution}

\eparts

\end{problem}

\endinput
