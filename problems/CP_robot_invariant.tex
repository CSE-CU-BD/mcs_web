%CP__robot_invariant

\documentclass[problem]{mcs}

\begin{pcomments}
  \pcomment{from: state machines notes problem}
\end{pcomments}

\pkeywords{
  invariant
  integer_grid
b}

%%%%%%%%%%%%%%%%%%%%%%%%%%%%%%%%%%%%%%%%%%%%%%%%%%%%%%%%%%%%%%%%%%%%%
% Problem starts here
%%%%%%%%%%%%%%%%%%%%%%%%%%%%%%%%%%%%%%%%%%%%%%%%%%%%%%%%%%%%%%%%%%%%%


\begin{problem}
A robot moves on the two-dimensional integer grid.  It starts out at
$(0,0)$, and is allowed to move in any of these four ways:
\begin{enumerate}
\item (+2,-1)   Right 2, down 1
\item (-2,+1)   Left 2, up 1 
\item (+1,+3) 
\item (-1,-3) 
\end{enumerate}

Prove that this robot can never reach (1,1).
\begin{solution}

A simple preserved invariant that does the job is defined on
  states $(i,j)$ by the predicate: $i + 2j$ is an integer multiple of $7$.
\end{solution}

\end{problem}

%%%%%%%%%%%%%%%%%%%%%%%%%%%%%%%%%%%%%%%%%%%%%%%%%%%%%%%%%%%%%%%%%%%%%
% Problem ends here
%%%%%%%%%%%%%%%%%%%%%%%%%%%%%%%%%%%%%%%%%%%%%%%%%%%%%%%%%%%%%%%%%%%%%

\endinput

