\documentclass[problem]{mcs}

\begin{pcomments}
  \pcomment{CP__robot_invariant}
  \pcomment{from: F09.state machines notes problem}
  \pcomment{similar to PS_robot_on_2D_grid}
\end{pcomments}

\pkeywords{
  state_machines
  unreachable_states
  invariant
  integer_grid
}

%%%%%%%%%%%%%%%%%%%%%%%%%%%%%%%%%%%%%%%%%%%%%%%%%%%%%%%%%%%%%%%%%%%%%
% Problem starts here
%%%%%%%%%%%%%%%%%%%%%%%%%%%%%%%%%%%%%%%%%%%%%%%%%%%%%%%%%%%%%%%%%%%%%


\begin{problem}
A robot moves on the two-dimensional integer grid.  It starts out at
$(0,0)$, and is allowed to move in any of these four ways:
\begin{enumerate}
\item $(+2,-1)$: right 2, down 1
\item $(-2,+1)$: left 2, up 1
\item $(+1,+3)$
\item $(-1,-3)$
\end{enumerate}

Prove that this robot can never reach $(1,1)$.
\begin{solution}

\textbf{Solution 1:}
Denote the robot's position by $(x,y)$.  Observe that the quantity $z=x+2y$ does not 
change when the robot undergoes an atomic movement of type 1 or type 2.  A type 3
movement increases $z$ by $1+2(3)=7$, while a type 4 movement decreases $z$ by 7.
Thus, the predicate 
\[
P(x,y)\eqdef 7\mid x+2y 
\]
is a preserved invariant for the robot.
Since $P(0,0)$ and the robot starts at the origin, therefore by the Invariant Principle,
$P$ must hold for all reachable states $(x,y)$ of the robot.  Since $P(1,1)$ is false,
therefore the robot cannot reach $(1,1)$.

(Alternatively, considering type 3 and type 4 movements before type 1 and 2 movements
leads to the preserved invariant 
\[
Q(x,y) \eqdef 7\mid 3x-y,
\] 
which also does the trick.)\\

\textbf{Solution 2:}
If the robot starts at the origin and makes $a_k$ atomic movements of
type $k$, $k\in\{1,2,3,4\}$, its position will be
\[
(2a_1-2a_2+a_3-a_4,-a_1+a_2+3a_3-3a_4)
\]
Letting $b_1 = a_1 - a_2$ and $b_2 = a_3 - a_4$, we have $(x,y) = (2b_1+b_2,-b_1+3b_2)$,
where $b_1$ and $b_2$ must be integers.
Now, witness that $x+2y = 7b_2$.  Thus, the robot's position $(x,y)$ must always
be such that $7\mid x+2y$.  (A preserved invariant that true wherever the robot goes!)
Since this requirement is not met with $(x,y)=(1,1)$, therefore the robot cannot reach
$(1,1)$.

(Alternatively, witness that $3x-y=7b_1$, so $7\mid 3x-y$ must also hold for
any location of the robot, yet it does not hold for $(x,y)=(1,1)$.)\\

\textbf{Solution 3:}
From Solution 2, the robot's position must always be 
\[
(2a_1-2a_2+a_3-a_4,-a_1+a_2+3a_3-3a_4)
\]
for some $a_1,a_2,a_3,a_4\in\naturals$.
Now, the system
\begin{align*}
2a_1-2a_2+a_3-a_4&=1\\
-a_1+a_2+3a_3-3a_4&=1
\end{align*}
has no solutions $(a_1,a_2,a_3,a_4)\in\naturals^4$.  (It has solutions in
$\reals^4$ given by $(a_1,a_2,a_3,a_4) = (\frac{2}{7}+s,s,\frac{3}{7}+t,t)$, 
for $s,t\in\reals$.)  Thus the robot, using only the four permissible atomic
movements, cannot possibly reach $(1,1)$. 

\end{solution}

\end{problem}

%%%%%%%%%%%%%%%%%%%%%%%%%%%%%%%%%%%%%%%%%%%%%%%%%%%%%%%%%%%%%%%%%%%%%
% Problem ends here
%%%%%%%%%%%%%%%%%%%%%%%%%%%%%%%%%%%%%%%%%%%%%%%%%%%%%%%%%%%%%%%%%%%%%

\endinput

