\documentclass[problem]{mcs}

\begin{pcomments}
  \pcomment{FP_combinatorial_nchoose3}
  \pcomment{ARM 11/20/05}
\end{pcomments}

\pkeywords{
  combinatorial
  sum
  chooose
  binomial
}

\begin{problem}
Give a combinatorial proof of
\[
%\sum_{m=2}^{n+2}
1\cdot 2 + 2\cdot 3 +3 \cdot 4 + \cdots + (n-1)\cdot n = 2\binom{n+1}{3}
\]

\hint Classify sets of three numbers from the integer interval
$\Zintv{0}{n}$ by their maximum element.

\begin{solution}
Let $T$ be the subsets of size 3 of the integer interval $\Zintv{0}{n}$.
Since there are $n+1$ elements in the interval, we have
\[
\card{T} = \binom{n+1}{3}.
\]

Now classify the subsets by their maximum element.  That is, let
\[
T_m \eqdef \set{\set{x,y,z} \subseteq \Zintv{0}{n} \suchthat x,y,z\text{
    are distinct integers}\text{ and } \max\set{x,y,z} = m}.
\]
So
\[
T = \lgunion_{m=2}^{n} T_m,
\]
and since the sets in the union are disjoint,
\[
\card{T} = \sum_{m=2}^{n} \card{T_m}.
\]
But, once the maximum element $m$ is chosen, there are $\binom{m}{2}$
possible choices for the remaining two elements, so
\[
\card{T_m} = \binom{m}{2}.
\]
Therefore,
\[
\binom{n+1}{3} = \sum_{m=2}^{n} \card{T_m} = \binom{2}{2} + \binom{3}{2} + \cdots + \binom{n}{2}.
\]
Mow multiplying both sides of this last equation by 2 completes the
proof.

\end{solution}

\end{problem}

\endinput


