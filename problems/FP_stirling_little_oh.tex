\documentclass[problem]{mcs}

\begin{pcomments}
  \pcomment{FP_stirling_little_oh}
  \pcomment{ARM 4/23/16}
\end{pcomments}

\pkeywords{
  asymptotics
  Stirling
  factorial
  polynomial
  exponential
}

%%%%%%%%%%%%%%%%%%%%%%%%%%%%%%%%%%%%%%%%%%%%%%%%%%%%%%%%%%%%%%%%%%%%%
% Problem starts here
%%%%%%%%%%%%%%%%%%%%%%%%%%%%%%%%%%%%%%%%%%%%%%%%%%%%%%%%%%%%%%%%%%%%%

\begin{problem}

\bparts  Let $p(n)$ be a polynomial and $a,b$ real numbers such that $1 \leq a < b$.

\ppart\label{pnanobn} Prove that
\[
p(n) \cdot a^n = o(b^n).
\]

\begin{solution}
We know
\[
p(n) = o(1+ \epsilon)^n
\]
for all polynomials, $p$, and real numbers $\epsilon > 0$, by
Lemma~\bref{xbax}.

Hence
\begin{equation}\label{pnano}
p(n)\cdot a^n = o(\paren{(1+ \epsilon)^n \cdot a^n}.
\end{equation}

But $b = (1+\epsilon) a$ where $\epsilon = (b/a - 1) > 0$.  Therefore
\[
(1+ \epsilon)^n \cdot a^n = ((1+ \epsilon)a)^n = b^n,
\]
and so
\[
p(n)\cdot a^n = o(b^n),
\]
by~\eqref{pnano}.
\end{solution}

\examspace[3in]

\ppart  Prove that
\[
n! = o\paren{\frac{n}{3}}^n.
\]

\hint Let $a = e$ and $b = 3/e$.

\begin{solution}
\begin{align*}
n! & \sim \sqrt{2 \pi n}\paren{\frac{n}{e}}^n  & \text{(Stirling)}\\
   & = o\paren{n\paren{\frac{n}{e}}^n}} & (\sqrt{2 \pi n} = o(n))\\
   & = o\paren{\paren{\frac{3}{e}\frac{n}{e}}^n} & \text{(by part~\eqref{pnano} since $3/e > 1$)} \\
   & = \paren{\frac{n}{3}}^n.
\end{align*}
\end{solution}

\eparts
\end{problem}

%%%%%%%%%%%%%%%%%%%%%%%%%%%%%%%%%%%%%%%%%%%%%%%%%%%%%%%%%%%%%%%%%%%%%
% Problem ends here
%%%%%%%%%%%%%%%%%%%%%%%%%%%%%%%%%%%%%%%%%%%%%%%%%%%%%%%%%%%%%%%%%%%%%

\endinput
