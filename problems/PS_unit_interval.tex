\documentclass[problem]{mcs}

\begin{pcomments}
  \pcomment{PS_unit_interval}
  \pcomment{from: S09.ps2, revised by ARM 9/19/09}
\end{pcomments}

\pkeywords{
  bijection
  surjections
  real plane
  unit interval
}

%%%%%%%%%%%%%%%%%%%%%%%%%%%%%%%%%%%%%%%%%%%%%%%%%%%%%%%%%%%%%%%%%%%%%
% Problem starts here
%%%%%%%%%%%%%%%%%%%%%%%%%%%%%%%%%%%%%%%%%%%%%%%%%%%%%%%%%%%%%%%%%%%%%

\begin{problem} In this problem you will prove a fact that may surprise
  you---or make you even more convinced that set theory is nonsense: the
  half-open unit interval is actually the ``\emph{same size}'' as the
  nonnegative quadrant of the real plane!\footnote{The half open unit
    interval, $(0,1]$, is $\set{r \in \reals
      \suchthat 0 < r \leq 1}$.  Similarly, $[0,\infty) \eqdef \set{r \in
      \reals \suchthat r \geq 0}$.}  Namely, there is a bijection
  from $(0,1]$ to $[0,\infty) \cross [0,\infty)$.

\bparts

\ppart\label{real-quadrant} Describe a bijection from $(0,1]$ to $[0,
\infty)$.

\hint $1/x$ almost works.

\begin{solution}
$f(x) \eqdef 1/x$ defines a bijection from $(0,1]$ to $[1, \infty)$, so
$g(x) \eqdef f(x) -1$ does the job.
\end{solution}

\ppart\label{surjection2a} An infinite sequence of the decimal digits
$\set{\STR{0},\STR{1},\dots,\STR{9}}$ will be called
\emph{long} if it does not end with all 0's.  An equivalent way to say
this is that a long sequence is one that has infinitely many
occurrences of nonzero digits.  Let $L$ be the set of all such long
sequences.  Describe a bijection from $L$ to the half-open real
interval $(0,1]$.

\hint Put a decimal point at the beginning of the sequence.

\begin{solution}
Putting a decimal point in front of a long sequence defines a
bijection from $L$ to $(0,1]$.  This follows because every real number
    in $(0,1]$ has a unique long decimal expansion.  Note that if we
      didn't exclude the non-long sequences, namely, those sequences
      ending with all zeroes, this wouldn't be a bijection.  For
      example, putting a decimal point in front of the sequences
      \STR{1000}\dots and \STR{099999}\dots maps both sequences
      to the same real number, namely, $1/10$.
\end{solution}

\ppart\label{surjection2b} Describe a surjective function from $L$ to
$L^2$ that involves alternating digits from two long sequences. \hint
The surjection need not be total.

\begin{editingnotes}
This would be clearer asking for a total injective function from $L^2
\to L$.
\end{editingnotes}

\begin{solution}
Given any long sequence $s = x_0, x_1, x_2, \dots$, let
\[
h_0(s) \eqdef x_0,x_2, x_4, \dots
\]
be the sequence of digits in even positions.  Similarly, let
\[
h_1(s) \eqdef x_1, x_3, x_5, \dots
\]
be the sequence of digits in odd positions.  Then $h$ is a surjective
function from $L$ to $L^2$, where
\begin{equation}
h(s) \eqdef \begin{cases}
  (h_1(s),h_2(s)), &  \text{if $h_1(s) \in L$ and $h_2(s) \in L$,}\\
  \text{undefined}, & \text{otherwise.}
 \end{cases}
\end{equation}

\end{solution}

\ppart\label{surjection2c} Prove the following lemma and use it to
conclude that there is a bijection from $L^2$ to $(0,1]^2$.

\begin{lemma}
  Let $A$ and $B$ be nonempty sets.  If there is a bijection from $A$
  to $B$, then there is also a bijection from $A \times A$ to $B \times
  B$.
\end{lemma}

\begin{solution}
\begin{proof}
  Suppose $f : A \to B$ is a bijection.  Let $g : A^2 \to B^2$ be the
  function defined by the rule $g(x,y) = (f(x),f(y))$.  It is easy to show
  that $g$ is a bijection:

\begin{itemize}
\item \textbf{$g$ is total}: Since $f$ is total, $f(a_1)$ and $f(a_2)$ exist
$\forall a_1, a_2 \in A$ and so $g(a_1,a_2) = (f(a_1), f(a_2))$ also exists.

\item \textbf{$g$ is surjective}: Since $f$ is surjective, for any $b_i
  \in B$ there exists $a_i \in A$ such that $b_i = f(a_i)$.  So for any
  $(b_1,b_2)$ in $B^2$, there is a pair $(a_1,a_2) \in A^2$ such that
  $g(a_1,a_2) \eqdef (f(a_1),f(a_2)) = (b_1,b_2)$.  This shows that $g$ is
  a surjection.

\item \textbf{$g$ is injective}:
\begin{align*}
  g(a_1,a_2) = g(a_3,a_4) &\qiff (f(a_1),f(a_2)) = (f(a_3),f(a_4))
       & \text{(by def of $g$)}\\
  &\qiff f(a_1) = f(a_3) \QAND f(a_2) = f(a_4)\\
  &\qiff a_1 =a_3 \QAND a_2 = a_4 \text{(since $f$ is injective)}\\
  & (a_1,a_2) = (a_3,a_4),
 \end{align*}
which confirms that $g$ is injective.
\end{itemize}
\end{proof}

Since it was shown in part~\eqref{surjection2a} that there is a bijection
from $L$ to $(0,1]$, an immediate corollary of the Lemma is
that there is a bijection from $L^2$ to $(0,1]^2$.
\end{solution}


\ppart\label{01012} Conclude from the previous parts that there is a
surjection from $(0,1]$ and $(0,1]^2$.  Then appeal to the
Schr\"oder-Bernstein Theorem to show that there is actually a bijection
from $(0,1]$ and $(0,1]^2$.

\begin{solution}
  There is a bijection between $(0,1]$ and $L$ by
  part~\eqref{surjection2a}, a surjective function from $L$ to $L^2$
  by part~\eqref{surjection2b}, and a bijection from from $L^2$ to
  $(0,1]^2$ by part~\eqref{surjection2c}.  These surjections compose to yield
  a surjection from $(0,1]$ to $(0,1]^2$.

  Conversely, there is obviously a surjective function $f: (0,1]^2 \to (0,1]$, namely
  \[
   f(\ang{x,y}) \eqdef x.
  \]

  The Schr\"oder-Bernstein Theorem now implies that there is a bijection
  from $(0,1]$ to $(0,1]^2$.
\end{solution}

\ppart Complete the proof that there is a bijection from $(0,1]$ to
$[0,\infty)^2$.

\begin{solution}
  There is a bijection from $(0,1]$ to $(0,1]^2$ by part~\eqref{01012},
  and there is a bijection from $(0,1]^2$ to $[0,\infty)^2$ by
  part~\eqref{real-quadrant} and the Lemma.  These
  bijections compose to yield a bijection from $(0,1]$ to $[0,\infty)^2$.
\end{solution}

\eparts
\end{problem}

%%%%%%%%%%%%%%%%%%%%%%%%%%%%%%%%%%%%%%%%%%%%%%%%%%%%%%%%%%%%%%%%%%%%%
% Problem ends here
%%%%%%%%%%%%%%%%%%%%%%%%%%%%%%%%%%%%%%%%%%%%%%%%%%%%%%%%%%%%%%%%%%%%%

\endinput
