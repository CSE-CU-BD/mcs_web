\documentclass[problem]{mcs}

\begin{pcomments}
  \pcomment{PS_pulverizer_machine}
  \pcomment{by ARM 2/26/11}
\end{pcomments}

\pkeywords{
  pulverizer
  Euclidean
  algorithm
  state_machine
  correctness
  termination
  extended_Euclidean
}

%%%%%%%%%%%%%%%%%%%%%%%%%%%%%%%%%%%%%%%%%%%%%%%%%%%%%%%%%%%%%%%%%%%%%
% Problem starts here
%%%%%%%%%%%%%%%%%%%%%%%%%%%%%%%%%%%%%%%%%%%%%%%%%%%%%%%%%%%%%%%%%%%%%

\begin{problem}
Define the Pulverizer State machine to have:
\begin{align*}
\text{states} & \eqdef \naturals^7\\
\text{start state} & \eqdef (a, b, 0, 1, 1, 0) 
             & \text{(where $a \ge b >0$)}\\
\text{transitions} & \eqdef (x,  y,  s,  t,  u,  v) \movesto\\
     &\qquad (y,\ \rem{x}{y},\ u - sq,\ v - tq,\ s,\ t)
             & \text{(for $q = \qcnt{x}{y}, y>0$)}.
\end{align*}

\bparts

\ppart Show that the following properties are preserved invariants of
the Pulverizer machine:
\begin{align}
\gcd(x,y) & =  \gcd(a,b), \label{XY}\\
sa+tb & =  y,\text{ and }\label{SaTb}\\
ua+vb & =  x. \label{uaVb}
\end{align}

\begin{solution}
To verify that these are preserved invariants, suppose
\[
(x,  y,  s,  t,  u,  v) \movesto (x',  y',  s',  t',  u',  v').
\]
Note that~\eqref{XY} is the same one we observed for the Euclidean
algorithm.  This leaves proving that~\eqref{SaTb} and~\eqref{uaVb}
hold for the new state $x',y',s',t',u',v'$.

Now according to the procedure, $u'=s,v'=t,x'=y$, so~\eqref{uaVb} holds
for $u',v',x'$ because of~\eqref{SaTb} for $s,t,y$.  Also, 
\[
s'= u - qs,\quad t'= v - qt,\quad y' = x - qy
\]
where $q = \qcnt(x,y)$,
so
\[
s'a+t'b = (u-qs)a + (v-qt)b =ua+vb - q(sa+tb) = x - qy = y',
\]
and therefore~\eqref{SaTb} holds for $s',t',y'$.
\end{solution}

\ppart Conclude that the Pulverizer machine is partially correct.

\begin{solution}
We claim that on termination, the values of $s$ and $t$ at
termination are the desired coefficients, that is, 
\[
\gcd(a,b) = sa+tb.
\]

To prove this, we first check that all three preserved invariants are
true just before the first time around the loop.  Namely, at the
start:
\begin{align*}
x      =a, y=b,s=0, t& =1 & \mbox{so}\\
sa+tb = 0a+1b=b& =y & \mbox{confirming~\eqref{SaTb}.}
\end{align*}
Also,
\begin{align*}
u     & =1, v=0, & \mbox{so} \\
ua+vb & = 1a+0b=a =x & \mbox{confirming~\eqref{uaVb}.  }
\end{align*}
Now by the Invariant Principle, they are true at termination.  But at
termination, $y \divides x$ so preserved invariants~\eqref{XY}
and~\eqref{SaTb} imply
\[
\gcd(a,b) = \gcd(x,y) = y = sa + tb.
\]
so we have the desired coefficients $s$ and $t$.
\end{solution}

\ppart Explain why the machine terminates after at most the same
number of transitions as the Euclidean algorithm.

\begin{solution}
Note that ${x,y}$ follows the transition rules of the Euclidean
algorithm state machine given in equation~(\bref{euclid_transition}),
except that this extended machine stops one step sooner.
\end{solution}

\eparts

\end{problem}


%%%%%%%%%%%%%%%%%%%%%%%%%%%%%%%%%%%%%%%%%%%%%%%%%%%%%%%%%%%%%%%%%%%%%
% Problem ends here
%%%%%%%%%%%%%%%%%%%%%%%%%%%%%%%%%%%%%%%%%%%%%%%%%%%%%%%%%%%%%%%%%%%%%

\endinput
