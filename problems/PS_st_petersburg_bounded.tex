\documentclass[problem]{mcs}

\begin{pcomments}
  \pcomment{PS_st_petersburg_bounded}
  \pcomment{related to CP_st_petersburg, CP_fair_st_petersburg}
  \pcomment{ARM 5/28/12}
\end{pcomments}

\pkeywords{
  st_peterburg
  paradox
  expectation
}

%%%%%%%%%%%%%%%%%%%%%%%%%%%%%%%%%%%%%%%%%%%%%%%%%%%%%%%%%%%%%%%%%%%%%
% Problem starts here
%%%%%%%%%%%%%%%%%%%%%%%%%%%%%%%%%%%%%%%%%%%%%%%%%%%%%%%%%%%%%%%%%%%%%

\begin{problem}


A gambler bets \$10 on ``red'' at a roulette table (the odds of red are
18/38, slightly less than even) to win \$10.  If he wins, he gets
back twice the amount of his bet and he quits.  Otherwise, he doubles his
previous bet and continues.

Suppose you had the good fortune to gamble against a fair roulette
wheel.  Then whatever your bet on a spin of the wheel, you are equally
likely to win or lose, and your expected win is 0.  This also means
that the expected win after any given number of spins remains zero, so
even playing the St. Peterburg strategy it seems your expected win
would be 0.


\end{problem}


%%%%%%%%%%%%%%%%%%%%%%%%%%%%%%%%%%%%%%%%%%%%%%%%%%%%%%%%%%%%%%%%%%%%%
% Problem ends here
%%%%%%%%%%%%%%%%%%%%%%%%%%%%%%%%%%%%%%%%%%%%%%%%%%%%%%%%%%%%%%%%%%%%%

\endinput
