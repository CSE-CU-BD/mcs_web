\documentclass[problem]{mcs}

\newcommand{\pairset}[2]{\text{pair}(#1,#2)}

\begin{pcomments}
  \pcomment{MQ_pair_predicate}
  \pcomment{related to CP_set_pairing}
  \pcomment{ARM 2/12/15}
\end{pcomments}

\pkeywords{
  pair
  ordered_pair
  logic
  sets
  set_theory
  predicate
  formula
}

\begin{problem}

\inhandout{ A \emph{formula of \idx{set theory}} is a predicate
  formula that only uses the predicate ``$x \in y$.''  The domain of
  discourse is the collection of sets, and ``$x \in y$'' is
  interpreted to mean that the set $x$ is one of the elements in the
  set $y$.

For example, since $x$ and $y$ are the same set iff they have the same
members, here's how we can express equality of $x$ and $y$ with a
formula of set theory:
\begin{equation}\label{x=xAz}
(x = y) \eqdef\ \forall z.\, (z \in x\ \QIFF\ z \in y).
\end{equation}
}

\bparts

\ppart Explain how to write a formula $\text{Members}(p,a,b)$ of set
theory\inbook{\footnote{See Section~\bref{ZFC_sec}.}} that means $p =
\set{a,b}$.

\hint Say that everything in $p$ is either $a$ or $b$.  It's OK to use
subformulas of the form ``$x = y$,'' since we can regard ``$x = y$''
as an abbreviation for a genuine set theory formula.

\begin{solution}
\[
a \in p \QAND b \in p \QAND \forall z.\, (z \in p \QIMPLIES (z = a \QOR z = b)).
\]
Alternatively,
\[
\forall z.\, (z \in p \QIFF (z = a \QOR z = b)).
\]
\end{solution}

\eparts

A \emph{\idx{pair}} $(a,b)$ is simply a sequence of length two whose
first item is $a$ and whose second is $b$.  \iffalse
\footnote{Some other texts refer to pairs as \emph{ordered} pairs.}
\fi
Sequences are a basic mathematical data type we take for granted, but
when we're trying to show how all of mathematics can be reduced to set
theory, we need a way to represent the ordered pair $(a,b)$ as a set.
One way that will work\footnote{Some similar ways that don't work are
  described in problem~\bref{CP_set_pairing}.} is to represent $(a,b)$
as
\[
\pairset{a}{b} \eqdef \set{a,\set{a,b}}.
\]

\bparts

\ppart Explain how to write a formula $\text{Pair}(p, a, b)$, of set
theory \inbook{\footnote{See Section~\bref{ZFC_sec}.}} that means $p =
\pairset{a}{b}$.

\hint Now it's OK to use subformulas of the form
``$\text{Members}(p,a,b)$.''

\begin{solution}
\[
\exists q.\, \text{Members}(q,a,b) \QAND \text{Members}(p,a,q).
\]
\end{solution}

\ppart Explain how to write a formula $\text{Second}(p, b)$, of set
theory that means $p$ is a pair whose second item is $b$.

\begin{solution}
\[
\exists a.\, \text{Pair}(p, a, b).
\]

\end{solution}

\eparts

\end{problem}

\endinput

