\documentclass[problem]{mcs}

\begin{pcomments}
  \pcomment{TP_remove_connected}
  \pcomment{subsumes PS_connected_induction}
  \pcomment{F12.rec6 by clumsuy induction; commented out of F12 final}
  \pcomment{ARM 4/9/14}
\end{pcomments}

\pkeywords{
  graph
  connected
  component
  induction
}

%%%%%%%%%%%%%%%%%%%%%%%%%%%%%%%%%%%%%%%%%%%%%%%%%%%%%%%%%%%%%%%%%%%%%
% Problem starts here
%%%%%%%%%%%%%%%%%%%%%%%%%%%%%%%%%%%%%%%%%%%%%%%%%%%%%%%%%%%%%%%%%%%%%

\begin{problem}
A simple graph $G$ is \emph{2-removable} iff it contains two vertices
$v \neq w$ such that $G-v$ is connected, and $G-w$ is also connected.
Prove that every connected graph with at least two vertices is
2-removable.

\begin{staffnotes}
\hint Consider a maximum length path.
\end{staffnotes}

\begin{solution}
If $G$ is connected and has at least two vertices, the maximum length
of a path in $G$ will be at least two, and in particular, the
endpoints of a maximum length path must be different.

Let $v$ be one of the endpoints of a maximum length path, $\walkv{p}$.
We need only show that $G-v$ is connected.

Now every vertex adjacent to $v$ must already be on $\walkv{p}$---or
else $\walkv{p}$ could be lengthened.  So for any path $\walkv{q}$
that goes through $v$, we can find another path in $G-v$ with the same
endpoints by replacing the edges entering and leaving $v$ by a subpath
of $\walkv{p}$ that does not include $v$.  So $G-v$ remains connected.

Note that the argument applies to the endpoints of any maxim\emph{al}
path---meaning a path that is not a subpath of any other path---not
just maximum length paths.  In fact, it's not hard to see that $G-v$ is
connected iff $v$ is the endpoint of a maximal path.
\end{solution}

\end{problem}


%%%%%%%%%%%%%%%%%%%%%%%%%%%%%%%%%%%%%%%%%%%%%%%%%%%%%%%%%%%%%%%%%%%%%
% Problem ends here
%%%%%%%%%%%%%%%%%%%%%%%%%%%%%%%%%%%%%%%%%%%%%%%%%%%%%%%%%%%%%%%%%%%%%

\endinput

