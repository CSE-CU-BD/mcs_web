\documentclass[problem]{mcs}

\begin{pcomments}
  \pcomment{CP_schedule_P4}
  \pcomment{subsumes FP_schedule_B4 7 FP_schedule_P4_conflict2}
  \pcomment{ARM 4/5/16}
  \pcomment{S16.mid3:FP version}
\end{pcomments}

\pkeywords{
  partial_orders
  scheduling
  constraints
  parallel_time
}

%%%%%%%%%%%%%%%%%%%%%%%%%%%%%%%%%%%%%%%%%%%%%%%%%%%%%%%%%%%%%%%%%%%%%
% Problem starts here
%%%%%%%%%%%%%%%%%%%%%%%%%%%%%%%%%%%%%%%%%%%%%%%%%%%%%%%%%%%%%%%%%%%%%

\begin{problem} 
Answer the following questions about the powerset,
$\power(\set{1,2,3,4})$, partially ordered by the strict subset relation
$\subset$.

\bparts

\iffalse
\ppart Prove that $\sqsubseteq$ is a partial order on $B^4$.

\examspace[2.0in]
\fi

\ppart\label{maxis5}
Give an example of a maximum length chain.

\examspace[1.0in]

\begin{solution}
\[
\emptyset, \set{1}, \set{1,2}, \set{1,2,3}, \set{1,2,3,4}.
\]

\end{solution}
\ppart Give an example of an antchain of size 6.

\examspace[1.0in]

\begin{solution}
\[
\set{1,2},\set{2,3},\set{3,4},\set{1,3},\set{2,4},\set{1,4}.
\]
\end{solution}

\ppart Describe an example of a topological sort of $\power(\set{1,2,3,4})$.

\examspace[1.0in]

\begin{solution}
The empty set, followed by the four 1-element sets in any order,
followed by the six 2-element sets in any order, followed by the
four 3-element sets in any order followed by $\set{1,2,3,4}$.
\end{solution}

\ppart Suppose the partial order describes scheduling constraints on
16 tasks.  That is, if
\[
A \subset B \subseteq \set{1,2,3,4},
\]
then $A$ has to be completed before $B$ starts.\footnote{As usual, we
  assume each task requires one time unit to complete.}  What is the
minimum number of processors needed to complete all the tasks in
minimum parallel time?\hfill\examrule

Prove it.

\examspace[3.0in]

\begin{solution}
\textbf{4}.

A minimum time schedule takes max chain length steps, which by
part~\eqref{maxis5} is 5.  There is a unique minimum task,
$\emptyset$, which must come first and a unique maximum task,
$\set{1,2,3,4}$, which must come last; this leaves 14 tasks which
require at least $\ceil{14/k}$ more parallel steps with $k$
processors.  So for min time, we need $\ceil{14/k} \leq 3$, which
implies that $k \ge 4$.  Moreover, there is a 4-processor schedule
that takes 5 steps:

For example, a length-6 4-processor schedule is:
\begin{align*}
\emptyset\\
1,\ 2,\ 3,\ 4 \\
12,\ 23, 13,\ 14\\
24,\ 34,\ 123\\
234,\ 124,\ 134\\
1234.
\end{align*}
\end{solution}

\ppart What is the length of a minimum time \textbf{3}-processor
schedule?\hfill\examrule

Prove it.

\examspace[2.0in]

\begin{solution}
\textbf{7}.  For example, a length-7 3-processor schedule is:
\begin{align*}
\emptyset\\
1,\ 2,\ 3\\
4,\ 12,\ 23\\
34,\ 14,\ 24\\
13,\ 124,\ 234\\
123,\ 134\\
1234.
\end{align*}

Moreover, no shorter schedule is possible: there is a unique minimum
task, $\emptyset$, which must come first and a unique maximum task,
$\set{1,2,3,4}$, which must come last; this leaves 14 tasks which
require at least $\ceil{14/3} = 5$ more parallel steps.
\end{solution}

\eparts

\end{problem} 


%%%%%%%%%%%%%%%%%%%%%%%%%%%%%%%%%%%%%%%%%%%%%%%%%%%%%%%%%%%%%%%%%%%%%
% Problem ends here
%%%%%%%%%%%%%%%%%%%%%%%%%%%%%%%%%%%%%%%%%%%%%%%%%%%%%%%%%%%%%%%%%%%%%

\endinput
