\documentclass[problem]{mcs}

\begin{pcomments}
  \pcomment{CP_planar_structural_induction}
  \pcomment{from: S09.cp7m}
\end{pcomments}

\pkeywords{
  structural_induction
  recursive_data
  planar_graphs
  planar_embeddings
}

%%%%%%%%%%%%%%%%%%%%%%%%%%%%%%%%%%%%%%%%%%%%%%%%%%%%%%%%%%%%%%%%%%%%%
% Problem starts here
%%%%%%%%%%%%%%%%%%%%%%%%%%%%%%%%%%%%%%%%%%%%%%%%%%%%%%%%%%%%%%%%%%%%%

\begin{problem}\label{planar-structural}
Prove the following assertions by structural induction on the definition
of planar embedding.

\bparts

\ppart\label{structind-twice}
In a planar embedding of a graph, each edge is traversed a total of two
times by the faces of the embedding.

\begin{solution}
\begin{proof}

The induction hypothesis is that if $\mathcal{E}$ is a planar
embedding of a graph, then each edge is traversed exactly twice by the
faces of $\mathcal{E}$.

\textbf{Base case:} There is one vertex and no edges, so this case holds vacuously.

\textbf{Constructor case:} (face-splitting) The only change is that one
face of $\mathcal{E}$ splits into two new faces, each traversing the new
edge once.

\textbf{Constructor case:} (bridge between two connected graphs) The only
change is that two faces merge into one face that makes two traversals of
the new bridging edge.  So the traversals of other edges are unchanged,
and the new edge is traversed twice by the new face.

So in any case, all edges of $\mathcal{E}$ are traversed exactly
twice.  This completes the proof of the Constructor case.  We conclude
by structural induction that for all planar embeddings, $\mathcal{E}$,
then each edge is traversed exactly twice by the faces of
$\mathcal{E}$.

\end{proof}
\end{solution}

\ppart\label{structind-face-length} In a planar embedding of a connected
graph with at least three vertices, each face is of length at least three.

\begin{solution}
\begin{proof}
The induction hypothesis is that if $\mathcal{E}$ is a planar embedding of
a graph with at least three vertices, then all faces in $\mathcal{E}$ are
of length at least three.

\textbf{Base case:} There is one vertex, so this case holds vacuously.

\textbf{Constructor case:} (face-splitting) An edge $\edge{a}{b}$ is
added between nonadjacent vertices $a,b$ on the same face.  This face
is replaced by two new faces of the form $abc\dots a$ and $abd\dots a$
where $c \neq d$ are vertices different from $a$ and $b$.  So both new
faces are of length at least 3; no other faces change.

\textbf{Constructor case:} (bridge between two connected graphs)

\textbf{case 1:} (both graphs have one vertex).  Connecting these
graphs with a bridge gives a graph with fewer than three vertices, so
this case holds vacuously.

\textbf{case 2:} (one graph has exactly two vertices and the other has
at most two vertices).  Connecting these graphs with a bridge yields a
line graph of length two or three whose unique embedding is a cycle of
length four or six going from one end of the graph to the other and
back.  In any case, the one face has length more than three.

\textbf{case 3:} (one graph has at most two vertices and the other has
at least three vertices).  Connecting replaces the face of the vertex
graph with at most two vertices and a face of the other graph with a
face of length at least $2 + 3 = 5$, and leaves all other faces
unchanged.  So all faces are indeed of length at least three.

\textbf{case 4:} (both graphs have at least three vertices).
Connecting replaces two faces of length at least three by a single
face of length at least $2 + 3 +3 = 7$, and leaves all other faces
unchanged.  So all faces are indeed of length at least three.

So in any case, all faces of connected planar embedding of graphs with at
least three vertices are indeed of length at least three.  This completes
the proof of the Constructor case and the structural induction.

\end{proof}

\end{solution}

\eparts
\end{problem}

%%%%%%%%%%%%%%%%%%%%%%%%%%%%%%%%%%%%%%%%%%%%%%%%%%%%%%%%%%%%%%%%%%%%%
% Problem ends here
%%%%%%%%%%%%%%%%%%%%%%%%%%%%%%%%%%%%%%%%%%%%%%%%%%%%%%%%%%%%%%%%%%%%%

\endinput
