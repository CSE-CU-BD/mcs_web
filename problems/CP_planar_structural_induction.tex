\documentclass[problem]{mcs}

\begin{pcomments}
  \pcomment{from: S09.cp7m}
\end{pcomments}

\pkeywords{
  structural_induction
  recursive_data
  planar_graphs
  planar_embeddings
}

%%%%%%%%%%%%%%%%%%%%%%%%%%%%%%%%%%%%%%%%%%%%%%%%%%%%%%%%%%%%%%%%%%%%%
% Problem starts here
%%%%%%%%%%%%%%%%%%%%%%%%%%%%%%%%%%%%%%%%%%%%%%%%%%%%%%%%%%%%%%%%%%%%%

\begin{problem}\label{structind}
Prove the following assertions by structural induction on the definition
of planar embedding.

\bparts

\ppart\label{structind-twice}
In a planar embedding of a graph, each edge is traversed a total of two
times by the faces of the embedding.

\begin{solution}
In the bridge case, the only change is that some face now makes
two traversals of a new edge.  In the face-splitting case, the only change
is that one face splits into two new faces, each traversing the same new
edge once.
\end{solution}

\ppart\label{structind-face-length} In a planar embedding of a graph with
at least three vertices, each face is of length at least three.

\begin{solution}
\textbf{Base case:} Check all possible embeddings of 3 vertex graphs.

\textbf{Constructor case:} (bridge) Two faces are replaced by a longer
face.

\textbf{Constructor case:} (face-splitting) The new faces are of the from
$ab...a$ where the three dots indicate at least one vertex since $a$ and
$b$ are not adjacent, so both new faces are of length at least 3; no other
faces change.
\end{solution}

\eparts
\end{problem}

%%%%%%%%%%%%%%%%%%%%%%%%%%%%%%%%%%%%%%%%%%%%%%%%%%%%%%%%%%%%%%%%%%%%%
% Problem ends here
%%%%%%%%%%%%%%%%%%%%%%%%%%%%%%%%%%%%%%%%%%%%%%%%%%%%%%%%%%%%%%%%%%%%%
