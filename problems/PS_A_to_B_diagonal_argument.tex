\documentclass[problem]{mcs}

\begin{pcomments}
  \pcomment{PS_A_to_B_diagonal_argument}
  \pcomment{from: S10.ps3 by ARM; revised 10.12.17 ARM}
  \pcomment{F17.cp7m}
\end{pcomments}

\pkeywords{
  set_theory
  diagonal
  Russells_paradox
  surjection
  powerset
}

%%%%%%%%%%%%%%%%%%%%%%%%%%%%%%%%%%%%%%%%%%%%%%%%%%%%%%%%%%%%%%%%%%%%%
% Problem starts here
%%%%%%%%%%%%%%%%%%%%%%%%%%%%%%%%%%%%%%%%%%%%%%%%%%%%%%%%%%%%%%%%%%%%%

\begin{problem}
  For any sets $A$ and $B$, let $[A \to B]$ be the set of total
  functions from $A$ to $B$.  Prove that if $A$ is not empty and $B$ has
  more than one element, then $A \strict [A \to B]$.

  \hint Suppose that $\sigma$ is a function from $A$ to $[A\to B]$
  mapping each element $a \in A$ to a function $\sigma_a:A \to B$.
  Suppose $b,c \in B$ and $b \neq c$.  Then define
  \[
  \text{diag}(a) \eqdef \begin{cases} c \text{ if } \sigma_a(a) = b,\\
                                      b \text{ otherwise}.
                        \end{cases}
  \]
  
\begin{solution}
One proof follows the hint:
\begin{proof}
  We need to prove that $\sigma$ is not a surjection.  We'll do this
  by exhibiting a function from $[A \to B]$ that is not in
  $\range{\sigma}$.  The function $\text{diag}:[A \to B]$ works, since
  by definition
  \[
  \text{diag}(a) \neq \sigma_a(a)
  \]
  for all $a \in A$, so $\text{diag}$ differs from every function,
  $\sigma_a$.
\end{proof}

Another proof uses the observation that $[A\to B] \surj \power(A)$.  For
example, the mapping that takes a function $f:A \to B$ to the set $\set{a
  \in A \suchthat f(a) = b}$ defines a surjection from $[A\to B]$ to
$\power(A)$.  So if $A \surj [A \to B]$ as well, then it would follow that
$A \surj \power(A)$, contradicting Theorem~\bref{powbig}.
\end{solution}

\end{problem}


%%%%%%%%%%%%%%%%%%%%%%%%%%%%%%%%%%%%%%%%%%%%%%%%%%%%%%%%%%%%%%%%%%%%%
% Problem ends here
%%%%%%%%%%%%%%%%%%%%%%%%%%%%%%%%%%%%%%%%%%%%%%%%%%%%%%%%%%%%%%%%%%%%%
\endinput
