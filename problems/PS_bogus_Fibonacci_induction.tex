\documentclass[problem]{mcs}

\begin{pcomments}
   \pcomment{PS_bogus_Fibonacci_induction}
   \pcomment{Adapted from TP_bogus_well_ordering_principle_proof by ARM 2/20/11}
\end{pcomments}

\pkeywords{
  bogus
  induction
  strong_induction
  false_theorem
  Fibonacci
}

\begin{problem}
The Fibonacci numbers
\[
0, 1, 1, 2, 3, 5, 8, 13, \dots
\]
are defined as follows.  Let $F(n)$ be the
$n$th Fibonacci number.  Then

\begin{align}
F(0) & \eqdef 0,\label{fibdef0}\\
F(1) & \eqdef 1,\label{fibdef1}\label{fib1def}\\
F(m) & \eqdef F(m-1) + F(m-2) &  \text{for } m \ge  2.\label{fibdefeqn}
\end{align}

Indicate exactly which sentence(s) in the following bogus proof contain
logical errors?  Explain.

\begin{falseclm*}
Every Fibonacci number is even.
\end{falseclm*}

\begin{bogusproof}
  Let all the variables $n,m,k$ mentioned below be nonnegative integer
  valued.  Let $\Even(n)$ mean that $F(n)$ is even.  The proof is by
  strong induction with induction hypothesis $\Even(n)$.

  \textbf{base case}: $F(0) = 0$ is an even number, so $\Even(0)$ is true.

  \textbf{inductive step}: We assume may assume the strong induction
  hypothesis
\[
\Even(k)$ for $0 \le k \le n,
\]
and we must prove $\Even(n+1)$

So assume $n+1 \ge 2$.  The by strong induction hypothesis, $\Even(n)$ and
$\Even(n-1)$ are true, that is, $F(n)$ and $F(n-1)$ are both even.  But by
the defining equation~\eqref{fibdefeqn}, $F(n+1)$ equals the sum, $F(n) +
F(n-1)$, of two even numbers, and so it is also even.  This proves
$\Even(n+1)$ as required.

Hence, $F(m)$ is even for all $m \in \naturals$ by the Strong Induction Principle.

\end{bogusproof}

\begin{solution}
  The only error in proof is the claim at the end of the inductive step
  that $\Even(n+1)$ has been proved.  The problem is that the inductive
  step began by assuming that $n+1 \ge 2$, and then never finished up by
  dealing with the case where $n+1 = 1$, the proof is supposed to show
  that
\begin{equation}\label{e0impe1}}
\Even(0) \QIMPLIES \Even(1).
\end{equation}
But in the $n+1=1$ case, equation~\eqref{Fibdefeqn} does not apply to
$Fib(n+1)$.  Instead, $Fib(n+1)$ is defined to be 1 by
equation~\eqref{fib1def}, making the implication~\eqref{e0impe1} false.

In short, the proof in the inductive step that $\Even(n) \QIMPLIES
\Even(n+1)$ doesn't work for $n=0$, invalidating the whole proof.

Incidentally, saying that the inductive step made a logical error when it
assumed $n+1 \ge 2$ is on the right track, but not quite right.  After the
base case $F(0)$, the orderly thing to have done would have been to
consider $F(1)$ next, and the proof didn't do that.  But technically,
starting with the $n+1 \ge 2$ case is an \emph{organizational}, or perhaps
\emph{strategic}, error because it skips the $n+1 = 1$ case.  But this is
not a \emph{logical} error, because the proof was correct for the $n+1 \ge
2$ case.  The logical error was in skipping the $n+1 =1$ case altogether.
This distinction between logical errors where there is a mistake in
reasoning, and strategic errors which will cause a later difficulty, is a
useful one to recognize.
\end{staffnotes}

\end{solution}

\end{problem}

\endinput
