\documentclass[problem]{mcs}

\begin{pcomments}
  \pcomment{FP_Emotional_Quotient}
\end{pcomments}

\pkeywords{
  probability
  Chebyshev Bound
  Chebyshev Inequality
}

\begin{problem}

There is a newly developed test called EQ \textit{(Emotional Quotient)}, that is used to decide whether or not you will be able to make friends outside MIT. $EQ$ is expressed as a nonnegative integer. The average EQ over all humans is 1000.

\bparts

\ppart Knowing only the average $EQ$, find an upper bound for the fraction of humans that have $EQ \geq1500$.

\begin{solution}

We can use the Markov's inequality to obtain the upper bound.

$$Pr\{X\geq cE[X]\} \leq \frac{1}{c}$$

setting $c=\frac{1500}{1000}=\frac{3}{2}$, we obtain the upper bound $\frac{2}{3}$.

\end{solution}

\ppart Give an example of a distribution that achieves the upper bound on part a.

\begin{solution}

One possible example is $3$ humans, $2$ with EQ equals to $1500$ and $1$ with EQ equal to $0$.

\end{solution}
 
\ppart Assuming that the standard deviation of $EQ$ is $100$, what fraction of humans can have $EQ \geq 3000$

\begin{solution}
We use the Chebyshev bound: 

$$Pr\{ | X - E[X] | \geq  c\sigma_x\} \leq \frac1{c^2}$$

with $c=20$ to get an upper bound of $\frac{1}{400}$, since $Pr(X - E[X] \leq -1000) = 0 $ because $E[X] = 2000$ and EQ is expressed as a
nonnegative integer.
  
\end{solution}

\ppart Give an example of a distribution that achieves the upper bound on part c.



\eparts

\end{problem}

%%%%%%%%%%%%%%%%%%%%%%%%%%%%%%%%%%%%%%%%%%%%%%%%%%%%%%%%%%%%%%%%%%%%%
% Problem ends here
%%%%%%%%%%%%%%%%%%%%%%%%%%%%%%%%%%%%%%%%%%%%%%%%%%%%%%%%%%%%%%%%%%%%%

\endinput
