\documentclass[problem]{mcs}

\begin{pcomments}
  \pcomment{FP_graph_colorableb}
  \pcomment{Forked from PS_graph_colorable}
  \pcomment{CH, 4/2/2014}
\end{pcomments}

\pkeywords{
 graph theory
 colorable
 width
}

\begin{problem}

A simple graph $G$ is said to have \emph{width} $w$ iff there is a
way to list all its vertices so that each vertex is adjacent to at
most $w$ vertices that appear earlier in the list.  For example, if
the degree of every vertex is at most $w$, then the graph clearly has
width at most $w$---just list the vertices in any order.

This problem will show that every graph with width $w$ is $(w +
1)$-colorable using \textbf{induction} on the number of vertices in
the graph.

\bparts

\ppart Clearly state the Induction Hypothesis, $P(n)$.

\begin{solution}

$P(n)$ is the proposition that for all $w \in \naturals$ and all
$n$-vertex graphs $G$ with width $w$, the graph $G$ is
$(w+1)$-colorable.

\end{solution}

\examspace[0.6in]

\ppart Prove the base case for $n=1$.

\begin{solution}
Every graph with $1$ vertex has width 0 and is $0 + 1 = 1$ colorable.
Therefore, $P(1)$ is true.
\end{solution}

\examspace[0.6in]

\ppart Prove the induction step.

\hint Remove the last vertex.

\begin{solution}

\begin{proof}

Assume that $P(n)$ is true for some $n \geq 1$ and let $G$ be an
$(n+1)$-vertex graph with width $w$.  We need only show that $G$ is
$(w+1)$-colorable.

Since $G$ has width $w$, the vertices of $G$ can be listed with each
vertex adjacent to at most $w$ earlier in the list.  Let $v$ be the
last vertex in the list, and let $G'$ be the graph obtained by
removing $v$, and all edges incident to $v$, from $G$.

Now $G'$ has $n$ vertices and still has width $w$ since the sequence
$S$ with its last vertex removed is a sequence consisting of all the
vertices of $G'$ with each vertex adjacent to exactly the same
previous vertices.  Therefore, by the Induction Hypothesis, $G'$ is
$(w+1)$-colorable.

We can define a $(w+1)$-coloring of $G$ as follows: color all the
vertices of $G$ besides $v$ using the $(w+1)$-coloring of $G'$.  Since
there are at most $w$ colors among the $w$ vertices adjacent to $v$,
there will always be one of the $w+1$ colors that differs from these
$w$ colors.  So assigning this color to $v$ yields the required
$(w+1)$-coloring of $G$.

This proves $P(n+1)$ and completes the induction step.

\end{proof}
\end{solution}

\iffalse Assume that the Induction Hypothesis $P(n)$ is true.  Let $G$
be an $(n+1)$-vertex graph with width at most $w$.  Arrange the
vertices of $G$ in a sequence $S$ with each vertex connected to at
most $w$ preceding vertices.  Next, let $G'$ be the subgraph obtained
by removing the last vertex $v$ and all edges incident to $v$.  Show
that $G'$ is $(w+1)$-colorable.

\begin{solution}
The subgraph $G'$ has $n$ vertices. It also has width at most $w$,
since the sequence $S$ with its last vertex removed is a sequence
consisting of all the vertices of $G'$ with each vertex adjacent to
exactly the same previous vertices.  Therefore, by the Induction
Hypothesis, $G'$ is $(w+1)$-colorable.
\end{solution}

\examspace[1.5in]

\ppart Argue why it is always possible to assign one of the $(w+1)$
colors in $G'$ to $v$. 

\begin{solution}
Since there are at most $w$ colors among the $w$ vertices adjacent to
$v$, there will always be a different one of the $w+1$ colors of $G'$
that we can assign to $v$.  So $G$ is $(w+1)$-colorable, which proves
$P(n+1)$.

\end{solution}

\examspace[1.5in]

This shows that any $(w+1)$-coloring of $G'$ can be extended to a $(w+1)$-coloring of
$G$. Therefore, $P(n+1)$ is true, completing the Induction Step.
\fi

\eparts

\end{problem}
\endinput
