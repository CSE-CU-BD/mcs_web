\documentclass[problem]{mcs}

\begin{pcomments}
  \pcomment{FP_graph_colorableb}
  \pcomment{Forked from PS_graph_colorable}
  \pcomment{CH, 4/2/2014}
\end{pcomments}

\pkeywords{
 graph theory
 colorable
 width
}

\begin{problem}

\bparts
% \ppart Let $G$ be any {tree} with $n \geq 2$ vertices. Argue that $G$ is
% 2-colorable. 
% \begin{solution}
% Since $G$ is a tree, it has no cycles. Therefore, it is bipartite
% (since it has no cycles of odd length).  This means that 2 colors are
% \emph{sufficient}, and so $G$ is 2-colorable. In fact, if the tree has at least one edge, 2 colors are also \emph{necessary}. 
% \end{solution}
% \examspace[1in]

A simple graph, $G$, is said to have \emph{width} $w$ iff there is a
way to list all its vertices so that each vertex is adjacent to at
most $w$ vertices that appear earlier in the list.  For example, if
the degree of every vertex is at most $w$, then the graph obviously
has width $w$---just list the vertices in any order.

We will prove that every graph with width $w$ is $(w +
1)$-colorable using \textbf{induction} on $n$, the number of vertices
in the graph.

\ppart Clearly state the Induction Hypothesis, $P(n)$.
\begin{solution}
$P(n)$ is the proposition that for all $w$, every $n$-vertex graph with width $w$ is
$(w+1)$-colorable. 
\end{solution}

\examspace[0.6in]

\ppart Prove the base case for $n=1$.

\begin{solution}
Every graph with $1$ vertex has width 0 and is $0 + 1 = 1$ colorable.  Therefore, $P(1)$ is true.
\end{solution}

\examspace[0.6in]

\ppart Prove the induction step.

\hint Consider a graph $G$ of width $w$ with $(n+1)$ vertices arranged
in a sequence. Remove the last vertex and its edges, and show that the remaining
graph is $(w+1)$-colorable.

\begin{solution}

Assume that the Induction Hypothesis $P(n)$ is true.  Let $G$ be
an $(n+1)$-vertex graph with width $w$.  Arrange the vertices
of $G$ in a sequence, $S$, with each vertex connected to at most $w$
preceding vertices. 

Consider the subgraph $G'$ obtained by removing the last vertex, $v$, and all edges incident to $v$.
The subgraph $G'$ has $n$ vertices. It also has width $w$, since the
sequence $S$ with its last vertex removed is a sequence consisting of all the
vertices of $G'$ with each vertex adjacent to exactly the same previous
vertices.  Therefore, by the Induction Hypothesis, $G'$ is $(w+1)$-colorable.  

Since there are at most $w$ colors among the $w$ vertices adjacent to
$v$, there will always be a different one of the $w+1$ colors of $G'$
that we can assign to $v$.  So $G$ is $(w+1)$-colorable, which proves
$P(n+1)$. This completes the induction step.

\end{solution}


\iffalse

 Assume that the Induction Hypothesis $P(n)$ is true.  Let $G$ be
an $(n+1)$-vertex graph with width at most $w$.  Arrange the vertices
of $G$ in a sequence, $S$, with each vertex connected to at most $w$
preceding vertices. Consider the subgraph $G'$ obtained by removing
the last vertex, $v$, and all edges incident to $v$. Show that $G'$ is
$(w+1)$-colorable.

\begin{solution}
The subgraph $G'$ has $n$ vertices. It also has width at most $w$, since the
sequence $S$ with its last vertex removed is a sequence consisting of all the
vertices of $G'$ with each vertex adjacent to exactly the same previous
vertices.  Therefore, by the Induction Hypothesis, $G'$ is $(w+1)$-colorable.  
\end{solution}

\examspace[1.5in]

\ppart Argue why it is always possible to assign one of the $(w+1)$
colors in $G'$ to $v$. 
\begin{solution}
Since there are at most $w$ colors among the $w$ vertices adjacent to
$v$, there will always be a different one of the $w+1$ colors of $G'$
that we can assign to $v$.  So $G$ is $(w+1)$-colorable, which proves
$P(n+1)$.
\end{solution}

\examspace[1.5in]

This shows that any $(w+1)$-coloring of $G'$ can be extended to a $(w+1)$-coloring of
$G$. Therefore, $P(n+1)$ is true, completing the Induction Step.
\fi


\eparts

\end{problem}
\endinput
