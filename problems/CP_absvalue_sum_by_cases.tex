\documentclass[problem]{mcs}

\begin{pcomments}
  \pcomment{CP_absvalue_sum_by_cases}
  \pcomment{F95.ps1 adapted by ARM 2/3/17}
\end{pcomments}

\pkeywords{
  cases
  absolute_value
}

%%%%%%%%%%%%%%%%%%%%%%%%%%%%%%%%%%%%%%%%%%%%%%%%%%%%%%%%%%%%%%%%%%%%%
% Problem starts here
%%%%%%%%%%%%%%%%%%%%%%%%%%%%%%%%%%%%%%%%%%%%%%%%%%%%%%%%%%%%%%%%%%%%%
\begin{problem}
Prove by cases that
\begin{equation}\tag{1}%\label{absrsplus}
\abs{r + s} \leq  \abs{r} + \abs{s}
\end{equation}
for all real numbers $r,s$.\footnote{The \term{absolute value}
  $\abs{r}$ of $r$ equals whichever of $r$ or $-r$ is not negative.}

\begin{staffnotes}
The following proof from F95 would be great for a student team to come
up with.  It does omit some details given in the solution proof, but
the solution may be overly belabored.

\begin{proof}
There are four cases.
\begin{enumerate}
\item $r \geq 0$ and $s \geq 0$.  Equality holds.
\item $r < 0$ and $s < 0$.  Equality holds.
\item $r \geq 0$ and $s < 0$.  If $r \geq -s$ then $|r| + |s| = r +
  (-s) > r + s = |r+s|.$ If $r < -s$ then $|r| + |s| = r + (-s) \geq
  -r + (-s) = |r+s|$.
\item $r < 0$ and $s \geq 0$. Same as the previous case with the roles
  of $r$ and $s$ reversed.
\end{enumerate}
\end{proof}
\end{staffnotes}

\begin{solution}
Note that by definition,
\begin{align}
\abs{t} = & t & \text{(if $t \geq 0$)},\tag{2}\\%\label{absg0}
\abs{t} = & \abs{-t}, \tag{3}%\label{abt-t}
\end{align}
for all real numbers $t$.

\inductioncase{Case 1}: (Both $r$ and $s$ are not negative).  In this
case, $r+s$ is also not negative, so
\begin{align*}
\abs{r + s}
   & = r + s              & \text{(by (2))}\\ %~\eqref{absg0}
   & = \abs{r} + \abs{s}, & \text{(again by (2))}\\ %~\eqref{absg0}
\end{align*}
which proves (1)
%~\eqref{absrsplus}
in this case.

\inductioncase{Case 2}: (Both $r$ and $s$ are negative).  In this case,
\begin{align*}
\abs{r + s}
   & = \abs{-(r + s)}      & \text{(by (3))}\\%~\eqref{abt-t}
   & = \abs{-r + -s}\\
   & = -r + -s             & \text{(by (2))}\\%~\eqref{absg0}
   & = \abs{-r} + \abs{-s} & \text{(again by (2))}\\%~\eqref{absg0}
   & = \abs{r} + \abs{s} & \text{(by (3))}.%~\eqref{abt-t}
\end{align*}
So~(1)
%~\eqref{absrsplus}
is true in this case.

\inductioncase{Case 3}: (Exactly one of $r$ and $s$ is negative).  By
symmetry, we can assume that $r\geq 0$ and $s < 0$.
\begin{quote}

\inductioncase{Case 3.1}: ($\abs{r} \geq \abs{s}$).
In this case $(r - -s) >0$, so we have
\begin{align*}
\abs{r + s}
   & = \abs{r - -s}\\
   & = r - -s         & \text{by (2)}\\%~\eqref{absg0}
   & < r + -s         & \text{(since $-s >0$)}\\
   & = \abs{r} + \abs{s}  & \text{by (2)}.\\%~\eqref{absg0}
\end{align*}
So
%~\eqref{absrsplus}
(1) holds in this case.

\inductioncase{Case 3.2}: ($\abs{r} < \abs{s}$).
In this case $(-s - r) >0$, so we have
\begin{align*}
\abs{r + s} & = \abs{-(r + s)} & \text{(by (3))} \\%~\eqref{abt-t}
   & = \abs{-s - r}\\
   & = -s - r            & \text{(by (2))}\\%~\eqref{absg0}
   & \leq -s + r         & \text{(since $r \geq 0$)}\\
   & = \abs{s} + \abs{r} & \text{(by (2))}\\%~\eqref{absg0}
   & = \abs{r} + \abs{s}.
\end{align*}
So~(1)
%~\eqref{absrsplus}
holds in this case.
\end{quote}
We conclude that, in any case,~(1) holds.
\end{solution}
\end{problem}

%%%%%%%%%%%%%%%%%%%%%%%%%%%%%%%%%%%%%%%%%%%%%%%%%%%%%%%%%%%%%%%%%%%%%
% Problem ends here
%%%%%%%%%%%%%%%%%%%%%%%%%%%%%%%%%%%%%%%%%%%%%%%%%%%%%%%%%%%%%%%%%%%%%

\endinput
