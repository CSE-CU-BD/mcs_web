\documentclass[problem]{mcs}

\begin{pcomments}
  \pcomment{TP_divisibility_and_congruence}
  \pcomment{renamed from TP_Divisibility_and_Congruence}
  \pcomment{Converted from ./00Convert/probs/practice8/prob6.scm by
    scmtotex and drewe 8/2/11}
  \pcomment{format ARM 8/24/11}
  \pcomment{better wrong choices inserted ARM 4/1/16}
\end{pcomments}

\pkeywords{
divides
multiple
modulo
congruence
}

\begin{problem}
Assume that
\[
a \equiv  b \pmod n,
\]
where $n > 1$ and $a$ and $b$ are integers.
      
\begin{center}
\exambox{2.0in}{0.5in}{0.0in}
\end{center}

\begin{staffnotes}
Rubric: +3 for each equivalence listed, -2 for each non-equivalence
\end{staffnotes}

\bparts

\ppart 
For what positive integer value(s) of $k$ does $2a \equiv 2b \pmod kn$ hold?

\begin{solution}
Only for $k=2$
\end{solution}

\ppart

For what positive integer value(s) of $k$ does $a^k \equiv b^k \pmod n$ hold? For what values are the two statements equivalent (remember that for the two statements to be equivalent both directions of implications must hold)?

\begin{solution}
The statement holds for all positive integer $k$, but the two statements are equivalent only in the trivial case where $k=1$.
\end{solution}

\ppart
Is this equivalent to saying that  $\gcd(a,n) = \gcd(b,n)$? Either show the equivalence or give a counterexample.

\begin{solution}
$gcd(3,11) = gcd(7,11) = 1$ but it is not the case that $3 \equiv  7 \pmod 11$
\end{solution}

\eparts

\end{problem}

\endinput
