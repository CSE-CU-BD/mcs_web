\documentclass[problem]{mcs}

\begin{pcomments}
  \pcomment{TP_Counting_Relations}
  \pcomment{Converted from ./00Convert/probs/practice3/prob7.scm
    by scmtotex and drewe
    on Thu 21 Jul 2011 12:05:47 PM EDT}
\end{pcomments}

\begin{problem}

How many relations are there on a set of size $n$ when:
\bparts

\ppart
$n=1$?
\begin{solution}
2.  Either the element is related to itself or not.
\end{solution}
\ppart
$n=2$?
\begin{solution}
16. 

Four pairs, ---AA, AB, BA, BB, for example ---each of which can be
present or absent, gives a total of $2^4=16$ possible binary
relations.
\end{solution}

\ppart
$n=3$?
\begin{solution}
$2^6$. 

Same reasoning as above, but now there are 6 pairs.
\end{solution}

\eparts

\end{problem}

\endinput
