\documentclass[problem]{mcs}

\begin{pcomments}
  \pcomment{FP_hot_cows_chebyshev}
  \pcomment{variant of FP_hot_cows_markov}
  \pcomment{Original: ARM May 19, 2013}
\end{pcomments}

\pkeywords{
  average
  Markov_bound
  deviation
  sample_space
  outcome
  probability_space
}

%%%%%%%%%%%%%%%%%%%%%%%%%%%%%%%%%%%%%%%%%%%%%%%%%%%%%%%%%%%%%%%%%%%%%
% Problem starts here
%%%%%%%%%%%%%%%%%%%%%%%%%%%%%%%%%%%%%%%%%%%%%%%%%%%%%%%%%%%%%%%%%%%%%


\begin{problem}

A herd of cows is stricken by an outbreak of \emph{hot cow disease}.
The disease raises the normal body temperature of a cow, and a cow
will die if its temperature goes above 90 degrees.  The disease
epidemic is so intense that it raised the average temperature of the
herd to $120$ degrees.  Body temperatures as high as $140$ degrees,
\textbf{but no higher}, were actually found in the herd.

\bparts

\ppart\label{2/10cows} Use Chebyshev's Theorem to
conclude that at most 2/5 of the cows could have survived.

\examspace[2in]

\begin{solution}
Let $T$ be the temperature of a cow chosen \emph{uniformly at random}. We apply
Chebyshev's Bound to $T$ as follows:
\begin{align*}
\prob{T \leq 90}
 & = \prob{140 - T \geq 140 -90}\\
 & \leq \frac{\expect{140-T}}{140-90} & \text{(Chebyshev's Theorem)}\\
 & = \frac{140-120}{140-90} & \text{(linearity of $\expect{}$)}\\
 & = 20/50 = 2/5.
\end{align*}

Since cows were chosen randomly, this implies that at most 2/5 of the
herd can have a temperature $\leq 90$.

\end{solution}


\ppart Notice that the conclusion of part~\eqref{2/10cows} is a purely
arithmetic facts about averages, not about probabilities.  But you
verified the claim of part~\eqref{2/10cows} by applying Markov's bound
on the deviation of a random variable.  Justify this approach by
explaining how to define a random variable, $T$, for the temperature
of a cow.  Carefully specify the probability space on which $T$ is
defined: what are the outcomes? what are their probabilities?  Explain
the precise connection between properties of $T$, average herd
temperature, and fractions of the herd with various temperatures, that
justify application of Markov's Bound.

\begin{solution}
The sample space for $T$ is the set of cows in the herd, that is, each
cow is an outcome.  The probabilities are defined to be
\emph{uniform}---the probability of any cow, $c$, is $1/n$ where $n$
is the size of the herd---and $T(c)$ is the temperature of cow $c$.
Since the probabilities are uniform, it follows that

\begin{itemize}

\item the average temperature of the herd equals $\expect{T}$.

\item the fraction of cows with temperatures $\leq t$ is the
  probability that $T \leq t$.

\end{itemize}

So the fact that $\prob{T \leq 90} \leq 2/5$ implies that at most 2/5
of the herd could have survived.

\end{solution}
\eparts

\end{problem}

%%%%%%%%%%%%%%%%%%%%%%%%%%%%%%%%%%%%%%%%%%%%%%%%%%%%%%%%%%%%%%%%%%%%%
% Problem ends here
%%%%%%%%%%%%%%%%%%%%%%%%%%%%%%%%%%%%%%%%%%%%%%%%%%%%%%%%%%%%%%%%%%%%%

\endinput
