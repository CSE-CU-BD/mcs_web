\documentclass[problem]{mcs}

\begin{pcomments}
  \pcomment{binomial and multinomial coefficients}
  \pcomment{by njoliat and ARM}
\end{pcomments}

\pkeywords{
  binomial_coefficients
}

%%%%%%%%%%%%%%%%%%%%%%%%%%%%%%%%%%%%%%%%%%%%%%%%%%%%%%%%%%%%%%%%%%%%%
% Problem starts here
%%%%%%%%%%%%%%%%%%%%%%%%%%%%%%%%%%%%%%%%%%%%%%%%%%%%%%%%%%%%%%%%%%%%%

\begin{problem}
\begin{problemparts}

\problempart
Write the term of $(x + y)^{40}$ which includes $x^3$.
\examspace[2in]

\begin{solution}
\[
\binom{40}{3}x^3y^{37}.
\]

\end{solution}

\problempart
Write the term of $(x+y+z)^{40}$ which includes $x^3y^5$.
\examspace[2in]

\begin{solution}
\[
\binom{40}{3,5,32}x^3y^5z^{32}
\]

\end{solution}

\problempart
Give a combinatorial proof that
\[
\sum_{i=0}^n \binom{n}{i} = 2^n.
\]

\hint Begin by finding a set whose cardinality is equal to the right
hand side of the equation.

\begin{solution}
Count the number of $n$-length bit strings.  For the LHS, we consider
the $i$th term of the sum to represent the bit strings which have $i$
zeros.
\end{solution}


\end{problemparts}

\end{problem}


%%%%%%%%%%%%%%%%%%%%%%%%%%%%%%%%%%%%%%%%%%%%%%%%%%%%%%%%%%%%%%%%%%%%%
% Problem ends here
%%%%%%%%%%%%%%%%%%%%%%%%%%%%%%%%%%%%%%%%%%%%%%%%%%%%%%%%%%%%%%%%%%%%%

\endinput
