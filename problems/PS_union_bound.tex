\documentclass[problem]{mcs}

\begin{pcomments}
  \pcomment{PS_union_bound}
  \pcomment{DO NOT USE: induction is weak approach and only proves finite case}
  \pcomment{from: F06.ps11 (ported by Rich)}
\end{pcomments}

\pkeywords{
  probability
  union_bound
}

%%%%%%%%%%%%%%%%%%%%%%%%%%%%%%%%%%%%%%%%%%%%%%%%%%%%%%%%%%%%%%%%%%%%%
% Problem starts here
%%%%%%%%%%%%%%%%%%%%%%%%%%%%%%%%%%%%%%%%%%%%%%%%%%%%%%%%%%%%%%%%%%%%%

\begin{problem}

Prove the following probabilistic identity, referred to as the \textbf{Union
Bound}.  You may assume the theorem
that the probability of a union of \emph{disjoint} sets is the sum of their
probabilities.

Let $A_1, \dots, A_n$ be a collection of events.  Then
\[
\prob{A_1 \union A_2 \union \cdots \union A_n} \leq \sum_{i=1}^n
\prob{A_i}.
\]
\hint Induction.

\begin{solution}
  For all $n \geq 1$, let $P(n)$ be the proposition that the claim is true.

  \inductioncase{Base case:} Trivially $\prob{A_1} \leq \prob{A_1}$, so $P(1)$ is true.

  \inductioncase{Induction step:} Assume that $P(n)$ is true and show $P(n+1)$ for
  $n \geq 1$.
  \begin{align*}
  \prob{ A_1 \union A_2 \union \cdots \union A_{n+1}} 
   &= \prob{(A_1 \union A_2 \union \cdots \union A_{n}) \union A_{n+1}} \\
   &= \prob{A_1 \union A_2 \union \cdots \union A_{n}} + \prob{A_{n+1}} \\
   &\quad - \prob{(A_1 \union A_2 \union \cdots \union A_{n}) \intersect A_{n+1}}
   &\text{(by~Inclusion-Exclusion)}\\
   &\leq \prob{A_1 \union A_2 \union \cdots \union A_{n}} + \prob{A_{n+1}}\\
   &\leq \sum_{i=1}^{n} \prob{A_{i}} + \prob{A_{n+1}} &\text{(by Ind. Hyp.)} \\
   &= \sum_{i=1}^{n+1} \prob{A_{i}} 
  \end{align*}
  Thus $P(n)$ is true and the result follows by induction.
\end{solution}

\end{problem}

%%%%%%%%%%%%%%%%%%%%%%%%%%%%%%%%%%%%%%%%%%%%%%%%%%%%%%%%%%%%%%%%%%%%%
% Problem ends here
%%%%%%%%%%%%%%%%%%%%%%%%%%%%%%%%%%%%%%%%%%%%%%%%%%%%%%%%%%%%%%%%%%%%%

\endinput
