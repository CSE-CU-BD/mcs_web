\documentclass[problem]{mcs}

\begin{pcomments}
  \pcomment{CP_findphi}
  \pcomment{ARM 3/11/12; source Shoup?}
\end{pcomments}

\pkeywords{
  number_theory
  modular_arithmetic
  primes
  RSA
  Eulers_function
  phi
  factoring
  security
}

%%%%%%%%%%%%%%%%%%%%%%%%%%%%%%%%%%%%%%%%%%%%%%%%%%%%%%%%%%%%%%%%%%%%%
% Problem starts here
%%%%%%%%%%%%%%%%%%%%%%%%%%%%%%%%%%%%%%%%%%%%%%%%%%%%%%%%%%%%%%%%%%%%%

\begin{problem}

\bparts

\ppart Just as RSA would be trivial to crack knowing the factorization
into two primes of $n$ in the public key, explain why RSA would also
be trivial to crack knowing $\phi(n)$.

\begin{staffnotes}
  \hint The answer is so obvious you may wonder if you misunderstood
  the question.
\end{staffnotes}

\begin{solution}
If you knew $\phi(pq) = (p-1)(q-1)$ you could find the private key~$d$
the same way the \textbf{Receiver} does using the Pulverizer to find
the inverse mod $(p-1)(q-1)$ of the public key $e$.
\end{solution}

\ppart Show that if you knew $n$, $\phi(n)$, and that $n$ was the
product of two primes, then you could easily factor $n$.

\begin{staffnotes}
\hint Suppose $n =pq$, replace $q$ by $n/p$ in the expression for
$\phi(n)$, and solve for $p$.
\end{staffnotes}

\begin{solution}
\begin{align*}
\phi(n) &  = (p-1)(q-1) = n - p - \frac{n}{p} + 1, \qquad \text{so}\\
p \phi(n) & = p n -p^2 -n + p\\
0 =  & p^2 + (\phi(n) -n - 1)p + n.
\end{align*}
Now we can solve for $p$ using the formula for the roots of a quadratic.
\end{solution}

\eparts

\end{problem}


%%%%%%%%%%%%%%%%%%%%%%%%%%%%%%%%%%%%%%%%%%%%%%%%%%%%%%%%%%%%%%%%%%%%%
% Problem ends here
%%%%%%%%%%%%%%%%%%%%%%%%%%%%%%%%%%%%%%%%%%%%%%%%%%%%%%%%%%%%%%%%%%%%%

\endinput
