\documentclass[problem]{mcs}

\begin{pcomments}
  \pcomment{PS_optimal_overhang}
  \pcomment{from ftl book draft}
\end{pcomments}

\pkeywords{
  stable_stack
  overhang
  optimal
}

%%%%%%%%%%%%%%%%%%%%%%%%%%%%%%%%%%%%%%%%%%%%%%%%%%%%%%%%%%%%%%%%%%%%%
% Problem starts here
%%%%%%%%%%%%%%%%%%%%%%%%%%%%%%%%%%%%%%%%%%%%%%%%%%%%%%%%%%%%%%%%%%%%%

\begin{problem}
\TBA{pending}

\begin{editingnotes}
\textbf{FROM FTL VERSION}

In general, the overhang of a stack of blocks is maximized by sliding
the entire stack rightward until its center of mass is at the edge of
the table.  The overhang will then be equal to the distance between
the center of mass of the stack and the rightmost edge of the
rightmost block.  We call this distance the \term{spread} of the
stack.  Note that the spread does not depend on the location of the
stack on the table---it is purely a property of the blocks in the
stack.  Of course, as we just observed, the maximum possible overhang
is equal to the maximum possible spread.  This relationship is
illustrated in Figure~\ref{fig:overhang}.

\begin{figure}

\graphic{Bookstack-2}

\caption{The overhang is maximized by maximizing the spread and then
  placing the stack so that the center of mass is at the edge of the
  table.}

\label{fig:overhang}

\end{figure}

%\subsection{A Recursive Solution}

Our goal is to find a formula for the maximum possible spread~$S_n$
that is achievable with a stable stack of $n$~blocks.

We already know that $S_1 = 1/2$ since the right edge of a single
block with length~1 is always distance~$1/2$ from its center of mass.
Let's see if we can use a recursive approach to determine~$S_n$ for
all~$n$.  This means that we need to find a formula for~$S_n$ in terms
of~$S_i$ where~$i < n$.

Suppose we have a stable stack~$\mathcal{S}$ of $n$~blocks with
maximum possible spread~$S_n$.  There are two cases to consider
depending on where the rightmost block is in the stack.

\subparagraph{Case 1:}

\emph{The rightmost block in~$\mathcal{S}$ is the bottom block.}
Since the center of mass of the top $n - 1$~blocks must be over the
bottom block for stability, the spread is maximized by having the
center of mass of the top $n - 1$~blocks be directly over the
\emph{left} edge of the bottom block.  In this case the center of mass
of~$\mathcal{S}$ is\footnote{The center of mass of a stack of blocks
  is the average of the centers of mass of the individual blocks.}
\[
\frac{ (n - 1) \cdot 1 + (1) \cdot \frac{1}{2} }{ n }
    = 1 - \frac{1}{2n}
\]
to the left of the right edge of the bottom block and so the spread
for~$\mathcal{S}$ is
\begin{equation}\label{eqn:9G15}
    1 - \frac{1}{2n}.
\end{equation}
For example, see Figure~\ref{fig:9G14}.

\begin{figure}

\graphic{Fig_G14}

\caption{The scenario where the bottom block is the rightmost block.
  In this case, the spread is maximized by having the center of mass
  of the top $n-1$~blocks be directly over the left edge of the bottom
block.}

\label{fig:9G14}

\end{figure}

In fact, the scenario just described is easily achieved by arranging
the blocks as shown in Figure~\ref{fig:9G15}, in which case we have
the spread given by equation~\ref{eqn:9G15}.  For example, the spread
is $3/4$ for 2~blocks, $5/6$ for 3~blocks, $7/8$ for 4~blocks, etc.

\begin{figure}

\graphic{Fig_G15}

\caption{A method for achieving spread (and hence overhang)~$1 - 1/2n$
  with $n$~blocks, where the bottom block is the rightmost block.}

\label{fig:9G15}

\end{figure}

Can we do any better?  The best spread in Case~1 is always less
than~1, which means that we cannot get a block fully out over the edge
of the table in this scenario.  \iffalse
Maybe our intuition was right that we
can't do better.  Before we jump to any false conclusions, however,
let's see what happens in the other case.
\fi

\subparagraph{Case 2:}

\emph{The rightmost block in~$\mathcal{S}$ is among the top $n -
  1$~blocks.}  In this case, the spread is maximized by placing the
top $n - 1$~blocks so that their center of mass is directly over the
\emph{right} end of the bottom block.  This means that the center of
mass for~$\mathcal{S}$ is at location
\[
    \frac{(n - 1) \cdot C + 1 \cdot \paren{C - \frac{1}{2}}}{n}
    = C - \frac{1}{2n}
\]
where $C$~is the location of the center of mass of the top $n -
1$~blocks.  In other words, the center of mass of~$\mathcal{S}$
is~$1/2n$ to the left of the center of mass of the top $n - 1$~blocks.
(The difference is due to the effect of the bottom block, whose center
of mass is $1/2$ unit to the left of~$C$.)  This means that the spread
of~$\mathcal{S}$ is $1/2n$ greater than the spread of the top $n -
1$~blocks (because we are in the case where the rightmost block is
among the top $n - 1$~blocks.)

Since the rightmost block is among the top $n - 1$ blocks, the spread
for~$\mathcal{S}$ is maximized by maximizing the spread for the top $n
- 1$~blocks.  Hence the maximum spread for~$\mathcal{S}$ in this case
is
\begin{equation}\label{eqn:9G16}
    S_{n - 1} + \frac{1}{2n}
\end{equation}
where $S_{n - 1}$~is the maximum possible spread for $n - 1$~blocks
(using any strategy).

We are now almost done.  There are only two cases to consider when
designing a stack with maximum spread and we have analyzed both of
them.  This means that we can combine equation~\ref{eqn:9G15} from
Case~1 with equation~\ref{eqn:9G16} from Case~2 to conclude that
\begin{equation}\label{eqn:9G17}
    S_n = \max \paren{1 - \frac{1}{2n}, \; S_{n - 1} + \frac{1}{2n}}
\end{equation}
for any $n > 1$.

Uh-oh. This looks complicated.  Maybe we are not almost done after
all!

Equation~\ref{eqn:9G17} is an example of a \term{recurrence}.  We will
describe numerous techniques for solving recurrences in a later chapter,
%Chapter~\ref{chap:recurrences},
but, fortunately, equation~\ref{eqn:9G17} is simple enough that we can
solve it directly.

One of the first things to do when you have a recurrence is to get a
feel for it by computing the first few terms.  This often gives clues
about a way to solve the recurrence, as it will in this case.

We already know that $S_1 = 1/2$.  What about~$S_2$?  From
equation~\ref{eqn:9G17}, we find that
\begin{align*}
S_2
    &= \max \left\{ 1 - \frac{1}{4}, \; \frac{1}{2} + \frac{1}{4} \right \} \\
    &= 3/4.
\end{align*}
Both cases give the same spread, albeit by different approaches.  For
example, see Figure~\ref{fig:9G20}.

\begin{figure}

\subfloat[]{\graphic{Fig_G20-a}}
\qquad
\subfloat[]{\graphic{Fig_G20-b}}

\caption{Two ways to achieve spread (and hence overhang)~$3/4$ with $n
  = 2$ blocks.  The first way~(a) is from Case~1 and the second~(b) is
  from Case~2.}

\label{fig:9G20}

\end{figure}

That was easy enough.  What about~$S_3$?
\begin{align*}
S_3 &= \max\left\{ 1 - \frac{1}{6}, \; \frac{3}{4} + \frac{1}{6} \right\} \\
    &= \max\left\{ \frac{5}{6}, \; \frac{11}{12} \right\} \\
    &= \frac{11}{12}.
\end{align*}
As we can see, the method provided by Case~2 is the best.  Let's check
$n = 4$.
\begin{align}
S_4 &= \max\left\{ 1 - \frac{1}{8}, \; \frac{11}{12} + \frac{1}{8} \right\}
    \notag \\
    &= \frac{25}{24}. \label{eqn:9G21}
\end{align}

Wow!  This is a breakthrough---for two reasons.  First,
equation~\ref{eqn:9G21} tells us that by using only 4~blocks, we can
make a stack so that one of the blocks is hanging out completely over
the edge of the table.  The two ways to do this are shown in
Figure~\ref{fig:9G22}.

\begin{figure}

\subfloat[]{\graphic{Fig_G22-a}}
\qquad
\subfloat[]{\graphic{Fig_G22-b}}

\caption{The two ways to achieve spread (and overhang)~$25/24$.  The
  method in~(a) uses Case~1 for the top 2~blocks and Case~2 for the
  others.  The method in~(b) uses Case~2 for every block that is added
to the stack.}

\label{fig:9G22}

\end{figure}

The second reason that equation~\ref{eqn:9G21} is important is that we
now know that $S_4 > 1$, which means that we no longer have to worry
about Case~1 for $n > 4$ since Case~1 never achieves spread greater
than~1.  Moreover, even for~$n \le 4$, we have now seen that the
spread achieved by Case~1 never exceeds the spread achieved by Case~2,
and they can be equal only for $n = 1$ and $n = 2$.  This means that
\begin{equation}\label{eqn:9G23}
    S_n = S_{n - 1} + \frac{1}{2n}
\end{equation}
for all $n > 1$ since we have shown that the best spread can always be
achieved using Case~2.
\end{editingnotes}


\end{problem}

%%%%%%%%%%%%%%%%%%%%%%%%%%%%%%%%%%%%%%%%%%%%%%%%%%%%%%%%%%%%%%%%%%%%%
% Problem ends here
%%%%%%%%%%%%%%%%%%%%%%%%%%%%%%%%%%%%%%%%%%%%%%%%%%%%%%%%%%%%%%%%%%%%%

\endinput


