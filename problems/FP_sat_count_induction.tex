\documentclass[problem]{mcs}

\begin{pcomments}
  \pcomment{FP_sat_count_induction}
  \pcomment{final.S05}
\end{pcomments}

\pkeywords{
  induction
  satisfiability
  predicate
  proposition
}

%%%%%%%%%%%%%%%%%%%%%%%%%%%%%%%%%%%%%%%%%%%%%%%%%%%%%%%%%%%%%%%%%%%%%
% Problem starts here
%%%%%%%%%%%%%%%%%%%%%%%%%%%%%%%%%%%%%%%%%%%%%%%%%%%%%%%%%%%%%%%%%%%%%

\begin{problem}
Consider the following sequence of predicates:
%
\[\begin{array}{ll}
Q_1(x_1)& \eqdef\  x_1 \\
Q_2(x_1, x_2)& \eqdef\  x_1 \QIMP x_2 \\
Q_3(x_1, x_2, x_3)& \eqdef\  (x_1 \QIMP x_2) \QIMP x_3 \\
Q_4(x_1, x_2, x_3, x_4)& \eqdef\  ((x_1 \QIMP x_2) \QIMP x_3) \QIMP x_4 \\
Q_5(x_1, x_2, x_3, x_4, x_5)& \eqdef\  (((x_1 \QIMP x_2) \QIMP x_3) \QIMP x_4) \QIMP x_5 \\
\dots \qquad & \qquad \qquad \dots
\end{array}\]
%
Let $T_n$ be the number of different true/false settings of the
variables $x_1, x_2, \dots, x_n$ for which $Q_n(x_1, x_2, \dots,
x_n)$ is true.  For example, $T_2 = 3$ since $Q_2(x_1, x_2)$ is true
for 3 different settings of the variables $x_1$ and $x_2$:
%
\[
\begin{array}{cc|c}
x_1 & x_2 & Q_2(x_1, x_2) \\ \hline
T & T & T \\
T & F & F \\
F & T & T \\
F & F & T
\end{array}
\]

\bparts

\ppart Express $T_{n+1}$ in terms of $T_n$, assuming $n \geq 1$.

\begin{solution}
We have:
%
\[
Q_{n+1}(x_1, x_2, \dots, x_{n+1})
    = Q_n(x_1, x_2, \dots, x_n) \QIMP x_{n+1}
\]
%
If $x_{n+1}$ is true, then $Q_{n+1}$ is true for all $2^n$ settings of
the variables $x_1, x_2, \dots, x_n$.  If $x_{n+1}$ is false, then
$Q_{n+1}$ is true for all settings of $x_1, x_2, \dots, x_n$
\textit{except} for the $T_n$ settings that make $Q_n$ true.  Thus,
altogether we have:
%
\[
T_{n+1} = 2^n + 2^n - T_n = 2^{n+1} - T_n
\]

\end{solution}

\ppart Use induction to prove that $T_n = \frac{1}{3}(2^{n+1} +
(-1)^n)$ for $n \geq 1$.  You may assume your answer to the previous
part without proof.

\begin{solution}
The proof is by induction.  Let $P(n)$ be the proposition
that $T_n = (2^{n+1} + (-1)^n) / 3$.

\noindent \textit{Base case:} There is a single setting of $x_1$ that
makes $Q_1(x_1) = x_1$ true, and $T_1 = (2^{1+1} + (-1)^1) / 3 = 1$.
Therefore, $P(0)$ is true.

\noindent \textit{Inductive step:} For $n \geq 0$, we assume $P(n)$
and reason as follows:
%
\begin{align*}
T_{n+1}
    & = 2^{n+1} - T_n \\
    & = 2^{n+1} - \paren{\frac{2^{n+1} + (-1)^n}{3}} \\
    & = \frac{2^{n+2} + (-1)^{n+1}}{3}
\end{align*}
%
The first step uses the result from the previous problem part, the
second uses the induction hypothesis $P(n)$, and the third is
simplification.  This implies that $P(n+1)$ is true.  By the principle
of induction, $P(n)$ is true for all $n \geq 1$.
\end{solution}

\eparts

\end{problem}

%%%%%%%%%%%%%%%%%%%%%%%%%%%%%%%%%%%%%%%%%%%%%%%%%%%%%%%%%%%%%%%%%%%%%
% Problem ends here
%%%%%%%%%%%%%%%%%%%%%%%%%%%%%%%%%%%%%%%%%%%%%%%%%%%%%%%%%%%%%%%%%%%%%

\endinput
