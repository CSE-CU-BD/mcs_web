\documentclass[problem]{mcs}

\begin{pcomments}
  \pcomment{FP_sat_count_induction}
  \pcomment{same as FP_sat_count_genfunc but demands induction}
  \pcomment{final.S05}
  \pcomment{edited ARM 5/23/12, revised to justify given recurrence 3/9/17}
\end{pcomments}

\pkeywords{
  induction
  satisfiability
  predicate
  proposition
}

%%%%%%%%%%%%%%%%%%%%%%%%%%%%%%%%%%%%%%%%%%%%%%%%%%%%%%%%%%%%%%%%%%%%%
% Problem starts here
%%%%%%%%%%%%%%%%%%%%%%%%%%%%%%%%%%%%%%%%%%%%%%%%%%%%%%%%%%%%%%%%%%%%%

\begin{problem}
We examine a series of propositional formulas
$F_1,\allowbreak F_2,\dots,F_n,\dots$ containing propositional variables
$P_1,\allowbreak P_2,\dots,P_n,\dots$ constructed as follows
\[\begin{array}{ll}
F_1(P_1)& \eqdef\quad   P_1 \\
F_2(P_1, P_2)& \eqdef\quad   P_1 \QIMP P_2 \\
F_3(P_1, P_2, P_3)& \eqdef\quad   (P_1 \QIMP P_2) \QIMP P_3 \\
F_4(P_1, P_2, P_3, P_4)& \eqdef\quad   ((P_1 \QIMP P_2) \QIMP P_3) \QIMP P_4 \\
F_5(P_1, P_2, P_3, P_4, P_5)& \eqdef\quad   (((P_1 \QIMP P_2) \QIMP P_3) \QIMP P_4) \QIMP P_5 \\
 & \vdots
\end{array}\]
Let $T_n$ be the number of different true/false settings of the
variables $P_1, P_2, \dots, P_n$ for which the statement $F_n(P_1, P_2, \dots,
P_n)$ is true.  For example, $T_2 = 3$ since $F_2(P_1, P_2)$ is true
for 3 different settings of the variables $P_1$ and $P_2$:
%
\[
\begin{array}{cc|c}
P_1 & P_2 & F_2(P_1, P_2) \\ \hline
T & T & T \\
T & F & F \\
F & T & T \\
F & F & T
\end{array}
\]

\bparts

\ppart Explain why

\begin{equation}\label{Tn+12n2n}
T_{n+1} = 2^{n+1} - T_n.
\end{equation}

\examspace[1.5in]

\begin{solution}
We have:
%
\[
F_{n+1}(P_1, P_2, \dots, P_{n+1})
    = F_n(P_1, P_2, \dots, P_n) \QIMP P_{n+1}
\]
%
If $P_{n+1}$ is true, then $F_{n+1}$ is true for all $2^n$ settings of
the variables $P_1, P_2, \dots, P_n$.  If $P_{n+1}$ is false, then
$F_{n+1}$ is true for all $2^n$ settings of $P_1, P_2, \dots, P_n$
\textit{except} for the $T_n$ settings that make $F_n$ true.  Thus,
altogether we have:
\[
T_{n+1} = 2^n + (2^n - T_n) = 2^{n+1} - T_n.
\]

\end{solution}

\ppart Use induction to prove that 
\begin{equation}\tag{*}
T_n = \frac{2^{n+1} +(-1)^n}{3}
\end{equation}
for $n \geq 1$.

\begin{solution}
\begin{proof}
The proof is by induction with $P(n)$ given by equation~(*) as the
induction hypothesis.

\inductioncase{Base case:} $(n=1)$.  There is a single setting of
$P_1$ that makes $F_1(P_1) = P_1$ true, so
\[
T_1 = 1 = \frac{2^{1+1} + (-1)^1}{3},
\]
which proves $P(1)$.

\inductioncase{Inductive step:} For $n \geq 1$, we assume~(*) and
reason as follows:
\begin{align*}
T_{n+1}
    & = 2^{n+1} - T_n
         & \text{(by~\eqref{Tn+12n2n})}\\
    & = 2^{n+1} - \frac{2^{n+1} + (-1)^n}{3}
         & \text{(by ind. hyp.)}\\
    & = \frac{3\cdot 2^{n+1} - 2^{n+1} - (-1)^{n}}{3}\\
    & = \frac{2^{n+2} + (-1)^{n+1}}{3},
\end{align*}
so~$P(n+1)$ holds.
\end{proof}

\end{solution}

\eparts

\end{problem}

%%%%%%%%%%%%%%%%%%%%%%%%%%%%%%%%%%%%%%%%%%%%%%%%%%%%%%%%%%%%%%%%%%%%%
% Problem ends here
%%%%%%%%%%%%%%%%%%%%%%%%%%%%%%%%%%%%%%%%%%%%%%%%%%%%%%%%%%%%%%%%%%%%%

\endinput
