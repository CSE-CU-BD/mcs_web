\documentclass[problem]{mcs}

\begin{pcomments}
  \pcomment{FP_sat_count_induction}
  \pcomment{final.S05}
  \pcomment{edited ARM 5/23/12}
\end{pcomments}

\pkeywords{
  induction
  satisfiability
  predicate
  proposition
}

%%%%%%%%%%%%%%%%%%%%%%%%%%%%%%%%%%%%%%%%%%%%%%%%%%%%%%%%%%%%%%%%%%%%%
% Problem starts here
%%%%%%%%%%%%%%%%%%%%%%%%%%%%%%%%%%%%%%%%%%%%%%%%%%%%%%%%%%%%%%%%%%%%%

\begin{problem}
Consider the following sequence of predicates:
%
\[\begin{array}{ll}
Q_1(x_1)& \eqdef\  x_1 \\
Q_2(x_1, x_2)& \eqdef\  x_1 \QIMP x_2 \\
Q_3(x_1, x_2, x_3)& \eqdef\  (x_1 \QIMP x_2) \QIMP x_3 \\
Q_4(x_1, x_2, x_3, x_4)& \eqdef\  ((x_1 \QIMP x_2) \QIMP x_3) \QIMP x_4 \\
Q_5(x_1, x_2, x_3, x_4, x_5)& \eqdef\  (((x_1 \QIMP x_2) \QIMP x_3) \QIMP x_4) \QIMP x_5 \\
\dots \qquad & \qquad \qquad \dots
\end{array}\]
%
Let $T_n$ be the number of different true/false settings of the
variables $x_1, x_2, \dots, x_n$ for which $Q_n(x_1, x_2, \dots,
x_n)$ is true.  For example, $T_2 = 3$ since $Q_2(x_1, x_2)$ is true
for 3 different settings of the variables $x_1$ and $x_2$:
%
\[
\begin{array}{cc|c}
x_1 & x_2 & Q_2(x_1, x_2) \\ \hline
T & T & T \\
T & F & F \\
F & T & T \\
F & F & T
\end{array}
\]

\bparts

\ppart Express $T_{n+1}$ in terms of $T_n$, assuming $n \geq 1$.

\begin{solution}
We have:
%
\[
Q_{n+1}(x_1, x_2, \dots, x_{n+1})
    = Q_n(x_1, x_2, \dots, x_n) \QIMP x_{n+1}
\]
%
If $x_{n+1}$ is true, then $Q_{n+1}$ is true for all $2^n$ settings of
the variables $x_1, x_2, \dots, x_n$.  If $x_{n+1}$ is false, then
$Q_{n+1}$ is true for all settings of $x_1, x_2, \dots, x_n$
\textit{except} for the $T_n$ settings that make $Q_n$ true.  Thus,
altogether we have:
\begin{equation}\label{Tn+12n2n}
T_{n+1} = 2^n + 2^n - T_n = 2^{n+1} - T_n
\end{equation}

\end{solution}

\ppart Use induction to prove that $T_n = \frac{1}{3}(2^{n+1} +
(-1)^n)$ for $n \geq 1$.  You may assume your answer to the previous
part without proof.

\begin{solution}
\begin{proof}
The proof is by induction.  Let $P(n)$ be the proposition
that
\[
T_n = \frac{2^{n+1} + (-1)^n}{3}.
\]

\inductioncase{Base case:} $(n=1)$.  There is a single setting of
$x_1$ that makes $Q_1(x_1) = x_1$ true, and $T_1 = (2^{1+1} + (-1)^1)
/ 3 = 1$.  Therefore, $P(1)$ is true.

\inductioncase{Inductive step:} For $n \geq 1$, we assume $P(n)$ and
reason as follows:
\begin{align*}
T_{n+1}
    & = 2^{n+1} - T_n
         & \text{(by~\eqref{Tn+12n2n})}\\
    & = 2^{n+1} - \frac{2^{n+1} + (-1)^n}{3}
         & \text{(by ind. hyp.)}\\
    & = \frac{3\cdot 2^{n+1} - 2^{n+1} + (-1)(-1)^{n}}{3}\\
    & = \frac{2^{n+2} + (-1)^{n+1}}{3}.
\end{align*}
That is, $P(n-1)$ holds.

\end{proof}
\end{solution}

\eparts

\end{problem}

%%%%%%%%%%%%%%%%%%%%%%%%%%%%%%%%%%%%%%%%%%%%%%%%%%%%%%%%%%%%%%%%%%%%%
% Problem ends here
%%%%%%%%%%%%%%%%%%%%%%%%%%%%%%%%%%%%%%%%%%%%%%%%%%%%%%%%%%%%%%%%%%%%%

\endinput
