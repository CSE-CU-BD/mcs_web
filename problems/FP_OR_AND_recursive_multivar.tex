\documentclass[problem]{mcs}

\begin{pcomments}
  \pcomment{FP_OR_AND_recursive_multivar}
  \pcomment{similar to CP_XOR_AND_recursive, CP_XOR_AND_formulas}
  \pcomment{ARM 3/9/17}
\end{pcomments}

\pkeywords{
  recursive
  structural_induction
  OR
  AND
  constant
  formula}

\newcommand{\MVAO}{\text{MVAO}}

\begin{problem}
A class of propositional formulas called the
Multivariable \QAND-\QOR\ (\MVAO) formulas are defined recursively as follows:
\begin{definition*}
\inductioncase{Base cases}: A single propositional variable, the
constants \True\ and \False are \MVAO\ formulas.

\inductioncase{Constructor cases}: If $G,H \in \MVAO$, then $G \QAND
H$ and $G \QOR H$ are \MVAO's.
\end{definition*}

For example,
\[
 (((P \QOR Q) \QAND P) \QOR (R \QAND \True)) \QOR (Q \QOR \False)
\]
is a \MVAO.

\begin{definition*}
A propositional formula $G$ is \emph{\textbf{weakly decreasing}} when
substituting the constant $\False$ for some occurrences of its
variables makes the formula ``more false.''  More precisely, if $G^f$
is the result of replacing some occurrences of variables in $G$ by
\False, then any truth assignment that makes $G$ false also makes
$G^f$ false.
\end{definition*}

\begin{staffnotes}
So $G$ is weakly decreasing iff $[G^f \QIMP G]$ is valid.  However,
this fact did not help me find a better proof than the one below.  Let
me know if you come up with one---ARM 3/9/17.
\end{staffnotes}

For example, the formula consisting of a single variable $P$ is weakly
decreasing since $P^f$ is the formula \False.  The formula $\bar{P}$
is not weakly decreasing since $\paren{\bar{P}}^f$ is the formula
$\bar{\False}$ which is true even under a truth assignment where
$\bar{P}$ is false.

\bigskip
\textbf{Prove by structural induction} \iffalse on the definition of \MVAO \fi
that every \MVAO\ formula $F$ is weakly decreasing.

\begin{solution}
\begin{proof}
\inductioncase{Base case}:
\begin{itemize}
\item ($F$ is a single variable). This was just given above.

\item ($F$ is a constant \True\ or \False).  Then $F^f$ is the same as
  $F$, so which trivially implies that $F^f$ is false whenever $F$ is
  false.
\end{itemize}

\inductioncase{Constructor case}: ($G\ \text{is}\ [F \QAND H]$).
Replacing some variable occurrences in $G$ by \False\ means that some
occurrences in each of $F$ and $H$ are replaced by \False.  In other
words, any $G^f$ equals $F^f \QAND H^f$ for some $F^f$ and $H^f$.

Now by definition of $\QAND$, $G^f$ is true under some true some
assignment $A$ iff both $F^f$ and $H^f$ are true under $A$.  By
structural induction hypothesis, if $F^f$ and $H^f$ are true under
$A$, then $F$ and $H$ are true under $A$.  Hence $F \QAND H$, that is
$G$, is true under $A$.  This shows that $G$ is weakly decreasing.

\inductioncase{Constructor case}: ($G\ \text{is}\ [F \QOR H]$).  As in
the \QAND\ case, we have that $G^f$ equals $F^f \QOR H^f$.

Now by definition of $\QOR$, $G^f$ is true under some true some
assignment $A$ iff either $F^f$ or $H^f$ is true under $A$.  Now if
$F^f$ is true under $A$, then by structural induction hypothesis, $F$
is true under $A$.  Hence $F \QOR H$, that is $G$, is true under $A$.
The same argument holds if $H^f$ is true under $A$.  This shows that
in any case, $G$ is weakly decreasing.
\end{proof}
\end{solution}

\end{problem}

\endinput
