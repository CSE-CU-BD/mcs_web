\documentclass[problem]{mcs}

\begin{pcomments}
  \pcomment{CP_maxweight_edge}
  \pcomment{shorter form of FP_maxweight_edge}
  \pcomment{ARM 2/20/17}
\end{pcomments}

\pkeywords{
  tree
  spanning_tree
  cycle
  MST
  weight
  maximum
}

%%%%%%%%%%%%%%%%%%%%%%%%%%%%%%%%%%%%%%%%%%%%%%%%%%%%%%%%%%%%%%%%%%%%%
% Problem starts here
%%%%%%%%%%%%%%%%%%%%%%%%%%%%%%%%%%%%%%%%%%%%%%%%%%%%%%%%%%%%%%%%%%%%%

\begin{problem}

Note that the two parts of this problem do not depend on each other.

\bparts

\ppart Let $G$ be a connected simple graph.  Prove that every spanning
tree of $G$ must include every cut edge of $G$.

\begin{solution}
\begin{proof}
If $e$ is a cut edge, then $G-e$ is not connected and so has no
spanning tree.  But a spanning tree of $G$ that did not include edge
$e$ would also be a spanning tree of $G-e$.  So every spanning tree of
$G$ must include $e$.
\end{proof}
\end{solution}

\ppart Suppose a connected, weighted graph $G$ has a unique
maximum-weight edge $e$.  Show that if $e$ is in a minimum weight
spanning tree of $G$, then $e$ is a cut edge.

\begin{solution}
Suppose $T$ is a MST of $G$ and $e$ is an edge of $T$.  Now suppose to
the contrary that $e$ is not a cut edge.  Then $e$ is on a cycle in
$G$, so the other edges in the cycle form a path between the endpoints
of $e$.  Now if $f$ is any edge on this path, then $T-e+f$ is a also a
spanning tree because it is connected and has the same number of edges
as $T$.  Moreover, $T-e+f$ has lower weight than $T$ because the
weight of $e$ is greater than the weight of $f$, contradicting the
fact that $T$ is an MST of $G$.
\end{solution}

\eparts
\end{problem}

%%%%%%%%%%%%%%%%%%%%%%%%%%%%%%%%%%%%%%%%%%%%%%%%%%%%%%%%%%%%%%%%%%%%%
% Problem ends here
%%%%%%%%%%%%%%%%%%%%%%%%%%%%%%%%%%%%%%%%%%%%%%%%%%%%%%%%%%%%%%%%%%%%%

\endinput
