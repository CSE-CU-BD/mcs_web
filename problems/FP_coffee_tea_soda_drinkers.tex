\documentclass[problem]{mcs}

\begin{pcomments}
  \pcomment{new problem for final exam by YingZ, May 2014}
\end{pcomments}

\pkeywords{
  expectation
  conditional expectation
  linearity of expectation
  deviation
  sampling
  confidence
  tree diagram (4-step method)
  sample_space
  outcome
  probability space
}

%%%%%%%%%%%%%%%%%%%%%%%%%%%%%%%%%%%%%%%%%%%%%%%%%%%%%%%%%%%%%%%%%%%%%
% Problem starts here
%%%%%%%%%%%%%%%%%%%%%%%%%%%%%%%%%%%%%%%%%%%%%%%%%%%%%%%%%%%%%%%%%%%%%


\begin{problem}

%\begin{staffnotes}
% Point distribution goes here!
%\end{staffnotes}

Members of MIT's Computer Science and Artificial Intelligence
Laboratory (CSAIL) are known to thrive on caffeine.  Sixty percent are
devoted coffee drinkers, who drink coffee and water exclusively and
with equal probability in the lab.  Ten percent are devoted tea
drinkers, who drink tea 100\% of the time.  Another ten percent of lab members are devoted soda drinkers, who drink soda 40\% of the time and water or juice with equal probability the rest of the time.  The remaining
twenty percent drink all three types of caffeinated beverages randomly 60\% of the time, and water and juice with equal probability the rest of the time.

\bparts

\ppart As a 6.042 student, you go to CSAIL for office hours.  As soon
as you get off the elevator, you see two people each holding a cup in
front of the white board in discussion of something mod something.  What
is the probability that both of them are drinking coffee?

\hint Draw a tree diagram first.

\examspace[2in]

\begin{solution}
$0.3+0.04=0.34$
\end{solution}

\ppart Assuming the two filled their cups independently.  What is the probability that their cups contain the same kind of beverage? 

\begin{solution}
\end{solution}

\ppart Next, you bump into someone holding a cup of hot tea, due to either your disorientation or his sleep deprivation, or both.  If he is a devoted tea drinker, he will offer you a cup of fine tea as he makes another cup for himself; else, you will have to find him an unopened tea bag.  What is the probability that you get to enjoy a cup of fine tea?

\begin{solution}
$\frac{0.1}{0.1+0.2 \cdot 0.6 \cdot 1/3} = \frac{0.1}{0.14} = \frac{5}{7}$
\end{solution}   

\ppart At office hours, as other students ask about a problem you already solved, you imagine pulling all 500 cups that people in CSAIL are drinking from and then counting the number of cups with just water.  Let $W$ be a random variable for such number.  What is the Chebyshev bound on the probability that $W \ge 100$?

\begin{solution}
\end{solution}

\ppart Not surprisingly, every student gains from office hours differently.  Let $K_{i}$ be the random variable for the amount of knowledge gained by the $i$-th student at the end of the office hours.  Assuming $n$ students are at the office hours, and $K_{i}$'s are pairwise independent but all have mean $\mu$ and standard deviation $\sigma$.  What is the upper bound on the probability that the average gain among the $n$ students deviates from the expected gain of individual students by at least $g$?  You could provide your answer in an expression containing any or none of $g$, $\mu$, $\sigma$, $n$. 

\begin{solution}
Let $\frac{S_{n}}{n}$ denotes the average gain among the $n$ students, where $S_{n}::=\sum_{i=1}^{n} K_{i}$.  By Pairwise Independent Sampling Theorem, $\pr{|\frac{S_{n}}{n} - \mu| \ge g} \le \frac{1}{n} \cdot \frac{\sigma^{2}}{g^{2}}$.

\end{solution}

\eparts

\end{problem}

%%%%%%%%%%%%%%%%%%%%%%%%%%%%%%%%%%%%%%%%%%%%%%%%%%%%%%%%%%%%%%%%%%%%%
% Problem ends here
%%%%%%%%%%%%%%%%%%%%%%%%%%%%%%%%%%%%%%%%%%%%%%%%%%%%%%%%%%%%%%%%%%%%%

\endinput
