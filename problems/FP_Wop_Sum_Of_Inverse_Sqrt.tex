\documentclass[problem]{mcs}

\begin{pcomments}
  \pcomment{FP_Wop_Sum_Of_Inverse_Sqrt}
  \pcomment{IS IT SUM or PRODUCT? NEEDS EDITING AND PROOFREAD}
  \pcomment{unknown provenance}
  \pcomment{partially edited but not checked by ARM 12/9/17}
\end{pcomments}

\pkeywords{
  WOP
  sum
  square_root
}

\begin{problem}

Prove by WOP that for any natural number $n \geq 2$,
\begin{equation}\label{1f1s2}
\paren{ 1 - \frac{1}{\sqrt{2}}} \paren{ 1 - \frac{1}{\sqrt{3}}} \dots
\paren{1 - \frac{1}{\sqrt{n}}} < \frac{2}{n^2}.
\end{equation}

\begin{solution}
Let $C$ be the set of counterexamples to the claim~\eqref{1f1s2}:
\[
C \eqdef \set{n \geq 2 \in \nngint \suchthat \QNOT(P(n))}.
\]
Assume for the sake of contradiction that $C$ is not empty.  Then by
WOP, there is a least element $m \in C$.

Now~\eqref{1f1s2} holds for $n=2$:
\[
\textbf{LHS of~\eqref{1f1s2} = 1 - \frac{1}{\sqrt{2}} = \frac{\sqrt{2}-1}{\sqrt{2}}
 = \frac{1}{2 + \sqrt{2}} < \frac{1}{2} = \frac{2}{2^2} = \textbf{RHS}.
\]
Therefore, $m>2$.

Since $m$ is a minimal, we have that~\eqref{1f1s2} holds for $n = m-1$.

$$\textbf{LHS of } P(k+1) =  \left( 1 - \frac{1}{\sqrt{2}} \right) \left( 1 - \frac{1}{\sqrt{3}} \right) \ldots \left( 1 - \frac{1}{\sqrt{k}} \right) \left( 1 - \frac{1}{\sqrt{k+1}} \right) $$

$$ = \textbf{(LHS of } P(k)) . \left( 1 - \frac{1}{\sqrt{k+1}} \right) $$

$$ < \textbf{(RHS of } P(k)) .\left( 1 - \frac{1}{\sqrt{k+1}} \right) $$

$$ = \frac{2}{k^2}  .\left( 1 - \frac{1}{\sqrt{k+1}} \right) $$

$$ =\frac{2(\sqrt{k+1} -1)(\sqrt{k+1} + 1)}{k^2 \sqrt{k+1} (\sqrt{k+1} +1)}$$

$$ =\frac{2(k+1-1)}{k^2 \sqrt{k+1} (\sqrt{k+1} +1) }$$

$$ =\frac{2}{k (k+1 +\sqrt{k+1}) }$$

$$ =\frac{2}{k (k+1) +k\sqrt{k+1} }$$

$$ \leq \frac{2}{k (k+1) +\sqrt{k+1} \sqrt{k+1} } \textrm{ (for }k \geq 2)$$

$$ = \frac{2}{k (k+1) + (k+1) }$$

$$ = \frac{2}{ (k+1)^2 }$$

$$ = \textbf{(RHS of } P(k+1))$$

Contradiction. Therefore $C$ is empty and $P(n)$ holds for all $n \geq 2 \in \mathbb{N}$.

\end{solution}

\eparts

\end{problem}

%%%%%%%%%%%%%%%%%%%%%%%%%%%%%%%%%%%%%%%%%%%%%%%%%%%%%%%%%%%%%%%%%%%%%
% Problem ends here
%%%%%%%%%%%%%%%%%%%%%%%%%%%%%%%%%%%%%%%%%%%%%%%%%%%%%%%%%%%%%%%%%%%%%

\endinput
