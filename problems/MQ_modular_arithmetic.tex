\documentclass[problem]{mcs}

\begin{pcomments}
  \pcomment{MQ_modular_arithmetic}
  \pcomment{from spring06 Quiz2}
\end{pcomments}

\pkeywords{
  modular_arithmetic
  multiplicative_inverse
 }


%%%%%%%%%%%%%%%%%%%%%%%%%%%%%%%%%%%%%%%%%%%%%%%%%%%%%%%%%%%%%%%%%%%%%
% Problem starts here
%%%%%%%%%%%%%%%%%%%%%%%%%%%%%%%%%%%%%%%%%%%%%%%%%%%%%%%%%%%%%%%%%%%%%

\begin{problem}
 
\bparts

\ppart
Prove that $22^{12001}$ has a multiplicative inverse modulo 
175.

\examspace[2in]
\begin{solution}
Since $\gcd(22^{12001},175) =
       \gcd(2^{12001} \cdot 11^{12001},5^{2} \cdot 7) =
       1$, $22^{12001}$ has a multiplicative inverse modulo 175.
\end{solution}

\ppart What is the value of $\phi(175)$, where $\phi$ is Euler's
function?

\examspace[2in]
\begin{solution}
Noting that $175=5^2\cdot 7$.  It follows
that $\phi(175)=(5^2 - 5^1)(7-1)= 20 \cdot 6 =120$.
\end{solution}

\ppart What is the remainder of $22^{12001}$ divided by 175?

\begin{solution}Since 22 and 175 are relatively prime, we have by Euler's
Theorem that $22^{120} \equiv 1 \pmod {175}$, and so
\[
22^{1201} = \paren{22^{120}}^{100}\cdot 22 \equiv 1^{100}\cdot 22 \equiv 22 \pmod {175}.
\]
\end{solution}

\eparts
\end{problem}
