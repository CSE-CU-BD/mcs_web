\documentclass[problem]{mcs}

\begin{pcomments}
  \pcomment{PS_drunken_sailor}
  \pcomment{from: S09.ps11}
  \pcomment{revised for F12.ps10 and again for S14.ps11 at the end; renamed PS_sailor_random_walk 4/29/14}
\end{pcomments}

\pkeywords{
  probability
  random_variables
  random_walk
  distribution - density function
  binomial_distribution
}

%%%%%%%%%%%%%%%%%%%%%%%%%%%%%%%%%%%%%%%%%%%%%%%%%%%%%%%%%%%%%%%%%%%%%
% Problem starts here
%%%%%%%%%%%%%%%%%%%%%%%%%%%%%%%%%%%%%%%%%%%%%%%%%%%%%%%%%%%%%%%%%%%%%

\begin{problem}
  An over-caffeinated sailor of Tech Dinghy wanders along Seaside Boulevard, which conveniently consists
  of the points along the $x$ axis with integral coordinates.  In each
  step, the sailor moves one unit left or right along the $x$ axis.  A
  particular \term{path} taken by the sailor can be described by a
  sequence of ``left'' and ``right'' steps.  For example,
  $\langle{left,left,right}\rangle$ describes the walk that goes left twice then
  goes right once.

  We model this scenario with a random walk graph whose vertices are the
  integers and with edges going in each direction between
  consecutive integers.  All edges are labelled $1/2$.

  The sailor begins his random walk at the origin.  This is described by
  an initial distribution that labels the origin with probability 1 and
  all other vertices with probability 0.  After one step, the sailor is
  equally likely to be at location 1 or $-1$, so the distribution after
  one step gives label 1/2 to the vertices 1 and $-1$ and labels all other
  vertices with probability 0.

\bparts 

\ppart Give the distributions after the 2nd, 3rd, and 4th step by filling in the
table of probabilities below, where omitted entries are 0.  For each row,
write all the nonzero entries so they have the same denominator.
\begin{center}
\begin{tabular}{l|ccccccccc}
  & \multicolumn{9}{c}{location} \\
  & -4 & -3 & -2 & -1 & 0 & 1 & 2 & 3 & 4 \\ \hline\hline
  initially & & & & & $1$ & & & & \\
  after 1 step & & & & $1/2$ & 0 & $1/2$ & & & \\
  after 2 steps & & & ? & ? & ? & ? & ? & & \\
  after 3 steps & & ? & ? & ? & ? & ? & ? & ? &  \\
  after 4 steps & ? & ? & ? & ? & ? & ? & ? & ? & ?
\end{tabular}
\end{center}
  
\begin{solution} \ 
\begin{center}
  \begin{tabular}{l|ccccccccc}
    & \multicolumn{9}{c}{location} \\
    & -4 & -3 & -2 & -1 & 0 & 1 & 2 & 3 & 4 \\ \hline\hline
    initially & & & & & $1$ & & & & \\
    after 1 step & & & & $1/2$ & 0 & $1/2$ & & & \\
    after 2 steps & & & $1/4$ & 0 & $2/4$ & 0 & $1/4$ & & \\
    after 3 steps & & $1/8$ & 0 & $3/8$ & 0 & $3/8$ & 0 & $1/8$ &  \\
    after 4 steps & $1/16$ & 0 & $4/16$ & 0 & $6/16$ & 0 & $4/16$ & 0 & $1/16$
  \end{tabular}
\end{center}
\end{solution}

\ppart \label{ppart:iright}\ Help the staff of the Sailing Pavilion locate the sailor by answering the following questions.  Provide your derivations and reasoning. 

\begin{enumerate}

\item  What is the final location of a $t$-step path that moves right exactly
  $i$ times?

\item How many different paths are there that end at that
location?

\item What is the probability that the sailor ends at this location?

\end{enumerate}

\begin{solution}
  If he takes $i$ steps to the right, then he takes $t-i$ steps to the
  left.  Since steps left and right cancel, he nets $i - (t-i)
  = 2i-t$ steps to the right, ending at location $2i-t$.  

  The number of paths is the number of length-$t$ sequences with $i$ ``right''s, which is $\binom{t}{i}$.  

  Each path is equally likely, so he takes the given path with
  probability $1/2^t$.  Thus, he ends at the location $2i-t$ with
  probability
\[
2^{-t}\binom{t}{i}.
\]
\end{solution}

\ppart Let $L$ be the random variable giving the sailor's location after
$t$ steps, and let $B ::= (L + t)/2$.  Use the answer to
part~\ref{ppart:iright} to show that $B$ has an unbiased binomial density
function. 

\begin{solution}
  From part~\ref{ppart:iright}, we have $\pr{L = 2x-t} =  2^{-t}
  \binom{t}{x}$ for $0 \leq x \leq t$, so:
  \begin{align*}
   PDF_B(x) &= \pr{B = x} \\
    &= \pr{(L+t)/2 = x} \\
    &= \pr{L = 2x - t}\\
    &= \frac{1}{2^t} \binom{t}{x}  
  \end{align*}
 This distribution is binomial. $B$ represents the probability of taking exactly $x$ steps to the right, which is the same as flipping a fair coin that comes up heads exactly $x$ times. Notice that the probability that $B=x$ is the same as the probability that the path ends at location $L=2x-t$, so our answer matches that from part~\ref{ppart:iright}.
\end{solution}

\eparts
\end{problem}

%%%%%%%%%%%%%%%%%%%%%%%%%%%%%%%%%%%%%%%%%%%%%%%%%%%%%%%%%%%%%%%%%%%%%
% Problem ends here
%%%%%%%%%%%%%%%%%%%%%%%%%%%%%%%%%%%%%%%%%%%%%%%%%%%%%%%%%%%%%%%%%%%%%

\endinput
