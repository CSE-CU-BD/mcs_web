\documentclass[problem]{mcs}

\begin{pcomments}
  \pcomment{PS_sailor_random_walk}
  \pcomment{revision 4/29/14 of PS_drunken_sailor}
  \pcomment{revised for F12.ps10, S14.ps11}
  \pcomment{revised 5/2/17 ARM to avoid random-walk graphs}
\end{pcomments}

\pkeywords{
  random_variable
  random_walk
  density_function
  pdf
  binomial_distribution
}

%%%%%%%%%%%%%%%%%%%%%%%%%%%%%%%%%%%%%%%%%%%%%%%%%%%%%%%%%%%%%%%%%%%%%
% Problem starts here
%%%%%%%%%%%%%%%%%%%%%%%%%%%%%%%%%%%%%%%%%%%%%%%%%%%%%%%%%%%%%%%%%%%%%

\begin{problem}
An over-caffeinated sailor of Tech Dinghy wanders along Seaside
Boulevard.  In each step, the sailor randomly moves one unit left or
right with equal probability.

\iffalse
The sailor's
movements can be described by a sequence of ``left'' and ``right''
steps.  For example, $\ang{\text{left, left, right}}$ means the sailor
went left twice then went right once.


We model this scenario with a random walk graph whose vertices are the
integers and with edges going in each direction between consecutive
integers.  All edges are labelled $1/2$.
\fi

We let the sailor's initial position be desiginated as location zero,
with successive positions to the right labelled 1,2,\dots, and
postions to the left labelled -1,-2,\dots.  Let $L_t$ be the random
variable giving the sailor's location after $t$ steps.  Before he
starts, the sailor is known to be at location zero, so
\[
\pdf_{L_0}(n) =
\begin{cases}
1 & \text{if }n = 0,\\
0 & \text{otherwise}.
\end{cases}
\]
After one step, the sailor is equally
likely to be at location 1 or $-1$, so
\[
\pdf_{L_1}(n) =
\begin{cases}
1/2 & \text{if } n = \pm 1,\\
0 & \text{otherwise}.
\end{cases}
\]

\bparts 

\ppart Give the distributions $\pdf_{L_t}$ for $t =2,3,4$ by filling
in the table of probabilities below, where omitted entries are 0.  For
each row, write all the nonzero entries so they have the same
denominator.
\begin{center}
\begin{tabular}{l|ccccccccc}
  & \multicolumn{9}{c}{location} \\
  & -4 & -3 & -2 & -1 & 0 & 1 & 2 & 3 & 4 \\ \hline\hline
  initially & & & & & $1$ & & & & \\
  after 1 step & & & & $1/2$ & 0 & $1/2$ & & & \\
  after 2 steps & & & ? & ? & ? & ? & ? & & \\
  after 3 steps & & ? & ? & ? & ? & ? & ? & ? &  \\
  after 4 steps & ? & ? & ? & ? & ? & ? & ? & ? & ?
\end{tabular}
\end{center}
  
\begin{solution}
\begin{center}
  \begin{tabular}{l|ccccccccc}
    & \multicolumn{9}{c}{location} \\
    & -4 & -3 & -2 & -1 & 0 & 1 & 2 & 3 & 4 \\ \hline\hline
    initially & & & & & $1$ & & & & \\
    after 1 step & & & & $1/2$ & 0 & $1/2$ & & & \\
    after 2 steps & & & $1/4$ & 0 & $2/4$ & 0 & $1/4$ & & \\
    after 3 steps & & $1/8$ & 0 & $3/8$ & 0 & $3/8$ & 0 & $1/8$ &  \\
    after 4 steps & $1/16$ & 0 & $4/16$ & 0 & $6/16$ & 0 & $4/16$ & 0 & $1/16$
  \end{tabular}
\end{center}
\end{solution}

\ppart \label{ppart:iright} Help the staff of the Sailing Pavilion
locate the sailor by answering the following questions.  Provide your
derivations and reasoning.

\begin{enumerate}[i]

\item What is the final location of a $t$-step walk that moves right
  exactly $i$ times?

\item How many different walks are there that end at that location?

\item What is the probability that the sailor ends at this location?

\item Let $B_t ::= (L_t + t)/2$.  Conclude that $B_t$ has an unbiased
  binomial distribution.  In other words, the probability that after
  $t$ steps the sailor finishes exactly $n$ steps to the right is the
  same as the probability that a fair coin flipped $t$ times comes up
  heads exactly $n$ times.
\end{enumerate}

\begin{solution}
If he takes $i$ steps to the right, then he takes $t-i$ steps to the
left.  Since steps left and right cancel, he nets $i - (t-i) = 2i-t$
steps to the right, ending at location $2i-t$.

The number of walks is the number of length-$t$ sequences with $i$
``right''s, which is $\binom{t}{i}$.

Each walk is equally likely, so he takes the given walk with
probability $1/2^t$.  Thus, he ends at the location $2i-t$ with
probability
\[
\binom{t}{i}2^{-t}.
\]

That is,
\[
\pr{L_t = 2i-t} = \binom{t}{i}2^{-t}
\]
for $0 \leq i \leq t$.  Therefore,
\begin{align*}
 \pdf_{B_t}(n)
  &= \pr{B_t = n} \\
  &= \pr{(L_t+t)/2 = n} \\
  &= \pr{L_t = 2n - t}\\
  &=  \binom{t}{n}2^{-t},
\end{align*}
which is an unbiased binomial distribution.
\end{solution}


\iffalse

\ppart Let $B_t ::= (L_t + t)/2$.  Use the answer to
part~\eqref{ppart:iright} to show that $B_t$ has an unbiased binomial
density function.  In other words, the probability that after $t$
steps the sailor finishes exactly $n$ steps to the right is the same a
fair coin flipped $t$ times comes up heads exactly $n$ times.

\begin{solution}
From part~\ref{ppart:iright}, we have $\pr{L_t = 2n-t} =
\binom{t}{n}2^{-t}$ for $0 \leq n \leq t$, so:
\begin{align*}
 \pdf_{B_t}(n)
  &= \pr{B_t = n} \\
  &= \pr{(L_t+t)/2 = n} \\
  &= \pr{L_t = 2n - t}\\
  &=  \binom{t}{n}2^{-t}
\end{align*}
This corresponds to an unbiased binomial distribution.  In other
words, we have shown that the probability of finishing exactly $n$
steps to the right is the same as flipping a fair coin that comes up
heads exactly $n$ times.

%Notice that the probability that $B=n$ is the same as the probability
%that the walk ends at location $L=2n-t$, so our answer matches that
%from part~\ref{ppart:iright}.

\end{solution}
\fi

\eparts
\end{problem}

%%%%%%%%%%%%%%%%%%%%%%%%%%%%%%%%%%%%%%%%%%%%%%%%%%%%%%%%%%%%%%%%%%%%%
% Problem ends here
%%%%%%%%%%%%%%%%%%%%%%%%%%%%%%%%%%%%%%%%%%%%%%%%%%%%%%%%%%%%%%%%%%%%%

\endinput
