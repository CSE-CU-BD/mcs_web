\documentclass[problem]{mcs}

\begin{pcomments}
\pcomment{MQ_simpsurj}
\pcomment{part(b) of TP_simpinj}
\pcomment{From class 2/27/17}

\end{pcomments}

\pkeywords{
 Functions
 Injections
 Surjections
}

%%%%%%%%%%%%%%%%%%%%%%%%%%%%%%%%%%%%%%%%%%%%%%%%%%%%%%%%%%%%%%%%%%%%%
% Problem starts here
%%%%%%%%%%%%%%%%%%%%%%%%%%%%%%%%%%%%%%%%%%%%%%%%%%%%%%%%%%%%%%%%%%%%%

\newcommand{\simpsurj}{\mrelt{simpsurj}}

\begin{problem}
The definition of $A \surj B$ requires that there be a surjective
\emph{function} from $A$ to $B$.  Suppose the function condition was
dropped, so we have a simpler definition $A \simpsurj B$ iff there is
a surjective $[\geq 1\ \text{in}]$ relation from $A$ to $B$.  For each
of the following items, give a simple description of the sets $B$ such
that

\begin{enumerate}[(i)]
\setlength{\itemsep}{0.1in}%
%\setlength{\parskip}{1cm}%

\item $\set{2, 7} \simpsurj B.$

\examspace[0.7in]

\item $\set{\emptyset} \simpsurj B.$

\examspace[0.7in]

\item $\emptyset \simpsurj B.$

\end{enumerate}

\begin{solution}
If $A$ is nonempty, then $A \simpsurj B$ is true for all sets $B$,
because there is an element $e$ in $A$, and the relation that has
arrows from $e$ to every element of $B$ has $[\geq 1\ \text{in}]$ and
is therefore surjective.

Also $\emptyset \simpsurj B$ is true iff $B = \emptyset$.
\end{solution}

\end{problem}

%%%%%%%%%%%%%%%%%%%%%%%%%%%%%%%%%%%%%%%%%%%%%%%%%%%%%%%%%%%%%%%%%%%%%
% Problem ends here
%%%%%%%%%%%%%%%%%%%%%%%%%%%%%%%%%%%%%%%%%%%%%%%%%%%%%%%%%%%%%%%%%%%%%
\endinput
