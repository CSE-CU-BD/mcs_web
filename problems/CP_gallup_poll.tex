\documentclass[problem]{mcs}

\begin{pcomments}
  \pcomment{CP_gallup_poll}
  \pcomment{from: F05 cp14f}
  \pcomment{from: S08 cp13f}
\end{pcomments}

\pkeywords{
}

%%%%%%%%%%%%%%%%%%%%%%%%%%%%%%%%%%%%%%%%%%%%%%%%%%%%%%%%%%%%%%%%%%%%%
% Problem starts here
%%%%%%%%%%%%%%%%%%%%%%%%%%%%%%%%%%%%%%%%%%%%%%%%%%%%%%%%%%%%%%%%%%%%%

\begin{problem}
	
A recent Gallup poll found that 35\% of the adult
population of the United States believes that the theory of evolution is
``well-supported by the evidence.''  Gallup polled $1928$ Americans
selected uniformly and independently at random.  Of these, $675$
asserted belief in evolution, leading to Gallup's estimate that the
fraction of Americans who believe in evolution is $675/1928 \approx
0.350$.  Gallup claims a margin of error of 3 percentage points, that
is, he claims to be confident that his estimate is within 0.03 of the
actual percentage.

\bparts

\ppart
Explain how to use the Binomial Sampling Theorem (available in the
Appendix) to determine the confidence level with which Gallup can make his
claim.  You do not actually have to do the calculation.

\begin{solution}
We let $\epsilon = 0.03$ and $n = 1928$ in the expression on the
righthand side of equation~\eqref{delta bound} in the Binomial Sampling
Theorem.  Evaluating the expression in Scheme (see Appendix), we see that
the probability that the error is 0.03 or more is less than $0.009983 <
0.01$, which means that 99\% of the time\footnote{An exact calculation
shows that the 99\% confidence level could have been achieved by polling
only $n=1863$ people.  With 1928 people, the estimate actually holds at
the $99.17\%$ level.} the fraction $p$ will lie within the specified range
$0.35 \pm 0.03$.
\end{solution}

\ppart
If we accept all of Gallup's polling data and calculations, can we
conclude that there is a high probability that the number of adult
Americans who believe in evolution is $35 \pm 3$ percent?

\begin{solution}
No.  As explained in Notes and lecture, the assertion that
fraction $p$ is in the range $0.35\pm 0.03$ is an assertion of fact that
is either true or false.  The number $p$ is an \emph{constant}.  We
don't know its value, and we don't know if the asserted fact is true or
false, but there is nothing probabilistic about the fact's truth or
falsehood.

We \textit{can} say that either the assertion is true or else a 1-in-100
event occurred during the poll.  Specifically, the unlikely event is
that Gallup's random sample was unrepresentative.  This may convince you
that $p$ is ``probably'' in the range $0.35 \pm 0.03$, but this informal
``probably'' is not a mathematical probability.
\end{solution}

\eparts

\end{problem}

%%%%%%%%%%%%%%%%%%%%%%%%%%%%%%%%%%%%%%%%%%%%%%%%%%%%%%%%%%%%%%%%%%%%%
% Problem ends here
%%%%%%%%%%%%%%%%%%%%%%%%%%%%%%%%%%%%%%%%%%%%%%%%%%%%%%%%%%%%%%%%%%%%%

\endinput
