\documentclass[problem]{mcs}

\begin{pcomments}
  \pcomment{CP_gallup_poll}
  \pcomment{from: F05 cp14f, S08 cp13f}
  \pcomment{revised a lot by ARM 12/5/09}
\end{pcomments}

\pkeywords{
 pairwise_independent
 sampling
 variance
 estimate
 confidence
 probability
 uniform
 independent
}

%%%%%%%%%%%%%%%%%%%%%%%%%%%%%%%%%%%%%%%%%%%%%%%%%%%%%%%%%%%%%%%%%%%%%
% Problem starts here
%%%%%%%%%%%%%%%%%%%%%%%%%%%%%%%%%%%%%%%%%%%%%%%%%%%%%%%%%%%%%%%%%%%%%

\begin{problem}
	
  A recent Gallup poll found that 35\% of the adult population of the
  United States believes that the theory of evolution is ``well-supported
  by the evidence.''  Gallup polled $1928$ Americans selected uniformly
  and independently at random.  Of these, $675$ asserted belief in
  evolution, leading to Gallup's estimate that the fraction of Americans
  who believe in evolution is $675/1928 \approx 0.350$.  Gallup claims a
  margin of error of 3 percentage points, that is, he claims to be
  confident that his estimate is within 0.03 of the actual percentage.

\bparts

\ppart\label{varindicator} What is the largest variance an indicator
variable can have?

\begin{solution}
\[
\frac{1}{4}
\]

By Lemma~\bref{bernoulli-variance}, $\variance{H} = pq$.

\begin{staffnotes}
Here's a proof starting from the definition of variance:
\begin{align*}
\variance{H}
     & = \expect{(H-p)^2}
                   & \text{(def of $\variance{}$)}\\
     & = \expect{H^2-2pH+p^2}\\
     & = \expectsq{H} - 2p\expect{H} + p^2
                   & \text{(linearity of $\expect{}$)}\\
     & = p - p^2   & \text{(since $\expect{H} =p$ and $H^2 = H$)}\\
     & = pq        & (q \eqdef 1-p).
\end{align*}

\end{staffnotes}
Noting that $d\, p(1-p)/dp = 2p-1$ is zero when $p=1/2$, it follows that
the maximum value of $p(1-p)$ must be at $p = 1/2$, so the maximum value
of $\variance{H}$ is $(1/2)(1 -(1/2)) = 1/4$.
\end{solution}

\iffalse
\ppart What is the variance of a binomial variable, $B_{n,p}$?
\begin{solution}
\[
B_{n,p} = H_1 + H_2 + \cdots + H_n
\]
where the $H_i$'s are mutually independent indicator variables with
$\pr{H_i=1} = p$.  By independence and part~\eqref{varindicator},
\[
\variance{H_1 + H_2 + \cdots + H_n} = \variance{H_1} + \variance{H_2} + \cdots +
\variance{H_n} = npq.
\]

\end{solution}
\fi

\ppart Use the Pairwise Independent Sampling Theorem to determine a
confidence level with which Gallup can make his claim.

\begin{solution}
By the Pairwise Independent Sampling, the probability that a sample of
size $n=1928$ is further than $x = 0.03$ of the actual fraction is at most
\[
\paren{\frac{\sigma}{x}}^2 \cdot \frac{1}{n}
\leq \paren{\frac{1/2}{0.03}}^2 \cdot \frac{1}{1928} \leq 0.144
\]
so we can be confident of Gallup's estimate at the 85.6\% level.
\end{solution}

\ppart Gallup actually claims greater than 99\% confidence in his
estimate.  How might he have arrived at this conclusion?  (Just explain
what quantity he could calculate; you do not need to carry out a
calculation.)

\begin{solution}
The variance, $\sigma^2$, of a single sample might really be as bad as
1/4 (this happens when $p=1/2$) and the variance of $n$ samples might
be as bad as $n\sigma^2$, so there is no better mileage to be gotten
out of the Chebyshev bound.

Instead, Gallup uses the fact that the sample has a binomial
distribution $B_{1928,p}$, where $p$ is the unknown quantity to be
estimated. \iffalse We estimate that the unknown fraction $p$ is about
$0.35$, and the question is how much confidence to hold in that
estimate.\fi The tails of a binomial distribution decrease much more
rapidly than arbitrary distributions (see
Section~\ref{binomial_distribution_section}),\iffalse and this
decrease is exponential rather than quadratic\fi so confidence degrees
calculated using this distribution will be higher than calculations
based solely on variance.

So Gallup wants an upper bound on
\[
\Prob{\abs{\frac{B_{1928,p}}{1928} - p} > 0.03}
\]
By part~\eqref{varindicator}, the variance of $B_{n,p}$ is largest
when $p =1/2$, which suggests that the probability that a binomial
random variable differs from its mean will be largest
when $p=1/2$.  This is in fact the case.  So Gallup will calculate
\begin{align*}
\lefteqn{\Prob{\abs{\frac{B_{1928,p}}{1928} - p} > 0.03}}
    & = \Prob{\abs{B_{1928,p} - 1928p} > 0.03(1928)}\\
    & \leq \Prob{\abs{B_{1928,1/2} - 1928(1/2)} > 0.03(1928)}\\
    & = \pr{906 \leq B_{1928,1/2} \leq 1021}\\
    & = \frac{\sum_{i=906}^{1021} \binom{1928}{i}}{2^{1928}} \approx 0.9912.
\end{align*}
\emph{Mathematica} will actually calculate this sum exactly.  There
are also simple ways to use Stirling's formula to get a good estimate
of this value.
\end{solution}

\ppart Accepting the accuracy of all of Gallup's polling data and
calculations, can you conclude that there is a high probability that the
number of adult Americans who believe in evolution is $35 \pm 3$ percent?

\begin{solution}
No.  As explained in Notes and lecture, the assertion that
fraction $p$ is in the range $0.35\pm 0.03$ is an assertion of fact that
is either true or false.  The number $p$ is a \emph{constant}.  We
don't know its value, and we don't know if the asserted fact is true or
false, but there is nothing probabilistic about the fact's truth or
falsehood.

We \emph{can} say that either the assertion is true or else a 1-in-100
event occurred during the poll.  Specifically, the unlikely event is
that Gallup's random sample was unrepresentative.  This may convince you
that $p$ is ``probably'' in the range $0.35 \pm 0.03$, but this informal
``probably'' is not a mathematical probability.
\end{solution}

\eparts

\end{problem}

%%%%%%%%%%%%%%%%%%%%%%%%%%%%%%%%%%%%%%%%%%%%%%%%%%%%%%%%%%%%%%%%%%%%%
% Problem ends here
%%%%%%%%%%%%%%%%%%%%%%%%%%%%%%%%%%%%%%%%%%%%%%%%%%%%%%%%%%%%%%%%%%%%%

\endinput
