\documentclass[problem]{mcs}

\begin{pcomments}
  \pcomment{from: S09.cp8t, S06.cp7m}
%  \pcomment{}
%  \pcomment{}
\end{pcomments}

\pkeywords{
  number_theory
  gcd
  lcm
}

%%%%%%%%%%%%%%%%%%%%%%%%%%%%%%%%%%%%%%%%%%%%%%%%%%%%%%%%%%%%%%%%%%%%%
% Problem starts here
%%%%%%%%%%%%%%%%%%%%%%%%%%%%%%%%%%%%%%%%%%%%%%%%%%%%%%%%%%%%%%%%%%%%%

\begin{problem}

\bparts

\ppart Let $m = 2^9 5^{24} 11^{7} 17^{12}$ and $n = 2^3 7^{22} 11^{211}
13^1 17^{9} 19^2$.  What is the $\gcd(m,n)$?  What is the \emph{least
common multiple}, $\lcm(m,n)$, of $m$ and $n$?  Verify that
\begin{equation}\label{gl}
\gcd(m,n) \cdot \lcm(m,n) = mn.
\end{equation}

\begin{solution}
\begin{align*}
g & = 2^3 11^7 17^9,\\
l & = 2^9 5^{24} 7^{22} 11^{211} 13^1 17^{12} 19^2\\
gl & = 2^{12} 5^{24} 7^{22} 11^{218} 13^1 17^{21} 19^2 = mn
\end{align*}

\end{solution}

\ppart Describe in general how to find the $\gcd(m,n)$ and $\lcm(m,n)$
from the prime factorizations of $m$ and $n$.  Conclude that
equation~\eqref{gl} holds for all positive integers $m,n$.

\iffalse \hint You may find it helpful to consider Lemma~\ref{mn}
below.
\fi

\begin{solution}
\iffalse By Lemma~\ref{mn}, \fi The divisors of $m$ correspond
to subsequences of the weakly increasing sequence of primes in the
factorization of $m$, and likewise for $n$.  So the factorization
$\gcd(m,n)$ is the largest common subsequence of the two factorizations.
This can be calculated by taking all the primes that appear in both
factorizations raised to the \emph{minimum} of the powers of that prime in
each factorization.

Likewise, the factorization of $\lcm(m,n)$ is the shortest sequence that
has the factorizations of $m$ and $n$ as subsequences.  So the
factorization of $\lcm(m,n)$ can be calculated by taking all the primes that
appear in either factorization raised to the \emph{maximum} of the powers
of that prime in each factorization.

So in the factorization of $\gcd(m,n) \cdot \lcm(m,n)$ each prime appears raised
to a power equal to the sum of its powers in the factorizations of $m$ and
$n$, which is precisely its power in the factorization of $mn$.
\end{solution}

\eparts
\end{problem}

%%%%%%%%%%%%%%%%%%%%%%%%%%%%%%%%%%%%%%%%%%%%%%%%%%%%%%%%%%%%%%%%%%%%%
% Problem ends here
%%%%%%%%%%%%%%%%%%%%%%%%%%%%%%%%%%%%%%%%%%%%%%%%%%%%%%%%%%%%%%%%%%%%%

\endinput
