\documentclass[problem]{mcs}

\begin{pcomments}
 \pcomment{FP_gambling_man}
 \pcomment{from: S09.cp14t, F09.cp14m}
 \pcomment{same as TP_markov_chebyshev_for_card_games but numbers doubled}
\end{pcomments}

\pkeywords{
  expectation
  variance
  Markov_bound
  Chebyshev
}

%%%%%%%%%%%%%%%%%%%%%%%%%%%%%%%%%%%%%%%%%%%%%%%%%%%%%%%%%%%%%%%%%%%%%
% Problem starts here
%%%%%%%%%%%%%%%%%%%%%%%%%%%%%%%%%%%%%%%%%%%%%%%%%%%%%%%%%%%%%%%%%%%%%

\begin{problem}
  Tom has a gambling problem. He plays 240 hands of draw poker, 120 hands
  of black jack, and 40 hands of stud poker per day.  He wins a hand of
  draw poker with probability 1/6, a hand of black jack with probability
  1/2, and a hand of stud poker with probability 1/5.

\bparts

\ppart What is the expected number of hands that Tom wins in a day?

\begin{center}
\exambox{1.0in}{0.5in}{-0.3in}
\end{center}

\begin{solution}
$240(1/6)+120(1/2)+40(1/5)=108$.
\end{solution}

\ppart What would the Markov bound be on the probability that Tom will win
at least 216 hands on a given day?

\begin{center}
\exambox{0.5in}{0.5in}{-0.3in}
\end{center}

\begin{solution}
The expected number of games won is 108, so
by Markov, $\pr{R \geq 216} \leq 108/216 = 1/2$.
\end{solution}

\ppart Assume the outcomes of the card games are pairwise independent.
What is the variance in the number of hands won per day?  You may answer with a
numerical expression that is not completely evaluated.

\begin{center}
\exambox{2.0in}{0.5in}{-0.3in}
\end{center}

\begin{solution}
The variance can also be calculated using linearity of variance.
For an individual hand the variance is $p(1-p)$ where $p$ is the
probability of winning.  Therefore the variance is
\[
240(1/6)(5/6) + 120(1/2)(1/2) + 40(1/5)(4/5) = 1046/15 = 69\ \tfrac{11}{15}.
\]
\end{solution}

\ppart What would the Chebyshev bound be on the probability that Tom
will win at least 216 hands on a given day?  You may answer with a
numerical expression that is not completely evaluated.

\begin{center}
\exambox{2.0in}{0.5in}{-0.3in}
\end{center}

\begin{solution}
\[
\pr{R - 108 \geq 108} \leq \pr{\abs{R - 108} \geq 108} \leq
\frac{V}{108^2} = \frac{1046}{15(108)^2} \approx 0.0059785.
\]

(A very slightly better bound of 0.0059430 comes from using the
one-sided Chebyshev bound from Problem~\bref{PS_Chebyshev_one_sided}.)

\end{solution}
\eparts
\end{problem}


%%%%%%%%%%%%%%%%%%%%%%%%%%%%%%%%%%%%%%%%%%%%%%%%%%%%%%%%%%%%%%%%%%%%%
% Problem ends here
%%%%%%%%%%%%%%%%%%%%%%%%%%%%%%%%%%%%%%%%%%%%%%%%%%%%%%%%%%%%%%%%%%%%%

\endinput
