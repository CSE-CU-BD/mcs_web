\documentclass[problem]{mcs}

\begin{pcomments}
  \pcomment{MQ_cycle_from_closed_walk}
  \pcomment{candidate problem introduced by ORM, S10}
  \pcomment{directed version of PS_shortest_undirected_closed_walk}
  \pcomment{revised ARM 3/11/15}
\end{pcomments}

\pkeywords{
  digraphs
  cycle
  walk
  closed_walk
}

\begin{problem}

\bparts

\ppart Give an example of a digraph that contains two vertices $u \neq v$ such
that there is a path from $u$ to $v$ and also a path from $v$ to $u$,
but no cycle containing both $u$ and $v$.

\begin{solution}
$u \to w \to v \to w \to u$
\end{solution}

\examspace[1.5in]

\ppart Prove that if there is a positive length walk in a digraph that
starts and ends at node $v$, then there is a cycle that contains $v$.

\iffalse
\inhandout{Recall that a \emph{cycle} is a positive length walk whose
  only repeat vertex is the start and end vertex.}
\fi

\begin{solution}
By the WOP, there is a \emph{shortest} positive length walk from $v$
to $v$.  We claim this walk must be a cycle.

Suppose to the contrary that some vertex $w \neq v$ occurred twice on
the walk.  Then removing the positive length walk from $w$ to $w$
would leave a shorter walk from $v$ to $v$, contradicting the choice
of shortest walk.  So if the shortest walk was not a cycle, the vertex
$v$ must have an occurrence that is not at the beginning or end of the
walk.  In that case, the walk from the initial $v$ to the non-end
occurrence of $v$ would be a shorter walk from $v$ to $v$, again
contradicting the choice of shortest walk.

Since both ways that the shortest positive length walk from $v$ to $v$
could fail to be a cycle led to contradictions, we conclude that this
shortest walk must be a cycle.

\end{solution}

\examspace[3in]

\eparts

\end{problem}

\endinput
