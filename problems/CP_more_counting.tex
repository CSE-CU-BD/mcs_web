\documentclass[problem]{mcs}

\begin{pcomments}
\pcomment{CP_more_counting}
\pcomment{from: S09.cp10m commented out; S07 cp10f? maybe use S08.10f instead?}
\pcomment{last part extended based on FP_more_counting.(c).(d) by ARM 4/7/10}
\end{pcomments}

\pkeywords{
  counting
  counting_rules
  bijection
}

%%%%%%%%%%%%%%%%%%%%%%%%%%%%%%%%%%%%%%%%%%%%%%%%%%%%%%%%%%%%%%%%%%%%%
% Problem starts here
%%%%%%%%%%%%%%%%%%%%%%%%%%%%%%%%%%%%%%%%%%%%%%%%%%%%%%%%%%%%%%%%%%%%%


\begin{problem}
Solve the following counting problems by defining an appropriate
mapping (bijective or $k$-to-1) between a set whose size you know and
the set in question.

\bparts 

\ppart How many different ways are there to select a dozen donuts if
four varieties are available?

\begin{solution}
There is a bijection from selections of a dozen donuts to 15-bit
sequences with exactly 3 ones.  In particular, suppose that the
varieties are glazed, chocolate, lemon, and Boston creme.  Then a
selection of $g$ glazed, $c$ chocolate, $l$ lemon, and $b$ Boston
creme maps to the sequence:
%
\[
(g\ 0's)\ 1\ (c\ 0's)\ 1\ (l\ 0's)\ 1\ (b\ 0's)
\]
%
Therefore, the number of selections is equal to the number of 15-bit
sequences with exactly 3 ones, which is:
%
\[
\frac{15!}{3!\ 12!} = \binom{15}{3}
\]

\end{solution}


\ppart In how many ways can Mr. and Mrs. Grumperson distribute 13
identical pieces of coal to their two ---no, three! ---children for
Christmas?

\begin{solution}
There is a bijection from 15-bit strings
with two ones.  In particular, the bit string $0^a10^b10^c$ maps to
the assignment of $a$ coals to the first child, $b$ coals to the
second, and $c$ coals to the third.  Therefore, there are
$\binom{15}{2}$ assignments.
\end{solution}

\ppart How many solutions over the nonnegative integers are there to the
inequality:

\begin{eqnarray*} 
x_1 + x_2 + \ldots + x_{10} & \leq & 100
\end{eqnarray*}

\begin{solution}
  There is a bijection from 110-bit sequences with 10 ones to solutions to
  this equation.  In particular, $x_i$ is the number of zeros before the
  $i$-th one but after the $(i-1)$-st one (or the beginning of the
  sequence).  Therefore, there are $\binom{110}{10}$ solutions.
\end{solution}

\ppart We want to count step-by-step paths between points in the plane
with integer coordinates.  Ony two kinds of step are allowed: a
right-step which increments the $x$ coordinate, and an up-step which
increments the $y$ coordinate.

\begin{itemize}

\item[(i)] How many paths are there from $(0, 0)$ to $(20, 30)$?

\begin{solution}
$\binom{50}{20}$.

There is a bijection from 50-bit sequences with 20 zeros and 30
ones.  The sequence $(b_1, \ldots, b_{30})$ maps to a path where the
$i$-th step is right if $b_i = 0$ and up if $b_i = 1$.  Therefore, the
number of paths is equal to $\binom{50}{20}$.
\end{solution}

\item[(ii)] How many paths are there from $(0, 0)$ to $(20, 30)$ that go through the point $(10, 10)$?

\begin{solution}
$\binom{20}{10}\cdot \binom{30}{10}$.

There is a bijection between the paths from $(20,30)$ that go through
$(10,10)$ and set of pairs of paths consisting of path from $(0,0)$ to
$(10,10)$ and a path from $(10,10)$ to $(20,30)$.  So the number of
paths through $(10, 10)$ is the product of the sizes of these two sets
of paths.
\end{solution}

\item[(iii)] How many paths are there from $(0, 0)$ to $(20, 30)$ that do
  \emph{not} go through either of the points $(10, 10)$ and
  $(15,20)$?

\hint Let $P$ be the set of paths from $(0, 0)$ to $(20, 30)$, $N_1$ be the
paths in $P$ that go through $(10,10)$ and $N_2$ be the paths in $P$
that go through $(15,20)$.

\end{itemize}

\begin{solution}
\[
\binom{50}{20} - 
\binom{20}{10} \cdot \binom{30}{10} - \binom{30}{15} \cdot \binom{15}{5}
+ \binom{20}{10} \cdot \binom{15}{5} \cdot \binom{15}{5}.
\]

$N_1 \intersect N_2$ is the set of paths from $(0, 0)$ to $(20, 30)$
that go through both $(10, 10)$ and $(15,20)$.  So $P - (N_1 \union
N_2)$ is the set of paths to be counted.  Now we have
\begin{align*}
\card{P - (N_1 \union N_2)}
     & = \card{P} - \card{N_1 \union N_2}\\
     & = \card{P} - \card{N_1} - \card{N_2} + \card{N_1 \intersect N_2}
           & \text{by Inclusion-Exclusion}.
\end{align*}
Part~(ii) shows how to calculate $\card{N_i}$.  Also, there is a
bijection between $N_1 \intersect N_2$ and the set of triples
consisting of a path $(0,0)$ to $(10,10)$, a path from $(10,10)$ to
$(15,20)$, and a path from $(15,20)$ to $(20,30)$.  So the size of
$N_1 \intersect N_2$ is the product of the sizes of these three sets
of paths.

\end{solution}

\eparts
\end{problem}

\endinput
