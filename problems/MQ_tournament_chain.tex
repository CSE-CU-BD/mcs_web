\documentclass[problem]{mcs}

\begin{pcomments}
  \pcomment{MQ_tournament_chain}
  \pcomment{excerpt of CP_tournament_chain}
  \pcomment{ARM 11/3/13}
\end{pcomments}

\pkeywords{
  digraph
  path
  tournament
  ranking
}

%%%%%%%%%%%%%%%%%%%%%%%%%%%%%%%%%%%%%%%%%%%%%%%%%%%%%%%%%%%%%%%%%%%%%
% Problem starts here
%%%%%%%%%%%%%%%%%%%%%%%%%%%%%%%%%%%%%%%%%%%%%%%%%%%%%%%%%%%%%%%%%%%%%

\begin{problem}
  In a round-robin tournament, every two distinct players play against
  each other just once.  For a round-robin tournament with no tied games,
  a record of who lost to whom can be described with a \term{tournament
    digraph}, where the vertices correspond to players, and there is an
  edge $\diredge{x}{y}$ iff $x$ lost to $y$ in their game.

  A \term*{\idx{ranking}} is a path that includes all the players.  So in
  a ranking, each player lost the game against the next player in the
  path, but may very well have won their games against other players
  further along the path.

\iffalse
---whoever does the ranking may have a lot of room to play favorites.
\fi
  
\bparts

\ppart Give an example of a tournament digraph with more than one
ranking.

\examspace[0.7in]

\begin{solution}
  Let $n=3$ with edges $\diredge{u}{v}$, $\diredge{v}{w}$ and
  $\diredge{w}{u}$.  Then both~$u,v,w$ and $v,w,u$ are rankings.
\end{solution}

\ppart\label{hasrank} \emph{Claim}: Every finite tournament digraph has a
ranking.  Prove the Claim by induction on the size of the tournament.

As usual, carefully indicate your induction hypothesis, the base case, and
the inductive step.

\begin{solution}
By induction on $n$ with induction hypothesis
\[
P(n) \eqdef \text{every tournament digraph with $n$ vertices has a
  ranking.}
\]
 
\inductioncase{base case} ($n=1$):  Trivial.

\inductioncase{inductive step}:  Let $G$ be a tournament digraph with $n+1$
vertices.  Remove one vertex,~$v$, to obtain the subgraph, $H$, with the
$n$ remaining vertices.  Since removing~$v$ does not change the edges
between the remaining vertices, $H$ is also a tournament digraph.  So by
induction hypothesis $H$ has a ranking.

If $v$ lost to the first player on the $H$-ranking path, then adding $v$
to the start of the path will be a ranking of $G$.  Otherwise, $v$ beat
the first player, so there will be a last player, $w$, on the path, that
$v$ beat.  Inserting~$v$ in the path just after $w$ then gives a ranking
for $G$.

Since $G$ was an arbitrary $n+1$ vertex tournament graph, we conclude that
$P(n+1)$ holds, which completes the proof.
\end{solution}

\eparts
\end{problem}

%%%%%%%%%%%%%%%%%%%%%%%%%%%%%%%%%%%%%%%%%%%%%%%%%%%%%%%%%%%%%%%%%%%%%
% Problem ends here
%%%%%%%%%%%%%%%%%%%%%%%%%%%%%%%%%%%%%%%%%%%%%%%%%%%%%%%%%%%%%%%%%%%%%

\endinput
