\documentclass[problem]{mcs}

\begin{pcomments}
  \pcomment{TP_countable_chain}
  \pcomment{ARM 3/15/17}
\end{pcomments}

\pkeywords{
  countable
  finite
  union
  chain
}

%%%%%%%%%%%%%%%%%%%%%%%%%%%%%%%%%%%%%%%%%%%%%%%%%%%%%%%%%%%%%%%%%%%%%
% Problem starts here
%%%%%%%%%%%%%%%%%%%%%%%%%%%%%%%%%%%%%%%%%%%%%%%%%%%%%%%%%%%%%%%%%%%%%


\begin{problem}
A collection $\mathcal{C}$ of sets is called a \emph{chain} when,
given any two sets in $\mathcal{C}$, one is a subset of the
other.  Prove that if $\mathcal{F}$ is chain of \emph{finite} sets,
then $\lgunion \mathcal{F}$ is countable.  (Notice that without the
chain condition, every set is the union of its finite subsets.)

\begin{solution}
Define the \emph{height} of an element $f \in \lgunion \mathcal{F}$ to
be the smallest $n$ such that $f \in F$ for some size $n$ set $F \in
\mathcal{F}$.  Every $f \in \lgunion \mathcal{F}$ has a well-defined
height by WOP.  Since $\mathcal{F}$ is a chain, no two different sets
can be the same size; this implies that there are only finitely many
elements of any given height.  So we can list the elements of
$\lgunion \mathcal{F}$ in order of height, with elements of the same
1height listed in arbitrary order, proving that $\lgunion \mathcal{F}$
is countable.
\end{solution}

\end{problem}

%%%%%%%%%%%%%%%%%%%%%%%%%%%%%%%%%%%%%%%%%%%%%%%%%%%%%%%%%%%%%%%%%%%%%
% Problem ends here
%%%%%%%%%%%%%%%%%%%%%%%%%%%%%%%%%%%%%%%%%%%%%%%%%%%%%%%%%%%%%%%%%%%%%

\endinput


