\documentclass[problem]{mcs}

\begin{pcomments}
  \pcomment{CP_max_ranvar_n}
  \pcomment{generalizes CP_max_ranvar}
  \pcomment{suitable as MQ with prior exposure to CP_max_ranvar}
  \pcomment{ARM 5/5/12}
\end{pcomments}

\pkeywords{
  random_variables
  mutually_independent
  density
  uniform
  maximum
}

%%%%%%%%%%%%%%%%%%%%%%%%%%%%%%%%%%%%%%%%%%%%%%%%%%%%%%%%%%%%%%%%%%%%%
% Problem starts here
%%%%%%%%%%%%%%%%%%%%%%%%%%%%%%%%%%%%%%%%%%%%%%%%%%%%%%%%%%%%%%%%%%%%%

\begin{problem}
Suppose $R_1,R_2,\dots, R_m$, are mutually independent random
variables with uniform distribution on the range $[1,n]$.
\begin{staffnotes}
That is,
\[
\pr{R_i=v} = \frac{1}{n}  \qquad\text{ for } v \in [1,n].
\]
\end{staffnotes}
Let $M=\max_{i=1]^m R_i$.

\bparts

\ppart Write a formula for $\pdf_M(1)$.

\examspace[0.5in]


\ppart\label{Mleqkin1nm}
Write a formula for $\pr{M \leq k}$ where $k \in [1,n]$.

\examspace[0.5in]

\begin{solution}
\[
\paren{\frac{k}{n}}^m .
\]

This follows $M \leq k$ iff $\forall i \in [1,n].\ R_i \leq k$.  Since
$R_i$ is uniform, $\pr{R_i} \leq k = k/n$, so by mutual independence,
\[
\pr{M \leq k} = \prod_{i=1}^m\pr{R_i \leq k} = \paren{\frac{k}{n}}^m .
\]
\end{solution}

\ppart Write a formula for $\pdf_M(k)$ for $k \in (1,n]$.

\begin{staffnotes}
\hint $M=k$ iff $M\leq k \QAND \QNOT(M \leq k-1)$.
\end{staffnotes}

\begin{solution}
\[
\pdf_M(k)  = \paren{\frac{k}{n}}^m - \paren{\frac{k-1}{n}}^m .
\]

This follows because 
\[
\pr{M \leq k} = \pr{M=k} + \pr{M \leq k-1}
\]
by the Disjoint Sum Rule, and so
\[
\pr{M = k} = \pr{M \leq k} - \pr{M \leq k-1}
\]
which equals~\eqref{pdfmkknm} by part~\eqref{Mleqkin1nm}.
\end{solution}

\eparts

\end{problem}

%%%%%%%%%%%%%%%%%%%%%%%%%%%%%%%%%%%%%%%%%%%%%%%%%%%%%%%%%%%%%%%%%%%%%
% Problem ends here
%%%%%%%%%%%%%%%%%%%%%%%%%%%%%%%%%%%%%%%%%%%%%%%%%%%%%%%%%%%%%%%%%%%%%

\endinput
