\documentclass[problem]{mcs}

\begin{pcomments}
  \pcomment{CP_max_ranvar_n}
  \pcomment{essentially the same as PS_max_ranvar}
  \pcomment{ARM 5/5/12, edited 11/27/13}
  \pcomment{solution corrected 4/30/14}
\end{pcomments}

\pkeywords{
  random_variables
  mutually_independent
  density
  uniform
  maximum
}

%%%%%%%%%%%%%%%%%%%%%%%%%%%%%%%%%%%%%%%%%%%%%%%%%%%%%%%%%%%%%%%%%%%%%
% Problem starts here
%%%%%%%%%%%%%%%%%%%%%%%%%%%%%%%%%%%%%%%%%%%%%%%%%%%%%%%%%%%%%%%%%%%%%


\begin{problem}
Let $R_1,R_2,\dots, R_m$, be mutually independent random variables
with uniform distribution on~$\Zintv{1}{n}$.
\begin{staffnotes}
That is,
\[
\pr{R_i=v} = \frac{1}{n}  \qquad\text{ for } v \in \Zintv{1}{n}.
\]
\end{staffnotes}
Let $M \eqdef \max\set{R_i \suchthat i \in \Zintv{1}{m}\, }$.

\bparts

\ppart Write a formula for $\pdf_M(1)$.

\examspace[0.75in]

\begin{solution}

\[
\paren{\frac{1}{n}}^m .
\]

$M= 1$ iff all $m$ random variables equal 1, and each variable equals
1 with probability $1/n$.

\end{solution}

\ppart More generally, write a formula for $\pr{M \leq k}$.

\examspace[.75in]

\begin{solution}
\[
\paren{\frac{k}{n}}^m .
\]

This follows because $M \leq k$ iff $\forall i \in \Zintv{1}{m}.\ R_i \leq k$.  Since
$R_i$ is uniform, $\pr{R_i \leq k} = k/n$, so by mutual independence,
\[
\pr{M \leq k} = \prod_{i=1}^m\pr{R_i \leq k} = \paren{\frac{k}{n}}^m .
\]
\end{solution}

\ppart For $k \in \Zintv{1}{n}$, write a formula for $\pdf_M(k)$ in
terms of expressions of the form ``$\pr{M \leq j}$" for $j \in
\Zintv{1}{n}$.

\examspace[0.75in]

\begin{solution}
\[
\pr{M \leq k} - \pr{M \leq k-1}.
\]

This follows because, by the Disjoint Sum Rule,
\[
\pr{M \leq k} = \pr{M=k} + \pr{M \leq k-1}\ .
\]

\end{solution}

\eparts

\end{problem}

%%%%%%%%%%%%%%%%%%%%%%%%%%%%%%%%%%%%%%%%%%%%%%%%%%%%%%%%%%%%%%%%%%%%%
% Problem ends here
%%%%%%%%%%%%%%%%%%%%%%%%%%%%%%%%%%%%%%%%%%%%%%%%%%%%%%%%%%%%%%%%%%%%%

\endinput
