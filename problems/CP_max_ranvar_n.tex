\documentclass[problem]{mcs}

\begin{pcomments}
  \pcomment{CP_max_ranvar_n}
  \pcomment{essentially the same as PS_max_ranvar}
  \pcomment{ARM 5/5/12, edited 11/27/13}
\end{pcomments}

\pkeywords{
  random_variables
  mutually_independent
  density
  uniform
  maximum
}

%%%%%%%%%%%%%%%%%%%%%%%%%%%%%%%%%%%%%%%%%%%%%%%%%%%%%%%%%%%%%%%%%%%%%
% Problem starts here
%%%%%%%%%%%%%%%%%%%%%%%%%%%%%%%%%%%%%%%%%%%%%%%%%%%%%%%%%%%%%%%%%%%%%


\begin{problem}
Let $R_1,R_2,\dots, R_m$, be mutually independent random variables
with uniform distribution on~$[1,n]$.
\begin{staffnotes}
That is,
\[
\pr{R_i=v} = \frac{1}{n}  \qquad\text{ for } v \in [1,n].
\]
\end{staffnotes}
Let $M \eqdef \max\set{R_i \suchthat i \in [1,m]\, }$.

\bparts

\ppart Write a formula for $\pdf_M(1)$.

\iffalse
\begin{center}
\exambox{0.5in}{0.5in}{0in}
\end{center}
\fi

\examspace[0.75in]

\begin{solution}

\[
\paren{\frac{1}{n}}^m .
\]
\end{solution}

\ppart More generally, write a formula for $\pr{M \leq k}$.

\iffalse
\begin{center}
\exambox{0.5in}{0.5in}{0in}
\end{center}
\fi

\examspace[.75in]

\begin{solution}
\[
\paren{\frac{k}{n}}^m .
\]

This follows $M \leq k$ iff $\forall i \in [1,n].\ R_i \leq k$.  Since
$R_i$ is uniform, $\pr{R_i} \leq k = k/n$, so by mutual independence,
\[
\pr{M \leq k} = \prod_{i=1}^m\pr{R_i \leq k} = \paren{\frac{k}{n}}^m .
\]
\end{solution}

\ppart For $k \in [1,n]$, write a formula for $\pdf_M(k)$ in terms of
expressions of the form ``$\pr{M \leq j}$ for $j \in [1,n]$.

\iffalse
\begin{center}
\exambox{2.0in}{0.5in}{0in}
\end{center}
\fi
\examspace[0.75in]

\begin{solution}
\[
\pr{M \leq k} - \pr{M \leq k-1}.
\]

This follows because, by the Disjoint Sum Rule,
\[
\pr{M \leq k} = \pr{M=k} + \pr{M \leq k-1}\ .
\]

\end{solution}

\eparts

\end{problem}

%%%%%%%%%%%%%%%%%%%%%%%%%%%%%%%%%%%%%%%%%%%%%%%%%%%%%%%%%%%%%%%%%%%%%
% Problem ends here
%%%%%%%%%%%%%%%%%%%%%%%%%%%%%%%%%%%%%%%%%%%%%%%%%%%%%%%%%%%%%%%%%%%%%

\endinput
