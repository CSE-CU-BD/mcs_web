\documentclass[problem]{mcs}

\begin{pcomments}
  \pcomment{CP_tournament_chain}
  \pcomment{by ARM 10/17/09 but I saw it somewhere}
\end{pcomments}

\pkeywords{
  digraphs
  relations
  chain
  tournament
}

%%%%%%%%%%%%%%%%%%%%%%%%%%%%%%%%%%%%%%%%%%%%%%%%%%%%%%%%%%%%%%%%%%%%%
% Problem starts here
%%%%%%%%%%%%%%%%%%%%%%%%%%%%%%%%%%%%%%%%%%%%%%%%%%%%%%%%%%%%%%%%%%%%%

\begin{problem} %\textbf{Tournament Graphs}

  \newcommand{\beats}{\!\rightarrow\!}

  In a \term{round-robin} tournament, every pair of distinct players play
  against each other just once.  For a round-robin tournament with with no
  tied games, a record of who beat whom can be described with a
  \term{tournament digraph}, where the vertices correspond to players and
  there is an edge $x \beats y$ if $x$ beat $y$ in their game.

  A \term*{ranking} is a directed simple path that includes all the
  players.

  \bparts

  \ppart Give an example of a tournament digraph with more than one
  ranking.

   \begin{solution}
   Let $n=3$ with edges $u \beats v \beats w \beats u$.  There are three
   different rankings starting with $u$, $v$, and $w$, respectively.
   \end{solution}

  \ppart If a tournament digraph is a DAG, then it has a unique ranking.  Explain.

\begin{solution}
  If it's a DAG, then its path relation defines a partial order, and since
  every pair of elements are comparable, this is in fact a total order.
  So there's no alternative but to rank the elements from smallest to
  largest.
\end{solution}

\ppart Prove that every tournament digraph has a ranking.
\hint Induction on the size of the tournament.

\begin{solution}
By induction on $n$ with induction hypothesis
\[
P(n) \eqdef \text{every tournament digraph with $n$ vertices has a
  ranking.}
\]
 
\textbf{base case} $n=1$:  Trivial.

\textbf{inductive step}: Let $G$ be a tournament digraph with $n+1$
vertices.  Remove one vertex, $v$, to obtain the subgraph, $H$, with the
$n$ remaining vertices.  Clearly, $H$ is also a tournament digraph, so by
induction hypothesis it has a ranking.  Now if the last player in this
$H$-ranking beat player $v$, then $v$ can be added at the end to form a
ranking in $G$.  On the other hand, if $v$ beat the last player in the
$H$-ranking, then there will (by WOP) be a first player in the $H$-ranking
that $v$ beats.  Inserting $v$ just before that first player gives a
ranking for $G$.  Since $G$ was an arbitrary $n+1$ vertex tournament
graph, we conclude that $P(n+1)$ holds, which completes the proof.
\end{solution}

\iffalse
  \ppart Give an example of a tournament that has a unique ranking and is
  \emph{not} a DAG.  \hint Try $n=4$.
\fi

  \eparts
\end{problem}

%%%%%%%%%%%%%%%%%%%%%%%%%%%%%%%%%%%%%%%%%%%%%%%%%%%%%%%%%%%%%%%%%%%%%
% Problem ends here
%%%%%%%%%%%%%%%%%%%%%%%%%%%%%%%%%%%%%%%%%%%%%%%%%%%%%%%%%%%%%%%%%%%%%

\endinput
