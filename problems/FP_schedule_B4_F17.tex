\documentclass[problem]{mcs}

\begin{pcomments}
  \pcomment{FP_schedule_B4_F17}
  \pcomment{ARM 4/5/16}
\end{pcomments}

\pkeywords{
  partial_orders
  scheduling
  constraints
  parallel_time
}

%%%%%%%%%%%%%%%%%%%%%%%%%%%%%%%%%%%%%%%%%%%%%%%%%%%%%%%%%%%%%%%%%%%%%
% Problem starts here
%%%%%%%%%%%%%%%%%%%%%%%%%%%%%%%%%%%%%%%%%%%%%%%%%%%%%%%%%%%%%%%%%%%%%

\begin{problem} 
Let $B^4$ be the length-4 binary vectors partially ordered
\emph{coordinatewise}, that is,
\[
b_1b_2b_3b_4 \leq c_1c_2c_3c_4
\]
iff $b_i \leq c_i$ for $i= 1,2,3,4$.
\iffalse
You can take it for granted that
$\leq$ defines a weak partial order on $B^4$.
\[
B^4 \eqdef \set{(b_1,b_2,b_3,b_4) \suchthat b_i \in \set{0,1} \text{for}\ i= 1,2,3,4},
\]\fi
For example,
\begin{align*}
\texttt{1001}  & \leq \texttt{1011} \leq \texttt{1111}\\
\texttt{1001}\ & \text{incomparable to}\ \texttt{0101}.
\end{align*}

\begin{staffnotes}
parts a,b,c,d worth 3,4,4,5 points
\end{staffnotes}

\bparts

\iffalse
\ppart Prove that $\sqsubseteq$ is a partial order on $B^4$.

\examspace[2.0in]
\fi

\ppart Give an example of a maximum length chain for in $B^4$.

\exambox{3.0in}{0.4in}{0in}

\begin{solution}
\[
0000,0001,0011,0111,1111
\]

\end{solution}

\ppart Give an example of an antchain of size 6 in $B^4$.

\exambox{3.5in}{0.4in}{0in}

\begin{solution}
\[
0011,0110,1100,1010,0101,1001
\]
\end{solution}

\iffalse
\ppart Describe an example of a topological sort of $B^4$.

\examspace[1.0in]
\begin{solution}
\texttt{0000} followed by the four vectors with one \texttt{1} in any
order, followed by the six vectors with two \texttt{1}'s in any order,
followed by the four vectors with three \texttt{1}'s in any order,
followed by \texttt{1111}.

A specific example is
\begin{align*}
0000, 0001, 0010, 0100, 1000, 0011,0110,1100,\\
\quad 1010,0101,1001, 1110,1101,1011,0111,1111
\end{align*}
\end{solution}
\fi

\ppart Suppose the partial order on $B^4$ describes scheduling
constraints on 16 unit-time tasks.  What is the length of a minimum
time \textbf{3}-processor schedule for $B^4$?

\begin{center}
\exambox{0.5in}{0.5in}{0.0in}
\end{center}

\begin{solution}
\textbf{7}.  For example, a length-7 3-processor schedule is:

\begin{staffnotes}
NEEDS CHECKING:
\end{staffnotes}
nn
\begin{align*}
0000\\
0001,0010,0100\\
1000,0110,0101\\
0011,1001,1010\\
1100,1011,0111\\
1110,1101\\
1111
\end{align*}
Moreover, no shorter schedule is possible: there is a unique minimum
task, \texttt{0000}, which must come first and a unique maximum task,
\texttt{1111}, which must come last; this leaves 14 tasks which
require at least $\ceil{14/3} = 5$ more parallel steps.
\end{solution}

\eparts

\end{problem} 


%%%%%%%%%%%%%%%%%%%%%%%%%%%%%%%%%%%%%%%%%%%%%%%%%%%%%%%%%%%%%%%%%%%%%
% Problem ends here
%%%%%%%%%%%%%%%%%%%%%%%%%%%%%%%%%%%%%%%%%%%%%%%%%%%%%%%%%%%%%%%%%%%%%

\endinput
