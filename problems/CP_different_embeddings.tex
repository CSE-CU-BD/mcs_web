\documentclass[problem]{mcs}

\begin{pcomments}
  \pcomment{from: S09.cp7m}
  \pcomment{from: S07.cp7m}
\end{pcomments}

\pkeywords{
  planar_graphs
  isomorphisms
  planar_embeddings
  recursive_data
}

%%%%%%%%%%%%%%%%%%%%%%%%%%%%%%%%%%%%%%%%%%%%%%%%%%%%%%%%%%%%%%%%%%%%%
% Problem starts here
%%%%%%%%%%%%%%%%%%%%%%%%%%%%%%%%%%%%%%%%%%%%%%%%%%%%%%%%%%%%%%%%%%%%%

\begin{problem}
Figures 1--4 show different pictures of planar graphs.

\mfigure{!}{6in}{figures/cp7mfigs-new}
        
\bparts

\ppart For each picture, describe its discrete faces (simple cycles that
define the region borders).

\solution{
Figs 1 \& 2: abda, bcdb, abcda.
Fig 3: abcdea, adea,abda,bcdb.  Fig 4: abcda, abdea, bdcb, adea.
}

\ppart Which of the pictured graphs are isomorphic?  Which pictures
represent the same \emph{planar embedding}? -- that is, they have the same
discrete faces.

\solution{
Figs 1 \& 2 have the same faces, so are different pictures of the
\emph{same} planar drawing.  Figs 3 \& 4 both have four faces, but they are
different, for example, Fig 3 has a face with 5 edges, but the
longest face in Fig 4 has 4 edges.}

\ppart Describe a way to construct the embedding in Figure 4 according to
the recursive definition of planar embedding (in the Appendix).

\solution{Here's one way.  By PS6, Problem 1, these steps could be done in
any order.

\[\begin{array}{ccr}
\text{recursive step} & & \text{faces}\\\hline
\text{vertex } a & \text{(base case)} & a\\
\text{vertex } b & \text{(base)} & b\\
\edge{a}{b} & \text{(bridge)} & aba\\
\text{vertex } c & \text{(base)} & c\\
\edge{b}{c} & \text{(bridge)} & abcba\\
\text{vertex } d & \text{(base)} & d\\
\edge{c}{d} & \text{(bridge)} & abcdcba\\
\edge{a}{d} & \text{(split)} & dabcd,\ dabcd\\
\edge{b}{d} & \text{(split)} & dabd,\ dbcd,\  abcda\\
\text{vertex } e & \text{(base)} & e\\
\edge{d}{e} & \text{(bridge)} & dedabd,\ dbcd,\  abcda\\
\edge{a}{e} & \text{(split)}  & abdea,\ adea,\ dbcd,\ abcda\\
\end{array}\]
}

\eparts
\end{problem}

%%%%%%%%%%%%%%%%%%%%%%%%%%%%%%%%%%%%%%%%%%%%%%%%%%%%%%%%%%%%%%%%%%%%%
% Problem ends here
%%%%%%%%%%%%%%%%%%%%%%%%%%%%%%%%%%%%%%%%%%%%%%%%%%%%%%%%%%%%%%%%%%%%%
