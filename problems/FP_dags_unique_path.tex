\documentclass[problem]{mcs}

\begin{pcomments}
  \pcomment{FP_dags_unique_path}
  \pcomment{ZDz 12/13/15}
  \pcomment{sugested for f15.final}
\end{pcomments}

\pkeywords{
  counting
  probability
  expectation
}

%%%%%%%%%%%%%%%%%%%%%%%%%%%%%%%%%%%%%%%%%%%%%%%%%%%%%%%%%%%%%%%%%%%%%
% Problem starts here
%%%%%%%%%%%%%%%%%%%%%%%%%%%%%%%%%%%%%%%%%%%%%%%%%%%%%%%%%%%%%%%%%%%%%

\begin{problem}
Let $S_n$ denote the set of all DAGs with $n$ nodes such that there
are no two different paths between any pair of nodes.

\bparts
 
\ppart
Let's apply the following transformation step to a graph $G \in S_n$.
Find vertices $u, v, w$ such that $u \rightarrow v \rightarrow w$
and replace edge $v \rightarrow w$ with edge $u \rightarrow w$.
If there is no such triple of vertices, nothing is done.
Prove that such obtained graph is also in $S_n$.

\examspace[3in]

\begin{solution}
Let $H$ be a graph after a transformation step.  We need to prove that
$H$ does not contain a pair of vertices, $x$ and $y$, such that there
are two different paths from $x$ to $y$.  Since that holds in $G$, we
only need to check if the added edge, $u \rightarrow w$, creates an
additional path from $x$ to $y$.  Let us assume it does.  If there are
two paths between $x$ and $y$ in $H$, one of them, $p_1$, must contain
$u \rightarrow w$, while the other one, $p_2$ cannot contain $v
\rightarrow w$ since $v \rightarrow w$ is not in $H$.  $p_2$ also
cannot contain $u \rightarrow w$ since that would imply that there
were two different paths either between $x$ and $u$, or between $w$
and $y$ in $G$.  Therefore $p_2$ is a path in $G$ as well.  However,
by replacing $u \rightarrow w$ with $u \rightarrow v$ and $v
\rightarrow w$ in $p_1$, $p_1$ would become a path in $G$ and still be
different from $p_2$ ($p_2$ does not contain $v \rightarrow w$), which
would imply that there are two different paths between $x$ and $y$ in
$G$---contradiction.
\end{solution}
 
 
\ppart Find an example of a graph $G \in S_n$ such that after applying
the transformation, the number of triples $u \rightarrow v \rightarrow
w$ increases.

\examspace[2in]

\begin{solution}
$G$ with the set of edges $\set{a \rightarrow c, b \rightarrow c, c
   \rightarrow d, d \rightarrow e}$.  Triples are $a \rightarrow c
  \rightarrow d$, $b \rightarrow c \rightarrow d$ and $c \rightarrow d
  \rightarrow e$.  After apply transformation to $c \rightarrow d
  \rightarrow e$, the obtained graph has edges $\set{a \rightarrow c,
   b \rightarrow c, c \rightarrow d, c \rightarrow e}$ and triples $a
  \rightarrow c \rightarrow d$, $b \rightarrow c \rightarrow d$, $a
  \rightarrow c \rightarrow e$ and $b \rightarrow c \rightarrow e$.
\end{solution}


\ppart Show that, by starting from graph $G \in S_n$, the procedure
that repeatedly applies transformation steps above until there are no
more triples $u \rightarrow v \rightarrow w$ will still terminate.

\hint Consider the number of pairs of nodes with path relation.

\examspace[3in]

\begin{solution}
By applying a transformation to $G$ on $u \rightarrow v \rightarrow w$,
path from $v$ to $w$ is removed (there could not exist an alternative path
due to the definition of $S_n$).  On the other hand, no other path relation
is added.  To prove that, let us assume for the sake of contradicition
that there is an added path relation from $x$ to $y$.  Then, the path from
$x$ to $y$, $p$, must include the added edge, $u \rightarrow w$.
By replacing $u \rightarrow w$ with $u \rightarrow v$
and $v \rightarrow w$ in $p$, a path from $x$ to $y$ in $G$ is obtained.
However, we assumed that there was no such path, hence a contradiction.  
Therefore, we showed that the number of pairs of nodes with path
relation is strictly decreasing.  Since that number is a nonnegative integer,
the procedure must terminate.
\end{solution}


\ppart
Let $m_n$ be the maximum number of edges of any graph from $S_n$.
Let $T_n \subset S_n$ be the set of DAGs in which there are no
three vertices $u, v, w$ such that $u \rightarrow v \rightarrow w$.
Show that there is a graph in $T_n$ that has $m_n$ edges.

\examspace[2in]

\begin{solution}
Let us assume that $G$ is a graph in $S_n$ with $m_n$ edges.
By applying the transformation steps above, the number of edges
does not change.  At the end of the procedure, a graph from $T_n$
is obtained (otherwise, there is a path $u \rightarrow v \rightarrow w$
and transformation can still be applied).  That graph has $m_n$ edges.
\end{solution}


\ppart
What is $m_n$ (as a function of $n$)?

\hint Note that there are two types of nodes in a graph from
  $T_n$---those with in-degree 0, and those with out-degree 0.

\examspace[3in]

\begin{solution}
Let $G$ be a graph from $T_n$.  Then, no vertex in $G$ can have both
incoming and outgoing edges (a path of the form $u \rightarrow v
\rightarrow w$ would exist otherwise).  Let $A$ be the set of vertices
of $G$ with in-degree 0.  Let $B$ be the set of vertices of $G$ with
out-degree 0.  Then, all edges in $G$ must go from a vertex in $A$ to
a vertex in $B$.  The number of edges is maximized if there is an edge
between all pairs of vertices from $A \times B$ (it is easy to show
that such a graph is still in $T_n$).  The number of edges is then
$\card{A} \cdot \card{B}$.  Since $\card{A} + \card{B} = n$, $\card{A}
\cdot \card{B}$ is maximized when $\card{A}$ is the nearest integer to
$n/2$.  For even $n$, the maximum number of edges attained by a graph
from $T_n$ is obtained when $\card{A} = n/2$, and is equal to $n^2/4$.
For odd $n$, it is obtained when $\card{A} = (n-1)/2$, and is equal to
$(n-1)(n+1)/4 = (n^2-1)/4$.  The maximum number of edges can be
written compactly as $m_n = \floor{n^2/4}$, for any $n$.  Due to the
result shown in the previous part, that is also the maximum number of
edges of any graph in $S_n$.
\end{solution}

\eparts

\end{problem}

%%%%%%%%%%%%%%%%%%%%%%%%%%%%%%%%%%%%%%%%%%%%%%%%%%%%%%%%%%%%%%%%%%%%%
% Problem ends here
%%%%%%%%%%%%%%%%%%%%%%%%%%%%%%%%%%%%%%%%%%%%%%%%%%%%%%%%%%%%%%%%%%%%%

\endinput
