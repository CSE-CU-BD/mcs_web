\documentclass[problem]{mcs}

\begin{pcomments}
  \pcomment{MQ_one_committee}
\end{pcomments}

\pkeywords{
}

%%%%%%%%%%%%%%%%%%%%%%%%%%%%%%%%%%%%%%%%%%%%%%%%%%%%%%%%%%%%%%%%%%%%%
% Problem starts here
%%%%%%%%%%%%%%%%%%%%%%%%%%%%%%%%%%%%%%%%%%%%%%%%%%%%%%%%%%%%%%%%%%%%%

\begin{problem}
Twenty people work at CantorCorp, a small, unsuccessful start-up.  A
six-person committee is to be formed, charged with the sole task of proving
the Continuum Hypothesis.  All the people serving on the committee are of 
equal standing.  

\begin{problemparts}

\problempart
Let $D$ denote the set of all committees that could conceivably be formed.  Find $|D|$.
\begin{solution}
Evidently,
\[|D| = \binom{20}{6}.\]
\end{solution}

\problempart
Two of the workers, Aleph and Beth, will be unhappy if they are put on the same committee.

Let $P$ denote the set of committees on which Aleph and Beth would serve together.
Find $|P|$.
\begin{solution}
With Aleph and Beth serving, there are $20-2=18$ employees left to choose from, and $6-2=4$ spots left in the committee, so
\[|P|=\binom{18}{4}.\]
\end{solution}

\problempart
Beth will also be unhappy if she has to serve on a committee with \textbf{both} Ferdinand and Georg.

Let $Q$ denote the set of committees on which Beth, Ferdinand, and Georg would all serve together.
Find $|Q|$.
\begin{solution}
By reasoning similar to that used to find $|P|$, obtain 
\[|Q|=\binom{17}{3}.\]
\end{solution}

\problempart
Find $|P\cap Q|$.
\begin{solution}
$P\cap Q$ is the set of all committees on which Aleph, Beth, Ferdinand, and Georg all serve together.  Clearly,
\[|P\cap Q|=\binom{16}{2}.\]
\end{solution}

\problempart
Let $S$ denote the set of \textbf{all} committees on which there is at least one unhappy employee.
Express $S$ in terms of $P$ and $Q$ \textbf{only}.
\begin{solution}
\[S = P\cup Q.\]
\end{solution}

\problempart
Find $|S|$.
\begin{solution}
Applying inclusion/exclusion, have
\begin{align*}
|S|       &= |P \cup Q| \\         
          &= |P| + |Q| - |P\cap Q| \\
          &= \binom{18}{4} + \binom{17}{3} - \binom{16}{2} \\
\end{align*}
\end{solution}

\problempart
How many committees can be formed on which no employee is unhappy? 
\begin{solution}
Let $R$ denote the set of committees on which no employee is unhappy.  Clearly, $D = R\cup S$ and 
$R\cap S = \emptyset$.  So apply the Sum Rule: $|D| = |R| + |S|$.  Hence:
\begin{align*}
|R| &= |D| - |S| \\
          &= \binom{20}{6} - \binom{18}{4} - \binom{17}{3} + \binom{16}{2}
\end{align*}
\end{solution}

\problempart
Suppose, finally, that we wanted to form two six-person committees instead of one.
(The second committee will work to disprove the Continuum Hypothesis, of course!)  
Each employee can serve on at most one committee.  How many ways are there to form 
such a pair of committees, if employee happiness is \textbf{not} taken into consideration?
\begin{solution}
There are $\displaystyle\binom{20}{6}$ ways to choose employees for the first committee.
For each of these, there are $20-6=1$4 people left over and so $\displaystyle\binom{14}{6}$ ways to assign
employees to the second committee.  By the Generalized Product Rule (remember, the two committes do different
things), the number of ways to form two six-person committes is
\[\binom{20}{6}\binom{14}{6}.\]
\end{solution}

\end{problemparts}

\end{problem}

%%%%%%%%%%%%%%%%%%%%%%%%%%%%%%%%%%%%%%%%%%%%%%%%%%%%%%%%%%%%%%%%%%%%%
% Problem ends here
%%%%%%%%%%%%%%%%%%%%%%%%%%%%%%%%%%%%%%%%%%%%%%%%%%%%%%%%%%%%%%%%%%%%%
