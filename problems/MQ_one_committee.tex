\documentclass[problem]{mcs}

\begin{pcomments}
  \pcomment{MQ_one_committee}
\end{pcomments}

\pkeywords{
}

%%%%%%%%%%%%%%%%%%%%%%%%%%%%%%%%%%%%%%%%%%%%%%%%%%%%%%%%%%%%%%%%%%%%%
% Problem starts here
%%%%%%%%%%%%%%%%%%%%%%%%%%%%%%%%%%%%%%%%%%%%%%%%%%%%%%%%%%%%%%%%%%%%%

\begin{problem}
Twenty people work at CantorCorp, a small, unsuccessful start-up.  A single
six-person committee is to be formed.  (It will be charged with the sole task of 
working to prove the Continuum Hypothesis.)  Employees appointed to serve on the
committee join as equals---they do not get assigned distinct roles or ranks.

\begin{problemparts}

\problempart
Let $D$ denote the set of all possible committees.  Find $\card{D}$.
\examspace[1in]
\begin{solution}
Evidently,
\[\card{D} = \binom{20}{6}.\]
\end{solution}

\problempart
Two of the workers, Aleph and Beth, will be unhappy if they are to serve together.

Let $P$ denote the set of all possible committees on which Aleph and Beth would serve together.
Find $\card{P}$.

\examspace[1in]

\begin{solution}
With Aleph and Beth serving, there are $20-2=18$ employees left to
choose from, and $6-2=4$ spots left in the committee, so
\[\card{P}=\binom{18}{4}.\]
\end{solution}

\problempart
Beth will also be unhappy if she has to serve with \textbf{both} Ferdinand and Georg.

Let $Q$ denote the set of all possible committees on which Beth, Ferdinand, and Georg would all serve 
together.
Find $|Q|$.
\examspace[1in]
\begin{solution}
By reasoning similar to that used to find $|P|$, obtain 
\[|Q|=\binom{17}{3}.\]
\end{solution}

\problempart
Find $|P\cap Q|$.
\examspace[1in]
\begin{solution}
$P\cap Q$ is the set of all possible committees on which Aleph, Beth, Ferdinand, and Georg all serve 
together.  Clearly,
\[|P\cap Q|=\binom{16}{2}.\]
\end{solution}

\problempart
Let $S$ denote the set of all possible committees on which there is at least one unhappy employee.
Express $S$ in terms of $P$ and $Q$ \textbf{only}.
\examspace[1in]
\begin{solution}
\[S = P\cup Q.\]
\end{solution}

\problempart
Find $|S|$.
\examspace[1.5in]
\begin{solution}
Applying inclusion/exclusion, have
\begin{align*}
|S|       &= |P \cup Q| \\         
          &= |P| + |Q| - |P\cap Q| \\
          &= \binom{18}{4} + \binom{17}{3} - \binom{16}{2} \\
\end{align*}
\end{solution}

\problempart
If we want to form a committee with no unhappy employees, how many choices do we have to choose from? 
\examspace[1.5in]
\begin{solution}
Let $R$ denote the set of all possible committees on which no employee is unhappy.  Clearly, $D = R\cup S$ 
and $R\cap S = \emptyset$.  So apply the Sum Rule: $|D| = |R| + |S|$.  Hence:
\begin{align*}
|R| &= |D| - |S| \\
          &= \binom{20}{6} - \binom{18}{4} - \binom{17}{3} + \binom{16}{2}
\end{align*}
\end{solution}

\problempart
Suddenly, we realize that it would be better to have two six-person committees instead of one.
(One committee would work on proving the Continuum Hypothesis, while the other would work to disprove 
it!)  Each employee can serve on at most one committee.  How many ways are there to form 
such a pair of committees, if employee happiness is \textbf{not} taken into consideration?
\examspace[2in]
\begin{solution}
There are $\displaystyle\binom{20}{6}$ ways to choose employees for the first committee.
For each of these, there are $20-6=1$4 people left over and so $\displaystyle\binom{14}{6}$ ways to assign
employees to the second committee.  By the Generalized Product Rule (remember, the two committes do different
things), the number of ways to form two six-person committes is
\[\binom{20}{6}\binom{14}{6}.\]
Alternatively, notice that we are just counting the $\paren{6,6,8}$-splits of the set of all employees.  So
another way to write the answer is
\[\binom{20}{6,6,8}.\]
\end{solution}

\end{problemparts}

\end{problem}

%%%%%%%%%%%%%%%%%%%%%%%%%%%%%%%%%%%%%%%%%%%%%%%%%%%%%%%%%%%%%%%%%%%%%
% Problem ends here
%%%%%%%%%%%%%%%%%%%%%%%%%%%%%%%%%%%%%%%%%%%%%%%%%%%%%%%%%%%%%%%%%%%%%
