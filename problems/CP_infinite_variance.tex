\documentclass[problem]{mcs}

\begin{pcomments}
  \pcomment{CP_infinite_variance}
  \pcomment{just like CP_sqrt_infinite_expectation}
  \pcomment{F07.rec14h (commented out), S06.cp13f, F03.rec11}
  \pcomment{edited 5/9/14}
\end{pcomments}

\pkeywords{
  random_variable
  expectation
  variance
  infinite
  pdf
}

%%%%%%%%%%%%%%%%%%%%%%%%%%%%%%%%%%%%%%%%%%%%%%%%%%%%%%%%%%%%%%%%%%%%%
% Problem starts here
%%%%%%%%%%%%%%%%%%%%%%%%%%%%%%%%%%%%%%%%%%%%%%%%%%%%%%%%%%%%%%%%%%%%%

\begin{problem}
Let $R$ be a positive integer valued random variable such that
\[
\pdf_R(n)  = \frac{1}{cn^3},
\]
where
\[
c \eqdef \sum_{n=1}^{\infty} \frac{1}{n^3}.
\]

\bparts

\ppart\label{ppart:exf} Prove that $\expect{R}$ is finite.

\begin{solution}
	
\[
\expect{R} \eqdef \sum_{n \in \nngint^{+}} n \cdot \frac{1}{cn^3}
 = \sum_{n \in \nngint^{+}} \frac{1}{cn^2}
 < 1 + \int_{1}^{\infty} \frac{1}{cx^2}\,dx  =  1 + \frac{1}{c} < \infty.
\]

\end{solution}

\ppart\label{ppart:varinf}
Prove that $\expect{R^2}$ and therefore $\variance{R}$ are both infinite.

\begin{solution}
Since
\[
\variance{R} = \expect{R^2} - \expectsq{R},
\]
and $\expectsq{R} < \infty$ by part~\eqref{ppart:exf}, we only need to show
that $\expect{R^2} = \infty$:
\[
\expect{R^2} \eqdef \sum_{n \in \nngint^{+}} n^2\frac{1}{cn^3}
 =  \sum_{n \in \nngint^{+}} \frac{1}{cn}
 = \frac{1}{c} \cdot \lim_{n \to \infty} H_n  = \infty.
\]
\end{solution}

A joking way to phrase the point of this example is ``the square root
of infinity may be finite.''  Namely, let $T \eqdef R^2$; then
part~\eqref{ppart:varinf} implies that $\expect{T} = \infty$ while
$\expect{\sqrt{T}} < \infty$ by~\eqref{ppart:exf}.

\eparts

\end{problem}

%%%%%%%%%%%%%%%%%%%%%%%%%%%%%%%%%%%%%%%%%%%%%%%%%%%%%%%%%%%%%%%%%%%%%
% Problem ends here
%%%%%%%%%%%%%%%%%%%%%%%%%%%%%%%%%%%%%%%%%%%%%%%%%%%%%%%%%%%%%%%%%%%%%

