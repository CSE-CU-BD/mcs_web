\documentclass[problem]{mcs}

\begin{pcomments}
  \pcomment{PS_bijective_FLT}
  \pcomment{Author: Justin Venezuela (jven@mit.edu)}
  \pcomment{Problem is well-known.}
\end{pcomments}

\pkeywords{
  counting
  Fermat's Little Theorem
  coloring
}

%%%%%%%%%%%%%%%%%%%%%%%%%%%%%%%%%%%%%%%%%%%%%%%%%%%%%%%%%%%%%%%%%%%%%
% Problem starts here
%%%%%%%%%%%%%%%%%%%%%%%%%%%%%%%%%%%%%%%%%%%%%%%%%%%%%%%%%%%%%%%%%%%%%

\begin{problem}
We have seen earlier in the class the following theorem:\\

\noindent\textbf{Theorem:} \textit{(Fermat's Little Theorem)} If $p$ is a prime and $a$ is a positive integer, 
then $a^p\equiv a\pmod{p}$.\\

As you no doubt remember, we proved this theorem using purely number theoretic 
techniques. We will herein seek to prove the same result combinatorially.

\bparts

\ppart
Suppose we have beads, each colored one of $a$ colors, where $a$ is a positive integer. 
Suppose further that we have infinitely many beads of each color. We seek to make a 
circular bracelet of length $p$, where $p$ is a prime number. We first string together $p$ beads in a row. 
Show that there are exactly $a^p-a$ such strings of beads that consist of at least two colors.

\begin{solution}
There are $a^p$ total strings of beads since, for each of the $p$ beads in the string, we choose 
its color independently in $a$ ways. Of these strings, exactly $a$ of them consist of exactly 
one color: there is one such string for each color. The other $a^p-a$ strings of beads must consist 
of at least two colors.
\end{solution}

\ppart
For each string of $p$ beads consisting of at least two colors, we make it into a bracelet 
by tying the two ends of the string together. Two bracelets are considered indistinguishable 
if one can be rotated to yield the other. (Note, however, that you \textbf{cannot} "flip" a bracelet 
over or reflect it.) Show that for every bracelet, there are exactly $p$ strings of beads that 
yield it.

Hint: You must here make use of the fact that $p$ is prime and that the bracelet consists 
of at least two colors.

\begin{solution}
Given a bracelet, choose any of its beads and cut the string right before it, thereby choosing 
that bead as the first bead of a string. By construction, this string yields the bracelet. Now 
consider the act of moving the first bead to the end of the string. Repeating this operation $p$ 
times, we generate $p$ strings, all of which clearly yield the bracelet. It is also clear that 
no other string can yield the bracelet.

\textit{STAFF NOTE: This part was a bit difficult to make precise (if I even succeeded in being 
precise). In any case, I would suggest being lenient in grading this section, giving full 
credit so long as it is clear that the student seems to have an understanding why we require 
$p$ prime and at least two colors. -JVen}

It suffices to show that each of these $p$ strings is unique. Indeed, suppose for sake of 
contradiction that some two of the strings were the same. That is, given one of these strings, 
applying the above operation some number of times less than $p$ yields the same string. 
Let $k$ be the smallest such number of times the operation must be applied to yield the 
same string. Note that applying the operation any multiple of $k$ times must also yield 
the same string. Let $tk$ be the greatest multiple of $k$ less than $p$. Note that applying 
the operation $p-tk=\rem{p}{k}$ times must also yield the same string. But $\rem{p}{k}<k$, 
so we must have $\rem{p}{k}=0$: that is, $k$ is a factor of $p$. \textit{Since} $p$ 
\textit{is prime,} $k$ must be $1$ or $p$. But \textit{since the string consists of 
at least two colors}, applying the operation once cannot yield the same string, so that 
$k\neq 1$. It follows that $k=p$, which contradicts the fact that $k<p$. We conclude 
that no two of the $p$ strings are the same, and we're done.
\end{solution}

\ppart
What does $\frac{a^p-a}{p}$ represent in this context? Conclude that this quantity 
must be an integer and that this implies $a^p\equiv a\pmod{p}$.

\begin{solution}
This quantity represents the number of indistinguishable bracelets, which must 
clearly be an integer. Thus $p$ divides $a^p-a$, or $a^p-a\equiv 0\implies 
a^p\equiv a\pmod{p}$, as desired.
\end{solution}

\eparts
\end{problem}

%%%%%%%%%%%%%%%%%%%%%%%%%%%%%%%%%%%%%%%%%%%%%%%%%%%%%%%%%%%%%%%%%%%%%
% Problem ends here
%%%%%%%%%%%%%%%%%%%%%%%%%%%%%%%%%%%%%%%%%%%%%%%%%%%%%%%%%%%%%%%%%%%%%

\endinput
