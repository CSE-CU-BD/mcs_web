\documentclass[problem]{mcs}

\begin{pcomments}
  \pcomment{FP_card_bij}
  \pcomment{ARM, 2/27/14}
\end{pcomments}

\pkeywords{
  bijection
  infinite
  cardinality
  countable
}

%%%%%%%%%%%%%%%%%%%%%%%%%%%%%%%%%%%%%%%%%%%%%%%%%%%%%%%%%%%%%%%%%%%%%
% Problem starts here
%%%%%%%%%%%%%%%%%%%%%%%%%%%%%%%%%%%%%%%%%%%%%%%%%%%%%%%%%%%%%%%%%%%%%

\begin{problem}

Describe which of the following sets have bijections between them:

\[\begin{array}{ll}
\integers\ \text{(integers)}, & \reals\ \text{(real numbers)},\\
\complexes\ \text{(complex numbers)}, & \rationals\ \text{(rational
  numbers)},\\
\power(\integers)\ \text{(all subsets of integers)}, &  \power(\emptyset),\\
\power(\power(\emptyset)), & \finbin\ \text{(finite binary sequences)},\\
\binw\ \text{(infinite binary sequences)} & \set{\true,\false}\ \text{(truth values)}\\
\power(\set{\true,\false}), & \power(\binw)
\end{array}
\]

\begin{solution}
\[\begin{array}{ll}
\text{size 2:} & \power(\emptyset), \set{\true,\false}\\
\text{size 4:} & \power(\power(\emptyset)), \power(\set{\true,\false})\\
\text{countably infinite:} & \integers, \rationals, \finbin\\
\text{continuum:} & \reals, \complexes, \power(\integers), \binw\\
\text{even bigger:} & \power(\binw)
\end{array}\]
\end{solution}

\iffalse
Give an example of two sets $A,B$, such that
\[
\nngint \strict A \strict B.
\]
\inhandout{
(Reminder: $A \strict B$ means there is no surjective function from
$A$ to $B$.)
}

\begin{solution}
Familiar values for $A$ are $\reals$, $\power(\nngint)$, and 
$\set{0,1}^{\omega}$. The corresponding values for $B$ would be
$\power(\reals)$, and $\power(\power(\nngint))$.
\end{solution}
\fi

\end{problem}

%%%%%%%%%%%%%%%%%%%%%%%%%%%%%%%%%%%%%%%%%%%%%%%%%%%%%%%%%%%%%%%%%%%%%
% Problem ends here
%%%%%%%%%%%%%%%%%%%%%%%%%%%%%%%%%%%%%%%%%%%%%%%%%%%%%%%%%%%%%%%%%%%%%

\endinput


\iffalse

\ppart Prove that if $A$ and $B$ are countable sets, then so is $A
\union B$.

\begin{solution}

\begin{proof}
Suppose we list the elements of $A$ as $a_0,a_1,\dots$ and the
elements of $B$ as $b_0,b_1, \dots$.  Then a list of all the elements in $A
\union B$ can be written as
\begin{equation}\label{FP_a0b0list}
a_0,b_0,a_1,b_1, \dots a_n,b_n, \dots.
\end{equation}
Of course, this list can potentially contain duplicates if $A$ and $B$ have elements
in common, but then deleting all but the first occurrences of each element in
list~\eqref{FP_a0b0list} leaves a list of all the {\em distinct} elements of
$A$ and $B$, which is the same as the list of elements in
$A \union B$.
\end{proof}

\end{solution}

\examspace[1.5in]
\fi


