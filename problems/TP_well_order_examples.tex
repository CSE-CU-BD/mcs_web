\documentclass[problem]{mcs}

\begin{pcomments}
  \pcomment{TP_well_order_examples}
  \pcomment{ARM 2/4/12}
\end{pcomments}

\pkeywords{
  well_order
  minimum
}

\begin{problem}

\inhandout{A set of numbers is \term{well ordered} when each of its nonempty
subsets has a minimum element.  The Well Ordering principle says, of
course, that the set of nonnegative integers is well ordered, but so
are lots of other sets, for example, the set $r\naturals$ of numbers
of the form $rn$, where $r$ is a positive real number and $n \in
\naturals$.}

Indicate which of the following sets of numbers have a minimum element
and which are well ordered.  For those that are not well ordered, give
an example of a subset with no minimum element.
     
\bparts
    
\ppart\label{geqsqrt2} The integers $\geq - \sqrt{2}$.

\begin{solution}
These have a minimum element -1, and are well ordered by
Corollary~\bref{lower_bound_cor}.
\end{solution}

\iffalse

\ppart The integers $> \sqrt{2}$.

\begin{solution}
\dots same as part~\eqref{geqsqrt2}.
\end{solution}
\fi

\ppart The rational numbers $\geq \sqrt{2}$.

\begin{solution}
These have no minimum element and so are not well ordered.

They have no minimum element because $\sqrt{2}$ is irrational
(Theorem~\bref{thm:sqrt2irr_by_contra}).  So if $r$ is rational number
$\geq \sqrt{2}$, then in fact $r > \sqrt{2}$.  This means that $r$ is
not a minimum element $\geq \sqrt{2}$ because the open-ended real
interval $(\sqrt{2},r)$ has positive length, and therefore contains a
rational number, $q$, which is less than $r$ and greater than
$\sqrt{2}$.
\end{solution}

\ppart The set of rationals of the form $1/n$ where $n$ is a
positive integer.

\begin{solution}
No minimum element.
\end{solution}

\ppart\label{fin-below} A set, $R$, of real numbers with the property
that every element of $R$ has only finitely many elements of $R$ below
it.  All the sets $r\naturals$ for real $r>0$ have this property.

\begin{solution}
Since there are only finitely many elements in $R$ below any element
of $R$, there can't be an infinite decreasing sequence of elements in
$R$.  This implies that $R$ is well ordered (see
Problem~\bref{CP_well_order_decreasing}).
\end{solution}

\iffalse

\ppart The set of rationals of the form $1/n$ where $n$ is a positive
integer less than or equal to a $g \eqdef 10^{100}$ (a \term{google}).

\begin{solution}
This is a finite set of size a google, and so is well ordered.  Its
minimum element is $1/g$.
\end{solution}
\fi

\ppart The set $G$ of rationals of the form $m/n$ where $m$ and $n$
are positive integers and $n \leq g \eqdef 10^{100}$ (a \term{google}).

\begin{solution}
$G$ has minimum element $1/g$.  Furthermore, there are only finitely
  many numbers in $G$ that are less than any given real number, $r$,
  so $G$ is well ordered by part~\eqref{fin-below}.

To see why there are only finitely many elements in $G$ below any
element of $G$, note that every number in $G$ can be written as its
quotient, $q$, on division by $g$ plus a remainder of at most $g$.  So
there are at $qg$ numbers in $G$ that are less than a given element of
$G$.
\end{solution}

\ppart The set $H$ of rational numbers of the form $n/(n+1)$ where $n$
is a nonnegative integer.

\begin{solution}
$H$ has minimum element 0.  Since there are only $n$ elements in $H$
  less than the element $n/(n+1) \in H$, the set $H$ is well ordered by
  part~\eqref{Fin-below}.
\end{solution}


\ppart The set $J \eqdef H \union \naturals$.  (Note that $J$ does not
satisfy the finiteness condition of part~\eqref{Fin-below}.)

\begin{solution}
$J$ has minimum element 0.  It does not satisfy the finiteness
  condition of part~\eqref{Fin-below} because there are infinitely
  many numbers in $J$ that are less then each intger in $J$, namely,
  all the elements of $H$ are numbers in $J$ that are less than 1.

  Even so, any decreasing sequence of elements $J$ must be finite, so
  $J$ is well ordered by Problem~\bref{CP_well_order_decreasing}.

  To see why such a decreasing sequence must be finite, suppose the
  sequence starts with an nonnegative integer $n$.  It can only
  contain $n$ more integers, and a number in the sequence after the
  last integer, if any, must be an element of $H$ of the form
  $m/(m+1)$, which will have only $m$ more elements below it.
\end{solution}

\ppart The set
\[
K \eqdef \set{n + h \suchthat n \in \naturals \QAND h \in H}.
\]

\begin{solution}
$K$ has minimum element 0.  To find the minimum of any subset $S
  \subseteq K$, let $n_S \in \naturals$ be the least nonnegative
  integer in $\set{m \suchthat m+h \in S \text{ for some } h \in H}$.
  Such an $n_S$ exists by the Well Ordering Principle.  Then let
  $h_S \in H$ be the least element of $H$ in $\set{h \in H \suchthat
    n_0+h \in S}$.  Such an $h_S$ exists since $H$ is well ordered.

  Now it's easy to verify that $n_S+h_S$ is the minimum element of $S$.
\end{solution}



\eparts

\end{problem}

\endinput
