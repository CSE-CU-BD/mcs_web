\documentclass[problem]{mcs}

\begin{pcomments}
  \pcomment{TP_well_order_examples}
  \pcomment{ARM 2/9/12, revised 2/12/12}
\end{pcomments}

\pkeywords{
  well_order
  minimum
  strictly_decreasing
  quotient
  remainder
  floor
}

\begin{problem}

\inhandout{A set of numbers is \term{well ordered} when each of its nonempty
subsets has a minimum element.  The Well Ordering Principle says, of
course, that the set of nonnegative integers is well ordered, but so
are lots of other sets, for example, the set $r\naturals$ of numbers
of the form $rn$, where $r$ is a positive real number and $n \in
\naturals$.}

Indicate which of the following sets of numbers have a minimum element
and which are well ordered.  For those that are not well ordered, give
an example of a subset with no minimum element.
     
\bparts
    
\ppart\label{geqsqrt2} The integers $\geq - \sqrt{2}$.

\begin{solution}
These have a minimum element -1, and are well ordered by
Corollary~\bref{lower_bound_cor}.
\end{solution}

\iffalse

\ppart The integers $> \sqrt{2}$.

\begin{solution}
\dots same as part~\eqref{geqsqrt2}.
\end{solution}
\fi

\ppart The rational numbers $\geq \sqrt{2}$.

\begin{solution}
These have no minimum element and so are not well ordered.

They have no minimum element because $\sqrt{2}$ is irrational
(Theorem~\bref{thm:sqrt2irr_by_contra}).  So if $r$ is rational number
$\geq \sqrt{2}$, then in fact $r > \sqrt{2}$.  This means that $r$ is
not a minimum element $\geq \sqrt{2}$ because the open-ended real
interval $(\sqrt{2},r)$ has positive length, and therefore contains a
rational number, $q$, which is less than $r$ and greater than
$\sqrt{2}$.
\end{solution}

\ppart The set of rationals of the form $1/n$ where $n$ is a
positive integer.

\begin{solution}
No minimum element.
\end{solution}


\iffalse

\ppart The set of rationals of the form $1/n$ where $n$ is a positive
integer less than or equal to a $g \eqdef 10^{100}$ (a \term{google}).

\begin{solution}
This is a finite set of size a google, and so is well ordered.  Its
minimum element $1/g$.
\end{solution}
\fi

\ppart The set $G$ of rationals of the form $m/n$ where $m,n >0$ and
$n \leq g$ where $g$ is a \term{google}, namely, $10^{100}$.

\begin{staffnotes}
\hint Express the elements in $G$ in terms of a common denominator.
\end{staffnotes}

\begin{solution}
$G$ has minimum element $1/g$ and is well ordered.

To see why $G$ is well ordered, note that all the numbers in $G$ can
be expressed with a common denominator.  Namely, let
\[
d \eqdef g! \eqdef 2\cdot 3\cdot 4 \cdots (g-1) \cdot g.
\]
Then each number in $G$ can be written in the form $k/d$ for some
\emph{unique} nonnegative integer numerator $k$.  So to find the
minimum element in any nonempty subset of $G$, express all the
elements with the common denominator $d$ and choose the one with
minimum numerator ---which exists by the WOP.
\end{solution}

\ppart The set $\twdone$ of fractions increasing to the limit 1:
\[
\frac01, \frac12, \frac23, \frac34, \dots, \frac{n}{n+1}, \dots.
\]

\begin{solution}
$\twdone$ has minimum element 0.  The minimum element of a nonempty
  subset of $\twdone$ is simply the one with the minimum numerator
  when expressed in the form $n/(n+1)$.
\end{solution}

\ppart The set $W$ consisting of the nonnegative integers %, $\naturals$,
along with all the fractions in $\twdone$.  Do you notice anything
different about $W$ compared to the earlier well ordered examples?

\begin{staffnotes}
If a hint is needed, suggest considering the two cases that a subset
includes a fraction from $\twdone$ or not.
\end{staffnotes}

\begin{solution}
$W$ has minimum element 0 and is well ordered.

To show it is well ordered, we must show that every nonempty subset of
$W$ has a minimum element.  But if a subset includes a fraction from
$\twdone$, then, because all the elements in $\twdone$ are less than
1, the minimum element in the subset is the same as the minimum
element among those in $\twdone$, which exists since $\twdone$ is
well ordered.  Otherwise, the subset consists solely of nonnegative
integers and has a minimum element by the WOP.

What's different about $W$ is that some elements, namely all the
positive integers, can be the first element in strictly
decreasing sequences of elements of arbitrary finite length.  For
example, the following decreasing sequences of elements in $W$
all start with 1:
\[\begin{array}{l}
1, 0.\\
1, \frac12, 0.\\
1, \frac23, frac12, 0.\\
1,  \frac34, \frac23, frac12, 0.\\
\qquad\vdots
\end{array}\]
Nevertheless, since $W$ is well ordered, it is impossible to find any
infinite decreasing sequence of elements in $W$, since the elements in
any such sequence would be a subset with no minimum element.
\end{solution}

\iffalse

\ppart\label{fin-below} A set, $R$, of real numbers with the property
that every element of $R$ has only finitely many elements of $R$ below
it.  All the sets $r\naturals$ for real $r>0$ have this property.

\hint You may assume that every finite set of real numbers has a
minimum element.  (If you feel this assumption requires a proof, see
Problem~\bref{TP_min_of_finite_WOP}).

\begin{staffnotes}
This is a point where students may get lost in unfamiliar reasoning
about abstract properties.  If they are stuck, give them $S_{\leq r}$
and have them verify that $\min S_{\leq r} = \min S$.
\end{staffnotes}

\begin{solution}
To show that $R$ is well ordered, let $S$ be a nonempty subset of $R$,
and suppose $r \in S$.  We claim $S$ has a minimum element.  To see
why, let $S_{\leq r}$ be the set of elements in $S$ that are less than
or equal to $r$.  A minimum number in $S_{\leq r}$ is obviously also a
minimum number in $S$.\footnote{Some things really are too obvious to
  belabor with a proof ---we're not trying to intimidate you into
  accepting something doubtful.}  But $S_{\leq r}$ is finite by
hypothesis, and therefore has a minimum element.
\end{solution}
\fi


\eparts

\end{problem}

\endinput
