\documentclass[problem]{mcs}

\begin{pcomments}
  \pcomment{PS_Lehmans_equation}
  \pcomment{from: S09.ps1, S04 (Eric Lehman)}
  \pcomment{minor edits by ARM in S09}
  \pcomment{Fix hard reference to Week 1 notes.}
\end{pcomments}

\pkeywords{
  well_ordering
  WOP
  contradiction
}

%%%%%%%%%%%%%%%%%%%%%%%%%%%%%%%%%%%%%%%%%%%%%%%%%%%%%%%%%%%%%%%%%%%%%
% Problem starts here
%%%%%%%%%%%%%%%%%%%%%%%%%%%%%%%%%%%%%%%%%%%%%%%%%%%%%%%%%%%%%%%%%%%%%

\begin{problem}
  \href{http://mathworld.wolfram.com/EulersSumofPowersConjecture.html}{\emph{Euler's
      Conjecture}} in 1769 was that there are no positive integer
  solutions to the equation
\[
a^4 + b^4 + c^4 =  d^4.
\]
Integer values for $a,b,c,d$ that do satisfy this equation were first
discovered in 1986.  So Euler guessed wrong, but it took more two hundred
years to prove it.

Now let's consider Lehman's equation, similar to Euler's but with some
coefficients:
\begin{equation}\label{lehman-eqn}
8 a^4 + 4 b^4 + 2 c^4  =  d^4
\end{equation}

Prove that Lehman's equation~\eqref{lehman-eqn} really does not have any positive
integer solutions.

\hint Consider the minimum value of $a$ among all possible solutions
to~\eqref{lehman-eqn}.

\begin{solution}
Suppose that there exists a solution.  Then there must be a
  solution in which $a$ has the smallest possible value.  We will show
  that, in this solution, $a$, $b$, $c$, and $d$ must all be even.
  However, we can then obtain another solution over the positive integers
  with a smaller $a$ by dividing $a$, $b$, $c$, and $d$ in half.  This is
  a contradiction, and so no solution exists.

All that remains is to show that $a$, $b$, $c$, and $d$ must all be even.
The left side of Lehman's equation is even, so $d^4$ is even, so $d$ must
be even.  Substituting $d = 2 d'$ into Lehman's equation gives:

\begin{equation}
8 a^4 + 4 b^4 + 2 c^4 = 16 d'^4
\end{equation}

Now $2 c^4$ must be a multiple of 4, since every other term is a
multiple of 4.  This implies that $c^4$ is even and so $c$ is also
even.  Substituting $c = 2 c'$ into the previous equation gives:

\begin{equation}
8 a^4 + 4 b^4 + 32 c'^4 = 16 d'^4
\end{equation}

Arguing in the same way, $4 b^4$ must be a multiple of 8, since every
other term is.  Therefore, $b^4$ is even and so $b$ is even.
Substituting $b = 2 b'$ gives:

\begin{equation}
8 a^4 + 64 b'^4 + 32 c'^4 = 16 d'^4
\end{equation}

Finally, $8 a^4$ must be a multiple of 16, $a^4$ must be even, and so
$a$ must also be even.  Therefore, $a$, $b$, $c$, and $d$ must all be
even, as claimed.
\end{solution}
\end{problem}

%%%%%%%%%%%%%%%%%%%%%%%%%%%%%%%%%%%%%%%%%%%%%%%%%%%%%%%%%%%%%%%%%%%%%
% Problem ends here
%%%%%%%%%%%%%%%%%%%%%%%%%%%%%%%%%%%%%%%%%%%%%%%%%%%%%%%%%%%%%%%%%%%%%

\endinput
