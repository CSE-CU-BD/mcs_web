\documentclass[problem]{mcs}

\begin{pcomments}
  \pcomment{FP_infinite_binary_sequences_bare}
  \pcomment{Suggested by ARM, solution written by zabel, 10/27/17 for f17 midterm 3}
  \pcomment{Less-guidance version of FP_infinite_binary_sequences_S14}
  \pcomment{subsumes FP_infinite_binary_sequences}
  \pcomment{ARM 5/15/14}
\end{pcomments}

\pkeywords{
  countable
  union
  infinite_sequence
  uncountable
}

%%%%%%%%%%%%%%%%%%%%%%%%%%%%%%%%%%%%%%%%%%%%%%%%%%%%%%%%%%%%%%%%%%%%%
% Problem starts here
%%%%%%%%%%%%%%%%%%%%%%%%%%%%%%%%%%%%%%%%%%%%%%%%%%%%%%%%%%%%%%%%%%%%%

\begin{problem}
  Let $\binw$ be the set of infinite binary sequences, and let $F$ be the set of sequences in $\binw$ that contain an \emph{infinite} number of 1s. Prove that $F$ is uncountable.

  \hint Consider a diagonal argument like those done in class.
  
%  \hint You may assume $\binw$ itself is uncountable. One approach is to use diagonalization with the ``slope'' trick. Another approach is to argue that $\binw\inj F$.
\end{problem}

\begin{solution}
 
  \begin{proof}
    We'll argue by diagonalization, using ``slope $-1/2$'' like we saw in class. For the sake of contradiction, assume $F$ is countable. Since $F$ is certainly infinite, it must be in bijection with $\mathbb{N}$, so we can write $F = \{s_0, s_1, s_2, \ldots\}$. Define the new sequence $t$ as follows:
    \begin{equation*}
      t \eqdef (\bar{s_0(0)}, 1, \bar{s_1(2)}, 1, \bar{s_2(4)}, 1, \ldots),
    \end{equation*}
    whose $2n$th entry is the \emph{opposite} of $s_n(2n)$, and whose $2n+1$st entry is $1$. This sequence $t$ contains infinitely many $1$s by construction, so $t\in F$. On the other hand, $t$ cannot equal $s_k$ for any $k\in\mathbb{N}$, because $t$ and $s_k$ differ in their $2k$th entry, again by construction. Thus, $t\notin\{s_0,s_1,s_2,\ldots\}$, i.e., $t\notin F$. This is a contradiction, so our assumption that $F$ is countable must be false, i.e., $F$ is countable.
  \end{proof}

  A similar, alternate proof does not rely on diagonalization at all.

  \begin{proof}
    We'll show that $\binw \inj F$, which suffices because we already know that $\binw$ is uncountable. Define the function $g: \binw\to F$ as follows: for every $s\in\binw$, define
    \begin{equation*}
      g(s) \eqdef (s(0), 1, s(1), 1, s(2), 1, \ldots),
    \end{equation*}
    which lies in $F$ because it has infinitely many 1s by construction. This function $g$ is certainly total. To see that $g$ is injective, observe that if $s\ne t \in\binw$, then there is some $k\in\mathbb{N}$ with $s(k)\ne t(k)$, and then $g(s)$ and $g(t)$ will differ at index $2k$ and will thus be different. So $g$ is a total injective relation, meaning $\binw \inj F$, as required.
  \end{proof}
\end{solution}
  


%%%%%%%%%%%%%%%%%%%%%%%%%%%%%%%%%%%%%%%%%%%%%%%%%%%%%%%%%%%%%%%%%%%%%
% Problem ends here
%%%%%%%%%%%%%%%%%%%%%%%%%%%%%%%%%%%%%%%%%%%%%%%%%%%%%%%%%%%%%%%%%%%%%

\endinput
