\documentclass[problem]{mcs}

\begin{pcomments}
  \pcomment{FP_gcd_TF}
  \pcomment{from FP_multiple_choice_unhidden_fall13}
  \pcomment{revised from FP_multiple_choice_unhidden by ARM 12/13/13}
  \pcomment{overlaps FP_graphs_short_answer}  
\end{pcomments}

\pkeywords{
  gcd
  divides
}

%%%%%%%%%%%%%%%%%%%%%%%%%%%%%%%%%%%%%%%%%%%%%%%%%%%%%%%%%%%%%%%%%%%%%
% Problem starts here
%%%%%%%%%%%%%%%%%%%%%%%%%%%%%%%%%%%%%%%%%%%%%%%%%%%%%%%%%%%%%%%%%%%%%

\begin{problem} \mbox{}

\textbf{\large Circle \textbf{true} or \textbf{false} for the
  following statements about the \textbf{greatest common divisor}, and
  \emph{provide counterexamples} for those that are \textbf{false}.}

\bparts

\ppart If $\gcd(a, b) \neq 1$ and $\gcd(b, c) \neq 1$, then $\gcd(a, c) \neq 1$. \hfill 
\textbf{true} \qquad  \textbf{false} \examspace[0.4in]

\begin{solution}
\textbf{false} $a=2\cdot 3, b=3\cdot 5, c=5\cdot 7$
\end{solution}

\iffalse
\ppart If $a \divides b c$ and $\gcd(a, b) = 1$, then $a \divides c$.  \hfill 
\textbf{true} \qquad \textbf{false} \examspace[0.4in]

\begin{solution}
\textbf{true}
\end{solution}
\fi

\ppart $\gcd(a^n,b^n) =  (\gcd(a,b))^n$  \hfill 
\textbf{true} \qquad \textbf{false} \examspace[0.4in]

\begin{solution}
\textbf{true}.
\end{solution}

\iffalse
\ppart $\gcd(ab, ac) = a \gcd(b, c)$.  \hfill 
\textbf{true} \qquad  \textbf{false} \examspace[0.4in]

\begin{solution}
\textbf{true}
\end{solution}
\fi

\ppart $\gcd(1 + a, 1 + b) = 1 + \gcd(a, b)$.  \hfill 
\textbf{true} \qquad  \textbf{false} \examspace[0.4in]

\begin{solution}
\textbf{false} $a=1,b=2$
\end{solution}


\iffalse
\ppart If an integer linear combination of $a$ and $b$ equals 1, then
  so does some integer linear combination of $a$ and $b^2$. \hfill
  \textbf{true} \qquad \textbf{false} \examspace[0.4in]

\begin{solution}
false $a=1,b=2$
\end{solution}

\begin{solution}
\textbf{true} An integer linear combination of $a$ and $b$ equals 1
iff $a,b$ are relatively prime.  But $a,b$ are relatively prime iff
$a,b^2$ are.
\end{solution}

\fi

\ppart If no integer linear combination of $a$ and $b$ equals 2, then
  neither does any integer linear combination of $a^2$ and
  $b^2$. \hfill \textbf{true} \qquad \textbf{false} \examspace[0.4in]

\begin{solution}
\textbf{true} No linear combination of $a,b$ is 2 iff $\gcd(a,b) >2$
iff $\gcd(a^2,b^2)>4$.
\end{solution}

\eparts

\end{problem}

\endinput
