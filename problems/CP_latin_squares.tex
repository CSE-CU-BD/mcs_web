\documentclass[problem]{mcs}

\begin{pcomments}
  \pcomment{from: S09.cp6r}
  \pcomment{from: S04.cp5f (edited ARM in S08)}
  \pcomment{This problem has a lot of commented out material that should be cleaned up.}
\end{pcomments}

\pkeywords{
  bipartite_matching
  degree-constrained
  graph_coloring
}

%%%%%%%%%%%%%%%%%%%%%%%%%%%%%%%%%%%%%%%%%%%%%%%%%%%%%%%%%%%%%%%%%%%%%
% Problem starts here
%%%%%%%%%%%%%%%%%%%%%%%%%%%%%%%%%%%%%%%%%%%%%%%%%%%%%%%%%%%%%%%%%%%%%

\begin{problem}

A \term{Latin square} is $n \times n$ array whose entries are the number
$1,\dots,n$.  These entries satisfy two constraints: every row contains
all $n$ integers in some order, and also every column contains all $n$
integers in some order.  Latin squares come up frequently in the design of
scientific experiments for reasons illustrated by a little story in a
footnote\footnote{At Guinness brewery in the eary 1900's, W. S. Gosset (a 
chemist) and E.  S. Beavan (a ``maltster'') were trying to improve the 
barley used to make the brew.  The brewery used different varieties of 
barley according to price and availability, and their agricultural 
consultants suggested a different fertilizer mix and best planting month 
for each variety.

Somewhat sceptical about paying high prices for customized fertilizer,
Gosset and Beavan planned a season long test of the influence of
fertilizer and planting month on barley yields.  For as many months as
there were varieties of barley, they would plant one sample of each
variety using a different one of the fertilizers.  So every month, they
would have all the barley varieties planted and all the fertilizers
used, which would give them a way to judge the overall quality of that
planting month.  But they also wanted to judge the fertilizers, so they
wanted each fertilizer to be used on each variety during the course of
the season.  Now they had a little mathematical problem, which we can
abstract as follows.

Suppose there are $n$ barley varieties and an equal number of
recommended fertilizers.  Form an $n \times n$ array with a column for
each fertilizer and a row for each planting month.  We want to fill in
the entries of this array with the integers 1,\dots,$n$ numbering the
barley varieties, so that every row contains all $n$ integers in some
order (so every month each variety is planted and each fertilizer is
used), and also every column contains all $n$ integers (so each
fertilizer is used on all the varieties over the course of the growing
season).}  

For example, here is a $4 \times 4$ Latin square:
{\Large
\[
\begin{array}{|c|c|c|c|}
\hline
1 & 2 & 3 & 4 \\ 
\hline
3 & 4 & 2 & 1 \\ 
\hline
2 & 1 & 4 & 3 \\ 
\hline
4 & 3 & 1 & 2 \\ 
\hline
\end{array}
\]
}

\bparts

\ppart

\iffalse
Before deciding to chuck it all and relax drinking a barrel of their
own product, the brewers has worked out the following \fi

Here are three rows of what could be part of a $5 \times 5$ Latin square:

{\Large
\[
\begin{array}{|c|c|c|c|c|}
\hline
2 & 4 & 5 & 3 & 1 \\ 
\hline
4 & 1 & 3 & 2 & 5 \\ 
\hline
3 & 2 & 1 & 5 & 4 \\ 
\hline
  &   &   &   &   \\ 
\hline
  &   &   &   &   \\
\hline
\end{array}
\]
}
Fill in the last two rows to extend this ``Latin rectangle'' to a complete
Latin square.

\begin{solution}
Here is one possible solution:

{\Large
\[
\begin{array}{|c|c|c|c|c|}
\hline
2 & 4 & 5 & 3 & 1 \\ 
\hline
4 & 1 & 3 & 2 & 5 \\ 
\hline
3 & 2 & 1 & 5 & 4 \\ 
\hline
1 & 5 & 2 & 4 & 3 \\ 
\hline
5 & 3 & 4 & 1 & 2 \\ 
\hline
\end{array}
\]
}
\end{solution}

%There are some nice connections between Latin squares and graph theory.
\iffalse 

\ppart
Construct a graph $G_n$ with $n^2$ vertices such that there is a
one-to-one correspondence between $n \times n$ Latin squares and valid
$n$-colorings of $G_n$.

\begin{solution}
Create a vertex in $G$ for each entry in the Latin Square.
Then connect each vertex to every other vertex in the same row and to
every other vertex in the same column.  Now color the graph with $n$
colors, each corresponding to a number between 1 and $n$.
Notice that every pair of vertices in the same row are connected, so
no two vertices in the same row can get the same color.  Similarly,
since every pair of vertices in the same column are connected, no two
vertices in the same column can get same color either.  These coloring
constraint match the constraints on Latin squares, so there is a
one-to-one correspondence between colorings of $G$ and $n \times n$
Latin squares.
\end{solution}

\eparts

Of course coloring problems can be difficult, so formulating Latin square
construction as a coloring problem is no direct help in finding them.  But
there's another approach to constructing Quadra\footnote{Get it? That's  
Latin for ``squares''.  Latin -- squares \smiley\ Okay, we'll stop making bad 
puns and let you get on with the problem.} based on bipartite matching.

\bparts
\fi

\ppart Show that filling in the next row of an $n \times n$ Latin
rectangle is equivalent to finding a matching in some $2n$-vertex
bipartite graph.

\begin{solution}
Construct a bipartite graph as follows. One set of vertices 
are  the columns of the Latin rectangle, and the other set is the 
numbers $1$  to $n$. Put an edge between a column and a number if the 
number has \emph{not yet appeared} in the column. Thus, a matching in
this graph would associate each column with a distinct number that has 
not yet appeared in that column. These numbers would form the next row 
of the Latin rectangle.
\end{solution}

\ppart Prove that a matching must exist in this bipartite graph and,
consequently, a Latin rectangle can always be extended to a Latin square.

\begin{solution}
Suppose the Latin rectangle has $k$ rows of width $n$. Then
each column-vertex has degree $n-k$ because its edges go to the $n - k$ 
numbers missing from the column.  Also, each number-vertex also has
degree $n - k$.  That's because each number appears exactly once in each
of the $k$ rows and at most once in each column, so each number must be
missing from exactly $n-k$ columns.

So the graph is degree-constrained and therefore has a matching.
This  implies that we can add rows to the Latin rectangle by the procedure
described above as long as $k < n$.  At that point, we have a Latin
square.
\end{solution}

\eparts
\end{problem}

%%%%%%%%%%%%%%%%%%%%%%%%%%%%%%%%%%%%%%%%%%%%%%%%%%%%%%%%%%%%%%%%%%%%%
% Problem ends here
%%%%%%%%%%%%%%%%%%%%%%%%%%%%%%%%%%%%%%%%%%%%%%%%%%%%%%%%%%%%%%%%%%%%%
