\documentclass[problem]{mcs}

\begin{pcomments}
  \pcomment{CP_AMM_root_2_proof}
  \pcomment{from: S09.cp1r}
\end{pcomments}

\pkeywords{
  root_2
  irrational
  well_ordering
  WOP
  contradiction
}

%%%%%%%%%%%%%%%%%%%%%%%%%%%%%%%%%%%%%%%%%%%%%%%%%%%%%%%%%%%%%%%%%%%%%
% Problem starts here
%%%%%%%%%%%%%%%%%%%%%%%%%%%%%%%%%%%%%%%%%%%%%%%%%%%%%%%%%%%%%%%%%%%%%

\begin{problem}
Here is a different proof that $\sqrt{2}$ is irrational, taken from the
American Mathematical Monthly, v.116, \#1, Jan. 2009, p.69:

\begin{proof}
  Suppose for the sake of contradiction that $\sqrt{2}$ is rational, and
  choose the least integer, $q>0$, such that $\paren{\sqrt{2}-1}q$ is a
  nonnegative integer.  Let $q' \eqdef \paren{\sqrt{2}-1}q$.  Clearly $0<
  q' < q$.  But an easy computation shows that $\paren{\sqrt{2}-1}q'$ is a
  nonnegative integer, contradicting the minimality of $q$.
\end{proof}

\bparts

\ppart This proof was written for an audience of college teachers, and
at this point it is a little more concise than desirable.  Write out a
more complete version which includes an explanation of each step.

\begin{solution}
The points that need justification are:

\begin{enumerate}
\item Why is there a positive integer, $q$, such that $\paren{\sqrt{2}-1}q$
  is a nonnegative integer?  \emph{Answer:} Since $\sqrt{2}$ is rational,
  so is $\sqrt{2}-1$.  So $\sqrt{2}-1$ can be expressed as an integer
  quotient with positive denominator; now just let $q$ be that
  denominator.

\item Why is there such a \emph{least} positive integer, $q$?
  \emph{Answer:} As long as there is one such positive integer, there has
  to be a least one.  This obvious fact is known as the
  \emph{Well Ordering Principle}.

\item Why is $0< q' < q$?  \emph{Answer:} We know that $1 < \sqrt{2} < 2$,
  so $0 < \sqrt{2}-1 < 1$.  Therefore, $0 < \paren{\sqrt{2}-1}r < r$ for any
  real number $r>0$.

\item Why is $(\sqrt{2}-1)q'$ a nonnegative integer?  \emph{Answer:} It's
  actually positive, because it is a product of positive numbers.  It's
  integer because
  \[
  \paren{\sqrt{2}-1}q' = \paren{\sqrt{2}-1}^2q = 2q - 2q\sqrt{2} + q =
  q-2\cdot\brac{\paren{\sqrt{2}-1}q}
  \]
  and the last term is of the form
  $\ang{\text{integer}-2 \cdot \brac{\text{integer}}}$.

\end{enumerate}
\end{solution}

\ppart Now that you have justified the steps in this proof, do you have a
preference for one of these proofs over the other?  Why?  Discuss these
questions with your teammates for a few minutes and summarize your team's
answers on your whiteboard.

\begin{staffnotes}
Push for at least an indication on the whiteboard of a pro and a con
for each proof.  (Don't settle for verbal comments.)
\end{staffnotes}

\begin{solution}
Both proofs seem about equally easy to understand.  The previous
  problems shows that the first proof generalizes pretty directly from
  square roots to $k$th roots, which doesn't seems as clear for the
  second proof.  On the other hand, the first proof requires appeal to
  Unique Prime Factorization, while the second just uses simple algebra.
\end{solution}

\eparts
\end{problem}

%%%%%%%%%%%%%%%%%%%%%%%%%%%%%%%%%%%%%%%%%%%%%%%%%%%%%%%%%%%%%%%%%%%%%
% Problem ends here
%%%%%%%%%%%%%%%%%%%%%%%%%%%%%%%%%%%%%%%%%%%%%%%%%%%%%%%%%%%%%%%%%%%%%

\endinput
