\documentclass[problem]{mcs}

\begin{pcomments}
  \pcomment{CP_bigger_number_game}
  \pcomment{from: S09.cp14m}
\end{pcomments}

\pkeywords{
  random_variables
  uniform
  four_step_method
  total_probability
}

%%%%%%%%%%%%%%%%%%%%%%%%%%%%%%%%%%%%%%%%%%%%%%%%%%%%%%%%%%%%%%%%%%%%%
% Problem starts here
%%%%%%%%%%%%%%%%%%%%%%%%%%%%%%%%%%%%%%%%%%%%%%%%%%%%%%%%%%%%%%%%%%%%%

%f07 rec13h
\textbox{
\begin{center}
\large{Guess the Bigger Number Game}
\end{center}

Team 1:
\begin{itemize}
\item  Write different integers between 0 and 7 on two pieces of paper.
\item  Put the papers face down on a table.
\end{itemize}

Team 2:
\begin{itemize}
\item Turn over one paper and look at the number on it.
\item Either stick with this number or switch to the unseen other
  number.
\end{itemize}

Team 2 wins if it chooses the larger number; else, Team 1 wins.
}

\begin{problem}
The analysis in section~\bref{bigger_number_subsec} implies that Team
2 has a strategy that wins 4/7 of the time no matter how Team 1 plays.
Can Team 2 do better?  The answer is ``no,'' because Team 1 has a
strategy that guarantees that it wins at least 3/7 of the time, no
matter how Team 2 plays.  Describe such a strategy for Team 1 and
explain why it works.

\begin{solution}
\begin{staffnotes}
Warn against assuming that Team 2 plays in any particular way such as
using thresholds.

To speed things up if need be, explain what the strategy is, and ask
for verification that $\pr{\text{win}} \geq 3/7$ \emph{no matter how
  Team 2 plays}.
\end{staffnotes}

Team 1 should randomly choose a number $Z \in \set{0,\dots,6}$ and
write $Z$ and $Z+1$ on the pieces of paper with all numbers equally
likely.  Then place the paper with $Z$ on it to the left or right with
equal probability.

To see why this works, let $N$ be the number on the paper that Team 2
turns over, and let $\text{OK}$ be the event that $N \in [1,6]$.  So
given event $\text{OK}$, that is, given that $1\leq N \leq 6$, Team
1's strategy ensures that half the time $N$ is the higher number and
half the time $N$ is the lower number.  So given event $\text{OK}$,
the probability that Team 1 wins is exactly 1/2 \emph{no matter how
  Team 2 chooses to play} (stick or switch).

Now we claim that
\begin{equation}\label{N}
  \pr{\text{OK}} = \frac{6}{7},
\end{equation}
which implies (by the Law of Total Probability) that the probability
that Team 1 wins is indeed at least $(1/2)(6/7)=3/7$.

To prove $\pr{\text{OK}}=6/7$, we can follow the four step method.
(Note that we couldn't apply this method to model the behavior of Team
2, since we don't know how they may play, and so we can't let our
analysis depend on what they do.)

The first level of the probability tree for this game will describe
the value of $Z$: there are seven branches from the root with equal
probability going to first level nodes corresponding to the seven
possible values of $Z$.  The second level of the tree describes the
choice of the number, $N$: each of the seven first-level nodes has two
branches with equal probability, one branch for the case that $N=Z$
and the other for the case that $N=Z+1$.  So there are 14 outcome
(leaf) nodes at the second level of the tree, each with probability
1/14.

\begin{figure}[h]
\graphic[height=3.5in]{probtree}
\end{figure}

Now only two outcomes are not $\text{OK}$, namely, when $Z=6$ and $N=7$,
and when $Z=0$ and $N=0$.  Each of the other twelve outcomes is
$\text{OK}$, and since each has probability 1/14, we conclude that
$\pr{\text{OK}}=12/14=6/7$, as claimed.
\end{solution}

\end{problem}

%part of problem from fall 05 cp13w

%%%%%%%%%%%%%%%%%%%%%%%%%%%%%%%%%%%%%%%%%%%%%%%%%%%%%%%%%%%%%%%%%%%%%
% Problem ends here
%%%%%%%%%%%%%%%%%%%%%%%%%%%%%%%%%%%%%%%%%%%%%%%%%%%%%%%%%%%%%%%%%%%%%

\endinput
