\documentclass[problem]{mcs}

\begin{pcomments}
  \pcomment{TP_forest_edges}
  \pcomment{ARM 4/11/14}
\end{pcomments}

\pkeywords{
  trees
  graphs
  connected
  componrnt
  vertices
  egdes
}

%%%%%%%%%%%%%%%%%%%%%%%%%%%%%%%%%%%%%%%%%%%%%%%%%%%%%%%%%%%%%%%%%%%%%
% Problem starts here
%%%%%%%%%%%%%%%%%%%%%%%%%%%%%%%%%%%%%%%%%%%%%%%%%%%%%%%%%%%%%%%%%%%%%
\begin{problem}
Prove that if $G$ is a forest and
\begin{equation}\label{vG-ceG+1}
\card{\vertices{G}} = \card{\edges{G}} +1,
\end{equation}
then $G$ is a tree.

\begin{solution}
Each connected component of $G$ is a tree.  In a tree
\begin{equation}\label{vT-ceT+1}
\card{\vertices{T}} = \card{\edges{T}} -1.
\end{equation}
by Theorem~\bref{th:connectivity}.\ref{treeprops:v=e+1}.

Summing the left and right-hand sides of~\eqref{vT-ceT+1} over the components of $G$
implies that
\[
\card{\vertices{G}} = \card{\edges{G}} - \#\text{components of $G$},
\] 
so~\eqref{vG-ceG+1} implies that \#components must be 1, namely, the
forest is a tree.
\end{solution}

\end{problem}

%%%%%%%%%%%%%%%%%%%%%%%%%%%%%%%%%%%%%%%%%%%%%%%%%%%%%%%%%%%%%%%%%%%%%
% Problem ends here
%%%%%%%%%%%%%%%%%%%%%%%%%%%%%%%%%%%%%%%%%%%%%%%%%%%%%%%%%%%%%%%%%%%%%

\endinput
