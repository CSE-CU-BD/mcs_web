\documentclass[problem]{mcs}

\begin{pcomments}
  \pcomment{CP_birthday-deviation}
  \pcomment{new S10 revised from book}
\end{pcomments}

\pkeywords{
  deviation
  pairwise_independent
  birthday
}

%%%%%%%%%%%%%%%%%%%%%%%%%%%%%%%%%%%%%%%%%%%%%%%%%%%%%%%%%%%%%%%%%%%%%
% Problem starts here
%%%%%%%%%%%%%%%%%%%%%%%%%%%%%%%%%%%%%%%%%%%%%%%%%%%%%%%%%%%%%%%%%%%%%
\begin{problem}
\begin{staffnotes}
This problem comes pretty directly from
Section~\bref{bday_deviation_subsec}.  If a team gets stuck, point
them there.
\end{staffnotes}

Suppose there are $n$ students and $d$ days in the year, and let $D$
be the number of pairs of students with the same birthday.  Let
$B_1,B_2,\dots,B_n$ be the birthdays of $n$ independently chosen
people, and let $E_{i,j}$ be the indicator variable for the event
$[B_i = B_j]$.

\bparts

\ppart What are $\expect{E_{i,j}}$ and $\variance{E_{i,j}}$?

\begin{solution}
By Lemma~\bref{expindic}, $\expect{E_{i,j}} = \pr{B_i = B_j}$, which
equals $1/d$ for $i \neq j$.  By Lemma~\bref{bernoulli-variance}.
\[
\variance{E_{i,j}} = \pr{B_i = B_j}(1- \pr{B_i = B_j})
   = \frac{1}{d}\paren{1- \frac{1}{d}}.
\]
\end{solution}

\ppart What is $\expect{D}$?

\begin{staffnotes}
\hint $D \eqdef \sum_{1\le i < j \le n} E_{i,j}$
\end{staffnotes}

\begin{solution}

\begin{equation}\label{Vn}
D \eqdef \sum_{1\le i < j \le n} E_{i,j}.
\end{equation}
So by linearity of expectation
\[
\expect{D} = \expect{\sum_{1\le i < j \le n} E_{i,j}} =
               \sum_{1\le i < j \le n} \expect{E_{i,j}} =
               \binom{n}{2}\cdot \frac{1}{d}.
\]

\end{solution}

\ppart  What is $\variance{D}$?

\begin{solution}
The $E_{i,j}$ were observed to be pairwise independent in
Section~\bref{birthday_principle_sec}, so by Theorem~\bref{th:varsum},
their variances are additive:
\begin{align*}
\variance{D}
    & = \variance{\sum_{1\le i < j \le n} E_{i,j}}\\
    & = \sum_{1\le i < j \le n} \variance{E_{i,j}}\\
    & = \binom{n}{2} \cdot \frac{1}{d}\paren{1-\frac{1}{d}}.
\end{align*}

\end{solution}

\ppart In a 6.01 class of 500 students, the youngest student was born
in 1995 and the oldest in 1975.  Let $S$ be the number of students in
the class who were born on exactly the same day.  What is the
probability that $4 \leq S \leq 32$?  (For simplicity, assume that the
distribution of birthdays is uniform over the 7305 days in the two
decade interval from 1975 to 1995.)

\begin{solution}
For a class of $n= 500$ students with $d=7305$ possible days, we have
\[
\expect{S} = 500\cdot 499/2 \cdot \frac{1}{7305} \approx 17.1
\]
and
\[
\variance{S} = 500\cdot 499/2 \cdot \frac{1}{7305} \cdot
\paren{1-\frac{1}{7305}} < 17.1\, .
\]
So by Chebyshev's Theorem
\[
\pr{\abs{S - 17.1} \geq 14} < \frac{17.1}{14^2} < 0.09.
\]
We conclude that there is a better than 91\% chance that there will be
between 4 and 32 pairs of students born on the same day.
\end{solution}

\eparts

\end{problem}

%%%%%%%%%%%%%%%%%%%%%%%%%%%%%%%%%%%%%%%%%%%%%%%%%%%%%%%%%%%%%%%%%%%%%
% Problem ends here
%%%%%%%%%%%%%%%%%%%%%%%%%%%%%%%%%%%%%%%%%%%%%%%%%%%%%%%%%%%%%%%%%%%%%

\endinput
