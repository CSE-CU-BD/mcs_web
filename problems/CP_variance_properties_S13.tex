\documentclass[problem]{mcs}

\begin{pcomments}
  \pcomment{CP_variance_properties_S13}
  \pcomment{variant of CP_variance_properties}
  \pcomment{ARM May 19, 2013}
\end{pcomments}

\pkeywords{
  variance
  expectation
  random_variable
}

%%%%%%%%%%%%%%%%%%%%%%%%%%%%%%%%%%%%%%%%%%%%%%%%%%%%%%%%%%%%%%%%%%%%%
% Problem starts here
%%%%%%%%%%%%%%%%%%%%%%%%%%%%%%%%%%%%%%%%%%%%%%%%%%%%%%%%%%%%%%%%%%%%%

\begin{problem}
Let $R$ be a positive integer valued random variable.
\begin{problemparts}

\begin{staffnotes}
(a) 3 pts (b) 3 (c) 2
\end{staffnotes}

\problempart
If $\expect{R}=2$, how large can $\variance{R}$ be?

\begin{solution}
The variance of $R$ is unbounded.  For example, suppose $R_n$ is a
random variable that equals $n+1$ with probability $1/n$ and equals 1
otherwise.  So $\expect{R_n} = (n-1)/n + n+1/n = 2$ as required.  However,
\[
\variance{R_n} = \expect{R_n^2} - \expectsq{R_n}= \frac{n^2}{n} - 2 > n+2.
\]
So $\variance{R_n}$ grows unboundedly.

In fact, there are positive integer valued random variables with
expectation 2 and infinite variance (see
Problem~\bref{CP_infinite_variance}).
\end{solution}

\problempart How large can $\expect{1/R}$ be?

\begin{solution}

$\expect{1/R}$ is maximized at 1 when $R = 1$ with probability $1$.
Intuitively, this can be seen by trying to shift some of the
probability mass away from $1$.  No matter how you do it, you will
always wind up with an expectation that is less than $1$.

Formally, we can prove this as follows. Suppose $\prob{R=1} = 1-\delta$
for some $\delta > 0$. Then
\begin{align*}
\expect{R}
& = 1\cdot(1-\delta) + \sum_{i=2}^{\infty}\frac{1}{n}\cdot \prob{R=n}\\
& \leq 1\cdot(1-\delta) + \sum_{i=2}^{\infty}\frac{1}{2}\cdot \prob{R=n}\\
& \leq 1\cdot(1-\delta) + \frac{1}{2}\sum_{i=2}^{\infty}\prob{R=n}\\
& \leq 1\cdot(1-\delta) + \frac{1}{2}\cdot\delta\\
& \leq 1-\delta\cdot\frac{3}{2}.
\end{align*}

So, for any positive $\delta$, $\expect{R}$ is less than 1.
Therefore, the given distribution maximizes $\expect{1/R}$.
\end{solution}

\ppart If $R \leq 2$, that is $R$ takes only the values 1 and 2, how
large can $\variance{R}$ be?

\begin{solution}
\[
\frac{1}{4}
\]

We know $\variance{R} = \variance{R-1}$ and $R-1$ is a Bernoulli
variable.  If $\pr{R=1} = p$, then we know the variance is $p(1-p)$
which is maximum for $p=1/2$ (see Problem~\bref{CP_gallup_poll}).
\end{solution}

\end{problemparts}
\end{problem}

\endinput
