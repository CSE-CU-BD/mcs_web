\documentclass[problem]{mcs}

\begin{pcomments}
  \pcomment{CP_variance_properties_S13}
  \pcomment{variant of CP_variance_properties}
  \pcomment{ARM May 19, 2013}
  \pcomment{updated 5/8/14}
\end{pcomments}

\pkeywords{
  variance
  expectation
  random_variable
}

%%%%%%%%%%%%%%%%%%%%%%%%%%%%%%%%%%%%%%%%%%%%%%%%%%%%%%%%%%%%%%%%%%%%%
% Problem starts here
%%%%%%%%%%%%%%%%%%%%%%%%%%%%%%%%%%%%%%%%%%%%%%%%%%%%%%%%%%%%%%%%%%%%%

\begin{problem}
Let $R$ be a positive integer-valued random variable.

\begin{problemparts}

\problempart How large can $\expect{1/R}$ be?

\examspace[1.5in]

\begin{solution}
The maximum value of $\expect{1/R}$ is 1.

Since $R \ge 1$, we have $1/R \le 1$, so $\expect{1/R} \leq 1$.
The maximum is achieved when $R$ is identically 1.

\iffalse
Intuitively, this can be seen by trying to shift some of the
probability mass away from $1$.  No matter how you do it, you will
always wind up with an expectation that is less than $1$.

Formally, we can prove this as follows. Suppose $\prob{R=1} = 1-\delta$
for some $\delta > 0$. Then
\begin{align*}
\expect{R}
& = 1\cdot(1-\delta) + \sum_{i=2}^{\infty}\frac{1}{n}\cdot \prob{R=n}\\
& \leq 1\cdot(1-\delta) + \sum_{i=2}^{\infty}\frac{1}{2}\cdot \prob{R=n}\\
& \leq 1\cdot(1-\delta) + \frac{1}{2}\sum_{i=2}^{\infty}\prob{R=n}\\
& \leq 1\cdot(1-\delta) + \frac{1}{2}\cdot\delta\\
& \leq 1-\delta\cdot\frac{3}{2}.
\end{align*}

So, for any positive $\delta$, $\expect{R}$ is less than 1.
Therefore, the given distribution maximizes $\expect{1/R}$.
\fi

\end{solution}

\ppart How large can $\variance{R}$ be if the only values of $R$ are 1
and 2?

\examspace[1in]

\begin{solution}
We know $\variance{R} = \variance{R-1}$ and $R-1$ is a Bernoulli
variable.  If $\pr{R=2} = \pr{R-1 = 1} = p$, then the variance of
$R-1$ is $p(1-p)$, which is maximized when $p=1/2$ (see
Problem~\bref{CP_gallup_poll}).  Thus, $\variance{R}$ can be as large
as 1/4.
\end{solution}

\problempart
How large can $\variance{R}$ be if $\expect{R}=2$?

\examspace[1.5in]

\begin{solution}
The variance of $R$ can be arbitrarily large.  For example, suppose
$R_n$ is a random variable that equals $n+1$ with probability $1/n$
and equals 1 otherwise.  So $\expect{R_n} = (n+1)/n + (n-1)/n = 2$ as
required.  Also,
\[
\expect{R_n^2} = (n+1)^2/n + 1^2(n-1)/n  = (n^2+3n)/n = n + 3.
\]
So,
\[
\variance{R_n} = \expect{R_n^2} - \expectsq{R_n}= (n+3) - 4 = n-1,
\]
and therefore $\variance{R_n}$ grows unboundedly with $n$.

In fact, there are positive integer valued random variables with
expectation 2 and infinite variance (see
Problem~\bref{CP_infinite_variance}).
\end{solution}

\end{problemparts}
\end{problem}

\endinput
