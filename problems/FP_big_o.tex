\documentclass[problem]{mcs}

\begin{pcomments}
  \pcomment{FP_asymptotics}
  \pcomment{Peter}
\end{pcomments}

\pkeywords{
  asymptotics
  big_oh
}

%%%%%%%%%%%%%%%%%%%%%%%%%%%%%%%%%%%%%%%%%%%%%%%%%%%%%%%%%%%%%%%%%%%%%
% Problem starts here
%%%%%%%%%%%%%%%%%%%%%%%%%%%%%%%%%%%%%%%%%%%%%%%%%%%%%%%%%%%%%%%%%%%%%

\begin{problem}
Recall that for functions $f,g$ on
$\naturals$, $f = O(g)$ iff
\begin{equation}\label{Oh}
\exists c \in \naturals\, \exists n_0 \in \naturals\,
\forall n \geq n_0\quad c \cdot g(n) \geq \abs{f(n)}.
\end{equation}

For each pair of functions below, determine whether $f = O(g)$ and whether
$g = O(f)$.  In cases where one function is O() of the other, indicate the
\emph{smallest nonnegative integer}, $c$, and for that smallest $c$, the
\emph{smallest corresponding nonnegative integer $n_0$} ensuring that
condition~\eqref{Oh} applies.

\begin{problemparts}

\problempart $f(n) = n^2, g(n) = 4n$.

$f = O(g)$ \hspace{.5in}YES \hspace{.5in}NO \hspace{.5in}
If YES, $c =$ \brule{.5in}, $n_0$ = \brule{.5in}

\begin{solution}
NO.
\end{solution}

$g = O(f)$ \hspace{.5in}YES \hspace{.5in}NO \hspace{.5in}
If YES, $c =$ \brule{.5in}, $n_0$ = \brule{.5in}

\begin{solution}
YES, with $c = 1$, $n_0 = 4$, which works
because $4^2 = 16$, $4 \cdot 4 = 16$.
\end{solution}

\problempart $f(n) = (5n + 6) / (2n - 3), g(n) = 3$

$f = O(g)$ \hspace{.5in}YES \hspace{.5in}NO \hspace{.5in}
If YES, $c =$ \brule{.5in}, $n_0$ = \brule{.5in}

\begin{solution}
YES, with $c = 2, n_0 = 0$

Since $\lim_{n \to \infty} f(n) = 2.5$, the smallest possible $c$ is 2.
\end{solution}

$g = O(f)$ \hspace{.5in}YES \hspace{.5in}NO \hspace{.5in}
If YES, $c =$ \brule{.5in}, $n_0$ = \brule{.5in}

\begin{solution}
YES, with $c =1, n_0 = 15.$


For $c = 1$, the smallest possible $n_0 = 15$ which follows from the
requirement that $f(n_0) \ge 3$.
\end{solution}

\problempart $f(n) = (n \cos(n\pi/2))^2, g(n) = 5n$

$f = O(g)$ \hspace{.5in}YES \hspace{.5in}NO \hspace{.5in}
If yes, $c =$ \brule{.5in} $n_0$ = \brule{.5in}

\begin{solution}
NO, because $f(2n) = (2n)^2 \neq O(n)$ which rules out
$f = O(g)$.
\end{solution}

$g = O(f)$ \hspace{.5in}YES \hspace{.5in}NO \hspace{.5in}
If yes, $c =$ \brule{.5in} $n_0$ = \brule{.5in}

\begin{solution}
NO, because $f(2n+1)=0$, which rules out $g =
O(f)$ since $g=\Theta(n)$.
\end{solution}

\end{problemparts}

\end{problem}

%%%%%%%%%%%%%%%%%%%%%%%%%%%%%%%%%%%%%%%%%%%%%%%%%%%%%%%%%%%%%%%%%%%%%
% Problem ends here
%%%%%%%%%%%%%%%%%%%%%%%%%%%%%%%%%%%%%%%%%%%%%%%%%%%%%%%%%%%%%%%%%%%%%

\endinput
