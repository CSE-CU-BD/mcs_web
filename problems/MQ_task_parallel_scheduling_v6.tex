\documentclass[problem]{mcs}

\begin{pcomments}
  \pcomment{MQ_task_parallel_scheduling_v6}
  \pcomment{variant of _v3}
  \pcomment{F17.final-conflict3}
\end{pcomments}

\pkeywords{
  DAG
  scheduling
  chains_and_antichains
}

%%%%%%%%%%%%%%%%%%%%%%%%%%%%%%%%%%%%%%%%%%%%%%%%%%%%%%%%%%%%%%%%%%%%%
% Problem starts here
%%%%%%%%%%%%%%%%%%%%%%%%%%%%%%%%%%%%%%%%%%%%%%%%%%%%%%%%%%%%%%%%%%%%%

\begin{problem}
The following DAG describes the prerequisites among tasks $\set{1, \dots, 10}$.

\begin{figure}[h]
\graphic[height=2.0in]{dag-v6}
\end{figure}

\bparts

\ppart If each task takes unit time to complete, what is the minimum
parallel time to complete all the tasks?  Briefly explain.

\exambox{0.7in}{0.4in}{0in}

\begin{solution}
\textbf{5}.  This is the size of the maximum chain.
\end{solution}

\examspace[2cm]

\ppart What is the minimum parallel time if no more than two tasks can
be completed in parallel?  Briefly explain.

\begin{solution}
  \textbf{6 days}. Tasks 8, 6, and 5 must be completed before initiating task 4, so this takes at least $\lceil 3/2\rceil = 2$ days. Task 4 must occur on its own day, since every remaining task depends on task 4. Finally, the remaining six tasks require at least $6/2 = 3$ more days, so $6$ days is a lower bound. It is not hard to find a 2-processor schedule that finishes in 6 time units, such as $8\,5,6,4,1\,2,9\,3,10\,7$.
\end{solution}

\exambox{0.7in}{0.4in}{0in}
\examspace[2cm]

\eparts

\end{problem}

%%%%%%%%%%%%%%%%%%%%%%%%%%%%%%%%%%%%%%%%%%%%%%%%%%%%%%%%%%%%%%%%%%%%%
% Problem ends here
%%%%%%%%%%%%%%%%%%%%%%%%%%%%%%%%%%%%%%%%%%%%%%%%%%%%%%%%%%%%%%%%%%%%%

\endinput
