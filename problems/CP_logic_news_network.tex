\documentclass[problem]{mcs}

\begin{pcomments}
  \pcomment{from: F09.ps2}
  \pcomment{from: S09.cp2t, F02.cp2w}
%  \pcomment{}
%  \pcomment{}
\end{pcomments}

\pkeywords{
  quantifiers
  domain_of_discourse
  predicate_calculus
  translating_english_statements
}

%%%%%%%%%%%%%%%%%%%%%%%%%%%%%%%%%%%%%%%%%%%%%%%%%%%%%%%%%%%%%%%%%%%%%
% Problem starts here
%%%%%%%%%%%%%%%%%%%%%%%%%%%%%%%%%%%%%%%%%%%%%%%%%%%%%%%%%%%%%%%%%%%%%

\begin{problem}
A media tycoon has an idea for an all-news television network called
LNN: The Logic News Network.  Each segment will begin with a
definition of the domain of discourse and a few predicates.  The day's
happenings can then be communicated concisely in logic notation.  For
example, a broadcast might begin as follows:

\begin{quotation}\noindent
``THIS IS LNN.  The domain of discourse is $\{
\mbox{Albert}, \mbox{Ben}, \mbox{Claire}, \mbox{David}, \mbox{Emily}
\}$.  Let $D(x)$ be a predicate that is true if $x$ is deceitful.  Let
$L(x, y)$ be a predicate that is true if $x$ likes $y$.  Let $G(x, y)$
be a predicate that is true if $x$ gave gifts to $y$.''
\end{quotation}

Translate the following broadcasted logic notation into (English) statements.

\bparts

\ppart
\[
(\neg (D(\mbox{Ben}) \vee D(\mbox{David}))) \implies
(L(\mbox{Albert}, \mbox{Ben}) \wedge L(\mbox{Ben}, \mbox{Albert}))
\]

\begin{solution}
If neither Ben nor David is deceitful, then Albert and Ben like each other.
\end{solution}

\ppart
\begin{eqnarray*}
\forall x\ (x  =   \mbox{Claire} \wedge \neg L(x, \mbox{Emily})) \vee
           (x \neq \mbox{Claire} \wedge      L(x, \mbox{Emily})) \wedge \\
\quad \quad
\forall x\ (x  =   \mbox{David} \wedge      L(x, \mbox{Claire})) \vee
           (x \neq \mbox{David} \wedge \neg L(x, \mbox{Claire}))
\end{eqnarray*}

\begin{solution}
Everyone except for Claire likes Emily, and no one except David likes Claire.
\end{solution}

\ppart
\[
\neg D(\mbox{Claire}) \implies
        (G(\mbox{Albert}, \mbox{Ben}) \wedge
        \exists\ x G(\mbox{Ben}, x))
\]

\begin{solution}
If Claire is not deceitful, then Albert gave gifts to
Ben, and Ben gave gifts to someone.
\end{solution}

\ppart
\[
\forall x \exists y \exists z\ (y \neq z) \wedge L(x, y) \wedge \neg L(x, z)
\]

\begin{solution}
Everyone likes someone and dislikes someone else.
\end{solution}

\ppart
How could you express ``Everyone except for Claire likes
Emily'' using just propositional connectives \emph{without} using any
quantifiers ($\forall, \exists$)?  Can you generalize to explain how
\emph{any} logical formula over this domain of discourse can be expressed
without quantifiers?  How big would the formula in the previous part be if
it was expressed this way?

\begin{solution}
\[
L(\mbox{Albert},\mbox{Emily}) \land L(\mbox{Ben},\mbox{Emily}) \land
L(\mbox{David},\mbox{Emily}) \land L(\mbox{Emily},\mbox{Emily}) \land
\neg L(\mbox{Claire},\mbox{Emily})
\]

In general, quantifiers can be eliminated by treating $\forall x\ P(x)$ as an
abbreviation for
\[
P(\mbox{Albert}) \land P(\mbox{Ben}) \land P(\mbox{Claire}) \land P(\mbox{David}) \land
P(\mbox{Emily}),
\]
and $\exists x\ P(x)$ as an abbreviation for
\[
P(\mbox{Albert}) \lor P(\mbox{Ben}) \lor P(\mbox{Claire}) \lor
P(\mbox{David}) \lor P(\mbox{Emily}).
\]

Expanded this way, the three-quantifier formula of the previous part would
expand by a factor of $5 \times 5 \times 5 = 125$.  So using quantifiers
can pay off even when they are not strictly necessary.
\end{solution}

\eparts

\end{problem}

%%%%%%%%%%%%%%%%%%%%%%%%%%%%%%%%%%%%%%%%%%%%%%%%%%%%%%%%%%%%%%%%%%%%%
% Problem ends here
%%%%%%%%%%%%%%%%%%%%%%%%%%%%%%%%%%%%%%%%%%%%%%%%%%%%%%%%%%%%%%%%%%%%%

\endinput
