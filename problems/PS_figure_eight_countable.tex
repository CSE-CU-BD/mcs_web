\documentclass[problem]{mcs}

\begin{pcomments}
  \pcomment{PS_figure_eight_countable}
  \pcomment{zabel 10/10/17}
\end{pcomments}

\pkeywords{
  sets
  countable
  rationals
  figure_eight
}

\newcommand\Circle{\operatorname{Circ}}
\newcommand\FE{\operatorname{FigEight}}
\newcommand\bR{\reals}
\newcommand\bQ{\rationals}

\begin{problem}
  For a point $p\in\bR^2$ in the plane\footnote{$\bR^2$ is shorthand for $\bR\times\bR$} and a positive real number $r > 0$, let $\Circle(p,r)$ be (the border of) the circle with center $p$ and radius $r$, i.e., $\Circle(p,r)\eqdef \{v\in\bR^2 \mid |v - p| = r\}$. Two circles $C,C' \subset \bR^2$ \emph{intersect} when they have a point (or two) in common, i.e., $C\cap C' \ne \emptyset$; they are \emph{disjoint} otherwise. For example, two tangent circles intersect, but two circles with one entirely enclosed within the other are disjoint.

  \bparts
  \ppart Show that it is possible to find \emph{uncountably infinitely} many circles in the plane where no two of these circles intersect each other.
  
  \begin{solution}
    We choose the set of circles $A = \{\Circle((0,0),r) | r > 0\}$. Different $r$ values give rise to different circles, so $A$ is uncountably infinite because the interval $(0,\infty)\subset\bR$ is uncountably infinite XXX WHY!!!. Any point $v\in\Circle((0,0),r)$ has $|v| = r$, so it cannot belong to $\Circle((0,0),r')$ when $r \ne r'$. Thus, no two of $A$'s circles intersect each other.
  \end{solution}

  \ppart
  If $p\ne q$ are two distinct points in the plane, define the \term{figure eight} shape $\FE(p,q)$ as the union of two equally-sized tangent circles centered at $p$ and $q$:
  \begin{equation*}
    \FE(p,q) \eqdef \Circle(p,r) \cup \Circle(q,r),
  \end{equation*}
  where $r = |p-q|/2$.
  In the remainder of this problem we'll answer the following question: can uncountably infinitely many disjoint figure eights fit in the plane, just like for circles? We'll show the answer is ``NO, only countably many''.


  \ppart \label{partRationalMark}
  To start out, let $E = \FE(p,q)$ be a figure eight as above. If $v$ and $w$ are two points in the plane, we say that $(v,w)$ is a \term{mark} for $E$ if $E$'s two circles enclose $v$ and $w$, respectively: $|v - p| < r \QAND |w - q| < r$. Show that it is possible to find a mark $(v,w)$ for $E$ where $v$'s and $w$'s coordinates are \emph{rational} numbers. \emph{Hint}: Any disk contains a square.

  \begin{solution}
    The square with side-length $r$ and center $p = (x,y)$ is enclosed by $\Circle(p,r)$, as may easily be verified geometrically. We may find a rational number $a\in [x-r/2, x+r/2]$ and a rational number $b \in[y - r/2, y+r/2]$, so that rational point $v=(a,b)$ lies in this square and is thus enclosed within $\Circle(p,r)$. A rational point $w$ enclosed by $\Circle(q,r)$ may similarly be found, making $(v,w)$ a mark for $E$ with rational coordinates.
  \end{solution}

  \ppart \label{partNoMarkDuplicates}
  Suppose $E = \FE(p,q)$ and $E' = \FE(p',q')$ are disjoint figure eights. Briefly argue that no $(v,w)$ can be a mark for both $E$ and $E'$. \emph{Hint}: There are two cases to consider: one of $E$ or $E'$ encloses the other, or neither does.

  \begin{solution}
    Suppose $(v,w)$ marked both $E$ and $E'$. We consider two cases based on how $E$ and $E'$ are drawn relative to each other: because $E$ and $E'$ are disjoint, either it holds that one of $E$ or $E'$ is enclosed within a circle of the other, or neither one is enclosed. In the first case, we may assume without loss of generality that $E'$ is enclosed by $\Circle(p,r)$, where $r = |p-q|/2$. Because $(v,w)$ marks $E'$, it follows that $v$ and $w$ are enclosed by $E'$ and thus by $\Circle(p,r)$. But then $(v,w)$ cannot mark $E$, a contradiction. In the other case, $v$ and $w$ are enclosed by $E$ because $(v,w)$ marks $E$, but this means they cannot be enclosed by $E'$, contradicting the fact that $(v,w)$ marks $E'$. In either case, the assumption that $(v,w)$ marks both $E$ and $E'$ cannot hold.
  \end{solution}
  
  \ppart Assume $A$ is any set of mutually disjoint figure eights. We can define a total function $f: A \to (\bQ^2)^2$ by picking a rational mark $f(E)$ for each figure eight $E\in A$, as in Part~\ref{partRationalMark}. Prove that $f$ is injective, and use this to prove that $A$ is countable. You may use the fact that if $Y$ and $Z$ are countable sets, then so is $Y\times Z$ (Problem~\ref{} ???? SOMETHING!).

  \begin{solution}
    It follows directly from Part~\ref{partNoMarkDuplicates} that $f$ is injective: any two figure eights $E$ and $E'$ in $A$ are disjoint, so the marks $f(E)$ and $f(E')$ for them cannot be equal. The inverse relation $f^{-1}$ thus proves that $(\bQ^2)^2 \surj A$. To conclude that $A$ is countable, it suffices to show that $(\bQ^2)^2$ itself is countable. But this is true by two applications of Problem~\ref{} ???? SOMETHING!: we know $\bQ$ is countable, so $\bQ\times \bQ = \bQ^2$ is countable, and thus $(\bQ^2)^2$ is countable.
  \end{solution}
  
  \eparts
\end{problem}

\endinput
