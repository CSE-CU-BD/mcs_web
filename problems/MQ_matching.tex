\documentclass[problem]{mcs}

\begin{pcomments}
  \pcomment{MQ_matching}
\end{pcomments}

\pkeywords{
  bipartite
  matching
  bottleneck
}

%%%%%%%%%%%%%%%%%%%%%%%%%%%%%%%%%%%%%%%%%%%%%%%%%%%%%%%%%%%%%%%%%%%%%
% Problem starts here
%%%%%%%%%%%%%%%%%%%%%%%%%%%%%%%%%%%%%%%%%%%%%%%%%%%%%%%%%%%%%%%%%%%%%

\begin{problem}
% 

\begin{problemparts}

\problempart Consider the bipartite graph $G$ in
Figure~\ref{fig:bipartite_with_bneck}.  Is it possible to find a
matching that covers $L(G)$?  If yes, explain what property of the
graph guarantees the existence of such a matching.  (Show that the
graph exhibits this property and what this implies.  Full credit will
not be given for merely identifying a matching.)  If no, identify a
bottleneck that prevents a matching.

\begin{figure}[h]
\graphic{MQ_bipartite_with_bottleneck}
\caption{$G$.\label{fig:bipartite_with_bneck}}
\end{figure}


\begin{solution}
It is not possible.  One bottleneck is $S=\{a,b,c,e\}$, since $N(S)=\{v,x,y\}$ and hence $|S|=4>3=|N(S)|$.  (It is easy to see that there
are no bottlenecks of size 1, 2, 3, or 5.)
\end{solution}

\examspace

\problempart
Consider the bipartite graph $H$ in Figure~\ref{fig:deg_constrained_bipartite}.  Is it possible to find a matching that covers $L(H)$?
If yes, explain what property of the graph guarantees the existence of such a matching.  (Show that the graph exhibits this property and what this implies.
Full credit will not be given for merely identifying a matching.)  If no, identify a bottleneck that prevents a matching.

\begin{figure}[h]
\graphic{MQ_degree_constrained_bipartite}
\caption{$H$.\label{fig:deg_constrained_bipartite}}
\end{figure}

\examspace[1in]
\begin{solution}
A matching is guaranteed to exist.  Each vertex in $L(H)$ has degree at least 3, while each vertex in $R(H)$ has degree at most 3.  Consequently,
the graph is degree-constrained.  There are therefore no bottlenecks and a matching must exist by Hall's Theorem. 
\end{solution}

\end{problemparts}
\end{problem}

\endinput
