\documentclass[problem]{mcs}

\begin{pcomments}
  \pcomment{PS_brents_theorem}
  \pcomment{from: F06 pset5Back.tex, maybe same as F04 pset5}
  \pcomment{major revision by ARM 10/3/09}
  \pcomment{need cleaner proof for last part}
  \pcomment{S11 email claims it is buggy?}
\end{pcomments}

\pkeywords{
  partial_orders
  chains
  antichain
  Brent
}

%%%%%%%%%%%%%%%%%%%%%%%%%%%%%%%%%%%%%%%%%%%%%%%%%%%%%%%%%%%%%%%%%%%%%
% Problem starts here
%%%%%%%%%%%%%%%%%%%%%%%%%%%%%%%%%%%%%%%%%%%%%%%%%%%%%%%%%%%%%%%%%%%%%

\begin{problem}
We want to schedule $n$ tasks with prerequisite constraints among the
tasks defined by a DAG.

\bparts

\ppart Explain why any schedule that requires only $p$ processors must
take time at least $\ceil{n/p}$.

\begin{solution}
At any given time step in a schedule, at most $p$ tasks can be completed, so the
maximum number of tasks that can be completed in $t$ parallel steps is
$tp$.  So to complete all the tasks, we require that $tp \geq n$, which
implies that $t \geq \ceil{n/p}$.
\end{solution}

\ppart\label{timeD} Let $D_{n,t}$ be the DAG with $n$
elements that consists of a chain of $t-1$ elements, with the bottom
element in the chain being a prerequisite of all the remaining elements as
in the following figure:

\begin{figure}[h]
\graphic[height=2.5in]{pset5-hasse}
\end{figure}

What is the minimum time schedule for $D_{n,t}$?  Explain why it is
unique.  How many processors does it require?

\begin{solution}
There is no choice but to schedule each of the $t-1$ vertices on
the path one at a time in order.  A minimum time schedule then does all
the remaining $n-(t-1)$ vertices at the $t$th time interval. Therefore, the number
of processors required is $n-t+1$.  
\end{solution}

\ppart Write a simple formula $M(n,t,p)$ for the minimum time of a
$p$-processor schedule to complete $D_{n,t}$.

\begin{solution}
As in part~\eqref{timeD}, there is no choice but to schedule each
of the $t-1$ vertices on the path one at a time in order.  A minimum time
schedule then does all the remaining $n-(t-1)$ vertices $p$ at a time, for
a total time of
\begin{equation}\label{tnp}
M(n,t,p) \eqdef (t-1) + \ceil{\frac{n-(t-1)}{p}}.
\end{equation}

\end{solution}

\ppart\label{Mntp-possible} Show that \emph{every} partial order with
$n$ vertices and maximum chain size $t$ has a $p$-processor schedule
that runs in time $M(n,t,p)$.

\hint Use induction on $t$.

\begin{staffnotes}
\textbf{There must be a cleaner proof.}
\end{staffnotes}

\begin{solution}

\begin{proof}
Induction on $t$ with induction hypothesis that the statement of this
problem part~\eqref{Mntp-possible} holds for all positive integers,
$n,p$.

\textbf{Base case} ($t=1$).  In this case there are $n$ vertices and no
edges between them.  So any partition of the vertices into $\ceil{n/p}$
blocks of size at most $p$ will be a $p$-processor schedule taking time
$\ceil{n/p} = 0 + \ceil{(n-0)/p} = M(n,1,p)$.

\textbf{Inductive step:}
Assume $P(t)$ and conclude $P(t+1)$ where $t \geq 1$.

Let $G$ be any partial order with $n$ elements and maximum chain size
$t+1$.  Suppose $k$ elements are endpoints of maximum-size chains in
$G$.  These elements must be maximal, for otherwise there would be a
chain of length one more than the maximum chain size.  Let $H$ be the
subset of $G$ obtained by removing these $k$ vertices.

Now $H$ is a partial order with $n-k$ elements and maximum chain size
$t$, so by Induction Hypothesis, there is a $p$-processor schedule for
$H$ taking time $M(n-k,t,p)$.

This $p$-processor schedule for $H$ can be extended to one for $G$ by
adding $\ceil{k/p}$ disjoint blocks of the endpoints, all of size $\leq p$.
So the time for this schedule for $G$ is
\begin{align}
\lefteqn{M(n-k,t,p) + \ceil{\frac{k}{p}}}\notag\\
  & = (t-1) + \ceil{\frac{n-k-(t-1)}{p}} + \ceil{\frac{k}{p}} & \text{(def of $M$)}\notag\\
  & = (t-1) + \ceil{\frac{n-t}{p} - \frac{k-1}{p}} + \ceil{\frac{k}{p}}\label{t1}
\end{align}

We complete the proof by showing that the expression~\eqref{t1} is $\leq
M(n, t+1, p)$.  To do this, we consider two cases:

\begin{itemize}

\item \textbf{Case 1:} ($k-1$ is not a multiple of $p$):
We have
\begin{equation}\label{k1}
\ceil{\frac{k-1}{p}} = \ceil{\frac{k}{p}},
\end{equation}
so
\begin{align*}
\text{\eqref{t1}}
   & \leq (t-1) + \paren{1+ \ceil{\frac{n-t}{p}} - \ceil{\frac{k-1}{p}}}
        + \ceil{\frac{k}{p}}
            & \text{(by~\eqref{cab})}\\
   & = (t-1) + \paren{1+ \ceil{\frac{n-t}{p}} - \ceil{\frac{k}{p}}}
       + \ceil{\frac{k}{p}}
            & \text{(by~\eqref{k1})}\\
   & = t + \ceil{\frac{n-t}{p}}\\
   & = M(n, t+1, p)
            & \text{(def of $M$).}
\end{align*}

Here the first inequality used the fact that
\begin{equation}\label{cab}
\ceil{a-b} \leq 1 + \ceil{a} - \ceil{b}
\end{equation}
for all real numbers $a,b$.

\item \textbf{Case 2:} ($k-1$ is a multiple of $p$): 
Now we have
\begin{equation}
\ceil{\frac{k}{p}}=1+\frac{k-1}{p},\label{k+}
\end{equation}
so
\begin{align*}
\text{\eqref{t1}}
  & = (t-1) + \left(\ceil{\frac{n-t}{p}} - \frac{k-1}{p}\right) +
       \ceil{\frac{k}{p}} & \text{(since $(k-1)/p \in \integers$)}\\
  & = (t-1) + \ceil{\frac{n-t}{p}} - \frac{k-1}{p} +
  \left(1+\frac{k-1}{p}\right) & \text{(by~\eqref{k+})}\\
   & = t + \ceil{\frac{n-t}{p}}\\
   & = M(n, t+1, p). & \text{(def of $M$)}
\end{align*}

\end{itemize}

\end{proof}

\end{solution}

\eparts

\end{problem}

%%%%%%%%%%%%%%%%%%%%%%%%%%%%%%%%%%%%%%%%%%%%%%%%%%%%%%%%%%%%%%%%%%%%%
% Problem ends here
%%%%%%%%%%%%%%%%%%%%%%%%%%%%%%%%%%%%%%%%%%%%%%%%%%%%%%%%%%%%%%%%%%%%%

\endinput
