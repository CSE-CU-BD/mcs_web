\documentclass[problem]{mcs}

\begin{pcomments}
  \pcomment{PS_connected_induction}
  \pcomment{clumsy induction: DON'T USE}
  \pcomment{F12.rec6; commented out of F12 final}
  \pcomment{revised ARM 4/8/14}
\end{pcomments}

\pkeywords{
  graph
  connected
  component
  induction
}

%%%%%%%%%%%%%%%%%%%%%%%%%%%%%%%%%%%%%%%%%%%%%%%%%%%%%%%%%%%%%%%%%%%%%
% Problem starts here
%%%%%%%%%%%%%%%%%%%%%%%%%%%%%%%%%%%%%%%%%%%%%%%%%%%%%%%%%%%%%%%%%%%%%

\begin{problem}
A simple graph $G$ is \emph{2-removable} iff it contains two vertices
$v \neq w$ such that $G-v$ is connected, and $G-w$ is also connected.

Prove by induction on the number of vertices that every connected
graph with at least two vertices is 2-removable.

\begin{staffnotes}
The induction proof below is pretty cumbersome; some figures would
make it easier to follow.

Actually, induction is not a good approach compared to the trivial
proof: choose the endpoints of a maximum length path in the graph.

Another simple proof without induction: if the graph has a positive
length cycle, then any one of the vertices on the cycle can be removed
w/o disconnecting the graph.  Otherwise it is a tree with at least two
vertices, so it must have at least 2 leaves
(Theorem~\bref{th:treeprops}), and any leaf can be removed.
\end{staffnotes}

\begin{solution}

\begin{proof} By strong induction on the number $n$ of vertices in graphs.

\inductioncase{Inductive Hypothesis:} $P(n) \eqdef$ Every connected
graph $G$ with $n$ vertices is 2-removable.

\inductioncase{Base Case:} $(n=2)$: Both vertices of $G$ satisfy the
condition, as a subgraph of one node is trivially connected.

\inductioncase{Inductive Step}: Let $G$ be a connected graph with
$n+1$ vertices for some $n\geq 2$.  We will show that $G$ is
2-removable.

\inductioncase{Case I:} If $G$ remains connected if any single vertex
is removed, then since $G$ has at least 2 vertices, it is 2-removable.
(We did not need to invoke the induction hypothesis for this case.)

\inductioncase{Case II:} Suppose that $G-v$ is not connected for some
vertex $v$.  Then $G-v$ must have at least two connected components,
each with fewer than $n$ vertices and with at least one vertex
adjacent to $v$. (The reader should take a moment to consider this
claim.)

Let $H$ be one of the components.  Now since $H$ is connected, every
vertex in $H$ must be connected (in $H$) to a vertex adjacent to $v$,
and likewise every vertex of $G-H$ is also connected (in $G-H$) to
$v$.

Let $H + v$ be the graph obtained by adding vertex $v$ to $H$ along
with those edges from $G$ that connect $v$ to adjacent vertices in
$H$.  Then $H + v$ is a connected graph with at least two and at most
$n$ vertices, so by strong induction, $H+v$ is 2-removable.  That is,
there is a vertex $w_H \neq v$ of $H+v$ such that $(H+v)-w_H$ is
connected.  This implies that $G-w_H$ is connected, since there is a
path between any vertex in $(H+v)-w_H$ to $v$ and then a path from $v$
to any vertex in $G-H$.  Since there is such a $w_H$ for each
connected component $H$, if follows that $G$ is 2-removable.
\end{proof}
\end{solution}

\end{problem}


%%%%%%%%%%%%%%%%%%%%%%%%%%%%%%%%%%%%%%%%%%%%%%%%%%%%%%%%%%%%%%%%%%%%%
% Problem ends here
%%%%%%%%%%%%%%%%%%%%%%%%%%%%%%%%%%%%%%%%%%%%%%%%%%%%%%%%%%%%%%%%%%%%%

\endinput

