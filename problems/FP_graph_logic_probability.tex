\documentclass[problem]{mcs}

\begin{pcomments}
  \pcomment{FP_graph_logic_probability}
  \pcomment{S08.final, F15.cp12m}
  \pcomment{rewritten ARM 12/01/15}
\end{pcomments}

\pkeywords{
  graph
  logic
  probability
  independence
}

%%%%%%%%%%%%%%%%%%%%%%%%%%%%%%%%%%%%%%%%%%%%%%%%%%%%%%%%%%%%%%%%%%%%%
% Problem starts here
%%%%%%%%%%%%%%%%%%%%%%%%%%%%%%%%%%%%%%%%%%%%%%%%%%%%%%%%%%%%%%%%%%%%%

\begin{problem} \inhandout{\textbf{Graphs, Logic \& Probability}}
  Let $G$ be a simple graph.  Let $E(x,y)$ mean that $G$ has an edge
  between vertices $x$ and $y$.  Since $G$ is simple, we have $E(x,y)
  \QIFF\ E(y,x)$ and also $\QNOT(E(x,x))$.

  Let $W(x,y)$ mean that there is a length-two walk in $G$ between $x$
  and $y$.

\bparts

\ppart Explain why $E(x,y)$ implies $W(x,x)$.

\begin{solution}
Going back and forth on edge $\edge{x}{y}$ is a length-two walk from
$x$ to $x$.
\end{solution}


\ppart Write a predicate-logic formula defining $W(x,y)$ in terms of $E(u,v)$.

\begin{solution}
\[
W(x,y) \eqdef \exists z.\, E(x,z) \QAND\ E(z,y).
\iffalse \QAND z \neq x \QAND\ z \neq y \QAND \fi
\]
\end{solution}

\eparts
\medskip

Now let $V$ be a fixed set of vertices and $G$ be a graph with
$\vertices{G} = V$ constructed randomly as follows: for every two
distinct vertices $u,v \in V$, independently include edge
$\edge{u}{v}$ in $\edges{G}$ with probability $p$.

So $\pr{E(u,v)} = p$, and $E(u,v)$ is independent of $E(r,s)$ when
$\edge{u}{v} \neq \edge{r}{s}$.

\begin{staffnotes}
Formally, our sample space is the set of simple graphs $G$ with
$\vertices{G} = V$ and $\pr{G} = p^k(1-p)^m$, where $k =
\card{\edges{G}}$ and $m = \binom{n}{2} - k$.

An event $E$ \emph{depends on an edge $\edge{u}{v}$} iff $E$ and
$E(u,v)$ are not independent.  If two events \textbf{depend on
  disjoint sets of edges}, then they are independent.
\end{staffnotes}

\bparts

\ppart Let $g$ be the number of edges of $G$.  Express $\pr{G}$ in
terms of $g, n$, and $p$.

\begin{solution}
\[
p^g(1-p)^{\binom{n}{2}-g},
\]
\end{solution}

\eparts
\medskip

In parts~\eqref{ExyW}--\eqref{Wwxyz}~below, \inbook{indicate which of}
\inhandout{circle} the event pairs below that are independent.  In
each case, briefly explain your answer.

\bparts

\ppart\label{ExyW} $E(x,y)$ versus $W(x,y)$ for $x \neq y$.

\begin{solution}
\True.  Since there are no self-loops, the edge $\edge{x}{y}$ cannot
be part of any length-two walk from $x$ to $y$.  So the existence of a
length-two walk from $x$ to $y$ does not depend on whether this edge
is present in $G$.
\end{solution}

\examspace[0.15in]

\iffalse
\ppart $W(w,x)$ versus $W(x,y)$ 

  \begin{solution}
\False, as demonstrated by the counterexample of
  $\card{V} = 4$ and $p = \frac{1}{2}$.  \TBA{explanation.}
\end{solution}
\fi

\ppart\label{EandE2}

$[E(x,w) \QAND\ E(w,y)]$ versus $[E(x,z)
  \QAND\ E(z,y)]$ for distinct vertices $w$, $x$, $y$ and $z$.

\begin{solution}
\True.  Since edges are chosen independently and since all the
following edges are distinct, the presence of edges $\edge{x}{w},
\edge{w}{y}$, is independent of the presence of edges $\edge{x}{z},
\edge{z}{y}$.
\end{solution}

\examspace[0.15in]

\ppart\label{Wwxyz} $W(w,x)$ versus $W(y,z)$ for distinct vertices $w$, $x$, $y$ and $z$.

\begin{solution}
\False.  The more randomly chosen edges there are in $G$, the more
likely length two walks become.  Now $W(w,x)$ implies the existence of
two edges---which might be incident to $y$ or $z$---and this makes
$W(y,z)$ more likely given $W(w,x)$.
\end{solution}

%\examspace

\ppart \label{nottwopath} \; Write a simple formula in terms of $n$
and $p$ for $\prob{\QNOT W(x,y)}$, for distinct vertices $x$ and $y$.

\hint Use part~\eqref{EandE2}.
 
\begin{solution}
Let $Z \eqdef V - \set{x,y}$ be the set of all
  the vertices other than $x$ and $y$.

\begin{align*}
\pr{\QNOT(W(x,y))}
  & = \pr{\QAND_{z \in Z} \QNOT(E(x,z) \QAND\ E(z,y))},\\
  & = \prod_{z \in Z} \pr{\QNOT(E(x,z) \QAND\ E(z,y))},
          & \text{(indep. by part~\eqref{EandE2})}\\
  & = \prod_{z \in Z} (1 - \pr{E(x,z) \QAND\ E(z,y)}),\\
  & = \prod_{z \in Z} (1-\pr{E(x,z)}\cdot \pr{E(y,z)}),
          & \text{(edges are independent)}\\
  & = \prod_{z \in Z} (1-p^2), \\
  & = (1-p^2)^{n-2}.
\end{align*}
\end{solution}

\ppart \label{3cycle} \; Express $x$ and $y$ being on a three-cycle as
a simple predicate formula involving $E(x,y)$ and $W(x,y)$.

\begin{solution}
$x$ and $y$ lie on a three-cycle iff $E(x,y) \QAND\ W(x,y)$.
\end{solution}

\examspace[0.15in]

\ppart Let $r \eqdef \pr{\QNOT(W(x,y))}$.  Write a simple expression
in terms of $p$ and $r$ for the probability that $x$ and $y$ lie on a
three-cycle in $G$.

\hint Parts~\eqref{ExyW} and~\eqref{3cycle}.

\begin{solution}
\[
p(1-r).
\]

Since $E(x,y)$ and $W(x,y)$ are independent,
\begin{align*} 
\pr{E(x,y) \QAND\ W(x,y)}
   & = \pr{E(x,y)} \cdot \pr{W(x,y)} \\
   & = p (1-r) .
\end{align*}
\end{solution}

\eparts
\end{problem}

%%%%%%%%%%%%%%%%%%%%%%%%%%%%%%%%%%%%%%%%%%%%%%%%%%%%%%%%%%%%%%%%%%%%%
% Problem ends here
%%%%%%%%%%%%%%%%%%%%%%%%%%%%%%%%%%%%%%%%%%%%%%%%%%%%%%%%%%%%%%%%%%%%%

\endinput


Original Solution: In this counterexample,
\[
W(w,x) \equiv (E(w,y) \QAND{} E(y,x)) \QOR{} (E(w,z) \QAND{} E(z,x))
\]
and
\[
W(x,y) \equiv (E(x,w) \QAND{} E(w,y)) \QOR{} (E(x,z) \QAND{} E(z,y)).
\]

By symmetry, we apply inclusion-exclusion to calculate the probability
for either of these events:
\[
\paren{\frac{1}{2}}^2 + \paren{\frac{1}{2}}^2 - \paren{ \frac{1}{2}}^4 = \frac{7}{16}.
\]

Now consider $\prcond{W(w,x)}{W(x,y)}$, the fraction of outcomes
satisfying $W(x, y)$ that also satisfy $W(w, x)$.  Partition the
outcomes satisfying $W(x, y)$ by whether they also satisfy $E(w, y)$.
Both sides of the partition are independent of $E(y, x)$ in the sense
formalized above, since $E(y, x)$ doesn't appear in the definition of
$W(x, y)$.  That means that the outcomes in the subcase for $E(w, y)$
can be partitioned into equally sized sets, one with $E(y, x)$ and the
other with $\bar{E(y, x)}$.  Clearly every element of the first set
satisfies $W(w, x)$, so
\[
\prcond{W(w, x)}{W(x, y) \QAND{} E(y, x)} \geq \frac{1}{2}.
\]

--------------
  For
  $|V| = 4$ and $p = \frac{1}{2}$, we have
\[
W(w,x) \equiv (E(w,y) \QAND\ E(y,x)) \QOR\ (E(w,z) \QAND\ E(z,x))
\]
 and
\[
W(y,z) \equiv (E(y,x) \QAND\ E(x,z)) \QOR\ (E(y,w) \QAND\ E(w,z)).
\]

\[
\prcond{W(w, x)}{W(y, z) \QAND{} (E(y,x) \QAND{} E(x,z))} = \frac{3}{4},
\]
since under these conditions the formula for $W(w,x)$ simplifies to
$E(w,y) \QOR E(w,z)$.
\[
\prcond{W(w, x)}{W(y, z) \QAND\ \QNOT (E(y,x) \QAND\ E(x,z))} = \frac{1}{2}
\]
since under these conditions we know $E(y,w) \QAND\ E(w,z)$, and the
formula for $W(w,x)$ simplifies to 
\[
(E(y,x) \QOR\ E(z,x)) \QAND\ \QNOT(E(y,x) \QAND\ E(z,x)).
\]
Both sides of the partition have probabilities no less than
$\frac{1}{2}$, so $\prcond{W(w, x)}{W(y, z)} \geq \frac{1}{2}$, which
again is above $\pr{W(w, x)}$, which can be computed as $\frac{7}{16}$
as in the prior case.
-----------

The outcomes in the subcase for $\bar{E(w, y)}$ must all have $E(x,
z)$, so, like above partitioning them based on $E(w, z)$, we get two
equal-size sets, where the set with $E(w, z)$ all satisfy $W(w, x)$,
and
\[
\prcond{W(w, x)}{W(x, y) \QAND\ \bar{E(y, x)}} \geq \frac{1}{2}.
\]
The true value of $\prcond{W(w, x)}{W(x, y)}$ must lie somewhere
between these two values, so it also must be no less than
$\frac{1}{2}$, and thus it must be greater than $\pr{W(w, x)} =
\frac{7}{16}$.

\vspace{0.05in} Less complicated explanation: We want to show that
$\prcond{W(w,x)}{W(x,y)} \neq \Pr{W(w,x)}$.  If $W(x,y)$, this
increases the probability of $E(z,x)$ and $E(w,y)$, which can be used
for $W(w,x)$.

\fi



% Exam version commented out for use as a class problem.
\iffalse
\ppart Circle the mathematical formula that best expresses the
definition of $W(x,y)$.

\begin{itemize}
\item $W(x,y) \eqdef \exists z.\ E(x,z) \QAND\ E(y,z)$
\examspace[0.15in]
\item $W(x,y) \eqdef x \neq y \QAND\ \exists z.\ E(x,z) \QAND\ E(y,z)$
\examspace[0.15in]
\item $W(x,y) \eqdef \forall z.\ E(x,z) \QOR\ E(y,z)$
\examspace[0.15in]
\item $W(x,y) \eqdef \forall z.\ x \neq y \QIMPLIES\ [E(x,z) \QOR\ E(y,z)]$
\end{itemize}
\fi
