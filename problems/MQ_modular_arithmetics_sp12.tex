\documentclass[problem]{mcs}

\begin{pcomments}
  \pcomment{MQ_modular_arithmetic}
  \pcomment{subsumes TP_Eulers_Theorem}
  \pcomment{from spring06 Quiz2}
\end{pcomments}

\pkeywords{
  modular_arithmetic
  multiplicative_inverse
  Eulers_theorem
  phi
  totient
 }


%%%%%%%%%%%%%%%%%%%%%%%%%%%%%%%%%%%%%%%%%%%%%%%%%%%%%%%%%%%%%%%%%%%%%
% Problem starts here
%%%%%%%%%%%%%%%%%%%%%%%%%%%%%%%%%%%%%%%%%%%%%%%%%%%%%%%%%%%%%%%%%%%%%

\begin{problem}

\bparts

\ppart
Prove that $2012^{1200}$ has a multiplicative inverse modulo
$77$. Note that $2012 = 503 \cdot 2^3$.

\examspace[2in]
\begin{solution}
Since $\gcd(2012^{1201},77) =
       \gcd(2^{3063} \cdot 503^{1201},7 \cdot 11) =
       1$, $2012^{1201}$ has a multiplicative inverse modulo 77.
\end{solution}

\ppart What is the value of $\phi(77)$, where $\phi$ is Euler's
function?

\examspace[2in]
\begin{solution}
Noting that $77=7 \cdot 11$, it follows
that $\phi(77)=(7-1)(11-1)= 6 \cdot 10 =60$.
\end{solution}

\ppart What is the remainder of $2012^{1201}$ divided by $77$?

\begin{solution}

Since 2012 and 77 are relatively prime, we have by Euler's Theorem that
$2012^{60} \equiv 1 \pmod {77}$, and so
\[
2012^{1200} = \paren{2012^{60}}^{20} \equiv 1^{20} \equiv 1 \pmod {77}.
\]
\end{solution}

\eparts
\end{problem}

\endinput
