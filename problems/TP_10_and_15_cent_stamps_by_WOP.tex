!\documentclass[problem]{mcs}

\begin{pcomments}
  \pcomment{TP_10_and_15_cent_stamps_by_WOP}
  \pcomment{same as F09: cp2m with 10 replacing 6}
\end{pcomments}

\pkeywords{
  well_ordering
  WOP
  postage_stamps
}

%%%%%%%%%%%%%%%%%%%%%%%%%%%%%%%%%%%%%%%%%%%%%%%%%%%%%%%%%%%%%%%%%%%%%
% Problems start here
%%%%%%%%%%%%%%%%%%%%%%%%%%%%%%%%%%%%%%%%%%%%%%%%%%%%%%%%%%%%%%%%%%%%%

\begin{problem}
  For practice using the Well Ordering Principle, fill in the template
  of an easy to prove fact: every amount of postage that
  can be assembled using only 10 cent and 15 cent stamps is
  divisible by 5.

  In particular, Let $S(n)$ mean that exactly $n$ cents postage can be
  assembled using only 10 and 15 cent stamps.  Then the proof shows that
%
\begin{equation}\tag{*}
S(n)\ \QIMPLIES\ 5 \divides n, \quad \text{for all nonnegative integers $n$}.
\end{equation}
Fill in the missing portions (indicated by ``\dots'') of the following
proof of~(*).

\begin{quote}
Let $C$ be the set of \emph{counterexamples} to~(*), namely
\[
C \eqdef \set{n \suchthat \dots}
\]

\begin{solution}
 $n$ is a counterexample to~(*) if $n$ cents postage can be
  made and $n$ is not divisible by 5, so the predicate
\[
S(n)\text{ and } \QNOT(5 \divides n)
\]
defines the set, $C$, of counterexamples.
\end{solution}

Assume for the purpose of obtaining a contradiction that $C$ is
nonempty.  Then by the WOP, there is a smallest number, $m \in C$.
This $m$ must be positive because \dots.

\begin{solution}
\dots $5 \divides 0$, so 0 is not a counterexample.
\end{solution}

But if $S(m)$ holds and $m$ is positive, then $S(m-10)$ or $S(m-15)$
must hold, because \dots.

\begin{solution}
\dots if $m>0$ cents postage is made from 10 and 15 cent
  stamps, at least one stamp must have been used, so removing this
  stamp will leave another amount of postage that can be made.
\end{solution}

So suppose $S(m-10)$ holds.  Then $5 \divides (m-10)$, because\dots

\begin{solution}
\dots if $\QNOT(5 \divides (m-10))$, then $m-10$ would be
  a counterexample smaller than $m$, contradicting the minimality of
  $m$.
\end{solution}

But if $5 \divides (m-10)$, then obviously $5 \divides m$,
contradicting the fact that $m$ is a counterexample.

Next, if $S(m-15)$ holds, we arrive at a contradiction in the same way.

Since we get a contradiction in both cases, we conclude that\dots

\begin{solution}
\dots $C$ must be empty.  That is, there are no counterexamples
to~(*),
\end{solution}

which proves that (*) holds.

\end{quote}
\end{problem}

%%%%%%%%%%%%%%%%%%%%%%%%%%%%%%%%%%%%%%%%%%%%%%%%%%%%%%%%%%%%%%%%%%%%%
% Problems end here
%%%%%%%%%%%%%%%%%%%%%%%%%%%%%%%%%%%%%%%%%%%%%%%%%%%%%%%%%%%%%%%%%%%%%

\endinput
