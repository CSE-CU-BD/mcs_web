\documentclass[problem]{mcs}

\begin{pcomments}
  \pcomment{FP_graph_variables}
  \pcomment{ARM 11/12/17}
\end{pcomments}

\pkeywords{
  graphs
  variables
  increasing
  decreasing
  connected
  chromatic
  components
}

%%%%%%%%%%%%%%%%%%%%%%%%%%%%%%%%%%%%%%%%%%%%%%%%%%%%%%%%%%%%%%%%%%%%%
% Problem starts here
%%%%%%%%%%%%%%%%%%%%%%%%%%%%%%%%%%%%%%%%%%%%%%%%%%%%%%%%%%%%%%%%%%%%%
\begin{problem}
  For any simple graph $G$, a ``new'' edge $e$ is one that connects
  nonadjacent vertices.  (More formally, $e$ is ``new'' when $e \in
  \vertices{G}^2 - \edges{G}$.)  let $G+e$ be the graph that results
  from adding the new edge $e$ to $\edges{G}$.  A real-valued function
  $f$ on finite simple graphs is
\begin{itemize}
 \item \emph{strictly edge-increasing} \inhandout{(\textbf{SI})}  when $f(G+e) > f(G)$,
 \item \emph{constant} \inhandout{(\textbf{C}) }when $f(G) = f(G+e)$
 \item \emph{weakly edge-increasing} \inhandout{(\textbf{WI})} when $f(G+e) \geq f(G)$ and $f$ is not constant,
 \item \emph{strictly edge-decreasing} \inhandout{(\textbf{SD})} when $f(G+e) < f(G)$,
 \item \emph{weakly edge-decreasing} \inhandout{(\textbf{WD})} when
   $f(G+e) \leq f(G)$ and $f$ is not constant,
\end{itemize}
for all finite simple graphs $G$ and new edges $e$.

For example, if $f(G)$ is $\card{\edges{G}}$, then $f(G+e) = 1 + f(G)
> f(G)$, so this $f$ is strictly increasing.  Similarly, if $f(G)$ is
$\card{\vertices{G}}$, then $f(G+e) = f(G)$ since adding an edge
between vertices leaves the set of vertices unchanged; this $f$ would
be constant.

For each of the following functions $f(G)$, indicate which of the above
properties it has\inbook{, if any}.  \inhandout{Write \textbf{N} for
  \emph{none}.}

\begin{enumerate}[(i)]
\item $\chi(G)$, the chromatic number of $G$.  \hfill\examrule \insolutions{\textbf{WI}}

\item the \emph{connectivity} of $G$: $\max\set{k \in \nngint
  \suchthat G\ \text{is $k$-connected}}$ \hfill\examrule
  \insolutions{\textbf{WI}}

\item the maximum path length in $G$ \hfill\examrule \insolutions{\textbf{WI}}

\item the largest \emph{distance}\footnote{The distance between two
  vertices is the length of the shortest path between them.  It is
  infinite if the vertices are not connected.} between two vertices of $G$.
  \hfill\examrule \insolutions{\textbf{WD}}

\item the \emph{girth} of $G$: the length of the smallest cycle in $G$
  \hfill\examrule \insolutions{\textbf{WD}}

\item  the number of connected components of $G$ \hfill\examrule \insolutions{\textbf{WD}}

\item the size of the largest complete subgraph of $G$: $\max\set{n
  \suchthat K_n\ \text{is a subgraph of}\ G}$ \hfill\examrule
  \insolutions{\textbf{WI}}

\item the sum of the vertex degrees of $G$ \hfill\examrule
  \insolutions{\textbf{SI}}

\item the number of spanning trees of $G$ \hfill\examrule \insolutions{
        \textbf{WI}\\
        ---will remain zero until $G$ becomes connected.}

\item the number of cut edges in $G$ \hfill\examrule \insolutions{\textbf{N}}

\end{enumerate}
\end{problem}
%%%%%%%%%%%%%%%%%%%%%%%%%%%%%%%%%%%%%%%%%%%%%%%%%%%%%%%%%%%%%%%%%%%%%
% Problem ends here
%%%%%%%%%%%%%%%%%%%%%%%%%%%%%%%%%%%%%%%%%%%%%%%%%%%%%%%%%%%%%%%%%%%%%

\endinput
