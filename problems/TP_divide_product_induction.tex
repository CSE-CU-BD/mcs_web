\documentclass[problem]{mcs}

\begin{pcomments}
  \pcomment{TP_divide_product_induction}
  \pcomment{minor revision of part(b) of PS_induction_mod_proof}
  \pcomment{by ARM 3/17/13}
\end{pcomments}

\pkeywords{
  divides
  prime
  induction
}

%%%%%%%%%%%%%%%%%%%%%%%%%%%%%%%%%%%%%%%%%%%%%%%%%%%%%%%%%%%%%%%%%%%%%
% Problem starts here
%%%%%%%%%%%%%%%%%%%%%%%%%%%%%%%%%%%%%%%%%%%%%%%%%%%%%%%%%%%%%%%%%%%%%

\begin{problem}
Prove by induction that if $p$ is prime, then for all
$a_1,a_2,\dots,a_n$ where $n\geq 1$, if $p \divides a_1 \cdot a_2
\cdots a_n$, then $p$ divides some $a_i$.  You may assume the case for
$n=2$ which was proved \inhandout{in the
  text}\inbook{Lemma~\bref{lem:prime-divides}}.
  
\begin{solution}
We proceed by induction on $n$ with induction hypothesis
\[
P(n) \eqdef\ (p\text{ is prime } \QIMPLIES \forall a_1, a_2 \dots,
a_n,\, (p \divides a_1 \cdot a_2 \cdots a_n) \QIMPLIES \exists i \in
[1,n].\, p \divides a_i).
\]

\inductioncase{Base case}: ($n = 1$).  $P(1)$ asserts that if $p \mid
a_1$, then $p \mid a_1$, which is trivially true.

\inductioncase{Inductive step}: Now we assume $P(n)$ holds for some $n
\geq 1$ and prove $P(n + 1)$.

So suppose that
\[
p \divides a_1 \cdot a_2 \cdots a_n \cdot a_{n+1}.
\]
Let $b \eqdef a_1 \cdots a_n \cdot a_{n}$.  Now $p \divides ba_{n+1}$,
which we know implies that $p \divides a_{n+1}$ or $p \divides b$.  If
$p \divides a_{n+1}$, then $P(n+1)$ follows immediately.  If $p
\divides b$, then the induction hypothesis $P(n)$ implies that $p
\divides a_i$ for some $i \in [1,n]$, which also implies $P(n+1)$.  So
in either case, $P(n+1)$ holds, which completes the inductive step.

By induction, the claim holds for all $n \geq 1$.
\end{solution}
  
\end{problem}

%%%%%%%%%%%%%%%%%%%%%%%%%%%%%%%%%%%%%%%%%%%%%%%%%%%%%%%%%%%%%%%%%%%%%
% Problem ends here
%%%%%%%%%%%%%%%%%%%%%%%%%%%%%%%%%%%%%%%%%%%%%%%%%%%%%%%%%%%%%%%%%%%%%

\endinput
