\documentclass[problem]{mcs}

\begin{pcomments}
  \pcomment{CP_harmonic_number_divergence}
  \pcomment{from: S07.cp9m, S09.cp9m, F09.cp9w,S10.cp9w, S12.cp8f}
  \pcomment{overlaps MQ/TP_Summation}
\end{pcomments}

\pkeywords{
  harmonic_numbers
  integral_method
}

%%%%%%%%%%%%%%%%%%%%%%%%%%%%%%%%%%%%%%%%%%%%%%%%%%%%%%%%%%%%%%%%%%%%%
% Problem starts here
%%%%%%%%%%%%%%%%%%%%%%%%%%%%%%%%%%%%%%%%%%%%%%%%%%%%%%%%%%%%%%%%%%%%%

\begin{problem}
There is a number $a$ such that $\sum_{i=1}^\infty i^p$ converges iff
$p < a$.  What is the value of $a$?  Prove it.

\hint Find a value for $a$ you think that works, then apply the integral bound.

\begin{solution}
$a= -1$.

For $p=-1$, the sum is the harmonic series which we know does not
converge.  Since the term $i^p$ is increasing in $p$ for $i>1$, the sum
will be larger, and hence also diverge for $p>-1$.

For $p<-1$ there exists an $\epsilon > 0$ such that $p =
-(1+\epsilon)$. By the integral method,

\begin{align*}
 \sum_{i=1}^\infty i^{-(1+\epsilon)}
    & \leq 1 + \int_{1}^{\infty} x^{-(1+\epsilon)}\, dx\\
    & = 1 + \epsilon^{-1} - \epsilon^{-1}
                \lim_{\alpha \to \infty} \alpha^{-\epsilon}\\
    & = 1 + \epsilon^{-1}\\
    & < \infty
\end{align*}
Hence the sum is bounded above, and since it is increasing, it has a
finite limit, that is, it converges.
\end{solution}

\end{problem}

%%%%%%%%%%%%%%%%%%%%%%%%%%%%%%%%%%%%%%%%%%%%%%%%%%%%%%%%%%%%%%%%%%%%%
% Problem ends here
%%%%%%%%%%%%%%%%%%%%%%%%%%%%%%%%%%%%%%%%%%%%%%%%%%%%%%%%%%%%%%%%%%%%%

\endinput
