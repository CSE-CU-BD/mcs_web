\documentclass[problem]{mcs}

\begin{pcomments}
  \pcomment{PS_conditional_aces}
  \pcomment{from S07.ps11}
  \pcomment{soln edited ARM 10/23/15}
\end{pcomments}

\pkeywords{
  probability
  probability_space
  conditional_probability
  outcomes
}

%%%%%%%%%%%%%%%%%%%%%%%%%%%%%%%%%%%%%%%%%%%%%%%%%%%%%%%%%%%%%%%%%%%%%
% Problem starts here
%%%%%%%%%%%%%%%%%%%%%%%%%%%%%%%%%%%%%%%%%%%%%%%%%%%%%%%%%%%%%%%%%%%%%

\newcommand{\ahrt}{\text{A$\heartsuit$}}
\newcommand{\asp}{\text{A$\spadesuit$}}

\begin{problem}

Suppose you have three cards: $\ahrt$, $\asp$, and a jack.  From these, you
choose a random hand (that is, each card is equally likely to be chosen) of
two cards, and let $n$ be the number of aces in your hand.  You then
randomly pick one of the cards in the hand and reveal it.

\bparts

\ppart\label{events} Describe a simple probability space (that is, outcomes and their
probabilities) for this scenario, and list the outcomes in each of the
following events:
\begin{enumerate}
\item $[n \geq 1]$, (that is, your hand has an ace in it),
\item \ahrt\ is in your hand,
\item the revealed card is an \ahrt,
\item the revealed card is an ace.
\end{enumerate}

\begin{solution}

Consider each outcome as a pair of cards, the first of which is the
revealed card. Each outcome is equally likely (probability $1/6$).

The sets of outcomes are then as follows:
\begin{enumerate}

\item $[n \geq 1]$: all pairs: 
\begin{align*}
\lefteqn{\left\{(\ahrt, \asp),(\ahrt, \text{jack}),(\asp, \ahrt),\right.} \\
 & \left. (\asp, \text{jack}),(\text{jack}, \ahrt),(\text{jack}, \asp)\right\},
\end{align*}

\item \ahrt\ is in your hand: $\set{(\ahrt, \asp),(\ahrt,
  \text{jack}),(\asp, \ahrt), (\text{jack}, \ahrt)}$,

\item the revealed card is an \ahrt: $\set{(\ahrt, \asp),(\ahrt, \text{jack})}$,


\item the revealed card is an ace: $\set{(\ahrt, \asp),(\ahrt,
  \text{jack}),(\asp, \ahrt),(\asp, \text{jack})}$.
\end{enumerate}
\end{solution}

\ppart Then calculate $\prcond{n=2}{E}$ for $E$ equal to each of the four
events in part~\eqref{events}.  Notice that most, but \emph{not all}, of
these probabilities are equal.

\begin{solution}

First, note that $\pr{n=2} = 1/3$.
\begin{enumerate}
\item $\prcond{n=2}{n \geq 1} = \pr{n=2}/1 = 1/3$, 
\item $\prcond{n=2}{\text{\ahrt\ is in your hand}} = \pr{n=2}/(2/3) = 1/2$,
\item $\prcond{n=2}{\text{the revealed card is an \ahrt}} = \pr{(\ahrt, \asp)}/(1/3)= 1/2$,
\item $\prcond{n=2}{\text{the revealed card is an ace}} = \pr{n=2}/(2/3)= 1/2$.
\end{enumerate}
\end{solution}

\eparts

Now suppose you have a deck with $d$ distinct cards, $a$ different
kinds of aces (including an \ahrt), you draw a random hand with $h$
cards, and then reveal a random card from your hand.

\bparts

\ppart\label{h/d} Prove that $\pr{\text{\ahrt\ is in your hand}} = h/d$.

\begin{solution}

The number, $N$, of hands is
\[
N = \binom{d}{h}.
\]

\begin{align*}
\lefteqn{\pr{\text{\ahrt\ is in your hand}}}\\
     & = \frac{\text{\# hands with \ahrt}}{N}\\
     & = \frac{\text{\# $(h-1)$-card hands from deck w/o \ahrt}}{N}\\
     & = \frac{\binom{d-1}{h-1}}{N}\\
     & = \frac{(d-1)! h! (d-h)!}{(h-1)! (d-h)! d!} & \text{(def. of $\textstyle{\binom{m}{n}}$)}\\
     &  = h/d.   & \text{(simplification)}
\end{align*}

\begin{editingnotes}
Alternatively
\[ = 1-prob{\text{\ahrt\ not in hand}}
              = 1- (d-1)/d \cdot (d-2)/(d-1) \cdots (d-h)/(d-h+1)
              = 1- (d-h)/d = h/d.
\]
\end{editingnotes}

\end{solution}

\ppart
Prove that
\begin{equation}\label{2dah}
\prcond{n=2}{\text{\ahrt\ is in your hand}} = \pr{n=2}\cdot\frac{2d}{ah}.
\end{equation}

\begin{solution}

\begin{align*}
\lefteqn{\prcond{n=2}{\text{\ahrt\ is in your hand}}}\\
  & = \frac{\pr{n=2 \text{ and \ahrt\ is in your hand}}}%
          {\pr{\text{\ahrt\ is in your hand}}}\\
  & = \frac{\pr{n=2 \text{ and \ahrt\ is in your hand}}}{h/d}
             &\text{(part~\eqref{h/d})}\\
 & = \frac{\pr{n=2}\cdot \prcond{\text{\ahrt\ is in your hand}}{n=2}}{h/d}\\
 & = \frac{\pr{n=2} \cdot 2/a}{h/d} & \text{(see below)}\\
 & = \pr{n=2} \cdot \frac{2d}{ah}
\end{align*}

The $\frac{2}{a}$ substitution above is justified by observing that in
a hand with two aces, each of the $\binom{a}{2}$ pairs of aces is
equally likely to be the one in the hand.  Of these pairs, $a-1$
include the $\ahrt$.  So
\[
\prcond{\text{\ahrt\ is in your hand}}{n=2} = \frac{a-1}{\binom{a}{2}} = \frac{2}{a}.
\]
\end{solution}

\ppart Conclude that
\[
\prcond{n=2}{\text{the revealed card is an ace}} = \prcond{n=2}{\text{\ahrt\ is in your hand}}.
\]

\begin{solution}

Note that
\begin{equation}\label{ad}
\pr{\text{the revealed card is an ace}} = \frac{a}{d},
\end{equation}
since the probability of revealing an ace from the random hand is simply
the probability that a random card is an ace.  Now,
\begin{align*}
\lefteqn{\prcond{n=2}{\text{the revealed card is an ace}}}\\
& = \frac{\pr{n=2 \text{ and the revealed card is an ace}}}
     {\pr{\text{the revealed card is an ace}}}\\
& = \frac{\pr{n=2 \text{ and the revealed card is an ace}}}
         {a/d} & \text{(by~\eqref{ad})}\\
& = \frac{\pr{n=2}\prcond{\text{the revealed card is an ace}}{n=2}}{a/d}\\
& = \frac{\pr{n=2}(2/h)}{a/d} & \text{(see below)}\\
& = \frac{2d}{ah}\cdot \pr{n=2}\\
& = \prcond{n=2}{\text{\ahrt\ is in your hand}}. & \text{(by~\eqref{2dah})}
\end{align*}

The $\frac{2}{h}$ substitution for $\prcond{\text{the revealed card is
    an ace}}{n=2}$ is justified by noting that, in any hand with 2
aces, that is, when $n=2$, the revealed card must come from one of exactly 2
positions (with aces) out of $h$ total positions.

\end{solution}

\eparts

\end{problem}

%%%%%%%%%%%%%%%%%%%%%%%%%%%%%%%%%%%%%%%%%%%%%%%%%%%%%%%%%%%%%%%%%%%%%
% Problem ends here
%%%%%%%%%%%%%%%%%%%%%%%%%%%%%%%%%%%%%%%%%%%%%%%%%%%%%%%%%%%%%%%%%%%%%

\endinput
