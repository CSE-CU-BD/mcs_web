\documentclass[problem]{mcs}

\begin{pcomments}
  \pcomment{TP_min_of_finite_WOP}
  \pcomment{ARM 2/12/12}
\end{pcomments}

\pkeywords{
  induction
  well_order
  minimum
}

\begin{problem}

Use the Well Ordering Principle to prove that every finite, nonempty
set of real numbers has a minimum element.

\begin{staffnotes}
\hint Consider a \emph{minimum} size, nonempty finite set of real
numbers with no minimum element.
\end{staffnotes}

\begin{solution}
\begin{proof}
A finite set by definition has $n$ elements for some $n \in
\nngint$.  So the claim will follow if we prove that every set of $n
\geq 1$ real numbers has a minimum element.

Suppose to the contrary that there was a set of $n>0$ real numbers
with no minimum element.  By the WOP, there is a minimum value $m>0$
and a set $R$ of $m$ real numbers with no minimum element.  Since
$m>0$, there is an element $r \in R$.

If $m=1$, then $r$ is the only element in $R$, and so $r$ is the
minimum element, contradicting the fact that $R$ has no minimum.  So
we conclude that $m>1$.

Now remove $r$ from $R$ to obtain a smaller set $R'$ with at least one element.
Since, $R$ was the smallest size set without a minimum, we know that
there is a minimum element $r' \in R'$.

But then the minimum element of $R$ is simply the smaller of $r$ and
$r'$, contradicting the fact that $R$ has no minimum.

This contradiction implies that there is no set of $n>1$ real numbers
that does not have a minimum.
\end{proof}

\end{solution}

\end{problem}

\endinput
