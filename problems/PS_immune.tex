\documentclass[problem]{mcs}

\begin{pcomments}
  \pcomment{PS_immune}
  \pcomment{variant of PS_immuneset} 
  \pcomment{ARM 4.15.17}
\end{pcomments}

\pkeywords{
  diagonal_argument
  diagonal
  complement
}

%%%%%%%%%%%%%%%%%%%%%%%%%%%%%%%%%%%%%%%%%%%%%%%%%%%%%%%%%%%%%%%%%%%%%
% Problem starts here
%%%%%%%%%%%%%%%%%%%%%%%%%%%%%%%%%%%%%%%%%%%%%%%%%%%%%%%%%%%%%%%%%%%%%

\begin{problem}
Suppose $\mathcal{S} = \set{S_0, S_1,\dots}$ is a countable set each
of whose elements $S_n$ is an infinite set of nonnegative integers.
Using a diagonal argument, we can find a ``new'' infinite set $U$ of
nonnegative integers that is not in $\mathcal{S}$.

In this problem we describe how to find an infinite set $U$ of
nonnegative integers that not only is not in $\mathcal{S}$, but does
not even have a subset that is in $\mathcal{S}$.

Rather than describing $U$ directly, it's a little easier if we
describe its complement $C$.  Then $U$ will be $\bar{C} \eqdef \nngint
- C$.

Define a function $f:\nngint \to \nngint$ as follows:
\[
f(n) \eqdef \min\set{k \in S_n \suchthat k \geq n^2},
\]
and define
\[
C \eqdef \range{f}.
\]

\bparts

\ppart Prove that no subset of $U$ is in $\mathcal{S}$.

\begin{solution}
We need to show that no $S_n$ is a subset of $U$.

Since $S_n$ is infinite, $f(n)$ will exist.  By definition, $f(n) \in
S_n \intersect C$.  So $S_n$ has an element $f(n)$ that is not in $U$,
and therefore $S_n$ is not a subset of $U$.
\end{solution}

\ppart Show that the limiting density of $U$ is one.  That is,
\[
\lim_{k \to \infty} \frac{\card{U \intersect \Zintv{0}{k}}}{k} = 1.
\]

\begin{solution}
The only numbers not in $U \intersect \Zintv{0}{k}$ are $f(n)$ such
  that $f(n) \leq k$.  But $n^2 \leq f(n)$ by definition, which means
  $n \leq \sqrt{k}$.  So
\[
\card{U \intersect \Zintv{0}{k}} \geq k -\sqrt{k}.
\]
Therefore,
\[
\lim_{k \to \infty} \frac{\card{U \intersect \Zintv{0}{k}}}{k}
 \geq \lim_{k \to \infty} \frac{k-\sqrt{k}}{k} = 1.
\]
\end{solution}

\eparts
\end{problem}

\endinput
