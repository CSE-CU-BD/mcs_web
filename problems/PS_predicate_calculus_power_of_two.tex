\documentclass[problem]{mcs}

\begin{pcomments}
  \pcomment{PS_predicate_calculus_power_of_two}
  \pcomment{from: S06.ps1}
  \pcomment{edited way down from PS_express_predicates_in_formal_logic_notation}
  \pcomment{conflicts with PS_predicate_calculus_power_of_prime}
\end{pcomments}

\pkeywords{
  predicates
  propositional_logic
  power_of
}

%%%%%%%%%%%%%%%%%%%%%%%%%%%%%%%%%%%%%%%%%%%%%%%%%%%%%%%%%%%%%%%%%%%%%
% Problem starts here
%%%%%%%%%%%%%%%%%%%%%%%%%%%%%%%%%%%%%%%%%%%%%%%%%%%%%%%%%%%%%%%%%%%%%

\newcommand{\isprime}{\text{\sc{is-prime}}}
\newcommand{\ispowerofprime}{\text{\sc{is-power-of-prime}}}
\newcommand{\hastwoprimefactors}{\text{\sc{has-two-prime-factors}}}

\begin{problem}
  Express each of the following predicates and propositions in formal
  logic notation.  The domain of discourse is the nonnegative integers,
  $\naturals$.  Moreover, in addition to the propositional operators,
  variables and quantifiers, you may define predicates using addition,
  multiplication, and equality symbols, and nonnegative integer
  \emph{constants} $\mtt{0}, \mtt{1},\dots$, but no \emph{exponentiation}
  (like $x^y$).  For example, the predicate ``n is an even number''
  could be defined by either of the following formulas:
\[
\exists m.\; (2m = n), \qquad \exists m.\; (m + m = n).
\]

\begin{problemparts}
\iffalse
\problempart 
$n$ is the sum of two cubes (a cube is a number equal to $k^3$ for some
integer $k$).

\begin{solution}
\[
\exists x \exists y.\; (x \cdot x \cdot x + y \cdot y \cdot y = n)
\]

Since the constant 0 is not allowed to appear explicitly, the predicate
``$x = 0$'' can't be written directly, but note that it could be expressed
in a simple way as:
\[
x + x = x.
\]
Then the predicate $x > y$ could be expressed
\[
\exists w.\; (y + w = x) \land (w \neq 0).
\]
Note that we've used ``$w \neq 0$'' in this formula, even though it's
technically not allowed.  But since ``$w \neq 0$'' is equivalent to the
allowed formula ``$\neg(w+w= w)$,'' we can use ``$w \neq 0$'' with the
understanding that it abbreviates the real thing.  And now that we've shown
how to express ``$x>y$,'' it's ok to use it too.
\end{solution}

\problempart $x = 1$.

\begin{solution}
One formula is $\forall y.\; xy=y$.  Another is $(x \cdot x = x) \conj (x
\neq 0)$.
\end{solution}
\fi

\ppart $m$ is a divisor of $n$ (notation: $m \divides n$)

\begin{solution}
\[
m \divides n 
\eqdef\quad 
\exists k.\; k \cdot m = n
\]
\end{solution}

\problempart 
$n$ is a prime number.

\begin{solution}
\[
\isprime(n)
\eqdef\quad 
(n\neq 1)\ \QAND\ \forall m.\; (m \divides n) \QIMPLIES (m=1 \QOR m=n).
\]
Note that $n\neq 1$ is an abbreviation of the formula $\QNOT(n=1)$.
\end{solution}

\problempart 
$n$ is a power of a prime.

\begin{solution}
Suppose that $n$ is a power of a prime $p$ (i.e., $n = p^k$ for
some $k > 0$). Then every divisor of $n$ (say, $m > 1$)
  is of the form $m = p^j$ for some $j \leq k$. Therefore, $m$ is itself divisible by $p$. In
  formal logic notation, we have:
\[
\ispowerofprime(n) 
\eqdef\quad
\exists p.\,[\isprime(p) \QAND (\forall m.\; (m \divides n \QAND m \neq 1)
\QIMPLIES p \divides m)].
\]
\end{solution}

\problempart
$n$ has exactly two distinct prime factors.

\begin{solution}
Suppose that $n$ has exactly two distinct prime factors. Then $n =
pq$, where $p$ and $q$ are powers of {\em distinct} primes. The
distinctness property ensures that $p$ cannot be a divisor of $q$, and vice versa. In
formal logic notation, we have:   
\[
\exists p, q.\,[n=pq~\QAND~\ispowerofprime(p)~\QAND~\ispowerofprime(q)~\QAND~\QNOT\left((p \divides q) \QOR (q \divides p) \right)].
\]
\end{solution}

\end{problemparts}
\end{problem}

%%%%%%%%%%%%%%%%%%%%%%%%%%%%%%%%%%%%%%%%%%%%%%%%%%%%%%%%%%%%%%%%%%%%%
% Problem ends here
%%%%%%%%%%%%%%%%%%%%%%%%%%%%%%%%%%%%%%%%%%%%%%%%%%%%%%%%%%%%%%%%%%%%%
