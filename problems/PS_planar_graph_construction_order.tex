\documentclass[problem]{mcs}

\begin{pcomments}
  \pcomment{PS_planar_graph_construction_order}
  \pcomment{from: S09.ps6, notesproblem planar_graphs}
  \pcomment{commented out in S09}
  \pcomment{solution missing}
\end{pcomments}

\pkeywords{
  planar_graph
  structural_induction
  embedding
}

%%%%%%%%%%%%%%%%%%%%%%%%%%%%%%%%%%%%%%%%%%%%%%%%%%%%%%%%%%%%%%%%%%%%%
% Problem starts here
%%%%%%%%%%%%%%%%%%%%%%%%%%%%%%%%%%%%%%%%%%%%%%%%%%%%%%%%%%%%%%%%%%%%%

\begin{problem}
\bparts

\ppart Prove Lemma~\bref{switch-edges}.  \hint There are four cases to analyze,
  depending on which two constructor operations are applied to add $e$ and
  then $f$.  Structural induction is not needed.

\ppart Prove Corollary~\bref{permute-edges}.

\hint By induction on the number of switches of adjacent elements needed
to convert the sequence 0,1,\dots,$n$ into a permutation
$\pi(0),\pi(1),\dots,\pi(n)$.

\eparts
\begin{editingnotes}
  \href{http://courses.csail.mit.edu/6.042/spring09/ln6.pdf#switch.edges}
  {Lemma 7.8} in Notes 6 established a key property of planar embeddings:
  two edges that could be successively added to an embedding can be added
  in either order.  Carefully prove this property for the case that the
  two edges are both added by the split-a-face constructor.
\end{editingnotes}

\begin{solution}
\textcolor{red}{TBA}
\end{solution}


\end{problem}

%%%%%%%%%%%%%%%%%%%%%%%%%%%%%%%%%%%%%%%%%%%%%%%%%%%%%%%%%%%%%%%%%%%%%
% Problem ends here
%%%%%%%%%%%%%%%%%%%%%%%%%%%%%%%%%%%%%%%%%%%%%%%%%%%%%%%%%%%%%%%%%%%%%

\endinput
