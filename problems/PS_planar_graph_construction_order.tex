\documentclass[problem]{mcs}

\begin{pcomments}
  \pcomment{PS_planar_graph_construction_order}
  \pcomment{from: S09.ps6, notesproblem planar_graphs}
  \pcomment{commented out in S09}
  \pcomment{solution missing}
\end{pcomments}

\pkeywords{
  planar_graph
  structural_induction
  embedding
}

%%%%%%%%%%%%%%%%%%%%%%%%%%%%%%%%%%%%%%%%%%%%%%%%%%%%%%%%%%%%%%%%%%%%%
% Problem starts here
%%%%%%%%%%%%%%%%%%%%%%%%%%%%%%%%%%%%%%%%%%%%%%%%%%%%%%%%%%%%%%%%%%%%%

\begin{problem}
\bparts

\ppart Prove %Lemma~\bref{switch-edges}.

\begin{lemma*}[Switch Edges] %\label{switch-edges}
 Suppose that, starting from some embeddings of planar graphs with
 disjoint sets of vertices, it is possible by two successive
 applications of constructor operations to add edges $e$ and then $f$
 to obtain a planar embedding $\embed{F}$.  Then starting from the
 same embeddings, it is also possible to obtain $\embed{F}$ by adding
 $f$ and then $e$ with two successive applications of constructor
 operations.
\end{lemma*}

\hint There are four cases to analyze,
  depending on which two constructor operations are applied to add $e$ and
  then $f$.  Structural induction is not needed.

\begin{solution}
\TBA{case analysis}

Another way of phrasing this result is that the add-edge transitions
relation between planar graphs are \emph{commuting} as defined in
Problem~\bref{PS_commuting_rewrite_confluence}.  The proof of
part~eqref{permedge} can be seen as a refinement of the proof in
Problem~\bref{PS_commuting_rewrite_confluence} that the transitions
are therefore \emph{confluent}.
\end{solution}

\ppart\label{permedge} Prove % Corollary~\bref{permute-edges}.

\begin{corollary*}[Permute Edges]  %\label{permute-edges}

Suppose that, starting from some embeddings of planar graphs with
disjoint sets of vertices, it is possible to add a sequence of edges
$e_0,e_1,\dots,e_n$ by successive applications of constructor
operations to obtain a planar embedding $\embed{F}$.  Then starting
from the same embeddings, it is also possible to obtain $\embed{F}$ by
applications of constructor operations that successively add any
permutation\footnote{If $\pi:\set{0,1,\dots,n} \to \set{0,1,\dots,n}$
  is a bijection, then the sequence
  $e_{\pi(0)},e_{\pi(1)},\dots,e_{\pi(n)}$ is called a
  \term{permutation} of the sequence $e_0,e_1,\dots,e_n$.} of the
edges $e_0,e_1,\dots,e_n$.
\end{corollary*}

\hint By induction on the number of switches of adjacent elements needed
to convert the sequence 0,1,\dots,$n$ into a permutation
$\pi(0),\pi(1),\dots,\pi(n)$.

\begin{solution}
\textcolor{red}{TBA}
\end{solution}

\ppart Prove 

\begin{corollary*}[Delete Edge] %\label{delete-edge}
Deleting an edge from a planar graph leaves a planar graph.
\end{corollary*}
\begin{solution}

\begin{proof}
  By Corollary [Permute Edges], we may assume the deleted edge was the
  last one added in constructing an embedding of the graph.  So the
  embedding to which this last edge was added must be an embedding of the
  graph without that edge.
\end{proof}

\end{solution}

\ppart Conclude that any subgraph of a planar graph is planar.

\begin{solution}
\textcolor{red}{TBA}
\end{solution}

\eparts

\end{problem}

%%%%%%%%%%%%%%%%%%%%%%%%%%%%%%%%%%%%%%%%%%%%%%%%%%%%%%%%%%%%%%%%%%%%%
% Problem ends here
%%%%%%%%%%%%%%%%%%%%%%%%%%%%%%%%%%%%%%%%%%%%%%%%%%%%%%%%%%%%%%%%%%%%%

\endinput
