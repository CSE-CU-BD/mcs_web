\documentclass[problem]{mcs}

\begin{pcomments}
  \pcomment{PS_emailed_exactly_2_others}
  \pcomment{perturbation of PS_emailed_at_most_2_others}
  \pcomment{from: S09.ps1}
\end{pcomments}

\pkeywords{
  quantifier
  predicate
  predicate_calculus
  domain
  domain_of_discourse
  translating_english_statements
}

%%%%%%%%%%%%%%%%%%%%%%%%%%%%%%%%%%%%%%%%%%%%%%%%%%%%%%%%%%%%%%%%%%%%%
% Problem starts here
%%%%%%%%%%%%%%%%%%%%%%%%%%%%%%%%%%%%%%%%%%%%%%%%%%%%%%%%%%%%%%%%%%%%%

\begin{problem}
Translate the following sentence into a predicate formula:
\begin{quote}
There is a student who has emailed exactly two other people in the class,
besides possibly herself.
\end{quote}

The domain of discourse should be the set of students in the class; in
addition, the only predicates that you may use are
\begin{itemize}
\item equality, and
\item $E(x,y)$, meaning that ``$x$ has sent e-mail to $y$.''
\end{itemize}

\begin{solution}
A good way to begin tackling this problem is by working
``top-down'' to translate the successive parts of the sentence.  First of
all, our formula must be of the form
\[
\exists x. P(x)
\]
where $P(x)$ should be a formula that says that ``student $x$ has
e-mailed  exactly two other people in the class, besides possibly
herself''.

One way to write $P(x)$ is to give names, say $y$ and $z$, to the two
students whom $x$ has emailed.  So we translate $P(x)$ as ``besides $x$,
there are two students, $y$ and $z$, and \dots'':
\[
\exists y,z.\ x \neq y \QAND x \neq z \QAND y \neq z \QAND \dots
\]
``$x$ has emailed both $y$ and $z$, and \dots'':
\[
E(x,y) \QAND E(x,z) \QAND \dots
\]
``if $x$ has emailed somebody, it's either $x$, $y$ or $z$.'':
\[
\forall s.\ E(x,s) \QIMP (s=x \lor s=y \lor s=z).
\]
Putting these together, we get:
\[\begin{array}{rll}
P(x)  \eqdef &\quad \exists y,z. & x \neq y \QAND x \neq z \QAND y \neq z\ \QAND\\
             &   & E(x,y) \QAND E(x,z)\ \QAND\\
             &   & [\forall s. E(x,s)  \QIMP (s=x \lor s=y \lor s=z)]
\end{array}\]
\end{solution}

\end{problem}

%%%%%%%%%%%%%%%%%%%%%%%%%%%%%%%%%%%%%%%%%%%%%%%%%%%%%%%%%%%%%%%%%%%%%
% Problem ends here
%%%%%%%%%%%%%%%%%%%%%%%%%%%%%%%%%%%%%%%%%%%%%%%%%%%%%%%%%%%%%%%%%%%%%

 \endinput
