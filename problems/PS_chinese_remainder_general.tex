\documentclass[problem]{mcs}

\begin{pcomments}
  \pcomment{PS_chinese_remainder_general}
  \pcomment{same as CP_chinese_remainder_general but has restatement of Chinese remainder with two}
  \pcomment{by ARM 3/1/11}
\end{pcomments}

\pkeywords{
  prime
  relatively_prime
  number_theory
  modular_arithmetic
  chinese_remainder
  remainder
  parallel
  residue
}

%%%%%%%%%%%%%%%%%%%%%%%%%%%%%%%%%%%%%%%%%%%%%%%%%%%%%%%%%%%%%%%%%%%%%
% Problem starts here
%%%%%%%%%%%%%%%%%%%%%%%%%%%%%%%%%%%%%%%%%%%%%%%%%%%%%%%%%%%%%%%%%%%%%

\begin{problem}
\inhandout{The \term{Chinese Remainder Theorem} asserts that if
  $a, b$ are relatively prime and greater than 1, then for all $m,n$,
  there is a \emph{unique} $x \in [0,ab)$ such that
  \begin{align*}
   x &\equiv m \pmod a,\\
   x &\equiv n \pmod b.
  \end{align*}
This was proved in Problem~\bref{CP_chinese_remainder}.
}

The general version of the Chinese Remainder theorem
\inbook{(Problem~\bref{CP_chinese_remainder})} extends to more than two
relatively prime moduli.  Namely,
\begin{theorem*}[General Chinese Remainder]
Suppose $a_1,\dots,a_k$ are integers greater than 1 and each is
relatively prime to the others.  Let $n \eqdef a_1 \cdot a_2 \cdots
a_k$.  Then for any integers $m_1,m_2,\dots,m_k$, there is a unique $x
\in [0,n)$ such that
\[
x \equiv m_i \pmod{a_i},
\]
for $1 \le i \le k$.
\end{theorem*}

The proof is a routine induction on $k$ using a fact that follows
immediately from unique factorization: if a number is relatively prime
to some other numbers, then it is relatively prime to their product.

Now suppose an $n$-bit number, $N$, was a product of relatively prime
$k$-bit numbers, where $n$ was big, but $k$ was small enough to be
handled by cheap and available arithmetic hardware units.  Suppose a
calculation requiring a large number of additions and multiplications
modulo $N$ had to be performed starting with some small set of $n$-bit
numbers.  For example, suppose we wanted to compute
\[
\rem{\ (x-3)^{110033}((y+7)^{27123} - z^{4328})\ }{N}
\]
which would require several dozen $n$-bit operations starting from the
three numbers $x,y,z$.

Doing a multiplication or addition modulo $N$ directly requires
breaking up the $n$-bit numbers $x,y,z$ and all the intermediate
results of the mod $N$ calculation into $k$-bit pieces, using the
hardware to perform the additions and multiplications on the pieces,
and then reassembling the $k$-bit results into an $n$-bit answer after
each operation.  Suppose $N$ was a product of $m$ relatively prime
$k$-bit numbers.

Explain how the General Chinese Remainder Theorem offers a far more
efficient approach to performing the required operations.

\begin{solution}
Suppose $N= a_1\cdot a_2 \cdots a_m$ where the $a_i$'s are relatively
prime $k$-bit numbers.  Represent any $n$-bit numbers, $x$, as a
vector
\[
(\rem{x}{a_1}, \rem{x}{a_2},\dots, \rem{x}{a_m}).
\]

Notice that if $x$ is represented by $(x_1,x_2,\dots,x_m)$ and likewise $y$ by 
$(y_1,y_2,\dots,y_m)$ then
$x+y$ is represented by
\[
(x_1+y+1,x_2+y+2, \dots, x_m+y+m),
\]
and likewise for $x \cdot y$.  So with this representation, additions
and multiplications of $n$-bit numbers can be performed by the
hardware units without need to break apart and reassemble the $n$-bit
numbers to perform each operation.  Moreover, the $k$-bit operations
can be performed \emph{in parallel}.  This is sometimes referred to as
\term{residue arithmetic}.

Doing residue arithmetic involves an initial cost of converting $x$,
$y$, and $z$ into the vectors of their remainders, but which requires
$m$ different GCD computations with $n$-bit numbers for each of $x$,
$y$, and $z$.  When the vector representation of the final value, $v$,
is found, it must be converted back into an $n$-bit representation,
say using the method from the proof of the chinese remainder theorem;
this requires another $m$ different extended-GCD computations with
$n$-digit numbers.  But these conversion are one time only for $x$,
$y$ and $z$, so their costs will be outweighed by the savings in doing
a large number of mod $N$ operations on the vector representations of
$x$, $y$ and $z$.

\end{solution}

\begin{staffnotes}
This problem started out aiming to ask for an induction proof of the
General Chinese Remainder Theorem, assuming the Chinese Remainder
Theorem for $k=2$.  Most of the proof appears below, but it turned out
to make a poor problem: the proof is easy but seems very cumbersome to
write out.

\begin{proof}
The proof is By induction on $k \ge 2$, with induction hypothesis,
$P(k)$ being the Theorem as stated.

\textbf{base case} ($k =2$): Given by the $a,b$-version stated at the
start.

\textbf{inductive step}: Suppose $a_1,\dots,a_k, a_{k+1}$ satisfy the
conditions for $P(k+1)$ and let $n_{k+1} \eqdef a_1 \cdot a_2 \cdots
a_k cdot a_{k+1} = n \cdot a_{k+1}$.  Then given $m_1,\dots,
m_k,m_{k+1}$, we want to prove there is a unique $x \in [0,n_{k+1})$
  such that
\begin{equation}\label{aemiaik+1}
x \equiv m_i \pmod{a_i},\quad \text{ for } 1 \le i \le k+1.
\end{equation}
Now by induction hypothesis, there is a unique
$x_k \in [0,n)$ satisfying the first $k$ congruences.  But $n$, and
  $a_{k+1}$ are relatively, so by the original Chinese Remainder
  Theorem, there is a unique $x_{k+1} \in [0,na_{k+1})$ such that
\begin{align*}
x_{k+1} \equiv x_k \pmod n,\\
x_{k+1} \equiv m_{k+1} \pmod{a_{k+1}}.
\end{align*}
But for any $c,d >$, if follows immediately from definition of congruence that
\[
y \equiv m \pmod{cd} \qimplies y \equiv m \pmod{c}.
\]
So in particular,
\[
x_{k+1} \equiv x_k \equiv m_i \pmod{a_i}
\]
for $1 \le i \le k$.  Hence, $x_{k+1}$ satisfies~\eqref{aemiaik+1}.


UNIQUENESS ARGUMENT TBA.

\end{proof}

\end{staffnotes}

\end{problem}

%%%%%%%%%%%%%%%%%%%%%%%%%%%%%%%%%%%%%%%%%%%%%%%%%%%%%%%%%%%%%%%%%%%%%
% Problem ends here
%%%%%%%%%%%%%%%%%%%%%%%%%%%%%%%%%%%%%%%%%%%%%%%%%%%%%%%%%%%%%%%%%%%%%

\endinput
