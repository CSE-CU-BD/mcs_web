\documentclass[problem]{mcs}

\begin{pcomments}
    \pcomment{TP_only_and_or_not}
    \pcomment{by ARM 2/5/11}
\end{pcomments}

\begin{problem}
A half dozen different operators may appear in propositional formulas,
but just \QAND, \QOR, and \QNOT\ are enough to do the job.  That is
because each of the operators is equivalent to a simple formula using
only these three operators.  For example, $A \QIMP B$ is equivalent to
$\QNOT(A) \QOR B$.  So all occurences of $\QIMP$ in a formula can be
replaced using just \QNOT\ and \QOR.

\bparts

\ppart Write formulas using only \QAND, \QOR, \QNOT\ that are
equivalent to each of $A \QIFF B$ and $A \QXOR B$.  Conclude that every
propositional formula is equivalent to an \QAND-\QOR-\QNOT\ formula.

\begin{solution}
\begin{align*}
A \QIFF B & \text{ is equivalent to } (A \QAND B) \QOR (\QNOT(A) \QAND \QNOT(B))\\
A \QXOR B & \text{ is equivalent to } (A \QAND \QNOT(B)) \QOR (\QNOT(A) \QAND B)
\end{align*}

\end{solution}

\ppart
Explain why you don't even need \QAND.

\begin{solution}
By DeMorgan's Law and double negation,
\[
A \QAND B \text{ is equivalent to } \QNOT(\QNOT(A) \QOR \QNOT(B)).
\]
\end{solution}

\ppart Explain how to get by with the single operator \QNAND\ where $A
\QNAND B$ is equivalent by definition to $\QNOT(A \QAND B)$.

\begin{solution}
By the previous parts, it's enough to express \QNOT\ and \QAND\ using
\QNAND.  To start $\QNOT(P)$ is equivalent to $P \QNAND P$.  Then $P \QAND
Q$ is equivalent to $\QNOT(P \QNAND Q)$ which in turn is equivalent to
$(P \QNAND Q) \QNAND (P \QNAND Q)$.
\end{solution}

\eparts

\end{problem}

\endinput
