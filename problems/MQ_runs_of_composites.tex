\documentclass[problem]{mcs}

\begin{pcomments}
  \pcomment{MQ_runs_of_composites}
  \pcomment{adapted from CP_runs_of_composites by kazerani}
  \pcomment{from: F05-PS5-P5}
\end{pcomments}

\pkeywords{
composite
prime
factorial
}

%%%%%%%%%%%%%%%%%%%%%%%%%%%%%%%%%%%%%%%%%%%%%%%%%%%%%%%%%%%%%%%%%%%%%
% Problem starts here
%%%%%%%%%%%%%%%%%%%%%%%%%%%%%%%%%%%%%%%%%%%%%%%%%%%%%%%%%%%%%%%%%%%%%


\begin{problem}
For $n > 2$, what is the maximum number of primes that may be in the 
sequence $(n!, n!+1, \ldots, n!+n)$? 

\begin{solution}

  Since $n > 2$, therefore $n! > n$.  Now let $k\in\nngint$ and $1 < k 
  \leq n$.

  Since $k \divides n!$ and $1 < k \leq n < n!$, therefore $n!$ is 
composite. 
  
  Since $k \divides k$ as well, therefore $k \divides (n! + k)$.  Together with
  $1 < k < n! + k$, this implies that $n!+k$ is composite.
  
  Thus, $n!$, along with $n!+2, n!+3, n!+4, \dots, n!+n$, must be composite.  

  Therefore, of the terms in the sequence $(n!, n!+1, \dots, n!+n)$, 
  only one, namely $n!+1$, can possibly be prime.  It is prime, for instance, 
  when $n=3$.

\end{solution}

\end{problem}


%%%%%%%%%%%%%%%%%%%%%%%%%%%%%%%%%%%%%%%%%%%%%%%%%%%%%%%%%%%%%%%%%%%%%
% Problem ends here
%%%%%%%%%%%%%%%%%%%%%%%%%%%%%%%%%%%%%%%%%%%%%%%%%%%%%%%%%%%%%%%%%%%%%

\endinput

