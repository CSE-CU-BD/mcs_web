\documentclass[problem]{mcs}

\begin{pcomments}
  \pcomment{MQ_runs_of_composites}
  \pcomment{adapted from CP_runs_of_composites by kazerani}
  \pcomment{from: F05-PS5-P5}
\end{pcomments}

\pkeywords{
composite
prime
factorial
}

%%%%%%%%%%%%%%%%%%%%%%%%%%%%%%%%%%%%%%%%%%%%%%%%%%%%%%%%%%%%%%%%%%%%%
% Problem starts here
%%%%%%%%%%%%%%%%%%%%%%%%%%%%%%%%%%%%%%%%%%%%%%%%%%%%%%%%%%%%%%%%%%%%%


\begin{problem}
For $n\geq 2$, what is the maximum number of primes that may be in the 
sequence $(n!, n!+1, \ldots, n!+n)$? 

\begin{solution}

  Let $k$ be some natural number such that $1 < k \leq n$.  Now, $k 
  \mid n!$, so $n!$ is composite.  Since $k \mid k$ as well, therefore 
  $k \mid (n! + k)$. Thus, the numbers $n!, n!+2, n!+3, n!+4, 
  \ldots, n!+n$ must all be composite.  

  Therefore, of the terms in the sequence $(n!, n!+1, \ldots, n!+n)$, 
  only one, $n!+1$, can possibly be prime.  It is prime, for instance, 
  when $n=2$.

\end{solution}

\end{problem}


%%%%%%%%%%%%%%%%%%%%%%%%%%%%%%%%%%%%%%%%%%%%%%%%%%%%%%%%%%%%%%%%%%%%%
% Problem ends here
%%%%%%%%%%%%%%%%%%%%%%%%%%%%%%%%%%%%%%%%%%%%%%%%%%%%%%%%%%%%%%%%%%%%%

\endinput

