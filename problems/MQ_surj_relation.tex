\documentclass[problem]{mcs}

\begin{pcomments}
  \pcomment{MQ_surj_relation}
  \pcomment{from: CP_surj_relation}
\end{pcomments}

\pkeywords{
  relations
  functions
  surjections
}

%%%%%%%%%%%%%%%%%%%%%%%%%%%%%%%%%%%%%%%%%%%%%%%%%%%%%%%%%%%%%%%%%%%%%
% Problem starts here
%%%%%%%%%%%%%%%%%%%%%%%%%%%%%%%%%%%%%%%%%%%%%%%%%%%%%%%%%%%%%%%%%%%%%

\begin{problem}
Define a \term{surjection relation} $\surj$ on sets by the rule
\begin{definition*}
  $A \surj B$ iff there is a surjective \textbf{function} from $A$ to $B$.
\end{definition*}

\label{surjsurj} Prove that if $A \surj B$ and $B \surj C$, then $A \surj C$.

\begin{solution}
By definition of $\surj$, there are surjective functions,
$F:A \to B$ and $G:B \to C$.

Let $H \eqdef G \compose F$ be the function equal to the composition of
$G$ and $F$, that is
\[
H(a) \eqdef G(F(a)).
\]
We show that $H$ is surjective, which will complete the proof.  So suppose
$c \in C$.  Then since $G$ is a surjection, $c = G(b)$ for some $b \in B$.
Likewise, $b = F(a)$ for some $a \in A$.  Hence $c = G(F(a)) = H(a)$,
proving that $c$ is in the range of $H$, as required.
\end{solution}

\end{problem}

%%%%%%%%%%%%%%%%%%%%%%%%%%%%%%%%%%%%%%%%%%%%%%%%%%%%%%%%%%%%%%%%%%%%%
% Problem ends here
%%%%%%%%%%%%%%%%%%%%%%%%%%%%%%%%%%%%%%%%%%%%%%%%%%%%%%%%%%%%%%%%%%%%%

\endinput
