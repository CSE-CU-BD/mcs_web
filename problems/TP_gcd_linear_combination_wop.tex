\documentclass[problem]{mcs}

\begin{pcomments}
  \pcomment{TP_gcd_linear_combination_wop}
  \pcomment{DO NOT USE: mistaken & subsumed by PS_gcd_union, ARM 3/16/16}
  \pcomment{ARM 3/17/13, edit to mention $n$ 2/12/14}
  \pcomment{break out PS_gcd_union ARM 3/15/16}
\end{pcomments}

\pkeywords{
  well_ordering
  gcd
  linear_combination
}

%%%%%%%%%%%%%%%%%%%%%%%%%%%%%%%%%%%%%%%%%%%%%%%%%%%%%%%%%%%%%%%%%%%%%
% Problem starts here
%%%%%%%%%%%%%%%%%%%%%%%%%%%%%%%%%%%%%%%%%%%%%%%%%%%%%%%%%%%%%%%%%%%%%

\begin{problem}
Use the Well Ordering Principle to prove that the gcd of a $n$
integers is an integer linear combination of these integers.

You may assume that the gcd of two integers is an integer linear
combination of them, which was proved in \inhandout{the
  text}\inbook{Theorem~\bref{gcd_is_lin_thm}}.  You may also assume
the easily verified fact\inbook{ (Problem~\bref{})} that
\begin{equation}
\gcd(\set{a} \union B) = \gcd(a,\gcd(B)),\tag{*}
\end{equation}  
for any integer $a$ and finite set $B$ of integers.

Be sure to define and clearly label the set of counterexamples that
you are assuming is nonempty.

\examspace

\begin{solution}
Let
\begin{align*}
P(n)  \eqdef & \text{the gcd of any set of $n$ of integers is a}\\
             & \text{linear combination of the integers in the set.}\\
C \eqdef & \set{n \geq 1 \suchthat \QNOT(P(n))},
\end{align*}
and assume for the sake of contradiction that the set $C$ of
counterexamples is not empty.

By the WOP, there is a least integer $m \in C$.  So there must be 
integers $ a_1, a_2,\dots, a_{m}$ such that $\gcd(a_1,a_2,\dots, a_m)$ is
not a linear combination of $a_1,a_2,\dots,a_m$.

Since $\gcd(\set{a_1}) = 1 \cdot a_1$, we know that $m > 1$, so $m-1 \geq 1$.
Since $m$ is the smallest element of $C$, it follows that
\[
\gcd(a_1,\dots,a_{m-1}) = s_1a_1 + s_2a_2 + \cdots + s_{m-1}a_{m-1}.
\]
Now
\begin{align*}
\lefteqn{\gcd(a_1,a_2,\dots,a_{m})}\\
  & = \gcd(\gcd(a_1,a_2,\dots,a_{m-1}), a_{m})
       & \text{(by~(*))}\\
  & = s \cdot \gcd(a_1,a_2,\dots,a_{m-1}) + t \cdot a_{m} \\
  &     \qquad \text{ for some } s,t \in \integers 
           &  \text{(the two element case)}\\

  & = s(s_1a_1 + s_2a_2 + \cdots + s_na_{m=1}) + t a_{m}\\
  & = (ss_1)a_1 + (ss_2)a_2 + \cdots + (ss_{m-1})a_{m-1} + t a_{m}.
\end{align*}
This shows that $\gcd(a_1,a_2,\dots,a_{m})$ is also a linear
combination of $a_1,a_2,\dots,a_{m}$, contradicting the choice of $m$.

The contradiction implies that $C$ must be empty, proving that $P(n)$
holds for all $n \geq 1$.
\end{solution}
  
\end{problem}

%%%%%%%%%%%%%%%%%%%%%%%%%%%%%%%%%%%%%%%%%%%%%%%%%%%%%%%%%%%%%%%%%%%%%
% Problem ends here
%%%%%%%%%%%%%%%%%%%%%%%%%%%%%%%%%%%%%%%%%%%%%%%%%%%%%%%%%%%%%%%%%%%%%

\endinput
