\documentclass[problem]{mcs}
\begin{pcomments}
  \pcomment{MQ_4_way_dependent_ions}
  \pcomment{rephrased from MQ_4_way_dependent by DIG 11/17/17}
  \pcomment{edited ARM 11/20/17}
\end{pcomments}

\pkeywords{
  independence
  pairwise
  expectation
  indicator
}

\begin{problem}
Suppose there are 4 ions in a crystal, laid out in the corners of a square
with corners $1$, $2$, $3$ and $4$. \iffalse
 as in Figure~\ref{fig:4desks}.

\begin{figure}

%\graphic{4_rectangles}

\caption{Arrangement of four ions.}

\label{fig:4desks}

\end{figure}\fi

Each corner is occupied by a sodium ion with probability $p>0$ or a
chloride ion with probability $q \eqdef 1-p > 0$.  A sodium ion and a
chloride ion \emph{bond} when they occupy adjacent corners of the
square.  Let $I_{12},I_{23},I_{34},I_{41}$ be the indicator variables
for a bonded pair being at the subscripted locations.

\bparts

\ppart\label{ei12} What is the $\expect{I_{12}}$?

\begin{solution}
$\mathbf{2pq}.$
\[
\expect{I_{12}} = \pr{I_{12} = 1} = pq+qp.
\]
\end{solution}

\exambox{0.5in}{0.5in}{0.0in}

\ppart\label{Ebpq} What is the expected number of bonded pairs in terms of $p$ and $q$?

\begin{solution}
$\mathbf{8pq}.$

The number of pairs is $I_{12}+I_{23}+I_{34}+I_{41}$.  By
part~\eqref{ei12} and symmetry, the expectation of each of these
indicator variables is $2pq$, so by linearity of expectation, the
expectation of the sum $4(2pq)$.
\end{solution}

\exambox{0.5in}{0.5in}{0.0in}


\ppart Prove that if $p = 1/2$ then the events $I_{12} = 1$ and $I_{23} =
1$ are independent.

\examspace[2in]

\begin{solution}
If $p = q = 1/2$ then $\pr{I_{12} = 1} = \pr{I_{23} = 1} = 1/2$ and
$\pr{I_{12} = 1 \QAND\ I_{23} = 1}$ can be calculated from the fact that
only C-S-C and S-C-S are possible when both pairs are bonded.  In
that case, we have $\pr{I_{12} = 1 \QAND\ I_{23} = 1} = 2/8 = 1/4 =
\pr{I_{12} = 1} \cdot \pr{I_{12} = 1}$.
\end{solution}

\ppart Prove conversely that if the events $I_{12} = 1$ and $I_{23} =
1$ are independent, then $p = 1/2$.

\begin{staffnotes}
\hint Working from the definition of independence, set up an equation
and solve.
\end{staffnotes}

\examspace[4in]

\begin{solution}
We can again compare $\pr{I_{12} = 1 \QAND\ I_{23} = 1}$ and $\pr{I_{12}
  = 1}\cdot \pr{I_{23} = 1}$.

As in the previous part, $I_{12} = 1 \QAND\ I_{23} = 1$ only happen when
we have a pattern of C-S-C or S-C-S for ions 1 2 and 3
respectively. These occur with total probablity $p^2q + pq^2$. On the
other hand, $I_{12}$ happens with probability $2pq$ total, accounting
for the two patterns possible, S-C and C-S. Hence, $I_{12}$ and
$I_{23}$ are independent iff $p^2q + pq^2 = pq(p+q) = 4p^2q^2$. By
manipulating the expression we get $p+q = 4pq$. Recall $p+q =
1$. Hence, we are dealing with $1 = 4p - 4p^2$.  The equation can be
factored into $(2p - 1)^2 = 0$, yielding $p = 1/2$.
\end{solution}

\ppart Let $N$ be the number of bonded pairs.   What is the \pdf\ of $N$?

\begin{solution}
\[
p_N(0) = p^4+q^4, p_N(1) = 0, p_N(2) = 4(p^3q + pq^3 + p^2q^2), p_N(3) = 0, p_N(4) = 2p^2q^2
\]
\end{solution}

\begin{staffnotes}
Use the pdf of $N$ to calculate $\expect{N}$ and verify that the
result matches part~\eqref{Ebpq}.
\[
\expect{N} = 2(4(p^3q + pq^3 + p^2q^2)) + 4(2p^2q^2) = 8pq(p^2+q^2+pq+ pq) = 8pq(p+q)^2 = 8pq.
\]
\end{staffnotes}

\eparts
\end{problem}

\endinput
