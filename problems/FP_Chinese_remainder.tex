\documentclass[problem]{mcs}

\begin{pcomments}
  \pcomment{FP_Chinese_remainder}
  \pcomment{adapted from: CP_chinese_remainder}
  \pcomment{used to appear
    as first part of PS_Euler_function_multiplicativity}
  \pcomment{adapted by: Kazerani 5/11, ARM 5/13/11}
\end{pcomments}

\pkeywords{
  prime
  relatively_prime
  number_theory
  modular_arithmetic
  chinese_remainder
  remainder
}

%%%%%%%%%%%%%%%%%%%%%%%%%%%%%%%%%%%%%%%%%%%%%%%%%%%%%%%%%%%%%%%%%%%%%
% Problem starts here
%%%%%%%%%%%%%%%%%%%%%%%%%%%%%%%%%%%%%%%%%%%%%%%%%%%%%%%%%%%%%%%%%%%%%

\begin{problem}

\iffalse
If $a$ and $b$ are relatively prime and greater than $1$, then for any
$x$,
\begin{equation}
\brac{x \equiv 0 \bmod a \ \QAND\ x \equiv 0 \bmod b} \qimplies x
\equiv 0 \bmod{ab}.\label{0cong}
\end{equation}
Using this,
\fi

Prove that if $a$ and $b$ are relatively prime integers and greater
than $1$, then
\begin{equation}
\brac{x \equiv x^\prime \bmod{a}\ \QAND\ x \equiv x^\prime \bmod b} \qimplies
x \equiv x^\prime \bmod{ab}.\label{primecong}
\end{equation}

\begin{staffnotes}
If needed suggest ``Look at $x^\prime - x$.''
\end{staffnotes}

\examspace[4in]


\begin{solution}
$x \equiv x^\prime \bmod{a}$ iff $a \divides (x - x^\prime)$, and
$x \equiv x^\prime \bmod{b}$ iff $b \divides (x - x^\prime)$.  But
$a,b$ are relatively prime, so $ab \divides (x - x^\prime)$
iff $\brac{x \equiv x^\prime \bmod{a}\ \QAND\ x \equiv x^\prime \bmod b}$.
That is,
\[
\brac{a \divides (x - x^\prime) \QAND b \divides (x - x^\prime)} \qiff
x \equiv x^\prime \bmod{ab}.
\]
\end{solution}

\end{problem}

%%%%%%%%%%%%%%%%%%%%%%%%%%%%%%%%%%%%%%%%%%%%%%%%%%%%%%%%%%%%%%%%%%%%%
% Problem ends here
%%%%%%%%%%%%%%%%%%%%%%%%%%%%%%%%%%%%%%%%%%%%%%%%%%%%%%%%%%%%%%%%%%%%%

\endinput
