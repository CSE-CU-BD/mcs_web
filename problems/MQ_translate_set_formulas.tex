\documentclass[problem]{mcs}

\begin{pcomments}
    \pcomment{MQ_translate_set_formulas}
    \pcomment{based on TP_basic_set_formulas}
    \pcomment{ARM 3/17/13}
\end{pcomments}

\pkeywords{
  logic
  sets
  set_theory
  predicate
  formula
  subset
  power_set
  union
}

\begin{problem}

Nine assertions about sets are bulleted below, with variables
$x,y,z\dots$ ranging over sets.  There are eleven predicate formulas
that express some of these assertions.  Write the number of the
formula next to the bulleted assertion it expresses.  For example, you
should write ``2'' next to the first assertion, since formula (2)
expresses the assertion that $x = y$.  More than one formula may
express the same bulleted assertion.

\begin{quote}


\begin{itemize}
\item $x = y$. \hfill\examrule[1in]

\item $x = \emptyset$. \hfill\examrule[1in]

\item $x = \set{y,z}$. \hfill\examrule[1in]

\item $x \subseteq y$. \hfill\examrule[1in]

\item $x = y \union z$. \hfill\examrule[1in]

\item $x = y - z$. \hfill\examrule[1in]

\item $x = \power(y)$. \hfill\examrule[1in]

\item $\card{x} \leq 3$. \hfill\examrule[1in]

\item $\card{x} > 3$. \hfill\examrule[1in]

\end{itemize}

\end{quote}


\begin{enumerate}
\item $\forall z.\, \QNOT(z\in x)$.   %x empty

\item $\forall z.\, (z \in x\ \QIFF\ z \in y)$.  %x=y

\item $\forall w.\, w \in x \QIFF (w=y \QOR\ w = z)$. %x = {y,z}

\item $\forall z.\, z \in x\ \QIMPLIES\ z \in y$.  %x \subseteq y

\item $\forall w.\, w \in x \QIFF (w \in y\ \QOR\ w \in z)$.  %x = y U z

\item $\forall w.\, w \in x\ \QIFF\  (w \in y\ \QAND\ \QNOT(w \in z))$. %x = y - z

\item $\exists z.\, (y = x \union z)$.  %$x \subseteq y$

\item $(x - y) = \emptyset$. %$x \subseteq y$

\item $\forall z.\, z \in x \QIFF z \subseteq y$. % $x = \power(y)$.

\item $\exists x_1,x_2,x_3.\, \forall z.\, z \in x \QIMPLIES (z = x_1 \QOR z = x_2 \QOR z = x_3)$.
 %$\card{x} \leq 3$

\item
$\forall x_1,x_2,x_3.\, \exists z.\, z \in x \QAND\ z \neq
  x_1\ \QAND\ z \neq x_2\ \QAND\ z \neq x_3$  %$\card{x} > 3$.

\end{enumerate}

\end{problem}

\endinput
