\documentclass[problem]{mcs}

\begin{pcomments}
  \pcomment{FP_hot_cows_markov}
  \pcomment{variant of CP_cold_cows_Markov}
  \pcomment{ARM May 19, 2013}
  \pcomment{Shortened by AC 5/17/15}
\end{pcomments}

\pkeywords{
  average
  Markov_bound
  deviation
  sample_space
  outcome
  probability_space
}

%%%%%%%%%%%%%%%%%%%%%%%%%%%%%%%%%%%%%%%%%%%%%%%%%%%%%%%%%%%%%%%%%%%%%
% Problem starts here
%%%%%%%%%%%%%%%%%%%%%%%%%%%%%%%%%%%%%%%%%%%%%%%%%%%%%%%%%%%%%%%%%%%%%


\begin{problem}

\begin{staffnotes}
(a) 4 pts, (b) 3
\end{staffnotes}

A herd of cows is stricken by an outbreak of \emph{hot cow disease}.
The disease raises the normal body temperature of a cow, and a cow
will die if its temperature goes above 90 degrees.  The disease
epidemic is so intense that it raised the average temperature of the
herd to $120$ degrees.  Body temperatures as high as $140$ degrees,
\textbf{but no higher}, were actually found in the herd.

Use Markov's Bound\inbook{~\bref{markovthm}} to
prove that at most 2/5 of the cows could have survived.

%\hint Let $T$ be the temperature of a random cow.

\examspace[2in]

\begin{solution}
Let $T$ be the temperature of a random cow.

\iffalse

Applying Markov's Bound to $T$:
  \[
  \prob{T \leq 90} \leq \frac{\expect{T}}{90} = \frac{85}{90} =
  \frac{17}{18}\, .
  \]
  But $17/18 > 3/4$, so this bound is not good enough.
\fi

We apply Markov's Bound to $140-T$:
\begin{align*}
\prob{T \leq 90}
 & = \prob{140 - T \geq 140 -90}\\
 & \leq \frac{\expect{140-T}}{140-90} & \text{(Markov)}\\
 & = \frac{140-120}{140-90} & \text{(linearity of $\expect{}$)}\\
 & = 20/50 = 2/5.
\end{align*}

Since cows were chosen randomly, this implies that at most 2/5 of the
herd can have a temperature $\leq 90$.

\end{solution}

\end{problem}

%%%%%%%%%%%%%%%%%%%%%%%%%%%%%%%%%%%%%%%%%%%%%%%%%%%%%%%%%%%%%%%%%%%%%
% Problem ends here
%%%%%%%%%%%%%%%%%%%%%%%%%%%%%%%%%%%%%%%%%%%%%%%%%%%%%%%%%%%%%%%%%%%%%

\endinput
