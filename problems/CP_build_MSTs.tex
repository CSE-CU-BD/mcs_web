\documentclass[problem]{mcs}

\begin{pcomments}
  \pcomment{CP_build_MSTs}
  \pcomment{ARM 10/27/11}
  \pcomment{F11.cp9m}
  \pcomment{do not use again until figure is inserted (3 copies)}
\end{pcomments}

\pkeywords{
spanning_tree
weighted_tree
minimum_weight
MST
}

%%%%%%%%%%%%%%%%%%%%%%%%%%%%%%%%%%%%%%%%%%%%%%%%%%%%%%%%%%%%%%%%%%%%%
% Problem starts here
%%%%%%%%%%%%%%%%%%%%%%%%%%%%%%%%%%%%%%%%%%%%%%%%%%%%%%%%%%%%%%%%%%%%%

\begin{problem}
Let $G$ be a $4 \times 4$ grid with vertical and horizontal
edges between neighboring vertices.  Formally,
\[
\vertices{G} = [0,3]^2 \eqdef \set{(k,j) \suchthat 0 \leq k,j \leq 3}.
\]
Letting $h_{i,j}$ be the horizontal edge $\edge{(i,j)}{(i+1,j)}$ and
$v_{j,i}$ be the vertical edge $\edge{(j,i)}{(j,i+1)}$ for $i\in[0,2],
j \in [0,3]$.   The weights of these edges are
\begin{align*}
w(h_{i,j}) & \eqdef  \frac{4i+j}{100},\\
w(v_{j,i}) & \eqdef 1+\frac{i+4j}{100}.
\end{align*}

(A picture of $G$ would help; you might like to draw one.)

\bparts

\ppart Construct a minimum weight spanning tree (MST) for $G$ by
initially selecting the minimum weight edge, and then successively
selecting the minimum weight edge that does not create a cycle with
the previously selected edges.  Stop when the selected edges form a
spanning tree of $G$.  (This is Kruskal's MST algorithm.)


\begin{solution}

The edges are in the order that they are constructed by the given algorithm.

Answer: $h_{0,0}h_{0,1}h_{0,2}h_{0,3}h_{1,0}h_{1,1}h_{1,2}h_{1,3}h_{2,0}h_{2,1}h_{2,2}h_{2,3}v_{0,0}v_{0,1}v_{0,2}$

\end{solution}

\ppart Grow an MST for $G$ starting with the tree consisting of the
single vertex $(1,2)$ and successively adding the minimum weight edge
with exactly one endpoint in the tree.  Stop when the tree spans $G$.
(This is Prim's MST algorithm.)


\begin{solution}

Answer: $h_{0,2}h_{1,2}h_{2,2}v_{0,1}h_{0,1}h_{1,1}h_{2,1}v_{0,0}h_{0,0}h_{1,0}h_{2,0}v_{0,2}h_{0,3}h_{1,3}h_{2,3}$

\end{solution}


\ppart Grow an MST for $G$ by treating the vertices $(0,0), (0,3),
(2,3)$ as single vertex trees and then successively adding, for each
tree in parallel, the minimum weight edge among the edges with one
endpoint in the tree.  Continue until the trees merge and form a
spanning tree of $G$.  (This is 6.042's parallel MST algorithm.)

\begin{solution}

Done in parallel:

T1@(0,0): $h_{0,3}$ (merges with T2)

T2@(0,3): $h_{1,3}h_{3,3}v_{0,2}h_{0,2}h_{1,2}h_{2,2}v_{0,1}$ (merges with T3)

T3@(2,3): $h_{0,0}h_{1,0}h_{2,0}v_{0,0}h_{0,1}h_{1,1}h_{2,1}$

\end{solution}

\ppart Verify that you got the same MST each time.

\begin{solution}
They are the same ---if no mistake was made.
Problem~\bref{PS_unique_MST} explains why there is a unique MST for
any finite connected weighted graph where no two edges have the same weight.
\end{solution}

\eparts

\end{problem}

%%%%%%%%%%%%%%%%%%%%%%%%%%%%%%%%%%%%%%%%%%%%%%%%%%%%%%%%%%%%%%%%%%%%%
% Problem ends here
%%%%%%%%%%%%%%%%%%%%%%%%%%%%%%%%%%%%%%%%%%%%%%%%%%%%%%%%%%%%%%%%%%%%%

\endinput
