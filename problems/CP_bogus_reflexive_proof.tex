\documentclass[problem]{mcs}

\begin{pcomments}
  \pcomment{CP_bogus_reflexive_proof}
  \pcomment{from: S02.cp3w}
\end{pcomments}

\pkeywords{
  bogus_proof
  reflexive
  transitive
  symmetric
  equivalence_relation
}

%%%%%%%%%%%%%%%%%%%%%%%%%%%%%%%%%%%%%%%%%%%%%%%%%%%%%%%%%%%%%%%%%%%%%
% Problem starts here
%%%%%%%%%%%%%%%%%%%%%%%%%%%%%%%%%%%%%%%%%%%%%%%%%%%%%%%%%%%%%%%%%%%%%

\begin{problem} 
Find the flaw in the following bogus proof, and give a counterexample
to the claim.

\begin{falseclm*}
Suppose $R$ is a relation on a set $A$.  If $R$ is symmetric and transitive, 
then $R$ is reflexive.
\iffalse
\footnote{That is, $R$ is an \idx{equivalence relation}.}
\fi

\end{falseclm*}

\begin{bogusproof}
Let $a$ be an arbitrary element of $A$.  Let $b$ be any element of $A$
such that $a\mrel{R}b$.  Since $R$ is symmetric, it follows that
$b\mrel{R}a$.  Then since $a\mrel{R}b$ and $b\mrel{R}a$, we conclude
by transitivity that $a\mrel{R}a$.  Since $a$ was arbitrary, we have
\iffalse (by \idx{Universal Generalization}) \fi
shown that $\forall a \in A.\; a\mrel{R}a$, which means that $R$ is
reflexive.
\end{bogusproof}

\begin{solution}

The flaw is assuming that $b$ exists.  It is possible that there is an
$a \in A$ that is not related by $R$ to anything.  No such $R$ will be
reflexive.  The simplest such $R$ that is also symmetric and
transitive is the empty relation on any nonempty set $A$.  We can
easily construct other examples, such as letting $A \eqdef \set{a,b,c}$ and
\[
\graph{R_0} \eqdef \set{(c,c),(c,b),(b,c),(b,b)}.
\] 
Now $R_0$ is not reflexive because $\QNOT(a \mrel{R_0}a)$.  So $R_0$
is a counterexamples to the claim.

Note that the theorem can be fixed: $R$ restricted to its domain of
definition \emph{is} reflexive.%, and hence an equivalence relation.

\end{solution}
\end{problem} 



%%%%%%%%%%%%%%%%%%%%%%%%%%%%%%%%%%%%%%%%%%%%%%%%%%%%%%%%%%%%%%%%%%%%%
% Problem ends here
%%%%%%%%%%%%%%%%%%%%%%%%%%%%%%%%%%%%%%%%%%%%%%%%%%%%%%%%%%%%%%%%%%%%%
\endinput
