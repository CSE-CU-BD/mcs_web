\documentclass[problem]{mcs}

\begin{pcomments}
\pcomment{TP_combination_identity}
\pcomment{renamed from CP_combination_identity}
\pcomment{from: S06.ps7}
\end{pcomments}

\pkeywords{
  combinations
  binomial_coefficients
}

%%%%%%%%%%%%%%%%%%%%%%%%%%%%%%%%%%%%%%%%%%%%%%%%%%%%%%%%%%%%%%%%%%%%%
% Problem starts here
%%%%%%%%%%%%%%%%%%%%%%%%%%%%%%%%%%%%%%%%%%%%%%%%%%%%%%%%%%%%%%%%%%%%%

\begin{problem}

Prove the following identity by algebraic manipulation and by giving a
combinatorial argument:
\[
\binom{n}{r}\binom{r}{k} = \binom{n}{k}\binom{n-k}{r-k}
\]

\begin{solution}
\textbf{Algebraic:}
\begin{align*}
\binom{n}{k}\binom{n-k}{r-k}
       &= \frac{n!}{k!(n-k)!}\frac{(n-k)!}{(r-k)!(n-r) !}\\
       &= \frac{n!}{(n-r)!k!(r-k)!}\\ &= &\frac{n!r!}{r!(n-r)!k!(r-k)!}\\
       &= \frac{n!}{r!(n-r)}\frac{r!}{k!(r-k)!}\\ &= &\binom{n}{r}\binom{r}{k}
\end{align*}

\textbf{Combinatorial:} With a group of $n$ balls, I wish to pick a group
of $k$ balls and another group of $r-k$ balls.  I could do it by picking a
group of $r$ balls and then partitioning it into sets of sizes $k$ and
$r-k$.  On the other hand, I can first pick $k$ balls and then pick $r-k$
balls from the remaining $n-k$ balls.
\end{solution}

\end{problem}

\endinput
