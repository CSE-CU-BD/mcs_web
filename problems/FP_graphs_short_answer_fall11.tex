\documentclass[problem]{mcs}

\begin{pcomments}
  \pcomment{FP_graphs_short_answer_fall11}
  \pcomment{overlaps FP_graphs_short_answer}
  \pcomment{excerpted from FP_multiple_choice_unhidden}
\end{pcomments}

\pkeywords{
  isomorphism
  partial_order  
  total_order
  path-total
  asymptotic
  vertices
  stationary_distribution
  countable
}

%%%%%%%%%%%%%%%%%%%%%%%%%%%%%%%%%%%%%%%%%%%%%%%%%%%%%%%%%%%%%%%%%%%%%
% Problem starts here
%%%%%%%%%%%%%%%%%%%%%%%%%%%%%%%%%%%%%%%%%%%%%%%%%%%%%%%%%%%%%%%%%%%%%

\begin{problem} \mbox{}

\bparts

\ppart Circle all the properties below that are preserved under graph
isomorphism.

\begin{itemize}

\item The vertices can be numbered 1 through 7.

\item There is a cycle that includes all the vertices.

\item The graph remains connected if a vertex is removed.

\item There are exacty two spanning trees.

\item The $\QOR$ of two properties that are preserved under isomorphism.

%% NOT USED S11
%\item There are two degree 8 vertices. % too easy
%\item The area enclosed by a graph equals some constant, $A$.  % true for trees?
%\item There are two simple cycles that do not share any vertices. % too easy
%\item There are two connected components. % too easy
%\item The graph can be drawn such that all the edges have the same length. % huh?
%\item The graph is 4-colorable. % too easy
%\item Adding an edge between any two vertices creates a cycle. % spanning tree one better
\end{itemize}

\begin{solution}
All are preserved.
\end{solution}

% FROM: Spring07 MQ-4/6-2a
%
% COMMENTS: Reduced number, changed statements, now asking for
% counterexample. Solution incomplete. Should we replace "Lemma 3.4"
% with an \eqref?
%
% Need to make boxes (make sure they know that counterexamples req'd): 
%     true        false__________________ <-- room for counterexample
%


% ~~~~~~~~~~~~~~~~~~~~~~~~~~~~~~~~~~~~~~~~~~~~~~~~~~~~~~~~~~~~~~~~~~~
% FROM: Spring07 MQ-3/14-5
%
% COMMENTS: Reduced number, DID NOT change statements, now asking for
% counterexample.
%

\ppart For the following six statements about finite simple graphs,
circle those that are \textbf{true}, and \emph{provide
  counterexamples} for those that are \textbf{false}.

\begin{itemize}

%S11
\item Every graph has a spanning tree.  \hfill \textbf{true} \qquad
  \textbf{false} \examspace[0.8in]

\begin{solution}
\textbf{false}.  Any disconnected graph is a counterexample.
\end{solution}

\iffalse  %S11
\item Any connected subgraph of a tree is a tree.  \hfill \textbf{true} \qquad
  \textbf{false} \examspace[0.8in]

\begin{solution}
\textbf{true.}
\end{solution}

\item Any subgraph of a tree is a tree.  \hfill \textbf{true} \qquad
  \textbf{false} \examspace[0.8in]

\begin{solution}
\textbf{false.  Choose a subgraph with 2 vertices and no edges.}
\end{solution}
\fi

\item A graph with three vertices cannot have exactly one
  vertex of degree 1.  \hfill \textbf{true} \qquad
  \textbf{false} \examspace[0.8in]

\begin{solution}
\textbf{true}.  There are only two such \emph{connected} graphs: a
triangle (with no degree 1 vertices) and a length three line graph
with two degree 1 vertices.  Otherwise, the graph has a degree 0
vertex in which case the other two have the same degree.}
\end{solution}

\item Adding an edge between two nonadjacent vertices in a tree
  creates a cycle. \hfill \textbf{true} \qquad
  \textbf{false} \examspace[0.8in]

\begin{solution}
\textbf{true.}
\end{solution}

\iffalse %S11
\item The number of vertices in a tree is one less than twice the number of
  leaves.  \hfill 
\textbf{true} \qquad \textbf{false}  \examspace[0.8in]

\begin{solution}
  \textbf{false}.  This property holds for full binary trees, but not in
  general.  A tree with two vertices is a counterexample.
\end{solution}
\fi

\item The number of leaves in a tree is not equal to the number of
  non-leaf vertices.  \hfill \textbf{true} \qquad
  \textbf{false} \examspace[0.8in]

\begin{solution}
  \textbf{false}.  A line graph with 4 vertices has 2 non-leaf
  vertices and 2 leaves.
\end{solution}

%S11
\item The number of vertices in a tree is one less than the number of
  edges.  \hfill \textbf{true} \qquad \textbf{false} \examspace[0.8in]

\begin{solution}
  \textbf{false}.  This got ``edges'' and ``vertices'' reversed.
Every tree is a counterexample.
\end{solution}

\item The minimum number of edges possible in a
nonplanar 2-colorable graph is 10.
\hfill \textbf{true} \qquad \textbf{false} \examspace[0.8in]

\begin{solution}
\textbf{false.}  $K_{3,3}$ is the smallest 2-colorable (bipartite)
nonplanar graph, and it has 9 edges.  $K_5$ is nonplanar with 10
edges, but not bipartite.
\end{solution}

\iffalse
%NOT USED S11 --tricky if you haven't seen it before
\item For every, possibly infinite, connected graph, there is one (a tree) that spans it.
\hfill \textbf{true} \qquad \textbf{false} \examspace[0.8in]

\begin{solution}
false.  Any dense order, for example, $\rationals$ with
$\diredge{r}{s}$ iff $r < s$
\end{solution}
\fi

\end{itemize}

\iffalse
%S11
\ppart What is the minimum number of \textbf{vertices} possible in a
nonplanar graph?  \hfill \examrule[0.5in]

\begin{solution}
5.  $K_5$ is the smallest nonplanar graph.
\end{solution}

%S11
\ppart What is the minimum number of \textbf{edges} possible in a
nonplanar graph that is 2-colorable?  \hfill \examrule[0.5in]

\begin{solution}
9.  $K_{3,3}$ is the smallest 2-colorable nonplanar graph.
\end{solution}

%S11
\ppart A \term{sink} in a digraph is a vertex with no edges leaving
it.  Circle whichever of the following assertions are true of
\idx{stationary distributions} on finite digraphs with exactly two sinks:

\begin{itemize}

\item there may not be any

\item there may be a unique one

\item there are exacty two

\item there may be a countably infinite number

\item there may be a uncountable number

\item there always is an uncountable number

\end{itemize}

\begin{solution}
The first four choices are false, and the last two are true.
That's because a distribution in which one sink has probability $r \in
[0,1] \subseteq \reals$ and the other sink has probability $1-r$ is
stable, and there are an uncountable number of real numbers in $[0,1]$.
\end{solution}
\fi

\eparts

\end{problem}

\endinput
