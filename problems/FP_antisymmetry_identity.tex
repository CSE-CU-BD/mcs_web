\documentclass[problem]{mcs}

\begin{pcomments}
  \pcomment{FP_antisymmetry_identity}
  \pcomment{by ARM 10/27/15}
\end{pcomments}

\pkeywords{
  symmetric
  antisymmetric
}

%%%%%%%%%%%%%%%%%%%%%%%%%%%%%%%%%%%%%%%%%%%%%%%%%%%%%%%%%%%%%%%%%%%%%
% Problem starts here
%%%%%%%%%%%%%%%%%%%%%%%%%%%%%%%%%%%%%%%%%%%%%%%%%%%%%%%%%%%%%%%%%%%%%

\begin{problem}
Let $D$ be a set of size $n >0$.  Explain why there are exactly $2^n$
binary relations on $D$ that are both symmetric and antisymmetric.

\begin{solution}
Suppose $R$ is symmetric and antisymmetric.  If $d \mrel{R} e$, then
symmetry implies that $e \mrel{R} d$, and then antisymmetry implies
$e=d$.  That is, the only arrows in $\graph{R}$ are self-loops
$\diredge{d}{d}$.  Conversely, if the only edges are self-loops, then
$R$ is symmetric and antisymmetric.  So each such $R$ is uniquely
determined by which subset of $D$ has self-loops.  There are $2^n$
subsets of $D$, so that is the number of symmetric and antisymmetric
relations on $D$.
\end{solution}

\end{problem}
%%%%%%%%%%%%%%%%%%%%%%%%%%%%%%%%%%%%%%%%%%%%%%%%%%%%%%%%%%%%%%%%%%%%%
% Problem ends here
%%%%%%%%%%%%%%%%%%%%%%%%%%%%%%%%%%%%%%%%%%%%%%%%%%%%%%%%%%%%%%%%%%%%%

\endinput
