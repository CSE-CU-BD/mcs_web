\documentclass[problem]{mcs}

\begin{pcomments}
  \pcomment{CP_Euler_theorem_not_rel_prime}
  \pcomment{by Drew Wolpert}
  \pcomment{edited by ARM 3/9/12}
  \pcomment{better suited as a pset problem --ARM}
\end{pcomments}

\pkeywords{
  Euler_theorem
  Eulers_theorem
  number_theory
  modular_arithmetic
  prime_power
  phi_function
}

%%%%%%%%%%%%%%%%%%%%%%%%%%%%%%%%%%%%%%%%%%%%%%%%%%%%%%%%%%%%%%%%%%%%%
% Problem starts here
%%%%%%%%%%%%%%%%%%%%%%%%%%%%%%%%%%%%%%%%%%%%%%%%%%%%%%%%%%%%%%%%%%%%%

\begin{problem}
In this problem we'll prove that for all integer $a,m$ where $m>1$,
\begin{equation}\label{amam-phim}
a^{m} \equiv a^{m - \phi(m)} \pmod{m}.
\end{equation}
Note that $a$ and $m$ need not be relatively prime.

Assume $m = p_1^{k_1} \cdots p_n^{k_n}$ for distinct primes,
$p_1,\dots,p_n$.

\bparts

\ppart Show that if $\gcd(a, p_i) = 1$ then $p_i^{k_i} \divides
a^{\phi(m)}-1$.

\begin{solution}
This follows from Euler's theorem. \TBA{more explanation}
\end{solution}

\ppart Show that if $ p_i\divides a$ then $ p_i^{k_i} \divides
a^{m-\phi(m)}.$

\begin{solution}
We need only show that $m-\phi(m) \geq k_i$.  But this follows because
there are at least $k_i+1$ numbers in $[0,m)$ that are not relatively
 prime to $m$, namely, $0, p_i, p_i^2, \dots, p_i^{k_i}$.
\end{solution}

\ppart Conclude~\eqref{amam-phim} from the facts above.

\hint $a^m-a^{m-\phi(m)} = a^{m-\phi(m)}(a^{\phi(m)}-1)$.

\begin{solution}
Let $b = a^m-a^{m-\phi(m)} = a^{m-\phi(m)}(a^{\phi(m)}-1)$.  To
prove~\eqref{amam-phim}, we need only show that $m \divides b$.  But
by the previous problem parts, $p_i^{k_i}$ divides $a^{m-\phi(m)}$ or
$a^{\phi(m)}-1$ for each $i \in [1,n]$, and therefore $p_i^{k_i}$
divides $b$ divides $a^{m-\phi(m)}$ or $a^{\phi(m)}-1$ for each $i \in
[1,n]$.  This implies that the product of the $p_i^{k_i}$ for $i\in
[1,n]$, namely $m$, divides $b$.
\end{solution}

\eparts
\end{problem}

%%%%%%%%%%%%%%%%%%%%%%%%%%%%%%%%%%%%%%%%%%%%%%%%%%%%%%%%%%%%%%%%%%%%%
% Problem ends here
%%%%%%%%%%%%%%%%%%%%%%%%%%%%%%%%%%%%%%%%%%%%%%%%%%%%%%%%%%%%%%%%%%%%%

\endinput
