\documentclass[problem]{mcs}

\begin{pcomments}
  \pcomment{CP_Euler_theorem_not_rel_prime}
  \pcomment{looks like someone's draft for what could be a good
    problem --ARM}
  \pcomment{formatted by ARM 3/8/12]
\end{pcomments}

\pkeywords{
  Eulers_theorem
  number_theory
  modular_arithmetic
}

%%%%%%%%%%%%%%%%%%%%%%%%%%%%%%%%%%%%%%%%%%%%%%%%%%%%%%%%%%%%%%%%%%%%%
% Problem starts here
%%%%%%%%%%%%%%%%%%%%%%%%%%%%%%%%%%%%%%%%%%%%%%%%%%%%%%%%%%%%%%%%%%%%%

\begin{problem}
In this problem we'll prove that for all pairs of integers $a, m$,
even if they aren't relatively prime,
\[
a^{m} \equiv a^{m - \phi(m )} \pmod{m}.
\]

\bparts

\ppart Show that if $\gcd(a, p_i) = 1$ then $p_i^{k_i} \divides
a^{\varphi(m)}-1$.
\begin{solution}
This follows from Euler's theorem.
\end{solution}

\ppart Show that if $ p_i \divides a$ then $ p_i^{k_i} \divides a^{m-\varphi(m)}.$

\begin{solution}
Let $b = a^m-a^{m-\varphi(m)} = a^{m-\varphi(m)}(a^{\varphi(m)}-1)$.
Assume $m = p_1^{k_1} \cdots p_n^{k_n}$.  We only need to prove that
if $\gcd(a, p_i)=1$ then $p_i^{k_i} \divides (a^{\varphi(m)}-1)$, and
if $ p_i\divides a$ then $ p_i^{k_i} \divides a^{m-\varphi(m)}.$  The
first assertion follows from Euler's theorem, while the second is easy
to check, since $ m-\varphi(m)\geqslant k_i$ because there are at
least $k_i$ numbers that are not relatively prime to $m$, namely,
$p_i,p_i^2,\ldots,p_i^{k_i}$).
\end{solution}

\end{problem}

%%%%%%%%%%%%%%%%%%%%%%%%%%%%%%%%%%%%%%%%%%%%%%%%%%%%%%%%%%%%%%%%%%%%%
% Problem ends here
%%%%%%%%%%%%%%%%%%%%%%%%%%%%%%%%%%%%%%%%%%%%%%%%%%%%%%%%%%%%%%%%%%%%%

\endinput
