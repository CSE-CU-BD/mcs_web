\documentclass[problem]{mcs}

\begin{pcomments}
  \pcomment{FP_startup_mcq}
  \pcomment{Variant of CP_startup}
\end{pcomments}

\pkeywords{
  binomial
  combinatorial_proof
  algebraic_proof
  bijection
}

%%%%%%%%%%%%%%%%%%%%%%%%%%%%%%%%%%%%%%%%%%%%%%%%%%%%%%%%%%%%%%%%%%%%%
% Problem starts here
%%%%%%%%%%%%%%%%%%%%%%%%%%%%%%%%%%%%%%%%%%%%%%%%%%%%%%%%%%%%%%%%%%%%%

\begin{problem}
You want to choose a team of $m$ people from
a pool of $n$ applicants, and from these $m$ people you want to choose
$k$ to be the team managers.  You know you can do this in
\[
\binom{n}{m}\binom{m}{k}
\]
ways.  

Your competitor, a graduate from Harvard Business School, claims that
he has an identical combinatorial formula. He gives you a bijection
that supposedly proves this, and challenges you to find out what his
combinatorial formula is.

He says: ``In your formula we count the number of pairs $(A,B)$, where
$A$ is a size $m$ subset of the pool of $n$ applicants, and $B$ is a
size $k$ subset of $A$. But in mine, we count pairs $(C,D)$, where $C$
is a size $k$ subset of the applicant pool, and $D$ is a size $(n-m)$
subset of the pool that is disjoint from $C$.  These two expressions
are equal because there is a bijection between the two kinds of pairs
that map $(A,B)$ to $(B, (A-B)')$.''

\bparts

\ppart Is your competitor correct?(Yes/No).  Give a one-line justification for your answer.

%If you answered `No' what is the mistake he is making?

\begin{solution}
He is correct.  The size $(n-m)$ subset corresponds to the complement
of the size $(m-k)$ subset.
\end{solution}


\ppart Circle the combinatorial identity that matches the description given by your competitor.

\begin{enumerate}
\item 
\[
\binom{n}{m}\binom{m}{k}
\]
\item 
\[
\binom{n}{k}\binom{n-k}{m-k}
\]

\item 
\[
\binom{n}{k}\binom{n-k}{n-m}
\]

\item 
\[
\binom{n}{n-k}\binom{k}{n-m}
\]

\item 
\[
\binom{n}{k}\binom{n-k}{n-m}
\]

\end{enumerate}

\begin{solution}
He is wrong, there is no combinatorial identity!

\end{solution}

\eparts
\end{problem}


%%%%%%%%%%%%%%%%%%%%%%%%%%%%%%%%%%%%%%%%%%%%%%%%%%%%%%%%%%%%%%%%%%%%%
% Problem ends here
%%%%%%%%%%%%%%%%%%%%%%%%%%%%%%%%%%%%%%%%%%%%%%%%%%%%%%%%%%%%%%%%%%%%%
\endinput
