\documentclass[problem]{mcs}

\begin{pcomments}
  \pcomment{MQ_simple_graphs_short_answer}
\end{pcomments}

\pkeywords{
}

%%%%%%%%%%%%%%%%%%%%%%%%%%%%%%%%%%%%%%%%%%%%%%%%%%%%%%%%%%%%%%%%%%%%%
% Problem starts here
%%%%%%%%%%%%%%%%%%%%%%%%%%%%%%%%%%%%%%%%%%%%%%%%%%%%%%%%%%%%%%%%%%%%%

\begin{problem}
\begin{problemparts}
\problempart
A graph has 8 vertices and 24 edges.  What is the average degree per vertex?
\examspace[1in]

\begin{solution}
By the Handshaking Lemma, the sum of the degrees of the vertices in any graph is equal to twice the number of edges.
So in this case, the sum of the degrees of the vertices is $2\times 24=48$.  With 8 vertices, the average degree per 
vertex is $\frac{48}{8}=6$.
\end{solution}

\problempart
A connected planar graph has 5 more edges than it has vertices.  How many faces does it have?
\examspace[1in]

\begin{solution}
Denoting the number of vertices by $v$, the number of edges by $e$, and the number of faces by $f$, Euler's Formula states that
$v-e+f=2$.  But here, $e=v+5$.  Substituting gives $v-(v+5)+f=2$ and hence $f=7$.  
\end{solution}

\problempart
A connected graph has one more vertex than it has edges.  Is it necessarily planar?
\examspace[1in]

\begin{solution}
Let $G$ denote any such graph.  Now, any graph with $v$ vertices but fewer than $v-1$ edges cannot possibly be connected.  So every edge in $G$ is a cut edge,
and therefore $G$ is acyclic.  So $G$ is a tree and must be planar.
\end{solution}

\problempart
If your answer to the previous part was \textit{yes}, then how many faces can such a graph have?  If your answer was \textit{no}, then give an example of a nonplanar connected graph whose vertices outnumber its edges by one.
\examspace[2in]

\begin{solution}
Since the graph is acyclic, it only has one face.
\end{solution}

\problempart
Consider the graph shown in Figure~\ref{fig:self_isomorphism}.  How many distinct isomorphisms exist between this graph and itself?  (Include the identity isomorphism.)
\begin{figure}[h]
\graphic{MQ_self_isomorphism}
\caption{\label{fig:self_isomorphism}}
\end{figure}
\examspace[2in]
\begin{solution}
Only vertex $f$ has degree 1, so in any self-isomorphism, $f$ must map to itself.  $b$ is the only vertex to be adjacent to a degree-1 vertex, so $b$ must also map to itself.
$a$ and $c$ are both degree-3 vertices, and $d$ and $e$ are both degree-2 vertices.  It is clear from examining the graph that $a$ can be mapped to $c$ and $c$ to $a$, or each of $a$ and $c$
can be mapped to itself.  Independently, and similarly, $d$ can be mapped to $e$ and $e$ to $d$, or each of $d$ and $e$ can be mapped to itself.  The only possible isomorphisms, then,
are obtained by choosing one of the two possible mappings for $a$ and $c$ and, independently, one of the two possible mappings for $d$ and $e$.  The result is $2\times2=4$ possible isomorphisms.
\end{solution}

\end{problemparts}
\end{problem}

\endinput
