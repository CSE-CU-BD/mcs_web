\documentclass[problem]{mcs}

\begin{pcomments}
  \pcomment{from: F07.ps2}
\end{pcomments}

\pkeywords{
  partial_order
  weak_partial_order
  reflexive
  antisymmetric
  transitive
  subset
  isomorphic
  inverse_image
}

%%%%%%%%%%%%%%%%%%%%%%%%%%%%%%%%%%%%%%%%%%%%%%%%%%%%%%%%%%%%%%%%%%%%%
% Problem starts here
%%%%%%%%%%%%%%%%%%%%%%%%%%%%%%%%%%%%%%%%%%%%%%%%%%%%%%%%%%%%%%%%%%%%%

\newcommand{\prinv}[1]{\text{L}#1}

\begin{problem}
This problem asks for a proof of Notes Lemma~\ref{rgb}:

Let $\preceq$ be a weak partial order on a set, $A$.  For any element $a
\in A$, let
\[
\prinv(a) \eqdef \set{b \in A \suchthat b \preceq a}.
\]
That is, $\prinv(a)$ is the inverse image of $a$ under the $\preceq$
relation.\footnote{We already have a notation for $\prinv(a)$, namely,
  $(\preceq\set{a})$, but this looks funny.}
Let $\mathcal{L}$ be the set of such inverse images, that is,
\[
\mathcal{L} \eqdef \set{\prinv(a) \suchthat a \in A}.
\]
Then the function $\prinv{}:A \to \mathcal{L}$ is an isomorphism from the
$\preceq$ relation on $A$, to the subset relation on $\mathcal{L}$.

\bparts

\ppart
Prove that the function $\prinv{}:A \to \mathcal{L}$ is a bijection.

\begin{solution}
  By definition, $\prinv{}$ is a surjective function onto $\mathcal{L}$.
  So all we have to do is prove it is an injection.  To prove this,
  suppose $\prinv(a) = \prinv(b)$.  Now since $a \in \prinv(a)$ by
  reflexivity, we also have $a \in \prinv(b)$.  This means $a \preceq b$.
  Likewise, $b \preceq a$.  Hence $a=b$, by antisymmetry.
\end{solution}

\ppart Complete the proof by showing that
\begin{equation}\label{apbiff}
a \preceq b  \qiff  \prinv(a) \subseteq \prinv(b)
\end{equation}
for all $a,b \in A$.

\begin{solution}
  For the left-to-right direction, suppose $a \preceq b$.  To prove that
  $\prinv(a) \subseteq \prinv(b)$, suppose $c \in \prinv(a)$, which means
  that $c \preceq a$.  So by transitivity, $c \preceq b$, and so $c \in
  \prinv(b)$.  Hence every $c \in \prinv(a)$ is also in $\prinv(b)$, which
  proves containment.

  For the right-to-left direction, suppose $\prinv(a) \subseteq
  \prinv(b)$.  But $a \in \prinv(a)$ by reflexivity, so $a \in \prinv(b)$,
  which means that $a \preceq b$.

\iffalse
%elegant, but harder to follow:

\begin{align*}
a \preceq b
  & \qiff \forall c \in A.\, c \preceq a \QIMPLIES c \preceq b 
           & \text{(because $a \preceq a$)}\\
  & \qiff \forall c \in A.\, c \in \prinv(a) \QIMPLIES c \in \prinv(b)
           & \text{(def of $\prinv$)}\\
  & \qiff \prinv(a) \subseteq \prinv(b)
\end{align*}
\fi

\end{solution}

\eparts
\end{problem}


%%%%%%%%%%%%%%%%%%%%%%%%%%%%%%%%%%%%%%%%%%%%%%%%%%%%%%%%%%%%%%%%%%%%%
% Problem ends here
%%%%%%%%%%%%%%%%%%%%%%%%%%%%%%%%%%%%%%%%%%%%%%%%%%%%%%%%%%%%%%%%%%%%%

\endinput
