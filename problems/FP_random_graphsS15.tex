\documentclass[problem]{mcs}

\begin{pcomments}
  \pcomment{FP_random_graphsS15}
  \pcomment{extends FP_random_graphsF99 to arbitrary n, omits final
    2 part about k-cliques and Big Oh}
\end{pcomments}

\pkeywords{
  probability
  random_variables
  expectation
  degree
  graph
}

\begin{problem}
A simple graph with $n$ vertices is constructed by randomly placing an
edge between every two vertices with probability $p$.  These random
edge placements are performed independently.

\bparts

\ppart\label{pgdeg2} What is the probability that a given vertex of the graph
has degree two?

\exambox{1in}{0.6in}{0in}

\begin{solution}
\[
\binom{n-1}{2} \cdot p^2\cdot ({p-1})^{n-3}
\]
The probability that the given vertex is adjacent to two other given
vertices is $p^2\cdot ({p-1})^{n-3}$.  There are $\binom{n-1}{2}$
possible such sets of two other vertices.

\end{solution}

\examspace[1in]

\ppart What is the expected number of nodes with degree two?  (You may
express your answer in terms of $t$, where $t$ is the answer to
part~\eqref{pgdeg2}.)

\exambox{2in}{1in}{0in}

\begin{solution}
$nt$, by linearity of expectations, because the expectation of the
  indicator variable for a given vertex having degree two is $t$.
\end{solution}

\examspace[2in]

\iffalse

\ppart 
What is the expected number of cliques of size $k$, that is,
subgraphs isomorphic to the complete graph with $k$ vertices?  Express
your answer as a closed form formula involving binomial
coefficients.

\exambox{1.5in}{0.7in}{0in}

\begin{solution}
The strategy for computing the expected number of cliques uses
linearity of expectation.  Define an indicator random variable for
every possible $k$-clique.  Every $k$-subset of the $n$ nodes is a
possible $k$-clique, so there are $\binom{n}{k}$ such random
variables.  The random variable for a particular $k$-clique is 1 if
the $k$-clique appears in the graph and 0 if not, so its expected
value is just the probability that a particular $k$-clique occurs in
the graph.  This happens if and only if all $k$ nodes are connected by
edges; there are $\binom{k}{2}$ such edges, so the probability of a
particular $k$-clique is $p^{\binom{k}{2}}$.  Summing the expected
values of all the $k$-clique indicators gives the expected number of
$k$-cliques:
\[
\binom{n}{k} p^{\binom{k}{2}}.
\]

\end{solution}
\examspace[1in]

\ppart The previous result implies an interesting fact which you may
assume without proof: if $p=1/2$, then the expected number of cliques of
size $2 \log n$ is greater than one, where $n$ is the total number of
vertices in the graph.  (In the previous parts we had $n=10$; now $n$ is
arbitrary.)  Oddly, even though we can expect there to be such a clique,
no procedure is known to \emph{find} some clique of size $2\log n$ in the
graph---when there is one---which is much more efficient than exhaustively
searching through all possible subsets of vertices of size $2 \log n$ and
checking each subset to see if it is a clique.  Circle each of the
following expressions which correctly bounds the number of subsets such an
exhaustive search would examine:
\begin{enumerate}
\item $O(2^n)$
\item $O(n^2)$
\item $O(n^{\log n})$
\item $O(n^{2 \log n})$
%\item $\Theta(e^{\log^2 n})$
%\item $\Theta((\log n)^n)$
\end{enumerate}

\begin{solution}

(a),(d).

The number of such subsets is
\[
\binom{n}{2 \log n} = \frac{n!}{(2 \log n)! (n-2 \log n)!} < \frac{n!}{(n-2 \log n)!}
< n^{2 \log n},
\]
and 
\[
\frac{n!}{(2 \log n)! (n-2 \log n)!} > \paren{\frac{(n- \log n)}{2 \log n}}^{2 \log
n} > (n^{0.9})^{2 \log n} = n^{1.8 \log n} \neq O(n^{\log n}).
\]

\end{solution}
\fi

\eparts
\end{problem}

\endinput
