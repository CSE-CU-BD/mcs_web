\documentclass[problem]{mcs}

\begin{pcomments}
\pcomment{PS_tree_degree_sequence}
\pcomment{by Diego Cifuentes 3/17/14; edits by ARM}
\end{pcomments}

\pkeywords{
 tree
 degree
 handshaking
}

%%%%%%%%%%%%%%%%%%%%%%%%%%%%%%%%%%%%%%%%%%%%%%%%%%%%%%%%%%%%%%%%%%%%%
% Problem starts here
%%%%%%%%%%%%%%%%%%%%%%%%%%%%%%%%%%%%%%%%%%%%%%%%%%%%%%%%%%%%%%%%%%%%%

\begin{problem}

%  The \emph{degree sequence} of a simple graph is the weakly
%  decreasing sequence of degrees of its vertices.  For example, the
%  degree sequence for the 5-vertex numbered tree pictured in the
%  Figure~\bref{codetrees} in Problem~\bref{CP_numbered_trees} is
%  $(2,2,2,1,1)$ and for the 7-vertex tree it is $(3,3,2,1,1,1,1)$.
%
%\inhandout{
%\begin{figure}[htb]
%\graphic{n-2}
%\caption{}
%\label{codetrees}
%\end{figure}
%}

Let $D = (d_1,d_2,\ldots,d_n)$ be a sequence of positive integers where $n\geq 2$.

\bparts
\ppart\label{occnum}

Suppose $D$ is a list of the degrees of vertices of some $n$-vertex
tree $T$, that is, $d_i$ is the degree of the $i$th vertex of $T$.
Explain why
\begin{equation}\label{eq:tree_degree_seq}
\sum_{i=1}^n d_i = 2(n-1)
\end{equation}

\begin{solution}
By the Handshaking Lemma\inbook{~\bref{sumedges}}, the sum of the
degrees of the vertices of $T$ is twice the number of edges, and since
$T$ is a tree with $n$ vertices, it has $n-1$ edges.
\end{solution}

\ppart\label{7pats} Prove conversely that if $D$ satisfies
equation~\eqref{eq:tree_degree_seq}, then $D$ is a list of the degrees
of the vertices of some $n$-vertex tree.  \hint Induction.

\begin{solution}
The proof will be by induction on $n$ with induction hypothesis
\[
P(n) \eqdef \forall D.\, D\
\text{ satisfies }~\eqref{eq:tree_degree_seq}\ \QIMPLIES\ \exists\text{ tree }
 T.\, D \text{ is a list of the vertex degrees of } T.
 \]

\inductioncase{Base case} ($n=2$): In this case $D$ can only be the
length two list $(1,1)$, which is the degree sequence of a 2-vertex
tree.

\inductioncase{Induction step}: Suppose $D = (d_1,d_2,\ldots,d_n)$
satisfies~\eqref{eq:tree_degree_seq}.  First notice that there must be
an $i$ such that $d_i = 1$, otherwise we would have that $d_{i}\geq 2$
for all $i$ and the sum would be too large.  Similarly, there must be
a $j$ such that $d_j\geq 2$, otherwise the sum would be too small.
Now we can assume without loss of generality that $d_{n}=1$ and $d_{n-1} \geq 2$.

Let $k \eqdef n-1$ and $D' = (d_1,d_2,\dots, d_{k-1},d_k - 1)$.
Observe that the sum of $D'$ equals $2(n-1)-1 -1 = 2(k-1)$, so $D'$
satisfies equation \eqref{eq:tree_degree_seq} for the case that $n$ is
$k$.  Assuming the induction hypothesis, $P(k)$, we know that there is
a $k$-vertex tree $T'$ with degree sequence $D'$.  Let $T$ be the
$n$-vertex tree obtained by adding a new leaf to $T'$ adjacent to the
$k$th vertex of $T'$.  Then $D$ is a list of the vertex degrees of $T$,
concluding the proof.
\end{solution}

\ppart\label{pat-decode} Assume that $D$ satisfies equation
\eqref{eq:tree_degree_seq}.  Show that it is possible to partition $D$
into two sets $S_1, S_2$ such that the sum of the elements in each set
is the same. \hint Trees are bipartite.

\begin{solution}
Using the previous part we know that there is a tree $T$ with degree
sequence $D$.  Since trees are bipartite graphs, there exists a partition
of the vertices into sets $V_1,V_2$ such that any edge connects a
vertex in $V_1$ with a vertex in $V_2$.  We argue that the sum of the
degrees of the vertices in $V_1$ is equal to the number of edges of
the graph.  The reason is that any edge in the graph contributes to
$1$ in the degree in exactly one of the vertices in $V_1$.  Similarly,
the sum of the degrees of the vertices in $V_2$ is also the number of
edges.  Thus the degrees corresponding to the partition $V_1,V_2$
determine the sets $S_1,S_2$ we were looking for.

A proof by induction similar to part~\eqref{pat-decode} is also
possible.
\end{solution}

\eparts

\end{problem}

%%%%%%%%%%%%%%%%%%%%%%%%%%%%%%%%%%%%%%%%%%%%%%%%%%%%%%%%%%%%%%%%%%%%%
% Problem ends here
%%%%%%%%%%%%%%%%%%%%%%%%%%%%%%%%%%%%%%%%%%%%%%%%%%%%%%%%%%%%%%%%%%%%%

\endinput
