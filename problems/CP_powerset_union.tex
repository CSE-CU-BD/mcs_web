\documentclass[problem]{mcs}

\begin{pcomments}
  \pcomment{CP_powerset_union}
  \pcomment{idea from Bender \& Williamson \cite{Bender2005}, Sets & Functions, prob 1.20}
  \pcomment{by ARM 3/29/13}
\end{pcomments}

\pkeywords{
  power_set
  union
  intersection
}

%%%%%%%%%%%%%%%%%%%%%%%%%%%%%%%%%%%%%%%%%%%%%%%%%%%%%%%%%%%%%%%%%%%%%
% Problem starts here
%%%%%%%%%%%%%%%%%%%%%%%%%%%%%%%%%%%%%%%%%%%%%%%%%%%%%%%%%%%%%%%%%%%%%

\begin{problem}
\emph{Powerset Properties.}

Let $A$ and $B$ be sets.

\bparts

\ppart Prove that
\[
\power(A \intersect B) = \power(A) \intersect \power(B).
\]

\begin{solution}
We show that $\power(A \intersect B)$ and $\power(A) \intersect
\power(B)$ have the same elements.  Namely,
\begin{align*}
S \in \power(A \intersect B)
  & \QIFF\  S \subseteq A \intersect B
      & \text{(def of power set)}\\
  & \QIFF\ (S \subseteq A) \QAND (S \subseteq B)
      & \text{(def of $\intersect$)}\\
  & \QIFF\ S \in \power(A) \QAND  S \in \power(B)
      & \text{(def of power set)}\\
  & \QIFF\ S \in \power(A) \intersect \power(B)
      & \text{(def of $\intersect$)}.
\end{align*}
\end{solution}

\ppart Prove that
\[
\power(A) \union \power(B)  \subseteq \power(A \union B),
\]
with equality holding iff one of $A$ or $B$ is a subset of the other.

\begin{solution}
To show $\subseteq$, we show that every element of $\power(A) \union \power(B)$ is an element of 
$\power(A \union B)$.  Namely,
\begin{align*}
S \in \power(A) \union \power(B)
   & \QIFF\ (S \in \power(A)) \QOR (S \in \power(B))
         & \text{(def of $\union$)}\\
  & \QIFF\ (S \subseteq A) \QOR (S \subseteq B)
      & \text{(def of power set)}\\
  & \QIMPLIES\ S \subseteq A \union B
      & \text{(def of $\subseteq$ and $\union$)}\\
  & \QIFF\  S \in \power(A \union B)
      & \text{(def of power set)}
\end{align*}

Also, if $A \subseteq B$, then $\power(A \union B) = \power(A) \union
\power(B)$ since both equal $\power(B)$.

Finally, if there are elements $a \in A-B$ and $b\in B-A$, then
$\set{a,b} \in \power(A \union B) - (\power(A) \union \power(B))$, so
\[
\power(A) \union \power(B)  \subset \power(A \union B)
\]
in this case.
\end{solution}



\eparts
\end{problem}

%%%%%%%%%%%%%%%%%%%%%%%%%%%%%%%%%%%%%%%%%%%%%%%%%%%%%%%%%%%%%%%%%%%%%
% Problem ends here
%%%%%%%%%%%%%%%%%%%%%%%%%%%%%%%%%%%%%%%%%%%%%%%%%%%%%%%%%%%%%%%%%%%%%

\endinput
