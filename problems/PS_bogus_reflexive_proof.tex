\documentclass[problem]{mcs}

\begin{pcomments}
  \pcomment{from: S02.cp3w}
\end{pcomments}

\pkeywords{
  bogus proof
  reflexive
  relation
}

%%%%%%%%%%%%%%%%%%%%%%%%%%%%%%%%%%%%%%%%%%%%%%%%%%%%%%%%%%%%%%%%%%%%%
% Problem starts here
%%%%%%%%%%%%%%%%%%%%%%%%%%%%%%%%%%%%%%%%%%%%%%%%%%%%%%%%%%%%%%%%%%%%%

\begin{problem} 
Find the flaw in the following false proof, and give a counterexample
to the claim.

\begin{claim*}
Suppose $R$ is a relation on $A$. If $R$ is symmetric and transitive, 
then $R$ is reflexive.
\end{claim*}

\begin{falseproof}
Let $x$ be an arbitrary element of $A$.  Let $y$ be any element of $A$
such that $xRy$.  Since $R$ is symmetric, it follows that $yRx$.  Then
since $xRy$ and $yRx$, we conclude by transitivity that $xRx$.  Since
$x$ was arbitrary, we have shown that $\forall x \in A\; (xRx)$, so
$R$ is reflexive.
\end{falseproof}

\solution{
The flaw is assuming that $y$ exists.  It is possible that there is an
$x \in A$ that is not related by $R$ to anything.  No such $R$ will be
reflexive.  The simplest such $R$ that is also symmetric and
transitive is the empty relation on any nonempty set $A$. We can
easily construct other examples, such as 

$R = \{(a,a),(a,b),(b,a),(b,b) \}$ 

on the set $A \eqdef \set{a,b,c}$, which is not reflexive because $(c,c)$
is not in $R$. These relations are counterexamples to the claim. 

Note that the theorem can be fixed: $R$ restricted to its domain of
definition is reflexive, and hence an equivalence relation.}
\end{problem} 



%%%%%%%%%%%%%%%%%%%%%%%%%%%%%%%%%%%%%%%%%%%%%%%%%%%%%%%%%%%%%%%%%%%%%
% Problem ends here
%%%%%%%%%%%%%%%%%%%%%%%%%%%%%%%%%%%%%%%%%%%%%%%%%%%%%%%%%%%%%%%%%%%%%