\documentclass[problem]{mcs}

\begin{pcomments}
  \pcomment{PS_discriminate_biased_coins_chebyshev}
  \pcomment{complements PS_discriminate_biased_coins_chernoff}
\end{pcomments}

\pkeywords{
  probability
  Chebyshev
  sampling
  variance
  binomial_distribution
}

%%%%%%%%%%%%%%%%%%%%%%%%%%%%%%%%%%%%%%%%%%%%%%%%%%%%%%%%%%%%%%%%%%%%%
% Problem starts here
%%%%%%%%%%%%%%%%%%%%%%%%%%%%%%%%%%%%%%%%%%%%%%%%%%%%%%%%%%%%%%%%%%%%%

\begin{problem}
There is a fair coin and a biased coin that flips heads with
probability $3/4$.  You are given one of the coins, but you don't know
which.  To determine which coin was picked, your strategy will be to
choose a number $n$ and flip the picked coin $n$ times.  If the number
of heads flipped is closer to $(3/4) n$ than to $(1/2) n$,
you will guess that the biased coin had been picked and otherwise you
will guess that the fair coin had been picked.

\begin{problemparts}

  \ppart\label{cheb95discrim} Use the Chebyshev Bound to find a value
  $n$ so that with probability 0.95 your strategy makes the correct
  guess, no matter which coin was picked.

\inhandout{\hint Let the random variable $F$ be the number of heads that would appear
in the first $n$ flips of the fair coin, and let $B$ denote the number
of heads that would appear in the first $n$ flips of the biased coin.
We must flip the coin sufficiently many times to ensure that if the
coin is biased, then with probability 0.95, the number of heads will
be closer to $(3/4) n$ than to $(1/2) n$, that is,
\begin{equation}\label{confreqB}
\Prob{B > (5/8)n} \ge 0.95\, .
\end{equation}
Similarly, if the coin is fair, then with probability 0.95, the number of heads will
\emph{not} be closer to $(3/4) n$ than to $(1/2) n$, that is,
\begin{equation}\label{confreqF}
\Prob{F < (5/8)n} \ge 0.95\, .
\end{equation}}

\begin{solution}
\inbook{Let the random variable $F$ be the number of heads that would appear
in the first $n$ flips of the fair coin, and let $B$ denote the number
of heads that would appear in the first $n$ flips of the biased coin.
We must flip the coin sufficiently many times to ensure that if the
coin is biased, then with probability 0.95, the number of heads will
be closer to $(3/4) n$ than to $(1/2) n$, that is,
\begin{equation}\label{confreqB}
\Prob{B > (5/8)n} \ge 0.95\, .
\end{equation}
Similarly, if the coin is fair, then with probability 0.95, the number of heads will
\emph{not} be closer to $(3/4) n$ than to $(1/2) n$, that is,
\begin{equation}\label{confreqF}
\Prob{F < (5/8)n} \ge 0.95\, .
\end{equation}
}

The variable $B$ has an $(n,3/4)$-binomial distribution, so its
expectation is $(3/4)n$ and its variance is $(3/4)(1-3/4)n=(3/16)n$.
\begin{align}
\text{Equation~\eqref{confreqB} holds}\ 
& \QIFF\ \prob{B \leq (5/8)n} < 0.05\notag\\
& \QIFF\ \prob{(3/4)n - B \geq (3/4)n -(5/8)n} < 0.05\notag\\
& \QIFF\ \prob{\expect{B} - B \geq n/8} < 0.05\, .\label{peB-B}
\end{align}
But
\[
[\expect{B} - B \geq n/8]\ \QIMPLIES\ [\abs{\expect{B} - B} \geq n/8]
\]
and therefore
\[
\prob{\expect{B} - B \geq n/8} \leq \prob{\abs{\expect{B} - B} \geq n/8}.
\]
So we need only find $n$ such that
\[
\prob{\abs{\expect{B} - B} \geq n/8} < 0.05\, .
\]
Using Chebyshev's inequality, we know
\begin{align*}
\prob{ \abs{B - \expect{B}} > n/8 }
& \leq \frac{\variance{B}}{(n/8)^2}  & \text{(by Chebyshev)}\\
& = \frac{(3/16)n}{n^2/64}
  = \frac{12}{n},
\end{align*}
so we only need $12/n < 0.05$, that is
\[
n \geq 240.
\]

Likewise, $F$ has an $(n,1/2)$-binomial distribution, so its expectation is
$n/2$ and its variance is $n/4$.  Reasoning in exactly the same way, 
we only need $16/n < 0.05$, that is
\[
n \geq 320.
\]
So knowing the results of $320$ flips of the chosen coin will
allow us to guess its identity with 95\% confidence.

Because the variance of the biased coin is less than that of the fair
coin, we can do slightly better if we increase our threshold from
$(5/8)n$ to about $0.634n$, which gives 95\% confidence with 279 coin
flips.
\end{solution}

\ppart Suppose you had access to a computer program that would
generate, in the form of a plot or table, the full binomial-$(n,p)$
probability density and cumulative distribution functions.  How would
you find the minimum number of coin flips needed to infer the identity
of the chosen coin with probability 0.95?  How would you expect the
number $n$ determined this way to compare to the number obtained in
part\eqref{cheb95discrim}?  (You do not need to determine the
numerical value of this minimum $n$, but we'd be interested to know if
you did.)

\begin{solution}
Again, we seek to determine the values of $n$ that satisfy
both~\eqref{confreqF} and~\eqref{confreqB}.  But~\eqref{confreqF} is
equivalent to
\begin{equation}\label{cdfF}
\cdf_F((5/8)n) \geq 0.95
\end{equation}
while~\eqref{confreqB} is equivalent to
\begin{equation}\label{cdfB}
\cdf_B((5/8)n) < 0.05
\end{equation}

Knowing that $F$ is $(n,1/2)$-binomially distributed and $B$ is
$(n,3/4)$-binomially distributed, we can use the computer program to
find the smallest $n$ that satisfies both~\eqref{cdfF}
and~\eqref{cdfB}.  We would expect these numbers to be smaller than
those on part~\eqref{cheb95discrim} because we are making use of
knowledge of the entire CDF of $F$ and $B$ instead of merely their
mean and variance.  Moreover, we know from the discussion in
Section~\bref{binomial_distribution_section} that the tails of a
binomial distribution decrease very rapidly, so we would expect the
numbers obtained using the binomial CDF to be a lot smaller than the
one obtained using Chebyshev's Bound in part~\eqref{cheb95discrim}.

In fact, the numbers of flips needed according to the binomial CDF's
are 40 and 38 respectively, and we conclude that merely knowing the
results of $40$ flips of the chosen coin will allow us to guess its
identity with 95\% confidence.
\end{solution}

\ppart Now that we have determined the proper number $n$, we will
assert that the picked coin was the biased one whenever the number of
Heads flipped is greater than $(5/8)n$, and we will be right with
probability 0.95.  What, if anything, does this imply about
\[
\prcond{\text{picked coin was biased}}{\text{\# Heads flipped }\geq
  (5/8)n}?
\]

\begin{solution}
The conditional probability depends on the probability of the biased
coin being picking, and we don't know what the latter probability is,
or even if the choice of coin is made probabilistically---see the
discussion of diagnostic testing in
Section~\bref{med_test-subsection}.
\end{solution}

\end{problemparts}

\end{problem}

%%%%%%%%%%%%%%%%%%%%%%%%%%%%%%%%%%%%%%%%%%%%%%%%%%%%%%%%%%%%%%%%%%%%%
% Problem ends here
%%%%%%%%%%%%%%%%%%%%%%%%%%%%%%%%%%%%%%%%%%%%%%%%%%%%%%%%%%%%%%%%%%%%%

\endinput

