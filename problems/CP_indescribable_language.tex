\documentclass[problem]{mcs}
\begin{pcomments}
  \pcomment{CP_indescribable_language}
  \pcomment{from: S09.cp2r}
  \pcomment{Can only be used AFTER pred calculus.
    Needs handy copy of Russell Paradox}
  \pcomment{revised by ARM 2/13/11, 3/1/16}
\end{pcomments}
\pkeywords{
  Russell
  paradox
  strings
  binary
  logic
}

%%%%%%%%%%%%%%%%%%%%%%%%%%%%%%%%%%%%%%%%%%%%%%%%%%%%%%%%%%%%%%%%%%%%%
% Problem starts here
%%%%%%%%%%%%%%%%%%%%%%%%%%%%%%%%%%%%%%%%%%%%%%%%%%%%%%%%%%%%%%%%%%%%%

\newcommand{\desc}{\text{desc}} % \def\all0s{\textbf{all-0s}}

\begin{problem}
There is a famous paradox about describing numbers which goes as
follows:

There are only so many possible definitions of nonnegative integers
that can be written out in English using no more than 161 characters
from the Roman alphabet, punctuation symbols, and spaces.  So there
have to be an infinite number of nonnegative integers that don't have
such short definitions.  By the Well Ordering Principle, there must be
a \emph{least} nonnegative integer $n$ that has no such short
definition.  But wait a minute,
\begin{quote}
``The \emph{least} nonnegative integer that cannot be defined in
  English using at most 161 characters from the Roman alphabet,
  punctuation symbols, and spaces.''
\end{quote}
is a definition of $n$ that uses 161 characters (count 'em).  So $n$
can't exist, and the Well Ordering Principle is unsound!

Now this ``paradox'' doesn't stand up to reason because it rests on
the decidedly murky concept of a ``definition in English.''  As usual,
when you don't know what you're talking about, reaching contradictory
conclusions is to be expected.

But we can extract from this paradox a well-defined and interesting
theorem about definability in predicate logic.  The method we use is
essentialy the same as the one used to prove Cantor's
Theorem~\bref{powbig}, \iffalse that the power set is ``bigger'' than
the set\fi and it leads to many other important results about the
logical limits of mathematics and computer science.  In particular,
we'll present a simple and precise description of a set of binary
strings that can't be described by ordinary logical formulas.  In
other words, we will give a precise description of an undescribable
set of strings, which sounds paradoxical, but won't be when we look at
it more closely.

Let's start by illustrating how a logical formula can describe the set
of binary strings that do not contain a \mtt{1}:
\begin{equation}%\tag{\all0s} %\label{no1}
  \QNOT[\exists y.\, \exists z.\, s = y\mtt{1}z]\,.
  \tag{$\text{no-1s}$}
\end{equation}
So the strings $s$ described by formula~(no-1s) are exactly the
strings consisting solely of \mtt{0}'s.

Formula~(no-1s) is an example of a ``string formula'' of the kind we
will use to describe properties of binary strings.  More precisely, an
\emph{atomic} string formula is a formula, like ``$s = y\mtt{1}z$''
above, that is of the general form
 \[
  ``xy\dots z = uv\dots w''
 \]
where $x,y,\dots,z,u,v,\dots,w$ may be the constants \mtt{0},\mtt{1},
or may be variables ranging over the set, \bins, of finite binary
strings.  A string formula in general is one like~(no-1s),
built up from atomic formulas using quantifiers and propositional
connectives.

When $G(s)$ is a string formula, we'll use the notation $\desc(G)$ for
the set of binary strings $s$ that satisfy $G$.  That is,
  \[
  \desc(G) \eqdef \set{s \in \bins \suchthat G(s)}.
  \]
A set of binary strings is \emph{describable} if it equals $\desc(G)$
for some string formula $G$.  For example, the set $\strings{0}$ of
finite strings of 0's is describable because
 \[
 \desc(\text{(no-1s)}) = \strings{0}.
 \]

The next important idea comes from the observation that a string
formula itself is a syntactic object, consisting of a string of
characters over some standard character alphabet.  Now coding
characters of an alphabet into binary strings is a familiar idea.  For
example, the characters of the ASCII alphabet have a standard coding
into the length eight binary strings.  Once its individual characters
are coded into binary, a complete string formula can be coded into a
binary string by concatenating the binary codes of its consecutive
characters---a very familiar idea to a computer scientist.

Now suppose $x$ is a binary string that codes some formula $G_x$.  The
details of how we extract $G_x$ from its code $x$ don't matter
much---we only require that there is some procedure to actually
display the string formula $G_{x}$ given its code $x$.

It's technically convenient to treat every string as the code of a
string formula, so if $x$ is not a binary string we would get from a
string formula, we'll arbitrarily define $G_x$ to be the
formula~\text{(no-1s)}.

\iffalse
Standard binary representations of formulas are often based on
character-by-character translation into binary, which means that only
a sparse set of binary strings actually represent string formulas.  It
will be technically convenient to have \emph{every} binary string
represent some string formula.  This is easy to do: tweak the display
procedure so it displays some default formula, say $\text{no-1s}$,
when it gets a binary string that isn't a standard representation of a
string formula.  With this tweak, \emph{every} binary string, $x$,
will now represent a string formula, $G_{x}$.
\fi

Now we have just the kind of situation where a Cantor-style diagonal
argument can be applied, namely, we'll ask whether a string describes a
property of \emph{itself}!  That may sound like a mind-bender, but all
we're asking is whether
\[
G_x(x)
\]
is true, or equivalently whether
\[
x \in \desc(G_{x}).
\]

For example, using character-by-character translations of formulas
into binary, neither the string $0000$ nor the string $10$ would be
the binary representation of a formula, so our convention implies that
\[
G_{0000} = G_{10} \eqdef \text{formula~(no-1s)}.
\]
So
\[
\desc(G_{0000})= \desc(G_{10}) = \strings{0}.
\]
This means that
\[
0000 \in \desc(G_{0000}) \quad\text{and}\quad 10 \notin \desc(G_{10}).
\]

Now we are in a position to give a precise mathematical description of
an ``undescribable'' set of binary strings, namely:
\begin{theorem*}
Define
\begin{equation}\label{Udefxngx}
U \eqdef \set{x \in \bins \suchthat\ x \notin \emph{\desc}(G_{x})}.
\end{equation}
The set $U$ is not describable.
\end{theorem*}

Use reasoning similar to Cantor's Theorem~\bref{powbig} to prove this
Theorem.

\hint Suppose $U =\desc(G_{x_U})$.

\begin{solution}
Suppose to the contrary that $U$ was describable.  This means there is
some string formula describing $U$, and that string formula would have
a binary code $x_U \in \bins$.  In other words
\[
U = \desc(G_{x_U}).
\]
Another way to phrase this is to say that, by definition of $x_U$,
\begin{equation}\label{xuxdgxu}
x \in U \qiff x \in \desc(G_{x_U})
\end{equation}
for all $x \bins$.

Also, by definition of $U$, we have
\begin{equation}\label{xUiffxGx}
x \in U \qiff x \notin \desc(G_{x}) 
\end{equation}
for $x\in \bins$.

Combining~\eqref{xuxdgxu} and~\eqref{xUiffxGx}, we conclude that
\begin{equation}\label{xdGiff}
x \notin \desc(G_{x}) \qiff x \in \desc(G_{x_U}),
\end{equation}
for all $x\in \bins$.

Now plugging in $x_U$ for $x$ in~\eqref{xdGiff} gives an immediate
contradiction:
\[
x_U \notin \desc(G_{x_u}) \qiff x_U \in \desc(G_{x_U}).
\]
\end{solution}


%%%%%%%%%%%%%%%%%%%%%%%%%%%%%%%%%%%%%%%%%%%%%%%%%%%%%%%%%%%%%%%%%%%%%
% Problem ends here
%%%%%%%%%%%%%%%%%%%%%%%%%%%%%%%%%%%%%%%%%%%%%%%%%%%%%%%%%%%%%%%%%%%%%

\endinput

\inhandout{
\textbox{
\begin{center}
\large Cantor's Theorem
\textbf{There is no bijection between any set $A$ and its powerset
$\power(A)$.}
\end{center}
\begin{proof}
  We show that if $g$ is a total function from $A$ to $\power(A)$,
  then $g$ does not have the $[\ge 1\ \text{in}]$, surjection
  property, and so is certainly not a bijection.
  Define
  \[
  A_g \eqdef \set{a \in A \suchthat a \notin g(a)}.
  \]
  Since $g$ is total, $A_g$ is a well-defined subset of $A$, which
  means it is a member of $\power(A)$.  We claim $A_g$ is not in the
  range of $g$, and so $g$ is not a surjection.
  To prove that $A_g \notin \range{g}$, assume to the contrary that it
  was in $\range{g}$.  That is,
\[
A_g = g(a_0)
\]
for some $a_0 \in A$.  Then by definition of $A_g$,
\[
a \in g(a_0) \qiff a \in A_g \qiff a \notin g(a)
\]
for all $a \in A$.  Now letting $a = a_0$ yields the contradiction
\[
a_0 \in g(a_0) \qiff a_0 \notin g(a_0).
\]
\end{proof}
}}
