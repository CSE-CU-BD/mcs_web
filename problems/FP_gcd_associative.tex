\documentclass[problem]{mcs}

\begin{pcomments}
  \pcomment{FP_gcd_associative}
  \pcomment{ARM 3/15/15}
\end{pcomments}

\pkeywords{
  number_theory
  gcd
  lcm
  prime
  factorization
}

%%%%%%%%%%%%%%%%%%%%%%%%%%%%%%%%%%%%%%%%%%%%%%%%%%%%%%%%%%%%%%%%%%%%%
% Problem starts here
%%%%%%%%%%%%%%%%%%%%%%%%%%%%%%%%%%%%%%%%%%%%%%%%%%%%%%%%%%%%%%%%%%%%%

\begin{problem}

Let 
\begin{align*}
m & = 2^9     5^{24} 7^4     11^{7},\\
n & = 2^3           7^{22}   11^{211}  19^7,\\
p & = 2^5 3^4       7^{6042}          19^{30}.
\end{align*}
\bparts

\ppart What is the $\gcd(m,n,p)$?

\examspace[0.3in]

\begin{solution}
\[
2^3 7^4
\]
\end{solution}

\ppart  What is the \emph{least common multiple} $\lcm(m,n,p)$?

\examspace[0.3in]

\begin{solution}
\[
2^9 3^4 5^{24} 7^{6042} 11^{211} 19^{30}
\]
\end{solution}

\eparts

\medskip

Let $\expof{k}{n}$ be the largest power of $k$ that divides
$n$, where $k>1$.   That is,
\[
\expof{k}{n} \eqdef \max\set{i \suchthat k^i\text{ divides } n}.
\]

If \iffalse $\emptyset \neq A \subseteq \nngint$, that is, \fi
$A$ is a nonempty set of positive integers, define
\[
\expof{k}{A} \eqdef \set{\expof{k}{a} \suchthat a \in A}.
\]

\bparts

\ppart\label{gcdexp}  Express $\expof{k}{\gcd(A)}$ in terms of $\expof{k}{A}$.

\examspace[0.5in]

\begin{solution}
\[
\expof{k}{\gcd(A)} = \min \expof{k}{A}.
\]
\end{solution}

\ppart\label{lcmexp} Let $p$ be a prime number.  Express
$\expof{p}{\lcm(A)}$ in terms of $\expof{p}{A}$.

\examspace[0.5in]

\begin{solution}
\[
\expof{p}{\lcm(A)} = \max \expof{p}{A}.
\]
\end{solution}

\ppart Give an example of integers $a,b$ where $\expof{6}{\lcm(a,b)} >
\max(\expof{6}{a}, \expof{6}{b})$.

\examspace[0.5in]

\begin{solution}
Let $a=2$, $b=3$.
\end{solution}

\inbook{
\ppart\label{Prodexp} Let $\prod A$ be the product of all the elements in $A$.
Express $\expof{p}{\prod A}$ in terms of $\expof{p}{A}$.

\begin{solution}
\[
\expof{p}{\prod A} = \sum \expof{p}{A}.
\]
\end{solution}

\ppart Let $B$ also be a nonempty set of nonnegative integers.
Conclude that
\begin{equation}\label{gcdAUB}
\gcd(A \union B) = \gcd(\gcd(A), \gcd(B)).
\end{equation}
\hint Consider $\expof{p}{}$ of the left and right-hand sides of~\eqref{gcdAUB}.
You may assume
\begin{equation}\label{minAuBminmin}
\min(A \union B) = \min(\min(A), \min(B)).
\end{equation}

\examspace[3in]

\begin{solution}
There is also a simple direct proof using the fact that every common
factor divides the gcd:
\begin{proof}
The common factors of $A \union B$ are the same as the numbers
that are common factors of both $A$ and $B$ (by definition of common
factor of a set of numbers).  We know the common factors of $A$ are
the same as the factors of $\gcd(A)$ and similarly for $B$.  So the
common factors of both $A$ and $B$ are the same as the common factors
of $\gcd(A)$ and $\gcd(B)$.  Since the common factors $A \union B$ and
$\set{\gcd(A),\gcd(B)}$ are the same, they have the same gcd.
\end{proof}

There is a nice alternative proof using exponent algebra:
\begin{proof}

We show that 
\[
\expof{p}{\text{left-hand side of~\eqref{gcdAUB}}}
   = \expof{p}{\text{right-hand side of~\eqref{gcdAUB}}}
\]
Since $p$ is an arbitrary prime, this implies~\eqref{gcdAUB}.
Now
\begin{align*}
\expof{p}{\gcd(A \union B)}
   & = \min(\expof{p}{A \union B}) & \text{(by~\eqref{gcdexp})}\\
   & = \min(\expof{p}{A} \union  \expof{p}{B}) & \text{(by def of $\expof{p}{A \union B}$)}\\
   & = \min(\min(\expof{p}{A}), \min(\expof{p}{B})) & \text{(by~\eqref{minAuBminmin})}\\
   & = \min(\expof{p}{\gcd(A)}, \expof{p}{\gcd(B)}) & \text{(by~\eqref{gcdexp})}\\
   & = \expof{p}{(\gcd(\gcd(A), \gcd(B)))}  & \text{(by~\eqref{gcdexp})}
\end{align*}

\end{proof}

\end{solution}
}

\iffalse

\ppart Which of the parts above still hold if $A$ is infinite?

\begin{solution}
Only part~\ref\label{gcdexp}, because $\lcm(A) = \prod A = \infty$ if $A$ is infinite.

\end{solution}

\fi

\eparts

\end{problem}


%%%%%%%%%%%%%%%%%%%%%%%%%%%%%%%%%%%%%%%%%%%%%%%%%%%%%%%%%%%%%%%%%%%%%
% Problem ends here
%%%%%%%%%%%%%%%%%%%%%%%%%%%%%%%%%%%%%%%%%%%%%%%%%%%%%%%%%%%%%%%%%%%%%

\endinput
