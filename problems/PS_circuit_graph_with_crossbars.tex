\documentclass[problem]{mcs}

\begin{pcomments}
  \pcomment{PS_circuit_graph_with_crossbars}
  \pcomment{requires too much ingenuity}
  \pcomment{fall07 pset5-5}
  \pcomment{renamed from PS_graph_questions}
  \pcomment{edited ARM 3/16/14}
\end{pcomments}

\pkeywords{
 graph theory
 simple_circuit
 diameter
 connectivity
}

\begin{problem}

  In the cycle $C_{2n}$ of length $2n$, we'll call two vertices
  \emph{opposite} if they are on opposite sides of the cycle, that is
  that are distance $n$ apart in $C_n$.  Let $G$ be the graph formed
  from $C_{2n}$ by adding an edge, which we'll call a \emph{crossing
    edge}, between each pair of opposite vertices.  So $G$ has $n$
  crossing edges.

\bparts

\ppart Give a simple description of the shortest path between any two
vertices of $G$.

\hint Argue that a shortest path between two vertices in $G$ uses at
most one crossing edge.

\begin{solution} 

  Suppose a path, $Q$, in $G$ begins with a crossing edge, then
  follows a path, $P$, in $C_{2n}$ and ends with another crossing
  edge.  Replacing every vertex in $P$ by its opposite, you get a walk
  in $C_{2n}$ between the endpoints of $Q$, but the new walk is two
  edges shorter than $Q$.  This implies that a \emph{shortest} path in
  $G$ uses at most one crossing edge.

\begin{editingnotes}
Revise the following solution to describe the shortest path, then
explain why it is shortest.  Formula for distance not really needed.
\end{editingnotes}

  Suppose $v$ and $w$ are two points in $G$ and $v'$ and $w'$ are their
  opposites.  Let $c_{xy}$ be the distance in $C_{2n}$ between two
  vertices $x,y$.  Let $d$ be the distance in $G$ from $v$ to $w$.

  Clearly $d \leq c_{vw}$.  Also, $d \leq c_{v'w}+1$ because there is a
  path in $G$ of length $c_{v'w}+1$ from $v$ to $w$ which begins by
  crossing from $v$ to $v'$ and then takes a shortest path in $C_{2n}$
  from $v'$ to $w$.

  Can we have $d < c_{vw}$ or $d < c_{v'w}+1$?  No.  If a shortest path,
  $S$, in $G$ between $v$ and $w$ does not use any crossing edge, then $S$
  is actually a path in $C_{2n}$ and thus $d = \text{length}(S) \geq
  c_{vw}$.  On the other hand, if $S$ uses one crossing edge
  $\edge{p}{p'}$, then $S$ consists of a path, $S_1$, in $C_{2n}$ from $v$
  to $p$ followed by the crossing from $p$ to $p'$, followed by a path,
  $S_2$, in $C_{2n}$ from $p'$ to $w$.  But then $S_1$ followed by the
  path $(S_2)'$ of opposites of vertices in $S_2$, is a path in $C_{2n}$
  from $v$ to $p$ to $w'$ of length $d-1$.  Thus $d-1 \geq c_{v'w}$, that
  is, $d \geq c_{v'w}+1$.

  Thus $d = \min(c_{vw},c_{v'w}+1) = \min(c_{vw}, n-c_{vw}+1)$.
\end{solution}

\ppart What is the \emph{\idx{diameter}} of $G$, that is, the largest
distance between two vertices?

\begin{solution} If $n=2k$ the diameter is $k$.  If $n = 2k + 1$ the diameter
  is also $k$.  (For $d \in [1,n)$ find the maximum value of
    $\min(d,n-d+1)$.)
\end{solution}

\ppart Prove that the graph is not 4-connected.

\begin{solution} Every vertex of the graph has degree $3$.  Removing all $3$
  edges incident to a vertex disconnects the graph.  (If it was
  $4$ connected, no matter what $3$ edges we remove, the graph would
  have to remain connected.)
\end{solution}

\ppart Prove that the graph is 3-connected.

\begin{solution} Let $G'$ be the graph that remains after removing two edges.
  Removing any two edges that were orginally part of $C_{2n}$ splits
  $C_{2n}$ into two connected components, each of them paths.  Call
  them $P$ and $P'$.  Since the length of the paths and the two
  removed edges sum to $2n$, one path must have length at most $n-1$.
  Call this one $P$.  Pick a vertex $v$ within $P$.  $v$ is distance
  $n$ to some other vertex, $w$, in $C_{2n}$.  So $w$ must be in the
  other path $P'$.  The paths $P$, $P'$, and the edge $\edge{v}{w}$
  form a spanning tree in $G'$ which then must be connected.

  Removing at most one edge from $C_{2n}$ leaves $C_{2n}$ connected,
  hence $G'$ remains connected too.
\end{solution}

\iffalse  %Chromatic num not defined yet.  This part is routine
          %anyway. -ARM 10/18/09

\ppart What is the chromatic number of $G$? (It may depend on $n$.)

\begin{solution} If $n=2$, $G$ is the complete graph on four vertices, and so
  has chromatic number $4$.  Assume $n > 2$ from now on.

  Since $G$ has an edge, $\chi(G) \geq 2$.

  If $n$ is odd, color the vertices $v_i$ with $i$ even, red, and the
  vertices $v_i$ with $i$ odd, blue.  In this way if $v_i$ and $v_j$
  are adjacent in $C_{2n}$ they are colored differently.  If vertices
  $v_i$ and $v_j$ are at distance $n$ from each other in $C_{2n}$ then
  they are endpoints of a path of $n+1$ vertices in $C_{2n}$.  Since
  this is an even number of vertices, they must be colored differently
  as well.  Thus no edge of $G$ is improperly colored and this is a
  $2$-coloring of $G$.  Hence $\chi(G)=2$.

  If $n$ is even, $v_1, \dots, v_{n+1}$ form a cycle of odd length in
  $G$, so $\chi(G) \geq 3$.  Color $v_1$ and $v_n$ green.  Color
  $v_{n+1}, \dots, v_{2n}$ alternately red and blue.

  For $1 \leq i \leq n-2$, color the vertices $v_i$ red if $v_{i+n}$ is
  blue and vice versa.

  Clearly this is a valid coloring for $C_{2n}$, and the edges
  $\edge{v_i}{v_{i+n}}$ of $G$ have also been colored properly, for $1
  \leq i \leq n$.  If $i = 1$ or $n$, this is true because $v_i$ is green
  whereas $v_{i+n}$ is red or blue.  On the other hand if $2 \leq i \leq
  n-1$, the colors of $v_i$ have been chosen to make the edge properly
  colored.  Thus $\chi(G) =3$.
\end{solution}
\fi


\eparts

\end{problem}
