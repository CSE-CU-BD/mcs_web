\documentclass[problem]{mcs}

\begin{pcomments}
  \pcomment{PS_counting_problems}
  \pcomment{subsumed by PS_alphabet, TP_committees_of_5}
  \pcomment{Source: f01-ps9-3 and s01-ps7-5}
  \pcomment{soln to first part revised 5/12/13 ARM}
\end{pcomments}

\pkeywords{
counting
division_rule
permutation
}

%%%%%%%%%%%%%%%%%%%%%%%%%%%%%%%%%%%%%%%%%%%%%%%%%%%%%%%%%%%%%%%%%%%%%
% Problem starts here
%%%%%%%%%%%%%%%%%%%%%%%%%%%%%%%%%%%%%%%%%%%%%%%%%%%%%%%%%%%%%%%%%%%%%

\begin{problem}

\bparts

\ppart How many ways can $2n$ people be paired up?

\begin{solution}

\begin{equation}\label{2nfacover} 
\frac{(2n)!}{n! 2^n}.
\end{equation}

There are $(2n)!$ permutations of the $2n$ people.  A permutation can
be mapped to a pairing up of the $2n$ people by pairing consecutive
people in the permutation.  That is, one pair consists of the first
and second people, another pair of the third and fourth people,
through an $n$th pair of the $(2n-1)$st and $2n$th people in the
permutation.

Two permutations will map to the same set of pairs iff one permutation
can be changed into the other permuting the order of the consecutive
pairs or by switching the elements of a pair.  Since there are $n$
consecutive pairs, there are $n!$ ways to permute the pairs and $2^$
ways to switch the order within pairs.  So the mapping from
permutations to sets of pairs is $n!2^n$.  Now the Division
Rule~\bref{division_rule_sec} implies that the number of ways to
divide $2n$ people into $n$ pairs is given by~\eqref{2nfacover}.
\end{solution}

\ppart How many ways can you choose $n$ out of $2n$ objects, given that
$n$ of the $2n $ objects are identical?

\begin{solution}
The answer is $2^n$, since you can pick any subset from the $n$
nonidentical objects, and make up the rest with the identical ones. And the
number of subsets of $n$ different objects is $2^n$. 
\end{solution}

\ppart
Six women and nine men are on the faculty of a school's
EECS department.  The individuals are distinguishable.
How many ways are there to select a committee of 5
members if at least 1 woman must be on the committee?

\begin{solution}
\iffalse
        
\textbf{First method: (by brute force)} We need to count all possible
combinations of people such that there is at least one woman in every
combination, but we must remember not to count any combinations
multiple times.

We can have committees with 

1 woman, 4 men: $ \binom{6}{1} \binom{9}{4}$\\
2 women, 3 men: $ \binom{6}{2} \binom{9}{3}$\\
3 women, 2 men: $ \binom{6}{3} \binom{9}{2}$\\
4 women, 1 man: $ \binom{6}{4} \binom{9}{1}$\\
5 women, 0 men: $ \binom{6}{5} \binom{9}{0}$\\

So there are 
$ \binom{6}{1} \binom{9}{4} 
+ \binom{6}{2} \binom{9}{3}
+ \binom{6}{3} \binom{9}{2}
+ \binom{6}{4} \binom{9}{1}
+ \binom{6}{5} \binom{9}{0} = 2877$ different possibilities for
committees.

\medskip

\textbf{Second method:} Another way to solve this problem is to say
that \fi

There are $\binom{15}{5}$ different committees, and $\binom{9}{5}$
committees of just men.  So there are $\binom{15}{5} - \binom{9}{5}=
2877$ different possibilities for committees.

\iffalse
Note that $6 \cdot \binom{14}{4} =6006$ is not a correct answer.  The
Product Rule only applies when you are choosing two separate sets of
items which are independent of each other.  In this case, a woman is
chosen and then the remainder of the committee is chosen, but since
there may be women in the last set of $4$ chosen, we will be double
counting.  For example, if $A$ and $B$ are women, and $C, D, $ and $E$
are men, then we will count $A$, $\set{BCDE}$ as well as $B$,
$\set{ACDE}$.
\fi

\end{solution}

\eparts
\end{problem}

%%%%%%%%%%%%%%%%%%%%%%%%%%%%%%%%%%%%%%%%%%%%%%%%%%%%%%%%%%%%%%%%%%%%%
% Problem ends here
%%%%%%%%%%%%%%%%%%%%%%%%%%%%%%%%%%%%%%%%%%%%%%%%%%%%%%%%%%%%%%%%%%%%%

\endinput
