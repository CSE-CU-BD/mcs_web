\documentclass[problem]{mcs}

\begin{pcomments}
  \pcomment{PS_increasing_sequences_bijection}
  \pcomment{zabel 3/17/2018 for s18 ps4}
\end{pcomments}

\pkeywords{
  bijection
  infinite sets
  sequences
  increasing
}

%%%%%%%%%%%%%%%%%%%%%%%%%%%%%%%%%%%%%%%%%%%%%%%%%%%%%%%%%%%%%%%%%%%%%
% Problem starts here
%%%%%%%%%%%%%%%%%%%%%%%%%%%%%%%%%%%%%%%%%%%%%%%%%%%%%%%%%%%%%%%%%%%%%

\newcommand\Inc{\text{Inc}}

\begin{problem}
  Let $\nngint^\omega$ be the set of infinite sequences of natural numbers, and call a sequence $(a_0,a_1,a_2,\ldots)\in\nngint^\omega$ \emph{strictly increasing} if $a_0 < a_1 < a_2 < \cdots$. Define the subset $\Inc\subset\nngint^\omega$ as the set of strictly increasing sequences. Prove that $\nngint^\omega \bij \Inc$ by \emph{explicitly} constructing a bijection $f:\nngint^\omega \to \Inc$. Carefully prove that
  \begin{itemize}
  \item $f(s)\in \Inc$ for each $s\in \nngint^\omega$, meaning $f$ has domain $\nngint^\omega$ and range $\Inc$, and
  \item $f$ is a function, total, injective, and surjective.
  \end{itemize}

  \noindent\hint Think sums, but remember that $0\in\nngint$.
  
  \begin{solution}
    For a sequence $s = (s_0,s_1,s_2,\ldots)$, define
    \begin{equation*}
      f(s) \eqdef (s_0, s_0+s_1+1, s_0+s_1+s_2+2, \ldots),
    \end{equation*}
    where the $n$th element equals $s_0 + s_1 + \cdots + s_n + n$. To verify that $f(s)\in \Inc$, we must check that $f(s)[k] < f(s)[k+1]$ for each $k\in\nngint$, i.e.,
    \begin{equation*}
      0 < f(s)[k+1] - f(s)[k] = (s_0 + s_1 + \cdots + s_{k+1} + (k+1)) - (s_0 + s_1 + \cdots + s_k + k) = s_{k+1} + 1.
    \end{equation*}
    But this is true because $s_{k+1}\in\nngint$, so $s_{k+1}+1 > 0$. (Note: this is why we have the $+n$ term when defining $f$. If $f(s)$ were defined simply as the sequence of partial sums $(s_0,s_0+s_1,\ldots)$, then $f(s)$ would only be \emph{weaking} increasing anywhere $s$ has a $0$ term.) We now verify the four required properties:

    \begin{description}
    \item[\boldmath $f$ is a function:] For each $s\in\nngint^\omega$, $f(s)$ takes only the specified value, so $f$ has at most one arrow out.
      
    \item[\boldmath $f$ is total:] For each $s\in\nngint^\omega$, the sequence $f(s)$ exists and belongs to $\Inc$ as shown above, so $f$ has at least one arrow out.
      
    \item[\boldmath $f$ is injective:] Suppose $f(s) = f(t)$ where $s=(s_0,s_1,\ldots)$ and $t=(t_0,t_1,\ldots)$ are two integer sequences. We must prove that $s = t$. Because $f(s) = f(t)$ it follows that $s_0 = f(s)[0] = f(t)[0] = t_0$. Similarly, for each $k \ge 1$ we have $s_{k} = f(s)[k]-f(s)[k-1] - 1$ and $t_{k} = f(t)[k]-f(t)[k-1] - 1$ as computed above, so $s_k = t_k$ as well. Thus $s=t$, as required.
      
    \item[\boldmath $f$ is surjective:] Let $v = (v_0,v_1,\ldots) \in \Inc$ be a strictly increasing sequence; we must find some sequence $u = (u_0,u_1,\ldots)\in\nngint^\omega$ with $f(u) = v$. As in the previous paragraph, let's choose the sequence with $u_0 = v_0$ and $u_{k} = v_{k} - v_{k-1} - 1$ for each $k \ge 1$. The elements of this new sequence $u$ are nonnegative because $v$ was assumed to be strictly increasing. We can verify that $f(u)[0] = u_0 = v_0$ and
      \begin{align*}
        f(u)[k] &= u_0 + u_1 + \cdots + u_k + k \\
                &= (v_0) + (v_1 - v_0 - 1) + (v_2 - v_1 - 1) + \cdots + (v_k - v_{k-1}-1) + k \\
                &= v_k
      \end{align*}
      because all other terms in the sum cancel out. Thus $f(u) = v$, as required.      
    \end{description}
    
    Note: if we define the function $g(v) \eqdef (v_0, v_1-v_0-1, v_2-v_1-1,\ldots)$, then the ``surjective'' section above verifies that $g$ has the correct domain and range (i.e., $g(v)\in\nngint^\omega$) and that $f(g(v)) = v$ for each $v\in \Inc$. The ``injective'' section verifies that $g(f(s))$ for every $s\in\nngint^\omega$. Taken together, this proves that $g = f^{-1}$. Usually when proving that a function $f$ is a bijection you'll end up describing the inverse function $f^{-1}$, whether implicitly (as above) or explicitly (as in this paragraph).
  \end{solution}

\end{problem}


%%%%%%%%%%%%%%%%%%%%%%%%%%%%%%%%%%%%%%%%%%%%%%%%%%%%%%%%%%%%%%%%%%%%%
% Problem ends here
%%%%%%%%%%%%%%%%%%%%%%%%%%%%%%%%%%%%%%%%%%%%%%%%%%%%%%%%%%%%%%%%%%%%%

\endinput
