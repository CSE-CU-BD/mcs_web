\documentclass[problem]{mcs}
\begin{pcomments}
  \pcomment{CP_leaves_vs_nodes}
  \pcomment{ARM 10/3/17}
\end{pcomments}

\pkeywords{
  trees
  binary_trees
  depth
  recursive_data
  structural_induction
  }

%%%%%%%%%%%%%%%%%%%%%%%%%%%%%%%%%%%%%%%%%%%%%%%%%%%%%%%%%%%%%%%%%%%%%
% Problem starts here
% %%%%%%%%%%%%%%%%%%%%%%%%%%%%%%%%%%%%%%%%%%%%%%%%%%%%%%%%%%%%%%%%%%%%

\begin{problem}
For $T \in \brnch$, define
\begin{align*}
\text{leaves}(T)   & \eqdef \set{S \in \subbrn{T} \suchthat S\ \text{is a leaf}}\\
\text{internal}(T) & \eqdef \set{S \in \subbrn{T} \suchthat S \in \brnchng}
\end{align*}

Prove that in a recursive tree, there is always one more leaf than
there are internal subtrees:

\begin{lemma*}
If $T \in \rectr$, then
\[
\card{\text{leaves}(T)} = 1 + \card{\text{internal}(T)}.
\]
\end{lemma*}
\begin{solution}

\begin{proof}
The proof is by structural induction on the definition of \rectr.

\TBA{rest of proof}
\end{proof} 
\end{solution}
\end{problem}

%%%%%%%%%%%%%%%%%%%%%%%%%%%%%%%%%%%%%%%%%%%%%%%%%%%%%%%%%%%%%%%%%%%%%
% Problem ends here
%%%%%%%%%%%%%%%%%%%%%%%%%%%%%%%%%%%%%%%%%%%%%%%%%%%%%%%%%%%%%%%%%%%%%

\endinput
