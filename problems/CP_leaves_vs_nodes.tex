\documentclass[problem]{mcs}
\begin{pcomments}
  \pcomment{CP_leaves_vs_nodes}
  \pcomment{prereq for generalization in PS_shared_leaves}
  \pcomment{ARM 10/5/17}
\end{pcomments}

\pkeywords{
  trees
  binary_trees
  depth
  recursive_data
  structural_induction
  }

%%%%%%%%%%%%%%%%%%%%%%%%%%%%%%%%%%%%%%%%%%%%%%%%%%%%%%%%%%%%%%%%%%%%%
% Problem starts here
% %%%%%%%%%%%%%%%%%%%%%%%%%%%%%%%%%%%%%%%%%%%%%%%%%%%%%%%%%%%%%%%%%%%%

\begin{problem}
For $T \in \brnch$, define
\begin{align*}
\text{leaves}(T)   & \eqdef \set{S \in \subbrn{T} \suchthat S \in \leafset}\\
\text{internal}(T) & \eqdef \set{S \in \subbrn{T} \suchthat S \in \brnchng}.
\end{align*}

\bparts

\ppart Explain why it follows immediately from the definitions that if
$T \in \brnchng$,
\begin{align}
\text{internal}(T)
& = \set{T} \union \text{internal}(\leftsub{(T)})
            \union \text{internal}(\rightsub{(T)}), \tag{trnlT}\\
\text{leaves}(T)
& =  \text{leaves}(\leftsub{(T)}) \union
     \text{leaves}(\rightsub{(T)}). \tag{lvT}
\end{align}

\begin{staffnotes}
It is a good idea to remark at the start that this is a triviality
intended just to review the definitions.
\end{staffnotes}

\begin{solution}
\inhandout{We know that}\inbook{By equation~(\bref{unionTLR}),}
\begin{equation}\tag{TLR}
\subbrn{T} = \set{T} \union \subbrn{\leftsub{(T)}} \union \subbrn{\rightsub{(T)}}.
\end{equation}
$T \in \brnchng$, so~(trnlT) follows by intersecting both sides of
this equality with \brnchng, and $T$ is not a leaf, so~(lvT) follows
by intersecting both sides of~(TLR) with \leafset.
\end{solution}

\ppart Prove by structural induction on the definition of \rectr\
\inbook{(Definition~\bref{def:rectree})}
that in a recursive tree, there is always one more leaf than
there are internal subtrees:

\begin{lemma*}
If $T \in \rectr$, then
\begin{equation}\tag{lf-vs-in}
\card{\text{leaves}(T)} = 1 + \card{\text{internal}(T)}.
\end{equation}
\end{lemma*}
\begin{solution}

\begin{proof}

\inductioncase{Base case}: ($T \in \leafset$).  The only subtree of
$T$ is $T$ itself, so $\text{leaves}(T) = \set{T}$ and
$\text{internal}(T) = \emptyset$.  Therefore,
\[
\card{\text{leaves}(T)} = 1 = 1 + 0 
= 1 + \card{\emptyset} = 1 + \card{\text{internal}(T)}.
\]

\inductioncase{Constructor case}: ($T \in \brnchng$).
We may assume by induction hypothesis that
\begin{align*}
\card{\text{leaves}(\leftsub{(T)})} & = 1 + \card{\text{internal}(\leftsub{(T)})},\\
\card{\text{leaves}(\rightsub{(T)})} & = 1 + \card{\text{internal}(\rightsub{(T)})},
\end{align*}
Also, for recursive $T$, the unions in~(trnlT) and~(lvT) are disjoint,
so
\begin{align}
\card{\text{internal}(T)}
& = 1+ \card{\text{internal}(\leftsub{(T)})} +
    \card{\text{internal}(\rightsub{(T)})},\tag{ctrnlT}\\
\card{\text{leaves}(T)}
& = \card{\text{leaves}(\leftsub{(T)})} +
    \card{\text{leaves}(\rightsub{(T)})}. \tag{clvT}\\
\end{align}
Therefore
\begin{align*}
\card{\text{leaves}(T)} 
& = \card{\text{leaves}(\leftsub{(T)})} +
     \card{\text{leaves}(\rightsub{(T)})}
        & \text{(by~(clvT))}\\
& = (1 + \card{\text{internal}(\leftsub{(T)})}) +
    (1 + \card{\text{internal}(\rightsub{(T)})})
        & \text{(by ind hypothesis)}\\
& = 1 + (1+ \card{\text{internal}(\leftsub{(T)})} +
        \card{\text{internal}(\rightsub{(T)})})\\
& = 1+ \card{\text{internal}(T)}
        & \text{(by~(ctrnlT))}.
\end{align*}
\end{proof} 

\end{solution}

\eparts

\end{problem}

%%%%%%%%%%%%%%%%%%%%%%%%%%%%%%%%%%%%%%%%%%%%%%%%%%%%%%%%%%%%%%%%%%%%%
% Problem ends here
%%%%%%%%%%%%%%%%%%%%%%%%%%%%%%%%%%%%%%%%%%%%%%%%%%%%%%%%%%%%%%%%%%%%%

\endinput
