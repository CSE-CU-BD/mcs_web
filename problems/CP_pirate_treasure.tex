\documentclass[problem]{mcs}

\begin{pcomments}
  \pcomment{CP_pirate_treasure}
  \pcomment{first used SP12}
  \pcomment{revised by drewe 2/21/2012}
\end{pcomments}

\pkeywords{
  number_theory
  modular_arithmetic
}

%%%%%%%%%%%%%%%%%%%%%%%%%%%%%%%%%%%%%%%%%%%%%%%%%%%%%%%%%%%%%%%%%%%%%
% Problem starts here
%%%%%%%%%%%%%%%%%%%%%%%%%%%%%%%%%%%%%%%%%%%%%%%%%%%%%%%%%%%%%%%%%%%%%

\begin{problem}
\bparts
\ppart
Ten pirates find a chest filled with gold and silver coins. There are twice as many silver coins in the chest as there are gold. They divide the gold coins in such a way that the difference in the number of coins given to any two pirates is not divisible by $10$.  They will only take the silver coins if it is possible to divide them the same way.  Is this possible, or will they have to leave the silver behind?  Prove your answer.

\begin{solution}
They can divide the coins in such a way only if the number of coins is $0+1+2+\ldots+9=45$ (mod $10$). Hence the number of gold coins is $5$ mod $10$, and the number of silver coins is therefore $0$ mod $10$.  As $0$ is not congruent to $5$ mod $10$, they cannot divide the silver coins in the same way.
\end{solution}

\ppart
There are also 3 sacks in the chest, containing 5, 49, and 51 rubies respectively.  The treasurer of the pirate ship is bored and decides to play a game with the following rules: 
\begin{enumerate}
\item He can join any two piles together into one pile and 
\item He can divide a pile with an even number of rubies into two piles of equal size.  
\end{enumerate} 
He makes one move every day, and he will finish the game when he has divided the rubies into 105 piles of one.  Is it possible for him to finish the game?

\begin{solution}
From $(5,49,51)$ you have only three choices: $(54, 51)$, $(49, 56)$, and $(5,100)$.  Consider $(54,51)$.  You can only combine two elements or divide one by two (if it's divisble by 2).  Notice that 54 and 51 are divisible by 3.  Any set of numbers you can possibly obtain from $(54,51)$ must all be divisible by 3 because adding two multiples of 3 produces a multiple of 3, and dividing an even multiple of 3 by 2 still produces a multiple of 3.  Similar reasoning rules out each of the other pairs, because each has components that share a prime factor beside 2.
\end{solution}

\eparts
\end{problem}

%%%%%%%%%%%%%%%%%%%%%%%%%%%%%%%%%%%%%%%%%%%%%%%%%%%%%%%%%%%%%%%%%%%%%
% Problem ends here
%%%%%%%%%%%%%%%%%%%%%%%%%%%%%%%%%%%%%%%%%%%%%%%%%%%%%%%%%%%%%%%%%%%%%


\endinput
