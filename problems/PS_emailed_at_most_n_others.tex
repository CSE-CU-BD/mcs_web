\documentclass[problem]{mcs}

\begin{pcomments}
  \pcomment{PS_emailed_at_most_n_others}
  \pcomment{from: S17.ps2}
  \pcomment{small perturbation of PS_emailed_at_most_2_others}
\end{pcomments}

\pkeywords{
  predicate
  formula
  translate
  sentence
  first-order logic
}

%%%%%%%%%%%%%%%%%%%%%%%%%%%%%%%%%%%%%%%%%%%%%%%%%%%%%%%%%%%%%%%%%%%%%
% Problem starts here
%%%%%%%%%%%%%%%%%%%%%%%%%%%%%%%%%%%%%%%%%%%%%%%%%%%%%%%%%%%%%%%%%%%%%

\begin{problem}
\bparts
\ppart Translate the following sentence into a predicate formula:
\begin{quote}
There is a student who has e-mailed at most $n$ other people in the class,
besides possibly himself.
\end{quote}

The domain of discourse should be the set of students in the class; in
addition, the only predicates that you may use are 
\begin{itemize}
\item equality,
\item $E(x,y)$, meaning that ``$x$ has sent e-mail to $y$.''
\end{itemize}

\begin{solution}
A good way to begin tackling this problem is by trying to translate parts
of the sentence. First of all, our formula must be of the form
\[
\exists x. \text{atmost}_n(x)
\]
where $\text{atmost}_n(x)$ should be a formula that says that ``student $x$ has
e-mailed at most $n$ other people in the class, besides possibly
himself''.

One way to express $\text{atmost}_n(x)$ is ``whenever we find a student
$s$ who has been e-mailed by $x$, this student is either $x$ or one of
a particular set of students $y_0, y_1 \dots, y_{n-1}$.''  A formula
expressing that $s$ is either $x$ or one of students $y_0, y_1 \dots,
y_{n-1}$ is
\[
s = x \QOR s = y_0 \QOR s = y_1 \QOR \dots \QOR s = y_{n-1}.
\]
So we can express ``whenever we find a student $s$ who has been
e-mailed by $x$, \dots'' with the formula
\begin{align*}
 \forall s.\,[
     & E(x,s) \QIMPLIES \dots].
\end{align*}

Adding existential quantifiers to say that there are such students
$y_0, y_1 \dots, y_{n-1}$, we finish with
\begin{align*}
\text{atmost}_n(x) \eqdef\
   & \exists y_0.\exists y_1.\dots\exists y_{n-1}.\\
   &    \quad \forall s.\,[E(x,s) \QIMPLIES\\
   &       \qquad  (s = x \QOR s = y_0 \QOR s = y_1 \QOR \dots \QOR s = y_{n-1})].
\end{align*}

At this point you may be thinking that this formula actually says that
``$x$ has e-mailed \emph{exactly} $n$ students besides possibly
himself.''  But remember, the $y_i$'s may not all be different
students.  Of course we could have added that constraint with a long
\QAND\ formula:
\[\begin{array}{rllcrl}
 y_0 \neq y_1\ \QAND & y_0 \neq y_2 & \QAND\ y_1 \neq y_3\ \QAND &\dots & \QAND\ y_0 &\neq y_{n-1}\ \QAND\\
                    & y_1 \neq y_2 & \QAND\ y_1 \neq y_3\ \QAND &\dots & \QAND\ y_1 &\neq y_{n-1}\ \QAND\\
                    &              &                           &\vdots\\
                    &              &                           &      &     y_{n-2} &\neq y_{n-1}.
\end{array}\]
In the next part of this problem, we'll see a simple expression that
does this job and would be much shorter to write out.
\end{solution}

\ppart Explain how you would use your predicate formula (or some variant of it)
to express the following two sentences.
\begin{enumerate}
\item\label{Esatn} There is a student who has emailed at least $n$ other people in the class,
besides possibly himself.
\item\label{Esexn} There is a student who has emailed exactly $n$ other people in the class,
besides possibly himself.
\end{enumerate}

\begin{solution}
Student $x$ has emailed at least $n$ students just when he has
\emph{not} emailed at most $n-1$ students.  So define
\[
\text{atleast}_n(x) \eqdef \QNOT(\text{atmost}_{n-1}(x)),
\]
and a formula for item~\ref{Esatn}.\ becomes
\[
\exists x.\, \text{atleast}_n(x).
\]

Now student $x$ has emailed \emph{exactly} $n$ students when he has
emailed at most and also at least $n$ students, so for
item~\ref{Esexn}.\ we have
\[
\text{exactly}_n \eqdef \exists x.\, \text{atmost}_n(x) \QAND \text{atleast}_n(x).
\]

\end{solution}

\eparts

\end{problem}

%%%%%%%%%%%%%%%%%%%%%%%%%%%%%%%%%%%%%%%%%%%%%%%%%%%%%%%%%%%%%%%%%%%%%
% Problem ends here
%%%%%%%%%%%%%%%%%%%%%%%%%%%%%%%%%%%%%%%%%%%%%%%%%%%%%%%%%%%%%%%%%%%%%

\endinput

