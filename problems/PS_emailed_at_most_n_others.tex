\documentclass[problem]{mcs}

\begin{pcomments}
  \pcomment{PS_emailed_at_most_n_others}
  \pcomment{small perturbation of PS_emailed_at_most_2_others}
\end{pcomments}

\pkeywords{
  predicate
  formula
  translate
  sentence
  first-order logic
}

%%%%%%%%%%%%%%%%%%%%%%%%%%%%%%%%%%%%%%%%%%%%%%%%%%%%%%%%%%%%%%%%%%%%%
% Problem starts here
%%%%%%%%%%%%%%%%%%%%%%%%%%%%%%%%%%%%%%%%%%%%%%%%%%%%%%%%%%%%%%%%%%%%%

\begin{problem}
\bparts
\ppart Translate the following sentence into a predicate formula:
\begin{quote}
There is a student who has e-mailed at most $n$ other people in the class,
besides possibly himself.
\end{quote}

The domain of discourse should be the set of students in the class; in
addition, the only predicates that you may use are 
\begin{itemize}
\item equality,
\item set membership ($\in$), and
\item $E(x,y)$, meaning that ``$x$ has sent e-mail to $y$.''
\end{itemize}

\begin{solution}
A good way to begin tackling this problem is by trying to translate parts
of the sentence. First of all, our formula must be of the form
\[
\exists x. P(x)
\]
where $P(x)$ should be a formula that says that ``student $x$ has
e-mailed at most $n$ other people in the class, besides possibly
himself''.

One way to write $P(x)$ is ``whenever we meet a student that has been e-mailed by $x$, this
student is either $x$ or $y_0$ or $y_1 \ldots$ or $y_{n-1}$, where $y_0 \ldots y_{n-1}$
are particular students in the class.'' 

To write the part ``whenever we meet a student that has been e-mailed by $x$, this
student is either $x$ or $y_0$ or $y_1 \ldots$ or $y_{n-1}$,'' we write 
\[
\forall s\, \paren{ E(x,s) \QIMPLIES s \in \{x, y_0, y_1, \ldots, y_{n-1}\}}.
\]
By adding the appropriate existential quantifier, we get
\[
P(x) \eqdef \quad\exists y_0.\exists y_1.\ldots\exists y_{n-1}. 
\forall s. \paren{ E(x,s) \QIMPLIES s \in \{x, y_0, y_1, \ldots, y_{n-1}\}}.
\]
At this point you may be thinking that $P(x)$ says that ``$x$ has
e-mailed \emph{exactly} $n$ students besides possibly
himself.''  However we did not require that $y_i$ and $y_j$
(where $i \neq j$) refer to distinct students, or that they
be different from $x$.

Overall the full formula is:
\begin{equation}\label{EEEAs}
\exists x.\, \exists y_0.\exists y_1.\ldots\exists y_{n-1}. \forall s. 
\paren{E(x,s) \QIMPLIES s \in \{x, y_0, y_1, \ldots, y_{n-1}\}}.
\end{equation}

\end{solution}


\ppart Explain how you would use your predicate formula (or some derivative of it)
to express the following two sentences. You may use additional predicates
as you see fit.
\begin{enumerate}
\item There is a student who has emailed at least $n$ other people in the class,
besides possibly himself.
\item There is a student who has emailed exactly $n$ other people in the class,
besides possibly himself.
\end{enumerate}

\begin{solution}
You can think of a student who has emailed at least $n$ students as a
student who has \emph{not} emailed at most $n-1$ students in the class. 
\begin{equation}\label{EEEBs}
\exists x.\, \exists y_0.\exists y_1.\ldots\exists y_{n-2}. \forall s. 
\paren{\QNOT{\paren{E(x,s) \QIMPLIES s \in \{x, y_0, y_1, \ldots, y_{n-2}\}}}}.
\end{equation}

You can think of a student who has emailed exactly $n$ students as a
someone who has emailed at least $n$ students \emph{and} at most $n$ students
in the class.
\begin{equation}\label{EEECs}
\exists x.\, \exists y_0.\exists y_1.\ldots\exists y_{n-1}. \forall s. 
\paren{\QNOT{\paren{E(x,s) \QIMPLIES s \in \{x, y_0, y_1, \ldots, y_{n-2}\}}}
\QAND \paren{E(x,s) \QIMPLIES s \in \{x, y_0, y_1, \ldots, y_{n-1}\}}}.
\end{equation}

You could also use the original formula \eqref{EEEAs} and require that
$x, y_0, y_1, \ldots, y_{n-1}$ all refer to distinct students.

\end{solution}

\eparts

\end{problem}

%%%%%%%%%%%%%%%%%%%%%%%%%%%%%%%%%%%%%%%%%%%%%%%%%%%%%%%%%%%%%%%%%%%%%
% Problem ends here
%%%%%%%%%%%%%%%%%%%%%%%%%%%%%%%%%%%%%%%%%%%%%%%%%%%%%%%%%%%%%%%%%%%%%

\endinput

