\documentclass[problem]{mcs}

\begin{pcomments}
  \pcomment{TP_Annuities}
  \pcomment{Converted from prob1.scm by
    scmtotex and dmj on Sun 13 Jun 2010 03:24:16 PM EDT}
  \pcomment{revised by drewe 2 August 2011}
  \pcomment{soln edited ARM 10/24/13}
\end{pcomments}

\pkeywords{
  geometric_series
  geometric_sum
  interest
  annuity
}


\begin{problem}
Assuming you could be sure of getting a 4\% return forever on money
you keep in the bank, what would be a proper price to pay for an annuity that
paid you \$10,000 a year forever starting a year from now?

% Answer without commas or dollar signs. For example, if you want to
% enter \$140,000, you should type

% \begin{equation*}
% 140000
% 
% \end{equation*}

\begin{solution}
\$250,000.

With a \%4 interest rate, the current value of \$10,000 paid out $i$
years from now is $10,000/(1.04)^{i}$.  Summing this for $i=1$ to
$\infty$, we get
\[
10,000 \paren{\frac{1}{1-1/1.04} - 1} = 1.04/0.04 - 1 =
10,000\frac{1}{0.04} = 250,000.
\]

Alternatively, you need to put \$$n$ in the bank today to collect
$0.04\, n = \$10,000$ every year starting a year from now, so $n=
10000/0.04 = 250,000$.
\end{solution}

\end{problem}

\endinput
