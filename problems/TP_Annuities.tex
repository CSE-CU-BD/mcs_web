\documentclass[problem]{mcs}

\begin{pcomments}
    \pcomment{TP_Annuities}
    \pcomment{Converted from prob1.scm
              by scmtotex and dmj
              on Sun 13 Jun 2010 03:24:16 PM EDT}
    \pcomment{revised by drewe 2 August 2011}
\end{pcomments}

\begin{problem}

%% type: short-answer
%% title: Annuities

Assuming you could be sure of getting a 4\% return forever on money
you keep in the bank, what would be a proper price to pay for an annuity that
paid you \$10,000 a year forever starting a year from now?

% Answer without commas or dollar signs. For example, if you want to
% enter \$140,000, you should type

% \begin{equation*}
% 140000
% 
% \end{equation*}

\begin{solution}
\$250,000

The current value of \$10,000 paid out $i$ years from now is
$10,000/(1.04)^{i}$.  Summing this for $i=1,2,\dots$ we get \$250,000.


Alternatively, \$250,000 put in the bank today will allow us to
collect $0.04(\$250,000) = \$10,000$ every year, starting a year from
now.
\end{solution}

\end{problem}

\endinput
