\documentclass[problem]{mcs}

\begin{pcomments}
  \pcomment{PS_equal_birthdays2}
  \pcomment{by ARM 4/23/11}
\end{pcomments}

\pkeywords{
  independence
  mutual
  pairwise
  uniform
  distribution
}

%%%%%%%%%%%%%%%%%%%%%%%%%%%%%%%%%%%%%%%%%%%%%%%%%%%%%%%%%%%%%%%%%%%%%
% Problem starts here
%%%%%%%%%%%%%%%%%%%%%%%%%%%%%%%%%%%%%%%%%%%%%%%%%%%%%%%%%%%%%%%%%%%%%

\begin{problem}
\begin{definition*}
A random variable, $R$, is independent of a set $\set{R_1,R_2,\dots}$ of
random variables iff the event $[R=r]$ is independent of the event
\[
[R_1 = r_1 \QAND R_2 = r_2 \QAND \cdots]
\]
for all values $r, r_1, r_2, \dots$.
\end{definition*}
Let $R$, $S$, and $T$ be random variables with the same codomain, $V$.

\bparts

\ppart\label{R=S=b} Prove that if $R$ is uniform ---that is, $\pr{R=b}
= p$ for all $b \in V$ where $p = 1/\card{V}$ ---and $R$ is
independent of $S$, then $\pr{R=S} = p$.

\begin{solution}
\begin{staffnotes}
Given the first line below as a hint if students are stuck.
\end{staffnotes}

\begin{proof}
\begin{align*}
\pr{R=S}
   & = \sum_{b \in V} \prcond{R=S}{S=b}\pr{S=b}
         & \text{(total probability law)}\\
   & = \sum_{b \in V} \prcond{R=b}{S=b} \cdot \pr{S=b}\\
   & = \sum_{b \in V} \pr{R=b} \cdot \pr{S=b}
          & \text{$R$ independent of $S$}\\
   & = \sum_{b \in V} p \cdot \pr{S=b}\\
   & = p \cdot \sum_{b \in V} \pr{S=b} = p \cdot 1 = p.
\end{align*}
The second equality follows from the fact that $[R=S] \intersect
[S=b]$ is the same event as $[R=b] \intersect [S=b]$.
\end{proof}

\end{solution}

\ppart Prove that if $R$ has a uniform distribution, and $R$ is
independent of $\set{S,T}$, then $[R=S]$ is independent of $[S=T]$.

\begin{staffnotes}
Give
\[
[R=S \QAND\ S=T] = \lgunion_{b \in V} [R=b \QAND\ S=b \QAND\ T=b]
\]
as a hint if students are stuck.

This proof follows directly enough from the definitions, but I have a
feeling I'm overlooking a simpler argument that does not require
part~\eqref{R=S=b} beforehand.  --ARM 4/25/11
\end{staffnotes}

\begin{solution}

\begin{align*}
\pr{R=S \QAND\ S=T}
    & = \Prob{\lgunion_{b \in V} [R=b \QAND\ S=b \QAND\ T=b]}\\
    & = \sum_{b \in V} \pr{R=b \QAND\ S=b \QAND\ T=b}
          & \text{(disjoint sum rule)}\\
    & = \sum_{b \in V} \pr{R=b} \cdot \pr{S=b \QAND\ T=b}
          & \text{($R$ independent of $\set{S,T}$)}\\
    & =  \sum_{b \in V} p \cdot \pr{S=b \QAND\ T=b}
             & (\text{$R$ is uniform})\\
    & = p \cdot \sum_{b \in V} \pr{S=b \QAND\ T=b}\\
    & = p \cdot \pr{\lgunion_{b \in V} [S=b \QAND\ T=b]}
          & \text{(disjoint sum rule)}\\
    & = p \cdot \pr{S=T}\\
    & = \pr{R=S} \cdot \pr{S=T}.
          & \text{(part~\eqref{R=S=b})}
\end{align*}

\end{solution}


\ppart Let $V = \set{1,2,3}$ and $R,S,T$ take the following values
with equal probability, namely, 1/6:
\[
111, 211, 123, 223, 132, 232.
\]
Verify that
\begin{enumerate}

\item $S$ and $T$ have a uniform distribution,\label{STunif}
\item $R$ is independent of $\set{S,T}$,\label{RindST}
\item The event $[R=S]$ is not independent of $[S=T]$.\label{R=SnotS=T}
\end{enumerate}

\begin{solution}
There are two outcomes with $S=i$ for each $i \in V$, so $\pr{S=i} =
1/3$ for all $i \in V$, that is, $S$ is uniform.  Likewise for $T$.
This proves~\ref{STunif}.

To prove independence, note that $1,s,t$ is a possible sequence of
values $R,S,T$ iff $2,s,t$ is.  This implies that $\pr{R=1} = \pr{R=2}
= 1/2$, and also that $\pr{R=i \QAND S=s \QAND T = t} =1/6 = $
\TBA{remaining cases}.  This proves~\ref{RindST}.

Finally, there are two outcomes out of six with $R = S$, so $\pr{R=S}
= 1/3$.  But of the two outcomes with $S=T$, only one has $R=S$,
namely when $R=S=T=1$, so $\prcond{R=S}{S=T} = 1/2 \neq 1/3 =
\pr{R=S}$.  So $[R=S]$ is not independent of $[S=T]$, which
proves~\ref{R=SnotS=T}.

\end{solution}

\eparts
\end{problem}


%%%%%%%%%%%%%%%%%%%%%%%%%%%%%%%%%%%%%%%%%%%%%%%%%%%%%%%%%%%%%%%%%%%%%
% Problem ends here
%%%%%%%%%%%%%%%%%%%%%%%%%%%%%%%%%%%%%%%%%%%%%%%%%%%%%%%%%%%%%%%%%%%%%

\endinput
