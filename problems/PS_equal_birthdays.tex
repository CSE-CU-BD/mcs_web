\documentclass[problem]{mcs}

\begin{pcomments}
  \pcomment{PS_equal_birthdays}
  \pcomment{PS_dependent_pairs is a sequel}
  \pcomment{CP_equal_ranvars covers first part}
  \pcomment{by ARM 4/23/11}
\end{pcomments}

\pkeywords{
  independence
  mutual
  pairwise
  uniform
  distribution
}

%%%%%%%%%%%%%%%%%%%%%%%%%%%%%%%%%%%%%%%%%%%%%%%%%%%%%%%%%%%%%%%%%%%%%
% Problem starts here
%%%%%%%%%%%%%%%%%%%%%%%%%%%%%%%%%%%%%%%%%%%%%%%%%%%%%%%%%%%%%%%%%%%%%

\begin{problem}
Let $R$, $S$, and $T$ be random variables with the same codomain, $V$.

\bparts

\ppart\label{R=S=b} Suppose $R$ is uniform---that is,
\[
\pr{R=b} = \frac{1}{\card{V}},
\]
for all $b \in V$---and $R$ is independent of $S$.  Originally this
text had the following argument:

\begin{quote}
The probability that $R=S$ is the same as the probability that $R$
takes whatever value $S$ happens to have, therefore
\begin{equation}\label{RS1V}
\pr{R=S} = \frac{1}{\card{V}}\,.
\end{equation}
\end{quote}

Are you convinced by this argument?  \inhandout{We decided to replace
  it by a reference to this problem.  We'd like your advice on whether
  it should be put back in the text.  Before advising us, w}%
\inbook{W}rite out a careful proof of~\eqref{RS1V}.

\hint The event $[R=S]$ is a disjoint union of events
\[
[R=S] = \lgunion_{b \in V} [R = b \QAND\ S=b].
\]

\begin{solution}

\begin{proof}
\begin{align*}
\pr{R=S}
   & = \Pr{\lgunion_{b \in V} [R=b \QAND\ S=b]}\\
   & = \sum_{b \in V} \pr{R=b \QAND\ S=b}
         & \text{(disjoint sum rule)}\\
   & = \sum_{b \in V} \pr{R=b} \cdot \pr{S=b}
          & \text{($R,S$ independent)}\\
   & = \sum_{b \in V} \frac{1}{\card{V}} \cdot \pr{S=b}
         &\text{($R$ is uniform)}\\
   & = \frac{1}{\card{V}} \cdot \sum_{b \in V} \pr{S=b}\\
   & = \frac{1}{\card{V}} \cdot 1 = \frac{1}{\card{V}}.
\end{align*}
This proves~\eqref{RS1V}.
\end{proof}

\begin{editingnotes}
We're now leaning toward putting the argument back in the text---along
with a reference to a problem asking for the proof above.
\end{editingnotes}

\end{solution}

\ppart\label{RindScT} Let $S \cross T$ be the random variable giving
the values of $S$ and $T$.\footnote{That is, $S \cross T: \sspace \to
  V \cross V$ where
\[
(S \cross T)(\omega) \eqdef (S(\omega), T(\omega))
\]
for every outcome $\omega \in \sspace$.}  Now suppose $R$ has a
uniform distribution, and $R$ is independent of $S \cross T$.  How
about this argument?

\begin{quote}
The probability that $R=S$ is the same as the probability that $R$
equals the first coordinate of whatever value $S \cross T$ happens to
have, and this probability remains equal to $1/\card{V}$ by
independence.  Therefore the event $[R=S]$ is independent of $[S=T]$.
\end{quote}

Write out a careful proof that $[R=S]$ is independent of $[S=T]$.

\begin{staffnotes}
If students are stuck getting started give:
\hint
\[
[R=S] \intersect [S=T] = \lgunion_{b \in V} [R=b \QAND\ S \cross T = (b,b)].
\]
\end{staffnotes}

\begin{solution}
We must show that:
\begin{equation}\label{RSST=RS.ST}
\pr{[R=S] \intersect [S=T]} = \pr{R=S} \cdot \pr{S=T}.
\end{equation}

\begin{proof}
\begin{align*}
\lefteqn{\pr{[R=S] \intersect [S=T]}}\\
    & = \Prob{\lgunion_{b \in V} [R=b \QAND\ S \cross T = (b,b)]}
           & \text{(by the hint)}\\
    & = \sum_{b \in V} \pr{R=b \QAND\ S \cross T = (b,b)}
          & \text{(disjoint sum rule)}\\
    & = \sum_{b \in V} \pr{R=b} \cdot \pr{S \cross T = (b,b)}
          & \text{($R$ independent of $S\cross T$)}\\
    & = \sum_{b \in V} \frac{1}{\card{V}}\cdot \pr{S \cross T = (b,b)}
             & (\text{$R$ is uniform})\\
    & = \frac{1}{\card{V}}\cdot \sum_{b \in V} \pr{S \cross T = (b,b)}\\
    & = \frac{1}{\card{V}}\cdot \Pr{\lgunion_{b \in V} [S \cross T = (b,b)]}
          & \text{(disjoint sum rule)}\\
    & = \frac{1}{\card{V}}\cdot \pr{S=T}\\
    & = \pr{R=S} \cdot \pr{S=T},
          & \text{(part~\eqref{R=S=b})}
\end{align*}
which proves~\eqref{RSST=RS.ST}.
\end{proof}
\end{solution}

\ppart\label{VRSTdepend} Let $V = \set{1,2,3}$ and $(R,S,T)$ take the
following triples of values with equal probability,
\[
(1,1,1), (2,1,1), (1,2,3), (2,2,3), (1,3,2), (2,3,2).
\]
Verify that
\begin{enumerate}
\item $R$ is independent of $S \cross T$,\label{RindST}
\item The event $[R=S]$ is not independent of $[S=T]$.\label{R=SnotS=T}
\item $S$ and $T$ have a uniform distribution.\label{STunif}
\end{enumerate}

\begin{solution}
To prove independence, note that $1,s,t$ is a possible sequence of
values $R,S,T$ iff $2,s,t$ and $1,t,s$ are also possible.
This implies that
\[
\pr{R=1} = \pr{R=2} = \frac{1}{2}.
\]
It also implies that if $s,t$ are possible values for $S,T$, then
\[
\pr{S=s \QAND T=t} = \frac{1}{3}.
\]
So if $i,s,t$ are possible values for $R,S,T$, then
\[
\pr{R=i \QAND S=s \QAND T = t} = \frac{1}{6}
    = \frac{1}{2}\cdot \frac{1}{3}
    = \pr{R=i} \cdot \pr{S=s \QAND T=t}.
\]
Likewise, if $i,s,t$ are \emph{not} possible values for $R,S,T$, then
either $\pr{R=i} = 0$ because $i=3$, or else $\pr{S=s \QAND T=t} = 0$
because $s,t$ are not possible values for $S,T$.  So in this case
\[
\pr{R=i \QAND S=s \QAND T = t} = 0
    = \pr{R=i} \cdot \pr{S=s \QAND T=t}.
\]
This proves~\ref{RindST}.

Finally, there are two outcomes out of six with $R = S$ and two
outcomes with $S=T$, so $\pr{R=S}= 1/3= \pr{S=T}$.
But the only outcome in $[R=S] \intersect [S=T]$ is $111$, so
\[
\pr{[R=S] \intersect [S=T]} = \frac{1}{6} \neq \frac{1}{9} = \pr{R=S}\cdot \pr{S=T}.
\]
This proves~\ref{R=SnotS=T}.

There are two outcomes with $S=i$ for each $i \in V$, so $\pr{S=i} =
1/3$ for all $i \in V$, that is, $S$ is uniform.  Likewise for $T$.
This proves~\ref{STunif}.

\begin{staffnotes}
Ask why part~\eqref{VRSTdepend} is not a counter-example to
part~\eqref{RindScT}---after all, by part~\eqref{VRSTdepend} $T$ is
uniform and independent of $R$, so part~\eqref{RindScT} would ``apply
to $T$ just as well as $R$'' and this would imply independence.  The
answer is that for part~\eqref{RindScT} to apply to $T$, we would need
$T$ to be independent of $R \cross S$, but it is obviously not even
independent of $S$.
\end{staffnotes}
\end{solution}

\eparts
\end{problem}


%%%%%%%%%%%%%%%%%%%%%%%%%%%%%%%%%%%%%%%%%%%%%%%%%%%%%%%%%%%%%%%%%%%%%
% Problem ends here
%%%%%%%%%%%%%%%%%%%%%%%%%%%%%%%%%%%%%%%%%%%%%%%%%%%%%%%%%%%%%%%%%%%%%

\endinput
