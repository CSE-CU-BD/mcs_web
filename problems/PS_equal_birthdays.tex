\documentclass[problem]{mcs}

\begin{pcomments}
  \pcomment{PS_equal_birthdays}
  \pcomment{by ARM 4/23/11}
\end{pcomments}

\pkeywords{
  independence
  mutual
  pairwise
  uniform
  distribution
}

%%%%%%%%%%%%%%%%%%%%%%%%%%%%%%%%%%%%%%%%%%%%%%%%%%%%%%%%%%%%%%%%%%%%%
% Problem starts here
%%%%%%%%%%%%%%%%%%%%%%%%%%%%%%%%%%%%%%%%%%%%%%%%%%%%%%%%%%%%%%%%%%%%%

\begin{problem}
Let $R$, $S$, and $T$ be random variables with the same codomain, $V$.

\bparts

\ppart\label{R=S=b} Suppose $R$ is uniform ---that is,
\[
\pr{R=b} = \frac{1}{\card{V}},
\]
for all $b \in V$.

Suppose $R$ is independent of $S$.  Originally this text had the
following argument:

\begin{quote}
The probability that $R=S$ is the same as the probability that $R$
takes whatever value $S$ happens to have, therefore
\begin{equation}\label{RS1V}
\pr{R=S} = \frac{1}{\card{V}}\,.
\end{equation}
\end{quote}

Are you convinced by this argument?  \inhandout{We decided to replace
  it by a reference to this problem.  We'd like your advice on whether
  it should be put back it the text.  Before advising us, w}%
\inbook{W}rite out a careful proof of~\eqref{RS1V}.

\hint The event $[R=S]$ is the same as the disjoint union of events
$[R = b \QAND S=b]$ for $b \in V$.

\begin{solution}

\begin{proof}
\begin{align*}
\pr{R=S}
   & = \Pr{\lgunion_{b \in V} [R=b \QAND S=b]}\\
   & = \sum_{b \in V} \pr{R=b \QAND S=b}
         & \text{(disjoint sum rule)}\\
   & = \sum_{b \in V} \pr{R=b} \cdot \pr{S=b}
          & \text{($R,S$ independent)}\\
   & = \sum_{b \in V} \frac{1}{\card{V}} \cdot \pr{S=b}
         &\text{($R$ is uniform)}\\
   & = \frac{1}{\card{V}} \cdot \sum_{b \in V} \pr{S=b}\\
   & = \frac{1}{\card{V}} \cdot 1 = \frac{1}{\card{V}}.
\end{align*}
This proves~\eqref{RS1V}.
\end{proof}

We're now leaning toward putting the argument back in the text
---along with a reference to a problem asking for the proof above.

\end{solution}

\eparts

\begin{definition*}
A random variable, $R$, is independent of a set $\set{R_1,R_2,\dots}$ of
random variables iff the event $[R=r]$ is independent of the event
\[
[R_1 = r_1 \QAND R_2 = r_2 \QAND \cdots]
\]
for all values $r, r_1, r_2, \dots$.
\end{definition*}

\bparts

\ppart Now suppose $R$ has a uniform distribution, and $R$ is
independent of $\set{S,T}$.  How about this argument?

\begin{quote}
The probability that $R=S$ is the same as the probability that $R$
takes whatever value $S$ and $T$ happen to have in common, and this
probability remains equal to $1/\card{V}$ by independence.  Therefore
the event $[R=S]$ is independent of $[S=T]$.
\end{quote}

Write out a careful proof that $[R=S]$ is independent of $[S=T]$.

\begin{staffnotes}
Give
\[
[R=S \QAND\ S=T] = \lgunion_{b \in V} [R=b \QAND\ S=b \QAND\ T=b]
\]
as a hint if students are stuck.
\end{staffnotes}

\begin{solution}
We must show that:
\begin{equation}\label{RSST=RS.ST}
\pr{[R=S] \intersect [S=T]} = \pr{R=S} \cdot \pr{S=T}.
\end{equation}

\begin{proof}
\begin{align*}
\lefteqn{\pr{[R=S] \intersect [S=T]}}\\
    & = \pr{R=S \QAND\ S=T}\\
    & = \Prob{\lgunion_{b \in V} [R=b \QAND\ S=b \QAND\ T=b]}\\
    & = \sum_{b \in V} \pr{R=b \QAND\ S=b \QAND\ T=b}
          & \text{(disjoint sum rule)}\\
    & = \sum_{b \in V} \pr{R=b} \cdot \pr{S=b \QAND\ T=b}
          & \text{($R$ independent of $\set{S,T}$)}\\
    & = \sum_{b \in V} \frac{1}{\card{V}}\cdot \pr{S=b \QAND\ T=b}
             & (\text{$R$ is uniform})\\
    & = \frac{1}{\card{V}}\cdot \sum_{b \in V} \pr{S=b \QAND\ T=b}\\
    & = \frac{1}{\card{V}}\cdot \Pr{\lgunion_{b \in V} [S=b \QAND\ T=b]}
          & \text{(disjoint sum rule)}\\
    & = \frac{1}{\card{V}}\cdot \pr{S=T}\\
    & = \pr{R=S} \cdot \pr{S=T},
          & \text{(part~\eqref{R=S=b})}
\end{align*}
which proves~\eqref{RSST=RS.ST}.
\end{proof}
\end{solution}

\ppart Let $V = \set{1,2,3}$ and $R,S,T$ take the following values
with equal probability,
\[
111, 211, 123, 223, 132, 232.
\]
Verify that
\begin{enumerate}
\item $R$ is independent of $\set{S,T}$,\label{RindST}
\item The event $[R=S]$ is not independent of $[S=T]$.\label{R=SnotS=T}
\item $S$ and $T$ have a uniform distribution,\label{STunif}
\end{enumerate}

\begin{solution}
To prove independence, note that $1,s,t$ is a possible sequence of
values $R,S,T$ iff $2,s,t$ and $1,t,s$ are also possible.
This implies that
\[
\pr{R=1} = \pr{R=2} = \frac{1}{2}.
\]
It also implies that if $s,t$ are possible values for $S,T$, then
\[
\pr{S=s \QAND T=t} = \frac{1}{3}.
\]
So if $i,s,t$ are possible values for $R,S,T$, then
\[
\pr{R=i \QAND S=s \QAND T = t} = \frac{1}{6}
    = \frac{1}{2}\cdot \frac{1}{3}
    = \pr{R=i} \cdot \pr{S=s \QAND T=t}.
\]
Likewise, if $i,s,t$ are \emph{not} possible values for $R,S,T$, then
either $\pr{R=i} = 0$ because $i=3$, or else $\pr{S=s \QAND T=t} = 0$
because $s,t$ are not possible values for $S,T$.  So in this case
\[
\pr{R=i \QAND S=s \QAND T = t} = 0
    = \pr{R=i} \cdot \pr{S=s \QAND T=t}.
\]
This proves~\ref{RindST}.

Finally, there are two outcomes out of six with $R = S$ and and two
outcomes with with $S=T$, so $\pr{R=S}= 1/3= \pr{S=T}$.
But the only outcome in $[R=S] \intersect [S=T]$ is $111$, so
\[
\pr{[R=S] \intersect [S=T]} = \frac{1}{6} \neq \frac{1}{9} = \pr{R=S}\cdot \pr{S=T}.
\]
This proves~\ref{R=SnotS=T}.

There are two outcomes with $S=i$ for each $i \in V$, so $\pr{S=i} =
1/3$ for all $i \in V$, that is, $S$ is uniform.  Likewise for $T$.
This proves~\ref{STunif}.

\end{solution}

\eparts
\end{problem}


%%%%%%%%%%%%%%%%%%%%%%%%%%%%%%%%%%%%%%%%%%%%%%%%%%%%%%%%%%%%%%%%%%%%%
% Problem ends here
%%%%%%%%%%%%%%%%%%%%%%%%%%%%%%%%%%%%%%%%%%%%%%%%%%%%%%%%%%%%%%%%%%%%%

\endinput
