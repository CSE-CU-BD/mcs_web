\documentclass[problem]{mcs}

\begin{pcomments}
  \pcomment{CP_recursive_prop_form_eval}
  \pcomment{ARM 3/4/16}
\end{pcomments}

\pkeywords{
  propositional_formula
  propositional_variable
  recursive
  environment
  structural_induction
  induction
  evaluate
}

\newcommand{\pvrbls}[1]{\mopt{pvar}(#1)}
%\newcommand{\ANDsym}{\ensuremath{\mathbin{\mathrm{\textbf{And}}}}}
%\newcommand{\NOTsym}{\ensuremath{\mathop{\mathrm{\textbf{Not}}}}}

\begin{problem}
In this problem we'll need to be careful about the propositional
\emph{operations} that apply to truth values and the corresponding
\emph{symbols} that appear in formulas.  We'll restrict ourselves to
formulas with \emph{symbols} \ANDsym\ and \NOTsym\ that correspond to
the operations \QAND, \QNOT.  We will also allow the constant symbols
\True\ and \False.

\bparts

\ppart Give a simple recursive definition of \emph{propositional
  formula} $F$ and the set $\pvrbls{F}$ of propositional
\emph{variables that appear} in it.

\begin{solution}

\inductioncase{Base cases:}
\begin{itemize}

\item A propositional \emph{variable} $P$ is a propositional formula
  and the set $\pvrbls{P}$ of variables that appear in it is $\set{P}$.

\item A propositional \emph{constants} \True, \False\ are propositional
  formulas, and
\[
\pvrbls{\True} = \pvrbls{\False} \eqdef \emptyset.
\]

\end{itemize}

\inductioncase{Constructor cases:} If $F,G$ are propositional
formulas, then so are
\begin{itemize}

\item $(F \ANDsym\ G)$, and $\pvrbls{(F \ANDsym G)} \eqdef \pvrbls{F} \union
  \pvrbls{G}$.

\item $\NOTsym(F)$, and $\pvrbls{\NOTsym(F)} \eqdef \pvrbls{F}$.
\end{itemize}
\end{solution}

\eparts

Let $V$ be a set of propositional variables.  A \emph{truth
  environment} $e$ over $V$ assigns truth values to all these
variables.  In other words, $e$ is a total function,
\[
e: V \to \set{\true,\false}.
\]

\bparts

\ppart Give a recursive definition of the \emph{truth value},
$\meval{F}{e}$, of propositional formula $F$ in an environment $e$
over a set of variables $V \supseteq \pvrbls{F}$.

\begin{solution}

\inductioncase{Base cases:}
\begin{align*}
\meval{P}{e} & \eqdef e(P),\\
\meval{\True}{e} & \eqdef \true,\\
\meval{\False}{e} & \eqdef \false.\\
\end{align*}

\inductioncase{Constructor cases:} If $F,G$ are propositional
formulas, then
\begin{align*}
 \meval{(F \ANDsym G)}{e} & \eqdef  \meval{F}{e} \QAND\ \meval{G}{e},\\
 \meval{\NOTsym(F)}{e} & \eqdef  \QNOT(\meval{F}{e}).
\end{align*}
\end{solution}
\eparts

Clearly the truth value of a propositional formula only depends on the
truth values of the variables in it.  How could it be otherwise?  But
it's good practice to work out a rigorous definition and proof of this
assumption.

\bparts

\ppart Give an example of a propositional formula containing the
variable $P$ but whose truth value does not depend on $P$.  Now give a
rigorous definition of the assertion that ``the truth value of 
propositional formula $F$ does not depend on propositional variable $P$.''

\hint Let $e_1,e_2$ be two environments whose values agree on all
variables other than $P$.

\begin{solution}
A simple example is the formula $P \ANDsym \NOTsym(P)$ whose truth
value is $\false$ in all environments.

Not ``depending on $P$'' means that whether $P$ is \true\ or \false,
the truth value of $F$ comes out the same.  More precisely this means
that if two environments agree on the truth values of all variables
other than $P$, then the value of $F$ is the same in both
environments:
\begin{definition*}
The truth value of a propositional formula \emph{$F$ does not depend
  on propositional variable $P$} iff whenever $e_1, e_2$ are
environments over $V_0 \supseteq \pvrbls{F} \union \set{P}$ and $e_1(Q)
= e_2(Q)$ for all variables $Q \in V_0$ other than $P$, then
\[
 \meval{F}{e_1} = \meval{F}{e_2}.
\]
\end{definition*}
\end{solution}

\ppart Give a rigorous definition of the assertion that ``the truth
value of a propositional formula only depends on the truth values of
the variables that appear in it,'' and then prove it by structural
induction on the definition of propositional formula.

\begin{solution}
\begin{definition*}
The truth value of a propositional formula $F$ \emph{depends only on
  the truth values of $\pvrbls{F}$} iff for all $P \notin \pvrbls{F}$,
$F$ does not depend on $P$.
\end{definition*}

To prove that all formulas $F$ have this property, suppose $P \notin
\pvrbls{F}$, and $e_1, e_2$ are environments over $V_0 \supseteq
\pvrbls{F} \union \set{P}$ and $e_1(Q) = e_2(Q)$ for all variables $Q
\in V_0$ other than $P$.  Then we will prove by structural induction on
$F$ that
\begin{equation}\label{Fe1Fe2}
 \meval{F}{e_1} = \meval{F}{e_2}.
\end{equation}

\inductioncase{Base case} $F$ can't be the variable $P$, so the only
cases are when $F$ is a truth constant. Now if $F$ is \True, then
\[
 \meval{F}{e_1} \eqdef \true =  \meval{F}{e_2}, 
\]
by definition of eval.  The same reasoning applies if $F$ is \False,
proving that~\eqref{Fe1Fe2} holds on the base case.

\inductioncase{Constructor case} ($F$ is $(G \ANDsym H)$).  Then by
structural induction hypothesis,
\begin{align*}
\meval{G}{e_1} & = \meval{G}{e_2}\\
\meval{H}{e_1} & = \meval{H}{e_2}, 
\end{align*}
so
\begin{align*}
\meval{F}{e_1}
  & = \meval{G}{e_1} \QAND\ \meval{H}{e_1}
     & \text{(def of eval)}\\
  & = \meval{G}{e_2} \QAND\ \meval{H}{e_2}
     & \text{(ind. hypothesis)}\\
  & = \meval{(G \ANDsym H)}{e_2}
     & \text{(def of eval)}\\
  & = \meval{F}{e_2}.
\end{align*}
proving~\eqref{Fe1Fe2} in this case.

\begin{staffnotes}
We've made implicit use of the fact  that
\[
\pvrbls{G} \subseteq \pvrbls{G \ANDsym H}
\]
by definition of $\pvrbls{F}$.
\end{staffnotes}

\inductioncase{Constructor case} ($F$ is $\NOTsym(G)$).  Similar,
easier proof.
\end{solution}

\ppart Now we can formally define $F$ being \emph{valid}.  Namely, $F$
is valid iff
\[
\forall e.\, \meval{F}{e} = \true.
\]
Give a similar formal definition of formula $G$ being
\emph{unsatisfiable}.  Then use the definition of eval to prove that a
formula $F$ is valid iff $\NOTsym(F)$ is unsatisfiable.

\begin{solution}
Let $e$ range over environments on $\pvrbls{G}$.  Then $G$ is
unsatisfiable iff
\[
\QNOT(\exists e.\,\meval{G}{e} = \true)
\]

\begin{align*}
F \text{is valid} 
   &  \qiff \forall e.\, \meval{F}{e} = \true\\
   &  \qiff \forall e.\, \QNOT(\meval{F}{e}) = \false
       & \text{(by def of operation $\QNOT$)}\\
   & \qiff  \forall e.\, \meval{\NOTsym(F)}{e} = \false
       & \text{(by def of $\meval{\NOTsym(F)}{e}$)}\\
   & \qiff  \QNOT(\exists e.\,\meval{\NOTsym(F)}{e} = \true)
       & \text{(De Morgan)}\\
   & \qiff \NOTsym(F) \text{ is unsatisfiable}
       & \text{(def of unsatisfiable)}.
\end{align*}
\end{solution}

\eparts

\end{problem}

\endinput
