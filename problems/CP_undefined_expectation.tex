\documentclass[problem]{mcs}

\begin{pcomments}
\pcomment{CP_undefined_expectation}
\pcomment{F99.PS12}
\pcomment{edited by ARM 5/17/17}
\end{pcomments}

\pkeywords{
 probability
 expectation
 convergent
 divergent
}


%%%%%%%%%%%%%%%%%%%%%%%%%%%%%%%%%%%%%%%%%%%%%%%%%%%%%%%%%%%%%%%%%%%%%
% Problem starts here
%%%%%%%%%%%%%%%%%%%%%%%%%%%%%%%%%%%%%%%%%%%%%%%%%%%%%%%%%%%%%%%%%%%%%

\begin{problem}
Ben Bitdiddle is asked to analyze a game in which a fair coin is tossed
until the first time a head turns up.  If this head occurs on the $n$th
toss, and $n$ is odd, then he wins $\$2^n/n$, but if $n$ is even, he loses
$\$2^n/n$.  Ben observes that the expected dollar win from this game is
\[
(1/2) \cdot 2 - (1/4) \cdot 2 + (1/8) \cdot 8/3 + \cdots \pm (1/2^n) \cdot 2^n/n
 = 1 - 1/2 + 1/3 - 1/4 + \cdots \pm 1/n.
\]
which is the alternating harmonic series---a series that converges to a
definite real number $r>0$.  Since $r >0$, Ben concludes that it's to his
advantage to play this game, but as usual, his shoot-from-the-hip analysis
is off the mark.  Explain.

\begin{solution}
This series is not absolutely convergent, so its expectation is not
defined, and the number $r$ is not relevant to the behavior of the
game.  Indeed, \emph{Riemann's Series Theorem}, \cite{Apostol67},
p.413 (see Problem~\bref{CP_conditional_convergence}),reveals that by
reordering its terms, the series can be made to converge to any real
value or to diverge to $\pm\infty$.
\end{solution}

\end{problem}
%%%%%%%%%%%%%%%%%%%%%%%%%%%%%%%%%%%%%%%%%%%%%%%%%%%%%%%%%%%%%%%%%%%%%
% Problem ends here
%%%%%%%%%%%%%%%%%%%%%%%%%%%%%%%%%%%%%%%%%%%%%%%%%%%%%%%%%%%%%%%%%%%%%

\endinput
