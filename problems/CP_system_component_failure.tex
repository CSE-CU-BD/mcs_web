\documentclass[problem]{mcs}

\begin{pcomments}
  \pcomment{CP_system_component_failure}
  \pcomment{from: S09.cp12r, F09}
\end{pcomments}

\pkeywords{
  probability
  union_bound
}

%%%%%%%%%%%%%%%%%%%%%%%%%%%%%%%%%%%%%%%%%%%%%%%%%%%%%%%%%%%%%%%%%%%%%
% Problem starts here
%%%%%%%%%%%%%%%%%%%%%%%%%%%%%%%%%%%%%%%%%%%%%%%%%%%%%%%%%%%%%%%%%%%%%

\begin{problem}
Suppose there is a system, built by Caltech graduates, with $n$~components. We know from past
experience that any particular component will fail in a given year with
probability~$p$. That is, letting $F_i$ be the event that the $i$th
component fails within one year, we have
\[
\pr{F_i} = p
\]
for $1 \leq i \leq n$.  The \emph{system} will fail if \emph{any one} of
its components fails.  What can we say about the probability that the
system will fail within one year?

Let $F$ be the event that the system fails within one year.  Without any
additional assumptions, we can't get an exact answer for $\pr{F}$.
However, we can give useful upper and lower bounds, namely,
\begin{equation}\label{cp12r_pFnp}
p \le \pr{F} \le np.
\end{equation}
We may as well assume $p < 1/n$, since the upper bound is trivial
otherwise.  For example, if $n = 100$ and $p = 10^{-5}$, we conclude that
there is at most one chance in 1000 of system failure within a year and at
least one chance in 100,000.

Let's model this situation with the sample space $\sspace \eqdef
\power([1,n])$ whose outcomes are subsets of positive integers
$\le n$, where $s \in \sspace$ corresponds to the indices of exactly those
components that fail within one year.  For example, $\set{2, 5}$ is the
outcome that the second and fifth components failed within a year and none
of the other components failed.  So the outcome that the system did not
fail corresponds to the empty set $\emptyset$.

\begin{staffnotes}
Encourage students to begin by stating explicitly what outcomes
(subsets of $[1,n]$) are in the event $F_i$ and $F$.
\end{staffnotes}

\bparts


\ppart Show that the probability that the system fails could be as small
as $p$ by describing appropriate probabilities for the outcomes.  Make
sure to verify that the sum of your outcome probabilities is 1.

\begin{solution}
According to the description,
\begin{align*}
F_i & = \set{s \in \sspace \suchthat i \in s}\\
F & = \union_{i=1}^n F_i = \set{s \in \sspace \suchthat s \neq \emptyset}
\end{align*}

There could be a probability $p$ of system failure if the individual
failures always occur together.  That is, if
\[
\pr{\set{1, \dots,n}}= p,\qquad \pr{\emptyset} = 1-p,
\]
and the probability of all other outcomes is zero.  Then
\[
\pr{F_i} = \pr{\set{1, \dots,n}}+ 0 + 0 + \cdots + 0 = \pr{\set{1, \dots ,n}} = p.
\]
Also, the only outcome with positive probability in $F$ is $\set{1,
  \dots ,n}$, so also $\pr{F} = p$, as required.
\end{solution}

\ppart Show that the probability that the system fails could actually be
as large as $np$ by describing appropriate probabilities for the outcomes.
Make sure to verify that the sum of your outcome probabilities is 1.

\begin{solution}
Suppose at most one component ever fails at a time.  That is,
$\pr{\set{i}}=p$ for $1 \le i \le n$, $\pr{\emptyset}=1-np$, and
probability of all other outcomes is zero.  The sum of the
probabilities of all the outcomes is one, so this is a well-defined
probability space.  Also, the only outcome in $F_i$ with positive
probability is $\set{i}$, so $\pr{F_i} = \pr{\set{i}} = p$ as
required.  Finally, $\pr{F} = np$ because $F$ contains all the $n$
outcomes of the form $\set{i}$.
\end{solution}

\ppart Prove inequality~\eqref{cp12r_pFnp}.

\begin{solution}
$F = \lgunion_{i=1}^n F_i$ so
\begin{align*}
p & = \pr{F_1}          & \text{(given)}\\
  & \le \pr{F}          & \text{(since $F_1 \subseteq F$)}\\
  & = \pr{\lgunion F_i} & \text{(def.\ of $F$)} \\
  & \le \sum_{i=1}^n \pr{F_i} & \text{(Union Bound)}\\
  & = np.
\end{align*}
\end{solution}

\eparts
\end{problem}


%%%%%%%%%%%%%%%%%%%%%%%%%%%%%%%%%%%%%%%%%%%%%%%%%%%%%%%%%%%%%%%%%%%%%
% Problem ends here
%%%%%%%%%%%%%%%%%%%%%%%%%%%%%%%%%%%%%%%%%%%%%%%%%%%%%%%%%%%%%%%%%%%%%

\endinput
