\documentclass[problem]{mcs}

\begin{pcomments}
  \pcomment{PS_invserse_of_11_mod_29_by_Fermat}
  \pcomment{modified from PS_invserse_of_13_mod_23_by_Fermat}
\end{pcomments}

\pkeywords{
  number_theory
  modular_arithmetic
  inverses
  Fermat_theorem
  remainder
}

%%%%%%%%%%%%%%%%%%%%%%%%%%%%%%%%%%%%%%%%%%%%%%%%%%%%%%%%%%%%%%%%%%%%%
% Problem starts here
%%%%%%%%%%%%%%%%%%%%%%%%%%%%%%%%%%%%%%%%%%%%%%%%%%%%%%%%%%%%%%%%%%%%%

\begin{problem}

Use \idx{Fermat's theorem} to find the inverse of 11 modulo 29
in $[1,29)$.

\begin{solution}
Since 29 is prime, Fermat's theorem implies
$11^{29-2} \cdot 11 \equiv 1 \pmod{29}$ and so $\rem{11^{29-2}}{29}$ is
the inverse of 11 in the range $\set{1,\dots,22}$.  Now using the method
of repeated squaring, we have the following congruences modulo 29:
\[
\begin{array}{lcl}
11^{2}  & =      & 121\\
	& \equiv & \rem{121}{29} = 5\\
&&\\
11^{4}  & \equiv & 5^2\\
	& =      & 25\\
	& \equiv & \rem{25}{29} = 25\\
&&\\
11^{8}  & \equiv & 25^2\\
	& =      & 625\\
	& \equiv & \rem{625}{29} = 16\\
&&\\
11^{16} & \equiv & 16^2\\
        & =      & 256\\
        & \equiv & \rem{256}{29} = 24\\
&&\\
11^{27} & =      & 11^{16} \cdot 11^{8} \cdot 11^{2} \cdot 11\\
	& \equiv & 24 \cdot 16 \cdot 5 \cdot 11\\
	& =      & (4 \cdot 6) \cdot 16 \cdot 5 \cdot 11\\
        & =      & (6 \cdot 5) \cdot (4 \cdot 16 \cdot 11)\\
	& =      & 30 \cdot 704\\
	& \equiv & 1 \cdot 704\\
	& \equiv & \rem{704}{29} = \fbox{$8$}.
\end{array}
\]

\end{solution}

\end{problem}

%%%%%%%%%%%%%%%%%%%%%%%%%%%%%%%%%%%%%%%%%%%%%%%%%%%%%%%%%%%%%%%%%%%%%
% Problem ends here
%%%%%%%%%%%%%%%%%%%%%%%%%%%%%%%%%%%%%%%%%%%%%%%%%%%%%%%%%%%%%%%%%%%%%

\endinput
