\documentclass[problem]{mcs}

\begin{pcomments}
  \pcomment{FP_conditional_beaver_fever}
  \pcomment{from F07.mq-nov28}
\end{pcomments}

\pkeywords{
  probability
  conditional_probability
  Bayes
}

%%%%%%%%%%%%%%%%%%%%%%%%%%%%%%%%%%%%%%%%%%%%%%%%%%%%%%%%%%%%%%%%%%%%%
% Problem starts here
%%%%%%%%%%%%%%%%%%%%%%%%%%%%%%%%%%%%%%%%%%%%%%%%%%%%%%%%%%%%%%%%%%%%%

\begin{problem}
 There is a rare and serious disease called Beaver Fever which
 afflicts about 1 person in 1000.  Victims of this disease start
 telling math jokes in social settings, believing other people will
 think they're funny.

  Doctor Meyer has some fairly reliable tests for this disease. In
  particular:

  \begin{itemize}
  \item If a person has Beaver Fever, the probability that Meyer diagnoses
    the person as having the disease is 0.99.

  \item If a person doesn't have it, the probability that Meyer diagnoses
    that person as not having Beaver Fever is 0.97.
  \end{itemize}

  Let $B$ be the event that a randomly chosen person has Beaver Fever,
  and $Y$ be the event that Meyer's diagnosis is ``Yes, that person
  has Beaver Fever,'' with $\bar{B}$ and $\bar{Y}$ the complements of
  these events.

\bparts

%\ppart[0.5] Some of the following probabilities are \emph{explicitly} given
%by the description above.  The probabilities that are not given explicitly
%can nevertheless be derived from the explicitly given ones.  Circle the
%explicitly given probabilities:
%
%\begin{itemize}
%\item $\pr{B}$
%
%\item $\pr{Y}$
%
%\item $\pr{B\text{ and }Y}$
%
%\item $\prcond{Y}{B}$
%
%\item $\prcond{B}{Y}$
%
%\item $\prcond{\bar{Y}}{\bar{B}}$
%
%\item $\prcond{\bar{B}}{\bar{Y}}$
%
%\end{itemize}
%
%\solution{
%$\pr{B}, \prcond{Y}{B}, \prcond{\bar{Y}}{\bar{B}}$
%}

%
%
%There is a rare and serious disease called \textit{Beaver Fever}
%which afflicts about 1 person in 1000.  Victims eventually start
%telling math jokes at cocktail parties, thinking others will find
%the jokes amusing.
%
%Doctor $X$ says he can test for the disease in the following
%sense:
%%
%\begin{itemize}
%\item If a person has Beaver Fever, doctor $X$ says ``yes'', meaning that the person has the disease, with probability 0.99.
%\item If a person doesn't have it, he says ``no'', meaning that the person doesn't have the disease, with probability 0.97.
%\end{itemize}
%%

\ppart\label{explicitly_given}  The description above explicitly gives the values of the
  following quantities.  What are their values?
\begin{center}
$\prob{B}$ \qquad $\prcond{Y}{B}$ \qquad $\prcond{\bar{Y}}{\bar{B}}$ 
\end{center}

\begin{solution}
$\prob{B} = 0.001$ \qquad $\prcond{Y}{B} = 0.99$ \qquad $\prcond{\bar{Y}}{\bar{B}} = 0.97$
\end{solution}

\ppart Write formulas for $\pr{\bar{B}}$ and $\prcond{Y}{\bar{B}}$
solely in terms of the explicitly given expressions.  Literally use
the expressions, not their numeric values.

\begin{center}
\exambox{2.0in}{1.0in}{-0.2in}
\end{center}
%\examspace[1.5in]

\begin{solution}
 $\pr{\bar{B}}= 1 - \pr{B}$,
$\prcond{Y}{\bar{B}} = 1 - \prcond{\bar{Y}}{\bar{B}}$. 
\end{solution}

\ppart Write a formula for the probability that Doctor Meyer
says a person has the disease solely in terms of $\pr{B}$,
$\pr{\bar{B}}$, $\prcond{Y}{B}$ and $\prcond{Y}{\bar{B}}$.


\begin{center}
\exambox{3.0in}{0.75in}{-0.2in}
\end{center}
%\examspace[1.5in]

\begin{solution}
By the Total Probability Law:
\[
\pr{Y}=\prcond{Y}{B}\pr{B}+ \prcond{Y}{\bar{B}}\pr{\bar{B}}
%        = 0.99(1/1000) + 0.03(1- 1/1000) = 0.03096.
\]
\end{solution}


\ppart Write a formula solely in terms of the expressions given in
part~\eqref{explicitly_given} for the probability that a person has
Beaver Fever given that Doctor Meyer says the person has it.

\begin{center}
\exambox{4.5in}{0.75in}{-0.2in}
\end{center}
%\examspace[2.5in]


\begin{solution}

\begin{align*}
\prcond{B}{Y}
    & = \frac{\pr{B\text{ and }Y}}{\pr{Y}}\\
    & = \frac{\prcond{Y}{B}\pr{B}}{\prcond{Y}{B}\pr{B}+ \prcond{Y}{\bar{B}}\pr{\bar{B}}}\\
    & = \frac{\prcond{Y}{B}\pr{B}}{\prcond{Y}{B}\pr{B}+ (1- \prcond{\bar{Y}}{\bar{B}})(1-\pr{B})}.
\end{align*}

\iffalse
\begin{align*}
\prcond{B}{Y}
    & = \frac{\pr{B\text{ and }Y}}{\pr{Y}}\\
    & = \frac{\prcond{Y}{B}\pr{B}}{}\\
    & = \frac{0.99(1/1000)}{0.03096}\\
    & = \frac{99}{3096}\\
    & = \frac{11}{344} \approx \frac{1}{32}.
\end{align*}
\fi

\end{solution}

\eparts

\end{problem}

%%%%%%%%%%%%%%%%%%%%%%%%%%%%%%%%%%%%%%%%%%%%%%%%%%%%%%%%%%%%%%%%%%%%%
% Problem ends here
%%%%%%%%%%%%%%%%%%%%%%%%%%%%%%%%%%%%%%%%%%%%%%%%%%%%%%%%%%%%%%%%%%%%%

\endinput
 

