\documentclass[problem]{mcs}

\begin{pcomments}
  \pcomment{FP_conditional_beaver_fever}
  \pcomment{from F07.mq-nov28}
  \pcomment{Revisions on grammar, layout, clarity 4/27/14}
  \pcomment{ARM added discussion in last part solution 5/3/13}
\end{pcomments}

\pkeywords{
  probability
  conditional_probability
  Bayes
}

%%%%%%%%%%%%%%%%%%%%%%%%%%%%%%%%%%%%%%%%%%%%%%%%%%%%%%%%%%%%%%%%%%%%%
% Problem starts here
%%%%%%%%%%%%%%%%%%%%%%%%%%%%%%%%%%%%%%%%%%%%%%%%%%%%%%%%%%%%%%%%%%%%%

\begin{problem}
 There is an unpleasant, degenerative disease called Beaver Fever
 which causes people to tell math jokes unrelentingly in social
 settings, believing other people will think they're funny.
 Fortunately, Beaver Fever is rare, afflicting only about 1 in 1000
 people.  Doctor Meyer has a fairly reliable diagnostic test to
 determine who is going to suffer from this disease:

  \begin{itemize}
  \item If a person will suffer from Beaver Fever, the probability
    that Dr.\ Meyer diagnoses this is 0.99.

  \item If a person will not suffer from Beaver Fever, the probability
    that Dr.\ Meyer diagnoses this is 0.97.
  \end{itemize}

  Let $B$ be the event that a randomly chosen person will suffer
  Beaver Fever, and $Y$ be the event that Dr.\ Meyer's diagnosis is
  ``Yes, this person will suffer from Beaver Fever,'' with $\bar{B}$
  and $\bar{Y}$ being the complements of these events.

\bparts

%\ppart[0.5] Some of the following probabilities are \emph{explicitly} given
%by the description above.  The probabilities that are not given explicitly
%can nevertheless be derived from the explicitly given ones.  Circle the
%explicitly given probabilities:
%
%\begin{itemize}
%\item $\pr{B}$
%
%\item $\pr{Y}$
%
%\item $\pr{B\text{ and }Y}$
%
%\item $\prcond{Y}{B}$
%
%\item $\prcond{B}{Y}$
%
%\item $\prcond{\bar{Y}}{\bar{B}}$
%
%\item $\prcond{\bar{B}}{\bar{Y}}$
%
%\end{itemize}
%
%\solution{
%$\pr{B}, \prcond{Y}{B}, \prcond{\bar{Y}}{\bar{B}}$
%}

%
%
%There is a rare and serious disease called \textit{Beaver Fever}
%which afflicts about 1 person in 1000.  Victims eventually start
%telling math jokes at cocktail parties, thinking others will find
%the jokes amusing.
%
%Doctor $X$ says he can test for the disease in the following
%sense:
%%
%\begin{itemize}
%\item If a person has Beaver Fever, doctor $X$ says ``yes'', meaning that the person has the disease, with probability 0.99.
%\item If a person doesn't have it, he says ``no'', meaning that the person doesn't have the disease, with probability 0.97.
%\end{itemize}
%%

\ppart\label{explicitly_given} The description above explicitly gives
the values of the following quantities.  What are their values?

\begin{center}
$\prob{B}$ \qquad $\prcond{Y}{B}$ \qquad $\prcond{\bar{Y}}{\bar{B}}$ 
\end{center}

\begin{solution}
$\prob{B} = 0.001$ \qquad $\prcond{Y}{B} = 0.99$ \qquad $\prcond{\bar{Y}}{\bar{B}} = 0.97$
\end{solution}

\ppart Write formulas for $\pr{\bar{B}}$ and $\prcond{Y}{\bar{B}}$
solely in terms of the explicitly given quantities in
part~\eqref{explicitly_given}---literally use their expressions, not
their numeric values.

\begin{center}
\exambox{2.0in}{1.0in}{-0.2in}
\end{center}
%\examspace[1.5in]

\begin{solution}
 $\pr{\bar{B}}= 1 - \pr{B}$,
$\prcond{Y}{\bar{B}} = 1 - \prcond{\bar{Y}}{\bar{B}}$. 
\end{solution}

\ppart Write a formula for the probability that Dr.\ Meyer says a
person will suffer from Beaver Fever solely in terms of $\pr{B}$,
$\pr{\bar{B}}$, $\prcond{Y}{B}$ and $\prcond{Y}{\bar{B}}$.


\begin{center}
\exambox{3.0in}{0.75in}{-0.2in}
\end{center}
%\examspace[1.5in]

\begin{solution}
By the Total Probability Law:
\[
\pr{Y}=\prcond{Y}{B}\pr{B}+ \prcond{Y}{\bar{B}}\pr{\bar{B}}
\]

The values turn out to be $0.99(1/1000) + 0.03(1- 1/1000) = 0.03096$.

\end{solution}


\ppart\label{prcondsuffer} Write a formula solely in terms of the
expressions given in part~\eqref{explicitly_given} for the probability
that a person will suffer Beaver Fever given that Doctor Meyer says
they will.  \inbook{Then calculate the numerical value of the formula.}

\begin{staffnotes}
Have students use a calculator to get the actual value.
\end{staffnotes}

\examspace[1.0in]

\begin{center}
\exambox{4.5in}{0.75in}{-0.2in}
\end{center}

\begin{solution}

\begin{align*}
\prcond{B}{Y}
    & = \frac{\pr{B\text{ and }Y}}{\pr{Y}}\\
    & = \frac{\prcond{Y}{B}\pr{B}}{\prcond{Y}{B}\pr{B}+ \prcond{Y}{\bar{B}}\pr{\bar{B}}}\\
    & = \frac{\prcond{Y}{B}\pr{B}}{\prcond{Y}{B}\pr{B}+ (1- \prcond{\bar{Y}}{\bar{B}})(1-\pr{B})}.
\end{align*}

The values turn out to be
\[
\prcond{B}{Y} = \frac{0.99(1/1000)}{0.03096} = \frac{99}{3096} \approx \frac{1}{32}.
\]

The low probability of actually suffering Beaver Fever even though the
(97\% accurate) test says you will is because there are way more
people who will not suffer the disease than those who will.  Among
1000 people, the number of false positives $(999 \times 3\%)$ is more
than 30 times the number of true positives $(1 \times 99\%)$.  So if
the test says you will suffer Beaver Fever, it's probably a false
positive.

Of course Dr.\ Meyer has a recourse to a 99.9\% accurate test that has
no false positives: simply telling everyone they won't get Beaver
Fever.
\end{solution}

\eparts

\medskip

Suppose there was a vaccine to prevent Beaver Fever, but the vaccine
was expensive or slightly risky itself.  If you were sure you were
going to suffer from Beaver Fever, getting vaccinated would be
worthwhile, but even if Dr.\ Meyer diagnosed you as a future sufferer
of Beaver Fever, the probability you actually will suffer Beaver Fever
remains low (about 1/32 by part~\eqref{prcondsuffer}).

In this case, you might sensibly decide not to be vaccinated---after
all, Beaver Fever is not \emph{that} bad an affliction.  So the
diagnostic test serves no purpose in your case.  You may as well not
have bothered to get diagnosed.  Even so, the test may be useful:

\bparts

\ppart Suppose Dr.\ Meyer had enough vaccine to treat 2\% of the
population.  If he randomly chose people to vaccinate, he could expect
to vaccinate only 2\% of the people who needed it.  But by testing
everyone and only vaccinating those diagnosed as future sufferers, he
can expect to vaccinate a much larger fraction people who were going
to suffer from Beaver Fever.  Estimate this fraction.

\examspace[2.0in]

\begin{solution}
$\mathbf{\approx 2/3}$.

The test will diagnose about 3\% of the population as future
sufferers.  This 3\% will include 99\% of the actual sufferers but
mostly include people who will not get Beaver Fever---the false
positives.  By giving the vaccine at random to only this 3\% that are
diagnosed as future sufferers, Dr.\ Meyer will have enough vaccine for
2/3 of them.  So he will be able to vaccinate nearly 2/3 of the people
who actually need it.

So even though the probability that a diagnosed person will suffer
Beaver Fever is small, the increased probability (from 1/1000 to about
1/32) provided by the diagnosis has significant public health value.
\end{solution}

\eparts

\end{problem}

%%%%%%%%%%%%%%%%%%%%%%%%%%%%%%%%%%%%%%%%%%%%%%%%%%%%%%%%%%%%%%%%%%%%%
% Problem ends here
%%%%%%%%%%%%%%%%%%%%%%%%%%%%%%%%%%%%%%%%%%%%%%%%%%%%%%%%%%%%%%%%%%%%%

\endinput
