\documentclass[problem]{mcs}

\begin{pcomments}
  \pcomment{TP_X_series}
  \pcomment{by ARM 10/3/12}
\end{pcomments}

\pkeywords{
  ring
  inverse
  power_series
  generating_function
}

%%%%%%%%%%%%%%%%%%%%%%%%%%%%%%%%%%%%%%%%%%%%%%%%%%%%%%%%%%%%%%%%%%%%%
% Problem starts here
%%%%%%%%%%%%%%%%%%%%%%%%%%%%%%%%%%%%%%%%%%%%%%%%%%%%%%%%%%%%%%%%%%%%%

\begin{problem}
Define the formal power series
\[
X \eqdef (0,1,0,0,\dots,0,\dots).
\]

\bparts 

\ppart Explain why $X$ has no reciprocal.

\hint What can you say about $x \cdot (g_0 + g_1x+ g_2 x^2 + \cdots)$?

\begin{solution}
\[
x \cdot (g_0 + g_1x+ g_2 x^2 + \cdots) = 0 + \text{multiples of $x$}
\neq 1,
\]
which in terms of formal power series means that
\[
X \otimes G \neq I
\]
for any series $G$.  That is, $X$ has no inverse.
\end{solution}

\ppart Use the definition of power series multiplication $\otimes$ to prove
carefully that
\[
X \otimes (g_0,g_1,g_2, \dots) = (0, g_0, g_1, g_2, \dots).
\]

\begin{solution}
\TBA{.}
\end{solution}

\ppart Recursively define $X^n$ for $n \in \nngint$ by
\begin{align*}
X^0 & \eqdef I \eqdef (1,0,0,\dots,0,\dots),\\
X^{n+1} & \eqdef X \otimes X^n.
\end{align*}
Verify that the monomial $x^n$ refers to the same power series as $X^n$.

\begin{solution}
\TBA{.}
\end{solution}

\eparts
\end{problem}


%%%%%%%%%%%%%%%%%%%%%%%%%%%%%%%%%%%%%%%%%%%%%%%%%%%%%%%%%%%%%%%%%%%%%
% Problem ends here
%%%%%%%%%%%%%%%%%%%%%%%%%%%%%%%%%%%%%%%%%%%%%%%%%%%%%%%%%%%%%%%%%%%%%
\endinput





