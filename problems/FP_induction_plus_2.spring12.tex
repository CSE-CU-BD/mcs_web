\documentclass[problem]{mcs}

\begin{pcomments}
  \pcomment{FP_induction_plus_2.spring12}
  \pcomment{from: F05.final}
  \pcomment{by adamc, 2012/01/13}
  \pcomment{wordsmithed by ARM 2/14/12}
\end{pcomments}

%%%%%%%%%%%%%%%%%%%%%%%%%%%%%%%%%%%%%%%%%%%%%%%%%%%%%%%%%%%%%%%%%%%%%
% Problem starts here
%%%%%%%%%%%%%%%%%%%%%%%%%%%%%%%%%%%%%%%%%%%%%%%%%%%%%%%%%%%%%%%%%%%%%

\begin{problem}

Suppose $P(n)$ is a predicate on nonnegative integers and suppose
\begin{equation}\label{Sk2}
\forall k. \; P(k) \QIMPLIES P(k+2).
\end{equation}
For each of the assertions below, determine whether it:
\begin{itemize}
\item \textbf{C}an hold (holds for some, but not all, $P$ satisfying~\eqref{Sk2})
\item \textbf{A}lways holds (for any such $P$), or
\item \textbf{N}ever holds (for any such $P$). 
\end{itemize}
Indicate which case applies \inbook{and briefly explain why}\inhandout{by
  \textbf{circling} the correct letter}.\\

\begin{problemparts}

%\begin{staffnotes}
%WAS COMMENTED OUT.
%\end{staffnotes}
%
%\problempart  $\forall n \ge 0.\; P(n)$ \inhandout{\hfill \textbf{C\quad A\quad N}}
%
%\begin{solution}
%\textbf{C}.  The assertion means that $P$ is always true.  So $P(k+2)$
%is always true, and therefore $P(k) \QIMPLIES P(k+2)$ is always true.
%So this case is possible.  But~\eqref{Sk2} also holds when $P$ is always
%false, so the asertion does not always hold when \eqref{Sk2} does.
%\end{solution}

%\begin{staffnotes}
%WAS COMMENTED OUT.
%\end{staffnotes}

\problempart  $\QNOT(P(0)) \QAND \forall n \ge 1.\; P(n)$ \inhandout{\hfill \textbf{C\quad A\quad N}}

\begin{solution}
\textbf{C}.  This formula says that $P$ is false at 0, but true
everywhere else.  So $P(k) \QIMPLIES P(k+2)$ still always holds
because $P(k+2)$ is still always true.  So this assertion can hold,
but not always, since~\eqref{Sk2} can hold when $P(0)$ is true.
\end{solution}

%\begin{staffnotes}
%WAS COMMENTED OUT.
%\end{staffnotes}
%
%\problempart $\forall n \ge 0.\; \QNOT(P(n))$  \inhandout{\hfill \textbf{C\quad A\quad N}}
%
%\begin{solution}
%\textbf{C.}  Now $P$ is always false.  So $P(k) \QIMPLIES P(k+2)$ is
%always true because $P(k)$ is false.  So this case is possible, but
%again does not always hold.
%\end{solution}

%\problempart $(\forall n \le 100.\; P(n)) \QAND (\forall n > 100.\; \QNOT(P(n)))$
%\inhandout{\hfill \textbf{C\quad A\quad N}}
%
%\begin{solution}
%\textbf{N}.  In this case, $P$ is true for $n$ up to 100 and false
%from 101 on.  So $P(99)$ is true, but $P(101)$ is false.  That means
%that $\QNOT[P(k) \QIMPLIES P(k+2)]$ for $k = 99$.  This case is
%impossible.
%\end{solution}

%\begin{staffnotes}
%WAS COMMENTED OUT.
%\end{staffnotes}

%\problempart $(\forall n \le 100.\; \QNOT(P(n))) \QAND (\forall n > 100.\; P(n))$
%\inhandout{\hfill \textbf{C\quad A\quad N}}
%
%\begin{solution}
%\textbf{C}.  In this case, $P$ is false for $n$ up to 100 and true
%from 101 on.  So $P(k) \QIMPLIES P(k+2)$ for $k \le 100$ because
%$P(k)$ is false, and $P(k) \QIMPLIES P(k+2)$ for $k \ge 99$ because
%$P(k+2)$ is true.  So this case is possible, but again does not always
%hold.
%\end{solution}

%\begin{staffnotes}
%WAS COMMENTED OUT.
%\end{staffnotes}

%\problempart $P(0) \QIMPLIES \forall n.\, P(n+2)$
%\inhandout{\hfill \textbf{C\quad A\quad N}}

%\begin{solution}
%\textbf{C}.  In this case, $P$ is false for $n$ up to 100 and true
%from 101 on.  So $P(k) \QIMPLIES P(k+2)$ for $k \le 100$ because
%$P(k)$ is false, and $P(k) \QIMPLIES P(k+2)$ for $k \ge 99$ because
%$P(k+2)$ is true.  So this case is possible, but again does not always
%hold.
%\end{solution}

%\begin{staffnotes}
%WAS COMMENTED OUT.
%\end{staffnotes}

\problempart $[\exists n.\, P(2n)] \QIMPLIES \forall n. \ P(2n+2)$
\inhandout{\hfill \textbf{C\quad A\quad N}}

\begin{solution}
\textbf{C}.  We see the case is possible by considering $P(n)$ that is always true.
A counterexample is $P(n)$ that holds iff $n > 3$, where the $\exists n$
is satisfied with $n = 2$, but the $\forall n$ fails for $n = 0$.
\end{solution}

\problempart $P(1) \QIMPLIES \forall n.\, P(2n+1)$
\inhandout{\hfill \textbf{C\quad A\quad N}}

\begin{solution}
  \textbf{A}.  This assertion says that if $P(1)$
  holds, then $P(n)$ holds for all odd $n$.  This case is always true.
\end{solution}

%\problempart $[\exists n.\, P(2n)] \QIMPLIES \forall n. \ P(2n+2)$
%\inhandout{\hfill \textbf{C\quad A\quad N}}
%
%\begin{solution}
%  \textbf{C}.  If $P(n)$ is always true, this assertion holds.  So
%  this case is possible.  If $P(n)$ is true only for even $n$ greater
%  than 4,~\eqref{Sk2} holds, but this assertion is false.  So this
%  case does not always hold.
%\end{solution}

\problempart $\exists n.\, \exists m > n.\, [P(2n) \QAND \QNOT(P(2m))]$
\inhandout{\hfill \textbf{C\quad A\quad N}}

\begin{solution}
  \textbf{N}.  This assertion says that $P$ holds for some even number
  $2n$ but not for some other larger even number $2m$.  However, if
  $P(2n)$ holds, we can apply~\eqref{Sk2} $n-m$ times to conclude
  $P(2m)$ also holds.  This case is impossible.
\end{solution}

\problempart $[\exists n.\, P(n)] \QIMPLIES \forall n.\, \exists m > n.\, P(m)$
\inhandout{\hfill \textbf{C\quad A\quad N}}

\begin{solution}
\textbf{A}.  This assertion says that if $P$ holds for some $n$, then
for every number, there is a larger number $m$ for which $P$ also
holds.  Since~\eqref{Sk2} implies that if there is one $n$ for which
$P(n)$ holds, there are an infinite, increasing chain of $k$'s for
which $P(k)$ holds, this case is always true.
\end{solution}

%\begin{staffnotes}
%WAS COMMENTED OUT.
%\end{staffnotes}

\problempart $\QNOT(P(0)) \QIMPLIES \forall n. \ \QNOT(P(2n))$
\inhandout{\hfill \textbf{C\quad A\quad N}}

\begin{solution}

%\TBA{solution}
\textbf{C}. If for all $n$, $P(n)$ is false, then the statement is true.
If $P(0)$ is false but $P(2)$ is true, then the statement is false. 
\end{solution}

\end{problemparts}

\end{problem}

%%%%%%%%%%%%%%%%%%%%%%%%%%%%%%%%%%%%%%%%%%%%%%%%%%%%%%%%%%%%%%%%%%%%%
% Problem ends here
%%%%%%%%%%%%%%%%%%%%%%%%%%%%%%%%%%%%%%%%%%%%%%%%%%%%%%%%%%%%%%%%%%%%%

\endinput
