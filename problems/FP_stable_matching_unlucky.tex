\documentclass[problem]{mcs}

\begin{pcomments}
  \pcomment{FP_stable_matching_unlucky}
  \pcomment{subsumed by PS_stable_matching_unlucky}
  \pcomment{Lemma made into problem TP_stable_wedding}
\end{pcomments}

\pkeywords{
  stable_matching
  state_machines
  termination
  partial_correctness
  invariant
}

\providecommand{\boys}{\text{Boys}}
\providecommand{\girls}{\text{Girls}}

%%%%%%%%%%%%%%%%%%%%%%%%%%%%%%%%%%%%%%%%%%%%%%%%%%%%%%%%%%%%%%%%%%%%%
% Problem starts here
%%%%%%%%%%%%%%%%%%%%%%%%%%%%%%%%%%%%%%%%%%%%%%%%%%%%%%%%%%%%%%%%%%%%%


\begin{problem}
  In a stable matching between $n$ boys and $n$ girls produced by the
  Mating Ritual, call a person \term*{lucky} if they are matched up
  with one of their top $\ceil{n/2}$ choices. Show that there must be
  at least one lucky person.

\begin{solution}

\begin{staffnotes}
Lemma is unnecessary, see PS\_ version.
\end{staffnotes}

We prove the following lemma first:

\begin{lemma*}
There exists a girl who gets serenaded for the first time on the
wedding day.
\end{lemma*}

\begin{proof}
Since there are an equal number of boys and girls, the wedding day is
when every girl is being serenaded by some (necessarily only one) boy.

Since ``being serenaded'' is a preserved invariant, any girl being
serenaded on some day continues being serenaded until the wedding day.
So if every girl has been serenaded on some day before the wedding
day, then all the girls are being serenaded on some day before the
wedding day, Thus the day before the wedding day should be the wedding
day, a contradiction.

\end{proof}

Assuming all boys are unlucky, the total number of rejections received
by the boys is at least $n(n/2)$.  Because one girl did not reject
anyone, these rejections need to be supplied by $n-1$ girls.  So on
average, every girl need to reject $n(n/2)/(n-1) > \floor{n/2}$ times.
At least one girl need to reject at least an average number of boys,
that is at least $\floor{n/2}$ times, and she is the lucky girl.
\end{solution}

\end{problem}


%%%%%%%%%%%%%%%%%%%%%%%%%%%%%%%%%%%%%%%%%%%%%%%%%%%%%%%%%%%%%%%%%%%%%
% Problem ends here
%%%%%%%%%%%%%%%%%%%%%%%%%%%%%%%%%%%%%%%%%%%%%%%%%%%%%%%%%%%%%%%%%%%%%


\endinput
