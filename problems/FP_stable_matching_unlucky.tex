%PS_stable_matching_unlucky.tex

\documentclass[problem]{mcs}

\begin{pcomments}
  \pcomment{PS_stable_matching_unlucky}
  \pcomment{from: F08.ps4; F03.ps4; F00.ps5; revised by ARM 10/15/09 based in part F08}
  \pcomment{revise after October 09 using F08 and Justin Zhang's suggestion}
\end{pcomments}

\pkeywords{
  stable_matching
  state_machines
  termination
  partial_correctness
  invariant
}

\providecommand{\boys}{\text{Boys}}
\providecommand{\girls}{\text{Girls}}

%%%%%%%%%%%%%%%%%%%%%%%%%%%%%%%%%%%%%%%%%%%%%%%%%%%%%%%%%%%%%%%%%%%%%
% Problem starts here
%%%%%%%%%%%%%%%%%%%%%%%%%%%%%%%%%%%%%%%%%%%%%%%%%%%%%%%%%%%%%%%%%%%%%


\begin{problem}
  In a stable matching between $n$ boys and girls produced by the
  Mating Ritual, call a person \term*{lucky} if they are matched up
  with one of their $\floor{n/2}$ top choices. Show that
\begin{theorem*}%\label{luckyperson}
  There must be at least one lucky person.
\end{theorem*}

\hint: you might want to prove the following lemma first: 
\begin{lemma*}
 There exists a girl, such that the wedding day is the first day she
 is being serenaded. 
\end{lemma*}

\begin{solution}
We try to prove the following lemma first:

\begin{lemma*}
There exists a girl, such that the wedding day is the first day
she is being serenaded. 
\end{lemma*}
Proof: 
If everyone has been serenaded before, then everyone was
serenaded the day before the wedding day, because ``being serenaded''
is a preserved invariant, thus the day before the wedding day should
be the wedding day, contradicting with the definition of the wedding day.

Thus there exists a girl who did not rejected anyone during the entire
ritual. Assuming all boys are unlucky, the total rejections received by
the boys is at least $n*\floor{n/2}$. Because one girl did not rejected
anyone, these rejections need to be supplied by $n-1$ girls. On
average every girl need to reject  $n/(n-1)*\floor{n/2}$ times.  At
least one girl need to reject more than $\floor{n/2}$ times, and she
is the lucky girl.  
\end{solution}

\end{problem}


%%%%%%%%%%%%%%%%%%%%%%%%%%%%%%%%%%%%%%%%%%%%%%%%%%%%%%%%%%%%%%%%%%%%%
% Problem ends here
%%%%%%%%%%%%%%%%%%%%%%%%%%%%%%%%%%%%%%%%%%%%%%%%%%%%%%%%%%%%%%%%%%%%%


\endinput
