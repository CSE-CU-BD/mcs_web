\documentclass[problem]{mcs}

\begin{pcomments}
  \pcomment{FP_RSA_TF}
  \pcomment{CH, spring '14; edits ARM 4/1/14 9:30PM}
\end{pcomments}

\pkeywords{
  RSA
  public key
  private key
  Euler's theorem
}

%%%%%%%%%%%%%%%%%%%%%%%%%%%%%%%%%%%%%%%%%%%%%%%%%%%%%%%%%%%%%%%%%%%%%
% Problem starts here
%%%%%%%%%%%%%%%%%%%%%%%%%%%%%%%%%%%%%%%%%%%%%%%%%%%%%%%%%%%%%%%%%%%%%

\begin{problem}

Indicate whether the following statements are \textbf{true} or
\textbf{false}\inhandout{ by circling \textbf{T} or \textbf{F}}.  Provide a brief
argument justifying your choice for each statement.

\bparts

\ppart Let $n$ and $a$ be positive integers.  If $n$ and $a$ are
relatively prime, then

\hspace{0.4in} $a^{(\phi(n)^2)} \equiv 1 \pmod{n}$. \inhandout{\hfill \textbf{T}
  \qquad \textbf{F}}

\begin{solution}

\textbf{True.}  Euler's Theorem states that if $n$ and $a$ are
relatively prime, then $a^{\phi(n)} \equiv 1 \pmod{n}$.  Raising both
sides to the $\phi(n)^{\text{th}}$ power, we get that
\[
a^{(\phi(n))^2} = \paren{a^{\phi(n)}}^{\phi(n)} \equiv 1^{\phi(n)} = 1 \pmod{n}.
\]
\end{solution}

\examspace[1.0in]

\ppart If $n$ is a product of two distinct primes, then $\phi(n)$ is
even.  \inhandout{\hfill \textbf{T} \qquad \textbf{F}}

\begin{solution}
\textbf{True}.  Let $n = pq$ for primes $p,q$.  Then, $\phi(n) =
\phi(p)\phi(q) = (p-1)(q-1)$.  Since 2 is the only even prime, at
least one of $p$ and $q$ must be odd, so at least one of $p-1$ and
$q-1$ must be even.  Therefore, $(p-1)(q-1)$ must be even.
\end{solution}

\examspace[1.0in]

\ppart Suppose $a, b, c, d$ are any four positive integers and $ac
\equiv bc \pmod{d}$.  Then

\hspace{0.4in} $a \equiv b \pmod{d}$.  \inhandout{\hfill \textbf{T} \qquad \textbf{F}}

\begin{solution}
  \textbf{False}.  To cancel $c$, we require that $c$ is relatively
  prime to $d$.  A counter-example is $a=1, b=2, d=3$, and $c=6$.
\end{solution}

\examspace[1.0in]

% \ppart In the RSA Cryptosystem, if $n=pq$ then the private key $d$ and the public key
% $(e,n)$ are chosen to satisfy:
% $d \cdot e \equiv 1 \pmod{pq}$. \inhandout{\hfill \textbf{T} \qquad \textbf{F}}
% \begin{solution}
%   \textbf{False}. The values of $d$ and $e$ are chosen to satisfy:
% \[
% d \cdot e \equiv 1 \pmod{(p-1)(q-1)} .
% \]
% \end{solution}
% \examspace[0.6in]

\ppart An efficient algorithm for \textsc{Factoring} would render RSA insecure. 
\inhandout{\hfill \textbf{T} \qquad \textbf{F}}

\begin{solution}
 \textbf{True}.  An eavesdropper could find the private key from the
 factorization of $n$ in the public key in the same way that the RSA
 Receiver does.  Namely, an eavesdropper could use the efficient
 factoring algorithm as follows: from the public key $(n, d)$,
 calculate the prime factors $p$ and $q$ of $n$, and compute the
 inverse of $d$ modulo $(p-1)(q-1)$ to get the private key $e$, after
 which the eavesdropper can decrypt any message transmitted by a
 sender.
\end{solution}

\examspace[1.0in]

\eparts

\end{problem}

\endinput
