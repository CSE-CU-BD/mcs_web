\documentclass[problem]{mcs}

\begin{pcomments}
  \pcomment{FP_trees_TF}
  \pcomment{from FP_multiple_choice_unhidden_fall13}
  \pcomment{revised from FP_multiple_choice_unhidden by ARM 12/13/13}
  \pcomment{overlaps FP_graphs_short_answer}
  \pcomment{revised ARM 11/20/17}
\end{pcomments}

\pkeywords{
  trees
  vertices
  leaf
  cycle
}

%%%%%%%%%%%%%%%%%%%%%%%%%%%%%%%%%%%%%%%%%%%%%%%%%%%%%%%%%%%%%%%%%%%%%
% Problem starts here
%%%%%%%%%%%%%%%%%%%%%%%%%%%%%%%%%%%%%%%%%%%%%%%%%%%%%%%%%%%%%%%%%%%%%

\begin{problem} \mbox{}

If $T$ is a \textbf{tree}, which of the following statements must be
true about $T$? \inbook{Indicate} \inhandout{Circle} \True\ or
\False\ for each, and \emph{provide counterexamples} for those
that are \False.

\bparts

\ppart Adding an edge between two nonadjacent vertices of $T$ creates a
  cycle. \hfill \True \qquad \False \examspace[0.7in]

\begin{solution}
\True
\end{solution}

\ppart Removing an edge of $T$ breaks a cycle. \hfill \True
\qquad \False \examspace[0.7in]

\begin{solution}
\False.  There are no cycles to break.  Every tree with at least one
edge is a counterexample.
\end{solution}

\ppart The number of vertices is one less than twice the number of
  leaves.  \hfill \True \qquad
  \False \examspace[0.7in]

\begin{solution}
  \False.  This property holds for full binary trees, but not
  in general.  A tree with two vertices is a counterexample.
\end{solution}

\iffalse
\ppart The number of leaves in $T$ is not equal to the number of
  non-leaf vertices.  \hfill \True \qquad
  \False \examspace[0.7in]

\begin{solution}
  \False.  A line graph with 4 vertices has 2 non-leaf
  vertices and 2 leaves.
\end{solution}
\fi

\ppart Any subgraph of $T$ is a tree.  \hfill \True \qquad
  \False \examspace[0.7in]

\begin{solution}
\False.  Any subgraph obtained by removing one or more edges is a
counterexample since it leaves a disconnected graph.
\end{solution}

%S11 %cut from Monday version
\ppart The number of vertices in $T$ is one less than the number of
  edges.  \hfill \True \qquad \False \examspace[0.7in]

\begin{solution}
  \False.  This got ``edges'' and ``vertices'' reversed.
Every tree is a counterexample.
\end{solution}

\iffalse
\ppart Any connected subgraph of $T$ is a tree.  \hfill \True \qquad
  \False \examspace[0.7in]

\begin{solution}
\True
\end{solution}
\fi


\eparts

\end{problem}

\endinput
