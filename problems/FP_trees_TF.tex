\documentclass[problem]{mcs}

\begin{pcomments}
  \pcomment{FP_trees_TF}
  \pcomment{from FP_multiple_choice_unhidden_fall13}
  \pcomment{revised from FP_multiple_choice_unhidden by ARM 12/13/13}
  \pcomment{overlaps FP_graphs_short_answer}  
\end{pcomments}

\pkeywords{
  isomorphism
  vertices
}

%%%%%%%%%%%%%%%%%%%%%%%%%%%%%%%%%%%%%%%%%%%%%%%%%%%%%%%%%%%%%%%%%%%%%
% Problem starts here
%%%%%%%%%%%%%%%%%%%%%%%%%%%%%%%%%%%%%%%%%%%%%%%%%%%%%%%%%%%%%%%%%%%%%

\begin{problem} \mbox{}

If $T$ is a \textbf{tree}, which of the following statements must be true about $T$? Circle \textbf{true} or \textbf{false} for each, and \emph{provide counterexamples} for those that are \textbf{false}.

\bparts

\ppart Any connected subgraph of $T$ is a tree.  \hfill \textbf{true} \qquad
  \textbf{false} \examspace[0.7in]

\begin{solution}
\textbf{true}
\end{solution}

\ppart Adding an edge between two nonadjacent vertices of $T$ creates a
  cycle. \hfill \textbf{true} \qquad \textbf{false} \examspace[0.7in]

\begin{solution}
\textbf{true}
\end{solution}

\ppart The number of vertices is one less than twice the number of
  leaves.  \hfill \textbf{true} \qquad
  \textbf{false} \examspace[0.7in]

\begin{solution}
  \textbf{false}.  This property holds for full binary trees, but not
  in general.  A tree with two vertices is a counterexample.
\end{solution}

\ppart The number of leaves in $T$ is not equal to the number of
  non-leaf vertices.  \hfill \textbf{true} \qquad
  \textbf{false} \examspace[0.7in]

\begin{solution}
  \textbf{false}.  A line graph with 4 vertices has 2 non-leaf
  vertices and 2 leaves.
\end{solution}

\ppart Any subgraph of $T$ is a tree.  \hfill \textbf{true} \qquad
  \textbf{false} \examspace[0.7in]

\begin{solution}
\textbf{false.  Choose a subgraph with 2 vertices and no edges.}
\end{solution}

%S11 %cut from Monday version
\ppart The number of vertices in $T$ is one less than the number of
  edges.  \hfill \textbf{true} \qquad \textbf{false} \examspace[0.7in]

\begin{solution}
  \textbf{false}.  This got ``edges'' and ``vertices'' reversed.
Every tree is a counterexample.
\end{solution}


\eparts

\end{problem}

\endinput
