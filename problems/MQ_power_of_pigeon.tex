\documentclass[problem]{mcs}

\begin{pcomments}
  \pcomment{MQ_power_of_pigeon.tex}
  \pcomment{variant of PS_pigeonhole-power_of_3}
\end{pcomments}

\pkeywords{
  pigeonhole
  power_of_2
}

%%%%%%%%%%%%%%%%%%%%%%%%%%%%%%%%%%%%%%%%%%%%%%%%%%%%%%%%%%%%%%%%%%%%%
% Problem starts here
%%%%%%%%%%%%%%%%%%%%%%%%%%%%%%%%%%%%%%%%%%%%%%%%%%%%%%%%%%%%%%%%%%%%%

\begin{problem}
Suppose $n+1$ numbers are selected from $\set{1,2,3, \dots,2n}$.  Show
that there must be two selected numbers whose quotient is a power of
two.

\examspace[3.5in]

\begin{solution}
  The $n+1$ selected numbers are the pigeons.  There are $n$ odd
  numbers between 1 and $2n$, namely, the numbers of the form $2k+1$
  for $0 \leq k \leq n-1$.  These will be the pigeonholes.  A pigeon
  will be assigned to the largest odd number that divides it.  For
  example, $31 \cdot 2^0, 31 \cdot 2^1, 31 \cdot 2^2,\dots$ are all
  assigned to pigeonhole 31.

  Two pigeons must assigned to some hole, and these can be the two
  selected numbers, since the quotient of the larger of these two by
  the smaller will be a power of two by definition.
\end{solution}

\end{problem}

\endinput
