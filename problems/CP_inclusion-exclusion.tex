\documentclass[problem]{mcs}

\begin{pcomments}
\pcomment{CP_inclusion-exclusion}
\pcomment{overlaps FP_combinatorial_doritos}
\pcomment{final.F13 but garbled so not graded}
\pcomment{adapted from S04.final}
\pcomment{ARM 12/13/13}
\end{pcomments}

\pkeywords{
  counting
  combinatorial_proof
  inclusion-exclusion
}

\begin{problem}
Each day, an MIT student selects a breakfast from among $b$
possibilities, lunch from among $l$ possibilities, and dinner from
among $d$ possibilities.  In each case one of the possibilities is
Doritos.  However, a legimate daily menu may include Doritos for at
most one meal.  Give a combinatorial (not algebraic) proof based on
the number of legimate daily menus that
\[\begin{array}{l}
bld-[(b-1)+(l-1)+(d-1)+1]\\
\quad = b(l-1)(d-1)  + (b-1)l(d - 1) + (b-1)(l-1)d\\
\qquad - 3(b-1)(l-1)(d-1) + (b-1)(l-1)(d-1)
\end{array}\]

\hint Let $M_b$ be the number of menus where, if Doritos appear at
all, they only appear at \emph{b}reakfast; likewise, for $M_{l},
M_{d}$.

\begin{solution}
Both sides of the equation equal the number of legimate daily menus.

The left hand side counts the number $bld$ of all possible menus minus
the number of illegimate menus.  The number of illegimate menus are
those including Doritos at exactly two meals (for example, $(b-1)$ is
the number menus with Doritos at lunch and dinner but not at
breakfast), and 1 is the number of illegimate menus with Doritos at
all three meals.

The set of legimate menus is $M_{b} \union M_{l} \union M_{d}$.  The
right hand side counts the size of this union using
inclusion-exclusion.  The first three terms are the respective sizes
of $M_{b}$, $M_{l}$, and $M_{d}$.  The next term is the sum of the
sizes of the three possible two-way intersections $M_b \intersect
M_l$, $M_b \intersect M_d$, $M_l \intersect M_d$, each of which equals
the same set, namely, the menus with no Doritos, and the last term is
the size of the three-way intersection $M_{b} \intersect M_{l}
\intersect M_{d}$, which also happens to be the set of menus with no
Doritos.
\end{solution}

\end{problem}

\endinput
