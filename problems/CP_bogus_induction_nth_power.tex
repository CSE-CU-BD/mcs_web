\documentclass[problem]{mcs}

\begin{pcomments}
  \pcomment{CP_bogus_induction_nth_power}
  \pcomment{from: F09.ps3; S02.ps2}
  \pcomment{edited by ARM to be simple induction 3/11/10}
  \pcomment{was called CP_flawed_induction_proof}
\end{pcomments}

\pkeywords{
  bogus
  induction
  false_proof
  power
  exponent
}

%%%%%%%%%%%%%%%%%%%%%%%%%%%%%%%%%%%%%%%%%%%%%%%%%%%%%%%%%%%%%%%%%%%%%
% Problem starts here
%%%%%%%%%%%%%%%%%%%%%%%%%%%%%%%%%%%%%%%%%%%%%%%%%%%%%%%%%%%%%%%%%%%%%

\begin{problem}
  Find the flaw in the following bogus proof that $a^n = 1$ for all
  nonnegative integers $n$, whenever $a$ is a nonzero real number.

\begin{bogusproof}
The proof is by induction on $n$, with hypothesis
\[
P(n) \eqdef \forall k \leq n.\, a^k = 1,
\]
where $k$ is a nonnegative integer valued variable.

\inductioncase{Base Case}: $(n=0)$.  $P(0)$ is equivalent to $a^0 =1$,
which is true by definition of $a^0$.  (By convention, this holds even
if $a=0$.)

\inductioncase{Inductive Step:} By induction hypothesis, $a^k = 1$ for all $k \in
\naturals$ such that $k \leq n$.  But then
\[
a^{n+1} = \frac{a^n \cdot a^n}{a^{n-1}} = \frac{1 \cdot 1}{1} = 1,
\]
which implies that $P(n+1)$ holds.  It follows by induction that
$P(n)$ holds for all $n \in \naturals$, and in particular, $a^n = 1$
holds for all $n \in \naturals$.

\end{bogusproof}

\begin{solution}
The flaw comes in the inductive step, where we implicitly assume
$n\geq 1$ in order to talk about $a^{n-1}$ in the denominator
(otherwise the exponent is not a nonnegative integer, so we cannot
apply the inductive hypothesis).  We checked the base case only for
$n=0$, so we are not justified in assuming that $n\geq 1$ when we try
to prove the statement for $n+1$ in the inductive step.  And of course
the proposition first breaks precisely at $n=1$.
\end{solution}

\end{problem}

%%%%%%%%%%%%%%%%%%%%%%%%%%%%%%%%%%%%%%%%%%%%%%%%%%%%%%%%%%%%%%%%%%%%%
% Problem ends here
%%%%%%%%%%%%%%%%%%%%%%%%%%%%%%%%%%%%%%%%%%%%%%%%%%%%%%%%%%%%%%%%%%%%%

 \endinput
