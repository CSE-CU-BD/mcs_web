\documentclass[problem]{mcs}

\begin{pcomments}
  \pcomment{CP_like_Lehmans_equation}
  \pcomment{from F05.cp3f}
  \pcomment{edited by ARM 3/14/12}
\end{pcomments}

\pkeywords{
  well_ordering
  WOP
  contradiction
}

%%%%%%%%%%%%%%%%%%%%%%%%%%%%%%%%%%%%%%%%%%%%%%%%%%%%%%%%%%%%%%%%%%%%%
% Problem starts here
%%%%%%%%%%%%%%%%%%%%%%%%%%%%%%%%%%%%%%%%%%%%%%%%%%%%%%%%%%%%%%%%%%%%%

\begin{problem}
Use the Well Ordering Principle to prove that there is no solution
over the positive integers to the equation:
\[
4a^3 + 2b^3 = c^3.
\]

\begin{solution}
Let $S$ be the set of all positive integers $a$ for which there are
positive integers $b$ and $c$ that satisfy this equation.  We want to
show that $S$ is empty.

Assume to the contrary that $S$ is nonempty.  Then by the
well-ordering principle, $S$ contains a smallest element $a_0$.  By
the definition of $S$, there are corresponding positive integers $b_0$
and $c_0$ such that:
\[ 
4a_0^3 + 2b_0^3 = c_0^3.
\] 
The left side of this equation is even, so $c_0^3$ is even, and
therefore $c_0$ is also even.  Thus, there exists an integer $c_1$
such that $c_0 = 2 c_1$.  Substituting $2 c_1$ for $c_0$ in the
preceding equation and then dividing both sides by 2 gives:
\[ 
2 a_0^3 + b_0^3 = 4 c_1^3.
\] 
Now $b_0^3$ must be even, so $b_0$ is even.  Thus, there is an integer
$b_1$ such that $b_0 = 2 b_1$.  Substituting for $b_0$ in the
preceding equation and dividing both sides by 2 gives:
\[ 
a_0^3 + 4 b_1^3 = 2 c_1^3 .
\] 
From this equation, we know that $a_0^3$ is even, so $a_0$ is also
even.  Thus, there exists an integer $a_1$ such that $a_0 = 2 a_1$.
Finally, substituting for $a_0$ in the previous equation and
dividing by 2 gives
\[ 
4 a_1^3 + 2 b_1^3 = c_1^3.
\] 

So $a = a_1$, $b = b_1$ and $c = c_1$ is another solution to the
original equation, which means that $a_1 \in S$ by definition.  But
$a_1 < a_0$, contradicting the fact that $a_0$ is the smallest element
of $S$.

This contradiction implies that $S$ nust be empty, which means the
original equation has no solutions over the positive integers.
\end{solution}

\end{problem}


%%%%%%%%%%%%%%%%%%%%%%%%%%%%%%%%%%%%%%%%%%%%%%%%%%%%%%%%%%%%%%%%%%%%%
% Problem ends here
%%%%%%%%%%%%%%%%%%%%%%%%%%%%%%%%%%%%%%%%%%%%%%%%%%%%%%%%%%%%%%%%%%%%%

\endinput


