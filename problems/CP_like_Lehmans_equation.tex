\documentclass[problem]{mcs}

\begin{pcomments}
  \pcomment{CP_like_Lehmans_equation}
  \pcomment{from F05, cp3f}
  \pcomment{edited by ARM 3/14/12}
\end{pcomments}

\pkeywords{
  well_ordering
  WOP
  contradiction
}

%%%%%%%%%%%%%%%%%%%%%%%%%%%%%%%%%%%%%%%%%%%%%%%%%%%%%%%%%%%%%%%%%%%%%
% Problem starts here
%%%%%%%%%%%%%%%%%%%%%%%%%%%%%%%%%%%%%%%%%%%%%%%%%%%%%%%%%%%%%%%%%%%%%

\begin{problem}
Use the Well Ordering Principle to prove that there is no solution
over the positive integers to the equation:
\[
4a^3 + 2b^3 = c^3.
\]

\begin{solution}
We use contradiction and the Well Ordering principle.  Let $S$ be the
set of all positive integers $a$ such that there exist positive
integers $b$ and $c$ that satisfy the equation.

Assume for the purpose of obtaining a contradiction that $S$ is nonempty.
Then $S$ contains a smallest element $a_0$ by the well-ordering principle.
sBy the definition of $S$, there exist corresponding positive integers
$b_0$ and $c_0$ such that:
\[ 
4a_0^3 + 2b_0^3 = c_0^3 
\] 
The left side of this equation is even, so $c_0^3$ is even, and therefore
$c_0$ is also even.  Thus, there exists an integer $c_1$ such that $c_0
= 2 c_1$.  Substituting into the preceding equation and then dividing both
sides by 2 gives:
\[ 
2 a_0^3 + b_0^3 = 4 c_1^3 
\] 
Now $b_0^3$ must be even, so $b_0$ is even.  Thus, there exists an
integer $b_1$ such that $b_0 = 2 b_1$.  Substituting into the preceding
equation and dividing both sides by 2 again gives:
\[ 
a_0^3 + 4 b_1^3 = 2 c_1^3 
\] 
From this equation, we know that $a_0^3$ is even, so $a_0$ is also even.
Thus, there exists an integer $a_1$ such that $a_0 = 2 a_1$.
Substituting into the previous equation one last time and dividing by 2
one last time gives:
% 
\[ 
4 a_1^3 + 2 b_1^3 = c_1^3 
\] 

So $a = a_1$, $b = b_1$ and $c = c_1$ is another solution to 
the original equation, and so $a_1$ is an element of $S$.  But this is 
a contradiction, because $a_1 < a_0$ and $a_0$ was defined to be the 
smallest element of $S$.  Therefore, our assumption was wrong, and the 
original equation has no solutions over the positive integers. 
\end{solution}

\end{problem}


%%%%%%%%%%%%%%%%%%%%%%%%%%%%%%%%%%%%%%%%%%%%%%%%%%%%%%%%%%%%%%%%%%%%%
% Problem ends here
%%%%%%%%%%%%%%%%%%%%%%%%%%%%%%%%%%%%%%%%%%%%%%%%%%%%%%%%%%%%%%%%%%%%%

\endinput


