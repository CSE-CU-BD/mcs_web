\documentclass[problem]{mcs}

\begin{pcomments}
  \pcomment{PS_drunken_sailor}
  \pcomment{from: S09.ps11}
\end{pcomments}

\pkeywords{
  probability
  random_variables
}

%%%%%%%%%%%%%%%%%%%%%%%%%%%%%%%%%%%%%%%%%%%%%%%%%%%%%%%%%%%%%%%%%%%%%
% Problem starts here
%%%%%%%%%%%%%%%%%%%%%%%%%%%%%%%%%%%%%%%%%%%%%%%%%%%%%%%%%%%%%%%%%%%%%

\begin{problem}
  A drunken sailor wanders along main street, which conveniently consists
  of the points along the $x$ axis with integral coordinates.  In each
  step, the sailor moves one unit left or right along the $x$ axis.  A
  particular \term{path} taken by the sailor can be described by a
  sequence of ``left'' and ``right'' steps.  For example,
  $\ang{\text{left,left,right}}$ describes the walk that goes left twice
  then goes right.

  We model this scenario with a \idx{random walk} graph whose vertices
  are the integers and with edges going in each direction between
  consecutive integers.  All edges are labelled $1/2$.

  The sailor begins his random walk at the origin.  This is described by
  an initial distribution which labels the origin with probability 1 and
  all other vertices with probability 0.  After one step, the sailor is
  equally likely to be at location 1 or $-1$, so the distribution after
  one step gives label 1/2 to the vertices 1 and $-1$ and labels all other
  vertices with probability 0.

\bparts

\ppart

Give the distributions after the 2nd, 3rd, and 4th step by filling in the
table of probabilities below, where omitted entries are 0.  For each row,
write all the nonzero entries so they have the same denominator.
\begin{center}
\begin{tabular}{l|ccccccccc}
  & \multicolumn{9}{c}{location} \\
  & -4 & -3 & -2 & -1 & 0 & 1 & 2 & 3 & 4 \\ \hline\hline
  initially & & & & & $1$ & & & & \\
  after 1 step & & & & $1/2$ & 0 & $1/2$ & & & \\
  after 2 steps & & & ? & ? & ? & ? & ? & & \\
  after 3 steps & & ? & ? & ? & ? & ? & ? & ? &  \\
  after 4 steps & ? & ? & ? & ? & ? & ? & ? & ? & ?
\end{tabular}
\end{center}

\begin{solution}\
\begin{center}
  \begin{tabular}{l|ccccccccc}
    & \multicolumn{9}{c}{location} \\
    & -4 & -3 & -2 & -1 & 0 & 1 & 2 & 3 & 4 \\ \hline\hline
    initially & & & & & $1$ & & & & \\
    after 1 step & & & & $1/2$ & 0 & $1/2$ & & & \\
    after 2 steps & & & $1/4$ & 0 & $2/4$ & 0 & $1/4$ & & \\
    after 3 steps & & $1/8$ & 0 & $3/8$ & 0 & $3/8$ & 0 & $1/8$ &  \\
    after 4 steps & $1/16$ & 0 & $4/16$ & 0 & $6/16$ & 0 & $4/16$ & 0 & $1/16$
  \end{tabular}
\end{center}
\end{solution}

\ppart\label{iright}\

\begin{enumerate}

\item  What is the final location of a $t$-step path that moves right exactly
  $i$ times?

\item How many different paths are there that end at that location?

\item What is the probability that the sailor ends at this location?

\end{enumerate}

\begin{solution}
  If he takes $i$ steps to the right, then he takes $t-i$ steps to the
  left.  Since steps to the left and right cancel, he nets $i - (t-i)
  = 2i-t$ steps to the right, ending at location $2i-t$.

  The number of paths is just the number of length-$t$ sequences that
  contain $i$ ``right''s, which is $\binom{t}{i}$.

  Each path is equally likely, so he takes the given path with
  probability $1/2^t$.  Thus, he ends at the location $2i-t$ with
  probability
\[
2^{-t}\binom{t}{i}.
\]
\end{solution}

\ppart Let $L$ be the random variable giving the sailor's location after
$t$ steps, and let $B \eqdef (L + t)/2$.  Use the answer to
part~\eqref{iright} to show that $B$ has an unbiased binomial density
function.

\begin{solution}
  From the previous part, we have $\pr{L = 2x-t} =
  2^{-t} \binom{t}{x}$ for $0 \leq x \leq t$.

  We have
  \begin{eqnarray*}
    \pdf_{B}(x) &=& \pr{B = x} \\
    &=& \pr{(L+t)/2 = x} \\
    &=& \pr{L = 2x - t}\\
    &=& \frac{1}{2^t} \binom{t}{x}  \ ,
  \end{eqnarray*}
  which is the binomial distribution.
\end{solution}

\ppart Again let $L$ be the random variable giving the sailor's
location after $t$ steps, where $t$ is even.  Show that
\[
\pr{\abs{L} < \frac{\sqrt{t}}{2}} < \frac{1}{2} \ .
\]
So there is a better than even chance that the sailor ends up at least
$\sqrt{t}/2$ steps from where he started.

\hint Work in terms of $B$.  Then you can use an estimate that bounds the
binomial distribution.  Alternatively, observe that the origin is
the most likely final location and then use the asymptotic estimate
\[
\pr{L = 0} = \pr{B = t/2} \sim \sqrt{\frac{2}{\pi t}}.
\]
%together with the Union Bound.

\begin{solution}
  Translating to $B$, we're looking for
\[
\pr{\abs{L} < \frac{\sqrt{t}}{2}} = \pr{\abs{2B-t} <
    \frac{\sqrt{t}}{2}} \ .
\]
  This expression $\abs{2B-t} < \sqrt{t}/2$ is true for $t/2 -
  \sqrt{t}/4 < B < t/2 + \sqrt{t}/4$, which allows
  $\sqrt{t}/2$ different values of $B$.

  Since $B$ follows the binomial distribution, we have $\pr{L = 0} =
  \pr{B = t/2} \leq \sqrt{\frac{2}{\pi t}}$.  Moreover, the peak of
  the binomial distribution occurs at $B=t/2$.  Thus, any other point is
  less likely.  Now we have

%  Now we just take a union bound to complete the proof:

\begin{eqnarray*}
    \pr{\abs{L} < \frac{\sqrt{t}}{4}}
   &=& \pr{\abs{2B-t} < \frac{\sqrt{t}}{4}}\\
   &=& \sum_{\set{x \suchthat\ \abs{2x-t} < \sqrt{t}/4}} \pr{B = x} \\
   &\leq& \sum_{\set{x \suchthat\ \abs{2x-t} < \sqrt{t}/4}} \pr{B = t/2} \\
    &\leq& \sum_{\set{x \suchthat\ \abs{2x-t} < \sqrt{t}/4}} \sqrt{\frac{2}{\pi t}} \\
    &\leq& \paren{\frac{\sqrt{t}}{2}} \sqrt{\frac{2}{\pi t}}\\
%    \hspace{1cm} \text{by Union Bound}\\
    &=& \frac{1}{\sqrt{2\pi}} < \frac{1}{2}.
  \end{eqnarray*}
\end{solution}

\eparts
\end{problem}

%%%%%%%%%%%%%%%%%%%%%%%%%%%%%%%%%%%%%%%%%%%%%%%%%%%%%%%%%%%%%%%%%%%%%
% Problem ends here
%%%%%%%%%%%%%%%%%%%%%%%%%%%%%%%%%%%%%%%%%%%%%%%%%%%%%%%%%%%%%%%%%%%%%

\endinput
