\documentclass[problem]{mcs}

\begin{pcomments}
  \pcomment{PS_drunken_sailor}
  \pcomment{revised 4/29/14 as PS_sailor_random_walk}
  \pcomment{S09.ps11, F12.ps10, S14.ps11}
  \pcomment{revised 5/2/17 ARM and renamed as drunken_sailor to avoid
    random-walk graphs}
\end{pcomments}

\pkeywords{
  random_variable
  random_walk
  density_function
  pdf
  binomial_distribution
}

%%%%%%%%%%%%%%%%%%%%%%%%%%%%%%%%%%%%%%%%%%%%%%%%%%%%%%%%%%%%%%%%%%%%%
% Problem starts here
%%%%%%%%%%%%%%%%%%%%%%%%%%%%%%%%%%%%%%%%%%%%%%%%%%%%%%%%%%%%%%%%%%%%%

\begin{problem}
An over-caffeinated sailor of Tech Dinghy wanders along Seaside
Boulevard.  In each step, the sailor randomly moves one unit left or
right with equal probability.

\iffalse
The sailor's
movements can be described by a sequence of ``left'' and ``right''
steps.  For example, $\ang{\text{left, left, right}}$ means the sailor
went left twice then went right once.

We model this scenario with a random walk graph whose vertices are the
integers and with edges going in each direction between consecutive
integers.  All edges are labelled $1/2$.
\fi

We let the sailor's initial position be designated location zero, with
successive positions to the right labelled 1,2,\dots, and positions to
the left labelled -1,-2,\dots.  Let $L_t$ be the random variable
giving the sailor's location after $t$ steps.  Before he starts, the
sailor is known to be at location zero, so
\[
\pdf_{L_0}(n) =
\begin{cases}
1 & \text{if }n = 0,\\
0 & \text{otherwise}.
\end{cases}
\]
After one step, the sailor is equally
likely to be at location 1 or $-1$, so
\[
\pdf_{L_1}(n) =
\begin{cases}
1/2 & \text{if } n = \pm 1,\\
0 & \text{otherwise}.
\end{cases}
\]

\bparts 

\ppart Give the distributions $\pdf_{L_t}$ for $t =2,3,4$ by filling
in the table of probabilities below, where omitted entries are 0.  For
each row, write all the nonzero entries so they have the same
denominator.
\begin{center}
\begin{tabular}{l|ccccccccc}
  & \multicolumn{9}{c}{location} \\
  & -4 & -3 & -2 & -1 & 0 & 1 & 2 & 3 & 4 \\ \hline\hline
  initially & & & & & $1$ & & & & \\
  after 1 step & & & & $1/2$ & 0 & $1/2$ & & & \\
  after 2 steps & & & ? & ? & ? & ? & ? & & \\
  after 3 steps & & ? & ? & ? & ? & ? & ? & ? &  \\
  after 4 steps & ? & ? & ? & ? & ? & ? & ? & ? & ?
\end{tabular}
\end{center}
  
\begin{solution}
\begin{center}
  \begin{tabular}{l|ccccccccc}
    & \multicolumn{9}{c}{location} \\
    & -4 & -3 & -2 & -1 & 0 & 1 & 2 & 3 & 4 \\ \hline\hline
    initially & & & & & $1$ & & & & \\
    after 1 step & & & & $1/2$ & 0 & $1/2$ & & & \\
    after 2 steps & & & $1/4$ & 0 & $2/4$ & 0 & $1/4$ & & \\
    after 3 steps & & $1/8$ & 0 & $3/8$ & 0 & $3/8$ & 0 & $1/8$ &  \\
    after 4 steps & $1/16$ & 0 & $4/16$ & 0 & $6/16$ & 0 & $4/16$ & 0 & $1/16$
  \end{tabular}
\end{center}
\end{solution}

\ppart \label{ppart:iright} Help the staff of the Sailing Pavilion
locate the sailor by answering the following questions.  Provide your
derivations and reasoning.

\begin{enumerate}[(i)]

\item What is the final location of a $t$-step walk that moves right
  exactly $i$ times?

\begin{solution}
If he takes $i$ steps to the right, then he takes $t-i$ steps to the
left.  Since steps left and right cancel, he nets $i - (t-i) = 2i-t$
steps to the right, ending at location $2i-t$.
\end{solution}

\item How many different length-$t$ walks are there that end at that
  location?

\begin{solution}
The number of walks is the number of length-$t$ sequences with $i$
``right''s, which is
\[
\binom{t}{i}.
\]
\end{solution}

\item What is the probability that the sailor ends at this location?

\begin{solution}
Each walk is equally likely, so he takes the given walk with
probability $1/2^t$.  That is,
\[
\pr{L_t = 2i-t} =  \binom{t}{i}2^{-t}.
\]
\end{solution}

\item Let $B_t ::= (L_t + t)/2$.  Conclude that $B_t$ has an unbiased
  binomial distribution.

\begin{solution}
\begin{align*}
 \pdf_{B_t}(i)
  &= \pr{B_t = i} \\
  &= \pr{(L_t+t)/2 = i} \\
  &= \pr{L_t = 2i-t}\\
  &=  \binom{t}{i}2^{-t},
\end{align*}
which is an unbiased binomial distribution.

To find the probability of ending at location $n$ after $t$ steps, let
$n = 2i-t$ so $i = (n+t)/2$.  Now
\[
\pr{L_t = n} = \binom{t}{(n+t)/2}2^{-t}
\]
for $\abs{n} \leq t$ and $(n+t)/2 \in \nngint$.  In other words, the
probability that after $t$ steps the sailor finishes exactly $n$ steps
to the right is the same as the probability that a fair coin flipped
$t$ times comes up heads exactly $(n+t)/2$ times.
\end{solution}

\end{enumerate}

\eparts
\end{problem}

%%%%%%%%%%%%%%%%%%%%%%%%%%%%%%%%%%%%%%%%%%%%%%%%%%%%%%%%%%%%%%%%%%%%%
% Problem ends here
%%%%%%%%%%%%%%%%%%%%%%%%%%%%%%%%%%%%%%%%%%%%%%%%%%%%%%%%%%%%%%%%%%%%%

\endinput
