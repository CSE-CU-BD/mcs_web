\documentclass[problem]{mcs}

\begin{pcomments}
  \pcomment{MQ_pigeonhole_cards_4_of_a_kind}
\end{pcomments}

\pkeywords{
  counting
  pigeonhole principle
  cards
  deck
  rank
}

%%%%%%%%%%%%%%%%%%%%%%%%%%%%%%%%%%%%%%%%%%%%%%%%%%%%%%%%%%%%%%%%%%%%%
% Problem starts here
%%%%%%%%%%%%%%%%%%%%%%%%%%%%%%%%%%%%%%%%%%%%%%%%%%%%%%%%%%%%%%%%%%%%%

\begin{problem}
In a standard 52-card deck, there are 13 ranks.  Use the Pigeonhole
Principle to find the smallest $k$ such that every size $k$ subset of
the cards contains \emph{four} cards of the same rank.  Clearly indicate
what are the pigeons, holes, and rules for assigning a pigeon to a
hole, and give the value of $k$.

\examspace[2.0in]

\begin{solution}
\[
k = 40.
\]

The cards in the size $k$ subset are the pigeons, the 13 card ranks are the
pigeon holes, and a pigeon is assigned to its rank. 
By the Generalized Pigeonhole Principle, we want the smallest integer $k$ such that
\[
\ceil{\frac{k}{13}} = 4,
\]
which is equivalent to
\[
\frac{k}{13} > 3.
\]
\end{solution}

\end{problem}

%%%%%%%%%%%%%%%%%%%%%%%%%%%%%%%%%%%%%%%%%%%%%%%%%%%%%%%%%%%%%%%%%%%%%
% Problem ends here
%%%%%%%%%%%%%%%%%%%%%%%%%%%%%%%%%%%%%%%%%%%%%%%%%%%%%%%%%%%%%%%%%%%%%

\endinput
