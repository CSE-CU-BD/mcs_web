\documentclass[problem]{mcs}

\begin{pcomments}
  \pcomment{PS_uncountable_infinite_sequences}
  \pcomment{from: S10 by wing}
\end{pcomments}

\pkeywords{
  Russells_paradox
}

%%%%%%%%%%%%%%%%%%%%%%%%%%%%%%%%%%%%%%%%%%%%%%%%%%%%%%%%%%%%%%%%%%%%%
% Problem starts here
%%%%%%%%%%%%%%%%%%%%%%%%%%%%%%%%%%%%%%%%%%%%%%%%%%%%%%%%%%%%%%%%%%%%%

\begin{problem}
  
  Let $ A^{\naturals} $ be the set of all countably infinite sequences of 
  elements from set $A$. \footnote {$ A^B $ is more formally defined as 
  the set of all total functions from $B$ to $A$.}
  
  For example, the set $ \{1,2,3\}^{\naturals} $ will include elements such 
  as $ (1, 1, 1, 1 ...) $, $ (2, 1, 1, 1 ...) $, $ (3, 1, 1, 1 ...) $, 
  $ (1, 2, 1, 1 ...) $, $ (2, 2, 1, 1 ...) $, $ (3, 2, 1, 1 ...) $, etc.
  
  Prove that, for any set $A$ with more than 1 element, there are
  uncountably many infinite sequences in the set $A^{\naturals}$ 
  (that $A^{\naturals}$ has a higher cardinality than $\naturals$).
  
  \hint Try to find a bijection from the power set $\power(A)$ to $\{1,2\}^{\naturals}$.
  
  \hint You may assume that $A^{\naturals}$ is as large as $B^{\naturals}$ 
  if $A$ is as large as $B$.

\end{problem}

\endinput
