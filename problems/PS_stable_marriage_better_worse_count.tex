\documentclass[problem]{mcs}

\begin{pcomments}
  \pcomment{PS_stable_marriage_better_worse_count}
  \pcomment{REVISE in face of MQ_stable_marriage_optimal_pessimal}
  \pcomment{alternative to PS_stable_marriage_better_worse, maybe simpler}
  \pcomment{by ARM 5/6/15}
  \pcomment{Lemma suggested by Leonid Levin; proof by ARM}

\end{pcomments}

\pkeywords{
 stable_matching
 optimal_spouse
 pessimal
}


%%%%%%%%%%%%%%%%%%%%%%%%%%%%%%%%%%%%%%%%%%%%%%%%%%%%%%%%%%%%%%%%%%%%%
% Problem starts here
%%%%%%%%%%%%%%%%%%%%%%%%%%%%%%%%%%%%%%%%%%%%%%%%%%%%%%%%%%%%%%%%%%%%%

\begin{problem}
Suppose there are two stable sets of marriages\iffalse , a first set
and a second set\fi.  So each man has a first wife and a second wife
\iffalse(they may be the same)\fi, and likewise each woman has a first
husband and a second husband.

Someone in a given marriage is a \emph{winner} when they prefer their
current spouse to their other spouse, and they are a \emph{loser} when
they prefer their other spouse to their current spouse.  (If someone
has the same spouse in both of their marriages, then they will be
neither a winner nor a loser.)

We will show that
\begin{equation}\tag{WL}
\text{In every marriage, someone is a winner iff their spouse is a loser.}
\end{equation}

\begin{staffnotes}
REVISE in light of:

\begin{quote}
These is a easy direct proof: Suppose David is Carol's optimal spouse.
Then in any stable set of marriages in which Carol is not marriaed to
David, Carol will prefer David to her spouse.  So if David is married
to anyone her likes less than Carol, he and Carol are a rogue couple.
To avoid this, Carol must be Ben's pessimal spouse.
\end{quote}
\end{staffnotes}

This will lead to an alternative proof of \inhandout{the
  theorem}\inbook{Theorem~\bref{boyopt}} that when men are married to
their optimal spouses, women must be married to their pessimal
spouses.  This alternative proof does not depend on the Mating
Ritual\inbook{ of Section~\bref{mating_ritual_sec}}.

\bparts

\ppart\label{2l2w} The left to right direction of~(WL) is
equivalent to the assertion that married partners cannot both be
winners.  Explain why this follows directly from the definition of
rogue couple.

\begin{solution}
If both partners in a marriage were winners, then they would, by
definition, be a rogue couple for the other set marriages,
contradicting stability.
\end{solution}

\medskip
\eparts

The right to left direction of~(WL) is equivalent to the
assertion that a married couple cannot both be losers.  This will
follow by comparing the number of winners and losers among the
marriages.

\bparts

\ppart Explain why the number of winners must equal the number of
losers among the two sets of marriages.

\begin{solution}
If a person has the same spouse in both marriages, then they add
nothing to the numbers of winners or losers.  Otherwise, a person is a
loser in one marriage and a winner in the other.  So the overall
number of winners and losers contributed by different people must be
equal.
\end{solution}

\ppart Complete the proof of~(WL) by showing that if some
married couple were both losers, then there must be another couple who
were both winners.

\begin{solution}
Since the number of winners and losers is the same, if some marriage
contributed two losers---and hence no winners---some other marriage
would have to contribute two winners, contradicting part~\eqref{2l2w}.
\end{solution}

\ppart Conclude that in a stable set of marriages, someone's spouse is
optimal iff they are pessimal for their spouse.

\begin{solution}
A man is not married to his optimal wife in a stable set of marriages
iff he has a better wife in some second set of stable marriages.
By~(WL), this holds iff his first wife has a worse second husband iff
he is not his first wife's pessimal husband.
\end{solution}

\eparts
\end{problem}

\endinput
