\documentclass[problem]{mcs}

\begin{pcomments}
    \pcomment{TP_basic_set_formulas}
    \pcomment{ARM 2/17/13}
    \pcomment{AC produced FP version 5/17/15}
\end{pcomments}

\pkeywords{
  logic
  sets
  set_theory
  predicate
  formula
  subset
  power_set
  union
}

\begin{problem}
\inhandout{A \emph{formula of set theory} is a predicate formula that
  only uses the predicate ``$x \in y$.''  The domain of discourse is
  the collection of sets, and ``$x \in y$'' is interpreted to mean the
  set $x$ is one of the elements in the set $y$.

For example, since $x$ and $y$ are the same set iff they have the same
members, here's how we can express equality of $x$ and $y$ with a
formula of set theory:
\begin{equation}\label{x=xAz}
(x = y) \eqdef\ \forall z.\, (z \in x\ \QIFF\ z \in y).
\end{equation}
}
Express each of the following assertions about sets by a formula of
set theory\inbook{\footnote{See Section~\bref{ZFC_sec}.}},
where each answer may use abbreviations introduced
earlier (e.g., it is legal to use $=$ because we just defined it).

\bparts

\ppart $x = \emptyset$.

\examspace[.7in]

\begin{solution}
$\forall z.\, \QNOT(z\in x)$.
\end{solution}

\ppart $x \subseteq y$.  ($x$ is a subset of $y$ that might equal $y$.)

\examspace[.7in]

\begin{solution}
$\forall z.\, z \in x\ \QIMPLIES\ z \in y$.
\end{solution}

\ppart $x = y \cap z$.

\examspace[.7in]

\begin{solution}
$\forall w.\, w \in x\ \QIFF\ (w \in y\ \QAND\ w \in z)$.
\end{solution}

\ppart $|x| = 1$.  ($x$ is a set of size 1.)

\examspace[.7in]

\begin{solution}
$\exists v.\, \forall u.\, u \in x \QIFF u = v$.
\end{solution}

\eparts

\end{problem}

\endinput
