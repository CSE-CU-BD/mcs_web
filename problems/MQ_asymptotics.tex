\documentclass[problem]{mcs}

\begin{pcomments}
  \pcomment{MQ_asymptotics}
\end{pcomments}

\pkeywords{
	asymptotics
}

%%%%%%%%%%%%%%%%%%%%%%%%%%%%%%%%%%%%%%%%%%%%%%%%%%%%%%%%%%%%%%%%%%%%%
% Problem starts here
%%%%%%%%%%%%%%%%%%%%%%%%%%%%%%%%%%%%%%%%%%%%%%%%%%%%%%%%%%%%%%%%%%%%%

\begin{problem}
\inbook{Indicate}\inhandout{Circle} every symbol on the left that
could correctly appear in the box to its right.  (Thus, for each of
the five parts you may \inbook{indicate}\inhandout{circle} zero, one,
or both of the symbols.)

{\Large
\begin{align*}
\mathbf{(a)} && \Theta \hspace{0.5in} \omega && \hspace{0.5in}
  10^n & \mybox{n^{10}} \\[0.35in]
\mathbf{(b)} && o \hspace{0.5in} O && \hspace{0.5in}
  \frac{1}{\log n} & \mybox{1} \\[0.35in]
\mathbf{(c)} && \Omega \hspace{0.5in} o && \hspace{0.5in}
  \sqrt{n} & \mybox{(\log n)^5} &  \\[0.35in]
\mathbf{(d)} && O \hspace{0.5in} \Omega && \hspace{0.5in}
  n^2 & \mybox{3 n^2 - 5 n + 7} \\[0.35in]
\mathbf{(e)} && \Theta \hspace{0.5in} O && \hspace{0.5in}
  n! & \mybox{n^n}
\end{align*}
}

\begin{solution}
Underlined symbols should be \inbook{indicated}\inhandout{circled}.
%
{\Large
\[
\begin{array}{cc}
\Theta & \underline{\omega} \\
\underline{o} & \underline{O} \\
\underline{\Omega} & o \\
\underline{O} & \underline{\Omega} \\
\Theta & \underline{O}
\end{array}
\]
}
\end{solution}

\end{problem}

\endinput
