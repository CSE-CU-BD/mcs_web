%PS_triangle_free_planar_graphs
%formerly PS_simple_triangle_free_planar_graphs
%subsumes PS_planar_triangle_free_graph_coloring

\documentclass[problem]{mcs}

\begin{pcomments}
  \pcomment{from: S09.ps6}
\end{pcomments}

\pkeywords{
  planar_graphs
  handshaking
  graph_coloring
  strong_induction
  induction_on_graphs
  colorable
}

%%%%%%%%%%%%%%%%%%%%%%%%%%%%%%%%%%%%%%%%%%%%%%%%%%%%%%%%%%%%%%%%%%%%%
% Problem starts here
%%%%%%%%%%%%%%%%%%%%%%%%%%%%%%%%%%%%%%%%%%%%%%%%%%%%%%%%%%%%%%%%%%%%%

\begin{problem} A simple graph is \emph{triangle-free} if it 
  has no subgraphs isomorphic to $K_3$.

\bparts

\ppart\label{e2v4} While a triangle-free graph is not necessarily bipartite, 
  explain why the result $e \leq 2v - 4$ still holds for any connected 
  triangle-free planar graph with $v>2$ vertices and $e$ edges.

\begin{solution}
The proof that $e \leq 2v - 4$ for any connected triangle-free 
  planar graph $G$ with more than two vertices is identical to the proof 
  of the same inequality for bipartite graph planar graphs which appealed 
  to the fact that any bipartite graph is triangle-free.
\end{solution}

\ppart\label{triangle-free} Show that any connected triangle-free planar 
  graph has at least one vertex of degree three or less.

\begin{solution}
If $v \leq 4$, \emph{all} vertices have degree at most three,
  so the claim is immediate for $v\leq 4$.

  Also, by the Handshaking Lemma, the sum of degrees is $2e$ so the average
  degree is $2e/v$.  By part~\eqref{e2v4}, $2e/v \leq (4v-8)/v < 4$ for
  $v>2$.  But the average degree can be less than 4 only if at least one
  vertex has degree less than 4.

  It follows that for all $v > 0$, there is a vertex of degree three or
  less.
\end{solution}

\ppart Prove by induction on the number of vertices that any connected 
  triangle-free planar graph is 4-colorable.

  \hint use part \eqref{triangle-free}.

\begin{solution}
\mbox{}

  \begin{proof} By strong induction on the number of vertices with the
  induction hypothesis that if a graph is connected, planar and 
  triangle-free then it is 4-colorable.

  {\bf base case:} A planar graph with a single vertex is trivially 
  connected, triangle-free and 1-colorable.

  {\bf inductive step:} Any connected triangle-free planar graph
  $G$ with 2 or more vertices has a vertex of degree 3 or less. 
  Removing this vertex and any incident edges results in a graph 
  $H$ whose connected components are subgraphs of a planar graph 
  and therefore planar. They are also triangle-free since removing 
  vertices/edges from a graph with no triangles cannot create 
  triangles. Since the components have strictly fewer vertices than $G$, 
  the induction hypothesis implies each connected component is 
  4-colorable and thus $H$ is 4-colorable.

  A 4-coloring of $G$ is then given by a 4-coloring of $H$ where the 
  removed vertex is colored with a color not used for the (at most 3) 
  adjacent vertices. \end{proof}
\end{solution}

\eparts

\end{problem}

%%%%%%%%%%%%%%%%%%%%%%%%%%%%%%%%%%%%%%%%%%%%%%%%%%%%%%%%%%%%%%%%%%%%%
% Problem ends here
%%%%%%%%%%%%%%%%%%%%%%%%%%%%%%%%%%%%%%%%%%%%%%%%%%%%%%%%%%%%%%%%%%%%%
