\documentclass[problem]{mcs}

\begin{pcomments}
  \pcomment{CP_power_set-tower}
  \pcomment{9/23/09 by ARM, from logic notesproblem}
  \pcomment{2/14/11 revised by ARM to use \strict}
\end{pcomments}

\pkeywords{
 powerset
 cardinality
 strictly_bigger
 infinite
}

%%%%%%%%%%%%%%%%%%%%%%%%%%%%%%%%%%%%%%%%%%%%%%%%%%%%%%%%%%%%%%%%%%%%%
% Problem starts here
%%%%%%%%%%%%%%%%%%%%%%%%%%%%%%%%%%%%%%%%%%%%%%%%%%%%%%%%%%%%%%%%%%%%%

\begin{problem}
There are lots of different sizes of infinite sets.  For example, starting
with the infinite set, $\naturals$, of nonnegative integers, we can build
the infinite sequence of sets
\[
\naturals \strict \power(\naturals) \strict \power(\power(\naturals))
\strict \power(\power(\power(\naturals))) \strict \dots.
\]
where each set is ``strictly smaller'' than the next one by
Theorem~\bref{powbig}.  Let $\power^n(\naturals)$ be the $n$th set in
the sequence, and
  \[
  U \eqdef \lgunion_{n=0}^\infty \power^n(\naturals).
  \]
Prove that
\[
\power^n(\naturals) \strict U
\]
for all $n \in \naturals$.

Now of course, we could take $U, \power(U), \power(\power(U)), \dots$
and keep on in this way building still bigger infinities indefinitely.

\begin{solution}
We have to prove that
\begin{equation}\label{Usp} 
U \surj \power^n(\naturals),
\end{equation}
and
\begin{equation}\label{nopsU}
\QNOT(\power^n(\naturals) \surj U)
\end{equation}
for all $n \in \naturals$.

Everything follows from a trivial observation: if $A \supseteq B$, then $A
\surj B$.  (Why is this trivial?)

So since $U \supseteq \power^n(\naturals)$, we have immediately
have~\eqref{Usp}.

To prove~\eqref{nopsU}, assume to the contrary that $\power^m(\naturals)
\surj U$ for some $m$.  Now we know from~\eqref{Usp} that $U \surj
\power^{m+1}(\naturals)$.  But this implies that
\[
\power^m(\naturals)
\surj \power^{m+1}(\naturals) = \power(\power^m(\naturals)).
\]
That is, $A \surj \pow(A)$ where $A = \power^m(\naturals)$, which
contradicts Cantor's Theorem~\bref{powbig}.
\end{solution}

\end{problem}

%%%%%%%%%%%%%%%%%%%%%%%%%%%%%%%%%%%%%%%%%%%%%%%%%%%%%%%%%%%%%%%%%%%%%
% Problem ends here
%%%%%%%%%%%%%%%%%%%%%%%%%%%%%%%%%%%%%%%%%%%%%%%%%%%%%%%%%%%%%%%%%%%%%

\endinput
