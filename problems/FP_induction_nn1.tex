\documentclass[problem]{mcs}

\begin{pcomments}
\pcomment{FP_induction_nn1}
\pcomment{rewrite of FP_wop_nn1}
\pcomment{from F04rec2}
\pcomment{edited by ARM 9/20/15}
\end{pcomments}

\pkeywords{
  induction
  series
  sum
}

%%%%%%%%%%%%%%%%%%%%%%%%%%%%%%%%%%%%%%%%%%%%%%%%%%%%%%%%%%%%%%%%%%%%%
% Problem starts here
%%%%%%%%%%%%%%%%%%%%%%%%%%%%%%%%%%%%%%%%%%%%%%%%%%%%%%%%%%%%%%%%%%%%%

\begin{problem}
Prove by induction that
\begin{equation}\label{nn1n2}
1 \cdot 2 + 2 \cdot 3 + 3 \cdot 4 + \dots + n (n+1)
    = \frac{n (n+1) (n+2)}{3}
\end{equation}
for all integers, $n\geq 1$.

\begin{solution}

\begin{proof}
We use induction.  Let the induction hypothesis $P(n)$ be equation~\eqref{nn1n2}.

\inductioncase{Base case} ($n=1$): $P(1)$ is true, because the lefthand
side of~\eqref{nn1n2} is $1 \cdot 2 = 2$, and the righthand side is
$(1\cdot 2 \cdot 3)/ 3 = 2$.

\inductioncase{Inductive step}:  Assuming the induction hypothesis~\eqref{nn1n2} holds 
for some $n \geq 1$, we must prove $P(n+1)$.  We reason as follows:
\begin{align*}
\lefteqn{1 \cdot 2 + 2 \cdot 3 + \dots + (n+1) (n+2)}\\
    & = [1 \cdot 2 + 2 \cdot 3 + \dots + n(n+1)] + (n+1) (n+2)\\
    & = \frac{n(n+1)(n+2)}{3} + (n+1) (n+2) & \text{by ind.\ hypothesis~\eqref{nn1n2}}\\
    & = \frac{n(n+1)(n+2)}{3} + \frac{3(n+1)(n+2)}{3} & \text{algebra}\\
    & = \frac{([(n+1)(n+2)](n + 3)}{3} & \text{algebra}\\
    & = \frac{(n+1)(n+2)(n+3)}{3} & \text{algebra}
\end{align*}
Therefore
\[
1 \cdot 2 + 2 \cdot 3 + \dots + (n+1) (n+2) = \frac{(n+1)(n+2)(n+3)}{3}
\]
which is $P(n+)$.

By the induction principle, we conclude that $P(n)$ is true for all $n
\ge 1$, which proves the claim.\footnote{We spelled out all three
  algebraic simplification steps here.  In general, such routine
  algebra steps would, and should, be skipped.}
\end{proof}

A good question is how someone came up with equation~~\eqref{nn1n2} in the first
place.  The proof above gives no hint about this, but it should be absolutely
convincing anyway.

\begin{editingnotes}
ref gen func chapter.  add problem about this.
\end{editingnotes}

\end{solution}

\end{problem}

%%%%%%%%%%%%%%%%%%%%%%%%%%%%%%%%%%%%%%%%%%%%%%%%%%%%%%%%%%%%%%%%%%%%%
% Problem ends here
%%%%%%%%%%%%%%%%%%%%%%%%%%%%%%%%%%%%%%%%%%%%%%%%%%%%%%%%%%%%%%%%%%%%%

\endinput
