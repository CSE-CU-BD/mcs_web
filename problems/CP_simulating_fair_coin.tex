\documentclass[problem]{mcs}

\begin{pcomments}
  \pcomment{CP_simulating_fair_coin}
  \pcomment{essentially same as CP_coin_flips}
  \pcomment{from: S09.cp12r}
\end{pcomments}

\pkeywords{
  probability
  infinite_sample_space
  probability
  sample_space
  subtree
  tree_diagram
}

%%%%%%%%%%%%%%%%%%%%%%%%%%%%%%%%%%%%%%%%%%%%%%%%%%%%%%%%%%%%%%%%%%%%%
% Problem starts here
%%%%%%%%%%%%%%%%%%%%%%%%%%%%%%%%%%%%%%%%%%%%%%%%%%%%%%%%%%%%%%%%%%%%%

% F09, S09

\begin{problem}
Suppose you need a fair coin to decide which door to choose in the
6.042 Monty Hall game.  After making everyone in your group empty
their pockets, all you managed to turn up are some old collaboration
statements, a few used tissues, and one penny.  However, the penny was
from Prof. Meyer's pocket, so it is \textbf{not} safe to assume that
it is a fair coin.

How can we use a coin of unknown bias to get the same effect as a fair
coin of bias $1/2$?  Draw the tree diagram for your solution, but
since it is infinite, draw only enough to see a pattern.

Suggestion: A neat trick allows you to sum all the outcome
probabilities that cause you to say ``Heads'': Let $s$ be the sum of
all ``Heads'' outcome probabilities in the whole tree.  Notice that
\emph{you can write the sum of all the ``Heads'' outcome probabilities
  in certain subtrees as a function of $s$}.  Use this observation to
write an equation in $s$ and then solve.

\begin{solution}
Flip Prof. Meyer's coin twice; if you see HT output Heads, if you see
TH output Tails, and if you see HH or TT start over.

In the tree diagram in Figure~\ref{f-c}, the small triangles represent
subtrees that are themselves complete copies of the whole tree.

\begin{figure}[htbp]
\graphic{fair-coin}
\caption{Simulating a Fair Coin.}
\label{f-c}
\end{figure}

Let $s$ equal the sum of all ``Heads'' probabilities in the whole tree.
There are two extra edges with probability $p$ on the path to each
outcome in the top subtree.  Therefore, the sum of ``Heads''
probabilities in the upper tree is $p^2 s$.  Similarly, the sum of
``Heads'' probabilities in the lower subtree is $(1-p)^2 s$.  This gives
the equation:
\[
s = p^2 s + (1-p)^2 s + p (1-p)
\]

The solution to this equation is $s = 1/2$, for all $p$ between 0 and
1.
\end{solution}

\end{problem}


%%%%%%%%%%%%%%%%%%%%%%%%%%%%%%%%%%%%%%%%%%%%%%%%%%%%%%%%%%%%%%%%%%%%%
% Problem ends here
%%%%%%%%%%%%%%%%%%%%%%%%%%%%%%%%%%%%%%%%%%%%%%%%%%%%%%%%%%%%%%%%%%%%%

\endinput
