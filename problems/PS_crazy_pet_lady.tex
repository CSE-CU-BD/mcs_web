\documentclass[problem]{mcs}

\begin{pcomments}
  \pcomment{PS_crazy_pet_lady}
  \pcomment{from: S09.ps10, F11, S13.ps11}
\end{pcomments}

\pkeywords{
  generating_functions
  partial_fractions
}

%%%%%%%%%%%%%%%%%%%%%%%%%%%%%%%%%%%%%%%%%%%%%%%%%%%%%%%%%%%%%%%%%%%%%
% Problem starts here
%%%%%%%%%%%%%%%%%%%%%%%%%%%%%%%%%%%%%%%%%%%%%%%%%%%%%%%%%%%%%%%%%%%%%

\begin{problem}

Miss McGillicuddy never goes outside without a collection of pets.  In
particular:

\begin{itemize}
\item She brings a positive number of songbirds, which always come in
pairs.

\item She may or may not bring her alligator, Freddy.

\item She brings at least 2 cats.

\item She brings two or more chihuahuas and labradors leashed together
  in a line.
\end{itemize}

Let $P_n$ denote the number of different collections of $n$ pets that
can accompany her, where we regard chihuahuas and labradors leashed 
in different orders as different collections.%, even if there are the same
%number of chihuahuas and labradors leashed in the line.

For example, $P_6 = 4$ since there are 4 possible collections of 6 pets:
\begin{itemize}
\item 2 songbirds, 2 cats, 2 chihuahuas leashed in line
\item 2 songbirds, 2 cats, 2 labradors leashed in line
\item 2 songbirds, 2 cats, a labrador leashed behind a chihuahua
\item 2 songbirds, 2 cats, a chihuahua leashed behind a labrador
\end{itemize}
% And $P_7 = 16$ since there are 16 possible collections of 7 pets:
% \begin{itemize}
% \item 2 songbirds, 3 cats, 2 chihuahuas leashed in line
% \item 2 songbirds, 3 cats, 2 labradors leashed in line
% \item 2 songbirds, 3 cats, a labrador leashed behind a chihuahua
% \item 2 songbirds, 3 cats, a chihuahua leashed behind a labrador
% \item 4 collections consisting of 2 songbirds, 2 cats, 1 alligator, and a
% line of 2 dogs
% \item 8 collections consisting of 2 songbirds, 2 cats, and a line of 3 dogs.
% \end{itemize}

\bparts

\ppart Let
\[
P(x) \eqdef P_0 + P_1 x + P_2 x^2 + P_3 x^3 + \cdots
\]
be the generating function for the number of Miss
McGillicuddy's pet collections.  Verify that
\[
P(x) = \frac{4x^6}{(1-x)^2(1-2x)}.
\]

\begin{solution}
\begin{align*}
P(x) & = \underbrace{(x^2 + x^4 + x^6 + x^8 +
\cdots)}_{\text{collections of songbirds}}\\
     & \cdot \underbrace{(1 + x)}_{\text{collections of gators}}\\
     &\ \cdot \underbrace{(x^2 + x^3 + x^4 + \cdots)}_{\text{collections of cats}}\\
     &\ \cdot \underbrace{(2^2x^2 + 2^3x^3 + 2^4x^4 + \cdots)}_{\text{lines of dogs}}\\
     & = \frac{x^2}{1 - x^2}\cdot (1+x) \cdot \frac{x^2}{1 - x} \cdot
        \frac{4x^2}{1-2x} \\
     & = \frac{x^2}{(1 - x)(1+x)}\cdot (1+x) \cdot \frac{x^2}{1 - x} \cdot
        \frac{4x^2}{1-2x}\\
     & =  \frac{4x^6}{(1 - x)^2(1-2x)}.
\end{align*}
\end{solution}


\ppart Find a closed form expression for $P_n$.

\begin{solution}
$P_n$ is the coefficient of $x^n$ in the power series for $4x^6/(1 -
x)^2(1-2x)$, which means it is 4 times the coefficient of $x^{n-6}$ in the
series for $1/(1 -x)^2(1-2x)$ when $n \geq 6$, and $P_n = 0$ for $n<6$.

We can express $1/(1 -x)^2(1-2x)$ using partial fractions as
\begin{equation}\label{ABC/}
\frac{1}{(1 -x)^2(1-2x)} = \frac{A}{1-x} + \frac{B}{(1-x)^2} + \frac{C}{1-2x}
\end{equation}
for some constants $A,B,C$, so $P_n$ will be 4 times the sum of the
coefficients of $x^{n-6}$ in each of $A/(1-x)$, $B/(1-x)^2$, and
$C/(1-2x)$.  In other words, 
\begin{equation}
P_n = 4([x^{n-6}]\frac{A}{1-x}+[x^{n-6}]\frac{B}{(1-x)^2}+[x^{n-6}]\frac{C}{1-2x}).
\end{equation}
Thus,
\begin{equation}\label{PnABC}
P_n = 4(A+B\binom{n-5}{1} + C2^{n-6}) = 4A + 4B(n-5)+ C2^{n-4}. 
\end{equation}

So we need only find the values of $A,B,C$.  But multiplying both sides
of~\eqref{ABC/} by the left-hand denominator $(1 -x)^2(1-2x)$ yields
\begin{equation}\label{ABCx}
1 = A(1-x)(1-2x) + B(1-2x) + C(1-x)^2.
\end{equation}
Now letting $x=1$ in ~\eqref{ABCx} gives $B=-1$.  Similarly, letting
$x=1/2$ gives $C=4$.  Finally, letting $x=0$ gives $A+B+C=1$ and so
$A=-2$.  Substituting these values into~\eqref{PnABC} finally gives
\[
P_n = 4(-2) - 4(n-5) + 4( 2^{n-4}) = 2^{n-2} - 4n + 12.
\]
\end{solution}

\eparts

\end{problem}

%%%%%%%%%%%%%%%%%%%%%%%%%%%%%%%%%%%%%%%%%%%%%%%%%%%%%%%%%%%%%%%%%%%%%
% Problem ends here
%%%%%%%%%%%%%%%%%%%%%%%%%%%%%%%%%%%%%%%%%%%%%%%%%%%%%%%%%%%%%%%%%%%%%

\endinput
