\documentclass[problem]{mcs}

\begin{pcomments}
    \pcomment{CP_sat_formulas_vs_circuits}
    \pcomment{by ARM 2/17/13, revised 2/7/16}
\end{pcomments}

\pkeywords{
  satisfiable
  circuit
  digital_circuit
  proposition
  formula
}

\begin{problem}
The circuit-SAT problem is the problem of determining, for any given
digital circuit with one output wire, whether there are truth values
that can be fed into the circuit input wires which will lead the
circuit to give output \true.

It's easy to see that any efficient way of solving the circuit-SAT
problem would yield an efficient way to solve the usual SAT problem
for propositional formulas (Section~\bref{SAT_sec}).  Namely, for any
formula $F$, just construct a circuit $C_F$ using that computes the
values of the formula.  Then there are inputs for which $C_F$ gives
output true iff $F$ is satisfiable.  Constructing $C_F$ from $F$ is
easy, using a binary gate in $C_F$ for each propositional connective
in $F$.  So an efficient circuit-SAT procedure leads to an efficient
SAT procedure.

It's also true that any efficient way of solving the regular SAT
problem for formulas would yield an efficient way to solve the
circuit-SAT problem, but this requires a somewhat trickier argument.

The basic idea is that given any digital circuit $C$ with binary gates
and one output, we can assign a distinct variable to each wire of $C$.
Then for each gate of $C$, we can set up a propositional formula that
represents the constraints that the gate places on the values of its
input and output wires.  For example, for an \QAND\ gate with input
wire variables $P$ and $Q$ and output wire variable $R$, the
constraint proposition would be
\begin{equation}\label{AQBIC}
(P \QAND\ Q) \QIFF R.
\end{equation}

\bparts

\ppart Given a circuit $C$, explain how to construct a formula $F_C$
such that $F_C$ is satisfiable iff $C$ gives output \true\ for some
set of input values.

\begin{solution}
The product, $P$, of all the gate constraints will have the property
that, for any assignment of truth values to the input variables of the
circuit, $P$ will be \true\ iff each wire variable has the value that
the circuit would compute for that wire.

Suppose $Q$ is the variable corresponding to the output wire of the
circuit.  Let $F_C$ be
\[
F_C \eqdef P \QAND Q.
\]
Now for any truth assignment for the input wire variables, $F_C$ will
be \true\ iff $C$ would output \true\ for those inputs, and moreover
each wire variable in $F_C$ is assigned the value that $C$ would
compute for that wire.

That is, there are inputs for which $C$ gives output \true\ iff $F_C$
is satisfiable.
\end{solution}

\ppart Conclude that any efficient way of solving SAT would yield an
efficient way to solve circuit-SAT.

\begin{solution}
The formula $F_C$ can be read off a diagram for the circuit $C$, and
the size of $F_C$ will be proportional to the number of wires in $C$.
So an efficient SAT procedure leads to an efficient circuit-SAT
procedure: given $C$ construct $F_C$ and use the efficient SAT
procedure on $F_C$
\end{solution}
\eparts
\end{problem}

\endinput
