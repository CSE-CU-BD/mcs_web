\documentclass[problem]{mcs}

\begin{pcomments}
  \pcomment{CP_logical_set_theory}
  \pcomment{initial parts commented out in favor of TP_basic_set_formulas}
  \pcomment{from: F01.ps1, F11, ps2}
  \pcomment{edited by Eli & ARM, then ARM 3/17/13}
\end{pcomments}

\pkeywords{
  logic
  predicate
  set_theory
  subset
  power_set
}

%%%%%%%%%%%%%%%%%%%%%%%%%%%%%%%%%%%%%%%%%%%%%%%%%%%%%%%%%%%%%%%%%%%%%
% Problem starts here
%%%%%%%%%%%%%%%%%%%%%%%%%%%%%%%%%%%%%%%%%%%%%%%%%%%%%%%%%%%%%%%%%%%%%

\begin{problem}

\inhandout{
A \emph{formula of \idx{set theory}}\footnote{Technically this is
  called a \term{first-order predicate formula} of set theory} is a
predicate formula that only uses the predicate ``$x \in y$.''  The
domain of discourse is the collection of sets, and ``$x \in y$'' is
interpreted to mean that $x$ and $y$ are variables that range over
sets, and $x$ is one of the elements in $y$.

For example, since $x$ and $y$ are the same set iff they have the same
members, here's how we can express equality of $x$ and $y$ with a
formula of set theory:
\begin{equation}\label{x=xAz}
(x = y) \eqdef\ \forall z.\, (z \in x\ \QIFF\ z \in y).
\end{equation}
}

\bparts

\ppart\label{threey} Write a formula, $\text{Three}(y)$, of set theory
\inbook{\footnote{See Section~\bref{ZFC_sec}.}} that means that $y$ has at least
three elements.

\begin{solution}
\[
\exists u, v, w.\; (u \in y) \QAND (v \in y)
\QAND (w \in y) \QAND (u \neq v) \QAND (u \neq w) \QAND (v \neq w).
\]
\end{solution}

\problempart Write a formula of set theory that
means that $y$ has \emph{exactly} two elements.

\begin{solution}
As in part~\eqref{threey}, it is even easier to write a formula,
$\text{Two}(y)$, that means that $y$ has at least two elements.  The
desired formula is just the conjunction of $\text{Two}(y)$ and $\QNOT
(\text{Three}(y))$.
\end{solution}

\iffalse

\problempart\label{memminxy} A set, $x$, is \emph{member-minimal} in a set $y$ iff $x$ is
a member of $y$ and no element of $y$ is a member of $x$.  Write a formula,
$\text{MM}(x,y)$, of set theory that means that $x$ is member-minimal in
$y$.

\begin{solution}
\[
\text{MM}(x,y) \eqdef [x \in y \QAND \QNOT(\exists z.\, z \in y \QAND z \in x)]
\]
\end{solution}

\ppart The \term{Foundation Axiom} of \idx{Zermelo-Fraenkel} set theory
asserts that every nonempty set has a member-minimal element.  Express the
Foundation Axiom as a formula of set theory.  (You may use the formula
$\text{MM}(x,y)$ of part~\eqref{memminxy} as an abbreviation in your
formula).

\begin{solution}
\[
\forall y.\, (\exists z \in y) \QIMPLIES (\exists x.\, \text{MM}(x,y))
\]
\end{solution}
\fi

\end{problemparts}

It's now easy to see how to write a formula, $D_n(x_1,\dots,x_n)$ of
set theory that means that $x_1,dots,x_n$ are distinct elements, and
then using $D_n$ to write a formula that means that $y$ has $n$
elements.  A problem is that the obvious way to write $D_n$ uses
subformulas ``$x_i \neq x_j$'' for $1 \leq i < j \leq n$, and there
are $n(n-1)/2$ such subformuls.

Here's a tricky, optional puzzle: find a way to express
$D_n(x_1,\dots,x_n)$ with a formula of size proportional to $n$
(instead of $n^2$).

\hint Find a propositional formula of size propositional to $n$ which
has $n$ designated propositional variables and is satisfiable iff one
of the designated variables gets assigned to be true.

\begin{solution}
Write a digital circuit with inputs $x_1,\dots,x_n$ which outputs 1
iff exactly one of the $x_i$'s equals 1.  This is easy to do with a
linear-size circuit resembling a ripple-carry adder.  

Then describe the structure of the circuit Translate this into a
linear-size propositional

\end{solution}

\end{problem}

%%%%%%%%%%%%%%%%%%%%%%%%%%%%%%%%%%%%%%%%%%%%%%%%%%%%%%%%%%%%%%%%%%%%%
% Problem ends here
%%%%%%%%%%%%%%%%%%%%%%%%%%%%%%%%%%%%%%%%%%%%%%%%%%%%%%%%%%%%%%%%%%%%%


\iffalse

Old initial parts:

\ppart Write a formula of set theory that means that $x$ is the
empty set.

\begin{solution}
\[
\forall y\; \QNOT (y \in x).
\]
\end{solution}

\ppart Write a formula of set theory that means that $x \subseteq y$, that
is, $x$ is a subset of $y$.

\begin{solution}

\begin{equation}\label{xsubeqy}
\forall z\; (z \in x) \QIMPLIES (z \in y)
\end{equation}

\end{solution}


\ppart Write a formula of set theory that means that $x \subset y$, that
is, $x$ is a \emph{proper} subset of $y$.

\begin{solution}
Take the conjunction ($\QAND$) of formulas~\eqref{xsubeqy}
  and~\eqref{setneq}.
\end{solution}

\ppart Write a formula of set theory that means that $y$ is the powerset
of $x$.

\begin{solution}
Let $P(x,y)$ be the formula~\eqref{xsubeqy} meaning $x \subseteq
  y$.  Then $y =\power(x)$ iff
\[
\forall z\; P(z,x) \QIMPLIES (z \in y).
\]

\end{solution}
\fi

\endinput
