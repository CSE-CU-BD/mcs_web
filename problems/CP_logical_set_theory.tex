\documentclass[problem]{mcs}

\begin{pcomments}
  \pcomment{PS_logical_set_theory}
  \pcomment{from: F01.ps1, edited by Eli & ARM}
\end{pcomments}

\pkeywords{
  logic
  predicate
  set_theory
  subset
  power_set
}

%%%%%%%%%%%%%%%%%%%%%%%%%%%%%%%%%%%%%%%%%%%%%%%%%%%%%%%%%%%%%%%%%%%%%
% Problem starts here
%%%%%%%%%%%%%%%%%%%%%%%%%%%%%%%%%%%%%%%%%%%%%%%%%%%%%%%%%%%%%%%%%%%%%

\begin{problem}

\bparts

The standard notation for the proposition that a set, $x$, is a member of
a set, $y$, is
\[
x \in y.
\]
Formulas built up from membership formulas of this form using logical
connectives and quantifiers are called the logical \term{formulas of set
  theory}.  For example, to say $x \neq y$ with a formula of set theory, we
could write
\begin{equation}\label{setneq}
\exists z\; (z \in x) \QXOR (z \in y).
\end{equation}

\ppart Write a formula of set theory that means that $x$ is the
empty set.

\begin{solution}
\[
\forall y\; \QNOT (y \in x).
\]
\end{solution}

\ppart Write a formula of set theory that means that $x \subseteq y$, that
is, $x$ is a subset of $y$.

\begin{solution}

\begin{equation}\label{xsubeqy}
\forall z\; (z \in x) \QIMPLIES (z \in y)
\end{equation}

\end{solution}


\ppart Write a formula of set theory that means that $x \subset y$, that
is, $x$ is a \emph{proper} subset of $y$.

\begin{solution}
Take the conjunction (\QAND) of formulas~\ref{xsubeqy}
  and~\ref{setneq}.
\end{solution}

\ppart Write a formula of set theory that means that $y$ is the powerset
of $x$.

\begin{solution}
Let $P(x,y)$ be the formula~\ref{xsubeqy} meaning $x \subseteq
  y$.  Then $y =\power(x)$ iff
\[
\forall z\; P(z,x) \IFF (z \in y).
\]

\end{solution}

\ppart Write a formula, $\text{Three}(y)$, of set theory that means
that $y$ has at least three elements.

\begin{solution}

\[
\exists u \exists v \exists w\; (u \in y) \QAND (v \in y)
\QAND (w \in y) \QAND (u \neq v) \QAND (u \neq w) \QAND (v \neq w).
\]

\end{solution}

\problempart Write a formula of set theory that
means that $y$ has \emph{exactly} two elements.

\begin{solution}
It is easy to write a formula, $\text{Two}(y)$, that means
  that $y$ has at least two elements.  The desired formula is just the
  conjunction of $\text{Two}(y)$ and $\QNOT \text{Three}(y)$.
\end{solution}

\end{problemparts}

\end{problem}

%%%%%%%%%%%%%%%%%%%%%%%%%%%%%%%%%%%%%%%%%%%%%%%%%%%%%%%%%%%%%%%%%%%%%
% Problem ends here
%%%%%%%%%%%%%%%%%%%%%%%%%%%%%%%%%%%%%%%%%%%%%%%%%%%%%%%%%%%%%%%%%%%%%

\endinput
