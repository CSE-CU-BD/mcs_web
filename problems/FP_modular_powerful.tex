\documentclass[problem]{mcs}

\begin{pcomments}
  \pcomment{FP_modular_powerful}
  \pcomment{verbatim from: F07 Final P7 with minor syntax edits}
\end{pcomments}

\pkeywords{
  euler
  modular_arithmetic
  relatively_prime
  exponential
}

%%%%%%%%%%%%%%%%%%%%%%%%%%%%%%%%%%%%%%%%%%%%%%%%%%%%%%%%%%%%%%%%%%%%%
% Problem starts here
%%%%%%%%%%%%%%%%%%%%%%%%%%%%%%%%%%%%%%%%%%%%%%%%%%%%%%%%%%%%%%%%%%%%%

\begin{problem}

\bparts

\ppart
Prove that $(-14)^{721}$ has a multiplicative inverse modulo 135.

\examspace[2in]

\begin{solution}
  A number has a multiplicative inverse modulo 135 iff it is relatively
  prime to 135.  But the only prime divisors of 135 are 3 and 5, and the
  only prime divisors of $(-14)^{721}$ are the prime divisors of 14, namely
  2 and 7, so $(-14)^{721}$ is relatively prime to 135.
\end{solution}

\ppart
What is the value of $\phi(135)$, where $\phi$ is Euler's function?

\examspace[2in]

\begin{solution}
  Noting that $135 = 3^3 \cdot 5$.  It follows
  that $\phi(135) = (3^3 - 3^2)(5-1) = 18 \cdot 4 = 72$.
\end{solution}

\ppart
What is the remainder of $(-14)^{721}$ divided by 135?

\begin{solution}
  121

  Since -14 and 135 are relatively prime, we have by Euler's
  Theorem that $(-14)^{72} \equiv 1 \pmod {135}$, and so
  \[
  (-14)^{721} = \paren{(-14)^{72}}^{10} \cdot (-14)
              \equiv 1^{10} \cdot (-14)
              \equiv 135-14 \equiv 121 \pmod {135}.
  \]
\end{solution}

\ppart
  Call a number from 0 to 134 \emph{powerful} iff some power
  of the number is congruent to 1 modulo 135.  What is the probability
  that a random number from 0 to 134 is powerful?

\begin{solution}
  $8/15$.

  Note that $x^k \equiv 1 \pmod n$ for some $k$ iff $x$ has an inverse
  modulo $n$ iff $x$ is relatively prime to $n$.  So being powerful is
  equivalent to being relatively prime to 135.  There are
  $\phi(135) = 72$ numbers from 0 to 134 that are relatively prime to
  135, so
  \[
  \pr{\text{powerful}} = \frac{72}{135} = \frac{8}{15}.
  \]
\end{solution}

\eparts
\end{problem}

%%%%%%%%%%%%%%%%%%%%%%%%%%%%%%%%%%%%%%%%%%%%%%%%%%%%%%%%%%%%%%%%%%%%%
% Problem ends here
%%%%%%%%%%%%%%%%%%%%%%%%%%%%%%%%%%%%%%%%%%%%%%%%%%%%%%%%%%%%%%%%%%%%%

\endinput
