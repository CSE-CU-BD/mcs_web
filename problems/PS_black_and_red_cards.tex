\documentclass[problem]{mcs}

\begin{pcomments}
  \pcomment{PS_black_and_red_cards}
  \pcomment{subsumed by PS_black_and_red_cards_revised}
  \pcomment{from: F07.ps10}
\end{pcomments}

\pkeywords{
  probability
  playing_cards
  total_probability
  strategy
}

%%%%%%%%%%%%%%%%%%%%%%%%%%%%%%%%%%%%%%%%%%%%%%%%%%%%%%%%%%%%%%%%%%%%%
% Problem starts here
%%%%%%%%%%%%%%%%%%%%%%%%%%%%%%%%%%%%%%%%%%%%%%%%%%%%%%%%%%%%%%%%%%%%%

\begin{problem}

  I have a deck of $52$ regular playing cards, $26$ red, $26$ black,
  randomly shuffled.  They all lie face down in the deck so that you
  can't see them.  I will draw a card off the top of the deck and turn
  it face up so that you can see it and then put it aside. I will
  continue to turn up cards like this but at some point while there
  are still cards left in the deck, you have to declare that you want
  the next card in the deck to be turned up.  If that next card turns
  up black you win and otherwise you lose.  Either way, the game is
  then over.

\bparts

\ppart Show that if you take the first card before you have seen any
cards, you then have probability $1/2$ of winning the game.

\begin{solution}
If we just record the sequence of black and red cards that
  will be drawn, there are $\binom{51}{25}$ sequences with first card
  black: $25$ positions for the black cards chosen from the $51$
  remaining positions.  Since there are $\binom{52}{26}$ sequences in
  all, the probability of winning on the first draw is
  $\binom{51}{25}/\binom{52}{26} = 26/52 = 1/2$.
\end{solution}

\ppart Suppose you don't take the first card and it turns up red.
Show that you then have a probability of winning the game that is
greater than $1/2$.

\begin{solution}
Suppose you take the next card after that.  There are
  $\binom{50}{25}$ sequences that start with a red card and then a
  black and there are $\binom{51}{26}$ sequences that start with a red
  card. So then there is a $\binom{50}{25}/\binom{50}{26} = 26/51 >
  1/2$ chance of winning.  Any optimum strategy would have to
  guarantee a probability of winning as least as big as that.
\end{solution}

\ppart If there are $r$ red cards left in the deck and $b$ black
cards, show that the probability of winning if you take the next card
is $b/(r+b)$.

\begin{solution}
The probability is $\binom{b+r-1}{b-1}/\binom{b+r}{b} =
b/(r+b)$.
\end{solution}

\ppart Either,
\begin{enumerate}
\item come up with a strategy for this game that gives you a
  probability of winning strictly greater than $1/2$ and prove that
  the strategy works, or,
\item come up with a proof that no such strategy can exist.
\end{enumerate}

\begin{solution}
There is no such strategy.  Let $S_{b,r}$ be a strategy that achieves
the best probability of winning when starting with $b$ black cards and
$r$ red cards.  The claim is that $\pr{\mbox{win by playing
    $S_{b,r}$}} = b/(r+b)$ for all $b,r$ with at least $b>0$ or $r>0$.

Clearly $\pr{\mbox{win by playing $S_{1,0}$}} = 1$ and $\pr{\mbox{win
    by playing $S_{0,1}$}} = 0$.  We prove the rest of the claim by
induction on $r+b$.  If the strategy $S_{b,r}$ is to take the next
card, then $\pr{\mbox{win by playing $S_{b,r}$}} = b/(r+b)$ as
claimed.  Suppose then that the strategy $S_{r,b}$ is to not take the
first card, but to keep playing.  Then by the law of \idx{total
probability},
\begin{align*}
\pr{\mbox{win by playing $S_{b,r}$}}
   & = \prcond{\mbox{win by playing $S_{b,r}$}}{\mbox{first card is black}} \pr{\mbox{first
      card is black}} +\\
   &\qquad \prcond{\mbox{win by playing
    $S_{b,r}$}}{\mbox{first card is red}} \pr{\mbox{first card is
    red}}\\
   & = \pr{\mbox{win by playing $S_{b-1,r}$}} (b/(r+b)) + \pr{\mbox{win
    by playing $S_{b,r-1}$}}(r/(b+r)).
\end{align*}
By induction, this is
\[
\pr{\mbox{win by playing $S_{b,r}$}} = ((b-1)/(b-1+r))(b/(r+b)) +
  (b/(b+r-1))(r/(b+r)) = b/(b+r),
\]
as claimed

Why is
\[
\prcond{\mbox{win by playing $S_{b,r}$}}{\mbox{first card is
    black}} = \pr{\mbox{win by playing $S_{b-1,r}$}}?
\]
\dots because if you have decided to see at least one more card, and
that card is black, this means you are starting the game over again
with $S_{b-1,r}$.
\end{solution}

\eparts

\end{problem}

%%%%%%%%%%%%%%%%%%%%%%%%%%%%%%%%%%%%%%%%%%%%%%%%%%%%%%%%%%%%%%%%%%%%%
% Problem ends here
%%%%%%%%%%%%%%%%%%%%%%%%%%%%%%%%%%%%%%%%%%%%%%%%%%%%%%%%%%%%%%%%%%%%%

\endinput
