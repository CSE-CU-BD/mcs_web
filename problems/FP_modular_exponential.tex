\documentclass[problem]{mcs}

\begin{pcomments}
  \pcomment{FP_modular_exponential}
  \pcomment{from: F05.Q2, prob 2}
  \pcomment{adapted by Steven F09}
\end{pcomments}

\pkeywords{
  euler
  modular_arithmetic
  relatively_prime
  exponential
}

%%%%%%%%%%%%%%%%%%%%%%%%%%%%%%%%%%%%%%%%%%%%%%%%%%%%%%%%%%%%%%%%%%%%%
% Problem starts here
%%%%%%%%%%%%%%%%%%%%%%%%%%%%%%%%%%%%%%%%%%%%%%%%%%%%%%%%%%%%%%%%%%%%%

\begin{problem}\mbox{}

\bparts 

\iffalse let's not test recipes

\ppart
Using the Pulverizer, find an $x$ in the range $\Zintvoc{0}{99}$ such 
that $x$ is an inverse of 19 modulo 100.

\examspace[3.5in]
\begin{solution}
There are two valid approaches.
\textbf{Solution 1:} Use the Pulverizer to find integers $s,t$ such that
$s19+t100 =1$.  Then $s \rem 100$ will be $19^{-1}$.
In this case the Pulverizer yields $-21\cdot 19 + 4 \cdot 100 =1$, so
$19^{-1} \equiv -21 \equiv 79 \pmod {100}$.
\textbf{Solution 2:} Find $k =\phi(100)$, so by Euler's Theorem, $19^{-1}
\equiv 19^{k-1} \pmod{100}$.  Then $19^{k-1} \rem 100$ will be $19^{-1}$.
This can be computed by at most $\floor{\log 100} = 6$ squarings modulo
100.
\end{solution}
\fi

\ppart What is the probability that an integer from 1 to 360 selected
with uniform probability is relatively prime to 360?

\exambox{0.5in}{0.4in}{0in}
\examspace[2in]

\iffalse
What is the value of $\phi(360)$, where $\phi$ is Euler's
function? \hfill\examrule{4em}
\fi

\begin{solution}
\[
\frac{4}{15}.
\]

Since $360=2^3\cdot3^2\cdot5$, it follows that
$\phi(360)=4\cdot6\cdot4=96$.
So the probability is 
\[
\frac{\phi(360)}{360}= \frac{96}{360} = \frac{4}{15}.
\]
\end{solution}

\ppart What is the value of $\rem{7^{98}}{360}$?

\exambox{0.5in}{0.4in}{0in}
\examspace[2in]

\begin{solution}
49.

Since 7 and 360 are relatively prime, we have by Euler's Theorem
that $7^{96} \equiv 1 \pmod {360}$, and so
\[
7^{98} = 7^{96}\cdot 7^2 \equiv 1 \cdot 49 \equiv 49 \pmod {360}.
\]
\end{solution}

\eparts
\end{problem}


%%%%%%%%%%%%%%%%%%%%%%%%%%%%%%%%%%%%%%%%%%%%%%%%%%%%%%%%%%%%%%%%%%%%%
% Problem ends here
%%%%%%%%%%%%%%%%%%%%%%%%%%%%%%%%%%%%%%%%%%%%%%%%%%%%%%%%%%%%%%%%%%%%%

\endinput
