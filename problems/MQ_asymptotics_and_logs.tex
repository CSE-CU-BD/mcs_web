\documentclass[problem]{mcs}

\begin{pcomments}
  \pcomment{MQ_asymptotics_and_logs}
  \pcomment{part of PS_Stirlings_and_log_n_factorial}
  \pcomment{ARM 4/7/15}
\end{pcomments}

\pkeywords{
  asymptotic
  logarithm
  Theta
}

%%%%%%%%%%%%%%%%%%%%%%%%%%%%%%%%%%%%%%%%%%%%%%%%%%%%%%%%%%%%%%%%%%%%%
% Problem starts here
%%%%%%%%%%%%%%%%%%%%%%%%%%%%%%%%%%%%%%%%%%%%%%%%%%%%%%%%%%%%%%%%%%%%%

\begin{problem}
Let $f,g$ be real-valued functions such that $f = \Theta(g)$ and
$\lim_{x \to \infty} f(x) = \infty$.  Prove that
\[
\ln f \sim \ln g.
\]

\begin{solution}
$f = \Theta(g)$ implies that there are constants $a,b$ such that
\[
af(x) \leq g(x) \leq bf(x)
\]
for all large $x \in \reals$.  Taking logs of the expressions above, we get
\[
\ln a + \ln f(x) \leq \ln g(x) \leq \ln b + \ln f(x),
\]
and dividing by $\ln f(x)$, we get 
\[
\frac{\ln a}{\ln f} + 1 \leq \frac{\ln g}{\ln f} \leq \frac{\ln b}{\ln f} + 1.
\]
Since $\lim_{x \to \infty} f(x) = \infty$, the limits of the left and
right-hand sides of the above inequality are both equal to 1, and
therefore $\lim_{x \to \infty} \ln g/\ln f = 1$.

\end{solution}

\end{problem}

%%%%%%%%%%%%%%%%%%%%%%%%%%%%%%%%%%%%%%%%%%%%%%%%%%%%%%%%%%%%%%%%%%%%%
% Problem ends here
%%%%%%%%%%%%%%%%%%%%%%%%%%%%%%%%%%%%%%%%%%%%%%%%%%%%%%%%%%%%%%%%%%%%%

\endinput
