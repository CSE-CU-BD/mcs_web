\documentclass[problem]{mcs}

\begin{pcomments}
  \pcomment{FP_red_black_tree_induction}
  \pcomment{F15.final-conflict2}
  \pcomment{same as PS_ but with hint}
  \pcomment{12/20/15 by ARM suggested by Zoran}
\end{pcomments}

\pkeywords{
  structural_induction
  tree
  red_black
}

%%%%%%%%%%%%%%%%%%%%%%%%%%%%%%%%%%%%%%%%%%%%%%%%%%%%%%%%%%%%%%%%%%%%%
% Problem starts here
%%%%%%%%%%%%%%%%%%%%%%%%%%%%%%%%%%%%%%%%%%%%%%%%%%%%%%%%%%%%%%%%%%%%%

\newcommand{\redl}{\textbf{red}}
\newcommand{\blackl}{\textbf{black}}
\newcommand{\RBT}{\text{RBT}}

\begin{problem} \mbox{}
The set \RBT\ of \emph{Red-Black Trees} is defined recursively as follows:

\inductioncase{Base cases}:
\begin{itemize}
\item $\ang{\redl} \in \RBT$, and
\item $\ang{\blackl} \in \RBT$.
\end{itemize}

\inductioncase{Constructor cases}:
$A, B$ are \RBT's, then
\begin{itemize}
\item if $A, B$ start with \blackl, then $\ang{\redl, A, B}$ is an \RBT.
\item if $A, B$ start with \redl, then $\ang{\blackl, A, B}$ is an \RBT.
\end{itemize}

For any \RBT\ $T$, let
\begin{itemize}
\item $r_T$ be the number of \redl\ labels in $T$,
\item $b_T$ be the number of \blackl\ labels in $T$, and
\item $n_T \eqdef r_T + b_T$ be the total number of labels in $T$.
\end{itemize}

Prove that
\begin{equation}\label{nT3red}
\text{If}\ T\ \text{starts with a}\ \redl\ \text{label},\ \text{then}\
\frac{n_T}{3} \leq r_T \leq \frac{2n_T +1}{3},
\end{equation}

\medskip

\hint \[n/3 \leq r\quad \QIFF\quad (2/3)n \geq n-r\]

\begin{solution}
\begin{proof}
The proof will be by structural induction on the definition of \RBT's.
The induction hypothesis will be~\eqref{nT3red} along with the
symmetric statement for \blackl:
\begin{equation}\label{nT3black}
\text{If}\ T\ \text{starts with a}\ \blackl\ \text{label},\ \text{then}\
\frac{n_T}{3} \leq b_T \leq \frac{2n_T +1}{3}.
\end{equation}

\inductioncase{Base case(s)}:
\begin{itemize}

\item ($T = \ang{\redl}$).  Then $r=1, b=0, n=1$ and
\[
\frac{1}{3} \leq 1 = \frac{2\cdot 0 + 1}{3},
\]
so~\eqref{nT3red} holds, as required.

\item ($T = \ang{\blackl}$).  Now $r=0, b=1, n=1$, so~\eqref{nT3black}
  holds similarly.
\end{itemize}

\inductioncase{Constructor case(s)}:
\begin{itemize}

\item ($T = \ang{\redl, A, B}$).  The constructor rules imply that
  $A,B$ are \RBT's that start with \blackl.  Now we have by induction
  hypothesis~\eqref{nT3black} that
\[\begin{array}{rcccl}
\dfrac{n_A}{3} & \leq & b_A & \leq &\dfrac{2n_A + 1}{3}\\
\dfrac{n_B}{3} & \leq & b_B & \leq &\dfrac{2n_B + 1}{3}.
\end{array}\]

Adding these, we get
\begin{align*}
\frac{n_A+n_b}{3}
   & =     \frac{n_A}{3} + \frac{n_B}{3}\\
   & \leq  b_A           + b_B\notag\\
   & \leq  \frac{2n_A + 1}{3}+\frac{2n_b + 1}{3}\\
   &  = \frac{2(n_A + n_b + 1)}{3}.
\end{align*}
But
\begin{align*}
n_T - 1 & = n_A+n_b,\\
b_T & = b_A + b_B,\\
n_T & = n_A + n_b + 1,
\end{align*}
and so,
\begin{equation}\label{nT13leq}
\frac{n_T - 1}{3}  \leq   b_T  \leq   \frac{2n_T}{3}.
\end{equation}
Now, as suggested by the hint, subtracting each of the three terms
in~\eqref{nT13leq} from $n_T$ reverses the inequalities and yields
\[
n_T -\frac{n_T - 1}{3}  \geq  n_T - b_T  \geq   n_T -\frac{2n_T}{3}.
\]
But $n_T - b_T = r_T$, and simplifying the fractions, we conclude that
\[
\frac{2n_T + 1}{3}      \geq    r_T     \geq   \frac{n_T}{3}
\]
which proves~\eqref{nT3red}

\item ($T = \ang{\blackl, A, B}$).  Now~\eqref{nT3black} follows by a
  symmetric argument.

\end{itemize}

We conclude by structural induction that both~\eqref{nT3red}
and~\eqref{nT3black} hold for all $T \in \RBT$.

\end{proof}

An alternative approach would be to just use~\eqref{nT3red} as the
induction hypothesis and consider constructor cases
\[
T = \begin{cases}
    \ang{\redl, \ang{\blackl, A_1,A_2}, \ang{\blackl, B_1,B_2}},\\
    \ang{\redl, \ang{\blackl, A_1,A_2}, \ang{\blackl}},\\
    \ang{\redl, \ang{\blackl}, \ang{\blackl, B_1,B_2}},\\
    \ang{\redl, \ang{\blackl}, \ang{\blackl}}.\\
    \end{cases}
\]
\end{solution}

\end{problem}

\endinput
