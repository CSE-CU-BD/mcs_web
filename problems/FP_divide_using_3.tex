\documentclass[problem]{mcs}

\begin{pcomments}
  \pcomment{FP_divide_using_3}
  \pcomment{excerpted from PS_divide_using_3}
\end{pcomments}

\pkeywords{
  state_machines
  termination
  partial_correctness
  invariant
  division
  algorithm
}

%%%%%%%%%%%%%%%%%%%%%%%%%%%%%%%%%%%%%%%%%%%%%%%%%%%%%%%%%%%%%%%%%%%%%
% Problem starts here
%%%%%%%%%%%%%%%%%%%%%%%%%%%%%%%%%%%%%%%%%%%%%%%%%%%%%%%%%%%%%%%%%%%%%

\begin{problem}
We describe a state machine for multiplying a real number, $x$, by a
nonnegative integer, $y$, using only addition, and multiplication and
division by 3.  The states are triples of real numbers $(r,s,a)$.  The
initial state is $(x,y,0)$.  The transitions are given by the rule
that for $s \in \integers^+$:
\[
(r,s,a)\rightarrow\begin{cases}
        (3r,s/3,a) &\text{ if } 3 \divides s\\
        (3r,(s-1)/3,a+r) &\text{ if } 3 \divides (s-1)\\
        (3r,(s-2)/3,a+2r) &\text{otherwise}.
       \end{cases}
\]

\bparts

\ppart Prove that the property that $s \in \integers^+$ is a preserved
invariant of this state machine.

\examspace[2in]

\ppart Prove that the machine terminates after at most $1+ \log_3 y$
transitions.

\examspace[2in]

\begin{solution}
Also, each transition corresponds to an execution of the \texttt{do}
statement body, and each transition reduces $s$ to at most $s/3$.
Hence, after at most $1+ \log_3 y$ executions of the body, the value
of $s$ is at most its initial value, $y$, times $\paren{1/3}^{1+
  \log_3 y} = 1/3y$.  That is, the value of $s$ is at most 1/3.  Since
$s \in \naturals$, it follows that $s$ will be 0 after this many
executions of the body, if it wasn't 0 earlier.  But with $s=0$, the
procedure terminates.
\end{solution}

\ppart Use the Invariant Method to prove that the algorithm is partially
correct ---that is, if $s = 0$, then $a = xy$.

\examspace[4in]

\begin{solution}
Let
\[
P((r,s,a)) \eqdef\quad [rs+a = xy].
\]

Clearly, $P$ holds for the start state because
\[
P((x,y,0)) \qiff [xy+0 = xy].
\]

Now, we show that $P$ is indeed a preserved invariant, namely, assuming
$P((r,s,a))$,
\begin{equation}\label{inv}
rs+a = xy,
\end{equation}
holds and $(r,s,a) \to (r',s',a')$ is a transition, then $P((a',b',p'))$,
\begin{equation}\label{inv'}
r's'+a' = xy,
\end{equation}
holds.

We consider three cases:

If $3 \divides s$, then we have that $r' = 3r, s' = s/3, a'=a$.
Therefore,
\begin{align*}
  r's' + a' & = 3r \cdot \frac{s}{3} + a\\
            & = rs+a\\
            & = xy & \text{(by~\eqref{inv})}.
\end{align*}

If $3 \divides s-1$, then $r' = 3r, s' = (s-1)/3,a = a+r$.  So:
\begin{align*}
  r's' + a'  & = 3r \cdot \frac{s-1}{3} + a+r\\
   & = r\cdot(s-1) + a + r\\
   & = rs+a\\
   & = xy & \text{(by~\eqref{inv})}.
\end{align*}

Otherwise, we have $r' = 3r, s' = (s-2)/3,a = a+2r$.  So:
\begin{align*}
  r's' + a'  & = 3r \cdot \frac{s-2}{3} + a+2r\\
   & = r\cdot(s-2) + a + 2r\\
   & = rs+a\\
   & = xy & \text{(by~\eqref{inv})}.
\end{align*}

So in all three cases,~\eqref{inv'} holds, proving that $P$ is indeed a
preserved invariant.

Since the procedure's only termination condition is that $s=0$,
partial correctness will follow if we can show that if $s=0$, then $a=xy$.
But this follows immediately from~\eqref{inv}.
\end{solution}


\eparts
\end{problem}

\endinput
