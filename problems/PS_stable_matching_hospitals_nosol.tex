%PS_stable_matching_hospitals

\documentclass[problem]{mcs}

\begin{pcomments}
  \pcomment{from: S08.ps4}
%  \pcomment{}
%  \pcomment{}
\end{pcomments}

\pkeywords{
   stable_matching
}

%%%%%%%%%%%%%%%%%%%%%%%%%%%%%%%%%%%%%%%%%%%%%%%%%%%%%%%%%%%%%%%%%%%%%
% Problem starts here
%%%%%%%%%%%%%%%%%%%%%%%%%%%%%%%%%%%%%%%%%%%%%%%%%%%%%%%%%%%%%%%%%%%%%

\begin{problem}
\textbf{This problem is \textcolor{green}{not due} with this problem set.  It will be
included in pset 5.}

The most famous application of stable matching was in assigning
graduating medical students to hospital residencies.  Each hospital has a
preference ranking of students and each student has a preference order of
hospitals, but unlike the setup in the notes where there are an equal
number of boys and girls and monogamous marriages, hospitals generally have
differing numbers of available residencies, and the total number of
residencies may not equal the number of graduating students.  Modify the
definition of stable matching so it applies in this situation, and explain
how to modify the Mating Ritual so it yields stable assignments of students
to residencies.  No proof is required.

\begin{solution}
Will appear with Pset 5.
\end{solution}

\end{problem}


%%%%%%%%%%%%%%%%%%%%%%%%%%%%%%%%%%%%%%%%%%%%%%%%%%%%%%%%%%%%%%%%%%%%%
% Problem ends here
%%%%%%%%%%%%%%%%%%%%%%%%%%%%%%%%%%%%%%%%%%%%%%%%%%%%%%%%%%%%%%%%%%%%%

\endinput
