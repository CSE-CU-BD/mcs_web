\documentclass[problem]{mcs}

\begin{pcomments}
  \pcomment{UNER REVISION: combining with TP_Flipping_coins}
  \pcomment{FP_flirt_dependence_deviation}
  \pcomment{sequel to CP_sixteen_desks}
  \pcomment{submumes FP_flirt_dependence}
  \pcomment{ARM 5/8/16}
\end{pcomments}

\pkeywords{
  probability
  independence
  mutual
  deviation
  Markov
  Chebyshev
  binomial
  distribution
}

\newcommand{\flirt}{\text{flirt}}

%%%%%%%%%%%%%%%%%%%%%%%%%%%%%%%%%%%%%%%%%%%%%%%%%%%%%%%%%%%%%%%%%%%%%
% Problem starts here
%%%%%%%%%%%%%%%%%%%%%%%%%%%%%%%%%%%%%%%%%%%%%%%%%%%%%%%%%%%%%%%%%%%%%

\begin{problem} A classroom has sixteen desks in a $4 \times 4$ arrangement as shown below.
%
\begin{center}
\begin{picture}(300,160)
% \put(0,0){\dashbox(270,198){}} % bounding box
\multiput(0,  0)(72,0){4}{\framebox(40,25){}}
\multiput(0, 40)(72,0){4}{\framebox(40,25){}}
\multiput(0, 80)(72,0){4}{\framebox(40,25){}}
\multiput(0,120)(72,0){4}{\framebox(40,25){}}
\end{picture}
\end{center}

If two desks are next to each other, vertically or horizontally, they
are called an \emph{adjacent pair}.  So there are three horizontally
adjacent pairs in each row, for a total of twelve horizontally
adjacent pairs.  Likewise, there are twelve vertically adjacent pairs.
An adjacent pair $D$ of desks is said to have a \emph{flirtation} when
there is a boy at one desk and a girl at the other desk.

\bparts

\ppart Suppose boys and girls are assigned to desks in some unknown
probabilistic way.  What is the Markov bound on the probability that
the number of flirtations is at least 33 1/3\% more than expected?

\begin{center}
\exambox{0.6in}{0.4in}{0in}
\end{center}

%\examspace[0.5in]

\begin{solution}
\textbf{3/4}.

The Markov bound on $\pr{R \geq (4/3)\expect{R}}$ is $1/(4/3)$.
\end{solution}
\eparts

\medskip

Suppose that boys and girls are actually assigned to desks mutually
independently, with probability $p$ of a desk being occupied by a boy,
where $0<p<1$.

\bparts

\ppart Express the expected number of flirtations in terms of $p$.

\hint Let $I_D$ be the indicator variable for a flirtation at $D$.

\begin{center}
\exambox{0.6in}{0.6in}{-0.3in}
\end{center}

\examspace[0.7in]

\begin{solution}
$\mathbf{48p(1-p)}$.

\[
\expect{I_D} = \pr{I_D=1} = p(1-p)+(1-p)p = 2p(1-p).
\]
Since there are 24 adjacent pairs, the expected number of flirtations
overall is $24\expect{I_D} = 48p(1-p)$.
\end{solution}

\eparts

\medskip
Different pairs $D$ and $E$ of adjacent desks are said to
\emph{overlap} when they share a desk.  For example, the first and
second pairs in each row overlap, and so do the second and third
pairs, but the first and third pairs do not overlap.

\ppart\label{DEp12} Prove that if $D$ and $E$ overlap, and $p=1/2$,
then $I_D$ and $I_E$ are independent.

\examspace[1.7in]

\begin{solution}
Let $F_D$ be the event of a flirtation at $D$.

The variables are independent iff 
\[
\pr{F_D \intersect F_E} = \pr{F_D} \cdot \pr{F_E}.
\]
Now
\[
\pr{F_D} \cdot \pr{F_E} = (2p(1-p))^2 = \frac{1}{4}
\]
and
\begin{align*}
\pr{F_D \intersect F_E}
 & = \pr{\text{shared desk has a boy and the other two desks have girls}} +\\
 &\quad\ \pr{\text{shared desk has a girl and the other two desks have boys}}\\
 & = p(1-p)^2 + (1-p)p^2\\
 & = p(1-p)((1-p)+p)\\
 & = p(1-p) = \frac{1}{4}.
\end{align*}
So the events are independent.
\end{solution}

\ppart When $p=1/2$, what is the variance of the number of
flirtations?

\begin{center}
\exambox{0.7in}{0.4in}{0in}
\end{center}

\examspace[0.75in]

\begin{solution}
\textbf{25}.

Let $I_{F_D}$ be the indicator variable for $F_D$ so the total number of flirtations

We know $\variance{F_D} = p(1-p) = 1/4$\inbook{ by Corollary~\bref{bernoulli-variance}}.


The total number of flirtations is $\sum_D I_{F_D}
$ where 
Since the flirtations are pairwise independent by part~\eqref{DEp12}, the variances add

Let $X$ and $X_{i}$ be as in part ~\ref{expected}.  Then, by the independence of
the $X_{i}$, we know
\[
\Var[X] = 
\Var[X_{1} + \dots  + X_{100}] = 
\Var[X_{1}]+ \dots  + \Var[X_{100}].
\]
Since the variance of an indicator with expectation $p$ is
$p(1-p)$\inbook{ by Corollary~\bref{bernoulli-variance}}, we have
$\Var[X_{i}] = (1/2)(1 - 1/2) = 1/4$.  Therefore, 
\[
\Var[X] = 100 \cdot (1/4) = 25.
\]
\end{solution}

\ppart

What upper bound does Chebyshev's Theorem give us on the probability
that the number of heads is either less than 30 or greater than 70?

\begin{center}
\exambox{0.7in}{0.4in}{0in}
\end{center}

\begin{solution}
\textbf{1/16 = 0.0625}

The mean is 50.  So $X$ is less than 30 and more than 70 iff
it deviates from its mean by at least $20=|30-50|=|70-50|$. How big is
this deviation in comparison to the \emph{standard deviation}?  Since
the variance of the number of heads is 25, its standard deviation is
5.  So 20 is 4 times the standard deviation.

Now, Chebyshev's Theorem says that the probability of the number of
heads deviating from its expected value by 4 times the standard
deviation is ${}\le 1/(4^{2}) = 1/16$.
\end{solution}








\ppart Let $D$ and $E$ be pairs of adjacent desks that overlap.  Prove
that if $p \neq 1/2$, then $F_D$ and $F_E$ are \emph{not} independent.

\begin{solution}
From the solution to part~\eqref{DEp12}, we that $F_D$ and $F_E$ are
independent iff iff $p(1-p) = (2p(1-p))^2$.  Since $p,(1-p)>0$ we can cancel $p(1-p)$
and conclude that the events are independent iff
\begin{equation}\label{14pq}
1 = 4p(1-p) = 4(p-p^2).
\end{equation}
But~\eqref{14pq} holds iff $p = (1-p)$.  This can be verified by solving
the quadratic $p^2-p+ 1/4 = 0$ or by noting that $p(1-p)$ is maximized
when $p=(1-p)$.
\end{solution}

\ppart Find four pairs of desks $D_1,D_2,D_3,D_4$ and explain why
$F_{D_1}, F_{D_2}, F_{D_3}, F_{D_4}$ are \emph{not} mutually
independent (even if $p=1/2$).

\examspace[1.5in]

\begin{solution}
Let $D_1,D_2,D_3,D_4$ form a $2 \times 2$ pattern of four desks.  Then
$\bar{F_{D_1}} \intersect \bar{F_{D_2}} \intersect \bar{F_{D_3}}$
implies that all four desks must be occupied by people of the same
sex, and therefore $\bar{F_{D_4}}$ is true.  Hence, the events are not independent.

\begin{staffnotes}
Not required for full credit:
\end{staffnotes}

More formally,
\[
\prcond{\bar{F_{D_4}}}{\bar{F_{D_1}} \intersect \bar{F_{D_2}} \intersect \bar{F_{D_3}}}
= 1 \neq \pr{\bar{F_{D_4}}}.
\]

\end{solution}

\eparts

\end{problem}

%%%%%%%%%%%%%%%%%%%%%%%%%%%%%%%%%%%%%%%%%%%%%%%%%%%%%%%%%%%%%%%%%%%%%
% Problem ends here
%%%%%%%%%%%%%%%%%%%%%%%%%%%%%%%%%%%%%%%%%%%%%%%%%%%%%%%%%%%%%%%%%%%%%

\endinput

%Wolfram Alpha: prob x>11 for x binomial with n=16 and p=.5
