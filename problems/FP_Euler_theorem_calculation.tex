\documentclass[problem]{mcs}

\begin{pcomments}
  \pcomment{FP_Euler_theorem_calculation}
  \pcomment{adapted from: PS_Euler_theorem_calculation (aka MQ_Euler_theorem_calculation)}
  \pcomment{from: F08 PS3 -> S11 CP5F}
  \pcomment{modified by: Kazerani 05/11}
  \pcomment{edited to Zmod{220} by ARM 3/28/13}
\end{pcomments}

\pkeywords{
  Euler_theorem
  number_theory
  remainder
  modular_arithmetic
  phi
  Euler_function
}

%%%%%%%%%%%%%%%%%%%%%%%%%%%%%%%%%%%%%%%%%%%%%%%%%%%%%%%%%%%%%%%%%%%%%
% Problem starts here
%%%%%%%%%%%%%%%%%%%%%%%%%%%%%%%%%%%%%%%%%%%%%%%%%%%%%%%%%%%%%%%%%%%%%

\begin{problem}
What is the remainder of $63^{9601}$ divided by $220$?\hfill \examrule

\iffalse
\hint{$9601= (80 \cdot 120) + 1$; \idx{Euler's theorem}}
\fi

\begin{solution} 
\[
63.
\]

Since $63=3^2 \cdot 7$ and $220=2^2 \cdot 5 \cdot 11$ are relatively prime,
Euler's theorem implies that
\[
63^{\phi(220)} = 1 \inzmod{220}
\]
where
\begin{align*}
\phi(220) & = \phi(2^2 \cdot 5 \cdot 11)\\
          & = \phi(2^2 \cdot 5)\phi(11)
                 & \text{(since $\gcd(2^2\cdot 5,11)=1$)}\\
          & = \phi(2^2)\phi(5)\phi(11) 
                 & \text{(since $\gcd(2^2,5)=1$)}\\
          & = (2^2-2^1)(5-1)(11-1)
                 & \text{(since $2$, $5$ and $11$ are prime)}\\
          & = (2)(4)(10) = 80.
\end{align*}
Therefore $63^{80}\equiv 1 \pmod{220}$.

Since $9601 = 80 \cdot 120 + 1$, therefore
%
\begin{align*}
63^{9601} & = 63 \cdot 63^{80 \cdot 120}\\
             & = 63 \cdot 1^{120}\\
             & = 63 \inzmod{220}.
\end{align*}
So the remainder of $63^{9601}$ divided by $220$ is $63$.
\end{solution}

\end{problem}


%%%%%%%%%%%%%%%%%%%%%%%%%%%%%%%%%%%%%%%%%%%%%%%%%%%%%%%%%%%%%%%%%%%%%
% Problem ends here
%%%%%%%%%%%%%%%%%%%%%%%%%%%%%%%%%%%%%%%%%%%%%%%%%%%%%%%%%%%%%%%%%%%%%


\endinput
