\documentclass[problem]{mcs}

\begin{pcomments}
  \pcomment{CP_XOR_AND_recursive}
  \pcomment{variant of CP_XOR_AND_CP_XOR_AND_formulas except this uses
    structural induction}
  \pcomment{S16.mid1, S17.ps3}
  \pcomment{ARM 2/27/17}
\end{pcomments}

\pkeywords{
  recursive
  structural_induction
  proposition
  XOR
  equivalence
  constant
  formula}

\newcommand{\PXA}{\text{PXA}}

\begin{problem}
Let $P$ be a propositional variable.
\bparts

\ppart Show how to express $\QNOT(P)$ using $P$ and a selection from
among the constant $\True$, and the connectives $\QXOR$ and $\QAND$.

\begin{solution}
\[
\QNOT(P) \quad\equiv\quad P \QXOR \True.
\]
\end{solution}

\examspace[0.6in]

 \eparts

\bigskip
The use of the constant \True\ above is essential.  To prove this, we
begin with a recursive definition of \QXOR-\QAND\ formulas that do not
use \True, called the \PXA\ formulas.
\begin{definition*}
\inductioncase{Base case}: The propositional variable $P$ is a \PXA\ formula.

\inductioncase{Constructor cases}  If $R,S \in \PXA$, then
\begin{itemize}
\item  $R \QXOR S$,
\item  $R \QAND S$
\end{itemize}
are \PXA 's.
\end{definition*}

For example,
\[  (((P \QXOR P) \QAND P) \QXOR (P \QAND P)) \QXOR (P \QXOR P)
\]
is a \PXA.

\bparts

\ppart Prove by structural induction on the definition of \PXA\ that
every \PXA\ formula $A$ is equivalent to $P$ or to \False.

\begin{solution}
\begin{proof}
\inductioncase{Base case}: ($A\ \text{is}\ P$).  $A$ is equivalent to
$P$ since it equals $P$.

\inductioncase{Constructor case}: ($A\ \text{is}\ [R \QAND S]$).
Each of $R$ and $S$ are equivalent either to $P$ or to \False\ by
structural induction hypothesis.  If either of them is equivalent to
\False, then $A$ is equivalent to \False.  If both are equivalent to
$P$, then $A$ is equivalent to $P$.  It follows that $A$ satisfies the
induction hypothesis.

\inductioncase{Constructor case}: ($A\ \text{is}\ [R \QXOR S]$).  Each
of $R$ and $S$ are equivalent either to $P$ or to \False\ by
structural induction hypothesis.  If $R$ is equivalent to \False, then
$A$ is equivalent to $S$, and therefore $A$ satisfies the induction
zhypothesis; likewise if $S$ is equivalent to \False, then $A$ is
equivalent to $R$.  If both $R$ and $S$ are equivalent to $P$, then $A$
is equivalent to $P \QXOR P$ which is equivalent to \False\ and
therefore satisfies the induction hypothesis.  It follows in any case
that $A$ satisfies the induction hypothesis.
\end{proof}
\end{solution}

\eparts
\end{problem}

\endinput

