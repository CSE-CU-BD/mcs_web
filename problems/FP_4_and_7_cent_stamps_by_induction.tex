\documentclass[problem]{mcs}

\begin{pcomments}
  \pcomment{FP_4_and_7_cent_stamps_by_induction}
  \pcomment{from: F09.mq3}
  \pcomment{similar to CP_10_and_15_cent_stamps_by_WOP but slightly
    different and uses induction}
\end{pcomments}

\pkeywords{
  induction
  strong_induction
}

%%%%%%%%%%%%%%%%%%%%%%%%%%%%%%%%%%%%%%%%%%%%%%%%%%%%%%%%%%%%%%%%%%%%%
% Problem starts here
%%%%%%%%%%%%%%%%%%%%%%%%%%%%%%%%%%%%%%%%%%%%%%%%%%%%%%%%%%%%%%%%%%%%%

\begin{problem}
  
  Let $S(n)$ mean that exactly $n$ cents of postage can be paid using
  only 4 and 7 cent stamps.  USe strong induction to prove that
  \[ 
    \forall n.\ ( n \geq 18 ) \QIMPLIES\ S(n).
  \]
    
% \ppart
% 
%   Let $S(n)$ mean that exactly $n$ cents of postage can be paid using 
%   only 5 and 3 cent stamps. 
%   
%   Find the smallest $k$ such that the following 
%   proposition is true.
%   \[ 
%   \forall n.\ ( n \geq k ) \QIMPLIES\ S(n).
%   \]
%   
% \vspace{1in}
  % Use induction to prove that your answer to the previous problem 
  % part is correct. 
  
\begin{solution}
  
  \begin{proof}
    The following proof is by strong induction on $n$.
    
    We can begin by observing that the following postage amounts can 
    be made by 4 and 7 cent stamps:

    \begin{align*}
    18 & = 4 + 7 + 7 \\
    19 & = 4 + 4 + 4 + 7 \\
    20 & = 4 + 4 + 4 + 4 + 4 \\
    21 & = 7 + 7 + 7 \\
    \end{align*}

    The 4 consecutive postage values $18, 19, 20, 21$ will be our base cases 
    for the induction proof.

    \textbf{Base cases:} $S(18)$, $S(19)$, $S(20)$ and $S(21)$ are shown to hold
    by explicit calculations.

    \textbf{Inductive step:} For all $n \geq 21$, we assume that $P(18)$,
    $\dots$, $P(n)$ are true in order to prove that $P(n+1)$ is true.

    By the assumption that $P(n-3)$ is true, we know that the postage value
    $n - 3$ can be paid with 4 and 7 cent stamps. By adding one 7 cent
    stamp to that postage, we will be able to pay for a postage of 
    $n - 3 + 4 = n + 1$ cents, showing that $P(n+1)$ is true. It follows
    by strong induction that $P(n)$ holds for all $n \geq 21$.
    
    We have therefore shown that all postage values $ \geq 18 $
    can be paid by 4 and 7 cent stamps.
  \end {proof}
  
  Therefore, we have shown that the postage amounts 4, 7, and any 
  $k \geq 8$ can be paid by 5 and 3 cent stamps.
\begin{editingnotes}  
  Note that we have actually seen a proof for this using WOP in the 
  \href{http://courses.csail.mit.edu/6.042/fall09/slides2m.pdf}{Well Ordering Principle}
  lecture.
\end{editingnotes}
\end{solution}

\end{problem}

%%%%%%%%%%%%%%%%%%%%%%%%%%%%%%%%%%%%%%%%%%%%%%%%%%%%%%%%%%%%%%%%%%%%%
% Problem ends here
%%%%%%%%%%%%%%%%%%%%%%%%%%%%%%%%%%%%%%%%%%%%%%%%%%%%%%%%%%%%%%%%%%%%%
