\documentclass[problem]{mcs}

\begin{pcomments}
  \pcomment{FP_4_and_7_cent_stamps_by_induction}
  \pcomment{from: F09.mq3}
  \pcomment{similar to CP_10_and_15_cent_stamps_by_WOP but slightly
    different and uses induction}
  \pcomment{revised to use class of teams by ARM 2/21/15}
\end{pcomments}

\pkeywords{
  induction
  strong_induction
}

%%%%%%%%%%%%%%%%%%%%%%%%%%%%%%%%%%%%%%%%%%%%%%%%%%%%%%%%%%%%%%%%%%%%%
% Problem starts here
%%%%%%%%%%%%%%%%%%%%%%%%%%%%%%%%%%%%%%%%%%%%%%%%%%%%%%%%%%%%%%%%%%%%%

\begin{problem}
  
  A class of any size of 18 or more can be assembled from student
  teams of sizes 4 and 7.  Prove this by \textbf{induction}, using the
  predicate
  \[
  S(n) \eqdef \text{a class of $n$ students can be
    assembled from teams of sizes 4 and 7}.
  \]

% \ppart
% 
%   Let $S(n)$ mean that exactly $n$ cents of postage can be paid using 
%   only 5 and 3 cent stamps. 
%   
%   Find the smallest $k$ such that the following 
%   proposition is true.
%   \[ 
%   \forall n.\ ( n \geq k ) \QIMPLIES\ S(n).
%   \]
%   
% \vspace{1in}
  % Use induction to prove that your answer to the previous problem 
  % part is correct. 
  
\begin{solution}
  
  \begin{proof}
    The following proof is by strong induction on $n$.
    
    We can begin by observing that the following size classes can 
    be made from teams of sizes 4 and 7:
    \begin{align*}
    18 & = 4 + 7 + 7 \\
    19 & = 4 + 4 + 4 + 7 \\
    20 & = 4 + 4 + 4 + 4 + 4 \\
    21 & = 7 + 7 + 7 \\
    \end{align*}

    The 4 consecutive class sizes $18, 19, 20, 21$ will be our base cases 
    for the induction proof.

    \inductioncase{Base cases:} $S(18)$, $S(19)$, $S(20)$ and $S(21)$
    were shown to hold by explicit calculations.

    \inductioncase{Inductive step:} For all $n \geq 21$, we assume
    that $S(18), \dots, S(n)$ are true in order to prove that $S(n+1)$
    is true.

    Since $n-3 \geq 18$, we have $S(n-3)$ by assumption.  Adding one
    team of 4 students to the class of size $n-3$, we get a class of
    size $n - 3 + 4 = n + 1$ cents, showing that $S(n+1)$ is true. It
    follows by strong induction that $S(n)$ holds for all $n \geq 18$.
    
    We conclude that classes of all sizes $ \geq 18 $ can be formed
    from teams of 4 and 7 students.
  \end{proof}
  
  Therefore, we have shown that the postage amounts 4, 7, and any 
  $k \geq 8$ can be paid by 5 and 3 cent stamps.

\begin{staffnotes}
  Note that we have actually seen a proof for this using WOP and
  stamps in the
  \href{https://courses.csail.mit.edu/6.042/spring15/well_ordering_2.pdf}{Well
    Ordering Principle} lecture.
\end{staffnotes}

\end{solution}

\end{problem}

%%%%%%%%%%%%%%%%%%%%%%%%%%%%%%%%%%%%%%%%%%%%%%%%%%%%%%%%%%%%%%%%%%%%%
% Problem ends here
%%%%%%%%%%%%%%%%%%%%%%%%%%%%%%%%%%%%%%%%%%%%%%%%%%%%%%%%%%%%%%%%%%%%%
