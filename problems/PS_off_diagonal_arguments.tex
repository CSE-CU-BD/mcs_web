\documentclass[problem]{mcs}

\begin{pcomments}
  \pcomment{PS_off_diagonal_arguments}
  \pcomment{ARM 3.14.17}
\end{pcomments}

\pkeywords{
  diagonal_argument
  diagonal
  countable
  uncountable
  subset
}

%%%%%%%%%%%%%%%%%%%%%%%%%%%%%%%%%%%%%%%%%%%%%%%%%%%%%%%%%%%%%%%%%%%%%
% Problem starts here
%%%%%%%%%%%%%%%%%%%%%%%%%%%%%%%%%%%%%%%%%%%%%%%%%%%%%%%%%%%%%%%%%%%%%

\begin{problem}
You don't really have to go down the diagonal in a ``diagonal'' argument.

Let's review the historic application of a diagonal argument to
one-way infinite sequences
\[
\ang{e_0,e_1,e_2, \dots, e_k, \dots}.
\]
The angle brackets appear above as a reminder that the sequence is not a
set: its elements appear in order, and the same element may appear
multiple times.\footnote{The right angle-bracket is not really
  visible, since the sequence does not have a right end.  If such
  one-way infinite sequences seem worrisome, you can replace them with
  total functions on $\nngint$.  So the sequence above simpy becomes the
  function $e$ on $\nngint$ where $e(n) \eqdef e_n$.}

The general setup for a diagonal argument is that we have some sequence
$S$ whose elements are themselves one-way infinite sequences.  We
picture the sequence $S = \ang{s_0,s_1,s_2,\dots}$ running vertically
downward, and each sequence $s_k \in S$ running horizontally to the right:
\[
s_k =  \ang{s_{k,0}, s_{k,1}, s_{k,2}, \dots}.
\]
So we have a 2-D matrix that is infinite down and to the right:

\[\begin{array}{r|ccccccl}
    &     0  &  1     & 2      & \hspace{0.2in}& \dots & \hspace{0.2in} &  k       \dots\\
\hline
s_0 &  s_{0,0} & s_{0,1} & s_{0,2} &                & \dots &                &  s_{0,k}  \dots\\
s_1 &  s_{1,0} & s_{1,1} & s_{1,2} &                & \dots &                &  s_{1,k}  \dots\\
s_2 &  s_{2,0} & s_{2,1} & s_{2,2} &                & \dots &                &  s_{2, k} \dots\\
\\
\vdots &                                     &&&&\vdots\\
\\
s_k                                         &&&&& \dots &                &  s_{k.k}   \dots
\end{array}\]

The diagonal argument explains how to find a ``new'' sequence, that is,
a sequence that is not in $S$.  Namely, create the new sequence by going
down the diagonal of the matrix and, for each element encountered, add a
differing element to the sequence being created.  In other words, the
diagonal sequence is
\[
D_S \eqdef \ang{\bar{s_{0,0}},\ \bar{s_{1,1}},\ \bar{s_{2,2}},\dots,\bar{s_{k,k}}, \dots}
\]
where $\bar{x}$ indicates some element that is not equal to $x$.  Now
$D_S$ is a sequence that is not in $S$ because it differs from every
sequence in $S$, namely, it differs from the $k$th sequence in $S$ at
position $k$.

For definiteness, let's say
\[
\bar{x} \eqdef \begin{cases}
               1 &\text{ if }x \neq 1,\\
               2 &\text{otherwise}.\\
               \end{cases}
\]
With this contrivance, we have gotten the diagonal sequence $D_S$ to be
a sequence of 1's and 2's that is not in $S$.

But as we said at the beginning, you don't have to go down the
diagonal.  You could, for example, follow a line with a slope of
$-1/4$ to get a new sequence
\[
T_S \eqdef \ang{\bar{s_{0,0}},\  t_1,\  t_2,\  t_3,\  \bar{s_{1,4}},\ 
  t_5,\  t_6,\  t_7,\  \bar{s_{2,8}},\ \dots,\  t_{4k-1},\ 
  \bar{s_{k,4k}},\  t_{4k+1},\  \dots}
\]
where the $t_i$'s can be anything at all.  Any such $T_s$ will be a new
sequence because it will differ from every row of the matrix, but this
time it differs from the $k$th row at position $4k$.

\bparts

\ppart By letting all the $t_i$'s be 2's, we get a new sequence $T_S$
of 1's and 2's whose elements (in the limit) are at least
three-quarters equal to 2.  Explain how to find an \emph{uncountable}
number of such sequences of 1's and 2's.

\hint Use slope $-1/8$ and fill six out of eight places with 2's.  The
abbreviation 
\[
2^{(n)} \eqdef\underbrace{2,2,\dots,2}_{\text{length}\ n}
\]
for a length-$n$ sequence of 2's may be helpful, in particular
$2^{(6)}$.

\begin{solution}
Using the hint, we are considering sequences of the form
\[
  \ang{\bar{s_{0,0}},\  t_1,\  2^{(6)}, \bar{s_{1,8}},\  t_9,\ 
    2^{(6)},\  \dots,\  \bar{s_{k,8k}},\  t_{8k+1},\  2^{(6)},\  \dots}.
\]
Since six out of every eight positions have 2's, any such a sequence
will be at least three quarters 2's.  Also, every such sequence will be
new, since it will still differ from the $k$th row at position $8k$.

Any particular sequence of this kind is exactly determined by the set
$K$ of nonnegative integers $k$ for which $t_{8k+1}$ is a 1.  This
defines a bijection between sequences of this form and the uncountably
many subsets $K \in \power(\nngint)$.
\end{solution}

\ppart Let's say a sequence has a \emph{negligible fraction of non-2
  elements} if, in the limit, it has a fraction of at most $\epsilon$
non-2 elements for \emph{every} $\epsilon>0$.  Describe how to define
a sequence not in $S$ that has a \emph{negligible fraction of non-2
  elements}.

\begin{solution}
One such sequence is
\[
\ang{\bar{s_{0,0}},\  \bar{s_{1,1}},\  2^{(2)},\  \bar{s_{2,4}},\ 
  2^{(4)},  \dots,\  \bar{s_{k,k^2}},\  2^{(2k)},\  \dots}
\]
This sequence is new because it differs from the $k$th row at position
$k^2$, and it has a negligible fraction of non-2's because among the
first $n$ elements, at most $(\sqrt{n}+1)/n$ can be non-2's, and this
fraction goes to zero as $n$ goes to infinity.

In fact, we could replace $k^2$ by any function of $k$ that grows more
than linearly.
\end{solution}

\ppart Describe how to find a sequence that differs \emph{infinitely
  many times} from every sequence in $S$.

\hint Divide \nngint\ into an infinite number of non-overlapping
infinite pieces.

\begin{solution}
Divide \nngint\ into an infinite number of non-overlapping infinite
pieces, and let the ``diagonal'' set differ from the $k$th row at all
the positions in the $k$th piece.

There are lots of ways to divide up \nngint\ in this way.

For example, one way is to use the bijection between $\nngint$ and pairs
$(m,n) \in \nngint^2$ and let the $k$th piece be the integers
corresponding to the pairs for the form $(k,n)$ for $n \in \nngint$.

Another way is to let the first piece be half of \nngint, the second
piece be half of the remaining half, the third piece be half of the
remaining quarter, and so forth.  So the $k$th piece would be the
integers of the form $2^k \cdot n + n$ for $n \in \nngint$.

And yet another way would be to let the first piece be the powers of
two, the second piece be the powers of three, \dots, the $n$th piece
be the powers of the $n$th prime number, and let the zeroth piece be
all the numbers that are not powers of a prime.

\end{solution}

\eparts

\end{problem}
\endinput
