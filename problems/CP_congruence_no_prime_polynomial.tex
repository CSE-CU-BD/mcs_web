\documentclass[problem]{mcs}

\begin{pcomments}
  \pcomment{CP_congruence_no_prime_polynomial}
  \pcomment{subsumed by simpler CP_polynomials_produce_multiples}
  \pcomment{edited ARM 3/1/12}
\end{pcomments}

\pkeywords{
  primes
  polynomial
  nonconstant
  congruence
  induction
  degree
}


%%%%%%%%%%%%%%%%%%%%%%%%%%%%%%%%%%%%%%%%%%%%%%%%%%%%%%%%%%%%%%%%%%%%%
% Problem starts here
%%%%%%%%%%%%%%%%%%%%%%%%%%%%%%%%%%%%%%%%%%%%%%%%%%%%%%%%%%%%%%%%%%%%%

\begin{problem}
  An \term{integer polynomial} is a polynomial with integer
  coefficients.  Let $q(x)$ be an integer polynomial.  This problem
  shows that the values of $q$ on nonnegative integers, that is the
  image $q(\naturals)$ includes infinitely many composite numbers.

\begin{problemparts}

\ppart\label{jknq>1} Prove that if $j \equiv k \pmod n$ then $q(j)
\equiv q(k) \pmod n$ for all $j, k, n \in \integers$ with $n>1$.

\hint If $q(x)$ is of degree $d>0$, then it equals $c + xp(x)$ for
some integer $c$ and integer polynomial $p(x)$ of degree $d-1$.

\begin{solution}
  The proof is by induction on $d$ with induction hypothesis
  that~\eqref{jknq>1} holds for all integer polynomials, $q$, of
  degree $d$.

  \inductioncase{Base case} ($d=0$): A polynomial $q$ of degree $0$ is a constant, so 
   $q(j) =q(k)$ for all $j,k$.

  \inductioncase{Inductive step}: We must prove that~\eqref{jknq>1} holds for
  any integer polynomial, $q(x)$, of degree $d+1$.

  So suppose $j \equiv k \pmod n$ where $n>1$.  Now by the hint, $q(x)
  = c + xp(x)$ for some integer polynomial, $p(x)$ of degree, $d$.
  Also,
\begin{equation}\label{}
p(j) \equiv p(k) \pmod n
\end{equation}
by induction hypothesis, so
\begin{equation}\label{jpjkpk}
jp(j) \equiv kp(k) \pmod n
\end{equation}
by multiplicativity of congruence
(Lemma~\bref{congruence_facts}.\bref{congruence_multiplicativity}.
Finally,
\begin{align*}
  q(j) & = c + jp(j) \equiv c + kp(k) \pmod n
          & \text{(by~\eqref{jpjkpk} and congruence additivity
             Lemma~\bref{congruence_facts}.bref{congruence_additivity})}\\
       & = q(k).
 \end{align*}
\end{solution}

\ppart\label{qunbmd} If $q$ is not constant, then $q(j)$ grows
unboundedly as $j$ goes to infinity.  Conclude from this that
$q(\naturals)$ contains infinitely many composite positive integers.
\hint Choose $j$ such that $q(j) = n > 1$ and consider $k$'s such that
$j \equiv k \pmod n$.

\begin{solution}
Since $q$ grows unboundedly, there will be $j \in \naturals$ such that
$1 < n \eqdef q(j)$, and for any such $j$, there will be an infinite
sequence of nonnegative integers $j_1, j_2,j_3,\dots$ all $\equiv j
\pmod n$ such that $q(j_1), q(j_2), q(j_3), \dots$ is a strictly
increasing sequence of integers greater than $n$.  But all the numbers
in this sequence will be composite, since by part~\eqref{jknq>1}, they
all have $n$ as a divisor.
\end{solution}

Part~\eqref{qunbmd} implies that if an integer polynomial is not
constant then it has infinitely many positive values on $\naturals$,
then infinite number of these values will not be prime.  This fact no
longer holds true for multivariate polynomials.  An amazing
consequence of Matiyasevich's~\cite{Matiyasevich} solution to
Hilbert's Tenth Problem, is that multivariate polynomials can be
understood as \emph{general purpose} programs for generating sets of
integers.  If a set of nonnegative integers can be generated by
\emph{any} program, then it equals the set of nonnegative integers in
the range of a multivariate integer polynomial!  In particular, there
is an integer polynomial $p(x_1,\dots,x_7)$ whose nonnegative values
as $x_1,\dots,x_7$ range over $\naturals$ are precisely the set of all
prime numbers!

\end{problemparts}

\end{problem}

%%%%%%%%%%%%%%%%%%%%%%%%%%%%%%%%%%%%%%%%%%%%%%%%%%%%%%%%%%%%%%%%%%%%%
% Problem ends here
%%%%%%%%%%%%%%%%%%%%%%%%%%%%%%%%%%%%%%%%%%%%%%%%%%%%%%%%%%%%%%%%%%%%%

\endinput
