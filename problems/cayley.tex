\documentclass[problem]{mcs}

\begin{pcomments}
  \pcomment{Cayley's formula}
  \pcomment{wikipedia}
\end{pcomments}

\pkeywords{
  tree
  forest
  Cayley
  counting
  double_counting
}

%\subsection{}
\begin{problem}

\begin{center}
Counting trees
\footnote{From
  \href{https://en.wikipedia.org/wiki/Double\_counting\_\%28proof\_technique\%29\#Counting\_trees}
{Double counting, wikipedia, Aug. 30, 2014}} see also
\href{https://en.wikipedia.org/wiki/Pr\%C3\%BCfer\_sequence}{Pr\:ufer Sequences}
\end{center}

What is the number $T_n$ of different trees that can be formed from a
set of $n$ distinct vertices? Cayley's formula gives the answer $T_n =
n^{n - 2}$.  Aigner \& Ziegler (1998) list four proofs of this fact; they
write of the fourth, a double counting proof due to Jim Pitman, that
it is “the most beautiful of them all.”

Pitman's proof counts in two different ways the number of different
sequences of directed edges that can be added to an empty graph on $n$
vertices to form from it a rooted tree.  One way to form such a
sequence is to start with one of the $T_n$ possible unrooted trees,
choose one of its n vertices as root, and choose one of the $(n - 1)!$
possible sequences in which to add its $n - 1$ edges.  Therefore, the
total number of sequences that can be formed in this way is
\[
T_n n(n - 1)! = T_n n!\, .
\]

Another way to count these edge sequences is to consider adding the
edges one by one to an empty graph, and to count the number of choices
available at each step.  If one has added a collection of $n - k$ edges
already, so that the graph formed by these edges is a rooted forest
with $k$ trees, there are $n(k - 1)$ choices for the next edge to add:
its starting vertex can be any one of the n vertices of the graph, and
its ending vertex can be any one of the $k - 1$ roots other than the
root of the tree containing the starting vertex.  Therefore, if one
multiplies together the number of choices from the first step, the
second step, etc., the total number of choices is
\[
   \prod_{k=2}^{n} n(k-1) = n^{n-1} (n-1)! = n^{n-2} n!\, .
\]
Equating these two formulas for the number of edge sequences results in Cayley's formula:
\[
   T_n n! = n^{n-2}n!
\]
and
\[
    T_n = n^{n-2}.
\]
As Aigner and Ziegler describe, the formula and the proof can be
generalized to count the number of rooted forests with $k$ trees, for
any $k$.

\end{problem}

\endinput
