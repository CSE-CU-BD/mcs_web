\documentclass[problem]{mcs}

\begin{pcomments}
  \pcomment{from: S09.cp14m}
  \pcomment{revised 5/2/2014}
\end{pcomments}

\pkeywords{
  expectation
}

%%%%%%%%%%%%%%%%%%%%%%%%%%%%%%%%%%%%%%%%%%%%%%%%%%%%%%%%%%%%%%%%%%%%%
% Problem starts here
%%%%%%%%%%%%%%%%%%%%%%%%%%%%%%%%%%%%%%%%%%%%%%%%%%%%%%%%%%%%%%%%%%%%%

\begin{problem}
  Here's a dice game with maximum payoff $k$: make three independent
  rolls of a fair die, and if you roll a six
\begin{itemize}

\item no times, then you lose 1 dollar;

\item exactly once, then you win 1 dollar;

\item exactly twice, then you win 2 dollars;

\item all three times, then you win $k$ dollars.

\end{itemize}
For what value of $k$ is this game fair?\footnote{This game is
  actually offered in casinos with $k=3$, where it is called Carnival
  Dice.}

\begin{solution}
Let the random variable $P$ be your payoff.
We can compute $\expect{P}$ as follows:
\begin{align*}
\expect{P}  & = -1 \cdot \pr{\text{0 sixes}} +
             1 \cdot \pr{\text{1 six}} +
             2 \cdot \pr{\text{2 sixes}} +
             k \cdot \pr{\text{3 sixes}} \\
        & = -1 \cdot \left(\frac{5}{6}\right)^3 +
             1 \cdot 3\paren{\frac{1}{6}}\paren{\frac{5}{6}}^2 +
             2 \cdot 3\paren{\frac{1}{6}}^2\paren{\frac{5}{6}} +
             k \cdot \paren{\frac{1}{6}}^3 \\
        & =  \frac{-125 + 75 + 30 + k}{216}
\end{align*}
The game is fair when $\expect{P} = 0$.  This happens when $k = 20$.
\end{solution}

\end{problem}


%%%%%%%%%%%%%%%%%%%%%%%%%%%%%%%%%%%%%%%%%%%%%%%%%%%%%%%%%%%%%%%%%%%%%
% Problem ends here
%%%%%%%%%%%%%%%%%%%%%%%%%%%%%%%%%%%%%%%%%%%%%%%%%%%%%%%%%%%%%%%%%%%%%
