\documentclass[problem]{mcs}

\begin{pcomments}
  \pcomment{CP_Theta_log_n_factorial}
  \pcomment{from: S09.cp9t, F03.ps6}
\end{pcomments}

\pkeywords{
  asymptotics
  factorial
  Theta
}

%%%%%%%%%%%%%%%%%%%%%%%%%%%%%%%%%%%%%%%%%%%%%%%%%%%%%%%%%%%%%%%%%%%%%
% Problem starts here
%%%%%%%%%%%%%%%%%%%%%%%%%%%%%%%%%%%%%%%%%%%%%%%%%%%%%%%%%%%%%%%%%%%%%

\begin{problem}
Give an elementary proof (without appealing to Stirling's formula)
that $\log (n!) = \Theta(n\log n)$.

\begin{solution}
One elementary proof goes as follows:

First,
\[
\log (n!) = \sum_{i=1}^n \log i < \sum_{i=1}^n \log n =n\log n.
\]

On the other hand,
\begin{align*}
\log (n!) &= \sum_{i=1}^n \log i > \sum_{i=\ceil{(n+1)/2}}^n \log i\\
    & > \sum_{i=\ceil{(n+1)/2}}^n \log (n/2) > \frac{n}{2} \cdot \log (n/2)\\
    & = \frac{n ((\log n) -1)}{2}  = \frac{n \log n}{2} - \frac{n}{2}\\
    & > \frac{n \log n}{2} -\frac{n \log n}{6} & \text{for $n>8$.}\\
    & = \frac{1}{3} \cdot n \log n.
\end{align*}

Therefore, $(1/3)n \log n < \log(n!) <n\log n$ for $n>8$, proving that
$\log(n!)=\Theta(n\log n)$.

\end{solution}

\end{problem}

%%%%%%%%%%%%%%%%%%%%%%%%%%%%%%%%%%%%%%%%%%%%%%%%%%%%%%%%%%%%%%%%%%%%%
% Problem ends here
%%%%%%%%%%%%%%%%%%%%%%%%%%%%%%%%%%%%%%%%%%%%%%%%%%%%%%%%%%%%%%%%%%%%%

\endinput
