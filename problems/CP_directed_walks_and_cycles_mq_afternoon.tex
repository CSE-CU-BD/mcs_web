\documentclass[problem]{mcs}

\begin{pcomments}
  \pcomment{CP_directed_walks_and_cycles} \pcomment{from: S09.cp7r}
  \pcomment{subsumed by text, class prob only}
\end{pcomments}

\pkeywords{
  digraphs
  cycles
}

%%%%%%%%%%%%%%%%%%%%%%%%%%%%%%%%%%%%%%%%%%%%%%%%%%%%%%%%%%%%%%%%%%%%%
% Problem starts here
%%%%%%%%%%%%%%%%%%%%%%%%%%%%%%%%%%%%%%%%%%%%%%%%%%%%%%%%%%%%%%%%%%%%%

\begin{problem}

Give an example showing that two vertices in a digraph may be
on the same closed walk, but \emph{not} necessarily on the same cycle.

\begin{solution}
Let the vertices be $a,b,c$ and edges be $(a,b), (b,a), (b,c), (c,b)$.
Now $a$ and $c$ are on the closed walk $a,b,c,b,a$, but every closed
walk from $a$ to $c$ must go through $b$ at least twice, and so will
not be a cycle.
\end{solution}


\end{problem}

%%%%%%%%%%%%%%%%%%%%%%%%%%%%%%%%%%%%%%%%%%%%%%%%%%%%%%%%%%%%%%%%%%%%%
% Problem ends here
%%%%%%%%%%%%%%%%%%%%%%%%%%%%%%%%%%%%%%%%%%%%%%%%%%%%%%%%%%%%%%%%%%%%%

\endinput
