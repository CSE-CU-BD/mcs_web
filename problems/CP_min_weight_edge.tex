\documentclass[problem]{mcs}

\begin{pcomments}
  \pcomment{CP_min_weight_edge}
  \pcomment{renamed from CP_min_weight_span_tree}
  \pcomment{cut down and revised from mistaken early version --ARM Oct. 31, '11}
  \pcomment{from F07.rec7t, S06.cp5f, S04.cp5w}
  \pcomment{special case of general theory in text}
\end{pcomments}

\pkeywords{
spanning_tree
weighted_tree
minimum_weight
MST
}

%%%%%%%%%%%%%%%%%%%%%%%%%%%%%%%%%%%%%%%%%%%%%%%%%%%%%%%%%%%%%%%%%%%%%
% Problem starts here
%%%%%%%%%%%%%%%%%%%%%%%%%%%%%%%%%%%%%%%%%%%%%%%%%%%%%%%%%%%%%%%%%%%%%

\begin{problem}

\iffalse
A \emph{weighted graph} is a graph with a function that assigns a
nonnegative real number to each edge.  The real number assigned to an
edge is called its \emph{weight}.  The weight of the graph is the sum
of the weights of its edges.

A \emph{minimum weight spanning tree} of a weighted graph is a
spanning tree with the smallest weight of any spanning tree of the
graph.
\fi

\begin{editingnotes}
An example would be nice.
\end{editingnotes}

Let $G$ be a weighted graph and suppose there is a unique edge $e \in
\edges{G}$ with smallest weight, that is, $w(e) < w(f)$ for all edges
$f \in \edges{G} - \set{e}$.  Prove that any minimum weight spanning
tree (MST) of $G$ must include $e$.

\begin{solution}
Suppose to the contrary that $e$ is not included in some MST, $T$.
Since $T$ is a spanning tree, if we add $e$ to $T$, everything stays
connected, so the new graph, $T+e$, is a connected spanning subgraph.
Moreover, $T+e$ now has too many edges to be a tree, so the edge $e$
must be on a cycle in $T+e$.

Let $f$ be another edge on this cycle.  Now remove $f$ to obtain the
graph $(T + e) - f$.  Then
\begin{enumerate}

\item $(T + e) - f$ is still a connected spanning subgraph of $G$,
  because removing an edge, $f$, on a cycle does not change
  connectedness.

\item $(T + e) - f$ is a tree, because it is a connected subgraph with
  the same number of vertices and edges as the tree $T$.

\item $w((T + e) - f) = w(T) - (w(f) - w(e)) < w(T)$.

\end{enumerate}
Hence $(T+e)-f$ is a spanning tree of $G$ with strictly smaller weight
than the MST, $T$, contradicting the minimality of $T$%.

\end{solution}

\end{problem}

\endinput
