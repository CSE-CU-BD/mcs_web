\documentclass[problem]{mcs}

\begin{pcomments}
    \pcomment{TP_Induction_by_n+3}
    \pcomment{Converted from prob1.scm by scmtotex and dmj
              on Sun 13 Jun 2010 10:06:06 AM EDT}
    \pcomment{revised by drewe on 18 july 2011}
    \pcomment{trivial edit, format by ARM 8/25/11}
\end{pcomments}

\pkeywords{
induction
strong_induction
predicate
}

\begin{problem}
Alice wants to prove by induction that a predicate $P$ holds for
certain nonnegative integers.  She has proven that for all nonnegative
integers $n = 0,1,\dots$

\begin{equation*}
P(n) \QIMPLIES P(n+3).
\end{equation*}

\bparts

\ppart Suppose Alice also proves that $P(5)$ holds.  Which of the
following propositions can she infer?

\begin{enumerate}

\item $P(n)$ holds for all $n \ge  5$

\item $P(3n)$ holds for all $n \ge  5$

\item $P(n)$ holds for $n = 8, 11, 14, \dots $

\item $P(n)$ does not hold for $n < 5$

\item $\forall n.\; P(3n + 5)$

\item $\forall n > 2.\; P(3n - 1)$

\item $P(0) \QIMPLIES \forall n.\; P(3n + 2)$

\item $P(0) \QIMPLIES \forall n.\; P(3n)$

\end{enumerate}

\begin{solution}
3,
5,
6,
8

$P$ will be true on 5 and all numbers that are greater by a multiple
of~3. That is, 5, 8, 11, 14, etc.  This is exactly what answer~(5)
says. Answers~(3) and~(6) talk of the same sequence except number~5,
so they are still propositions that Alice can infer, although not the
strongest possible. Answer~(8) is also a valid conclusion: if Alice
knows $P$ is true on~0, she knows it will also be true on 3, 6, 9,
etc.; so, it will be true on all multiples of~3 (note that the truth
of $P$ on~5 is irrelevant here).
\end{solution}

\ppart
Which of the following could Alice prove in order to conclude that
$P(n)$ holds for all $n \ge 5$?

\begin{enumerate}

\item $P(0)$

\item $P(5)$

\item $P(5)$ and $P(6)$

\item $P(0)$, $P(1)$ and $P(2)$

\item $P(5)$, $P(6)$ and $P(7)$

\item $P(2)$, $P(4)$ and $P(5)$

\item $P(2)$, $P(4)$ and $P(6)$

\item $P(3)$, $P(5)$ and $P(7)$

\end{enumerate}

\begin{solution}

4,
5,
7,
8

Once Alice proves $P(5)$, $P(6)$ and $P(7)$, she can conclude that
$P(n)$ holds for all $n \ge 5$.  But $P(5)$, $P(6)$ and $P(7)$ can
also follow from proving $P$ of any three nonnegative integers up to~7 that
leave remainders on division by~3 equal to each of 0, 1 and 2.
\end{solution}

\eparts

\end{problem}

\endinput
