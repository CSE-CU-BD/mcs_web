\documentclass[problem]{mcs}

\begin{pcomments}
  \pcomment{from: S09.ps6}
  \pcomment{from: S08.ps6}
  \pcomment{from: S06.ps4 (adapted)}
\end{pcomments}

\pkeywords{
  digraphs
  DAGs
  topological_sort
  transitive_closure
  paths
}

%%%%%%%%%%%%%%%%%%%%%%%%%%%%%%%%%%%%%%%%%%%%%%%%%%%%%%%%%%%%%%%%%%%%%
% Problem starts here
%%%%%%%%%%%%%%%%%%%%%%%%%%%%%%%%%%%%%%%%%%%%%%%%%%%%%%%%%%%%%%%%%%%%%

\begin{problem}
  Vertices $u, v$ in a digraph are said to be \emph{connected} when there
  is a directed path (in either direction) between $u$ and $v$.

The following procedure can be applied to any digraph, $G$:

Pick any two vertices $u, v$ of $G$.
\begin{enumerate}

\item\label{del} If there is an edge $(u,v)$ of $G$ and there is also a
directed path connecting $u$ and $v$ (in either direction) which does
\emph{not} traverse this edge, then delete the edge $(u,v)$, or

\item\label{add} if $u$ and $v$ are not connected, then add the edge
$(u,v)$.
\end{enumerate}
Repeat these operations until it is no longer possible to find two
vertices $u \neq v$ to which an operation applies.

This procedure can be modeled as a state machine.  The start state is
$G$, and the states are all possible digraphs with the same vertices as
$G$.

\bparts

\ppart Let $G$ be the graph with vertices $\set{1,2,3,4}$ and edges
\[
\set{(1,2),(3,4)}
\]
What are the possible states reachable in one step from start state $G$?
What are the possible final states reachable from $G$?

\solution{It's not possible to delete any edge.  At the first step, the
procedure can only add an edge $(i,j)$ or $(j,i)$, where $i \in \set{1,2}$
and $j \in \set{3,4}$.

The possible final states are all graphs consisting of a single directed
path, where 1 precedes 2 and 3 precedes 4.  The possible such paths are
1234, 1324, 1342, 3124, 3142, 3412.}

\eparts

To prove termination, let's say that two vertices are \term{undirected}
iff either they are unconnected or they lie on the same directed cycle.
Otherwise they are \term{directed}.

\bparts

\ppart For any state, $G'$, let $c$ be its number of directed cycles, 
$e$ be its number of edges, and $p$ its
number of sets of two undirected vertices.  Show that $(c,p,e)$ is a
strictly decreasing derived variable under lexicographic order.  Conclude
that the procedure terminates started on any finite digraph, $G$.

\solution{
Since $G$ is finite, the value of the derived variable $(c,p,e)$ is
a trio of nonnegative integers.  We will show that $(c,p,e)$ is strictly
decreasing under lexicographic order.  Termination then follows from the
fact that lexicographic order is well-founded.

There are two cases depending on the operation that was used.

\textbf{Case} \emph{Operation~\ref{del}}: Since $e$ decreases by~1, it is
enough to show that $c$ stays the same or decreases, and $p$ stays the same 
or decreases if $c$ stays the same.  This will follow if we show that .  
If our edge removes a cycle, then $c$ will decrease, so we can confine our 
remaining cases to when the edge removed does not remove a cycle.  But 
deleting an edge creates no new cycles, so there are no new pairs that are
undirected because they both lie on the same new cycle.  Also, since an
edge between $u$ and $v$ gets deleted only if a directed path between them
remains after the deletion, no new unconnected pairs are created
since there was no cycle containing the edge removed.  So any
pair that was undirected before the edge deletion remains undirected after
deletion.

\textbf{Case} \emph{Operation~\ref{add}}: Since $e$ increases by~1, we
must show that $p$ decreases while $c$ stays the same or that $c$ decreases
. In other words, $G'$ has strictly fewer undirected pairs of vertices.

Now since $u$ and $v$ are unconnected in $G$, they are an undirected pair.
Also, adding an edge $\diredge{u}{v}$ cannot create a new cycle in $G'$.
This follows because any new cycle would have to traverse the new edge,
and the rest of the cycle would be a directed path from $v$ to $u$ in $G$,
contradicting the fact that $u$ and $v$ are unconnected in $G$  This shows 
that $c$ will remain constant, so it only remains to show that $p$ decreases
.  The added edge between $u$ and $v$ makes them a directed pair in $G'$ 
since it does not create any cycles.  Moreover, since the new edge creates 
no new cycle, it creates no new undirected pairs that lie on a cycle.  
And of course adding an edge cannot create any new unconnected pairs.  Hence,
 adding the edge creates no new undirected pairs and removes at least one 
undirected pair, and so reduces the number of undirected pairs by at least one.}

\ppart Let an edge be called a generalized covering edge if removing that edge
leaves no path in either direction between its endpoints.  Let g be the number
of generalized coverging edges in our graph.  Give an alternative termination proof using the derived variable
$v+e-2g$.   Briefly indicate why this proof is more informative than the
previous one using lexicographic order. (\textbf{Correction: this answer relies on
 $G$ being a DAG, which we did not assume.  The fix using a different derived variable
will be available before the end of Spring Break.})

\solution{
We let $G$ and $G'$ be as in the previous part and distinguish the same
cases.

\textbf{Case} \emph{Operation~\ref{del}}: We delete an edge, so $e$ decreases
by 1 and $v$ remains the same, so it suffices to show that $g$ remains the same.
By our definition of the deletion operation, we only delete an edge if another path
will remain between its endpoints after it is removed, so the deleted edge cannot
be a generalized covering edge.

\textbf{Case} \emph{Operation~\ref{add}}: We add an edge, so $e$ increases by 1 and
again $v$ remains the same, meaning we must show that $g$ also increases by 1 to show
strict decrease of $v+e-2g$.  We only add an edge if there is no connection between our
two endpoints, so by definition the added edge will be a generalized covering edge.

Hence, $v+e-2g$ is strictly decreasing, nonnegative integer valued derived
variable, which implies the procedure always terminates.

This proof is more informative because the $v$, $e$ and $g$ for a starting
graph give us the information that the procedure will terminate in at most
$v+e-2g$ steps.  The argument using well-founded of lexicographic order gave
no bound on the number of steps until termination.}

\eparts

A \emph{line graph} is a graph whose edges can all be traversed by a
simple path.  For example, the two-ended graph below is a line graph of
length 4.

\mfigure{!}{0.75in}{figures/ps4-path}

\bparts
\ppart\label{lineg}
Prove that for any finite, $G$, the procedure terminates with a line graph
with the same vertices as $G$.

\solution{To prove that $H$ is a line graph, it is sufficient to prove
  that (1)~$H$ is acyclic, (2)~$H$ is connected, and (3)~every vertex in
  $H$ has in-degree and out-degree at most ~1. (Why is this sufficient?)
  The proof follows from the definition of the termination condition.

  In order for the procedure to terminate, there are no two vertices $u
  \neq v$ to which an operation applies. We prove each property by
  contradiction.
\begin{enumerate}

\item Assume $H$ has a cycle at termination. Then there exist two adjacent
  vertices $u$, $v$, with an edge $(u,v)$ between them. By the definition
  of cycle, there is also another path $\pi$ from $v$ to $u$ that does not
  traverse $(u,v)$, so \emph{Operation~\ref{del}} applies. This
  contradicts the condition for termination, so $H$ is acyclic.

\item Assume $H$ is not connected at termination. Then by definition of
  connected, there exist two vertices $u, v$ that are not connected by a
  directed path, so \emph{Operation~\ref{add}} applies. This contradicts
  the condition for termination, so $H$ is connected.

\item Assume there exists a vertex in $H$ with in-degree or out-degree
  greater than 1. If the vertex $u$ has in-degree of 2 or more, then $H$
  contains at least two distinct edges $(u,v)$ and $(u',v)$, where there
  exists a path $\pi$ between $u$ and $u'$ by Claim~(2).

\begin{itemize}

\item If $\pi$ goes from $u$ to $u'$ through $v$, then there is a cycle
  $v,\ldots,u',v$, contradicting Claim~(1).

\item Otherwise, $\pi$ does not visit $v$ and there is a path
  $u,\ldots,u',v$ that does not traverse $(u,v)$, so
  \emph{Operation~\ref{del}} applies, contradicting the condition for
  termination.
\end{itemize}

A similar argument applies for out-degree of 2 or more. Therefore, every
vertex in $H$ has in-degree and out-degree of at most 1.
\end{enumerate}

}

\ppart Prove that $G$ being a DAG is a preserved invariant of the procedure.

\solution{Deleting an edge cannot create a cycle, and neither can adding
  an edge between unconnected vertices.  So if there was no cycle in $G$,
  there wouldn't be any after one state transition.  }

\ppart Prove that if $G$ is a DAG, then the path relation of the final line
graph is a topological sort of $G$.

\hint Verify that the predicate
\[
P(u,v)\eqdef \text{there is a directed path from $u$ to $v$}
\]
is a preserved invariant of the procedure, for any two vertices $u,v$ of a DAG.

\solution{
\begin{proof}
  To prove $P(u,v)$ is an invariant, suppose $P(u,v)$ holds in $G$ and
  consider the two operations that the procedure might perform:

  \textbf{Case} \emph{Operation~\ref{del}}: Suppose the edge-deletion
  operation~\ref{del} is applied to an edge $(x,y)$ in $G$, yielding a
  digraph $G'$.

Now by definition of the condition for applying the edge-deletion
operation, there must be a directed path, $\pi$, between $x$ and $y$ which
is in both $G$ and $G'$. This directed path must start at $x$ and end at
$y$ (if it went from $y$ to $x$, together with $(x,y)$ it would have
formed a directed cycle $(y,\ldots,x,y)$ in $G$, contrary to the fact that
$G$ is a DAG).

\begin{itemize}

\item If the path from $u$ to $v$ does not traverse $(x,y)$, then it
  remains in $G'$, proving that $P(u,v)$ still holds in $G'$.

\item If the path from $u$ to $v$ \emph{does} traverse $(x,y)$, then
  removing $(x,y)$ and replacing it with the other path $\pi$ yields a
  path in $G'$ from $u$ to $v$, proving that $P(u,v)$ still holds in $G'$.
\end{itemize}

This completes the proof that $P$ is preserved by the edge-deletion
operation~\ref{del}.

\textbf{Case} \emph{Operation~\ref{add}}: Adding an $(x,y)$ to a graph
preserves all previously existing paths, so $P(u,v)$ will hold in $G'$.
So the edge adding operation~\ref{add} preserves $P$

We conclude that $P(u,v)$ is indeed a preserved invariant of the
procedure.

From part~\eqref{lineg}, the final graph is a line graph, which implies
that its path relation, $L^*$, is a total order.  Also, if $G$ is a DAG,
then its path relation, $G^*$, is a partial order.  Then $L^*$ is a
topological sort of $G^*$ means that $u\,G^*\,v$ implies $u\,L^*\,v$ for
all vertices $u,v$.

But $u\,G^*\,v$ is equivalent to $P(u,v)$ holding in $G$.  So by
invariance, $u\,G^*\,v$ implies $P(u,v)$ holds in the final line graph,
that is, $u\,L^*\,v$, as required.
\end{proof}}

\eparts
\end{problem}

%%%%%%%%%%%%%%%%%%%%%%%%%%%%%%%%%%%%%%%%%%%%%%%%%%%%%%%%%%%%%%%%%%%%%
% Problem ends here
%%%%%%%%%%%%%%%%%%%%%%%%%%%%%%%%%%%%%%%%%%%%%%%%%%%%%%%%%%%%%%%%%%%%%
