\documentclass[problem]{mcs}

\begin{pcomments}
  \pcomment{MQ_graph_state_machine_f15}
  \pcomment{variant of MQ_graph_state_machine}
  \pcomment{Proposed by ZDz for Miderm 3 in F15}
\end{pcomments}

\pkeywords{
graphs
state_machines
connectivity
well_order
termination
partial_correctness
}

%%%%%%%%%%%%%%%%%%%%%%%%%%%%%%%%%%%%%%%%%%%%%%%%%%%%%%%%%%%%%%%%%%%%%
% Problem starts here
%%%%%%%%%%%%%%%%%%%%%%%%%%%%%%%%%%%%%%%%%%%%%%%%%%%%%%%%%%%%%%%%%%%%%

\begin{problem}
Starting with some \textbf{simple graph} $G$ with two or more
vertices, keep applying the following operations: pick two vertices $u
\neq v$ such that either

\renewcommand{\theenumi}{\roman{enumi}}
\renewcommand{\labelenumi}{(\theenumi)}

\begin{enumerate}

\item\label{deleteedgeuv} there is an edge between $u$ and $v$, and there is also a path
from $u$ to $v$ which does \emph{not} include this edge; in this case,
delete the edge $\edge{u}{v}$.

\item\label{addedgeuv} there is no path from $u$ to $v$; in this case, add the edge
  $\edge{u}{v}$.
\end{enumerate}
Continue until it is no longer possible to find two vertices $u \neq
v$ to which an operation applies.

\bparts

\ppart Let $c$ be the number of connected components and $e$ the
number of edges of a simple graph.  Prove that 
\[
3c+2e
\]
is a strictly decreasing derived variable for this process.

\examspace[2.5in]

\begin{solution}
Operation~\eqref{deleteedgeuv} decreases $e$ by one and leaves $c$
unchanged since $u$ and $v$ remain connected.  So $3c+2e$ decreases by
two.

Operation~\eqref{addedgeuv} decreases $c$ by one since the two
connected components it connects become a single component.  It
increases $e$ by one.  So $3c+2e$ changes by $(3 \cdot -1) +2 = -1$,
that is, it decreases by one.
\end{solution}

\ppart Explain why, starting from any finite simple graph, the
procedure above terminates.

\examspace[0.75in]

\begin{solution}
Since each step decreases $3c+2e$ by one, the largest possible number
of steps before termination is the initial value of $3c+2e$.

More abstractly, $3c+2e$ is a nonnegative integer-valued, strictly
decreasing derived variable, so Theorem~\bref{th:decr} implies
termination.
\end{solution}

\ppart Prove that when the procedure terminates, the remaining graph is a tree.

\examspace[2in]

\begin{solution}
We use the characterization of a tree as a cycle-free, connected,
simple graph.

A final graph must be connected, because otherwise there would be two
vertices with no path between them, and then a transition adding the edge
between them would be possible.

A final graph can't have a cycle, because deleting any edge on the
cycle would be a possible operation.
\end{solution}

\eparts
\end{problem}

%%%%%%%%%%%%%%%%%%%%%%%%%%%%%%%%%%%%%%%%%%%%%%%%%%%%%%%%%%%%%%%%%%%%%
% Problem ends here
%%%%%%%%%%%%%%%%%%%%%%%%%%%%%%%%%%%%%%%%%%%%%%%%%%%%%%%%%%%%%%%%%%%%%

\endinput
