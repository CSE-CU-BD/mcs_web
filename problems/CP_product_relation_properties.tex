\documentclass[problem]{mcs}

\begin{pcomments}
  \pcomment{CP_product_relation_properties}
  \pcomment{quickie and remarked in partial order notes}
\end{pcomments}

\pkeywords{
  product_relation
  irreflexive
  transitivity
  reflexivity
  antisymmetry
  preserved
}

%%%%%%%%%%%%%%%%%%%%%%%%%%%%%%%%%%%%%%%%%%%%%%%%%%%%%%%%%%%%%%%%%%%%%
% Problem starts here
%%%%%%%%%%%%%%%%%%%%%%%%%%%%%%%%%%%%%%%%%%%%%%%%%%%%%%%%%%%%%%%%%%%%%

\begin{problem}
  Let $R_1$, $R_2$ be binary relations on the same set $A$.  A relational
  property is preserved under product, if $R_1 \times R_2$ has the
  property whenever both $R_1$ and $R_2$ have the property.

\bparts

\ppart Verify that each of the following properties are preserved under
product.
\begin{enumerate}

\item reflexivity,

\item antisymmetry,

\item transitivity.             %

\end{enumerate}

\begin{solution}
These facts follows directly from the definitions.  We'll write out
just the case of antisymmetry.  So suppose $R_1,R_2$ are antisymmetric.

\begin{proof}
To prove $R_1 \cross R_2$ is antisymmetric, suppose
\begin{align}
(r_1,r_2) \mrel{[R_1 \cross R_2]} (s_1,s_2) \quad\text{ and also}\label{r1r2R}\\
(s_1,s_2) \mrel{[R_1 \cross R_2]} (r_1,r_2).\label{s1s2R}
\end{align}
We need to show that $(r_1,s_1)=(r_2,s_2)$.

By~\eqref{r1r2R} and the definition of $R_1 \cross R_2$, we know that
$r_i \mrel{R_i} s_i$ for $i=1,2$.  Similarly, by~\eqref{r1r2R} $s_i
\mrel{R_i} r_i$.  Since $R_i$ is antisymmetric, it follows that $r_i =
s_i$ for $i=1,2$.  That is, $(r_1,s_1)=(r_2,s_2)$.
\end{proof}

\end{solution}

\ppart Verify that if \emph{either} of $R_1$ or $R_2$ is irreflexive,
then so is $R_1 \cross R_2$.

\begin{solution}
We may as well assume $R_1$ is irreflexive.  This means that
$\QNOT(r_1\mrel{R_1}r_1)$ for every $r_1 \in \domain{R_1}$.  So by
definition of relational product,
\[
\QNOT[(r_1,r_2) \mrel{[R_1 \cross R_2]} (r_1,s_2)]
\]
for all $r_1 \in \domain{R_1}$ and $r_2,s_2 \in \domain{R_2}$.  In particular
\[
\QNOT[(r_1,r_2) \mrel{[R_1 \cross R_2]} (r_1,r_2)],
\]
which implies that $R_1 \cross R_2$ is irreflexive.
\end{solution}
\eparts

Note that it now follows immediately that if if $R_1$ and $R_2$ are
partial orders and at least one of them is strict, then $R_1 \cross
R_2$ is a strict partial order.

\end{problem}


%%%%%%%%%%%%%%%%%%%%%%%%%%%%%%%%%%%%%%%%%%%%%%%%%%%%%%%%%%%%%%%%%%%%%
% Problem ends here
%%%%%%%%%%%%%%%%%%%%%%%%%%%%%%%%%%%%%%%%%%%%%%%%%%%%%%%%%%%%%%%%%%%%%

\endinput
