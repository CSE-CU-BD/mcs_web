\documentclass[problem]{mcs}
\begin{pcomments}
  \pcomment{PS_search_1-n}
  \pcomment{part(a) similar to FP_unique_search_tree}
  \pcomment{ARM 10/24/17}
\end{pcomments}

\pkeywords{
  trees
  binary_tree
  search_tree
  insertion
  deletion
  }

\newcommand{\nlbla}[1]{\text{num}_{\text{alt}}{#1}}
\newcommand{\nlblsa}[1]{\text{nums}_{\text{alt}}{#1}}

%%%%%%%%%%%%%%%%%%%%%%%%%%%%%%%%%%%%%%%%%%%%%%%%%%%%%%%%%%%%%%%%%%%%%
% Problem starts here
% %%%%%%%%%%%%%%%%%%%%%%%%%%%%%%%%%%%%%%%%%%%%%%%%%%%%%%%%%%%%%%%%%%%%

\begin{problem}

\bparts

\ppart Edit the labels in this size 11 tree $T$ so it becomes a
search tree for the set of labels $\Zintv{1}{11}$.

\begin{center}
\begin{verbatim}
                        T              
                       / \             
                      /   \            
                     A     B
                    / \   / \          
                   /   \ E   F
                  C     D
                 / \   / \
                G   H I   J
\end{verbatim}         
\end{center}

%\examspace[2in]

\begin{solution}

\begin{center}
\noindent
\begin{verbatim}
                          8       
                         / \      
                        /   \     
                       4     10    
                      / \   / \   
                     /   \ 9  11
                    2     6       
                   / \   / \      
                  1   3 5   7      
\end{verbatim}         
\end{center}
\end{solution}

\instatements{Reminder:

\begin{definition*}%\label{defsearchtree}
The Search trees $T \in \brnch$ are defined recursively as follows:

\inductioncase{Base case}: ($T \in \leafset$).  $T$ is a Search tree.

\inductioncase{Constructor case}: ($T \in \brnchng$).
If $\leftsub{(T)}$ and $\rightsub{(T)}$ are both Search trees, and
\[
\text{max}(\leftsub{(T)}) < \nlbl{(T)} < \text{min}(\rightsub{(T)}),
\]
then $T$ is a Search tree.
\end{definition*}
}

\ppart Let $T$ be a search tree whose labels are the integers in the
interval $\Zintv{1}{n}$ where $n = \sz{T}$.  Prove by structural
induction on the definition of search tree
\inbook{(Definition~\bref{defsearchtree})} that the leaves of $T$ are
precisely the subtrees whose label is an odd number:
\begin{equation}\tag{odd}
\text{leaves}(T) = \set{S \in \subbrn{T} \suchthat \nlbl{(S)}\text{ is odd}}.
\end{equation}

\inhandout{You may use the fact that the size of a search tree is an
  odd number (see Problem~\bref{CP_leaves_vs_nodes}).  You may also
  assume that if $T$ is a search tree under some labelling, then it
  remains a search tree when the same constant $r \in \reals$ is added
  to each label.}

\begin{solution}
\begin{proof}
The induction hypothesis is $P(T) \eqdef \text{equation~(odd)}$.

\inductioncase{Base case}: ($T \in \leafset$).  Since $T$ is a search
tree for $\Zintv{1}{1}$, the label $\nlbl{(T)}$ equals 1, which is an
odd number.

\inductioncase{Constructor case}: ($T \in \brnchng$).  $T$ is a search
tree for $\Zintv{1}{\sz{T}}$, so by definition of search tree,
\begin{align*}
\leftsub{(T)} & \text{ is a search tree for labels } \Zintv{1}{\sz{\leftsub{(T)}}},\\
\nlbl{(T)} & = \sz{\leftsub{(T)}} + 1,\\
\rightsub{(T)} & \text{ is a search tree for labels }
      \Zintv{(\nlbl{(T)}+1)}{(\nlbl{(T)}+\sz{\rightsub{(T)}})}.\\
\end{align*}
By induction hypothesis, the leaves of $\leftsub{(T)}$ are the odd
numbers in $\Zintv{1}{\sz{\leftsub{(T)}}}$.  Also,
$\sz{\leftsub{(T)}}$ is odd, so $\nlbl{(T)}$ is even.

Since $\rightsub{(T)}$ is a search tree, it remains a search tree if
$-\nlbl{(T)}$ is added to each label, that is, it becomes a search
tree on $\Zintv{1}{\sz{\rightsub{(T)}}}$.  By induction hypothesis,
under this labelling its leaves are labelled with the odd numbers in
this interval.  Hence the leaves of $\rightsub{(T)}$ must originally
have been labelled with the odd numbers in
$\Zintv{(\nlbl{(T)}+1)}{(\nlbl{(T)}+\sz{\rightsub{(T)}})}$.
This implies
that the labels of the leaves of $T$ are the odd numbers in
\begin{align*}
\Zintv{1}{\sz{\leftsub{(T)}}}
          & \union \Zintv{(\nlbl{(T)}+1)}{(\nlbl{(T)}+\sz{\rightsub{(T)}})}\\
          & = \Zintv{1}{\sz{T}} - \set{\nlbl{(T)}}.
\end{align*}
But since $\nlbl{(T)}$ is even, these are the odd numbers in
$\Zintv{1}{\sz{T}}$ which proves $P(T)$.
\end{proof}

\end{solution}

\inhandout{\textbf{(Continued on next page)}}

\examspace

\ppart Let $T$ be a search tree of size $n$ with labels
$\Zintv{1}{n}$.  Deleting the value 1 and inserting $n+1$ yields the
updated set of labels $\Zintv{2}{(n+1)}$.  A search tree for these
updated values can be obtained by adding 1 to each of the labels in
$T$.  Let $U$ be any search tree that matches this
description.\footnote{More precisely, $U$ is any search tree with
  labels $\Zintv{2}{(n+1)}$ that is isomorphic to $T$.}  Explain why
$T$ and $U$ have no leaves in common and therefore have no shared
subtree.

\begin{solution}
By equation~(odd), the leaves of $T$ all have odd number labels, so
the leaves of $U$ all have even number labels.  Therefore, $T$ and $U$
have no leaves in common.  If they had a subtree in common, they would
have its leaves in common, so they can't share any subtree.
\end{solution}

\eparts

\inhandout{\begin{staffnotes}
So if we tried to maintain the shape of a search tree $T$, then after
just one deletion and one insertion we might have to find $\sz{T}$ new
subtrees.  This contrasts dramatically with the case of AVL trees,
where by allowing the shape of $U$ to differ somewhat from the shape
of $T$, the search trees $T$ and $U$ can share all but proportional to
$\log_2(\sz{T})$ subtrees.
\end{staffnotes}
}

\inbook{So if we tried to maintain the shape of a search tree $T$, then after
just one deletion and one insertion we might have to find $\sz{T}$ new
subtrees.  This contrasts dramatically with the case of AVL trees,
where by allowing the shape of $U$ to differ somewhat from the shape
of $T$, the search trees $T$ and $U$ can share all but proportional to
$\log_2(\sz{T})$ subtrees.}

\end{problem}

%%%%%%%%%%%%%%%%%%%%%%%%%%%%%%%%%%%%%%%%%%%%%%%%%%%%%%%%%%%%%%%%%%%%%
% Problem ends here
%%%%%%%%%%%%%%%%%%%%%%%%%%%%%%%%%%%%%%%%%%%%%%%%%%%%%%%%%%%%%%%%%%%%%

\endinput


