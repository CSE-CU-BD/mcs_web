\documentclass[problem]{mcs}

\begin{pcomments}
  \pcomment{FP_inverse17mod29}
  \pcomment{variant of S11.ps4.prob2}
  \pcomment{by Tigran Sloyan 5/12/11}
\end{pcomments}

\pkeywords{
  number_theory
  Pulverizer
  modular_arithmetic
  inverses
  Fermat_theorem
  remainder
}

%%%%%%%%%%%%%%%%%%%%%%%%%%%%%%%%%%%%%%%%%%%%%%%%%%%%%%%%%%%%%%%%%%%%%
% Problem starts here
%%%%%%%%%%%%%%%%%%%%%%%%%%%%%%%%%%%%%%%%%%%%%%%%%%%%%%%%%%%%%%%%%%%%%

\begin{problem}
Find the inverse of 17 modulo 29 in the interval $\Zintv{1}{28}$.

\begin{solution}
We first use the Pulverizer to find $s,t$ such that $\gcd(17,29) = s\cdot 17 +
t\cdot 29$, namely,
\[
1 = 12 \cdot 17 - 7 \cdot 29.
\]
This implies that $s = 12$ is an inverse of 17 modulo 29.

Let $a=29, b=17$.  Here is the Pulverizer calculation:
\[
\begin{array}{ccccrcl}
x & \quad & y & \quad & \rem{x}{y} & = & x - q \cdot y \\ \hline
29 && 17 && 12  & = &   a - b \\
17 && 12 && 5   & = &   b - 12 \\
&&&&            & = &   b - (a - b)\\
&&&&            & = &  (-1)\cdot a + 2 \cdot b\\
12 && 5  && 2   & = &   12 - 2 \cdot 5 \\
&&&&            & = &   (a - b) - 2 \cdot ((-1)\cdot a + 2 \cdot b))\\
&&&&            & = &  3 \cdot a - 5 \cdot b  \\
5  && 2  && 1   & = &  5 - 2 \cdot 2 \\
&&&&            & = &  (-1)\cdot a + 2 \cdot b - 2\cdot(3 \cdot a - 5 \cdot b) \\
&&&&            & = &  -7 \cdot a + 12 \cdot b \\
2 && 1 && 0     & & \\
\end{array}
\]

So the inverse is \textbf{12}.
\end{solution}

\end{problem}

%%%%%%%%%%%%%%%%%%%%%%%%%%%%%%%%%%%%%%%%%%%%%%%%%%%%%%%%%%%%%%%%%%%%%
% Problem ends here
%%%%%%%%%%%%%%%%%%%%%%%%%%%%%%%%%%%%%%%%%%%%%%%%%%%%%%%%%%%%%%%%%%%%%

\endinput
