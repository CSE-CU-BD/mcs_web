\documentclass[problem]{mcs}

\begin{pcomments}
  \pcomment{from: S09.cp7m}
\end{pcomments}

\pkeywords{
  planar_graphs
  Eulers_formula
}

%%%%%%%%%%%%%%%%%%%%%%%%%%%%%%%%%%%%%%%%%%%%%%%%%%%%%%%%%%%%%%%%%%%%%
% Problem starts here
%%%%%%%%%%%%%%%%%%%%%%%%%%%%%%%%%%%%%%%%%%%%%%%%%%%%%%%%%%%%%%%%%%%%%

\begin{problem}

\bparts

\ppart Show that if a connected planar graph with more than two vertices
is bipartite, then
\begin{equation}\label{2v}
e \leq 2v -4.
\end{equation}

\hint Similar to the proof that $e \leq 3v-6$\inhandout{ (see the Appendix)}.  
Use Problem~\ref{structind}.

\solution{By Problem~\ref{structind}.\ref{structind-face-length}, every
face is of length at least 3.  But all cycles in a bipartite graph are of
even length, and so every face of an embedding must be of length at least
4.

Each edge is traversed by exactly two faces, so
\begin{equation}\label{4f}
2e = \sum_{f \in\text{ faces}} \text{length}(f) \geq
\sum_{f \in\text{ faces}} 4 = 4f.
\end{equation}
By Euler's formula, $f = e-v+2$, so
substituting for $f$ in~\eqref{4f}, yields
\[
2e \geq 4(e-v+2),
\]
which simplifies to~\eqref{2v}.
}

\ppart Conclude that that $K_{3,3}$ is not planar.  ($K_{3,3}$ is the
graph with six vertices and an edge from each of the first three vertices
to each of the last three.)

\solution{$K_{3,3}$ is bipartite and connected. Also, it has 9 edges and 6
vertices, and since $9 > 8 = 2 \cdot 6 - 4$, it does not
satisfy~\eqref{2v}, and so cannot be planar.}

\eparts

\end{problem}

%%%%%%%%%%%%%%%%%%%%%%%%%%%%%%%%%%%%%%%%%%%%%%%%%%%%%%%%%%%%%%%%%%%%%
% Problem ends here
%%%%%%%%%%%%%%%%%%%%%%%%%%%%%%%%%%%%%%%%%%%%%%%%%%%%%%%%%%%%%%%%%%%%%
