\documentclass[problem]{mcs}

\begin{pcomments}
    \pcomment{TP_Practice_with_Bounds}
    \pcomment{Converted from markov.scm
              by scmtotex and dmj
              on Sun 13 Jun 2010 05:11:14 PM EDT}
    \pcomment{edited by drewe 11.07.2011}
\end{pcomments}

\begin{problem}

%% type: short-answer
%% title: Practice with Bounds

Suppose 120 students take a final exam and the mean of their scores is
90.  You have no other information about the students and the exam,
that is, you should not assume that the highest possible score is 100.
You may, however, assume that exam scores are nonnegative.

\bparts

\ppart

State the best possible upper bound on the number of students who
scored at least 180.

\begin{solution}

60

Let $R$ be the score of a student chosen at random.  According to
Markov's Theorem:

\begin{equation*}
    \prob{R \ge 180} \le \expect{R}/180 = 90/180 = 1/2.
\end{equation*}
So at most $(1/2) \cdot 120 = 60$ students scored greater than or
equal to 180.
\end{solution}

\ppart

Now suppose somebody tells you that the lowest score on the exam
is 30.  Compute the new best possible upper bound on the number of
students who scored at least 180.

\begin{solution}

48

Let $R$ be the same as before. We can apply Markov's Theorem to the
variable $R-30$:

\begin{equation*}
    \pr{R \ge 180} = \pr{R-30 \ge 150} \le \Ex[R-30]/150 = 60/150 = 2/5. 
\end{equation*}
So at most $(2/5) \cdot 120 = 48$ students scored 180 or higher.
\end{solution}

\eparts

\end{problem}

\endinput
