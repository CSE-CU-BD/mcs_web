\documentclass[problem]{mcs}

\begin{pcomments}
  \pcomment{CP_covering_edges}
  \pcomment{from: S09.cp7r. S07.cp7r}
  \pcomment{edited by ARM 10/17/09}
\end{pcomments}

\pkeywords{
  relations
  digraphs
  covering_edges
  transitive_closure
  DAGs
}

%%%%%%%%%%%%%%%%%%%%%%%%%%%%%%%%%%%%%%%%%%%%%%%%%%%%%%%%%%%%%%%%%%%%%
% Problem starts here
%%%%%%%%%%%%%%%%%%%%%%%%%%%%%%%%%%%%%%%%%%%%%%%%%%%%%%%%%%%%%%%%%%%%%

\begin{problem}
Let $D$ be a finite Directed Acyclic Graph (DAG),

\bparts

\ppart Explain in one (maybe two) sentences why the positive path relation
of $D$ is obviously a strict partial order.
\eparts

If $a$ and $b$ are distinct nodes of a digraph, then $a$ is said to
\emph{cover} $b$ if there is an edge from $a$ to $b$ and every path from
$a$ to $b$ traverses this edge.  If $a$ covers $b$, the edge from $a$ to
$b$ is called a covering edge.

\bparts

\ppart What are the covering edges in the following DAG?

%\mfigure{!}{3in}{div-dag}

\mfigure{!}{2.5in}{divisibility}

\ppart Describe two graphs with vertices $\set{1,2}$ which have the same
set of covering edges, but not the same positive path relation (\hint They
can't be DAG's.)

\begin{solution}
Let one graph have edges $\set{(1,2), (1,1)}$ and the other
$\set{(1,1),(2,1)}$.  They have the same set of covering edges, namely,
none.  The reason is that if a vertex is on a positive length cycle, then
no edge incident to it can be a covering edge.
\end{solution}

\ppart\label{+path=} Show that if two DAG's have the same positive path
relation, then they have the same set of covering edges.

\begin{solution}
Suppose two DAG's have the same positive path relation, and
consider any covering edge in the first DAG.  By definition of covering,
it is the unique path between its endpoints in \emph{both} DAG's, since
they agree on positive length paths.  But this implies it is a covering
edge in the second DAG as well.
\end{solution}

\newcommand{\covering}[1]{\text{covering}\paren{#1}}

\ppart\label{cover-ok} Let $\covering{D}$ be the subgraph of $D$ consisting
of only the covering edges.  Explain why $\covering{D}$ has the same
positive path relation as $D$.

\hint Consider \emph{longest} paths between a pair of vertices.

\begin{solution}
What we need to show is that if there is a path in $D$ between
vertices $a \neq b$, then there is a path consisting only of covering
edges from $a$ to $b$.  But since $D$ is a finite DAG, there must be a
\emph{longest} path from $a$ to $b$.  Now every edge on this path must be a
covering edge or it could be replaced by a path of length 2 or more,
yielding a longer path from $a$ to $b$.
\end{solution}

\ppart Conclude $\covering{D}$ is the unique DAG with the smallest number
of edges among all digraphs with the same positive path relation as $D$.

\begin{solution}
  By part~\eqref{+path=}, any DAG with the same positive path relation as
  $D$ must contain all the edges of $\covering{D}$, so the unique
  minimality of $\covering{D}$ follows immediately from
  part~\eqref{cover-ok}.

\end{solution}

\iffalse
\ppart Show that the previous result is not true in general infinite
DAG's.

\hint Consider the DAG for the total order on the rational numbers.

\begin{solution}
In the DAG for $<$ on the $\rationals$, there are no covering
edges, so $\widehat{<}$ has no edges.
\end{solution}
\fi


\eparts
\end{problem}

%%%%%%%%%%%%%%%%%%%%%%%%%%%%%%%%%%%%%%%%%%%%%%%%%%%%%%%%%%%%%%%%%%%%%
% Problem ends here
%%%%%%%%%%%%%%%%%%%%%%%%%%%%%%%%%%%%%%%%%%%%%%%%%%%%%%%%%%%%%%%%%%%%%
