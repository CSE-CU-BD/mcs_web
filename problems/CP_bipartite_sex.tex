\documentclass[problem]{mcs}

\begin{pcomments}
  \pcomment{CP_bipartite_sex}
  \pcomment{CP version of FP_bipartite_matching_sex without perfect match parts}
  \pcomment{by ARM 3/15/11}
\end{pcomments}

\pkeywords{
  graphs
  bipartite
  matching
  Halls_theorem
  sexual
}

%%%%%%%%%%%%%%%%%%%%%%%%%%%%%%%%%%%%%%%%%%%%%%%%%%%%%%%%%%%%%%%%%%%%%
% Problem starts here
%%%%%%%%%%%%%%%%%%%%%%%%%%%%%%%%%%%%%%%%%%%%%%%%%%%%%%%%%%%%%%%%%%%%%

\begin{problem}
  A researcher analyzing data on heterosexual sexual behavior in a group
  of $m$ males and $f$ females found that within the group, the male
  average number of female partners was 10\% larger that the female
  average number of male partners.

  \bparts

  \ppart Comment on the following claim.  ``Since we're assuming that
  each encounter involves one man and one woman, the average numbers
  should be the same, so the males must be exaggerating.''

\begin{solution}
The averages won't be the same.  According to equation~(\bref{avgsexMF}),
\begin{equation}\label{avgsexMF_CP}
\text{Avg. \# male partners} = \frac{\card{F}}{\card{M}} \cdot \text{Avg. \# female partners}
\end{equation}
So the averages simply reflect the relative sizes of the male and
female populations.  This means that the males could truthfully report
a higher average if there where more females.

Of course if the males exaggerate, then their reported average could
be as large as they choose to fantasize, whatever the size of the
female population.
\end{solution}

  \ppart\label{partner-ratio_CP} For what constant $c$ is $m = c\cdot f$?

\iffalse
  \ppart\label{partner-ratio_CP} Circle all of the assertions below that are
  implied by the above information on average numbers of partners:

\renewcommand{\theenumi}{\roman{enumi}}
\renewcommand{\labelenumi}{(\theenumi)}

\begin{enumerate}
\item males exaggerate their number of female partners
\item $m = (9/10)f$
\item\label{m10.11} $m = (10/11)f$
\item $m = (11/10)f$


\item there cannot be a perfect matching with each male matched to one of
  his female partners
\item\label{no-female-match} there cannot be a perfect matching with each
  female matched to one of her male partners
\end{enumerate}
\fi

\begin{solution}
By equation~\eqref{avgsexMF_CP}, the men's average number of partners
is $f/m$ times the female's average, so $f/m = 1.1$ which implies $m =
(1/1.1)f$ and $c = 10/11$.
\end{solution}

\ppart The data shows that approximately 20\% of the females were virgins,
while only 5\% of the males were.  The researcher wonders how excluding
virgins from the population would change the averages.  If he knew graph
theory, the researcher would realize that the nonvirgin male average
number of partners will be $x(f/m)$ times the nonvirgin female average number
of partners.  What is $x$?
\iffalse

\begin{center}
\exambox{0.5in}{0.5in}{-0.2in}
\end{center}
\fi

\begin{solution}
  The male average number of partners is $f/m$ times the female
  average number of partners.  (According to part~\eqref{partner-ratio_CP},
  $f/m = 1.1$, but this number isn't needed here.)  When virgins are
  excluded, the ratio of the male's average to the females' average will
  be
\[
\frac{f - .2f}{m  - .05m} =  \frac{.8f}{.95m} = \frac{4/5}{19/20} \cdot \frac{f}{m},
\]
so $x = 80/95 = 16/19$.
\end{solution}

\ppart For purposes of further research, it would be helpful to pair
each female in the group with a unique male in the group.  Explain why
this is not possible.

\begin{solution}
There are more females than males, so there cannot be an injective
function from the females to the males.
\end{solution}
\eparts

\end{problem}

%%%%%%%%%%%%%%%%%%%%%%%%%%%%%%%%%%%%%%%%%%%%%%%%%%%%%%%%%%%%%%%%%%%%%
% Problem ends here
%%%%%%%%%%%%%%%%%%%%%%%%%%%%%%%%%%%%%%%%%%%%%%%%%%%%%%%%%%%%%%%%%%%%%

\endinput
