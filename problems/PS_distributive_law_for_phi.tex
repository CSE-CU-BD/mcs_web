\documentclass[problem]{mcs}

\begin{pcomments}
  \pcomment{from: S09.ps8}
  \pcomment{S08.ps7}
\end{pcomments}

\pkeywords{
  primes
  number_theory
  Eulers_theorem
  modular_arithmetic
  bijections
}

%%%%%%%%%%%%%%%%%%%%%%%%%%%%%%%%%%%%%%%%%%%%%%%%%%%%%%%%%%%%%%%%%%%%%
% Problem starts here
%%%%%%%%%%%%%%%%%%%%%%%%%%%%%%%%%%%%%%%%%%%%%%%%%%%%%%%%%%%%%%%%%%%%%

\begin{problem}
  Suppose $m,n$ are relatively prime.  In the problem you will prove the
  key property of Euler's function that $\phi(mn)=\phi(m)\phi(n)$.

\bparts

\ppart Prove that for any $a,b$, there is an
$x$ such that
\begin{align}
x & \equiv a \pmod{m}, \label{xa}\\
x & \equiv b \pmod{n}. \label{xb}
\end{align}

\hint Congruence~\eqref{xa} holds iff
\begin{equation}\label{xj}
x = jm + a.
\end{equation}
for some $j$.  So there is such an $x$ only if
\begin{equation}\label{jn}
j m +a \equiv b \pmod{n}.
\end{equation}
Solve~\eqref{jn} for $j$.

\solution{
\begin{proof}
\footnote{Adapted from
\href{http://www.cut-the-knot.org/blue/chinese.shtml}{\texttt{http://www.cut-the
-knot.org/blue/chinese.shtml}}.}
Since $m,n$ are relatively prime, there is an inverse, $m'$, modulo $n$ of
$m$.  So $j = m'(b-a)$ satisfies~\eqref{jn}.  Now~\eqref{xj} leads to the
definition
\[
x_1 \eqdef m'(b-a)m + a.
\]
So
\[
x_1 = (m'(b-a))m + a \equiv a \pmod{m},
\]
and
\[
x_1 = m'(b-a)m + a = m'm(b-a) +a \equiv 1\cdot(b-a) + a \equiv b \pmod{n},
\]
proving that $x_1$ satisfies the congruences~\eqref{xa} and~\eqref{xb}.
\end{proof}}

\ppart\label{x0} Prove that there is an $x$ satisfying the
congruences~\eqref{xa} and~\eqref{xb} such that $0 \leq x < mn$.

\solution{
Let
\[
x_0 \eqdef \rem{x_1}{mn},
\]
where $x_1$ satisfies~\eqref{xa} and~\eqref{xb}.

Now $0 \leq x_0 < mn$ by definition of remainder.  Further, we know $x_0
\equiv x_1 \pmod{mn}$, which immediately implies that $x_0 \equiv x_1 \pmod{m}$
and $x_0 \equiv x_1 \pmod{n}$.  So $x_0$ also satisifies~\eqref{xa}
and~\eqref{xb}, and is therefore the desired solution.
}

\ppart\label{uniq} Prove that the $x$ satisfying part~\eqref{x0} is unique.

\solution{Assume $x_0,y$ both satisfy congruences~\eqref{xa}
and~\eqref{xb}.  Taking the differences we see that
\[
x_0 - y \equiv 0 \pmod{m} \text{  and  } x_0 - y \equiv 0 \pmod{n}.
\]
So by definition, both $m$ and $n$ divide $x_0 - y$, and since $m$ and
$n$ are relatively prime, this implies $mn \divides (x_0 - y)$.  But
if $x_0$ and $y$ are both in the range $0$ to $mn-1$, then $mn >
\card{x_0 - y}$, so it must be that $y = x_0$, as required.}

\ppart For an integer $k$, let $k^*$ be the integers between $1$ and
$k-1$ that are relatively prime to $k$.  Conclude from part~(c) that
the function
\[
f: (mn)^* \rightarrow m^* \times n^*
\]
defined by
\[
f(x) \eqdef (\rem{x}{m},\rem{x}{n})
\]
is a bijection.

\solution{
For any positive integer, $k$, let
\[
[0,k) \eqdef \set{0,1,\dots, k-1}.
\]
By part~\eqref{uniq}, the mapping from $x$ to
$(\rem{x}{m},\rem{x}{n})$ is a bijection between $[0,mn)$ and $[0,m)
\cross [0,n)$.  Moreover, since $x$ is relatively prime to $mn$ iff
$x$ is relatively prime to $m$ and $x$ is relatively prime to $n$,
this mapping also defines a bijection between the integers in $[0,mn)$
that are relatively prime to $mn$ and the pairs of integers in $[0,m)
\cross [0,n)$ that are relatively prime to $m$ and $n$, respectively.
}

\ppart Conclude from the preceding parts of this problem
that
\[
\phi(mn)=\phi(m)\phi(n).
\]

\solution{ The mapping $f$ defines a bijection between numbers
  integers in $[0,mn)$ that are relatively prime to $mn$ and the pairs
  of integers in $[0,m) \cross [0,n)$ that are relatively prime to $m$
  and $n$, respectively.  In particular the number, $\phi(mn)$ of
  numbers in $[0,mn)$ that are relatively prime to $mn$ is the same as
  the number $\phi(m)\phi(n)$ of pairs of integers in $[0,m) \cross
  [0,n)$ whose first coordinate is relatively prime to $m$ and whose
  second coordinate is relatively prime to $n$.  }

\eparts
\end{problem}

%%%%%%%%%%%%%%%%%%%%%%%%%%%%%%%%%%%%%%%%%%%%%%%%%%%%%%%%%%%%%%%%%%%%%
% Problem ends here
%%%%%%%%%%%%%%%%%%%%%%%%%%%%%%%%%%%%%%%%%%%%%%%%%%%%%%%%%%%%%%%%%%%%%
