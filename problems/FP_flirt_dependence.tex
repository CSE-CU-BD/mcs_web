\documentclass[problem]{mcs}

\begin{pcomments}
  \pcomment{FP_flirt_dependence}
  \pcomment{sequel to CP_sixteen_desks}
  \pcomment{subsumed by FP_flirt_dependence_deviation}
  \pcomment{ARM 5/7/16}
\end{pcomments}

\pkeywords{
  probability
  independence
  mutual
  event
}

\newcommand{\flirt}{\text{flirt}}

%%%%%%%%%%%%%%%%%%%%%%%%%%%%%%%%%%%%%%%%%%%%%%%%%%%%%%%%%%%%%%%%%%%%%
% Problem starts here
%%%%%%%%%%%%%%%%%%%%%%%%%%%%%%%%%%%%%%%%%%%%%%%%%%%%%%%%%%%%%%%%%%%%%

\begin{problem} A classroom has sixteen desks in a $4 \times 4$ arrangement as shown below.
%
\begin{center}
\begin{picture}(300,160)
% \put(0,0){\dashbox(270,198){}} % bounding box
\multiput(0,  0)(72,0){4}{\framebox(40,25){}}
\multiput(0, 40)(72,0){4}{\framebox(40,25){}}
\multiput(0, 80)(72,0){4}{\framebox(40,25){}}
\multiput(0,120)(72,0){4}{\framebox(40,25){}}
\end{picture}
\end{center}

If two desks are next to each other, vertically or horizontally, they
are called an \emph{adjacent pair}.  So there are three horizontally
adjacent pairs in each row, for a total of twelve horizontally
adjacent pairs.  Likewise, there are twelve vertically adjacent pairs.

Boys and girls are assigned to desks mutually independently, with
probability $p > 0$ of a desk being occupied by a boy and probability
$q \eqdef 1-p > 0$ of being occupied by a girl.  An adjacent pair $D$
of desks is said to have a \emph{flirtation} when there is a boy at
one desk and a girl at the other desk.  Let $F_D$ be the event that
$D$ has a flirtation.

\bparts

\iffalse
\ppart What is the expected number of flirtations?

\begin{solution}
$\mathbf{48pq}$.

For any adjacent pair $D$,
\[
\pr{F_D} = pq+qp = 2pq,
\]
and so 
\[
\expect{F_D} = 2pq.
\]
Since there are 24 adjacent pairs, the expected number of flirtations
overall is $\expect{F_D} = 48pq$.
\end{solution}
\fi

\ppart Different pairs $D$ and $E$ of adjacent desks are said to
\emph{overlap} when they share a desk.  For example, the first and
second pairs in each row overlap, and so do the second and third
pairs, but the first and third pairs do not overlap.  Prove that if
$D$ and $E$ overlap, then $F_D$ and $F_E$ are independent events iff
$p=q$.

\examspace[1.7in]

\begin{solution}
The events are independent iff 
\[
\pr{F_D \intersect F_E} = \pr{F_D} \cdot \pr{F_E}.
\]
Now
\[
\pr{F_D} \cdot \pr{F_E} = (2pq)^2,
\]
and
\begin{align*}
\pr{F_D \intersect F_E}
 & = \pr{\text{shared desk has a boy and the other two desks have girls}} +\\
 &\quad\ \pr{\text{shared desk has a girl and the other two desks have boys}}\\
 & = pq^2 + qp^2\\
 & = pq(q+p) = pq.
\end{align*}
So the events are independent iff $pq = (2pq)^2$.  Since $p,q>0$ we
can cancel $pq$ and conclude that the events are independent iff
\begin{equation}\label{14pq}
1 = 4pq = 4(p-p^2).
\end{equation}
But~\eqref{14pq} holds iff $p = q$.  This can be verified by solving
the quadratic $p^2-p+ 1/4 = 0$ or by noting that $pq$ is maximized
when $p=q$.
\end{solution}

\ppart Find four pairs of desks $D_1,D_2,D_3,D_4$ and explain why
$F_{D_1}, F_{D_2}, F_{D_3}, F_{D_4}$ are \emph{not} mutually
independent (even if $p=q=1/2$).

\examspace[1.5in]

\begin{solution}
Let $D_1,D_2,D_3,D_4$ form a $2 \times 2$ pattern of four desks.  Then
$\bar{F_{D_1}} \intersect \bar{F_{D_2}} \intersect \bar{F_{D_3}}$
implies that all four desks must be occupied by people of the same
sex, and therefore $\bar{F_{D_4}}$ is true.  Hence, the events are not independent.

\begin{staffnotes}
Not required for full credit:
\end{staffnotes}

More formally,
\[
\prcond{\bar{F_{D_4}}}{\bar{F_{D_1}} \intersect \bar{F_{D_2}} \intersect \bar{F_{D_3}}}
= 1 \neq \pr{\bar{F_{D_4}}}.
\]

\end{solution}

\eparts

\end{problem}

%%%%%%%%%%%%%%%%%%%%%%%%%%%%%%%%%%%%%%%%%%%%%%%%%%%%%%%%%%%%%%%%%%%%%
% Problem ends here
%%%%%%%%%%%%%%%%%%%%%%%%%%%%%%%%%%%%%%%%%%%%%%%%%%%%%%%%%%%%%%%%%%%%%

\endinput
