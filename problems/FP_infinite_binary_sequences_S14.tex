\documentclass[problem]{mcs}

\begin{pcomments}
  \pcomment{FP_infinite_binary_sequences_S14}
  \pcomment{subsumes FP_infinite_binary_sequences}
  \pcomment{ARM 5/15/14}
\end{pcomments}

\pkeywords{
  countable
  union
  infinite_sequence
  uncountable
}

%%%%%%%%%%%%%%%%%%%%%%%%%%%%%%%%%%%%%%%%%%%%%%%%%%%%%%%%%%%%%%%%%%%%%
% Problem starts here
%%%%%%%%%%%%%%%%%%%%%%%%%%%%%%%%%%%%%%%%%%%%%%%%%%%%%%%%%%%%%%%%%%%%%

%\newcommand{\binw}{\set{0,1}^\omega}

\begin{problem}

\inhandout{\mbox{}
\bparts

\ppart\label{unioncount} Explain why the union of two countable sets
is countable.
\begin{editingnotes}
Same as TP\_union\_two\_countable
\end{editingnotes}

\examspace[2.0in]

\begin{solution}

\begin{proof}
Countable and nonempty means can be listed, possibly with repeats.
Suppose a list of all the elements of $A$ is $a_0,a_1,\dots$ and
a list of $B$ is $b_0,b_1, \dots$.  Then a list of all the elements
in $A \union B$ is just
\[
a_0,b_0,a_1,b_1, \dots a_n,b_n, \dots.
\]
\end{proof}
\end{solution}

\eparts
}

Let $\finbin$ be the set of finite binary sequences, $\binw$ be the
set of infinite binary sequences, and $F$ be the set of sequences in
$\binw$ that contain only a \textbf{f}inite number of occurrences of
$\mtt{1}$'s.

\bparts

\ppart\label{binwsurjbarF} Describe a simple surjective function from
$\finbin$ to $F$.

\examspace[1in]

\begin{solution}
Add an infinite sequence of $\mtt{0}$'s to the end of a finite sequence
to make it into an infinite sequence with only finitely many
$\mtt{1}$'s.  That is, $x \in \finbin$ maps to $x 0^\omega \in F$.
\end{solution}

\ppart The set $\bar{F} \eqdef \binw - F$ consists of all the infinite
binary sequences with \emph{infinitely} many $\mtt{1}$'s.  Use the
previous problem part\inhandout{s} to prove that $\bar{F}$ is
uncountable.

\hint We know that $\finbin$ is countable and $\binw$ is not.

\begin{solution}
By part~\eqref{binwsurjbarF}, we have $\finbin \surj F$, and so $F$ is
countable (Lemma~\bref{NsurjC}).  Now if $\bar{F}$ was also countable,
then by
\inbook{Lemma~\bref{TP_union_two_countable}}
\inhandout{part~\eqref{unioncount}},
$F \union \bar{F} = \binw$ would be countable, a contradiction.

There is also a simple diagonal argument that proves that $\bar{F}$ is
uncountable using a $30^o$ diagonal instead of the basic $45^o$
diagonal.  Supposing there was a list of the elements of $\bar{F}$,
define a ``diagonal'' sequence, $s$, that differs from the $n$th
sequence in the list at position $\mathbf{2}n$.  This still ensures
that $s$ is not in the list, but it leaves all the odd numbered bits
of $s$ unspecified.  So we can define all the odd numbered bits of $s$
to be 1, guaranteeing that $s$ has infinitely many 1's.

\begin{staffnotes}
Consider why the usual $45^o$ diagonal argument does not work here.
If $s'$ is the diagonal sequence that differs from the $n$th sequence
in the list at position $n$, for all $n \in \naturals$, then $s'$ is
not in the list, \emph{but there is no guarantee that $s'$ has
  infinitely many 1's}.  So the diagonalization may not yield a
``missing'' sequence, and the proof breaks down.
\end{staffnotes}
\end{solution}

\eparts

\end{problem}

%%%%%%%%%%%%%%%%%%%%%%%%%%%%%%%%%%%%%%%%%%%%%%%%%%%%%%%%%%%%%%%%%%%%%
% Problem ends here
%%%%%%%%%%%%%%%%%%%%%%%%%%%%%%%%%%%%%%%%%%%%%%%%%%%%%%%%%%%%%%%%%%%%%

\endinput
