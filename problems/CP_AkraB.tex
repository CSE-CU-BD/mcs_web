\documentclass[problem]{mcs}

\begin{pcomments}
  \pcomment{CP_AkraB}
  \pcomment{formatted, but unedited from fall04 pset7 rec11}
\end{pcomments}

\pkeywords{
  recurrence
  Akra-Bazzi
}

%%%%%%%%%%%%%%%%%%%%%%%%%%%%%%%%%%%%%%%%%%%%%%%%%%%%%%%%%%%%%%%%%%%%%
% Problem starts here
%%%%%%%%%%%%%%%%%%%%%%%%%%%%%%%%%%%%%%%%%%%%%%%%%%%%%%%%%%%%%%%%%%%%%


\begin{problem}
We have devised an error-tolerant version of%
\index{Merge Sort} 
\CMD{MergeSort}.  We call our exciting new algorithm \CMD{OverSort}.

Here is how the new algorithm works.  The input is a list of $n$
distinct numbers.  If the list contains a single number, then there is
nothing to do.  If the list contains two numbers, then we sort them
with a single comparison.  If the list contains more than two numbers,
then we perform the following sequence of steps.

\begin{itemize}
\item We make a list containing the first $\frac{2}{3}n$
numbers and sort it recursively.

\item We make a list containing the last $\frac{2}{3}n$ numbers
and sort it recursively.

\item We make a list containing the first $\frac{1}{3}n$ numbers and
the last $\frac{1}{3}n$ numbers and sort it recursively.

\item We merge the first and second lists, throwing out duplicates.

\item We merge this combined list with the third list, again throwing out
duplicates.
\end{itemize}

The final, merged list is the output.  What's great is that because
multiple copies of each number are maintained, even if the sorter
occasionally forgets about a number, \CMD{OverSort} can still output a
complete, sorted list.

\bparts

\ppart Let $T(n)$ be the maximum number of comparisons that
\CMD{OverSort} could use to sort a list of $n$ distinct numbers,
assuming the sorter never forgets a number and $n$ is a power of 3.
What is $T(3)$?  Write a recurrence relation for $T(n)$.  (\hint
Merging a list of $j$ distinct numbers and a list of $k$ distinct
numbers, and throwing out duplicates of numbers that appear in both
lists, requires $j+k-d$ comparisons, when $d>0$ is the number of
duplicates.)

\begin{solution}
When $n=3$, we begin with a list $(a,b,c)$ of 3 distinct numbers.
\CMD{OverSort} starts by forming the lists $(a,b)$, $(b,c)$, and
sorts each of them.  Sorting a list of length two takes 1 comparison,
so the total number of comparisons required to sort both lists is 2.
Next it merges these two lists.  There must be exactly one duplicate
in the two lists, so, using the Hint, we conclude that this takes
$2+2-1= 3$ more comparisons and yields a sorted length three list of
all the elements.

\CMD{OverSort} then forms the list $(a,c)$, uses 1 comparison to
sort it, and merges this sorted list of length two with the previous
length three list.  However, merging these two lists takes only
$2+3-2= 3$ comparisons, because there are two duplicates.  So $T(3)$,
the worst case number of comparisons, is $2+3+1+3=9$.

Now for $n>3$, \CMD{OverSort} will form and then sort three
different lists of length $(2/3)n$.  The first two overlap by
$(1/3)n$, so merging them takes $(2/3)n+(2/3)n -(1/3)n = n$
comparisons and yields a list of length $n$.  This list and the
remaining list of length $(2/3)n$ overlap by $(2/3)n$, so merging them
requires only another $n$ comparisons.  So
\[
T(n) = 3 T\left(\frac{2}{3}n\right) + 2n.
\]
\end{solution}

\ppart Now we're going to apply the \idx{Akra-Bazzi Theorem} to find a
$\Theta$ bound on $T(n)$.  Begin by identifying the following
constants and functions in the Akra-Bazzi
recurrence~\eqref{A-B_recurrence}:

\begin{itemize}

\item The constant $k$.

\begin{solution}
$k = 1$
\end{solution}

\item The constants $a_i$.

\begin{solution}
$a_1 = 3$
\end{solution}

\item The constants $b_i$.

\begin{solution}
$b_1 = 2/3$
\end{solution}

\item The functions $h_i$.

\begin{solution}
$h_1(x) = 0$
\end{solution}

\item The function $g$.

\begin{solution}
$g(x) = 2x$
\end{solution}

\item The constant $p$.  You can leave $p$ in terms of logarithms, but
you'll need a rough estimate of its value later on.

\begin{solution}
\begin{align*}
3 \cdot \left(\frac{2}{3}\right)^p & = 1 \\
\ln 3 + p (\ln 2 - \ln 3) & = 0 \\
p & = \frac{\ln 3}{\ln 3 - \ln 2} = 2.7095\dots
\end{align*}

\end{solution}

\end{itemize}

\ppart Does the condition $\abs{g'(x)} = O(x^c)$ for some $c \in
\naturals$ hold?

\begin{solution}
Yes.  $g'(x) = 2 = O(x^0)$.
\end{solution}

\ppart Does the condition $\abs{h_i(x)} = O(x / \log^2 x)$ hold?

\begin{solution}
Yes.  $0 = O(x / \log^2 x)$.
\end{solution}

\ppart Determine a $\Theta$ bound on $T(n)$ by integration.

\begin{solution}
\begin{align*}
T(n)	& = \Theta\left(n^p \cdot \left(1 +
		\int_1^n \frac{2u}{u^{1+p}} \ du \right)\right) \\
        & = \Theta\left(n^p \cdot \left(1 +
		2 \int_1^n \frac{1}{u^p} \ du \right)\right) \\
	& = \Theta(n^p) \\
	& = \Theta\left(n^{2.7095\dots}\right)
\end{align*}
%
Note that since $p > 1$, the integral nonnegative and bounded above by a
constant, no matter how large $n$ grows.  Thus, everything except $n^p$ is
absorbed by the $\Theta$.
\end{solution}

\eparts

\end{problem}

%%%%%%%%%%%%%%%%%%%%%%%%%%%%%%%%%%%%%%%%%%%%%%%%%%%%%%%%%%%%%%%%%%%%%
% Problem ends here
%%%%%%%%%%%%%%%%%%%%%%%%%%%%%%%%%%%%%%%%%%%%%%%%%%%%%%%%%%%%%%%%%%%%%
\endinput
