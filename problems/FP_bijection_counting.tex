\documentclass[problem]{mcs}

\begin{pcomments}
  \pcomment{FP_bijection_counting}
  \pcomment{subsumed by FP_string_inclusion_exclusion}
  \pcomment{verbatim from: S07 Final P5 with minor syntax edits}
\end{pcomments}

\pkeywords{
  combinatorics
  counting
  bijection
  inclusion-exclusion
}

%%%%%%%%%%%%%%%%%%%%%%%%%%%%%%%%%%%%%%%%%%%%%%%%%%%%%%%%%%%%%%%%%%%%%
% Problem starts here
%%%%%%%%%%%%%%%%%%%%%%%%%%%%%%%%%%%%%%%%%%%%%%%%%%%%%%%%%%%%%%%%%%%%%

\begin{problem}

\bparts

\ppart
Define a bijection between the nonnegative integers and all the even integers.

\examspace[3in]

\begin{solution}
  One such bijection $f$ is defined by lining up
  the nonnegative integers against all the even integers as follows:

  \[\begin{array}{ccccccccl}
  0 & 1  & 2 & 3 & 4  & 5  & 6 & 7 &  \dots\\
  0 & -2 & 2 & -4 & 4 & -6 & 6 & -8 & \dots
  \end{array}\]

  We can also define $f: \nngint \to \integers$ by the formula,
  \begin{align*}
  f(2k) & \eqdef 2k,\\
  f(2k+1) & \eqdef -2k-2,
  \end{align*}
  for $k \in \nngint$.  An equivalent formulation is:
  \[
  f(n) \eqdef \begin{cases}
             n, & \text{if $n$ is even};\\
            -(n+1), & \text{if $n$ is odd}.
         \end{cases}
  \]
\end{solution}

\iffalse

\ppart
How many length $n$ binary strings are there in which
\STR{011} occurs starting at the 4th position?

\begin{solution}
  $2^{n-3}$
\end{solution}
\fi

\ppart
Let $A_i$ be the set of length $n$ binary strings in which
\STR{011} occurs starting at the $i$th position.  (So $A_i$ is empty
for $i > n-2$.)  For $i < j$, the intersections $A_i \intersect A_j$ that
are nonempty are all the same size.  What is $\card{A_i \intersect A_j}$
in this case?

\examspace[1in]

\begin{solution}
  To be nonempty, the copies of \STR{011} at $i$
  and $j$ use up 6 positions, leaving $n-6$ positions that can contain any
  pattern of bits.  So $\card{A_i \intersect A_j} = 2^{n-6}$.
\end{solution}

\ppart \label{emptyij}
Let $t$ be the number of intersections $A_i \intersect
A_j$ that are nonempty, where $i < j$.  Express $t$ as a binomial
coefficient:

\examspace[1in]

\begin{solution}
  \[
  \binom{n-4}{2}.
  \]
  This is the same as asking how many ways there are to place two copies of
  \STR{011} in a length $n$ binary sequence.  Since the copies can't
  overlap, this is the same as the number of sequences of $n-6$
  indistinguishable positions for single bits and 2 indistinguishable
  positions for the copies, which, by the Bookkeeper Principle, is
  $\binom{(n-6)+2}{2}$.
\end{solution}

\ppart
How many length 9 binary strings that contain the substring
\STR{011} are there?  You should express your answer as an integer or
as a simple expression which may include the constant $t$
of part~\eqref{emptyij}.

\hint Inclusion-exclusion for $\card{\lgunion_1^7 A_i}$.

\examspace[2in]

\begin{solution}
  By Inclusion-exclusion
  \begin{equation}\label{bg}
  \card{\lgunion_1^9 A_i} = \sum_1^9 \card{A_i} - \sum_{i \neq j} \card{A_i
  \intersect A_j} + \sum_{i \neq j \neq k} \card{A_i
  \intersect A_j \intersect A_k}.
  \end{equation}

  Since $A_8=A_9=\emptyset$, there are 7 terms in the first sum
  in~\eqref{bg}, and each term is $2^{n-3}$, namely the number of patterns
  of the remaining $n-3$ bits besides the substring \STR{011}.

  There are $t$ terms in the second sum in~\eqref{bg}, each of size
  $2^{n-6}$.

  Finally, among the terms in the third sum, only $A_1 \intersect A_4
  \intersect A_7$ is nonempty, and it is of size 1, corresponding to the
  only length 9 string containing 3 occurrences of \STR{011}.  So,

  \[
  \card{\lgunion_1^9 A_i} = 7 \cdot 2^{6} - t \cdot 2^{3} +1 = 369.
  \]

  %$\binom{7}{1} 2^6 + \binom{5}{2} 2^3 - \binom{3}{3} 2^0 = 512 - 448 + 80 - 1 = 143$
\end{solution}

\eparts

\end{problem}

%%%%%%%%%%%%%%%%%%%%%%%%%%%%%%%%%%%%%%%%%%%%%%%%%%%%%%%%%%%%%%%%%%%%%
% Problem ends here
%%%%%%%%%%%%%%%%%%%%%%%%%%%%%%%%%%%%%%%%%%%%%%%%%%%%%%%%%%%%%%%%%%%%%

\endinput
