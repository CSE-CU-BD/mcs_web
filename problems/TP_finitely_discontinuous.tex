\documentclass[problem]{mcs}

\begin{pcomments}
  \pcomment{TP_finitely_discontinuous}
  \pcomment{provenence unknown, ARM revised 3/15/17}
\end{pcomments}

\pkeywords{
  countable
  finite
  continuous
  bijection
  union
  chain
}

%%%%%%%%%%%%%%%%%%%%%%%%%%%%%%%%%%%%%%%%%%%%%%%%%%%%%%%%%%%%%%%%%%%%%
% Problem starts here
%%%%%%%%%%%%%%%%%%%%%%%%%%%%%%%%%%%%%%%%%%%%%%%%%%%%%%%%%%%%%%%%%%%%%

\begin{problem}
Let $A$ to be some infinite set.  We know \inbook{from
  Lemma~\bref{AUb}} that
\begin{equation}\tag{Aub}
A \bij (A \union \set{b_0})
\end{equation}
for any element $b_0$.  An easy induction implies that
\begin{equation}\tag{$\text{Aub}_\text{i}$}
A \bij (A \union \set{b_0,b_1,\dots,b_n})
\end{equation}
for any finite set $\set{b_0,b_1,\dots,b_n}$.

Students sometimes assume that this implies that
\begin{equation}\label{AuB}
A \bij (A \union B),
\end{equation}
but it doesn't.  For example,~(AuB) is not true if $A$ is $\nngint$
and $B$ is the real numbers $\reals$.\footnote{It happens that(AuB) is
true \emph{if $B$ is countable},
  \inbook{(Problem~\bref{PS_add_countable_elements})}, but this is not
  completely obvious and takes some proving.}

A collection $\mathcal{C}$ of sets is called a \emph{chain} when, given any
two sets in $\mathcal{C}$, one is a subset of the other.  A predicate
$P(C)$ is said to be \emph{finitely continuous} if, whenever
$\mathcal{F}$ is a chain of \emph{finite} sets, and $P(F)$ is true for
every $F \in \mathcal{F}$, then $P(\lgunion \mathcal{F})$ is true.
Claiming that $\text{Aub}_\text{i}$ implies~(AuB) amounts to claiming
that the predicate $P_A(C)$ is finitely continuous, where
\[
P_A(C) \eqdef A \bij (A \union C).
\]
But it isn't, as the example with $A = \nngint$ and $C = \reals$
demonstrates.

Briefly explain which of the following predicates $P(C)$ is
finitely \textbf{c}ontinuous and which \textbf{n}ot.

\begin{enumerate}
\item $C$ is finite.
\begin{solution}
\textbf{n}
\end{solution}

\item $C$ is uncountable.
\begin{solution}
\textbf{c} vacuously, because $P(F)$ is false for all finite $F$.
\end{solution}

\item $C = \emptyset$.
\begin{solution}
\textbf{c}, because a union of empty sets is empty.
\end{solution}

\item There is a minimum element $b \in C \intersect \nngint$.
\begin{solution}
\textbf{c}, by WOP, even if $C=\emptyset$.
\end{solution}

\item There is a minimum element $b \in C \intersect \integers$.
\begin{solution}
\textbf{n}. Letting $C = \integers$ is a counter-example.
\end{solution}

\item $\pi/2 \in C$.
\begin{solution}
\textbf{c}.
\end{solution}

\item $\exists \epsilon > 0\, \forall a,b \in C \intersect \reals.\,
  \abs{a - b} > \epsilon$.
\begin{solution}
\textbf{c}: Because if $a,b$ are in the union of a chain, they must be
in at least one of the sets in the chain.
\end{solution}

\item $C \union \nngint$ is finite.
\begin{solution}
vacuously \textbf{c} since $P(F)$ is always false.
\end{solution}

\item $C \subseteq \nngint$.
\begin{solution}
\textbf{c}.  a union of subsets of a set $S$ is also a subset of $S$.
\end{solution}

\item $C \subset \nngint$.
\begin{solution}
\textbf{n}  $\nngint = \lgunion_{n\in \nngint} \Zintv{0}{n}$.
\end{solution}

\item $C$ is countable.
\begin{solution}
\textbf{c} because a union of a chain of finite sets is
countable\inbook{(Problem~\bref{TP_countable_chain})}.
\end{solution}
\end{enumerate}
\end{problem}

%%%%%%%%%%%%%%%%%%%%%%%%%%%%%%%%%%%%%%%%%%%%%%%%%%%%%%%%%%%%%%%%%%%%%
% Problem ends here
%%%%%%%%%%%%%%%%%%%%%%%%%%%%%%%%%%%%%%%%%%%%%%%%%%%%%%%%%%%%%%%%%%%%%

\endinput


