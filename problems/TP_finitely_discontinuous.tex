\documentclass[problem]{mcs}

\begin{pcomments}
  \pcomment{TP_finitely_discontinuous}
  \pcomment{draft ARM 3/17/15}
\end{pcomments}

\pkeywords{
  countable
  finite
  predicate
  bijection
  union
}

%%%%%%%%%%%%%%%%%%%%%%%%%%%%%%%%%%%%%%%%%%%%%%%%%%%%%%%%%%%%%%%%%%%%%
% Problem starts here
%%%%%%%%%%%%%%%%%%%%%%%%%%%%%%%%%%%%%%%%%%%%%%%%%%%%%%%%%%%%%%%%%%%%%

\begin{problem}
We know \inbook{from Lemma~\bref{AUb}} that if $A$ is an infinite set,
then $A \bij (A \union \set{b_0})$ for any element $b_0$.  It follows
by induction that $A \bij (A \union \set{b_0,b_1,\dots,b_n})$ for any
finite set $\set{b_0,b_1,\dots,b_n}$.  Students sometimes get confused
and think that this proves that $A \bij (A \union B)$ for a countably
infinite set $B \eqdef \set{b_0,b_1,\dots,b_n,\dots}$.  Now it's true
that $A \bij (A \union B)$ for any countable set $B$
(Problem~\bref{PS_add_countable_elements}), but the facts above do not
prove it.

What's bogus about this ``proof'' is that just because some predicate
$P(C)$ holds for sets $C = A \union F$ for every \emph{finite} subset
$F \subset B$, it need not hold for $C = A \union B$.  Such predicates
might be called \emph{finitely discontinuous} at $A \union B$.  Let
$A$ be the integers and $B$ be the rational numbers.  Indicate which
of the predicates $P(C)$ are finitely discontinuous at $A \union B$.

\begin{itemize}
\item $C$ contains only finitely many non-integers.
\item There is a maximum non-integer in $C$.
\item 
\end{itemize}

\end{problem}

%%%%%%%%%%%%%%%%%%%%%%%%%%%%%%%%%%%%%%%%%%%%%%%%%%%%%%%%%%%%%%%%%%%%%
% Problem ends here
%%%%%%%%%%%%%%%%%%%%%%%%%%%%%%%%%%%%%%%%%%%%%%%%%%%%%%%%%%%%%%%%%%%%%

\endinput

