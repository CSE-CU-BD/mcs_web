\documentclass[problem]{mcs}

\begin{pcomments}
  \pcomment{TP_infinite_repeat}
  \pcomment{by ARM 5/8/11}
\end{pcomments}

\pkeywords{
  infinite
  expectation
}

%%%%%%%%%%%%%%%%%%%%%%%%%%%%%%%%%%%%%%%%%%%%%%%%%%%%%%%%%%%%%%%%%%%%%
% Problem starts here
%%%%%%%%%%%%%%%%%%%%%%%%%%%%%%%%%%%%%%%%%%%%%%%%%%%%%%%%%%%%%%%%%%%%%

\begin{problem}

%\bparts %\ppart

You have a biased coin with nonzero probability $p<1$ of coming up
heads.  You toss until a head comes up, and then, as in
Section~\bref{infinite_expect_sec}, you keep tossing until you get a
long run of tails, but this time let ``long run'' mean a run of tails
that is at least $k-10$ when your initial run was length $k$.  Prove
that the expected number of times you toss a head and start over is
still infinite.

\begin{solution}
Let $T$ be the length of your initial run of tails.  If $T=k$, then
the expected number of tries until getting $k-10$ tails will be the
mean time to ``failure,'' $q^{k-10}$, because the probability of
``failing'' by tossing $k-10$ tails in a row is $q^{-(k-10)}$, where $q
\eqdef 1-p$.  Letting $R$ be the number of restarts, we have
\[
\expect{R}
     = \sum_{k \in \naturals} \expcond{R}{T=k} \cdot \prob{T=k}
     = \paren{\sum_{k < 10} q^kp} + \sum_{k \geq 10}  \frac{1}{q^{k-10}} \cdot q^kp
     = \text{constant} +\sum_{k \geq 10} \frac{p}{q^{10}} = \infty.
\]
\end{solution}

\begin{editingnotes}
\ppart We can formalize the process of sampling a random variable $T$
and then counting the number of repeat samples needed until we get a
value larger than the initial one as follows.  Let $T_1, T_2,\dots$ be
mutually independent random variables with the same distribution as
$T$ and define
\[
R \eqdef \min\set{k \geq 1 \suchthat T_k > T}.
\]
Suppose $\pdf_{T}(n) = \Theta(n^a)$ for some real number $a >1$.
Prove that $\expect{R} = \infty$.\footnote{You may not appeal to
  Problem~\bref{PS_infinite_repeat_expectation} which shows that for
  essentially \emph{every} random variable $T$, the corresponding
  repeat variable $R$ has infinite expectation.}

\begin{solution}
\TBA{}
\end{solution}

\eparts

\end{editingnotes}

\end{problem}

%%%%%%%%%%%%%%%%%%%%%%%%%%%%%%%%%%%%%%%%%%%%%%%%%%%%%%%%%%%%%%%%%%%%%
% Problem ends here
%%%%%%%%%%%%%%%%%%%%%%%%%%%%%%%%%%%%%%%%%%%%%%%%%%%%%%%%%%%%%%%%%%%%%

\endinput
 
