\documentclass[problem]{mcs}

\begin{pcomments}
  \pcomment{PS_shortest_undirected_closed_walk}
  \pcomment{first part is MQ_closed_walk_to_cycle}
  \pcomment{S15.mid3}
\end{pcomments}

\pkeywords{
  walk
  path
  cycle
  closed_walk
  simple_graph
  undirected
}

%%%%%%%%%%%%%%%%%%%%%%%%%%%%%%%%%%%%%%%%%%%%%%%%%%%%%%%%%%%%%%%%%%%%%
% Problem starts here
%%%%%%%%%%%%%%%%%%%%%%%%%%%%%%%%%%%%%%%%%%%%%%%%%%%%%%%%%%%%%%%%%%%%%

\begin{problem}

\bparts

\ppart Give an example of a simple graph that has two vertices $u \neq
v$ and two distinct paths between $u$ and $v$, but neither $u$ nor $v$
is on a cycle.

\inhandout{\hint There is an example with five vertices.}

\examspace[1in]

\begin{solution}

\begin{editingnotes}
INSERT:

\begin{figure}
\graphic{Fig_walkpath}
\caption{$u, v$ connected by Distinct paths but neither is on a cycle.}
\label{fig:pathcycle}
\end{figure}

As in Figure~\ref{fig:pathcycle}, define
\end{editingnotes}

Take a triangle and attach $u$ by an edge to one corner and $v$ by an
edge to another corner.  Formally, define
\begin{align*}
V & \eqdef \set{u,v,a,b,c},\\
E & \eqdef \set{\edge{u}{a}, \edge{a}{b}, \edge{b}{c}, \edge{c}{a}, \edge{c}{v} }.
\end{align*}
Two paths from $u$ to $v$ are
\[
u \edge{u}{a} a \edge{a}{c} c \edge{c}{v} v
\]
and
\[
u \edge{u}{a} a \edge{a}{b} b \edge{b}{c} c \edge{c}{v} v.
\]
\end{solution}

\ppart Prove that if there are different paths between two vertices
in a simple graph, then the graph has a cycle.

\examspace[3in]

\begin{solution}

\begin{proof}
  Call two vertices $u \neq v$ a \emph{different-path-pair} (dpp) if
  there are distinct paths between them.  Suppose $u, v$ is a dpp
  whose distance is minimum among all dpp's---here we are implicitly
  applying the WOP---and let $\walkv{p}$ be a shortest path between
  $u$ and $v$.  By definition of dpp, there must be another path
  $\walkv{q} \neq \walkv{p}$ between $u$ and $v$.

  We claim that, other than $u$ and $v$, there cannot be a vertex that
  appears in both paths $\walkv{p}$ and $\walkv{q}$.  This implies
  that $\merge{\walkv{q}}{\text{reverse}(\walkv{p})}$ is a cycle.

  So we just have to prove the claim: suppose to the contrary there
  was such a vertex $w$ appearing in both $\walkv{p}$ and
  $\walkv{q}$.  This means that
\[
\walkv{p} = \catv{\walkv{p_1}}{w}{\walkv{p_2}}
\]
and
\[
\walkv{q} = \catv{\walkv{q_1}}{w}{\walkv{q_2}}
\]
for some walks $\walkv{p_1}$, $\walkv{q_1}$ that start at $u$ and end
at $w$, and walks $\walkv{p_2}$, $\walkv{q_2}$ that start at $w$ and
end at $v$.  But since $\walkv{p} \neq \walkv{q}$, either $\walkv{p_1}
\neq \walkv{q_1}$ or $\walkv{p_2} \neq \walkv{q_2}$, which implies
that either $u,w$ is a dpp or $w,v$ is a dpp, and this dpp will
have a shorter path between them than $u,v$.  This contradicts the
fact that among all dpp's, $u,v$ have a shortest length path between
them.  So the claim must be true.

\end{proof}

Note that the proof above is a rephrasing of the proof that a simple
graph is a tree iff there is a unique path between any two vertices,
Theorem~\bref{th:treeprops}.\bref{treeprops:uniquepath}.  In fact,
appeal to this Theorem yields an immediate proof of this part: if
$u,v$ are a dpp, then by the Theorem, the connected component
containing $u,v$ is not a tree, which by definition means it contains
a cycle.

\end{solution}

\eparts

\end{problem}

%%%%%%%%%%%%%%%%%%%%%%%%%%%%%%%%%%%%%%%%%%%%%%%%%%%%%%%%%%%%%%%%%%%%%
% Problem ends here
%%%%%%%%%%%%%%%%%%%%%%%%%%%%%%%%%%%%%%%%%%%%%%%%%%%%%%%%%%%%%%%%%%%%%

\endinput
