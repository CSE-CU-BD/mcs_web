\documentclass[problem]{mcs}

\begin{pcomments}
  \pcomment{new by ARM 9/8/09, revised hint 9/15/09}
  \pcomment{Fix hard reference to Week 1 notes.}
\end{pcomments}

\pkeywords{
  primes
  polynomials
  proof_by_cases
}



%%%%%%%%%%%%%%%%%%%%%%%%%%%%%%%%%%%%%%%%%%%%%%%%%%%%%%%%%%%%%%%%%%%%%
% Problem starts here
%%%%%%%%%%%%%%%%%%%%%%%%%%%%%%%%%%%%%%%%%%%%%%%%%%%%%%%%%%%%%%%%%%%%%


\begin{problem}
  For $n=40$, the value of polynomial $p(n) \eqdef
  n^2+n+41$ % ~\ref{pn41},
  is not prime, as noted in Chapter~\bref{proofs_chap} of the Course 
Text.
  But we could have predicted based on general principles that no
  nonconstant polynomial, $q(n)$, with integer coefficients can map each
  nonnegative integer into a prime number.  Prove it.

  \hint Let $c \eqdef q(0)$ be the constant term of $q$.  Consider two
  cases: $c$ is not prime, and $c$ is prime.  In the second case, note
  that $q(cn)$ is a multiple of $c$ for all $n \in \integers$.  You may
  assume the familiar fact that the magnitude (absolute value) of any
  nonconstant polynomial, $q(n)$, grows unboundedly as $n$ grows.
 
\begin{solution}
%from ARM email reply S07 to Timan Goshit:
\begin{proof}
The proof is by cases following the hint.  

\textbf{Case 1} ($c$ is not prime): Then $q(0)$ is not a prime, so $q$
does not map all nonnegative integers to primes.

\textbf{Case 2} ($c$ is prime): In this case, $q(cm)$ has $c$ as a factor
for all integers, $m$.  Because $q$ is not constant, $\abs{q(cm)}$
grows unboundedly as $m$ increases.  But as soon as $\abs{q(cm)}$
grows bigger than $c$, it won't be prime because it has $c$ as a factor.
\end{proof}

Note that the proof for Case 2 shows something stronger: there are
infinitely many nonprimes in $\range{q}$ as long as $c > 1$.  It's also
easy to see that there will be infinitely many nonprimes in $\range{q}$
when $c=0$.  We will bet that this is also true when $c=1$, but we haven't
found a proof.  Let us know if you find one.

\end{solution}

\end{problem}

%%%%%%%%%%%%%%%%%%%%%%%%%%%%%%%%%%%%%%%%%%%%%%%%%%%%%%%%%%%%%%%%%%%%%
% Problem ends here
%%%%%%%%%%%%%%%%%%%%%%%%%%%%%%%%%%%%%%%%%%%%%%%%%%%%%%%%%%%%%%%%%%%%%
