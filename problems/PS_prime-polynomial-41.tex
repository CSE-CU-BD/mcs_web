\documentclass[problem]{mcs}

\begin{pcomments}
  \pcomment{new by ARM 9/8/09}
  \pcomment{Fix hard reference to Week 1 notes.}
\end{pcomments}

\pkeywords{
  primes
  polynomials
  proof_by_cases
}



%%%%%%%%%%%%%%%%%%%%%%%%%%%%%%%%%%%%%%%%%%%%%%%%%%%%%%%%%%%%%%%%%%%%%
% Problem starts here
%%%%%%%%%%%%%%%%%%%%%%%%%%%%%%%%%%%%%%%%%%%%%%%%%%%%%%%%%%%%%%%%%%%%%


\begin{problem}
  In notes Chapter~\ref{pn41}, it took until $n=40$ to find a nonnegative
  integer argument such that $p(n) \eqdef n^2 + n + 41$ was not prime.
  But we could have predicted based on general principles that, no
  nonconstant polynomial, $q(n)$, with integer coefficients can map each
  nonnegative integer into a prime number.  Prove it.

  \hint Let $c$ be the constant term in $q(n)$.  Consider two cases: $c=0$
  and $c \neq 0$.  In the second case, note that $q(cn)$ is a multiple of
  $c$ for all $n \in \integers$.  You may assume the familiar fact that
  the magnitude (in absolute value) of any nonconstant polynomial, $q(n)$,
  grows unboundedly as $n$ grows.
 
\begin{solution}
%from ARM email reply S07:
\begin{proof}
Suppose $q$ is a polynomial of degree $d$. So
\[
q(n) = \sum_{i=0}^d c_in^i
\]
for some integer constants $c_i$.  Let $c \eqdef c_0$ be the constant term
of $q$.

The proof is by cases following the hint.  

\textbf{Case 1} ($c = 0$): Then all the terms in $q(n)$ are multiples of
$n$, so $p(2m)$ is always even.  Since $q$ is not constant, $\abs{q(2m)}$
grows unboundedly as $m$ increases.  But as soon as $\abs{q(2m)}$
grows bigger than 2, it won't be prime because it has 2 as a factor.

\textbf{Case 2} ($c \neq 0$): In this case, $q(cm)$ has $c$ as a factor
for all integers, $m$.  Because $q$ is not constant, $\abs{q(cm)}$
grows unboundedly as $m$ increases.  But as soon as $\abs{q(2m)}$
grows bigger than $c$, it won't be prime because it has $c$ as a factor.
\end{proof}

\end{solution}

\end{problem}

%%%%%%%%%%%%%%%%%%%%%%%%%%%%%%%%%%%%%%%%%%%%%%%%%%%%%%%%%%%%%%%%%%%%%
% Problem ends here
%%%%%%%%%%%%%%%%%%%%%%%%%%%%%%%%%%%%%%%%%%%%%%%%%%%%%%%%%%%%%%%%%%%%%
