\documentclass[problem]{mcs}

\begin{pcomments}
  \pcomment{MQ_10_and_15_cent_stamps_by_WOP}
  \pcomment{variant of TP_10_and_15_cent_stamps_by_WOP}
  \pcomment{same as F09: cp2m with 10 replacing 6}
\end{pcomments}

\pkeywords{
  well_ordering
  WOP
  postage_stamps
}

%%%%%%%%%%%%%%%%%%%%%%%%%%%%%%%%%%%%%%%%%%%%%%%%%%%%%%%%%%%%%%%%%%%%%
% Problems start here
%%%%%%%%%%%%%%%%%%%%%%%%%%%%%%%%%%%%%%%%%%%%%%%%%%%%%%%%%%%%%%%%%%%%%

\begin{problem}
  Except for an easily repaired omission, the following proof using
  the Well Ordering Principle shows that every amount of postage that
  can be paid exactly using only 10 cent and 15 cent stamps, is
  divisible by 5.

  Namely, let the notation ``$j \divides k$'' indicate that integer
  $j$ is a divisor of integer $k$, and let $S(n)$ mean that exactly
  $n$ cents postage can be assembled using only 10 and 15 cent stamps.
  Then the proof shows that
\begin{equation}\label{SnQ5n}
S(n)\ \QIMPLIES\ 5 \divides n, \quad \text{for all nonnegative integers $n$}.
\end{equation}
Fill in the missing portions (indicated by ``\dots'') of the following
proof of~\eqref{SnQ5n}, and at the end, identify the minor mistake in
the proof and how to fix it.

\begin{quote}
Let $C$ be the set of \emph{counterexamples} to~\eqref{SnQ5n}, namely
\[
C \eqdef \set{n \suchthat S(n)\text{ and } \QNOT(5 \divides n)}
\]

Assume for the purpose of obtaining a contradiction that $C$ is
nonempty.  Then by the WOP, there is a smallest number, $m \in C$.
Then $S(m-10)$ or $S(m-15)$ must hold, because the $m$ cents postage
is made from 10 and 15 cent stamps, so we remove one.

So suppose $S(m-10)$ holds.  Then $5 \divides (m-10)$, because\dots

\examspace[0.3in]
\examrule[6in]

\begin{solution}
\dots if $\QNOT(5 \divides (m-10))$, then $m-10$ would be
  a counterexample smaller than $m$, contradicting the minimality of
  $m$.
\end{solution}

But if $5 \divides (m-10)$, then  $5 \divides m$, because\dots

\examspace[0.3in]
\examrule[6in]

\begin{solution}
\dots $ 5 \divides (m-10) $ and $5 \divides 10$, so $5 \divides (m - 10  + 10)$.
\end{solution}

contradicting the fact that $m$ is a counterexample.

Next suppose $S(m-15)$ holds.  Then the proof for $m-10$ carries over
directly for $m-15$ to yield a contradiction in this case as well.
Since we get a contradiction in both cases, we conclude that $C$ must
be empty.  That is, there are no counterexamples to~\eqref{SnQ5n},
which proves that~\eqref{SnQ5n} holds.

\end{quote}

The proof makes an implicit assumption about the value of $m$.  State
the assumption and justify it in one sentence.

\begin{solution}
The claim that $S(m)$ implies $S(m - 10) \QOR\ S(m -15)$ makes the
proof implicit assumption that $m>0$.  But this assumption can be
justified because $5 \divides 0$, so 0 cannot be a counterexample
to~\eqref{SnQ5n}.
\end{solution}

\examspace[0.7in]

\end{problem}

%%%%%%%%%%%%%%%%%%%%%%%%%%%%%%%%%%%%%%%%%%%%%%%%%%%%%%%%%%%%%%%%%%%%%
% Problems end here
%%%%%%%%%%%%%%%%%%%%%%%%%%%%%%%%%%%%%%%%%%%%%%%%%%%%%%%%%%%%%%%%%%%%%

\endinput
