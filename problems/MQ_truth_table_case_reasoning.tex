\documentclass[problem]{mcs}

\begin{pcomments}
  \pcomment{MQ_truth_table_case_reasoning}
  \pcomment{for MQ spring '11}
  \pcomment{by Oscar}
\end{pcomments}

\pkeywords{
  satisfiable
  truth_assignment
  case_reasoning
}

\begin{problem}

Show that there are exactly two truth assignments for the variables P,Q,R,S that satisfy the following formula:

\[ (\overline P \mbox{ OR } Q) \mbox{ AND } (\overline Q \mbox{ OR } R) \mbox{ AND } (\overline R  \mbox{ OR } S ) \mbox{ AND } (\overline S \mbox{ OR } P) \]

\hint A truth table will do the job, but it will have a bunch of rows.  A proof by cases can be quicker; if you do use cases,
be sure each one is clearly specified.

\begin{solution}
You can deduce the  only two possibilities by cases: 

If $P$ is false, then in order to have any chance of satisfying clause 4,  $S$ must be false. Similarly, if $S$ is 
false, then in order to satisfy clause 3, $R$ must be false. And similarly, $Q$ must be false. On the other hand, if $P$ is true, then $Q$ must be true to make clause 1 true and have any chances of making the overall expression true. Similarly, If $Q$ is true, then  $R$ must be true and if $R$ is true then $S$ is true. \\

Those arguments prove there are at most 2 cases, but you need to show the assignments we are left with  actually satisfy the formula. 
This can be easily done, by plugging the values into the formula: \\

If all variables are set to true, then since clause 1 has $Q$ clause 2 has $R$, clause 3 has $S$, and clause 4 has $P$, then every clause is satisfied, and the full
$\mbox{AND}$ is satisfied. If all are false, then since clause 1 has $\overline P$, clause 2 has $\overline Q$ , clause 3 has $\overline R$ and clause 4 has $\overline S$,
then again every clause is satisfied and the overall proposition is satisfied. So both of those satisfy the proposition.


\end{solution}

\end{problem}

%%%%%%%%%%%%%%%
% The former two-part problem
%%%%%%%%%%%%%%%

%\begin{problem}
%Consider the following formula:

%\[ (\overline P \mbox{ OR } Q) \mbox{ AND } (\overline Q \mbox{ OR } R) \mbox{ AND } (\overline R  \mbox{ OR } S ) \mbox{ AND } (\overline S \mbox{ OR } P) \]

%
%\bparts

%\ppart Exhibit two truth assignments (that is, truth environments) under which the formula evaluates to True.

%\examspace[3in]

%\begin{solution}
%The two satisfying assignments are: all false or all true. You can plug them into the formula to check they satisfy it.
%\end{solution}

%\ppart Explain \emph{briefly} why those are the only two satisfying assignments. Be as clear as you can, while remaining brief.

%\begin{solution}
%You can prove this by cases: assume $P$ is true, and with that show everyone else must be true. 
%Then show if $P$ is false, everyone else is also false.
%\end{solution}

%\eparts

%\end{problem}

\endinput
