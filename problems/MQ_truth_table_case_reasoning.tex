\documentclass[problem]{mcs}

\begin{pcomments}
  \pcomment{MQ_truth_table_case_reasoning}
  \pcomment{for MQ S11, S13}
  \pcomment{by Oscar}
  \pcomment{common in student solutions: way too many students decided
    to do this manually, you may want to increase the problem size to
    guide them to do by cases}
  \pcomment{revised S13 by ARM to require cases}
\end{pcomments}

\pkeywords{
  satisfiable
  truth_assignment
  case_reasoning
}

\begin{problem}
\begin{claim*}
 There are exactly two truth environments (assignments) for the variables $M,N,P,Q,R,S$
 that satisfy the following formula:
 \[
 \underbrace{(\bar{P} \QOR\ Q)}_{\text{clause (1)}} \QAND\ 
 \underbrace{(\bar{Q} \QOR\ R)}_{\text{clause (2)}} \QAND\ 
 \underbrace{(\bar{R} \QOR\ S)}_{\text{clause (3)}} \QAND\
 \underbrace{(\bar{S} \QOR\ P)}_{\text{clause (4)}} \QAND\
 M \QAND\ \bar{N}
 \]
 \end{claim*}

\bparts

\ppart This claim could be proved by truth-table.  How many rows would
the truth table have?\examrule

\begin{solution}
A truth table for a formula with 6 variables has $2^6 = 64$ rows.
\end{solution}

\ppart Instead of a truth-table, prove this claim with an argument by
cases according to the truth value of $P$.

\begin{solution}
Obviously $M$ must be true and $N$ must be false.  Now we have:

\textbf{Case 1} ($P$ is false):
In order to have any chance of satisfying clause (4), $S$ must be false.
Similarly, if $S$ is false, then in order to satisfy clause (3), $R$
must be false; similarly, $Q$ must be false.

\textbf{Case 2} ($P$ is true): 
$Q$ must be true to make clause (1) true and have any chances of
making the overall expression true.  Similarly, if $Q$ is true, then
$R$ must be true and if $R$ is true then $S$ is true.

Those arguments prove there are \emph{at most} two satisfying truth
environments, but we need to show the two environments we were left with
actually satisfy the formula.  This can be easily done, by plugging
the values into the formula:

If all variables $P,Q,R,S$ are set to true, then since clause (1) has
$Q$ clause (2) has $R$, clause (3) has $S$, and clause (4) has $P$,
then every clause is satisfied, and the full $\mbox{AND}$-combination
is satisfied.  If all are false, then since clause (1) has $\bar P$,
clause (2) has $\bar Q$, clause (3) has $\bar R$ and clause (4) has
$\bar S$, then again every clause is satisfied and the overall
proposition is satisfied.  So both of those satisfy the proposition.
\end{solution}

\eparts

\end{problem}

%%%%%%%%%%%%%%%
% The former two-part problem
%%%%%%%%%%%%%%%

%\begin{problem}
%Consider the following formula:

%\[ (\bar P \QOR\ Q) \QAND\ (\bar Q \QOR\ R) \QAND\ (\bar R  \QOR\ S ) \QAND\ (\bar S \QOR\ P) \]

%
%\bparts

%\ppart Exhibit two truth assignments (that is, truth environments) under which the formula evaluates to True.

%\examspace[3in]

%\begin{solution}
%The two satisfying assignments are: all false or all true. You can plug them into the formula to check they satisfy it.
%\end{solution}

%\ppart Explain \emph{briefly} why those are the only two satisfying assignments. Be as clear as you can, while remaining brief.

%\begin{solution}
%You can prove this by cases: assume $P$ is true, and with that show everyone else must be true. 
%Then show if $P$ is false, everyone else is also false.
%\end{solution}

%\eparts

%\end{problem}

\endinput
