\documentclass[problem]{mcs}

\begin{pcomments}
  \pcomment{CP_a_baseball_series}
  \pcomment{from: S09.cp12r}
\end{pcomments}

\pkeywords{
  probability
  sample_space
  event
}

%%%%%%%%%%%%%%%%%%%%%%%%%%%%%%%%%%%%%%%%%%%%%%%%%%%%%%%%%%%%%%%%%%%%%
% Problem starts here
%%%%%%%%%%%%%%%%%%%%%%%%%%%%%%%%%%%%%%%%%%%%%%%%%%%%%%%%%%%%%%%%%%%%%

% F09, S09

\begin{problem}
The New York Yankees and the Boston Red Sox are playing a
two-out-of-three series.  In other words, they play until one team has
won two games.  Then that team is declared the overall winner and the
series ends.  Assume that the Red Sox win each game with probability
$3/5$, regardless of the outcomes of previous games.

Answer the questions below using the four step method.  You can use
the same tree diagram for all three problems.

\begin{staffnotes}
The point of this problem is to spell out the 4-step method, after
which the actual probability calculations are trivial.  Make sure
students explicitly exhibit a tree, assign probabilities to outcomes
(the leaves of the tree), and list the outcomes in each of the 3
events given below.
\end{staffnotes}

\bparts

\ppart What is the probability that a total of 3 games are played?

\ppart What is the probability that the winner of the series loses the
first game?

\ppart What is the probability that the \emph{correct} team wins the
series?

\eparts

\begin{solution}
A tree diagram is worked out below.
%
\begin{figure}[h]
\graphic{series}
\end{figure}
%
From the tree diagram, we get:
%
\begin{align*}
\pr{\text{3 games played}}
    & = \frac{12}{125} + \frac{18}{125} + \frac{12}{125} + \frac{18}{125}
      = \frac{12}{25} \\
\pr{\text{winner lost first game}}
    & = \frac{18}{125} + \frac{12}{125}
      = \frac{6}{25} \\
\pr{\text{correct team wins}}
    & = \frac{18}{125} + \frac{18}{125} + \frac{9}{25} = \frac{81}{125}
\end{align*}

\end{solution}

\end{problem}

%%%%%%%%%%%%%%%%%%%%%%%%%%%%%%%%%%%%%%%%%%%%%%%%%%%%%%%%%%%%%%%%%%%%%
% Problem ends here
%%%%%%%%%%%%%%%%%%%%%%%%%%%%%%%%%%%%%%%%%%%%%%%%%%%%%%%%%%%%%%%%%%%%%

\endinput
