\documentclass[problem]{mcs}

\begin{pcomments}
  \pcomment{CP_runs_of_composites}
  \pcomment{from: F05-PS5-P5}
\end{pcomments}

\pkeywords{
  prime
  composite
  factorial
  divides}

%%%%%%%%%%%%%%%%%%%%%%%%%%%%%%%%%%%%%%%%%%%%%%%%%%%%%%%%%%%%%%%%%%%%%
% Problem starts here
%%%%%%%%%%%%%%%%%%%%%%%%%%%%%%%%%%%%%%%%%%%%%%%%%%%%%%%%%%%%%%%%%%%%%


\begin{problem}
Here is a long run of composite numbers:

\[
114, 115, 116, 117, 118, 119, 120, 121, 122, 123, 124, 125, 126
\]

Prove that there exist arbitrarily long runs of composite numbers.
Consider numbers a little bigger than $n!$ where $n! = n \cdot (n - 1)
\cdots 3 \cdot 2 \cdot 1$.


\begin{solution}
  Let $k$ be some nonnegative integer number such that $1 < k \leq n$.  We know $k
  \divides (n! + k)$ because $k \mid n!$ and $k \mid k$. Thus, the numbers
  $n!+2, n!+3, n!+4, \ldots, n!+n$ must all be composite.  This is a
  run of $n-1$ consecutive composite numbers. Because we can
  arbitrarily choose $n$, we know arbitrarily long runs of compisite
  numbers exist.

\end{solution}

\end{problem}


%%%%%%%%%%%%%%%%%%%%%%%%%%%%%%%%%%%%%%%%%%%%%%%%%%%%%%%%%%%%%%%%%%%%%
% Problem ends here
%%%%%%%%%%%%%%%%%%%%%%%%%%%%%%%%%%%%%%%%%%%%%%%%%%%%%%%%%%%%%%%%%%%%%

\endinput

