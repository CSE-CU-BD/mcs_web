\documentclass[problem]{mcs}

\begin{pcomments}
  \pcomment{MQ_task_parallel_scheduling_v4}
  \pcomment{variant of _v3}
  \pcomment{F15.final-conflict2}
\end{pcomments}

\pkeywords{
  DAG
  scheduling
  chains_and_antichains
}

%%%%%%%%%%%%%%%%%%%%%%%%%%%%%%%%%%%%%%%%%%%%%%%%%%%%%%%%%%%%%%%%%%%%%
% Problem starts here
%%%%%%%%%%%%%%%%%%%%%%%%%%%%%%%%%%%%%%%%%%%%%%%%%%%%%%%%%%%%%%%%%%%%%

\begin{problem}
The following DAG describes the prerequisites among tasks $\set{1, \dots, 9}$.

\begin{figure}[h]
\graphic[height=2.0in]{sequence-poset-down2}
\end{figure}

\bparts

\ppart If each task takes \textbf{one hour} to complete, what is the minimum
parallel time to complete all the tasks?  Briefly explain.

\exambox{0.7in}{0.4in}{0in}

\begin{solution}
\textbf{5}.  This is the size of the maximum chain.
\end{solution}

\examspace[2cm]

\ppart What is the minimum parallel time if no more than two tasks can
be completed in parallel?  Briefly explain.

\begin{solution}
\textbf{6}.  Any schedule must start with \textbf{9} and then
\textbf{8}.  This leaves 7 tasks which will require at least
$\ceil{7/2} = 4$ more time units, so a minimum 2-processor schedule
must take parallel time 6.  It is not hard to find a 2-processor
schedule that finishes in 6 time units.
\end{solution}

\exambox{0.7in}{0.4in}{0in}
\examspace[2cm]

\eparts

\end{problem}

%%%%%%%%%%%%%%%%%%%%%%%%%%%%%%%%%%%%%%%%%%%%%%%%%%%%%%%%%%%%%%%%%%%%%
% Problem ends here
%%%%%%%%%%%%%%%%%%%%%%%%%%%%%%%%%%%%%%%%%%%%%%%%%%%%%%%%%%%%%%%%%%%%%

\endinput
