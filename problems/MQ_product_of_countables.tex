\documentclass[problem]{mcs}

\begin{pcomments}
  \pcomment{MQ_product_of_countables}
  \pcomment{created by Ali Kazerani; related to CP_rationals_are_countable}
  \pcomment{revised by ARM, 3/1/11}
\end{pcomments}

\pkeywords{
  bijection
  product
  countable
}

%%%%%%%%%%%%%%%%%%%%%%%%%%%%%%%%%%%%%%%%%%%%%%%%%%%%%%%%%%%%%%%%%%%%%
% Problem starts here
%%%%%%%%%%%%%%%%%%%%%%%%%%%%%%%%%%%%%%%%%%%%%%%%%%%%%%%%%%%%%%%%%%%%%

\begin{problem}
Let A and B denote two countably infinite sets:
\begin{align*}
A & = \set{a_0, a_1, a_2, a_3, \dots}\\
B & = \set{b_0, b_1, b_2, b_3, \dots}
\end{align*}
Show that their product, $A \times B$, is also a countable set by
showing how to list the elements of $A \times B$.  You need only show
enough of the initial terms in your sequence to make the pattern clear
---a half dozen or so terms usually suffice.

\iffalse
first six elements of this
  (infinitely long) list, as long as the underlying pattern is
  clear.\\\\ Recall that the set product $S_1\times
  S_2\times\cdots\times S_n$ is just the set of all possible sequences
  of length $n$ whose $i$th element is drawn from $S_i$.  In
  particular, $A\times B$ is the set of all possible ordered pairs
  $(a_i, b_j)$, $i=0, 1, 2, \dots$, $j=0, 1, 2, \dots$.
\fi

\begin{solution}
    The elements of $A\times B$ can be arranged as follows:
\[\begin{array}{ccccc}
{\color{Red}(a_0, b_0)} & {\color{Blue}(a_0, b_1)} &
{\color{Green}(a_0, b_2)}  & {\color{Magenta}(a_0, b_3)} & \dots\\
{\color{Blue}(a_1, b_0)} & {\color{Green}(a_1, b_1)} &
{\color{Magenta}(a_1, b_2)} & {\color{MidnightBlue}(a_1, b_3)} & \dots\\
{\color{Green}(a_2, b_0)} & {\color{Magenta}(a_2, b_1)} &
{\color{MidnightBlue}(a_2, b_2)} & {\color{OliveGreen}(a_2, b_3)} & \dots\\
{\color{Magenta}(a_3, b_0)} & {\color{MidnightBlue}(a_3, b_1)} &
{\color{OliveGreen}(a_3, b_2)} & {\color{Fuchsia}(a_3, b_3)} & \dots\\
	\vdots & \vdots & \vdots & \vdots & \ddots
\end{array}\]
Traversing this grid along successive lower-left to upper-right
diagonals yields the required list:

\[{\color{Red}(a_0, b_0)}, {\color{Blue}(a_1, b_0)}, 
  {\color{Blue}(a_0, b_1)}, {\color{Green}(a_2, b_0)}, 
  {\color{Green}(a_1, b_1)}, {\color{Green}(a_0, b_2)}, 
  {\color{Magenta}(a_3, b_0)}, {\color{Magenta}(a_2, b_1)}, 
  {\color{Magenta}(a_1, b_2)}, {\color{Magenta}(a_0, b_3)}, \dots
\]
Obviously, traveling in the opposite direction along each diagonal
yields an equally acceptable list:
\[
{\color{Red}(a_0, b_0)}, {\color{Blue}(a_0, b_1)},
{\color{Blue}(a_1, b_0)}, {\color{Green}(a_0, b_2)},
{\color{Green}(a_1, b_1)}, {\color{Green}(a_2, b_0)},
{\color{Magenta}(a_0, b_3)}, {\color{Magenta}(a_1, b_2)},
{\color{Magenta}(a_2, b_1)}, {\color{Magenta}(a_3, b_0)}, \dots
\]
\end{solution}
  
\end{problem}

%%%%%%%%%%%%%%%%%%%%%%%%%%%%%%%%%%%%%%%%%%%%%%%%%%%%%%%%%%%%%%%%%%%%%
% Problem ends here
%%%%%%%%%%%%%%%%%%%%%%%%%%%%%%%%%%%%%%%%%%%%%%%%%%%%%%%%%%%%%%%%%%%%%

\endinput
