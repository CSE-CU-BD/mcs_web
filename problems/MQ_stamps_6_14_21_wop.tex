\documentclass[problem]{mcs}

\begin{pcomments}
  \pcomment{MQ_stamps_6_14_21_wop}
  \pcomment{F16.midterm1}
  \pcomment{ARM 2/24/16}
\end{pcomments}

\pkeywords{
  WOP
  stamp
  makeable
}

\begin{problem}
Prove using the Well Ordering Principle that, using 6\textcent,
14\textcent, and 21\textcent\ stamps, it is possible to make any
amount of postage over 50\textcent.  To save time, you may specify
\emph{assume without proof} that 50\textcent, 51\textcent, \dots
100\textcent\ are all makeable, but you should clearly indicate which
of these assumptions your proof depends on.

\begin{solution}
\begin{proof}
Assume to the contrary that some amount of postage of
$50$\textcent\ or more is not makeable.  So by WOP, there will be a
\emph{least} unmakeable amount $m \geq 50$.  If we assume
50--55\textcent\ is makeable, then we can conclude that $m \geq 56$.
So $m-6 \geq 50$ and therefore is makeable, because if $m > k \geq
50$, then $k$ is makeable by definition of $m$.  Now since $m-6$ is
makeable, we can add a 6\textcent\ stamp and make $(m-6)+6 =
m$\textcent, contradicting the fact that $m$ is unmakeable.  So there
cannot be such a minimum $m$, which proves that all amounts $\geq
50$\textcent\ are makeable.
\end{proof}
\end{solution}

\end{problem}

\endinput
