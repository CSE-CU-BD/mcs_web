\documentclass[problem]{mcs}

\begin{pcomments}
  \pcomment{PS_3_friends}
  \pcomment{combinatorial proof of friends-and-strangers theorem}
  \pcomment{CH, Spring '14, edited ARM 4/26/14; revised 5/17/15}
\end{pcomments}

\pkeywords{
  counting
  degree
  complete_graph
  pigeon_hole
  ramsey
  friends
  strangers
}

%%%%%%%%%%%%%%%%%%%%%%%%%%%%%%%%%%%%%%%%%%%%%%%%%%%%%%%%%%%%%%%%%%%%%
% Problem starts here
%%%%%%%%%%%%%%%%%%%%%%%%%%%%%%%%%%%%%%%%%%%%%%%%%%%%%%%%%%%%%%%%%%%%%

\begin{problem}
Let $G$ be a simple graph with $6$ vertices and an edge between every
pair of vertices (that is, $G$ is a \emph{complete} graph).  A
length-3 cycle in $G$ is called a \emph{triangle}.

A set of two edges that share a vertex is called an \emph{incident
  pair} (i.p.); the shared vertex is called the \emph{center} of the
i.p.\   That is, an i.p.\ is a set,
\[
\set{\edge{u}{v}, \edge{v}{w}},
\]
where $u,v$ and $w$ are distinct vertices, and its center is
$v$.

\bparts

\ppart\label{20trik6} How many triangles are there?\hfill\examrule[0.5in]

\examspace[0.75in]

\begin{solution}
\textbf{20.}

Every set of three vertices in $G$ determines a triangle, so there are
\[
\binom{6}{3} = 20
\]
triangles.
\end{solution}

\ppart\label{60ip6} How many incident pairs are there? \hfill\examrule[0.5in]

\examspace[0.75in]

\begin{solution}
\textbf{60.}

An i.p.\ is determined by its center and the set of the other two
endpoints.  There are 6 possible centers and, for each center, there
are $\binom{5}{2}$ sets of two possible endpoints, for a total of
\[
6\binom{5}{2} = 60.
\]

Alternatively, an i.p.\ is determined by the three endpoints of its
edges and the endpoint that is its center.  There are $\binom{6}{3}$
sets of endpoints and three possible center among these, so there are
\[
\binom{6}{3}\cdot 3 = 60
\]
i.p.'s.
\end{solution}

\eparts

\medskip

Now suppose that every edge in $G$ is colored either red or blue.  A
triangle or i.p.\ is called \emph{multicolored} when its edges are not
all the same color.

\bparts

\ppart\label{2-to-1multi} Map the i.p.
\[
\set{\edge{u}{v}, \edge{v}{w}}
\]
to the triangle
\[
\set{\edge{u}{v}, \edge{v}{w}, \edge{u}{w}}.
\]
Notice that multicolored i.p.'s map to multicolored triangles.
Explain why this mapping is 2-to-1 on these multicolored objects.

\examspace[1.2in]

\begin{solution}
A multicolored triangle $\set{\edge{u}{v}, \edge{v}{w}, \edge{u}{w}}$
has two edges of one color and one edge, say, $\edge{u}{v}$, of
another color.  Of the three i.p.'s that map to this triangle, only
the two i.p.'s that include this other-colored edge, that is, the
i.p.'s $\set{\edge{u}{v}, \edge{v}{w}}$ and $\set{\edge{u}{v},
  \edge{u}{w}}$, are multicolored.  So precisely these two
multicolored i.p.'s map to the multicolored triangle.
\end{solution}

\ppart If two people are not friends, they are called
\emph{strangers}.  If every pair of people in a group are friends, or
if every pair are strangers, the group is called \emph{uniform}.  The
final part of this problem will show that
\begin{quote}
There are at most 36 multicolored i.p.'s.
\end{quote}
Explain why this fact and the results of parts~\eqref{20trik6}
and~\eqref{2-to-1multi} imply that
\begin{quote}
Every set of six people includes \emph{two} uniform three-person
groups.
\end{quote}

\begin{solution}
Let the vertices of $G$ be the 6 people.  Color the edge between two
people red if they are friends and blue if they are strangers.  Then,
the statement is equivalent to the claim that there are at least two
triangles that are \emph{not} multicolored.

Since there are at most 36 the multicolored i.p's, and these map
2-to-1 to multicolored triangles, there are at most eighteen
multicolored triangles.  But there are twenty triangles, so at least
two must not be multicolored.
\end{solution}

\ppart\label{6edgeset} Show that at most six multicolored i.p.'s can
have the same center.  Conclude that there are at most 36 possible
multicolored i.p.'s.

\hint A vertex incident to $r$ red edges and $b$ blue edges is the
center of $r \cdot b$ different multicolored i.p.'s.

\examspace[1.0in]

\begin{solution}
If vertex $v$ is incident to $r$ red edges and $b$ blue edges, then it
is the center of exactly $r \cdot b$ i.p.'s.  But the degree of $v$ is
five, so $r+b =5$.  Under this constraint, it's easy to see that the
maximum value of $r \cdot b$ is six.  So the number of multicolored
i.p.'s is at most six times the number of vertices.
\end{solution}

\eparts

\end{problem}

%%%%%%%%%%%%%%%%%%%%%%%%%%%%%%%%%%%%%%%%%%%%%%%%%%%%%%%%%%%%%%%%%%%%%
% Problem ends here
%%%%%%%%%%%%%%%%%%%%%%%%%%%%%%%%%%%%%%%%%%%%%%%%%%%%%%%%%%%%%%%%%%%%%

\endinput
