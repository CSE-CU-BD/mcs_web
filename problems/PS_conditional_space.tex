\documentclass[problem]{mcs}

\begin{pcomments}
  \pcomment{PS_conditional_space}
  \pcomment{from: new S10.ps11 by Rich}
\end{pcomments}

\pkeywords{
  probability
}

%%%%%%%%%%%%%%%%%%%%%%%%%%%%%%%%%%%%%%%%%%%%%%%%%%%%%%%%%%%%%%%%%%%%%
% Problem starts here
%%%%%%%%%%%%%%%%%%%%%%%%%%%%%%%%%%%%%%%%%%%%%%%%%%%%%%%%%%%%%%%%%%%%%

\newcommand{\prb}[1]{\mathop{\textup{Pr}_B}\nolimits\left\{#1\right\}}

\begin{problem}

  Suppose the sample space, $\sspace$ and together with the function
  $\pr{}: \sspace\to [0,1]$ form a \textit{probability space}
  according to Definition~\bref{LN12:probsp}.

  Let $B$ be an event such that $\pr{B} > 0$, and define a function
  $\prb{\cdot}$ on events $A \subset \sspace$ by the rule:
\begin{equation}\label{prsubbdef}
  \prb{A} \eqdef \frac{\pr{A \intersect B}}{\pr{B}}.
\end{equation}
  
  Prove that $\sspace$ together with $\prb{\cdot}$ also forms a
  probability space.
  
\begin{solution}
  To show that $\prb{\cdot})$ defines a probability space on $\sspace$,
  we must show that
\begin{itemize}
\item[(i)] $0 \leq \prb{A} \leq 1$  or all events $A \subseteq \sspace$, and

\item[(ii)] 
\begin{equation}\label{prbss=1}
  \sum_{w \in \sspace} \prb{w} = 1.
\end{equation}
\end{itemize}  

  To prove~(i), note that $\prb{A} \geq 0$ since both the numerator,
  $\pr{A \intersect B}$, and the denominator, $\pr{B}$,
  in~\eqref{prsubbdef} are nonnegative.  Further, the numerator is
  less than or equal to the denominator, that is,
\begin{align*}
\pr{A \intersect B} \leq \pr{B} & \text{(by~\bref{LN12:subsetbound})},
\end{align*}
  and so $\prb{A} \leq 1$.
  
  Also, ~(ii) holds because
  \begin{align*}
    \sum_{w \in \sspace} \prb{w}
       & = \sum_{w \in B} \prb{\set{w} \intersect B}
           + \sum_{w \notin B} \prb{\set{w} \intersect B}\\
       & = \frac{\sum_{w \in B} \pr{w} + \sum_{w \notin B} \pr{\emptyset}}{\pr{B}}\\
       & = \frac{\pr{B} + \sum_{w \notin B} 0}{\pr{B}}\\
       &= \frac{\pr{B} + 0}{\pr{B}} = 1.
  \end{align*}

\end{solution}

\end{problem}

%%%%%%%%%%%%%%%%%%%%%%%%%%%%%%%%%%%%%%%%%%%%%%%%%%%%%%%%%%%%%%%%%%%%%
% Problem ends here
%%%%%%%%%%%%%%%%%%%%%%%%%%%%%%%%%%%%%%%%%%%%%%%%%%%%%%%%%%%%%%%%%%%%%

\endinput
