\documentclass[problem]{mcs}

\begin{pcomments}
  \pcomment{PS_conditional_space}
  \pcomment{from: new S10.ps11 by Rich}
\end{pcomments}

\pkeywords{
  probability_function
  sample_space
}

%%%%%%%%%%%%%%%%%%%%%%%%%%%%%%%%%%%%%%%%%%%%%%%%%%%%%%%%%%%%%%%%%%%%%
% Problem starts here
%%%%%%%%%%%%%%%%%%%%%%%%%%%%%%%%%%%%%%%%%%%%%%%%%%%%%%%%%%%%%%%%%%%%%

\newcommand{\prb}[1]{\mathop{\textup{Pr}_B}\nolimits\left\{#1\right\}}

\begin{problem}
  Suppose $\pr{}: \sspace\to [0,1]$ is a \idx{probability function} on
  a sample space, $\sspace$, and let $B$ be an event such that $\pr{B}
  > 0$.  Define a function $\prb{\cdot}$ on events outcomes $w
  \in \sspace$ by the rule:
\begin{equation}\label{prsubbdef}
  \prb{w} \eqdef \begin{cases} \pr{w}/\pr{B} &\text{ if } w \in B,\\
                               0 & \text{ if } w \notin B.
                  \end{cases}
\end{equation}

\bparts

\ppart Prove that $\prb{\cdot}$ is also a probability function on
$\sspace$ according to Definition~\bref{LN12:probsp}.

\begin{solution}
  We must show that $\prb{w} \geq 0$ for all outcomes $w \in \sspace$,
  and
\begin{equation}\label{prbss=1}
  \sum_{w \in \sspace} \prb{w} = 1.
\end{equation}

  But obviously $\prb{w} \geq 0$ since both the numerator, $\pr{w}$,
  and the denominator, $\pr{B}$, in~\eqref{prsubbdef} are nonnegative.
  Also~\eqref{prbss=1} holds because
  \begin{align*}
    \sum_{w \in \sspace} \prb{w}
       & = \sum_{w \in B} \prb{w} + \sum_{w \notin B} \prb{w}\\
       & = \sum_{w \in B} \frac{\pr{w}}{\pr{B}} + \sum_{w \notin B} 0\\
       & = \frac{\pr{B}}{\pr{B}} + 0 = 1.
  \end{align*}

\end{solution}

\ppart Prove that
\[
\prb{A} = \frac{\pr{A \intersect B}}{\pr{B}}
\]
for all $A \subseteq \sspace$.

\begin{solution}

\begin{align*}
\prb{A}
   & \eqdef \sum_{w \in A} \frac{\pr{w}}{\pr{B}}\\
   & = \sum_{w \in A \intersect B} \prb{w} + \sum_{w \in A - B} \prb{w}\\
   & = \sum_{w \in A \intersect B} \frac{\pr{w}}{\pr{B}} + \sum_{w \in A - B} \frac{\pr{w}}{\pr{B}}\\
   & = \frac{\sum_{w \in A \intersect B} \pr{w}}{\pr{B}} + \sum_{w \in A - B} 0\\
   & = \frac{\pr{A \intersect B}}{\pr{B}}.\\
\end{align*}

\end{solution}
\eparts

\end{problem}

%%%%%%%%%%%%%%%%%%%%%%%%%%%%%%%%%%%%%%%%%%%%%%%%%%%%%%%%%%%%%%%%%%%%%
% Problem ends here
%%%%%%%%%%%%%%%%%%%%%%%%%%%%%%%%%%%%%%%%%%%%%%%%%%%%%%%%%%%%%%%%%%%%%

\endinput
