\documentclass[problem]{mcs}

\begin{pcomments}
\pcomment{MQ_inclusion-exclusion}
\pcomment{adapted from S04.final}
\pcomment{ARM 12/13/13}
\end{pcomments}

\pkeywords{
  counting
  combinatorial_proof
  inclusion-exclusion
}

\begin{problem}
Each day, an MIT student selects a breakfast from among $b$
possibilities, lunch from among $l$ possibilities, and dinner from
among $d$ possibilities.  In each case one of the possibilities is
Doritos.  However, a legimate daily menu may include Doritos for at
most one meal.  Give a combinatorial (not algebraic) proof based on
the number of legimate daily menus that
\[
(b-1)(l-1)(d-1) + bl + bd +  ld = bld-(b+l+d)+1
\]

\begin{solution}
Both sides of the equation equal the number of legimate daily menus.

The terms on the left hand side respectively count the number of menus
with no Doritos, Doritos for only breakfast, Doritos only for lunch,
and Doritos only for dinner, so their sum counts the number of
legimate menus.

On the right hand side, $bld$ counts the number of possible menus
allowing any number of Doritos, $(b+l+d)$ is the number of
illegimate menus including Doritos at two or more meals (for example,
$b$ is the number menus with Doritos at both lunch and dinner), and 1
is the number of menus with Doritos at all three meals.  So by
inclusion-exclusion, these three terms combine to count the number of
legimate menus.
\end{solution}

\end{problem}

\endinput
