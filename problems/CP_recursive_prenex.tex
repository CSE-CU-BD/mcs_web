\documentclass[problem]{mcs}

\begin{pcomments}
   \pcomment{CP_recursive_prenex}
   \pcomment{depends on CP_prenex} 
   \pcomment{PS_recursive_variable_convention is a sequel}
   \pcomment{ARM 3/4/16}
\end{pcomments}

\pkeywords{ predicate_formula recursive free_varianble structural_induction }

\newcommand{\xbod}{\ensuremath{\mathop{\mathrm{bod}}}}
\newcommand{\qnts}{\ensuremath{\mathop{\mathrm{qnt}}}}

\begin{problem}
Defining predicate formulas as a recursive data type leads to straightforward
recursive definitions of free variables, prenex forms, and many other formula
properties.

We'll only consider the propositional connectives $\ANDsym, \NOTsym$,
and $\IMPsym$, since these are enough to illustrate how to handle all
the other propositional connectives.  For simplicity, the only atomic
formulas allowed will be predicate symbols applied to variables.  So
function symbols and the equality symbol ``$\mathbf{=}$'' will not
appear.

\begin{definition*}
\inductioncase{Base case}: $P(u,\dots,v)$ is a predicate formula,
where $P$ is a predicate symbol and $u,\dots,v$ are variable symbols,
not necessarily distinct.

\inductioncase{Constructor step}: 
If $F$ and $G$ are predicate formulas, then so are
\[
 \NOTsym(F),
 (F \ANDsym G),
 (F \IMPsym G),
 (\exists x.\, F),
 (\forall x.\, F),
\]
where $x$ is a variable.
\end{definition*}

\bparts

\ppart Give a recursive definition of the set $\fvrbls(F)$ of \emph{free
  variables} in a predicate formula $F$.  (A variable $x$ is free iff it has an
occurrence that is not inside any subformula of the form $\exists x.[\dots]$ or
$\forall x.[\dots]$.)

\begin{solution}
\inductioncase{Base cases} 
\[
\fvrbls(P(u,\dots,v)) \eqdef \set{u,\dots,v}.
\]

\inductioncase{Constructor cases}
\begin{align*}
\fvrbls(\NOTsym(F))   & \eqdef \fvrbls(F),\\
\fvrbls((F \ANDsym G)) & \eqdef \fvrbls(F) \union \fvrbls(G),\\
\fvrbls((F \IMPsym G)) & \eqdef \fvrbls(F) \union \fvrbls(G),\\
\fvrbls(\exists x.\, F) & \eqdef \fvrbls(F) - \set{x},\\
\fvrbls(\forall x.\, F) &  \eqdef \fvrbls(F) - \set{x}.
\end{align*}
\end{solution}

\eparts

A predicate formula is in \emph{prenex form} when all its quantifiers
appear at the beginning of the formula; some examples were given in
Problem~\bref{CP_prenex}.  More formally, a \emph{prenex formula} is a predicate
formula of the form
\[
\qnts(F).\,\xbod(F),
\]
where $\xbod(F)$ is a quantier-free formula, and $\qnts{F}$ is a
(possibly empty) sequence of quantifiers
\[
Q_1 x_1.\, Q_2 x_2.\, \dots Q_n x_n.\, 
\]
where $Q_i$ is ``$\forall$'' or ``$\exists$,'' and the $x_i$'s are
distinct variables.

It is easy to convert a formula into an equivalent prenex formula if the given
formula satisfies the \emph{unique variable convention}: for each variable $x$,
there is at most one occurrence of either a ``$\forall x$'' or an ``$\exists
x$,'' and that if $x$ is a free variable, there is no occurrence of a
``$\forall x$'' or an ``$\exists x$'' anywhere else in the formula.  (A way of
renaming variables to convert a formula into an equivalent one satisfying the
unique variable convention was illustrated in
Problem~\bref{CP_variable_convention}.  The general renaming procedure is given
in Problem~\bref{PS_recursive_variable_convention}.)

\bparts

\ppart Let $F$ be a predicate formula satisfying the unique variable convention.
Define a recursive procedure based on the recursive definition of predicate
formulas that converts $F$ into an equivalent prenex formula whose body is the
same as $F$'s with the quantifiers erased.  (That is, the propositional
connectives of $F$ don't change, in contrast to the method suggested in
Problem~\bref{CP_prenex}.)

It will help to use the notation ``$\overline{\qnts(F)}$'' for the sequence
\[
\overline{Q_1} x_1.\, \overline{Q_2} x_2.\, \dots \overline{Q_n} x_n.\, 
\]
where $\overline{\forall} \eqdef \exists$ and $\overline{\exists}
\eqdef \forall$.

For example, the variable convention ensures that if $(Q x. F)$ is a
subformula somewhere, then there are no free occurrences of $x$
anywhere.  Therefore,
\[
(Q x. F) \ANDsym G\quad  \text{is equivalent to} \quad Q x. (F \ANDsym G),
\]
because we can think of $G$ as simply being \True\ or \False.  This
explains why the following definition of $\qnts((F \ANDsym G))$ and
$\xbod((F \ANDsym G))$ preserves equivalence:
\begin{align*}
\xbod((F \ANDsym G)) & \eqdef (\xbod(F) \ANDsym \xbod(G)),\\
\qnts((F \ANDsym G)) & \eqdef \qnts(F),\qnts(G).
\end{align*}

\begin{solution}
\inductioncase{Base cases} 
\begin{align*}
\xbod(P(u,\dots,v))& \eqdef P(u,\dots,v),\\
\qnts(P(u,\dots,v)) & \eqdef \emptystring.
\end{align*}

\inductioncase{Constructor cases}
\begin{align*}
\xbod(\NOTsym(F))   & \eqdef \NOTsym(\xbod(F)),\\
\qnts(\NOTsym(F)))   & \eqdef \overline{\qnts}(F),\\
\xbod((F \ANDsym G)) & \eqdef (\xbod(F) \ANDsym \xbod(G)),\\
\qnts((F \ANDsym G)) & \eqdef \qnts(F),\qnts(G)
      &\text{(concatenation of the quantifier lists)}\\
\xbod((F \IMPsym G)) & \eqdef \xbod(F) \IMPsym \xbod(G),\\
\qnts((F \IMPsym G)) & \eqdef \overline{\qnts(F)},\qnts(G)\\
\xbod(\exists x.\, F) & \eqdef \xbod(F)\\
\qnts(\exists x.\, F) & \eqdef \exists x,\qnts(F)\\
\xbod(\forall x.\, F) &  \eqdef \xbod(F)\\
\qnts(\forall x.\, F) & \eqdef \forall x,\qnts(F).
\end{align*}
\end{solution}

\begin{staffnotes}
The variable convention justifies the other cases similarly.
\end{staffnotes}

\eparts

\end{problem}

\endinput 
