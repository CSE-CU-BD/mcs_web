\documentclass[problem]{mcs}

\begin{pcomments}
  \pcomment{FP_tree_kcolor}
  \pcomment{subsumes FP_tree_color_induction}
  \pcomment{ARM 5/9/14}
\end{pcomments}

\pkeywords{
  coloring
  tree
  induction
}

%%%%%%%%%%%%%%%%%%%%%%%%%%%%%%%%%%%%%%%%%%%%%%%%%%%%%%%%%%%%%%%%%%%%%
% Problem starts here
%%%%%%%%%%%%%%%%%%%%%%%%%%%%%%%%%%%%%%%%%%%%%%%%%%%%%%%%%%%%%%%%%%%%%

\begin{problem}
Prove by induction that, using a fixed set of $n>1$ colors, there are
exactly $n\cdot (n-1)^{m-1}$ different
colorings\inhandout{\footnote{That is, an assignment of colors to
    vertices so that no two adjacent vertices are assigned the same
    color.}} of any tree with $m$ vertices.

\begin{solution}

\begin{proof}
By induction on the number of vertices, $m$.  

\inductioncase{Induction hypothesis}: $P(m) \eqdef$ For all $m$-vertex
trees, $T$, there are $n\cdot (n-1)^{m-1}$ different colorings of $T$.

\inductioncase{Base case}: ($m=1$).  There are $n = n (n-1)^{1-1}$
ways to color one vertex.

\inductioncase{Induction step}: Let $T$ be a tree with $m+1$ vertices
for some $m \geq 1$.  Let $v$ be a leaf of $T$.  Then removing $v$ and
its incident edge, we obtain a tree $T-v$ with $m$ vertices.  We may
assume by induction that there are $n \cdot (n-1)^{m-1}$ colorings
of $T-v$.  For each such coloring of $T-v$, there are $n-1$ ways
to assign a color to $v$ to obtain an coloring of $T$, so there
are $n \cdot (n-1)^{m-1} \cdot (n-1) = n \cdot (n-1)^m$ colorings of
$T$, which proves $P(m+1)$.

\inductioncase{Conclusion}: Thus, there are exactly $n\cdot
(n-1)^{m-1}$ different colorings of any tree with $m$ vertices.  
\end{proof}

\end{solution}

\end{problem}

%%%%%%%%%%%%%%%%%%%%%%%%%%%%%%%%%%%%%%%%%%%%%%%%%%%%%%%%%%%%%%%%%%%%%
% Problem ends here
%%%%%%%%%%%%%%%%%%%%%%%%%%%%%%%%%%%%%%%%%%%%%%%%%%%%%%%%%%%%%%%%%%%%%

\endinput
