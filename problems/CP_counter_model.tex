\documentclass[problem]{mcs}

\begin{pcomments}
  \pcomment{CP_counter_model}
  \pcomment{lightweight practice problem, not so much for class}
  \pcomment{from: S09.cp2t}
\end{pcomments}

\pkeywords{
  counter_model
  quantifiers
  predicate_calculus
}

%%%%%%%%%%%%%%%%%%%%%%%%%%%%%%%%%%%%%%%%%%%%%%%%%%%%%%%%%%%%%%%%%%%%%
% Problem starts here
%%%%%%%%%%%%%%%%%%%%%%%%%%%%%%%%%%%%%%%%%%%%%%%%%%%%%%%%%%%%%%%%%%%%%

\begin{problem}
Show that
\[
(\forall x \exists y.\; P(x,y)) \QIMPLIES \forall z.\; P(z,z)
\]
is not valid by describing a counter-model.

\begin{solution}
Let $P(x,y)$ mean $x \neq y$.  Then the conclusion $\forall z.\; z \neq z$
is always false, but in any domain with two or more elements, the
hypothesis is true.
\end{solution}
\end{problem}

%%%%%%%%%%%%%%%%%%%%%%%%%%%%%%%%%%%%%%%%%%%%%%%%%%%%%%%%%%%%%%%%%%%%%
% Problem ends here
%%%%%%%%%%%%%%%%%%%%%%%%%%%%%%%%%%%%%%%%%%%%%%%%%%%%%%%%%%%%%%%%%%%%%

\endinput
