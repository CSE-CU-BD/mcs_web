\documentclass[problem]{mcs}

\begin{pcomments}
  \pcomment{CP_expected_number_of_keys}
  \pcomment{overlaps PS_n_keys, FP_expected_number_of_keys}
  \pcomment{from: S08.cp14m, F15.final}
  \pcomment{revised ARM 12/12/15}
\end{pcomments}

\pkeywords{
  random_variable
  expectation
  indicator_variable
  failure
}

%%%%%%%%%%%%%%%%%%%%%%%%%%%%%%%%%%%%%%%%%%%%%%%%%%%%%%%%%%%%%%%%%%%%%
% Problem starts here
%%%%%%%%%%%%%%%%%%%%%%%%%%%%%%%%%%%%%%%%%%%%%%%%%%%%%%%%%%%%%%%%%%%%%

\begin{problem}
A man has a set of $n$ keys, one of which fits the door to his
apartment.  He tries the keys randomly until he finds the key that
fits.  Let $T$ be the number of times he tries keys until he finds the
right key.

\bparts

\ppart Suppose each time he tries a key that does not fit the door, he
simply puts it back.  This means he might try the same ill-fitting key
several times before he finds the right key.  What is $\expect{T}$?

\hint Mean time to failure.

\examspace[1in]

\begin{solution}
Take finding the right key to be a ``failure''.  So the probability of
failure on the any given try, given ``success'' on the previous tries, is
$1/n$.  Therefore $\expect{T} = 1/(1/n) = n$.
\end{solution}
\eparts
\medskip

Now suppose he throws away each ill-fitting key that he tries.  That
is, he chooses keys randomly from \emph{among those he has not yet
  tried}.  This way he is sure to find the right key within $n$ tries.

\bparts

\ppart\label{mkleft} If he hasn't found the right key yet and there
are $m$ keys left, what is the probability that he will find the right
key on the next try?
\begin{solution}
\[
\frac{1}{m}.
\]
\end{solution}

\ppart\label{prT>kT>k1} Given that he did not find the right key on
his first $k-1$ tries, verify that the probability that he does not
find it on the $k$th trial is given by
\[
\prcond{T > k}{T > k-1} = \frac{n-k}{n-(k-1)}.
\]

\examspace[1.0in]

\begin{solution}
Given that the right key has not been picked in the first $k-1$
trials, there are $n-(k-1)$ keys left, so by part~\eqref{mkleft}, the
probability that the $k$th trial is also not the right key is
\[
1- \frac{1}{n-(k-1)} = \frac{n-k}{n-(k-1)}.
\]
\end{solution}

\ppart Prove that 
\begin{equation}\label{T>kn-kn}
\pr{T > k} =  \frac{n-k}{n}.
\end{equation}

\hint This can be argued directly, but if you don't see how, induction
using part~\eqref{prT>kT>k1} will work.

\examspace[2.0in]

\begin{solution}

Suppose the man keeps going until he has tried all $n$ keys.  In this
case, it's clear that the right key is equally likely to come up on
each try, hence the probability that it comes up after the $k$th try is
\[
(n-k)\cdot \frac{1}{n} = \frac{n-k}{n}.
\]

Alternatively, we can prove~\eqref{T>kn-kn} by induction on $k$,
using~\eqref{T>kn-kn} as the induction hypothesis.

\inductioncase{Base case} {$k=0$}: He has to try at least one key,
which means $\pr{T>0} = 1 = (n-0)/n$, so~\eqref{T>kn-kn} holds in this
  case.

\inductioncase{Inductive step}:
We must show that for $k\geq 0$,
\[
\pr{T > k+1} =  \frac{n-(k+1)}{n}.
\]

Now we have
\begin{align*}
\pr{T > k+1}
 & = \prcond{T > k+1}{T > k} \cdot \pr{T > k}
        & \text{(Conditional Product Rule)} \\
 & = \frac{n-(k+1)}{n-k} \cdot \pr{T > k}
        & \text{(part~\eqref{prT>kT>k1})}\\
 & = \frac{n-(k+1)}{n-k}\frac{n-k}{n}
        & \text{(induction hypothesis)}\\
 & = \frac{n-(k+1)}{n}.
\end{align*}
\end{solution}

\ppart Conclude that in this case
\[
\expect{T} = \frac{n+1}{2}.
\]

\examspace[1.5in]

\begin{solution}
We've already argued that
\[
\pr{T = k} = \frac{1}{n}.
\]
This can also be proved using~\eqref{T>kn-kn}:
\begin{align*}
\pr{T = k} 
  & = \pr{T > k-1} - \pr{T > k}\\
  & = \frac{n-(k-1)}{n} - \frac{n-k}{n}\\
  & = \frac{1}{n}.
\end{align*}
Therefore,
\[
\expect{T} = \sum_{k=1}^n k \cdot \pr{T=k}  = \frac{1}{n}\sum_{k=1}^n{k} = \frac{n+1}{2}.
\]

\end{solution}

\eparts

\end{problem}

\endinput
