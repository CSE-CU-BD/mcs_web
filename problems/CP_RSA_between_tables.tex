\documentclass[problem]{mcs}

\begin{pcomments}
  \pcomment{CP_RSA_between_tables}
  \pcomment{from: S09.cp8r, S06.cp7w}
\end{pcomments}

\pkeywords{
  number_theory
  modular_arithmetic
  primes
  RSA
  Pulverizer
}

%%%%%%%%%%%%%%%%%%%%%%%%%%%%%%%%%%%%%%%%%%%%%%%%%%%%%%%%%%%%%%%%%%%%%
% Problem starts here
%%%%%%%%%%%%%%%%%%%%%%%%%%%%%%%%%%%%%%%%%%%%%%%%%%%%%%%%%%%%%%%%%%%%%

\begin{problem}

Let's try out RSA!  \inhandout{There is a complete description of the
  algorithm in the text box.  You'll probably need extra paper.
  \textbf{Check your work carefully!}}

\bparts

\ppart As a team, go through the \textbf{beforehand} steps.

\begin{itemize}

\item Choose primes $p$ and $q$ to be relatively small, say in the
range 10-40.  In practice, $p$ and $q$ might contain several hundred
digits, but small numbers are easier to handle with pencil and paper.

\item Try $e = 3, 5, 7, \dots$ until you find something that works.
Use Euclid's algorithm to compute the gcd.

\item Find $d$ (using the Pulverizer---see appendix for a reminder on how
the Pulverizer works---or Euler's Theorem).

\end{itemize}

When you're done, put your public key on the board.  This lets another
team send you a message.

\ppart Now send an encrypted message to another team using their
public key.  Select your message $m$ from the codebook below:

\begin{itemize}

\item 2 = Greetings and salutations!

\item 3 = Yo, wassup?

\item 4 = You guys are slow!

\item 5 = All your base are belong to us.

\item 6 = Someone on \emph{our} team thinks someone on \emph{your} team
is kinda cute.

\item 7 = You \emph{are} the weakest link.  Goodbye.

\end{itemize}

\ppart Decrypt the message sent to you and verify that you received
what the other team sent!

\eparts

\instatements{\textbox{
%\begin{minipage}{\textwidth}
\textboxheader{The RSA Cryptosystem}

\begin{description}

\item[Beforehand] The receiver creates a public key and a secret key
as follows.

\begin{enumerate}

\item Generate two distinct primes, $p$ and $q$.  Since they can be
used to generate the secret key, they must be kept hidden.

\item Let $n = pq$.

\item Select an integer $e$ such that $\gcd(e, (p-1)(q-1)) = 1$.\\ The
\emph{public key} is the pair $(e, n)$.  This should be distributed
widely.

\item Compute $d$ such that $de \equiv 1 \pmod{(p-1)(q-1)}$.  This
can be done using the Pulverizer.\\ The
\emph{secret key} is the pair $(d, n)$.  This should be kept hidden!

\end{enumerate}

\item[Encoding]
Given a message~$m$, the sender first checks that $\gcd(m, n) =
1$.

\iffalse
\footnote{It would be very bad if $\gcd(m, n)$ equals $p$ or $q$
since then it would be easy for someone to use the encoded message to
compute the secret key  If $\gcd(m, n) = n$, then the encoded message
would be~0, which is fairly useless.  For very large values of~$n$, it
is extremely unlikely that $\gcd(m, n) \ne 1$.  If this does happen,
you should get a new set of keys or, at the very least, add some bits
to~$m$ so that the resulting message is relatively prime to~$n$.}
\fi

The sender then encrypts message $m$ to produce $m^*$ using
the public key:
\[
m^* = \rem{m^e}{n}.
\]

\item[Decoding] The receiver decrypts message $m^*$ back to message $m$
using the secret key:
\[
m = \rem{(m^*)^d}{n}.
\]

\end{description}
%\end{minipage}
}}


\end{problem}

%%%%%%%%%%%%%%%%%%%%%%%%%%%%%%%%%%%%%%%%%%%%%%%%%%%%%%%%%%%%%%%%%%%%%
% Problem ends here
%%%%%%%%%%%%%%%%%%%%%%%%%%%%%%%%%%%%%%%%%%%%%%%%%%%%%%%%%%%%%%%%%%%%%

\endinput
