\documentclass[problem]{mcs}

\begin{pcomments}
  \pcomment{CP_pred_calc_model_N}
\end{pcomments}

\pkeywords{
  predicate
  logic
  nonnegative
  model
}

%%%%%%%%%%%%%%%%%%%%%%%%%%%%%%%%%%%%%%%%%%%%%%%%%%%%%%%%%%%%%%%%%%%%%
% Problem starts here
%%%%%%%%%%%%%%%%%%%%%%%%%%%%%%%%%%%%%%%%%%%%%%%%%%%%%%%%%%%%%%%%%%%%%

\begin{problem}
Suppose $L(,)$ is a predicate symbol taking two arguments.  Write down
formulas of predicate calculus involving only the predicate $L$ and
equality whose possible models contain a copy of the consecutive
nonnegative integers $\naturals$ and as little else in the model
domain as possible.  (If you think you have formulas whose only models are exactly like
$\naturals$, look again, because that is impossible.)

\hint Write formula so that $L(x,y)$ behaves as much as possible like
$<$ on $\naturals$.  For this purpose, it's suggestive to write $x\,
L\, y$ instead of $L(x,y)$.

For example to express $y = x+1$ you could write
\[
x\,L\,y \QAND \forall z.\, (z\,L\,y \QIMPLIES (z=x  \QOR z\,L\,x).
\]


\begin{solution}
Assert that $L$ is \emph{transitive}
\[
\forall x,y,z.\, (x\,L\,y \QAND y\,L\,z) \QIMPLIES x\,L\,z,
\]
and that $L$ is \emph{irreflexive}
\[
forall x.\, \QNOT(x\,L\,x).
\]

Further every two different elements are related by $L$ one way or the
other:
\[
\forall x,y.\, x\,L\,y \QOR y\,L\,x \QOR x = y.
\]

Then say that $x$ is $L$-minimum element:
\[
M(x) \eqdef \forall y.\, \QNOT(y\, L\, x), 
\]
and there is only one $L$-minimum element, which we call $\mathbf{0}$:
\[
\forall x,y. (M(x) \QAND M(y)) \QIMPLIES x = y.
\]

Now the only models of all these formulas is a copy of $\naturals$:
\[
\mathbf{0}, \mathbf{0+1}, \mathbf{(0+1)+1}, \dots
\]
along with some two-way infinite chains
\[
\dots c_{-2}\, L \, c_{-1}\, L\, c_0\, L\, c_1\, L\, c_2\, \dots
\]
all of whose element are $L$-bigger than all the elements
corresponding to nonnegative integers.

These chains will have the property that every element of one chain
will be $L$ of every member of the other chain.  So the chains
themselves are ``ordered'' by $L$, but there is nothing further that
can be said of how chains are ordered: there might be a finite number,
even 0, such chains, or there might be an infinite number of them
ordered like the integers, or nonnegative integers, or real
numbers,\dots.

\end{solution}

\end{problem}
a
%%%%%%%%%%%%%%%%%%%%%%%%%%%%%%%%%%%%%%%%%%%%%%%%%%%%%%%%%%%%%%%%%%%%%
% Problem ends here
%%%%%%%%%%%%%%%%%%%%%%%%%%%%%%%%%%%%%%%%%%%%%%%%%%%%%%%%%%%%%%%%%%%%%

\endinput
