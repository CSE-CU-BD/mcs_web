\documentclass[problem]{mcs}

\begin{pcomments}
  \pcomment{CP_pred_calc_model_N}
  \pcomment{variant of CP_pred_calc_model_N_arrows}
\end{pcomments}

\pkeywords{
  predicate
  logic
  nonnegative
  model
}

\newcommand{\zerosym}{\textbf{\~0}}
\newcommand{\nextsym}{\textbf{next}}
\newcommand{\nextfun}{\textit{nextf}}
\newcommand{\prevsym}{\textbf{prev}}
\newcommand{\prevfun}{\textit{prevf}}
\newcommand{\numerl}[1]{\widetilde{\text{#1}}}
\newcommand{\nextarrow}{\stackrel{\nextsym}{\longrightarrow}}
\newcommand{\prevarrow}{\stackrel{\prevsym}{\longleftarrow}}
\newcommand{\nparrow}{\stackrel{\nextarrow}{\prevarrow}}
\newcommand{\ltsym}{\mrel{\textbf{less}}}


\newcommand{\nextel}{\emph{next}}
\newcommand{\prev}{\emph{prev}}


%%%%%%%%%%%%%%%%%%%%%%%%%%%%%%%%%%%%%%%%%%%%%%%%%%%%%%%%%%%%%%%%%%%%%
% Problem starts here
%%%%%%%%%%%%%%%%%%%%%%%%%%%%%%%%%%%%%%%%%%%%%%%%%%%%%%%%%%%%%%%%%%%%%

\begin{problem}
Let $\zerosym$ be a constant symbol, $\nextsym()$ and $\prevsym()$ be
function symbols taking one argument.

The aim of this problem is to develop a series of predicate formulas
using only these symbols whose models must contain contain a ``copy''
of the consecutive nonnegative integers $\nngint$.  Moreover, the
model must assign $\nextsym()$ and $\prevsym()$ to be the plus one and minus
one functions on the copy.

To start, we can introduce some abbreviations for certain terms called
\emph{numerals} built up using these symbols: namely, let
\begin{align*}
\numerl{1} & \eqdef \nextsym(\zerosym)\\
\numerl{2} & \eqdef \nextsym(\numerl{1})\\
\numerl{3} & \eqdef \nextsym(\numerl{2})\\
           &\vdots
\end{align*}
For example, ``$\numerl{3}$'' is an abbreviation for the numeral
\[
\nextsym(\nextsym(\nextsym(\zerosym))).
\]

Now we will make $\nextsym$ act like the ``add one'' function on values
of these terms.  In particular, adding one to two different elements
will result in two different elements:
\begin{equation}\label{next1to1}
\forall x,y.\, x \neq y \QIMPLIES\ \nextsym(x) \neq \nextsym(y)\ .
\end{equation}
We also make $\prevsym$ act like the ``subtract one'' function on these
values, namely subtracting one from zero has no effect:
\begin{equation}\label{prev00}
\prevsym(\zerosym)= \zerosym,
\end{equation}
and otherwise adding one undoes subtracting one.

\bparts

\ppart\label{previnvnext-part} Write a predicate formula expressing
the requirement that adding one undoes subtracting one from nonzero elements.
\begin{solution}
\begin{equation}\label{previnvnext}
\forall x\neq \zerosym.\, \nextsym(\prevsym(x)) = x\ .
\end{equation}
\end{solution}

\iffalse
\ppart Write a formula expressing the fact that subtracting one from
two different nonzero elements will result in two different elements.

I was going to add ``Explain why this formula is implied the earlier
ones,'' but no time to check and problem long enough already.

\begin{solution}
\begin{equation}\label{prev1to1}
\forall x,y \neq \zerosym.\, x \neq y \QIMPLIES\ \prevsym(x) \neq \prevsym(y)\ .
\end{equation}
\end{solution}
\fi

\eparts

Now any interpretation that satisifies these formulas must assign
different domain elements to each of the numerals.  (This is not
completely obvious, by the way, but we will take it for granted in the
rest of the problem.)  Moreover, interpretations of $\nextsym$ and
$\prevsym$ must act like plus one and minus one on these elements.

But although the values of the numerals in every model satisfying the
formulas of~\eqref{next1to1}, \eqref{prev00}, and
part~\eqref{previnvnext-part} form a copy of $\nngint$, a model may
have other elements that lead to strange behavior.  For example, a
model satisfying all the above formulas could have \emph{two} copies
that act like $\nngint$, with $\nextsym$ and $\prevsym$ acting like add
one and subtract one on each copy.  In this model, there would be two
elements that act like zero.

So let's fix this:
\bparts

\ppart Write a formula that forces there to be only one copy of
$\nngint$ in the model.

\begin{solution}
It's enough to say there is only one element that can start a copy:
\begin{equation}\label{uniqz}
\forall x.\, \prevsym(x) = x \QIMPLIES x = \zerosym.
\end{equation}
\end{solution}
\eparts

But there still might be other elements with funny properties.  For
example, there might be two elements, each of which was ``plus one''
of the other:
\[
\exists x, y.\, \nextsym(x) = y \QAND \nextsym(y) = x.
\]
or a cycle of three:
\[
\exists x, y,z.\, \nextsym(x) = y \QAND\ \nextsym(y) = z \QAND\ \nextsym(z) = x\ .
\]
We could easily write a formula forbidding such cycles of length
three, or forbidding cycles of any given length, but we would need an
infinite number of formulas to forbid cycles of all lengths.

To forbid all finite cycles using only a fixed number of formulas, we
will need something further: we will allow a binary relation symbol
$\ltsym$ and make it act like a less-than relation.  For this purpose,
it's suggestive to write $x\ltsym y$ instead of $\textbf{less}(x,y)$.
Then one requirement will be that for all $n,m \in \nngint$, if $n <
m$, then $\numerl{n}\ltsym \numerl{m}$ must be true.  But we also want
$\ltsym$ to define a relation that has the ``less-than'' properties on
all domain elements.  For example, we will say that adding one makes
numbers bigger with the formula
\begin{equation}\label{Lnext}
\forall x.\ x\ltsym \nextsym(x).
\end{equation}

\iffalse  CHECK THIS:
\bparts

\ppart Explain why the formulas above imply 
\begin{equation}\label{Lprev}
\forall x\neq \zerosym.\ \prevsym(x)\ltsym x\ .
\end{equation}

\begin{solution}
If $x\ltsym\nextsym(x)$ and $x \neq \zerosym$, then
\end{solution}
\eparts
\fi

Then we can forbid all finite cycles by requiring
\begin{equation}\label{irrform}
\forall x.\, \QNOT(x\ltsym x)\ .
\end{equation}

\bparts

\ppart Write down formulas of predicate calculus with only the symbols
above whose models must have properties as much like $\nngint$ as
you can manage.  In particular, make sure that all models of your
formulas force $\ltsym$ to mean $<$ on the numeral values and ensure
that~\eqref{irrform} implies there are no finite cycles of
$\nextsym$'s.  (If you think you have formulas whose only models are
exactly like $\nngint$, look again, because that is impossible.)

\begin{solution}
Assert that $\ltsym$ is \emph{transitive}
\[
\forall x,y,z.\, (x\ltsym y \QAND y\,L\,z) \QIMPLIES x\ltsym z,
\]

Further every two different elements are related by $L$ one way or the
other:
\begin{equation}\label{trichot}
\forall x,y.\, x\ltsym y \QOR y\ltsym ,x \QOR x = y.
\end{equation}

Now the only models of all these formulas has the a copy of
$\nngint$ along with some two-way infinite chains
\[
\dots \ltsym c_{-2}\, \ltsym c_{-1}\, \ltsym c_0\, \ltsym c_1\, \ltsym c_2\, \ltsym \dots.
\]

All the elements in one of these chains must be $L$-bigger than all
the numeral values.  Also,~\eqref{trichot} ensures all these chains
will have the property that every element of one chain will be $L$ of
every member of the other chain.  So the chains themselves are
``ordered'' by $L$, but there is nothing further that can be said of
how chains are ordered: there might be none, some finite number, or
even an infinite number of them.  And the chains might be ordered
among themselves like the integers, or nonnegative integers, or real
numbers,\dots.

\end{solution}

\ppart Describe a model that satisfies all your formulas but has
elements that are not the numeral values.

\begin{solution}
The numeral values along with a single two-way infinite chain of
successive elements ordered by $\ltsym$ is an example for the formulas
given above.
\end{solution}

\eparts

\end{problem}

%%%%%%%%%%%%%%%%%%%%%%%%%%%%%%%%%%%%%%%%%%%%%%%%%%%%%%%%%%%%%%%%%%%%%
% Problem ends here
%%%%%%%%%%%%%%%%%%%%%%%%%%%%%%%%%%%%%%%%%%%%%%%%%%%%%%%%%%%%%%%%%%%%%

\endinput
