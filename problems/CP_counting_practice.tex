\documentclass[problem]{mcs}

\begin{pcomments}
  \pcomment{CP_counting_practice}
  \pcomment{from: S09:cp10r}
  \pcomment{2nd part subsumed by CP_nonadjacent_books_counting_sequel now commented out}
\end{pcomments}

\pkeywords{
  multinomial
  counting
  bijection
  division_rule
  bookkeeper
  permutations
  combinations
}

%%%%%%%%%%%%%%%%%%%%%%%%%%%%%%%%%%%%%%%%%%%%%%%%%%%%%%%%%%%%%%%%%%%%%
% Problem starts here
%%%%%%%%%%%%%%%%%%%%%%%%%%%%%%%%%%%%%%%%%%%%%%%%%%%%%%%%%%%%%%%%%%%%%

\begin{problem}
Solve the following counting problems.  Define an appropriate mapping
(bijective or $k$-to-1) between a set whose size you know and the set
in question.

\bparts

\ppart An independent living group is hosting nine new candidates for
membership.  Each candidate must be assigned a task: 1 must wash pots, 2
must clean the kitchen, 3 must clean the bathrooms, 1 must clean the
common area, and 2 must serve dinner.  Write a multinomial
coefficient for the number of ways this can be done.

\begin{solution}
There is a bijection from sequences containing one
$P$, two $K$'s, three $B$'s, a $C$, and two $D$'s.  In any such
sequence, the letter in the $i$th position specifies the task assigned to
the $i$th candidate.  Therefore, the number of
possible assignments is:
\[
\binom{9}{1, 2, 3, 1, 2} \eqdef \frac{9!}{1!\ 2!\ 3!\ 1!\ 2!}
\]

\end{solution}

\iffalse

\ppart Write a multinomial coefficient for the number of nonnegative
integer solutions for the equation:

\begin{equation}\label{x5}
x_1 + x_2 + x_3 + x_4 + x_5 =  8.
\end{equation}

\begin{solution}
There is a bijection from solutions over $\naturals$ for~\eqref{x5} to
bit strings with eight \texttt{0}'s and four \texttt{1}'s.  Namely,
letting $\mathtt{0}^{x}$ represent a string of $x$ zeroes,
\[
(x_1,x_2,x_3,x_4,x_5) \in \naturals^5 \mapsto
\mathtt{0}^{x_1} \mathtt{1} \mathtt{0}^{x_2} \mathtt{1} \mathtt{0}^{x_3}
\mathtt{1} \mathtt{0}^{x_4} \mathtt{1} \mathtt{0}^{x_5}
\]

Therefore, there are
\[
\binom{12}{4}
\]
nonnegative integer solutions to~\eqref{x5}.
\end{solution}
\fi


\ppart How many nonnegative integers less than 1,000,000 have exactly
one digit equal to 9 and have a sum of digits equal to 17?


\begin{solution}
  We identify the nonnegative integers less than 1,000,000 with the length
  6 strings of decimal digits.  Then there is a bijection with pairs:
\[
(\text{position of the 9}, \text{successive values of other 5 digits})
\]

The sum of the other 5 digits is equal to 8, so the number of ways to
choose their values is equal to the number of solutions over the
nonnegative integers to
\begin{equation}
x_1 + x_2 + x_3 + x_4 + x_5 =  8,
\end{equation}
namely, $\binom{12}{4}$.  So by the product rule there are
\[
6 \cdot \binom{12}{4}
\]
such integers.
\end{solution}

\eparts

\end{problem}

%%%%%%%%%%%%%%%%%%%%%%%%%%%%%%%%%%%%%%%%%%%%%%%%%%%%%%%%%%%%%%%%%%%%%
% Problem ends here
%%%%%%%%%%%%%%%%%%%%%%%%%%%%%%%%%%%%%%%%%%%%%%%%%%%%%%%%%%%%%%%%%%%%%

\endinput
