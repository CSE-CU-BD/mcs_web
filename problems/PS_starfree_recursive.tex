\documentclass[problem]{mcs}

\begin{pcomments}
  \pcomment{PS_starfree_recursive}
  \pcomment{recursive data version of PS_starfree_language}
  \pcomment{F16ps3}
  \pcomment{same as closed under op version PS_starfree_language that
    is omitted from book; labels conflict}
  \pcomment{ARM 2/26/16}
\end{pcomments}

\pkeywords{
  sequence
  word
  complement
  union
  concatenation
  language
  starfree
}

\newcommand{\CDlang}{\text{C-D}}

%%%%%%%%%%%%%%%%%%%%%%%%%%%%%%%%%%%%%%%%%%%%%%%%%%%%%%%%%%%%%%%%%%%%%
% Problem starts here
%%%%%%%%%%%%%%%%%%%%%%%%%%%%%%%%%%%%%%%%%%%%%%%%%%%%%%%%%%%%%%%%%%%%%

\begin{problem}
At the lowest level of abstraction, documents in a written
language---natural or computer language--just consist of a sequence of
symbols.  In formal language theory, such sequences of symbols are
referred to as \emph{words}, though they could equally well be called
``phrases,'' or ``sentences.''

A \emph{binary word} is a finite sequence of \STR{0}'s and
\STR{1}'s.  For example, $(\STR{1},\STR{1},\STR{0})$ and
$(\STR{1})$ are words of length three and one, respectively.  We
usually omit the parentheses and commas in the descriptions of words,
so the preceding binary words would just be written as $\STR{110}$
and $\STR{1}$.

The basic operation of placing one word immediately after another is
called \term{concatentation}.  For example, the concatentation of
$\STR{110}$ and $\STR{1}$ is $\STR{1101}$, and the
concatentation of $\STR{110}$ with itself is $\STR{110110}$.

We can extend this basic operation on words to an operation on
\emph{sets} of words.  To emphasize the distinction between a word and
a set of words, from now on we'll refer to a set of words as a
\emph{language}.  Now if $R$ and $S$ are languages, then $R \cdot S$
is the language consisting of all the words you can get by
concatenating a word from $R$ with a word from $S$.  That is,
\[
R \cdot S \eqdef \set{rs \suchthat r \in R \QAND s \in S}.
\]
For example,
\[
\set{\STR{0},\STR{00}}\cdot \set{\STR{00},\STR{000}} =
\set{\STR{000},\STR{0000},\STR{00000}}
\]

Another example is $D \cdot D$, abbreviated as $D^2$, where $D \eqdef
\set{\STR{1}, \STR{0}}$ is just the two binary digits.
\[
D^2 = \set{\STR{00},\STR{01},\STR{10},\STR{11}}.
\]
In other words, $D^2$ is the language consisting of all the length two
words.  More generally, $D^n$ will be the language of length $n$
words.

If $S$ is a language, the language you can get by concatenating any
number of copies of words in $S$ is called $S^*$---pronounced ``$S$
star.''  (By convention, the empty word, $\emptystring$, always
included in $S^*$.)  For example, $\set{\STR{0},\STR{11}}^*$ is the
language consisting of all the words you can make by stringing
together $\STR{0}$'s and $\STR{11}$'s.  This language could also be
described as consisting of the words whose blocks of $\STR{1}$'s are
always of even length.  Another example is $(D^2)^*$, which consists
of all the even length words.  Finally, the language, $B$, of
\emph{all} binary words is just $D^*$.

The \emph{Concatenation-Definable} (\emph{\CDlang}) languages are defined recursively:

\begin{itemize}

\item \textbf{Base case}: Every finite language is a \CDlang.

\item \textbf{Constructor cases}:  If $L$ and $M$ are \CDlang's, then
\[
L\cdot M,\quad L \union M,\quad \text{and   } \bar{L}
\]
are \CDlang's.
\end{itemize}
Note that the $^*$-operation is \emph{not} allowed.  For this reason,
the c-d languages are also called the ``star-free languages,''
\cite{Meyer69}.

Lots of interesting languages turn out to be concatenation-definable,
but some very simple languages are not.  This problem ends with the
conclusion that the language $\set{\STR{00}}^*$ of even length words
whose bits are all \STR{0}'s is not a C-D language.

\begin{problemparts}
\ppart Show that the set $B$ of all binary words is C-D.
\hint The empty set is finite.
\begin{solution}
The empty set, $\emptyset$, is fiite and therefore C-D by base case
definition.  Therefore, $\bar{\emptyset}$ which equals $B$ is C-d
using the set-complement constructor
\end{solution}

Now a more interesting example of a C-D set is the language of all
binary words that include three consecutive \STR{1}'s:
\[
B\STR{111}B.
\]
Notice that the proper expression here is ``$B \cdot
\set{\STR{111}} \cdot B$.''  But it causes no confusion and helps
readability to omit the dots in concatenations and the curly braces
for sets with only one element.

\bparts 

\ppart Show that the language consisting of the binary words that
start with \STR{0} and end with \STR{1} is C-D.

\begin{solution}
$0B1$.
\end{solution}

\ppart Show that $\STR{0}^*$ is C-D.

\begin{solution}
This is simply the binary words that do not contain a \STR{1}
\[
\STR{0}^* = \bar{B \STR{1} B}.
\]
\end{solution}

\ppart\label{RiScd}  Show that if $R$ and $S$ are C-D, then so is $R \intersect S$.

\begin{solution}
By DeMorgan's Law for sets
\[
R \intersect S = \bar{\bar{R} \union \bar{S}}.
\]
\end{solution}

\ppart Show that $\set{\STR{01}}^*$ is C-D.

\begin{solution}
This language consists of the words that do not include \STR{00} or
\STR{11}, and start with \STR{0}, and end with \STR{1}, along with the
empty word, $\emptystring$, which is the complement of the set of
words of length one or more:
\begin{align*}
\set{\emptystring} & = \overline{\set{\STR{0},\STR{1}}B},\\
\set{\STR{01}}^*   & = (\bar{B\set{\STR{00},\STR{11}}B} \intersect 0B1) \union \set{\emptystring}.
\end{align*}

Another way to say this is that $\set{\STR{01}}^*$ consists of the
words that do \emph{not} start wrong, end wrong, or contain a wrong
substring:
\[
\set{\STR{01}}^* = \bar{1B \union B0 \union B\set{\STR{00},\STR{11}}B }
\]
\end{solution}

\eparts

Let's say a language $S$ is \STR{0}-\emph{finite} when it includes
only a finite number of words whose bits are all \STR{0}'s, that is,
when $S \intersect \STR{0}^*$ is a finite set of words.  A langauge
$S$ is \STR{0}-\emph{boring}---boring, for short---when either $S$ or
$\bar{S}$ is \STR{0}-finite.

\bparts

\ppart Explain why $\set{\STR{00}}^*$ is not boring.

\begin{solution}
The language $\set{\STR{00}}^*$ is an infinite set consisting of all
even length all-\STR{0} words, and so is not \STR{0}-finite.  Its
complement contains the infinite set of all the odd length all-\STR{0}
words, and so is also not \STR{0}-finite.
\end{solution}

\ppart\label{uc0simp} Verify that if $R$ and $S$ are boring, then so
is $R \union S$.

\begin{solution}
There are two cases:

\inductioncase{Case 1}: (Both $R$ and $S$ are \STR{0}-finite.)

Since the union of finite sets is finite, $R \union S$ must also be
\STR{0}-finite, so $R \union S$ is boring.

\inductioncase{Case 2}: (At least one $\bar{R}$ and $\bar{S}$ is
\STR{0}-finite.)

We can safely assume that it is $\bar{R}$ that is \STR{0}-finite.
But $\bar{R \union S} \subseteq \bar{R}$, so $\bar{R \union S}$ must
also be \STR{0}-finite.  Therefore $R \union S$ is boring in this
case as well.
\end{solution}

\ppart Verify that if $R$ and $S$ are boring, then so is $R \cdot S$.

\hint By cases: whether $R$ and $S$ are both \STR{0}-finite,
whether $R$ or $S$ contains no all-\STR{0} words at all (including
the empty word $\emptystring$), and whether neither of these cases
hold.

\begin{solution}
There are four cases.

\inductioncase{Case 1}: (Both $R$ and $S$ are \STR{0}-finite.)

Since the concatenation of two finite sets of words is finite, $R
\cdot S$ is \STR{0}-finite and therefore is boring in this case.

\inductioncase{Case 2}: (Either $R$ or $S$ contains no all-\STR{0} words.)

In this case, $R \cdot S$ won't contain any all-\STR{0} words
either, which means it is \STR{0}-finite and boring.

\inductioncase{Case 3}: ($\bar{R}$ is \STR{0}-finite.)

Let's say the length of the longest all-\STR{0} word in $\bar{R}$
is $n$.  Then $R$ must contain every all-\STR{0} word longer than
$n$.  We can assume that $S$ includes some all-\STR{0} word, say of
length $k$, since otherwise Case 2.\ would apply.  But this means that
that $R \cdot S$ contains every all-\STR{0} word longer than $n+k$.
So every all-\STR{0} word in $\bar{R \cdot S}$ has length at most
$n+k$, and there are only a finite number of such words.  Hence $R
\cdot S$ is \STR{0}-finite and therefore is boring.

\inductioncase{Case 4}: ($\bar{S}$ is \STR{0}-finite.)  Same proof as
for Case 3.
\end{solution}

\ppart\label{cdimp0s} Conclude by structural inductin that all C-D
languages are boring.

\begin{solution}
The starting c-d languages are the finite sets, and these are boring
by definition.  Also, $S$ is boring iff $\bar{S}$ is boring, because
$S = \bar{\bar{S}}$.  By part~\eqref{uc0simp}, we conclude that all
the operations for constructing c-d languages preserve boredome.
Hence all c-d languages are boring.
\end{solution}

So we have proved that the set $(\STR{00})^*$ of even length
all-\STR{0} words is not a C-D language.

\end{problemparts}

\end{problem}

%%%%%%%%%%%%%%%%%%%%%%%%%%%%%%%%%%%%%%%%%%%%%%%%%%%%%%%%%%%%%%%%%%%%%
% Problem ends here
%%%%%%%%%%%%%%%%%%%%%%%%%%%%%%%%%%%%%%%%%%%%%%%%%%%%%%%%%%%%%%%%%%%%%

\endinput


\iffalse

\eparts

Let $w$ be a binary word and $n$ a nonnegative integer.  A set $S$ of
words \emph{$n$-repeats $w$}, a word with $n$ consective occurrences
of $w$ is in $S$ iff that word with an extra occurrence of $w$ is also
in $S$.  That is,
\[
xw^ny \in S \QIFF xw^nwy \in S.
\]
We say a set of words is \emph{$n$-repeating} if it $n$-repeats every
word $w$.  For example, $\STR{0}^*$ is 1-repeating and $B$ is actually 0-repeating.

\begin{problemparts}
\ppart Verify that $\set{\STR{01}}^*$ is 2-repeating.

\begin{solution}
TBA
\end{solution}

\ppart Explain why $\set{\STR{00}}^*$ is not $n$-repeating for any $n$.

\begin{solution}
$\set{\STR{00}}^*$ does not $n$-repeat \STR{0} for any $n$,
  because \STR{0}^n is in $\set{\STR{00}}^*$ but $\STR{0}^n
  \STR{0}$ is not.
\end{solution}
\fi

 
