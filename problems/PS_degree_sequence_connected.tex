\documentclass[problem]{mcs}

\begin{pcomments}
\pcomment{PS_degree_sequence_connected}
\pcomment{part(a) is same as FP_degree_sequences}
\pcomment{ARM 4/5/17}
\end{pcomments}

\pkeywords{
 simple_graph
 tree
 degree
 handshaking
 induction
}

%%%%%%%%%%%%%%%%%%%%%%%%%%%%%%%%%%%%%%%%%%%%%%%%%%%%%%%%%%%%%%%%%%%%%
% Problem starts here
%%%%%%%%%%%%%%%%%%%%%%%%%%%%%%%%%%%%%%%%%%%%%%%%%%%%%%%%%%%%%%%%%%%%%

\begin{problem}

The \emph{degree sequence} of a simple graph $G$ with $n$ vertices is
the length-$n$ sequence of the degrees of the vertices listed in
weakly increasing order.  For example, if $G$ is a 3-vertex line
graph, then its degree sequence is $\ang{1,1,2}$.  On the other hand,
$\ang{0,0,2}$ is not a degree sequence, since in any graph with an
edge, there are at least two vertices of positive degree, namely, the
endpoints of the edge.

\bparts

\ppart Briefly explain why each of the following sequences is
\textbf{not} a degree sequence of any \emph{connected} simple graph.

\begin{enumerate}[(i)]

\item $\ang{1, 2, 3, 4, 5, 6, 7}$

\begin{solution}
There are only 7 vertices, so the degree of any vertex is at most 6.
\end{solution}

\examspace[1in]

\item $\ang{0, 2, 2, 2, 2}$
\begin{solution}
There is a vertex with degree 0, and there is more than one vertex, so
the graph is not connected.
\end{solution}

\examspace[1in]

\item $\ang{1, 3, 3, 4, 4, 4}$
\begin{solution}
By the Handshaking Lemma, the sum of degrees in any simple graph must
be even, which is not true in this case since $1 + 3 + 3+ 4 + 4 + 4 =
19$.
\end{solution}

\examspace[1in]

\item A sequence of $n$ nonnegative integers whose sum is less than
  $2n - 2$.

\begin{solution}
There are too few edges.  The sum of the degrees is twice the number
of edges by the Handshaking Lemma~\bref{sumedges}.  However, the
number of edges in any $n$-vertex connected graph is at least $n-1$
(Corollary~\bref{cor:n-1}).  Therefore, the sum of the degrees must be
at least $2n-2$.
\end{solution}

\examspace[1in]

\item $\ang{1, 2, 3, 4, 4}$

\begin{solution}
There are five vertices, two of which have degree 4.  So both degree-4
vertices have to be connected to \emph{all} the other vertices.  That
means that the degree of every vertex is greater than one, which is
violated in this case.

\end{solution}
\examspace[1in]

\end{enumerate}

\ppart Give an example of two simple graphs $G$ and $H$ with the same
degree sequence such that $G$ is not connected and $H$ is connected.

\examspace[1.5in]

\begin{staffnotes}
I think the following 5-vertex graphs are the smallest possible based
on a quick (read: ``unreliable'') analysis of simple graphs with at
most 4 vertices.  Would be nice to have a clean proof of this.
\end{staffnotes}

\begin{solution}
Let one component of $G$ be a triangle and the other a length-one line
graph.  Let $H$ be a length-4 line graph.  More formally:
\begin{align*}
\vertices{G} & \eqdef \Zintv{1}{5} = \vertices{H},\\
\edges{G}    & \eqdef \set{\edge{1}{2}, \edge{2}{3}, \edge{1}{3}, \edge{4}{5}},\\
\edges{H}    & \eqdef \set{\edge{1}{2}, \edge{2}{3}, \edge{3}{4}, \edge{4}{5}}.
\end{align*}
So $D \eqdef (1,1,2,2,2)$.
\end{solution}

\textbf{Optional}: Argue that your example has the smallest possible
number of vertices.

\eparts

\end{problem}

%%%%%%%%%%%%%%%%%%%%%%%%%%%%%%%%%%%%%%%%%%%%%%%%%%%%%%%%%%%%%%%%%%%%%
% Problem ends here
%%%%%%%%%%%%%%%%%%%%%%%%%%%%%%%%%%%%%%%%%%%%%%%%%%%%%%%%%%%%%%%%%%%%%

\endinput
 
