\documentclass[problem]{mcs}

\begin{pcomments}
  \pcomment{CP_partial_order_mq2}
  \pcomment{from: Rich Chan \& Megumi Ando(?) 10/5/09}
\end{pcomments}

\pkeywords{
  binary_relations
  relational_properties
  partial_orders
  minimal
  maximal
  total_order
  linear
}

%%%%%%%%%%%%%%%%%%%%%%%%%%%%%%%%%%%%%%%%%%%%%%%%%%%%%%%%%%%%%%%%%%%%%
% Problem starts here
%%%%%%%%%%%%%%%%%%%%%%%%%%%%%%%%%%%%%%%%%%%%%%%%%%%%%%%%%%%%%%%%%%%%%

\begin{problem} 

\mbox{}

\bparts

\ppart Each row in the following table starts with a partial order.
\iffalse
The second row starts with the divides relation on $\nngint$ where
$a \divides b$ iff $b=ak$ for some $k \in \nngint$.\fi
(The relation in the third row is on $\nngint$ and
the complex number $i$ where $a < b$ iff $a, b \in \nngint$ and $a
< b$.)

Fill in the remaining entries in each row.

\renewcommand\arraystretch{2}

\[\begin{array}{| c | c | c | c |}
    \hline
    \textbf{partial order} & \textbf{linear order}\ \text{(YES, NO)} & \hspace{0.2in}\textbf{minimal(s)}\hspace{0.2in} &\hspace{0.2in} \textbf{maximal(s)}\\ \hline
    \subseteq \text{ on } \power(\set{1, 2, 3})  & & &  \\ \hline
    \text{divides on } \nngint & &  & \\ \hline
    < \text{ on } \nngint \union \set{i}  & & &  \\
    \hline
\end{array}\]


\begin{solution}
\[\begin{array}{| c | c | c | c |}
    \hline
    \textbf{partial order} & \textbf{linear order}\ \text{(YES, NO)} & \hspace{0.2in}\textbf{minimal(s)}\hspace{0.2in} &\hspace{0.2in} \textbf{maximal(s)}\\ \hline
    \subseteq \text{ on } \power(\set{1, 2, 3})  & NO & \set{}& \set{1, 2, 3} \\ \hline
    \text{divides on } \nngint & NO & 1 & 0 \\ \hline
    < \text{ on } \nngint \union \set{i} &  NO & 0, i & i \\
    \hline
\end{array}\]
\end{solution}

\examspace[0.5in]
\eparts

The next problem parts ask about the subset relation $\subseteq$ on
$\power(\set{1, 2, 3})$.

\bparts

\ppart Give an example of a maximum length \emph{chain}.

\examspace[2in]

\begin{solution}
\[
\emptyset, \set{1}, \set{1, 2}, \set{1, 2, 3}.
\]

Another is
\[
\emptyset, \set{2}, \set{2,3}, \set{1, 2, 3}.
\]

\end{solution}

\ppart Give an example of a maximum size \emph{anti-chain}.

\examspace[2in]

\begin{solution}
\[
\set{1}, \set{2}, \set{3}.
\]

Another is 
\[
\set{1,2}, \set{2,3}, \set{1,3}.
\]
\end{solution}

\eparts

\end{problem} 


%%%%%%%%%%%%%%%%%%%%%%%%%%%%%%%%%%%%%%%%%%%%%%%%%%%%%%%%%%%%%%%%%%%%%
% Problem ends here
%%%%%%%%%%%%%%%%%%%%%%%%%%%%%%%%%%%%%%%%%%%%%%%%%%%%%%%%%%%%%%%%%%%%%

\endinput
