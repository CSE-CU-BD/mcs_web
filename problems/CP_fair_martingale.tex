\documentclass[problem]{mcs}

\begin{pcomments}
  \pcomment{DRAFT}
  \pcomment{CP_fair_martingale}
  \pcomment{close to FP_martingale}
  \pcomment{should resolve with hidden material in RV section on Infinite Sums}
  \pcomment{ARM 5/28/12, from material removed from deviation chapter}
  \pcomment{revision 5/1/14, may be subsumed by discussion in book: RV or DEV chapters}
\end{pcomments}

\pkeywords{
  martingale
  deviation
  pairwise_independent
  paradox
  roulette
  mean_time_to_failure
}

%%%%%%%%%%%%%%%%%%%%%%%%%%%%%%%%%%%%%%%%%%%%%%%%%%%%%%%%%%%%%%%%%%%%%
% Problem starts here
%%%%%%%%%%%%%%%%%%%%%%%%%%%%%%%%%%%%%%%%%%%%%%%%%%%%%%%%%%%%%%%%%%%%%

\begin{problem}
A gambler bets on the toss of a fair coin: if the toss is Heads, the
gambler gets back the amount he bet along with an additional the
amount equal to his bet.  Otherwise he loses the amount bet.  For
example, the gambler bets \$10 and wins, he gets back \$20 for a net
profit of \$10.  If he loses, he gets back nothing for a net profit of
$-\$10$---that is, a net loss of \$10.

Gamblers often try to develop betting strategies to beat the odds is
such a game.  A well known strategy of this kind is \emph{bet
  doubling}, namely, if he wins his first \$10 bet, he stops playing
and leaves with his \$10 profit.  If he loses the first bet, he bets
\$20 on the second toss.  Now if the second toss is Heads, he gets his
\$20 bet plus \$20 back and again walks away with a net profit of
$20-10 = \$10$.  If he loses the second toss, he bets \$40 on the
third toss, and so on.

You would think that any such strategy will be doomed: in a fair game
your expected win by definition is zero, so no strategy should have
nonzero expectation.  We can make this reasoning more precise as
follows:

\begin{quote}
Let $W_n$ be a random variable equal to the amount won in the
$n$th coin toss.  So with the bet doubling strategy starting with a
\$10 bet, $W_1 = \pm 10$ with equal probability.  If the betting ends
before the $n$th bet, define $W_n=0$.  So $W_2$ is zero with
probability 1/2, is 10 with probability 1/4, and is $-10$ with
probability 1/4.  Now letting $W$ be the amount won when the gambler
stops betting, we have
\[
W = W_1 + W_2 +\cdots+ W_n +\cdots.
\]
Furthermore, since each toss is fair,
\[
\expect{W_n} = 0
\]
for all $n>0$.  Now by linearity of expectation, we have
\begin{equation}\label{zeroexpectsum}
\expect{W} = \expect{W_1} + \expect{W_2} +\cdots+\expect{W_n} +\cdots = 0 + 0 + \cdots + 0 + \cdots = 0,
\end{equation}
confirming that with fair tosses, the expected win is zero.
\end{quote}

But wait a minute!
\bparts

\ppart Explain why the gambler is certain to win eventually if he
keeps betting.

\begin{solution}
The probability that he will never win is less than the probability
that he will have to bet more than $n$ times, which equals $(1/2)^n$.
Since this quantity approaches zero as $n$ grows, the probability of
never winning must be zero, and therefore the probability that he does
eventually win is one.
\end{solution}

\ppart Prove that when the gambler finally wins a bet, his net profit
is \$10.

\begin{solution}
On the $n$th bet before winning, the gambler bets $10\cdot 2^{n-1}$.
So prior to his $n$th bet, he has bet and lost
\[
\sum_{k=1}^{n-1} 10 \cdot 2^{k-1} = 10 \sum_{k=0}^{n-2} 2^k = 10 \cdot 2^{n-1} - 10,
\]
which is \$10 less than his $n$th bet.  So if he wins on the $n$th bet,
his net profit is \$10.
\end{solution}

\ppart Since the gambler's profit is always \$10 when he wins, and he
is certain to win, his expected profit is also \$10.  That is
\[
\expect{W} = 10,
\]
contradicting~\eqref{zeroexpectsum}.  So what's wrong with the
reasoning that led to the false conclusion~\eqref{zeroexpectsum}?

\begin{solution}
Equation~\eqref{zeroexpectsum} assumes that linearity holds for the
expected value of the sum of the \emph{infinite} sequence of random
variables $W_1,W_2,\cdots$, but this is a case where infinite
linearity fails.  The infinite linearity Theorem~(\bref{linexp})
requires that the sum of expectations of \emph{\idx{absolute value}s}
converges.  That is, infinite linearity would follow if the sum
\begin{equation}\label{sumabsC}
\sum_{n=1}^\infty \expect{\, \abs{W_n}\, }
\end{equation}
converged.  But
\begin{align*}
\expect{\, \abs{W_n}\, }
  & = (\abs{10 \cdot 2^{n-1}}) \cdot \prob{\text{1st Head on the $n$th toss}}\\
  &\quad  + (\abs{-10 \cdot 2^{n-1}}) \cdot \prob{\text{1st Head after the $n$th toss}}\\
  &\quad  + 0 \cdot \prob{\text{1st Head before the $n$th toss}}\\
  & = (10 \cdot 2^{n-1}) \cdot 2^{-n}
       + (10 \cdot 2^{n-1}) \cdot 2^{-n}
       + 0 = 10,
\end{align*}
so the sum~\eqref{sumabsC} equals $10+10+\cdots 10 + \cdots$ which
diverges---rapidly.

It would be 0 if the number of bets had a fixed bound.  If the gambler
only had enough money to make $b$ bets, then his expectation in the
fair game would be the sum of his expected wins on at most $b$ bets.
This is $\expect{\sum_{n=1}1^b W_b}$.  By linearity of expectation for
finite sums, the expectation is $b \cdot 0 = 0$.  But there is no such
fixed bound, and that changes things.

Probability theory truly leads to this absurd conclusion: a game
allowing an unbounded number of fair bets may not be fair in the end.
In fact, even against an \emph{unfair} wheel, as long as there is some
fixed positive probability of red on each spin, you are certain to win
$\$10$ playing the Martingale strategy!  \iffalse
(See Problem~\ref{MQ_martingale}.)\fi

\end{solution}

\end{problem}

%%%%%%%%%%%%%%%%%%%%%%%%%%%%%%%%%%%%%%%%%%%%%%%%%%%%%%%%%%%%%%%%%%%%%
% Problem ends here
%%%%%%%%%%%%%%%%%%%%%%%%%%%%%%%%%%%%%%%%%%%%%%%%%%%%%%%%%%%%%%%%%%%%%

\endinput


\iffalse ---also known as the St. Peterberg
strategy---because of an alleged attempt to bankrupt the famous
St. Petersberg casino in 19th century Russia.\fi 
