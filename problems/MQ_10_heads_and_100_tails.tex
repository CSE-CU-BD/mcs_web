\documentclass[problem]{mcs}

\begin{pcomments}
  \pcomment{MQ_10_heads_and_100_tails}
  \pcomment{contributed by Rich Chan 10/18/09; edited by ARM 10/18/09}
  \pcomment{revised ARM 5/14/10}
\end{pcomments}

\pkeywords{
  state_machines
  increasing
  decreasing
  derived_variables
  weakly_decreasing
  weakly_increasing
  constant
}

%%%%%%%%%%%%%%%%%%%%%%%%%%%%%%%%%%%%%%%%%%%%%%%%%%%%%%%%%%%%%%%%%%%%%
% Problem starts here
%%%%%%%%%%%%%%%%%%%%%%%%%%%%%%%%%%%%%%%%%%%%%%%%%%%%%%%%%%%%%%%%%%%%%

\begin{problem}
Start with some coins on a table, $H_0$ of them showing heads and
$T_0$ of them showing tails.

There are two ways to change the coins:

\begin{enumerate}
\item[(i)] Remove $10$ \emph{heads} and $10$ \emph{tails}.

\item[(ii)] If there are \emph{more tails than heads} on the table, add
  more heads so the number of heads is doubled.
\end{enumerate}

\bparts

\ppart
Model this situation as a state machine, carefully defining the set of
states, the start state and the possible state transitions.

\textbf{Reminder}: be sure to state the conditions of the state
transitions.

\examspace[1.5in]

\ppart Let $H \eqdef \text{the number of heads}$ on the table and
likewise $T \eqdef \text{the number of heads}$.  For each of the
derived variables below, indicate whether it is:
\begin{itemize}
\item \textbf{SI}: Strictly Increasing
\item \textbf{SD}: Strictly Decreasing
\item \textbf{WI}: Weakly Increasing but not constant
\item  \textbf{WD}: Weakly Decreasing but not constant
\item  \textbf{C}: Constant
\item  \textbf{N}: None of these
\end{itemize}
% \renewcommand{\labelenumi}{(\roman{enumi})}

\renewcommand{\theenumi}{\roman{enumi}}
\renewcommand{\labelenumi}{(\theenumi)}

\begin{enumerate}

% \item $H$
% \vspace{0.5in}

\item $T$ \hfill \examrule[0.5in]

\item $T+H$ \hfill \examrule[0.5in]

\item $T-H$ \hfill \examrule[0.5in]

\item $2T-H$ \hfill \examrule[0.5in]

\end{enumerate}

\begin{staffnotes}
This next part on termination had been commented out.  I like it, but
keeping things nonnegative may be tricky enough to warrant a further
hint.  Maybe add some invariants to identify such as $T \leq T_0$,
$2T-H \leq 2T_0 + H_0$,\dots
\end{staffnotes} 

 \ppart Prove that the machine will terminate, that is, no infinite
 sequence of transitions is possible.  \iffalse : any long enough
 sequence of transitions will arrive at a state in which no transition
 is possibleby providing a strictly decreasing derived variable and
 proving that it has a minimum value.\fi

\textbf{warning}: To prove termination using a strictly decreasing
derived variable and the Well Ordering Principle, its values must be
nonnegative.
 
\begin{solution}
$2T-H$ is SD and therefore so is $2T-H + 2T_0 + H_0 \geq 0$.
\end{solution}
 
\eparts
\end{problem}

%%%%%%%%%%%%%%%%%%%%%%%%%%%%%%%%%%%%%%%%%%%%%%%%%%%%%%%%%%%%%%%%%%%%%
% Problem ends here
%%%%%%%%%%%%%%%%%%%%%%%%%%%%%%%%%%%%%%%%%%%%%%%%%%%%%%%%%%%%%%%%%%%%%

\endinput

