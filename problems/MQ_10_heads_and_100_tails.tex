\documentclass[problem]{mcs}

\begin{pcomments}
  \pcomment{MQ_10_heads_and_100_tails}
  \pcomment{contributed by Rich Chan 10/18/09; edited by ARM 10/18/09}
\end{pcomments}

\pkeywords{
  state_machines
  increasing
  decreasing
  derived_variables
  weakly_decreasing
  weakly_increasing
  constant
}

%%%%%%%%%%%%%%%%%%%%%%%%%%%%%%%%%%%%%%%%%%%%%%%%%%%%%%%%%%%%%%%%%%%%%
% Problem starts here
%%%%%%%%%%%%%%%%%%%%%%%%%%%%%%%%%%%%%%%%%%%%%%%%%%%%%%%%%%%%%%%%%%%%%

\begin{problem}
Start with 110 coins on a table, 10 showing heads and 100 showing tails.

There are two ways to change the coins:

\begin{enumerate}
\item[(i)] Remove $10$ \emph{heads} and $10$ \emph{tails}.

\item[(ii)] If there are \emph{more tails than heads} on the table, add
  more heads so the number of heads is doubled.
\end{enumerate}

\bparts

\ppart
Model this situation as a state machine, carefully defining the set of
states, the start state and the possible state transitions.

\textbf{Reminder}: be sure to state the conditions of the state
transitions.

\instatements{\vspace{1.5in}}

\ppart Let $H \eqdef \text{the number of heads}$ and $T \eqdef \text{the
  number of heads}$.  For each of the derived variables below, indicate
which property it satisfies.

The choices for properties are: \textbf{SI}: Strictly Increasing;
\textbf{SD}: Strictly Decreasing; \textbf{WI}: Weakly Increasing but not
constant; \textbf{WD}: Weakly Decreasing but not constant; \textbf{C}:
Constant; \textbf{N}: None of these.

% \renewcommand{\labelenumi}{(\roman{enumi})}

\renewcommand{\theenumi}{\roman{enumi}}
\renewcommand{\labelenumi}{(\theenumi)}

\begin{enumerate}

% \item $H$
% \vspace{0.5in}

\item $T$ \hfill \brule{0.5in}

\item $H+T$ \hfill \brule{0.5in}

\item $T-H$ \hfill \brule{0.5in}

\item $2T-H$ \hfill \brule{0.5in}

\end{enumerate}
% 
% \ppart
% Prove that any long enough sequence of transitions will arrive at a 
% state in which no transition is possible: by providing a strictly decreasing
% derived variable and proving that it has a minimum value.
% 
% 
% \begin{solution}
% TBA
% \end{solution}
% 
\eparts
\end{problem}

%%%%%%%%%%%%%%%%%%%%%%%%%%%%%%%%%%%%%%%%%%%%%%%%%%%%%%%%%%%%%%%%%%%%%
% Problem ends here
%%%%%%%%%%%%%%%%%%%%%%%%%%%%%%%%%%%%%%%%%%%%%%%%%%%%%%%%%%%%%%%%%%%%%

\endinput
