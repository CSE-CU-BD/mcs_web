\documentclass[problem]{mcs}

\begin{pcomments}
  \pcomment{MQ_planar_face_length_short}
  \pcomment{ARM 4/1/11, adapted 11/2011 by drewe from MQ_planar_face_length}
\end{pcomments}

\pkeywords{
 planar
 face
 edge
}

%%%%%%%%%%%%%%%%%%%%%%%%%%%%%%%%%%%%%%%%%%%%%%%%%%%%%%%%%%%%%%%%%%%%%
% Problem starts here
%%%%%%%%%%%%%%%%%%%%%%%%%%%%%%%%%%%%%%%%%%%%%%%%%%%%%%%%%%%%%%%%%%%%%

\begin{problem}
\bparts
Define the length of a planar embedding, $\embed{E}$, of a graph $G$ to be
the sum of the lengths of the faces of $\embed{E}$.
\ppart Give a formula for $\embed{E}$ in terms of $v$, $c$ and $e$ (number of $v$ertices, $c$omponents, and $e$dges in $G$). \hint{the length of face \emph{abca} is 3.}
\examspace[2in]
\begin{solution}
$\embed{E} = 2e.$
\end{solution}
\ppart
Conclude that all embeddings of the same planar graph have the same length.
\examspace
\begin{solution}
Each edge has exactly two occurrences on the faces of an embedding, so
the length of an embedding is always twice the number of edges in the
graph, which is the same for any embedding of the same graph.
\end{solution}

\eparts

\end{problem}

%%%%%%%%%%%%%%%%%%%%%%%%%%%%%%%%%%%%%%%%%%%%%%%%%%%%%%%%%%%%%%%%%%%%%
% Problem ends here
%%%%%%%%%%%%%%%%%%%%%%%%%%%%%%%%%%%%%%%%%%%%%%%%%%%%%%%%%%%%%%%%%%%%%

\endinput
