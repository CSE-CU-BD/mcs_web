\documentclass[problem]{mcs}

\begin{pcomments}
  \pcomment{PS_immortal_probability}
  \pcomment{overlaps of PS_ethernet}
  \pcomment{F12.rec22}
\end{pcomments}

\pkeywords{
  probability
  sample_space
  tie_breaking
}

\begin{problem}

There were $n$ Immortal Warriors born into our world, but in the end
there can be \emph{only one}.  The Immortals' original plan was to
stalk the world for centuries, dueling one another with ancient swords
in dramatic landscapes until only one survivor remained.  However,
after a thought-provoking discussion probability, they opt to give the
following protocol a try:

\begin{enumerate}
\item[(i)] The Immortals forge a coin that comes up heads with probability
$p$.
\item[(ii)] Each Immortal flips the coin once.
\item[(iii)] If \emph{exactly one} Immortal flips heads, then they are
  declared The One.  Otherwise, the protocol is declared a failure, and
  they all go back to hacking each other up with swords.
\end{enumerate}

One of the Immortals (Kurgan from the Russian steppe) argues that as $n$
grows large, the probability that this protocol succeeds must tend to
zero.  Another (McLeod from the Scottish highlands) argues that this need
not be the case, provided $p$ is chosen carefully.

\bparts

\ppart A natural sample space to use to model this problem
is $\set{H,T}^n$ of length-$n$ sequences of H and T's, where the
successive H's and T's in an outcome correspond to the Head or Tail
flipped on each one of the $n$ successive flips.  Explain how a tree
diagram approach leads to assigning a probability to each outcome that
depends only on $p, n$ and the number $h$ of H's in the outcome.

\begin{solution}
The tree would have depth $n$, with each vertex having a branch assigned
probability $p$ to a child labelled H and a branch assigned probability
$1-p$ to a child labelled T.  An outcome with $h$ H's must have $n - h$
T's and would therefore be assigned a probability of
\[  
p^h (1-p)^{n-h } .
\]      
\end{solution}

\ppart
What is the probability that the experiment succeeds as a
function of $p$ and $n$?

\begin{solution}
Let $E$ be the event that the experiment successfully selects The One.
Then $E$ consists of the $n$ outcomes which contain a single head.  Each
of these has probability $p (1-p)^{n-1 }$, so the probability that the
procedure succeeds is
\begin{equation}\label{Enp1p}
\pr{E} = n \paren{p (1-p)^{n-1}} .
\end{equation}
\end{solution}

\ppart How should $p$, the bias of the coin, be chosen in
order to maximize the probability that the experiment succeeds? 

\begin{staffnotes}
You're going to have to compute a derivative!
\end{staffnotes}

\begin{solution}
We compute the derivative of the success probability:
\begin{eqnarray*}
\frac{d}{dp}\ n p (1-p)^{n-1}
	& = & n (1-p)^{n-1} - n p (n-1) (1-p)^{n-2} \\
\end{eqnarray*}
Now we set the right side equal to zero to find the best probability $p$:
\begin{align*}
n (1-p)^{n-1} & =  n p (n-1) (1-p)^{n-2} \\
       (1-p) & =  p (n-1) \\
           p & =  \frac{1}{n} .
\end{align*}
This answer makes some intuitive sense, since we want the coin to come up
heads exactly 1 time in $n$.
\end{solution}

\ppart What is the probability of success if $p$ is chosen
in this way?  What quantity does this approach when $n$, the number of
Immortal Warriors, grows large?

\begin{solution}
Setting $p = 1/n$ in the formula~\eqref{Enp1p} for the probability that
the experiment succeeds gives:
\[
\pr{E} =  \paren{1-\frac{1}{n}}^{n-1}
\]
In the limit, this tends to $1/e$.  McLeod is right.
\end{solution}

\eparts


\end{problem}

\endinput


\iffalse
What does your intuition tell you?
\solution[\vspace{0.5in}]{Your intuition tells you that it is not to be
  trusted and that there \emph{can be only one} way to solve this problem:
  do the math.}
\item[b.
\fi
