\documentclass[problem]{mcs}

\begin{pcomments}
\pcomment{MQ_sets_to_membership_no_intro-afternoon}
\pcomment{concise version of MQ_sets_to_membership by ARM 9/20/11}
\pcomment{variant of CP_proving_basic_set_id}
\pcomment{S13, mq2 ARM 2/21/13}
\end{pcomments}

\pkeywords{
  logic
  set_theory
  identity
  propositional
  chain_of_iff
  difference
}

%%%%%%%%%%%%%%%%%%%%%%%%%%%%%%%%%%%%%%%%%%%%%%%%%%%%%%%%%%%%%%%%%%%%%
% Problem starts here
%%%%%%%%%%%%%%%%%%%%%%%%%%%%%%%%%%%%%%%%%%%%%%%%%%%%%%%%%%%%%%%%%%%%%

\begin{problem}
  You've seen how certain set identities follow from corresponding
  propositional equivalences.  For example, you proved by a chain of
  iff's that
  \[
  (A-B) \union (A \intersect B) = A
  \]
  using the fact that the propositional formula $(P \QAND \bar{Q}) \QOR (P
  \QAND Q)$ is equivalent to $P$.

State a similar propositional equivalence that would justify the key
step in a proof for the following set equality organized as a chain of
iff's:
\[
\setcomp{A \intersect B \intersect C} = \setcomp{A} \union (\setcomp{B} - A) \union \setcomp{C}.
\]

(You are \emph{not} being asked to write out an iff-proof of the
equality or to write out a proof of the propositional equivalence.
Just state the equivalence.)

\begin{solution}
\[
\QNOT(P \QAND Q \QAND R} = \bar{P} \QOR (\QNOT(Q) \QAND \QNOT(A) \QOR \bar{R}.
\]
\end{solution}

\end{problem}

%%%%%%%%%%%%%%%%%%%%%%%%%%%%%%%%%%%%%%%%%%%%%%%%%%%%%%%%%%%%%%%%%%%%%
% Problem ends here
%%%%%%%%%%%%%%%%%%%%%%%%%%%%%%%%%%%%%%%%%%%%%%%%%%%%%%%%%%%%%%%%%%%%%

\endinput
