\documentclass[problem]{mcs}

\begin{pcomments}
  \pcomment{FP_neighborhood_census}
  \pcomment{F15.cfinal, S12.final, S11.ps10, F09.ps12, F07.ps10}
  \pcomment{this simplified version would be better as a class problem---ARM April '10}
\end{pcomments}

\pkeywords{
  probability
  independence
  conditional_probability
  mistake
  bug
}

%%%%%%%%%%%%%%%%%%%%%%%%%%%%%%%%%%%%%%%%%%%%%%%%%%%%%%%%%%%%%%%%%%%%%
% Problem starts here
%%%%%%%%%%%%%%%%%%%%%%%%%%%%%%%%%%%%%%%%%%%%%%%%%%%%%%%%%%%%%%%%%%%%%

\begin{problem}
You are organizing a neighborhood census and instruct your census
takers to knock on doors and note the sex of any child that answers
the knock.  Assume that there are two children in every household,
that a random child is equally likely to be a girl or a boy, and that
the two children in a household are equally likely to be the one that
opens the door.

A sample space for this experiment has outcomes that are triples whose
first element is either \texttt{B} or \texttt{G} for the sex of the
elder child, whose second element is either \texttt{B} or \texttt{G}
for the sex of the younger child, and whose third element is
\texttt{E} or \texttt{Y} indicating whether the \emph{e}lder child or
\emph{y}ounger child opened the door.  For example,
$(\mathtt{B},\mathtt{G},\mathtt{Y})$ is the outcome that the elder
child is a boy, the younger child is a girl, and the girl opened the
door.

\bparts

\ppart Let \emph{O} be the event that a girl opened the door, and let
\emph{T} be the event that the household has two girls.  List the
outcomes in \emph{O} and \emph{T}.

\begin{solution}
$O=\set{GGE,GGY,GBE,BGY}$, $T=\set{GGE,GGY}$
\end{solution}

\examspace[1in]

\ppart What is the probability $\prcond{T}{O}$, that both children are
girls, given that a girl opened the door?
\begin{solution}
1/2
\end{solution}

\examspace[1in]

\ppart What mistake is made in the following argument?  (Merely
stating the correct probability is not \emph{an explanation} of the
mistake.)

\begin{quote}
If a girl opens the door, then we know that there is at least one girl in
the household.  The probability that there is at least one girl is
\[
1 - \prob{\text{both children are boys}} = 1 - (1/2 \times 1/2) = 3/4.
\]
So,
\begin{align*}
\lefteqn{\prcond{T}{\text{there is at least one girl in the household}}}\\
& = \frac{\prob{T \intersect \text{there is at least one girl in the household}}}
{\pr{\text{there is at least one girl in the household}}}\\
& = \frac{\prob{T}}{\pr{\text{there is at least one girl in the household}}}\\
& = (1/4) / (3/4) = 1/3.
\end{align*}
Therefore, given that a girl opened the door, the probability that there
are two girls in the household is \textup{1/3}.
\end{quote}

\begin{solution}
The argument is a correct proof that
\[
\prcond{T}{\text{there is at least one girl in the household}} = 1/3.
\]
The problem is that the event $H$ that the household has at least one girl,
namely,
\[
H \eqdef \set{\mathtt{GGE,GGY,GBE,GBY,BGE,BGY}},
\]
is not equal to the event, \emph{O}, that a girl opens the door.  These
two events differ:
\[
H-O = \set{\mathtt{BGE,GBY}},
\]
and their probabilities are different.  So the fallacy is in the final
conclusion where the value of $\prcond{T}{H}$ is taken to be the same as
the value $\prcond{T}{O}$.  Actually, $\prcond{T}{O} = 1/2$.
\end{solution}

\eparts
\end{problem}

%%%%%%%%%%%%%%%%%%%%%%%%%%%%%%%%%%%%%%%%%%%%%%%%%%%%%%%%%%%%%%%%%%%%%
% Problem ends here
%%%%%%%%%%%%%%%%%%%%%%%%%%%%%%%%%%%%%%%%%%%%%%%%%%%%%%%%%%%%%%%%%%%%%

\endinput
