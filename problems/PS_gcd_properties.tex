\documentclass[problem]{mcs}

\begin{pcomments}
  \pcomment{PS_gcd_properties}
  \pcomment{from: S06.ps5.03}
  \pcomment{substantial revision by ARM 3/4/12}
\end{pcomments}

\pkeywords{
  number_theory
  prime_factor
  fundamental_theorem
  arithmetic
  relatively_prime
}

%%%%%%%%%%%%%%%%%%%%%%%%%%%%%%%%%%%%%%%%%%%%%%%%%%%%%%%%%%%%%%%%%%%%%
% Problem starts here
%%%%%%%%%%%%%%%%%%%%%%%%%%%%%%%%%%%%%%%%%%%%%%%%%%%%%%%%%%%%%%%%%%%%%

\begin{problem}
  
  Suppose $m$ and $n$ are relatively prime.  Use the Fundamental
  Theorem of Arithmetic (Unique Prime Factorization) to give simple
  proofs of:
  
  \bparts
  
  \ppart
\begin{align*}
\lefteqn{  k \text{ is relatively prime to } mn\quad \QIFF}\\
  & k \text{ is relatively  prime to } m\ \QAND\ k \text{ is relatively prime to } n.
\end{align*}
  
  \begin{solution}
    By unique factorization, a prime divides $mn$ iff it divides $m$
    or $n$.
    \begin{align*}
      \lefteqn{k \text{ is relatively prime to } mn\ \ \QIFF}\\
      & \text{ no prime that divides $k$ also divides } mn\ \ \QIFF\\
      & \text{ no prime that divides $k$ in divides $m$ or } n\ \ \QIFF\\
      & k \text{ is relatively prime to } m\ \ \QAND\ \ k \text{ is relatively prime to } n.
    \end{align*}
  \end{solution}

  \ppart
  \[
  [m \divides a\ \QAND\ n \divides a]\ \QIMP\  mn \divides a.
  \]

\hint By unique factorization, $j \divides k$ iff the sequence of
primes in the factorization of $j$ is a subsequence of the sequence of
primes in the factorization of $k$.

\begin{staffnotes}
Ask for counterexample when $m,n$ not relatively prime, for example,
$m=2\cdot 3, n=3\cdot 5, a = 2 \cdot 3 \cdot 5$.
\end{staffnotes}

  \begin{solution}
    If $m \divides a$, then the sequence of primes in the
    factorization of $m$ appears in the factorization of $a$ (by the
    hint).  Likewise for the sequence of primes in the factorization
    of $n$.  But since $m$ and $n$ are relatively prime, the two
    sequences do not overlap, and therefore the ``merge'' of the two
    sequences is a subsequence in the factorization of $a$.  But by
    unique factorization, this merge is the sequence of primes in the
    factorization of $mn$.  Using the hint again, this implies that
    $mn \divides a$.
  \end{solution}
  
  \eparts
  
\end{problem}

%%%%%%%%%%%%%%%%%%%%%%%%%%%%%%%%%%%%%%%%%%%%%%%%%%%%%%%%%%%%%%%%%%%%%
% Problem ends here
%%%%%%%%%%%%%%%%%%%%%%%%%%%%%%%%%%%%%%%%%%%%%%%%%%%%%%%%%%%%%%%%%%%%%

\endinput

\iffalse
    In what follows, let the unique prime factorizations of $m$
    and $n$ be as follows:
    \begin{align*}
      m & = p_1^{a_1} p_2^{a_2} \cdots p_s^{a_s}\\
      n & = q_1^{b_1} q_2^{b_2} \cdots q_t^{b_t}
    \end{align*}
    Since $m$ and $n$ are relatively prime, we know that $p_i \neq q_j$,
    for all $i,j$.  

    Furthermore, the unique prime factorization of $mn$
    is the ``concatenation'' of the two disjoint prime factorizations of
    $m$ and $n$:
    \[
    mn = (p_1^{a_1} p_2^{a_2} \cdots p_s^{a_s})(q_1^{b_1} q_2^{b_2} \cdots q_t^{b_t}).
    \]

    If $mn \divides a$, then the entire sequence $(p_1^{a_1} p_2^{a_2}
    \cdots p_s^{a_s})(q_1^{b_1} q_2^{b_2} \cdots q_t^{b_t})$ appears in
    the prime factorization of $a$.  But this implies that the prime
    factorizations of $m$ and $n$ both appear in the prime factorization
    of $a$, so both $m$ and $n$ divide $a$.

    Conversely, 
\fi
