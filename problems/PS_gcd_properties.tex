\documentclass[problem]{mcs}

\begin{pcomments}
  \pcomment{PS_gcd_properties}
  \pcomment{from: S06.ps5.03}
\end{pcomments}

\pkeywords{
  gcd
  linear_combinations
  divides
  number_theory
  common_divisor
}

%%%%%%%%%%%%%%%%%%%%%%%%%%%%%%%%%%%%%%%%%%%%%%%%%%%%%%%%%%%%%%%%%%%%%
% Problem starts here
%%%%%%%%%%%%%%%%%%%%%%%%%%%%%%%%%%%%%%%%%%%%%%%%%%%%%%%%%%%%%%%%%%%%%

\begin{problem}
  
  Suppose $m$ and $n$ are relatively prime.  Use the Fundamental Theorem of
  Arithmetic (that the factorization into primes of an integer greater than
  1 is unique) to give simple proofs of:
  
  \bparts
  
  \ppart
  \[
  mn \divides a \qiff m \divides a \text{ and } n \divides a.
  \]
  
  \begin{solution}
    In what follows, let the unique prime factorizations of $m$
    and $n$ be as follows:
    \begin{eqnarray*}
      m & = & p_1^{a_1} p_2^{a_2} \dotsb p_s^{a_s}\\
      n & = & q_1^{b_1} q_2^{b_2} \dotsb q_t^{b_t}
    \end{eqnarray*}
    Since $m$ and $n$ are relatively prime, we know that $p_i \neq q_j$,
    for any $i,j$.  Furthermore, the unique prime factorization of $mn$
    is the ``concatenation'' of the two disjoint prime factorizations of
    $m$ and $n$:
    $$mn = (p_1^{a_1} p_2^{a_2} \dotsb p_s^{a_s})(q_1^{b_1} q_2^{b_2}
    \dotsb q_t^{b_t}).$$

    If $mn \divides a$, then the entire sequence $(p_1^{a_1} p_2^{a_2}
    \dotsb p_s^{a_s})(q_1^{b_1} q_2^{b_2} \dotsb q_t^{b_t})$ appears in
    the prime factorization of $a$.  But this implies that the prime
    factorization of $m$ and $n$ both appear in the prime factorization
    of $a$, so both $m$ and $n$ divide $a$.

    Conversely, if $m \divides a$ and $n \divides a$, then both
    $p_1^{a_1} p_2^{a_2} \dotsb p_s^{a_s}$ and $q_1^{b_1} q_2^{b_2}
    \dotsb q_t^{b_t}$ appear in the prime factorization of $a$.  Since
    these two sequences are disjoint (i.e. do not share any common
    terms), their ``concatenation'' $(p_1^{a_1} p_2^{a_2} \dotsb
    p_s^{a_s})(q_1^{b_1} q_2^{b_2} \dotsb q_t^{b_t})$ also appears in
    the prime factorization, and this implies that $mn$ divides $a$.
  \end{solution}
  
  \ppart
  \[
  x \text{ is relatively prime to } mn \qiff [x\text{ is relatively
  prime to } m\text{ and } x \text{ is relatively prime to } n]
  \]
  
  \begin{solution}
    Let the unique prime factorization of $x$ be
    $r_1^{c_1}r_2^{c_2}\dotsb r_u^{c_u}$. Suppose $x$ is relatively
    prime to $mn$.  This implies that $r_i \neq p_j$ for any $i,j$, i.e.
    $x$ does not share any prime factors with $m$, so $x$ and $m$ are
    relatively prime.  Furthermore, it also implies that $r_i \neq q_j$,
    for any $i,j$, i.e. $x$ does not share any prime factors with $n$,
    so $x$ and $n$ are relatively prime.

    Conversely, suppose $x$ is relatively prime to $m$ and $x$ is
    relatively prime to $n$.  Again, $r_i \neq p_j$ for any $i,j$, and
    $r_i \neq q_j$ for any $i,j$.  Since the prime factorization of $mn
    =(p_1^{a_1} p_2^{a_2} \dotsb p_s^{a_s})(q_1^{b_1} q_2^{b_2} \dotsb
    q_t^{b_t})$, none of the prime factors of $x$ appear in the prime
    factorization of $mn$, so $x$ is relatively prime to $mn$.
  \end{solution}

  \eparts
  
\end{problem}

%%%%%%%%%%%%%%%%%%%%%%%%%%%%%%%%%%%%%%%%%%%%%%%%%%%%%%%%%%%%%%%%%%%%%
% Problem ends here
%%%%%%%%%%%%%%%%%%%%%%%%%%%%%%%%%%%%%%%%%%%%%%%%%%%%%%%%%%%%%%%%%%%%%

\endinput
