\documentclass[problem]{mcs}

\begin{pcomments}
  \pcomment{PS_order_divides_phi}
  \pcomment{ARM 3/27/12, hint added 3/4/14}
\end{pcomments}

\pkeywords{
  Euler_theorem
  Eulers_theorem
  number_theory
  modular_arithmetic
  prime_power
  phi_function
}

%%%%%%%%%%%%%%%%%%%%%%%%%%%%%%%%%%%%%%%%%%%%%%%%%%%%%%%%%%%%%%%%%%%%%
% Problem starts here
%%%%%%%%%%%%%%%%%%%%%%%%%%%%%%%%%%%%%%%%%%%%%%%%%%%%%%%%%%%%%%%%%%%%%

\begin{problem}

\begin{definition*}
Define the \emph{order} of $k$ over
  $\Zmod{n}$ to be
\[
\ordmod{k}{n} \eqdef \min \set{m > 0 \suchthat k^m = 1 \inzmod{n}}.
\]
If no positive power of $k$ equals 1 in $\Zmod{n}$, then
$\ordmod{k}{n} \eqdef \infty$.
\end{definition*}

\bparts

\ppart Show that $k \in \relpr{n}$ iff $k$ has finite order in
$\Zmod{n}$.

\begin{solution}
If $k$ has finite order in $\Zmod{n}$, then $k^{\ordmod{k}{n}-1}$ is
an inverse of $k$, so $k \in \relpr{n}$ by
Theorem~\bref{thm:mod_inverses}.

Conversely, since $\Zmod{n}$ has $n$ elements, some number must occur
twice in the list
\[
k^0, k^1,\ k^2,\ \dots,\ k^{n} \inzmod{n}.
\] 
That is,
\begin{equation}\label{kiki+m}
k^{i} = k^{i+m} \inzmod{n}
\end{equation}
for some $i, m \in \Zintv{1}{n}$.  But if $k \in \relpr{n}$, then $k$ is
cancellable over $\Zmod{n}$, so we can cancel the first $i$ of the
$k$'s on both sides of~\eqref{kiki+m} to get
\[
1 = k^m \inzmod{n}.
\]
It follows that $k$ has order $< n \inzmod{n}$.
\end{solution}

\ppart Prove that for every $k \in \relpr{n}$, the order of $k$ over
$\Zmod{n}$ divides $\phi(n)$.

\hint Let $m= \ordmod{k}{n}$.  Consider the quotient and remainder of
$\phi(n)$ divided by $m$.

\begin{solution}

\begin{proof}
Let $m= \ordmod{k}{n}$.  Now we have
\begin{align}
1 & = k^{\phi(n)} & \text{(Euler)}\notag\\
  & = k^{m\qcnt{\phi(n)}{m} + \rem{\phi(n)}{m}} & \text{(Division Theorem)}\notag\\
  & = \paren{k^m}^{\qcnt{\phi(n)}{m}
} \cdot k^{\rem{\phi(n)}{m}}\notag\\
  & = 1^{\qcnt{\phi(n)}{m}} \cdot k^{\rem{\phi(n)}{m}} & \text{(Def of $m$)}\notag\\
  & = k^{\rem{\phi(n)}{m}}.\label{kremphi}
\end{align}
But $\rem{\phi(n)}{m} < m$ and $m$ is the smallest positive power of
  $k$ equal to 1 in $\Zmod{n}$, so~\eqref{kremphi} implies that
  $\rem{\phi(n)}{m}$ must equal 0, which means that $m \divides
    \phi(n)$.
\end{proof}

\end{solution}

\eparts
\end{problem}


%%%%%%%%%%%%%%%%%%%%%%%%%%%%%%%%%%%%%%%%%%%%%%%%%%%%%%%%%%%%%%%%%%%%%
% Problem ends here
%%%%%%%%%%%%%%%%%%%%%%%%%%%%%%%%%%%%%%%%%%%%%%%%%%%%%%%%%%%%%%%%%%%%%

\endinput

