\documentclass[problem]{mcs}

\begin{pcomments}
  \pcomment{PS_planar_coloring}
  \pcomment{Source(s): S02 PS4-2}
  \pcomment{edited ARM 10/22/11}   
\end{pcomments}

\pkeywords{
  planar_graphs 
  coloring
  6_coloring
  subgraph
}

%%%%%%%%%%%%%%%%%%%%%%%%%%%%%%%%%%%%%%%%%%%%%%%%%%%%%%%%%%%%%%%%%%%%%
% Problem starts here
%%%%%%%%%%%%%%%%%%%%%%%%%%%%%%%%%%%%%%%%%%%%%%%%%%%%%%%%%%%%%%%%%%%%%

\begin{problem}

\bparts

\iffalse  Now in book Lemma~\bref{lem:pg5}
\ppart 

Show that any planar graph has a node of degree at most $5$.
\hint Any planar graph with $n$
vertices and $m$ edges satisfies $m \leq 3n-6$.

\begin{solution}Suppose that every vertex has degree at least $6$.  Then
\begin{align*}
2m &= \sum_{v \in V}\textrm{deg}(v)\geq\sum_{v \in V}6=6n\\
m  &\geq 3n
\end{align*}
contradicting our assertion that $m \leq 3n-6$.
\end{solution}
\fi

\ppart Using Lemma~\bref{lem:pg5} that every planar graph has a node
of degree at most five, to prove that any planar graph can be colored
with six colors.

\begin{solution}
The proof is by induction on the number of vertices $n$.
\[
P(n) \eqdef \text{A planar graph of $n$ vertices can be colored with
  at most $6$ colors}.
\]

\inductioncase{Base case}: $P(1)$ is true because a single vertex
 can be colored with $1\leq 6$ colors.

\inductioncase{Inductive step}: Assume $P(n)$ is true in order to show
that $P(n+1)$ is true.

Let $G$ be a planar graph with $n+1$ vertices and remove a vertex $v$
of degree $5$ (or less).  The remaining $n$-vertex graph will be
planar (see By Lemma~\bref{planar-subgraph}), and so can be colored
with $6$ colors by the inductive hypothesis.  Re-attach $v$.  Since
$v$ is adjacent to at most $5$ vertices which are collectively colored
with at most 5 colors.  Hence there will a sixth color to assign to
$v$ that differs from the colors all five neighbors.  This proves $P(n+1)$, completing
the induction step.

We conclude by induction that every planar graph of finite size $n$
can be 6-colored.
\end{solution}

\eparts
\end{problem}


%%%%%%%%%%%%%%%%%%%%%%%%%%%%%%%%%%%%%%%%%%%%%%%%%%%%%%%%%%%%%%%%%%%%%
% Problem ends here
%%%%%%%%%%%%%%%%%%%%%%%%%%%%%%%%%%%%%%%%%%%%%%%%%%%%%%%%%%%%%%%%%%%%%

\endinput
