\documentclass[problem]{mcs}

\begin{pcomments}
  \pcomment{PS_planar_coloring}
  \pcomment{Source(s): S02 PS4-2}
\end{pcomments}

\pkeywords{
  planar_graphs 
  coloring
}

%%%%%%%%%%%%%%%%%%%%%%%%%%%%%%%%%%%%%%%%%%%%%%%%%%%%%%%%%%%%%%%%%%%%%
% Problem starts here
%%%%%%%%%%%%%%%%%%%%%%%%%%%%%%%%%%%%%%%%%%%%%%%%%%%%%%%%%%%%%%%%%%%%%

\begin{problem}

\bparts
\ppart 

Show that any planar graph has a node of degree at most $5$.
\hint Any planar graph with $n$
vertices and $m$ edges satisfies $m \leq 3n-6$.

\begin{solution}Suppose that every vertex has degree at least $6$.  Then
\begin{align*}
2m &= \sum_{v \in V}\textrm{deg}(v)\geq\sum_{v \in V}6=6n\\
m  &\geq 3n
\end{align*}
contradicting our assertion that $m \leq 3n-6$.
\end{solution}

\ppart Using this fact, prove that any planar graph can be colored
with six colors.

\begin{solution}
The proof is by induction on the number of vertices $n$.

$P(n) = $ ``A planar graph of $n$ vertices can be colored with at most
$6$ colors''.

Base case: $P(1)$ is true because a single vertex
 can be colored with $1$ color.

Inductive step: 
Assume $P(n)$ is true in order to show that $P(n+1)$ is true.

Let $G$ be a planar graph with $n+1$ vertices and remove a vertex $v$
of degree $5$ (or less).  We know the remaining $n$-vertex graph will
be planar, and so can be colored with $6$ colors by the inductive
hypothesis.  Re-attach $v$.  $v$ is adjacent to at most $5$ vertices,
occupying at most $5$ out of the six colors.  We can use the remaining
color to color $v$.

We have shown that $P(n) \rightarrow P(n+1)$, so the proof is complete.
\end{solution}

\eparts
\end{problem}


%%%%%%%%%%%%%%%%%%%%%%%%%%%%%%%%%%%%%%%%%%%%%%%%%%%%%%%%%%%%%%%%%%%%%
% Problem ends here
%%%%%%%%%%%%%%%%%%%%%%%%%%%%%%%%%%%%%%%%%%%%%%%%%%%%%%%%%%%%%%%%%%%%%

\endinput
