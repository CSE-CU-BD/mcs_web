\documentclass[problem]{mcs}

\begin{pcomments}
  \pcomment{PS_rewrite_assertions_prime_goldbach_bertrand_fermat}
  \pcomment{from: S03.ps4, f03.ps1, f09.ps2}
  \pcomment{subsumed PS_translate_to_predicate_logic}
\end{pcomments}

\pkeywords{
  predicate
  quantifier
  Goldbach
  Bertrand
  Fermat
}

%%%%%%%%%%%%%%%%%%%%%%%%%%%%%%%%%%%%%%%%%%%%%%%%%%%%%%%%%%%%%%%%%%%%%
% Problem starts here
%%%%%%%%%%%%%%%%%%%%%%%%%%%%%%%%%%%%%%%%%%%%%%%%%%%%%%%%%%%%%%%%%%%%%

\begin{problem}
Rewrite the following assertions using only variables, logic symbols,
and arithmetic expressions involving constants, addition,
multiplication, exponentiation, and comparison.  In all parts, the
domain of discourse is $\naturals$.  In the first part, you are asked
to write a definition of a prime number.  Then, in the second and
subsequent parts, you may make use of $\text{Prime}(n)$, a predicate
that is true if and only if $n$ is a prime number.

\begin{problemparts}

\problempart $p$ is a prime number.

\begin{solution}
\[
(p > 1)
\QAND
\QNOT(\exists m.\, \exists n.\, (m > 1 \QAND n > 1 \QAND mn = p))
\]
\end{solution}

\problempart There is no largest prime number.

\begin{solution}
%The domain of discourse is $\mathbb{Z}$.

\[
\QNOT(\exists p.\, (\text{Prime}(p) \QAND (\forall q.\, (\text{Prime}(q) \QIMPLIES p \geq q))))
\]
\end{solution}

\problempart (\idx{Goldbach Conjecture}) Every even nonnegative
integer $n \geq 4$ can be expressed as the sum of two primes.

\begin{solution}
\[
\forall n \left( (n \geq 4 \QAND \exists k\ n = 2k) \implies \exists p
  \exists q (\text{Prime}(p) \QAND \text{Prime}(q) \QAND (n = p + q))
\right)
\]
\end{solution}

\problempart (\idx{Bertrand's Postulate}) If $n > 1$, then there is always
at least one prime $p$ such that $n < p < 2n$.

\begin{solution}
The domain of discourse is $\mathbb{Z}$.

\[
\forall n.\,
\( (n > 1) \implies (\exists p ( \text{Prime}(p)  \QAND (n < p) \QAND (p < 2n))) 
)
\]
\end{solution}

\problempart (\idx{Fermat's Last Theorem}) There are no solutions to the
equation:

\begin{equation*}
    x^n + y^n = z^n
\end{equation*}

where $n > 2$ and $x$, $y$, and $z$ are positive.

\begin{solution}
The domain of discourse is $\mathbb{Z}$.

\[
\forall x, y, z, n.\,
    (
    (x > 0 \QAND y > 0 \QAND z > 0 \QAND n > 2)
    \QIMPLIES
    \QNOT(x^n + y^n = z^n)
    )
\]
\end{solution}

\end{problemparts}

\end{problem}

%%%%%%%%%%%%%%%%%%%%%%%%%%%%%%%%%%%%%%%%%%%%%%%%%%%%%%%%%%%%%%%%%%%%%
% Problem ends here
%%%%%%%%%%%%%%%%%%%%%%%%%%%%%%%%%%%%%%%%%%%%%%%%%%%%%%%%%%%%%%%%%%%%%
