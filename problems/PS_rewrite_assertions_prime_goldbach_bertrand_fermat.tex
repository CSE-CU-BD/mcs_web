\documentclass[problem]{mcs}

\begin{pcomments}
  \pcomment{from: f03.ps1}
\end{pcomments}

\pkeywords{
  rewrite_assertions
  predicates
  quantifiers
}

%%%%%%%%%%%%%%%%%%%%%%%%%%%%%%%%%%%%%%%%%%%%%%%%%%%%%%%%%%%%%%%%%%%%%
% Problem starts here
%%%%%%%%%%%%%%%%%%%%%%%%%%%%%%%%%%%%%%%%%%%%%%%%%%%%%%%%%%%%%%%%%%%%%

\begin{problem}

Rewrite the following assertions using only variables, logic symbols,
and arithmetic expressions involving constants, addition,
multiplication, exponentiation, and comparison.  In all parts, the
domain of discourse is $\mathbb{N}$.  In the first part, you are asked
to write a definition of a prime number.  Then, in the second and
subsequent parts, you may make use of $\text{Prime}(n)$, a predicate
that is true if and only if $n$ is a prime number.

\begin{problemparts}

\problempart $p$ is a prime number.

\begin{solution}
\[
(p > 1)
\wedge
\neg \left( \exists m \exists n (m > 1 \wedge n > 1 \wedge mn = p) \right)
\]
\end{solution}

\problempart There is no largest prime number.

\begin{solution}
The domain of discourse is $\mathbb{Z}$.

\[
\neg \left(\exists p (Prime(p) \wedge (\forall q (Prime(q) \implies p \geq q)))
\right)
\]
\end{solution}

\problempart (Goldbach Conjecture) Every even natural number $n \geq
4$ can be expressed as the sum of two primes.

\begin{solution
\[
\forall n 
\left(
(n \geq 4 \wedge \exists k\ n = 2k) \implies \exists p \exists q (Prime(p) \wedge Prime(q) \wedge (n = p + q))
\right)
\]
\end{solution}

\problempart (Bertrand's Postulate) If $n > 1$, then there is always
at least one prime $p$ such that $n < p < 2n$.

\begin{solution}
The domain of discourse is $\mathbb{Z}$.

\[
\forall n
\left( (n > 1) \implies (\exists p ( Prime(p)  \wedge (n < p) \wedge (p < 2n))) 
\right)
\]
\end{solution}

\problempart (Fermat's Last Theorem) There are no solutions to the
equation:

\begin{eqnarray*}
x^n + y^n & = & z^n
\end{eqnarray*}

where $n > 2$ and $x$, $y$, and $z$ are positive.

\begin{solution}
The domain of discourse is $\mathbb{Z}$.

\[
\forall x \forall y \forall z \forall n
    \left(
    (x > 0 \wedge y > 0 \wedge z > 0 \wedge n > 2)
    \rightarrow
    \neg (x^n + y^n = z^n)
    \right)
\]
\end{solution}

\end{problemparts}

\end{problem}

%%%%%%%%%%%%%%%%%%%%%%%%%%%%%%%%%%%%%%%%%%%%%%%%%%%%%%%%%%%%%%%%%%%%%
% Problem ends here
%%%%%%%%%%%%%%%%%%%%%%%%%%%%%%%%%%%%%%%%%%%%%%%%%%%%%%%%%%%%%%%%%%%%%