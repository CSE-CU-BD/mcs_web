\documentclass[problem]{mcs}

\begin{pcomments}
  \pcomment{PS_counting_colored_dice}
  \pcomment{from: S05PS8 -> S06 -> S07PS9}
\end{pcomments}

\pkeywords{
  counting
  bijection  
  product
  generalized_product
}

%%%%%%%%%%%%%%%%%%%%%%%%%%%%%%%%%%%%%%%%%%%%%%%%%%%%%%%%%%%%%%%%%%%%%
% Problem starts here
%%%%%%%%%%%%%%%%%%%%%%%%%%%%%%%%%%%%%%%%%%%%%%%%%%%%%%%%%%%%%%%%%%%%%

\begin{problem}
Suppose you have seven dice---each a different color of the rainbow;
otherwise the dice are standard, with faces numbered 1 to 6.  A
\emph{roll} is a sequence specifying a value for each die in rainbow
(ROYGBIV) order.  For example, one roll is $(3,1,6,1,4,5,2)$
indicating that the red die showed a 3, the orange die showed 1, the
yellow 6,\dots.

For the problems below, describe a bijection between the specified set
of rolls and another set that is easily counted using the Product,
Generalized Product, and similar rules.  Then write a simple
arithmetic formula, possibly involving factorials and
binomial coefficients, for the size of the set of rolls.  You do
not need to prove that the correspondence between sets you describe is
a bijection, and you do not need to simplify the expression you come
up with.

For example, let $A$ be the set of rolls where 4 dice come up showing
the same number, and the other 3 dice also come up the same, but with
a different number.  Let $R$ be the set of seven rainbow colors and
$S \eqdef [1,6]$ be the set of dice values.

Define $B \eqdef P_{S,2} \cross R_3$, where $P_{S,2}$ is the set of 
2-permutations of $S$ and $R_3$ is the set of size-3 subsets of $R$.  
Then define a bijection from $A$ to $B$ by
mapping a roll in $A$ to the sequence in $B$ whose first element is
a pair consisting of the number that came up 
three times followed by the number that came up four times, and whose 
second element is the set of colors of the three matching dice.

For example, the roll
\[
(4,4,2,2,4,2,4) \in A
\]
maps to
\[
((2,4),\set{\text{yellow,green,indigo}}) \in B.
\]

Now by the Bijection rule $\card{A} = \card{B}$, and by the
Generalized Product and Subset rules,
\[
\card{B} = 6 \cdot 5 \cdot \binom{7}{3}.
\]

\bparts

\ppart\label{66} For how many rolls do \emph{exactly} two dice have
the value 6 and the remaining five dice all have different values?
Remember to describe a bijection and write a simple arithmetic
formula.

Example: $(6, 2, 6, 1, 3, 4, 5)$ is a roll of this type, but $(1, 1, 2, 6,
3, 4, 5)$ and $(6, 6, 1, 2, 4, 3, 4)$ are not.

\begin{solution}
As in the example, map a roll into an element of $B \eqdef R_2
\cross P_5$ where $P_5$ is the set of permutations of $\set{1,\dots,5}$.  A roll
maps to the pair whose first element is the set of colors of the two dice
with value 6, and whose second element is the sequence of values of the
remaining dice (in rainbow order).  So $(6, 2, 6, 1, 3, 4, 5)$ above maps
to $(\set{\text{red,yellow}}, (2,1,3,4,5))$.  By the Product rule,
\[
\card{B} = \binom{7}{2}\cdot 5!.
\]
\end{solution}


\ppart For how many rolls do two dice have the same value and the
remaining five dice all have different values?  Remember to describe a
bijection and write a simple arithmetic formula.

Example: $(4, 2, 4, 1, 3, 6, 5)$ is a roll of this type, but $(1, 1, 2, 6,
1, 4, 5)$ and $(6, 6, 1, 2, 4, 3, 4)$ are not.  

\begin{solution}
Map a roll into a triple whose first element is in $S$,
indicating the value of the pair of matching dice, whose second element is the
set of colors of the two matching dice, and whose third element is the
sequence of the remaining five dice values (in rainbow order).

So $(4, 2, 4, 1, 3, 6, 5)$ above maps to $(4, \set{\text{red,yellow}},
(2,1,3,6,5))$.  Notice that the number of choices for the third element of
a triple is the number of permutations of the remaining five values,
namely $5!$.  This mapping is a bijection, so the number of such rolls
equals the number of such triples.
By the Generalized Product rule, the number of such triples is
\[
6 \cdot \binom{7}{2} \cdot 5!.
\]

Alternatively, we can define a map from rolls in this part to the
rolls in part~\eqref{66}, by replacing the value of the duplicated values
with 6's and replacing any 6 in the remaining values by the value of the
duplicated pair.  So the roll $(4, 2, 4, 1, 3, 6, 5)$ would map to the
roll $(6, 2, 6, 1, 3, 4, 5)$.  Now a type~\ref{66} roll $r$ is mapped to
by exactly the rolls obtainable from $r$ by exchanging occurrences of 6's
and $i$'s, for $i = 1,\dots,6$.  So this map is 6-to-1, and by the
Division rule, the number of rolls here is 6 times the number of rolls in
part~\eqref{66}.

\iffalse

An useful variation of this idea is to map a roll into a triple in the set
$B \eqdef S cross R_2 \cross P_5$.  Again, the first element of a triple
in $B$ is the value of the matching dice, the second element is the set of
colors of the two matching dice, and the third element is the sequence of
\emph{rankings}\footnote{Given a set of $n$ numbers, the \emph{rank} of a
number in the set is its position when the numbers are listed in
increasing order.  So the rank of the smallest number is 1, the rank of
the second smallest is 2, \dots, and the rank of the largest is $n$.} of
the remaining five dice values (in rainbow order).

So now $(4, 2, 4, 1, 3, 6, 5)$ above maps to $(4, \set{\text{red,yellow}},
(2,1,3,5,4))$.  By the Product rule, $\card{B} = \binom{7}{2}\cdot 5!$.

The use of ranking let us define $B$ without having the possible values of
third elements in triples depend on the first elements.  This let us give
a simpler definition of the set $B$ of triples corresponding to rolls.
With the simpler definition of $B$, we could calculate its size using the
Product rule instead of the Generalized Product rule.
\fi
\end{solution}

\ppart For how many rolls do two dice have one value, two different
dice have a second value, and the remaining three dice a third value?
Remember to describe a bijection and write a simple arithmetic
formula.

Example: $(6, 1, 2, 1, 2, 6, 6)$ is a roll of this type, but $(4, 4, 4, 4,
1,3,5)$ and $(5, 5, 5, 6, 6,1,2)$ are not.

\begin{solution}
Map a roll of this kind into a 4-tuple whose first element is
the set of two numbers of the two pairs of matching dice, whose second
element is the set of two colors of the pair of matching dice with the
smaller number, whose third element is the set of two colors of the larger
of the matching pairs, and whose fourth element is the value of the
remaining three dice.  For example, the roll $(6, 1, 2, 1, 2, 6, 6)$ maps
to the triple
\[
(\set{1,2},\set{\text{orange,green}},\set{\text{yellow,blue}}, 6).
\]

There are $\binom{6}{2}$ possible first elements of a triple,
$\binom{7}{2}$ second elements, $\binom{5}{2}$ third elements since the
second set of two colors must be different from the first two, and 4 ways
to choose the value of the three dice since their value must differ from
the values of the two pairs.  So by the Generalized Product rule, there
are
\[
\binom{6}{2} \cdot \binom{7}{2} \cdot \binom{5}{2} \cdot 4
\]
possible rolls of this kind.
\end{solution}

\eparts

\end{problem}

%%%%%%%%%%%%%%%%%%%%%%%%%%%%%%%%%%%%%%%%%%%%%%%%%%%%%%%%%%%%%%%%%%%%%
% Problem ends here
%%%%%%%%%%%%%%%%%%%%%%%%%%%%%%%%%%%%%%%%%%%%%%%%%%%%%%%%%%%%%%%%%%%%%

\endinput
