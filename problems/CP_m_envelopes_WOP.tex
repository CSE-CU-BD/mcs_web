\documentclass[problem]{mcs}

\begin{pcomments}
  \pcomment{CP_m_envelopes_WOP}
  \pcomment{CH, S14. Adapted from a question on stackexchange.}
  \pcomment{revised ARM 1/30/15}
\end{pcomments}

\pkeywords{
  WOP
  envelopes
  dollars
  binary representation of integers
}

%%%%%%%%%%%%%%%%%%%%%%%%%%%%%%%%%%%%%%%%%%%%%%%%%%%%%%%%%%%%%%%%%%%%%
% Problem starts here
%%%%%%%%%%%%%%%%%%%%%%%%%%%%%%%%%%%%%%%%%%%%%%%%%%%%%%%%%%%%%%%%%%%%%

\begin{problem}
You are given $m$ envelopes, respectively containing $1, 2, 4, \dots,
2^{m-1}$ dollars.  Define
\begin{quote}
\textbf{Property $m$}: For any nonnegative integer less than $2^m$,
there is a selection of envelopes whose contents add up to
\emph{exactly} that number of dollars.
\end{quote}
Use the Well Ordering Principle (WOP) to prove that Property $m$ holds
for all positive integers $m$.

\begin{staffnotes}
An easy way to see why this holds is to think about $m$ digit binary
numbers.  The binary representation of the number of dollars contained
in a selection of envelopes has a 1 for its $i$th digit iff the $i$th
envelope is in the selection.  So there is a selection of envelopes
whose contents add up to any given $m$ digit binary number.

This is worth pointing out at some point, maybe before, or maybe
after, students work out the WOP proof.
\end{staffnotes}


\begin{solution}
Let $C$ be the set of positive integers $m$ such that Property $m$ is
\emph{not} true, and assume for the sake of contradiction that $C$ is
non-empty.  Then by the Well Ordering Principle (WOP), $C$ has a
smallest element $m_0$.

The first thing to notice is that Property 1 holds: in this case the
nonnegative integers less than $2^1$ are zero and one.  Also, there is
just one envelope and it contains $2^{1-1}$, that is, $1$, dollar.  So
we can get the desired number of zero dollars or one dollar by
selecting or not selecting the envelope.

So $m_0$ must be greater than 1, and $m_0-1$ must be a positive
integer.  The definition of $m_0$ now implies that Property $m_0-1$
must hold.

We will now prove that Property $m_0$ holds, contradicting the
definition of $m_0$.  To do this, we show how to find a selection of
envelopes whose contents add up to any nonnegative integer $n$ less
than $2^{m_0}$.

There are two cases:

\begin{enumerate}

\item $n < 2^{m_0-1}$.  Now since Property $m_0-1$ holds, we can find a
  subset of the first $m_0-1$ envelopes that add up to $n$.

\item $2^{m_0 - 1} \leq n < 2^{m_0}$.  So $n - 2^{m_0 - 1}$ is a
  nonnegative integer less than $2^{m_0-1}$.  Therefore by Case 1,
  there is a selection of the first $m_0-1$ envelopes that adds up to
  $n - 2^{m_0 - 1}$ dollars.  Now add the $m_0$th envelope to the
  selection to obtain a selection that adds to $n$ dollars.
\end{enumerate}
Therefore, for any $n < 2^{m_0}$ we can find a selection of envelopes
whose contents add up to $n$ dollars.  This means that Property $m_0$
holds, contradicting the choice of $m_0$.

This contradition implies that $C$ must be empty, proving that
Property $m$ holds for all positive integers $m$.
\end{solution}

\end{problem}

%%%%%%%%%%%%%%%%%%%%%%%%%%%%%%%%%%%%%%%%%%%%%%%%%%%%%%%%%%%%%%%%%%%%%
% Problem ends here
%%%%%%%%%%%%%%%%%%%%%%%%%%%%%%%%%%%%%%%%%%%%%%%%%%%%%%%%%%%%%%%%%%%%%
