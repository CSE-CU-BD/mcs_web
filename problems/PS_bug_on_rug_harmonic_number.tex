\documentclass[problem]{mcs}

\begin{pcomments}
  \pcomment{from: S08.ps8}
\end{pcomments}

\pkeywords{
  harmonic_number
  bug_on_rug
}

%%%%%%%%%%%%%%%%%%%%%%%%%%%%%%%%%%%%%%%%%%%%%%%%%%%%%%%%%%%%%%%%%%%%%
% Problem starts here
%%%%%%%%%%%%%%%%%%%%%%%%%%%%%%%%%%%%%%%%%%%%%%%%%%%%%%%%%%%%%%%%%%%%%

\begin{problem}
There is a bug on the edge of a 1-meter rug.  The bug wants to cross
to the other side of the rug.  It crawls at 1 cm per second.  However,
at the end of each second, a malicious first-grader named Mildred
Anderson \emph{stretches} the rug by 1 meter.  Assume that her
action is instantaneous and the rug stretches uniformly.  Thus, here's
what happens in the first few seconds:

\begin{itemize}

\item The bug walks 1 cm in the first second, so 99 cm remain ahead.

\item Mildred stretches the rug by 1 meter, which doubles its length.
So now there are 2 cm behind the bug and 198 cm ahead.

\item The bug walks another 1 cm in the next second, leaving 3 cm
behind and 197 cm ahead.

\item Then Mildred strikes, stretching the rug from 2 meters to 3
meters.  So there are now $3 \cdot (3 / 2) = 4.5$ cm behind the bug
and $197 \cdot (3/2) = 295.5$ cm ahead.

\item The bug walks another 1 cm in the third second, and so on.

\end{itemize}

Your job is to determine this poor bug's fate.

\begin{problemparts}

\problempart
During second $i$, what \emph{fraction} of the rug does the
bug cross?

\begin{solution}
During second $i$, the length of the rug is $100i$ cm and
the bug crosses 1 cm.  Therefore, the fraction that the bug crosses is
$1 / 100i$.
\end{solution}

\problempart
Over the first $n$ seconds, what fraction of the rug does the
bug cross altogether?  Express your answer in terms of the Harmonic
number $H_n$.

\begin{solution}
The bug crosses $1/100$ of the rug in the first second,
$1/200$ in the second, $1/300$ in the third, and so forth.  Thus, over
the first $n$ seconds, the fraction crossed by the bug is:
%
\[
\sum_{k=1}^{n} \frac{1}{100k} = H_n / 100
\]
%
(This formula is valid only until the bug reaches the far side of the
rug.)
\end{solution}

\problempart
The known universe is thought to be about $3 \cdot 10^{10}$ light
years in diameter.  How many universe diameters must the bug travel to
get to the end of the rug? (This distance is NOT the inflated distance caused by the stretching but only the actual walking done by the bug).

\begin{solution}
The bug arrives at the far side when the fraction it has
crossed reaches 1.  This occurs when $n$, the number of seconds
elapsed, is sufficiently large that $H_n / 100 \geq 1$.  Now $H_n$ is
approximately $\ln n$, so the bug reaches the end of the rug when:
%
\begin{align*}
\frac{\ln n}{100} & \geq 1 \\
\ln n & \geq 100 \\
n & \geq e^{100} \approx 10^{43.4} \text{ seconds},
\end{align*}
The bug must travel for $10^{43.4}$ seconds; this is also the number of cm
the bug travels. Now, the universe diameter is $3 \cdot 10^{10}$ lightyears,
which is approximately $3 \cdot 10^{28}$ cm. That gives us that the
bug must travel about $10^{15}$ universe diameters.

\iffalse
(exp 100)
;Value: 2.6881171418161614e43 cm

(/ (* 100 1000) .62)
;Value: 161290.32258064515 cm/mi

(* 186000 60 60 24 365)
;Value: 5865696000000  mi/light-year

(* (/ (* 100 1000) .62) (* 186000 60 60 24 365))
;Value: 946080000000000000. cm/light-year

(/ (exp 100)
   (* (/ (* 100 1000) .62) (* 186000 60 60 24 365)))
;Value: 2.841321179832743e25 light years

(/ 2.841321179832743e25 3e10)
;Value: 947107059944247.6\fi
\end{solution}

\end{problemparts}

\end{problem}

%%%%%%%%%%%%%%%%%%%%%%%%%%%%%%%%%%%%%%%%%%%%%%%%%%%%%%%%%%%%%%%%%%%%%
% Problem ends here
%%%%%%%%%%%%%%%%%%%%%%%%%%%%%%%%%%%%%%%%%%%%%%%%%%%%%%%%%%%%%%%%%%%%%
