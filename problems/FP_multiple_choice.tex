\documentclass[problem]{mcs}

\begin{pcomments}
  \pcomment{FP_multiple_choice}
  \pcomment{from: S07.MQ-3/14-3}
  \pcomment{partial order part from: S07.MQ-2/28-1}
  \pcomment{Tom Brown F10: changed a few, reordered them.  Picked favorites.
           Was worth 20 points on the S07 final}
\end{pcomments}

\pkeywords{
  isomorphism
  partial_order  
  total_order
  linear
  asymptotic
  vertices
  gcd
  modular
  mod_n
}

%%%%%%%%%%%%%%%%%%%%%%%%%%%%%%%%%%%%%%%%%%%%%%%%%%%%%%%%%%%%%%%%%%%%%
% Problem starts here
%%%%%%%%%%%%%%%%%%%%%%%%%%%%%%%%%%%%%%%%%%%%%%%%%%%%%%%%%%%%%%%%%%%%%

\begin{problem}\mbox{}

\begin{staffnotes}
(a) 1.5pts  (b) 1/2, 1, 1/2, 1, 3/2, 2
\end{staffnotes}

\bparts

\ppart Circle all the properties below that are preserved under
isomorphism of simple graphs.
\begin{itemize}

\item There is a simple cycle that includes all the vertices.

\item Two edges are of equal length.

\item The graph remains connected if any two edges are removed.

\item There exists an edge that is an edge of every spanning tree.

\item The negation of a property that is preserved under isomorphism.

%% Removed
%\item There are two degree 8 vertices. % too easy
%\item The area enclosed by a graph equals some constant, $A$.  % true for trees?
%\item There are two simple cycles that do not share any vertices. % too easy
%\item There are two connected components. % too easy
%\item The graph can be drawn such that all the edges have the same length. % huh?
%\item The graph is 4-colorable. % too easy
%\item Adding an edge between any two vertices creates a cycle. % spanning tree one better
\end{itemize}

\begin{solution}
All but the second one are preserved.
\end{solution}


% FROM: Spring-07 MQ-2/28-1
%
% COMMENTS: Change to ask if they are weak/strict partial/total orders?
%           f=O(g), f=o(g), f~g, and subsets 

\iffalse

\ppart For each of the relations below, indicate whether it is
\emph{transitive} but not a partial order (\textbf{Tr}), a \emph{linear
order} (\textbf{Tot}), a \emph{strict partial order} that is not linear
(\textbf{S}), a \emph{weak partial order} that is not linear (\textbf{W}),
or \emph{none} of the above (\textbf{N}).
\begin{itemize}

\item the ``is a subgraph of'' relation on finite graphs.  (Note that every graph
is a considered a subgraph of itself.) \examrule{0.5in}
\end{itemize}

Let $f,g$ be nonnegative functions on the real numbers.
\begin{itemize}

\item  the ``Big Oh'' relation, $f=O(g)$, \examrule{0.5in}

\item  the ``Little Oh'' relation, $f=o(g)$, \examrule{0.5in}

\item the ``asymptotically equal'' relation, $f \sim g$. \examrule{0.5in}

%\set{(a,b) | a<b} \text{ on the set of integers }\quad & \examrule{1in}\\
%% Added
%\set{(a,b) | a^2 \leq b^2} \text{ on the set of real numbers }\quad & \examrule{1in}\\

\end{itemize}

\begin{solution}big Oh is \textbf{Tr}, little Oh is \textbf{S}, sim is
\textbf{Tr}, and subgraph is \textbf{W}.
\end{solution}

% FROM: Spring07 MQ-4/6-2a
%
% COMMENTS: Reduced number, changed statements, now asking for
% counterexample. Solution incomplete. Should we replace "Lemma 3.4"
% with an \eqref?
%
% Need to make boxes (make sure they know that counterexamples req'd): 
%     true        false__________________ <-- room for counterexample
%
\fi

\ppart \textbf{Circle} and \emph{provide counterexamples} for all the
\textbf{false} statements below.

% ~~~~~~~~~~~~~~~~~~~~~~~~~~~~~~~~~~~~~~~~~~~~~~~~~~~~~~~~~~~~~~~~~~~
% FROM: Spring07 MQ-3/14-5
%
% COMMENTS: Reduced number, DID NOT change statements, now asking for
% counterexample.
%

The following statements about trees:

\begin{itemize}

\iffalse
\item Any connected subgraph is a tree.

\examspace[0.7in]

\begin{solution}
\textbf{false}.  Any nontree is a counterexample, with the subgraph
being the whole graph.
\end{solution}

\fi

\item Adding an edge between two vertices creates a cycle.

\examspace[0.7in]

\begin{solution}
\textbf{true}.
\end{solution}

\item The number of vertices is one less than twice the number of leaves.

\examspace[0.7in]

\begin{solution}
  \textbf{false}.  This property holds for full binary trees, but not in
  general.  A tree with two vertices is a counterexample.
\end{solution}

\end{itemize}

The following statements about the greatest common divisor:

\begin{itemize}

\iffalse

\item $\gcd(1 + a, 1 + b) = 1 + \gcd(a, b)$.

\begin{solution}
\textbf{false}: $a=1,b=2$
\end{solution}

\item If $a \divides b c$ and $\gcd(a, b) = 1$, then $a \divides c$.
\begin{solution}
\textbf{true}
\end{solution}
\fi

\item $\gcd(a^n,b^n) =  (\gcd(a,b))^n$ 

\examspace[0.7in]

\begin{solution}
\textbf{true}.
\end{solution}

\item If $\gcd(a, b) \neq 1$ and $\gcd(b, c) \neq 1$, then $\gcd(a, c) \neq 1$.

\examspace[0.7in]

\begin{solution}
\textbf{false}: $a=2\cdot 3, b=3\cdot 5, c=5\cdot 7$
\end{solution}

%%Removed
%\item $\gcd(a, b) = \gcd(b, \rem{a}{b})$. % too easy
%\item $\gcd(a,b) \gcd(c,d) =  \gcd(ac,bd)$ % too many vars for counterexample
%\item $\gcd(a, b) = \gcd(a+b, a-b)$ % what does it test?
\end{itemize}

\iffalse

\begin{solution}

\begin{itemize}
\item The first one is not valid. counterexample: $a=2$, $b=6$, $c=3$
\item The second one is not valid. counterexample: $k=1$, $a=1$, $b=2$
\item The third and fourth ones are valid.
\end{itemize}

\end{solution}
\fi

% ~~~~~~~~~~~~~~~~~~~~~~~~~~~~~~~~~~~~~~~~~~~~~~~~~~~~~~~~~~~~~~~~~~~
% FROM: Spring07 MQ-4/6-2b
%
% COMMENTS: Reduced number, changed statements, now asking for
% counterexample. Solution needs update. Should we replace "Lemma 7.3"
% "Corollary 7.2" and "Corollary 10.3" with \eqref's?  Is the third
% one too hard?  Do we need m>=0?
%
% Need to make boxes (make sure they know that counterexamples req'd): 
%     true        false__________________ <-- room for counterexample
%

The following statements about equivalence $\pmod n$, where $n > 1$.

\begin{itemize}

\item If $a c \equiv b c \pmod{n}$ and $n$ does not divide $c$, then $a \equiv b \pmod{n}$.

\examspace[0.7in]

\begin{solution}
\textbf{false}.  $n=2 \cdot 3, a=0, b=2, c=3$
\end{solution}

\item If $a \equiv b \pmod{\phi(n)}$ for $a, b > 0$, then $c^a \equiv c^b
\pmod{n}$.

\examspace[0.7in]

\begin{solution}
  \textbf{false}.  $n=4$, so $\phi(n) = 2$; $a=1, b=3$, so $a \equiv b
  \pmod \phi(n)$, $c = 2$, so $c^a = 2 \not\equiv 0 = c^b \pmod 4$.
\end{solution}

\iffalse
\item If $a \equiv b \pmod{n}$, then $P(a) \equiv P(b) \pmod{n}$ for any
polynomial $P(x)$ with integer coefficients.

\begin{solution}
\textbf{true}
\end{solution}

\item If $a \equiv b \pmod{nm}$, then $a \equiv b \pmod{n}$, for $m,n \geq 1$.

\begin{solution}
\textbf{true}
\end{solution}
\fi

%%Removed
%\item If $a \equiv b \pmod{n}$ and $b \equiv c \pmod{n}$, then $a \equiv c \pmod{n}$. % too easy
%\item If $a \equiv b \pmod{n}$ and $c \equiv d \pmod{n}$, then $a^c \equiv b^d \pmod{n}$. % replaced with phi() one
%\item $\rem{a}{n} \equiv a \pmod{n}$. % too easy
%\item $\gcd(a \pmod{n}, b \pmod{n}) \equiv \gcd(a,b) \pmod{n}$
%\item If $a \equiv b \pmod{n}$ and $a \equiv b \pmod{m}$, then $a \equiv b \pmod{nm}$. % othera m*n one better
\end{itemize}

\iffalse
\begin{solution}

\begin{itemize}
\item The first one is not valid. counterexample: $a=1$, $b=2$, $c=2$, $n=2$.
\item The second one is not valid. counterexample: $n=4, a=1, b=3, c=2$.
\item The third one is valid. A polynomial is constructed with only addition and multiplication.
\item The fourth one is valid. An equivalent statment is ``If $mn | a$, then $m | a$.''
\end{itemize}
% (Corollary 7.2 and Lemma 7.3 in lecture notes 7). The
%first statement does not hold in general if $c$ and $n$ are not
%relatively prime (Corollary 10.3 in lecture notes 7). 
%The third holds.
%The third does not hold, for instance, if $n=3$, $a=2$, $b=5$, $c=0$ and $d=3$.
\end{solution}
\fi

\eparts

\end{problem}

\endinput
