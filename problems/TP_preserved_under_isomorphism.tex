\documentclass[problem]{mcs}

\begin{pcomments}
  \pcomment{TP_preserved_under_isomorphism}
  \pcomment{overlaps FP_graphs_short_answer, FP_multiple_choice_unhidden}
  \pcomment{by ARM 3/29/13 for first simple graph lecture}
\end{pcomments}

\pkeywords{
  simple_graph
  isomorphism
  vertices
  preserved
  degree
  edge
  path
}

%%%%%%%%%%%%%%%%%%%%%%%%%%%%%%%%%%%%%%%%%%%%%%%%%%%%%%%%%%%%%%%%%%%%%
% Problem starts here
%%%%%%%%%%%%%%%%%%%%%%%%%%%%%%%%%%%%%%%%%%%%%%%%%%%%%%%%%%%%%%%%%%%%%
\begin{problem}
Which of the items below are simple-graph properties preserved under
isomorphism?

\bparts

\ppart The vertices can be numbered 1 through 7.

\ppart There is a cycle that includes all the vertices.

\begin{staffnotes}
If asked, explain that simple graph cycles can be defined in the
essentially same way as for digraphs.  The only difference is that going
back and forth on the same edge---a length 2 ``cycle''---is not
considered to be a cycle.
\end{staffnotes}

\ppart There are two degree 8 vertices.

\ppart\label{eqlen} Two edges are of equal length.

\begin{staffnotes}
Not a property of simple graphs since edges don't have length.
\end{staffnotes}

%\ppart There are exacty two spanning trees.

\ppart No matter which edge is removed, there is a path between any two
  vertices.

\ppart There are two cycles that do not share any vertices. % too easy

%\ppart There are two connected components. % too easy

%\ppart\label{edgesubset} One edge is a subset of another one.

\ppart\label{vertexsubset} One vertex is a subset of another one.

\begin{staffnotes}
If need be, remind students that we used sets as vertices when we
proved that every poset can be represented sets under containment.
\end{staffnotes}

\ppart The graph can be pictured in a way that all the edges have the same length.

%\ppart The graph is 4-colorable. % too easy

%\ppart Adding an edge between any two vertices creates a cycle. % spanning tree one better

\ppart\label{isoOR} The $\QOR$ of two properties that are preserved under isomorphism.

\begin{staffnotes}
When students have figured out that item~\eqref{isoOR} is in the positive category, ask them to do a careful proof of that fact.  See solution for an example.
\end{staffnotes}

\ppart The negation of a property that is preserved under isomorphism.

\eparts

\begin{solution}
Item~\eqref{eqlen} is not a property of simple graphs, since edges
don't have length.  

Item~\eqref{vertexsubset} is not preserved under isomorphism.
Although it can be useful to use sets as vertices---as was done for
representing DAGs in Theorem~\bref{thm:posetrepsets}---vertices need
not be represented by sets, and since isomorphism does not depend on what
vertices are made of, vertices being sets and any set-theoretic
properties they may have are not going to be preserved.

\iffalse
Note that the version of~\eqref{vertexsubset} originally used in class
asked whether one \emph{edge} was a subset of another one.  Since an
edge in a simple graph is defined to be a set of size two, namely the
set consisting of its endpoints, and no set of size two can be a
subset of another subset of size two, the property in this version of
the problem was \emph{always false} and so was \emph{vacously
  preserved}.
\fi

All the others are preserved.  We'll prove this just for
item~\eqref{isoOR}:
\begin{proof}
Suppose $P$ and $Q$ are graph properties preserved under isomorphism,
and $G$ and $H$ are isomorphic simple graphs.  Let $R \eqdef P \QOR
Q$.  Then
\[\begin{array}{rcll}
R(G) 
  & \QIFF & P(G) \QOR Q(G)
     & \text{(def of $R$)}\\
  & \QIMPLIES & P(H) \QOR Q(H) 
     & \text{(since $P$, $Q$ are preserved)}\\
  & \QIFF & R(H)
     & \text{(def of $R$)},
\end{array}\]
so $R$ is preserved, as claimed.
\end{proof}
\end{solution}

\end{problem}

\endinput
