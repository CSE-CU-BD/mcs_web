\documentclass[problem]{mcs}

%TP_inverse_relation_table.tex

\begin{pcomments}
  \pcomment{from: excerpted from CP_relational_properties_table
            do not use with that problem}

%  \pcomment{}
\end{pcomments}

\pkeywords{
  relations
  total
  relational_properties
  functions
  injections
  surjections
  bijections
}

%%%%%%%%%%%%%%%%%%%%%%%%%%%%%%%%%%%%%%%%%%%%%%%%%%%%%%%%%%%%%%%%%%%%%
% Problem starts here
%%%%%%%%%%%%%%%%%%%%%%%%%%%%%%%%%%%%%%%%%%%%%%%%%%%%%%%%%%%%%%%%%%%%%

\begin{problem}
  The \term{inverse}, $\inv{R}$, of a binary relation, $R$, from $A$ to
  $B$, is the relation from $B$ to $A$ defined by:
\[
b \mrel{\inv{R}} a \qiff a \mrel{R} b.
\]
In other words, you get the diagram for $\inv{R}$ from $R$ by ``reversing
the arrows'' in the diagram describing $R$.  Now many of the relational
properties of $R$ correspond to different properties of $\inv{R}$.  For
example, $R$ is an \emph{total} iff $\inv{R}$ is a \emph{surjection}.

Fill in the remaining entries is this table:
\begin{center}
\begin{tabular}{l|cl}
$R$ is  & iff & $\inv{R}$ is \\ \hline
total                    && a surjection\\
a function\\
a surjection\\
an injection\\
a bijection
\end{tabular}
\end{center}

\hint Explain what's going on in terms of ``arrows'' from $A$ to $B$ in
the diagram for $R$.

\begin{solution}

\begin{center}
\begin{tabular}{l|cl}
$R$ is  & iff & $\inv{R}$ is \\ \hline
total                    && a surjection\\
a function               && \insolutions{an injection}\\
a surjection             && \insolutions{total}\\
an injection             && \insolutions{a function}\\
a bijection              && \insolutions{a bijection}
\end{tabular}
\end{center}

\end{solution}

\end{problem}


%%%%%%%%%%%%%%%%%%%%%%%%%%%%%%%%%%%%%%%%%%%%%%%%%%%%%%%%%%%%%%%%%%%%%
% Problem ends here
%%%%%%%%%%%%%%%%%%%%%%%%%%%%%%%%%%%%%%%%%%%%%%%%%%%%%%%%%%%%%%%%%%%%%

\endinput
