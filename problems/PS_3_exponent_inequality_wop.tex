\documentclass[problem]{mcs}

\begin{pcomments}
  \pcomment{PS_3_exponent_inequality_wop}
  \pcomment{renamed from PS_3_exponent_inequality}
  \pcomment{adapted by ARM from induction version in FP_3_exponent_inequality_induction}
\end{pcomments}

\pkeywords{
 well_ordering
 inequality
}

%%%%%%%%%%%%%%%%%%%%%%%%%%%%%%%%%%%%%%%%%%%%%%%%%%%%%%%%%%%%%%%%%%%%%
% Problem starts here
%%%%%%%%%%%%%%%%%%%%%%%%%%%%%%%%%%%%%%%%%%%%%%%%%%%%%%%%%%%%%%%%%%%%%

\begin{problem}
Use the Well Ordering Principle to prove that
\begin{equation}\label{n3n3}
n \leq 3^{n/3}
\end{equation}
for every nonnegative integer, $n$.

\hint Verify~\eqref{n3n3} for $n \leq 4$ by explicit calculation.

\begin{solution}
  Suppose to the contrary that~\eqref{n3n3} failed for some
  nonnegative integer.  Then by the WOP, there is a least such
  nonnegative integer, $m$.

  But $0 \leq 3^{0/3}$, so $m \neq 0$.  Also, $1^3 \leq 3^1$, so taking
  cube roots, $1 \leq 3^{1/3}$, which implies $m \neq 1$.  Likewise, $2^3
  \leq 3^2$, so taking cube roots, $2 \leq 3^{2/3}$, which implies $m \neq
  2$.  Similar simple calculations show that $m \neq 3,4$, so we know that
  $m \geq 5$.

  Now since $m > m-3 \geq 0$ and $m$ is the least nonnegative integer
  for which the inequality~\eqref{n3n3} fails, the inequality must
  hold when $n = m-3$. So
\begin{align}
3^{m/3} & = 3 \cdot 3^{(m-3)/3} \notag\\
    & \geq 3 \cdot (m-3)  & \text{(by~\eqref{n3n3} for $n = m-3$)}\label{g3m3}
\end{align}
Also,
\begin{align}
3 \cdot (m-3) & = 3m - 9\notag\\
              & > 3m - 2m & \text{since $m > 9/2$}\notag\\
              & = m.\label{3m3gm}
\end{align}

Combining~\eqref{g3m3} and~\eqref{3m3gm}, we get
\[
m \leq 3^{m/3},
\]
contradicting the assumption that~\eqref{n3n3} fails for $n = m$.

This contradiction implies that there cannot be a nonnegative integer for
which~\eqref{n3n3} fails.  By the WOP, this means that~\eqref{n3n3} must
hold for all nonnegative integers.

\end{solution}

\end{problem}

%%%%%%%%%%%%%%%%%%%%%%%%%%%%%%%%%%%%%%%%%%%%%%%%%%%%%%%%%%%%%%%%%%%%%
% Problem ends here
%%%%%%%%%%%%%%%%%%%%%%%%%%%%%%%%%%%%%%%%%%%%%%%%%%%%%%%%%%%%%%%%%%%%%

\endinput
