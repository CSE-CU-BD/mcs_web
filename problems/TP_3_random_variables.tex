\documentclass[problem]{mcs}

\begin{pcomments}
  \pcomment{TP_3_random_variables}
  \pcomment{from: S09.cp14m}
\end{pcomments}

\pkeywords{
  random_variables
  mutually_independent
  density
}

%%%%%%%%%%%%%%%%%%%%%%%%%%%%%%%%%%%%%%%%%%%%%%%%%%%%%%%%%%%%%%%%%%%%%
% Problem starts here
%%%%%%%%%%%%%%%%%%%%%%%%%%%%%%%%%%%%%%%%%%%%%%%%%%%%%%%%%%%%%%%%%%%%%


\begin{problem}
Suppose $X_1$, $X_2$, and $X_3$ are three mutually independent random
variables, each having the uniform distribution
\[
\forall k, \ k \in{\set{1,2,3}}. \ \pr{ X_i = k } = \frac{1}{3}.
\]

Let $M$ be another random variable giving the maximum of these three
random variables.  What is the probability density function of $M$?

\begin{solution}
\begin{align*}
\pdf_M(1) & = \frac{1}{27}\\
\pdf_M(2) & = \frac{7}{27} \\
\pdf_M(3) & = \frac{19}{27}
\end{align*}

This can be hashed out by counting the possible outcomes.  Alternatively,
we can reason as follows:

The event $M=1$ is the event that all three of the variables equal 1, and
since they are mutually independent, we have
\[
\pr{M=1}\ \ =\ \ \pr{X_1=1}\cdot \pr{X_2=1}\cdot \pr{X_3=1} =
\paren{\frac{1}{3}}^3 = \frac{1}{27}.
\]

To compute $\pr{M=2}$, we first compute $\pr{M \leq 2}$.  Now the event
$[M \leq 2]$ is the event that all three of the variables is at most 2, so
by mutual independence we have
\[
\pr{M\leq 2}\ \ =\ \ \pr{X_1 \leq 2}\cdot \pr{X_2\leq 2}\cdot \pr{X_3 \leq 2} =
\paren{\frac{2}{3}}^3 = \frac{8}{27}.
\]
Therefore,
\[
\pr{M=2}\ \ =\ \ \pr{M \leq 2} - \pr{M=1} = \frac{8}{27} - \frac{1}{27} =
\frac{7}{27}.
\]

Finally,
\[
\pr{M=3}\ \ =\ \ 1- \pr{M\leq 2} = 1- \frac{8}{27} = \frac{19}{27}.
\]

\end{solution}

\end{problem}

%%%%%%%%%%%%%%%%%%%%%%%%%%%%%%%%%%%%%%%%%%%%%%%%%%%%%%%%%%%%%%%%%%%%%
% Problem ends here
%%%%%%%%%%%%%%%%%%%%%%%%%%%%%%%%%%%%%%%%%%%%%%%%%%%%%%%%%%%%%%%%%%%%%

\endinput
