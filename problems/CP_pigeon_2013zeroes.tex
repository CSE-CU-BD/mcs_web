\documentclass[problem]{mcs}

\begin{pcomments}
  \pcomment{CP_pigeon_2013zeroes}
  \pcomment{Dusan_problem}
  \pcomment{formatted by ARM 3/9/15}
\end{pcomments}

\pkeywords{
  pigeon_hole
  decimal_expansion
  modulo
}

%%%%%%%%%%%%%%%%%%%%%%%%%%%%%%%%%%%%%%%%%%%%%%%%%%%%%%%%%%%%%%%%%%%%%
% Problem starts here
%%%%%%%%%%%%%%%%%%%%%%%%%%%%%%%%%%%%%%%%%%%%%%%%%%%%%%%%%%%%%%%%%%%%%

%Problem Proposal for the Final Exam: Dusan Milijancevic, May 18, 2013}

\begin{problem}
The aim of this problem is to prove that there exist a natural number
$n$ such that $3^n$ has at least $2013$ consecutive zeros in its
decimal expansion.

\bparts

\ppart\label{n3n110} Prove that there exist a nonnegative integer $n$ such that
\[
3^n\equiv 1 pmod{10^{2014}}.
\]

\hint Use pigeonhole principle or Euler's theorem.

\begin{solution}
There are two ways to solve this problem as indicated in the hint.
\begin{itemize}

\item \textbf{Pigeonhole principle}.  Let
\[
A\eqdef \set{3^i \suchthat i \in \Zintvcc{0}{10^{2014}}.
\]
At least two numbers from $A$ have the same reminder when divided by
$10^{2013}$, by the pigeon hole principle.  Call them $3^i$ and $3^j$
for $i>j$.  Then $10^{2014} \divides 3^j(3^{i-j}-1)$, so $10^{2014}
\divides 3^{i-j}-1$.  Hence $3^n\equiv 1 \pmod{10^{2014}}$ for
$n=i-j$.

\item \textbf{Euler's theorem}. As $gcd(3,10^{2014})=1$ for
  $n=\phi(10^{2014})$ we have $3^{n}\equiv 1 \pmod{10^{2014}}$.

\end{itemize}

\end{solution}

\ppart Conclude that there exist a natural number $n$ such that $3^n$
has at least $2013$ consecutive zeros.

\begin{solution}
For $n$ in part~\eqref{n3n110}, we have
\[
3^n-1=***...\underbrace{000....000}_{2014},
\]
so the decimal expansion of $3^n$ ends with least 2013 consecutive
zeros followed by a one.
\end{solution}

\eparts

\end{problem}
%%%%%%%%%%%%%%%%%%%%%%%%%%%%%%%%%%%%%%%%%%%%%%%%%%%%%%%%%%%%%%%%%%%%%
% Problem ends here
%%%%%%%%%%%%%%%%%%%%%%%%%%%%%%%%%%%%%%%%%%%%%%%%%%%%%%%%%%%%%%%%%%%%%

\endinput
