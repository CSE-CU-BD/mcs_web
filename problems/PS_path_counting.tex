\documentclass[problem]{mcs}

\begin{pcomments}
  \pcomment{PS_path_counting}
  \pcomment{S13,ps11, S09.ps10}
  \pcomment{minor variation of CP_inclusion_exclusion_paths}
\end{pcomments}

\pkeywords{
  Inclusion-exclusion
}

%%%%%%%%%%%%%%%%%%%%%%%%%%%%%%%%%%%%%%%%%%%%%%%%%%%%%%%%%%%%%%%%%%%%%
% Problem starts here
%%%%%%%%%%%%%%%%%%%%%%%%%%%%%%%%%%%%%%%%%%%%%%%%%%%%%%%%%%%%%%%%%%%%%

\begin{problem}
  How many paths are there from point $(0, 0)$ to $(50, 50)$ if each
  step along a path increments one coordinate and leaves the other
  unchanged?  How many are there when there are impassable boulders
  sitting at points $(10, 11)$ and $(21, 20)$?  (You do not have to
  calculate the number explicitly; your answer may be an expression
  involving binomial coefficients.)

\hint Inclusion-Exclusion.

\begin{staffnotes}
\hint Suggest counting the number of paths going through $(10,11)$ and
through both $(10,11)$ and $(21,20)$.
\end{staffnotes}

\begin{solution}
We use Inclusion-Exclusion.  The total number of paths is
$\binom{100}{50}$, but we must subtract off the obstructed paths.
There are $\binom{21}{10} \cdot \binom{79}{40}$ paths through the
first boulder, since there are $\binom{21}{10}$ paths from the start
to the first boulder and $\binom{79}{40}$ paths from the boulder to
the finish.  Similarly, there are $\binom{41}{20} \cdot
\binom{59}{30}$ paths through the second boulder.  However, we must
subtract off paths going through both boulders.  The number of these
is the number of paths from the origin to the first boulder times the number
of paths from the first boulder to the second boulder times the number
of paths from the second boulder to the end, namely
\[
\binom{21}{10} \cdot \binom{20}{9} \cdot \binom{59}{30}.
\]
Therefore, the total number of paths is: %
\[
\binom{100}{50} - \binom{21}{10} \cdot \binom{79}{40} - \binom{41}{20}
\cdot \binom{59}{30} + \binom{21}{10} \cdot \binom{20}{9} \cdot
\binom{59}{30}
\]

\end{solution}

\end{problem}


%%%%%%%%%%%%%%%%%%%%%%%%%%%%%%%%%%%%%%%%%%%%%%%%%%%%%%%%%%%%%%%%%%%%%
% Problem ends here
%%%%%%%%%%%%%%%%%%%%%%%%%%%%%%%%%%%%%%%%%%%%%%%%%%%%%%%%%%%%%%%%%%%%%

\endinput
