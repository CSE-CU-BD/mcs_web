\documentclass[problem]{mcs}

\begin{pcomments}
  \pcomment{CP_transitive_irreflexive_implies_asymmetric}
  \pcomment{renamed from CP_transitive_irreflexive_implies_SPO}
  \pcomment{from: S09.cp3r}
  \pcomment{subsumes CP_strict_PO_irreflexive}
\end{pcomments}


\pkeywords{
  relational_properties
  partial_orders
  irreflexive
  transitive
}

%%%%%%%%%%%%%%%%%%%%%%%%%%%%%%%%%%%%%%%%%%%%%%%%%%%%%%%%%%%%%%%%%%%%%
% Problem starts here
%%%%%%%%%%%%%%%%%%%%%%%%%%%%%%%%%%%%%%%%%%%%%%%%%%%%%%%%%%%%%%%%%%%%%

\begin{problem}
  Prove directly from the definitions (without appealing to DAG
  properties) that if a binary relation $R$ on a set $A$ is
  transitive and irreflexive, then it is asymmetric.%
\index{binary relation!properties}%
\index{transitive relation}%
\index{asymmetric relation}%
\index{irreflexive relation}

\begin{solution}
  Suppose $R$ is transitive and irreflexive.  To show that $R$ is a
  strict partial order, we need only show that it is asymmetric.  So
  suppose $a\mrel{R}b$ holds for some $a,b\in A$.  We need to prove
  $\QNOT(b \mrel{R} a)$.

  Assume to the contrary that $b\mrel{R}a$ holds.  Now $a\mrel{R}b$
  and $b\mrel{R}a$, imply $a\mrel{R}a$ by transitivity. This
  contradicts the fact that $R$ is irreflexive.
\end{solution}

\end{problem}

%%%%%%%%%%%%%%%%%%%%%%%%%%%%%%%%%%%%%%%%%%%%%%%%%%%%%%%%%%%%%%%%%%%%%
% Problem ends here
%%%%%%%%%%%%%%%%%%%%%%%%%%%%%%%%%%%%%%%%%%%%%%%%%%%%%%%%%%%%%%%%%%%%%

\endinput
