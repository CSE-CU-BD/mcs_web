\documentclass[problem]{mcs}

\begin{pcomments}
  \pcomment{PS_credit_union}
  \pcomment{from: S07.ps7}
\end{pcomments}

\pkeywords{
  interest_rate
  geometric_sum
  geometric_series
  interest_rate
}

%%%%%%%%%%%%%%%%%%%%%%%%%%%%%%%%%%%%%%%%%%%%%%%%%%%%%%%%%%%%%%%%%%%%%
% Problem starts here
%%%%%%%%%%%%%%%%%%%%%%%%%%%%%%%%%%%%%%%%%%%%%%%%%%%%%%%%%%%%%%%%%%%%%

\begin{problem}
Suppose you deposit \$100 into your MIT Credit Union account
today, \$99 in one month from now, \$98 in two months from now, and so on.
Given that the \idx{interest rate} is constantly 0.3\% per month, how long will
it take to save \$5,000?

\begin{solution}
First note that you will certainly manage to have saved \$5,000
\emph{some} day, since, even without your earnings from the interest,
you will have $100+99+\cdots+1=100\times 101/2 = 5,050$ dollars after
99~months.

But fewer months will be needed: after the first deposit you will have
\$100.  After the second deposit, you will have \$$(100\times 1.003 +
99)$.  After your third deposit, your saved money will be
\[
\$(100\times 1.003 + 99)\times 1.003 + 98)
= \$\paren{100\times (1.003)^2 + 99\times 1.003 + 98},
\]
and so on.  So after the $n$th deposit,
\[
S_n \eqdef \sum_{i=0}^{n-1} (100 - i) (1.003)^{n-i-1}
\]
dollars will be in your account.  Substituting $j=n-i-1$, we can rewrite
this as
\[
\sum_{j=0}^{n-1} \bigl( 100 - (n-j-1)\bigr) (1.003)^{j},
\]
and then as
\[
(101-n)
\paren{\sum_{j=0}^{n-1} (1.003)^{j}} +
\paren{\sum_{j=0}^{n-1} j (1.003)^{j}}.
\]
Using the closed form from Problem~\ref{CP_neat_trick_for_geometric_sum},
we can finally write $S_n$ as
\[
(101-n)
\paren{\frac{1-1.003^n}{1-1.003}} +
\paren{\frac{1.003-n1.003^n+(n-1)1.003^{n+1}}{(1-1.003)^2}}.
\]
Solving the inequality $S_n \geq 5,000$ for $n$, we get $n\geq 67$.  That
is, you'll need more than 5.5~years to save~\$5,000.
\end{solution}
\end{problem}

%%%%%%%%%%%%%%%%%%%%%%%%%%%%%%%%%%%%%%%%%%%%%%%%%%%%%%%%%%%%%%%%%%%%%
% Problem ends here
%%%%%%%%%%%%%%%%%%%%%%%%%%%%%%%%%%%%%%%%%%%%%%%%%%%%%%%%%%%%%%%%%%%%%

\endinput
