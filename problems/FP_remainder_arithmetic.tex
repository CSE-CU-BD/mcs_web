\documentclass[problem]{mcs}

\begin{pcomments}
  \pcomment{FP_remainder_arithmetic}
  \pcomment{smaller numbers than CP_remainder_computation_practice}
  \pcomment{by ARM 4/2/17}
\end{pcomments}

\pkeywords{
  number_theory
  modular_arithmetic
  exponent
  remainder
}

%%%%%%%%%%%%%%%%%%%%%%%%%%%%%%%%%%%%%%%%%%%%%%%%%%%%%%%%%%%%%%%%%%%%%
% Problem starts here
%%%%%%%%%%%%%%%%%%%%%%%%%%%%%%%%%%%%%%%%%%%%%%%%%%%%%%%%%%%%%%%%%%%%%

\begin{problem}
Compute the remainder
\begin{equation}\tag{rem15}
\rem{24989^{184637} \cdot 673459^{8447}}{15},
\end{equation}
carefully explaining the steps in your computation.  \inhandout{A
  correct explanation will get nearly full credit even if there is an
  arithmetic error in the final answer.}

\begin{solution}
The remainder is 11.

Following the General Principle of Remainder Arithmetic from
Section~\bref{remainder_arithmetic_sec}, first replace the numbers
being raised to powers by their remainders.  Since $\rem{24989}{15} =
14$ and $\rem{673459}{15} = 4$, this implies that~(rem15) equals
\[
\rem{14^{184637} \cdot 4^{8447}}{15}.
\]

Looking next at the remainders of powers of 14:
\begin{align*}
\rem{14^1}{15} & = 14,\\
\rem{14^2}{15} & = 1,\\
\rem{14^3}{15} & = 14,\\
\rem{14^4}{15} & = 1,\\
               & \vdots
\end{align*}
So the remainder on division by 15 of 14 raised to any odd power will be
14.  In particular,
\[
\rem{14^{184637}}{15} = 14.
\]

Similarly,
\begin{align*}
\rem{4^1}{15} & = 4,\\
\rem{4^2}{15} & = 1,\\
\rem{4^3}{15} & = 4,\\
\rem{4^4}{15} & = 1,\\
               & \vdots
\end{align*}
So the remainder on division by 15 of 14 raised to any odd power will
be 4.  In particular,
\[
\rem{4^{8447}}{15} = 4.
\]

Therefore the remainder in~(rem15) equals
\[
\rem{14 \cdot 4}{15} = 11.
\]
\end{solution}

\end{problem}

%%%%%%%%%%%%%%%%%%%%%%%%%%%%%%%%%%%%%%%%%%%%%%%%%%%%%%%%%%%%%%%%%%%%%
% Problem ends here
%%%%%%%%%%%%%%%%%%%%%%%%%%%%%%%%%%%%%%%%%%%%%%%%%%%%%%%%%%%%%%%%%%%%

\endinput
