\documentclass[problem]{mcs}

\begin{pcomments}
  \pcomment{FP_probable_equivalence}
  \pcomment{F03.final}
\end{pcomments}

\pkeywords{
  equivalence
  equivalence_relation
  reflexive
  symetric
  transitive
  random
}

\begin{problem}
Let $\mathcal{S}$ be the set $\set{a, b, c}$.  Suppose that we
construct a binary relation $R$ on $\mathcal{S}$ with a randomized
procedure.  In particular, for each $x, y \in \mathcal{S}$, the
relation $x\mrel{R}y$ holds with probability $p$ and fails to hold
with probability $q \eqdef 1-p$ mutually independently.  So, for
example, the probability that $b\mrel{R}c$ and no other relationships
hold is $pq^8$.  Write formulas in terms of $p$ and $q$ for the
probability of each of the following events.

\bparts

\ppart $R$ is reflexive.

\examspace[2in]

\begin{solution}
$R$ is reflexive if and only if $a\mrel{R}a$, $b\mrel{R}b$, and
  $c\mrel{R}c$ all hold.  Since each of these events occurs
  independently with probability $p$, the probability that $R$ is
  reflexive is $p^3$.
\end{solution}

\ppart $R$ is symmetric.

 \examspace[2in]
\begin{solution}
$R$ is symmetric if:

\begin{itemize}
\item $a\mrel{R}b$ and $b\mrel{R}a$ both hold or both do not hold
\item $b\mrel{R}c$ and $c\mrel{R}b$ both hold or both do not hold
\item $a\mrel{R}c$ and $c\mrel{R}a$ both hold or both do not hold
\end{itemize}

Each of these events occurs independently with probability $p^2 +
q^2$.  Therefore, all three occur with probability $(p^2 + q^2)^3$.
\end{solution}

\ppart $R$ is an equivalence relation with the equivalence classes
$\set{b}$ and $\set{a,c}$.

\begin{solution}
$p^5q^4$.

There is only one relation $R$ satisfying these conditions.  Of the nine
possible pairs in $\mathcal{S} \cross \mathcal{S}$, 5 must be in $R$,
namely, $(a,a),(b,b),(c,c),(a,c),(c,a)$, and the others must not be in
$R$.  
\end{solution}

\eparts

\end{problem}
