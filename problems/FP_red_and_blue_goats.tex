\documentclass[problem]{mcs}

\begin{pcomments}
  \pcomment{FP_red_and_blue_goats}
  \pcomment{FP_red_and_blue_goats_fall11 uses part(b)}
  \pcomment{from S09.final}
  \pcomment{minor edit ARM 12/15/11}
\end{pcomments}

\pkeywords{
  probability
  four_step_method
  Monty_Hall
  outcome  
}

%%%%%%%%%%%%%%%%%%%%%%%%%%%%%%%%%%%%%%%%%%%%%%%%%%%%%%%%%%%%%%%%%%%%%
% Problem starts here
%%%%%%%%%%%%%%%%%%%%%%%%%%%%%%%%%%%%%%%%%%%%%%%%%%%%%%%%%%%%%%%%%%%%%

\begin{problem}
  Suppose that {\em Let's Make a Deal} is played according to slightly
  different rules and with a red goat and a blue goat.  There are
  three doors, with a prize hidden behind one of them and the goats
  behind the others.  No doors are opened until the contestant makes a
  final choice to stick or switch.  The contestant is allowed to pick
  a door and ask a certain question that the host then answers
  honestly.  The contestant may then stick with their chosen door, or
  switch to either of the other doors.

  \bparts \ppart If the contestant asks ``is there is a goat behind
  one of the unchosen doors?'' and the host answers ``yes,'' is the
  contestant more likely to win the prize if they stick, switch, or
  does it not matter?  Clearly identify the probability space of
  outcomes and their probabilities you use to model this situation.
  What is the contestant's probability of winning if he uses the best
  strategy?

\examspace[2in]

\begin{solution}
  It doesn't matter.  There will always be a goat behind one of the
  unchosen doors and so there is no new information provided.  Whether
  they stick or switch, there is a 1/3 chance they pick the prize.
\end{solution}

  \ppart If the contestant asks ``is the \emph{red} goat behind one of
  the unchosen doors?'' and the host answers ``yes,'' is the
  contestant more likely to win the prize if they stick, switch, or
  does it not matter?  Clearly identify the probability space of
  outcomes and their probabilities you use to model this situation.
  What is the contestant's probability of winning if he uses the best
  strategy?

\examspace

\begin{solution}
They are more likely to win if they stick.
  
To model the stick strategy, we can use three equally likely outcomes
corresponding to the contestant picking the door with the prize, the
red goat or the blue goat.  The contestant wins in one outcome---when
he picks the prize.  Being given that the red goat is behind an
unchosen door excludes the outcome of picking the red goat, leaving
only a winning outcome and an equally likely losing outcome, so
sticking leads to a probability of winning equal to $1/2$.

For the switch strategy, we can use six equally likely outcomes
corresponding to what's behind the picked door and then what's behind
the switched-to door.  Given that the red goat is behind an unchosen
door rules out two of the outcomes, leaving four equally likely
outcomes for which the red goat is not behind the door first picked by
the contestant.  In two of these outcomes, the contestant first picks
the door with the prize, so these are both losing outcomes under the
switch strategy.  In the other two outcomes, the contestant first
picks the door with the blue goat and then switches either to the red
goat or the prize.  The contestant wins only in the outcome where he
switches to the prize.  So the contestant wins in only one of the four
outcomes, and therefore switching wins with probability only $1/4$.
\end{solution}

\eparts
\end{problem}


%%%%%%%%%%%%%%%%%%%%%%%%%%%%%%%%%%%%%%%%%%%%%%%%%%%%%%%%%%%%%%%%%%%%%
% Problem ends here
%%%%%%%%%%%%%%%%%%%%%%%%%%%%%%%%%%%%%%%%%%%%%%%%%%%%%%%%%%%%%%%%%%%%%

\endinput
