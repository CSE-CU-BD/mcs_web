\documentclass[problem]{mcs}

\begin{pcomments}
  \pcomment{MQ_st_petersburg}
  \pcomment{renamed from CP_st_petersbErg}
  \pcomment{F07.rec14h}
  \pcomment{subsumed by sec:st_petersburg}}
\end{pcomments}

\pkeywords{
  deviation
  pairwise_independent
  paradox
  roulette
  mean_time_to_failure
}

%%%%%%%%%%%%%%%%%%%%%%%%%%%%%%%%%%%%%%%%%%%%%%%%%%%%%%%%%%%%%%%%%%%%%
% Problem starts here
%%%%%%%%%%%%%%%%%%%%%%%%%%%%%%%%%%%%%%%%%%%%%%%%%%%%%%%%%%%%%%%%%%%%%

\begin{problem}
A gambler bets \$10 on ``red'' at a roulette table (the odds of red are
18/38, slightly less than even) to win \$10.  If he wins, he gets
back twice the amount of his bet, and he quits.  Otherwise, he doubles his
previous bet and continues.

For example, if he loses his first two bets but wins his third bet,
the total spent on his three bets is $10+20+40$ dollars, but he gets
back $2 \cdot 40$ dollars after his win on the third bet, for a net
profit of\$10.

\bparts

\ppart What is the expected number of bets the gambler makes before he
wins?

\begin{solution}
This is mean time to failure, with failure being a red number coming up.  So
the expected time (number of bets) is
\[
\frac{1}{18/38} = 2\ \frac{1}{9}.
\]
\end{solution}

\ppart What is his probability of winning?

\begin{solution}
He is certain to win, since $\pr{> k \text{ bets}} = (20/38)^{k}$ which
goes to zero as $k$ goes to infinity.  More fully,
\[
\pr{\text{win}} \geq \pr{\text{win in $\leq k$ bets}}
= 1 - \pr{> k \text{ bets}}
\]
and this last expression goes to 1 as $k$ goes to infinity.
\end{solution}

\ppart What is his expected final profit (amount won minus amount lost)?

\begin{solution}
His final profit is always \$10 whenever he finally wins, and he is
certain to win, so \$10 is also his expected final profit.
\end{solution}

\ppart The fact that the gambler's expected profit is positive,
despite the fact that the game is biased against him, is explained in
Section~\ref{sec:martingale}.  The apparent paradox is explained by the fact
that bet doubling is not a feasible strategy: prove that the expected
size of the gambler's last bet is infinite.

\begin{solution}
It sounds plausible that, since his expected number of bets is less
than three, the expected size of his bet would be less than the size
of his third bet, that is, $10\cdot 2^2 = 40$ dollars.  But this is a
sloppy ---and wrong ---argument.  What it overlooks is that later bets,
though progressively less likely, grow tremendously, and so contribute
heavily to the expected bet size.

To get the answer, we go back to the definition of expected bets.  Let $B$
be the size of his last bet in dollars.  Now if he wins his $\$10$ final
profit on the $k$th bet, then $B=10\cdot 2^{k-1}$, so
\[
\pr{B=10\cdot 2^{k-1}} = (20/38)^{k-1}(18/38)
\]
So
\begin{align*}
\expect{B} & = \sum_{k \in \naturals^{+}} 10\cdot
      2^{k-1}\paren{\frac{20}{38}}^{k-1}(18/38)\\
  & = 10(18/38) \sum_{k \in \naturals} 2^k\paren{(20/38)^k}\\
  & > \sum_{k \in \naturals} \paren{1+ \frac{1}{19}}^k= \infty
\end{align*}
so the gambler has to have an infinite bank account to win \$10 with
certainty.
\end{solution}

\eparts

\end{problem}


%%%%%%%%%%%%%%%%%%%%%%%%%%%%%%%%%%%%%%%%%%%%%%%%%%%%%%%%%%%%%%%%%%%%%
% Problem ends here
%%%%%%%%%%%%%%%%%%%%%%%%%%%%%%%%%%%%%%%%%%%%%%%%%%%%%%%%%%%%%%%%%%%%%

\endinput
