\documentclass[problem]{mcs}

\begin{pcomments}
  \pcomment{PS_emailed_at_most_2_others}
  \pcomment{formerly called PS_translate_sentence_into_predicate_formula}
  \pcomment{small perturbation of PS_emailed_exactly_2_others}
  \pcomment{from: S06.ps1}
  \pcomment{edited ARM 2/13/12}
\end{pcomments}

\pkeywords{
  predicate
  formula
  translate
  sentence
  first-order logic
}

%%%%%%%%%%%%%%%%%%%%%%%%%%%%%%%%%%%%%%%%%%%%%%%%%%%%%%%%%%%%%%%%%%%%%
% Problem starts here
%%%%%%%%%%%%%%%%%%%%%%%%%%%%%%%%%%%%%%%%%%%%%%%%%%%%%%%%%%%%%%%%%%%%%

\begin{problem}
Translate the following sentence into a predicate formula:
\begin{quote}
There is a student who has e-mailed at most two other people in the class,
besides possibly himself.
\end{quote}

The domain of discourse should be the set of students in the class; in
addition, the only predicates that you may use are 
\begin{itemize}
\item equality, and
\item $E(x,y)$, meaning that ``$x$ has sent e-mail to $y$.''
\end{itemize}

\begin{solution}
A good way to begin tackling this problem is by trying to translate parts
of the sentence. First of all, our formula must be of the form
\[
\exists x. P(x)
\]
where $P(x)$ should be a formula that says that ``student $x$ has
e-mailed at most two other people in the class, besides possibly
himself''.

One way to write $P(x)$ is to write that ``whenever we meet a student
that has been e-mailed by $x$, this student is either $x$ himself or
$y$ or $z$, where $y$ and $z$ are particular students in the class.''

To write the part ``whenever we meet a student that has been e-mailed
by $x$, this student is either $x$ himself or $y$ or $z$'' we write
\[
\forall s\, \paren{ E(x,s) \QIMPLIES (s=x \QOR s=y \QOR s=z)}.
\]
The part ``where $y$ and $z$ are two students in the class'' simply
means that there exist two such students; so by adding the appropriate
existential quantifiers, we get
\[
P(x) \eqdef \quad\exists y. \exists z.\; 
\forall s\, \paren{ E(x,s) \QIMPLIES (s=x \QOR s=y \QOR s=z)}.
\]
At this point you may be thinking that $P(x)$ says that ``$x$ has
e-mailed \emph{exactly} two students besides possibly
himself.''  However we did not require that $y$ and $z$ be
distinct, or that they be different from $x$.  So our formula 
describes all possibilities: 
\begin{itemize}
\item $x$ and exactly 2 other students: $x\neq y$, $x\neq z$, $y \neq z$.
\item $x$ and exactly 1 other student: $x\neq y$, $x\neq z$, $y=z$.
\item $x$ and no other student: $x=y=z$.
\end{itemize}

Overall the full formula is:
\begin{equation}\label{EEEAs}
\exists x.\, \exists y. \, \exists z.\, \forall s. 
\paren{E(x,s) \QIMPLIES (s=x \QOR s=y \QOR s=z)}.
\end{equation} 

%\qed

\iffalse
An alternate approach to defining the desired formula is arguably more
straightforward than~\eqref{EEEAs}, but also more long-winded.  Saying
that "student $x$ has emailed at most two people besides himself" is
the same as saying that "$x$ has \emph{not} emailed three different,
other people.

Let's give these other people names $a, b$ and $c$.  Saying that $a,
b$ and $c$ are "three different, other people" just means that $a,b,c$
and $x$ are all different.  Saying that "$x$ \emph{has} emailed each
of these people translates directly into a formula $R(x,a,b,c)$ of the
required kind:
\begin{align*}
R(x,a,b,c) & \eqdef
      E(x,a) \QAND E(x,b) \QAND E(x,c) \QAND\\
     & (a \neq b) \QAND (a \neq c) \QAND(b \neq c) \QAND\\
     & (x \neq a) \QAND (x \neq b) \QAND(x \neq c).
\end{align*}
So the final formula we have been looking for is
\begin{equation}\label{ENEEE}
\exists x.\, \QNOT(\exists a,b,c.\, R(x,a,b,c)).
\end{equation}

The formulas~\eqref{EEEAs} and~\eqref{ENEEE} are different ways of
expressing the same fact about student email recipients---they are
equivalent formulas.  The formula~\eqref{EEEAs} is arguably a little
easier to understand than~\eqref{ENEEE}, but it's longer.  Moreover,
this difference in length becomes more dramatic if we consider
generalizing from two students to $n$ students as in Problem~\bref{PS_emailed_at_most_n_others}

\begin{quote}
There is a student who has e-mailed at most $n$ other people in the class,
besides possibly himself.
\end{quote}
Now the approach of~\eqref{ENEEE} leads to a formula with $n+1$
variables all of which are supposed to designate different students.
We say they are all different by including about $n^2/2$ ``not-equal''
formulas between variables.  On the other hand, the approach
of~\eqref{EEEAs} leads to a formula of overall size proportional to
$n$.
\fi

\end{solution}

\end{problem}

%%%%%%%%%%%%%%%%%%%%%%%%%%%%%%%%%%%%%%%%%%%%%%%%%%%%%%%%%%%%%%%%%%%%%
% Problem ends here
%%%%%%%%%%%%%%%%%%%%%%%%%%%%%%%%%%%%%%%%%%%%%%%%%%%%%%%%%%%%%%%%%%%%%

\endinput
