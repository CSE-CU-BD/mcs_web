\documentclass[problem]{mcs}

\begin{pcomments}
  \pcomment{DG_validity_by_cases}
  \pcomment{variant of 2003 final problem 1(a)}
\end{pcomments}

\pkeywords{
  validity
  cases
}

%%%%%%%%%%%%%%%%%%%%%%%%%%%%%%%%%%%%%%%%%%%%%%%%%%%%%%%%%%%%%%%%%%%%%
% Problem starts here
%%%%%%%%%%%%%%%%%%%%%%%%%%%%%%%%%%%%%%%%%%%%%%%%%%%%%%%%%%%%%%%%%%%%%

\begin{problem}%\label{generprob}
Here is a valid formula

\[
(X \AND \neg(\overline{X} \QIMPLIES Y) \AND W ) \QIMPLIES (Z \AND U
\AND V \AND R \AND S \AND T \AND A \AND B \AND C \QAND D)
\]

\begin{problemparts}

\problempart Explain why verifying the validity of this formula
\emph{by truth table} would be very hard for one person to do on paper
(no computers).

\begin{solution}
The number of entries in a truth table here would be $2^{14}$ or about
16,000 since there are 14 variables.  This could take weeks for one
person to do by hand.
\end{solution}

\problempart Verify the formula by cases according to the truth value
of $X$.  Briefly explain your reasoning in each case.

\begin{solution}

\begin{proof}
The proof is by cases on the value of X.
    \begin{enumerate}
        \item[X] is \True: Since (\False implies \True) is \True, the
          expression $\\QNOT(\overline{X} \QIMPLIES Y)$ evaluates to
          \False.  Therefore the left side of the implication is
          \False, and the formula is \True\ in all assignments with
          $X$ assigned \True.

        \item[X] is \False: The left side of the formula is trivially
          \False, and the formula is \True\ in all assignments with
          $X$ assigned \False.

    \end{enumerate}
    Since $X$ must be \True\ or \False\ in any truth assignment, the
    formula is \True\ in any case.  That is, it is valid.
\end{proof}

\end{problemparts}
\end{solution}

\end{problem}

\%%%%%%%%%%%%%%%%%%%%%%%%%%%%%%%%%%%%%%%%%%%%%%%%%%%%%%%%%%%%%%%%%%%%%
% Problem ends here
%%%%%%%%%%%%%%%%%%%%%%%%%%%%%%%%%%%%%%%%%%%%%%%%%%%%%%%%%%%%%%%%%%%%%

\endinput
