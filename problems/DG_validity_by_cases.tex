\documentclass[problem]{mcs}

\begin{pcomments}
  \pcomment{variant of 2003 final problem 1 a)}
\end{pcomments}

\pkeywords{
  validity
  cases
}

%%%%%%%%%%%%%%%%%%%%%%%%%%%%%%%%%%%%%%%%%%%%%%%%%%%%%%%%%%%%%%%%%%%%%
% Problem starts here
%%%%%%%%%%%%%%%%%%%%%%%%%%%%%%%%%%%%%%%%%%%%%%%%%%%%%%%%%%%%%%%%%%%%%

\begin{problem}\label{generprob}
Here is a valid formula
$(X \AND \neg(\overline{X} \rightarrow Y) \AND W ) \rightarrow (Z \AND W \AND U \AND V \AND R \AND S \AND T \AND A \AND B \AND  C)$

\begin{problemparts}
\problempart Explain why verifying the validity of this formula BY TRUTH TABLE could not be done with reasonable effort.

\begin{solution}
The number of entries in a truth table is on the order of $2^n$ where n is the number of variables in the expression. Writing out every value would take far too long to be useful.
\end{solution}

\problempart Verify the formula by cases according to the truth value of X.
Briefly explain your reasoning in each case

\begin{solution}

\begin{proof}
The proof is by cases on the value of x.
    \begin{enumerate}
        \item[X] is True: Since (False implies True) is True, the expression $\neg(\overline{X} \rightarrow Y)$ evaluates to False. Therefore the left side of the implication is False, and the formula is valid in this case.
        \\
        \item[X] is False: The left side of the formula is trivially False, and the formula is again valid in this case.
    \end{enumerate}
    Since X must be True or False, these cases are exhaustive and the formula is proven valid.
\end{proof}

\end{problemparts}
\end{solution}

\end{problem}

%%%%%%%%%%%%%%%%%%%%%%%%%%%%%%%%%%%%%%%%%%%%%%%%%%%%%%%%%%%%%%%%%%%%%
% Problem ends here
%%%%%%%%%%%%%%%%%%%%%%%%%%%%%%%%%%%%%%%%%%%%%%%%%%%%%%%%%%%%%%%%%%%%%

\endinput
