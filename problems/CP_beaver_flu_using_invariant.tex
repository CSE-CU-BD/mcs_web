\documentclass[problem]{mcs}

\begin{pcomments}
  \pcomment{CP_beaver_flu_using_invariant}
  \pcomment{revised from CP_beaver_flu by ARM 2/18/11}
  \pcomment{from: S09.cp5r; F06.ps2}
\end{pcomments}

\pkeywords{
  state_machines
  unreachable_states
  increasing_decreasing_variables
}

%%%%%%%%%%%%%%%%%%%%%%%%%%%%%%%%%%%%%%%%%%%%%%%%%%%%%%%%%%%%%%%%%%%%%
% Problem starts here
%%%%%%%%%%%%%%%%%%%%%%%%%%%%%%%%%%%%%%%%%%%%%%%%%%%%%%%%%%%%%%%%%%%%%

\begin{problem}
  A classroom is designed so students sit in a square arrangement.  An
  outbreak of beaver flu sometimes infects students in the class;
  beaver flu is a rare variant of bird flu that lasts forever, with
  symptoms including a yearning for more quizzes and the thrill of
  late night problem set sessions.

  Here is an illustration of a $6 \times 6$-seat classroom with seats
  represented by squares.  The locations of infected students are
  marked with an asterisk.

\[
\begin{array}{|c|c|c|c|c|c|}
\hline
\ast& & & &\ast& \\ \hline
 &\ast& & & & \\ \hline
& &\ast&\ast& & \\ \hline
& & & & & \\ \hline
& &\ast& & & \\ \hline
& & &\ast& &\ast \\ \hline
\end{array}
\]

Outbreaks of infection spread rapidly step by step.  A student is infected
after a step if either

\begin{itemize}
\item the student was infected at the previous step (since beaver flu
  lasts forever), or

\item the student was adjacent to \emph{at least two} already-infected
  students at the previous step.

\end{itemize}
Here \emph{adjacent} means the students' individual squares share an
edge (front, back, left or right); they are not adjacent if they only
share a corner point.  So each student is adjacent to 2, 3 or 4
others.

In the example, the infection spreads as shown below.
%
\[
\begin{array}{|c|c|c|c|c|c|}
\hline
\ast& & & &\ast& \\ \hline
 &\ast& & & & \\ \hline
& &\ast&\ast& & \\ \hline
& & & & & \\ \hline
& &\ast& & & \\ \hline
& & &\ast& &\ast \\ \hline
\end{array}
\Rightarrow
\begin{array}{|c|c|c|c|c|c|}
\hline
\ast&\ast& & &\ast& \\ \hline
\ast&\ast&\ast& & & \\ \hline
&\ast&\ast&\ast& & \\ \hline
& &\ast& & & \\ \hline
& &\ast&\ast& & \\ \hline
& &\ast&\ast&\ast&\ast \\ \hline
\end{array}
\Rightarrow
\begin{array}{|c|c|c|c|c|c|}
\hline
\ast&\ast&\ast& &\ast& \\ \hline
\ast&\ast&\ast&\ast& & \\ \hline
\ast&\ast&\ast&\ast& & \\ \hline
&\ast&\ast&\ast& & \\ \hline
& &\ast&\ast&\ast& \\ \hline
& &\ast&\ast&\ast&\ast \\ \hline
\end{array}
\]
%
In this example, over the next few time-steps, all the students in class
become infected.

\begin{theorem*}
  If fewer than $n$ students among those in an $n \times n$ arrangment are
  initially infected in a flu outbreak, then there will be at least one
  student who never gets infected in this outbreak, even if students
  attend all the lectures.
\end{theorem*}

Prove this theorem.

\hint Think of the state of an outbreak as an $n \times n$ square
above, with asterisks indicating infection.  The rules for the spread
of infection then define the transitions of a state machine.  Show
that
\begin{align*}
R(q) \eqdef & \text{The ``perimeter'' of the ``infected region''}\\
            & \text{of state $q$ is at most $k$},
\end{align*}
is a preserved invariant.

\begin{solution}
\begin{proof}
  Define the \term{perimeter} of an infected set of students to be the
  number of edges with infection on exactly one side.  Let $\nu$ be the
  perimeter.

  We claim that $\nu$ never gets bigger.  This follows because the
  perimeter changes after a transition only because some squares
  became newly infected.  By the rules above, each newly infected
  square is adjacent to at least two previously infected squares.
  Thus, for each newly infected square, at least two edges are removed
  from the perimeter of the infected region, and at most two edges are
  added to the perimeter.  Therefore, the perimeter of the infected
  region cannot increase, so if it is $\le k$ in some state, it
  stays that way.

  Now if an $n \times n$ grid is completely infected, then the perimeter
  of the infected region is $4n$.  Thus, the whole grid can become
  infected only if the perimeter is initially at least $4n$.  Since each
  square has perimeter 4, at least $n$ squares must be infected initially
  for the whole grid to become infected.
\end{proof}

\end{solution}

\end{problem}

%%%%%%%%%%%%%%%%%%%%%%%%%%%%%%%%%%%%%%%%%%%%%%%%%%%%%%%%%%%%%%%%%%%%%
% Problem ends here
%%%%%%%%%%%%%%%%%%%%%%%%%%%%%%%%%%%%%%%%%%%%%%%%%%%%%%%%%%%%%%%%%%%%%

\endinput
