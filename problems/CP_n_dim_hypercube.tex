\documentclass[problem]{mcs}

\begin{pcomments}
  \pcomment{CP_n_dim_hypercube}
  \pcomment{from: S09.cp6t}
\end{pcomments}

\pkeywords{
  graphs
  connectivity
  binary_strings
  spanning_trees
  hypercube
  spanning_tree
}


%%%%%%%%%%%%%%%%%%%%%%%%%%%%%%%%%%%%%%%%%%%%%%%%%%%%%%%%%%%%%%%%%%%%%
% Problem starts here
%%%%%%%%%%%%%%%%%%%%%%%%%%%%%%%%%%%%%%%%%%%%%%%%%%%%%%%%%%%%%%%%%%%%%

\begin{problem}

  The $n$-dimensional \term{hypercube}, $H_n$, is a graph whose vertices are
  the binary strings of length $n$.  Two vertices are adjacent if and only
  if they differ in exactly $1$ bit.  For example, in $H_3$, vertices
  \texttt{111} and \texttt{011} are adjacent because they differ only in
  the first bit, while vertices \texttt{101} and \texttt{011} are not
  adjacent because they differ at both the first and second bits.

\iffalse
The \term{$n$-dimensional hypercube} is a graph, $H_n$, whose vertices are
the length $n$ binary strings, with an edge between two vertices iff they
differ in only one position.  For $n=4$, for example, vertex \texttt{0000}
is adjacent to the length four binary strings with exactly one \texttt{1},
namely, the four strings \texttt{1000}, \texttt{0100}, \texttt{0010},
\texttt{0001}.
\fi

\bparts

\ppart Prove that it is impossible to find two spanning trees of $H_3$
that do not share some edge.

\begin{solution}
$H_3$ has 8 vertices so every spanning tree has 7 edges.  But $H_3$ has
only 12 edges, so any two sets of 7 edges must overlap.

\end{solution}

\ppart Verify that for any two vertices $x \neq y$ of $H_3$, there are $3$
paths from $x$ to $y$ in $H_3$, such that, besides $x$ and $y$, no two of
those paths have a vertex in common.

\begin{solution}
Define the distance between two binary strings of length $n$ to be the
number of positions at which they differ (this is known as the
\term{Hamming distance} between the strings).

To show that there are 3 paths between any two distance 1 strings, we can,
by symmetry, just consider paths between the vertices \texttt{000} and
\texttt{001}.

Paths from \texttt{000} to \texttt{001}:
\begin{align*}
\mtt{000, 001}\\
\mtt{000, 010, 011, 001}\\
\mtt{000, 100, 101, 001}
\end{align*}

Likewise for distance 2, it is enough to find paths between
\texttt{000} and \texttt{011}:

\begin{align*}
\mtt{000, 010, 011}\\
\mtt{000, 001, 011}\\
\mtt{000, 100, 110, 111, 011}
\end{align*}

Finally, for distance 3 from \texttt{000} to \texttt{111}:
\begin{align*}
\mtt{000, 001, 011, 111}\\
\mtt{000, 010, 110, 111}\\
\mtt{000, 100, 101, 111}
\end{align*}
\end{solution}

\ppart Conclude that the connectivity of $H_3$ is 3.

\begin{solution}
Since there are three paths from $x$ to $y$ in $H_3$ that share
  no edges with one another, removing any two edges will leave one of
  these paths intact, so $x$ and $y$ remain connected.  So removing two
  edges from $H_3$ does not disconnect it.

  On the other hand, removing all $3$ edges incident to any vertex,
  disconnects that vertex.  Thus the minimum number of edges necessary
  to disconnect $H_3$ is $3$.
\end{solution}

\ppart Try extending your reasoning to $H_4$.  (In fact, the
connectivity of $H_n$ is $n$ for all $n \geq 1$.  A proof appears in
the problem solution.)

\begin{solution}
Two walks in a graph are said to \emph{cross} when they have a vertex
in common other than their endpoints.  A set of walks in a graph
\emph{don't cross} when no two walks in the set cross.  A graph is
\emph{$k$-routed} if between every pair of distinct vertices in the
graph there is a set of $k$ paths that don't cross.

We'll show that
\begin{lemma*}
\[
H_n \text{ is $n$-routed for all } n \geq 1.
\]
\end{lemma*}

Since $H_n$ can be disconnected by deleting the $n$ edges incident to any
vertex, this implies that $H_n$ has connectivity $n$.

\begin{proof}
The proof is by induction on $n$ with induction hypothesis,
\[
P(n) \eqdef H_n \text{ is $n$-routed}.
\]

\textbf{Base case} [$n=1$]: Since $H_1$ consists of two
vertices connected by an edge, $P(1)$ is immediate.

\textbf{Base case} [$n=2$]: $H_2$ is a square.  Vertices on opposite
corners are connected by two length 2 paths that don't cross, and
adjacent vertices are connected by a length 1 path and a length 3 path.

\textbf{Inductive step:} We prove $P(n+1)$ for $n \geq 2$ by letting $v$
and $w$ be two vertices of $H_{n+1}$ and describing $n+1$ paths between them
that don't cross.

Let $R$ by any positive length path in $H_n$, say
\[
R = r_0,r_1,\dots, r_k.
\]
For $b \in \set{\mtt{0,1}}$ define the $H_{n+1}$ path
\[
bR \eqdef br_0, br_1,  \dots, br_k.
\]

\textbf{Case 1:} The distance from $v$ to $w$ is $d \leq n$.  In this
case, the $(n+1)$-bit strings $v$ and $w$ agree in one or more positions.
By symmetry, we can assume without loss of generality that $v$ and $w$
both start with \texttt{0}.  That is $v=\mtt{0}v'$ and $w=\mtt{0}w'$ for
some $n$-bit strings $v', w'$.  Now by induction, there are paths, $Q_i$
for $1 \leq i \leq n$, that don't cross going between $v'$ and $w'$ in
$H_n$.

Define the first $n$ paths in $H_{n+1}$ between $v$ and $w$ to be
\[
\pi_i \eqdef \mtt{0}Q_i
\]
for $1 \leq i \leq n$.  These paths don't cross since the $Q_i$'s don't
cross.

Then define the $n+1$st path
\[
\pi_{n+1} \eqdef v,\ \mtt{1}\pi_{v',w'},\ w
\]
where $\pi_{v',w'}$ is any path from $v'$ to $w'$ in $H_n$.
Then $\pi_{n+1}$ does not cross any of the other paths since
$\mtt{1}\pi_{v',w'}$ is vertex disjoint from $\mtt{0}Q_i$ for $1 \leq i \leq n$.

This proves that $P(n+1)$ holds in this case.

\textbf{Case 2:} The distance from $v$ to $w$ is $n+1$.  By symmetry, we
can assume without loss of generality that $v=\mtt{0}^{n+1}$ and
$w=\mtt{1}^{n+1}$.

Now by induction, there are $n$ paths from $\mtt{0}^n$ to $\mtt{1}^n$
in $H_n$ that don't cross in $H_n$.
\iffalse We can assume wlog\footnote{\emph{without loss of
    generality}} that each of these walks is a path.  \fi
Removing the shared first vertex, $\mtt{0}^n$, of these paths yields paths
$R_1,R_2,\dots,R_n$.  Now the $R_i$'s are vertex disjoint except for their
common endpoint, $1^n$.  Let $s_i$ be the start vertex of $R_i$ for $1
\leq i \leq n$.

We now define $n+1$ paths in $H_{n+1}$ from $\mtt{0}^{n+1}$ to
$\mtt{1}^{n+1}$ that don't cross.

The first of these paths will be
\[
\pi_1 \eqdef \mtt{0}^{n+1}, \mtt{1}\mtt{0}^{n}, 1R_1.
\]

For $2 \leq i \leq n$, the $i$th of these paths will be
\[
\pi_i \eqdef \mtt{0}^{n+1}, \mtt{0}s_i, 1R_i.
\]
These paths don't cross because

\begin{itemize}

\item the paths $1R_i$ for $1 \leq i \leq n$ are vertex disjoint except
  for their common endpoint, $\mtt{1}^{n+1}$, because the $R_i$'s are
  vertex disjoint except for their common endpoint, $\mtt{1}^n$,

\item a vertex $0s_i$ does not appear on $\pi_j$ for any for $j \neq i$
  because the $s_i \neq s_j$ for $j \neq i$, and the other vertices on
  the $\pi_j$'s start with \texttt{1},

\item the vertex $\mtt{1}\mtt{0}^{n}$ appears only on $\pi_1$.  This follows
  because if it appeared on $\pi_i$ for $i \neq 1$ it must appear on $1R_i$.
  That would imply that $\mtt{0}^{n}$ appears on $R_i$, contradicting the
  fact that the original path $\mtt{0}^{n}, R_i$ in $H_n$ is simple.

\end{itemize}

Finally, the $n+1$st path will be
\[
\pi_{n+1} \eqdef \mtt{0}^{n+1},  0R_1, \mtt{1}^{n+1}.
\]
Note that, since all but the final vertex on $\pi_{n+1}$ start with
\texttt{0}, the only vertices besides the endpoints that $\pi_{n+1}$ could
share with another path would be $\mtt{0}s_i$ for $2 \leq i \leq n$.  But
none of these appear on $\pi_{n+1}$ because, except for their shared
endpoint, $R_1$ is vertex disjoint from all the other $R_i$'s.

This proves that $P(n+1)$ holds in case~2, and therefore holds in all
cases, which completes the proof by induction.
\end{proof}

This proof implicitly defines a recursive procedure that, for
any two vertices in $H_n$, finds between the two vertices $n$ paths
of length at most $n+1$ that don't cross.

The proof rests on the fact that $H_{n+1} = \gprod{H_1}{H_n}$ where
$\gprod{}{}$ is a Cartesian product of graphs.
\begin{definition*}
The \term{Cartesian product} $\gprod{G}{H}$ of simple graphs $G$ and
$H$ has
\begin{align*}
\vertices{\gprod{G}{H}} & \eqdef  \vertices{G} \cross \vertices{H},\\
\edges{\gprod{G}{H}}
& \eqdef \set{\edge{(g_1,h)}{(g_2,h)} \suchthat \edge{g_1}{g_2} \in \edges{G}}\ \union\\
& \qquad \set{\edge{(g,h_1)}{(g,h_2)} \suchthat \edge{h_1}{h_2} \in \edges{H}}.
\end{align*}
\end{definition*}
The $n$-connectedness of $H_n$ then follows immediately from a known
result of graph theory that is both simpler and more general:
\begin{theorem*}[Cartesian routedness]
If $G$ is $m$-routed and $H$ is $n$-routed, then
$\gprod{G}{H}$ is $(m+n)$-routed.
\end{theorem*}
We expect that the proof above for $H_n$ can be extended to prove
this Cartesian routedness Theorem.

\begin{editingnotes}
$n$-routedness obviously implies $n$-vertex-connectedness.  Menger's
  Theorem, gives the converse: if $G$ is $n$-vertex-connected, then
  $G$ is $n$-routed.

The precise result (reference needed) is that
\[
\conntd{\gprod{G}{H}} = \min\{\conntd{G} \cdot v_H,
                              \conntd{H} \cdot v_G,
                               d_G+d_H\}
\]
where $\conntd{}$ is the vertex connectedness, $v_G =
\card{\vertices{G}}$ and $d_G$ is the minumum degree of the vertices
of $G$, with $v_H$ and $d_H$ are defined correspondingly.
\end{editingnotes}

\end{solution}

\eparts

\end{problem}

%%%%%%%%%%%%%%%%%%%%%%%%%%%%%%%%%%%%%%%%%%%%%%%%%%%%%%%%%%%%%%%%%%%%%
% Problem ends here
%%%%%%%%%%%%%%%%%%%%%%%%%%%%%%%%%%%%%%%%%%%%%%%%%%%%%%%%%%%%%%%%%%%%%

\endinput
