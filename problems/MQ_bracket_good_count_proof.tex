\documentclass[problem]{mcs}

\begin{pcomments}
  \pcomment{MQ_bracket_good_count_proof}
  \pcomment{edited from MQ_bracket_good_count by Wielian Chu Oct 2017}
  \pcomment{part of PS_bracket_good_count}
  \pcomment{similar structure to CP_erasable_strings}
\end{pcomments}

\pkeywords{
  induction
  matching
  recursive_data
  strings
  structural_induction
  count
}

%%%%%%%%%%%%%%%%%%%%%%%%%%%%%%%%%%%%%%%%%%%%%%%%%%%%%%%%%%%%%%%%%%%%%
% Problem starts here
%%%%%%%%%%%%%%%%%%%%%%%%%%%%%%%%%%%%%%%%%%%%%%%%%%%%%%%%%%%%%%%%%%%%%
\begin{problem}
One way to determine if a string has matching brackets, that is, if it is
in the set, \RM\inhandout{\footnote{$\RM$ is defined recursively: \textbf{Base case:}
    $\emptystring \in\RM$. \textbf{Constructor case:} If $s,t \in \RM$,
    then $\lefbrk s\, \rhtbrk t \in \RM.$ }},\inbook{ of Definition~\bref{RM_def}}
 is to start with 0 and read
the string from left to right, adding 1 to the count for each left bracket
and subtracting 1 from the count for each right bracket.  For example,
here are the counts for two sample strings:
\[\begin{array}{rrrrrrrrrrrrr}
& \lefbrk & \rhtbrk & \rhtbrk & \lefbrk & \lefbrk & \lefbrk & \lefbrk &
\lefbrk & \rhtbrk & \rhtbrk & \rhtbrk & \rhtbrk\\
0 & 1 & 0 & -1 & 0 & 1 & 2 & 3 & 4 & 3 & 2 & 1 & 0\\
\\
\\
& \lefbrk & \lefbrk & \lefbrk & \rhtbrk & \rhtbrk & \lefbrk & \rhtbrk &
\rhtbrk & \lefbrk & \rhtbrk\\
0 & 1 & 2 & 3 & 2 & 1 & 2 & 1 & 0 & 1 & 0
\end{array}\]
A string has a \term{good count} if its running count never goes
negative and ends with 0.  So the second string above has a good count, but
the first one does not because its count went negative at the third step.
The empty string $\emptystring$ has a good count of 0.
Let
\[
\GC \eqdef \set{ s \in \brkts \suchthat s\ \text{has a good count}}.
\]

One way to prove that $\RM = \GC$ is to show that each includes the
other.

\bparts

\ppart\label{recstruct} What inductive hypothesis would you use to prove RecMatch
$\subseteq$ GoodCount by structural induction?

\examspace[1.0in]

\begin{solution}
The structural induction hypothesis is simply.
\[
P(s) \eqdef s \in \GC.
\]
\end{solution}

\ppart Use your inductive hypothesis from part~\ref{recstruct} to
prove RecMatch $\subseteq$ GoodCount by structural induction.

\examspace[3.0in]
\begin{solution}

\begin{proof}
The proof is by structural induction on the recursive definition of
$\RM$.

\textbf{Base Case}: $P(\emptystring)$ holds since the count of the
empty string ends where it starts at zero.

\textbf{Inductive Step:} Assume $P(s)$ and $P(t)$ are true.  We need to show
that $P(\lefbrk s\, \rhtbrk t)$ is true.

The count values for $\lefbrk s\, \rhtbrk t$ start with 0.  Reading
the initial left bracket yields 1 as the next count value.  This 1
serves as the start of a series of count values exactly equal to the
count values of $s$, with each value incremented by one.  Since $s \in
\GC$ by induction hypothesis, these incremented count values begin
with 1, always stay positive, and end with 1.  The right bracket
immediately after $s$ reduces the ending count to 0.  This 0 serves as
the start of the remaining count values which are exactly the count
values of $t$.  Since by induction hypothesis $t \in \GC$, these
remaining values never go negative and end at 0.  Hence the entire
sequence of count values for $\lefbrk s\, \rhtbrk t$ starts with 0,
never goes negative, and ends with 0, which proves that $\lefbrk s\,
\rhtbrk t \in \GC$.
\end{proof}

\end{solution}

\ppart The other inclusion, GoodCount $\subseteq$ RecMatch, can be
proved by strong induction. What induction hypothesis would you use
for this? No proof is needed.

\examspace[1.0in]

\begin{solution}
A straightforward strong induction hypothesis that works is
\[
Q(n) \eqdef \forall r \in \GC.\, \lnth{r} = n \QIMPLIES r \in \RM.
\]

The proof was not required, but here it is:

\begin{proof}

\textbf{Base Case} $n = 0$: There is only one string of length 0,
namely the empty string, which is in $\RM$ by definition, proving
$Q(0)$.

\textbf{Inductive Step:} Assume that $Q(k)$ is true for all $k\leq n$, we
need to prove that $Q(n+1)$ is also true.

So suppose $r$ is a length $n+1$ string that counts well.  We must prove
that $r \in \RM$.

Now since $r$ has a good count, it must start with a left bracket (or else
the count would immediately go negative).  Likewise, since the count for
$r$ returns to the value 0 by the end, $r$ must end with right
bracket.  So there must be a \emph{first} right bracket in $r$
after which the count returns to 0.  Let $s$ be the substring of $r$
between the initial left bracket and this right bracket.  That is,
\[
r = \lefbrk s\rhtbrk t
\]
for some string $t$.

Since counts only change by one as each bracket character is read, and
the count for $r$ \emph{first} returns to 0 after the right bracket
following $s$, the count during $s$ must start and end with 1 and must
stay \emph{positive} in between.  But this implies that a count for $s$
alone, which would start with 0, would also end with 0 and stay
\emph{nonnegative} in between.  That is, $s$ by itself has a good count.
Since the length of $s \in \GC$ is less than the length of $r$, we
have by strong induction that $s \in \RM$.

Further, we know the count for $r$ returns to 0 after the right
bracket following $s$, and since $r \in \GC$, the count ends with 0
again and stays nonnegative in between.  But this implies that $t$ has
a good count, and since the length of $t$ is less than the length of
$r$, we have by strong induction that $t \in \RM$.  Now by the second
case in the definition of $\RM$, we conclude $r = \lefbrk s\, \rhtbrk
t \in \RM$.
\end{proof}

\end{solution}

\end{problemparts}
\end{problem}

%%%%%%%%%%%%%%%%%%%%%%%%%%%%%%%%%%%%%%%%%%%%%%%%%%%%%%%%%%%%%%%%%%%%%
% Problem ends here
%%%%%%%%%%%%%%%%%%%%%%%%%%%%%%%%%%%%%%%%%%%%%%%%%%%%%%%%%%%%%%%%%%%%%

\endinput
