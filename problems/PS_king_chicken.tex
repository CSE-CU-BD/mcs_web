\documentclass[problem]{mcs}

\begin{pcomments}
  \pcomment{PS_king_chicken}
  \pcomment{edited by ARM from FTL digraph chapter}
\end{pcomments}

\pkeywords{
  king_chicken
  digraph
  tournament
  outdegree
}

%%%%%%%%%%%%%%%%%%%%%%%%%%%%%%%%%%%%%%%%%%%%%%%%%%%%%%%%%%%%%%%%%%%%%
% Problem starts here
%%%%%%%%%%%%%%%%%%%%%%%%%%%%%%%%%%%%%%%%%%%%%%%%%%%%%%%%%%%%%%%%%%%%%

\begin{problem}
Suppose that there are $n$ chickens in a farmyard.  Chickens are
rather aggressive birds that tend to establish dominance in
relationships by pecking; hence the term ``pecking order.''  In
particular, for each pair of distinct chickens, either the first pecks
the second or the second pecks the first, but not both.  We say that
chicken~$u$ \emph{virtually pecks} chicken~$v$ if either:
\begin{itemize}

\item Chicken $u$ directly pecks chicken~$v$, or

\item Chicken $u$ pecks some other chicken~$w$ who in turn pecks
  chicken~$v$.

\end{itemize}
A chicken that virtually pecks every other chicken is called a
\emph{king chicken}.

We can model this situation with a \emph{chicken digraph} whose
vertices are chickens with an edge from chicken~$u$ to chicken~$v$
precisely when $u$ pecks $v$.  In the graph in Figure~\ref{fig:6EE3},
three of the four chickens are kings.  Chicken~$c$ is not a king in
this example since it does not peck chicken~$b$ and it does not peck
any chicken that pecks chicken~$b$.  Chicken~$a$ \emph{is} a king
since it pecks chicken~$d$, who in turn pecks chickens $b$ and~$c$.

In general, a \emph{\idx{tournament digraph}} is a digraph with exactly one
edge between each pair of distinct vertices.

\begin{figure}[h]

\graphic{Fig_EE3}

\caption{A 4-chicken tournament in which chickens $a$, $b$ and~$d$
  are kings.}

\label{fig:6EE3}.

\end{figure}

\bparts

\ppart Define a 10-chicken tournament graph with a king chicken that has outdegree 1.

\begin{solution}
1 pecks 2 and 2 pecks 3--10 and 3--10 peck 1. The directions of edges amongst 3-10 are irrelevant.
\end{solution}

\ppart Describe a 5-chicken tournament graph in which every player is a king.

\begin{solution}

\iffalse We claim any digraph in which each vertex has two edges in
and two edges out will work.  To see this, consider any chicken $u$ in
this graph.  Then $u$ will directly peck two other chickens, $a$ and
$b$.  WLOG, assume $a$ pecks $b$.  Then $b$ will peck the remaining
chickens $c$ and $d$, since $b$ must have two arrows out.  Thus the
king pecks $a$ and $b$ directly, and $b$ pecks $c$ and $d$, so $u$ is
a king.  This logic can apply to any vertex, and thus every chicken is
a king.  \fi

An example is illustrated in Figure~\ref{fig:6EE4}.

Notice that every vertex has outdegree two.  By the King Chicken
Theorem below, this implies that all vertices are kings.

\begin{figure}

\graphic{Fig_EE4}

\caption{A 5-chicken tournament in which every chicken is a king.}

\label{fig:6EE4}

\end{figure}

\end{solution}

\ppart Prove
\begin{theorem*}[King Chicken Theorem]%\label{thm:king_chicken}
Any chicken with maximum outdegree in a tournament is a king.
\end{theorem*}
\begin{solution}

\begin{proof}
By contradiction.  Let $u$ be a node in a tournament graph $G = (V,
E)$ with maximum outdegree and suppose that $u$ is not a king.  Let $Y
= \set{v \suchthat \diredge{u}{v} \in E}$ be the set of chickens that
chicken~$u$ pecks.  Then $\outdegr{u} = \card{Y}$.

Since $u$ is not a king, there is a chicken $x \notin Y$ (that is, $x$
is not pecked by chicken~$u$) and that is not pecked by any chicken
in~$Y$.  Since for any pair of chickens, one pecks the other, this
means that $x$ pecks~$u$ as well as every chicken in~$Y$.  This means
that
\begin{align*}
    \outdegr{x} = \card{Y} + 1 % \\
                > \outdegr{u}.
\end{align*}

But $u$ was assumed to be the node with the largest degree in the
tournament, so we have a contradiction.  Hence, $u$ must be a king.
\end{proof}

\end{solution}

The King Chicken Theorem means that if the player with the most
victories is defeated by another player~$x$, then at least he/she
defeats some third player that defeats~$x$.  In this sense, the player
with the most victories has some sort of bragging rights over every
other player.  Unfortunately, as Figure~\ref{fig:6EE3} illustrates,
there can be many other players with such bragging rights, even some
with fewer victories.

\eparts  

\end{problem}

%%%%%%%%%%%%%%%%%%%%%%%%%%%%%%%%%%%%%%%%%%%%%%%%%%%%%%%%%%%%%%%%%%%%%
% Problem ends here
%%%%%%%%%%%%%%%%%%%%%%%%%%%%%%%%%%%%%%%%%%%%%%%%%%%%%%%%%%%%%%%%%%%%%
 
