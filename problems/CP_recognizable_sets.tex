\documentclass[problem]{mcs}

\begin{pcomments}
  \pcomment{CP_recognizable_sets}
  \pcomment{subsumed by notes section~\bref{halting_sec} & CP_computable_reducibility}
  \pcomment{from: S06.cp2f, F09(?)}
  \pcomment{revised by ARM 9/20/09 and more 3/16/10}
  \pcomment{needs handy copy of Russell paradox or Theorem~\bref{powbig}}
\end{pcomments}

\pkeywords{
  sets
  set_theory
  diagonal_argument
  Russell_paradox
  recognizable
  halting_problem
}

%%%%%%%%%%%%%%%%%%%%%%%%%%%%%%%%%%%%%%%%%%%%%%%%%%%%%%%%%%%%%%%%%%%%%
% Problem starts here
%%%%%%%%%%%%%%%%%%%%%%%%%%%%%%%%%%%%%%%%%%%%%%%%%%%%%%%%%%%%%%%%%%%%%

\begin{problem}
Let's refer to a programming procedure (written in your favorite
programming language---C++, or Java, or Python, \dots) as a
\term{string procedure} when it is applicable to data of type
\textbf{string} and only returns values of type \textbf{boolean}.
When a string procedure $P$ applied to a \textbf{string} $s$
returns \True, we'll say that $P$ \term{recognizes} $s$.  If
$\mathcal{R}$ is the set of strings that $P$ recognizes, we'll call
$P$ a \term{recognizer} for $\mathcal{R}$.

\bparts

\ppart\label{double_letters} Describe how a recognizer would work for
the set of strings containing only lowercase Roman letters
---\STR{a,b,\dots,z}---such that each letter occurs twice in a
row.  For example, \STR{aaccaabbzz}, is such a string, but
\STR{abb}, \STR{00bb}, \STR{AAbb}, and \STR{a} are not.
(Even better, actually write a recognizer procedure in your favorite
programming language).

\begin{solution}
  All the standard programming languages have built-in operations for
  scanning the characters in a string.  So simply write a procedure
  that checks an input string left to right, verifying that successive
  pairs of characters in the string are duplicated, lowercase roman characters.

%\textbf{ACTUAL PROGRAM TBA}

\end{solution}

\eparts

A set of strings is called \emph{recognizable} if there is a
recognizer procedure for it.  So the program you described above
proves that the set of strings with doubled letters from
part~\eqref{double_letters} is recognizable.

When you actually program a procedure, you have to type the program
text into a computer system.  This means that every procedure is
described by some string of typed characters.  If a string $s$ is
actually the typed description of some string procedure, let's refer
to that procedure as $P_s$.  You can think of $P_s$ as the result of
compiling $s$.\footnote{The string $s$ and the procedure, $P_s$,
  have to be distinguished to avoid a type error: you can't apply a
  string to string.  For example, let $s$ be the string that you wrote
  as your program to answer part~\eqref{double_letters}.  Applying $s$
  to a string argument, say \STR{aabbccdd}, should throw a type
  exception; what you need to do is apply the procedure $P_s$ to
  \STR{aabbccdd}.  This should result in a returned value \True,
  since \STR{aabbccdd} consists of consecutive pairs of lowercase
  roman letters.}

In fact, it will be helpful to associate every string $s$ with a
procedure, $P_s$.  So if string $s$ is not the typed description of a
string procedure, we will define $P_s$ to be some fixed string
procedure---say one that always returns \False; so if $s$ is an
ill-formed string, $P_s$ will be a recognizer for the empty set of
strings.

The result of this is that we have now defined a total function $f$
mapping every string $s$ to the set, $f(s)$, of
strings recognized by $P_s$.  That is we have a total
function,
\begin{equation}\label{f:s-to-Ps}
f: \textbf{string} \to \power(\textbf{string}).
\end{equation}

\bparts

\ppart Explain why $\range{f}$ is the set of all recognizable sets of
strings.

\begin{solution}
Since $f(s)$ is the set of strings recognized by $P_s$, everything in
$\range{f}$ is a recognizable set.  Conversely, every recognizable set
is in $\range{f}$: if $\mathcal{R}$ is a recognizable set, then by
definition, there is a procedure that recognizes $R$.  Let $r$ be the
typed input from which such a recognizer was compiled, that is,
suppose $P_r$ is the recognizer for $R$.  This means $\mathcal{R} =
f(r) \in \range{f}$.
\end{solution}

\eparts

This is exactly the set up we need to apply the reasoning behind Russell's
Paradox to define a set that is not in the range of $f$, that is, a set of
strings, $\mathcal{N}$, that is \emph{not} recognizable.

\bparts

\ppart\label{Ps-to-s}
Let
\[
\mathcal{N} \eqdef \set{s \in \textbf{string} \suchthat s \notin f(s)}.
\]

Prove that $\mathcal{N}$ is not recognizable.

\hint Similar to Russell's paradox or the proof of Theorem~\bref{powbig}.

\begin{staffnotes}
Refer students to the Russell paradox slide or the text for Theorem~\bref{powbig}.
\end{staffnotes}

\begin{solution}
By definition of $\mathcal{N}$,
\begin{equation}\label{Psnf}
s \in \mathcal{N} \qiff s \notin f(s).
\end{equation}
for every \textbf{string} $s$.

Now assume to the contrary that $\mathcal{N}$ was recognizable by some
string procedure.  This procedure must have a string $w$ that
describes it, so we have
\begin{align}
s \in \mathcal{N} & \qiff P_w \text{ applied to $s$ returns \True},\notag\\
                  & \qiff s \in f(w) \label{Psfw} & \text{(by def.\ of $f$)}
\end{align}
for all \textbf{string}'s $s$.

Combining~\eqref{Psnf} and ~\eqref{Psfw}, we have that for every string $s$
\begin{equation}\label{snfsf}
s \notin f(s) \qiff s \in f(w),
\end{equation}
for all \textbf{string}s $s$.

Now letting $s$ be $w$ in~\eqref{snfsf}, we reach the contradiction
\[
w \notin f(w) \qiff w \in f(w).
\]
This contradiction implies that the assumption that $\mathcal{N}$ was
recognizable must be false.

\end{solution}

\ppart Discuss what the conclusion of part~\eqref{Ps-to-s} implies about
the possibility of writing ``program analyzers'' that take programs
as inputs and analyze their behavior.

\begin{solution}
  Let's call a programming procedure ``unconscious'' if it does not
  return \True\ when applied to its own textual definition.

  Rephrased informally, the conclusion of part~\eqref{Ps-to-s} says
  that it is logically impossible to design a \emph{general} program
  analyzer, which takes as input the text of an arbitrary program, and
  recognizes when the program is unconscious.  This implies that it is
  impossible to write a program which does the more general analysis
  of how an arbitrary procedure behaves when applied to arbitrary
  arguments.

  By the way, it \emph{is} feasible to write a general procedure that
  recognizes the texts of ``conscious'' procedures, namely texts for
  procedures that \emph{do} return \True\ when applied to their own
  descriptions.  The ``consciousness'' recognizer simply takes an
  input text $s$ and then simulates $P_s$ applied to $s$.  In other
  words, this general procedure just acts like a virtual machine
  simulator or ``interpreter'' for the programming language of its
  input programs.  Note that this naive procedure runs forever on all
  input programs that run forever!

  It's also important to recognize that there's no hope of getting around
  this by switching programming languages.  For example, by
  part~\eqref{Ps-to-s}, no C++ program can analyze arbitrary C++ programs,
  and no Java program can analyze Java programs, but you might wonder if a
  language like C++, which allows more intimate manipulation of computer
  memory than Java, might therefore allow a C++ program to analyze general
  Java programs.  But there is no loophole here: since it's possible to
  write a Java program that is a simulator/interpreter for C++ programs,
  if a C++ program could analyze Java programs, so could the Java program
  that simulated the C++ program, contradicting~\eqref{Ps-to-s}.
\end{solution}

\eparts

\end{problem}

\endinput
