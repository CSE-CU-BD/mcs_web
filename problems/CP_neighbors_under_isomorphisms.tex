\documentclass[problem]{mcs}

\begin{pcomments}
  \pcomment{from: S09.cp6m}
\end{pcomments}

\pkeywords{
  isomorphisms
  degree
}

%%%%%%%%%%%%%%%%%%%%%%%%%%%%%%%%%%%%%%%%%%%%%%%%%%%%%%%%%%%%%%%%%%%%%
% Problem starts here
%%%%%%%%%%%%%%%%%%%%%%%%%%%%%%%%%%%%%%%%%%%%%%%%%%%%%%%%%%%%%%%%%%%%%

\begin{problem}
\bparts

\ppart\label{fhat} For any vertex, $v$, in a graph, let $N(v)$ be
the set of \term{neighbors} of $v$, namely, the vertices adjacent to $v$:
\[
N(v) \eqdef \set{u \suchthat \edge{u}{v}\text{ is an edge of the
graph}}.
\]

Suppose $f$ is an isomorphism from graph $G$ to graph $H$.
Prove that $f(N(v)) = N(f(v))$.

Your proof should follow by simple reasoning using the definitions of
isomorphism and neighbors ---no pictures or handwaving.

\hint Prove by a chain of iff's that
\[
h \in N(f(v)) \qiff h \in f(N(v))
\]
for every $h \in V_H$.  Use the fact that $h=f(u)$ for some $u \in V_G$.

\begin{solution}
\begin{proof}
Suppose $h \in V_H$.  By definition of isomorphism, there is a
  unique $u \in V_G$ such that $f(u)=h$.  Then
\begin{align*}
h \in N(f(v))
    & \qiff \edge{h}{f(v)} \in E_H          & \text{(def of $N$)}\\
    & \qiff \edge{f(u)}{f(v)} \in E_H       & \text{(def of $u$)}\\
    & \qiff \edge{u}{v} \in E_V             & \text{(since $f$ is an isomorphism)}\\
    & \qiff u \in N(v)                      & \text{(def of $N$)}\\
    & \qiff f(u) \in f(N(v))                & \text{(def of $f$-image)}\\
    & \qiff h \in f(N(v))                   & \text{(def of $u$)}
\end{align*}
So $N(f(v))$ and $f(N(v))$ have the same members and therefore are equal.

\end{proof}
\end{solution}

\ppart Conclude that if $G$ and $H$ are isomorphic graphs, then for each
$k \in \naturals$, they have the same number of degree $k$ vertices.

\begin{solution}
By definition, $\degr{v}= \card{N(v)}$.  Since an isomorphism is
  a bijection, any set of vertices and its image under an isomorphism will
  be the same size (by the Mapping Rule from Week 2 Notes), so the Lemma
  of part~\eqref{fhat} implies that an isomorphism, $f$, maps degree $k$
  vertices to degree $k$ vertices.  This means that the image under $f$ of
  the set of degree $k$ vertices of $G$ is precisely the set of degree $k$
  vertices of $H$.  So by the Mapping Rule again, there are the same
  number of degree $k$ vertices in $G$ and $H$.
\end{solution}

\eparts

\end{problem}

%%%%%%%%%%%%%%%%%%%%%%%%%%%%%%%%%%%%%%%%%%%%%%%%%%%%%%%%%%%%%%%%%%%%%
% Problem ends here
%%%%%%%%%%%%%%%%%%%%%%%%%%%%%%%%%%%%%%%%%%%%%%%%%%%%%%%%%%%%%%%%%%%%%
