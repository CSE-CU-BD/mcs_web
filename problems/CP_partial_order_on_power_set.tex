\documentclass[problem]{mcs}

\begin{pcomments}
  \pcomment{CP_partial_order_on_power_set}
  \pcomment{from: F07.ps2}
\end{pcomments}

\pkeywords{
  partial_order
  power_set
  antichain
  maximal_and_minimal_elements
}

%%%%%%%%%%%%%%%%%%%%%%%%%%%%%%%%%%%%%%%%%%%%%%%%%%%%%%%%%%%%%%%%%%%%%
% Problem starts here
%%%%%%%%%%%%%%%%%%%%%%%%%%%%%%%%%%%%%%%%%%%%%%%%%%%%%%%%%%%%%%%%%%%%%

\begin{problem}
Consider the proper subset partial order, $\subset$, on the power set
$\power{\set{1,2,\dots,6}}$.

\begin{problemparts}
\problempart 
What is the size of a maximal chain in this partial order?
Describe one.

\begin{solution}
Size 7, for example,
\[
\set{\emptyset, \set{1}, \set{1,2},
  \set{1,2,3},\set{1,2,3,4},\set{1,2,3,4,5}, \set{1,2,3,4,5,6}}.
\]
\end{solution}


\problempart

\iffalse An \emph{antichain} in a partial order is a set
of elements such that any two elements in the set are incomparable.\fi

Describe the largest antichain you can find in this partial order.

\begin{solution}
All the size 3 subsets of $\set{1,2,\dots, 6}$ form an antichain
of size 20.  These are actually the largest, though proving this can be
a challenge, especially trying to generalize to the power set of an $n$
element set.
\end{solution}


\problempart
What are the maximal and minimal elements?  Are they maximum and
minimum?

\begin{solution}
$\emptyset$ is minimum and $\set{1,2,\dots,6}$ is maximum.
\end{solution}


\problempart 
Answer the previous part for the $\subset$ partial order on the set
$\power{\set{1,2,\dots, 6}} - \emptyset$.

\begin{solution}
Now the six size 1 subsets are minimal and there is no minimum.
$\set{1,2,\dots,6}$ is still maximum.
\end{solution}

\end{problemparts}
\end{problem}


%%%%%%%%%%%%%%%%%%%%%%%%%%%%%%%%%%%%%%%%%%%%%%%%%%%%%%%%%%%%%%%%%%%%%
% Problem ends here
%%%%%%%%%%%%%%%%%%%%%%%%%%%%%%%%%%%%%%%%%%%%%%%%%%%%%%%%%%%%%%%%%%%%%

\endinput
