\documentclass[problem]{mcs}

\begin{pcomments}
  \pcomment{FP_graph_width_one}
  \pcomment{Related to FP_graph_colorable}
  \pcomment{ARM 4/11/14; slightly edited 5/17/15}
\end{pcomments}

\pkeywords{
 graph theory
 colorable
 width
}

\begin{problem}
A simple graph, $G$, is said to have \emph{width} $1$ iff there is a
way to list all its vertices so that each vertex is adjacent to at
most one vertex that appears earlier in the list.  All the graphs
mentioned below are assumed to be finite.

\bparts

\ppart\label{w1forest} Prove that every graph with width one is a forest.

\hint By induction, removing the last vertex.

\examspace[3in]

\begin{solution}

\begin{proof}
By induction on the number of vertices, $n$.  The induction hypothesis
is
\begin{center}
$P(n) \eqdef$ all $n$-vertex graphs $G$ with width one are forests.
\end{center}

\inductioncase{Base case}: ($n=1$).  A graph with one vertex is
acyclic and therefore is a forest.

\inductioncase{Induction step}.  Assume that $P(n)$ is true for some
$n \geq 1$ and let $G$ be an $(n+1)$-vertex graph with width one.  We
need only show that $G$ is acyclic.

The vertices of $G$ can be listed with each vertex adjacent to at most
one vertex earlier in the list.  Let $v$ be the last vertex in the
list.  Since \emph{all} the vertices adjacent to $v$ appear earlier in
the list, it follows that $\degr{v} \leq 1$.

Now removing a vertex won't increase width, so $G-v$ still has width
one, so it is acyclic by Induction Hypothesis.  But no degree-one
vertex is in a cycle, so adding $v$ back to $G-v$ will not create a
cycle.  Hence $G$ is acyclic, as claimed.

This proves $P(n+1)$ and completes the induction step.

\end{proof}

\end{solution}

\ppart Prove that every finite tree has width one.  Conclude that a
graph is a forest iff it has width one.

\begin{solution}
By induction on the number of vertices, $n$.  The induction hypothesis
is
\begin{center}
$Q(n) \eqdef$ all $n$-vertex trees $T$ have width one.
\end{center}

\inductioncase{Base case}: ($n=1$).  Trivial.

\inductioncase{Induction step}.  Assume that that $Q(n)$ is true for
some $n \geq 1$ and let $T$ be an $(n+1)$-vertex tree.  We need only
show that $T$ has width one.

By Theorem~\bref{th:treeprops}, every tree with at least two vertices has a leaf.  Let $v$
be a leaf of $T$.  Then $T-v$ has width one by Induction Hypothesis,
so its vertices can be listed with each vertex adjacent to at most one
vertex earlier in the list.  Since $v$ has degree one, we can add it
to the end of the list of vertices for $T-v$ to obtain the required
list for $T$.  Hence $T$ has width one, as claimed.

This proves $Q(n+1)$ and completes the induction step.

Note that if all the connected components of a graph have width $1$,
then so does the whole graph: just append the lists of vertices for
each successive component.  In particular, since every tree has width
one, so does every forest.  Therefore by part~\eqref{w1forest}, a
graph is a forest iff it has width one.
\end{solution}

\eparts

\end{problem}
\endinput
