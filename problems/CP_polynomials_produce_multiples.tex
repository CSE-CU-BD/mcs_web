\documentclass[problem]{mcs}

\begin{pcomments}
  \pcomment{CP_polynomials_produce_multiples}
  \pcomment{from: F07 Rec8Th P5}
\end{pcomments}

\pkeywords{
}

%%%%%%%%%%%%%%%%%%%%%%%%%%%%%%%%%%%%%%%%%%%%%%%%%%%%%%%%%%%%%%%%%%%%%
% Problem starts here
%%%%%%%%%%%%%%%%%%%%%%%%%%%%%%%%%%%%%%%%%%%%%%%%%%%%%%%%%%%%%%%%%%%%%


\begin{problem}
  You may remember from Week 1 Notes the polynomial $p(n) \eqdef n^2 + n +
  41$ whose values on nonnegative integers were all prime until $p(40)$.
  Well, $p$ didn't work, but are there any other polynomials whose values
  are always prime?  No way!  In fact, we'll prove a much stronger claim:

  Suppose $q$ is a polynomial with integer coefficients whose domain is
  restricted to be the nonnegative integers.  We'll say that $q$
  \emph{produces multiples} if, for every nonzero value in the range of $q$,
  there are infinitely many multiples of that value also in the range.

  For example, if $q$ produces multiples and $q(4) = 7$, then there are
  infinitely many different multiples of 7 in the range of $q$, and of
  course, except for 7 itself, none of these multiples is prime.

  We claim that if $q$ is not a constant function, then $q$ produces
  multiples.

  \bparts

  \ppart\label{jk} Prove that if $j \equiv k \pmod n$, then $q(j) \equiv
  q(k) \pmod n$.

  \hint The set, $A$, of polynomial functions with integer coefficients can be
  defined recursively:
\begin{itemize}
\item \textbf{Base cases}:
  \begin{itemize}

   \item the identity function, $i(x) \eqdef x$ is in $A$.

   \item for any integer, $k$, the constant function, $c(x) \eqdef k$ is in $A$.
  \end{itemize}

\item \textbf{Constructor cases}.  If $r,s \in A$, then $r+s$ and $r \cdot
  s \in A$.
\end{itemize}

\begin{solution}

The proof is by structural induction on the definition of $A$.  The
hypothesis $P(q)$ is that
\[
\text{for all } k,n \in \naturals,\text{ if } j \equiv k \pmod n, \text{
then } q(j) \equiv q(k) \pmod n.
\]

\begin{itemize}
\item \textbf{Base cases}: $P(i)$ and $P(c)$ both hold trivially.

\item \textbf{Constructor cases}.  Suppose $P(r)$ and $P(t)$ hold, and let
  $t \eqdef r+s$.  To show $P(t)$, suppose $j \equiv k \pmod n$.  Since
  $P(r)$ holds, we have that $r(j) \equiv r(k) \pmod n$.  Likewise, $s(j)
  \equiv s(k) \pmod n$.  So
\[
r(j) + s(s) \equiv r(j) + s(k) \pmod n,
\]
that is, $t(j) \equiv t(k) \pmod n$.

The proof for $t \eqdef r \cdot s$ is the same.

\end{itemize}

\end{solution}

\ppart Prove the claim.

\begin{solution}
 Suppose $q$ is a nonconstant polynomial.  Since $q$ produces
  multiples iff $-q$ produces multiples, we can assume without loss of
  generality that the leading coefficient of $p$ is positive.  Now we use
  the following familiar fact polynomials of positive degree whose leading
  coefficient is positive: once $x$ is above a certain bound, the function
  $p(x)$ is strictly increasing.  \footnote{We'll prove this and similar
    ``growth rate'' facts about polynomials and other functions next
    week.}

Now assume that $v > 0$ is the absolute value of some integer in the range
of $q$, that is, $0 < v = \abs{q(k)}$ for some $k \in \naturals$.
Therefore,
\[
q(k) \equiv 0 \pmod v.
\]

Since $q$ strictly increases once its argument is big enough, the sequence
\[
q(k),q(k+v),q(k+2v),q(k+3v),\dots
\]
will include an infinite number of values.

But by part~\eqref{jk}, each of the elements in the sequence is $\equiv 0
\pmod v$.  So the sequence includes an infinite number of multiples of
$v$, which proves that $q$ multiplies.
\end{solution}

\eparts

\end{problem}


%%%%%%%%%%%%%%%%%%%%%%%%%%%%%%%%%%%%%%%%%%%%%%%%%%%%%%%%%%%%%%%%%%%%%
% Problem ends here
%%%%%%%%%%%%%%%%%%%%%%%%%%%%%%%%%%%%%%%%%%%%%%%%%%%%%%%%%%%%%%%%%%%%%

\endinput

