\documentclass[problem]{mcs}

\begin{pcomments}
  \pcomment{CP_polynomials_produce_multiples}
  \pcomment{strengthens CP_congruence_no_prime_polynomial and naive
    proof of PS_prime_polynomial_41}
  \pcomment{from: F07 Rec8Th P5}
  \pcomment{edit by ARM 3/1/12}
\end{pcomments}

\pkeywords{
  structural_induction
  polynomial
  multiple
  congruence 
}

%%%%%%%%%%%%%%%%%%%%%%%%%%%%%%%%%%%%%%%%%%%%%%%%%%%%%%%%%%%%%%%%%%%%%
% Problem starts here
%%%%%%%%%%%%%%%%%%%%%%%%%%%%%%%%%%%%%%%%%%%%%%%%%%%%%%%%%%%%%%%%%%%%%

\begin{problem}
  The values of polynomial $p(n) \eqdef n^2 + n + 41$ are prime for
  all the integers from 0 to 39 (see Section~\bref{prop_sec}).  Well,
  $p$ didn't work, but are there any other polynomials whose values
  are always prime?  No way!  In fact, we'll prove a much stronger
  claim.

  Suppose $q$ is a polynomial with integer coefficients whose domain is
  restricted to be the nonnegative integers.  We'll say that $q$
  \emph{produces multiples} if, for every nonzero value in the range of $q$,
  there are infinitely many multiples of that value also in the range.

  For example, if $q$ produces multiples and $q(4) = 7$, then there are
  infinitely many different multiples of 7 in the range of $q$, and of
  course, except for 7 itself, none of these multiples is prime.
\begin{claim*}\label{qconmult}
  If $q$ is not a constant function, then $q$ produces
  multiples.
\end{claim*}

  \bparts

  \ppart\label{jk} Prove that if $j \equiv k \pmod n$, then $q(j) \equiv
  q(k) \pmod n$.

  \hint The set, $A$, of polynomial functions with integer coefficients can be
  defined recursively:
\begin{itemize}
\item \inductioncase{Base cases}:
  \begin{itemize}

   \item the identity function, $i(x) \eqdef x$ is in $A$.

   \item for any integer, $k$, the constant function, $c(x) \eqdef k$ is in $A$.
  \end{itemize}

\item \inductioncase{Constructor cases}.  If $r,s \in A$, then $r+s$
  and $r \cdot s \in A$.
\end{itemize}

\begin{solution}

The proof is by structural induction on the definition of $A$.  The
hypothesis $P(q)$ is that
\[
\text{for all } k,n \in \naturals,\text{ if } j \equiv k \pmod n, \text{
then } q(j) \equiv q(k) \pmod n.
\]

\begin{itemize}
\item \inductioncase{Base cases}: $P(i)$ and $P(c)$ both hold trivially.

\item \inductioncase{Constructor cases}.  Suppose $P(r)$ and $P(t)$ hold, and let
  $t \eqdef r+s$.  To show $P(t)$, suppose $j \equiv k \pmod n$.  Since
  $P(r)$ holds, we have that $r(j) \equiv r(k) \pmod n$.  Likewise, $s(j)
  \equiv s(k) \pmod n$.  So
\[
r(j) + s(s) \equiv r(j) + s(k) \pmod n,
\]
by additivity of congruences
Lemma~\bref{mod_congruence_lem}.\bref{mod_congruence_lem+}, that is,
$t(j) \equiv t(k) \pmod n$.

The proof for $t \eqdef r \cdot s$ is the same.

\end{itemize}

\end{solution}

\ppart Prove the Claim~\ref{qconmult}.

\begin{staffnotes}
If need be, suggest use of the fact that once $x$ is above a certain
bound, the magnitude of a polynomial becomes strictly increasing.
\end{staffnotes}

\begin{solution}
 Suppose $q$ is a nonconstant polynomial.  Since $q$ produces
 multiples iff $-q$ produces multiples, we can assume without loss of
 generality that the leading coefficient of $p$ is positive.  Now we
 use the following familiar fact polynomials of positive degree whose
 leading coefficient is positive: once $x$ is above a certain bound,
 the function $p(x)$ is strictly increasing.  \footnote{We'll prove
   this and similar ``growth rate'' facts about polynomials and other
   functions in Chapter~\bref{chap:asymptotics}.}

Now assume that $v > 0$ is the absolute value of some integer in the range
of $q$, that is, $0 < v = \abs{q(k)}$ for some $k \in \naturals$.
Therefore,
\[
q(k) \equiv 0 \pmod v.
\]

Since $q$ strictly increases once its argument is big enough, the sequence
\[
q(k),q(k+v),q(k+2v),q(k+3v),\dots
\]
will include an infinite number of values.

But by part~\eqref{jk}, each of the elements in the sequence is $\equiv 0
\pmod v$.  So the sequence includes an infinite number of multiples of
$v$, which proves that $q$ multiplies.
\end{solution}

\eparts

Claim~\ref{qconmult} implies that if an integer polynomial is not
constant then its range includes infinitely many nonprimes.  This fact
no longer holds true for multivariate polynomials.  An amazing
consequence of Matijesevich's solution to Hilbert's Tenth Problem,
\TBA{reference}, is that multivariate polynomials can be understood as
\emph{general purpose} programs for generating sets of integers.  If a
set of nonnegative integers can be generated by \emph{any} program,
then it equals the set of nonnegative integers in the range of a
multivariate integer polynomial!  In particular, there is an integer
polynomial $p(x_1,\dots,x_7)$ whose nonnegative values as
$x_1,\dots,x_7$ range over $\naturals$ are precisely the set of all
prime numbers!

\end{problem}


%%%%%%%%%%%%%%%%%%%%%%%%%%%%%%%%%%%%%%%%%%%%%%%%%%%%%%%%%%%%%%%%%%%%%
% Problem ends here
%%%%%%%%%%%%%%%%%%%%%%%%%%%%%%%%%%%%%%%%%%%%%%%%%%%%%%%%%%%%%%%%%%%%%

\endinput

