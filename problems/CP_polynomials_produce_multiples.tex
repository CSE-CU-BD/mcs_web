\documentclass[problem]{mcs}

\begin{pcomments}
  \pcomment{CP_polynomials_produce_multiples}
  \pcomment{strengthens CP_congruence_no_prime_polynomial and naive
    proof of PS_prime_polynomial_41}
  \pcomment{from: F07 Rec8Th P5}
  \pcomment{revised by ARM 3/12/13}
\end{pcomments}

\pkeywords{
  structural_induction
  polynomial
  multiple
  congruence 
}

%%%%%%%%%%%%%%%%%%%%%%%%%%%%%%%%%%%%%%%%%%%%%%%%%%%%%%%%%%%%%%%%%%%%%
% Problem starts here
%%%%%%%%%%%%%%%%%%%%%%%%%%%%%%%%%%%%%%%%%%%%%%%%%%%%%%%%%%%%%%%%%%%%%

\begin{problem}
\inbook{The values of polynomial $p(n) \eqdef n^2 + n + 41$ are prime for
  all the integers from 0 to 39 (see Section~\bref{prop_sec}).  Well,
  $p$ didn't work, but are there any other polynomials whose values
  are always prime?  No way!  In fact, we'll prove a much stronger
  claim.}

\begin{definition*}
The set, $P$, of integer polynomials can be defined recursively:

\inductioncase{Base cases}:
  \begin{itemize}

   \item the identity function, $i(x) \eqdef x$ is in $P$.

   \item for any integer, $k$, the constant function, $c_k(x) \eqdef k$ is in $P$.
  \end{itemize}

\inductioncase{Constructor cases}.  If $r,s \in P$, then $r+s$
  and $r \cdot s \in P$.

\end{definition*}

  \bparts

  \ppart\label{jk} Using the recursive definition of integer
  polynomial given below, prove by structural induction that for all
  $q \in P$,
\[
j \equiv k \pmod{n}\quad \QIMPLIES\quad q(j) \equiv q(k) \pmod n,
\]
for all integers $j,k,n$ where $n>1$.

Be sure to clearly state and label your Induction Hypothesis, Base
case(s), and Constructor step.

\examspace[3.0in]

\begin{solution}
The proof is by structural induction on the definition of $P$.  The
hypothesis $H(q)$ is:
\begin{align*}
H(q) & \eqdef\  j\equiv k \pmod{n}\ \QIMPLIES q(j) \equiv q(k) \pmod n,\\
     & \qquad\quad \text{for all } j,k,n \in \integers,\text{ where } n>1.
\end{align*}

\begin{itemize}
\item \inductioncase{Base cases}: $H(i)$ holds because $j = i(j)$ and
  $k = i(k)$.  $H(c_k)$ holds because $c_k(i) = c_k(j)$.

\item \inductioncase{Constructor cases}.  Suppose $H(r)$ and $H(t)$ hold, and let
  $t \eqdef r+s$.  To show $H(t)$, suppose $j \equiv k \pmod n$.  Since
  $H(r)$ holds, we have that $r(j) \equiv r(k) \pmod n$.  Likewise, $s(j)
  \equiv s(k) \pmod n$.  So
\[
r(j) + s(s) \equiv r(j) + s(k) \pmod n,
\]
by additivity of congruences
Lemma~\bref{mod_congruence_lem}(\bref{mod_congruence_lem+}), that is,
$t(j) \equiv t(k) \pmod n$.

The proof for $t \eqdef r \cdot s$ is the same with ``$\cdot$'' replacing ``$+$.''

\end{itemize}

\end{solution}

\ppart\label{qconmult} We'll say that $q$ \emph{produces multiples}
if, for every integer greater than one in the range of $q$, there are
infinitely many different multiples of that integer in the range.  For
example, if $q(4) = 7$ and $q$ produces multiples, then there are
infinitely many different multiples of 7 in the range of $q$\inbook{,
  and of course, except for 7 itself, none of these multiples is
  prime}.

Prove that if $q$ has positive degree and positive leading
coefficient, then $q$ produces multiples.  You may assume that every
such polynomial is strictly increasing for large arguments.

\inhandout{\hint Observe that all the elements in the sequence
\[
q(k), q(k + v), q(k+2v), q(k+3v),\dots,
\]
are congruent modulo $v$.  Let $v=q(k)$.}

\begin{solution}
If $1< v \in \range{q}$, then $v = q(k)$ for some integer $k$, and
we have immediately that
\[
q(k) \equiv 0 \pmod v.
\]

Since $k \equiv k+nv \pmod v$, part~\eqref{jk} implies that each of
the elements in the sequence
\[
q(k), q(k + v), q(k+2v), q(k+3v),\dots
\]
is $\equiv q(k) \equiv 0 \pmod v$.  So all the elements in the
sequence are multiples of $v$.

Since $q(k)$ is strictly increasing\footnote{We'll prove this and
  similar ``growth rate'' facts about polynomials and other functions
  in Chapter~\bref{chap:asymptotics}.} for $k\geq b$ for some bound,
$b>0$, all the elements in the sequence are different after a certain
point (no later than after $(b/2)$ elements).  So there are
arbitrarily large multiples of $v$ in the range of $q$.  Since $v>1$
was an arbitrary element, we conclude there are infinitely many
multiples of every element $v>1$ in the range of $q$.  That is, $q$
produces multiples.
\end{solution}

\eparts

\insolutions{
\medskip
Part~\eqref{qconmult} implies that an integer polynomial with positive
leading coefficient and degree has infinitely many nonprimes in its
range.  This fact no longer holds true for multivariate polynomials.
An amazing consequence of Matijesevich's~\cite{Matijesevich} solution
to Hilbert's Tenth problem, is that multivariate polynomials can be
understood as \emph{general purpose} programs for generating sets of
integers.  If a set of nonnegative integers can be generated by
\emph{any} program, then it equals the set of nonnegative integers in
the range of a multivariate integer polynomial!  In particular, there
is an integer polynomial $p(x_1,\dots,x_7)$ whose nonnegative values
as $x_1,\dots,x_7$ range over $\integers$ are precisely the set of all
prime numbers!}

\inbook{
\medskip

Part~\eqref{qconmult} implies that an integer polynomial with positive
leading coefficient and degree has infinitely many nonprimes in its
range.  This fact no longer holds true for multivariate polynomials.
An amazing consequence of Matijesevich's solution to Hilbert's Tenth
Problem~\cite{Matijesevich}, is that multivariate polynomials can be
understood as \emph{general purpose} programs for generating sets of
integers.  If a set of nonnegative integers can be generated by
\emph{any} program, then it equals the set of nonnegative integers in
the range of a multivariate integer polynomial!  In particular, there
is an integer polynomial $p(x_1,\dots,x_7)$ whose nonnegative values
as $x_1,\dots,x_7$ range over $\naturals$ are precisely the set of all
prime numbers!}

\end{problem}


%%%%%%%%%%%%%%%%%%%%%%%%%%%%%%%%%%%%%%%%%%%%%%%%%%%%%%%%%%%%%%%%%%%%%
% Problem ends here
%%%%%%%%%%%%%%%%%%%%%%%%%%%%%%%%%%%%%%%%%%%%%%%%%%%%%%%%%%%%%%%%%%%%%

\endinput

