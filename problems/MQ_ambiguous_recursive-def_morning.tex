\documentclass[problem]{mcs}

\begin{pcomments}
  \pcomment{MQ_ambiguous_recursive-def}
  \pcomment{contributed by ARM 10/18/09}
\end{pcomments}

\pkeywords{
 recursive_data
 matching_parenteses
 ambiguous_definition
}

\newcommand{\MB}[1]{\text{MB}_{#1}}
\newcommand{\lefnum}{\text{numleft}}
\newcommand{\derlen}{\text{two-t}}


%%%%%%%%%%%%%%%%%%%%%%%%%%%%%%%%%%%%%%%%%%%%%%%%%%%%%%%%%%%%%%%%%%%%%
% Problem starts here
%%%%%%%%%%%%%%%%%%%%%%%%%%%%%%%%%%%%%%%%%%%%%%%%%%%%%%%%%%%%%%%%%%%%%
\begin{problem}
\begin{definition*}

  Recursively define the set, $\MB{1}$, of strings of ``matching'' brackets
  as follows:

\begin{itemize}

\item \textbf{Base case:} $\emptystring \in\MB{1}$.

\item \textbf{Constructor case:} If $s,t \in\MB{1}$, then
  $\mtt{[}s\mtt{]}t \in \MB{1}$.

\end{itemize}

Recursively define the set, $\MB{2}$, of strings of brackets as follows:

\begin{itemize}

\item \textbf{Base case:} $\emptystring \in \MB{2}$,

\item \textbf{Constructor cases:}
If $s\neq \emptystring$, $t \neq \emptystring$, and $s,t \in \MB{2}$, then

\renewcommand{\theenumi}{\roman{enumi}}
\renewcommand{\labelenumi}{(\theenumi)}

\begin{enumerate}

\item $\mtt{[}s\mtt{]}$ is in $\MB{2}$, and
\item $st$ is in $\MB{2}$.

\end{enumerate}

\end{itemize}

\end{definition*}

These two definitions both specify the same set of strings, that is, $\MB{1}
= \MB{2}$.

\bparts

\begin{comment}
\ppart To prove that $\MB{1} = \MB{2}$, you could first prove that
\[
\forall r \in \MB{1}.\ r \in \MB{2},
\]
and then prove that
\[
\forall u \in \MB{2}.\ u \in \MB{1}.
\]
Of these two statements, circle the one that would be simpler to prove by
structural induction directly from the definitions.

\iffalse
\hint The harder of the two proofs requires first working out an idea like
``good counting'' to find matching brackets.
\fi

\begin{solution}
  It's easy to prove that $\forall r \in \MB{1}.\ r \in \MB{2}$
  by structural induction on $r \in \MB{1}$.

  That's because if $s,t \in \MB{2}$, then it's obvious that
  $\mtt{[}s\mtt{]}t \in \MB{2}$, and this immediately proves the inductive
  step of showing that if $\mtt{[}s\mtt{]}t \in \MB{1}$ then $\mtt{[}s\mtt{]}t \in
  \MB{2}$.

  On the other hand, proving that if $s,t \in \MB{1}$ then $st \MB{1}$ is
  what's needed in the inductive step that if $st \in \MB{2}$ then $st \in
  \MB{1}$.  But proving this requires working out a definition of which
  bracket matches the initial bracket in a string in $\MB{1}$.  We saw how
  to do this using the idea of ``good counting,'' but it took some work.
\end{solution}
\end{comment}

\ppart To prove that $\MB{1} = \MB{2}$, you could first prove that
\[
\MB{1} \subseteq \MB{2},
\]
and then prove that
\[
\MB{2} \subseteq \MB{1}.
\]
Suppose structural induction was being used to prove that $\MB{2}
\subseteq \MB{1}$.  Circle the one predicate below that would fit the
format for a structural induction hypothesis in such a proof.

\begin{itemize}

\item $P_0(s) \eqdef s \in \MB{1}$.

\item $P_1(s) \eqdef s \in \MB{2}$.

\item $P_2(s) \eqdef\ (s \in \MB{1} \QIMPLIES s \in \MB{2})$.

\item $P_3(n) \eqdef  \lnth{s} \leq n \QIMPLIES s \in \MB{1}$.

\item $P_4(n) \eqdef  \lnth{s} \leq n \QIMPLIES s \in \MB{2}$.

\end{itemize}

\begin{solution}
  $\MB{2} \subseteq \MB{1}$ means that $\forall s \in \MB{2}.\ s \in
  \MB{1}$.  To prove this, the only hypothesis above that fits the format
  for a structural induction would be $P_0$ in a structural induction on
  the definition of $\MB{2}$.
\end{solution}

\instatements{\newpage}
\ppart One of these recursive definitions is \emph{ambiguous}.  Verify
this by giving two different derivations for the string "$\mtt{[\ ][\ ][\ ]}$"
according to one of these recursive definitions.

\instatements{\vspace{6in}}

\begin{solution}
  $\MB{2}$ is ambiguous, because "$\mtt{[\ ][\ ][\ ]}" \in \MB{2}$ can be derived
  from the second constructor with $s = \mtt{[\ ]}$ and $t = [\ ][\ ]$, but also
  with $s = \mtt{[\ ][\ ]}$ and $t = [\ ]$.
\end{solution}

\ppart Explain in one or two sentences (preferably one) \emph{why} we need
be concerned about ambiguous recursive definitions.

\instatements{\vspace{2in}}

\begin{solution}

A function defined recursively from an ambiguous definition of a data type
may not be well-defined. For example, the following function:
  
\[\begin{array}{rl}
f(r) & \eqdef \begin{cases}
                   0 & \text{ if } r = \emptystring\\
                   1 + f(s)
                     & \text{ if } r = \mtt{[}s\mtt{]},\\
                   f(s) + 2 \cdot f(t)
                     & \text{ if } r = st\text{ and } s,t \neq \emptystring.
                   \end{cases}
\end{array}\]  

is not well-defined because

\[
5= 3+2\cdot 1 =f([\ ][\ ]\cdot[\ ]) = f([\ ]\cdot[\ ][\ ]) = 1+2\cdot 3 =7.
\]

\end{solution}

% \ppart Ambiguous definitions can lead to ill-defined recursive definitions
% of functions.  Here are apparent definitions of four functions $\lefnum_i$
% and $\derlen_i$
% \iffalse from $\MB{i} \to \naturals$\fi
% based on the recursive definition of $\MB{i}$, for $i=1,2$.  Circle all the
% apparent definitions which do not, in fact, uniquely determine a function.
% 
% \[\begin{array}{rl}
% \lefnum_1(r)  & \eqdef \begin{cases}
%                    0 & \text{ if } r = \emptystring\\
%                    1 + \lefnum_1(s) + \lefnum_1(t)
%                      & \text{ if } r = \mtt{[}s\mtt{]}t.
%                    \end{cases}\\
% \\
% \lefnum_2(r)  & \eqdef \begin{cases}
%                    0 & \text{ if } r = \emptystring\\
%                    1 + \lefnum_2(s)
%                      & \text{ if } r = \mtt{[}s\mtt{]},\\
%                    \lefnum_2(s) + \lefnum_2(t)
%                      & \text{ if } r = st.
%                    \end{cases}\\
% \\
% \derlen_1(r)  & \eqdef \begin{cases}
%                    0 & \text{ if } r = \emptystring\\
%                    1+ \derlen_1(s) + 2\cdot\derlen_1(t)
%                      & \text{ if } r = \mtt{[}s\mtt{]}t.
%                    \end{cases}\\
% \\
% \derlen_2(r)  & \eqdef \begin{cases}
%                    0 & \text{ if } r = \emptystring\\
%                    1 + \derlen_2(s)
%                      & \text{ if } r = \mtt{[}s\mtt{]},\\
%                    \derlen_2(s) + 2 \cdot \derlen_2(t)
%                      & \text{ if } r = st\text{ and } s,t \neq \emptystring.
%                    \end{cases}
% \end{array}\]
% 
% \begin{solution}
% $\lefnum_1$ and $\derlen_1$ are well-defined because the definition of
% $\MB{1}$ is unambiguous.
% 
% $\derlen_2$ is ill-defined, because
% \[
% 5= 3+2\cdot 1 =\derlen_2([\ ][\ ]\cdot[\ ]) = \derlen_2([\ ]\cdot[\ ][\ ]) = 1+2\cdot 3 =7.
% \]
% 
% $\lefnum_2(r)$ is well-defined because it equals the number of left
% brackets in $r$, which could be proved by structural induction on the
% definition of $\MB{2}$.
% \end{solution}

\eparts

\end{problem}

%%%%%%%%%%%%%%%%%%%%%%%%%%%%%%%%%%%%%%%%%%%%%%%%%%%%%%%%%%%%%%%%%%%%%
% Problem ends here
%%%%%%%%%%%%%%%%%%%%%%%%%%%%%%%%%%%%%%%%%%%%%%%%%%%%%%%%%%%%%%%%%%%%%

\endinput
