\documentclass[problem]{mcs}

\begin{pcomments}
  \pcomment{PS_sevenish_maps}
  \pcomment{variant of PS_triangle_free_planar_graphs}
  \pcomment{from: S10.ps7}
\end{pcomments}

\pkeywords{
  planar_graphs
  handshaking
  graph_coloring
  strong_induction
  induction_on_graphs
  colorable
}

%%%%%%%%%%%%%%%%%%%%%%%%%%%%%%%%%%%%%%%%%%%%%%%%%%%%%%%%%%%%%%%%%%%%%
% Problem starts here
%%%%%%%%%%%%%%%%%%%%%%%%%%%%%%%%%%%%%%%%%%%%%%%%%%%%%%%%%%%%%%%%%%%%%

\begin{problem} A simple graph is \emph{sevenish} when it has no
cycles of length less than seven.  A \term{map} is a planar graph 
whose faces are all cycles (no dongles or bridges).  Let $M$ be
a sevenish, connected, map with $v>2$ vertices, $e$ edges and $f$
faces.

\bparts

\ppart\label{e2v4big} Prove that
\begin{equation}\label{e7v14}
e \leq (7v - 14)/5.
\end{equation}

\hint Similar to the proof of Theorem~\bref{th:e3v}, that $e \leq 3v-6$
in planar graphs.

\begin{solution}

\begin{proof}
  Since the faces are cycles, the length of each face is at least 7.
  Each edge is occurs in exactly two faces, so
  \begin{equation}\label{4fbig}
    2e = \sum_{C \in\text{ faces}} \text{length}(C) \geq
    \sum_{C \in\text{ faces}} 7 = 7f.
  \end{equation}

  By Euler's formula, $f = e-v+2$, so
  substituting for $f$ in~\eqref{4fbig}, yields
\[
2e \geq 7(e-v+2),
\]
which simplifies to~\eqref{e7v14}.

\end{proof}
\end{solution}

\ppart\label{sevenishd2} Show that $M$ has a vertex of degree at most
two.

\begin{solution}
Multiplying both sides of equation~\eqref{e7v14} by $2/v$, we get
\[
2e/v \leq 2(7v-14)/5v < 3.
\]

But by the Handshaking Lemma, the sum of degrees is $2e$, so the
average degree is $2e/v$.  But the average degree can be strictly less
than 3 only if at least one vertex has degree $\leq 2$.
\end{solution}

\ppart Part~\eqref{sevenishd2} can be used to prove that $M$ is
3-colorable by induction on $v$.  The proof is slightly complicated by
the fact that subgraphs of $M$ may not be sevenish, connected, maps.
For each of the following properties, briefly explain why all
connected subgraphs of $M$ have the property, or give an example of a
connected subgraph of an $M$ that does not have the property.

\begin{enumerate}
\item map
\item planar
\item sevenish
\item connected % take it out if reusing.. implied by "connected" subgraphs
\item 3-colorable
\end{enumerate}

\begin{solution}
  
  \begin{enumerate}
  \item
    No. Consider a graph with a length seven cycle. The graph will no longer be
    a map if we remove one of its edges.
  \item
    Yes. A subgraph of a planar graph is always planar.
  \item
    Yes. Removing edges or vertices will not create new cycles.  Therefore,
    the cycles in the subgraph must also exist in the original graph and
    none of them can have a length greater than seven since the original graph
    was sevenish.
  \item
    Yes. All \emph{connected} subgraph is clearly connected.
  \item
    Yes. If we assume that $M$ is 3-colorable, then clearly any subgraph
    must be 3-colorable as well.
  \end{enumerate}

\iffalse
\mbox{}

NEEDS REVISION:

  \begin{proof} By strong induction on the number $v$ of vertices with the
  induction hypothesis that every $v$-vertex sevenish, connected,
  planar graph is 3-colorable.

  \inductioncase{Base cases} ($v\leq 3$):  A graph with at most 3 vertices is
  trivially be 3-colorable.

  \inductioncase{Inductive step}: By part~\eqref{sevenishd2}, $G$ has a vertex,
  $u$, of degree 2 or less.  Remove this vertex and any incident
  edges, and let $H$ be any connected component of the resulting
  graph.  We know that since $G$ is planar, so is $H$.  Moreover,
  removing an edge may remove some cycles, but won't create new
  ones, so $H$ is still sevenish.  \textbf{But may not be a map}!!

  So by strong induction, $H$ is 3-colorable.  Now to 3-color $G$,
  3-color each of the connected components $H$ obtained when $u$ was
  removed.  Except for $u$, adjacent vertices in $G$ are also adjacent
  in some connected component $H$ and will therefore be assigned
  different colors.  Finally, to complete the 3-coloring of $G$,
  assigning to $u$ a color that differs from those of the at most two
  vertices adjacent to $u$.
\end{proof}\fi

\end{solution}

\eparts

\end{problem}

%%%%%%%%%%%%%%%%%%%%%%%%%%%%%%%%%%%%%%%%%%%%%%%%%%%%%%%%%%%%%%%%%%%%%
% Problem ends here
%%%%%%%%%%%%%%%%%%%%%%%%%%%%%%%%%%%%%%%%%%%%%%%%%%%%%%%%%%%%%%%%%%%%%
\endinput
