\documentclass[problem]{mcs}

\begin{pcomments}
  \pcomment{MQ_inverse_by_pulverizer}
  \pcomment{from: S09.mqapr9, F07.mq4}
\end{pcomments}

\pkeywords{
  number_theory
  Pulverizer
  modular_arithmetic
  inverses
  Fermat_theorem
}

%%%%%%%%%%%%%%%%%%%%%%%%%%%%%%%%%%%%%%%%%%%%%%%%%%%%%%%%%%%%%%%%%%%%%
% Problem starts here
%%%%%%%%%%%%%%%%%%%%%%%%%%%%%%%%%%%%%%%%%%%%%%%%%%%%%%%%%%%%%%%%%%%%%

\begin{problem}
\bparts
\ppart\label{Pul40&7}
Use the Pulverizer to find integers $s, t$ such that
\[
40 s + 7 t = \gcd(40, 7).
\]
Show your work.
\iffalse
You will be graded on the correctness of the method; if you obtain the
correct answer without implementing the Pulverizer, you will not receive
most of the credit for this problem.
\fi

\examspace{4in}
\begin{solution}
$s=3$ and $t=-17$

 Here is the table produced by the Pulverizer:
 \[
 \begin{array}{ccccrcl}
 x & \quad & y & \quad & \rem{x}{y} & = & x - q \cdot y \\ \hline
  40 &&   7 &&  5 & = &    40 - 5 \cdot  7 \\
   7 &&   5 &&  2 & = &    7 - 5 \\
 &&&&             & = &    7 - (40 - 5 \cdot  7) \\
 &&&&             & = &   -1 \cdot 40 + 6 \cdot 7 \\
   5 &&   2 &&  1 & = &   5  - 2 \cdot 2  \\
 &&&&             & = &   (40 - 5 \cdot  7)
                          -2 \cdot (-1 \cdot 40 + 6 \cdot 7) \\
 &&&&             & = &   3\cdot 40 -17 \cdot 7 \\
   2 &&   1 &&  0 &   &
 \end{array}
 \]
\end{solution}

\ppart
Adjust your answer to part~\eqref{Pul40&7} to find an inverse modulo 40 of 7 in
the range $\set{1,\dots,39}$.

\examspace{2in}

\begin{solution}
 \begin{align*}
  1  & = 3 \cdot 40 - 17 \cdot 7\\
     & = 3 \cdot 40 - 7 \cdot 40 + 40 \cdot 7 - 17 \cdot 7\\
     & = (3-7) \cdot 40 + (40-17) \cdot 7\\
     & = -4 \cdot 40 + 23 \cdot 7
 \end{align*}
 Therefore, $23 \cdot 7 \equiv 1 \pmod{40}$ and 23 is the inverse
 of 7 modulo 40.

Alternatively, since $-17$ is an inverse, so is $\rem{-17}{40} =23$.

\end{solution}

\eparts
\end{problem}
