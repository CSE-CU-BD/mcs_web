\documentclass[problem]{mcs}

\begin{pcomments}
\pcomment{MQ_truth_table_and_predicate_multipart}
\pcommnet{from MQ1 sp09}
\end{pcomments}

\pkeywords{
  truth_table
  quantifier
  propositional
  predicate
}

\begin{problem}
\bparts

\ppart Use a truth table to verify that $(P \QXOR (P \QXOR Q))$ is
equivalent to $Q$.

\examspace[3in]

\begin{solution}
\[                                                                                                                                                                                
\begin{array}{|c|c|ccc|}                                                                                                                                                          
\hline                                                                                                                                                                            
P         &\QXOR        & \text{(}P  & \QXOR       & Q\text{)}\\ \hline
\true     &\true       & \true      & \false     & \true    \\ \hline
\true     &\false      & \true      & \true      & \false   \\ \hline                                                                                                             
\false    &\true       & \false     & \true      & \true    \\ \hline
\false    &\false      & \false     & \false     & \false   \\ \hline
\end{array}  
\]

Compare the $(P \QXOR (P \QXOR Q))$ column with the $Q$ column to see
that they are equivalent.

\end{solution}

\ppart Find a counter model showing the following is not valid.
\[
\exists x . P(x) \QIMPLIES \forall x . P(x)
\]

\begin{solution}

 Define $P(x) \eqdef ``x=1''$ and let the domain of discourse be
  the two element set $\set{1,2}$.  Then the left hand side of the implies
  is true since there is an $x$, namely $x=1$, for which $P(x)$ is true
  and the right hand side is false because every element in $\set{1,2}$ is
  not equal to 1.  Therefore the implication does not hold and the
  statement is not a validity.  
\end{solution}

\eparts
\end{problem}