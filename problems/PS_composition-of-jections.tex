%PS_composition-of-jections.tex

\begin{pcomments}
  \pcomment{from: S09 ln3 notesproblem;
            mostly subsumed by CP_surj_relation}
\end{pcomments}

\pkeywords{
composition
surjection
bijection
inverse}

%%%%%%%%%%%%%%%%%%%%%%%%%%%%%%%%%%%%%%%%%%%%%%%%%%%%%%%%%%%%%%%%%%%%%
% Problem starts here
%%%%%%%%%%%%%%%%%%%%%%%%%%%%%%%%%%%%%%%%%%%%%%%%%%%%%%%%%%%%%%%%%%%%%

\begin{problem}
  Let $f:A \to B$ and $g: B \to C$ be functions and $h:A \to C$ be their
  composition, namely, $h(a) \eqdef g(f(a))$ for all $a \in A$.
\bparts
  \ppart Prove that if $f$ and $g$ are surjections, then so is $h$.

  \ppart Prove that if $f$ and $g$ are bijections, then so is $h$.

  \ppart If $f$ is a bijection, then define $f':B \to A$ so that
  \[
  f'(b) \eqdef\text{ the unique } a \in A \text{ such that } f(a)=b.
  \]
  Prove that $f'$ is a bijection.  (The function $f'$ is called the
  \emph{inverse} of $f$.  The notation $f^{-1}$ is often used for the
  inverse of $f$.)
\eparts

\begin{solution}
TBA
\end{solution}

\end{problem}

%%%%%%%%%%%%%%%%%%%%%%%%%%%%%%%%%%%%%%%%%%%%%%%%%%%%%%%%%%%%%%%%%%%%%
% Problem ends here
%%%%%%%%%%%%%%%%%%%%%%%%%%%%%%%%%%%%%%%%%%%%%%%%%%%%%%%%%%%%%%%%%%%%%

\endinput
