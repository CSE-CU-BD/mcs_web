\documentclass[problem]{mcs}

\begin{pcomments}
  \pcomment{TP_divide_product_wop}
  \pcomment{wop version of TP_divide_product_induction}
  \pcomment{ARM 3/17/13}
\end{pcomments}

\pkeywords{
  divides
  prime
  well_ordering
}

%%%%%%%%%%%%%%%%%%%%%%%%%%%%%%%%%%%%%%%%%%%%%%%%%%%%%%%%%%%%%%%%%%%%%
% Problem starts here
%%%%%%%%%%%%%%%%%%%%%%%%%%%%%%%%%%%%%%%%%%%%%%%%%%%%%%%%%%%%%%%%%%%%%

\begin{problem}
Use the Well Ordering Principle to prove that if $p$ is prime, then
for all $a_1,a_2,\dots,a_n$ where $n\geq 1$, if $p \divides a_1 \cdot
a_2 \cdots a_n$, then $p$ divides some $a_i$.  You may assume the case
for $n=2$ which was proved \inhandout{in the
  text}\inbook{Lemma~\bref{lem:prime-divides}}.
  
Be sure to define and clearly label the set of counterexamples you are
assuming is nonempty.

\examspace

\begin{solution}
Let
\begin{align*}
C &\eqdef \{n \geq 1 \suchthat\\
  & \qquad \QNOT(\text{if $p$ is prime, then for all $a_1,a_2,\dots,a_n$}\\
  & \qquad       \text{where $n\geq 1$, if $p \divides a_1 \cdot a_2 \cdots a_n$,}\\
  & \qquad       \text{then $p$ divides some $a_i$})\},
\end{align*}
and assume for the sake of contradiction that $C$ is not empty.

By the WOP, there is a least integer $m \in C$.
So there must be integers $ a_1, a_2, a_{m}$ such that
\[
p \divides a_1 \cdot a_2 \cdots a_{m},
\]
but $\QNOT(p \divides a_i)$ for all $i \in \Zintv{1}{m}$.

Since the claim is trivially true for $n=1$, we know that $m-1 \geq
1$.

Let $b \eqdef a_1 \cdots a_n \cdot a_{m-1}$.  Now $p \divides ba_m$,
which we know implies that $p \divides a_m$ or $p \divides b$.  Since
$\QNOT(p \divides a_m)$ so it must be that $p \divides a_1 \cdots a_2
\cdot a_{m-1}$.  But $m-1 \notin C$, so we have that $p \divides a_i$
for some $i \in \Zintv{1}{m}$, contradicting the fact that $\QNOT(p \divides
  a_i)$ for all $i \in \Zintv{1}{m}$.

The oontradiction implies that $C$ must be empty, proving that the
claim holds for all $n \geq 1$.

\end{solution}
  
\end{problem}

%%%%%%%%%%%%%%%%%%%%%%%%%%%%%%%%%%%%%%%%%%%%%%%%%%%%%%%%%%%%%%%%%%%%%
% Problem ends here
%%%%%%%%%%%%%%%%%%%%%%%%%%%%%%%%%%%%%%%%%%%%%%%%%%%%%%%%%%%%%%%%%%%%%

\endinput
