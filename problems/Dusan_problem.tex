\documentclass[problem]{mcs}

\begin{pcomments}
  \pcomment{Dusan_problem}
  \pcomment{formatted by ARM 3/9/15}
\end{pcomments}

\pkeywords{
  pigeon_hole
  decimal_expansion
  modulo
}

%%%%%%%%%%%%%%%%%%%%%%%%%%%%%%%%%%%%%%%%%%%%%%%%%%%%%%%%%%%%%%%%%%%%%
% Problem starts here
%%%%%%%%%%%%%%%%%%%%%%%%%%%%%%%%%%%%%%%%%%%%%%%%%%%%%%%%%%%%%%%%%%%%%

%Problem Proposal for the Final Exam: Dusan Milijancevic, May 18, 2013}

\begin{problem}
The aim of this problem is to prove that there exist a natural number
$n$ such that $3^n$ has at least $2013$ consecutive zeros in its
decimal expansion.

\bparts

\ppart Prove that there exist a nonnegative integer $n$ such that
$3^n\equiv 1 \; \; mod \; \; 10^{2014}$.

\hint Use pigeonhole principle or Euler's theorem.

\begin{solution}
There are two ways to solve this problem as indicated in the hint.
\begin{itemize}


\item \textbf{Pigeonhole principle}. Let $A\eqdef \set{3^i \suchthat i
  \in \Zintvcc{0}{10^{2014}}$.  At least two number from $A$ have the
  same reminder when divided by $10^{2013}$, by the pigeon hole
  principle.  Call them $3^i$ and $3^j$ for $i>j$. Then
  $10^{2014}|3^j(3^{i-j}-1)$, so $10^{2014}|3^{i-j}-1$. Hence for
  $n=i-j$, $3^n\equiv 1 \; \; mod \; \; 10^{2014}$.

\item \textbf{Euler's theorem}. As $gcd(3,10^{2014})=1$ for
  $n=\phi(10^{2014})$ we have $3^{n}\equiv \; \; 1 \; \; mod{10^{2014}}$.

\end{itemize}

\end{solution}

\ppart Conclude that there exist a natural number $n$ such that
$3^n$ has at least $2013$ consecutive zeros.

\begin{solution}
For $n$ in part a) we have
\[
3^n=***...\underbrace{000....000}_{2013}1,
\]
so $3^n$ has at least 2013 consecutive zeros.

\end{solution}

\end{problem}
%%%%%%%%%%%%%%%%%%%%%%%%%%%%%%%%%%%%%%%%%%%%%%%%%%%%%%%%%%%%%%%%%%%%%
% Problem ends here
%%%%%%%%%%%%%%%%%%%%%%%%%%%%%%%%%%%%%%%%%%%%%%%%%%%%%%%%%%%%%%%%%%%%%

\endinput
