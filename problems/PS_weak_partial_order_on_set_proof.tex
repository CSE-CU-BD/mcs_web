\documentclass[problem]{mcs}

\begin{pcomments}
  \pcomment{from: F07.ps2}
\end{pcomments}

\pkeywords{
  partial_order
  weak_partial_order
  reflexive
  antisymmetric
  transitive
}

%%%%%%%%%%%%%%%%%%%%%%%%%%%%%%%%%%%%%%%%%%%%%%%%%%%%%%%%%%%%%%%%%%%%%
% Problem starts here
%%%%%%%%%%%%%%%%%%%%%%%%%%%%%%%%%%%%%%%%%%%%%%%%%%%%%%%%%%%%%%%%%%%%%

\begin{problem}
Prove the following assertion:

If $R$ is a weak partial order on a set, $A$, then
\begin{equation}\label{rgb}
a\, R\, b  \qiff  R\set{a} \subseteq R\set{b}
\end{equation}
holds for all $a,b \in A$.

\begin{solution}
Suppose $R$ is a weak partial order. In other words, $R$ is
reflexive, antisymmetric, and transitive. We'll prove \eqref{rgb}.

For the left-to-right direction of the ``iff'': Assume $aRb$. To prove
that $R\set{a} \subseteq R\set{b}$, we need to prove that
\[
\text{every $c$ in $R\{a\}$ is also in $R\{b\}$.}
\]
So, pick any $c$ in $R\{a\}$. Then $cRa$ (by the definition of
$R\{a\}$). Therefore, $cRa$ and $aRb$ (by our assumption). By
transitivity, we conclude that $cRb$, too. Therefore $c\in R\{b\}$ (by
the definition of $R\{b\}$).

For the right-to-left direction of the ``iff'': Assume
$R\set{a}\subseteq R\set{b}$. We need to prove that $aRb$. We start by
noting that $aRa$ holds (by reflexivity). Therefore $a\in R\{a\}$ (by
the definition of $R\{a\}$). By our assumption that $R\set{a}\subseteq
R\set{b}$, we conclude that $a\in R\{b\}$, too. By the definition of
$R\{b\}$, this means that $aRb$.
\end{solution}

\end{problem}


%%%%%%%%%%%%%%%%%%%%%%%%%%%%%%%%%%%%%%%%%%%%%%%%%%%%%%%%%%%%%%%%%%%%%
% Problem ends here
%%%%%%%%%%%%%%%%%%%%%%%%%%%%%%%%%%%%%%%%%%%%%%%%%%%%%%%%%%%%%%%%%%%%%