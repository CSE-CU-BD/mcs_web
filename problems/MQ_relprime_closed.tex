\documentclass[problem]{mcs}

\begin{pcomments}
    \pcomment{MQ_relprime_closed}
    \pcomment{for an actual quiz, ask for at most 2 out of 3 proofs}
    \pcomment{ARM 1/26/12}
\end{pcomments}

\pkeywords{
  modular_arithmetic
  relatively_prime
  inverse
}

\begin{problem}
Prove that if $k_1$ and $k_2$ are relatively prime to $n$, then
so is $k_1 \cdot_n k_2$,

\bparts

\ppart \dots using the fact that $k$ is relatively prime to $n$ iff
$k$ has an inverse modulo $n$

\hint Recall that $k_1k_2 \equiv k_1 \cdot_n k_2 \pmod{n}$.

\begin{solution}
If $j_1$ is an inverse of $k_1$ modulo $n$, that is
\[
j_1k_1 \equiv 1 \pmod{n},
\]
and likewise $j_2$ is an inverse of $k_2$, then it follows immediately that
\[
(j_2j_1)(k_1k_2) \equiv 1 \pmod{n}.
\]
That is, $k_1k_2$ also has an inverse.  Since we know that $k_1k_2
\equiv k_1 \cdot_n k_2 \pmod{n}$, any inverse of $k_1k_2$ will also be
an inverse of $k_1 \cdot_n k_2$.
\end{solution}

\ppart \dots using the fact that $k$ is relatively prime to $n$ iff
$k$ is cancellable modulo $n$.

\begin{solution}
If $k_1$ and $k_2$ are cancellable modulo $n$, then you can cancel
$k_1k_2$ by first cancelling $k_1$ and then cancelling $k_2$.  Also,
it follows from the Congruence Lemma~\bref{mod_congruence_lem}, the if
$k$ is cancellable than so is anything congruent to $k$ modulo $n$, so
by the previous Hint, $k_1 \cdot_n k_2$ is cancellable.
\end{solution}

\ppart \dots using the Unique Factorization Theorem and the basic GCD
properties\inbook{ such as Lemma~\bref{lem:gcdrem}}.

\begin{solution}
We know that $k$ is relatively prime to $n$ iff $\gcd(k,n)=1$.

By Unique Factorization, the primes divisors of $k_1 \cdot k_2$ are
the same as the prime divisors of $k_1$ or of $k_2$.  If $k_1$ and
$k_2$ are relatively prime to $n$, they have no prime divisors in
common with $n$, then neither does $k_1k_2$, so $k_1k_2$ is relatively
prime to $n$ and $1=\gcd(k_1k_2,n)$.  But $k_1 \cdot_n k_2\eqdef
\rem{k_1k_2,n}$ and $\gcd(n, \rem{k_1k_2,n}) = \gcd(k_1k_2,n)$ by
Lemma~\bref{lem:gcdrem}, so $\gcd(n, \rem{k_1k_2,n}) = 1$.
\end{solution}

\eparts

\end{problem}

\endinput
