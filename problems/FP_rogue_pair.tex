\documentclass[problem]{mcs}

\begin{pcomments}
  \pcomment{FP_rogue_pair}
  \pcomment{excerpted From F07.final, unchanged from S07.final and
    S06.final}
  \pcomment{\textbf{Stable Marriages}}
\end{pcomments}

\pkeywords{
 stable_matching
 stable_marriage
 mating_ritual
 rogue_pair
}

%%%%%%%%%%%%%%%%%%%%%%%%%%%%%%%%%%%%%%%%%%%%%%%%%%%%%%%%%%%%%%%%%%%%%
% Problem starts here
%%%%%%%%%%%%%%%%%%%%%%%%%%%%%%%%%%%%%%%%%%%%%%%%%%%%%%%%%%%%%%%%%%%%%

\begin{problem}
  Four unfortunate children want to be adopted by four foster families of
  ill repute.  A child can only be adopted by one family, and a family can
  only adopt one child.  Here are their preference rankings (most-favored
  to least-favored):
\begin{center}
\begin{tabular}{r|l}
Child & Families \\ \hline
Bottlecap:   & Hatfields, McCoys, Grinches, Scrooges \\
Lucy:     & Grinches, Scrooges, McCoys, Hatfields \\
Dingdong:  & Hatfields, Scrooges, Grinches, McCoys \\
Zippy:    & McCoys, Grinches, Scrooges, Hatfields
\end{tabular}
\end{center}

\begin{center}
\begin{tabular}{r|l}
Family   & Children \\ \hline
Grinches:    & Zippy, Dingdong, Bottlecap, Lucy \\
Hatfields:   & Zippy, Bottlecap, Dingdong, Lucy \\
Scrooges: & Bottlecap, Lucy, Dingdong, Zippy \\
McCoys:   & Lucy, Zippy, Bottlecap, Dingdong
\end{tabular}
\end{center}

\bparts

\ppart\label{childfamtwo}
Exhibit two different stable matching of Children and Families.

\begin{center}
\begin{tabular}{r|c|c|}
Family       &   Child in 1st match & Child in 2nd match\\ \hline
Grinches:    &                      & \\
Hatfields:   &                      & \\
Scrooges:    &                      & \\
McCoys:      &                      &
\end{tabular}
\end{center}

%\examspace[2in]

\begin{solution}
Treat Families as Girls and the result is the following assignment:
\begin{center}
\begin{tabular}{c|c}
Family & Children\\ \hline
Grinches:    & Lucy \\
Hatfields:   & Bottlecap\\
Scrooges: & Dingdong\\
McCoys:   & Zippy
\end{tabular}
\end{center}

Treat Families as Boys and the result is the following assignment:
\begin{center}
\begin{tabular}{c|c}
Family & Children \\ \hline
Grinches:    & Dingdong\\
Hatfields:   & Bottlecap\\
Scrooges: & Lucy\\
McCoys:   & Zippy
\end{tabular}
\end{center}
\end{solution}

\examspace[1in]

\ppart Examine the matchings from part~\eqref{childfamtwo}, and explain
why these matchings are the only two possible stable matchings between
Children and Families.

\hint In general, there may be many more than two stable matchings for
the same set of preferences.

\begin{solution}
The two stable matchings of part~\eqref{childfamtwo} are respectively
Child optimal and Child pessimal.  Since the matchings agree that
Hatfields adopt Bottlecap, this means that the Hatfields are both the
best and worst, and therefore the \emph{only}, possible adoption for
Bottlecap in any stable matching.  Likewise for the McCoys adopting
Zippy.  This leaves only two ways to match up the remaining two
children and families, so the two matchings from
part~\eqref{childfamtwo} are the only ones possible.

\iffalse
An alternate approach was to use proof by contradiction.  Say, for
contradiction, that there is some assignment of children to families
such that Bottlecap does not wind up with the Hatfields, but the
assignment has no rogue pairs.  The Hatfields must get some other
child than Bottlecap in the assignment. We examine the cases:
\begin{enumerate}
\item The Hatfields get Dingdong: The Hatfields prefer Bottlecap over
  Dingdong, and Bottlecap prefers the Hatfields over the other
  families, so Bottlecap and the Hatfields would be a rogue pair.
\item The Hatfields get Lucy: This case is the same as for Dingdong.
\item The Hatfields get Zippy: The Grinches like Zippy better than any
  other child, and Zippy likes anybody better than the Hatfields, so
  Zippy and the Grinches would be a rogue pair.
\end{enumerate}
Hence, any assignment of a child that is not Bottlecap to the
Hatfields results in a rogue pair, a contradiction.
\fi

\end{solution}

\iffalse

Suppose each family adopts one of these children.  Explain
why Bottlecap must be adopted by the Hatfields in any stable matching
of families and children.
\fi

\iffalse
, or else there will
be a \emph{rogue pair}, that is, there will be a child who prefers
another family to his or her adopted family, and that other family
prefers that child to their own.

\fi

\iffalse

\examspace
Suppose an assignment of children to families is made uniformly at
random, with no regard to the preferences.  In other words, each
possible assignment occurs with the same probability.

Let $H$ be the indicator variable for the event that the
\textbf{H}atfields get their first choice.
\iffalse and $L$ be the indicator variable for the event that
\textbf{L}ucy gets her first choice\fi
Similarly define indicator variables $G$, $S$ and $M$ for the other
families
\iffalse and $B$, $D$ and $Z$ for the other children\fi
(according to the first letter in each of their names).

\parbox{5.5in}{
\ppart\label{ranmarr} For a given assignment, let $X$ be the
  number of families that receive their first choice. What is the
  expected value of $X$?
}\qquad\
$\fbox{\makebox[.5in]{\rule[-0.2in]{0in}{0.5in}
\insolutions{\textbf{1}}}}$ \vspace{.2in}

\begin{solution}[\vspace{1in}]
  Note that $X = H + M + G + S$.  But $\expect{H} = \pr{H=1} = 1/4$ since
  each of the four children are equally likely to be assigned to the
  Hatfields.  The same is true of the other three indicator variable in
  the sum.  So by linearity of expectation,
  \[\expect{X} = \expect{H} + \expect{M} + \expect{G} + \expect{S} =
    4\cdot \frac{1}{4} =1.
  \]
\end{solution}

\parbox{5.5in}{
\ppart\label{HG} If an assignment is chosen at random as in
 part~\eqref{ranmarr}, what is the probability that the Hatfields and
 Grinches both get their first choice?
}\qquad\
$\fbox{\makebox[.5in]{\rule[-0.2in]{0in}{0.5in}
\insolutions{\textbf{0}}}}$ \vspace{.2in}

\begin{solution}[\vspace{1in}]
Zippy can be assigned to at most one of
  those families, so both families cannot get their first choice,
  Zippy, at the same time.
\end{solution}

\ppart\label{M} Sometimes the upper bound found using Markov's 
Theorem is tight and sometimes it's not. Calculate the Markov bound 
on $\pr{X \geq 4}$ and comment on the quality of the approximation.

\begin{solution}
By part~\eqref{ranmarr} $\expect{X}=1$ and
  so by Markov's Theorem,
  \[
  \pr{X\geq 4} \leq \frac{\expect{X}}{4}=\frac{1}{4}.
  \]
  As was shown in part (\ref{HG}) it is not possible for all the
  families to get their first choice ($\pr{X\geq 4}=0$) and 
  we conclude that the bound implied by Markov's Theorem is
  not very tight.
\end{solution}
\fi

\eparts

\end{problem}

%%%%%%%%%%%%%%%%%%%%%%%%%%%%%%%%%%%%%%%%%%%%%%%%%%%%%%%%%%%%%%%%%%%%%
% Problem ends here
%%%%%%%%%%%%%%%%%%%%%%%%%%%%%%%%%%%%%%%%%%%%%%%%%%%%%%%%%%%%%%%%%%%%%

\endinput
