v\documentclass[problem]{mcs}

\begin{pcomments}
  \pcomment{revision of PS_50_point_games by ARM 3/23/13}
\end{pcomments}

\pkeywords{
  recursive_data
  structural_induction
  games
  perfect_information
  max-value
  min-value
}

%%%%%%%%%%%%%%%%%%%%%%%%%%%%%%%%%%%%%%%%%%%%%%%%%%%%%%%%%%%%%%%%%%%%%
% Problem starts here
%%%%%%%%%%%%%%%%%%%%%%%%%%%%%%%%%%%%%%%%%%%%%%%%%%%%%%%%%%%%%%%%%%%%%

\newcommand{\VG}{\ensuremath{\text{VG}}}

\begin{problem}
We're going to characterize a large category of games as a recursive
data type and then prove, by structural induction, a fundamental
theorem about game strategies.  The games we'll consider are known as
\index{deterministic games|textbf}{\emph{deterministic}} \term{games of
  perfect information}, because at each move, the complete game
situation is known to the players, and this information completely
determines how the rest of the game can be played.  Games like chess,
checkers, GO, and tic-tac-toe fit this description.  In contrast, most
card games do not fit, since card players usually do not know exactly
what cards belong to the other players.  Neither do games involving
random features like dice rolls, since a player's move does not
uniquely determine what happens next.

Chess counts as a deterministic game of perfect information because
at any point of play, both players know whose turn it is to move and
the location of every chess piece on the board.\footnote{In order to
  prevent the possibility of an unending game, chess rules specify a
  limit on the number of moves, or a limit on the number of times a
  given board postion may repeat.  So the number of moves or the
  number of position repeats would count as part of the game situation
  known to both players.}  At the start of the game, there are 20
possible first moves: the player with the White pieces can move one of
his eight pawns forward 1 or 2 squares or one of his two knights
forward and left or forward and right.  For the second move, the Black
player can make one of the 20 corresponding moves of his own pieces.
The White player would then make the third move, but now the number of
possible third moves depends on what the first two moves happened to
be.

A nice way to think of these games is to regard each game situation as
a game in its own right.  For example, after five moves in a chess
game, we think of the players as being at the start of a new ``chess''
game determined by the current board position and the fact that it is
Black's turn to make the next move.

At the end of a chess game, we might assign a score of 1 if the White
player won, $-1$ if White lost, and 0 if the game ended in a stalemate
(a tie).  Now we can say that White's objective is to maximize the
final score and Black's objective is to minimize it.  We might also
choose to score the game in a more elaborate way, taking into account
not only who won, but also how many moves the game took, or the final
board configuration.

This leads to an elegant abstraction of this kind of game.  We suppose
there are two players, called the \emph{max-player} and the
\emph{min-player}, whose aim is, respectively, to maximize and
minimize the final score.  A game will specify its set of possible
first moves, each of which will simply be another game.  A game with
no possible moves is called an \emph{ended game}, and will just have a
final score.  Strategically, all that matters about an ended game is
its score.  If a game is not ended, it will have a label \texttt{max}
or \texttt{min} indicating which player is supposed to move first.

This motivates the following formal definition:

\begin{definition*}%\label{def:vg}
Let $V$ be a nonempty set of real numbers.  The class \VG\ of
\emph{$V$-valued} \emph{deterministic max-min games} \emph{of}
\emph{perfect information} is defined recursively as follows:

\inductioncase{Base case}:
A value $v \in V$ is a \VG, and is called an \emph{ended game}.

\inductioncase{Constructor case}: If $\set{G_0,G_1,\dots}$ is a
  nonempty set of \VG's, and $a$ is a label equal to \texttt{max} or
  \texttt{min}, then
\[
G \eqdef (a, \set{G_0,G_1,\dots})
\]
is a $\VG$.  Each game $G_i$ is called a possible \emph{first move} of
$G$.
\end{definition*}

In all the games like this that we're familiar with, there are only a
finite number of possible first moves.  It's worth noting that the
definition of \VG\ does not require this.  Since finiteness is not
needed to prove any of the results below, it would arguably be
misleading to assume it.  Later, we'll suggest how games with an
infinite number of possible first moves might come up.

A \emph{play} of a game is a sequence of legal moves that either goes
on forever or finishes with an ended game.  More formally:

\begin{definition*}%\label{def:play}
A \emph{play} of a game $G \in \VG$ is defined recursively on the
definition of $\VG$:

\inductioncase{Base case}: ($G$ is an ended game.)
Then the length one sequence $(G)$ is a \emph{play} of $G$.

\inductioncase{Constructor case}: ($G$ is not an ended game.)
  Then a \emph{play} of $G$ is a sequence that starts with a possible first
  move, $G_i$, of $G$ and continues with the elements of a play of $G_i$.

If a play does not go on forever, its \emph{payoff} is defined to be
the value it ends with.
\end{definition*}

Let's first rule out the possibility of playing forever.  Namely,
every play will have a payoff.

\bparts

\ppart Prove that every play of a $G \in \VG$ is a finite sequence that
ends with a value in $V$.  \hint By structural induction on the
definition of \VG.

\begin{solution}
We prove by structural induction on $G \in \VG$ that every play of $G$
is finite and ends with a value.

\inductioncase{Base case}: [$G$ is the ended game $v \in V$.]  Then
there is only one play of $G$, namely the length one play $(v)$, which
certainly ends with a value.

\inductioncase{Constructor case}: [$G = (a, \set{G_0,G_1,\dots})$.]  A
play of $G$ by definition consists of a sequence that starts with some
first move $G_j$ and continues with a play of $G_j$.  By structural
induction, we know that this play of $G_j$ is a sequence of some
finite length $n$ that ends with a value, so this play of $G$ is a
length $n+1$ sequence that ends with the same value.
\end{solution}

\eparts

A \emph{strategy} for a game is a rule that tells a player which move
to make when it's his turn.  Formally:
\begin{definition*}
If $a$ is one of the labels \texttt{max} or \texttt{min}, then an
$a$-\emph{strategy} is a function $s:\VG \to \VG$ such that
\[
s(G)\text{ is } \begin{cases}
                \text{a first move of $G$} & \text{if $G$ has label $a$,}\\
                \text{undefined}, & \text{otherwise.}
                \end{cases}
\]
\end{definition*}

Any pair of strategies for the two players determines a unique play of
a game, and hence a unique payoff, in an obvious way.  Namely, when it
is a player's turn to move in a game $G$, he chooses the move
specified by his strategy.  A strategy for the max-player is said to
\emph{ensure} payoff $v$ when, paired with \emph{any} strategy for the
min-player, the resulting payoff is \emph{at least} $v$.  Dually, a
strategy for the min-player \emph{caps} payoff at $v$ when, paired
with any strategy for the max-player, the resulting payoff is \emph{at
  most} $v$.

\begin{editingnotes}
\begin{definition}
If $s_a$ and $s_b$ are strategies for labels $a \neq b$, and $G \in
\VG$, then the play, $p$, of $G$ they determine is defined recursively on the
definition of $\VG$:

\inductioncase{Base case}: If $G$ is an \emph{ended game}, then $p$ is
the unique play of $G$.

\inductioncase{Constructor case}: If $G$ is not an ended game, and the
label of $G$ is $c$, then $p$ is the play that starts with $s_c(G)$
followed by the elements in the play of $s_c(G)$ determined by the two
strategies.
\end{definition}
\end{editingnotes}

Assuming for simplicity that the set $V$ of possible values of a game
is finite, the WOP (Section~\bref{well_ordering_sec}) implies there
will be a strategy for the max-player that ensures the largest
possible payoff; this is called the \emph{max-ensured-value} of the
game.  Dually, there will also be a strategy for the min-player that
caps the payoff at the smallest possible value, which is called the
\emph{min-capped-value} of the game.

The max-ensured-value of course cannot be larger than the
min-capped-value.  A unique value can be assigned to a game when these
two values agree:
\begin{definition*}
If the max-ensured-value and min-capped-value of a game are equal, their
common value is called the \emph{value of the game}.
\end{definition*}

So if both players play optimally in a game with that has a value,
$v$, then there is actually no point in playing.  Since the payoff is
ensured to be at least $v$ and is also capped to be at most $v$, it
must be exactly $v$.  So the min-player may as well skip playing and
simply pay $v$ to the max-player (a negative payment means the
max-player is paying the min-player).

The punch line of our story is that the max-ensured-value and the
min-capped-value are \emph{always} equal.

\begin{theorem*}[Fundamental Theorem for Deterministic Min-Max
Games of Perfect Information]
Let $V$ be a finite set of real numbers.  Every $V$-valued
deterministic max-min game of perfect information has a value.
\end{theorem*}

\bparts

\ppart Prove this Fundamental Theorem for \VG's by structural
induction.

\begin{solution}
The proof is by structural induction on the definition of a $G \in
\VG$.  The induction hypothesis is that there is that
\begin{quote}
  $G$'s max-ensured-value equals $G$'s min-capped-value.
\end{quote}

\inductioncase{Base case}: [$G$ is the ended game $v \in V$.]  The
only possible play is $(v)$.  So the max-ensured-value and the
min-capped-value are both $v$.

\inductioncase{Constructor case}: [$G = (a, \set{G_0,G_1,\dots})$].  By
structural induction we may assume that the max-ensured-value and
min-capped-value are equal for each of the possible first moves $G_i$.

\inductioncase{Case} ($a = \texttt{max}$.)  Let $v$ be the maximum of
the values of the first moves of $G$.  This max will exist because $V$
is assumed to be finite.

Now a strategy for the max-player that ensures $v$ is:
\begin{quote}
Choose a first move, $G_i$, that has the maximum value $v$, and then
follow the strategy that ensures $v$ for $G_i$.
\end{quote}

Dually, a strategy whereby the min-player can cap payoff at $v$ is:
\begin{quote}
Let $G_j$ be whatever first move is chosen by the max-player, and let
$w$ be the value of $G_j$.  Now follow the strategy in $G_j$ that caps
the value at $w$.  But $w \leq v$ since $v$ is the maximum value among
the first moves, so this strategy will also cap the value at $v$.
\end{quote}
So $v$ can be both ensured and capped at, and hence it is the value of
$G$.

\inductioncase{Case} ($a = \texttt{min}$):
Let $v$ be the minimum of the values of the next moves of $G$.  This
minimum exists for the same reasons as in the previous case.

Now a strategy for the min-player to cap payoff to $v$ is to choose a
first move, $G_i$, with value $v$, and then to follow the strategy in
$G_i$ that caps the payoff at $v$.

Dually, a strategy whereby the max-player can ensure payoff $v$ is to
let $G_j$ be whatever first move is chosen by the min player and then
to follow the strategy in $G_j$ that ensures the value of $G_j$, which
is at least $v$.  

So $v$ can be both capped at and ensured, and hence is the value of
$G$.

So in any case, the game $G$ has a value, which completes the
constructor case of the structural induction.
\end{solution}

\ppart Conclude immediately that in chess, there is a winning strategy
for White, or a winning strategy for Black, or both players have
strategies that guarantee at least a stalemate.  (The only difficulty
is that no one knows which case holds.)

\begin{solution}
By the fundamental theorem, the value of chess must be 1, $-1$, or 0.
\end{solution}
\eparts

So where do we come upon games with an infinite number of first
moves?  Well, suppose we play a tournament of $n$ chess games for some
positive integer $n$.  This tournament will be a \VG\ if we agree on a
rule for combining the payoffs of the $n$ individual chess games into
a final payoff for the whole tournament.

There still are only a finite number of possible moves at any stage of
the $n$-game chess tournament, but we can define a
\emph{meta-chess-tournament}, whose first move is a choice of any
positive integer $n$, after which we play an $n$-game tournament.  Now
the meta-chess-tournament has an infinite number of first moves.

Of course only the first move in the meta-chess-tournament is
infinite, but then we could set up a tournament consisting of $n$
meta-chess-tournaments.  This would be a game with $n$ possible
infinite moves.  And then we could have a
\emph{meta-meta}-chess-tournament whose first move was to choose how
many meta-chess-tournaments to play.  This meta-meta-chess-tournament
will have an infinite number of infinite moves.  Then we could move on
to meta-meta-meta-chess-tournaments \dots.

As silly or weird as these meta games may seem, their weirdness
doesn't disqualify the Fundamental Theorem: each of these games will
still have a value.

\bparts

\ppart State some reasonable generalization of the Fundamental Theorem
to games with an infinite set $V$ of possible payoffs.
\emph{Optional}: Prove your generalization.

\begin{solution}
The obvious generalization would redefine the max-value as the lub of
the ensured values, and the min-value as the glb of the limits to
payoffs.  The result is that some games may now have a value $v$ that
is positive or negative infinity, and that $v$ can't exactly be
ensured or limted to, but rather that for any $\epsilon >0$ there will
be a strategy that ensures a value of at least $v - \epsilon$ and a
strategy that limits payoff to at most $v + \epsilon$.
\end{solution}

\eparts

\end{problem}

%%%%%%%%%%%%%%%%%%%%%%%%%%%%%%%%%%%%%%%%%%%%%%%%%%%%%%%%%%%%%%%%%%%%%
% Problem ends here
%%%%%%%%%%%%%%%%%%%%%%%%%%%%%%%%%%%%%%%%%%%%%%%%%%%%%%%%%%%%%%%%%%%%%
\endinput
