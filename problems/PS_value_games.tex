v\documentclass[problem]{mcs}

\begin{pcomments}
  \pcomment{PS_value_games}
  \pcomment{revision of PS_50_point_games by ARM 3/23/13}
\end{pcomments}

\pkeywords{
  recursive_data
  structural_induction
  games
  perfect_information
  max-value
  min-value
}

%%%%%%%%%%%%%%%%%%%%%%%%%%%%%%%%%%%%%%%%%%%%%%%%%%%%%%%%%%%%%%%%%%%%%
% Problem starts here
%%%%%%%%%%%%%%%%%%%%%%%%%%%%%%%%%%%%%%%%%%%%%%%%%%%%%%%%%%%%%%%%%%%%%

\newcommand{\VG}{\ensuremath{\text{VG}}}

\begin{problem}
We're going to characterize a large category of games as a recursive
data type and then prove, by structural induction, a fundamental
theorem about game strategies.  The games we'll consider are known as
\index{deterministic games|textbf}{\emph{deterministic}} \term{games of
  perfect information}, because at each move, the complete game
situation is known to the players, and this information completely
determines how the rest of the game can be played.  Games like chess,
checkers, GO, and tic-tac-toe fit this description.  In contrast, most
card games do not fit, since card players usually do not know exactly
what cards belong to the other players.  Neither do games involving
random features like dice rolls, since a player's move does not
uniquely determine what happens next.

\iffalse
Chess counts as a deterministic game of perfect information because
at any point of play, both players know whose turn it is to move and
the location of every chess piece on the board.\footnote{In order to
  prevent the possibility of an unending game, chess rules specify a
  limit on the number of moves, or a limit on the number of times a
  given board postion may repeat.  So the number of moves or the
  number of position repeats would count as part of the game situation
  known to both players.}  At the start of the game, there are 20
possible first moves: the player with the White pieces can move one of
his eight pawns forward 1 or 2 squares or one of his two knights
forward and left or forward and right.  For the second move, the Black
player can make one of the 20 corresponding moves of his own pieces.
The White player would then make the third move, but now the number of
possible third moves depends on what the first two moves happened to
be.
\fi

A nice way to think of these games is to regard each game situation as
a game in its own right.  For example, after five moves in a chess
game, we think of the players as being at the start of a new ``chess''
game determined by the current board position.\iffalse
 and the fact that it is
Black's turn to make the next move.\fi

At the end of a chess game, let's assign a score of 1 if the White
player won, $-1$ if White lost, and 0 if the game ended in a stalemate
(a tie).  This means that White's objective is to maximize the final
score, and Black's objective is to minimize it.  We might also choose
to score the game in a more elaborate way, taking into account
additional information such as what pieces they had won.

This leads to an elegant abstraction of this kind of game.  We suppose
there are two players, called the \emph{max-player} and the
\emph{min-player}, whose aim is, respectively, to maximize and
minimize the final score.  A game will specify its set of possible
first moves, each of which will simply be another game.  A game with
no possible moves is called a \emph{finish}.  Finishes will have final
scores indicating how much was won.  If a game is not ended, it will
have a label \STR{max} or \STR{min} indicating which player is
supposed to make the next move.

Here's the formal definition:

\begin{definition*}%\label{def:vg}
Let $V$ be a nonempty set of real numbers.  The class \VG\ of
\emph{$V$-valued} \emph{deterministic max-min games} \emph{of}
\emph{perfect information} is defined recursively as follows:

\inductioncase{Base case}:
A value $v \in V$ is a \VG\ known as a \emph{finish}.

\inductioncase{Constructor case}: If $\mathcal{M}$ is a
  nonempty set of \VG's, and $a$ is a label equal to \STR{max} or
  \STR{min}, then
\[
(a, \mathcal{M})
\]
is a $\VG$.  Each game $M \in \mathcal{M}$ is called a possible
\emph{first move} of the \VG.
\end{definition*}

\iffalse
In all the games like this that we're familiar with, there are only a
finite number of possible first moves.  It's worth noting that the
definition of \VG\ does not require this.  Since finiteness is not
needed to prove any of the results below, it would arguably be
misleading to assume it.  Later, we'll suggest how games with an
infinite number of possible first moves might come up.
\fi

A \emph{play} of a game is finite sequence of legal moves coming to a
finish, or it can go on forever without finishing.  More formally:

\begin{definition*}%\label{def:play}
A \emph{play} of a \VG, $G$, is defined recursively on the definition
of $\VG$:

\inductioncase{Base case}: ($G = v \in V$ is a finish.)
Then the length one sequence $(v)$ is a \emph{play} of $G$.

\inductioncase{Constructor case}: ($G = (a, \mathcal{M})$) Then a
\emph{play} of $G$ is a sequence that starts with a possible first
move, $M \in \mathcal{M}$, of $G$ and continues with the elements of a
play of $M$.

If a play does not go on forever, its \emph{payoff} is defined to be
the value it finishes with.
\end{definition*}

The basic rules of some games do allow plays that go on forever.  In
chess for example, a player might just moving the same piece back and
forth, and if his opponent did the same, the play could go on forever.
Real chess tournaments rule this out by setting an advance limit on
the number of moves, or by forbidding repetitions of the same position
more than twice.  But the recursive definition of\VG's actually rules
out the possibility of infinite play.

\begin{lemma*}
Every play of a \VG\ is a finite sequence finishing with a value.
\end{lemma*}

The Lemma has an amazingly simple proof by structural induction.

\begin{proof}
We prove by structural induction on $G \in \VG$ that every play of $G$
is finite and ends with a value.

\inductioncase{Base case}: [$G= v \in V$.]  Then there is only one
play of $G$, namely the length one play $(v)$, which ends with value
$v$.

\inductioncase{Constructor case}: [$G = (a, \mathcal{M})$.]  A play of
$G$ by definition consists of a sequence that starts with some first
move $M \in \mathcal{M}$ and continues with a play of $M$.  By
structural induction, this play of $M$ is a finite sequence of
some length $n$ that finishes value $v$.   So this play of $G$ is
a length $n+1$ sequence that finishes with $v$.
\end{proof}

Notice that the proof of the Lemma above did \emph{not} assume that
the number of next moves was always finite.  The proof shows that even
infinite \VG's cannot have infinite plays.  We're not going to run
into infinite games much in Discrete Math, it would be misleading to
add an irrelevant finiteness assumption to the
proof.\footnote{Infinite games are useful in set theory and logic.}

\begin{staffnotes}
An infinite \VG might have plays of every finite length, but it won't
have an infinite play.
\end{staffnotes}

An strategy for a player is a rule that tells the player which move to
make when it's their turn.  Matching up any \STR{max}-strategy with a
\STR{min}-strategy will then determine a unique play for
each \VG.

\begin{staffnotes}
More precisely, let $a$ be one of the
labels \STR{max} or \STR{min}.  An $a$-\emph{strategy} is a function
$s:\VG \to \VG$ such that
\[
s((a,\mathcal{M}) \in \mathcal{M}.
\]
\end{staffnotes}

\begin{staffnotes}
Namely, when it is a player's turn to move in a game $G$, he chooses
the move $s(G)$ specified by his strategy.
\end{staffnotes}

\begin{editingnotes}
A strategy for the max-player is said to \emph{ensure} payoff $v$
when, paired with \emph{any} strategy for the min-player, the
resulting payoff is \emph{at least} $v$.  Likewise, a strategy for the
min-player \emph{caps} payoff at $v$ when, paired with any strategy
for the max-player, the resulting payoff is \emph{at most} $v$.

\begin{definition}
If $s_a$ and $s_b$ are strategies for labels $a \neq b$, and $G \in
\VG$, then the play, $p$, of $G$ they determine is defined recursively on the
definition of $\VG$:

\inductioncase{Base case}: If $G= v$, then $(v)$ is
the unique play of $G$.

\inductioncase{Constructor case}: If $G$ is not an ended game, and the
label of $G$ is $c$, then $p$ is the play that starts with $s_c(G)$
followed by the play of $s_c(G)$ determined by the two strategies.
\end{definition}

Assuming for simplicity that the set $V$ of possible values of a game
is finite,\iffalse the WOP (Section~\bref{well_ordering_sec})
implies\fi there must be a largest value that the max-player can
ensure.  \iffalse this is called the \emph{max-ensured-value} of the
game.\fi Likewise, there must also be smallest possible cap the
min-player can guarantee.
\end{editingnotes}

\iffalse

which is called the \emph{min-capped-value} of the game.

The max-ensured-value of course cannot be larger than the
min-capped-value.  A unique value can be assigned to a game when these
two values agree:
\begin{definition*}
If the max-ensured-value and min-capped-value of a game are equal, their
common value is called the \emph{value of the game}.
\end{definition*}

So if both players play optimally in a game with that has a value, $v$, then
there is actually no point in playing.  Since the payoff is ensured to be at
least $v$ and is also capped to be at most $v$, it must be exactly $v$.  So
the min-player may as well skip playing and simply pay $v$ to the max-player
(a negative payment means the max-player is paying the min-player).

The punch line of our story is that the max-ensured-value and the
min-capped-value are \emph{always} equal.
\fi

\begin{theorem*}[Fundamental Theorem for \VG's]
Let $V$ be a finite set of real numbers and $G$ be a \VG.  Then there is a
value $v \in V$ is called \emph{the value of $G$} such that
\begin{itemize}
\item the max-player has a strategy that, matched with \emph{any}
  min-player strategy, will define a play of $G$ that finishes with a
  value of at least $v$,

\item the min-player has a strategy that, matched with \emph{any}
  max-player strategy, will define a play of $G$ that finishes with a
  value of at most $v$.
\end{itemize}
\end{theorem*}

\textbf{Prove Fundamental Theorem for \VG's by structural induction.}

\hint You can assume by induction that each first move $M$ in a game $G$ has
a value $v_M$.

\begin{solution}
The proof is by structural induction on the definition of a $G \in
\VG$.  The induction hypothesis is that there $G$ has a value.

\inductioncase{Base case}: [$G = v \in V$.]  Then all strategies finish with
the value $v$, so $v$ is the value of $G$.

\inductioncase{Constructor case}: [$G = (a, \mathcal{M})$].  By structural
induction we may assume that each $M \in \mathcal{M}$ has a value $v_M$.

\inductioncase{Case} ($a = \STR{max}$.)  Let 
\[
v \eqdef \max{v_M \suchthat M \in \mathcal{M}}.
\]
This max will exist because $V$ is finite.

Now a strategy for the max-player that finishes with a value $\ge v$ is:
\begin{quote}
Choose a first move, $M$ such that $v = v_M$.  Now by structural induction
hypothesis, there is a max-strategy for $M$ that guarantees a finish with a
value of at least $v$.  Follow that strategy.
\end{quote}

Similarly, a strategy for the min-player that finishes with a value $\leq v$ is:
\begin{quote}
Let $M$ be whatever first move is chosen by the max-player.  Now by
structural induction hypothesis, there is a min-strategy for $M$ that
guarantees a finish with a value of at most $v$.  Follow that strategy.
\end{quote}

\inductioncase{Case} ($a = \STR{min}$):
Let 
\[
v \eqdef \min{v_M \suchthat M \in \mathcal{M}}.
\]
This min will exist because $V$ is finite.

Now a strategy for the min-player that always finishes with a value $\leq v$
is:
\begin{quote}
Choose a first move, $M$ such that $v=v_M$. Now by structural induction
hypothesis, there is a min-strategy for $M$ that guarantees a finish with a
value $\leq v$.  Follow that strategy.
\end{quote}

Similarly, a strategy for the max-player that finishes with a value $\ge v$
is:
\begin{quote}
Let $M$ be whatever first move is chosen by the min-player.  Now by
structural induction hypothesis, there is a max-strategy for $M$ that
guarantees a finish with a value $\geq v$.  Follow that strategy.
\end{quote}

So in any case, $G$ has a value, which completes the constructor case of the
structural induction.
\end{solution}

\iffalse

\ppart Conclude immediately that in chess, there is a winning strategy
for White, or a winning strategy for Black, or both players have
strategies that guarantee at least a stalemate.  (The only difficulty
is that no one knows which case holds.)

\begin{solution}
By the fundamental theorem, the value of chess must be 1, $-1$, or 0.
\end{solution}
\eparts

So where do we come upon games with an infinite number of first
moves?  Well, suppose we play a tournament of $n$ chess games for some
positive integer $n$.  This tournament will be a \VG\ if we agree on a
rule for combining the payoffs of the $n$ individual chess games into
a final payoff for the whole tournament.

There still are only a finite number of possible moves at any stage of
the $n$-game chess tournament, but we can define a
\emph{meta-chess-tournament}, whose first move is a choice of any
positive integer $n$, after which we play an $n$-game tournament.  Now
the meta-chess-tournament has an infinite number of first moves.

Of course only the first move in the meta-chess-tournament is
infinite, but then we could set up a tournament consisting of $n$
meta-chess-tournaments.  This would be a game with $n$ possible
infinite moves.  And then we could have a
\emph{meta-meta}-chess-tournament whose first move was to choose how
many meta-chess-tournaments to play.  This meta-meta-chess-tournament
will have an infinite number of infinite moves.  Then we could move on
to meta-meta-meta-chess-tournaments \dots.

As silly or weird as these meta games may seem, their weirdness
doesn't disqualify the Fundamental Theorem: each of these games will
still have a value.

\bparts

\ppart State some reasonable generalization of the Fundamental Theorem
to games with an infinite set $V$ of possible payoffs.
\emph{Optional}: Prove your generalization.

\begin{solution}
The obvious generalization would redefine the max-value as the lub of
the ensured values, and the min-value as the glb of the limits to
payoffs.  The result is that some games may now have a value $v$ that
is positive or negative infinity, and that $v$ can't exactly be
ensured or limted to, but rather that for any $\epsilon >0$ there will
be a strategy that ensures a value of at least $v - \epsilon$ and a
strategy that limits payoff to at most $v + \epsilon$.
\end{solution}

\eparts
\fi

\end{problem}

%%%%%%%%%%%%%%%%%%%%%%%%%%%%%%%%%%%%%%%%%%%%%%%%%%%%%%%%%%%%%%%%%%%%%
% Problem ends here
%%%%%%%%%%%%%%%%%%%%%%%%%%%%%%%%%%%%%%%%%%%%%%%%%%%%%%%%%%%%%%%%%%%%%
\endinput
