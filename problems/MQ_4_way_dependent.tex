\documentclass[problem]{mcs}

\begin{pcomments}
  \pcomment{MQ_4_way_dependent}
  \pcomment{new in spring 11; edited ARM 5/2/11}
  \pcomment{some of the text has references to other spring11 problems}
\end{pcomments}

\pkeywords{
  independence
  pairwise_independence
  distribution
}

\begin{problem}

Suppose there are 4 desks in a classroom, laid out in the corners of a square
with corners $1$ $2$ $3$ and $4$. 

Each desk is occupied by a male with probability $p>0$ or a female
with probability $q \eqdef 1-p > 0$.  A male and a female \emph{flirt}
when they occupy desks in adjacent corners of the square.  Let
$I_{12},I_{23},I_{34},I_{41}$ be the indicator variables of  whether there is a
flirting couple at the indicated adjacent desks.

\bparts

\ppart Show rigorously that $I_{12}$ and $I_{23}$ are independent iff $p = q$. Hint: work from the definitions.
\examspace[5in]

\begin{solution}

We can compare $\pr{I_{12} = 1 \&\ I_{23} = 1}$ and 
 $\pr{I_{12} = 1}\cdot \pr{I_{23} = 1}$. 

$I_{12} = 1 \&\ I_{23} = 1$ only happen when we have a pattern of F-M-F or M-F-M for students
1 2 and 3 respectively. These occur with total probablity $p^2q + pq^2$. On the other hand, $I_{12}$ happens with
probability $2pq$ total, accouting for the two patterns possible, M-F and F-M. Hence,
 $I_{12}$ and $I_{23}$ are independent  iff $p^2q + pq^2 = pq(p+q) = 4p^2q^2$. By manipulating 
the expression we get $p+q = 4pq$. Recall $p+q = 1$. Hence, we are dealing with $1 = 4p - 4p^2$.
The equation can be factored into $(2p - 1)^2 = 0$, yielding $p = 1/2$.

\end{solution}

\ppart What is the expected number of flirting couples in terms of $p$ and $q$?
\examspace[2in]
\begin{solution}
The expected number of couples is $8pq$ by linearity of expectation.
\end{solution}

\iffalse
\ppart What is the probability distribution function for the random variable $N$, the number
of flirting pairs.
\begin{solution}
$p_N(0) = 2/16, p_N(1) = 0 p_N(2) = 12/16, p_N(3) = 0, p_N(4) = 2/16$
\end{solution}
\fi

\eparts
\end{problem}

\endinput
