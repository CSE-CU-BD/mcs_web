\documentclass[problem]{mcs}

\begin{pcomments}
   \pcomment{TP_bogus_well_ordering_fibonacci_proof}
   \pcomment{Converted from bogus-WOP.scm by scmtotex and dmj
             on Sat 12 Jun 2010 09:47:19 PM EDT}
   \pcomment{edited for repository by ARM 7/7/10}
\end{pcomments}

\pkeywords{
  well_ordering
  bogus
  false_theorem
  Fibonacci
}

\begin{problem}
The \term{Fibonacci numbers} $F(0),F(1), F(2),\dots$ are defined as follows:
\begin{align}
F(0) & \eqdef 0,\notag\\%\label{wopfibdef0}\\
F(1) & \eqdef 1,\notag\\%\label{wopfibdef1}\\
F(n) & \eqdef F(n-1) + F(n-2) &  \text{for } n \ge  2.\label{wopfibdefeqn}
\end{align}

Exactly which sentence(s) in the following bogus proof contain
logical errors?  Explain.

\begin{falseclm*}
Every Fibonacci number is even.
\end{falseclm*}

\begin{bogusproof}
  Let all the variables $n,m,k$ mentioned below be nonnegative integer
  valued.
\begin{enumerate}

\item The proof is by the WOP.

\item Let $\Even(n)$ mean that $F(n)$ is even.

\item
Let $C$ be the set of counterexamples to the assertion that
$\Even(n)$ holds for all $n \in  \naturals$, namely,
\[
C \eqdef \set{n \in \naturals \suchthat  \QNOT(\Even(n)) }.
\]

\item We prove by contradiction that $C$ is empty.  So assume that $C$ is
  not empty.

\item By WOP, there is a least nonnegative integer, $m \in C$,

\item Then $m > 0$, since $F(0) = 0$ is an even number.

\item  Since $m$ is the minimum counterexample, $F(k)$ is even for
  all $k<m$.

\item\label{fm1fm2even} In particular, $F(m-1)$ and $F(m-2)$ are both
  even.

\item
But by the defining equation~\eqref{wopfibdefeqn}, $F(m)$ equals the sum
$F(m-1) + F(m-2)$ of two even numbers, and so it is also even.

\item That is, $\Even(m)$ is true.

\item This contradicts the condition in the definition of $m$ that
$\QNOT(\Even(m))$ holds.

\item This contradition implies that $C$ must be empty.  Hence, $F(n)$ is
  even for all $n \in \naturals$.

\end{enumerate}

\end{bogusproof}

\begin{solution}
  The error is in step~\ref{fm1fm2even}.  The mistake is in assuming
  that equation~\eqref{wopfibdefeqn} applies to $m$.
  But~\eqref{wopfibdefeqn} only applies if $m \ge 2$.  All that's been
  proved previously is that $m>0$, so the proof didn't cover the case
  when $m=1$.  And of course 1 is in $C$, so no contradiction will
  come from assuming $m \in C$.

\begin{staffnotes}
Saying that step~7 contains a logical error is on the
right track.  The natural place to handle the case $F(1)$ would have
been right after line~6.  But the proof explicitly avoided the
case $m = 1$, by saying, ``suppose $m \ge 2$.''

Technically, there is no \emph{logical} error in line~7: it is simply
the beginning of a proof for the case when $m \ge 2$.  On the other
hand, it's reasonable to say that line~7 is the place where the proof
makes an \emph{organizational}, or perhaps \emph{strategic}, error
because it skips the $m = 1$ case.  This distinction between
logical errors and strategic errors which will cause a later difficulty is
a useful one to recognize.
\end{staffnotes}

\end{solution}

\end{problem}

\endinput
