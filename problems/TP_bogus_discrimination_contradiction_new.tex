\documentclass[problem]{mcs}

\begin{pcomments}
  \pcomment{TP_bogus_discrimination_contradiction}
  \pcomment{renamed from PS_}
  \pcomment{revised ARM 11/10/15}
\end{pcomments}

\pkeywords{
  conditional_probability
  paradox
  Simpson
}

%%%%%%%%%%%%%%%%%%%%%%%%%%%%%%%%%%%%%%%%%%%%%%%%%%%%%%%%%%%%%%%%%%%%%
% Problem starts here
%%%%%%%%%%%%%%%%%%%%%%%%%%%%%%%%%%%%%%%%%%%%%%%%%%%%%%%%%%%%%%%%%%%%%

\begin{problem}
\inhandout{Define the following events as in Section~\bref{discrimination_subsec}:
%
\begin{itemize}
\item $A \eqdef$ a random applicant is admitted,
\item $F_{EE} \eqdef$ the applicant is a woman for the EE department,
\item $F_{CS} \eqdef$ the applicant is a woman for the CS department,
\item $M_{EE} \eqdef$ the applicant is a man for the EE department,
\item $M_{CS} \eqdef$ the applicant is a man for the CS department,
\end{itemize}
where all applicants are assumed to be either men or women, and no
applicants is in both departments.}

\inbook{Define the events $A, F_{EE}, F_{CS}, M_{EE}$, and $M_{CS}$ as
  in Section~\bref{discrimination_subsec}.}

In these terms, the plaintiff in a discrimination suit against a
university makes the argument that in both departments, the
probability that a female is admitted is less than the probability for
a male.  That is,
\begin{align}
\prcond{A}{F_{EE}} & < \prcond{A}{M_{EE}} \quad\text{and}\label{AFEE<}\\
\prcond{A}{F_{CS}} & < \prcond{A}{M_{CS}}.\label{AFCS<}
\end{align}

The university's defence attorneys retort that \emph{overall}, a female
applicant is \emph{more} likely to be admitted than a male,
namely, that
\begin{equation}\label{AFEEunion}
    \prcond{A}{F_{EE} \union F_{CS}} > \prcond{A}{M_{EE} \union M_{CS}}.
\end{equation}

\bparts

\ppart Are the two statements contradictory? If they are contradictory, provide an explanation why. 
Otherwise give an example where both claims can be true.
\begin{solution}
They are not contradictory. This is an example of Simpson's paradox.
\end{solution}

A 6.042 TA makes a bogus claim that 
\begin{falseclm*}
If $B$ and $C$ are disjoint events, then for any event $A$,
\begin{equation}\label{backwardssum}
\prcond{A}{B \union C} = \prcond{A}{B} + \prcond{A}{C}.
\end{equation}
\end{falseclm*}

\ppart How would you revise the right hand side of the TA's equation to make it true?
\begin{solution}
$\prcond{A}{B \union C} = \prcond{A}{B}\prcond{B}{B \union C} + \prcond{A}{C}\prcond{C}{B \union C}.$
\end{solution} 
\eparts
\end{problem}

%%%%%%%%%%%%%%%%%%%%%%%%%%%%%%%%%%%%%%%%%%%%%%%%%%%%%%%%%%%%%%%%%%%%%
% Problem ends here
%%%%%%%%%%%%%%%%%%%%%%%%%%%%%%%%%%%%%%%%%%%%%%%%%%%%%%%%%%%%%%%%%%%%%

\endinput
