\documentclass[problem]{mcs}

\begin{pcomments}
  \pcomment{PS_relational_properties_proofs}
  \pcomment{some overlap with PS_function_composition}
  \pcomment{from: F09.ps2, S03.ps5, S15.ps3}
  \pcomment{ARM deleted ``finite'' modifying ``sets'' 9/17/11}
\end{pcomments}

\pkeywords{
  relations
  relational_properties
  mapping_lemma
  functions
  injections
  surjections
  bijections
}

%%%%%%%%%%%%%%%%%%%%%%%%%%%%%%%%%%%%%%%%%%%%%%%%%%%%%%%%%%%%%%%%%%%%%
% Problem starts here
%%%%%%%%%%%%%%%%%%%%%%%%%%%%%%%%%%%%%%%%%%%%%%%%%%%%%%%%%%%%%%%%%%%%%

\begin{problem}

Let $A$, $B$ and $C$ be sets, and let $f : B \to C$ and $g : A \to B$
be functions.  Let $h:A \to C$ be the composition $f \compose g$;
that is, $h(x) \eqdef f(g(x))$ for $x \in A$.  Prove or disprove the
following claims:

\inhandout{ \hint Arguments based on ``arrows'' using
  Definition~\bref{archery-def} are fine.  }

\bparts
\ppart
If $h$ is surjective, then $f$ must be surjective.

\begin{solution}
\emph{True.}

For all $x$ in $C$: Since $h$ is surjective, there exists $y$ in $A$ such that 
$h(y) = x$. Therefore, by definition of $h$, $f(g(y)) = x$, so $x$ is in
the range of $f$.

Therefore, all of $C$ is in the image of $f(C)$, so $f$ is surjective.
\end{solution}

\ppart
If $h$ is surjective, then $g$ must be surjective.

\begin{solution}
\emph{False.}

Suppose $A = C = \set{1}$ and $B = \set{1, 2}$.  Let $f$ be such that
$f(1) = f(2) = 1$, and $g$ such that $g(1) = 1$. In this case $h$ is
indeed surjective, as $h(1) = 1$, but $g$ is not surjective as it
doesn't map anything to 2.
\end{solution}

\ppart
If $h$ is injective, then $f$ must be injective.

\begin{solution}
\emph{False.}

Taking the same example as in the previous case. $h$ is injective, because
only 1 maps to 1.  However, $f$ is not injective as $f(1) = f(2).$
\end{solution}

\ppart
If $h$ is injective and $f$ is total, then $g$ must be injective.

\begin{solution}
\emph{True.}

For all $x$ and $y$: If $g(x) = g(y)$ then since $f$ is total,
$f$ is defined on $g(x)$ and
\[
h(x) = f(g(x)) = f(g(y)) = h(y),
\]

so $x = y$ because $h$ is injective.  This means, $g$ is injective.

Note that $g$ need not be injective when $f$ is not total (see
Problem~\bref{PS_function_composition}).
\end{solution}

\eparts
\end{problem}

%%%%%%%%%%%%%%%%%%%%%%%%%%%%%%%%%%%%%%%%%%%%%%%%%%%%%%%%%%%%%%%%%%%%%
% Problem ends here
%%%%%%%%%%%%%%%%%%%%%%%%%%%%%%%%%%%%%%%%%%%%%%%%%%%%%%%%%%%%%%%%%%%%%

\endinput
