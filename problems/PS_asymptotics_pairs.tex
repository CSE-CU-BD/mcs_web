\documentclass[problem]{mcs}

\begin{pcomments}
  \pcomment{PS_asymptotics_pairs} 
  \pcomment{from S08, pset8}
\end{pcomments}

\pkeywords{
  asymptotics
  sitrling's approximation
}

%%%%%%%%%%%%%%%%%%%%%%%%%%%%%%%%%%%%%%%%%%%%%%%%%%%%%%%%%%%%%%%%%%%%%
% Problem starts here
%%%%%%%%%%%%%%%%%%%%%%%%%%%%%%%%%%%%%%%%%%%%%%%%%%%%%%%%%%%%%%%%%%%%%
\begin{problem}

\bparts
\ppart

Either prove or disprove each of the following statements.
\begin{itemize}
\item $n! = O((n+1)!)$
\item $(n+1)! = O(n!)$
\item $n! = \Theta((n+1)!)$
\item $n! = o((n+1)!)$
\item $(n+1)! = o(n!)$
\end{itemize}

\begin{solution}Observe that:
\[
\lim_{n \to \infty} n!/(n+1)! = \lim_{n \to \infty} 1/(n+1) = 0.
\]
This gives us $n! = o((n+1)!)$ as well as $n! = O((n+1)!)$.

Further,
\[
\lim_{n \to \infty} (n+1)!/n! = \lim_{n \to \infty} n+1 = \infty.
\]
This implies that $(n+1)! = O(n!)$ and $(n+1)! = o(n!)$ are false. Combining 
the two limits, we get that $n! = \Theta((n+1)!)$ must also be false. 
\end{solution}

\ppart
Show that $\paren{\frac{n}{3}}^{n+e} = o(n!)$.

\begin{solution}By Stirling's formula,
\[
n! \sim \sqrt{2 \pi n} \paren{\frac{n}{e}}^{n}
\]

On the other hand, note that $\paren{\frac{n}{3}}^{n+e} =
\paren{\frac{n}{3}}^e \paren{\frac{n}{3}}^n$.  Dividing this quantity by $n!$, we get:
\[
\frac{\paren{\frac{n}{3}}^e \paren{\frac{n}{3}}^n}{\sqrt{2 \pi n} \paren{\frac{n}{e}}^{n}}
 = \frac{n^{e-1/2}}{3^e\sqrt{2\pi}} \cdot \paren{\frac{e}{3}}^n,
\]
This expression goes to $0$ in the limit as $e < 3$.  Thus,
$\paren{\frac{n}{3}}^{n+e}= o(n!)$.

\end{solution}

\eparts
\end{problem}

%%%%%%%%%%%%%%%%%%%%%%%%%%%%%%%%%%%%%%%%%%%%%%%%%%%%%%%%%%%%%%%%%%%%%
% Problem ends here
%%%%%%%%%%%%%%%%%%%%%%%%%%%%%%%%%%%%%%%%%%%%%%%%%%%%%%%%%%%%%%%%%%%%%

\endinput
