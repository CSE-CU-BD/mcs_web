\documentclass[problem]{mcs}

\begin{pcomments}
    \pcomment{Converted from ../problems/00Convert/probs/practice13/prob1-spiders.scm
              by scmtotex and drewe
              on Fri 15 Jul 2011 12:09:19 PM EDT}
\end{pcomments}

\begin{problem}

%% type: multi-part
%% title: Spiders and Flies

The spider is expecting dinner guests and wants to catch 500 flies.  100 flies pass by her web every hour.  60 of these flies are
 quite small and are caught with probability 1/6 each.  40 of the flies
 are big and are caught with probability 3/4 each.  Assume all fly
 interceptions are mutually independent. She has 10 hours before the dinner party is set to begin. % Using this information, the
% methods from lecture can show
% that the poor spider has only about 1 chance in 100,000 of catching 500
% flies within 10 hours.
\bparts

\ppart\label{ppart:first}
%% type: multiple-choice
%% title: 
Which of the following methods will provide useful information about the spider's chances of catching 500 flies in 10 hours?


\begin{enumerate}
\item Estimation of the binomial density $F_{n,p}$
\item Markov's bound
\item Chebyshev's bound
\item Chernoff's bound
\end{enumerate}
\begin{solution} \begin{enumerate}
\item No. The sum of two binomial distributions with different values of $p$ is not binomial.  We're considering a random variable that is the sum of
$1000$ Bernoulli variables, $600$ with $p = 1/6$ and $400$ with $p = 3/4$.
\item No.  Markov's bound merely tells us that the probability is at most 0.8, which is an absurd overestimate.
\item Yes.
\item Yes.
\end{enumerate}
\end{solution}
\ppart
%% type: multiple-choice
%% title: 
What is that bound? 

\begin{solution}

$e^{-(5/4 ln(5/4) - 5/4 + 1)400}$


This is the Chernoff bound.  The expected number of flies caught every hour is $(1/6)60+(3/4)40 = 40$, so the expected number of flies in 10 hours is
400.  So, 
\begin{equation*}
  Pr{X >= 500} <= Pr{X >= 5/4 400} <= e^{-(5/4 ln(5/4) - 5/4 + 1)400}
\end{equation*}
\end{solution}

\ppart
Should the spider be concerned?
\begin{solution}
Yes, if it's important that she have at least 500 flies.  The bound above comes out to about $9.4\times 10^{-6}$, so it's very unlikely she'll meet her goal.
\end{solution}
%% </font>

\eparts


\end{problem}

\endinput
