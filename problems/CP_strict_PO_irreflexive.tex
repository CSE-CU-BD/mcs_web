\documentclass[problem]{mcs}

\begin{pcomments}
  \pcomment{CP_strict_PO_irreflexive}
  \pcomment{from: F09.cp4w, prob5; S09.cp3r; S06.ps2 (by wing)}
  \pcomment{edited by ARM 3/13/10}
\end{pcomments}

\pkeywords{
  partial_orders
  irreflexive
  transitive
}

%%%%%%%%%%%%%%%%%%%%%%%%%%%%%%%%%%%%%%%%%%%%%%%%%%%%%%%%%%%%%%%%%%%%%
% Problem starts here
%%%%%%%%%%%%%%%%%%%%%%%%%%%%%%%%%%%%%%%%%%%%%%%%%%%%%%%%%%%%%%%%%%%%%

\begin{problem} 
  A binary relation, $R$, on a set, $A$, is \emph{irreflexive} iff
  $\QNOT(a \mrel{R} a)$ for all $a \in A$.  Prove that if a binary
  relation on a set transitive and irreflexive, then it is strict
  partial order.

\begin{solution}
\begin{proof}

Suppose $R$ is a transitive and irreflexive relation on a set, $A$.
Since $R$ is transitive, to show it is a strict partial order, we need
only verify that it is asymmetric.

To prove that it is asymmetric, suppose $a\mrel{R}b$ holds for some
$a,b\in A$.  We need to prove $\QNOT(b \mrel{R} a)$.

So assume to the contrary that $b\mrel{R}a$ holds.  Then $a\mrel{R}b$
and $b\mrel{R}a$, so by transitivity, $a\mrel{R}a$, contradicting the
fact that $R$ is irreflexive.  So $b\mrel{R}a$ does not hold, as
claimed.
\end{proof}
\end{solution}
 
\end{problem}

%%%%%%%%%%%%%%%%%%%%%%%%%%%%%%%%%%%%%%%%%%%%%%%%%%%%%%%%%%%%%%%%%%%%%
% Problem ends here
%%%%%%%%%%%%%%%%%%%%%%%%%%%%%%%%%%%%%%%%%%%%%%%%%%%%%%%%%%%%%%%%%%%%%

\endinput
