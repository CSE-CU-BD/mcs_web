\documentclass[problem]{mcs}

\begin{pcomments}
  \pcomment{MQ_voldemort_returns}
  \pcomment{by njoliat and ARM}
\end{pcomments}

\pkeywords{
  conditional_probability
  tree_diagram
  four-step_method
}

%%%%%%%%%%%%%%%%%%%%%%%%%%%%%%%%%%%%%%%%%%%%%%%%%%%%%%%%%%%%%%%%%%%%%
% Problem starts here
%%%%%%%%%%%%%%%%%%%%%%%%%%%%%%%%%%%%%%%%%%%%%%%%%%%%%%%%%%%%%%%%%%%%%

\begin{problem}
We revisit Sauron, Voldemort, and Bunny Foo Foo as in the class
problem.  As before, the guard is going to release exactly two of the
three prisoners, and he's equally likely to release any set of two
prisoners.

The guard offers to tell Voldemort the name of one of the prisoners to
be released.  The guard's rule for which name he chooses:

1. The guard will never say that Voldemort will be released.

2. If both Foo Foo and Sauron are getting released, the guard will
always give Foo Foo's name.

We're interested in which characters are released, and in which
character the guard says will be released.

\begin{problemparts}
\problempart
Draw a tree to represent the sample space.  Indicate, in your drawing,
which outcomes correspond to the following events:

i. The guard tells Voldemort that Foo Foo will be released

ii. The guard tells Voldemort that Sauron will be released

iii. Voldemort is released 

\examspace[4.5in]
\begin{solution}
%
\begin{center}
\begin{picture}(360,175)(0,-40)
%\put(0,-40){\dashbox(360,175){}} % bounding box
\put(0,60){\line(1,1){60}}
\put(0,60){\line(1,0){60}}
\put(0,60){\line(1,-1){60}}
\put(30,-10){\makebox(0,0){released}}
\put(60,120){\line(1,0){60}}
\put(60,60){\line(1,0){60}}
\put(60,0){\line(1,0){60}}
\put(90,-25){\makebox(0,0){guard says}}
\put(11,90){\makebox(0,0){$F,V$}}
\put(40,68){\makebox(0,0){$F,S$}}
\put(11,30){\makebox(0,0){$V,S$}}
\put(52,96){\makebox(0,0){$1/3$}}
\put(40,50){\makebox(0,0){$1/3$}}
\put(52,24){\makebox(0,0){$1/3$}}
\put(90,128){\makebox(0,0){$F$}}
\put(90,68){\makebox(0,0){$F$}}
\put(90,-10){\makebox(0,0){$S$}}
\put(90,110){\makebox(0,0){$1$}}
\put(90,50){\makebox(0,0){$1$}}
\put(90,8){\makebox(0,0){$1$}}
\put(150,120){\makebox(0,0){$1/3$}}
\put(150,60){\makebox(0,0){$1/3$}}
\put(150,0){\makebox(0,0){$1/3$}}
\put(150,-20){\makebox(0,0){prob.}}
\put(210,120){\makebox(0,0){$\times$}}
\put(210,68){\makebox(0,0){$\times$}}
\put(210,0){\makebox(0,0){}}
\put(210,-25){\makebox(0,0){\shortstack{guard says\\"Foo-foo"}}}
\put(270,120){\makebox(0,0){}}
\put(270,68){\makebox(0,0){}}
\put(270,0){\makebox(0,0){$\times$}}
\put(270,-25){\makebox(0,0){\shortstack{guard says\\"Sauron"}}}
\put(330,120){\makebox(0,0){$\times$}}
\put(330,68){\makebox(0,0){}}
\put(330,0){\makebox(0,0){$\times$}}
\put(330,-25){\makebox(0,0){\shortstack{Voldemort\\released}}}
\end{picture}
\end{center}
\end{solution}

\problempart %B
What is the probability that Voldemort is released, given that the
guard says Foo-foo will be released?
\examspace[3in]

\begin{solution}
\begin{align*}
\prcond{\text{V released}}{\text{says foofoo}} & = \frac{ \text{V
    released} \intersect \text{says foofoo} }{ \text{says foofoo} }\\
     &  = \frac{ 1/3 }{ 2/3 } = \frac{1}{2}
\end{align*}
\end{solution}

\problempart %C
What is the probability Voldemort is released, given that the guard
says Sauron will be released?
\examspace[3in]

\begin{solution}
\begin{align*}
\prcond{\text{V released}}{text{says sauron}} & = \frac{ \text{V released}
  \intersect \text{says sauron} }{ \text{says sauron} } \\
 & =\frac{ 1/3 }{ 1/3 } = 1.
\end{align*}
\end{solution}

\problempart %D
Use the above calculations, and the Law of Total Probability, to find
the total probability that Voldemort will be released.
\examspace[3in]

\begin{solution}

Still 2/3, by law of total probability.
\begin{align*}
\prob{\text{V released}}
       & = \prcond{\text{V released}}{\text{says foofoo}}
              \cdot \prob{\text{says foofoo}}\\
       & \quad + \prcond{\text{V released}}{\text{says sauron}}
              \cdot \prob{\text{says sauron}}
\end{align*}

\begin{editingnotes}
complete the last calculation
\end{editingnotes}

\end{solution}
\end{problemparts}

\end{problem}


%%%%%%%%%%%%%%%%%%%%%%%%%%%%%%%%%%%%%%%%%%%%%%%%%%%%%%%%%%%%%%%%%%%%%
% Problem ends here
%%%%%%%%%%%%%%%%%%%%%%%%%%%%%%%%%%%%%%%%%%%%%%%%%%%%%%%%%%%%%%%%%%%%%

\endinput
