\documentclass[problem]{mcs}

\RequirePackage{enumerate}

\begin{pcomments}
  \pcomment{MQ_voldemort_returns}
  \pcomment{by njoliat and ARM}
  \text{revised ARM 5/5/12}
\end{pcomments}

\pkeywords{
  conditional_probability
  tree_diagram
  four-step_method
}

%%%%%%%%%%%%%%%%%%%%%%%%%%%%%%%%%%%%%%%%%%%%%%%%%%%%%%%%%%%%%%%%%%%%%
% Problem starts here
%%%%%%%%%%%%%%%%%%%%%%%%%%%%%%%%%%%%%%%%%%%%%%%%%%%%%%%%%%%%%%%%%%%%%

\begin{problem}
We revisit Sauron, Voldemort, and Bunny Foo Foo \inbook{from
  problem~\bref{CP_conditional_prob_says_so_bug}} \inhandout{from a
  class problem}.  As before, the guard is going to release exactly
two of the three prisoners, and he's equally likely to release any set
of two prisoners.

\ppart\label{Vreleaseprob} What is the probability that Voldemort will be released?

The guard offers to tell Voldemort the name of one of the prisoners to
be released.  The guard's rule for which name he chooses:
\begin{enumerate}

\item The guard will never say that Voldemort will be released.

\item If both Foo Foo and Sauron are getting released, the guard will
  always give Foo Foo's name.

\end{enumerate}
We're interested in which characters \emph{are} released and which
character the guard \emph{says} will be released.

\begin{problemparts}
\problempart
Draw a tree to represent the sample space.  Indicate, in your drawing,
which outcomes correspond to the following events:
\begin{enumerate}[i.\ ]

\item The guard tells Voldemort that Foo Foo will be released.

\item The guard tells Voldemort that Sauron will be released.

\item Voldemort is released.

\end{enumerate}

\examspace[4.5in]

\begin{solution}
%
\begin{center}
\begin{picture}(360,175)(0,-40)
%\put(0,-40){\dashbox(360,175){}} % bounding box
\put(0,60){\line(1,1){60}}
\put(0,60){\line(1,0){60}}
\put(0,60){\line(1,-1){60}}
\put(30,-10){\makebox(0,0){released}}
\put(60,120){\line(1,0){60}}
\put(60,60){\line(1,0){60}}
\put(60,0){\line(1,0){60}}
\put(90,-25){\makebox(0,0){guard says}}
\put(11,90){\makebox(0,0){$F,V$}}
\put(40,68){\makebox(0,0){$F,S$}}
\put(11,30){\makebox(0,0){$V,S$}}
\put(52,96){\makebox(0,0){$1/3$}}
\put(40,50){\makebox(0,0){$1/3$}}
\put(52,24){\makebox(0,0){$1/3$}}
\put(90,128){\makebox(0,0){$F$}}
\put(90,68){\makebox(0,0){$F$}}
\put(90,-10){\makebox(0,0){$S$}}
\put(90,110){\makebox(0,0){$1$}}
\put(90,50){\makebox(0,0){$1$}}
\put(90,8){\makebox(0,0){$1$}}
\put(150,120){\makebox(0,0){$1/3$}}
\put(150,60){\makebox(0,0){$1/3$}}
\put(150,0){\makebox(0,0){$1/3$}}
\put(150,-20){\makebox(0,0){prob.}}
\put(210,120){\makebox(0,0){$\times$}}
\put(210,68){\makebox(0,0){$\times$}}
\put(210,0){\makebox(0,0){}}
\put(210,-25){\makebox(0,0){\shortstack{guard says\\"Foo-foo"}}}
\put(270,120){\makebox(0,0){}}
\put(270,68){\makebox(0,0){}}
\put(270,0){\makebox(0,0){$\times$}}
\put(270,-25){\makebox(0,0){\shortstack{guard says\\"Sauron"}}}
\put(330,120){\makebox(0,0){$\times$}}
\put(330,68){\makebox(0,0){}}
\put(330,0){\makebox(0,0){$\times$}}
\put(330,-25){\makebox(0,0){\shortstack{Voldemort\\released}}}
\end{picture}
\end{center}
\end{solution}

\problempart\label{VgiveFF}
What is the probability that Voldemort is released, given that the
guard says Foo-foo will be released?
\examspace[3in]

\begin{solution}
\begin{align*}
\prcond{\text{V released}}{\text{says foofoo}} & = \frac{ \pr{\text{V
    released} \QAND \text{says foofoo} }}{ \pr{\text{says foofoo}} }\\
     &  = \frac{ 1/3 }{ 2/3 } = \frac{1}{2}\ .
\end{align*}
\end{solution}

\problempart\label{VgiveSa}
What is the probability Voldemort is released, given that the guard
says Sauron will be released?
\examspace[3in]

\begin{solution}
\begin{align*}
\prcond{\text{V released}}{text{says sauron}}
   & = \frac{ \pr{\text{V released} \QAND \text{says sauron}} }{\pr{\text{says sauron}} } \\
 & =\frac{ 1/3 }{ 1/3 } = 1.
\end{align*}
\end{solution}

\problempart Show how to use the Law of Total Probability to combine
your answers to parts~\eqref{VgiveFF} and~\eqref{VgiveSa} to verify
that the result matches the answer to part~\eqref{Vreleaseprob}.

\examspace[3in]

\begin{solution}
\begin{align*}
\frac{2}{3}
 & = \prob{\text{V released}}
      & (part~\eqref{Vreleaseprob}\\
 & = \prcond{\text{V released}}{\text{says foofoo}}
              \cdot \prob{\text{says foofoo}} + \\
     \prcond{\text{V released}}{\text{says sauron}}
              \cdot \prob{\text{says sauron}}
      & \text{(Total Probability)}\\
 & = \frac{1}{2}\frasc{2}{3} + 1 \cdot \frac{1}{3}\\
       & \text{(parts~\eqref{VgiveFF} and~\eqref{VgiveSa})}
 & = \frac{2}{3}
\end{align*}

\end{solution}
\end{problemparts}p

\end{problem}


%%%%%%%%%%%%%%%%%%%%%%%%%%%%%%%%%%%%%%%%%%%%%%%%%%%%%%%%%%%%%%%%%%%%%
% Problem ends here
%%%%%%%%%%%%%%%%%%%%%%%%%%%%%%%%%%%%%%%%%%%%%%%%%%%%%%%%%%%%%%%%%%%%%

\endinput
