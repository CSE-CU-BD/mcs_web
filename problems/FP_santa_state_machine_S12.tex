\documentclass[problem]{mcs}

\begin{pcomments}
  \pcomment{FP_santa_state_machine_S12}
  \pcomment{excerpted FP_santa_state_machine}
\end{pcomments}

\pkeywords{
  state_machines
  unreachable_states
  strictly_increasing
  strictly_decreasing
  constant
  weakly_increasing
  weakly_decreasing
  invariant
}

%%%%%%%%%%%%%%%%%%%%%%%%%%%%%%%%%%%%%%%%%%%%%%%%%%%%%%%%%%%%%%%%%%%%%
% Problem starts here
%%%%%%%%%%%%%%%%%%%%%%%%%%%%%%%%%%%%%%%%%%%%%%%%%%%%%%%%%%%%%%%%%%%%%

\begin{problem}
  Santa's elves have decided to primp his ride.  They add spinners,
  hydraulics and a huge array of lights to the side of Santa's sleigh.
  The elves start off with 98 green and 4 red lights.  The lights can
  switch between Green and Red, and are controlled by two big
  buttons:
\begin{itemize}

%\begin{enumerate}
%\item[(i)]

\item Switch the color of any ten lights, chosen by winking at them.

%\item[(ii)]

\item Let $n$ be the number of green lights showing.  Add $n+1$
  additional red lights (if there aren't enough lights on the sleigh to do
  this, then add some more).
%\end{enumerate}

\end{itemize}

For example, a possible first state transition would be to flip nine green
lights and one red, which leaves 90 green and 12 red lights on.  A next
possible transition would be to add 91 red lights, thereby leaving 90
green and 103 red lights on.

\iffalse

\bparts


\ppart
Model this situation as a state machine, carefully defining the set of
states, the start state and the possible state transitions.

\examspace[1.5in]

\begin{solution}
This can be modeled by a state machine.  The state of the
  machine is the number of green and red lights.  The start state is $(98,4)$,
  and the transitions are:
\[
(g,r) \to
\begin{cases}
(g-a+(10-a),r+a-(10-a)) & \mbox{for } 10 \leq g+r\ \QAND\
                         \max(0, 10-r) \leq a \leq \min(10, g).\\
(g,r+g+1).
\end{cases}
\]

\begin{staffnotes}
Most students forgot to specify the range of $a$ precisely
for the first transition.
\end{staffnotes}

\end{solution}

\ppart Explain how to reach a state with exactly one red light showing.

\examspace[1.5in]

\begin{solution}
One way is to:
\begin{enumerate}

\item Do operation 2 three times, yielding $(98, 4+3\cdot99)= (98,301)$.

\item Repeat 30 times: Do operation 1 to flip 10 red lights to green. This
will result in the state $(398,1)$, which is the desired state.
\end{enumerate}

\end{solution}


\ppart\label{derivedvars_part}\fi

In any given state, let $G$ be the number of green lights and $R$ be
the number of red lights.  For the following six derived variables
\[\begin{array}{|l|l|}\hline
 G   &  \rem{G}{2},\\
 R   &  \rem{R}{2},\\
 R+G &  \rem{R+G}{2},\\\hline
\end{array}\]
indicate which are\dots

\begin{itemize}\renewcommand{\itemsep}{0pt}

\item strictly increasing \hfill \examrule[1in]

\begin{solution}
NONE
\end{solution}

\item weakly increasing  \hfill \examrule[1in]

\begin{solution}
$R+G$, $\rem{G}{2}$

\textbf{Comment:} Notice that a constant variable like $\rem{G}{2}$ is
also weakly decreasing and weakly increasing, by definition.
\end{solution}

\item strictly decreasing  \hfill \examrule[1in]

\begin{solution}
NONE
\end{solution}

\item weakly decreasing  \hfill \examrule[1in]

\begin{solution}
$\rem{G}{2}$
\end{solution}

\item constant  \hfill \examrule[1in]

\begin{solution}
$\rem{G}{2}$
\end{solution}
\end{itemize}

\iffalse
\examspace

\ppart Prove that it is not possible to reach a state in which there is
exactly one green light showing.  (If you appeal to one of your answers to
part~\eqref{derivedvars_part}, you must prove it.) 

\examspace[3in]

\begin{solution}
  We claimed above that $\rem{G}{2}$ is a constant, that is, it does not
  change under state transitions.  To prove this, let $(g,r)$ be a state
  with $g$ even.  For the next state, we have two cases to consider:
\begin{enumerate}
\item The first operation is executed:
$(g,r)\rightarrow(g-2a+10,r+2a-10)$.  Since $-2a+10$ is even, $\rem{(g,r)}{2}=
\rem{(g-2a+10, r+2a-10)}{2}$.

\item The second operation is executed: $(g,r)\rightarrow(g,r+g+1)$. The
  number of green lights does not change in this case, so $\rem{(g,r)}{2}$
  does not change.
\end{enumerate}
Since the initial number of green lights, 98, is even, that is,
$\rem{98}{2}=0$, the Invariant Method now implies that the number of
green lights in a reachable state is always even.  But since 1 is odd, it
is not possible to reach a state in which there is exactly 1 green light
showing.

\end{solution}

\eparts
\fi

\end{problem}

%%%%%%%%%%%%%%%%%%%%%%%%%%%%%%%%%%%%%%%%%%%%%%%%%%%%%%%%%%%%%%%%%%%%%
% Problem ends here
%%%%%%%%%%%%%%%%%%%%%%%%%%%%%%%%%%%%%%%%%%%%%%%%%%%%%%%%%%%%%%%%%%%%%

\endinput
