\documentclass[problem]{mcs}

\begin{pcomments}
\pcomment{MQ_induction_nn1}
\pcomment{rewrite of FP_induction_nn1}
\pcomment{rewrite of FP_wop_nn1}
\pcomment{from F04rec2}
\pcomment{edited by ARM 9/20/15}
\end{pcomments}

\pkeywords{
  induction
  series
  sum
}

%%%%%%%%%%%%%%%%%%%%%%%%%%%%%%%%%%%%%%%%%%%%%%%%%%%%%%%%%%%%%%%%%%%%%
% Problem starts here
%%%%%%%%%%%%%%%%%%%%%%%%%%%%%%%%%%%%%%%%%%%%%%%%%%%%%%%%%%%%%%%%%%%%%

\begin{problem}
Prove by induction that
\begin{equation}\label{nn1n2ind}
1 \cdot 2 + 2 \cdot 3 + 3 \cdot 4 + \dots + n (n+1)
    = \frac{n (n+1) (n+2)}{3}
\end{equation}
for all integers $n\geq 1$. Use the equation itself as the induction hypothesis $P(n)$.

\bparts

\ppart  Prove the \inductioncase{base case} $(n=1)$.

\begin{solution}
\begin{proof}
$P(1)$ is true because the left-hand
side of~\eqref{nn1n2ind} is $1 \cdot 2 = 2$, and the right-hand side is
$(1\cdot 2 \cdot 3)/ 3 = 2$.
\end{proof}
\end{solution}

\examspace[2in]

\ppart Now prove the \inductioncase{inductive step}.

\begin{solution}

\begin{proof}
  Assuming the induction hypothesis~\eqref{nn1n2ind} holds 
for some $n \geq 1$, we must prove $P(n+1)$.  We reason as follows:
\begin{align*}
\lefteqn{1 \cdot 2 + 2 \cdot 3 + \dots + (n+1) (n+2)}\\
    & = [1 \cdot 2 + 2 \cdot 3 + \dots + n(n+1)] + (n+1) (n+2)\\
    & = \frac{n(n+1)(n+2)}{3} + (n+1) (n+2) & \text{by ind.\ hypothesis~\eqref{nn1n2ind}}\\
    & = \frac{n(n+1)(n+2)}{3} + \frac{3(n+1)(n+2)}{3} & \text{algebra}\\
    & = \frac{([(n+1)(n+2)](n + 3)}{3} & \text{algebra}\\
    & = \frac{(n+1)(n+2)(n+3)}{3} & \text{algebra}
\end{align*}
Therefore
\[
1 \cdot 2 + 2 \cdot 3 + \dots + (n+1) (n+2) = \frac{(n+1)(n+2)(n+3)}{3}
\]
which is $P(n+1)$.

By the induction principle, we conclude that $P(n)$ is true for all $n
\ge 1$, which proves the claim.
\end{proof}
\end{solution}


\eparts



\end{problem}

%%%%%%%%%%%%%%%%%%%%%%%%%%%%%%%%%%%%%%%%%%%%%%%%%%%%%%%%%%%%%%%%%%%%%
% Problem ends here
%%%%%%%%%%%%%%%%%%%%%%%%%%%%%%%%%%%%%%%%%%%%%%%%%%%%%%%%%%%%%%%%%%%%%

\endinput
