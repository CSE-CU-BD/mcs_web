\documentclass[problem]{mcs}
\begin{pcomments}
  \pcomment{PS_update_balanced}
  \pcomment{ARM 10/8/17}
\end{pcomments}

\pkeywords{
  trees
  binary_tree
  search_tree
  insertion
  deletion
  }

%%%%%%%%%%%%%%%%%%%%%%%%%%%%%%%%%%%%%%%%%%%%%%%%%%%%%%%%%%%%%%%%%%%%%
% Problem starts here
% %%%%%%%%%%%%%%%%%%%%%%%%%%%%%%%%%%%%%%%%%%%%%%%%%%%%%%%%%%%%%%%%%%%%
\begin{problem}
The \emph{isomorphism} relation between recursive tree $T,U \in
\rectr$ is defined recursively

\inductioncase{Base case}: ($T \in \leafset$).  $T$ is isomorphic to
$U$ iff $U$ is a leaf.

\inductioncase{Constructor case}: ($T \in \brnchng$).  $T$ is
isomorphic to $U$ iff
\[
U \in \brnchng\ \QAND\
 \leftsub{U}\ \text{isomorphic to}\ \leftsub{T}\ \QAND\
\rightsub{U}\ \text{isomorphic to}\ \rightsub{T}.
\]

\bparts

\ppart Prove that if $T$ and $U$ are isomorphic, then $\sz{T} = \sz{U}$.

\begin{solution}
Trivial by structural induction:

\inductioncase{Base case}: ($T \in \leafset$).  If $T$ is isomorphic
to $U$, then $U$ must also be a leaf, so $\sz{T} = 1 = \sz{U}$.

\inductioncase{Constructor case}: ($T \in \brnchng$).  If $U$
isomorphic to $T$, then by def. of isomorphism, $\leftsub{T}$ is isomorphic to $\leftsub{U}$, and so 
\[
\sz{\leftsub{T}} = \sz{\leftsub{U}}
\]
by induction hypothesis.  Likewise, 
\[
\sz{\rightsub{T}} = \sz{\rightsub{U}}.
\]
Therefore,
\[
\sz{T} = 1 + \sz{\leftsub{T}} + \sz{\rightsub{T}} = 1 + \sz{\leftsub{U}} + \sz{\rightsub{U}} = \sz{U}.
\]
\end{solution}

\ppart Show that if $T$ and $U$ are isomorphic and are labelled to be
search trees for the same set of values, then $\nlbl{T} = \nlbl{U}$.

\hint By def of isomorphism and search tree.  No induction needed.

\begin{solution}

\inductioncase{Base case}: ($T \in \leafset$).  There is only one
possible labelling of $T$ and $U$ with the same value.

\inductioncase{Constructor case}: ($T \in \brnchng$).  For $T$ to be
labelled as a search tree, all the values $\nlbls{(\leftsub{T})}$ in
$\leftsub{T}$ must be less than the label $nlbl{(T)}$ of $T$, and
likewise all the labels in $\nlbls{(\rightsub{T})}$ must be greater
than $nlbl{(T)}$.  So $\nlbls{(\leftsub{T})}$ must be the smallest
$\sz{\leftsub{T}}$ values in $\nlbls{(T)}$, and $nlbl{(T)}$ must be
the $\sz{\leftsub{T}}+1$st smallest value.  Since $T$ and $U$ are
isomorphic, the sizes of their corresponding branches are the same, so
by the same reasoning, $\nlbl{(U)}$ must also be the
$\sz{\leftsub{T}}+1$st smallest value.
\end{solution}

\ppart Suppose $T,U \in \rectr$ have the same label.  $T,U$
\emph{clash} when only one of them is a leaf, or else both are
branching and
\begin{itemize}
\item $\nlbl{(\leftsub{T})} \neq \nlbl{(\leftsub{U})}$, or
\item $\nlbl{(\rightsub{T})} \neq \nlbl{(\rigtsub{U})}$.
\end{itemize}

Let $R$ and $S$ be isomorphic trees that are each labelled as search
trees for their respective sets of values $\nlbls{R}$ and $\nlbls{S}$.
Now suppose $\nlbls{S}$ is the set that results from deleting
$\min(R)$ from $\nlbls{(R)}$ and then inserting some element $m >
\max(R)$, that is,
\[
\nlbls{(S)} \eqdef (\nlbls{(R)} - \set{\min(R)) \union \set{m}.
\]

Prove that if $T$ is a subtree of $R$ with the same label as a subtree
$U$ of $S$, then $T$ and $U$ clash.

\hint Induction again.

\begin{solution}
\inductioncase{Base case}: ($R \in \leafset$).  Then $\nlbl{(S)} = m >
\nlbl{(R)}$, so there are no subtrees with the same label.

\inductioncase{Constructor case}: ($R \in \brnchng$).  We know that
$\leftsub{(R)}$ will a search tree isomorphic to the search tree
$\leftsub{(S)}$.  Also
\[
\nlbls{(\leftsub{S})} = (\nlbls{(\leftsub{R})} - \min\set{R}) \union \nlbl{(R)}.
\]
So by induction hypothesis, any subtree of $\leftsub{R}$ clashes with
the subtree of $\leftsub{S}$ with the same label.  Likewise for
subtrees of $\rightsub{R}$ and $\rightsub{S}$ with the same label.

All that remains is to show that the $R$ and $S$ clash \TBA{TBA\dots}.
\end{solution}

\iffalse

numberings clash $\nlbl{}_V(T)$.
Suppose $R$ is the unique subtree with $\nlbl{R}_W = \nlbl{}_V(T)$.
We know that $\nlbl{}_W(T)$ is the next largest value in $V$ after
$\nlbl{}_V(T)$.  In particular, $\nlbl{}_W(T) \neq \nlbl{}_V(T)$, so
$R \neq T$.  Assume wlog that $R$ is a subtree of $\leftsub{T}$.
Since $\nlbl{}_W$ is a search tree labelling, 
\[
\nlbl{}_W(\rightsub{R}) < \nlbl{}_W(T),
\]
so 
\begin{equation}\tag{*}
\nlbl{}_W(\rightsub{R}) < \leq \nlbl{}_V(T).
\end{equation}
But since $\nlbl{}_V$ is a search tree labelling, 
\begin{equation}\tag{**}
\nlbl{}_V(T) < \nlbl{}_V(\rightsub{T}).
\end{equation}
Now~(*) and~(**) imply
\[
\nlbl{}_W(\rightsub{R}) \neq \nlbl{}_V(\rightsub{T}),
\]
confirming that $\nlbl{}_W$ and $\nlbl{}_V$ clash at $\nlbl{}_V(T)$.


\begin{staffnotes}
The point of this is to show that two isomorphic trees $T$, $U$ with
$T$ a search tree for $V$ and $U$ a search tree under the same
labelling function for $V - \min\set{V}) \union m$, then $T$ and $U$
have no subtree in common.  So if we tried to maintain the shape of a
search tree after just one deletion and one insertion, we might have
to change all $\card{V}$ subtrees instead of only $\log_2 \card{V}$ as
with AVL trees, where the tree shape may change after an insertion and
deletion.
\end{staffnotes}

\eparts
\fi

\end{problem}

%%%%%%%%%%%%%%%%%%%%%%%%%%%%%%%%%%%%%%%%%%%%%%%%%%%%%%%%%%%%%%%%%%%%%
% Problem ends here
%%%%%%%%%%%%%%%%%%%%%%%%%%%%%%%%%%%%%%%%%%%%%%%%%%%%%%%%%%%%%%%%%%%%%

\endinput





