\documentclass[problem]{mcs}
\begin{pcomments}
  \pcomment{PS_update_balanced}
  \pcomment{ARM 10/5/17}
\end{pcomments}

\pkeywords{
  trees
  binary_tree
  search_tree
  insertion
  deletion
  }

%%%%%%%%%%%%%%%%%%%%%%%%%%%%%%%%%%%%%%%%%%%%%%%%%%%%%%%%%%%%%%%%%%%%%
% Problem starts here
% %%%%%%%%%%%%%%%%%%%%%%%%%%%%%%%%%%%%%%%%%%%%%%%%%%%%%%%%%%%%%%%%%%%%
\begin{problem}
\bparts

\ppart\label{zint12n-1} Prove that for any given set $V$ of $2^n-1$
real numbers and fully balanced tree $B$ of size $2^n-1$, there is
\emph{only one way} to label $B$ with these values so that $B$ is a
search tree.

\hint Induction on $n$.

\begin{solution}
Let $V_{low}$ be the $2^{n-1}-1$ smallest values in $V$ and $V_{high}$
be the $2^{n-1}-1$ largest values.  In a balanced search tree for $V$,
the values in the left branch must be $V_{low}$, the values in the
left branch must be $V_{high}$, and the tree itself must be labelled
with the median value.  Since in a balanced tree, the left and right
hand branches must also be balanced, there is, by induction, only way
way to label the branches of $B$ with these respective sets of values.
This uniquely determines the labelling of $B$.
\end{solution}

\ppart Let $W$ be a set of values that results from deleting
$\min\set{V}$ and inserting some element $m > \max\set{V}$ into $V$,
that is,
\[
W \eqdef (V - \set{\min\set{V}}) \union \set{m}
\]

Let $\text{num}_V$ be the unique search tree labelling of $B$ with
values $V$, and likewise for $\text{num}_W$.  Let
\[
\text{leafnums}_V \eqdef
\set{r \suchthat r = \text{num}_V(T)\ \text{for some leaf of}\ B}
\]
and likewise for $\text{leafnums}_V$.
Show that
\[
\text{leafnums}_V \intersect \text{leafnums}_W = \emptyset.
\]

\hint Induction again.

\begin{solution}
Note that
\[
W_{\text{low}} = (V_{\text{low}} - \set{\min\set{V}}) \union \set{r_{\text{low}}}
\]
where $r_{\text{low}}$ is the median element of $V$.  

By induction, we have
\[
\text{leafnums}_{V_{\text{low}}} \intersect \text{leafnums}_{W_{\text{low}}} = \emptyset.
\]
But $\text{num}_V$ and $\text{num}_{V_{\text{low}}}$ must agree on
$\leftsub{B}$ and likewise for $\rightsub{B}$, so \TBA{to be continued}.

\end{solution}

\eparts

\end{problem}

%%%%%%%%%%%%%%%%%%%%%%%%%%%%%%%%%%%%%%%%%%%%%%%%%%%%%%%%%%%%%%%%%%%%%
% Problem ends here
%%%%%%%%%%%%%%%%%%%%%%%%%%%%%%%%%%%%%%%%%%%%%%%%%%%%%%%%%%%%%%%%%%%%%

\endinput





