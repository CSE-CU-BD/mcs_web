\documentclass[problem]{mcs}

\begin{pcomments}
  \pcomment{MQ_closed_walk_to_cycle}
\end{pcomments}

\pkeywords{
}

%%%%%%%%%%%%%%%%%%%%%%%%%%%%%%%%%%%%%%%%%%%%%%%%%%%%%%%%%%%%%%%%%%%%%
% Problem starts here
%%%%%%%%%%%%%%%%%%%%%%%%%%%%%%%%%%%%%%%%%%%%%%%%%%%%%%%%%%%%%%%%%%%%%

\begin{problem}
% 

\begin{problemparts}

\problempart
Draw a simple graph in which there is a closed walk that includes two distinct vertices, $u$ and $v$,
but no cycle that includes both $u$ and $v$.  Se sure to label $u$ and $v$ on your drawing.
\examspace[2in]
\begin{solution}
One possibility is shown in Figure~\ref{fig:closed_walk_graph}.  Here, a closed walk that includes $u$ and $v$ is $uabcvbu$.  (This closed walk in particular is not a cycle because it is not a path: the vertex $b$ appears twice.)
\begin{figure}[h]
\graphic{MQ_closed_walk_graph}
\caption{One possibility.\label{fig:closed_walk_graph}}
\end{figure}
\end{solution}

\textbf{NOTE:} At this stage, I was supposed to say ``Prove that in any 
simple graph, if there is a positive-length closed walk that includes a 
vertex $u$, then there is also a cycle that includes $u$.''  We probably 
shouldn't
ask this directly, but ask one easily-answered question at a time.  In 
any case, I'm stuck, because I don't even see how this statement is 
true.  Consider the connected graph with two vertices.  
(This also seems to provide another, simpler answer to part (a).)  
I may be missing something really 
obvious.

\end{problemparts}
\end{problem}

\endinput
