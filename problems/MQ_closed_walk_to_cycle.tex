\documentclass[problem]{mcs}

\begin{pcomments}
  \pcomment{MQ_closed_walk_to_cycle}
  \pcomment{first part of PS_shortest_undirected_closed_walk}
  \pcomment{generalizes for both uv off cycle, ARM 4.20.17}

\end{pcomments}

\pkeywords{
  digraph
  walk
  cycle
}

%%%%%%%%%%%%%%%%%%%%%%%%%%%%%%%%%%%%%%%%%%%%%%%%%%%%%%%%%%%%%%%%%%%%%
% Problem starts here
%%%%%%%%%%%%%%%%%%%%%%%%%%%%%%%%%%%%%%%%%%%%%%%%%%%%%%%%%%%%%%%%%%%%%

\begin{problem} 
Draw a simple graph in which there are at least two distinct paths
between two of the graph's vertices $u$ and $v$, but neither $u$ nor
$v$ is on a cycle.  Be sure to label $u$ and $v$ on your drawing.

\hint There is a five-vertex graph that exhibits this property.

\examspace[2in]

\begin{solution}
Take a triangle and attach $u$ by an edge to one corner and $v$ by an
edge to another corner.

\iffalse
One possibility is shown in Figure~\ref{fig:closed_walk_graph}.  Here,
$uabv$ and $ubv$ are two paths between $u$ and $v$.  Since $v$ has
degree $1$, it cannot be part of a cycle, though it can be (and is)
part of a positive-length closed walk.

\begin{figure}[h]
\graphic{MQ_closed_walk_graph}
\caption{One possibility.\label{fig:closed_walk_graph}}
\end{figure}
\fi

\end{solution}

\end{problem}

\endinput
