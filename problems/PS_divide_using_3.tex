\documentclass[problem]{mcs}

\begin{pcomments}
  \pcomment{PS_divide_using_3}
  \pcomment{from: F07.ps4}
\end{pcomments}

\pkeywords{
  state_machines
  termination
  partial_correctness
  invariant
  division
  algorithm
}

%%%%%%%%%%%%%%%%%%%%%%%%%%%%%%%%%%%%%%%%%%%%%%%%%%%%%%%%%%%%%%%%%%%%%
% Problem starts here
%%%%%%%%%%%%%%%%%%%%%%%%%%%%%%%%%%%%%%%%%%%%%%%%%%%%%%%%%%%%%%%%%%%%%

\begin{problem}
In the late 1960s, the military junta that ousted the government of the
small republic of Nerdia completely outlawed built-in multiplication
operations, and also forbade division by any number other than $3$.
Fortunately, a young dissident found a way to help the population multiply
any two nonnegative integers without risking persecution by the junta.  The
procedure he taught people is:

\examspace

\begin{tabbing}
XXX\=XXX\=XXX\=XXX\kill
\CMD{procedure} {\it multiply}$(x, y$: nonnegative integers$)$ \\
$r:=x$; \\
$s:=y$; \\
$a:=0$;\\
\CMD{while} $s \neq 0$ \CMD{do} \\
   \>\CMD{if} $3 \divides s$ \CMD{then} \\
   \>\>   $r := r+r+r$; \\
   \>\>   $s := s / 3$;\\
   \>\CMD{else if} $3 \divides (s-1)$ \CMD{then} \\
   \>\>   $a := a + r$; \\
   \>\>   $r := r+r+r$; \\
   \>\>   $s := (s - 1) / 3$; \\
   \>\CMD{else} \\
   \>\>   $a := a + r+ r$; \\
   \>\>   $r := r+r+r$; \\
   \>\>   $s := (s - 2) / 3$; \\
\CMD{return} $a$; \\
\end{tabbing}

We can model the algorithm as a state machine whose states are triples of
nonnegative integers $(r,s,a)$.  The initial state is $(x,y,0)$.  The
transitions are given by the rule that for $s>0$:
\[
(r,s,a)\rightarrow\begin{cases}
        (3r,s/3,a) &\text{ if } 3 \divides s\\
        (3r,(s-1)/3,a+r) &\text{ if } 3 \divides (s-1)\\
        (3r,(s-2)/3,a+2r) &\text{otherwise}.
       \end{cases}
\]

\bparts

\ppart
List the sequence of steps that appears in the execution of the
algorithm for inputs $x = 5$ and $y = 10$.

\begin{solution}
$(5, 10, 0) \movesto
(15, 3, 5) \movesto
(45 ,1, 5) \movesto
(135, 0, 50)$
\end{solution}

\ppart Use the Invariant Method to prove that the algorithm is partially
correct---that is, if $s = 0$, then $a = xy$.

\begin{solution}
Let
\[
P((r,s,a)) \eqdef\quad [rs+a = xy].
\]

Clearly, $P$ holds for the start state because
\[
P((x,y,0)) \qiff [xy+0 = xy].
\]

Now, we show that $P$ is indeed a preserved invariant, namely, assuming
$P((r,s,a))$,
\begin{equation}\label{inv}
rs+a = xy,
\end{equation}
holds and $(r,s,a) \to (r',s',a')$ is a transition, then $P((a',b',p'))$,
\begin{equation}\label{inv'}
r's'+a' = xy,
\end{equation}
holds.

We consider three cases:

If $3 \divides s$, then we have that $r' = 3r, s' = s/3, a'=a$.
Therefore,
\begin{align*}
  r's' + a' & = 3r \cdot \frac{s}{3} + a\\
            & = rs+a\\
            & = xy & \text{(by~\eqref{inv})}.
\end{align*}

If $3 \divides s-1$, then $r' = 3r, s' = (s-1)/3,a = a+r$.  So:
\begin{align*}
  r's' + a'  & = 3r \cdot \frac{s-1}{3} + a+r\\
   & = r\cdot(s-1) + a + r\\
   & = rs+a\\
   & = xy & \text{(by~\eqref{inv})}.
\end{align*}

Otherwise, we have $r' = 3r, s' = (s-2)/3,a = a+2r$.  So:
\begin{align*}
  r's' + a'  & = 3r \cdot \frac{s-2}{3} + a+2r\\
   & = r\cdot(s-2) + a + 2r\\
   & = rs+a\\
   & = xy & \text{(by~\eqref{inv})}.
\end{align*}

So in all three cases,~\eqref{inv'} holds, proving that $P$ is indeed a
preserved invariant.

Since the procedure's only termination condition is that $s=0$,
partial correctness will follow if we can show that if $s=0$, then $a=xy$.
But this follows immediately from~\eqref{inv}.
\end{solution}

\ppart
Prove that the algorithm terminates after at most $1+ \log_3 y$ executions
of the body of the \CMD{do} statement.

\begin{solution}
We first notice that $s \in \nngint$ is a preserved invariant.
Also, each transition corresponds to an execution of the \CMD{do}
statement body, and each transition reduces $s$ to at most $s/3$.
Hence, after at most $1+ \log_3 y$ executions of the body, the value
of $s$ is at most its initial value $y$ times $\paren{1/3}^{1+
  \log_3 y} = 1/3y$.  That is, the value of $s$ is at most 1/3.  Since
$s \in \nngint$, it follows that $s$ will be 0 after this many
executions of the body, if it wasn't 0 earlier.  But with $s=0$, the
procedure terminates.
\end{solution}

\eparts
\end{problem}

\endinput
