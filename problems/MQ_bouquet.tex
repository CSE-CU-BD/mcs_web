\documentclass[problem]{mcs}

\begin{pcomments}
  \pcomment{MQ_bouquet}
  \pcomment{from: S08.Quiz2}
\end{pcomments}

\pkeywords{
  generating function
  convolution
  polynomial
 }


%%%%%%%%%%%%%%%%%%%%%%%%%%%%%%%%%%%%%%%%%%%%%%%%%%%%%%%%%%%%%%%%%%%%%
% Problem starts here
%%%%%%%%%%%%%%%%%%%%%%%%%%%%%%%%%%%%%%%%%%%%%%%%%%%%%%%%%%%%%%%%%%%%%

\begin{problem}

You would like to buy a bouquet of flowers. You find an
online service that will make bouquets of \textbf{lilies}, \textbf{roses}
and \textbf{tulips}, subject to the following constraints:
\begin{itemize}
\item there must be at most 3 lilies,
\item there must be an odd number of tulips,
\item there can be any number of roses.
\end{itemize}

Example: A bouquet of 3 tulips, 5 roses and no lilies satisfies
the constraints.

Let $f_n$ be the number of possible bouquets with $n$ flowers that fit the
service's constraints. Express $F(x)$, the \idx{generating function}
corresponding to $\ang{f_0, f_1, f_2, \dots}$, as a quotient of
polynomials (or products of polynomials). You do not need to simplify this
expression.

\begin{solution}Generating function for the number of ways to choose
lilies:
\[
F_L (x) = 1+ x+ x^2 + x^3 = \frac{1-x^4}{1-x}
\]

Generating function for the number of ways to choose roses:
\[
F_R (x) = 1+x+ x^2+x^3 + x^4 + \cdots = \frac{1}{1-x}
\]

Generating function for the number of ways to choose tulips:
\[
F_W (x) = x + x^3 + x^5 + \cdots = \frac{x}{1-x^2}
\]

By the \index{convolution} Convolution Property, the generating function
for $f_n$ is therefore the product of these functions, namely,
\begin{align*}
F(x) & = F_L(x) F_R(x) F_W(x) \\
     & = \frac{x (1+ x+ x^2 + x^3)}{(1-x)(1-x^2)} \\
     & = \frac{x (1-x^4)}{(1-x)^2(1-x^2)}.
\end{align*}
\iffalse
\begin{eqnarray*}
F(x) &=& F_L(x) F_R(x) F_W(x) \\
     &=& \frac{x (1+ x+ x^2 + x^3)}{(1-x)(1-x^2)}\\
     &=& \frac{x(1 -x^4)}{(1-x)^3(1+x)}
\end{eqnarray*}
\fi

\end{solution}

\end{problem}

\endinput
