\documentclass[problem]{mcs}

\begin{pcomments}
  \pcomment{MQ_bouquet}
  \pcomment{from: S08.Quiz2}
\end{pcomments}

\pkeywords{
  generating function
  convolution
  polynomial
 }

%%%%%%%%%%%%%%%%%%%%%%%%%%%%%%%%%%%%%%%%%%%%%%%%%%%%%%%%%%%%%%%%%%%%%
% Problem starts here
%%%%%%%%%%%%%%%%%%%%%%%%%%%%%%%%%%%%%%%%%%%%%%%%%%%%%%%%%%%%%%%%%%%%%

\begin{problem}

You would like to buy a bouquet of flowers.  You find an
online service that will make bouquets of \textbf{lilies}, \textbf{roses}
and \textbf{tulips}, subject to the following constraints:
\begin{itemize}
\item there must be at most 3 lilies,
\item there must be an odd number of tulips,
\item there can be any number of roses.
\end{itemize}

Example: A bouquet of 3 tulips, 5 roses and no lilies satisfies
the constraints.

Express $B(x)$, the \idx{generating function} for the number of ways
to select a bouquet of $nR$ flowers, as a quotient of polynomials (or
products of polynomials).  You do not need to simplify this
expression.

\begin{solution}
Generating function for the number of ways to choose
lilies:
\[
L(x) = 1+ x+ x^2 + x^3 = \frac{1-x^4}{1-x}
\]

Generating function for the number of ways to choose roses:
\[
R (x) = 1+x+ x^2+x^3 + x^4 + \cdots = \frac{1}{1-x}
\]

Generating function for the number of ways to choose tulips:
\[
T(x) = x + x^3 + x^5 + \cdots = \frac{x}{1-x^2}
\]

By the \index{convolution} Convolution Property, the generating function
$B(x)$ is the product of these functions, namely,
\begin{align*}
B(x) & = L(x) R(x) T(x) \\
     & = \frac{x (1+ x+ x^2 + x^3)}{(1-x)(1-x^2)} \\
     & = \frac{x (1-x^4)}{(1-x)^2(1-x^2)}.
\end{align*}
\iffalse
\begin{align*}
F(x) &= F_L(x) F_R(x) F_W(x) \\
     &= \frac{x (1+ x+ x^2 + x^3)}{(1-x)(1-x^2)}\\
     &= \frac{x(1 -x^4)}{(1-x)^3(1+x)}
\end{align*}
\fi

\end{solution}

\end{problem}

\endinput
