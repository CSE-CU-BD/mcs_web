\documentclass[problem]{mcs}

\begin{pcomments}
  \pcomment{FP_dice_rolling}
\end{pcomments}

\pkeywords{
  combinatorics
  counting
}

%%%%%%%%%%%%%%%%%%%%%%%%%%%%%%%%%%%%%%%%%%%%%%%%%%%%%%%%%%%%%%%%%%%%%
% Problem starts here
%%%%%%%%%%%%%%%%%%%%%%%%%%%%%%%%%%%%%%%%%%%%%%%%%%%%%%%%%%%%%%%%%%%%%

\begin{problem}

\textbf{Combinatorics and Counting}

\bparts

\ppart
Suppose that we are flipping a fair coin $n$ times. What is 
the probability that there are exactly $k$ heads, where the heads
must be separated by at least 2 tails?

\begin{solution}
  There are $\binom{n-2k+2}{k}$ number of ways to choose where to 
  place the heads (consider a bijection with having $k$ 1s 
  in an $n-2k+2$ bit binary string). Note that it is $n-2(k-1)$ 
  because the last head does not need to be followed by 2 tails.
  
  The total number of results is $2^n$, so the probability is:
  
  \[ \binom{n-2k+2}{k} \over {2^n} \]
\end{solution}

\ppart
An evil TA was trying to make a difficult probability question for the
Finals. The question goes as follows:

``Suppose that we are flipping a fair coin $n$ times. What is 
the probability that there are at least $k$ consecutive heads?
(e.g. there are \textit{7} consecutive heads in the result: 
\textit{THTTHHHHHHHT}.)''

The TA then gave a solution that turned out to be wrong:

\begin{quote}
Consider the possible results as an $n$ length binary string.
The total of all possible results is therefore $2^n$.

We can then count the total number of strings that have $k$ 
consecutive ones: there are $n - k + 1$ places to begin the 
$k$ length substring; and $2^{(n-k)}$ ways to fill in the 
remaining $n-k$ results. There are, therefore, a total of 
$ (n-k+1) \cdot 2^{(n-k)}$ number of possible strings.

The probability is the number of results with $k$ consecutive ones 
divided by the number of all possible results:

\[ (n-k+1) \cdot 2^{(n-k)} \over 2^n \]
\end{quote}

What is wrong with the solution? Give a counter-example and then
explain what is wrong. (you do \textit{not} need to solve for
the actual answer.)

\begin{solution}
  The mistake is in ``double-counting'' strings with more than k
  consecutive ones. For instance, for n = 4, k = 2; the string
  \textit{0111} will be counted in the case where the substring 
  starts at the first 1 and again in the case where the substring
  starts at the second 1.
\end{solution}

\eparts
\end{problem}