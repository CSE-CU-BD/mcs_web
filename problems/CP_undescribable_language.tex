\documentclass[problem]{mcs}

\begin{pcomments}
  \pcomment{CP_undescribable_language}
  \pcomment{from: S09.cp2r}
  \pcomment{Can only be used AFTER pred calculus and the referenced class
    problem.  Needs handy copy of Russell Paradox}
  \pcomment{revised by ARM 2/13/11}

\end{pcomments}

\pkeywords{
  Russell
  paradox
  strings
  binary
  logic
}

%%%%%%%%%%%%%%%%%%%%%%%%%%%%%%%%%%%%%%%%%%%%%%%%%%%%%%%%%%%%%%%%%%%%%
% Problem starts here
%%%%%%%%%%%%%%%%%%%%%%%%%%%%%%%%%%%%%%%%%%%%%%%%%%%%%%%%%%%%%%%%%%%%%

\begin{problem}
  \newcommand{\desc}{\mopt{strings}}
%  \def\all0s{\textbf{all-0s}}

  The method used to prove Cantor's Theorem that the power set is
  ``bigger'' than the set, leads to many important results in logic
  and computer science.  In this problem we'll apply that idea to
  describe a set of binary strings that can't be described by ordinary
  logical formulas.  To be provocative, we could say that we will
  describe an undescribable set of strings!

  The following logical formula illustrates how a formula can describe a set of strings.  The formula
  \begin{equation}%\tag{\all0s} %\label{no1}
    \QNOT[\exists y.\, \exists z.\, s = y1z] \tag{$\text{no-1s}(s)$},
  \end{equation}
  where the variables range over the set, \bins, of finite binary
  strings, 
says that the binary string, $s$, does not contain a 1.

  We'll call such a predicate formula, $G(s)$, about strings a
  \emph{string formula}, and we'll use the notation $\desc(G)$ for the set
  of binary strings with the property described by $G$.  That is,
\[
\desc(G) \eqdef \set{s \in \bins \suchthat G(s)}.
\]
A set of binary strings is \term{describable} if it equals $\desc(G)$
for some string formula, $G$.  So the set, $0^*$, of finite strings of
0's is describable because it equals
$\desc(\text{no-1s})$.\footnote{$\text{no-1s}$ and similar formulas
  were examined in Problem~\bref{CP_assertions_about_binary_strings},
  but it is not necessary to have done that problem to do this one.}

The idea of representing data in binary is a no-brainer for a computer
scientist, so it won't be a stretch to agree that any string formula can
be represented by a binary string.  We'll use the notation $G_{x}$ for
the string formula with binary representation $x \in \bins$.  The details
of the representation don't matter, except that there ought to be a
display procedure that can actually display $G_{x}$ given $x$.

Standard binary representations of formulas are often based on
character-by-character translation into binary, which means that only
a sparse set of binary strings actually represent string formulas.  It
will be technically convenient to have \emph{every} binary string
represent some string formula.  This is easy to do: tweak the display
procedure so it displays some default formula, say $\text{no-1s}$,
when it gets a binary string that isn't a standard representation of a
string formula.  With this tweak, \emph{every} binary string, $x$,
will now represent a string formula, $G_{x}$.

Now we have just the kind of situation where a Cantor-style diagonal
argument can be applied, namely, we'll ask whether a string describes a
property of \emph{itself}!  That may sound like a mind-bender, but all
we're asking is whether $x \in \desc(G_{x})$.

For example, using character-by-character translations of formulas into
binary, neither the string $0000$ nor the string $10$ would be the binary
representation of a formula, so the display procedure applied to either of
them would display $\text{no-1s}$.  That is, $G_{0000} = G_{10} =
\text{no-1s}$ and so $\desc(G_{0000})= \desc(G_{10}) = 0^*$.  This means
that
\[
0000 \in \desc(G_{0000}) \quad\text{and}\quad 10 \notin \desc(G_{10}).
\]

Now we are in a position to give a precise mathematical description
of an ``undescribable'' set of binary strings, namely, let
\begin{theorem*}
Define
\begin{equation}\label{Udefxngx}
U \eqdef \set{x \in \bins \suchthat\ x \notin \desc(G_{x})}.
\end{equation}
The set $U$ is not describable.
\end{theorem*}

Use reasoning similar to Cantor's theorem (repeated below) to prove
this Theorem.

\begin{solution}
By definition~\eqref{Udefxngx}, 
\begin{equation}\label{xUiffxGx}
x \in U \qiff x \notin \desc(G_{x}).
\end{equation}
for $x\in \bins$.

Also, $U = \desc(G_{x_U})$ by assumption.  This means:
\begin{equation}\label{xUqxGx}
x \in U \qiff x \in \desc(G_{x_U}).
\end{equation}

Combining~\eqref{xUqxGx} and~\eqref{xUiffxGx}, we have
\begin{equation}\label{xdGiff}
x \notin \desc(G_{x}) \iff x \in \desc(G_{x_U}),
\end{equation}
for all $x\in \bins$.  Now plugging in $x_U$
for $x$ in~\eqref{xdGiff} gives an immediate contradiction.

So there cannot be any formula that describes $U$.
\end{solution}
\end{problem}

\iffalse

\textbox{
\begin{center}
\large Cantor's Theorem

\textbf{There is no bijection between any set $A$ and its powerset
$\power(A)$.}
\end{center}

\begin{proof}
  We show that if $g$ is a total function from $A$ to $\power(A)$,
  then $g$ does not have the $[\ge 1\ \text{in}]$, surjection
  property, and so is certainly not a bijection.

  Define
  \[
  A_g \eqdef \set{a \in A \suchthat a \notin g(a)}.
  \]
  Since $g$ is total, $A_g$ is a well-defined subset of $A$, which
  means it is a member of $\power(A)$.  We claim $A_g$ is not in the
  range of $g$, and so $g$ is not a surjection.

  To prove that $A_g \notin \range{g}$, assume to the contrary that it
  was in $range{g}$.  That is,
\[
A_g = g(a_0)
\]
for some $a_0 \in A$.  Then by definition of $A_g$,
\[
a \in g(a_0) \qiff a \in A_g \qiff a \notin g(a)
\]
for all $a \in A$.  Now letting $a = a_0$ yields the contradiction
\[
a_0 \in g(a_0) \qiff a_0 \notin g(a_0).
\]

\end{proof}
}
\fi

%%%%%%%%%%%%%%%%%%%%%%%%%%%%%%%%%%%%%%%%%%%%%%%%%%%%%%%%%%%%%%%%%%%%%
% Problem ends here
%%%%%%%%%%%%%%%%%%%%%%%%%%%%%%%%%%%%%%%%%%%%%%%%%%%%%%%%%%%%%%%%%%%%%

\endinput
