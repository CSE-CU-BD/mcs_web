\documentclass[problem]{mcs}

\begin{pcomments}
  \pcomment{CP_undescribable_language}
  \pcomment{from: S09.cp2r}
  \pcomment{Can only be used AFTER pred calculus and the referenced class
    problem.  Needs handy copy of Russell Paradox}
  \pcomment{The students had some trouble with this one and sent
            suggestions via email in S09 (maybe Noah Caplan?) about
            their difficulties and how to fix this problem.  It seems
            like some of their suggestions were discussed at the end (?)
            but it could probably still use a good revision.}
\end{pcomments}

\pkeywords{
  Russells_paradox
  strings
  binary
  logic
}

%%%%%%%%%%%%%%%%%%%%%%%%%%%%%%%%%%%%%%%%%%%%%%%%%%%%%%%%%%%%%%%%%%%%%
% Problem starts here
%%%%%%%%%%%%%%%%%%%%%%%%%%%%%%%%%%%%%%%%%%%%%%%%%%%%%%%%%%%%%%%%%%%%%

\begin{problem}
%  \newcommand{\nns}{\mop{bins}}
  \newcommand{\desc}{\mopt{ok-strings}}
%  \def\all0s{\textbf{all-0s}}

  Though it was a serious challenge for set theorists to overcome
  Russells' Paradox, the idea behind the paradox led to some important
  (and correct \smiley\ ) results in logic and computer science.  In this
  problem we'll apply that idea to describing a set of strings that can't
  be described by ordinary logical formulas about binary strings!

  For example, the using binary strings as the domain of discourse, the
  formula
  \begin{equation}%\tag{\all0s} %\label{no1}
    \QNOT[\exists y\, \exists z.\, s = y1z] \tag{$\sc{no-1s}(s)$}
  \end{equation}
  says that the binary string, $s$, does not contain a 1.

  In general, when $G(s)$ is any predicate formula that defines a property of a
  binary string, $s$, we'll say it \emph{describes} the set of binary
  strings with that property:
\[
\desc(G) \eqdef \set{s \suchthat G(s)}.
\]
So $\desc(\sc{no-1s})$\footnote is the set of all strings of
0's.\footnote{$\sc{no-1s}$ and similar formulas were examined in
  Problem~\bref{CP_assertions_about_binary_strings}, but it is not
  necessary to have done that problem to do this one.}

Now the idea of representing data in binary should be completely familiar
to a computer scientist, so it won't be a stretch to claim that any
string formula, $G(s)$, can be represented by a binary string.  We'll use
the notation $G_{\mathbf{x}}(s)$ for the string formula with binary
representation $\mathbf{x}$.  The details of the representation don't
matter, except that there ought to be a display procedure that can
actually display $G_{\mathbf{x}}(s)$ given $\mathbf{x}$.  With typical
representations not every binary string will represent a string formula,
$G(s)$, but we'll fix this by assuming that the display procedure has been
tweaked to display some default formula, say $\sc{no-1s}(s)$, when it gets
given a string that otherwise wouldn't represent a $G(s)$.  So
\emph{every} binary string, $\mathbf{x}$, will now represent a string
formula, $G_{\mathbf{x}}(s)$.

Now we have just the kind of situation where a Cantor-style diagonal
argument can be applied, namely, we'll ask whether a string describes a
property of \emph{itself}!  That may sound like a mind-bender, but all
we're asking is whether $\mathbf{x} \in \desc(G_{\mathbf{x}})$.

For example, using typical schemes for binary representation of a string
formula, neither the string $0000$ nor the string $10$ would represent an
actual formula, so the display procedure applied to either of them would
display $\sc{no-1s}$.  That is, $G_{0000} = G_{10} = \sc{no-1s}$.  So
$0000 \in \desc(G_{0000})$ and $10 \notin \desc(G_{10})$.

Now we're in a position to give a perfectly clear and precise description
of an ``undescribable'' set of binary strings, namely, let
\begin{equation}\label{Udefxngx}
U \eqdef \set{\mathbf{x} \suchthat\ \mathbf{x} \notin
  \desc(G_{\mathbf{x}})}.
\end{equation}

Use reasoning similar to Russell's paradox to show that $U$ is not
described by any predicate formula $G(s)$ about binary strings, that is, 
$U \neq \desc(G)$.

\begin{solution}
By definition~\eqref{Udefxngx}, 
\begin{equation}\label{tGH}
\mathbf{x} \in U \qiff \mathbf{x} \notin \desc(G_{\mathbf{x}}).
\end{equation}

for all formulas, $G$, that define a property of strings.  Substituting
$H$ for $G$ in~\eqref{tGH} now yields the immediate contradiction
\[
t_H \in \desc(H) \qiff t_H \notin \desc(H).
\]
So there cannot be any formula, $H$ that describes $V$.
\end{solution}
\end{problem}

%%%%%%%%%%%%%%%%%%%%%%%%%%%%%%%%%%%%%%%%%%%%%%%%%%%%%%%%%%%%%%%%%%%%%
% Problem ends here
%%%%%%%%%%%%%%%%%%%%%%%%%%%%%%%%%%%%%%%%%%%%%%%%%%%%%%%%%%%%%%%%%%%%%

\endinput
