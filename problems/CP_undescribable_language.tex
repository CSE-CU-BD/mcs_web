\documentclass[problem]{mcs}

\begin{pcomments}
  \pcomment{CP_undescribable_language}
  \pcomment{from: S09.cp2r}
  \pcomment{Can only be used AFTER pred calculus and the referenced class
    problem.  Needs handy copy of Russell Paradox}
  \pcomment{The students had some trouble with this one and sent 
            suggestions via email in S09 (maybe Noah Caplan?) about 
            their difficulties and how to fix this problem.  It seems
            like some of their suggestions were discussed at the end (?)
            but it could probably still use a good revision.}
\end{pcomments}

\pkeywords{
  Russells_paradox
  strings
  binary
  logic
}

%%%%%%%%%%%%%%%%%%%%%%%%%%%%%%%%%%%%%%%%%%%%%%%%%%%%%%%%%%%%%%%%%%%%%
% Problem starts here
%%%%%%%%%%%%%%%%%%%%%%%%%%%%%%%%%%%%%%%%%%%%%%%%%%%%%%%%%%%%%%%%%%%%%

\begin{problem}
%  \newcommand{\nns}{\mop{bins}}
  \newcommand{\desc}{\text{ok-strings}}
  \def\all0s{\textbf{all-0s}}

  Though it was a serious challenge for set theorists to overcome
  Russells' Paradox, the idea behind the paradox led to some important
  (and correct \smiley\ ) results in Logic and Computer Science.

  To show how the idea applies, let's recall the formulas from 
  Problem~\bref{CP_assertions_about_binary_strings}
  that made assertions about binary strings.  For
  example, one of the formulas in that problem was
  \begin{equation}\tag{\all0s} %\label{no1}
  \QNOT[\exists y\, \exists z. s = y1z]
  \end{equation}
  This formula defines a property of a binary string, $s$, namely that $s$
  has no occurrence of a 1.  In other words, $s$ is a string of (zero or
  more) 0's.  So we can say that this formula \emph{describes} the set of
  strings of 0's.

  More generally, when $G$ is any formula that defines a string property,
  let $\desc(G)$ be the set of all the strings that have this property.  A
  set of binary strings that equals $\desc(G)$ for some $G$ is called a
  \term{describable} set of strings.  So, for example, the set of all
  strings of 0's is describable because it equals $\desc(\all0s)$.

\iffalse
\[
\desc(G) \eqdef \set{s} \suchthat G(s)\ \text{is true}}.
\]
\fi

Now let's shift gears for a moment and think about the fact that
formula~$\all0s$ appears above.  This happens because instructions for
formatting the formula were generated by a computer text processor
(for this text, we used the \LaTeX\ text processing system), and then
an image suitable for printing or display was constructed according to
these instructions.  Since everybody knows that data is stored in
computer memory as binary strings, this means there must have been
some binary string in computer memory ---call it $t_{\all0s}$ ---that
enabled a computer to display formula~$\all0s$ once $t_{\all0s}$ was
retrieved from memory.

In fact, it's not hard to find ways to represent \emph{any} formula, $G$,
by a corresponding binary word, $t_G$, that would allow a computer to
reconstruct $G$ from $t_G$.  We needn't be concerned with how this
reconstruction process works; all that matters for our purposes is that
every formula, $G$, has a representation as binary string, $t_G$.

Now let
\[
V \eqdef \set{t_G \suchthat\ \text{$G$ defines a property of strings and
    $t_G \notin \desc(G)$}}.
\] 
Use reasoning similar to Russell's paradox to show that $V$ is not
describable.

\begin{solution}
\hint  By definition of $V$,
\begin{equation}\label{nV}
t_G \in V \qiff t_G \notin \desc(G),
\end{equation}    
for every formula, $G$, that defines a property of strings.

\hint Suppose to the contrary that $V = \desc(H)$ for some formula, $H$.

So from~\eqref{nV}, we have
\begin{equation}\label{tGH}
t_G \in \desc(H) \qiff t_G \notin \desc(G).
\end{equation}
for all formulas, $G$, that define a property of strings.  Substituting
$H$ for $G$ in~\eqref{tGH} now yields the immediate contradiction
\[
t_H \in \desc(H) \qiff t_H \notin \desc(H).
\]
So there cannot be any formula, $H$ that describes $V$.
\end{solution}
\end{problem}

%%%%%%%%%%%%%%%%%%%%%%%%%%%%%%%%%%%%%%%%%%%%%%%%%%%%%%%%%%%%%%%%%%%%%
% Problem ends here
%%%%%%%%%%%%%%%%%%%%%%%%%%%%%%%%%%%%%%%%%%%%%%%%%%%%%%%%%%%%%%%%%%%%%

\endinput
