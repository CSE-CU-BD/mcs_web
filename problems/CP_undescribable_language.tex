\documentclass[problem]{mcs}

\begin{pcomments}
  \pcomment{CP_undescribable_language}
  \pcomment{from: S09.cp2r}
  \pcomment{Can only be used AFTER pred calculus and the referenced class
    problem.  Needs handy copy of Russell Paradox}
  \pcomment{revised by ARM 2/13/11}

\end{pcomments}

\pkeywords{
  Russell
  paradox
  strings
  binary
  logic
}

%%%%%%%%%%%%%%%%%%%%%%%%%%%%%%%%%%%%%%%%%%%%%%%%%%%%%%%%%%%%%%%%%%%%%
% Problem starts here
%%%%%%%%%%%%%%%%%%%%%%%%%%%%%%%%%%%%%%%%%%%%%%%%%%%%%%%%%%%%%%%%%%%%%

\begin{problem}
  \newcommand{\desc}{\mopt{by}}
%  \def\all0s{\textbf{all-0s}}

  Though it was a serious challenge for set theorists to overcome
  Russell's Paradox, the idea behind the paradox led to some important
  (and correct \smiley\ ) results in logic and computer science.  In this
  problem we'll apply that idea to describe a set of strings that can't be
  described by ordinary logical formulas.

  For example, the using the set, \bins, of finite binary strings as the
  domain of discourse, the formula
  \begin{equation}%\tag{\all0s} %\label{no1}
    \QNOT[\exists y\, \exists z.\, s = y1z] \tag{$\text{no-1s}(s)$}
  \end{equation}
  says that the binary string, $s$, does not contain a 1.

  We'll call such a predicate formula, $G(s)$, about strings a
  \emph{string formula}, and we'll use the notation $\desc(G)$ for the set
  of binary strings with the property described by $G$.  That is,
\[
\desc(G) \eqdef \set{s \in \bins \suchthat G(s)}.
\]
A set of binary strings is \term{describable} if it equals $\desc(G)$ for
some string formula, $G$.  So the set, $0^*$, of finite strings of
0's is describable because it equals
$\desc(\sc{no-1s})$.\footnote{$\text{no-1s}$ and similar formulas were
  examined in Problem~\bref{CP_assertions_about_binary_strings}, but it is
  not necessary to have done that problem to do this one.}

The idea of representing data in binary is a no-brainer for a computer
scientist, so it won't be a stretch to agree that any string formula, can
be represented by a binary string.  We'll use the notation $G_{x}$ for
the string formula with binary representation $x \in \bins$.  The details
of the representation don't matter, except that there ought to be a
display procedure that can actually display $G_{x}$ given $x$.

Standard binary representations of formulas are often based on
character-by-character translation into binary, which means that only a
sparse set of binary strings actually represent string formulas.  It will
be technically convenient for us to use \emph{every} binary string as a
representation of some string formula.  This is easy to do: tweak the
display procedure so it displays some default formula, say
$\text{no-1s}(s)$, when it gets a binary string that isn't a standard
representation of a string formula.  With this tweak, \emph{every} binary
string, $x$, will now represent a string formula, $G_{x}$.

Now we have just the kind of situation where a Cantor-style diagonal
argument can be applied, namely, we'll ask whether a string describes a
property of \emph{itself}!  That may sound like a mind-bender, but all
we're asking is whether $x \in \desc(G_{x})$.

For example, using character-by-character translations of formulas into
binary, neither the string $0000$ nor the string $10$ would be the binary
representation of a formula, so the display procedure applied to either of
them would display $\text{no-1s}$.  That is, $G_{0000} = G_{10} =
\text{no-1s}$ and so $\desc(G_{0000})= \desc(G_{10}) = 0^*$.  This means
that
\[
0000 \in \desc(G_{0000}) \quad\text{and}\quad 10 \notin \desc(G_{10}).
\]

Now we are in a position to give a precise mathematical description
of an ``undescribable'' set of binary strings, namely, let
\begin{theorem*}
Define
\begin{equation}\label{Udefxngx}
U \eqdef \set{x \in \bins \suchthat\ x \notin \desc(G_{x})}.
\end{equation}
The set $U$ is not describable.
\end{theorem*}

Assume for contradiction that $U$ was described by some string formula,
$G_{x_U}$ and the use reasoning similar to Russell's paradox (repeated
below) to prove this Theorem.

\begin{solution}
By definition~\eqref{Udefxngx}, 
\begin{equation}\label{xUiffxGx}
x \in U \qiff x \notin \desc(G_{x}).
\end{equation}
for $x\in \bins$.

Also, $U = \desc(G_{x_U)}$ by assumption.  This means:
\begin{equation}\label{xUqxGx}
x \in U \qiff x \in \desc{G_{x_U}}.
\end{equation}

Combining~\eqref{xUqxGx} and~\eqref{xUiffxGx}, we have
\begin{equation}\label{xdGiff}
x \notin \desc{G_{x}} \iff x \in \desc{G_{x_U}},
\end{equation}
for all $x\in \bins$.  Now plugging in $x_U$
for $x$ in~\eqref{xdGiff} gives an immediate contradiction.

So there cannot be any formula that describes $U$.
\end{solution}
\end{problem}

\begin{staffnotes}
  Students may not be able to do this if all they've seen is Russell's
  paradox.  If they are stuck, it would be OK to give
  them~\eqref{Udefxngx} as a hint, and if necessary also~\eqref{xUiffxGx}.
\end{staffnotes}


%%%%%%%%%%%%%%%%%%%%%%%%%%%%%%%%%%%%%%%%%%%%%%%%%%%%%%%%%%%%%%%%%%%%%
% Problem ends here
%%%%%%%%%%%%%%%%%%%%%%%%%%%%%%%%%%%%%%%%%%%%%%%%%%%%%%%%%%%%%%%%%%%%%

\textbox{
\begin{center}
\large Russell's Paradox
\end{center}

\begin{quote}
Let $S$ be a variable ranging over all sets, and define
\[
W \eqdef \set{S \suchthat S \not\in S}.
\]
So by definition,
\[
S \in W  \mbox{  iff  } S \not\in S,
\]
for every set $S$.  In particular, we can let $S$ be $W$, and obtain
the contradictory result that
\[
W \in W  \mbox{  iff  } W \not\in W.
\]
\end{quote}}

\endinput
