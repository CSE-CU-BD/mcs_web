\documentclass[problem]{mcs}

\begin{pcomments}
    \pcomment{TP_Predicate_Logic}
    \pcomment{Converted from prob3.scm by scmtotex and dmj
              on Sun 13 Jun 2010 10:18:32 AM EDT}
    \pcomment{revised by drewe 19 july 2011}
    \pcomment{format ARM 8/25/11}
\end{pcomments}

\pkeywords{
valid
quantifier
}

\begin{problem}

Which of the following are \emph{valid}?
\bparts

\ppart $\exists x \exists y. \; P(x, y) \QIMPLIES \exists y \exists x. \;
P(x, y)$
\begin{solution}
Valid. 

Quantifiers of the \emph{same} type (existential or universal) can
be reordered without altering the meaning of the statement.
\end{solution}

\ppart
$\forall x \exists y. \; Q(x, y) \QIMPLIES \exists y \forall x.\; Q(x, y)$
\begin{solution}
\emph{Not} valid. 

For example, let $Q(x, y)$ be $(y > x)$ and suppose the domain is
nonnegative integers.  Then the left side asserts that for every
nonnegative integer, there is a larger nonnegative integer, which is
true.  The right-hand side asserts that there exists a nonnegative integer
greater than every other nonnegative integer, which is false.
Therefore, the implication as a whole is false, and the statement is
not valid.
\end{solution}

\ppart
$\exists x \forall y. \; R(x, y)
    \QIMPLIES \forall y \exists x. \; R(x, y)$
\begin{solution}
Valid. 

Suppose that the left side is true.  Then there exists an $x_{0}$
such that for all $y$, $R(x_{0}, y)$ is true.  Thus, for all $y$,
there exists an $x$ (namely, $x_{0}$), such that $R(x, y)$ is true.
Therefore, the right-hand side is also true, and the statement is valid.
\end{solution}

\ppart
$\QNOT (\exists x \; S(x))\ \QIFF\ \forall x \; \QNOT(S(x))$
\begin{solution}
Valid. 

If it's not true that there is an element in the domain with
property $S$, then \emph{every} element in the domain must \emph{not}
have property $S$.  Conversely, if \emph{every} element in the domain
does \emph{not} have property~$S$, it can't be true that some element in
the domain has property $S$. 
\end{solution}

\eparts


\end{problem}

\endinput
