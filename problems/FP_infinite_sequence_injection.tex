\documentclass[problem]{mcs}

\begin{pcomments}
  \pcomment{FP_infinite_sequence_injection}
  \pcomment{slightly related to PS_unit_interval}
\end{pcomments}

\pkeywords{
  bijection
  surjection
  infinite_sequence
}

%%%%%%%%%%%%%%%%%%%%%%%%%%%%%%%%%%%%%%%%%%%%%%%%%%%%%%%%%%%%%%%%%%%%%
% Problem starts here
%%%%%%%%%%%%%%%%%%%%%%%%%%%%%%%%%%%%%%%%%%%%%%%%%%%%%%%%%%%%%%%%%%%%%

\begin{problem}
The set $\infstrings{\set{\texttt{1},\texttt{2},\texttt{3}}}$ consists
of the \textbf{infinite} sequences of the digits
\texttt{1},\texttt{2}, and \texttt{3}, and likewise
$\infstrings{\set{\texttt{4},\texttt{5}}}$ is the set of infinite
sequences of the digits \texttt{4},\texttt{5}.  For example
\[\begin{array}{ll}
\texttt{123123123}    \dots & \in \infstrings{\set{\texttt{1},\texttt{2},\texttt{3}}},\\
\texttt{222222222222} \dots &\in \infstrings{\set{\texttt{1},\texttt{2},\texttt{3}}},\\
\texttt{4554445554444}\dots & \in \infstrings{\set{\texttt{4},\texttt{5}}}.
\end{array}\]

\bparts

\ppart Give an example of a total injective function
\[
f: \infstrings{\set{\texttt{1},\texttt{2},\texttt{3}}} \to
\infstrings{\set{\texttt{4},\texttt{5}}}.
\]

\examspace[1.0in]

\begin{solution}
Code each of the digits $\texttt{1},\texttt{2},\texttt{3}$ into a
    pair of digits $\texttt{4},\texttt{5}$, for example
\begin{align*}
\texttt{1} & \leftrightarrow 44\\
\texttt{2} & \leftrightarrow 45\\
\texttt{3} & \leftrightarrow 55,
\end{align*}
and apply this digit by digit to map a sequence in
$\infstrings{\set{\texttt{1},\texttt{2},\texttt{3}}}$ to one in
$\infstrings{\set{\texttt{4},\texttt{5}}}$.  For example.
\[
f(\texttt{1231133}\dots) = 44\,45\,55\,44\, 44\, 55\, 55\dots.
\]
\end{solution}

\ppart\label{interlv} Give an example of a bijection $g:
(\infstrings{\set{\texttt{1},\texttt{2},\texttt{3}}} \times
\infstrings{\set{\texttt{1},\texttt{2},\texttt{3}}}) \to
\infstrings{\set{\texttt{1},\texttt{2},\texttt{3}}}$.

\examspace[1.0in]

\begin{solution}
Let $g$ be the interleaving operation on two sequences, that is
\[
g(x_0x_1x_2x_3\dots\ \mathbf{{\large ,}}\ y_0y_1y_2y_3\dots) \eqdef x_0 y_0 x_1 y_1 x_2 y_2 x_3 \dots.
\]
\end{solution}

\ppart Explain why there is a bijection between
$\infstrings{\set{\texttt{1},\texttt{2},\texttt{3}}} \times
\infstrings{\set{\texttt{1},\texttt{2},\texttt{3}}}$ and
$\infstrings{\set{\texttt{4},\texttt{5}}}$.  (You need not explicitly
define the bijection.)

\examspace[1.5in]

\begin{solution}
The composition $f \compose g$ is a total injective function, so
\[
(\infstrings{\set{\texttt{1},\texttt{2},\texttt{3}}} \times
\infstrings{\set{\texttt{1},\texttt{2},\texttt{3}}}) \inj\
\infstrings{\set{\texttt{4},\texttt{5}}}.
\]

Conversely, mapping a sequence $s \in \infstrings{\set{\texttt{4},\texttt{5}}}$ to the sequence $h(s)$
obtained by replacing \texttt{4}'s by \texttt{1}'s and \texttt{5}'s by
\texttt{2}'s defines a total injective function
\[
h: \infstrings{\set{\texttt{4},\texttt{5}}} \to 
\infstrings{\set{\texttt{1},\texttt{2},\texttt{3}}}.
\]
Now
\[
g^{-1} \compose h: \infstrings{\set{\texttt{4},\texttt{5}}} \to 
(\infstrings{\set{\texttt{1},\texttt{2},\texttt{3}}} \times
\infstrings{\set{\texttt{1},\texttt{2},\texttt{3}}})
\]
is a total injective function, so
\[
\infstrings{\set{\texttt{4},\texttt{5}}} \inj\
(\infstrings{\set{\texttt{1},\texttt{2},\texttt{3}}} \times
 \infstrings{\set{\texttt{1},\texttt{2},\texttt{3}}})
\]

The Schr\"oder-Bernstein Theorem~\bref{S-B_thm} now implies that
\[
\infstrings{\set{\texttt{4},\texttt{5}}} \bij\
(\infstrings{\set{\texttt{1},\texttt{2},\texttt{3}}} \times
\infstrings{\set{\texttt{1},\texttt{2},\texttt{3}}})
\]
\end{solution}

\eparts

\end{problem}
%%%%%%%%%%%%%%%%%%%%%%%%%%%%%%%%%%%%%%%%%%%%%%%%%%%%%%%%%%%%%%%%%%%%%
% Problem ends here
%%%%%%%%%%%%%%%%%%%%%%%%%%%%%%%%%%%%%%%%%%%%%%%%%%%%%%%%%%%%%%%%%%%%%

\endinput
