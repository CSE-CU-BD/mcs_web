\documentclass[problem]{mcs}

\begin{pcomments}
  \pcomment{PS_probabilistic_proof}
  \pcomment{\textbf{Random Tournaments}}
  \pcomment{from S01.tut12}
  \pcomment{added to F02 repository by Tina Wang, S02}
  \pcomment{edited by ARM 5/18/12}
\end{pcomments}

\pkeywords{
  probability
  probabilistic_method
  Boole
  Booles_inequality
  union_bound
  tournament
}

%%%%%%%%%%%%%%%%%%%%%%%%%%%%%%%%%%%%%%%%%%%%%%%%%%%%%%%%%%%%%%%%%%%%%
% Problem starts here
%%%%%%%%%%%%%%%%%%%%%%%%%%%%%%%%%%%%%%%%%%%%%%%%%%%%%%%%%%%%%%%%%%%%%

\begin{problem}           
                                                                         
\begin{staffnotes}
(a) 2 pts, (b) 3, (c) 3, (d) 2
\end{staffnotes}

The results of a round robin tournament in which every two people play
each other and one of them wins can be modelled a \term{tournament
  digraph}---a digraph with exactly one edge between each pair of
distinct vertices, but we'll continue to use the language of players
beating each other.

An $n$-player tournament is \emph{$k$-neutral} for some $k \in [0,n)$,
  when, for every set of $k$ players, there is another player who
  beats them all.  For example, being 1-neutral is the same as not
  having a ``best'' player who beats everyone else.

This problem shows that for any fixed $k$, if $n$ is large enough,
there will be a $k$-neutral tournament of $n$ players.  We will do
this by reformulating the question in terms of probabilities.  In
particular, for any fixed $n$, we assign probabilities to each
$n$-vertex tournament digraph by choosing a direction for the edge
between any two vertices, independently and with equal probability for
each edge.

\begin{problemparts}

\problempart\label{prbsn} For any set $S$ of $k$ players, let $B_S$ be
the event that no contestant beats everyone in $S$.  Express
$\prob{B_S}$ in terms of $n$ and $k$.

\examspace[1in]

\begin{solution}
The probability that a player outside $S$ beats everyone in $S$ is
$(1/2)^k$.  So the probability such a player did not beat everyone in
the group is $1 - (1/2)^k$.  There are $n-k$ players outside of the
group, so
\[
\pr{B_S} = \brac{1 - \paren{\frac{1}{2}}^k}^{n-k} \,.
\]
\end{solution}

\problempart\label{qkalpha} Let $Q_k$ be the event equal to the set of $n$-vertex
tournament digraphs that are \emph{not} $k$-neutral.
Prove that
\[
\prob{Q_k} \leq \binom{n}{k} \alpha^{n-k},
\]
where $\alpha \eqdef 1 - (1/2)^k$.

\hint Let $S$ range over the size-$k$ subsets of players, so
\[
Q_k = \lgunion_S B_S \, .
\]
Use Boole's inequality.

\examspace[1.5in]

\begin{solution}

\begin{align*}
\prob{Q_k} 
 & =\prob{\lgunion_{S} B_S} & \text{(hint)}\\
 & \leq \sum_{S} \prob{B_S} & \text{(Boole's Inequality)}\\
 & = \card{\set{S \suchthat \card{S} = k}}\cdot \prob{B_S}\\
 & = \binom{n}{k} \alpha^{n-k} & \text{(part~\eqref{prbsn})}
\end{align*}

\end{solution}

\ppart Conclude that if $n$ is enough larger than $k$, then
$\pr{Q_k} < 1$.

\examspace[2.0in]

\begin{solution}
\begin{align*}
\pr{Q_k}
   & \leq \binom{n}{k}\alpha^{n-k}
       & \text{(by part~\eqref{qkalpha})}\\
   & = (1/\alpha)^k \binom{n}{k}\alpha^n\\
   & \leq (1/\alpha)^kn^k \alpha^n\,.
\end{align*}
But for any fixed $k$, this last term approaches 0 as $n$ goes to
infinity, so $\pr{Q_k}$ will be less than 1 for all large $n$.

To see why it approaches 0, note that $0< \alpha < 1$, so $1/\alpha >
1$ and hence $n^k = o((1/\alpha)^n)$.  Therefore,
\[
n^k\alpha^n = \frac{n^k}{(1/\alpha)^n} \to 0.
\]
\end{solution}

\problempart Explain why the previous result implies that for every
integer $k$, there is a $k$-neutral tournament.

\examspace[1in]

\begin{solution}
Suppose $n$ is large enough that $\pr{Q_k} < 1$.  Then $\pr{\bar{Q_k}}
> 0$, which implies there must be at least one tournament graph
(outcome) in $\bar{Q_k}$, that is, at least one tournament graph that
is $k$-neutral.  In fact, as $n$ grows, it follows that almost all
$n$-player tournaments will be $k$-neutral.

\medskip 
For interest, some numbers that work are:
\begin{align*}
k = 1, n = 3; k = 2, n = 21;  k = 3, n = 33; k = 4, n = 46; \\
k = 5, n = 59; k = 6, n = 72; k = 7, n = 85; k = 8, n = 98.
\end{align*}

\end{solution}

\end{problemparts}

\end{problem}
                                 

%%%%%%%%%%%%%%%%%%%%%%%%%%%%%%%%%%%%%%%%%%%%%%%%%%%%%%%%%%%%%%%%%%%%%
% Problem ends here
%%%%%%%%%%%%%%%%%%%%%%%%%%%%%%%%%%%%%%%%%%%%%%%%%%%%%%%%%%%%%%%%%%%%%

\endinput
