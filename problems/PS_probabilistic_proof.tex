\documentclass[problem]{mcs}

\begin{pcomments}
  \pcomment{PS_probabilistic_proof}
  \pcomment{\textbf{Round Robin Tournaments}}
  \pcomment{from S01.tut12}
  \pcomment{added to F02 repository by Tina Wang, S02}
  \pcomment{formatted by ARM 5/7/12}
\end{pcomments}

\pkeywords{
  probability
  probabilistic_method
  Boole
  Booles_inequality
  union_bound
  tournament
}

%%%%%%%%%%%%%%%%%%%%%%%%%%%%%%%%%%%%%%%%%%%%%%%%%%%%%%%%%%%%%%%%%%%%%
% Problem starts here
%%%%%%%%%%%%%%%%%%%%%%%%%%%%%%%%%%%%%%%%%%%%%%%%%%%%%%%%%%%%%%%%%%%%%
                                                                         
\begin{problem}           
A round robin tournament of $n$ contestants is one in which every two
contestants play each other exactly once and one of them wins.  For a
fixed integer $k < n$, a question of interest is whether there is
tournament for every $k$ players, there is another player who beats
them all.  This problem shows that if
\[
\binom{n}{k} \brac{1 - \paren{\frac{1}{2}}^k}^{n-k} < 1,
\]
then such an outcome is possible.

\begin{problemparts}

\problempart
Start by numbering the sets of $k$ contestants.  How many such sets are
there?

\begin{solution}
\[
\binom{n}{k}
\]
\end{solution}

\problempart
Let $B_i$ be the event that no contestant beat all the $k$ contestants
in set $i$.  Compute $\prob{B_i}$.  (Note that you must choose
probabilities for each match in order to compute this).

\begin{solution}
Suppose that the results of the game are independent and that each
game is equally likely to be won by either contestant.  This is an
arbitrary choice, but it is easy to work with (and any probability
will suffice to prove existance).  The probability that a person
inside group $i$ beats everyone in $i$ is clearly $0$.  The
probability that a person outside group $i$ beats everyone in $i$ is
$(1/2)^k$, so the probability the person they did not beat everyone in
$i$ is $1 - (1/2)^k$.  There are $n-k$ people outside of group $i$.
Thus, $B_i$ has probability
\[
\brac{1 - \paren{\frac{1}{2}}^k}^{n-k}
\]
\end{solution}

\problempart
Give an upper bound on $\prob{\lgunion B_i}$.

\begin{solution}
Use Boole's inequality:
\[
\Prob{\lgunion B_i} \leq \sum \Prob{B_i}.
\]

In other words, $\prob{\lgunion B_i}$ can be no greater than if all of
the $B_i$ are disjoint.  (Overlap will merely reduce the total
probability of the union).  Since the expression for $B_i$ does not
depend on $i$---the probability is the same for each $k$-sized group
$i$---the sum for all of the $B_i$ is simply
\[
\binom{n}{k} \brac{1 - \paren{\frac{1}{2}}^k }^{n-k}
\]
\end{solution}

\problempart
Explain why this result can be used to prove the existence of the desired
tournament outcome.

\begin{solution}
Probabilistic proof.  If the overall probability of $\lgunion B_i$ is
less than 1, then there must be an outcome that is not in $\lgunion
B_i$.  It does not matter that we chose the probabilities arbitrarily;
the fact that there is \emph{any} positive probability at all means
some outcome that we had not accounted for is possible.

\medskip 
For interest, some numbers that work are:
\begin{align*}
k = 1, n = 3; k = 2, n = 21;  k = 3, n = 33; k = 4, n = 46; \\
k = 5, n = 59; k = 6, n = 72; k = 7, n = 85; k = 8, n = 98
\end{align*}
\end{solution}

\end{problemparts}
\end{problem}
                                 

%%%%%%%%%%%%%%%%%%%%%%%%%%%%%%%%%%%%%%%%%%%%%%%%%%%%%%%%%%%%%%%%%%%%%
% Problem ends here
%%%%%%%%%%%%%%%%%%%%%%%%%%%%%%%%%%%%%%%%%%%%%%%%%%%%%%%%%%%%%%%%%%%%%

\endinput
