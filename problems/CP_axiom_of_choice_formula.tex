%CP_axiom_of_choice_formula

\documentclass[problem]{mcs}

\begin{pcomments}
  \pcomment{9/23/09 by ARM, from logic notesproblem}
\end{pcomments}

\pkeywords{
 Choice
 Set Theory
 ZFC
 member
}

%%%%%%%%%%%%%%%%%%%%%%%%%%%%%%%%%%%%%%%%%%%%%%%%%%%%%%%%%%%%%%%%%%%%%
% Problem starts here
%%%%%%%%%%%%%%%%%%%%%%%%%%%%%%%%%%%%%%%%%%%%%%%%%%%%%%%%%%%%%%%%%%%%%

\def\goesto{translates into}

\begin{problem} 
  The Axiom of Choice can say that if $s$ is a set whose members are
  nonempty sets that are \term{pairwise disjoint} ---that is no two
  sets in $s$ have an element in common ---then there is a set, $c$,
  consisting of exactly one element from each set in $s$.

 In formal logic, we could describe $s$ with the formula,
 \[
 \text{pairwise-disjoint}(s) \eqdef\quad
 \forall x \in s.\, x \neq \emptyset
 QAND \forall x,y \in s. (x \neq y) \QIMPLIES (x \intersect y = \emptyset).
 \]
 Similarly we could describe $c$ with the formula
 \[
 \text{choice-set}(c,s) \eqdef \quad \forall x \in s.\, \exists! z.\ z \in c
 \intersect x.
 \]
 Here ``$\exists! z.$" is fairly standard notation for ``there exists a
 \emph{unique} $z$.

Now we can give the formal definition:
\begin{definition*}[Axiom of Choice]
\[
\forall s.\, \text{pairwise-disjoint}(s) \QIMPLIES \exists c.\,
\text{choice-set}(c,s).
\]
\end{definition*}

The only issue here is that Set Theory is technically supposed to be
expressed in terms of \emph{pure} formulas in the language of sets, which
means formula that uses only the membership relation, $\in$, propositional
connectives, and the two quantifies $\forall$ and $\exists$.  Verify that
the Axiom of Choice can be expressed as a pure formula, by explaining how
to replace all impure subformulas above with equivalent pure formulas.

For example, the formula $x = y$ could be replaced with the pure formula
$\forall z.\, z \in x \QIFF z \in y$.

\begin{solution}
Here is how the impure subformulas used in the above definition of the
Axiom of Choice can be translated into pure formulas:

\[
x \neq \emptyset \goesto \exists y/\, y \in x.
\]

\[
[x \intersect y = \emptyset] \goesto \QNOT(\exists z.\, z \in x \QAND z \in y).
\]

\[
[z \in x \intersect y] \goesto z \in x \QAND z \in y.
\]

\[
\exists! z.\, P(z) \goesto  \exists z.\, P(z) \QAND \forall w.\, P(w)
\QIMPLIES w = z.
\]
This last formula is not pure because it uses $=$, but this is ok since we
know it can be replaced by a pure formula.

\end{solution}
\end{problem}

%%%%%%%%%%%%%%%%%%%%%%%%%%%%%%%%%%%%%%%%%%%%%%%%%%%%%%%%%%%%%%%%%%%%%
% Problem ends here
%%%%%%%%%%%%%%%%%%%%%%%%%%%%%%%%%%%%%%%%%%%%%%%%%%%%%%%%%%%%%%%%%%%%%

\endinput
