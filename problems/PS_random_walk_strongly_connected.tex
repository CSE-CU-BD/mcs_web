\documentclass[problem]{mcs}

\begin{pcomments}
  \pcomment{PS_random_walk_strongly_connected}
  \pcomment{from: S08.ps12}
\end{pcomments}

\pkeywords{
  random_walk
  connected
  strongly_connected
}

%%%%%%%%%%%%%%%%%%%%%%%%%%%%%%%%%%%%%%%%%%%%%%%%%%%%%%%%%%%%%%%%%%%%%
% Problem starts here
%%%%%%%%%%%%%%%%%%%%%%%%%%%%%%%%%%%%%%%%%%%%%%%%%%%%%%%%%%%%%%%%%%%%%

\begin{problem}
  A digraph is \term{strongly connected} iff there is a directed path
  between every pair of distinct vertices.  In this problem we consider a
  finite random walk graph that is strongly connected.

\bparts

\ppart\label{zgamma} Let $d_1$ and $d_2$ be distinct distributions for the
graph, and define the \term{maximum dilation}, $\gamma$, of $d_1$ over
$d_2$ to be
\[
\gamma \eqdef \max_{x \in V} \frac{d_1(x)}{d_2(x)} \ .
\]
Call a vertex, $x$, \emph{dilated} if $d_1(x)/d_2(x) = \gamma$.  Show that
there is an edge, $\diredge{y}{z}$, from an undilated vertex $y$ to a
dilated vertex, $z$.  \hint Choose any dilated vertex, $x$, and consider
the set, $D$, of dilated vertices connected to $x$ by a directed path
(going to $x$) that only uses dilated vertices.  Explain why $D \neq V$,
and then use the fact that the graph is strongly connected.

\begin{solution}
By construction, there is at least one dilated vertex.  First, let's
argue that there's at least one \emph{undilated} vertex.  Suppose all
vertices are dilated.  Then we have $d_1(x) = \gamma d_2(x)$, for all
$x$.  Thus, $1 = \sum_x d_1(x) = \gamma \sum_x d_2(x) = \gamma \cdot
1$, which implies that $\gamma = 1$, and hence the distributions are
the same.

Consider an undilated vertex and a dilated vertex.  Since the graph is
strongly connected, there is a path from the undilated vertex to the
dilated vertex.  Thus, there be an edge along the path that goes from
undilated to dilated.

\iffalse
  The proof is algorithmic.  We'll show how to find $y$ and $z$.

  By construction, there exists a vertex $z$ with $d_1(z)/d_2(z) =
  \gamma$.  Start with this vertex.  Consider all vertices with edges
  pointing to $z$.  If one of them satisfies $d_1(y)/d_2(y) < \gamma$, we
  are done.  Otherwise, choose each of these as a starting point for $z$
  and recurse.

  If the procedure finds $y$ and $z$, we are done.  Assume for the
  sake of contradiction that it does not.

  Since the graph is strongly connected, there is a path from every
  vertex to the starting vertex.  Thus, if the graph has $n$ vertices,
  the recursive procedure will be called on every vertex within a
  recursive depth of $n$.  Thus, it will terminate eventually.

  If the procedure terminates without returning an answer, then we
  conclude that $\gamma = d_1(x)/d_2(x)$ for all $x$.  If $\gamma >
  1$, then $\sum_x d_1(x) > \sum_x d_2(x)$, and hence either $d_1$ or $d_2$
  is not a distribution.  Similarly for $\gamma < 1$.  If $\gamma =
  1$, then $d_1$ and $d_2$ are the same distribution.  Any of these
  generate a contradiction.
\fi

\end{solution}

\ppart Prove that the graph has \emph{at most one} stationary
distribution.  (There always \emph{is} a stationary distribution, but
we're not asking you prove this.) \hint Let $d_1$ be a stationary
distribution and $d_2$ be a different distribution.  Let $z$ be the vertex
from part~\eqref{zgamma}.  Show that starting from $d_2$, the
probability of $z$ changes at the next step.  That is, $\widehat{d_2}(z)
\neq d_2(z)$.

\begin{solution}

  Assume for the sake of contradiction that there are two stationary
  distributions $d_1$ and $d_2$.  Let $\gamma$ be the maximum dilation of
  $d_1$ over $d_2$.

  From the previous part, there exist vertices $y$ and $z$ with
  $d_1(y)/d_2(y) < \gamma$ and $d_1(z)/d_2(z) = \gamma$.
  We apply one step of the random walk to $d_1$ and $d_2$ to get
  \begin{equation}\label{d1}
    d_1(z) = \widehat{d_1}(z) = \sum_{y: y\rightarrow z} d_1(y) p(y,z) \ ,
  \end{equation}
  and
  \begin{equation}\label{d2}
    d_2(z) = \widehat{d_2}(z) = \sum_{y: y\rightarrow z} d_2(y) p(y,z) \ .
  \end{equation}
  Substituting $d_1(z) = \gamma d_2(z)$ into \eqref{d1}, and recalling
  that $d_1(x) \leq \gamma d_2(x)$ for all $x$, we have
  \begin{align*}
    \gamma d_2(z)
        &= \sum_{x: x\rightarrow z}d_1(x) p(x,z) \\
        &= \left(\sum_{x \neq y: x\rightarrow z}
          d_1(x)p(x,z)\right) + d_1(y) p(y,z) \\
        &\leq \left(\sum_{x \neq y: x\rightarrow z} \gamma d_2(x)
          p(x,z)\right) + d_1(y) p(y,z) \\
        &= \left(\left(\gamma \sum_{x: x\rightarrow z} d_2(x)
          p(x,z)\right) - \gamma d_2(y)p(y,z)\right) + d_1(y) p(y,z) \\
        &= \gamma d_2(z) - \gamma d_2(y)p(y,z) + d_1(y) p(y,z)\\
        &< \gamma d_2(z) - \gamma d_2(y)p(y,z) + \gamma d_2(y) p(y,z)\\
        &= \gamma d_2(z).
  \end{align*}
  The strict inequality at the next to last step of the derivation
  follows from the assumption that $d_1(y) < \gamma d_2(y)$, and
  $p(y,z) > 0$.  Thus, we have shown $d_2(z) < d_2(z)$, which is a
  contradiction.

\end{solution}

\eparts

\end{problem}

%%%%%%%%%%%%%%%%%%%%%%%%%%%%%%%%%%%%%%%%%%%%%%%%%%%%%%%%%%%%%%%%%%%%%
% Problem ends here
%%%%%%%%%%%%%%%%%%%%%%%%%%%%%%%%%%%%%%%%%%%%%%%%%%%%%%%%%%%%%%%%%%%%%

\endinput

