\documentclass[problem]{mcs}

\begin{pcomments}
  \pcomment{FP_token_state_machine2}
  \pcomment{F15.final}
  \pcomment{variation of FP_token_state_machine}
\end{pcomments}

\pkeywords{
  state_machine
  invariant
  preserved_invariant
  induction
  congruence
  reachable
}

%%%%%%%%%%%%%%%%%%%%%%%%%%%%%%%%%%%%%%%%%%%%%%%%%%%%%%%%%%%%%%%%%%%%%
% Problem starts here
%%%%%%%%%%%%%%%%%%%%%%%%%%%%%%%%%%%%%%%%%%%%%%%%%%%%%%%%%%%%%%%%%%%%%

\begin{problem}
``Switch Token Replacing'' is a process for updating a set of black
  and white tokens.  The process starts with a single black token.  At
  each step,

\renewcommand{\theenumi}{\roman{enumi}}
\renewcommand{\labelenumi}{(\theenumi)}

\begin{enumerate}
\item one black token can be replaced with two white tokens, or
\item if the numbers of white and black tokens are not the same, the
  colors of all the tokens can be switched: all the black tokens
  become white, and the white tokens become black.
\end{enumerate}

  We can model this game as a state machine whose states are pairs
  $(b,w)$ of nonnegative integers, where $b$ equals the number of
  black tokens, and $w$ equals the number of white tokens.  So the
  start state is $(1,0)$.

\bparts
  
\ppart \inbook{Indicate}\inhandout{Circle} each of the following
states can be reached from the start state in \emph{exactly} two
steps:
\[
(0,0),\ (1,0),\ (0,1),\ (1,1),\ (0,2),\ (2,0),\ (2,1),\ (1,2),\ (0,3),\ (3,0)
\]

\begin{solution}
\[(0,2),\ (2,0),\ (2,1)\]
\end{solution}

\ppart
Define the predicate $F(b,w)$ by the rule:
\[
F(b,w) \eqdef\ [b - w  \not\equiv 0 \pmod{3}].
\]

Prove the following
\begin{claim*}
If $F(b,w)$, then state $(b,w)$ is reachable from the state state.
\end{claim*}

\examspace[3.5in]

\begin{solution}
The proof will be by induction in $n$ using induction hypothesis $P(n)
\eqdef$
\[
 \forall (n_b, n_w).\,
 [(n_b + n_w = n) \QAND  F(n_b,n_w)]
     \QIMPLIES (n_b,n_w) \text{ is reachable}.
\]

\begin{proof}

\inductioncase{Base case} ($n \leq 1$):
There are only two states with $n \leq 1$ that satisfy $F$:
\[
(1,0),\ (0,1).
\]
But $(1,0)$ is reachable since it is the start state, and $(0,1)$ is
reachable in one step by switching white and black tokens.

\inductioncase{Inductive step}: Suppose $b + w = n+1$ for some $n \geq
1$ and $F(b,w)$ holds.  We want to show that $(b, w)$ is reachable.

Now $F(b,w)$ implies that $b \neq w$, and since $n+1 \geq 2$, either
$b \geq 2$ or $w \geq 2$.  Since $(b,w)$ is reachable iff $(w,b)$ is
reachable, we can assume that $w \geq 2$.  This implies that
$(b+1,w-2)$ is a state that transitions to $(b,w)$ in one step.
So we need only show that $(b+1,w-2)$ is reachable.

Now $F(b+1,w-2)$ holds because 
\[
(b+1)-(w-2) = (b-w) + 3  \equiv b-w \not\equiv 0 \pmod{3}.
\]
Also, since
\[
(b+1)+(w-2) = (b+w) - 1 = n,
\]
we conclude by induction hypothesis $P(n)$ that $(b+1,w-2)$ is
reachable.
\end{proof}
\end{solution}

\ppart Explain why $(11^{6^{7777}}, 5^{10^{88}})$ is \emph{not}
  reachable.

\hint We have not proved that $F$ is a preserved invariant, so you may
not assume it.

\begin{solution}
It is easy to see that $\rem{b-w}{3}$ remains unchanged by the
transitions, which implies that $F$ is indeed a preserved invariant.
Since $F$ holds for the start state $(1,0)$, it follows that it holds
for all reachable states.

We claim that $F(11^{6^{7777}}, 5^{10^{88}})$ is false, which proves that
$(11^{6^{7777}}, 5^{10^{88}})$ is not reachable.

To prove $\QNOT(F(11^{6^{7777}}, 5^{10^{88}}))$, note that $11 \equiv
2 \equiv 5 \pmod{3}$, so both $11^{6^{7777}}$ and $5^{10^{88}}$ are
congruent to $2^{\text{even number}} \equiv 1 \pmod{3}$.  So
\[
11^{6^{7777}} - 5^{10^{88}} \equiv 1 - 1 = 0 \pmod{3},
\]
as claimed.
\end{solution}

\eparts

\end{problem}

%%%%%%%%%%%%%%%%%%%%%%%%%%%%%%%%%%%%%%%%%%%%%%%%%%%%%%%%%%%%%%%%%%%%%
% Problem ends here
%%%%%%%%%%%%%%%%%%%%%%%%%%%%%%%%%%%%%%%%%%%%%%%%%%%%%%%%%%%%%%%%%%%%%

\endinput
