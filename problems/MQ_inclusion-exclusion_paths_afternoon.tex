\documentclass[problem]{mcs}

\begin{pcomments}
\pcomment{MQ_inclusion-exclusion_paths}
\pcomment{simplification of CP_inclusion-exclusion_paths}
\pcomment{minor variation of PS_path_counting}
\pcomment{was part of from CP_more_counting}
\pcomment{perturbed from MQ_inclusion-exclusion_paths}
\end{pcomments}

\pkeywords{
  counting
  inclusion_exclsuion
  paths
  multinomial
}

%%%%%%%%%%%%%%%%%%%%%%%%%%%%%%%%%%%%%%%%%%%%%%%%%%%%%%%%%%%%%%%%%%%%%
% Problem starts here
%%%%%%%%%%%%%%%%%%%%%%%%%%%%%%%%%%%%%%%%%%%%%%%%%%%%%%%%%%%%%%%%%%%%%

\begin{problem}

\bparts

We want to count step-by-step paths between points with integer
coordinates in three dimensions.  Let $O,a,b,c$ be the following points:
\begin{align*}
O & \eqdef (0,0,0),\\
a & \eqdef (4,9,14),\\
b & \eqdef (14,9,4),\\
c & \eqdef (9,19,29).
%a & \eqdef (5,10,15),\\
%b & \eqdef (15,10,5),\\
%c & \eqdef (10,20,30).
\end{align*}

A step may move a unit distance in the positive $x$, $y$, or $z$
direction.  For example, a step from point $a$ in the $y$ direction
leads to $(4,10,14)$.

For any set $S \subseteq \set{O,a,b,c}$ of the these points, let
\[
P_S \eqdef \text{the paths that go through all the points in } S.
\]
For example, $P_{O, (0,1,1),(1,2,1)}$ is the set of following four paths.
\begin{align*}
O, (0,1,0), (0,1,1), (1,1,1), (1,2,1) \\
O, (0,1,0), (0,1,1), (0,2,1), (1,2,1) \\
O, (0,0,1), (0,1,1), (1,1,1), (1,2,1) \\
O, (0,0,1), (0,1,1), (0,2,1), (1,2,1) \\
\end{align*}

\ppart Express $\card{P_{a,c}}$ as a multinomial coefficient.

\examspace[1in]

\begin{solution}
There must be exactly $c_x-a_x=9-4=5$ right steps, $c_y-a_y=19-9=10$
up steps, and $c_z-a_z=29-14=15$ forward steps, for a total of 30
%There must be exactly $c_x-a_x=10-5=5$ right steps, $c_y-a_y=20-10=10$
%up steps, and $c_z-a_z=30-15=15$ forward steps, for a total of 30
steps.  These steps may be taken in any order, so the number of paths
is the same as the number of permutations of the letters $R^5 U^{10}
F^{15}$, namely,
\[
\binom{30}{5,10,15}\,.
\]
\end{solution}

\ppart\label{poacpoa}  Express $\card{P_{O,a,c}}$ in terms of $\card{P_{O,a}}$ and $\card{P_{a,c}}$.

\examspace[0.5in]

\begin{solution}
\begin{equation}\label{cqcpcn}
\card{P_{O,a,c}} = \card{P_{O,a}} \cdot \card{P_{a,c}}.
\end{equation}

Every path in $P_{O,a,c}$ consists of a path in $P_{O,a}$ followed by a path in $P_{a,c}$,
and this correspondence between $P_{O,a,c}$ and $P_{O,a} \cross P_{a,c}$ is a bijection,
so~\eqref{cqcpcn} follows by the Product Rule.
\end{solution}

\ppart\label{pab0} What is $\card{P_{a,b}}$? \hfill\examrule

\begin{solution}
\[
0
\]

Since $a_x < b_x$ there is no path from $b$ to $a$.  Since $b_z < a_z$
there is no path from $a$ to $b$.  So no path can go between $a$ and $b$.
\end{solution}

\ppart Let $N$ be the paths in $P_{O,c}$ that do \emph{not} go through $a$ or $b$.
Express $\card{N}$ in terms of the sizes $\card{P_{p,q}}$ for $p,q \in \set{O,a,b,c}$.

\examspace[2in]

\begin{solution}
\begin{align*}
\card{N} & = \card{P_{O,c}} - (\card{P_{O,a,c}} + \card{P_{O,b,c}} - \card{P_{O,a,b,c}})
                &  \text{(inclusion-exclusion)}\\
& = \card{P_{O,c}} - (\card{P_{O,a}}\card{P_{a,c}} + \card{P_{O,b}}\card{P_{b,c}} - 0)
               & \text{(parts~\eqref{poacpoa} and~\eqref{pab0})}
\end{align*}

\begin{comment}
\[
\binom{50}{20} -
\binom{20}{10} \cdot \binom{30}{10} - \binom{35}{15} \cdot \binom{15}{5}
+ \binom{20}{10} \cdot \binom{15}{5} \cdot \binom{15}{5}.
\]

$N_1 \intersect N_2$ is the set of paths from $(0, 0)$ to $(20, 30)$
that go through both $(10, 10)$ and $(15,20)$.  So $P - (N_1 \union
N_2)$ is the set of paths to be counted.  Now we have
\begin{align*}
\Card{P - (N_1 \union N_2)}
     & = \card{P} - \Card{N_1 \union N_2}\\
     & = \card{P} - \card{N_1} - \card{N_2} + \Card{N_1 \intersect N_2}
           & \text{by Inclusion-Exclusion}.
\end{align*}
Part~\ref{paths_through} shows how to calculate $\card{N_i}$.  Also,
there is a bijection between $N_1 \intersect N_2$ and the set of
triples consisting of a path $(0,0)$ to $(10,10)$, a path from
$(10,10)$ to $(15,20)$, and a path from $(15,20)$ to $(20,30)$.  So
the size of $N_1 \intersect N_2$ is the product of the sizes of these
three sets of paths.
\end{comment}
\end{solution}

\eparts
\end{problem}

\endinput
