\documentclass[problem]{mcs}

\begin{pcomments}
\pcomment{MQ_inclusion-exclusion_paths_afternoon}
\pcomment{perturbed from MQ_inclusion-exclusion_paths}
\end{pcomments}

\pkeywords{
  counting
  inclusion_exclsuion
  paths
  multinomial
}

%%%%%%%%%%%%%%%%%%%%%%%%%%%%%%%%%%%%%%%%%%%%%%%%%%%%%%%%%%%%%%%%%%%%%
% Problem starts here
%%%%%%%%%%%%%%%%%%%%%%%%%%%%%%%%%%%%%%%%%%%%%%%%%%%%%%%%%%%%%%%%%%%%%

\begin{problem}
We want to count step-by-step paths between points with integer
coordinates in three dimensions.  A step may move a unit distance in
the positive $x$, $y$ or $z$ direction.  For example, a step from
point $(2,3,7)$ in the $y$ direction leads to $(2,4,7)$.

For any set $S$ of the points, let
\[
P_S \eqdef \text{the paths that go through all the points in } S.
\]
For example, letting O be the origin $(0,0,0)$ and $p$ be the point
$(0,1,1)$, there are four paths in $P_{O, p,(1,2,1)}$:
\begin{align*}
O, (0,1,0), p, (1,1,1), (1,2,1). \\
O, (0,1,0), p, (0,2,1), (1,2,1). \\
O, (0,0,1), p, (1,1,1), (1,2,1). \\
O, (0,0,1), p, (0,2,1), (1,2,1).
\end{align*}

Let $O,a,b,c$ be the following points:
\begin{align*}
O & \eqdef (0,0,0),\\
a & \eqdef (4,9,14),\\
b & \eqdef (14,9,4),\\
c & \eqdef (9,19,29).
\end{align*}

\bparts

\ppart Express $\card{P_{a,c}}$ as a multinomial coefficient.

\examspace[2in]

\begin{solution}
There must be exactly $c_x-a_x=9-4=5$ $x$-steps, $c_y-a_y=19-9=10$
$y$-steps, and $c_z-a_z=29-14=15$ $z$-steps, for a total of 30
steps.  These steps may be taken in any order, so the number of paths
is the same as the number of permutations of the letters $x^5 y^{10}
z^{15}$, namely,
\[
\binom{30}{5,10,15}\,.
\]
\end{solution}

\ppart\label{aftpoacpoa} Express $\card{P_{O,a,c}}$ in terms of
$\card{P_{O,a}}$ and $\card{P_{a,c}}$.

\examspace[2in]

\begin{solution}
\begin{equation}\label{aftcqcpcn}
\card{P_{O,a,c}} = \card{P_{O,a}} \cdot \card{P_{a,c}}.
\end{equation}
Every path in $P_{O,a,c}$ parses uniquely into a path in $P_{O,a}$
followed by\inbook{\footnote{\emph{\idx{merged}} with, see
    Section~\bref{sec:diwalks}}} a path in $P_{a,c}$,
so~\eqref{aftcqcpcn} follows by the Product Rule.
\end{solution}

\examspace
\ppart\label{aftpab0} What is $\card{P_{a,b}}$? \hfill\examrule[0.75in]

\begin{solution}
\[
0
\]

The coordinates are weakly increasing under the step rules, so since
$a_x < b_x$ there is no path from $b$ to $a$.  Since $b_z < a_z$ there
is no path from $a$ to $b$.  So no path can go between $a$ and $b$.
\end{solution}


\ppart Let $N$ be the paths in $P_{O,c}$ that do \emph{not} go through $a$ or $b$.
Express $\card{N}$ in terms of the sizes $\card{P_{p,q}}$ for $p,q \in \set{O,a,b,c}$.

\examspace[2in]

\begin{solution}
\begin{align*}
\card{N} & = \card{P_{O,c}} - (\card{P_{O,a,c}} + \card{P_{O,b,c}} - \card{P_{O,a,b,c}})
                &  \text{(inclusion-exclusion)}\\
& = \card{P_{O,c}} - (\card{P_{O,a}}\card{P_{a,c}} + \card{P_{O,b}}\card{P_{b,c}} - 0)
               & \text{(parts~\eqref{aftpoacpoa} and~\eqref{aftpab0})}
\end{align*}

\end{solution}

\eparts

\end{problem}

\endinput
