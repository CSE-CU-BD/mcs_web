\documentclass[problem]{mcs}

\begin{pcomments}
  \pcomment{from: S09.ps6}
  \pcomment{from: F07.ps6 (revised)}
\end{pcomments}

\pkeywords{
  bipartite_matching
  Halls_Theorem
}

%%%%%%%%%%%%%%%%%%%%%%%%%%%%%%%%%%%%%%%%%%%%%%%%%%%%%%%%%%%%%%%%%%%%%
% Problem starts here
%%%%%%%%%%%%%%%%%%%%%%%%%%%%%%%%%%%%%%%%%%%%%%%%%%%%%%%%%%%%%%%%%%%%%

\begin{problem}

  Take a regular deck of $52$ cards.  Each card has a suit and a
  value. The suit is one of four possibilities: heart, diamond, club,
  spade. The value is one of $13$ possibilities, $A, 2, 3, \dots, 10,
  J, Q, K$.  There is exactly one card for each of the $4 \times
  13$ possible combinations of suit and value.

  Ask your friend to lay the cards out into a grid with 4 rows and 13
  columns.  They can fill the cards in any way they'd like.  In this 
  problem you will show that you can always pick out 13 cards, one 
  from each column of the grid, so that you wind up with cards of all 
  13 possible values.

  \bparts

  \ppart Explain how to model this trick as a bipartite matching 
  problem between the 13 column vertices and the 13 value vertices.  
  Is the graph necessarily degree constrained?

  \begin{solution}
Create a simple bipartite graph with 13 column vertices and
  13 value vertices.  Connect a column to a value by a single edge iff
  a card with that value is contained in that column.  A perfect 
  matching would then indicate the value of the card you would choose
  from each column.

  The graph may not be degree constrained if any one of the columns
  contains more than one card with the same value.  In the case where 
  the matching indicates a value that appears more than once in the 
  column it is matched to, you can arbitrarily pick any card of that 
  value in that column.
\end{solution}

  \ppart Show that any $n$ columns must contain at least $n$ different
  values and prove that a matching must exist.

  \begin{solution}
If $S$ is a set of columns, they contain $4|S|$ cards.  No
  card value repeats more than four times, so at least $|S|$ values
  must appear among those cards.  Thus $|N(S)| \geq |S|$ and Hall's
  theorem gives us a matching.
\end{solution}

\eparts

\end{problem}

%%%%%%%%%%%%%%%%%%%%%%%%%%%%%%%%%%%%%%%%%%%%%%%%%%%%%%%%%%%%%%%%%%%%%
% Problem ends here
%%%%%%%%%%%%%%%%%%%%%%%%%%%%%%%%%%%%%%%%%%%%%%%%%%%%%%%%%%%%%%%%%%%%%
