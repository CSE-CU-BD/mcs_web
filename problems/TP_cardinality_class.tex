\documentclass[problem]{mcs}

\begin{pcomments}
  \pcomment{TP_cardinality_class}
  \pcomment{CH 2/26/14}
\end{pcomments}

%%%%%%%%%%%%%%%%%%%%%%%%%%%%%%%%%%%%%%%%%%%%%%%%%%%%%%%%%%%%%%%%%%%%%
% Problem starts here
%%%%%%%%%%%%%%%%%%%%%%%%%%%%%%%%%%%%%%%%%%%%%%%%%%%%%%%%%%%%%%%%%%%%%
\begin{problem}

Group the following sets according to their cardinality. You only need
to write down the corresponding number in the correct column.

\begin{enumerate}

\item \label{itm:roots} The set of roots to the equation $x^2 = 1$.

\item \label{itm:naturals} The set of natural numbers $\naturals$.

\item \label{itm:rationals} The set of rational numbers $\rationals$.

\item \label{itm:reals} The set of real numbers $\reals$.

\item \label{itm:ints} The set of integers $\integers$.

\item \label{itm:complex} The set of complex numbers $\mathbb{C}$.

\item \label{itm:char} The set of words in the English language no more
  than 20 characters long.

\item \label{itm:self_bij} The set of all possible bijections from
  $\{1,2,\ldots,10\}$ onto itself. 

\item \label{itm:total_surj} A set $S$ with the property that there exists
  a total surjective function $f : \naturals \rightarrow S$.

\item \label{itm:union_count_uncount} The set $A \union B$, such that
  $A$ is a countable set and $B$ is an uncountable set.  

\end{enumerate}

\begin{center}
\begin{tabular}{c |c | c}
Finite & Countably infinite & Uncountable \\ \hline
  &  \ref{itm:naturals}  & \\
 && \\
&& \\
&&  \\
&&  \\
&&  \\
&&  
\end{tabular}
\end{center}

\end{problem}

\begin{solution}

\begin{center}
\begin{tabular}{c |c | c}
Finite & Countably infinite & Uncountable \\ \hline
\ref{itm:roots}  &  \ref{itm:naturals}  & \ref{itm:reals}\\
\ref{itm:char} & \ref{itm:ints} & \ref{itm:complex}\\
\ref{itm:self_bij}& \ref{itm:rationals} &  \ref{itm:union_count_uncount}\\
& \ref{itm:total_surj} &  
\end{tabular}
\end{center}

\end{solution}


%%%%%%%%%%%%%%%%%%%%%%%%%%%%%%%%%%%%%%%%%%%%%%%%%%%%%%%%%%%%%%%%%%%%%
% Problem ends here
%%%%%%%%%%%%%%%%%%%%%%%%%%%%%%%%%%%%%%%%%%%%%%%%%%%%%%%%%%%%%%%%%%%%%
\endinput
