\documentclass[problem]{mcs}

\begin{pcomments}
  \pcomment{MQ_Zn*prod}
  \pcomment{DRAFT ARM 10/3/13}
  \pcomment{overlaps MQ_relprime_closed}
\end{pcomments}

\pkeywords{
  modular_arithmetic
  relatively_prime
  inverse
  cancellable
  ring_mod_n
}

\begin{problem}
\bparts

\ppart\label{mrelprnrem} Prove that if $m$ is relatively prime to $n$,
then so is $\rem{m}{n}$.

\begin{solution}
A number is relatively prime to $n$ iff its gcd with $n$ is 1.
But by Lemma~\bref{lem:gcdrem},
\[
\gcd(m,n) = \gcd(\rem{m}{n},n),
\]
so if the left hand side equals 1, then so does the right hand side.
\end{solution}

\ppart Let $\relpr{n}$ be the ring of integers relatively prime to $n$.
Prove that if $k,m \in \relpr{n}$, then so is $k \cdot_n m \inhandout{
  \eqdef \rem{km}{n}}$.  \hint There are many ways to do this, but
using the result of part~\eqref{mrelprnrem}


\begin{solution}
\dots using the \emph{Unique Factorization Theorem}.

By Unique Factorization, the prime divisors of $k \cdot m$ are the
same as the prime divisors of $k$ along with the prime divisors of
$m$.  If $k$ and $m$ are relatively prime to $n$, they have no
prime divisors in common with $n$, so neither does $km$, that is,
$km$ is relatively prime to $n$.  But $\gcd(a,n) = \gcd(a,
\rem{a}{n})$ (Lemma~\bref{lem:gcdrem}), so $\rem{km}{n}$ \TBA{FIX}

\medskip

\dots using the fact that \emph{$k$ is relatively prime to $n$ iff
$k$ has an inverse modulo $n$.}

If $j_1$ is an inverse of $k_1$ modulo $n$, that is
\[
j_1k_1 \equiv 1 \pmod{n},
\]
and likewise $j_2$ is an inverse of $k_2$, then it follows immediately that
\[
(j_2j_1)(k_1k_2) \equiv 1 \pmod{n}.
\]
That is, $k_1k_2$ also has an inverse.  Since we know that $k_1k_2
\equiv k_1 \cdot_n k_2 \pmod{n}$, any inverse of $k_1k_2$ will also be
an inverse of $k_1 \cdot_n k_2$.

\medskip

\dots using the fact that \emph{$k$ is relatively prime to $n$ iff
$k$ is cancellable modulo $n$.}

If $k_1$ and $k_2$ are cancellable modulo $n$, then you can cancel
$k_1k_2$ by first cancelling $k_1$ and then cancelling $k_2$.
\end{solution}

\end{problem}

\endinput
