\documentclass[problem]{mcs}

\begin{pcomments}
  \pcomment{PS_N_to_A_diagonal_argument}
  \pcomment{from: S10.ps3 by ARM}
\end{pcomments}

\pkeywords{
  set_theory
  diagonal
  Russells_paradox
  surjection
  powerset
}

%%%%%%%%%%%%%%%%%%%%%%%%%%%%%%%%%%%%%%%%%%%%%%%%%%%%%%%%%%%%%%%%%%%%%
% Problem starts here
%%%%%%%%%%%%%%%%%%%%%%%%%%%%%%%%%%%%%%%%%%%%%%%%%%%%%%%%%%%%%%%%%%%%%

\begin{problem}
  For any sets, $A$, and $B$, let $[A \to B]$ be the set of total
  functions from $A$ to $B$.  Prove that if $A$ is not empty and $B$ has
  more than one element, then $\QNOT(A \surj [A \to B])$.

  \hint Suppose there is a function, $\sigma$, that maps each element $a
  \in A$ to a function $\sigma_a:A \to B$.  Pick any two elements of $B$;
  call them 0 and 1.  Then define
  \[
  \text{diag}(a) \eqdef \begin{cases} 0 \text{ if } \sigma_a(a) = 1,\\
                                      1 \text{ otherwise}.
                        \end{cases}
  \]

\begin{solution}

One proof follows the pattern of Theorem~\bref{powbig}:

\begin{proof}
  Suppose there is a function, $\sigma$, that maps each element $a \in A$
  to a function $\sigma_a:A \to B$.  We want to show that $\sigma$ is not
  a surjection.  In particular, we will define a total function
  $\text{diag}:A \to B$ such that $\text{diag} \neq \sigma_a$ for any $a
  \in A$.

  To do this, pick any two elements of $B$; call them 0 and 1.  Then define
  \[
  \text{diag}(a) \eqdef \begin{cases} 0 \text{ if } \sigma_a(a) = 1,\\
                                      1 \text{ otherwise}.
                        \end{cases}
  \]
  So by definition,
  \[
  \text{diag}(a) \neq \sigma_a(a)
  \]
  for all $a \in A$.  That is, $\text{diag}$ is a function in $[A \to B]$
  that is not in the range of $\sigma$, which proves that $\sigma$ is not
  a surjection, as claimed.
\end{proof}

Another proof uses the observation that $[A\to B] \surj \power(A)$.  For
example, the mapping that takes a function $f:A \to B$ to the set $\set{a
  \in A \suchthat f(a) \neq 0}$ defines a surjection from $[A\to B]$ to
$\power(A)$.  So if also, $A \surj [A \to B]$, then it would follow that
$A \surj \power(A)$, contradicting Theorem~\bref{powbig}.
\end{solution}

\end{problem}


%%%%%%%%%%%%%%%%%%%%%%%%%%%%%%%%%%%%%%%%%%%%%%%%%%%%%%%%%%%%%%%%%%%%%
% Problem ends here
%%%%%%%%%%%%%%%%%%%%%%%%%%%%%%%%%%%%%%%%%%%%%%%%%%%%%%%%%%%%%%%%%%%%%
\endinput
