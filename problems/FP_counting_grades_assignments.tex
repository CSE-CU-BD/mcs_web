\documentclass[problem]{mcs}

\begin{pcomments}
  \pcomment{FP_counting_grades_assignments}
  \pcomment{ZDz 11/13/15}
  \pcomment{f15.mid4}
\end{pcomments}

\pkeywords{
  counting
  bijection
  binomial
  sum_of_numbers
}

%%%%%%%%%%%%%%%%%%%%%%%%%%%%%%%%%%%%%%%%%%%%%%%%%%%%%%%%%%%%%%%%%%%%%
% Problem starts here
%%%%%%%%%%%%%%%%%%%%%%%%%%%%%%%%%%%%%%%%%%%%%%%%%%%%%%%%%%%%%%%%%%%%%

\begin{problem}
In $6.043$ there are $n$ students with distinct total scores.  The
staff needs to assign one of eleven letter grades \emph{A+, A, A-, B+,
  \dots, C-, D, F} to each student according to their scores---that
is, a student with a higher score cannot get a lower grade.  How many
grade assignments are possible?

\begin{solution}
\[
\binom{n+10}{10}.
\]

The grades assigned to the $n$ students in decreasing order of their
scores must be a \emph{weakly} decreasing sequence of grades.  So we
want a weakly decreasing sequences of $n$ integers in the interval
$\Zintv{0}{10}$.  We showed in class that there is a bijection between
such weakly decreasing sequences and bit strings of length $n+10$ with
ten one's.
\end{solution}

\iffalse

\bparts

\ppart
There must be at least one student with each grade.

\examspace[3in]

\begin{solution}
Let $(b_1, b_2, \dots, b_n)$ be a permutation of distinct student scores such that
$b_1 < b_2 < \cdots < b_n$, that is, the scores sorted in increasing order.

For the given example of scores, this list is $25, 30, 37, 50, 91$.
The problem of assigning grades is then equivalent to the problem of
choosing $k-1$ positions between scores that split all scores into $k$
nonempty groups. There are $n-1$ positions to choose from (between
$b_1$ and $b_2$, $b_2$ and $b_3$, \dots, $b_{n-1}$ and $b_n$).
Since chosen positions must be distinct so that $k$ groups of scores are
nonempty, there are $\binom{n-1}{k-1}$ ways to make a selection.
Therefore, the number of ways to assign grades such that there is
at least one student with each grade is $\binom{n-1}{k-1}$.

\textbf{Alternative solution.}
Let $(b_1, b_2, \dots, b_n)$ be a permutation of distinct student scores such that
$b_1 < b_2 < \cdots < b_n$, that is, the scores sorted in increasing order.

For the given example of scores, this list is $25, 30, 37, 50, 91$.
Let $g_i$ be a grade assigned to the student with score $b_i$, for $i=1,\ldots,n$.
Then, $g_1 \leq g_2 \leq \ldots \leq g_n$ must hold.
Note that the sequence of grades is $1, 1, 2, 3, 3$ for the given example.
Let $p_j$ be the position of the rightmost grade $j$ in this sequence.
In order to have at least one student with each grade
$1 \leq p_1 < p_2 < \ldots < p_k = n$ must hold. Therefore each assignment
of grades corresponds to a selection of integer numbers $p_1, \dots, p_{k-1}$
such that $1 \leq p_1 < p_2 < \cdots < p_{k-1} < n$.
The number of such selections is the number of ways to choose $k-1$ distinct
numbers out of $n-1$ possible values, which is $\binom{n-1}{k-1}$.
\end{solution}

\ppart
There are no other constraints.

\examspace[3.0in]

\begin{solution}
Let $x_1, \dots, x_k$ be the number of students assigned grade $1, \dots, k$,
respectively.  There is a bijection between grades assignments and the set
\[
A \eqdef \set{(x_1, \dots, x_k) | x_1, \dots, x_k \geq 0 \QAND \sum_{j=1}^k x_j = n}.
\]

To explain this bijection, let $(b_1, b_2, \ldots, b_n)$ be a
permutation of students scores such that $b_1 < b_2 < \cdots < b_n$,
that is, the list of scores sorted in increasing order.  Let $g_1,
\dots, g_n$ be grades assigned to these scores.  For any $(x_1, \dots,
x_k) \in A$, the corresponding assignment of grades is $g_i = 1$ for
$i \in \Zintv{1}{x_1}$, $g_i = 2$ for $i \in \Zintv{x_1+1}{x_1+x_2}$,
\dots, $g_i = k$ for $i \in
\Zintv{\paren{\sum_{j=1}^{k-1}x_j+1}}{\paren{\sum_{j=1}^{k}x_j}}$.  In
other words, the lowest $x_1$ scores are assigned grade $1$, the next
$x_2$ scores are assigned grade $2$, and so on.  This assignment is
unique since $g_1 \leq g_2 \leq \dots \leq g_n$ must hold (only the
lowest $x_1$ scores can be assigned $1$, etc.), and therefore the
mapping is a bijection.  Since there is a bijection between $A$ and
the set of binary strings of length $n+k-1$ with $k-1$ zeros,
$\card{A} = \binom{n+k-1}{k-1}$, which is also the number of possible
grades assignments.

\textbf{Alternative solution to part (a).}  If every grade must be
assigned to at least one student, there is a bijection between grades
assignments and the set $A = \set{(x_1, \dots, x_k) | x_1, \dots, x_k
  \geq 1 \QAND \sum_{j=1}^k x_j = n}$.  But, there is a bijection
between $A$ and the set $B = \set{(y_1, \dots, y_k) | y_1, \dots, y_k
  \geq 0 \QAND \sum_{j=1}^k y_j = n-k}$ by simply taking $y_j = x_j -
1$ for $j \in \Zintv{1}{k}$.  Since $\card{B} = \binom{n-k+k-1}{k-1} =
\binom{n-1}{k-1}$, that is also the number of possible assignments of
grades in which there is at least one student with each grade.
\end{solution}

\eparts
\fi

\end{problem}

%%%%%%%%%%%%%%%%%%%%%%%%%%%%%%%%%%%%%%%%%%%%%%%%%%%%%%%%%%%%%%%%%%%%%
% Problem ends here
%%%%%%%%%%%%%%%%%%%%%%%%%%%%%%%%%%%%%%%%%%%%%%%%%%%%%%%%%%%%%%%%%%%%%

\endinput
