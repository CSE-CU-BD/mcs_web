\documentclass[problem]{mcs}

\begin{pcomments}
  \pcomment{FP_counting_grades_assignments}
  \pcomment{ZDz 11/13/15}
  \pcomment{f15.mid4}
\end{pcomments}

\pkeywords{
  counting
  bijection
  binomial
  sum_of_numbers
}

%%%%%%%%%%%%%%%%%%%%%%%%%%%%%%%%%%%%%%%%%%%%%%%%%%%%%%%%%%%%%%%%%%%%%
% Problem starts here
%%%%%%%%%%%%%%%%%%%%%%%%%%%%%%%%%%%%%%%%%%%%%%%%%%%%%%%%%%%%%%%%%%%%%

\begin{problem}
In $6.043$ there are $n$ students, who receive final scores $a_1, a_2, \ldots, a_n$.
It happens that these scores are all different ($a_i \neq a_j$ if $i \neq j$).
$6.043$ staff needs to assign final grades. Each student must receive one of the grades
$1,\ldots,k$ such that for any two students a student with a higher score
cannot receive a lower grade. For example, if there are $n=5$ students with scores
$30, 50, 25, 91, 37$, and $k=3$, then $1, 3, 1, 3, 2$ is a valid assignment of grades,
while $1, 3, 1, 2, 2$ is not.
Compute the number of possible assignments of grades to students (and provide
explanation) if:

\bparts

\ppart
There must be at least one student with each grade.

\examspace[3in]

\begin{solution}
Let $(b_1, b_2, \ldots, b_n)$ be a permutation of students scores such that
$b_1 < b_2 < \ldots < b_n$, i.e., the list of scores sorted in increasing order.
For the given example of scores, this list is $25, 30, 37, 50, 91$.
The problem of assigning grades is then equivalent to the problem of
choosing $k-1$ positions between scores that split all scores into $k$
nonempty groups. There are $n-1$ positions to choose from (between
$b_1$ and $b_2$, $b_2$ and $b_3$, ...., $b_{n-1}$ and $b_n$).
Since chosen positions must be distinct so that $k$ groups of scores are
non-empty, there is ${n-1 \choose k-1}$ ways to make a selection.
Therefore, the number of ways to assign grades such that there is
at least one student with each grade is ${n-1 \choose k-1}$.

\textbf{Alternative solution.}
Let $(b_1, b_2, \ldots, b_n)$ be a permutation of students scores such that
$b_1 < b_2 < \ldots < b_n$, i.e., the list of scores sorted in increasing order.
For the given example of scores, this list is $25, 30, 37, 50, 91$.
Let $g_i$ be a grade assigned to the student with score $b_i$, for $i=1,\ldots,n$.
Then, $g_1 \leq g_2 \leq \ldots \leq g_n$ must hold.
Note that the sequence of grades is $1, 1, 2, 3, 3$ for the given example.
Let $p_j$ be the position of the rightmost grade $j$ in this sequence.
In order to have at least one student with each grade
$1 \leq p_1 < p_2 < \ldots < p_k = n$ must hold. Therefore each assignment
of grades corresponds to a selection of integer numbers $p_1, \ldots, p_{k-1$
such that $1 \leq p_1 < p_2 < \ldots < p_{k-1} \leq n-1$.
The number of such selections is the number of ways to choose $k-1$ distinct
numbers out of $n-1$ possible values, which is ${n-1 \choose k-1}$.
\end{solution}

\ppart
There are no other constraints.

\examspace[3.0in]

\begin{solution}
Let $x_1, \ldots, x_k$ be the number of students assigned grade $1, \ldots, k$,
respectively. There is a bijection between grades assignments and the set
$A = \set{(x_1, \ldots, x_k) | x_1, \ldots, x_k \geq 0 \QAND \sum_{j=1}^k x_j = n}$.
To see that, 
let $(b_1, b_2, \ldots, b_n)$ be a permutation of students scores such that
$b_1 < b_2 < \ldots < b_n$, i.e., the list of scores sorted in increasing order.
Let $g_1, \ldots, g_n$ be grades assigned to these scores.
For any $(x_1, \ldots, x_k) \in A$, the corresponding assignment of grades
is $g_i = 1$ for $i  \in \left[1, x_1\right]$, $g_i = 2$ for $i \in \left[x_1+1, x_1+x_2\right]$,
...,  $g_i = k$ for $i \in\left[(\sum_{j=1}^{k-1}x_j)+1, \sum_{j=1}^{k}x_j\right]$.
In other words, the lowest $x_1$ scores are assigned grade $1$, the next
$x_2$ scores are assigned grade $2$, and so on.
This assignment is unique since $g_1 \leq g_2 \leq \ldots \leq g_n$ must
hold (only the lowest $x_1$ scores can be assigned $1$, etc.),
and therefore the mapping is a bijection.
Since there is a bijection between $A$ and the set of binary strings of 
length $n+k-1$ with $k-1$ zeros, $\card{A} = {n+k-1 \choose k-1}$, which
is also the number of possible grades assignments.

\textbf{Alternative solution to part (a).}
If every grade must be assigned to at least one student, there is a bijection
between grades assignments and the set
$A = \set{(x_1, \ldots, x_k) | x_1, \ldots, x_k \geq 1 \QAND \sum_{j=1}^k x_j = n}$.
But, there is a bijection between $A$ and the set
$B = \set{(y_1, \ldots, y_k) | y_1, \ldots, y_k \geq 0 \QAND \sum_{j=1}^k y_j = n-k}$
by simply taking $y_j = x_j - 1$ for $j = 1, \ldots, k$.
Since $\card{B} = {n-k+k-1 \choose k-1} = {n-1 \choose k-1}$, that is also the
number of possible assignments of grades in which there is at least
one student with each grade.
\end{solution}

\eparts

\end{problem}

%%%%%%%%%%%%%%%%%%%%%%%%%%%%%%%%%%%%%%%%%%%%%%%%%%%%%%%%%%%%%%%%%%%%%
% Problem ends here
%%%%%%%%%%%%%%%%%%%%%%%%%%%%%%%%%%%%%%%%%%%%%%%%%%%%%%%%%%%%%%%%%%%%%

\endinput
