\documentclass[problem]{mcs}

\begin{pcomments}
  \pcomment{FP_random_grid_walk}
  \pcomment{ZDz 12/08/15, rewritten by ARM 12/11/15}
  \pcomment{f15.final}
\end{pcomments}

\pkeywords{
  state_machine
  preserved_invariant
  counting
  probability
}

%%%%%%%%%%%%%%%%%%%%%%%%%%%%%%%%%%%%%%%%%%%%%%%%%%%%%%%%%%%%%%%%%%%%%
% Problem starts here
%%%%%%%%%%%%%%%%%%%%%%%%%%%%%%%%%%%%%%%%%%%%%%%%%%%%%%%%%%%%%%%%%%%%%

\begin{problem}
We are interested in paths in the plane starting at $(0,0)$ that go
one unit right or one unit up at each step.  To model this, we use a
state machine whose states are $\nngint \times \nngint$, whose
start state is $(0,0)$, and whose transitions are
\begin{align*}
(x, y) & \to (x+1,y),\\
(x, y) & \to (x,y+1).
\end{align*}
\bparts

\ppart How many length $n$ paths are there starting from the origin?
\hfill \examrule[0.7in]

\begin{solution}
$2^n$.
\end{solution}

\ppart How many states are reachable in exactly $n$ steps? \hfill
\examrule[0.7in]

\begin{solution}
$\card{\set{(x,y) \suchthat x + y = n}} = n+1$.
\end{solution}

\ppart How many states are reachable in at most $n$ steps?\hfill
\examrule[0.9in]

\examspace[0.5in]

\begin{solution}
\[\frac{(n+1)(n+2)}{2}.\]

\begin{align*}
\#\text{states reachable in $\leq n$ steps}
& = \lefteqn{\card{{\set{(x,y) \suchthat x + y \leq n}}}}\\
& = \card{{\set{(x,y,z) \suchthat x + y + z =  n}}}\\
& = \binom{n+2}{2} = \frac{(n+1)(n+2)}{2}.
\end{align*}
\end{solution}

\ppart If transitions occur independently at random, going right with
probability $p$ and up with probability $q \eqdef 1-p$ at each step,
what is the probability of reaching position $(x,y)$?  \hfill
\examrule[1.0in]

\examspace[0.75in]

\begin{solution}
\[
\pr{(x,y)} = \binom{x+y}{x}p^xq^y.
\]

There have to be exactly $x$ right moves and $y$ up moves to get from
$(0,0)$ to $(x,y)$.  Each such path has probability $p^{x}q^{y}$ and
there are $\binom{x+y}{x}$ such paths.
\end{solution}

\ppart What is the probability of reaching state $(x,y)$ \emph{given}
that the path to $(x,y)$ reached $(m,n)$ before getting to $(x,y)$?
\hfill \examrule[1.0in]

\examspace[0.75in]

\begin{solution}
\[
\binom{x+y - (m+n)}{x-m}p^{x-m}q^{y-n},
\]

There have to be exactly $x-m$ right moves and $y-n$ up moves to get
from $(m,n)$ to $(x,y)$.  Each such path has probability
$p^{x-m}q^{y-n}$ and there are $\binom{(x-m)+(y -n ))}{x-m}$ such
paths.

Note that this conditional probability is zero if it is not possible
to get from $(m,n)$ to $(x,y)$, that is, if $m >x$ or $n>y$.
\end{solution}

\ppart Show that the probability that a path ending at $(x,y)$ went
through $(m,n)$ is the same for all $p$.

\examspace[2.0in]

\begin{solution}
All paths from the origin to $(x,y)$ have the same probability, so the
regardless of $p$, the probability of passing through $(m,n)$ \emph{en
  route} to $(x,y)$ is just the fraction of these paths that goes
through $(m,n)$.

Even without this insight, the claim can be verified using Bayes Theorem:

\begin{align*}
\prcond{(m,n)}{(x,y)}
   & = \frac{\pr{(m,n)}\prcond{(x,y)}{(m,n)}}{\pr{(x,y)}}\\
   & = \frac{\binom{m+n}{m}p^mq^n \cdot \binom{x+y - (m+n)}{x-m}p^{x-m}q^{y-n}}{\binom{x+y}{x}p^xq^y}\\
   & = \frac{\binom{m+n}{m} \cdot \binom{x+y - (m+n)}{x-m}}{\binom{x+y}{x}}.
\end{align*}

\end{solution}

\eparts

\end{problem}

%%%%%%%%%%%%%%%%%%%%%%%%%%%%%%%%%%%%%%%%%%%%%%%%%%%%%%%%%%%%%%%%%%%%%
% Problem ends here
%%%%%%%%%%%%%%%%%%%%%%%%%%%%%%%%%%%%%%%%%%%%%%%%%%%%%%%%%%%%%%%%%%%%%

\endinput
