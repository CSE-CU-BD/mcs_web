\documentclass[problem]{mcs}

\begin{pcomments}
  \pcomment{PS_pigeonhole-diff_2}
  \pcomment{ZDz, 11/24/15, F15.mid4.conflict}
  \pcomment{soln edited ARM 3/24/16}
\end{pcomments}

\pkeywords{
  pigeonhole
  difference
}

%%%%%%%%%%%%%%%%%%%%%%%%%%%%%%%%%%%%%%%%%%%%%%%%%%%%%%%%%%%%%%%%%%%%%
% Problem starts here
%%%%%%%%%%%%%%%%%%%%%%%%%%%%%%%%%%%%%%%%%%%%%%%%%%%%%%%%%%%%%%%%%%%%%

\begin{problem}
Suppose $2n+1$ numbers are selected from $\set{1,2,3, \dots,4n}$.
Using the Pigeonhole Principle, show that there must be two selected
numbers whose difference is $2$.  Clearly indicate what are the
pigeons, holes, and rules for assigning a pigeon to a hole.

\begin{solution}
The $2n+1$ selected numbers are the pigeons.  The pigeonholes will be
the 2-element $\set{2k+1,2k+3}, \set{2k+2,2k+4}$ for $k = 0,4,8\dots
4(n-1)$.  There are $2n$ pairwise disjoint holes, and their union is
$\set{1,2,3, \dots,4n}$, so assigning a pigeon to the hole that
contains it is an unambigous pigeonhole assignment.

Since there are more pigeons than holes, there must be two assigned to
some hole, which means their difference is $2$.
\end{solution}

\end{problem}

\endinput
