\documentclass[problem]{mcs}

\begin{pcomments}
  \pcomment{exit-diagnostic-short}
  \pcomment{JD 5/14/14; revised by CH}
\end{pcomments}

\pkeywords{
  probability
  counting
  logic
  relation
}

%%%%%%%%%%%%%%%%%%%%%%%%%%%%%%%%%%%%%%%%%%%%%%%%%%%%%%%%%%%%%%%%%%%%%
% Problem starts here
%%%%%%%%%%%%%%%%%%%%%%%%%%%%%%%%%%%%%%%%%%%%%%%%%%%%%%%%%%%%%%%%%%%%%
\begin{problem}

\bparts

\ppart \mbox{}

\begin{romanlist}
\item The power set of the set $\set{a, b, \emptyset}$ is:
\begin{center}
\exambox{5.0in}{0.4in}{0.0in}
\end{center}

\begin{solution}
\[
\set{ \emptyset, \set{a}, \set{b}, \set{\emptyset}, \set{a,b},
  \set{\emptyset, a}, \set{\emptyset, b}, \set{a, b, \emptyset} }
\]
\end{solution}

\examspace[0.1in]

\item 
What is the number of elements in the power set of $\set{2, 4,
  \dots, 14 }$?\hfill \examrule[0.5in]

\begin{solution}
\[
2^7 = 128 .
\]
\end{solution}

\item Circle the condition(s) under which the size of the intersection of
two finite sets $A$ and $B$ is the sum of their sizes.

\begin{center}
Always \qquad $A \union B = \emptyset$ \qquad $A \subset B$ \qquad
$A \intersect B \neq \emptyset$ \qquad Never
\end{center}

\begin{solution}
\[
A \union B = \emptyset.
\]
\end{solution}

\end{romanlist}

\ppart In a small village there live 20 families. 4 families have 1
child each, 5 families have 2 children with different ages each, 2
families have twins, 3 families have 3 children each, and 1 family has
4 children.

A family was chosen uniformly at random. What is the probability of
the family having:
\begin{romanlist}

\item no children at all?\hfill \examrule[0.5in]

\examspace[0.5in]

\begin{solution}
\[
\frac{5}{20} = \frac{1}{4} .
\]
\end{solution}

\item 2 or 3 children?\hfill \examrule[0.5in]

\examspace[0.5in]

\begin{solution}
\[
\frac{1}{2}
\]
\end{solution}

\item fewer than 3 children?\hfill \examrule[0.5in]

\examspace[0.5in]

\begin{solution}
\[
\frac{16}{20} = \frac{4}{5} .
\]
\end{solution}

\end{romanlist}

\ppart Explain the following concepts.

\begin{romanlist}
\item Walk relation in a digraph:

\begin{center}
\exambox{6.0in}{1.0in}{0in}
\end{center}

\begin{solution}
\TBA{}
\end{solution}

\item Mutual independence:

\begin{center}
\exambox{6.0in}{1.0in}{0in}
\end{center}

\begin{solution}
\TBA{}
\end{solution}

\end{romanlist}

\ppart For each of the following formulas, state whether it is valid
or not.

\begin{romanlist}
\item $\QNOT(\forall x \in A.\, P(x))\ \QIFF\ \exists x \in A.\, \QNOT(P(x))$.\hfill \examrule[0.5in]
\item $(P(x) \QAND \QNOT(Q(x)))\  \QIFF\  \QNOT(P(x) \QIMPLIES Q(x))$.\hfill \examrule[0.5in]
\item $\forall x \, \exists y. \, (P(x) \QOR Q(y))\ \QIFF\  \exists x \,
  \forall y \, \QNOT(P(x)) \QAND \QNOT(Q(y))$ . \hfill \examrule[0.5in]

\item $\forall x \, \exists y. \ x^2 + y^2 < 17$ where the domain of
  discourse is $\set{1, 2, 3, 4}$.\hfill \examrule[0.5in]
\end{romanlist}

\begin{solution}
valid; valid; not valid; not valid.
\end{solution}

\ppart What propositional connector has the following truth table? \hfill\examrule[0.5in]
\[\begin{array}{|c|c|c|}
\hline
P & Q & P\ \mathbf{?}\ Q\\
\hline
\true & \true & \false\\
\true & \false & \true\\
\false & \true & \true\\
\false & \false & \false\\
\hline
\end{array}\]

\begin{solution}
The $\QXOR$ connector.
\end{solution}

\ppart The probability that a move in a certain game will succeed is
2/3.  Three such moves are done independently.  What is the
probability that:

\begin{romanlist}

\item Not all three moves will succeed?\hfill \examrule[0.7in]

\examspace[0.5in]

\begin{solution}
\[
1 - (2/3)^2 = 19/27 .
\]
\end{solution}

\item At most one move will succeed? \hfill \examrule[0.7in]

\examspace[0.5in]

\begin{solution}
\[
(1/3)^3 + 3 (2/3) (1/3)^2 = 7/27.
\]
\end{solution}

\end{romanlist}

\ppart Let $p(n) \eqdef n^2 - 16$.  What is the largest prime factor
of $p(51)$? \hfill \examrule[0.7in]

\begin{solution}
\textbf{47}.

$p(n) = (n - 4)(n + 4)$ So $p(51) = 47 \cdot 55 = 47 \cdot 11 \cdot 5 $.
\end{solution}

\ppart Express the value of following sum in terms of a Harmonic number.\hfill \examrule[0.7in]
\[
2/1 + 2/2 + 2/3 + \dots + 2/53.
\]

\begin{solution}
\[
2 H_{53} .
\]
\end{solution}

\ppart Circle all the expressions below that equal the number of
size-5 subsets of the set $\set{1,2, 3, 4, 5, 6, 7}$.

\begin{center}
$7!/(2! 5!)$
\qquad $3 \cdot 7$
\qquad $5 \cdot 5!$
\qquad $7!/( 5! \cdot 5!)$
\qquad $7!/( 3! \cdot 4!)$
\end{center}

\begin{solution}
\[
7!/(2! 5!),~3 \cdot 7 .
\]
\end{solution}

\ppart A shooter shoots two bullets at a target. 

The probability that he will hit the target in the first shot is
0.70. If he hits the target in his first shot, then the probability
that he will hit the target in the second shot is 0.80. If he misses
the target in his first shot, then the probability that he will hit
the target in the second shot is $p$.

The probability that he will hit the target in the second shot is
0.62.  Circle the value of $p$ determined by this information.

\begin{center}
0.80 \qquad 0.62 \qquad 0.20 \qquad 0.56 \qquad Other
\end{center}

\begin{solution}
0.20 .

\TBA{EXPLANATION}
\end{solution}

\ppart A jar contains an equal number of green and red balls,
indistinguishable except for color.  George is blindfolded and asked to
pick a ball from a jar and guess its color.  Before guessing, he gets
told the color of the ball by two friends, Anna and Sam.  Anna is
colorblind, so her answer is random.  Sam can see the ball George
draws and answers correctly 3/4 of the time.  If Sam and Anna do not
agree, George returns the ball to the jar, and picks again.  If they
do agree, George will use their answer.

\begin{romanlist}

\item What is the probability that George will respond correctly?\hfill \examrule[0.6in]

\begin{solution}
3/8.
\end{solution}

\item Suppose that there is only one ball in the jar.  Should your
  previous answer change?  Explain:

\begin{center}
\exambox{6.0in}{1.0in}{0in}
\end{center}
\begin{solution}
Doesn't matter.  The previous reasoning that led to the 3/8
probability did not depend on how many balls of either color were in
the jar.
\end{solution}

\end{romanlist}
\eparts

\end{problem}
%%%%%%%%%%%%%%%%%%%%%%%%%%%%%%%%%%%%%%%%%%%%%%%%%%%%%%%%%%%%%%%%%%%%%
% Problem ends here
%%%%%%%%%%%%%%%%%%%%%%%%%%%%%%%%%%%%%%%%%%%%%%%%%%%%%%%%%%%%%%%%%%%%%

\endinput
