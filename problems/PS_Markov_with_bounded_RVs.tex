\documentclass[problem]{mcs}

\begin{pcomments}
  \pcomment{}
\end{pcomments}

\pkeywords{
}

%%%%%%%%%%%%%%%%%%%%%%%%%%%%%%%%%%%%%%%%%%%%%%%%%%%%%%%%%%%%%%%%%%%%%
% Problem starts here
%%%%%%%%%%%%%%%%%%%%%%%%%%%%%%%%%%%%%%%%%%%%%%%%%%%%%%%%%%%%%%%%%%%%%

\begin{problem}
Suppose $R$ is a nonnegative random variable.  Then for any $x > 
\expect{R}$\footnote{The less demanding constraint $x > 0$ would still
allow Markov's Theorem to be used, but the resulting bound would be 
useless if $x\leq\expect{R}$.}, direct application of Markov's Theorem gives 
\begin{equation}\label{MarkovBound}
\pr{R\geq x}\leq\frac{\expect{R}}{x}
\end{equation}
\begin{problemparts}

\ppart\label{boundpart} Suppose it is possible to find some $l > 0$ for which $R-l$ is
nonnegative.  Then Markov's Theorem could be applied to $R-l$
rather than to $R$.  Show that this approach yields a new upper bound 
on $\pr{R\geq x}$. 
\begin{solution}
Define
\[
T\eqdef R-1.
\]
Then $T$ is a nonnegative random variable with expectation
\[
\expect{T} = \expect{R - l} = \expect{R} - l.
\]
Obviously, since $x > \expect{R}$, therefore 
\[
x - l > \expect{R} - l = \expect{T} \geq 0
\] 
Markov's Theorem can therefore be applied to $T$ to give
\begin{align*}
\prob{T \geq x - l}
&\leq \frac{\expect{T}}{x - l} \\
&=    \frac{\expect{R} - l}{x - l}.
\end{align*}
However,
\begin{align*}
\prob{R \geq x}
&= \prob{R - l \geq x - l} \\
&= \prob{T \geq x - l}.
\end{align*}
Thus, we have a new upper bound:
\[
\prob{R \geq x} \leq \frac{\expect{R} - l}{x - l}.
\]
\end{solution} 

\ppart\label{tightpart} Show that this new upper bound is tighter (smaller)
than that given by~\eqref{MarkovBound}.
\begin{solution}
We are being asked to show that
\[
\frac{\expect{R} - l}{x - l} < \frac{\expect{R}}{x}
\]
Since $x$, $l$, and $x-l$ are all positive, therefore
\begin{align*}
&& \frac{\expect{R} - l}{x - l} &< \frac{\expect{R}}{x}\\
& \qiff & x\expect{R} - lx &< x\expect{R} - l\expect{R}\\
& \qiff & -lx &< -l\expect{R}\\
& \qiff & x &> \expect{R}
\end{align*}
Since $x$ \textit{is} larger than $\expect{R}$, therefore 
\[
\frac{\expect{R} - l}{x - l} < \frac{\expect{R}}{x},
\]
as required.
\end{solution}

\ppart Suppose $R$ is never zero.  Explain why choosing $l$ equal to the minimum value of 
$R$ yields the tightest possible upper bound on $\prob{R \geq x}$ 
that can be achieved using the approach of part~\ref{boundpart}.
\begin{solution}
Let $m$ denote the minimum value of $R$.  Since $R$ is nonnegative and nonzero, therefore $m > 0$.  
We demanded that $l > 0$ and $R-l$ be nonnegative.  These requirements are met iff $l\in(0,m]$.

Choose $l_1,l_2\in(0,m]$ such that $l_1<l_2$.  Now, we know from parts~\ref{boundpart} and~\ref{tightpart}
that Markov's Theorem can be applied to $R-l_1$ to give $\prob{R \geq x}\leq\frac{\expect{R} - l_1}{x - l_1}$,
and to $R-l_2$ to give $\prob{R \geq x}\leq\frac{\expect{R} - l_2}{x - l_2}$, both of which give
tighter bounds than is obtained by applying Markov's Theorem to $R$ directly to obtain ~\eqref{MarkovBound}.

Let $R_1=R-l_1$, $\Delta l = l_2-l_1$, and $x_1=x-l_1$.  Clearly, $R_1$ is nonnegative, as is $R_1-\Delta l=R_2$.  Also,
$\Delta l > 0$ and $x_1 = x - l_1 > \expect{R} - l_1 = \expect{R-l_1} = \expect{R_1}$.  So we can apply our approach and results
here: Markov's Theorem on $R_1-\Delta l$ gives $\prob{R_1 \geq x_1}\leq\frac{\expect{R_1} - \Delta l}{x_1 - \Delta l}$, and
this bound is tighter than that given by $\prob{R_1 \geq x_1}\leq\frac{\expect{R_1}}{x_1}$.
Now, $\pr{R_1 \geq x_1}=\pr{R - l_1 \geq x - l_1}=\pr{R \geq x}$, $\frac{\expect{R_1} - \Delta l}{x_1 - \Delta l}=\frac{\expect{R} - l_2}{x - l_2}$, and
$\frac{\expect{R_1}}{x_1}=\frac{\expect{R-l_1}}{x-l_1}$.  Thus we have that $\pr{R \geq x}\leq\frac{\expect{R} - l_2}{x - l_2}$
gives a tighter upper bound for  $\pr{R \geq x}$ than $\pr{R \geq x}\leq\frac{\expect{R} - l_1}{x - l_1}$ does.  
Therefore, applying Markov's Theorem to $R-l_2$ yields a tighter upper bound on the probability of interest than does applying
Markov's Theorem to $R-l_1$.  

So to get the tightest possible upper bound by this approach, we must choose the largest possible $l > 0$ that keeps $R-l$ nonnegative -- that is,
the largest possible $l\in(0,m]$.  So $l=m$ gives the tightest bound.
\end{solution} 
\end{problemparts}
\end{problem} 

%%%%%%%%%%%%%%%%%%%%%%%%%%%%%%%%%%%%%%%%%%%%%%%%%%%%%%%%%%%%%%%%%%%%% 
% Problem ends here 
%%%%%%%%%%%%%%%%%%%%%%%%%%%%%%%%%%%%%%%%%%%%%%%%%%%%%%%%%%%%%%%%%%%%%

\endinput
