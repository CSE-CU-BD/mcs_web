\documentclass[problem]{mcs}

\begin{pcomments}
\pcomment{TP_inc-exc-empty}
\end{pcomments}

\pkeywords{
  inclusion-exclusion
  sum_rule
  disjoint
  empty_set
}

%%%%%%%%%%%%%%%%%%%%%%%%%%%%%%%%%%%%%%%%%%%%%%%%%%%%%%%%%%%%%%%%%%%%%
% Problem starts here
%%%%%%%%%%%%%%%%%%%%%%%%%%%%%%%%%%%%%%%%%%%%%%%%%%%%%%%%%%%%%%%%%%%%%

\begin{problem}
Explain in a couple of sentences why the Sum Rule for two disjoint sets is a
special case of the Inclusion-exclusion Rule for two sets.

\examspace[2.0in]

\begin{solution}
If $A$ and $B$ are disjoint sets, then $\card{A \intersect B} = 0$, so
the Inclusion-exclusion Rule
\[
\card{A \union B} = \card{A} + \card{B} - \underbrace{\card{A \intersect B}}_0
\]
reduces to the Sum Rule
\[
\card{A \union B} = \card{A} + \card{B}.
\]
\end{solution}
\end{problem}

\endinput
