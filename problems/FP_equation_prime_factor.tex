\documentclass[problem]{mcs}

\begin{pcomments}
  \pcomment{FP_equation_prime_factor}
  \pcomment{F16, midterm1}
  \pcomment{ARM 2/24/16}
\end{pcomments}

\pkeywords{
  irrational
  contradiction
  prime_factor
  root
  rational
}

\begin{problem}
A familiar proof that $\sqrt[3]{7^2}$ is irrational depends on the
fact that a certain equation among those below is unsatisfiable by
integers $a, b > 0$.  (Note that more than one is unsatisfiable, but
only one of them is relevant.)  \inbook{Indicate}\inhandout{Circle}
the equation that would appear in the proof, and explain why it is
unsatisfiable.  (Do \emph{not} assume that $\sqrt[3]{7^2}$ is
irrational.)
\begin{enumerate}[i.]
\item $a^2 = 7^2+ b^2$
\item $a^3 = 7^2 + b^3$
\item $a^2 = 7^2b^2$
%\item $a^3 = 7b^3$
\item $a^3 = 7^2b^3$
\item $a^3 = 7^3b^3$
\item $(ab)^3 = 7^2$
\end{enumerate}

\begin{staffnotes}
20pt problem, 2pts (that is, -18pt) for picking an irrelevant
equation, say $(ab)^3 = 7^2$, and showing it is unsatisfiable.
\end{staffnotes}

\begin{solution}
\[
a^3 = 7^2b^3.
\]

By the unique factorization theorem, the prime factorization of $a^3$
is obtained by repeating the prime factors of $a$ three times.  So
every prime in the factorization of $a^3$ must appear three times.
Likewise for $b^3$.

But the number $7^2b^3$ of appearances of 7 in the factorization of
the right-hand side leaves a remainder of two when divided by three,
and so will not equal the number of appearances of 7 in factorization
of the left-hand side $a^3$.
\end{solution}
\end{problem}

\endinput
