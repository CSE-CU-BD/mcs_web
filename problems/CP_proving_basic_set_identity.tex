\documentclass[problem]{mcs}

\begin{pcomments}
  \pcomment{from: S09.cp2r}
%  \pcomment{}
%  \pcomment{}
\end{pcomments}

\pkeywords{
  logic
  set_theory
}

%%%%%%%%%%%%%%%%%%%%%%%%%%%%%%%%%%%%%%%%%%%%%%%%%%%%%%%%%%%%%%%%%%%%%
% Problem starts here
%%%%%%%%%%%%%%%%%%%%%%%%%%%%%%%%%%%%%%%%%%%%%%%%%%%%%%%%%%%%%%%%%%%%%

\begin{problem}
\emph{Set Formulas and Propositional Formulas.}
\bparts

\ppart\label{verprop}

Verify that the propositional formula $(P\ \QAND\ \,
\QNOT(Q))\ \QOR\ (P\ \QAND\ Q)$ is equivalent to $P$.

\begin{solution}
There is a simple verification by truth table with 4 rows which we omit.

There is also a simple cases argument: if $Q$ is \true, then the formula
simplifies to $(P\ \QAND\ \false)\ \QOR\ (P\ \QAND\ \true)$ which further
simplifies to $(\false\ \QOR\ P)$ which is equivalent to $P$.

Otherwise, if $Q$ is \textcolor{red}{F}, then the formula simplifies to
$(P\ \QAND\ \true)\ \QOR\ (P\ \QAND\ \false)$ which
is likewise equivalent to $P$.
\end{solution}

\ppart Use part~\eqref{verprop} to prove that
\[
A = (A-B) \union (A \intersect B)
\]
for any sets, $A,B$, where
\[
A-B \eqdef \set{a \in A \suchthat a \notin B}.
\]

\begin{solution}
We need only show that the two sets have the same elements, that is $x$ is
in one set iff $x$ is in the other set, for any $x$.

Let $P$ be $x \in A$ and $Q$ be $x \in B$.  Then
\begin{align*}
\lefteqn{x \in (A-B) \union (A \intersect B)}\\
 & \qiff x \in (A-B)\ \QOR\ x \in (A
\intersect B) & \text{(by def of $\union$)}\\
& \qiff (x \in A\ \QAND\  \QNOT(x \in B))\ \QOR\ (x \in A\ \QAND\  x \in B) 
  & \text{(by def of $\intersect$ and $-$)}\\
& \qiff (P\ \QAND\  \QNOT(Q))\ \QOR\ (P\ \QAND\ Q) & \text{(by def of
  $P$ and $Q$)}\\
&\qiff P & \text{(by part~\eqref{verprop})}\\
& \qiff x \in A & \text{(by def of $P$)}.
\end{align*}

\end{solution}

\eparts
\end{problem}

%%%%%%%%%%%%%%%%%%%%%%%%%%%%%%%%%%%%%%%%%%%%%%%%%%%%%%%%%%%%%%%%%%%%%
% Problem ends here
%%%%%%%%%%%%%%%%%%%%%%%%%%%%%%%%%%%%%%%%%%%%%%%%%%%%%%%%%%%%%%%%%%%%%

\endinput
