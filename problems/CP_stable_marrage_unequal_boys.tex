\documentclass[problem]{mcs}

\begin{pcomments}
  \pcomment{CP_stable_marrage_unequal_boys}
  \pcomment{renamed from CP_matching_more_boys}
  \pcomment{by ARM 3/15/11, revised 4/6/15}
  \pcomment{similar to CP_stable_marriage_invariant_review,
            but requires slightly more insight}
\end{pcomments}

\pkeywords{
 stable_matching
 Mating_ritual
 invariant
}


%%%%%%%%%%%%%%%%%%%%%%%%%%%%%%%%%%%%%%%%%%%%%%%%%%%%%%%%%%%%%%%%%%%%%
% Problem starts here
%%%%%%%%%%%%%%%%%%%%%%%%%%%%%%%%%%%%%%%%%%%%%%%%%%%%%%%%%%%%%%%%%%%%%

\begin{problem}
The Mating Ritual \inbook{of Section~\bref{mating_ritual_sec}} for
finding stable marriages works even when the numbers of men and women
are not equal.  As before, a set of (monogamous) marriages between men
and women is called stable when it has no ``rogue couples.''

\bparts

\ppart\label{genrogue} Extend the definition of \emph{rogue couple} so
it covers the case of unmarried men and women.  Verify that in a
stable set of marriages, either all the men are married or all the
women are married.

\begin{solution}
A rogue couple for a given a set of marriages is a woman, Alice, and a
man, Bob, such that
\begin{itemize}
\item Alice is unmarried or has a spouse she likes less than Bob, and also
\item Bob is unmarried or has a spouse he likes less than Alice.
\end{itemize}
By this definition, if both a man and a woman are unmarried, they are
a rogue couple, so in a stable set of marriages, either all the men
are married or all the women are married.
\end{solution}

\ppart Explain why even in the case of unequal numbers of men and
women, applying the Mating Ritual will yield a stable matching.

\begin{solution}
The preserved invariant\inhandout{\footnote{For every woman, $w$, and
    every man, $m$, if $w$ is crossed off $m$'s list, then $w$ has a
    suitor whom she prefers over~$m$.}}
\inbook{(Definition~\bref{def:P8})} and termination proof for the
Mating Ritual apply in this case without change, so we just need to
verify that there are no rogue couples of the more general kind in
part~\eqref{genrogue}.

As before, on the Wedding Day, no man will be in a rogue couple with a
woman \emph{not on} his final list because, by the preserved
invariant, any such woman has a spouse she prefers to him.  But no man
will be in a rogue couple with a woman still on his final list because
he must be married to that woman or to a woman higher on his list who,
by definition, he finds more preferable.
\end{solution}
\eparts
\end{problem}

\endinput

\inhandout{\instatements{
\newpage
\section*{Appendix: The Mating Ritual}

The \emph{Mating Ritual} takes place over several days.  The
following events happen each day:

{\bf Morning: } Each girl stands on her balcony.  Each boy stands under
the balcony of his favorite among the girls on his list, and he serenades
her.  If a boy has no girls left on his list, he stays home and does his
6.042 homework.

{\bf Afternoon: } Each girl who has one or more suitors serenading her,
says to her favorite suitor, ``We might get engaged.  Come back
tomorrow.''  To the others, she says, ``No.  I will never marry you!  Take
a hike!''

\textbf{Evening}: Any boy who is told by a girl to take a hike, crosses that
girl off his list.

\textbf{Termination condition}: When every girl has at most one suitor,
the ritual ends with each girl marrying her suitor, if she has one.
}}
