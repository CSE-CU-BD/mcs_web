\documentclass[problem]{mcs}

\begin{pcomments}
  \pcomment{CP_relation_images}
  \pcomment{from: S09.ps2}
\end{pcomments}

\pkeywords{
  relations
  set_theory
}

%%%%%%%%%%%%%%%%%%%%%%%%%%%%%%%%%%%%%%%%%%%%%%%%%%%%%%%%%%%%%%%%%%%%%
% Problem starts here
%%%%%%%%%%%%%%%%%%%%%%%%%%%%%%%%%%%%%%%%%%%%%%%%%%%%%%%%%%%%%%%%%%%%%

\begin{problem} 

\bparts

\ppart Give an example of a binary relation $R$ from $A$ to $B$ such
that $RB \neq A$.

\begin{solution}
Any relation that has an element in $A$ that does not relate
to any element in $B$, and is therefore not total, will work.

For example, consider the less-than relation $R$ from $A=\set{0,1}$ to 
$B=\set{0,1}$. $RB$ consists of the elements in $A$ that relate to some
element in $B$ and since $1 \in A$ is not less than either 0 or 1, 
$RB = \set{0} \neq A$.
\end{solution}

\ppart Prove that $R(AR) = RB$ for any binary relation $R$ from $A$ to 
$B$.

\begin{solution}
We proceed by showing $x \in R(AR) \iff x \in RB$ using a
chain of if and only if statements.

\begin{align*}
x \in R(AR) & \qiff \exists\, b \in AR \;.\; x\,R\,b\\ 
            & \qiff \exists\, b \in B, a \in A \;.\; a\,R\,b \ \QAND\  x\,R\,b\\
            & \qiff \exists\, b \in B \;.\; x\,R\,b\\
            & \qiff x \in RB
\end{align*}

The first and second iff's are simply applications of the definitions 
of the inverse image and image (respectively).  The third follows 
directly from the fact that $x \in A$: if $x\,R\,b$ then there must exist 
an $a \in A$, namely $a=x$, such that $a\,R\,b$.
\end{solution}

\eparts
\end{problem}

%%%%%%%%%%%%%%%%%%%%%%%%%%%%%%%%%%%%%%%%%%%%%%%%%%%%%%%%%%%%%%%%%%%%%
% Problem ends here
%%%%%%%%%%%%%%%%%%%%%%%%%%%%%%%%%%%%%%%%%%%%%%%%%%%%%%%%%%%%%%%%%%%%%
\endinput
