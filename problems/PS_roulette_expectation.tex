\documentclass[problem]{mcs}

\begin{pcomments}
\pcomment{PS_roulette_expectation}
\pcomment{F95.ps11}
\pcomment{edited by ARM 5/9/12}
\end{pcomments}

\pkeywords{
 probability
 expectation
 linearity
}


%%%%%%%%%%%%%%%%%%%%%%%%%%%%%%%%%%%%%%%%%%%%%%%%%%%%%%%%%%%%%%%%%%%%%
% Problem starts here
%%%%%%%%%%%%%%%%%%%%%%%%%%%%%%%%%%%%%%%%%%%%%%%%%%%%%%%%%%%%%%%%%%%%%

\begin{problem}
A wheel-of-fortune has the numbers from $1$ to $2n$ arranged in a
circle.  The wheel has a spinner, and a spin randomly determines the
two numbers at the opposite ends of the spinner.  How would you
arrange the numbers on the wheel to maximize the expected value of:

\bparts
\ppart the sum of the numbers chosen?  What is this maximum?

\begin{solution}
It makes no difference how you arrange the numbers; the expected value
is the same.  Why?  If we label the ends of the spinner we can define
$2$ random variables $X$ and $Y$: $X$ is the number pointed to by one
end of the spinner and $Y$ is the other number.  We wish to maximize
$\expect{X+Y}$.  But since $\expect{X+Y} = \expect{X} + \expect{Y}$
and $\expect{X} = \expect{Y} = (2n+1)/2$, all arangements give the
same expected value of $2n+1$.
\end{solution}

\ppart\label{ppart:product} the product of the numbers chosen?  What
is this maximum?

\hint For part \ref{ppart:product}, verify that the sum of the
products of numbers oppposite each other is maximized when
successive integers are on the opposite ends of the spinner, that is,
$1$ is opposite $2$, $3$ is opposite 4, 5 is opposite 6, \dots.

\begin{solution}
The expected product of opposite numbers is the average value of the
products of opposite numbers, and this expectation will be maximized
when the sum of the products of the opposite pairs of numbers is
maximized.

We claim that this sum of products of all the opposite numbers is
maximized exactly when each odd number $k$ is opposite $k+1$, that is
$1$ is opposite $2$, $3$ is opposite 4, \dots, $2n-1$ is opposite
$2n$.  Let's call this the \emph{successor arrangement}.  We'll show
that any arrangement that differs from the successor arrangement can
be altered to have a larger sum of products.  This implies that the
successor arrangement must be the unique arrangement with maximum sum
of products.

Given any arrangement, let's say a number is OK if it is opposite the
same number as in the successor arrangement.  So in any arrangement,
opposite numbers are both OK, or neither is OK.

Any given arrangement not equal to the successor arrangement must have
a smallest number, $k$, that is not-OK.  Since $k$ is the smallest
number that is not-OK, it must be opposite some larger not-OK number
$p$.  Also, $k$ must be odd, or else $k-1$ would be a smaller number
that is not-OK.  Since $k$ is odd, it is opposite $k+1$ in the
successor arrangement, and therefore $p > k+1$.  Moreover, $k+1$ must
be opposite some $q > k+1$.

Now, let's create a new arrangement by putting $k$ opposite $k+1$ and
$p$ opposite $q$, leaving all the other opposites unchanged.  We claim
the new arrangement will have a larger sum of products of opposites
than the given one.

Since nothing but the opposite pairings of $k, p, k+1, q$ changes, we
can just show that
\[
kp + (k+1)q < k(k+1)+pq.
\]
Letting $p = k+a$ and $q=k+b$, the above inequality simplifies to
\[
b < ab
\]
which follows immediately since $a,b \geq 2$.

Now the expected value of the product of a random pair of opposites in
the successor arrangement is 1/$n$th of
\[
\sum_{i=1}^n (2i-1)2i = 4 \sum_{i=1}^n i^2 - 2 \sum_{i=1}^n i = 4 \frac{n(n+1)(2n+1)}{6} - 2 \frac{n(n+1)}{2},
\]
namely\footnote{See Problem~\bref{sum-of-sq} for the sum-of-squares formula.},
\[
\frac{(n+1)(4n-1)}{3} = \frac{4}{3} n^2 + n - \frac{1}{3}.
\]

\end{solution}

\eparts
\end{problem}


%%%%%%%%%%%%%%%%%%%%%%%%%%%%%%%%%%%%%%%%%%%%%%%%%%%%%%%%%%%%%%%%%%%%%
% Problem ends here
%%%%%%%%%%%%%%%%%%%%%%%%%%%%%%%%%%%%%%%%%%%%%%%%%%%%%%%%%%%%%%%%%%%%%

\endinput
