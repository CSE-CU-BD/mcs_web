\documentclass[problem]{mcs}

\begin{pcomments}
  \pcomment{PS_generalized_stamps}
  \pcomment{from: F07 practice final, S06 ps4; q1 S04, S04 cp5w}
\end{pcomments}

\pkeywords{
  postage
  stamps
  congruence
  permute
}

%%%%%%%%%%%%%%%%%%%%%%%%%%%%%%%%%%%%%%%%%%%%%%%%%%%%%%%%%%%%%%%%%%%%%
% Problem starts here
%%%%%%%%%%%%%%%%%%%%%%%%%%%%%%%%%%%%%%%%%%%%%%%%%%%%%%%%%%%%%%%%%%%%%


\begin{problem} \textbf{The Generalized Postage Problem}

  Several other problems (\bref{PS_3_and_5_stamps_by_WOP},
  \bref{TP_10_and_15_cent_stamps_by_WOP}, %\bref{TP_4-5-stamps_by_induction},
  \bref{FP_4_and_7_cent_stamps_by_induction}) work out which amounts
  of postage can be formed using two stamps of given denominations.
  In this problem, we generalize this to two stamps with arbitrary
  positive integer denominations $a$ and $b$ cents.  Let's call an
  amount of postage that can be made from $a$ and $b$ cent stamps a
  \emph{makeable} amount.

  \begin{lemma*} (Generalized Postage) If $a$ and $b$ are relatively prime
    positive integers, then any integer greater than $ab-a-b$ is
    makeable.
\end{lemma*}

To prove the Lemma, consider the following array with $a$ infinite rows:
\[\begin{array}{rrrrr}
 0     &     a  &    2a  &    3a  & \dots\\
 b     &   b+a  &   b+2a &  b+3a  & \dots\\
2b     &  2b+a  &  2b+2a & 2b+3a  & \dots\\
3b     &  3b+a  &  3b+2a & 3b+3a  & \dots\\
\vdots & \vdots & \vdots & \vdots & \dots\\
(a-1)b & (a-1)b + a & (a-1)b + 2a & (a-1)b + 3a & \dots
\end{array}\]
Note that every element in this array is clearly makeable.

\bparts

\ppart \label{grid} Suppose that $n$ is at least as large as, and also
congruent mod~$a$ to, the first element in some row of this array.
Explain why $n$ must appear in the array.

\begin{solution}
Each row of the array contains all the integers that are least as
large as, and are congruent to modulo~$a$ to, the first element in that
row.  So $n$ appears in the row whose first element is $\leq n$ and is
$\equiv n \pmod a$.
\end{solution}

\ppart\label{0a-1} Prove that every integer from 0 to $a-1$ is congruent
modulo $a$ to one of the integers in the first column of this array.

\begin{solution}
Since $b$ has an inverse modulo $a$, we know that the first column of
integers $0,b,2b, (a-1)b$ includes numbers congruent to the each of
the integers $0,1,\dots a-1$ (Lemma~\bref{lem:cardks}).
\end{solution}

\ppart Complete the proof of the Generalized Postage Lemma by using
parts~\eqref{grid} and~\eqref{0a-1} to conclude that every integer $n
>ab-a-b$ appears in the array, and hence is makeable.

\hint Suppose $n$ is congruent mod $a$ to the first element in some row.
Assume $n$ is less than that element, and then show that $n \leq ab-a-b$.

\begin{solution}
By~\eqref{0a-1}, any integer $n$ is congruent mod $a$ to the first
element of some row.  If $n \geq$ that first element, then by
part~\eqref{grid}, $n$ appears in the array.  So we need only consider
$n$ that are congruent mod $a$ to the first element of some row and
are less than that first element.  But in that case, $n$ must be at
least $a$ less than this first element.  Since the largest first
element is $(a-1)b$, it follows that $n \leq (a-1)b -a = ab-a-b$.  So
every $n >ab-a-b$ must be in the array.
\end{solution}

\ppart\label{opt} (Optional) What's more, $ab-a-b$ is not makeable.
Prove it.

\begin{solution}
The array clearly contains all the postage amounts makeable using at
most $a-1$ of the $b$-stamps.  But any amount of postage makeable
using $a$ or more $b$-stamps can also be made with at most $a-1$ of
the $b$-stamps---just keep replacing $a$ of the $b$-stamps by $b$ of
the $a$-stamps until there are at most $a-1$ of the $b$-stamps.  In
short, the array above actually contains \emph{all} the makeable
integers.

So all we need to show is that $ab-a-b$ does not appear in the array.
But the row an integer appears on is uniquely determined: it must
appear in the row whose first element is congruent to it mod $a$.  Now
$ab-a-b = (a-1)b - a \equiv (a-1)b \pmod a$, which means that $ab-a-b$
would have to appear in the last row.  But since it is smaller than
the first element in the last row, it follows that it doesn't appear
at all.
\end{solution}

\ppart  Explain why the following even more general lemma follows directly
from the Generalized Lemma and part~\eqref{opt}.

\begin{lemma*} (Generalized$^2$ Postage) If $m$ and $n$ are positive
  integers and $g \eqdef \gcd(m,n)>1$, then with $m$ and $n$ cent stamps,
  you can only make amounts of postage that are multiples of $g$.  You can
  actually make any amount of postage greater than $(mn/g)-m-n$ that is a
  multiple of $g$, but you cannot make $(mn/g)-m-n$ cents postage.
\end{lemma*}

\begin{solution}
We know that any integer linear combination of $m$ and $n$ will be a
multiple of $g$.  But any amount of postage that can be made from $m$
and $n$ cent stamps is an integer linear combination of $m$ and $n$
with nonnegative coefficients.  So any such amount of postage must be
a multiple of $g$.

  Next, let $a \eqdef m/g$ and $b \eqdef n/g$.  Now $k$ cents postage
  can be made of $a$ and $b$ cent stamps iff $kg$ cents postage can be
  made of corresponding numbers of $ag=m$ and $bg=n$ cent stamps.  But
  $a$ and $b$ are relatively prime by the definition of gcd, so if an
  amount of postage is makeable from $a$ and $b$ cent stamps according
  to the Generalized Lemma then that amount times $g$ can be made from
  $m$ and $n$ cent stamps.  This proves the Generalized$^2$ Lemma for
  $m$ and $n$.
\end{solution}

  \ppart \textbf{Optional and possibly unknown.}  Suppose you have three
  denominations of stamps, $a,b,c$ and $\gcd(a,b,c) =1$.  Give a formula for
  the smallest number $n_{abc}$ such that you can make every amount of
  postage $\geq n_{abc}$.

\begin{solution}
   The general stamp problem of finding the smallest number,
     $n_{a_1a_2\dots a_k}$ that is not makeable using $k$ stamps of
   denominations $a_1,a_2,\dots,a_k$ is known to be NP-complete (see
   Section~\bref{SAT_sec}).  Good luck with the $k=3$ case---and
   notify the authors if get somewhere interesting with it.
\end{solution}

\eparts

\end{problem}


%%%%%%%%%%%%%%%%%%%%%%%%%%%%%%%%%%%%%%%%%%%%%%%%%%%%%%%%%%%%%%%%%%%%%
% Problem ends here
%%%%%%%%%%%%%%%%%%%%%%%%%%%%%%%%%%%%%%%%%%%%%%%%%%%%%%%%%%%%%%%%%%%%%

\endinput

