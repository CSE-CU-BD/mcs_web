\documentclass[problem]{mcs}

\begin{pcomments}
  \pcomment{FP_simple_graphs_asymptotics}
  \pcomment{adapted into a proof-based question from FP_simple_graphs_TF} 
  \pcomment{CH, Spring 14}
\end{pcomments}

\pkeywords{
  vertices
  edge
  coloring
  chromatic_number
  big_Oh
}

%%%%%%%%%%%%%%%%%%%%%%%%%%%%%%%%%%%%%%%%%%%%%%%%%%%%%%%%%%%%%%%%%%%%%
% Problem starts here
%%%%%%%%%%%%%%%%%%%%%%%%%%%%%%%%%%%%%%%%%%%%%%%%%%%%%%%%%%%%%%%%%%%%%

\begin{problem} 

In class, we have only used asymptotic notation for functions defined
over the real numbers. But there is a straightforward way to extend
this notion for functions defined over \emph{graphs}.  
Let $\mathbb{G}$ be the set of all simple connected graphs, and let $f,
g: \mathbb{G} \to \mathbb{R}^+$. Formally, we say that $f = O(g)$ iff
there exists a constant $c \geq 0$ and a positive integer $n_0$ such
that for any $n > n_0$, $f(G) \leq c g(G)$ for all $n$-vertex graphs $G$.

For the following questions, full formal proofs are not necessary and only a brief argument will suffice.

\bparts

\ppart Argue that $\card{\vertices{G}} = O(\card{\edges{G}})$.

\begin{solution}
Let $f(G) = \card{\vertices{G}} = n$.  The number of edges,
$\card{\edges{G}}$, is at least $n-1$ (since $G$ is
connected). Therefore,  $f(G) = n = O(n-1)$ and the result follows.
\end{solution}

\examspace[1in]

\ppart Argue that the reverse assertion: 
\[
\card{\edges{G}} = O(\card{\vertices{G}}) 
\]
is false. 
\begin{solution}
Consider the complete graph on $n$ vertices $K_n$. Then, $\card{\vertices{K_n}} = n = o(n^2)$, but
$\card{\edges{K_n}} = \Theta(n^2)$. Therefore, the assertion is false.
\end{solution}


\examspace[1in]

\ppart Let $\chi(G)$ be the chromatic number of $G$, where $G$ is a
{tree} with $n$ vertices and $n$ is very large. What is the chromatic
number of $G$?
\begin{solution}
Since $G$ is a tree, it has no cycles. Therefore, it is bipartite
(since it has no cycles of odd length).  This means 2 colors are \emph{sufficient}. Moreover, if the tree
has at least one edge, 2 colors are also \emph{necessary}.  So the
chromatic number $\chi(G)$ is~2.
\end{solution}

\examspace[1in]

\ppart Argue that the assertion 
\[
\card{\vertices{G}} = O(\chi(G)) .
\]
is false. 
\begin{solution}
There are examples of graphs with an arbitrary large number of
vertices $\chi = 2$. Every tree (and more generally, every bipartite
graph with at least 1 edge) has a chromatic number $\chi(G) = 2$. 
\end{solution}

\eparts

\end{problem}

\endinput
