\documentclass[problem]{mcs}

\begin{pcomments}
  \pcomment{FP_simple_graphs_asymptotics}
  \pcomment{adapted into an explanation question from FP_simple_graphs_TF} 
  \pcomment{ARM/CH, 5/16/14, def if O() revised ARM 11/1/15}
  \pcomment{f15.midterm3}
\end{pcomments}

\pkeywords{
  vertices
  edge
  coloring
  chromatic_number
  big_Oh
}

%%%%%%%%%%%%%%%%%%%%%%%%%%%%%%%%%%%%%%%%%%%%%%%%%%%%%%%%%%%%%%%%%%%%%
% Problem starts here
%%%%%%%%%%%%%%%%%%%%%%%%%%%%%%%%%%%%%%%%%%%%%%%%%%%%%%%%%%%%%%%%%%%%%

\begin{problem} 
Let $f,g$ be positive real-valued functions on finite,
\emph{connected}, simple graphs.  We will extend the $O()$ notation to
such graph functions as follows: $f = O(g)$ iff
there is a constant $c>0$ such that
\[
f(G) \leq c \cdot g(G) \text{ for all connected simple graphs } G \text{ with more than one vertex}.
\]
For each of the following assertions, state whether it is \True\ or
\False\, and briefly explain your answer.  You are \textbf{not}
expected to offer a careful proof or detailed counterexample.

\emph{Reminder}: $\vertices{G}$ is the set of vertices and $\edges{G}$
is the set of edges of $G$, and $G$ is connected.

\bparts

\ppart $\card{\vertices{G}} = O(\card{\edges{G}})$.

\begin{solution}
\True.

Since $G$ is connected, the number of edges $\card{\edges{G}}$, is at
least one less than the number of vertices.  That is,
\[
\card{\vertices{G}} \leq \card{\edges{G}} + 1 = O(\card{\edges{G}}).
\]

\end{solution}

\examspace[1in]

\ppart $\card{\edges{G}} = O(\card{\vertices{G}})$.

\begin{solution}
\False.

The complete graph $K_n$ has $n$ vertices has $n(n-1)/2 \neq O(n)$ edges.
\end{solution}

\examspace[1in]

\ppart $\card{\vertices{G}} = O(\chi(G))$, where $\chi(G)$ is the
chromatic number of $G$.

\begin{solution}
\False.

For example, every tree with more than one vertex has a chromatic
number 2.
\end{solution}

\examspace[1in]

\ppart $\chi(G) = O(\card{\vertices{G}})$.

\begin{solution}
\True.

Assigning a different color to each vertex gives a valid coloring, so
$\chi(G) \leq \card{\vertices{G}}$.
\end{solution}

\eparts

\end{problem}

\endinput
