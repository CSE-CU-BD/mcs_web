\documentclass[problem]{mcs}

\begin{pcomments}
  \pcomment{FP_simple_graphs_asymptotics}
  \pcomment{adapted into a proof-based question from FP_simple_graphs_TF} 
  \pcomment{CH, Spring 14}
\end{pcomments}

\pkeywords{
  vertices
  edge
  coloring
  chromatic_number
  big_Oh
}

%%%%%%%%%%%%%%%%%%%%%%%%%%%%%%%%%%%%%%%%%%%%%%%%%%%%%%%%%%%%%%%%%%%%%
% Problem starts here
%%%%%%%%%%%%%%%%%%%%%%%%%%%%%%%%%%%%%%%%%%%%%%%%%%%%%%%%%%%%%%%%%%%%%

\begin{problem} 
There is a straightforward way to extend $O()$ notation to functions
on \emph{graphs}.  Let $\mathbb{G}$ be the set of all
\emph{connected}, simple graphs, and let $f, g: \mathbb{G} \to
\reals^+$.  Then $f = O(g)$ iff there exists a constant $c > 0$ and an
integer $n_0 > 0$ such that for any $n > n_0$, $f(G) \leq c g(G)$ for
all $n$-vertex graphs $G$.

For each of the following assertions, state whether it is true or
false, and \emph{briefly} explain your answer.  You are not expected
to offer a careful proof or detailed counterexample.

\bparts

\ppart $\card{\vertices{G}} = O(\card{\edges{G}})$.

\begin{solution}
\textbf{TRUE}.

Since $G$ is connected, the number of edges, $\card{\edges{G}}$, is at
most one less than the number of vertices.  That is,
\[
\card{\vertices{G}} \leq \card{\edges{G}} + 1 = O(\card{\edges{G}}).
\]

\end{solution}

\examspace[1in]

\ppart $\card{\edges{G}} = O(\card{\vertices{G}})$.

\begin{solution}
\textbf{FALSE}.

The complete graph $K_n$ with $n$ vertices has $n(n-1)/2$ edges.
\end{solution}

\examspace[1in]

\ppart $\card{\vertices{G}} = O(\chi(G))$, where $\chi(G)$ is the
chromatic number of $G$.

\begin{solution}
\textbf{FALSE}.  For example, every tree with more than one vertex has
a chromatic number 2.
\end{solution}

\examspace[1in]

\ppart $\chi(G) = O(\card{\vertices{G}})$.

\begin{solution}
\textbf{TRUE}.

Assigning a different color to each vertex gives a valid coloring, so
$\chi(G) \leq \card{\vertices{G}}$.
\end{solution}

\eparts

\end{problem}

\endinput
