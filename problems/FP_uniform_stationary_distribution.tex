\documentclass[problem]{mcs}

\begin{pcomments}
\pcomment{FP_uniform_stationary_distribution}
\pcomment{from: S08.final}
\end{pcomments}

\pkeywords{
  random_walk
  uniform
  stationary_distribution
}

%%%%%%%%%%%%%%%%%%%%%%%%%%%%%%%%%%%%%%%%%%%%%%%%%%%%%%%%%%%%%%%%%%%%%
% Problem starts here
%%%%%%%%%%%%%%%%%%%%%%%%%%%%%%%%%%%%%%%%%%%%%%%%%%%%%%%%%%%%%%%%%%%%%

\begin{problem}
  For which of the graphs in Figure~\ref{4statgraphs} is the uniform
  distribution over nodes a stationary distribution?  The edges are
  labeled with transition probabilities.  Explain your reasoning.

\begin{figure}
\subfloat{\graphic[height=1in]{sttnry0}}
\qquad
\subfloat{\graphic[height=1in]{sttnry1}}

\subfloat{\graphic[height=1in]{sttnry2}}
\qquad
\subfloat{\graphic[height=1in]{sttnry3}}
\caption{Which ones have uniform stationary distribution?}
\label{4statgraphs}
\end{figure}

\begin{solution}
All except the last one (bottom right).

One way of approaching this problem is by performing a single update
step according to the rule
\[ \widehat{d}(v) = \sum_{u\;\text{s.t.}\;\diredge{u}{v}} d(u)
p(u,v),
\]
where $d$ is the stationary distribution ($1/2$ for all vertices on
the left graphs, $1/3$ for all vertices on the right), $\widehat{d}$
is the distribution after one step, and $p(u,v)$ is the edge
probability.  If $\widehat{d} = d$, then by definition, the uniform
distribution is stationary.

Alternatively, you could observe that the uniform distribution is
stationary if and only if the incoming-edge probabilities sum to 1
(see Problem~\bref{CP_uniform_stationary}, covered in class).
\end{solution}

\end{problem}

\endinput
