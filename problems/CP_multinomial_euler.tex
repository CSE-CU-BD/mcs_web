\documentclass[problem]{mcs}

\begin{pcomments}

  \pcomment{old name: CP_multinomial_theorem_2}
  \pcomment{from: S09.cp12m}
\end{pcomments}

\pkeywords{
  multinomial
  generating functions
  counting
}

%%%%%%%%%%%%%%%%%%%%%%%%%%%%%%%%%%%%%%%%%%%%%%%%%%%%%%%%%%%%%%%%%%%%%
% Problem starts here
%%%%%%%%%%%%%%%%%%%%%%%%%%%%%%%%%%%%%%%%%%%%%%%%%%%%%%%%%%%%%%%%%%%%%

\begin{problem}
\bparts 
\ppart Prove that
\begin{equation}\label{xp}
(x_1 + x_2 + \cdots +x_n)^p \equiv x_1^p + x_2^p + \cdots +x_n^p \pmod p
\end{equation}
for all primes $p$.

\hint Explain why $\binom{p}{k_1, k_2, \dots, k_m}$ is divisible by $p$
if all the $k_i$'s are less than $p$.


\begin{solution}
By the Multinomial Theorem~\ref{ml}, $(x_1 + x_2 + \cdots +x_n)^p$ is a
sum of monomials in $x_1,\dots,x_m$ whose coefficients are
\[
\binom{p}{k_1, k_2, \dots, k_m}
\]
where the sum of the $k_i$'s is $p$.  But if all the $k_i$'s are less than
$p$, then none of the denominator terms $k_1!$ divides the numerator, $p$, and
so the multinomial coefficient is divisible by $p$.  So the only
coefficients not divisible by $p$ are the coefficients of the terms
$x_i^p$, and all the other terms are $\equiv 0 \pmod p$.
\end{solution}

\ppart Explain how~\eqref{xp} immediately proves Fermat's Little Theorem:
$n^{p-1} \equiv 1 \pmod p$ for all $n$ relatively prime to $p$.

\begin{solution}
Let $x_1=x_2=\cdots x_m = 1$.  Then~\eqref{xp} implies $n^p \equiv n\cdot
1^p =n \pmod p$.  If $n$ is relatively prime to $p$, we can then cancel
$n$ to get $n^{p-1} \equiv 1 \pmod p$.
\end{solution}

\eparts
\end{problem}


%%%%%%%%%%%%%%%%%%%%%%%%%%%%%%%%%%%%%%%%%%%%%%%%%%%%%%%%%%%%%%%%%%%%%
% Problem ends here
%%%%%%%%%%%%%%%%%%%%%%%%%%%%%%%%%%%%%%%%%%%%%%%%%%%%%%%%%%%%%%%%%%%%%
\endinput
