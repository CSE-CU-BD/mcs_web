\documentclass[problem]{mcs}

\begin{pcomments}
  \pcomment{PS_recursive_set_data_type}
  \pcomment{ARM 3/8/16, revised 3/9/17}
  \pcomment{basically a rephrasing of Fundamental Thm for win-lose
    games in the text}
  
\end{pcomments}

\pkeywords{
  set_theory
  recursive_data
  structural_induction
  game
  strategy
}

\newcommand{\recset}{\text{Recs}}

%%%%%%%%%%%%%%%%%%%%%%%%%%%%%%%%%%%%%%%%%%%%%%%%%%%%%%%%%%%%%%%%%%%%%
% Problem starts here
%%%%%%%%%%%%%%%%%%%%%%%%%%%%%%%%%%%%%%%%%%%%%%%%%%%%%%%%%%%%%%%%%%%%%

\begin{problem}
In this problem, structural induction and the Foundation Axiom of set theory provide simple
proofs about some utterly infinite objects.

\begin{definition*}
The class of `recursive set-like'' objects, \recset, is defined recursively as follows:

\inductioncase{Base case}: The empty set $\emptyset$ is a \recset.

\inductioncase{Constructor step}: If $S$ is a nonempty set of
\recset's, then $S$ is a \recset.
\end{definition*}

\bparts

\ppart Prove that \recset\ satisfies the Foundation Axiom: there is no
infinite sequence of \recset, $r_o,r_1,\dots,r_{n-1}, r_n,\dots$ such
that
\begin{equation}\label{indecr}
\dots r_n \in r_{n-1} \in \dots r_1 \in r_0.
\end{equation}

\hint Structural induction.

\begin{solution}
We prove by structural induction on the definition of \recset, that no
\recset $R$ can be the start $r_0$ of an infinite decreasing
sequence of members shown in~\eqref{indecr}.

\inductioncase{Base case}: ($R = \emptyset$).  $R$ cannot be $r_0$
because there is no $r_1$.

\inductioncase{Constructor step}: ($R$ is a nonempty set of \recset's).

If $R=r_0$ then $r_1 \in R$, and by structural induction hypothesis,
$r_1$ cannot be the start of an infinite decreasing sequence of
members, so neither can $R$.
\end{solution}

\ppart In set theory, a set $S$ is said to be a \emph{pure set} if all elements of $S$ are sets, all elements of elements of $S$ are sets, all elements of elements of elements of $S$ are sets, and so on. In detail, a set $S$ is a pure set if for every integer $n\ge 1$ and every sequence $S \ni a_1 \ni a_2 \ni \cdots \ni a_n$, all of $S,a_1,a_2,\ldots,a_n$ are sets.

Note that $\emptyset$ is vacuously a pure set, since there are no sequences to check. Note also that all elements of pure sets must also be pure sets.

Prove that every pure set is a \recset.

% of $S$'s elements are sets, all elements of $S$'s elements are sets, 


% Prove that \emph{every} \emph{pure} set is a \recset.\footnote{A
%   ``pure'' set is empty or is a set whose elements are all pure sets.}

\hint Use the Foundation axiom.

\begin{solution}
  If there were a pure set $S_1$ that was not a \recset, then $S_1$ must be nonempty (since $\emptyset\in\recset$), and $S_1$ must have some element $S_2\in S_1$ that is not a \recset\ (otherwise the \recset\ constructor case would apply to $S_1$). So $S_2$ is a pure set and not a \recset, meaning the same argument may be applied: there exists an element $S_3\in S_2$ that is pure but not a \recset, an element $S_4\in S_3$ with the same properties, and so on for $S_5\ni S_6\ni S_7 \ni \cdots$. This violates the Foundation axiom. 
%
% If there was a pure set that was not a \recset, then by Foundation
% there would be a member-minimal set $S \notin \recset$.  By
% minimality, every member of $S$ is a member of \recset, and therefore
% by the constructor step, the set $S$ itself is a \recset, a
% contradiction.
\end{solution}

\ppart Every \recset\ $R$ defines a special kind of two-person game of
perfect information called a \emph{uniform} game.  The initial ``board
position'' of the game is $R$ itself.  A player's move consists of
choosing any member $R$.  The two players alternate moves, with the
player whose turn it is to move called the \emph{Next} player.  The
Next player's move determines a game in which the other player, called
the \emph{Previous} player, moves first.

The game is called ``uniform'' because the two players have the
\emph{same} objective: to leave the other player stuck with no move to
make.  That is, whoever moves to the empty set is a winner, because
then the next player has no move.

Prove that in every uniform game, either the Previous player or the
Next player has a winning strategy.

\begin{solution}
Call a \recset\ a \emph{P-game} if the Previous player has a winning
strategy and an \emph{N-game} if the Next player has a winning
strategy.  We prove that every \recset $R$ is a P-game or and
N-game, by structural induction on the definition of \recset.

\inductioncase{Base case} ($R = \emptyset$): $R$ is an P-game since
the Next player cannot move.

\inductioncase{Constructor step}: ($R$ is a nonempty set of
\recset's): By structural induction, we may assume that every $s \in
R$ is a P-Game or an N-game.  If there is an $s \in R$ that is a
P-Game, then the Next player wins by picking $s$ (since he will be the
Previous player in $s$), and following the P-winning strategy in $s$.
This makes $R$ an N-game.

Otherwise, every $s \in R$ is an N-game, and the Previous player will
have a winning strategy to follow regardless of which game the Next
player selects.  This makes $R$ a P-game.
\end{solution}

\eparts

\end{problem}

%%%%%%%%%%%%%%%%%%%%%%%%%%%%%%%%%%%%%%%%%%%%%%%%%%%%%%%%%%%%%%%%%%%%%
% Problem ends here
%%%%%%%%%%%%%%%%%%%%%%%%%%%%%%%%%%%%%%%%%%%%%%%%%%%%%%%%%%%%%%%%%%%%%

\endinput
