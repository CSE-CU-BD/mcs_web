\documentclass[problem]{mcs}

\begin{pcomments}
  \pcomment{PS_recursive_set_data_type}
  \pcomment{ARM 3/8/16}
\end{pcomments}

\pkeywords{
  set_theory
  recursive_data
  structural_induction
}

\newcommand{\recset}{\text{Recs}}

%%%%%%%%%%%%%%%%%%%%%%%%%%%%%%%%%%%%%%%%%%%%%%%%%%%%%%%%%%%%%%%%%%%%%
% Problem starts here
%%%%%%%%%%%%%%%%%%%%%%%%%%%%%%%%%%%%%%%%%%%%%%%%%%%%%%%%%%%%%%%%%%%%%

\begin{problem}
In this problem, structural induction and the Foundation Axiom of set theory provide simple
proofs about some utterly infinite objects.

\begin{definition*}
The class of `recursive-set-like'' objects, \recset, is defined recursively as follows

\inductioncase{Base case}: The empty set $\emptyset$ is a \recset.

\inductioncase{Constructor step}: If $P$ is a property of \recset's that is not identically
false, then
\[
\set{s \in \recset \suchthat P(s)}
\]
is a \recset.
\end{definition*}

\bparts

\ppart Prove that \recset\ satisfies the Foundation Axiom: there is no infinite sequence of
\recset, $r_o,r_1,\dots,r_{n-1}, r_n,\dots$ such that
\begin{equation}\label{indecr}
\dots r_n \in r_{n-1} \in \dots r_1 \in r_0.
\end{equation}

\hint Structural induction.

\begin{solution}
We prove by structural induction on the definition of \recset, that no
\recset, $R$, can be the start, $r_0$, of an infinite decreasing
sequence of members shown in~\eqref{indecr}.

\inductioncase{Base case} ($R = \emptyset$):  $R$ cannot be $r_0$
because there is no $r_1$.

\inductioncase{Constructor step} ($R = \set{s \in \recset \suchthat P(s)}$): If $R=r_0$ then
$r_1 = s \in recset$, and by structural induction hypothesis $r_1$ cannot be the start of an
infinite decreasing sequence of members, so neither can $R$.
\end{solution}

\ppart Prove that every set is a \recset.

\hint Use the Foundation axiom.

\begin{solution}
If there was a set that was not a \recset, then by Foundation there would be a member-minimal
set $S \notin \recset$.  By minimality, every member of $S$ is a member of \recset, and
therefore by the constructor step (with $P(s) \eqdef s \in S$), the set $S$ itself is a
\recset, a contradiction.
\end{solution}

\ppart Every \recset\ $R$ defines a special kind of two-person game of perfect information
called a \emph{uniform} game.  The initial ``board position'' of the game is $R$ itself.  A
player's move consists of choosing any member $R$.  The two players alternate moves, with the
player whose turn it is to move is called the \emph{Next} player.  The Next player's move
defines a game where the other player, called the \emph{Previous} player, moves first.

The game is called ``uniform'' because the two players have the \emph{same} objective: to leave
the other player stuck with no move to make.  That is, whoever moves to the empty set is a
winner, because then the next player has no move.

Prove that in every uniform game, either the previous player or the next player has a winning
strategy.

\begin{solution}
Call a \recset\ a \emph{P-game} if the Previous player has a winning strategy and an
\emph{N-game} if the Next player has a winning strategy.  We prove that every \recset, $R$, is
a P-game or and N-game, by structural induction on the definition of \recset.

\inductioncase{Base case} ($R = \emptyset$): $R$ is an P-game since the Next player cannot
move.

\inductioncase{Constructor step} ($R = \set{s \in \recset \suchthat P(s)}$).  By structural
induction, we may assume that every $s \in R$ is a P-Game or an N-game.  If there is an $s \in
R$ that is a P-Game, then the Next player wins by picking $s$ (since he will be the Previous
player in $s$), and following the P-winning strategy in $s$.  This makes $R$ an N-game.

Otherwise, every $s \in R$ is an N-game, and the Previous player will have a winning strategy
regardless of which game the Next player selects.  This makes $R$ a P-game.
\end{solution}

\eparts

\end{problem}

%%%%%%%%%%%%%%%%%%%%%%%%%%%%%%%%%%%%%%%%%%%%%%%%%%%%%%%%%%%%%%%%%%%%%
% Problem ends here
%%%%%%%%%%%%%%%%%%%%%%%%%%%%%%%%%%%%%%%%%%%%%%%%%%%%%%%%%%%%%%%%%%%%%

\endinput
