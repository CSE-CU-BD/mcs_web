\documentclass[problem]{mcs}

\begin{pcomments}
  \pcomment{PS_disjoint_cartesian_products}
%  \pcomment{TP_which_are_partial_orders}
  \pcomment{From: S00.ps3}
  \pcomment{Practice with elementary set concepts and proofs.
            Otherwise unmotivated and uninteresting}

\end{pcomments}

\pkeywords{
  cartesian_product
  disjoint
  sets
}

%%%%%%%%%%%%%%%%%%%%%%%%%%%%%%%%%%%%%%%%%%%%%%%%%%%%%%%%%%%%%%%%%%%%%
% Problem starts here
%%%%%%%%%%%%%%%%%%%%%%%%%%%%%%%%%%%%%%%%%%%%%%%%%%%%%%%%%%%%%%%%%%%%%

\begin{problem}
% topic: Cartesian products, disjoint, direct proof
% source: Velleman 4.1.8

Prove that for any sets $A$, $B$, $C$, and $D$, if the \idx{Cartesian product}s
$A \cross B$ and $C \cross D$ 
are disjoint, then either $A$ and $C$ are disjoint or $B$ and $D$ are disjoint.

\begin{solution}
\begin{proof}
We will prove the contrapositive.  In other words, we will
assume
\begin{equation}\label{neitherempty}
[(A \intersect C) \neq \emptyset \QAND (B \intersect D) \neq \emptyset]
\end{equation}
and prove that
\begin{equation}\label{prodnonempty}
(A \cross B) \intersect (C \cross D) \neq \emptyset.
\end{equation}

Now by~\eqref{neitherempty}, there must be an element $e$ in both $A$
and $C$, as well as an element $f$ in $B$ and $D$.  So, $(e, f) \in A
\cross B$ by definition of Cartesian product, and likewise $(e, f) \in
C \cross D$.  This means that
\[
(e,f) \in (A \cross B) \intersect (C \cross D),
\]
so 
$(A \cross B) \intersect (C \cross D) \neq \emptyset$

\end{proof}
\end{solution}
\end{problem}


%%%%%%%%%%%%%%%%%%%%%%%%%%%%%%%%%%%%%%%%%%%%%%%%%%%%%%%%%%%%%%%%%%%%%
% Problem ends here
%%%%%%%%%%%%%%%%%%%%%%%%%%%%%%%%%%%%%%%%%%%%%%%%%%%%%%%%%%%%%%%%%%%%%
