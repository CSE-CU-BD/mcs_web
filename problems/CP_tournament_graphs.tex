\documentclass[problem]{mcs}

\begin{pcomments}
  \pcomment{CP_tournament_graphs}
  \pcomment{revised by ARM 10/16/09, slightly again 3/8/11}
  \pcomment{from: S09.cp7r}
\end{pcomments}

\pkeywords{
  digraphs
  relations
  relational_properties
  partial_orders
  cycles
}

%%%%%%%%%%%%%%%%%%%%%%%%%%%%%%%%%%%%%%%%%%%%%%%%%%%%%%%%%%%%%%%%%%%%%
% Problem starts here
%%%%%%%%%%%%%%%%%%%%%%%%%%%%%%%%%%%%%%%%%%%%%%%%%%%%%%%%%%%%%%%%%%%%%

\begin{problem} %\textbf{Tournament Graphs}

%  \newcommand{\beats}{\!\rightarrow\!}

  In an $n$-player \term{round-robin tournament}, every pair of
  distinct players compete in a single game.  Assume that every game
  has a winner---there are no ties.  The results of such a tournament
  can then be represented with a \term{tournament digraph} where the
  vertices correspond to players and there is an edge $\diredge{x}{y}$
  iff $x$ beat $y$ in their game.

  \bparts

  \ppart Explain why a tournament digraph cannot have cycles of length
  1 or 2.

  \begin{solution}
    There are no self-loops in a tournament graph since no player plays
    himself, so no length 1 cycles.  Also, it cannot be that $x$ beats $y$
    and $y$ beats $x$ for $x \neq y$, since every pair competes exactly
    once and there are no ties.  This means there are no length 2 cycles.

\iffalse
    \footnote{Since there are no self-loops, any cycle of length 2 or 3
      must necessarily be simple.}
\fi

\end{solution}

  \ppart Is the ``beats'' relation for a tournament 
  graph always/sometimes/never:
  \begin{itemize}
%  \item symmetric,
  \item asymmetric?
  \item reflexive?
  \item irreflexive?
  \item transitive?
  \end{itemize}
Explain.

  \begin{solution}
  No self-loops implies the relation is irreflexive.  It 
  is also asymmetric since it is irreflexive and for every pair of 
  distinct players, exactly one game is played and results in a win
  for one of the players.  Some tournament graphs represent 
  transitive relations and others don't.
\end{solution}

\ppart Show that a tournament graph represents a linear order iff
there are no cycles of length 3.

  \begin{solution}
  As observed in the previous part, the ``beats'' relation whose graph
  is a tournament is asymmetric and irreflexive.  Since every pair of
  players is comparable, the relation is a linear order iff it is
  transitive.

  ``Beats'' is transitive iff for any players $x$, $y$ and $z$, $x$
  beats $y$ and $y$ beats $z$ implies that $x$ beats $z$, and
  consequently that there is no edge $\diredge{z}{x}$.  Therefore,
  ``beats'' is transitive iff there are no cycles of length 3.
\end{solution}

  \eparts
\end{problem}

%%%%%%%%%%%%%%%%%%%%%%%%%%%%%%%%%%%%%%%%%%%%%%%%%%%%%%%%%%%%%%%%%%%%%
% Problem ends here
%%%%%%%%%%%%%%%%%%%%%%%%%%%%%%%%%%%%%%%%%%%%%%%%%%%%%%%%%%%%%%%%%%%%%

\endinput
