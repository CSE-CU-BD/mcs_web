\documentclass[problem]{mcs}

\begin{pcomments}
  \pcomment{CP_tournament_graphs}
  \pcomment{revised by ARM 10/16/09, 3/8/11, 10/31/13, 4/23/17}
  \pcomment{S09.cp7r, S17.mid4}
\end{pcomments}

\pkeywords{
  digraph
  relation
  transitive
  path
  partial_order
  cycle
}

%%%%%%%%%%%%%%%%%%%%%%%%%%%%%%%%%%%%%%%%%%%%%%%%%%%%%%%%%%%%%%%%%%%%%
% Problem starts here
%%%%%%%%%%%%%%%%%%%%%%%%%%%%%%%%%%%%%%%%%%%%%%%%%%%%%%%%%%%%%%%%%%%%%

\begin{problem} %\textbf{Tournament Graphs}

%  \newcommand{\beats}{\!\rightarrow\!}

  In an $n$-player \emph{round-robin tournament}, every pair of
  distinct players compete in a single game.  Assume that every game
  has a winner---there are no ties.  The results of such a tournament
  can then be represented with a \term{tournament digraph} where the
  vertices correspond to players and there is an edge $\diredge{x}{y}$
  iff $x$ beat $y$ in their game.

  \bparts

  \ppart Briefly explain why a tournament digraph cannot have cycles
  of length one or two.

  \examspace[0.75in]

  \begin{solution}
    There are no self-loops in a tournament graph since no player plays
    himself, so no length one cycles.  Also, it cannot be that $x$ beats $y$
    and $y$ beats $x$ for $x \neq y$, since every pair competes exactly
    once and there are no ties.  This means there are no length two cycles.

\iffalse
    \footnote{Since there are no self-loops, any closed walk of length
      two or three must necessarily be a cycle.}
\fi

\end{solution}

  \ppart\label{beats-relation} Briefly explain whether the ``beats''
  relation of  a tournament graph
  \textbf{always}/\textbf{sometimes}/\textbf{never}\dots
  \begin{itemize}
%  \item symmetric,
  \item \dots is asymmetric.
    \examspace[0.5in]

    \begin{solution}
      Always asymmetric since every pair of distinct players play each
      other only once---so exactly one edge between distinct
      players---and no player plays themself---so no self-loops.
    \end{solution}

  \item \dots is reflexive.
    \begin{solution}
      Never reflexive since there are no self-loops.
    \end{solution}
    \examspace[0.5in]

  \item \dots is irreflexive.
    \begin{solution}
      Always irreflexive since there are no self-loops.
    \end{solution}
    \examspace[0.5in]

  \item \dots is transitive.
    \begin{solution}
      Sometimes transitive.  It follows from
      part~\eqref{linear_order_length3} that a tournament graph is
      transitive iff it is a DAG.
    \end{solution}
    \examspace[0.5in]

%   \item \dots has a unique directed path including all the vertices.

%     \begin{solution}
%       Sometimes.  It follows from part~\eqref{linear_order_length3}
%       that it will have a unique such a path iff it is a DAG.

%       Tournament graphs do turn out always to have \emph{at least} one
%       such path, see Problem~\bref{CP_tournament_chain}.
%     \end{solution}

%     \examspace[0.5in]

% \begin{staffnotes}
%   The following part should only be used if
%   Problem~\bref{CP_tournament_chain} has previously been
%   assigned.

%   \item \dots has a directed path including all the vertices.
%     \begin{solution}
%       Always.  See Problem~\bref{CP_tournament_chain}.
%     \end{solution}
%     \examspace[0.5in]
% \end{staffnotes}

  \end{itemize}

  \ppart\label{linear_order_length3} If a tournament graph has no cycles of length three, prove that it is a partial order.

  \begin{solution}
  As observed in part~\eqref{beats-relation}, the ``beats'' relation
  defined by a tournament graph is asymmetric.  So it will be a
  partial order iff it is transitive.

  For the ``beats'' to be transitive, if player $x$ beats $y$ and
  player $y$ beats $z$, then player $x$ must beat $z$.  But in a
  tournament, $x$ beats $z$ iff there is no edge $\diredge{z}{x}$.
  Therefore, ``beats'' is transitive iff there are no cycles of length
  three.

  % Finally, since there is an edge between every two players, if the
  % graph defines a partial order, it will be a linear partial order.


\end{solution}

  \eparts
\end{problem}

%%%%%%%%%%%%%%%%%%%%%%%%%%%%%%%%%%%%%%%%%%%%%%%%%%%%%%%%%%%%%%%%%%%%%
% Problem ends here
%%%%%%%%%%%%%%%%%%%%%%%%%%%%%%%%%%%%%%%%%%%%%%%%%%%%%%%%%%%%%%%%%%%%%

\endinput
