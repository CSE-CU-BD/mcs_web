\documentclass[problem]{mcs}

\begin{pcomments}
  \pcomment{CP_set_DeMorgan}
  \pcomment{similar to CP_proving_basic_set_identity}
  \pcomment{ARM 3/2/14}
\end{pcomments}

\pkeywords{
  logic
  set_theory
  identity
  propositional
  chain_of_iff
  difference
}

%%%%%%%%%%%%%%%%%%%%%%%%%%%%%%%%%%%%%%%%%%%%%%%%%%%%%%%%%%%%%%%%%%%%%
% Problem starts here
%%%%%%%%%%%%%%%%%%%%%%%%%%%%%%%%%%%%%%%%%%%%%%%%%%%%%%%%%%%%%%%%%%%%%

\begin{problem}
Prove De Morgan's Law for set equality
\begin{equation}\label{CP-set-demorgan}
\bar{A \intersect B} = \bar{A} \union \bar{B}.
\end{equation}
by showing with a chain of $\QIFF$'s that $x \in$ the left-hand side
of~\eqref{CP-set-demorgan} iff $x \in$ the right-hand side.  You may
assume the propositional version\inbook{~(\bref{DeMQAND})} of De Morgan's Law\inbook{.}\inhandout{:
\[
\QNOT(P \QAND Q)\ \text{is equivalent to}\ \bar{P} \QOR \bar{Q}.
\]
}

\begin{solution}
\begin{align*}
\lefteqn{x \in \bar{A \intersect B}}\\
 & \qiff \QNOT(x \in A \intersect B) & \text{def of set complement)}\\
 & \qiff \QNOT(x \in A \QAND x \in B) & \text{def of $\intersect$)}\\
 & \qiff \QNOT(P \QAND Q) & \text{(where $P \eqdef [x \in A]$ and $Q \eqdef [x \in B]$)}\\ 
 & \qiff \QNOT(P) \QOR \QNOT(Q) & \text{(De Morgan's Law for $\QAND$~(\bref{DeMQAND}))}\\
 & \qiff \QNOT(x \in A) \QOR \QNOT(x \in B) & \text{(def of $P,Q$)}\\
 & \qiff x \in \bar{A}  \QOR x \in \bar{B} & \text{def of set complement)}\\
 & \qiff x \in \bar{A} \union \bar{B}  & \text{def of $\union$)}\\
\end{align*}

\end{solution}

\begin{staffnotes}
Ask your students if they can now see how a computer could
automatically check such equalities between set formulas involving the
basic set operators like $\union, \intersect, -, \dots$?  The answer
is that proving such equalities reduces to verifying equivalence of
corresponding propositional formulas as above.
\end{staffnotes}

\end{problem}

%%%%%%%%%%%%%%%%%%%%%%%%%%%%%%%%%%%%%%%%%%%%%%%%%%%%%%%%%%%%%%%%%%%%%
% Problem ends here
%%%%%%%%%%%%%%%%%%%%%%%%%%%%%%%%%%%%%%%%%%%%%%%%%%%%%%%%%%%%%%%%%%%%%

\endinput
