\documentclass[problem]{mcs}

\begin{pcomments}
  \pcomment{from: S09.cp2m}
%  \pcomment{}
%  \pcomment{}
\end{pcomments}

\pkeywords{
  circuits
  logic
  boolean
  binary
}

%%%%%%%%%%%%%%%%%%%%%%%%%%%%%%%%%%%%%%%%%%%%%%%%%%%%%%%%%%%%%%%%%%%%%
% Problem starts here
%%%%%%%%%%%%%%%%%%%%%%%%%%%%%%%%%%%%%%%%%%%%%%%%%%%%%%%%%%%%%%%%%%%%%

\begin{problem}
  Propositional logic comes up in digital circuit design using the
  convention that \true\ corresponds to 1 and \false\ to 0.  A simple
  example is a 2-bit half-adder circuit.  This circuit has $3$ binary
  inputs, $a_1,a_0$ and $b$, and $3$ binary outputs, $c, o_1,o_0$.  The
  2-bit word $a_1a_0$ gives the binary representation of an integer, $s$
  between 0 and 3.  The 3-bit word $c\, o_1\, o_0$ gives the binary
  representation of $s+b$.
  The output $c$ is called the \emph{final carry bit}.

  So if $s$ and $b$ were both 1, then the value of $a_1a_0$ would be
  \texttt{01} and the value of the output $c\, o_1\, o_0$ would
  $\texttt{010}$, namely, the 3-bit binary representation of $1+1$.

  In fact, the final carry bit equals 1 only when all three binary inputs
  are 1, that is, when $s=3$ and $b=1$.  In that case, the value of $c\,
  o_1\, o_0$ is \texttt{100}, namely, the binary representation of $3+1$.

  This 2-bit half-adder could be described by the following formulas:
\begin{align*}
c_0 & = b \\
o_0 & = a_0\ \QXOR\ c_0\\
c_1 & = a_0\ \QAND\ c_0  & \text{the carry into  column 1}\\
o_1 & = a_1\ \QXOR\ c_1\\
c_2 & = a_1\ \QAND\ c_1 & \text{the carry into column 2}\\
c   & = c_2.
\end{align*}

\bparts

\ppart\label{CP_binary_adder_logic:anb} Generalize the above construction 
of a 2-bit half-adder to an $n+1$ bit half-adder with inputs $a_n,\dots, 
a_1, a_0$ and $b$ for arbitrary $n \geq 0$.  That is, give simple formulas 
for $o_{i}$ and $c_{i}$ for $0 \leq i \leq n+1$, where $c_i$ is the carry 
into column $i$ and $c=c_{n+1}$.

\solution{ The $n+1$-bit word $a_n \dots a_1 a_0$ will be the binary
  representation of an integer, $s$, between 0 and $2^{n+1}-1$.  The
  circuit will have $n+2$ outputs $c, o_n, \dots, o_1, o_0$ where the
  $n+2$-bit word $c o_n \dots o_1 o_0$ gives the binary representation of
  $s+b$.

Here are some simple formulas that define such a half-adder:
\begin{align*}
c_0 = b,\\
o_{i} & = a_{i}\ \QXOR\ c_{i}  & \text{for } 0 \leq i \leq n,\\
c_{i+1} &= a_i\ \QAND\ c_i & \text{for } 0 \leq i \leq n,\\
c & = c_{n+1}.
\end{align*}
}

\ppart\label{CP_binary_adder_logic:anbn} Write similar definitions for 
the digits and carries in the sum of two $n+1$-bit binary numbers 
$a_n\dots a_1a_0$ and $b_n\dots b_1b_0$.

\solution{Define
\begin{align*}
c_0 & = 0\\
o_{i} & = a_{i}\ \QXOR\ b_{i}\ \QXOR\ c_{i}
                & \text{for } 0 \leq i \leq n,\\
c_{i+1} & = (a_i\ \QAND\ b_i)\ \QOR\ (a_i\ \QAND\ c_i) \lor (b_i\ \QAND\ c_i)
                & \text{for } 0 \leq i \leq n,
c & = c_{n+1}.
\end{align*}
}

\eparts

Visualized as digital circuits, the above adders consist of a sequence of
single-digit half-adders or adders strung together in series.  These
circuits mimic ordinary pencil-and-paper addition, where a carry into a
column is calculated directly from the carry into the previous column, and
the carries have to ripple across all the columns before the carry into
the final column is determined.  Circuits with this design are called
``ripple-carry'' adders.  Ripple-carry adders are easy to understand and
remember and require a nearly minimal number of operations.  But the
higher-order output bits and the final carry take time proportional to $n$
to reach their final values.

\bparts 
\ppart How many of each of the propositional operations does your adder
from part~\eqref{CP_binary_adder_logic:anbn} use to calculate the sum?

\solution{The scheme given in the solution to 
part~\eqref{CP_binary_adder_logic:anbn} uses
  $3(n+1)$ AND's, $2(n+1)$ XOR's, and $2(n+1)$ OR's for a total of
  $7(n+1)$ operations.\footnote{Because $c_{0}$ is always 0, you could
    skip all the operations involving it.  Then the counts are $3n+1$
    AND's, $2n+1$ XOR's, and $2n$ OR's for a total of $7n+2$ operations.}
}

\eparts
\end{problem}

%%%%%%%%%%%%%%%%%%%%%%%%%%%%%%%%%%%%%%%%%%%%%%%%%%%%%%%%%%%%%%%%%%%%%
% Problem ends here
%%%%%%%%%%%%%%%%%%%%%%%%%%%%%%%%%%%%%%%%%%%%%%%%%%%%%%%%%%%%%%%%%%%%%

\endinput
