\documentclass[problem]{mcs}

\begin{pcomments}
  \pcomment{CP_binary_adder_logic}
  \pcomment{from S10.cp2m, S09.cp2m}
\end{pcomments}

\pkeywords{
  circuits
  logic
  boolean
  binary
}

%%%%%%%%%%%%%%%%%%%%%%%%%%%%%%%%%%%%%%%%%%%%%%%%%%%%%%%%%%%%%%%%%%%%%
% Problem starts here
%%%%%%%%%%%%%%%%%%%%%%%%%%%%%%%%%%%%%%%%%%%%%%%%%%%%%%%%%%%%%%%%%%%%%

\begin{problem}
  Propositional logic comes up in digital circuit design using the
  convention that \true\ corresponds to 1 and \false\ to 0.  A simple
  example is a 2-bit \term{half-adder} circuit.  This circuit has $3$
  binary inputs, $a_1,a_0$ and $b$, and $3$ binary outputs, $c,
  s_1,s_0$.  The 2-bit word $a_1a_0$ gives the binary representation
  of an integer, $k$, between 0 and 3.  The 3-bit word $c s_1 s_0$ gives
  the binary representation of $k+b$.  The third output bit, $c$, is
  called the final \emph{carry bit}.

  So if $k$ and $b$ were both 1, then the value of $a_1a_0$ would be
  \STR{01} and the value of the output $cs_1s_0$ would
  $\STR{010}$, namely, the 3-bit binary representation of $1+1$.

  In fact, the final carry bit equals 1 only when all three binary inputs
  are 1, that is, when $k=3$ and $b=1$.  In that case, the value of
  $cs_1s_0$ is \STR{100}, namely, the binary representation of $3+1$.

  This 2-bit half-adder could be described by the following formulas:
\begin{align*}
c_0 & = b \\
s_0 & = a_0\ \QXOR\ c_0\\
c_1 & = a_0\ \QAND\ c_0  & \text{the carry into  column 1}\\
s_1 & = a_1\ \QXOR\ c_1\\
c_2 & = a_1\ \QAND\ c_1 & \text{the carry into column 2}\\
c   & = c_2.
\end{align*}

\bparts

\ppart\label{CP_binary_adder_logic:anb} Generalize the above construction
of a 2-bit half-adder to an $n+1$ bit half-adder with inputs $a_n,\dots,
a_1, a_0$ and $b$ and outputs $c, s_n, \dots, s_1, s_0$.
\iffalse
for arbitrary $n \geq 0$
\fi
That is, give simple formulas for $s_{i}$ and $c_{i}$ for $0 \leq i
\leq n+1$, where $c_i$ is the carry into column $i+1$, and
$c=c_{n+1}$.

\begin{staffnotes}
If students are confused, have them add two binary numbers by hand, such as 10101 plus 10011.  Make sure students understand binary addition, or they will not be able to proceed with the problem.
\end{staffnotes}

\begin{solution}
 The $n+1$-bit word $a_n \dots a_1 a_0$ will be the binary
  representation of an integer, $s$, between 0 and $2^{n+1}-1$.  The
  circuit will have $n+2$ outputs $c, s_n, \dots, s_1, s_0$ where the
  $n+2$-bit word $c s_n \dots s_1 s_0$ gives the binary representation of
  $s+b$.  The variable $c_i$ will be the ``carry'' from the $i$th column.

Here are some simple formulas that define such a half-adder:
\begin{align*}
c_0 & = b,\\
s_{i} & = a_{i}\ \QXOR\ c_{i}  & \text{for } 0 \leq i \leq n,\\
c_{i+1} &= a_i\ \QAND\ c_i & \text{for } 0 \leq i \leq n,\\
c & = c_{n+1}.
\end{align*}

\end{solution}

\ppart\label{CP_binary_adder_logic:anbn} Write similar definitions for
the digits and carries in the sum of two $n+1$-bit binary numbers
$a_n\dots a_1a_0$ and $b_n\dots b_1b_0$.

\begin{solution}
Define
\begin{align*}
c_0     & =  0\\
s_{i}   & = a_{i}\ \QXOR\ b_{i}\ \QXOR\ c_{i}
                & \text{for } 0 \leq i \leq n,\\
c_{i+1} & = (a_i\ \QAND\ b_i)\ \QOR\\
       &\qquad  (a_i\ \QAND\ c_i) \QOR (b_i\ \QAND\ c_i)
                & \text{for } 0 \leq i \leq n,\\
c      & = c_{n+1}.
\end{align*}

\end{solution}

\eparts

Visualized as digital circuits, the above adders consist of a sequence
of single-digit half-adders or adders strung together in series.
These circuits mimic ordinary pencil-and-paper addition, where a carry
into a column is calculated directly from the carry into the previous
column, and the carries have to ripple across all the columns before
the carry into the final column is determined.  Circuits with this
design are called \term{ripple-carry} adders.  Ripple-carry adders are
easy to understand and remember and require a nearly minimal number of
operations.  But the higher-order output bits and the final carry take
time proportional to $n$ to reach their final values.

\bparts

\ppart How many of each of the propositional operations does your
adder from part~\eqref{CP_binary_adder_logic:anbn} use to calculate
the sum?

\begin{solution}
The scheme given in the solution to
part~\eqref{CP_binary_adder_logic:anbn} uses
  $3(n+1)$ AND's, $2(n+1)$ XOR's, and $2(n+1)$ OR's for a total of
  $7(n+1)$ operations.\footnote{Because $c_{0}$ is always 0, you could
    skip all the operations involving it.  Then the counts are $3n+1$
    AND's, $2n+1$ XOR's, and $2n$ OR's for a total of $7n+2$ operations.}

\end{solution}

\eparts
\end{problem}

%%%%%%%%%%%%%%%%%%%%%%%%%%%%%%%%%%%%%%%%%%%%%%%%%%%%%%%%%%%%%%%%%%%%%
% Problem ends here
%%%%%%%%%%%%%%%%%%%%%%%%%%%%%%%%%%%%%%%%%%%%%%%%%%%%%%%%%%%%%%%%%%%%%

\endinput
