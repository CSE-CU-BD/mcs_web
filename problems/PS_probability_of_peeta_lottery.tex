\documentclass[problem]{mcs}

\begin{pcomments}
  \pcomment{PS_probability_of_peeta_lottery}
  \pcomment{UNUSED}
  \pcomment{originate by dheins, Apr 2012}
  \pcomment{modified by djcshi}
\end{pcomments}

\pkeywords{
  probability
  expectation
}

%%%%%%%%%%%%%%%%%%%%%%%%%%%%%%%%%%%%%%%%%%%%%%%%%%%%%%%%%%%%%%%%%%%%%
% Problem starts here
%%%%%%%%%%%%%%%%%%%%%%%%%%%%%%%%%%%%%%%%%%%%%%%%%%%%%%%%%%%%%%%%%%%%%

\begin{problem}
Every year, Peeta is now forced to enter a lottery, where he picks a
number from 1 to $m$ randomly. If he picks a 1, he wins!

\begin{problemparts}

\ppart How many years should Peeta expect to play this lottery until he wins?
\begin{solution}
$m$ years.
\end{solution}

\ppart Now assume that even if Peeta wins (draws a 1), he keeps
playing.  If he plays $k$ years in a row, what is the chance that he
will win exactly $h$ times?  Assume $0 \le h \le k$.
\begin{solution}
\[
\binom{k}{h} \paren{\frac{1}{m}}^{h} \paren{\frac{m-1}{m}}^{k-h}
\]
\end{solution}

Let $P(m, k, h)$ be the probability that Peeta wins $h$ out of $k$ years
with the probability to win each year being $1/m$.

\ppart For $h=1$, what is the chance that Peeta wins $h$ out of $m$ years
with the probability to win each year being $1/m$ as $m$ approaches infinity?
\begin{solution}
\[
\lim_{m\to\infty}{P(m, m, 1)}
= \lim_{m\to\infty}{\left(\frac{m-1}{m}\right)^{m-1}}
= \left( \lim_{m\to\infty}{\frac{m}{m-1}} \right)
  \left( \lim_{m\to\infty}{\left(1 - \frac{1}{m}\right)^{m}} \right)
= 1\left(\frac{1}{e}\right) = \frac{1}{e}
\]
\end{solution}

\ppart How about for $h=0$ and $h=2$?
\begin{solution}
\[
\lim_{m\to\infty}{P(m, m, 0)}
= \lim_{m\to\infty}{\left(\frac{m-1}{m}\right)^{m}}
= \lim_{m\to\infty}{\left(1 - \frac{1}{m}\right)^{m}}
= \frac{1}{e}
\]

\[
\lim_{m\to\infty}{P(m, m, 2)}
= \lim_{m\to\infty}{\frac{m(m+1)}{2} \left(\frac{1}{m}\right)^2
  \left(\frac{m-1}{m}\right)^{m-2}}
= \left( \lim_{m\to\infty}{\frac{m(m+1)}{(m-1)^2}} \right)
  \left( \lim_{m\to\infty}{\left(1 - \frac{1}{m}\right)^{m}} \right)
= \frac{1}{2}\left(\frac{1}{e}\right) = \frac{1}{2e}
\]
\end{solution}

\ppart Conclude that $\sum_{i=0}^{\infty}{\frac{1}{i!} = e}$
\begin{solution}
Since
\[
\lim_{m\to\infty}{P(m, m, h)} = \frac{1}{h!e},
\]
and
\[
\sum_{h=0}^{m}{P(m, m, h)} = 1,
\]
we have 
\[
1 = \lim_{m\to\infty}{\sum_{h=0}^{m}{P(m, m, h)}}
= \sum_{h=0}^{\infty}{\lim_{m\to\infty}{P(m, m, h)}}
= \sum_{h=0}^{\infty}{\frac{1}{h!e}}.
\]
Therefore,
\[
\sum_{i=0}^{\infty}{\frac{1}{i!} = e}.
\]
\end{solution}

\end{problemparts}

\end{problem} 

%%%%%%%%%%%%%%%%%%%%%%%%%%%%%%%%%%%%%%%%%%%%%%%%%%%%%%%%%%%%%%%%%%%%% 
% Problem ends here 
%%%%%%%%%%%%%%%%%%%%%%%%%%%%%%%%%%%%%%%%%%%%%%%%%%%%%%%%%%%%%%%%%%%%%

\endinput
