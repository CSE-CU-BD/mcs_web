\documentclass[problem]{mcs}

\begin{pcomments}
  \pcomment{FP_bipartite_matching_sex}
  \pcomment{from: Meyer draft from F09}
  \pcomment{first part appears in CP_bipartite_sex}
\end{pcomments}

\pkeywords{
  graphs
  bipartite
  matching
  Halls_theorem
  sexual
}

%%%%%%%%%%%%%%%%%%%%%%%%%%%%%%%%%%%%%%%%%%%%%%%%%%%%%%%%%%%%%%%%%%%%%
% Problem starts here
%%%%%%%%%%%%%%%%%%%%%%%%%%%%%%%%%%%%%%%%%%%%%%%%%%%%%%%%%%%%%%%%%%%%%

\begin{problem}
  A researcher analyzing data on heterosexual sexual behavior in a group
  of $m$ males and $f$ females found that within the group, the male
  average number of female partners was 10\% larger that the female
  average number of male partners.

  \bparts

  \ppart[2]\label{partner-ratio} Circle all of the assertions below that are
  implied by the above information on average numbers of partners:

\renewcommand{\theenumi}{\roman{enumi}}
\renewcommand{\labelenumi}{(\theenumi)}

\begin{enumerate}
\item males exaggerate their number of female partners
\item $m = (9/10)f$
\item\label{m10.11} $m = (10/11)f$
\item $m = (11/10)f$
\item there cannot be a perfect matching with each male matched to one of
  his female partners
\item\label{no-female-match} there cannot be a perfect matching with each
  female matched to one of her male partners
\end{enumerate}

\begin{solution}
We know that the men's average number of partners is $f/m$ times the
female's average, so $f/m = 1.1$ which implies
\begin{itemize}

\item \eqref{m10.11}: the number of males is 10/11 times the number of
  females.

\end{itemize}
Also, since there are more females than males,

\begin{itemize}

\item \eqref{no-female-match}: there cannot be a perfect matching with each
  female matched to one of her male partners.

\end{itemize}

\end{solution}

\ppart[3] The data shows that approximately 20\% of the females were virgins,
while only 5\% of the males were.  The researcher wonders how excluding
virgins from the population would change the averages.  If he knew graph
theory, the researcher would realize that the nonvirgin male average
number of partners will be $x(f/m)$ times the nonvirgin female average number
of partners.  What is $x$?
\begin{center}
\exambox{0.5in}{0.5in}{-0.2in}
\end{center}

\begin{solution}
  The male average number of partners is $f/m$ times the female
  average number of partners.  (According to part~\eqref{partner-ratio},
  $f/m = 1.1$, but this number isn't needed here.)  When virgins are
  excluded, the ratio of the male's average to the females' average will
  be
\[
\frac{f - .2f}{m  - .05m} =  \frac{.8f}{.95m} = \frac{4/5}{19/20} \cdot \frac{f}{m},
\]
so $x = 80/95 = 16/19$.
\end{solution}

\eparts

\end{problem}

%%%%%%%%%%%%%%%%%%%%%%%%%%%%%%%%%%%%%%%%%%%%%%%%%%%%%%%%%%%%%%%%%%%%%
% Problem ends here
%%%%%%%%%%%%%%%%%%%%%%%%%%%%%%%%%%%%%%%%%%%%%%%%%%%%%%%%%%%%%%%%%%%%%

\endinput
