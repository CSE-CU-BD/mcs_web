\documentclass[problem]{mcs}

\begin{pcomments}
  \pcomment{FP_bipartite_matching_sex}
  \pcomment{perturbation of CP_bipartite_sex part(a)}
  \pcomment{from F11, edited ARM 11/3/13}
\end{pcomments}

\pkeywords{
  graph
  bipartite
  matching
  Halls_theorem
  sex
}

%%%%%%%%%%%%%%%%%%%%%%%%%%%%%%%%%%%%%%%%%%%%%%%%%%%%%%%%%%%%%%%%%%%%%
% Problem starts here
%%%%%%%%%%%%%%%%%%%%%%%%%%%%%%%%%%%%%%%%%%%%%%%%%%%%%%%%%%%%%%%%%%%%%

\begin{problem}
  A researcher analyzing data on heterosexual sexual behavior in a group
  of $m$ males and $f$ females found that within the group, the male
  average number of female partners was 15\% larger that the female
  average number of male partners.

  Circle all of the assertions below that are implied by the above
  information.  No explanation is required.

\renewcommand{\theenumi}{\roman{enumi}}
\renewcommand{\labelenumi}{(\theenumi)}

\begin{enumerate}
\item\label{mexagg} Males exaggerate their number of female partners.
\item\label{mf85} $m = 0.85f$
\item\label{mf115} $m = 1.15f$
\item\label{mf/115} $m = f/1.15$
\item\label{mf/1011} $m = (10/11)f$
%\item\label{mf/1720} $m = (17/20)f$
%\item\label{mf2023} $m = (20/23)f$
\item\label{male-match} There cannot be a matching with each male matched to one of
  his female partners.
\item\label{no-female-match} There cannot be a matching with each
  female matched to one of her male partners.
\end{enumerate}

\begin{solution}
  The implied assertions are~\eqref{mf/115} and~\eqref{no-female-match}.

  We know that the men's average number of partners is $f/m$ times the
  female's average, so $f/m = 1.15$ which implies~\eqref{mf/115}.

  Also, since there are 1.15 as many females as males, there are not
  enough males to match uniquely with females, so~\eqref{no-female-match}
  holds.
\end{solution}

\end{problem}

%%%%%%%%%%%%%%%%%%%%%%%%%%%%%%%%%%%%%%%%%%%%%%%%%%%%%%%%%%%%%%%%%%%%%
% Problem ends here
%%%%%%%%%%%%%%%%%%%%%%%%%%%%%%%%%%%%%%%%%%%%%%%%%%%%%%%%%%%%%%%%%%%%%

\endinput
