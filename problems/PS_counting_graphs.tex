\documentclass[problem]{mcs}

\begin{pcomments}
  \pcomment{PS_counting_graphs}
  \pcomment{from: F08 PS9 -> S09 PS9; S09 CP7R}
  \pcomment{part(c) rewritten by ARM 4/8/11}
\end{pcomments}

\pkeywords{
  counting
  graph
  asymmetric
  path-total
  partial_order
  permutation
  simple_graph
  digraph
}

%%%%%%%%%%%%%%%%%%%%%%%%%%%%%%%%%%%%%%%%%%%%%%%%%%%%%%%%%%%%%%%%%%%%%
% Problem starts here
%%%%%%%%%%%%%%%%%%%%%%%%%%%%%%%%%%%%%%%%%%%%%%%%%%%%%%%%%%%%%%%%%%%%%

\begin{problem}
In this problem, all graphs will have vertices $[1,n] \eqdef
\set{1,2,\dots,n}$; equivalently, all binary relations are on this set
$[1,n]$.


\bparts

\ppart How many simple undirected graphs are there?

\begin{solution}
There are $\binom{n}{2}$ potential edges, each of which may or
may not appear in a given graph.  Therefore, the number of graphs is:
\[
2^{\binom{n}{2}}
\]
\end{solution}

\ppart How many digraphs are there?

\begin{solution}

There are $n^2$ potential edges, each of which may or
may not appear in a given graph.  Therefore, the number of graphs is:
\[
2^{n^2}
\]

\end{solution}

\ppart How many asymmetric binary relations are there?

\begin{solution}
There are no self-loops in a tournament graph and 
for each of the $\binom{n}{2}$ pairs of distinct vertices $a$ and $b$,
either
\begin{enumerate}
\item $a \mrel{R} b$, or
\item $b \mrel{R} a$, or
\item neither,
\end{enumerate}
but not both.  Therefore, the number of tournament graphs is
\[
3^{\binom{n}{2}}
\]
\end{solution}

\iffalse

\newcommand{\beats}{\rightarrow}

\ppart Consider a $n$-player round-robin tournament where every pair of
  distinct players compete in a single game that doesn't allow for a tie.
  We can model the results of such a tournament using either a ``beats''
  relation or a digraph (called a \emph{tournament graph}).  The players
  are represented by vertices and there is an edge $x \beats y$ if $x$
  beat $y$.  How many tournament graphs are there with $V=\set{1,2,\dots,n}$?

\begin{solution}
There are no self-loops in a tournament graph and 
for each of the $\binom{n}{2}$ pairs of distinct vertices $a$ and $b$,
either $a \beats b$ or $b \beats a$ but not both.  Therefore, the 
number of tournament graphs is:
\[
2^{\binom{n}{2}}
\]
\end{solution}

\ppart How many acyclic tournament graphs are there with 
$V=\set{1,2,\dots,n}$?

\hint We showed in the class problems that an acyclic tournament 
graph defines a total order.

\begin{solution}
For any path from $x$ to $y$ in a tournament graph, an edge
$y \beats x$ would create a cycle.  Therefore in any acyclic tournament
graph, the existence of a path between distinct vertices $x$ and $y$ 
would require the edge $x \beats y$ also be in the graph.  That is, the
"beats" relation for such a graph would be transitive.  Since each 
pair of distinct players are comparable (either $x \beats y$ or 
$y \beats x$) we can uniquely rank the players $x_1 < x_2 < \cdots < x_n$.
There are $n!$ such rankings.
\fi

\ppart How many path-total strict partial orders are there?
\begin{solution}

$n!$.

Since the partial order is path-total, there is a unique listing of
the elements in decreasing partial order.  This listing defines a
bijection between the path-total strict partial orders and the
permutations of $[1,n]$.

\end{solution}

\eparts
\end{problem}

%%%%%%%%%%%%%%%%%%%%%%%%%%%%%%%%%%%%%%%%%%%%%%%%%%%%%%%%%%%%%%%%%%%%%
% Problem ends here
%%%%%%%%%%%%%%%%%%%%%%%%%%%%%%%%%%%%%%%%%%%%%%%%%%%%%%%%%%%%%%%%%%%%%

\endinput
