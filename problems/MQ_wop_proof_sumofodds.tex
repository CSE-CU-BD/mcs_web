\documentclass[problem]{mcs}

\begin{pcomments}
  \pcomment{MQ_wop_proof_sumofodds}
  \pcomment{by ARM 8/22/11; revised into parts 9/15/11}
\end{pcomments}

\pkeywords{
  well ordering
  sum
  contradiction
  summation
}

%%%%%%%%%%%%%%%%%%%%%%%%%%%%%%%%%%%%%%%%%%%%%%%%%%%%%%%%%%%%%%%%%%%%% 
% Problem starts here
%%%%%%%%%%%%%%%%%%%%%%%%%%%%%%%%%%%%%%%%%%%%%%%%%%%%%%%%%%%%%%%%%%%%%


\begin{problem}
We'll prove that for very positive integer, $n$, 
\iffalse
\emph{by the Well Ordering Principle}
\footnote{Proofs by other
  methods such by appeal to known formulas for similar sums or by induction
  will not receive full credit.}\fi
the sum of the first $n$ odd numbers is $n^2$, that is,
\begin{equation}\label{sumofodds}
\sum_{i=1}^{n} (2(i-1)+1) = n^2,
\end{equation}

Assume to the contrary that equation~\eqref{sumofodds} failed for some
positive integer, $n$.  Let $m$ be the least such number.

\bparts

\ppart Why must there be such an $m$?

\examspace[0.5in]
\begin{solution}
We're assuming the set of positive integers $n$ for which
equation~\eqref{sumofodds} fails is nonempty, so by WOP, there is a
least such number.
\end{solution}

\ppart\label{part-mgeq2} Explain why $m \geq 2$.

\examspace[0.5in]

\begin{solution}
When $n=1$, both sides of~\eqref{sumofodds} equal 1, so it must be
that $m \neq 1$, and since by definition $m$ is positive, it must be
$\geq 2$. 
\end{solution}

\ppart Explain why part~\eqref{part-mgeq2} implies that
\begin{equation}\label{sumofodds-to-m-1}
\sum_{i=1}^{m-1} (2(i-1)+1) = (m-1)^2.
\end{equation}
  
\examspace[0.75in]

\begin{solution}
By part~\eqref{part-mgeq2}, $m-1$ is positive.  Of course it is also
less than $m$.  Since $m$ is the least number for which
equation~\eqref{sumofodds} fails, the equation must hold when $n =
m-1$, namely equation~\eqref{sumofodds-to-m-1} must hold.
\end{solution}

\ppart\label{addnextodd} What term should be added to the left hand side
of~\eqref{sumofodds-to-m-1} so the result equals
\[
\sum_{i=1}^{m} (2(i-1)+1)?
\]

\examspace[0.5in]

\begin{solution}
$2(m-1)+1$
\end{solution}

\ppart Now prove that
\begin{equation}\label{contradict-sum-m}
\sum_{i=1}^{m} (2(i-1)+1) = m^2.
\end{equation}

\examspace

\begin{solution}
By part~\eqref{addnextodd}, adding $2(m-1)+1$ to both sides
of~\eqref{sumofodds-to-m-1} yields an equality whose left hand side is
the left hand side of~\eqref{contradict-sum-m}.  The right hand side
is
\[
(m-1)^2 + (2(m-1)+1) = (m^2 -2m + 1) + (2m - 2 + 1) = m^2,
\]
which proves~\eqref{contradict-sum-m}.
\end{solution}

\ppart Conclude that equation~\eqref{sumofodds} holds for all positive
integers, $n$.

\examspace[1.0in]

\begin{solution}
Equation~\eqref{contradict-sum-m} contradicts the fact
that~\eqref{sumofodds} did \emph{not} hold when $n = m$.  So the
assumption that~\eqref{sumofodds} failed for some positive integer, $n$,
led to a contradiction.  That means is must actually hold for
\emph{all} such $n$.

\end{solution}

\eparts

\end{problem}


%%%%%%%%%%%%%%%%%%%%%%%%%%%%%%%%%%%%%%%%%%%%%%%%%%%%%%%%%%%%%%%%%%%%%
% Problem ends here
%%%%%%%%%%%%%%%%%%%%%%%%%%%%%%%%%%%%%%%%%%%%%%%%%%%%%%%%%%%%%%%%%%%%%

\endinput
