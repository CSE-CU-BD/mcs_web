Content-Type: text/enriched
Text-Width: 70

<x-color><param>Cyan1</param>\documentclass</x-color>[problem]{<x-color><param>LightSteelBlue</param>mcs</x-color>}

<x-color><param>Cyan1</param>\begin</x-color>{<x-color><param>LightSkyBlue</param>pcomments</x-color>}
  <x-color><param>Cyan1</param>\pcomment</x-color>{MQ_<x-display><param>(raise -0.2)</param>w</x-display>op_<x-display><param>(raise -0.2)</param>p</x-display>roof_<x-display><param>(raise -0.2)</param>s</x-display>umofodds}
  <x-color><param>Cyan1</param>\pcomment</x-color>{by ARM 8/22/11; revised into parts 9/15/11}
<x-color><param>Cyan1</param>\end</x-color>{<x-color><param>LightSkyBlue</param>pcomments</x-color>}

<x-color><param>Cyan1</param>\pkeywords</x-color>{
  well ordering
  sum

  contradiction

  summation
}

%%%%%%%%%%%%%%%%%%%%%%%%%%%%%%%%%%%%%%%%%%%%%%%%%%%%%%%%%%%%%%%%%%%%%<x-color><param>chocolate1</param>
</x-color>% <x-color><param>chocolate1</param>Problem starts here
</x-color>%%%%%%%%%%%%%%%%%%%%%%%%%%%%%%%%%%%%%%%%%%%%%%%%%%%%%%%%%%%%%%%%%%%%%<x-color><param>chocolate1</param>
</x-color>
<x-color><param>Cyan1</param>\begin</x-color>{<x-color><param>LightSkyBlue</param>problem</x-color>}
We'll prove that for very positive integer, $n$, 
<x-color><param>Cyan1</param>\iffalse</x-color>
<x-color><param>Cyan1</param>\emph</x-color>{<italic>by the Well Ordering Principle</italic>}

<x-color><param>Cyan1</param>\footnote</x-color>{Proofs by other
  methods such by appeal to known formulas for similar sums or by induction
  will not receive full credit.}<x-color><param>Cyan1</param>\fi</x-color>
the sum of the first $n$ odd numbers is $n^<x-display><param>(raise 0.2)</param>2</x-display>$, that is,
<x-color><param>Cyan1</param>\begin</x-color>{<x-color><param>LightSkyBlue</param>equation</x-color>}<x-color><param>Cyan1</param>\label</x-color>{<x-color><param>Aquamarine</param>sumofodds</x-color>}
<x-color><param>Cyan1</param>\sum</x-color>_<x-display><param>(raise -0.2)</param>{i=1}</x-display>^<x-display><param>(raise 0.2)</param>{n}</x-display> (2(i-1)+1) = n^<x-display><param>(raise 0.2)</param>2</x-display>,
<x-color><param>Cyan1</param>\end</x-color>{<x-color><param>LightSkyBlue</param>equation</x-color>}
assume to the contrary that equation~<x-color><param>Cyan1</param>\eqref</x-color>{<x-color><param>Aquamarine</param>sumofodds</x-color>} failed for some
positive integer, $n$.  Let $m$ be the least such number.

<x-color><param>Cyan1</param>\bparts</x-color>

<x-color><param>Cyan1</param>\ppart</x-color> Why must there be such an $m$.

<x-color><param>Cyan1</param>\begin</x-color>{<x-color><param>LightSkyBlue</param>solution</x-color>}
We're assuming the set of positive integers $n$ for which
equation~<x-color><param>Cyan1</param>\eqref</x-color>{<x-color><param>Aquamarine</param>sumofodds</x-color>} fails is nonempty, so by WOP, there is a
least such number.
<x-color><param>Cyan1</param>\end</x-color>{<x-color><param>LightSkyBlue</param>solution</x-color>}

<x-color><param>Cyan1</param>\ppart\label</x-color>{<x-color><param>Aquamarine</param>part-mgeq2</x-color>} Explain why $m \geq 2$.

<x-color><param>Cyan1</param>\begin</x-color>{<x-color><param>LightSkyBlue</param>solution</x-color>}
When $n=1$, both sides of~<x-color><param>Cyan1</param>\eqref</x-color>{<x-color><param>Aquamarine</param>sumofodds</x-color>} equal 1, so it must be
that $m \neq 1$, and since by definition $m$ is positive, it must be
$\geq 2$. 
<x-color><param>Cyan1</param>\end</x-color>{<x-color><param>LightSkyBlue</param>solution</x-color>}

<x-color><param>Cyan1</param>\ppart</x-color> Explain why part~<x-color><param>Cyan1</param>\eqref</x-color>{<x-color><param>Aquamarine</param>part-mgeq2</x-color>} implies that
<x-color><param>Cyan1</param>\begin</x-color>{<x-color><param>LightSkyBlue</param>equation</x-color>}<x-color><param>Cyan1</param>\label</x-color>{<x-color><param>Aquamarine</param>sumodds-to-m-1</x-color>}
<x-color><param>Cyan1</param>\sum</x-color>_<x-display><param>(raise -0.2)</param>{i=1}</x-display>^<x-display><param>(raise 0.2)</param>{m-1}</x-display> (2(i-1)+1) = (m-1)^<x-display><param>(raise 0.2)</param>2</x-display>.
<x-color><param>Cyan1</param>\end</x-color>{<x-color><param>LightSkyBlue</param>equation</x-color>}

<x-color><param>Cyan1</param>\begin</x-color>{<x-color><param>LightSkyBlue</param>solution</x-color>}
By part~<x-color><param>Cyan1</param>\eqref</x-color>{<x-color><param>Aquamarine</param>part-mgeq2</x-color>}, $m-1$ is positive.  Of course it is also
less than $m$.  Since $m$ is the least number for which
equation~<x-color><param>Cyan1</param>\eqref</x-color>{<x-color><param>Aquamarine</param>sumofodds</x-color>} fails, the equation must hold when $n =
m-1$, namely equation~<x-color><param>Cyan1</param>\eqref</x-color>{<x-color><param>Aquamarine</param>sumofodds-to-m-1</x-color>} must hold.
<x-color><param>Cyan1</param>\end</x-color>{<x-color><param>LightSkyBlue</param>solution</x-color>}

<x-color><param>Cyan1</param>\ppart\label</x-color>{<x-color><param>Aquamarine</param>addnextodd</x-color>} What term should be added to the left hand side
of~<x-color><param>Cyan1</param>\reqref</x-color>{sumodds-to-m-1} so the result equals $\sum_<x-display><param>(raise -0.2)</param>{i=1}</x-display>^<x-display><param>(raise 0.2)</param>{m}</x-display>
(2(i-1)+1)$?
<x-color><param>Cyan1</param>\begin</x-color>{<x-color><param>LightSkyBlue</param>solution</x-color>}
$2(m-1)+1$
<x-color><param>Cyan1</param>\end</x-color>{<x-color><param>LightSkyBlue</param>solution</x-color>}

<x-color><param>Cyan1</param>\ppart</x-color> Now prove that


<x-color><param>Cyan1</param>\begin</x-color>{<x-color><param>LightSkyBlue</param>equation</x-color>}<x-color><param>Cyan1</param>\label</x-color>{<x-color><param>Aquamarine</param>contradict-sum-m</x-color>}
<x-color><param>Cyan1</param>\sum</x-color>_<x-display><param>(raise -0.2)</param>{i=1}</x-display>^<x-display><param>(raise 0.2)</param>{m}</x-display> (2(i-1)+1) = m^<x-display><param>(raise 0.2)</param>2</x-display>.
<x-color><param>Cyan1</param>\end</x-color>{<x-color><param>LightSkyBlue</param>equation</x-color>}


<x-color><param>Cyan1</param>\begin</x-color>{<x-color><param>LightSkyBlue</param>solution</x-color>}

By part~<x-color><param>Cyan1</param>\eqref</x-color>{<x-color><param>Aquamarine</param>addnextodd</x-color>}, adding $2(m-1)+1$ to both sides
of~<x-color><param>Cyan1</param>\eqref</x-color>{<x-color><param>Aquamarine</param>sumodds-to-m-1</x-color>} yields an equality whose left hand side is
the left hand side of~<x-color><param>Cyan1</param>\eqref</x-color>{<x-color><param>Aquamarine</param>contradict-sum-m</x-color>}.  The right hand side
is

<x-color><param>Cyan1</param>\[</x-color>
(m-1)^<x-display><param>(raise 0.2)</param>2</x-display> + (2(m-1)+1) = (m^<x-display><param>(raise 0.2)</param>2</x-display> -2m + 1) + (2m - 2 + 1) = m^<x-display><param>(raise 0.2)</param>2</x-display>,
<x-color><param>Cyan1</param>\]</x-color>
which proves~<x-color><param>Cyan1</param>\eqref</x-color>{<x-color><param>Aquamarine</param>contradict-sum-m</x-color>}.

<x-color><param>Cyan1</param>\end</x-color>{<x-color><param>LightSkyBlue</param>solution</x-color>}


<x-color><param>Cyan1</param>\ppart</x-color> Conclude that equation~<x-color><param>Cyan1</param>\eqref</x-color>{<x-color><param>Aquamarine</param>sumodds</x-color>} holds for all positive
integers, $n$.

<x-color><param>Cyan1</param>\begin</x-color>{<x-color><param>LightSkyBlue</param>solution</x-color>}
Equation~eqref{contradict-sum-m} contradicts the fact
that~<x-color><param>Cyan1</param>\eqref</x-color>{<x-color><param>Aquamarine</param>sumodds</x-color>} did <x-color><param>Cyan1</param>\emph</x-color>{<italic>not</italic>} hold when $n = m$.  So the
assumption that~<x-color><param>Cyan1</param>\eqref</x-color>{<x-color><param>Aquamarine</param>sumodds</x-color>} failed for some positive integer, $n$,
lead to a contradiction.  That means is must actually hold for
<x-color><param>Cyan1</param>\emph</x-color>{<italic>all</italic>} such $n$
<x-color><param>Cyan1</param>\end</x-color>{<x-color><param>LightSkyBlue</param>solution</x-color>}

<x-color><param>Cyan1</param>\eparts</x-color>


<x-color><param>Cyan1</param>\end</x-color>{<x-color><param>LightSkyBlue</param>problem</x-color>}


%%%%%%%%%%%%%%%%%%%%%%%%%%%%%%%%%%%%%%%%%%%%%%%%%%%%%%%%%%%%%%%%%%%%%<x-color><param>chocolate1</param>
</x-color>% <x-color><param>chocolate1</param>Problem ends here
</x-color>%%%%%%%%%%%%%%%%%%%%%%%%%%%%%%%%%%%%%%%%%%%%%%%%%%%%%%%%%%%%%%%%%%%%%<x-color><param>chocolate1</param>
</x-color><x-color><param>Cyan1</param>\endinput</x-color>
