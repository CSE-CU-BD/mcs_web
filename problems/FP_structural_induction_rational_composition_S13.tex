\documentclass[problem]{mcs}

\begin{pcomments}
  \pcomment{FP_structural_induction_rational_composition_S13}
  \pcomment{rewording of non_S13 version}
  \pcomment{subsumed by CP_structural_induction_rational_composition}
  \pcomment{This is part(a) of CP_structural_induction_rational_composition}
  \pcomment{S15.mid2, F15.mid2}
  \pcomment{ARM 1/31/12, adamc 3/17/15, ARM revise10/18/15}
\end{pcomments}

\pkeywords{
  structural_induction
  functions
  composition
  rational_function}

%%%%%%%%%%%%%%%%%%%%%%%%%%%%%%%%%%%%%%%%%%%%%%%%%%%%%%%%%%%%%%%%%%%%%
% Problem starts here
%%%%%%%%%%%%%%%%%%%%%%%%%%%%%%%%%%%%%%%%%%%%%%%%%%%%%%%%%%%%%%%%%%%%%

\providecommand{\RAF}{\ms{RAF}}

\begin{problem}

\begin{definition*}
The set $\RAF$ of \emph{rational functions} of one real variable is
the set of functions defined recursively as follows:

\inductioncase{Base cases:}
\begin{itemize}
\item The identity function, $\ide(r) \eqdef r$ for $r \in
  \reals$ (the real numbers), is an $\RAF$,
\item any constant function on $\reals$ is an $\RAF$.
\end{itemize}

\inductioncase{Constructor cases:}
If $f,g$ are $\RAF$'s, then so are
\begin{itemize}
\item $f + g$ $fg$ and $f/g$.\label{RAF+*}
%\item the inverse function $f^{(-1)}$,\label{L:inversefunc}
%\item the composition $f \compose g$.\label{cmp}
\end{itemize}
\end{definition*}

Prove by structural induction that $\RAF$ is closed under composition.
That is, using the induction hypothesis,
\[
P(h) \eqdef \forall g \in \RAF.\ h \compose g \in \RAF,
\]
prove that $P(h)$ holds for all $h \in \RAF$.  Make sure to indicate
explicitly
\begin{itemize}
\item each of the base cases, and
\item each of the constructor cases.
\end{itemize}

\examspace[3.5in]
\begin{solution}

\begin{staffnotes}
Base cases 2pts (1pt each); knowing what the constructor cases are:
2pts; explaning one or more constructor cases correctly: 2pts
\end{staffnotes}

\begin{proof}
  \inductioncase{base cases}: We must show $P(\ide_\reals)$ and
  $P(\text{constant-function})$.  But this follows immediately from
  the fact that $\ide_\reals \compose g = g$ and the composition of a
  constant function with any function $g$ is constant.
  
  \inductioncase{constructor cases}: Given $e, f \in \RAF$, we may
  assume by structural induction that $P(e)$ and $P(f)$ both hold, and
  must prove $P(h)$ for $h= e \circleast f$ in the three cases where
  $\circleast = +$ $\cdot$ or $\div$.  But the same proof works for
  all three cases.  The key observation is that
\[
(e \circleast f) \compose g = (e \compose g) \circleast (f \compose g).
\]
This follows simply  by definition of composition and $\circleast$ of
functions.\footnote{It's worth stopping a moment to realize that the
  similar-looking identity
\[
\textcolor{red}{
g \compose (e \circleast f) = (g \compose e) \circleast (g \compose f)
}
\]
is altogether \textbf{false}.}
Now since $(e \compose g), (f \compose g) \in \RAF$ for all $g \in
\RAF$ by hypothesis, so is their combination using the operator
$\circleast$---that is, their sum/product/quotient---by the
constructor rule~\eqref{RAF+*}.  This proves $P(h)$ for all three of
the constructor cases $\circleast = +$ $\cdot$ or $\div$.

This completes all the constructor cases, and so $\forall h \in
\RAF.\, P(h)$ follows by structural induction.

\end{proof}

\end{solution}
\end{problem}
%%%%%%%%%%%%%%%%%%%%%%%%%%%%%%%%%%%%%%%%%%%%%%%%%%%%%%%%%%%%%%%%%%%%%
% Problem ends here
%%%%%%%%%%%%%%%%%%%%%%%%%%%%%%%%%%%%%%%%%%%%%%%%%%%%%%%%%%%%%%%%%%%%%

\endinput
