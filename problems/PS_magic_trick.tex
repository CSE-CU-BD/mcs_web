\documentclass[problem]{mcs}

\begin{pcomments}
  \pcomment{from: S09 ps9}
  \pcomment{subsumed by PS_magic_trick_4cards}
\end{pcomments}

\pkeywords{
  card_magic
  matching
  bipartite
  degree-constrained
  degree_constrained
}

%%%%%%%%%%%%%%%%%%%%%%%%%%%%%%%%%%%%%%%%%%%%%%%%%%%%%%%%%%%%%%%%%%%%%
% Problem starts here
%%%%%%%%%%%%%%%%%%%%%%%%%%%%%%%%%%%%%%%%%%%%%%%%%%%%%%%%%%%%%%%%%%%%%


\begin{problem}

\newcommand{\pcard}[1]{\,\fbox{\parbox[c][.5in][c]{.3in}{\centering#1}}\,}

The Magician and his Assistant are determined to find a way to make the 
trick work when the audience selects four cards from a 52-card deck and 
the Assistant reveals only three in a way that lets the Magician ``guess'' 
the hidden fourth card.  As shown in the lecture notes this is not possible.  
They decide to change the rules slightly: instead of the Assistant lining 
three cards face-up for the Magician to see, he will line up all four 
cards with one card face down and the other three visible. 

Suppose the audience members had selected the cards $9\heartsuit$,
$10\diamondsuit$, $A\clubsuit$ $5\clubsuit$.  Then Assistant could choose
to arrange the 4 cards in any order so long as one is face down and the
others are visible.  Two possibilities are:
\begin{center}
\pcard{$A\clubsuit$}\pcard{?}\pcard{$10\diamondsuit$}\pcard{$5\clubsuit$}

\pcard{?}\pcard{$5\clubsuit$}\pcard{$9\heartsuit$}\pcard{$10\diamondsuit$}
\end{center}

Explain how the Assistant can communicate the hidden card to the Magician
in this slightly modified version of the magic trick. We only need you
to prove the existence of such method instead of giving an actual
algorithm. (Giving an actual algorithm will be merited extra credit.)


\begin{solution}
We proceed by again modeling the trick as a bipartite matching 
problem.  The vertices $L$ are the collection of all possible sets 
of 4 cards, representing the possible sets of cards selected by the 
audience.  The vertices $R$ are the collection of all possible 
length 4 sequences where the first 3 positions contain distinct cards 
representing the ordering of the visible cards and the 4th position 
is an element of $\set{1,2,3,4}$ denoting the position of the hidden 
card.

There is an edge between $l \in L$ and $r \in R$ if the set of visible 
cards in $r$ is a subset of $l$. 

A matching from $L$ to $R$ would represent a unique way to encode each
set of 4 cards as a sequence of 1 hidden and 3 visible cards.  For each
$l \in L$,
\[
\degr{l} = \binom{4}{3} 3! \cdot 4 = 96
\]
and for each $r \in R$,
\[
\degr{r} = \binom{52-3}{1} = 49.
\]

Therefore the graph is degree-constrained and therefore satisfies 
Hall's matching condition.
\end{solution}

\end{problem}

%%%%%%%%%%%%%%%%%%%%%%%%%%%%%%%%%%%%%%%%%%%%%%%%%%%%%%%%%%%%%%%%%%%%%
% Problem ends here
%%%%%%%%%%%%%%%%%%%%%%%%%%%%%%%%%%%%%%%%%%%%%%%%%%%%%%%%%%%%%%%%%%%%%

\endinput
