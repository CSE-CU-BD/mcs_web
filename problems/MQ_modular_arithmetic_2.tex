\documentclass[problem]{mcs}

\begin{pcomments}
  \pcomment{MQ_modular_arithmetic_2}
\end{pcomments}

\pkeywords{
  modular_arithmetic
  gcd
  phi
  Euler
}

%%%%%%%%%%%%%%%%%%%%%%%%%%%%%%%%%%%%%%%%%%%%%%%%%%%%%%%%%%%%%%%%%%%%%
% Problem starts here
%%%%%%%%%%%%%%%%%%%%%%%%%%%%%%%%%%%%%%%%%%%%%%%%%%%%%%%%%%%%%%%%%%%%%

\begin{problem}
% Tests understanding of gcd and modular arithmetic

\begin{problemparts}

\problempart
\label{phi}
Calculate the value of $\phi(100)$.
\examspace[3in]

\begin{solution}
\[
\phi(100) = \phi(25) \phi(4) = \phi(5^2) \phi(2^2) =  (5^2 - 5)(2^2 - 2) = 40.
\]
\end{solution}


\problempart Assume an integer $k > 9$ is relatively prime to 100.  Explain why
the last two digits of $k$ and $k^{121}$ are the same.

\hint Use your solution from part (\ref{phi}).

\begin{solution}
Notice that all we have to prove is that $k$ and $k^{121}$ are
congruent mod 100, implying they have the same last two digits.
\[
k^{121} \equiv k^{40\cdot 3 + 1} \equiv k(k^{40})^3 \pmod{100}.
\]  By
Euler's Theorem, since $k$ and 100 are relatively prime,
$n^{\phi(100)} \equiv 1 \pmod{100}$.  By part (a), we have that
$\phi(100) = 40$, implying $k^{40} \equiv 1 \pmod{100}$.  Hence,
$n(k^{40})^3 \equiv k (1^3) \equiv k \pmod{100}$.
\end{solution}

\end{problemparts}
\end{problem}

\endinput
