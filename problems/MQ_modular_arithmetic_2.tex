\documentclass[problem]{mcs}

\begin{pcomments}
  \pcomment{}
\end{pcomments}

\pkeywords{
	modular_arithmetic
}

%%%%%%%%%%%%%%%%%%%%%%%%%%%%%%%%%%%%%%%%%%%%%%%%%%%%%%%%%%%%%%%%%%%%%
% Problem starts here
%%%%%%%%%%%%%%%%%%%%%%%%%%%%%%%%%%%%%%%%%%%%%%%%%%%%%%%%%%%%%%%%%%%%%

\begin{problem}
% [Properties]
% Tests understanding of gcd and modular arithmetic

\begin{problemparts}

\problempart
Calculate the value of $\phi(100)$.
\begin{solution}
[\vspace{1.5in}]{  $\phi(100) = \phi(25) \phi(4) = \phi(5^2) \phi(4^2) =  (5^2 - 5)(2^2 - 2) = 40$.
\end{solution}


\problempart
Assume $n$ is a positive integer greater than 9, and relatively prime to 100. Explain why the last two digits of $n$ and $n^{121}$ are the same.
\begin{solution}
Notice that all we have to prove is that $n$ and $n^{121}$ are congruent mod 100, implying they have the same last two digits.

$n^{121} \equiv n^{40\cdot 3 + 1} \equiv n(n^{40})^3 \pmod{100}$. 
By Euler's Theorem, since $n$ and 100 are relatively prime, $n^{\phi(100)} \equiv 1 \pmod{100}$.
By part (a), we have that $\phi(100) = 40$, implying  $n^{40} \equiv 1 \pmod{100}$.
Hence, $n(n^{40})^3 \equiv n (1^3) \equiv n \pmod{100}$. 
\end{solution}


\end{problemparts}
\end{problem}