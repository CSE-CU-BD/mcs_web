\documentclass[problem]{mcs}

\begin{pcomments}
  \pcomment{FP_counting_poker_high_cards}
\end{pcomments}

\pkeywords{
  counting
}

%%%%%%%%%%%%%%%%%%%%%%%%%%%%%%%%%%%%%%%%%%%%%%%%%%%%%%%%%%%%%%%%%%%%%
% Problem starts here
%%%%%%%%%%%%%%%%%%%%%%%%%%%%%%%%%%%%%%%%%%%%%%%%%%%%%%%%%%%%%%%%%%%%%

\begin{problem}
In a standard 52-card deck (13 ranks and 4 suits),
a hand is a 5-card subset of the set of 52 cards.
Express the answer to each part as a formula using
factorial, binomial, or multinomial notation.

\bparts

\ppart
Let $H$ be the set of all hands. What is $\card{H}$?\hfill\examrule
\begin{solution}
$\card{H} = \binom{52}{5}$
\end{solution}

\ppart
Let $H_{NP}$ be the set of all hands that does not include a pair,
that is,  no two card in the hand have the same rank. What is
$\card{H_{NP}}$?  \hfill\examrule[1.0in]

\begin{solution}
$\card{H_{NP}} = \binom{13}{5} \binom{4}{1}^5$
\end{solution}

\ppart
Let $H_{S}$ be the set of all hands that is a straight,
i.e. the rank of the five cards are consecutive.
The order of the ranks is $(A,2,3,4,5,6,7,8,9,10,J,Q,k,A)$,
note that $A$ is appears twice.
What is $\card{H_{S}}$?\hfill\examrule[1.0in]
\begin{solution}
$\card{H_{S}} = \binom{10}{1} \binom{4}{1}^5$
\end{solution}

\ppart
Let $H_{F}$ be the set of all hands that is a flush,
i.e. the suit of the five cards are identical.
What is $\card{H_{F}}$? \hfill\examrule[1.0in]
\begin{solution}
$\card{H_{F}} = \binom{13}{5} \binom{4}{1}$
\end{solution}

\ppart
Let $H_{SF}$ be the set of all straight flush hands
that is both a straight and a flush.
What is $\card{H_{SF}}$? \hfill\examrule[1.0in]
\begin{solution}
$\card{H_{SF}} = \binom{10}{1} \binom{4}{1}$
\end{solution}

\ppart
Let $H_{HC}$ be the set of all high card hands
that is hands that do not include a pair,
are not straights, and are not flushs.
What is $\card{H_{HC}}$?\hfill\examrule[1.0in]

\begin{solution}
\begin{align*}
\card{H_{HC}} & = \card{H_{NP}} - \card{H_{S}} - \card{H_{F}} + \card{H_{SF}}\\
             & = \binom{13}{5} \binom{4}{1}^5 - \binom{10}{1}\binom{4}{1}^5\\
             & \quad  - \binom{13}{5} \binom{4}{1} + \binom{10}{1}\binom{4}{1}\\
             & = \paren{\binom{4}{1}^5 - \binom{4}{1}} \paren{\binom{13}{5} - \binom{10}{1}}
\end{align*}
\end{solution}

\eparts
\end{problem}

%%%%%%%%%%%%%%%%%%%%%%%%%%%%%%%%%%%%%%%%%%%%%%%%%%%%%%%%%%%%%%%%%%%%%
% Problem ends here
%%%%%%%%%%%%%%%%%%%%%%%%%%%%%%%%%%%%%%%%%%%%%%%%%%%%%%%%%%%%%%%%%%%%%

\endinput
