\documentclass[problem]{mcs}

\begin{pcomments}
    \pcomment{TP_CNF_and_DNF}
    \pcomment{by zabel 2/17/18}
    \pcomment{S18.mid1}
\end{pcomments}

\pkeywords{
  Conjunctive Normal Form
  CNF
  Disjunctive Normal Form
  DNF
  Full CNF
}

\begin{problem}

The five-variable propositional formula
\[
    P\eqdef (A\QAND B \QAND \bar{C} \QAND D \QAND \bar{E}) \QOR
    (\bar{A} \QAND B \QAND \bar{C} \QAND \bar{E})
\]
is in Disjunctive Normal Form with two ``$\QAND$-of-literal''
clauses.

\bparts

\ppart
Find a \textbf{Full}~Disjunctive Normal Form that is
equivalent to $P$, and explain your reasoning.

\hint Can you narrow in on the important parts of the truth table without writing all of it? Alternatively, can you avoid the truth table altogether?

\begin{solution}
The formula $P$ is not a Full DNF because the second clause does not
mention variable $D$.  So let's put in the possibilities for $D$.
Using our equivalence axioms, $P$'s second clause can be equivalently
written as
\begin{align*}
\lefteqn{(\bar{A} \QAND B \QAND \bar{C} \QAND \bar{E})}\\
& \leftrightarrow (\bar{A} \QAND B \QAND \bar{C} \QAND \bar{E}) \QAND \True \\
& \leftrightarrow (\bar{A} \QAND B \QAND \bar{C} \QAND \bar{E}) \QAND (D \QOR \bar{D})\\
& \leftrightarrow (\bar{A} \QAND B \QAND \bar{C} \QAND \mathbf{D} \QAND \bar{E}) \QOR
                  (\bar{A} \QAND B \QAND \bar{C} \QAND \mathbf{\bar{D}} \QAND \bar{E}),
\end{align*}
where the last line uses the Distributive rule.\footnote{It also uses
  the Associative and Commutative rules for \QAND, but we take these
  ``formatting'' rules for granted.}  So a full DNF is
\begin{align*}
P &\leftrightarrow (A\QAND B \QAND \bar{C} \QAND D \QAND \bar{E}) \QOR {}\\
  &\mathbin{\phantom{\leftrightarrow}}
                   (\bar{A} \QAND B \QAND \bar{C} \QAND D \QAND \bar{E})\QOR {}\\
  &\mathbin{\phantom{\leftrightarrow}}
                   (\bar{A} \QAND B \QAND \bar{C} \QAND \bar{D} \QAND \bar{E}).
  \end{align*}
Note that writing ``$P = \cdots$'' instead of ``$P \leftrightarrow
\cdots$'' would not quite be correct: $P$ and the Full DNF are
\emph{different} propositional formulas, even though they are
logically equivalent.


As an alternate solution, let's consider the truth table for $P$ without fully writing it out. The first clause of $P$ is satisfied by a single row of the truth table, namely $T,T,F,T,F$. To satisfy the second clause, the values of $A,B,C,E$ must be $F,T,F,F$ respectively, but $D$ can be either $T$ or $F$, so this clause is satisfied by two different rows, $F,T,F,T,F$ and $F,T,F,F,F$. These are precisely the rows of the truth table where $P$ evaluates to $\True$, so $P$ is equivalent to the Full DNF $A^{TTFTF}\QOR A^{FTFTF}\QOR A^{FTFFF}$, which can be expressed more fully as above.
\end{solution}



\begin{staffnotes}
The problem asks for and provides \textbf{\emph{a}} Full DNF.  Full
DNF is not unique (canonical) until each of the $\QAND$-of-literal
clauses is alphabetized, and the clauses themselves are sorted in some
standard order without duplicates.  The Full DNF given above is
canonical because its three clauses are distinct, alphabetized, and
appear in dictionary order determined by having each variable
alphabetically precede its complemented form, that is, $V$ precedes
$\bar{V}$ for each variable $V$ so the dictionary order of the
literals above is $A, \bar{A}, B, \bar{B}, C, \bar{C},\dots$.
\end{staffnotes}

\examspace[3in]

\ppart Let~$C$ be a \textbf{Full}~\textbf{Con}junctive Normal Form
that is equivalent to~$P$.  Assume that $C$ has been simplified so
that none of its ``$\QOR$-of-literals'' clauses are equivalent to each
other. How many clauses are there in $C$? (Please \textbf{don't} try to write out any of these clauses.)

\exambox{1.5in}{0.5in}{0.3in}

Briefly explain your answer.





\begin{solution}
  Answer: $29$ clauses.
    
As $P$ has five variables, $P$'s truth table has $2^5 = 32$ rows.  We've
shown that precisely $3$ rows evaluate to True, since these correspond
to the clauses of $P$'s Full DNF.  The remaining $2^5 - 3 = 29$ rows
correspond to the clauses of $P$'s Full CNF.
\end{solution}

\eparts

\end{problem}

\endinput
