\documentclass[problem]{mcs}

\begin{pcomments}
  \pcomment{CP_minimal_maximal_elements}
  \pcomment{from: S09.cp4m}
\end{pcomments}

\pkeywords{
  partial_orders
  chains_and_antichains
  primes
  relations
}

%%%%%%%%%%%%%%%%%%%%%%%%%%%%%%%%%%%%%%%%%%%%%%%%%%%%%%%%%%%%%%%%%%%%%
% Problem starts here
%%%%%%%%%%%%%%%%%%%%%%%%%%%%%%%%%%%%%%%%%%%%%%%%%%%%%%%%%%%%%%%%%%%%%

\begin{problem}

\bparts

\ppart What are the maxim\emph{al} and minim\emph{al} elements, if
any, of the power set $\power(\set{1,\dots,n})$, where $n$ is a
positive integer, under the \term{empty relation}?

\begin{solution}
The power set is a red herring.  With an empty relation on any set, every
element is maximal and minimal.
\end{solution}

\ppart What are the maxim\emph{al} and minim\emph{al} elements, if
any, of the set, $\naturals$, of all nonnegative integers under
divisibility?  Is there a minim\emph{um} or maxim\emph{um} element?

\begin{solution}
The minimum (and therefore unique minimal) element is 1 since 1
divides all natural numbers.  The maximum (and therefore unique
maximal) element is 0, since all numbers divide 0.
\end{solution}

\ppart What are the minimal and maximal elements, if any, of the set
of integers greater than 1 under divisibility?

\begin{solution}
All prime numbers are minimal elements, since no numbers divide
them.

There is no maximal element, because for any $n > 1$, there is  a
``larger'' number under the divisibility partial order, for example,
$2n$.
\end{solution}

\iffalse
\ppart
What is the size of the longest chain that is guaranteed to exist in any
partially ordered set of $n>0$ elements?  What about the largest antichain?

\begin{solution}
Chain size is 1 in the empty relation on a set of any size, since no two
elements are comparable.  Antichain size is 1 if the whole partial
order is a chain.

\end{solution}
\fi

\ppart Describe a partially ordered set that has no minimal or maximal
elements.

\begin{solution}
$\integers$, $\reals$, etc.
\end{solution}

\ppart Describe a partially ordered set that has a \emph{unique minimal}
element, but no minimum element. \hint It will have to be infinite.

\begin{solution}
  $\integers \union \set{i}$ where $i$ is a root of $-1$, under the usual
  order $\integers$.  So $i$ is incomparable to everything but itself, and
  is therefore minimal ---and maximal too.  The remaining elements are the
  integers, and none of them are minimal since $n-1 < n$, which makes $i$
  unique.
\end{solution}

\eparts
\end{problem}

%%%%%%%%%%%%%%%%%%%%%%%%%%%%%%%%%%%%%%%%%%%%%%%%%%%%%%%%%%%%%%%%%%%%%
% Problem ends here
%%%%%%%%%%%%%%%%%%%%%%%%%%%%%%%%%%%%%%%%%%%%%%%%%%%%%%%%%%%%%%%%%%%%%

\endinput
