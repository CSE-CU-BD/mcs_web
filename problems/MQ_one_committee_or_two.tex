\documentclass[problem]{mcs}

\begin{pcomments}
  \pcomment{MQ_one_committee_or_two}
\end{pcomments}

\pkeywords{
}

%%%%%%%%%%%%%%%%%%%%%%%%%%%%%%%%%%%%%%%%%%%%%%%%%%%%%%%%%%%%%%%%%%%%%
% Problem starts here
%%%%%%%%%%%%%%%%%%%%%%%%%%%%%%%%%%%%%%%%%%%%%%%%%%%%%%%%%%%%%%%%%%%%%

\begin{problem}
Twenty people work at CantorCorp, a small, unsuccessful start-up.  A single
six-person committee is to be formed.  (It will be charged with the sole task of 
working to prove the Continuum Hypothesis.)  Employees appointed to serve on the
committee join as equals -- they do not get assigned distinct roles or ranks.

\begin{problemparts}

\problempart
Let $D$ denote the set of all possible committees.  Find $|D|$.
\begin{solution}
Evidently,
\[|D| = \binom{20}{6}.\]
\end{solution}

\problempart
Two of the workers, Aleph and Beth, will be unhappy if they are to serve together.

Let $P$ denote the set of all possible committees on which Aleph and Beth would serve together.
Find $|P|$.
\begin{solution}
With Aleph and Beth serving, there are $20-2=18$ employees left to choose from, and $6-2=4$ spots left in the committee, so
\[|P|=\binom{18}{4}.\]
\end{solution}

\problempart
Beth will also be unhappy if she has to serve with \textbf{both} Ferdinand and Georg.

Let $Q$ denote the set of all possible committees on which Beth, Ferdinand, and Georg would all serve 
together.
Find $|Q|$.
\begin{solution}
By reasoning similar to that used to find $|P|$, obtain 
\[|Q|=\binom{17}{3}.\]
\end{solution}

\problempart
Find $|P\cap Q|$.
\begin{solution}
$P\cap Q$ is the set of all possible committees on which Aleph, Beth, Ferdinand, and Georg all serve 
together.  Clearly,
\[|P\cap Q|=\binom{16}{2}.\]
\end{solution}

\problempart
Let $S$ denote the set of all possible committees on which there is at least one unhappy employee.
Express $S$ in terms of $P$ and $Q$ \textbf{only}.
\begin{solution}
\[S = P\cup Q.\]
\end{solution}

\problempart
Find $|S|$.
\begin{solution}
Applying inclusion/exclusion, have
\begin{align*}
|S|       &= |P \cup Q| \\         
          &= |P| + |Q| - |P\cap Q| \\
          &= \binom{18}{4} + \binom{17}{3} - \binom{16}{2} \\
\end{align*}
\end{solution}

\problempart
If we want to form a committee with no unhappy employees, how many choices do we have to choose from? 
\begin{solution}
Let $R$ denote the set of all possible committees on which no employee is unhappy.  Clearly, $D = R\cup S$ 
and $R\cap S = \emptyset$.  So apply the Sum Rule: $|D| = |R| + |S|$.  Hence:
\begin{align*}
|R| &= |D| - |S| \\
          &= \binom{20}{6} - \binom{18}{4} - \binom{17}{3} + \binom{16}{2}
\end{align*}
\end{solution}

\problempart
Suddenly, we realize that it would be better to have two six-person committees instead of one.
(One committee would work on proving the Continuum Hypothesis, while the other would work to disprove 
it!)  Each employee can serve on at most one committee.  How many ways are there to form 
such a pair of committees, while keeping all employees happy?
\begin{solution}
Let's call a particular valid assignment of employees to the two committees an ``arrangement''.

Let $D'$ denote the set of all possible arrangements.  There are $\displaystyle\binom{20}{6}$ ways to choose employees for the first committee.
For each of these, there are 14 people left over and so $\displaystyle\binom{14}{6}$ ways to assign
employees to the second committee.  By the Generalized Product Rule (remember, the committees
do different things),
\[|D'| = \binom{20}{6}\binom{14}{6}.\]

For a more rigorous approach, start by assigning a unique identifier (ID) to each
employee at CantorCorp.  Denote the set of all IDs by $I$.  Then establish the obvious bijection
(with an obvious bijection rule) between the set, $D'$, of all possible arrangements and the set, $X$, of all ordered pairs $\paren{S_1,S_2}$, where
\begin{itemize}
\item $S_1 \subseteq I$.
\item $S_2 \subseteq I$.
\item $|S_1|=|S_2|=6$.
\item $S_1\cap S_2 = \emptyset$.
\end{itemize}
There are then $\displaystyle\binom{20}{6}$ ways to choose $S_1$, and for each of these, $\displaystyle\binom{14}{6}$
ways to choose $S_2$.   Thus $\displaystyle  |X|=\binom{20}{6}\binom{14}{6}$ and hence $\displaystyle  |D'|=\binom{20}{6}\binom{14}{6}$.

Let $P'$ denote the set of all possible arrangments in which Aleph and Beth would serve on the same committee.  Either Aleph and Beth serve 
on the first committee or they serve on the second.  If the first, then
there are 18 employees left to choose from to fill the remaining four spots on that committee.  Thus
there are $\displaystyle\binom{18}{4}$ ways to build the first committee (to choose $S_1$).  For each
such selection of the first committee, there are 14 employees left for the second committee.  Of these,
six must be chosen.  There are $\displaystyle\binom{14}{6}$ ways to do this (to choose $S_2$, for each choice of $S_1$).
So (by the Generalized Product Rule), there are $\displaystyle\binom{18}{4}\binom{14}{6}$ possible arrangements
(or ordered pairs $\paren{S_1,S_2}$) where Aleph and Beth serve together on the first committee.  Similarly,
there are  $\displaystyle\binom{18}{4}\binom{14}{6}$ possible arrangements (or ordered pairs $\paren{S_1,S_2}$)
where Aleph and Beth serve together on the second committee.  So the number of possible arrangements (or ordered pairs $\paren{S_1,S_2}$) in which
Aleph and Beth serve on the same committee is just, by the Sum Rule,
\[|P'|=2\binom{18}{4}\binom{14}{6}\].
(It should be obvious that the disjointness requirement needed to apply the Sum Rule is met here.)

Let $Q'$ denote the set of all possible arrangements in which Beth, Ferdinand, and Georg would all serve
on the same committee.  By reasoning similar to that used to find $|P'|$, obtain
\[|Q'|=2\binom{17}{3}\binom{14}{6}.\]

$P'\cap Q'$ is the set of all possible arrangements in which Aleph, Beth, Ferdinand, and Georg all serve
on the same committee.  Evidently,
\[|P'\cap Q'|=2\binom{16}{2}\binom{14}{6}.\]

$P'\cup Q'$ is the set of all possible arrangements in which there is at least one unhappy employee.  Applying inclusion/exclusion, have
\begin{align*}
|P' \cup Q'|&= |P'| + |Q'| - |P'\cap Q'| \\
            &= 2\binom{18}{4}\binom{14}{6} + 2\binom{17}{3}\binom{14}{6} - 2\binom{16}{2}\binom{14}{6} \\
            &= 2\paren{\binom{18}{4} + \binom{17}{3} - \binom{16}{2}}\binom{14}{6}
\end{align*}

Let $R'$ denote the set of all possible arrangements in which no employee is unhappy.  Clearly, $D' = R'\cup \paren{P' \cup Q'}$
and $R'\cap \paren{P' \cup Q'} = \emptyset$.  So apply the Sum Rule: $|D'| = |R'| + |P' \cup Q'|$.  Hence:
\begin{align*}
|R'| &= |D'| - |P' \cup Q'| \\
    &= \binom{20}{6}\binom{14}{6} - 2\paren{\binom{18}{4} + \binom{17}{3} - \binom{16}{2}}\binom{14}{6} \\
    &= \paren{\binom{20}{6} - 2\paren{\binom{18}{4} + \binom{17}{3} - \binom{16}{2}}}\binom{14}{6}
\end{align*}

\end{solution}

\end{problemparts}

\end{problem}

%%%%%%%%%%%%%%%%%%%%%%%%%%%%%%%%%%%%%%%%%%%%%%%%%%%%%%%%%%%%%%%%%%%%%
% Problem ends here
%%%%%%%%%%%%%%%%%%%%%%%%%%%%%%%%%%%%%%%%%%%%%%%%%%%%%%%%%%%%%%%%%%%%%
