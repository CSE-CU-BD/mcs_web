\documentclass[problem]{mcs}

\begin{pcomments}
  \pcomment{CP_validity}
  \pcomment{these are first two parts of CP_valid_vs_satisfiable revised ARM, 3/5/11}
  \pcomment{CP_valid_vs_satisfiable from S09.cp2m, but commented out}
\end{pcomments}

\pkeywords{
  valid
  satisfiable
}

%%%%%%%%%%%%%%%%%%%%%%%%%%%%%%%%%%%%%%%%%%%%%%%%%%%%%%%%%%%%%%%%%%%%%
% Problem starts here
%%%%%%%%%%%%%%%%%%%%%%%%%%%%%%%%%%%%%%%%%%%%%%%%%%%%%%%%%%%%%%%%%%%%%

\begin{problem}
\bparts

\ppart Verify by truth table that
\[
(P\ \QIMP\ Q)\ \QOR\ (Q \ \QIMP\ P)
\]
is valid.

\begin{solution}
\[
\begin{array}{cc|c|c|c}
P      & Q          & (P  \QIMP  Q) & \QOR      & (Q  \QIMP  P)\\ \hline
\true  & \true      &      \true    & \lgtrue   &      \true\\
\true  & \false     &      \false   & \lgtrue   &      \true\\
\false & \true      &      \true    & \lgtrue   &      \false\\
\false & \false     &      \true    & \lgtrue   &      \true
\end{array}
\]
\end{solution}

\ppart Let $P$ and $Q$ be propositional formulas.  Describe a single
formula, $R$, using \QAND's, \QOR's, and \QNOT's such that $R$ is valid
iff $P$ and $Q$ are equivalent.

\begin{solution}
\[
R \eqdef (P \QAND  Q) \QOR (\QNOT(P) \QAND \QNOT(Q))
\]

\end{solution}


\iffalse
\ppart\label{sat} A propositional formula is \emph{satisfiable} iff
there is an assignment of truth values to its variables---an
\emph{environment}---which makes it true.  Explain why
\begin{quote}
$P$ is valid iff $\QNOT(P)$ is \emph{not} satisfiable.
\end{quote}

\begin{solution}
To prove the iff, we prove first, that the left hand statement implies the
right hand one, and second, vice-versa.

\textbf{(left-to-right case)}: If $P$ is valid, then $\QNOT(P)$ is \emph{not}
satisfiable.

\begin{proof}
Now $P$ is true in an environment iff $\QNOT(P)$ is false in that
environment.  Since $P$ is valid, it is true in every environment, which
means that $\QNOT(P)$ is false in every environment.  So no environment makes
$\QNOT(P)$ true, which means that $\QNOT(P)$ is \emph{not} satisfiable.
\end{proof}

\textbf{(right-to-left case)}: If $\QNOT(P)$ is \emph{not}
satisfiable, the $P$ is valid.

\begin{proof}
  Since $\QNOT(P)$ is not satisfiable, every truth assignment makes it
  false.  This implies that every truth assignment makes $P$ true, that
  is, $P$ is valid.
\end{proof}

\end{solution}

\ppart A set of propositional formulas $P_1,\dots,P_k$ is
\emph{consistent} iff there is an environment in which they are all
true.  Write a formula, $S$, so that the set $P_1,\dots,P_k$ is \emph{not}
consistent iff $S$ is valid.

\begin{solution}
Note that the set $P_1,\dots,P_k$ is consistent iff
\[
(P_1\ \QAND\ P_2\ \QAND\ \dots\ \QAND\ P_k)
\]
is satisfiable.  So by part~\eqref{sat}
\[
S \eqdef \QNOT(P_1\ \QAND\ P_2\ \QAND\ \dots\ \QAND\ P_k)
\]
is the desired formula.  In more concise notation this would written
\[
\QNOT\paren{\Land_{i=1}^k P_i}
\]
or, using DeMorgan's Law:
\[
\Lor_{i=1}^k \bar{P_i}.
\]
\end{solution}
\fi

\eparts
\end{problem}

%%%%%%%%%%%%%%%%%%%%%%%%%%%%%%%%%%%%%%%%%%%%%%%%%%%%%%%%%%%%%%%%%%%%%
% Problem ends here
%%%%%%%%%%%%%%%%%%%%%%%%%%%%%%%%%%%%%%%%%%%%%%%%%%%%%%%%%%%%%%%%%%%%%

\endinput
