\documentclass[problem]{mcs}

\begin{pcomments}
  \pcomment{FP_robot_state_probability}
  \pcomment{ZDz 12/08/15}
  \pcomment{f15.final}
\end{pcomments}

\pkeywords{
  state_machine
  preserved_invariant
  counting
  probability
}

%%%%%%%%%%%%%%%%%%%%%%%%%%%%%%%%%%%%%%%%%%%%%%%%%%%%%%%%%%%%%%%%%%%%%
% Problem starts here
%%%%%%%%%%%%%%%%%%%%%%%%%%%%%%%%%%%%%%%%%%%%%%%%%%%%%%%%%%%%%%%%%%%%%

\begin{problem}
Our old friend Robot steps between integer positions in 2-dimensional
space.  Each step of Robot increments one coordinate and leaves the
other one unchanged. (Note that he cannot make diagonal steps anymore,
since that was too complicated for him -- yes, we blame Robot for it!).

  We can model Robot as a state machine whose states are 2-D positions
  $(x, y)$ where $x$ and $y$ are integers.

\bparts
  

\ppart
Which of the following predicates are preserved invariants?
Briefly explain your answer.

\begin{align}
x  & = 0 \\
y  & \geq 7 \\
x & > 3  \QAND y > -3 \\
x + 2y  & > 5 \\
|x - y|  &  \leq 5 \\
|x - y|  &  > 5 \\
x \cdot y  & > 10
\end{align}

\examspace[2in]

\begin{solution}
Predicates 2, 3, and 4 are preserved invariants, because $x$ and $y$ never decrease.
Predicate 1 is not, since $x$ may increase.
None of the predicated 5 and 6 are preserved invariants, since the absolute difference
between $x$ and $y$ may change arbitrarily.
Predicate 7 is also not a preserved invariant, since the product of two integers may be
positive when they are both negative, but would then decrease when one of the
numbers increases. E.g., if $x=-3$ and $y=-4$, then $x \cdot y = 12 > 10$, but after
increasing $x$ by one, $x \cdot y = (-2) \cdot (-4) = 8 < 10$.

\end{solution}

\medskip Now \textbf{assume} that Robot starts at position $(0,0)$.


\ppart
What is the set of reachable states after $N$ steps?

\examspace[1in]

\begin{solution}
\[
S = \set{(x,y) \in \integers^2 \,|\, x \geq 0 \QAND y \geq 0 \QAND x+y = N} \, ,
\]
i.e., the set of all pairs of nonnegative integers whose sum equals $N$.

Alternatively, $S = \set{(0, N), (1, N-1), (2, N-2), \ldots, (N, 1)}$.

Alternatively, the states of the form $(x, N-x)$, for $0 \leq x \leq N$.
\end{solution}


\ppart
If Robot chooses makes his steps randomly (i.e., increase either coordinate with probability 0.5 at each step),
what is the probability that after $M+N$ steps he will end up at position $(M, N)$?

\examspace[2in]

\begin{solution}
\[
\frac{\binom{M+N}{M}}{2^{M+N}}
\]
Let $0$ represent a step along $x$ coordinate and $1$ a step along $y$ coordinate.
Each sequence of $M+N$ 0's and 1's is equally likely. There are $2^{M+N}$
possible such sequences, out if which $\binom{M+N}{M}$ lead to the position
$(M,N)$. Thus, the probability of ending up at $(M,N)$ is ${\binom{M+N}{M}}/{2^{M+N}}$. 
\end{solution}


\ppart

Let $f(x, y)$ denote the probability that Robot ends up in state $(x, y)$ after $x+y$ steps.
If Robot ends up at the position $(M, N)$ after $M+N$ steps,
what is the probability that he was at the position $(m, n)$ after $m+n$ steps?
Assume that $0 \leq m \leq M$ and $0 \leq n \leq N$.

\hint{Use Bayes' Theorem.}

\examspace[2in]

\begin{solution}
Let $A_{x,y}$ denote the event that Robot is at position $(x,y)$ after $x+y$ steps.
By Bayes' Theorem,
\begin{align*}
\prcond{A_{m,n}}{A_{M,N}} & = \frac{\pr{A_{m,n} \QAND A_{M, N}}}{\pr{A_{M, N}}} \\
					  & = \frac{\pr{A_{m,n}} \prcond{A_{M, N}}{A_{m,n}}}{\pr{A_{M, N}}} \\
					  & = \frac{f(m,n) f(M-m, N-n)}{f(M, n)} \, .
\end{align*}
\end{solution}


\begin{staffnotes}
The following part is perhaps not for the exam.
\end{staffnotes}
\ppart

If Robot increases $x$ coordinate with probability 0.3 and $y$ coordinate with
probability 0.7 at each step, what is the probability that after $M+N$ steps he will end up at position
$(M, N)$?

\examspace[2in]

\begin{solution}
\[
\binom{M+N}{M} \cdot 0.3^M \cdot 0.7^N
\]
Let $0$ represent a step along $x$ coordinate and $1$ a step along $y$ coordinate.
There are $\binom{M+N}{M}$ sequences of length $M+N$ that lead to the position
$(M,N)$. The probability of each such sequence is $0.3^M \cdot 0.7^N$, since there
are exactly $M$ steps along $x$ and $N$ steps along $y$ coordinate (these choices
are made independently and so their probabilities can be multiplied).
Since these paths are disjoint events, the total probability that Robot ends up at
position $(M,N)$ after $M+N$ steps is is $\binom{M+N}{M} \cdot 0.3^M \cdot 0.7^N$. 
\end{solution}

\eparts

\end{problem}

%%%%%%%%%%%%%%%%%%%%%%%%%%%%%%%%%%%%%%%%%%%%%%%%%%%%%%%%%%%%%%%%%%%%%
% Problem ends here
%%%%%%%%%%%%%%%%%%%%%%%%%%%%%%%%%%%%%%%%%%%%%%%%%%%%%%%%%%%%%%%%%%%%%

\endinput
