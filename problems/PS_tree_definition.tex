\documentclass[problem]{mcs}

\begin{pcomments}
  \pcomment{spring02 pset4-2}
\end{pcomments}

\pkeywords{
 graph theory
}


\begin{problem}
We provide an alternative definition for a tree.

{\em Definition:} A {\em tree} is a connected graph such that
$\forall u,v \in V$, there is a unique path connecting $u$ to $v$.

Prove that this definition is equivalent to the one that defines a tree
to be an acyclic graph of $n$ vertices that has $n-1$ edges.

\begin{solution}
Let $G$ be a graph  of $n$ vertices such that
$\forall u,v \in V$, there is a unique path connecting $u$ to $v$.

First we show that $G$ is acyclic.
Suppose, for the purposes of contradiction, that $G$ has a cycle.
Pick two vertices $u$ and $v$ on the cycle.  Then there are at least
two paths connecting $v$ and $u$, contradicting the uniqueness assumption.
So $G$ is acyclic.

Also we note that the existence of paths between every two vertices
 implies that
$G$ is connected.  So $G$ is a connected acyclic graph, and from class
we know that it has $n-1$ edges.

Conversely, let $G$ be an acyclic graph with $n$ vertices and $n-1$ edges.
We know from class that such a graph is connected so there exists at least 
one path between any two vertices.

Suppose, for the purposes of contradiction, that there are two distinct
paths $C$ and $C'$ connecting the vertices $u$ and $v$. 
Starting at $u$ consider the first vertex $u'$ where $C$ differs 
from $C'$ (such a vertex exists because the graphs are distinct).  
Also consider the first vertex $v'$ where $C$ and $C'$ meet again
(such a vertex exists because both paths end at $v$).  Then there is a
simple cycle that starts at $u'$ follows $C$ until it reaches $v'$ and
returns to $u'$ following $C'$.
This contradicts the acyclicity of the graph.


\end{solution}

\end{problem} 
