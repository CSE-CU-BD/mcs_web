\documentclass[problem]{mcs}

\begin{pcomments}
    \pcomment{TP_divides_n_square_then_n}
    \pcomment{see also TP_divides_n_square_not_n}
    \pcomment{ARM 9/7/13}
\end{pcomments}

\pkeywords{
divisor
square
prime
}

\begin{problem}
Let $n$ be a nonnegative integer.

\bparts

\ppart\label{even_square} Explain why if $n^2$ is even---that is, a multiple of 2---then
$n$ is even.

\begin{solution}
We know that the product of two odd numbers is odd, and the product of
an even number and an odd number is even.  So the claim follows from
by proof by contradiction: if $n$ was odd, then $n \cdot n$ would also
be odd, contradicting the fact that $n^2$ is even.
\end{solution}

\ppart Explain why if $n^2$ is a multiple of 3, then
$n$ must be a multiple of 3.

\begin{solution}
We know that every integer greater than 1 has a \emph{unique}
factorization into primes (see the Fundamental Theorem of Arithmetic,
Section~\ref{fundamental_theorem_sec}).  Since a factorization of
$n^2$ can be obtained from two copies of the unique factorization of
$n$ into primes, it follows that the unique factorization of $n^2$
into primes consists of two copies of the factorization of $n$.  In
particular, if 3 is a prime in the factorization of $n^2$, then 3 must
be a prime in the factorization of $n$, and so 3 must appear a
positive, even number of times in the factorization of $n^2$.

Of course this argument holds for any prime in place of 3.  In
particular, it holds for the prime 2, which provides another
explanation for part~\ref{even_square}.
\end{solution}

\eparts

\end{problem}

\endinput
