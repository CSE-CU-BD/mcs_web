\documentclass[problem]{mcs}

\begin{pcomments}
  \pcomment{FP_demorgan_set_iff}
  \pcomment{F16, midterm1}
  \pcomment{ARM 2/24/16}
  \pcomment{was CP_TBA5}

\end{pcomments}

\pkeywords{
   set
   intersection
   complement
   chain_of_iff
   iff
}

\begin{problem}
The set equation 
\[
\overline{A \intersect B} = \overline{A} \union\overline{B}
\]
follows from a certain equivalence between propositional formulas.

\bparts

\ppart What is the equivalence?

\begin{solution}
DeMorgan's Law
\[
\QNOT(A \QAND B) = \QNOT(A) \QOR \QNOT(B).
\]
\end{solution}

\examspace[0.5in]

\ppart Show how to derive the equation from this equivalence.

\begin{solution}
We will prove the equality by showing that the left-hand and right
hand sets have the same members.  That is, we will prove:
\[
x \in \overline{A \intersect B} \qiff x \in \bar{A} \union \bar{B}.
\]

\begin{proof}
\begin{align*}
\lefteqn{x \in \overline{A \intersect B}}\\
   & \qiff \QNOT(x \in A \intersect\ B)
       & \text{(def of set complement)}\\
  & \qiff \QNOT(x \in A \QAND x \in B)
       & \text{(def of $\intersect$)}\\
  & \qiff \QNOT(x \in A) \QOR \QNOT(x \in B)
       & \text{(DeMorgan's Law)}\\
  & \qiff x \in \bar{A} \QOR x \in \bar{B}
       & \text{(def of set complement)}\\
  & \qiff x \in \bar{A} \union \bar{B}
       & \text{(def of $\union$)}.
\end{align*}
\end{proof}
\end{solution}

\eparts

\end{problem}

\endinput
