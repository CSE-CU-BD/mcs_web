\documentclass[problem]{mcs}

\begin{pcomments}
  \pcomment{MQ_planar_face_length}
  \pcomment{ARM 4/1/11}
\end{pcomments}

\pkeywords{
 planar
 face
 edge
}

%%%%%%%%%%%%%%%%%%%%%%%%%%%%%%%%%%%%%%%%%%%%%%%%%%%%%%%%%%%%%%%%%%%%%
% Problem starts here
%%%%%%%%%%%%%%%%%%%%%%%%%%%%%%%%%%%%%%%%%%%%%%%%%%%%%%%%%%%%%%%%%%%%%

\begin{problem}
\bparts

\ppart Give an example of a planar graph with two planar embeddings,
where the first embedding has a face whose length is not equal to the
length of any face in the secoind embedding.  Draw the two embeddings
to demonstrate this.

\examspace

\begin{solution}
Class problem~\bref{CP_planar_embedding_isomorphism} has one example,
and slides
\href{http://courses.csail.mit.edu/6.042/spring11/slides8w.pdf}{Slides
  8W} has another.
\end{solution}

\ppart Define the length of a planar embedding $\embed{E}$ to be
the sum of the lengths of the faces of $\embed{E}$.  Prove that all
embeddings of the same planar graph have the same length.
\examspace
\begin{solution}
Each edge has exactly two occurrences on the faces of an embedding, so
the length of an embedding is always twice the number of edges in the
graph.
\end{solution}

\eparts

\end{problem}

%%%%%%%%%%%%%%%%%%%%%%%%%%%%%%%%%%%%%%%%%%%%%%%%%%%%%%%%%%%%%%%%%%%%%
% Problem ends here
%%%%%%%%%%%%%%%%%%%%%%%%%%%%%%%%%%%%%%%%%%%%%%%%%%%%%%%%%%%%%%%%%%%%%

\endinput
