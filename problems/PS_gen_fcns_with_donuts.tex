%PS_binomial_problem

\documentclass[problem]{mcs}

\begin{pcomments}
  \pcomment{from: S08.ps10}
\end{pcomments}

\pkeywords{
  generating functions
}

%%%%%%%%%%%%%%%%%%%%%%%%%%%%%%%%%%%%%%%%%%%%%%%%%%%%%%%%%%%%%%%%%%%%%
% Problem starts here
%%%%%%%%%%%%%%%%%%%%%%%%%%%%%%%%%%%%%%%%%%%%%%%%%%%%%%%%%%%%%%%%%%%%%

% new problem (derived mostly from Gen Fxn lec. notes)

\begin{problem} \textbf{The Doughnut Number}

Recall from the lecture notes that the $n$th coefficient of the 
generating function for selecting $n$ items (doughnuts) from a 
$k$-element set (flavors) with repetition allowed is equal to
the number of $n+k-1$-bit sequences with exactly $k-1$ zeros.
In this problem you will be asked to generalize this result to
find the $n$th coefficient of an extremely useful generating 
function.

For nonnegative integers $n$, $k$ and $\alpha$, let $g_n$ be the 
number of length $n+k-1$ sequences of integers between $0$ and 
$\alpha$ (inclusive) that have exactly $k-1$ zeros.

\bparts

\ppart\label{S08_ps10_donut1} Find a closed form for $g_n$. 

\begin{solution}
\[
g_n = \binom{n+k-1}{k-1} \cdot \alpha^n
\]

There are $\binom{n+k-1}{k-1}$ choices for the placement of the
zeros.  Each of the remaining $n$ integers can then take any value 
from the $\alpha$-element set $\{1,2,\ldots,\alpha\}$.
\end{solution}

\ppart\label{S08_ps10_donut2} Find a closed form for the 
generating function corresponding to $\ang{g_0, g_1, g_2, \ldots}$.

\begin{solution}
\[
G(x) \eqdef \frac{1}{(1 - \alpha x)^k}
\]

By simple substitution,
\begin{eqnarray*}
 \sum_{n=0}^{\infty} \binom{n+k-1}{k-1} x^n 
   & \corresp & \frac{1}{(1 - x)^k}\\
 \sum_{n=0}^{\infty} \binom{n+k-1}{k-1} \alpha^n x^n
   & \corresp & \frac{1}{(1 - (\alpha x))^k}
\end{eqnarray*}
\end{solution}

\ppart Use a Taylor expansion to show your answers to parts
$\eqref{S08_ps10_donut1}$ and $\eqref{S08_ps10_donut2}$
agree.

\begin{solution}

By Taylor's Theorem,
\[
G(x) = G(0) + G'(0) x + \frac{G''(0)}{2!} x^2 + \cdots
+ \frac{G^{(n)}(0)}{n!} x^n + \cdots.
\]

\begin{align*}
G'(x)      & = \frac{k \alpha}
                    {(1 - \alpha x)^{k+1}} \\
G''(x)     & = \frac{k (k+1) \alpha^2} 
                    {(1 - \alpha x)^{k+2}} \\
G'''(x)    & = \frac{k (k+1) (k+2) \alpha^3}
                    {(1 - \alpha x)^{k+3}} \\
G^{(n)}(x) & = \frac{k (k+1) \cdots (k + n - 1) \alpha^n} 
                    {(1 - \alpha x)^{k+n}}
\end{align*}

Therefore,
\[
\frac{G^{(n)}(0)}{n!} = \binom{n+k-1}{k-1} \alpha^n 
\]
as claimed.
\end{solution}

\eparts
\end{problem}

%%%%%%%%%%%%%%%%%%%%%%%%%%%%%%%%%%%%%%%%%%%%%%%%%%%%%%%%%%%%%%%%%%%%%
% Problem ends here
%%%%%%%%%%%%%%%%%%%%%%%%%%%%%%%%%%%%%%%%%%%%%%%%%%%%%%%%%%%%%%%%%%%%%

\endinput
