\documentclass[problem]{mcs}

\begin{pcomments}
  \pcomment{CP_nonadjacent_books_counting_sequel}
  \pcomment{best used after CP_nonadjacent_books and
    CP_inequality_string_bijections, otherwise probably needs a hint}
  \pcomment{by ARM 4/5/11}
\end{pcomments}

\pkeywords{
  counting
  counting_rules
  bijection
  bit_string
  integer_solution
  integer_equation
}

%%%%%%%%%%%%%%%%%%%%%%%%%%%%%%%%%%%%%%%%%%%%%%%%%%%%%%%%%%%%%%%%%%%%%
% Problem starts here
%%%%%%%%%%%%%%%%%%%%%%%%%%%%%%%%%%%%%%%%%%%%%%%%%%%%%%%%%%%%%%%%%%%%%

\begin{problem}

\bparts

\ppart There are 30 books arranged in a row on a shelf.  In how many ways
can eight of these books be selected so that there are at least two unselected
books between any two selected books?

\begin{solution}
The answer is the number of length 16 bitstrings with eight \STR{1}'s,
namely
\[
\binom{16}{8}.
\]

As in Problem~\bref{CP_nonadjacent_books}, a selection of eight among
thirty books on a shelf can be represented by a length 30 bit-string
with eight \STR{1}'s indicating the selected book positions.  But there is a
bijection between length 30 bit-strings with eight \STR{1}'s that are at
least two apart and length 16 bit-strings with eight \STR{1}'s, namely,
insert two \STR{0}'s after the first seven \STR{1}'s in a 16 bit-string with eight
\STR{1}'s, to obtain a 30 bit-string with 8 occurrences of \STR{1}'s that are at
least two apart.  That is, map
\[
\STR{0}^{x_1}\STR{1}\STR{0}^{x_2}\STR{1}\STR{0}^{x_3}\dots \STR{1}\STR{0}^{x_8}\STR{1}\STR{0}^{x_9}
\longrightarrow
\STR{0}^{x_1}\STR{1}\STR{0}^{x_2+2}\STR{1}\STR{0}^{x_3+2}\STR{1}\dots \STR{1}\STR{0}^{x_8+2}\STR{1}\STR{0}^{x_9},
\]
where $x_1+x_2+\cdots+x_9 = 16$.

\end{solution}

\ppart\label{countintsolns} How many nonnegative integer solutions are
there for the following equality?
\begin{equation}\label{intsolncountmk}
x_1 + x_2 + \cdots + x_m = k.
\end{equation}

\begin{solution}
There are
\[
\binom{m+k-1}{k}
\]
nonnegative integer solutions to~\eqref{intsolncountmk}.

As in Problem~\bref{CP_inequality_string_bijections}, mapping
$(x_1,x_2,\dots,x_m) \in \naturals^m$ to
$\mathtt{0}^{x_1}\mathtt{1}\mathtt{0}^{x_2}\mathtt{1} \dots
\mathtt{0}^{x_m}$ defines a bijection between solutions
to~\eqref{intsolncountmk} and length $k+m-1$ bit strings with $k$
$\mathtt{0}$'s.
\end{solution}

\ppart How many nonnegative integer solutions are there for the
following inequality?
\begin{equation}\label{intsoln_ineq_countmk}
x_1 + x_2 + \cdots + x_m \leq k.
\end{equation}

\begin{solution}
There are
\[
\binom{m+k}{k}
\]
nonnegative integer solutions to~\eqref{intsoln_ineq_countmk}.

Taking an integer solution to the equality
\[
x_1 + x_2 + \cdots + x_m + x_{m+1} = k,
\]
and ignoring the value of $x_{m+1}$ yields a unique solution to the
inequality~\eqref{intsoln_ineq_countmk}.  This correspondence between
solutions is in fact a bijection, so the answer follows as in
part~\eqref{countintsolns}.
\end{solution}

\ppart How many length $m$ weakly increasing sequences of nonnegative
integers $\leq k$ are there?

\begin{solution}
There are
\[
\binom{m+k}{k}
\]
length $m$ weakly increasing sequences of nonnegative integers $\leq
k$.

As in Problem~\bref{CP_inequality_string_bijections}, there is a
bijection between these weakly increasing sequences and solutions
to~\eqref{intsoln_ineq_countmk}, namely, map a solution $(x_1,x_2,\dots,
x_m)$ to a sequence
$(x_1,\ x_1+x_2,\ x_1+x_2+x_3,\ \dots,\ \sum_{i=1}^m x_i)$.   So the
answer follows from part~\eqref{countintsolns}.

\end{solution}

\eparts

\end{problem}

%%%%%%%%%%%%%%%%%%%%%%%%%%%%%%%%%%%%%%%%%%%%%%%%%%%%%%%%%%%%%%%%%%%%%
% Problem ends here
%%%%%%%%%%%%%%%%%%%%%%%%%%%%%%%%%%%%%%%%%%%%%%%%%%%%%%%%%%%%%%%%%%%%%
