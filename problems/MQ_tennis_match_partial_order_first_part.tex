\documentclass[problem]{mcs}

\begin{pcomments}
  \pcomment{MQ_tennis_match_partial_order}
  \pcomment{from F11, ps6}
  \pcomment{don't use with MQ_reindeer_games_partial_order}
\end{pcomments}

\pkeywords{
  incomparable
  comparable
  maximal
  maximum
  minimal
  minimum
  chain
  linear
  topological_sort
  partial_order
  antichain
  maximal
  irreflexive
  reflexive
}

%%%%%%%%%%%%%%%%%%%%%%%%%%%%%%%%%%%%%%%%%%%%%%%%%%%%%%%%%%%%%%%%%%%%%
% Problem starts here
%%%%%%%%%%%%%%%%%%%%%%%%%%%%%%%%%%%%%%%%%%%%%%%%%%%%%%%%%%%%%%%%%%%%%

\begin{problem}
Tennis tournaments consists of a series of matches between two
players.  Usually the objective is to determine a single best player
among a set of players.  The organizers of the Math for Computer
Science tournament want to do more: they want to find a linear ranking
of all the players.  To avoid controversy, they want to avoid the
awkward situation of having a sequence of players each of whom beats
the next player in the sequence and then having last player beat the
first.  So the organizers will keep a running record of who beat whom
during the tournament and never allow simultaneous matches whose
outcomes could lead to an awkward situation.

Knowledge of binary relations can help the organizers in arranging the
tournament.  Namely, at any stage of the tournament, the organizers
have a record of who lost to whom.  Mathematically, we can say that
there is a binary relation $L$ on players where $p\mrel{L}q$ means
that player $p$ lost a match to player $q$.  No awkward situations
means that the positive length walk relation $L^+$ is a
strict partial order.  Which of the following partial order concepts
correspond to the properties~\eqref{lostall}--\eqref{finalrank} of the
partial order $L^+$?
\begin{center}
Partial Order Concepts
\end{center}
\begin{quote}
comparable \quad incomparable \quad  maximum \quad  maximal \quad minimum\quad minimal\\
a chain \quad an antichain \quad a topological sort \quad reflexive\quad irreflexive\quad asymmetric
\end{quote}

\bparts

\iffalse
\ppart An unbeaten player is a \brule{1in} element of the partial order.
\begin{solution}
maximal
\end{solution}
\fi

\ppart\label{lostall}  A player who has lost every match he was in is a \brule{1in}
element.
\begin{solution}
minimal
\end{solution}

\ppart A player who has beaten all the others is
a \brule{1in} element.
\begin{solution}
maximum
\end{solution}

\ppart Players can be matched in the next stage of the
tournament only if they are \brule{1in} elements.
\begin{solution}
incomparable
\end{solution}

\ppart A set of players whose rankings relative to each other are unique
is \brule{1in}.
\begin{solution}
a chain
\end{solution}

\iffalse
\ppart A set of players any two of whom could be paired up to play
the next match is \brule{1in}.
\begin{solution}
an antichain
\end{solution}
\fi

\ppart The fact that no player loses to himself corresponds to $L^+$
being \brule{1in}.
\begin{solution}
irreflexive
\end{solution}

\ppart\label{finalrank} The final ranking at the end of the tournament will be
\brule{1in}.

\begin{solution}
a topological sort
\end{solution}

\eparts
\end{problem}

\endinput
