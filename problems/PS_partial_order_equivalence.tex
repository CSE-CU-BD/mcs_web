\documentclass[problem]{mcs}

\begin{pcomments}
  \pcomment{PS_partial_order_equivalence}
  \pcomment{rephrasing of FP_equality_relation}
  \pcomment{by ARM 10/18/13}
\end{pcomments}


\pkeywords{
  partial_order
  weak_partial_order  
  equivalence
}

%%%%%%%%%%%%%%%%%%%%%%%%%%%%%%%%%%%%%%%%%%%%%%%%%%%%%%%%%%%%%%%%%%%%%
% Problem starts here
%%%%%%%%%%%%%%%%%%%%%%%%%%%%%%%%%%%%%%%%%%%%%%%%%%%%%%%%%%%%%%%%%%%%%

\begin{problem}
Prove that for any nonempty set $D$, there is a unique binary relation
on $D$ that is both a weak partial order and also an equivalence
relation.

\begin{solution}
The unique relation is the identity relation $\ident{D}$.  The
identity relation is obviously an equivalence relation and is
vacuously a weak partial order (in which no element is comparable to
any other element).

So it is only necessary to show uniqueness, that is, if $R$ is an
equivalence relation and a weak partial order on $D$, then $R =
\ident{D}$.  Since $R$ is obviously reflexive, we need only show that
if $a \mrel{R} b$, then $a = b$.

So suppose $a \mrel{R} b$.  Since $R$ is an equivalence relation, it
is symmetric, and so $b \mrel{R} a$.  Since $R$ is a weak partial
order, it is antisymmetric, so from $a \mrel{R} b$ and $b \mrel{R} a$,
we conclude that $a=b$, as required.
\end{solution}
\end{problem}

%%%%%%%%%%%%%%%%%%%%%%%%%%%%%%%%%%%%%%%%%%%%%%%%%%%%%%%%%%%%%%%%%%%%%
% Problem ends here
%%%%%%%%%%%%%%%%%%%%%%%%%%%%%%%%%%%%%%%%%%%%%%%%%%%%%%%%%%%%%%%%%%%%%

\endinput
