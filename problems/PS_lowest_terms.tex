\documentclass[problem]{mcs}

\begin{pcomments}
  \pcomment{from: S09.ps3}
  \pcomment{revise reference to notes}
\end{pcomments}

\pkeywords{
  WOP
  well-ordering
  rationals
  contradiction
  false_proof
}

%%%%%%%%%%%%%%%%%%%%%%%%%%%%%%%%%%%%%%%%%%%%%%%%%%%%%%%%%%%%%%%%%%%%%
% Problem starts here
%%%%%%%%%%%%%%%%%%%%%%%%%%%%%%%%%%%%%%%%%%%%%%%%%%%%%%%%%%%%%%%%%%%%%

\begin{problem}
There was a proof in the notes that used Well Ordering to
show that all positive rational numbers can be written in ``lowest terms,''
that is, as a ratio of positive integers with no common factor prime
factor.  Below is a different proof which also arrives at this correct
conclusion, but this proof is bogus.   Identify every step at which the
proof makes an unjustified inference.

\begin{bogusproof}
  Suppose to the contrary that there was positive rational, $q$, such that
  $q$ cannot be written in lowest terms.  Now let $C$ be the set of such
  rational numbers that cannot be written in lowest terms.  Then $q \in
  C$, so $C$ is nonempty.  So there must be a smallest rational, $q_0 \in
  C$.  So since $q_0/2 < q_0$, it must be possible to express $q_0/2$ in
  lowest terms, namely,
\begin{equation}\label{q02}
\frac{q_0}{2} = \frac{m}{n}
\end{equation}
for positive integers $m,n$ with no common prime factor.  Now we consider
two cases:

\textbf{Case 1:} [$n$ is odd].  Then $2m$ and $n$ also have no common prime
factor, and therefore
\[
q_0 = 2\cdot \paren{\frac{m}{n}} = \frac{2m}{n}
\]
expresses $q_0$ in lowest terms, a contradiction.

\textbf{Case 2:} [$n$ is even].  Any common prime factor of $m$ and $n/2$
would also be a common prime factor of $m$ and $n$.  Therefore $m$ and
$n/2$ have no common prime factor, and so
\[
q_0 = \frac{m}{n/2}
\]
expresses $q_0$ in lowest terms, a contradiction.

Since the assumption that $C$ is nonempty leads to a contradiction, it
follows that $C$ is empty ---that is, there are no counterexamples.
\end{bogusproof}

\solution{The proof applies Well Ordering to the positive rationals.
  Unfortunately, the positive rationals are not Well Ordered, thatis, $<$
  is not well-founded on the positive rationals.  For example, there is no
  least positive rational.  Aside from that, the other steps in the
  argument are correctly reasoned.}

\end{problem}

%%%%%%%%%%%%%%%%%%%%%%%%%%%%%%%%%%%%%%%%%%%%%%%%%%%%%%%%%%%%%%%%%%%%%
% Problem ends here
%%%%%%%%%%%%%%%%%%%%%%%%%%%%%%%%%%%%%%%%%%%%%%%%%%%%%%%%%%%%%%%%%%%%%
