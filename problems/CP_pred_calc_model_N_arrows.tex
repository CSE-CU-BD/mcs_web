\documentclass[problem]{mcs}

\begin{pcomments}
  \pcomment{CP_pred_calc_model_N_arrows}
  \pcomment{variant of CP_pred_calc_model_N}
  \pcomment{ARM 2/17/16}
\end{pcomments}

\pkeywords{
  predicate
  logic
  nonnegative
  model
  arrow
}

\newcommand{\zerosym}{\textbf{\~0}}
\newcommand{\nextsym}{\textbf{next}}
\newcommand{\nextfun}{\textit{nextf}}
\newcommand{\prevsym}{\textbf{prev}}
\newcommand{\prevfun}{\textit{prevf}}
\newcommand{\numerl}[1]{\widetilde{\text{#1}}}
\newcommand{\nextarrow}{\stackrel{\nextsym}{\longrightarrow}}
\newcommand{\prevarrow}{\stackrel{\prevsym}{\longleftarrow}}
\newcommand{\nparrow}{\stackrel{\nextarrow}{\prevarrow}}
\newcommand{\ltsym}{\mrel{\textbf{less}}}

%%%%%%%%%%%%%%%%%%%%%%%%%%%%%%%%%%%%%%%%%%%%%%%%%%%%%%%%%%%%%%%%%%%%%
% Problem starts here
%%%%%%%%%%%%%%%%%%%%%%%%%%%%%%%%%%%%%%%%%%%%%%%%%%%%%%%%%%%%%%%%%%%%%

\begin{problem}
Let $\zerosym$ be a constant symbol, $\nextsym$ and $\prevsym$ be
function symbols taking one argument.

We can picture any model for these symbols by representing domain
elements as points.  The model must interpret $\zerosym$ as some point
$e_0$.  It also interprets the symbols $\nextsym$ and $\prevsym$ as
total functions $\nextfun$ and $\prevfun$ on the domain.  We can
picture the functions by having an arrow labelled $\nextsym$ go out
from each point $e$ into the point $\nextfun(e)$, and an arrow
labelled $\prevsym$ go out from each $e$ into $\prevfun(e)$.  In
particular,

\emph{Every point has exactly one $\nextsym$-arrow, and exactly one
  $\prevsym$-arrow, going out of it.}

The aim of this problem is to develop a series of predicate formulas
using just the symbols $\zerosym$, $\nextsym$ and $\prevsym$ such that
every model satisfying these formulas will contain contain a copy of
the nonnegative integers $\nngint$, with $\nextfun$ acting on the copy
as the ``$+1$'' or \emph{successor function} and $\prevfun$ acting as
the ``$-1$'' or \emph{predecessor function}.

More precisely, the ``copy'' of $\nngint$ in a model will look like an
infinite sequence of distinct points starting with $e_0$, with a
$\nextsym$-arrow going from each point to the next in the sequence:
\[
e_0 \nextarrow e_1 \nextarrow e_2 \nextarrow  \dots \nextarrow e_n \nextarrow \dots
\]
so that $\nextfun$ acts like plus one.  We also want $\prevfun$ to act
like minus one: whenever a $\nextsym$-arrow goes into an element $e$,
then the $\prevsym$-arrow out of $e$ goes back to the beginning of the
$\nextsym$-arrow:
\begin{equation}\label{Ncopyshape}
e_0 \nparrow e_1 \nparrow e_2 \nparrow  \dots \nparrow e_n \nparrow \dots.
\end{equation}
Not shown is a further $\prevsym$-arrow self-loop from $e_0$ to $e_0$,
reflecting the convention for nonnegative integers that subtracting
from zero has no effect.

There is a simple way to express this requirement as a predicate
formula:
\begin{equation}\label{previnvnext}
\forall x.\, \prevsym(\nextsym(x)) = x\ .
\end{equation}
Formula~\eqref{previnvnext} means that the $\prevsym$-arrow out of any
point $e$ goes back to the beginning of any $\nextsym$-arrow into $e$.
Of course this will not be possible if there is more than one
$\nextsym$-arrow into $e$.

\begin{staffnotes}
\textbf{Remark:}

The formula~\eqref{previnvnext} implies that at most one
$\nextsym$-arrow can go into any point $e$.  Another way to say this
is that $\nextsym$-arrows that go out from different points have to go
into different points:
\begin{equation}\label{next1to1}
\forall x,y.\, \nextsym(x) = \nextsym(y)\ \QIMPLIES\ x = y.
\end{equation}

So if a model satisfies~\eqref{previnvnext}, then it will
satisfy~\eqref{next1to1}.  In other words,
\[
\text{\eqref{previnvnext}} \QIMPLIES\ \text{\eqref{next1to1}}
\]
is a \emph{valid formula}.
\end{staffnotes}

There are some standard terms called \emph{numerals} used to describe
the elements $e_0, e_1,\dots$ in \eqref{Ncopyshape}.  Namely, the
numeral for $e_0$ will be $\zerosym$, and the numeral for $e_n$ will
be the application $n$ times of $\nextsym$ to $\zerosym$.  For example, 
the numeral for three would be:
\begin{equation}\label{3numer}
\nextsym(\nextsym(\nextsym(\zerosym))).
\end{equation}
We'll refer to the numeral for $e_n$ as $\numerl{n}$.  So $\numerl{3}$
refers to the term~\eqref{3numer}.

But we don't quite have the desired situation pictured
in~\eqref{Ncopyshape}: there is nothing so far that prevents all the
numerals having the same value.  In other words, the formulas above
are consistent with the formula $\nextsym(\zerosym) = \zerosym$.
(You should check this yourself right now.)  We might try to fix this
by requiring that no $\nextsym$-arrow can begin and end at the same
element, but there still could be a pair of numeral values, each of which
was $\nextfun$ of the other.  That is, the model might satisfy
\textcolor{red}{
\[
\numerl{2} \eqdef \nextsym(\nextsym(\zerosym)) = \zerosym.
\]
}
We could go on to forbid such length two cycles of $\nextsym$-arrows,
but then there might be a cycle of three:
\iffalse
\[
\exists x, y,z.\, \nextsym(x) = y \QAND\ \nextsym(y) = z \QAND\ \nextsym(z) = x\ ,
\]
\fi

\textcolor{red}{
\[
\numerl{3} \eqdef \nextsym(\nextsym(\nextsym(\zerosym))) = \zerosym,
\]}
and so on.  Fortunately, something we want to do anyway will
fix this potential problem with the numeral values: we haven't yet
provided a formula that will make the interpretion $e_0$ of
$\zerosym$ behave like zero.

\bparts

\ppart\label{zero-part} Write a predicate formula expressing the fact
that the value of $\zerosym$ is not plus-one of anything.  Also, by
convention, subtracting one from zero has no effect.

\begin{solution}
\[
\QNOT(\exists x.\, \nextsym(x) = \zerosym)\ \QAND\ \prevsym(\zerosym) = \zerosym.
\]
\end{solution}
\eparts

The formulas of part~\eqref{zero-part} and~\eqref{previnvnext}
together imply that the numeral values will yield the
copy~\eqref{Ncopyshape} of $\nngint$ we want.  To verify this, we
just need to show that the values of all the numerals are distinct.

\bparts

\ppart Explain why two different numerals must have different values
in any model satisfying~\eqref{previnvnext} and
part~\eqref{zero-part}.

\hint It helps to describe the meaning of~\eqref{previnvnext} by what
it says about arrows.  There is also an explanation based on the Well
Ordering Principle.

\begin{solution}
There are a couple of ways to confirm this claim, one using arrows and
the other well-ordering and the given formulas.

An arrow explanation goes as follows:
\begin{proof}
Suppose we start at the point denoted by $\zerosym$ and proceed to
generate a sequence of points by following successive $\nextsym$-arrows
out of each point and into the next.  These points are, by definition,
the values of all the numerals.  But the formula of
part~\eqref{zero-part} implies that no $\nextsym$-arrow goes into the
starting point.  Also, no $\nextsym$-arrow between points in the sequence
can go into any other earlier point in the sequence, because all the
earlier nonzero points already a $\nextsym$-arrow going into it.  So each
$\nextsym$-arrow goes into a new point and therefore no point has a
repeat occurrence.
\end{proof}

Here is an alternate explanation based on the Well Ordering Principle.
\begin{proof}
Suppose to the contrary that two different numerals had the same value
in some interpretation satisfying~\eqref{previnvnext} and
part~\eqref{zero-part}.  Then there is a minimum $n \in \nngint$
such that $\numerl{n}$ has the same value as numeral $\numerl{m}$ for
some $n < m$.

But $n \neq 0$ because by part~\eqref{zero-part}, the interpretation
of $\zerosym$ is not $\nextfun$ of anything, while all the other
numerals are by definition $\nextfun$ of the previous numeral.

Now with $n,m>0$ the interpretation
of the numerals will satisfy
\begin{align}
\prevsym(\numerl{n}) & \eqdef \prevsym(\nextsym(\numerl{n-1})) = \numerl{n-1}
& \text{(by~\eqref{previnvnext})},\label{nneqn-1}\\
\prevsym(\numerl{m}) & \eqdef \prevsym(\nextsym(\numerl{m-1})) = \numerl{m-1},
& \text{(by~\eqref{previnvnext})},\label{mmeqm-1}
\end{align}
So by~\eqref{nneqn-1},~\eqref{mmeqm-1}, the assumption that
$\numerl{n} = \numerl{m}$ is true in the model implies
\[
 \numerl{n-1} = \numerl{m-1}
\]
is also true, contradicting the minimality of $n$.

So the assumption leads to a contradiction, and we conclude that in
every model satisfying~\eqref{previnvnext} and part~\eqref{zero-part},
the numerals all have different values.

\iffalse
\footnote{A careful look
at this argument reveals that for each $n\neq m \in \nngint$,
there is a direct proof that
  \[
  \QNOT(\numerl{n} = \numerl{m}
   \]
using part~\eqref{zero-part} once and~\eqref{previnvnext} $n$ times
\emph{without} appeal to Well Ordering.  This observation leads to the
conclusion that provability even in a very limited proof system with
just a few axioms and elementary logical rules, without Well Ordering
or Induction, is undecidable.}\fi
\end{proof}
\end{solution}

\eparts

So we have verified that any model satisfying~\eqref{previnvnext} and
the formula of part~\eqref{zero-part} has the desired copy of
$\nngint$.

\begin{staffnotes}
FYI; will be inserted into text at some point:

This helps explain why verifying valid formulas of predicate calculus
is so hard.  The explanation begins with the observation that lots of
famous mathematical problems---Fermat's Last Theorem, the Goldbach
Conjecture or the Twin Primes Conjecture, for example---can easily be
expressed in predicate logic using a symbols $\nextsym,\prevsym$ along
with symbols for plus and times (see Problem
\bref{CP_express_goldbach_twinprime_in_logic}).  So it is not
surprising that, in general, it is very hard\footnote{In
  Chapter~\bref{halting_sec} we'll explain why it is actually
  impossible to verify truth of arithmetic formulas.} to determine
whether a predicate formula expressing properties of plus and times on
the nonnegative integers is \emph{true}.

But if we use formula~\eqref{previnvnext} and the formula of
part~\eqref{zero-part} as axioms that force a model to have a copy of
$\nngint$, and we throw in a few more axioms describing some defining
properties of plus and times, we can conclude that to determine if
Fermat's Last Theorem or Goldbach's Conjecture is a \emph{true} when
the domain of discourse is $\nngint$, all we have to do is determine
if Goldbach's Conjecture is a \emph{valid} consequence of the axioms.
So if we accept that determining truth is generally very hard, then
determining if a predicate formula is valid is also going to be very
hard.
\end{staffnotes}

The distinction between formulas that are \emph{true} over the domain
$\nngint$ and formulas that are valid, that is true over \emph{all
  domains} can be confusing on first sight.  To highlight the
distinction, it is worth seeing that the formula of
part~\eqref{zero-part} together with~\eqref{previnvnext} do \emph{not}
ensure that their models consist \emph{solely} of a copy of $\nngint$.

For example, a model might consist of two separate copies of
$\nngint$.  We can stifle this particular possibility of extra copies
of $\nngint$ pretty easily.  Any such copy has to start with a
zero-like element, namely one that is not in the range of $\nextfun$.
So we just assert that there is the only zero-like element.

\bparts

\ppart\label{uniquezero-part} Write a formula such that any model has
only one copy of $\nngint$.

\begin{solution}
\begin{equation}\label{uniquezero}
\forall x.\, \QNOT(\exists y.\, \nextsym(y) = x)\ \QIMPLIES\ x = \zerosym.
\end{equation}
\end{solution}

\eparts

But the additional axiom of part~\eqref{uniquezero-part} is still leaves
room for models that, besides having a copy of $\nngint$, also have
``extra stuff'' with weird properties.

\bparts

\ppart Describe a model that satisfies the formula of
part~\eqref{uniquezero-part} along with~\eqref{previnvnext},
part~\eqref{zero-part} and also satisfies
\begin{equation}\label{x=nx}
\exists x.\, x =\nextsym(x).
\end{equation}

\begin{solution}
\begin{staffnotes}
More explanation would be helpful.
\end{staffnotes}
$\nngint \union \set{\infty}$ will serve.
\end{solution}

\eparts
 
\begin{center}
  \textbf{Supplemental Part}
\end{center}

\bparts

\ppart Prove that
\begin{equation}\label{nextinvprev}
  \forall x \neq \zerosym.\ \nextsym(\prevsym(x)) = x.
\end{equation}

\begin{solution}
We know that every element $e \neq e_0$ equals
$\nextfun(e')$ for some $e'$.  So
\begin{align*}
\nextfun(\prevfun(e))
   & = \nextfun(\prevfun(\nextfun(e')))
        & \text{(def of $e'$)}\\
   & = \nextfun(e')
        & \text{(by~\eqref{previnvnext})}\\  
   & = e.
        & \text{(def of $e'$)}.
\end{align*}
\end{solution}

\eparts

\iffalse

The above formulas serve to control the properties of the numeral
values that make up our copy~\eqref{Ncopyshape} of $\nngint$, but
they do not do much to ensure that all the elements in the domain have
properties like the numeral values.

For example, there might still be a $\nextsym$ cycle of non-numeral
values, that is,
\begin{equation}\label{n3x=x}
\exists x.\, \nextsym(\nextsym(\nextsym(x))) = x.
\end{equation}
might still be satisfiable.

We can forbid models from having finite cycles of $\nextsym$-arrows
like~\eqref{x=nx} or~\eqref{n3x=x} if we introduce another symbol
into formulas, namely, the binary relation symbol $\ltsym$.  Our
objective will be to force $\ltsym$ to act like the less-than
relation $<$ on numbers.  For this purpose, it's suggestive to write
$x \ltsym y$ instead of $\textbf{ltsym}(x,y)$.

One requirement will be that $\ltsym$ behave as expected in the copy
of $\nngint$.  Namely, for all $n,m \in \nngint$, if $n < m$, then
$\numerl{n}\ltsym \numerl{m}$ must be true.  But we also want $\ltsym$
to define a relation that has the ``less-than'' properties on all
domain elements.  One key property is called \emph{transitivity}
\[
\forall x,y,z.\, (x \ltsym y \QAND y\ltsym z) \QIMPLIES x\ltsym z.
\]

\bparts

\ppart Another key property is that the $\ltsym$ relation interacts
properly with $\nextsym$, namely, that if $y$ can be obtained by
repeatedly adding one to $x$, then $x \ltsym y$.  Write a formula that
ensures $\ltsym$ has this property.

\begin{solution}
It's enough to say that $\ltsym$ respects adding one only once,
because if doing it once is OK, transitivity will imply that repeated
adding will be OK.  So we just write:
\begin{equation}\label{Lnext}
\forall x.\ x \ltsym \nextsym(x),
\end{equation}
along with
\[
\forall x.\, \QNOT(x \ltsym \zerosym.)
\]
\end{solution}

\eparts

Now we see that if an element is in a cycle of $\nextsym$-arrows, then
by transitivity, it will be $\ltsym$ itself.  By forbidding this, we
also forbid $\nextsym$-arrow cycles.

\bparts

\ppart Write a formula that is unsatisfiable in an model with a cycle
of $\nextsym$-arrows.

\begin{solution}
By the preceding discussion, The following formula will do the job:
\begin{equation}\label{irrform}
\forall x.\, \QNOT(x\ltsym x)
\end{equation}
\end{solution}

\ppart The final key property is \emph{trichotomy}: every given
element is either bigger or smaller than each other element.  Write a
formula that expresses trichotomy.

\begin{solution}
\begin{equation}\label{trichot}
\forall x,y.\, x\ltsym y\ \QOR\ y \ltsym x\ \QOR\ x = y.
\end{equation}
\end{solution}

\eparts

At this point, the only models that satisfy all the above formulas
must have a single copy of $\nngint$ along with some two-way
infinite chains
\[
\dots \nextarrow c_{-2} \nextarrow c_{-1} \nextarrow c_0 \nextarrow
c_1 \nextarrow c_2 \nextarrow \dots.
\]

\bparts \ppart Verify that the set $\reals^{\geq 0}$ of nonnegative
real numbers is a model satisfying all the above formulas.

\begin{solution}
Easy to check formula by formula.
\end{solution}

\begin{staffnotes}
Might not want stuff between $x$ and $\nextfun(x)$:
\[
\forall x, y.\, \QNOT(x \ltsym y \ltsym \nextsym(x)).
\]

In this case, the trichotomy property,~\eqref{trichot} ensures all
these chains will have the property that every element of one chain
will be $\ltsym$ of every member of the other chain.  So the chains
themselves are ``ordered'' by $\ltsym$.

There seems to be nothing further that can be said of how chains are
ordered: there might be none, some finite number, or even an infinite
number of them.  And the chains might be ordered among themselves like
the integers, nonnegative integers, real numbers,\dots.
\end{staffnotes}

\eparts
\fi

\end{problem}

%%%%%%%%%%%%%%%%%%%%%%%%%%%%%%%%%%%%%%%%%%%%%%%%%%%%%%%%%%%%%%%%%%%%%
% Problem ends here
%%%%%%%%%%%%%%%%%%%%%%%%%%%%%%%%%%%%%%%%%%%%%%%%%%%%%%%%%%%%%%%%%%%%%

\endinput

\iffalse
\[
\zerosym, \nextsym(\zerosym), \nextsym(\nextsym(\zerosym)),
\nextsym(\nextsym(\nextsym(\zerosym))), dots.
\]
\fi 

\iffalse
The values that a model gives to these numerals will be called its
\emph{numeral values}.  We will use the numeral values to be the
elements $e_0,e_1,\dots$ in the copy~\eqref{Ncopyshape} of $\nngint$
we are trying to create, with $e_n$ being the value of $\numerl{n}$.
\fi

\iffalse
\ppart Write a formula expressing the fact that subtracting one from
two different nonzero elements will result in two different elements.

I was going to add ``Explain why this formula is implied the earlier
ones,'' but no time to check and problem long enough already.

\begin{solution}
\begin{equation}\label{prev1to1}
\forall x,y \neq \zerosym.\, x \neq y \QIMPLIES\ \prevsym(x) \neq \prevsym(y)\ .
\end{equation}
\end{solution}
\fi

\iffalse  CHECK THIS:
\bparts

\ppart Explain why the formulas above imply 
\begin{equation}\label{Lprev}
\forall x\neq \zerosym.\ \prevsym(x)\,L\,x\ .
\end{equation}

\begin{solution}
If $x\,L\,\nextsym(x)$ and $x \neq \zerosym$, then
\end{solution}
\eparts
\fi

\iffalse
\ppart Explain why every such model has exactly one copy of
$\nngint$.
\begin{solution}
A copy must start with a point with no $\nextsym$-arrow coming in, and
only $e_0$ has this property.
\end{solution}
\fi

\iffalse
 Write down formulas of predicate calculus with only the symbols
above whose models must have properties as much like $\nngint$ as
you can manage.  In particular, make sure that all models of your
formulas force $\ltsym$ to mean $<$ on the numeral values and ensure
that~\eqref{irrform} implies there are no finite cycles of
$\nextsym$'s.  (If you think you have formulas whose only models are
exactly like $\nngint$, look again, because that is impossible.)

All the elements in one of these chains must be $ltsym$ all the
numeral values.
, but there is nothing further that can be said
of how chains are ordered: there might be none, some finite number, or
even an infinite number of them.  And the chains might be ordered
among themselves like the integers, or nonnegative integers, or real
numbers,\dots.
\fi
