\documentclass[problem]{mcs}

\begin{pcomments}
  \pcomment{TP_divisibility_partial_order}
  \pcomment{renamed from CP}
  \pcomment{9/25/09 from partial order notes problem, edited by ARM}
\end{pcomments}

\pkeywords{
 partial_order
 divisibility
 antisymmetry
 antisymmetric
 reflexive
 transitive
}

%%%%%%%%%%%%%%%%%%%%%%%%%%%%%%%%%%%%%%%%%%%%%%%%%%%%%%%%%%%%%%%%%%%%%
% Problems start here
%%%%%%%%%%%%%%%%%%%%%%%%%%%%%%%%%%%%%%%%%%%%%%%%%%%%%%%%%%%%%%%%%%%%%

\begin{problem}
\bparts

\ppart
Verify that the divisibility relation on the set of nonnegative integers is
a weak partial order.

\begin{solution}
Divisibility is reflexive since $n \divides n$.

It is transitive by Lemma~\bref{lem:div}.\bref{lem:divtrans}.

It is anti-symmetric since if $n \divides m$, then $n \leq m$ for all
positive integers $m$ and nonnegative $n$.  So if $n \divides m$ and
$m \divides n$, then $m \le n$ and $n \le m$, that is, $n=m$.  Also,
if $n \divides 0$ then $n=0$, which confirms anti-symmetry when
$m=-0$.

\end{solution}

\ppart What about the divisibility relation on the set of integers?

\begin{solution}
Divisibility is not antisymmetric on the integers, since $n \divides -n$.
\end{solution}

\eparts

\end{problem}


%%%%%%%%%%%%%%%%%%%%%%%%%%%%%%%%%%%%%%%%%%%%%%%%%%%%%%%%%%%%%%%%%%%%%
% Problems end here
%%%%%%%%%%%%%%%%%%%%%%%%%%%%%%%%%%%%%%%%%%%%%%%%%%%%%%%%%%%%%%%%%%%%%

\endinput
