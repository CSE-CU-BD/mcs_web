\documentclass[problem]{mcs}

\begin{pcomments}
  \pcomment{PS_shuffle_any_5_digit_number}
  \pcomment{from: S09.ps7}
  \pcomment{from: S08}
\end{pcomments}

\pkeywords{
  number_theory
  permutations
  modular_arithmetic
}

%%%%%%%%%%%%%%%%%%%%%%%%%%%%%%%%%%%%%%%%%%%%%%%%%%%%%%%%%%%%%%%%%%%%%
% Problem starts here
%%%%%%%%%%%%%%%%%%%%%%%%%%%%%%%%%%%%%%%%%%%%%%%%%%%%%%%%%%%%%%%%%%%%%

\begin{problem}
Albert decides to entertain the class with a magic trick.  He says:
\begin{enumerate}
\item Pick any 5 digit number containing at least two different digits.
\item Shuffle the digits to obtain a different number.
\item Subtract the smaller number from the larger.
\item Now sum the digits of the result.
\item Repeat step 4 until you have only one digit, and write down your answer.
\end{enumerate}
He announces the right answer without seeing the paper, but the Math
for Computer Science students are not impressed.  This problem
demonstrates why they were not impressed with Albert's "magic."

\bparts

\ppart\label{shuffle} Show that taking \emph{any} nonnegative integer (not
necessarily a 5-digit number), rearranging its digits to form a new
number, and finding the difference between the two numbers, will always
result in a multiple of 9.

\begin{solution}
Let $n$ be the nonnegative integer, and let $m$ be the number
obtained after rearrangement of the digits of $n$.  We want to show that $n-m$
is divisible by 9.

To this end, let
\begin{align*}
n & =  10^0 \cdot a_0 + 10^1 \cdot a_1 + \ldots + 10^k \cdot a_k\\
m & = 10^{\pi(0)} \cdot a_0 + 10^{\pi(1)} \cdot a_1 + \ldots + 10^{\pi(k)} \cdot a_k
\end{align*}
where $\pi$ is a permutation mapping the $k$ digits of $n$ to $k$ digits
of $m$.  So,
\[
n-m = \sum_i a_i \paren{10^i - 10^{\pi(i)}}.
\]
Since $10^k \equiv 1^k = 1 \pmod{9}$ for any $k$, we have that $10^i -
10^{\pi(i)} \equiv 0 \pmod{9}$.  That is, each term, $a_i
\paren{10^i-10^{\pi(i)}}$, in the sum is equivalent $0 \pmod{9}$, and so
the whole sum is also $\equiv 0 \pmod{9}$.  That is,
\[
n-m \equiv 0 \pmod{9},
\]
which means that $n-m$ is divisible by 9.
\end{solution}

\ppart\label{congruent-sum} Show that summing the digits of a positive integer results in an
integer that is congruent to it modulo 9.

\begin{solution}
Let $n = 10^0 \cdot a_0 + 10^1 \cdot a_1 + \ldots + 10^k \cdot
a_k$.  We want to show that:
\[
\sum_i a_i \equiv \sum_i 10^i a_i \pmod{9}.
\]
Since $10 \equiv 1 \pmod{9}$, we have that $10^i \equiv 1^i = 1 \pmod{9}$.
Therefore, $10^i a_i \equiv a_i \pmod{9}$, and we have our desired result:
$\sum_i a_i \equiv \sum_i 10^i a_i \pmod{9}$.
\end{solution}

\ppart Show that for any 5 digit number, this procedure always terminates
with the same digit.  What would happen if the starting number had more
than 5 digits?

\begin{solution}
After completing steps 1-3, we have arrived at a number that is
divisible by 9, as shown in part~\eqref{shuffle} of this problem.
Moreover, this number will be positive, since the original and shuffled
numbers are different.  Then, by part~\eqref{congruent-sum}, summing the
digits of this number will still yield a positive number divisible by 9.
Furthermore, summing the digits of a number results in a smaller positive
number, so the procedure is guaranteed to terminate with the number 9.

All of the previous reasoning holds no matter how many digits the original
number had.
\end{solution}

\eparts

\end{problem}

%%%%%%%%%%%%%%%%%%%%%%%%%%%%%%%%%%%%%%%%%%%%%%%%%%%%%%%%%%%%%%%%%%%%%
% Problem ends here
%%%%%%%%%%%%%%%%%%%%%%%%%%%%%%%%%%%%%%%%%%%%%%%%%%%%%%%%%%%%%%%%%%%%%
