\documentclass[problem]{mcs}

\begin{pcomments}
  \pcomment{PS_cases_sum_of_squares}
  \pcomment{F16.ps1 and earlier even-year Fall terms}
\end{pcomments}

\pkeywords{
  cases
  sum
  square
  even
}

\begin{problem}

\bparts

\ppart\label{abcd-even} Suppose that
\[
a+b+c =d,
\]
where $a,b,c,d$ are nonnegative integers.

Let $P$ be the assertion that $d$ is even.  Let $W$ be the assertion
that exactly one among $a,b,c$ are even, and let $T$ be the assertion
that all three are even.

Prove by cases that
\[
P \QIFF [W \QOR T].
\]

\begin{solution}
\inductioncase{Case 1}: (None of $a,b,c$ are even).  Since neither $W$
nor $T$ is true in this case, the right hand side of the \QIFF\ is
false.  But the sum of an odd number of odd numbers---that is the
product of two odd numbers---is odd, so the sum $d$ is odd.  So $P$ is
also false.  Since both the left and right hand sides of the
\QIFF\ are false, the \QIFF\ assertion itself is true.

\inductioncase{Case 2}:. (Exactly one among $a,b,c$ is even).  Now $W$
is true, which means that the right hand side of the \QIFF\ is true.
But the sum of an even number of odd numbers is even, and the sum of
even numbers is also even, and this means the sum $d$ is even.  So $P$
is also true.  Since both the left and right hand sides of the
\QIFF\ are true, the \QIFF\ assertion itself is true.

\inductioncase{Case 3}: (Exactly two among $a,b,c$ are even). Since
neither $W$ nor $T$ is true in this case, the right hand side of the
\QIFF\ is false.  The sum of the two even numbers is even, and the sum
of an even and an odd number is odd, so the sum $d$ is odd.  So $P$ is
also false.  Since both the left and right hand sides of the
\QIFF\ are false, the \QIFF\ assertion itself is true.

\inductioncase{Case 4}: (All three of $a,b,c$ are even).  Now $T$ is
true, which means that the right hand side of the \QIFF\ is true.
Also, a sum of even numbers is even, so $d$ is even.  That is, $P$ is
also true.  Since both the left and right hand sides of the \QIFF\ are
true, the \QIFF\ assertion itself is true.
\end{solution}

\ppart Now suppose that 
\[
w^2 + x^2 + y^2 = z^2,
\]
where $w,x,y,z$ are nonnegative integers.  Let $P$ be the assertion
that $z$ is even, and let $R$ be the assertion that all three of $w,
x, y$ are even.  Prove by cases that
\[
P\ \QIFF\ R.
\]
\hint An odd number equals $2m+1$ for some integer $m$, so its square
equals $4(m^2 + m) + 1$.

\begin{solution}
The cases will be the same as in part~\eqref{abcd-even}, using the
squares of $w,x,y,z$ for $a,b,c,d$.

Since a number $n$ is even iff $n^2$ is even, part~\eqref{abcd-even}
implies that the \QIFF\ is true in Cases 1,3 and 4.  So we need to
show that it holds in Case 2.  In Case 2, $w,x,y$ are not all even, so
the right side $R$ is false.  So we have to show that in Case 2, the
left hand side $P$ is also false, namely, that $z$ is not even:

\begin{proof}
It's safe to assume in Case 2 that $w,x$ are the two odd numbers, and
$y$ is the even one.  So $w=2m+1,x=2n+1$ and $y = 2p$ for some
nonnegative integes $m,n,p$.  Now $z^2$ is the sum
\[
(2i + 1)^2 + (2j + 1)^2 + (2k)^2,
\]
which in turn equals
\[
4(i^2 + i + j^2 + j) + 2.
\]
But the square of any even number is a multiple of $4$.  Since $z^2$
leaves a remainder of 2 on division by 4, it is not of multiple
of 4, and therefore must not be even.
\end{proof}

\end{solution}
\eparts

\end{problem}

\endinput
