\documentclass[problem]{mcs}

\begin{pcomments}
  \pcomment{PS_cases_sum_of_squares}
  \pcomment{F16.ps1 and earlier even-year Fall terms}
  \pcomment{edited by ARM and fncv 2/5/17}
\end{pcomments}

\pkeywords{
  cases
  sum
  square
  even
}

\begin{problem}

\bparts

\ppart\label{abcd-even} Suppose that
\[
a+b+c =d,
\]
where $a,b,c,d$ are nonnegative integers.

Let $P$ be the assertion that $d$ is even.  Let $W$ be the assertion
that exactly one among $a,b,c$ are even, and let $T$ be the assertion
that all three are even.

Prove by cases that
\[
P \QIFF [W \QOR T].
\]

\begin{solution}
\inductioncase{Case 1}: (None of $a,b,c$ are even).  Since neither $W$
nor $T$ is true in this case, the right hand side of the \QIFF\ is
false.  But the sum of an odd number of odd numbers---that is the
product of two odd numbers---is odd, so the sum $d$ is odd.  So $P$ is
also false.  Since both the left and right hand sides of the
\QIFF\ are false, the \QIFF\ assertion itself is true.

\inductioncase{Case 2}: (Exactly one among $a,b,c$ is even).  Now $W$
is true, which means that the right hand side of the \QIFF\ is true.
But the sum of an even number of odd numbers is even, and the sum of
even numbers is also even, and this means the sum $d$ is even.  So $P$
is also true.  Since both the left and right hand sides of the
\QIFF\ are true, the \QIFF\ assertion itself is true.

\inductioncase{Case 3}: (Exactly two among $a,b,c$ are even). Since
neither $W$ nor $T$ is true in this case, the right hand side of the
\QIFF\ is false.  The sum of the two even numbers is even, and the sum
of an even and an odd number is odd, so the sum $d$ is odd.  So $P$ is
also false.  Since both the left and right hand sides of the
\QIFF\ are false, the \QIFF\ assertion itself is true.

\inductioncase{Case 4}: (All three of $a,b,c$ are even).  Now $T$ is
true, which means that the right hand side of the \QIFF\ is true.
Also, a sum of even numbers is even, so $d$ is even.  That is, $P$ is
also true.  Since both the left and right hand sides of the \QIFF\ are
true, the \QIFF\ assertion itself is true.
\end{solution}

\ppart Now suppose that 
\[
w^2 + x^2 + y^2 = z^2,
\]
where $w,x,y,z$ are nonnegative integers.  Let $P$ be the assertion
that $z$ is even, and let $R$ be the assertion that all three of $w,
x, y$ are even.  Prove by cases that
\[
P\ \QIFF\ R.
\]
\hint An odd number equals $2m+1$ for some integer $m$, so its square
equals $4(m^2 + m) + 1$.

\begin{solution}
The cases will be the same as in part~\eqref{abcd-even}, using the
squares of $w,x,y,z$ for $a,b,c,d$. That is, let $a=w^2, b=x^2, c=y^2,$ and $d=z^2$. Note that a number $n$ is even iff $n^2$ is even. Therefore, we can directly use what we proved in part~\eqref{abcd-even} to see that:\\

\inductioncase{Case 1}: if $w,x,y$ are all odd $\implies$ $a,b,c$ are all off $\implies$ $d$ is odd $\implies$ $z$ is odd. $P$ and $R$ are both False so the \QIFF\ assertion is True.

\inductioncase{Case 3}: if two of $w,x,y$ are even $\implies$ two of $a,b,c$ are even $\implies$ d is odd $\implies$ z is odd. $P$ and $R$ are both False so the \QIFF\ assertion is True.

\inductioncase{Case 4}: if $w,x,y$ are all even $\implies$ $a,b,c$ are all even $\implies$ d is even $\implies$ z is even. $P$ and $R$ are both True so the \QIFF\ assertion is True. 

\inductioncase{Case 2}: if one of $w,x,y$ is even $\implies$ one of the $a,b,c$ is even $\implies$ d is even $\implies$ z is even. In this case, let's assume without loss of generality that $a$ and $b$ are odd and $c$ is even. This means that $w$ and $x$ are odd and $y$ is even. Then let $w=2m+1$, $x=2n+1$, and $y=2p$ for some nonnegative integers $m,n,p$. Then,\\
$z^2 = (2m+1)^2 + (2n+1)^2 + (2p)^2$\\
$z^2 = 4(m^2+m+n^2+n+p^2) + 2$\\
But the square of any even number must be a multiple of 4. Since $z^2$ leaves a remainder of 2 on division by 4, it is not a multiple of 4, so this is a contradiction. Therefore, no such $z$ can exist. That is, there exist no integers $w,x,y$ such that one is even, two are odd, and $w^2+x^2+y^2$ is a square. So if \QIFF\ assertion is vacuously true.\\

We have showed that in all cases, $P$ \QIFF $R$.
\end{solution}
\eparts

\end{problem}

\endinput
