\documentclass[problem]{mcs}

\begin{pcomments}
  \pcomment{PS_odd_length_walk}
  \pcomment{by ARM 3/13/11}
\end{pcomments}

\pkeywords{
  walk
  path
  cycle
  closed_walk
  digraph
}

%%%%%%%%%%%%%%%%%%%%%%%%%%%%%%%%%%%%%%%%%%%%%%%%%%%%%%%%%%%%%%%%%%%%%
% Problem starts here
%%%%%%%%%%%%%%%%%%%%%%%%%%%%%%%%%%%%%%%%%%%%%%%%%%%%%%%%%%%%%%%%%%%%%

\begin{problem}

\bparts

\ppart Give an example of a digraph in which a vertex $v$ is on a even
positive length closed walk, but not on any even positive length
cycle.

\begin{staffnotes}
\hint There is an example with three vertices.
\end{staffnotes}

\begin{solution}
A graph consisting of three vertices in a cycle is an example. That
is, the vertices are $a,b,c$ and the edges are $\diredge{a}{b},
\diredge{b}{c},\diredge{c}{a}$.  The \emph{only} cycle in the graph is
of length three, but of course going around it twice gives an even
lenth closed walk.

\iffalse

\begin{figure}
\graphic{Fig_walkpath}
\caption{An even positive length  $v$ to $v$ walk, but no even positive length $v$ to $v$  cycle.}
\label{fig:walkpath}
\end{figure}

As in Figure~\ref{fig:walkpath}, let
\begin{align*}
V & \eqdef \set{u,v,w,x},\\
E & \eqdef \set{\diredge{u}{w}, \diredge{w}{x}, \diredge{x}{u}, \diredge{v}{w}, \diredge{x}{v}}.
\end{align*}
There is even positive length closed walk from $v$ to $v$:
\[
v \diredge{v}{w} w \diredge{w}{x} x \diredge{x}{u} u  \diredge{u}{w} w \diredge{w}{x} x \diredge{x}{v}
\]
It is easy to check that the \emph{only} including $v$ is the odd length cycle
\[
v \diredge{v}{w} w \diredge{w}{x} x \diredge{x}{v}.
\]
There certainly is an
 There is no even positive length cycle from $v$ to $v$
from~$u$ to~$u$ \emph{that contains~$v$}.  The reason is that the sole
edge out of $u$ goes to $w$, and the sole edge out of~$v$ likewise
goes to $w$, so any walk from~$u$ to~$u$ that goes through~$v$ must go
through $w$ at least twice and therefore won't be a cycle.
\fi

\end{solution}

\ppart Prove that the shortest odd-length closed walk through a vertex is an
odd-length cycle.

\begin{solution}
\begin{proof}
As in the proof of Problem~\bref{PS_shortest_undirected_closed_walk}, if the
shortest odd-length closed walk contained a ``repeat'' vertex, then
the walk could be split at that vertex into two positive length closed
subwalks the sum of whose lengths was the length of the given shortest
walk.  But then exactly one of the two subwalks would be a shorter
odd-length walk, contradicting the minimality of the shortest
odd-length walk.
\end{proof}

\end{solution}

\eparts

\end{problem}

%%%%%%%%%%%%%%%%%%%%%%%%%%%%%%%%%%%%%%%%%%%%%%%%%%%%%%%%%%%%%%%%%%%%%
% Problem ends here
%%%%%%%%%%%%%%%%%%%%%%%%%%%%%%%%%%%%%%%%%%%%%%%%%%%%%%%%%%%%%%%%%%%%%

\endinput
