documentclass[problem]{mcs}

\begin{pcomments}
  \pcomment{CP_expected_number_of_keys_deviation_no_replace}
  \pcomment{overlaps CP_expected_number_of_keys, PS_n_keys}
  \pcomment{ARM 12/19/15}
\end{pcomments}

\pkeywords{
  random_variable
  expectation
  indicator_variable
  failure
  deviation
  variance
}

%%%%%%%%%%%%%%%%%%%%%%%%%%%%%%%%%%%%%%%%%%%%%%%%%%%%%%%%%%%%%%%%%%%%%
% Problem starts here
%%%%%%%%%%%%%%%%%%%%%%%%%%%%%%%%%%%%%%%%%%%%%%%%%%%%%%%%%%%%%%%%%%%%%

\begin{problem}
A man has a set of $n$ keys, one of which fits the door to his
apartment.  He tries the keys randomly throwing away each ill-fitting
key that he tries until he finds the key that fits.  That is, he
chooses keys randomly from \emph{among those he has not yet tried}.
This way he is sure to find the right key within $n$ tries.

\bparts

\ppart\label{mkleftnr} If he hasn't found the right key yet and there
are $m$ keys left, what is the probability that he will find the right
key on the next try?
\begin{solution}
\[
\frac{1}{m}.
\]
\end{solution}

\ppart\label{prT>kT>k1nr} Given that he did not find the right key on
his first $k-1$ tries, verify that the probability that he does not
find it on the $k$th trial is given by
\[
\prcond{T > k}{T > k-1} = \frac{n-k}{n-(k-1)}.
\]

\examspace[1.0in]

\begin{solution}
Given that the right key has not been picked in the first $k-1$
trials, there are $n-(k-1)$ keys left, so by part~\eqref{mkleftnr}, the
probability that the $k$th trial is also not the right key is
\[
1- \frac{1}{n-(k-1)} = \frac{n-k}{n-(k-1)}.
\]
\end{solution}

\ppart Prove that 
\begin{equation}\label{T>kn-knnr}
\pr{T > k} =  \frac{n-k}{n}.
\end{equation}

\hint This can be argued directly, but if you don't see how, induction
using part~\eqref{prT>kT>k1nr} will work.

\examspace[2.0in]

\begin{solution}
Suppose the man keeps going until he has tried all $n$ keys.  In this
case, it's clear that the right key is equally likely to come up on
each try, that is
\begin{equation}\label{prT=k1nnr}
\pr{T=k} = 1/n
\end{equation}
for $k \in \Zintv{1}{n}$, and hence the probability that it comes up
after the $k$th try is
\[
(n-k)\cdot \frac{1}{n} = \frac{n-k}{n}.
\]

Alternatively, we can prove~\eqref{T>kn-knnr} by induction on $k$,
using~\eqref{T>kn-knnr} as the induction hypothesis.

\inductioncase{Base case} {$k=0$}: He has to try at least one key,
which means $\pr{T>0} = 1 = (n-0)/n$, so~\eqref{T>kn-knnr} holds in
this case.

\inductioncase{Inductive step}:
We must show that for $k\geq 0$,
\[
\pr{T > k+1} =  \frac{n-(k+1)}{n}.
\]

Now we have
\begin{align*}
\pr{T > k+1}
 & = \prcond{T > k+1}{T > k} \cdot \pr{T > k}
        & \text{(Conditional Product Rule)} \\
 & = \frac{n-(k+1)}{n-k} \cdot \pr{T > k}
        & \text{(part~\eqref{prT>kT>k1nr})}\\
 & = \frac{n-(k+1)}{n-k}\frac{n-k}{n}
        & \text{(induction hypothesis)}\\
 & = \frac{n-(k+1)}{n}.
\end{align*}
\end{solution}

\ppart Conclude that in this case
\[
\expect{T} = \frac{n+1}{2}.
\]

\examspace[1.5in]

\begin{solution}
We have from~\eqref{prT=k1nnr} that $\pr{T = k} = \frac{1}{n}$.

This can also be proved using~\eqref{T>kn-knnr}:
\begin{align*}
\pr{T = k} 
  & = \pr{T > k-1} - \pr{T > k}\\
  & = \frac{n-(k-1)}{n} - \frac{n-k}{n}\\
  & = \frac{1}{n}.
\end{align*}
Therefore,
\[
\expect{T} = \sum_{k=1}^n k \cdot \pr{T=k}  = \frac{1}{n}\sum_{k=1}^n{k} = \frac{n+1}{2}.
\]
\end{solution}

\ppart Write a closed formula for $\variance{T}$.

\begin{solution}
\begin{align*}
\variance{T} & = \expectsq{T} - \expect{T}^2,\\
\expectsq{T}
   & \eqdef \sum_{k=1}^n k^2\pr{T=k}\\
   & = \sum_{k=1}^n k^2\frac{1}{n} & \text{(part~\eqref{prT=k1nnr})}\\
   & = \frac{\sum_{k=1}^n k^2}{n}\\
   & = \frac{n(n+1)(2n+1)}{6n}
           & \text{(Problem~\bref{CP_sum_of_squares} or \bref{CP_gen_func_sum_of_squares})}\\
   & = \frac{(n+1)(2n+1)}{6}.
\variance{T}
    & = \frac{(n+1)(2n+1)}{6} - \paren{\frac{n+1}{2}}^2\\
    & = n(n+1)\frac{n - 1}{12n}
%\frac{\binom{n}{3}}{2n}
         & \text{(algebra)}.
\end{align*}
\end{solution}

\begin{staffnotes}
\begin{align*}
\lefteqn{\frac{n(n+1)(2n+1)}{6n} - \paren{\frac{n+1}{2}}^2}\\
& = n(n+1)\frac{2(2n+1)}{12n} - \frac{3n(n+1)^2}{12n}\\
& = n(n+1)\paren{\frac{4n+2}{12n} - \frac{3(n+1)}{12n}}\\
& = n(n+1)\frac{4n+2 - 3n-3}{12n}\\
& = n(n+1)\frac{n - 1}{12n}\\
& = \frac{\binom{n}{3}}{2n}
\end{align*}

\end{staffnotes}

\eparts

\end{problem}

\endinput
