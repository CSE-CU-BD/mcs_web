\documentclass[problem]{mcs}

\begin{pcomments}
  \pcomment{MQ_modular_arithmetic_2}
\end{pcomments}

\pkeywords{
  modular_arithmetic
  gcd
  phi
  Euler
}

%%%%%%%%%%%%%%%%%%%%%%%%%%%%%%%%%%%%%%%%%%%%%%%%%%%%%%%%%%%%%%%%%%%%%
% Problem starts here
%%%%%%%%%%%%%%%%%%%%%%%%%%%%%%%%%%%%%%%%%%%%%%%%%%%%%%%%%%%%%%%%%%%%%

\begin{problem}
% Tests understanding of gcd and modular arithmetic

\begin{problemparts}

\problempart
\label{MQ:phi}
Calculate the value of $\phi(6042)$. 

\hint $53$ is a factor of $6042$.
\examspace[3in]

\begin{solution}
\[
\phi(6042) = \phi(2) \phi(3) \phi(19) \phi(53) =  (2 - 1)(3 - 1)(19 - 1)(53 - 1) = 1872.
\]
\end{solution}


\problempart Consider an integer $k > 0$ that is relatively prime to 6042.  Explain why
$k^{9361} \equiv k \pmod{6042}$.

\hint Use your solution to part~\eqref{MQ:phi}.

\begin{solution}
\[
k^{9361} \equiv k^{1872\cdot 5 + 1} \equiv k(k^{1872})^5 \pmod{6042}.
\]  By
Euler's Theorem, since $k$ and 6042 are relatively prime,
$k^{\phi(6042)} \equiv 1 \pmod{6042}$.  By part (a), we have that
$\phi(6042) = 1872$, implying $k^{1872} \equiv 1 \pmod{6042}$.  Hence,
$k(k^{1872})^5 \equiv k (1^5) \equiv k \pmod{6042}$.
\end{solution}

\end{problemparts}
\end{problem}

\endinput
