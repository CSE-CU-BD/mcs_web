\documentclass[problem]{mcs}

\begin{pcomments}
  \pcomment{PS_strongly_connected}
  \pcomment{ARM 4/4/17}
  \pcomment{abstract path properties; needs connection ot Computer Science}
\end{pcomments}

\pkeywords{
  digraph
  connected
  dag
  strongly_connected
  }

\newcommand{\mutcon}{\ensuremath{\mrel{\overset{*}{\longleftrightarrow}}}}
\newcommand{\reach}{\ensuremath{\mrel{\rightsquigarrow}}}

%%%%%%%%%%%%%%%%%%%%%%%%%%%%%%%%%%%%%%%%%%%%%%%%%%%%%%%%%%%%%%%%%%%%%
% Problem starts here
%%%%%%%%%%%%%%%%%%%%%%%%%%%%%%%%%%%%%%%%%%%%%%%%%%%%%%%%%%%%%%%%%%%%%
\begin{problem}
Say that vertices $u,v$ in a digraph $G$ are \emph{mutually connected}
and write
\[
u \mutcon v,
\]
when there is a path from $u$ to $v$ and also a path from $v$ to $u$.

\bparts

\ppart Prove that \mutcon\ is an equivalence relation on
$\vertices{G}$.

\examspace[3.0in]

\begin{solution}
To show that \mutcon\ is an equivalence relation, we must show that $u
\mutcon v$ is reflexive, symmetric, and transitive.

\textbf{Reflexive}: There is a length-zero path from $u$ to itself, so
$u \mutcon u$.

\textbf{Symmetric}: \mutcon\ is symmetric by definition.

\textbf{Transitive}: Suppose $u \mutcon v$ and $v \mutcon w$.  We must
show that $u \mutcon w$.

Now $u \mutcon v$ in particular implies there is a path from $u$ to
$v$, and $v \mutcon w$ implies there is a path from $v$ to $w$.  The
merge of these paths will be a walk from $u$ to $w$.  But since there
is a walk from $u$ to $w$, there is also a path from $u$ to $w$.  The
same reasoning shows there is a path from $w$ to $u$.  Hence, $u
\mutcon w$.
\end{solution}

\ppart The blocks of the equivalence relation \mutcon\ are called the
\emph{strongly connected components} of $G$.  Define a relation
\reach\ on the strongly connected components of $G$ by the rule
\[
C \reach D\quad \QIFF \text{ there is a path from some vertex in $C$
  to some vertex in $D$}.
\]

Prove that \reach\ is a weak partial order on the strongly connected
components.

\begin{solution}
To show that \reach\ is a weak partial order, we must show that it is
reflexive, antisymmetric, and transitive.


\textbf{Reflexive}: Blocks are nonempty, so for any block $C$, there
is a vertex $c \in C$, and since there is a length-zero path from $c$
to $c$, we have $C \reach C$.


\textbf{Antisymmetric}: Suppose $C \reach D$ and $C \reach D$ for
blocks $C,D$ of $\mutcon$.  We need to show that $C=D$.

Since $C \reach D$, there is a path $\walkv{p_{c_1,d_1}}$ from some
vertex $c_1 \in C$ to some vertex $d_1 \in D$.

Also, by the definition of block, there is a path $\walkv{p_{c,c_1}}$
from any $c \in C$ to $c_1$.  Likewise, there is a path
$\walkv{p_{d_1,d}}$ from $d_1$ to any $d \in D$.  Merging the paths
$\walkv{p_{c,c_1}}$, $\walkv{p_{c_1,d_1}}$ and $\walkv{p_{d_1,d}}$
yields a walk from $c$ to $d$.  Therefore there is a path from $c$ to
$d$.

Likewise, since $D \reach C$, there is a path from any $d \in D$ to
any $c \in C$.  Therefore $c \mutcon d$ for all $c \in C$ and $d \in
D$, so $C$ and $D$ are the same block of $\mutcon$.

\textbf{Transitive}: Suppose $C \reach D$ and $D \reach E$.  So there
is a path $\walkv{p_{c,d_1}}$ from some vertex $c \in C$ to some
vertex $d_1 \in D$ and also a path $\walkv{p_{d_2,e}}$ from some
vertex $d_2 \in D$ to some vertex $e \in E$.  Since $D$ is a block of
\mutcon, there must be a path $\walkv{p_{d_1,d_2}}$ from $d_1$ to
$d_2$.  Merging $\walkv{p_{c,d_1}}$, $\walkv{p_{d_1,d_2}}$ and
$\walkv{p_{d_2,e}}$ yields a walk from $c$ to $e$, which proves that
$C \reach E$, as required.
\end{solution}

\eparts
\end{problem}


%%%%%%%%%%%%%%%%%%%%%%%%%%%%%%%%%%%%%%%%%%%%%%%%%%%%%%%%%%%%%%%%%%%%%
% Problem ends here
%%%%%%%%%%%%%%%%%%%%%%%%%%%%%%%%%%%%%%%%%%%%%%%%%%%%%%%%%%%%%%%%%%%%%

\endinput
