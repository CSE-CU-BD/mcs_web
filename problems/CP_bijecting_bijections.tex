\documentclass[problem]{mcs}

\begin{pcomments}
  \pcomment{CP_bijecting_bijections}
  \pcomment{by orm, Spring '11}
\end{pcomments}

\pkeywords{
  counting
  bijection
  product rule
  permutations
}

%%%%%%%%%%%%%%%%%%%%%%%%%%%%%%%%%%%%%%%%%%%%%%%%%%%%%%%%%%%%%%%%%%%%%
% Problem starts here
%%%%%%%%%%%%%%%%%%%%%%%%%%%%%%%%%%%%%%%%%%%%%%%%%%%%%%%%%%%%%%%%%%%%%

\begin{problem}
The following problems give you an idea of how many relations and functions there are given two sets.
\bparts


\ppart Let $X$, $Y$ be two finite sets. How many relations of $X$ and $Y$ are there. 

\begin{solution}
The set of all pairs $X\times Y$ has $|X||Y|$ elements. Any subset of it is a relation, hence there are $2 ^ {|X||Y|}$
\end{solution}

\ppart Let $X$ and $Y$ be two arbitrary finite sets. Propose a bijection between 
the set $F_{X,Y}$  of all {\em total functions} from X to Y and the set $Y^{|X|}$. Recall 
$Y^n $ is the cartesian product of Y $n$ times. 

Based on that, how many {\em total functions} are there from set $X$ to set $Y$?

\begin{solution}
We can encode a given function from $X$ to $Y$ by first giving an ordering to elements in $X$, say, 
calling them $x_1$, $x_2$, $\cdots$, $x_{|X|}$. 

Now given an element $f \in F_{X,Y}$ we can associate it with and element $g \in Y^{|X|}$ by following the rule $g[i] = f(x_i)$, where $g[i]$ is the $i$th entry of the vector. 

This is a total, bijective function, since it is defined for every $f \in F_{X,Y}$. It is also surjective and injective, as we show next.

To prove it is surective consider an arbitrary element of  $Y$. Write the element as  $(y_1, y_2, y_3,\cdots,y_{|X|})$. Now, the function $h \in X$ with $h(x_i) :=  y_i$ will map to it under our definition. To prove it is injective, suppose $g,h \in X$ map to the same vector $(z_1, z_2, z_3, \cdots, z_{|X|})$, then based on our rule we know $g(x_i) = h(x_i)  = z_i$ for all $x_i \in X$. Hence $g = h$. 

Based on this bijection we can easily count the number of total functions $f:X\to Y$ by counting the elements of $Y^{|X|}$. Since we know how to count cartesian products, we know the answer is  $|Y|^{|X|}$.  In fact, in many book, the set of all total functions from a set  $X$ to a set $Y$ is often denoted as $X^Y$
\end{solution}

\ppart Using the previous part how many {\em functions}, not necessarily total, are there from $X$ to $Y$. How does the fraction of functions vs. total functions grow  as the size of $X$ grows.  Is it $O(1)$, is it $O(|X|)$, is it $O(2^{|X|})$?

\begin{solution}
We can model this by adding a dummy element to $Y$, which indicates whether a given $x\in X$ has an actual image or not. After using the previous part, we get there are $(|Y|+1)^{|X|}$ functions, not necessarily total.   By taking the ratio of this answer and the previous questions, we see the ratio is $\left( \frac{|Y|+1}{|Y|}  \right)^{|X|}$ so it is not $O(1)$ nor $O(|x|)$ with respect to $|X|$,  rather it is exponential. Also, since $|Y| + 1 \leq 2|Y|$, then the ratio above is indeed $O(2^{|X|})$
\end{solution}


\ppart Show a bijection between $\mathcal{P}(X)$ and the set $F_{X,\{0,1\}}$ of functions  $f: X \to \{0,1\}$. 

\begin{solution}
Consider bijection $b: \mathcal{P}(X) \to  F_{X,\{0,1\}}$ defined as follows.  Let  $s\in \mathcal{P}(x)$, then let $b_s(x_i) = 1$ iff $x_i \in S$. We make $b_s(x_i) = 0$ otherwise. It can be shown this correspondnace is a bijection. //TODO.

This and the previous part show why $\mathcal{P}(x)$ is sometimes denoted as $2^X$.
\end{solution}

\ppart Let $X = \{1, 2, \cdots, n\}$. In this problem we count how many bijectons there are from $X$ to itself. 
Consider the set $B_{X,X}$ of all {\em bjections} from set $X$ to set $X$. Show a bijection from $B_{X,X}$ to the set of all permuations of $X$ (as defined in the notes). Using that, count $B_{X,X}$.

\begin{solution}
//TODO. 

In many other contexts, a permutation is defined as a bijection from a set to itself rather than as an ordered list of non repeating elements. Now you know why the two definitions are equivalent. 
\end{solution}


\eparts
\end{problem}

%%%%%%%%%%%%%%%%%%%%%%%%%%%%%%%%%%%%%%%%%%%%%%%%%%%%%%%%%%%%%%%%%%%%%
% Problem ends here
%%%%%%%%%%%%%%%%%%%%%%%%%%%%%%%%%%%%%%%%%%%%%%%%%%%%%%%%%%%%%%%%%%%%%

\endinput
