\documentclass[problem]{mcs}

\begin{pcomments}
  \pcomment{FP_monty_hall_variant}
  \pcomment{F01.final}
  \pcomment{edited ARM 5/17/15}
  \pcomment{last part requires mean-time-to-failure, commented out in book}
\end{pcomments}

\pkeywords{
  conditional_probability
  Monty_Hall
  total_probability
  failure
  expectation
}

%%%%%%%%%%%%%%%%%%%%%%%%%%%%%%%%%%%%%%%%%%%%%%%%%%%%%%%%%%%%%%%%%%%%%
% Problem starts here
%%%%%%%%%%%%%%%%%%%%%%%%%%%%%%%%%%%%%%%%%%%%%%%%%%%%%%%%%%%%%%%%%%%%%

\begin{problem}
Here's a variation of Monty Hall's game: the contestant still picks
one of three doors, with a prize randomly placed behind one door and
goats behind the other two.  But now, instead of always opening a door
to reveal a goat, Monty instructs Carol to \emph{randomly} open one of
the two doors that the contestant hasn't picked.  This means she may
reveal a goat, or she may reveal the prize.  If she reveals the prize,
then the entire game is \emph{restarted}, that is, the prize is again
randomly placed behind some door, the contestant again picks a door,
and so on until Carol finally picks a door with a goat behind it.
Then the contestant can choose to \emph{stick} with his original
choice of door or \emph{switch} to the other unopened door.  He wins
if the prize is behind the door he finally chooses.

To analyze this setup, we define two events:
\begin{description}
\item[$GP$:] The event that the contestant \textbf{g}uesses the door with
the \textbf{p}rize behind it on his first guess.

\item[$OP$:] The event that the game is restarted at least once.  Another
way to describe this is as the event that the door Carol first
\textbf{o}pens has a \textbf{p}rize behind it.
\end{description}

Give the values of the following probabilities:

\bparts

\ppart $\pr{GP}$ \hfill\examrule[0.7in]

\examspace[0.4in]

\begin{solution}
\textbf{1/3}.
\end{solution}

\ppart $\prcond{OP}{\setcomp{GP}}$ \hfill\examrule[0.7in]

\examspace[0.4in]
\begin{solution}
\textbf{1/2}.
\end{solution}

\ppart $\pr{OP}$  \hfill \examrule[0.7in]

\examspace[0.6in]

\begin{solution}
\textbf{1/3}.

\[
\pr{OP} = \prcond{OP}{GP}\pr{GP} +
\prcond{OP}{\setcomp{GP}}\pr{\setcomp{GP}} = 0\cdot 1/3 + 1/2 \cdot
2/3 = 1/3.
\]
\end{solution}

\ppart the probability that the game will continue forever \hfill\examrule[0.7in]

\examspace[0.6in]

\begin{solution}
0.  There is a $(\pr{OP})^n$ probability that the game will restart at
least $n$ times.  This probability goes to zero as $n$ goes to
infinity.
\end{solution}

\ppart When Carol finally picks the goat, the contestant has the
choice of sticking or switching.  Let's say that the contestant adopts
the strategy of sticking.  Let $W$ be the event that the contestant
wins with this strategy, and let $w \eqdef \pr{W}$.  Express the
following conditional probabilities as simple closed forms in terms of
$w$.
\begin{enumerate}
\renewcommand{\labelenumi}{\roman{enumi})}

\item $\prcond{W}{GP}$ \hfill\examrule[0.7in]

\examspace[0.3in]

\begin{solution}
1
\end{solution}

\item $\prcond{W}{\setcomp{GP}\intersect OP}$ \hfill\examrule[0.7in]

\examspace[0.3in]

\begin{solution}
$w$
\end{solution}

\item $\prcond{W}{\setcomp{GP}\intersect \setcomp{OP}}$ \hfill\examrule[0.7in]

\examspace[0.3in]

\begin{solution}
0
\end{solution}
\end{enumerate}

\ppart What is the value of $\pr{W}$? \hfill\examrule[0.7in]

\examspace[0.8in]

\begin{solution}
1/2, because
\[\begin{array}{rcll}
w & = & \prcond{W}{GP}\pr{GP} 
        & +\ \prcond{W}{\setcomp{GP}\intersect OP}\pr{\setcomp{GP}\intersect OP}\\
  &   & & +\ \prcond{W}{\setcomp{GP}\intersect \setcomp{OP}}\pr{\setcomp{GP}\intersect \setcomp{OP}}\\
  & = & 1 \cdot 1/3 & +\ w \cdot 2/3 \cdot 1/2 + 0 \cdot 2/3 \cdot 1/2\\
  & = & 1/3 + w/3.
\end{array}\]
So $w(1- 1/3) = 1/3$.
\end{solution}

\inbook{
\ppart For any final outcome where the contestant wins with a ``stick''
strategy, he would lose if he had used a ``switch'' strategy, and vice
versa.  In the original Monty Hall game, we concluded immediately that the
probability that he would win with a ``switch'' strategy was $1-\pr{W}$.
Why isn't this conclusion quite as obvious for this new, restartable game?
Is this conclusion still sound?  Briefly explain.

\begin{solution}
Switching strategies turns wins to losses for \emph{terminated} games.
So the probability of win with switch is $t-\pr{W}$ where $t$ is the
probability of termination.  In original Monty Hall, the game
terminated after one door-opening, so $t$ was 1.  The extra
complication here is that it is possible for the game to run forever,
but this event has probability 0, so $t$ is still 1, and the
conclusion is still sound.
\end{solution}
}

\inhandout{\ppart Let $R$ be the number of times the game is restarted
  before Carol picks a goat.

  What is $\expect{R}$? \hfill \examrule[0.8in]

  (You may express the answer as a simple closed form in terms of $p
  \eqdef \pr{OP}$.)

\examspace[0.6in]

\begin{solution}
$1/(1-p) - 1 = 3/2 -1 = 1/2$.

Think of \emph{not} having to restart as a failure.  So
$\pr{\text{failure}} = 1-p$, and $\expect{R}$ is mean time to failure
minus one---because we are only counting the number of
``successes''---namely, $1/\pr{\text{failure}} - 1$.
\end{solution}
}

\eparts
\end{problem}

%%%%%%%%%%%%%%%%%%%%%%%%%%%%%%%%%%%%%%%%%%%%%%%%%%%%%%%%%%%%%%%%%%%%%
% Problem ends here
%%%%%%%%%%%%%%%%%%%%%%%%%%%%%%%%%%%%%%%%%%%%%%%%%%%%%%%%%%%%%%%%%%%%%

\endinput
