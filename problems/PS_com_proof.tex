\documentclass[problem]{mcs}

\begin{pcomments}
  \pcomment{PS_com_proof}
  \pcomment{from: S04 ps9}
\end{pcomments}

\pkeywords{
 combinatorial_proof
 binomial_coefficient
 bit_strings
}

%%%%%%%%%%%%%%%%%%%%%%%%%%%%%%%%%%%%%%%%%%%%%%%%%%%%%%%%%%%%%%%%%%%%%
% Problem starts here
%%%%%%%%%%%%%%%%%%%%%%%%%%%%%%%%%%%%%%%%%%%%%%%%%%%%%%%%%%%%%%%%%%%%%

\begin{problem}
Give combinatorial proofs of the identities below.  Use the following
structure for each proof.  First, define an appropriate set $S$.
Next, show that the left side of the equation counts the number of
elements in $S$.  Then show that, from another perspective, the right
side of the equation also counts the number of elements in set $S$.
Conclude that the left side must be equal to the right, since both are
equal to $\card{S}$.

\bparts

\ppart

\[
\binom{2n}{n} = \sum_{k=0}^n \binom{n}{k} \cdot \binom{n}{n-k}
\]

\iffalse
A similar identity is proved in Rosen as a corollary of Vandermonde's
Identity, but your proof should be ``from scratch''.
\fi

\begin{solution}
We are counting the number of $2n$-bit strings with an equal
number of zeros and ones.  On one hand, the number of such strings is
equal to $\binom{2n}{n}$, since we must choose $n$ of the $2n$ bits to
be 1's.  This gives the expression on the left side of the equation.
On the other hand, the number of 1's in the first half of the string
must be equal to some number $k$ between 0 and $n$.  The number of
$2n$-bit strings with exactly $k$ ones in the first half and,
consequently, $n-k$ ones in the second half is equal to $\binom{n}{k}
\cdot \binom{n}{n-k}$.  Summing over all possible values of $k$ gives
the expression on the right side of the equation above.  Since the two
sides of this equation count the elements of the same set in two
different ways, the two sides must be equal.  
\end{solution}

\ppart

\[
\sum_{i=0}^r \binom{n+i}{i}  =  \binom{n+r+1}{r}
\]

\begin{solution}
Let $S$ be the set of all $(n + r + 1)$-bit sequences with
exactly $r$ ones.  Clearly, the right side is equal to $|S|$.  On the
other hand, we can partition the sequences in $S$ based on the
position of the last zero.  In particular, the last zero must appear
in position $n + i + 1$ for some $i \in \set{0, \dots, r}$.  The
number of sequences with a zero in position $n + i + 1$ is $\binom{n +
i}{i}$, since such a sequence consists of $i$ ones distributed
arbitrarily in the first $n + i$ positions followed by a 0 and an
unbroken run of 1's.  Therefore, $|S|$ is equal to the left side as
well.  Thus, the right side must equal the left side.
\end{solution}

\eparts

\end{problem}

%%%%%%%%%%%%%%%%%%%%%%%%%%%%%%%%%%%%%%%%%%%%%%%%%%%%%%%%%%%%%%%%%%%%%
% Problem ends here
%%%%%%%%%%%%%%%%%%%%%%%%%%%%%%%%%%%%%%%%%%%%%%%%%%%%%%%%%%%%%%%%%%%%%

\endinput
