\documentclass[problem]{mcs}

\begin{pcomments}
  \pcomment{PS_RM_equal_AM}
  \pcomment{subsumes MQ_RM_subs_M}
  \pcomment{by ARM 2/23/11; revised 3/3/11}
  \pcomment{was called PS_M_equal_RM}
  \pcomment{soln by madksev mar 2017, edited ARM 3/24/17}
\end{pcomments}

\pkeywords{
  string
  matched
  bracket
  structural_induction
  induction
  concatenation
  ambiguous
}

%%%%%%%%%%%%%%%%%%%%%%%%%%%%%%%%%%%%%%%%%%%%%%%%%%%%%%%%%%%%%%%%%%%%%
% Problem starts here
%%%%%%%%%%%%%%%%%%%%%%%%%%%%%%%%%%%%%%%%%%%%%%%%%%%%%%%%%%%%%%%%%%%%%

\begin{problem}
We prove that two recursive definitions of strings of matched
brackets, $\RM$\inhandout{\footnote{The set $\RM$ of strings of
    brackets is defined recursively by:

\inductioncase{Base case}: The empty string $\emptystring \in\RM$.

\inductioncase{Constructor case}: If $s,t \in \RM$, then $\lefbrk s\,
\rhtbrk t \in \RM.$ }}\inbook{ of
  Definition~\bref{RM_def}}
 and $\AM$\inhandout{\footnote{The set $\AM$ of matched bracket strings
is defined recursively by:

\inductioncase{Base case}: The empty string $\emptystring \in \AM$,

\inductioncase{Constructor cases}: if $s,t \in \AM$, then
  the strings $\lefbrk s\, \rhtbrk$ and $s \cdot t$ are also in $\AM$.}}
\inbook{ of Definition~\bref{AM_def}} define the same set.

\bparts

\ppart\label{RMcat} Prove that the set $\RM$ is closed under string
concatenation.  Namely, $\forall s,t \in \RM.\, s\cdot t \in \RM$.

\examspace[3in]

\begin{solution}
The proof is by structural induction on the definition of $s \in \RM$.

\inductioncase{Induction hypothesis}: $P(s) \eqdef \forall t \in \RM, s \cdot t \in \RM$.

\inductioncase{Base case}: ($s = \emptystring$). $P(s)$ holds, since
$\emptystring \cdot t = t$ for all $t$, and so $s \cdot t \in \RM$ for
all $t \in \RM$.

\inductioncase{Constructor case}: Assume that $s = \lefbrk a \rhtbrk b$ for
$a, b \in \RM$.  We may assume by structural induction that $P(a)$ and
$P(b)$ both hold.  We must prove $P(s)$.

If $t \in \RM$, then $bt \in \RM$ by the induction hypothesis $P(b)$.
Now,
\[
s \cdot t = (\lefbrk a \rhtbrk b) t = \lefbrk a \rhtbrk (b t),
\]
and the right hand term describes a string in $\RM$ by the
constructor.  This proves $P(s)$.

By structural induction, we conclude that $P(s)$ is true for all $s
\in \RM$, that is,
\[
\forall s, t \in \RM.\ s \cdot t \in \RM 
\]
as claimed.
\end{solution}

\ppart\label{AMsubRM} Prove by structural induction that $\AM \subseteq \RM$.

\inhandout{\hint part~\eqref{RMcat}.}

\examspace[3in]

\begin{solution}
We need to show that $\forall s \in \AM.\ s \in \RM$.  The proof is
by structural induction on the definition of $s \in \AM$.

\inductioncase{Induction hypothesis}: $P(s) \eqdef s \in \RM$.

\inductioncase{Base case}: ($s = \emptystring$).  $P(s)$ holds by the base
case for $\RM$.

\inductioncase{Constructor case}: ($s = \lefbrk a \rhtbrk$ for $a \in \AM$).

We may assume by structural induction that $P(a)$ holds.   Now
\[
s = s \cdot \emptystring  = \lefbrk a \rhtbrk \emptystring,
\]
and the right hand term describes a string in $\RM$ by the
constructor.  This proves $P(s)$, as required.

\inductioncase{Constructor case}: ($s = ab$ for $a,b \in \AM$).

We may assume $P(a)$ and $P(b)$ are true By induction hypothesis.
Therefore, $a,b \in \RM$.  By part~\eqref{RMcat}, we conclude that $ab
\in \RM$.  This proves $P(s)$, as required.

We conclude by structural induction that $\forall s \in \AM.\ s \in
\RM$, or equivalently, $\AM \subseteq \RM$.
\end{solution}

\ppart  Prove that $\RM = \AM$.

\begin{solution}
  All that's needed is a proof that $\RM \subseteq \AM$, since the
  converse inclusion was already proved in part~\eqref{AMsubRM}.

\begin{proof}
We need only show that $\forall s \in \RM.\ s \in \AM$.  The proof is
by structural induction on the definition of $s \in \RM$.

\inductioncase{Induction hypothesis}: $P(s) \eqdef s \in \AM$.

\inductioncase{Base case}: ($s = \emptystring$). $P(s)$ holds since
$\emptystring$ is a base case for $\AM$.

\inductioncase{Constructor case}: Assume that $s = \lefbrk a \rhtbrk b$ for
$a, b \in \RM$.  We may assume by structural induction that $P(a)$ and
$P(b)$ both hold.  We must prove $P(s)$, that is
\[
\lefbrk a \rhtbrk b \in \AM.
\]
We know $\lefbrk a \rhtbrk \in \AM$ by the first constructor of
$\AM$.  Now since $\lefbrk a \rhtbrk, b \in \AM$, we know $s = (\lefbrk a
\rhtbrk) \cdot b \in \AM$ by the second constructor of $\AM$.  Thus $P(s)$
holds.

We can conclude by structural induction that $\forall s \in \RM.\ s \in
\AM$, or equivalently, $\RM \subseteq \AM$.
\end{proof}

Since $\AM \subseteq \RM$ and $\RM \subseteq \AM$, it follows that $\RM
= \AM$.

\end{solution}

\eparts

\end{problem}

%%%%%%%%%%%%%%%%%%%%%%%%%%%%%%%%%%%%%%%%%%%%%%%%%%%%%%%%%%%%%%%%%%%%%
% Problem ends here
%%%%%%%%%%%%%%%%%%%%%%%%%%%%%%%%%%%%%%%%%%%%%%%%%%%%%%%%%%%%%%%%%%%%%

\endinput
