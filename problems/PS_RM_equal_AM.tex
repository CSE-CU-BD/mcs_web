\documentclass[problem]{mcs}

\begin{pcomments}
  \pcomment{PS_RM_equal_AM}
  \pcomment{by ARM 2/23/11; revised 3/3/11}
  \pcomment{was called PS_M_equal_RM}
\end{pcomments}

\pkeywords{
  string
  matched
  bracket
  structural_induction
  induction
  concatenation
  ambiguous
}

%%%%%%%%%%%%%%%%%%%%%%%%%%%%%%%%%%%%%%%%%%%%%%%%%%%%%%%%%%%%%%%%%%%%%
% Problem starts here
%%%%%%%%%%%%%%%%%%%%%%%%%%%%%%%%%%%%%%%%%%%%%%%%%%%%%%%%%%%%%%%%%%%%%

\begin{problem}

\bparts

\ppart\label{RMcat} Prove that the set \RM\ of matched strings\inbook{ of
  Definition~\bref{RM_def}} is closed under string
concatenation.\inhandout{\footnote{The set $\RM$ of strings of brackets
    is defined recursively as follows:
\begin{itemize}

\item \textbf{Base case:} $\emptystring \in\RM$.

\item \textbf{Constructor case:} If $s,t \in \RM$, then $\lefbrk s\, \rhtbrk t \in \RM.$
\end{itemize}}}
Namely, if $s,t \in \RM$, then $s\cdot t \in \RM$.

\begin{solution}
The proof is by structural induction on the definition of $s \in \RM$.

\textbf{Induction hypothesis:} $P(s) ::= $\forall t \in \RM, s \cdot t \in \RM$.

\textbf{Base case:} ($s = \emptystring$). $P(s)$ holds, since $\emptystring \cdot t = t$ for all $t$, and so $s \cdot t \in \RM$ for all $t \in \RM$.

\textbf{Constructor case:} Assume that $s = \lefbrk a \rhtbrk b$ for $a, b \in \RM$. We may assume by structural induction that $P(a)$ and $P(b)$ both hold. We must prove $P(s)$, that is
\[ \forall t \in \RM, s \cdot t \in \RM. \]

For all $t$ we have
\[ s \cdot t = \lefbrk a \rhtbrk b t. \]
If $t \in \RM$, then $bt \in \RM$ by the induction hypothesis. Now since $a, bt \in \RM$, we have $\lefbrk a \rhtbrk b t$ is in $\RM$ by the constructor. Thus, $s \cdot t \in \RM$ if $t \in \RM$. This proves $P(s)$.

By structural induction, we can conclude
\[ \forall s \in \RM, P(s) \]
or equivalently,
\[ \forall s, t \in \RM, s \cdot t \in \RM \]
\end{solution}

\ppart\label{AMsubRM}  Prove $\AM \subseteq \RM$, where \AM\ is the set of ambiguous
matched strings\inbook{ of Definition~\bref{AM_def}}.\inhandout{\footnote{
The set, $\AM \subseteq \brkts$ is defined recursively as follows:
\begin{itemize}

\item \textbf{Base case:} $\emptystring \in \AM$,

\item \textbf{Constructor cases:} if $s,t \in \AM$, then
  the strings $\lefbrk s\, \rhtbrk$ and $st$ are also in $\AM$.
\end{itemize}
}}

\begin{solution}
It is equivalent to show $\forall s \in \AM, s \in \RM$. The proof is by structural induction on the definition of $s \in \AM$.

\textbf{Induction hypothesis:} $P(s) ::= s \in \RM$.

\textbf{Base case:} ($s = \emptystring$). We know $P(s)$ holds since $\emptystring$ is a base case for $\RM$.

\textbf{Constructor cases:} There are two constructor cases to consider for $\AM$, $s = \lefbrk a \rhtbrk$ and $s = ab$.

Assume that $s = \lefbrk a \rhtbrk$ for $a \in \AM$. We may assume by structural induction that $P(a)$ holds. We must prove $P(s)$. Note that
\[ s = \lefbrk a \rhtbrk = \lefbrk a \rhtbrk \emptystring. \]
Thus we know that $s \in \RM$ since $a \in \RM$ (by the induction hypothesis) and $\emptystring \in \RM$, by the constructor case for $\RM$. Hence, $P(s)$ holds for this constructor case.

Assume that $s = ab$ for $a, b \in \AM$. We may assume by structural induction that $P(a)$ and $P(b)$ both hold. We must prove $P(s)$. Note that in part~\eqref{RMcat} we proved that if $a, b \in \RM$ then $ab \in \RM$. Thus, $P(s)$, since $a,b \in \RM$ by the induction hypothesis.

For both constructor cases, we showed $P(s)$.

We can conclude by structural induction that $\forall s \in \AM, s \in \RM$, or equivalently, $\AM \subseteq \RM$.
\end{solution}

\ppart  Prove that $\RM = \AM$.

\begin{solution}
  All that's needed is a proof that $\RM \subseteq \AM$, since the
  converse inclusion was already proved in part~\eqref{AMsubRM}.

\begin{proof}
It is equivalent to show $\forall s \in \RM, s \in \AM$. The proof is by structural induction on the definition of $s \in \RM$.

\textbf{Induction hypothesis:} $P(s) ::= s \in \AM$.

\textbf{Base case:} ($s = \emptystring$). $P(s)$ holds since $\emptystring$ is a base case for $\AM$.

\textbf{Constructor case:} Assume that $s = \lefbrk a \rhtbrk b$ for $a, b \in \RM$. We may assume by structural induction that $P(a)$ and $P(b)$ both hold. We must prove $P(s)$, that is
\[ \lefbrk a \rhtbrk b \in \AM. \]
We know $\lefbrk a \rhtbrk \in \AM$ by the first constructor of $\AM$. Now since $\lefbrk a \rhtbrk, b \in \AM$, we know $\lefbrk a \rhtbrk b \in \AM$ by the second constructor of $\AM$. Thus $P(s)$ holds. 

We can conclude by structural induction that $\forall s \in \RM, s \in \AM$, or equivalently, $\RM \subseteq \AM$.

\end{proof}

Since $\AM \subseteq \RM$ and $\RM \subseteq \AM$, $\RM = \AM$.

\end{solution}

\eparts

\end{problem}

%%%%%%%%%%%%%%%%%%%%%%%%%%%%%%%%%%%%%%%%%%%%%%%%%%%%%%%%%%%%%%%%%%%%%
% Problem ends here
%%%%%%%%%%%%%%%%%%%%%%%%%%%%%%%%%%%%%%%%%%%%%%%%%%%%%%%%%%%%%%%%%%%%%

\endinput
