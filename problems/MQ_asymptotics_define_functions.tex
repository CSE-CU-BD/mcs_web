\documentclass[problem]{mcs}

\begin{pcomments}
  \pcomment{MQ_asymptotics_define_functions}
  \pcomment{perturbation of FP_asymptotics_define_functions parts(a)(b)}
\end{pcomments}

\pkeywords{
  asymptotics
  little_oh
  big_oh
  Theta
  asymptotically_equal
  partial_order
  equivalence_relation
  implies
}

%%%%%%%%%%%%%%%%%%%%%%%%%%%%%%%%%%%%%%%%%%%%%%%%%%%%%%%%%%%%%%%%%%%%%
% Problem starts here
%%%%%%%%%%%%%%%%%%%%%%%%%%%%%%%%%%%%%%%%%%%%%%%%%%%%%%%%%%%%%%%%%%%%%

\begin{problem}

\mbox{}

\begin{problemparts} 

  \ppart\label{mqeswn} Indicate which of the following asymptotic
  relations on the set of positive real-valued functions are equivalence
  relations, (\textbf{E}), strict partial orders (\textbf{S}), weak
  partial orders (\textbf{W}), or \emph{none} of the above (\textbf{N}).

\begin{itemize}

\item $f=o(g)$, the ``little Oh'' relation. \hfill \examrule

\begin{solution}
\textbf{S}
\end{solution}

\item $f=O(g)$, the ``big Oh'' relation. \hfill \examrule

\begin{solution}
\textbf{N} because it is neither symmetric nor antisymmetric.
\end{solution}

\item $f=\Theta(g)$, the ``Theta'' relation. \hfill \examrule

\begin{solution}
\textbf{E}
\end{solution}

\item $f \sim g$, the ``asymptotically equal'' relation. \hfill \examrule
\begin{solution}
\textbf{E}
\end{solution}

\item $f=g \QOR [f=O(g) \QAND \QNOT(g=O(f))]$. \hfill \examrule

\begin{solution}
\textbf{W}.
\end{solution}

\end{itemize}

\ppart Indicate the implications among the assertions in
part~\eqref{mqeswn}

\examspace[1.0in]

\begin{solution}
\begin{align*}
f\sim g & \QIMPLIES f = \Theta(g)\ \QIMPLIES\ f= O(g),\\
%f = o(g) & \QIMPLIES\ f= O(g),\\
f = o(g) & \QIMPLIES\ f=O(g) \QAND \QNOT(g=O(f))\\
         & \QIMPLIES\ f = g \QOR\ [f=O(g) \QAND \QNOT(g=O(f))]\\
         & \QIMPLIES\ f= O(g).
\end{align*}

\end{solution}

\end{problemparts}
\end{problem}


%%%%%%%%%%%%%%%%%%%%%%%%%%%%%%%%%%%%%%%%%%%%%%%%%%%%%%%%%%%%%%%%%%%%%
% Problem ends here
%%%%%%%%%%%%%%%%%%%%%%%%%%%%%%%%%%%%%%%%%%%%%%%%%%%%%%%%%%%%%%%%%%%%%

\endinput
