\documentclass[problem]{mcs}

\begin{pcomments}
  \pcomment{PS_non_unique_factoring}
  \pcomment{by ARM 10/28/11}
\end{pcomments}

\pkeywords{
  prime
  factorization
  unique
}

\def\sqmf{\sqrt{-5}}
\def\zmf{\integers[\sqmf]}

%%%%%%%%%%%%%%%%%%%%%%%%%%%%%%%%%%%%%%%%%%%%%%%%%%%%%%%%%%%%%%%%%%%%%
% Problem starts here
%%%%%%%%%%%%%%%%%%%%%%%%%%%%%%%%%%%%%%%%%%%%%%%%%%%%%%%%%%%%%%%%%%%%%

\begin{problem}

\begin{definition*}
$\zmf \eqdef \set{m+n\sqmf \suchthat m,n \in \integers}$
\end{definition*}

Then $\zmf$ is closed under addition and multiplication.  Also,
\begin{equation}\label{3292}
3^2 = 9 = (2 + \sqmf) (2 - \sqmf).
\end{equation}

We claim that 3, $(2 + \sqmf)$ and $(2 - \sqmf)$ are all irreducike 

Let $r$ and $s$ be real numbers, $i \eqdef \sqrt{-1}$ and $c = r+si
\in \complexes$.  Then the norm $\abs{c}$ of $c$ is $\sqrt{r^2 + s^2}$.

\begin{lemma}\label{prodofnorms}
For $c,d \in \complexes$
\[
\abs{cd} = \abs{c} \abs{d}.
\]
\end{lemma}

\begin{proof}
For $c=r_c+s_c i$, $d=r_d+s_d i$,
\begin{align*}
\abs{c}\abs{d}
     & = \sqrt{r_c^2+s_c^2} \sqrt{r_d^2+s_d^2}\\
     & = \sqrt{(r_c^2+s_c^2)(r_d^2+s_d^2)}\\
     & = \sqrt{r_c^2r_d^2 + (r_c^2s_d^2 + r_d^2s_c^2) + s_c^2s_d^2}\\
     & = \sqrt{(r_cr_d)^2 + (r_cs_d)^2 + (r_ds_c)^2 + (s_cs_d)^2}\\
     & = \sqrt{((r_cr_d)^2 -2r_cr_ds_cs_d + (s_cs_d)^2) +
                (r_ds_c)^2 +2r_cr_ds_cs_d + (r_cs_d)^2}\\
     & = \sqrt{(r_cr_d - s_cs_d)^2 + (r_ds_c + r_cs_d)^2}\\
     & = \abs{(r_cr_d - s_cs_d) + (r_ds_c + r_cs_d)i}\\
     & = \abs{(r_c+s_c i) (r_d+s_d i)}\\
     & = \abs{cd}
\end{align*}
\end{proof}

\begin{lemma}\label{xz5}
If $x \in \zmf$ and $\abs{x}=1$, then $x = \pm 1$.
\end{lemma}

\begin{proof}
Say $x = m + n\sqmf$ for $m,n \integers$.  So $\abs{x} = m^2 +
5n^2$.  But $m^2 +5n^2 > 1$ if $n \neq 0$, so $\abs{x} = 1$ implies
$m^2=1$.  That is, $x = m = \pm 1$.
\end{proof}

\begin{lemma}\label{xy3pm1}
For $x,y \in \zmf$, if $xy =3$, then $x=\pm 1$ or $y= \pm 1$.

\begin{proof}
Let $x = m+n\sqmf$ for $m,n \in \integers$.
\begin{align*}
  xy = 3
     & \qimplies \abs{x} \abs{y} = \abs{3} = 3
        & \text{by Lemma~\ref{prodofnorms}}\\
     & \qimplies \abs{x}^2 \abs{y}^2 = 3^2\\
     & \qimplies \abs{x}^2 = 1 \QOR \abs{x}^2 = 3 \QOR \abs{x}^2 = 3^2
           & \text{since } \abs{x}^2, \abs{y}^2 \in \naturals\\
     & \qimplies \abs{x}^2 = 1 \QOR \abs{x}^2 = 3 \QOR \abs{y}^2 =1\\
     & \qimplies x = \pm 1 \QOR m^2+5n^2 = 3 \QOR y = \pm 1
         & \text{by Lemma~\ref{xz5}}\\
     & \qimplies x = \pm 1 \QOR y = \pm 1.
\end{align*}

\end{proof}

\end{lemma}

\begin{lemma*}
For $x,y \in \zmf$,
if $xy = 2 \pm \sqmf$, then $x=\pm 1$ or $y= \pm 1$.
\begin{align*}
  xy = 2 + \sqmf
     & \qimplies \abs{x} \abs{y} = \abs{2 + \sqmf} = 3
        & \text{by Lemma~\ref{prodofnorms}}\\
     & \qimplies \abs{x}^2 \abs{y}^2 = 3^2\\
     & \qimplies x = \pm 1 \QOR y = \pm 1
        & \text{as in the proof of Lemma~\ref{xy3pm1}}
\end{align*}
\end{lemma*}


\end{problem}

%%%%%%%%%%%%%%%%%%%%%%%%%%%%%%%%%%%%%%%%%%%%%%%%%%%%%%%%%%%%%%%%%%%%%
% Problem ends here
%%%%%%%%%%%%%%%%%%%%%%%%%%%%%%%%%%%%%%%%%%%%%%%%%%%%%%%%%%%%%%%%%%%%%


\endinput
