\documentclass[problem]{mcs}

\begin{pcomments}
  \pcomment{adapted from fall10 recitation 16.}
\end{pcomments}

\pkeywords{
  combinatorial proof
}

%%%%%%%%%%%%%%%%%%%%%%%%%%%%%%%%%%%%%%%%%%%%%%%%%%%%%%%%%%%%%%%%%%%%%
% Problem starts here
%%%%%%%%%%%%%%%%%%%%%%%%%%%%%%%%%%%%%%%%%%%%%%%%%%%%%%%%%%%%%%%%%%%%%

\begin{problem}
Give a combinatorial proof for this identity:
\[\sum_{i=0}^n \binom{k+i}{k} = \binom{k+n+1}{k+1}\]

\begin{solution}

We can prove it with a combinatorial approach:

\begin{proof}
We give a combinatorial proof.  Let $S$ be the set of all binary
sequences with exactly $n$ zeroes and $k + 1$ ones.

On the one hand, we know from a previous recitation that the number of
such sequences is equal to $\binom{k + n + 1}{k+1}$.

On the other hand, the number, $i$, of zeroes to the left of the
rightmost one ranges from 0 to $n$.  For a fixed value of $i$, there
are $\binom{k + i}{k}$ possible choices for the sequence of bits
before the rightmost one.  If we sum over all possible $i$, we find
that $|S| = \sum_{i = 0}^n \binom{k + i}{k}$.

Equating these two expressions for $\size{S}$ proves the theorem.
\end{proof}

\end{solution}
\end{problem}
\endinput
