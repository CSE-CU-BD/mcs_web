\documentclass[problem]{mcs}

\begin{pcomments}
    \pcomment{TP_inclusion-exclusion}
    \pcomment{Converted from inclusion-exclusion.scm
              by scmtotex and dmj
              on Sun 13 Jun 2010 03:52:26 PM EDT}
    \pcomment{revised by drewe 2 Aug 2011}
    \pcomment{ARM format 8/24/11}
\end{pcomments}

\begin{problem}
Let $A_{1}$, $A_{2}$, $A_{3}$ be sets with $\card{A_{1}} = 100$,
$\card{A_{2}} = 1,000$, and $\card{A_{3}} = 10,000$.

Determine $\card{A_{1} \union A_{2} \union A_{3}}$ in each of the
following cases:

%%  If there is insufficient data to determine the answer, 
%%  write \textbf{nd} for \textbf{n}ot \textbf{d}etermined. 
%%  
%%  Type in your answers as numbers \emph{without} commas.

\bparts

\ppart
$A_{1} \subset A_{2} \subset A_{3}$.

\begin{solution}

10000.

Since all the elements of $A_{1}$ are in $A_{2}$ and all elements in
$A_{2}$ are in $A_{3}$, the only elements in their union are the
elements of $A_{3}$.  Therefore, the size of the union is simply the
size of $A_{3}$.
\end{solution}

\ppart 
The sets are pairwise disjoint.

\begin{solution}
11100.

Since the sets are pairwise disjoint (i.e. $A_{1}$ shares no elements
with $A_{2}$ or $A_{3}$ and $A_{2}$ shares no elements with $A_{3}$),
the Sum Rule implies that the size of the union is the sum of the
sizes of the individual sets.
\end{solution}

\ppart 
For any two of the sets, there is exactly one element in both.

\begin{solution}
Undetermined.

The number depends on whether the element is common to all three, and
so we cannot determine the answer from the information given.  From
the viewpoint of the inclusion-exclusion formula, we don't know if the
value of $\card{A_{1} \intersect A_{2} \intersect A_{3}}$ is 0 or 1
\end{solution}

\ppart 
There are two elements common to each pair of sets and one element in
all three sets.

\begin{solution}
11095.

To solve this, we apply the inclusion-exclusion formula:
\begin{align*}
\lefteqn{\card{A_{1} \union A_{2} \union A_{3}}}\\
 & = \card{A_{1}} + \card{A_{2}} + \card{A_{3}}
    - \card{A_{1} \intersect A_{2}}
    - \card{A_{1} \intersect A_{3}}
    - \card{A_{2} \intersect A_{3}}
       + \card{A_{1} \intersect A_{2} \intersect A_{3}} \\
 & = 100+1000+10000-2-2-2+1 = 11095.
\end{align*}
\end{solution}

\eparts

\end{problem}

\endinput
