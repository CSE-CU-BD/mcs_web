\documentclass[problem]{mcs}

\begin{pcomments}
  \pcomment{CP_Sk_equiv_-1_mod_p}
  \pcomment{from: S09.cp8t, F04.ps3}
  \pcomment{commented out in S09 - proofread before using}
%  \pcomment{}
\end{pcomments}

\pkeywords{
  modular_arithmetic
  number_theory
  series
  prime
  Fermat
  Fermats_Theorem
}

%%%%%%%%%%%%%%%%%%%%%%%%%%%%%%%%%%%%%%%%%%%%%%%%%%%%%%%%%%%%%%%%%%%%%
% Problem starts here
%%%%%%%%%%%%%%%%%%%%%%%%%%%%%%%%%%%%%%%%%%%%%%%%%%%%%%%%%%%%%%%%%%%%%

%F04 ps3

\begin{problem}
  Let $S_k = 1^k + 2^k + \ldots + (p-1)^k$, where $p$ is an odd prime and
  $k$ is a positive multiple of $p - 1$.  Use \idx{Fermat's theorem} to
  prove that $S_k \equiv -1 \pmod{p}$.

\begin{solution}
Fermat's theorem says that $x^{p-1} \equiv 1 \pmod{p}$
when $1 \leq x \leq p - 1$.  Since $k$ is a multiple of $p-1$,
raising each side to a suitable power proves that $x^k \equiv 1
\pmod{p}$.  Thus:
%
\begin{align*}
1^k + 2^k + \ldots + (p-1)^k
    & \equiv \underbrace{1 + 1 + \ldots + 1}_{\text{$p-1$ terms}} \pmod{p} \\
    & \equiv p - 1 \pmod{p} \\
    & \equiv - 1 \pmod{p}
\end{align*}

\end{solution}

\end{problem}

%%%%%%%%%%%%%%%%%%%%%%%%%%%%%%%%%%%%%%%%%%%%%%%%%%%%%%%%%%%%%%%%%%%%%
% Problem ends here
%%%%%%%%%%%%%%%%%%%%%%%%%%%%%%%%%%%%%%%%%%%%%%%%%%%%%%%%%%%%%%%%%%%%%

\endinput
