\documentclass[problem]{mcs}
\begin{pcomments}
  \pcomment{MQ_coin_toss_game_asymmetric}
  \pcomment{new in spring 11}
  \pcomment{some of the text has references to other spring11 problems}
\end{pcomments}

\pkeywords{
  event_tree
  total_expectation
  total_probability
  expectation
}

\begin{problem}
Consider the following 2 player game.  A fair coin is tossed
repeatedly.  Turns alternate between the two players.  The game stops
after the first Heads come up.  If the first time the coin came up

Heads is during one of player 1's turns, player 1 wins.  On the other
hand, if the first time the coin came up Heads is during one of player
2's turns then player 2 wins.

\bparts

\iffalse

\ppart What is the probability $p$ the game ends. Hint: the game ends
when the first Heads comes up.
\examspace[2in]

\begin{solution}
The probability is 1. This is simply the probability that no heads come up ever.
\end{solution}
\fi

\ppart What is the expected number of turns $N$ until the game ends?
\examspace[2in]

\begin{solution}
This is just mean time to failure (a Head), so by
Lemma~\bref{lem:exp_time_to_fail}, the expected number of steps is
$\expect{N} = 1/(1/2) = 2$.
\end{solution}

\ppart What is the probability $p_1$ that player 1 wins? \hint draw an event tree.
\examspace[2in]

\begin{solution}
The tree can be described by $A = H_1 + T_1(H_2 + T_2A)$.  The
probability of winning can be found via the law of total probability.

$p_1 = (1/2)\cdot 1 +  (1/2)(1/2\cdot 0 + 1/2\cdot p_1)$

Hence $(3/4)\cdot p_1 = 1/2$, so $p_1 = 2/3$

\end{solution}

\ppart What is $\expcond{N}{1}$, the expected number $N$ of rounds in
the game given player 1 wins?  You can assume that the game ends with
probability 1 and that $\expcond{N}{2} = \expcond{N}{1} + 1$. \hint
Law of total Expectation.

\begin{solution}
From the law of total expectation, we know $\expect{N} =
\expcond{N}{1}p_1 + \expcond{N}{2}p_2$.  Now we know $p_1 = 2/3$, $p_2 =
1/3$ and $\expect{N} = 2$ and the hint.

We get $(2/3 +1/3)\expcond{N}{1} = 2 - 1/3$ so $\expcond{N}{1} = 5/3$.
\end{solution}

\eparts
\end{problem}


\endinput
