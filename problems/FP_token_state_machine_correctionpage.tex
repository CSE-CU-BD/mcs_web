\documentclass[problem]{mcs}

\begin{pcomments}
  \pcomment{FP_token_state_machine_correctionpage}
  \pcomment{F15.midterm2}
  \pcomment{author: Zoran Dzunic, edited ARM 10/15/15}
\end{pcomments}

\pkeywords{
  state_machine
  invariant
  induction
  congruence
}

%%%%%%%%%%%%%%%%%%%%%%%%%%%%%%%%%%%%%%%%%%%%%%%%%%%%%%%%%%%%%%%%%%%%%
% Problem starts here
%%%%%%%%%%%%%%%%%%%%%%%%%%%%%%%%%%%%%%%%%%%%%%%%%%%%%%%%%%%%%%%%%%%%%

\begin{problem}
\hspace{1in}

\bparts


\ppart \dots

\medskip

Now assume the game starts with a single black token, that is, the
start state is $(1,0)$.

\ppart  \dots

\ppart
Define the predicate $T(n_b,n_w)$ by the rule:
\[
T(n_b,n_w) \eqdef\  [n_w - n_b  \equiv 2 \pmod{3}].
\]

We will now prove the following:
\begin{claim*}
If $T(n_b, n_w)$, then state $(n_b, n_w)$ is reachable.
\end{claim*}

Note that this claim is different from the claim that $T$ is a
preserved invariant.

The proof of the Claim will be by induction in $n$ using induction
hypothesis $P(n) \eqdef$
\[
 \forall (n_b, n_w).\,
 [(n_b + n_w = n) \QAND  T(n_b,n_w)]
     \QIMPLIES (n_b,n_w) \text{ is reachable}.
\]

The base cases will be when $n \leq 2$.  

\begin{itemize}

\item Assuming that the base cases have been verified, complete the
  \textbf{Inductive Step}.

\begin{solution}
\begin{proof}
Assume that the induction hypothesis holds for some $n \ge 2$.
Suppose $n_b + n_w = n+1$ and $T(n_b,n_w)$ holds.  We want to show
that $(n_b, n_w)$ is reachable.

Since $n+1 \geq 3$, either $n_b \geq 2$ or $n_w \geq 2$.

In the case that $n_b \geq 2$, we have $n_b-2 \geq 0$, so
$(n_b-2,n_w+1)$ is a state. Also, $T(n_b-2,n_w+1)$ holds because
\[
(n_b-2)-(n_w+1) = n_b-n_w -3 \equiv n_b-n_w \equiv 2 \pmod{3}.
\]
Since
\[
(n_b-2)+(n_w+1) = n_b+n_w - 1 = n,
\]
we conclude by induction hypothesis $P(n)$ that $(n_b-2,n_w+1)$ is
reachable.  But $(n_b-2,n_w+1)$ transitions in one step to
$(n_b,n_w)$, which proves that $(n_b,n_w)$ is reachable.

The same argument applies in the case that $n_w \geq 2$.

We conclude that in any case $(n_b,n_w)$ is reachable, which completes
the induction step.
\end{proof}
\end{solution}

\item Now verify the \textbf{Base Cases}: $P(n)$ for $n \leq 2$.

\begin{solution}
There are only six states with $n \leq 2$:
[
(0,0, (1,0), (0,1), (1,1), (0,2), (2,0).
\]
Of these, only $T(1,0)$ and $T(0,2)$ hold, and $(1,0)$ is reachable
since it is the start state, and $(0,2)$ is reachable in one step from
the start state.  So all the states with $n \leq 2$ and satisfying
property $T$ are reachable.
\end{solution}

\end{itemize}
\eparts

\end{problem}

%%%%%%%%%%%%%%%%%%%%%%%%%%%%%%%%%%%%%%%%%%%%%%%%%%%%%%%%%%%%%%%%%%%%%
% Problem ends here
%%%%%%%%%%%%%%%%%%%%%%%%%%%%%%%%%%%%%%%%%%%%%%%%%%%%%%%%%%%%%%%%%%%%%

\endinput
