\documentclass[problem]{mcs}

\begin{pcomments}
  \pcomment{MQ_coloring}
\end{pcomments}

\pkeywords{
}

%%%%%%%%%%%%%%%%%%%%%%%%%%%%%%%%%%%%%%%%%%%%%%%%%%%%%%%%%%%%%%%%%%%%%
% Problem starts here
%%%%%%%%%%%%%%%%%%%%%%%%%%%%%%%%%%%%%%%%%%%%%%%%%%%%%%%%%%%%%%%%%%%%%

\begin{problem}

Consider the graph shown in Figure~\ref{fig:to_color}.  Determine a valid coloring of the graph, 
using as few colors as possible. (Simply write your proposed color for each 
vertex next to that vertex.  You may use $r$ for red, $g$ for green, etc.)

\begin{figure}[h]
\graphic{MQ_coloring}
\caption{\label{fig:to_color}}
\end{figure}
\examspace[1in]
\begin{solution}
There are odd-length cycles in the graph, so at least three colors will be needed.
So assume that three colors are sufficient.  (If we encounter a contradiction under this assumption, 
we will need to use more colors.)  Start with the length-3 cycle $abda$.  
All of its vertices must be colored differently, so assign red to $a$, blue to $b$, and green to $d$.  The 
length-3 cycle $bdhb$ now forces $h$ to be colored red.  $f$ must now be colored green and $g$ must be 
colored blue.  The coloring is valid so far.  $c$ is adjacent to a blue vertex and a green vertex, and no 
others, it must be colored red.  Finally, $e$ is not adjacent to any other vertices, so it can be assigned 
any of the three colors.  Choosing red for $e$, the result is shown in Figure~\ref{fig:colored}.  
There is no pair of like-colored adjacent vertices, so this coloring is valid.   
\begin{figure}[h]
\graphic{MQ_coloring_sol}
\caption{A valid coloring.\label{fig:colored}}
\end{figure}
\end{solution}

\end{problem}

\endinput
