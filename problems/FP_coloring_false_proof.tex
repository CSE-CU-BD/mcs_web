\documentclass[problem]{mcs}

\begin{pcomments}
\pcomment{FP_coloring_false_proof}
\pcomment{from: S08.final; F07.final}
\end{pcomments}

\pkeywords{
graph
coloring
induction
falseproof
vertex_degree
}

%%%%%%%%%%%%%%%%%%%%%%%%%%%%%%%%%%%%%%%%%%%%%%%%%%%%%%%%%%%%%%%%%%%%%
% Problem starts here
%%%%%%%%%%%%%%%%%%%%%%%%%%%%%%%%%%%%%%%%%%%%%%%%%%%%%%%%%%%%%%%%%%%%%

\begin{problem}

  Recall that a \term{coloring} of a graph is an assignment of a color to
  each vertex such that no two adjacent vertices have the same color.  A
  \term{$k$-coloring} is a coloring that uses at most $k$ colors.

\begin{falseclm*}
  Let $G$ be a graph whose vertex degrees are all $\leq k$.  If $G$
  has a vertex of degree strictly less than $k$, then $G$ is
  $k$-colorable.
\end{falseclm*}

\bparts \ppart[2] \label{counterexample} Give a counterexample to the
False Claim when $k=2$.

\examspace[1in]

\begin{solution}
One node by itself, and a separate triangle
($K_3$).  The graph has max degree 2, and a node of degree zero, but is not
2-colorable.
\end{solution}

\ppart[4] Underline the exact sentence or part of a sentence that is
the first unjustified step in the following ``proof'' of the False
Claim.

%goes wrong:

\begin{falseproof}

Proof by induction on the number $n$ of vertices:

\textbf{Induction hypothesis}:

$P(n) \eqdef$ ``Let $G$ be an $n$-vertex graph whose vertex degrees are
all $\leq k$.  If $G$ also has a vertex of degree strictly less than $k$,
then $G$ is $k$-colorable.''

\textbf{Base case}: ($n=1$) $G$ has one vertex, the degree of which is 0.
Since $G$ is 1-colorable, $P(1)$ holds.

\textbf{Inductive step}:

We may assume $P(n)$.  To prove $P(n+1)$, let $G_{n+1}$ be a graph with
$n+1$ vertices whose vertex degrees are all $k$ or less.  Also, suppose
$G_{n+1}$ has a vertex, $v$, of degree strictly less than $k$.  Now we
only need to prove that $G_{n+1}$ is $k$-colorable.

To do this, first remove the vertex $v$ to produce a graph, $G_n$, with
$n$ vertices.  Let $u$ be a vertex that is adjacent to $v$ in $G_{n+1}$.
Removing $v$ reduces the degree of $u$ by 1.  So in $G_n$, vertex $u$ has
degree strictly less than $k$.  Since no edges were added, the vertex
degrees of $G_n$ remain $\leq k$.  So $G_n$ satisfies the conditions of
the induction hypothesis, $P(n)$, and so we conclude that $G_n$ is
$k$-colorable.

Now a $k$-coloring of $G_n$ gives a coloring of all the vertices of
$G_{n+1}$, except for $v$.  Since $v$ has degree less than $k$, there will
be fewer than $k$ colors assigned to the nodes adjacent to $v$.  So among
the $k$ possible colors, there will be a color not used to color these
adjacent nodes, and this color can be assigned to $v$ to form a
$k$-coloring of $G_{n+1}$.
\end{falseproof}

\examspace[1in]
\begin{solution}
 The flaw is that if $v$ has degree 0, then no
  such $u$ exists.  In such a case, removing $v$ will not reduce the
  degree of any vertex, and so there may not be any vertex of degree less
  than $k$ in $G_n$, as in the counterexample of
  part~\eqref{counterexample}.

  So the mistaken sentence is ``Let $u$ be a vertex that is adjacent to
  $v$ in $G_{n+1}$.''

  Alternatively, you could say that it's OK to reason about a nonexistent
  $u$, and the only mistake is the claim that $u$ exists.  This claim is
  hidden in the phrase ``So $G_n$ satisfies the conditions of the
  induction hypothesis, $P(n)$''.
\end{solution}

\examspace
\ppart[4]  With a slightly strengthened condition, the preceding proof
of the False Claim could be revised into a sound proof of the following
Claim:

\begin{claim*}
  Let $G$ be a graph whose vertex degrees are all $\leq k$.  If
  $\ang{\emph{statement inserted from below}}$ has a vertex of degree
  strictly less than $k$, then $G$ is $k$-colorable.
\end{claim*}
Circle each of the statements below that could be inserted to make the
Claim true.

\begin{itemize}

\item $G$ is connected and

\item $G$ has no vertex of degree zero and

\item $G$ does not contain a complete graph on $k$ vertices and

\item every connected component of $G$

\item some connected component of $G$

\end{itemize}

\begin{solution}
Either the first statement ``$G$ is connected and'' or the fourth
statement ``every connected component of $G$'' will work.
\end{solution}

\eparts
\end{problem}

\endinput
