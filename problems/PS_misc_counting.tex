\documentclass[problem]{mcs}

\begin{pcomments}
\pcomment{PS_misc_counting}
\pcomment{from: S07.ps9}
\end{pcomments}

\pkeywords{
  counting
  counting_rules
  factorials
  binomial_coefficients
}

%%%%%%%%%%%%%%%%%%%%%%%%%%%%%%%%%%%%%%%%%%%%%%%%%%%%%%%%%%%%%%%%%%%%%
% Problem starts here
%%%%%%%%%%%%%%%%%%%%%%%%%%%%%%%%%%%%%%%%%%%%%%%%%%%%%%%%%%%%%%%%%%%%%

\begin{problem}

Answer the following questions with a number or a simple formula involving
factorials and binomial coefficients.  Briefly explain your answers.

\bparts

\ppart How many ways are there to order the 26 letters of the
alphabet so that no two of the vowels {\tt a}, {\tt e}, {\tt i},
{\tt o}, {\tt u} appear consecutively and the last letter in the
ordering is not a vowel?

\hint  Every vowel appears to the left of a consonant.

\begin{solution}
The constraint on where vowels can appear is equivalent to the
requirement that every vowel appears to the left of a consonant.  So given
a sequence of the 21 consonants, there are $\binom{21}{5}$ positions where
the 5 vowels can be placed.  After determining such a placement, we can
reorder the consonants and vowels in any order.  Thus, the number is:
\[
\binom{21}{5}\cdot 21! \cdot 5!.
\]
\end{solution}

\ppart How many ways are there to order the 26 letters of the alphabet
so that there are {\em at least two} consonants immediately following
each vowel?

\begin{solution}
The pattern of consonants and vowels in any permutation of the
26 letters of the alphabet can be indicated by a binary string with 5
ones indicating where the vowels occur and 21 zeros where the consonants
occur.  Patterns where every vowel has at least two consonants to its
right can be constructed by taking a sequence of 16 zeros and inserting
``10'' to the left of 5 of the 16 zeros.  There are $\binom{16}{5}$ ways to
do this.  For any such pattern, there are $5!$ ways to place the vowels
in the positions where ones occur and $21!$ ways to place the consonants
where the ones occur.  Thus, the final answer is:
\[
\binom{16}{5}\cdot 5! \cdot 21!.
\]
\end{solution}

\ppart In how many different ways can the letters in the name of the
popular 1980's band \texttt{BANANARAMA} be arranged?

\begin{solution}
There are 5 $A$'s, 2 $N$'s, 1 $B$, 1 $R$, and 1 $M$.
Therefore, the number of arrangements is
%
\[
\frac{10!}{5!\ 2!\ 1!\ 1!\ 1!}
\]
%
by the Bookkeeper Rule.
\end{solution}


\ppart In how many different ways can $2n$ students be paired
up?

\begin{solution}
Pair up students by the following procedure.  Line up the
students and pair the first and second, the third and fourth, the
fifth and sixth, etc.  The students can be lined up in $(2n)!$ ways.
However, this overcounts by a factor of $2^n$, because we would get
the same pairing if the first and second students were swapped, the
third and fourth were swapped, etc.  Furthermore, we are still
overcounting by a factor of $n!$, because we would get the same
pairing even if pairs of students were permuted, e.g. the first and
second were swapped with the ninth and tenth.  Therefore, the number
of pairings is:
%
\[
\frac{(2n)!}{2^n \cdot n!}
\]
\end{solution}

\ppart How many different solutions over the natural numbers are
there to the following equation?
%
\[
x_1 + x_2 + x_3 + \ldots + x_{8} = 100
\]
%
A solution is a specification of the value of each variable $x_i$.
Two solutions are different if different values are specified for some
variable $x_i$.

\begin{solution}
There is a bijection between sequences containing 100 zeros
and 7 ones.  Specifically, the 7 ones divide the zeros into 8
segments.  Let $x_i$ be the number of zeros in the $i$-th segment.
Therefore, the number of solutions is:
%
\[
\binom{100+7}{7}
\]
\end{solution}

\ppart How many simple graphs are there with $n$ vertices numbered $1,
\dots n$?

\begin{solution}
There are $\binom{n}{2}$ potential edges, each of which may or
may not appear in a given graph.  Therefore, the number of graphs is:
\[
2^{\binom{n}{2}}
\]
\end{solution}

\eparts

\end{problem}

\endinput