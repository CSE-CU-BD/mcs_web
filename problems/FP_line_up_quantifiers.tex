\documentclass[problem]{mcs}

\begin{pcomments}
  \pcomment{FP_line_up_quantifiers}
  \pcomment{by ARM 12/23/11 from FP_}
\end{pcomments}

\pkeywords{
 predicate_calculus
 quantifier
}

%%%%%%%%%%%%%%%%%%%%%%%%%%%%%%%%%%%%%%%%%%%%%%%%%%%%%%%%%%%%%%%%%%%%%
% Problem starts here
%%%%%%%%%%%%%%%%%%%%%%%%%%%%%%%%%%%%%%%%%%%%%%%%%%%%%%%%%%%%%%%%%%%%%

\begin{problem}
Some students from a large class will be lined up left to right.
There will be at least two students in the line.  Translate each of
the following assertions into predicate formulas with the set of
students in the class as the domain of discourse.  The only predicates
you may use are
\begin{itemize}
\item equality and,
\item $F(x,y)$, meaning that ``$x$ is somewhere to the left of $y$ in the line.''
For example, in the line ``CDA'', both $F(C,A)$ and $F(C,D)$ are true.
\end{itemize}
Once you have defined a formula for a predicate $P$ you may use the
abbreviation ``$P$'' in further formulas.

\bparts

\ppart Student $x$ is in the line.

\inhandout{
\begin{center}
\exambox{3in}{0.5in}{0.3in}
\end{center}
}

\begin{solution}
  \[\text{inline}(x) \eqdef \exists y.\, F(y,x) \QOR F(x,y) \]
\end{solution}

\ppart
Student $x$ is first in line.

\inhandout{\begin{center}
\exambox{3in}{0.5in}{0.3in}
\end{center}
}

\begin{solution}
\[\text{first}(x) \eqdef \text{inline}(x) \QAND \forall y.\, \QNOT F(y,x) \]
\end{solution}

\ppart
Student $x$ is immediately to the right of student $y$.

\inhandout{
\begin{center}
\exambox{3in}{0.5in}{0.3in}
\end{center}
}

\begin{solution}
  \[\text{isnext}(x,y) \eqdef F(y,x) \QAND \forall z.\, \QNOT(F(y,z) \QAND F(z,x))
  \]
\end{solution}

\ppart
Student $x$ is second.

\begin{center}
\exambox{3in}{0.5in}{0.3in}
\end{center}

\begin{solution}
  \[
 \exists y. \text{first}(y) \QAND \text{isnext}(x,y)
 \]
\end{solution}

\eparts
\end{problem}

\endinput
