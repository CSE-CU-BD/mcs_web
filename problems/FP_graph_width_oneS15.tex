\documentclass[problem]{mcs}

\begin{pcomments}
  \pcomment{FP_graph_width_oneS15}
  \pcomment{part (a) of FP_graph_width_one}
\end{pcomments}

\pkeywords{
 graph theory
 colorable
 width
}

\begin{problem}
A simple graph $G$ is said to have \emph{width} $1$ iff there is a
way to list all its vertices so that each vertex is adjacent to at
most one vertex that appears earlier in the list.

%\bparts

Prove that every finite graph with width one is a forest.

\hint By induction, removing the last vertex.

\examspace[4in]

\begin{solution}

\begin{proof}
By induction on the number of vertices $n$.  The induction hypothesis
is
\begin{center}
$P(n) \eqdef$ all $n$-vertex graphs $G$ with width one are forests.
\end{center}

\inductioncase{Base case}: ($n=1$).  A graph with one vertex is
acyclic and therefore is a forest.

\inductioncase{Induction step}.  Assume that $P(n)$ is true for some
$n \geq 1$ and let $G$ be an $(n+1)$-vertex graph with width one.  We
need only show that $G$ is acyclic.

The vertices of $G$ can be listed with each vertex adjacent to at most
one vertex earlier in the list.  Let $v$ be the last vertex in the
list.  Since \emph{all} the vertices adjacent to $v$ appear earlier in
the list, it follows that $\degr{v} \leq 1$.

Now removing a vertex won't increase width, so $G-v$ still has width
one, so it is acyclic by Induction Hypothesis.  But no degree-one
vertex is in a cycle, so adding $v$ back to $G-v$ will not create a
cycle.  Hence $G$ is acyclic, as claimed.

This proves $P(n+1)$ and completes the induction step.

\end{proof}

\end{solution}

\end{problem}
\endinput
