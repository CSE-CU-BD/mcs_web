\documentclass[problem]{mcs}

\begin{pcomments}
  \pcomment{FP_connected_induction}
  \pcomment{S01.final}
  \pcomment{added, edited ARM 5/20/17}
\end{pcomments}

\pkeywords{
  connect
  connected
}

%%%%%%%%%%%%%%%%%%%%%%%%%%%%%%%%%%%%%%%%%%%%%%%%%%%%%%%%%%%%%%%%%%%%%
% Problem starts here
%%%%%%%%%%%%%%%%%%%%%%%%%%%%%%%%%%%%%%%%%%%%%%%%%%%%%%%%%%%%%%%%%%%%%

\begin{problem}
MIT Information Services \& Technology (IS\&T) wants to assemble a
computer cluster of $n$ computers connected by wires and hubs so that
every pair of computers are connected by a path of wires and hubs.
Each computer must have exactly one wire attached to it, while each
hub can have up to five wires attached.  IS\%T wants to minimize the
number of hubs used to build this network,

\bparts

\ppart\label{ISTMmax} Suppose IS\&T uses $m$ hubs.  Write a simple
formula in terms of $m$ and $n$ for the maximum number of wires that
can be attached to the hubs and computers without leaving dangling
ends

\begin{center}
\exambox{1.2in}{0.4in}{-0.2in}
\end{center}

%\examspace[0.4in]

\begin{solution}
\[
\frac{5m+n}{2}\, .
\]

The sum of the hub degrees is $5m$ and the computer degrees is $n$.
The number of edges is half the degree sum.
\end{solution}

\ppart\label{wire-tree} Suppose we had hibs available with no
restriction on the number of wires that may be attached.  Write a
simple formula in terms of $m$ and $n$ for the for the minimum number
of wires needed to connect all the computers and hubs to each other so
that there is a path along the wires from any computer to all the hubs
and other computers.

\begin{center}
\exambox{1.2in}{0.4in}{-0.2in}
\end{center}

%\examspace[0.4in]

\begin{solution}
\[
m + n -1\, .
\]

This is the number of edges in a tree with the $n$ computers and $m$
hubs as vertices.
\end{solution}

\inbook{\ppart Compare the previous two quantities to prove that at
least $\ceil{(n-2)/3}$ hubs are needed to hook up the cluster.

\examspace[2.0in]

\begin{solution}
We want to minimize $m$ subject to the constraint
\[
m+n-1 \leq (5m+n)/2.
\]
This will happen when these two terms are equal:
\begin{align*}
m+n-1       & = (5m+n)/2,\\
n - 1 - n/2 & = 5m/2 -m,\\
n/2 - 1     & = 3m/2,\\
n-2         & = 3m,\\
\frac{n-2}{3} & = m.
\end{align*}
But $m$ must be an integer, so we get
\[
m = \ceil{\frac{n-2}{3}}.
\]
\end{solution}
}

\ppart Prove by induction that $\ceil{(n-2)/3}$ hubs are sufficient to
build the cluster of $n$ computers..

\examspace[4.0in]

\begin{solution}

\begin{proof}
The proof will be by strong induction on $n$ with hypothesis
\[
P(n) \eqdef n \text{ computers can be clustered using } \ceil{\frac{n-2}{3}} \text{ degree-5 hubs}.
\]

\inductioncase{Base Cases}.
($n = 1,2$): Zero hubs are needed, and $0 = \ceil{(1-2)/3} = \ceil{(2-2)/3}$.

($n = 3$): The three computer can be connected to one hub, and $1 = \ceil{(3-2)/3}$.

\inductioncase{Induction step}.  We must prove $P(n+1)$ assuming $P(k)$ for $0 \leq
k \leq n$, where $n\geq 3$.  In particular, we may assume $P(n-2)$.  That is, there is a
cluster $C$ of $n-2\geq 1$ computers using at most
\[
\ceil{\frac{(n-2)-2}{3}} = \ceil{\frac{(n+1)-2}{3}}-1
\]
hubs.  If we can rearrange the cluster $C$ to include one more hub and
three more computers, we will have a cluster with $n+1$ computers and
$\ceil{((n+1)-2)/3}$ hubs, thereby proving $P(n+1)$.

To rearrange $C$ in this way, we replace the computer $R$ at the end
of some wire with a new hub $H$.  This leaves room to attach four more
wires to $H$, allowing us to attach $R$ and three new computers to the
ends of these wires.
\end{proof}

\inhandout{Note that equating the formulas from parts~\eqref{ISTMmax}
  and~\eqref{wire-tree} and solving for $m$ proves that
  $\ceil{(n-2)/3}$ hubs are also necessary to form a cluster.  }

\end{solution}

\eparts

\end{problem}

\endinput
