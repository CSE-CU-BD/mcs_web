\documentclass[problem]{mcs}

\begin{pcomments}
  \pcomment{from: S02.ps5}
\end{pcomments}

\pkeywords{
  state_machines
}

%%%%%%%%%%%%%%%%%%%%%%%%%%%%%%%%%%%%%%%%%%%%%%%%%%%%%%%%%%%%%%%%%%%%%
% Problem starts here
%%%%%%%%%%%%%%%%%%%%%%%%%%%%%%%%%%%%%%%%%%%%%%%%%%%%%%%%%%%%%%%%%%%%%


\begin{problem} \textbf{Provide recursive definitions of the following sets and functions:}

\bparts

\ppart The set $S = \{ 2^a 3^b 5^c \mid a, b, c \in \mathbb{N}
\}$.

\begin{solution}
 We can define the set $S$ recursively as follows:

\begin{enumerate}
\item $1 \in S$
\item If $n \in S$, then $2n$, $3n$ and $5n$ are in $S$.
\end{enumerate}
\end{solution}

\ppart The set $L = \{ (a, b) \in \mathbb{Z} \times \mathbb{Z}
\mid \text{$a + 2b = 3k$ for some $k \in \mathbb{Z}$} \}$.

\begin{solution}
\begin{enumerate}
\item $(0, 0) \in L$
\item If $(a, b) \in L$, then $(a + 3, b)$, $(a - 3, b)$, $(a + 2, b -
1)$, and $(a - 2, b + 1)$ are in $L$.
\end{enumerate}
\end{solution}

\ppart The set $\Sigma^*$, which consists of all finite strings
of symbols drawn from the set $\{ p, n, d, q \}$.  (We'll use this set
$\Sigma^*$ in the next three problem parts as well.)

\begin{solution}
\begin{enumerate}
\item The empty string $\emptystring$ is in $\Sigma^*$.
\item If $\alpha \in \Sigma^*$, then $p\alpha$, $n\alpha$, $d\alpha$,
$q\alpha$ are in $\Sigma^*$.
\end{enumerate}
\end{solution}


\ppart Recursively define a function $l : \Sigma^* \to
\mathbb{N}$ that maps each string in $\Sigma^*$ to its length.

\begin{solution}
\begin{enumerate}
\item $l(\emptystring) = 0$.
\item If $\alpha \in \Sigma^*$, and $\beta \in \{p, n, d, q\}$, then
$l(\beta \alpha) = 1 + l(\alpha)$.
\end{enumerate}
\end{solution}

\ppart Suppose that $p$ stands for penny, $n$ stands for nickel,
$d$ stands for dime, and $q$ stands for quarter.  Recursively define a
function $m : \Sigma^* \to \mathbb{N}$ that maps each string in
$\Sigma^*$ to the monetary value of the corresponding pile of change.

\begin{solution}
\begin{enumerate}
\item $m(\emptystring) = 0$.
\item If $\alpha \in \Sigma^*$, then $m(p \alpha) = 1 + m(\alpha)$.
\item If $\alpha \in \Sigma^*$, then $m(n \alpha) = 5 + m(\alpha)$.
\item If $\alpha \in \Sigma^*$, then $m(d \alpha) = 10 + m(\alpha)$.
\item If $\alpha \in \Sigma^*$, then $m(q \alpha) = 25 + m(\alpha)$.
\end{enumerate}
\end{solution}

\ppart Recursively define a function $r : \Sigma^* \to \Sigma^*$
that maps each string $\sigma_1 \sigma_2 \ldots \sigma_{n}$ in
$\Sigma^*$ to its reverse:

\[
\sigma_n \sigma_{n-1} \ldots \sigma_1
\]

\begin{solution}
\begin{enumerate}
\item $r(\emptystring) = \emptystring$.
\item If $\alpha \in \Sigma^*$ and $\beta \in \{p, n, d, q\}$, then
$r(\beta \alpha) = r(\alpha) \beta$.
\end{enumerate}
\end{solution}

\eparts
\end{problem}

%%%%%%%%%%%%%%%%%%%%%%%%%%%%%%%%%%%%%%%%%%%%%%%%%%%%%%%%%%%%%%%%%%%%%
% Problem ends here
%%%%%%%%%%%%%%%%%%%%%%%%%%%%%%%%%%%%%%%%%%%%%%%%%%%%%%%%%%%%%%%%%%%%%
