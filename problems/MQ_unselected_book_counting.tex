\documentclass[problem]{mcs}

\begin{pcomments}
  \pcomment{MQ_unselected_book_counting}
  \pcomment{from: S09.mq5} edited/shortened by Tom Brown 11/14/09}
\end{pcomments}

\pkeywords{
  counting
  bijection
}

%%%%%%%%%%%%%%%%%%%%%%%%%%%%%%%%%%%%%%%%%%%%%%%%%%%%%%%%%%%%%%%%%%%%%
% Problem starts here
%%%%%%%%%%%%%%%%%%%%%%%%%%%%%%%%%%%%%%%%%%%%%%%%%%%%%%%%%%%%%%%%%%%%%

\begin{problem}

Suppose you want to select $k$ out of $n$ books on a shelf so that
there are always at least 3 unselected books between selected books.
Describe a  bijection between book selection and bit strings of length L 
containing exactly M 1s, so that counting the number of all such bit strings 
gives us the number of book selections. Find L and M and briefly explain why it works.

 
(Assume $n$ is large enough for this to be possible.)


\begin{solution}
%\[
%\binom{n-3(k-1)}{k}
%\]
%for $n \geq 4k-3$.
\[
L = n-3(k-1),\ M = k
\]

A selection of $k$ among $n$ books on a shelf corresponds in
  an obvious way to an $n$-bit string with exactly $k$ \texttt{1}'s.

  So the problem reduces to finding a bijection between $n$-bit strings
  with $k$ \texttt{1}'s that are at least 3 apart, and $(n-3(k-1))$-bit
  strings with $k$ \texttt{1}'s.

  But in a string, $s$, with $k$ \texttt{1}'s that are at least 3 apart,
  all but the last \texttt{1} must have a \texttt{000} to its right.  So we
  can map $s$ to a string with $k$ \texttt{1}'s and $3(k-1)$
  fewer \texttt{0}'s by erasing the \texttt{000}'s immediately to the right
  of each of the first $k-1$ \texttt{1}'s.

  This map is a bijection because given any $(n-3(k-1))$-bit string with
  $k$ \texttt{1}'s, there is a unique $n$-bit string with \texttt{1}'s that
  are at least 3 apart that maps to it, namely, the string obtained by
  replacing each of the first $k-1$ \texttt{1}'s in the $(n-3(k-1))$-bit
  string by a \texttt{1000}.
\end{solution}

\end{problem}

%%%%%%%%%%%%%%%%%%%%%%%%%%%%%%%%%%%%%%%%%%%%%%%%%%%%%%%%%%%%%%%%%%%%%
% Problem ends here
%%%%%%%%%%%%%%%%%%%%%%%%%%%%%%%%%%%%%%%%%%%%%%%%%%%%%%%%%%%%%%%%%%%%%

\endinput
