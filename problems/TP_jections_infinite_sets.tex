\documentclass[problem]{mcs}

\begin{pcomments}
  \pcomment{TP_jections_infinite_sets}
  \pcomment{CH, S14}
  \pcomment{mashup of smaller problems}
\end{pcomments}

\pkeywords{
  composition
  surjection
  injection
  bijection
  infinite sets
}

%%%%%%%%%%%%%%%%%%%%%%%%%%%%%%%%%%%%%%%%%%%%%%%%%%%%%%%%%%%%%%%%%%%%%
% Problem starts here
%%%%%%%%%%%%%%%%%%%%%%%%%%%%%%%%%%%%%%%%%%%%%%%%%%%%%%%%%%%%%%%%%%%%%

\begin{problem}

\bparts

Let $f: A \to B$ and $g: B \to C$ be functions. In Problem Set 3, we have seen that if the composition $g \compose f$ is a bijection,
then $f$ has to be a total injection and $g$ is a surjection.

\ppart 
Give an example demonstrating that $f$ need not be a surjection.

\begin{solution}

\begin{proof}

A simple way to construct the example is to make $B$ somewhat ``larger'' than $A$ in the
following way: Let $B$ have an element {\em not} in the range of $f$,
and having no $g$-arrow out. Here, $f$ is not a surjection but the composite function $g \compose
f$ is left unchanged.

\end{proof}

\end{solution}

\examspace[1.5in]

\ppart Suppose that we relax $g$ to be an arbitrary relation $G$. Show an 
example demonstrating that $G$ need not be
total, or even a function.

\begin{solution}

\begin{proof}

Assume $g$ to be a more general relation $G$ consisting of a bunch of arrows between $B$ and $C$.
As in the above proof, the key idea is to assume that $B$ is
``larger'' than $A$.  Let $B$ have an element not in the range of $f$
with no $g$-arrow out. Therefore, $g$ is not total.

Further, let another element in $B$ not in the range of
$f$ have {\em two} $G$-arrows out. Therefore $g$ is not a function.

\end{proof}

Basically, this example demonstrates that for $g \compose f$ to be a bijection, all that matters is that $g$
behaves ``nicely'' over the range of $f$.

\end{solution}

\examspace[1.5in]

\ppart Prove that if $A$ and $B$ are countable sets, then so is $A
\union B$.

\begin{solution}

\begin{proof}
Suppose we list the elements of $A$ as $a_0,a_1,\dots$ and the
elements of $B$ as $b_0,b_1, \dots$.  Then a list of all the elements in $A
\union B$ can be written as
\begin{equation}\label{FP_a0b0list}
a_0,b_0,a_1,b_1, \dots a_n,b_n, \dots.
\end{equation}
Of course, this list can potentially contain duplicates if $A$ and $B$ have elements
in common, but then deleting all but the first occurrences of each element in
list~\eqref{FP_a0b0list} leaves a list of all the {\em distinct} elements of
$A$ and $B$, which is the same as the list of elements in
$A \union B$.
\end{proof}

\end{solution}

\eparts

\end{problem}

%%%%%%%%%%%%%%%%%%%%%%%%%%%%%%%%%%%%%%%%%%%%%%%%%%%%%%%%%%%%%%%%%%%%%
% Problem ends here
%%%%%%%%%%%%%%%%%%%%%%%%%%%%%%%%%%%%%%%%%%%%%%%%%%%%%%%%%%%%%%%%%%%%%

\endinput
