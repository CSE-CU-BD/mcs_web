\documentclass[problem]{mcs}

\begin{pcomments}
  \pcomment{TP_jections_infinite_sets}
  \pcomment{CH, S14}
  \pcomment{mashup of smaller problems}
\end{pcomments}

\pkeywords{
  composition
  surjection
  injection
  bijection
  total
  function
  infinite
}

%%%%%%%%%%%%%%%%%%%%%%%%%%%%%%%%%%%%%%%%%%%%%%%%%%%%%%%%%%%%%%%%%%%%%
% Problem starts here
%%%%%%%%%%%%%%%%%%%%%%%%%%%%%%%%%%%%%%%%%%%%%%%%%%%%%%%%%%%%%%%%%%%%%

\begin{problem}
Let $R: A \to B$ and $S: B \to C$ be binary relations such that $S
\compose R$ is a bijection and $\card{A} = 2$.

\bparts

\ppart Give an example of such $R,S$ with where neither $R$ nor $S$ is
a bijection.  Indicate exactly which properties---total, surjection,
function, and injection---your example of $R$ and your example of $S$
does \textbf{not} have.

\hint Let $\card{B}=4$.

\begin{solution}
It is easy to see that $R$ must be total ($[\geq 1\ \text{out}]$) and
$S$ must be a surjection ($[\ge 1\ \text{in}]$).

$C$ must have 2 elements since $A \bij C$.

For the example with $R, S$ are not bijections, let $\card{B} = 4$,
and let there be an $R$-arrow from the first element of $A$ to the first
element of $B$ and an $S$-arrow from this first element of $B$ to the
element of $C$; likewise for the second elements of $A,B,C$.

Let the third element in $B$ have no $R$-arrow in and two $S$-arrows
out; so $R$ is not total and $S$ is not a function or an injection.
Let the fourth element of $B$ have two $R$-arrows in and no $S$-arrows
out; so $R$ is not a function or an injection and $S$ is not total.

So neither $R$ nor $S$ need have any additional ``jection''
properties. \iffalse
  That is,
\begin{itemize}

\item $R$ is not a function ($[\leq 1\ \text{out}]$), surjection,
  or injection ($[\leq 1\ \text{in}]$), and

\item $S$ need not be a function, total, or injection.

\end{itemize}
\fi

\end{solution}

\examspace[1.5in]

\iffalse

\ppart Prove that if $A$ and $B$ are countable sets, then so is $A
\union B$.

\begin{solution}

\begin{proof}
Suppose we list the elements of $A$ as $a_0,a_1,\dots$ and the
elements of $B$ as $b_0,b_1, \dots$.  Then a list of all the elements in $A
\union B$ can be written as
\begin{equation}\label{FP_a0b0list}
a_0,b_0,a_1,b_1, \dots a_n,b_n, \dots.
\end{equation}
Of course, this list can potentially contain duplicates if $A$ and $B$ have elements
in common, but then deleting all but the first occurrences of each element in
list~\eqref{FP_a0b0list} leaves a list of all the {\em distinct} elements of
$A$ and $B$, which is the same as the list of elements in
$A \union B$.
\end{proof}

\end{solution}

\examspace[1.5in]
\fi

\ppart Give an example of two sets $A,B$, such that
\[
\naturals \strict A \strict B.
\]

\iffalse
(Reminder: $A \strict B$ means there is no surjective function from
$A$ to $B$.)
\fi

\begin{solution}
Familiar values for $A$ are $\reals, \power(\naturals),
\set{0,1}^{\omega}$, and corresponding values for $B$ would be
$\power(\reals), \power(\power(\naturals))$.
\end{solution}

\examspace[0.8in]

\eparts

\end{problem}

%%%%%%%%%%%%%%%%%%%%%%%%%%%%%%%%%%%%%%%%%%%%%%%%%%%%%%%%%%%%%%%%%%%%%
% Problem ends here
%%%%%%%%%%%%%%%%%%%%%%%%%%%%%%%%%%%%%%%%%%%%%%%%%%%%%%%%%%%%%%%%%%%%%

\endinput
