\documentclass[problem]{mcs}

\begin{pcomments}
  \pcomment{FP_partial_order_short_answer}
  \pcomment{slight overlap with FP_partial_order_short_answer_S12}
  \pcomment{based on CP_partial_orders}
  \pcomment{edited, last part added by 10/15/15 ARM}
\end{pcomments}

\pkeywords{
  partial_orders
  weak_partial_order  
  strict_partial_order
  transitive 
  asymmetric
  path_total
}

%%%%%%%%%%%%%%%%%%%%%%%%%%%%%%%%%%%%%%%%%%%%%%%%%%%%%%%%%%%%%%%%%%%%%
% Problem starts here
%%%%%%%%%%%%%%%%%%%%%%%%%%%%%%%%%%%%%%%%%%%%%%%%%%%%%%%%%%%%%%%%%%%%%

\begin{problem}

\bparts

\ppart Indicate which of the following relations below are equivalence
relations, (\textbf{Eq}), strict partial orders (\textbf{SPO}), weak
partial orders (\textbf{WPO}).  For the partial orders, also indicate
whether it is \emph{linear} (\textbf{Lin}).

If a relation is none of the above, indicate whether it is
\emph{transitive} (\textbf{Tr}), \emph{symmetric} (\textbf{Sym}), or
\emph{asymmetric} (\textbf{Asym}).

\begin{enumerate}
\item The relation $a = b + 1$ between integers, $a$, $b$,
\hfill \examrule[0.7in]

\begin{solution}
\textbf{Asym}.

Not transitive since $4 = 3 + 1, 3 = 2 + 1,$ but $4 \neq 2 + 1$.
\end{solution}

\item The superset relation, $\supseteq$ on the power set of the integers.
\hfill \examrule[0.7in]

\begin{solution}
\textbf{WPO}.

Not linear: for example, if $A \neq B$ are two finite sets of the same
size, then neither $A \supseteq B$, nor $B \supseteq A$.
\end{solution}

\iffalse

\item The relation $\expect{R} < \expect{S}$ between real-valued
  random variables $R,S$.
\hfill \examrule[0.7in]

\begin{solution}
\textbf{Tr, Asym}
\end{solution}
\fi

\item The \idx{empty relation} on the set of rationals.
\hfill \examrule[0.7in]

\begin{solution}
\textbf{SPO}.

Not linear, since no rational is related to anything, including
itself.
\end{solution}

\item The divides relation on the nonegative integers, $\naturals$.  \hfill \examrule[0.7in]

\begin{solution}
\textbf{WPO}

Not linear since, for example, neither $2 \divides 3$ nor $3 divides
2$.
\end{solution}

\item The divides relation on all the integers, $\integers$.  \hfill
\examrule[0.7in]

\begin{solution}
\textbf{Tr}.

Not a \textbf{WPO} since $3 \divides -3$ and $-3 \divides 3$, but $3
\neq -3$.
\end{solution}

\item The divides relation on the positive powers of 4.  \hfill \examrule[0.7in]

\begin{solution}
\textbf{WPO, Lin}
\end{solution}

\item The relatively prime relation on the nonnegative integers.  \hfill \examrule[0.7in]

\begin{solution}
\textbf{Sym}

Not transitive, since $2$ and $3$ are relatively prime, $3$ and $4$
are relatively prime, but $2$ and $4$ are not relatively prime.
\end{solution}

\item The relation ``has the same prime factors'' on the integers.

\begin{solution}
\textbf{Eq}
\end{solution}

\end{enumerate}

\ppart A set of functions $f,g:\reals \to \reals$ can be partially ordered
by the $\leq$ relation, where
\[
[f \leq g] \eqdef \forall r \in \reals.\, f(r) \leq g(r).
\]
Let $L$ be the set of functions $f:\reals \to \reals$ of the form
\[
f(x) = ax+b
\]
for real constants $a,b \neq 0$.

Describe an infinite chain and an infinite anti-chain in $L$.

\inhandout{\hint Think about parallel lines.}

\begin{solution}
If $f,g \in L$, then $f$ are comparable $g$ iff the lines defined by
$f$ and $g$ are parallel.  That is, if $f(x) = ax +b$ and $g(x) = cx +
d$, then $f \leq g$ iff $a = c$ and $b \leq d$.

So the set $L_a$ of functions of the form $ax +1$ for $a \in \reals$,
for example, form an infinite anti-chain; in fact, $L_a$ is a
\emph{maximal} anti-chain, since any function in $L - L_a$ would be
parallel to, and hence comparable to, some function in $L_a$.

Likewise, the set of functions $x + b$ for $b \in \reals$ is an
example of an infinite chain, and in fact a \emph{maximal} chain.
\end{solution}

\iffalse

\vspace{0.1in}

For the next three parts, let $f,g$ be nonnegative functions from the
integers to the real numbers.

\ppart  The ``Big Oh'' relation, $f=O(g)$,  \hfill \examrule[0.7in]

\begin{solution}
\textbf{Tr}
\end{solution}

\ppart  The ``Little Oh'' relation, $f=o(g)$,  \hfill \examrule[0.7in]

\begin{solution}
\textbf{SPO}
\end{solution}

\ppart The ``asymptotically equal'' relation, $f \sim g$.  \hfill \examrule[0.7in]

\begin{solution}
\textbf{Tr, Sym}
\end{solution}

\fi

\eparts

\end{problem}

%%%%%%%%%%%%%%%%%%%%%%%%%%%%%%%%%%%%%%%%%%%%%%%%%%%%%%%%%%%%%%%%%%%%%
% Problem ends here
%%%%%%%%%%%%%%%%%%%%%%%%%%%%%%%%%%%%%%%%%%%%%%%%%%%%%%%%%%%%%%%%%%%%%

\endinput
