\documentclass[problem]{mcs}

\begin{pcomments}
  \pcomment{FP_partial_order_short_answers}
  \pcomment{renamed from FP_partial_orders}
  \pcomment{based on CP_partial_orders}
\end{pcomments}

\pkeywords{
  partial_orders
  weak_partial_order  
  strict_partial_order
  transitive
  asymmetric
  path_total
}

%%%%%%%%%%%%%%%%%%%%%%%%%%%%%%%%%%%%%%%%%%%%%%%%%%%%%%%%%%%%%%%%%%%%%
% Problem starts here
%%%%%%%%%%%%%%%%%%%%%%%%%%%%%%%%%%%%%%%%%%%%%%%%%%%%%%%%%%%%%%%%%%%%%

\begin{problem}

%\set{(a,b) | a<b} \text{ on the set of integers }\quad & \examrule{1in}\\
%% Added
%\set{(a,b) | a^2 \leq b^2} \text{ on the set of real numbers }\quad & \examrule{1in}\\

For each of the relations below, indicate whether it is
\begin{quote}
a \emph{weak partial order} (\textbf{WPO}), \quad a \emph{strict
  partial order} (\textbf{SPO}),

and if so, whether it is \emph{path-total} (\textbf{Tot})
\end{quote}

If it is neither (\textbf{WPO}) nor (\textbf{SPO}), indicate whether it is
\begin{quote}
\emph{transitive} (\textbf{Tr}), \qquad  \emph{symmetric}
(\textbf{Sym}), \qquad  \emph{asymmetric} (\textbf{Asym})
\end{quote}

\bparts

\ppart The relation $a = b + 1$ between integers, $a$, $b$,
\hfill \examrule[0.7in]

\begin{solution}
\textbf{Asym}
\end{solution}

\ppart The superset relation, $\supseteq$ on the power set of the integers.
\hfill \examrule[0.7in]

\begin{solution}
\textbf{WPO}
\end{solution}

\ppart The relation $\expect{R} < \expect{S}$ between real-valued
  random variables $R,S$.
\hfill \examrule[0.7in]

\begin{solution}
\textbf{Tr, Asym}
\end{solution}

\ppart The \idx{empty relation} on the set of rationals.
\hfill \examrule[0.7in]

\begin{solution}
\textbf{SPO}
\end{solution}

\ppart The divides relation on the positive powers of 4.  \hfill \examrule[0.7in]

\begin{solution}
\textbf{WPO, Tot}
\end{solution}
\eparts

\vspace{0.1in}

For the next three parts, let $f,g$ be nonnegative functions from the
integers to the real numbers.

\bparts
\ppart  The ``Big Oh'' relation, $f=O(g)$,  \hfill \examrule[0.7in]

\begin{solution}
\textbf{Tr}
\end{solution}

\ppart  The ``Little Oh'' relation, $f=o(g)$,  \hfill \examrule[0.7in]

\begin{solution}
\textbf{SPO}
\end{solution}

\ppart The ``asymptotically equal'' relation, $f \sim g$.  \hfill \examrule[0.7in]

\begin{solution}
\textbf{Tr, Sym}
\end{solution}

\eparts

\end{problem}

%%%%%%%%%%%%%%%%%%%%%%%%%%%%%%%%%%%%%%%%%%%%%%%%%%%%%%%%%%%%%%%%%%%%%
% Problem ends here
%%%%%%%%%%%%%%%%%%%%%%%%%%%%%%%%%%%%%%%%%%%%%%%%%%%%%%%%%%%%%%%%%%%%%

\endinput
