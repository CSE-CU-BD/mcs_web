\documentclass[problem]{mcs}

\begin{pcomments}
  \pcomment{FP_partial_order_short_answer}
  \pcomment{slight overlap with FP_partial_order_short_answer_S12}
  \pcomment{based on CP_partial_orders}
\end{pcomments}

\pkeywords{
  partial_orders
  weak_partial_order  
  strict_partial_order
  transitive
  asymmetric
  path_total
}

%%%%%%%%%%%%%%%%%%%%%%%%%%%%%%%%%%%%%%%%%%%%%%%%%%%%%%%%%%%%%%%%%%%%%
% Problem starts here
%%%%%%%%%%%%%%%%%%%%%%%%%%%%%%%%%%%%%%%%%%%%%%%%%%%%%%%%%%%%%%%%%%%%%

\begin{problem}

Indicate which of the following relations below are equivalence
relations, (\textbf{E}), strict partial orders (\textbf{S}), weak
partial orders (\textbf{W}).  For the partial orders, also indicate
whether it is \emph{path-total} (\textbf{T}).

If a relation is none of the above, indicate whether it is
\emph{transitive} (\textbf{Tr}), \qquad  \emph{symmetric}
(\textbf{Sym}), \qquad  \emph{asymmetric} (\textbf{Asym}).

\bparts

\ppart The relation $a = b + 1$ between integers, $a$, $b$,
\hfill \examrule[0.7in]

\begin{solution}
\textbf{Asym}
\end{solution}

\ppart The superset relation, $\supseteq$ on the power set of the integers.
\hfill \examrule[0.7in]

\begin{solution}
\textbf{W}
\end{solution}

\iffalse

\ppart The relation $\expect{R} < \expect{S}$ between real-valued
  random variables $R,S$.
\hfill \examrule[0.7in]

\begin{solution}
\textbf{Tr, Asym}
\end{solution}
\fi

\ppart The \idx{empty relation} on the set of rationals.
\hfill \examrule[0.7in]

\begin{solution}
\textbf{SPO}
\end{solution}


\ppart The divides relation on the nonegatitve integers.  \hfill \examrule[0.7in]

\begin{solution}
\textbf{W}
\end{solution}


\ppart The divides relation on the integers.  \hfill \examrule[0.7in]

\begin{solution}
\textbf{Tr}
\end{solution}


\ppart The divides relation on the positive powers of 4.  \hfill \examrule[0.7in]

\begin{solution}
\textbf{W, T}
\end{solution}


\ppart The relatively prime relation on the nonnegative integers.  \hfill \examrule[0.7in]

\begin{solution}
\textbf{Sym}
\end{solution}
\eparts

\ppart The less-than, $<$, relation on real-valued functions, $f(x)$, of the
form $f(x) = ax+b$ for constants $a,b \in reals$.

\begin{solution}
\textbf{S}
\end{solution}

\ppart The relation ``has the same prime factors'' on the integers.

\begin{solution}
\textbf{E}
\end{solution}


\iffalse

\vspace{0.1in}

For the next three parts, let $f,g$ be nonnegative functions from the
integers to the real numbers.

\bparts
\ppart  The ``Big Oh'' relation, $f=O(g)$,  \hfill \examrule[0.7in]

\begin{solution}
\textbf{Tr}
\end{solution}

\ppart  The ``Little Oh'' relation, $f=o(g)$,  \hfill \examrule[0.7in]

\begin{solution}
\textbf{SPO}
\end{solution}

\ppart The ``asymptotically equal'' relation, $f \sim g$.  \hfill \examrule[0.7in]

\begin{solution}
\textbf{Tr, Sym}
\end{solution}

\eparts
\fi

\end{problem}

%%%%%%%%%%%%%%%%%%%%%%%%%%%%%%%%%%%%%%%%%%%%%%%%%%%%%%%%%%%%%%%%%%%%%
% Problem ends here
%%%%%%%%%%%%%%%%%%%%%%%%%%%%%%%%%%%%%%%%%%%%%%%%%%%%%%%%%%%%%%%%%%%%%

\endinput
