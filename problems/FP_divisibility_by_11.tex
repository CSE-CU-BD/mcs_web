\documentclass[problem]{mcs}

\begin{pcomments}
  \pcomment{FP_divisibility_by_11}
  \pcomment{CH Spring '14; edited ARM 4/1/14 9:15PM}
  \pcomment{based on CP_multiples_of_9_and_11}
\end{pcomments}

\pkeywords{
  divisibility
  modular arithmetic
}


%%%%%%%%%%%%%%%%%%%%%%%%%%%%%%%%%%%%%%%%%%%%%%%%%%%%%%%%%%%%%%%%%%%%%
% Problem starts here
%%%%%%%%%%%%%%%%%%%%%%%%%%%%%%%%%%%%%%%%%%%%%%%%%%%%%%%%%%%%%%%%%%%%%

\begin{problem}
\iffalse Divisibility tests for small numbers such as 3, 4, 5, and 9 are
well-known.
\begin{staffnotes}
This will be demoralizing for the many students who don't know about
divisibility tests and is unnecessary for those that do know.
\end{staffnotes}
\fi
In this problem, we will construct a simple test for divisibility by 11.

\bparts

\ppart Explain why $10^n \equiv (-1)^n \pmod{11}$, for all
nonnegative integers $n$. 

\begin{staffnotes}
ARM: I've replaced the call for proof by induction by ``explain why.''

It would have been clearer to ask for a proof that
\[
a \equiv b \pmod{c}\ \QIMPLIES\ a^n \equiv b^n \pmod{c},
\]
but this of course is so obvious that proving it by induction is
perverse.

\end{staffnotes}

\begin{solution}
$10 \equiv -1 \pmod{11}$ and since congruence is preserved under
  product, $10^n \equiv (-1)^n \pmod{11}$.
\end{solution}

\iffalse

\begin{solution}
Let $P(n)$ be the assertion that:
\begin{quote}
 $10^n \equiv (-1)^n \pmod{11}$ for all nonnegative integers $n$.
\end{quote}

\inductioncase{Base case} ($n = 0$) This follows easily from the fact
that $10^0 = (-1)^0 = 1$, which is congruent to $1 \pmod{11}$. 

\inductioncase{Inductive step} Assume that $P(n)$ is true, i.e., $10^n \equiv (-1)^n
\pmod{11}$ for some nonnegative integer $n$. Then, since $10 \equiv -1
\pmod{11}$, we know that
\begin{align*}
10^{n+1} &\equiv 10 \cdot 10^{n} \pmod{11} \\
 & \equiv (-1) \cdot (-1)^n \pmod{11}~~~\text{by Induction Hypothesis} \\
 & \equiv (-1)^{n+1} \pmod{11} .
\end{align*}
In other words, $P(n+1)$ is true. Therefore, by induction $P(n)$ is
true for all nonnegative integers $n$.
\end{solution}
\fi

\examspace[0.75in]

\ppart Take a big number, such as 47262938151.  Sum the digits, where
every other digit is negated:
\[
4 + (-7) + 2 + (-6) + 2 + (-9) + 3 + (-8) + 1 + (-5) + 1  =  -22.
\]
Explain why the original number is a multiple of 11 if and only if
this sum is a multiple of 11.  For example, this number 47262938151 is
divisible by 11 since -22 is divisible by 11.

\begin{solution}
A number in decimal has the form:
\[
d_k \cdot 10^k + d_{k-1} \cdot 10^{k-1} + \ldots + d_1 \cdot 10 + d_0
\]

Since $10 \equiv -1 \pmod{11}$, we know that:
\begin{align*}
\lefteqn{d_k \cdot 10^k + d_{k-1} \cdot 10^{k-1} + \cdots + d_1 \cdot 10 + d_0} \\
    & \equiv d_k \cdot (-1)^k + d_{k-1} \cdot (-1)^{k-1} + \cdots  + d_1 \cdot (-1)^1 + d_0 \cdot (-1)^0\\
    & \equiv d_k - d_{k-1} + \cdots - d_1 + d_0  \pmod{11}
\end{align*}
when $k$ is even.  The case where $k$ is odd is the same with the
signs reversed.

So the procedure given in the problem computes $\pm$ this alternating
sum of digits.  In particular, the original number is congruent to
zero mod 11 iff $\pm$ the alternating sum is congruent to zero mod 11.
Of course being congruent to zero mod 11 is the same as being
divisible by 11.
\end{solution}

\eparts

\end{problem}

%%%%%%%%%%%%%%%%%%%%%%%%%%%%%%%%%%%%%%%%%%%%%%%%%%%%%%%%%%%%%%%%%%%%%
% Problem ends here
%%%%%%%%%%%%%%%%%%%%%%%%%%%%%%%%%%%%%%%%%%%%%%%%%%%%%%%%%%%%%%%%%%%%%

\endinput
