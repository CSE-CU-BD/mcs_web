\documentclass[problem]{mcs}

\begin{pcomments}
  \pcomment{FP_divisibility_by_11}
  \pcomment{CH Spring '14}
  \pcomment{based on CP_multiples_of_9_and_11}
\end{pcomments}

\pkeywords{
  divisibility
  modular arithmetic
}


%%%%%%%%%%%%%%%%%%%%%%%%%%%%%%%%%%%%%%%%%%%%%%%%%%%%%%%%%%%%%%%%%%%%%
% Problem starts here
%%%%%%%%%%%%%%%%%%%%%%%%%%%%%%%%%%%%%%%%%%%%%%%%%%%%%%%%%%%%%%%%%%%%%

\begin{problem}

 Take a big number, such as 37273761261.  Sum the digits, where
every other digit is negated:
\[
3 + (-7) + 2 + (-7) + 3 + (-7) + 6 + (-1) + 2 + (-6) + 1  =  -11 .
\]
Explain why the original number is a multiple of 11 if and only
if this sum is a multiple of 11.

\hint $10 \equiv -1 \pmod{11}$.

\begin{solution}
A number in decimal has the form:
\[
d_k \cdot 10^k + d_{k-1} \cdot 10^{k-1} + \ldots + d_1 \cdot 10 + d_0
\]

Since $10 \equiv -1 \pmod{11}$, we know that:
\begin{align*}
\lefteqn{d_k \cdot 10^k + d_{k-1} \cdot 10^{k-1} + \cdots + d_1 \cdot 10 + d_0} \\
    & \equiv d_k \cdot (-1)^k + d_{k-1} \cdot (-1)^{k-1} + \cdots  + d_1 \cdot (-1)^1 + d_0 \cdot (-1)^0 \pmod{11} \\
& \equiv d_k - d_{k-1} + \cdots - d_1 + d_0  \pmod{11}
\end{align*}
assuming $k$ is even.  The case where $k$ is odd is the same with the signs
reversed.

The procedure given in the problem computes this alternating sum of
digits (or its negative). Therefore, it yields a number divisible by 11 ($\equiv 0 \pmod {11}$)
iff the original number was divisible by 11.
\end{solution}

\end{problem}

%%%%%%%%%%%%%%%%%%%%%%%%%%%%%%%%%%%%%%%%%%%%%%%%%%%%%%%%%%%%%%%%%%%%%
% Problem ends here
%%%%%%%%%%%%%%%%%%%%%%%%%%%%%%%%%%%%%%%%%%%%%%%%%%%%%%%%%%%%%%%%%%%%%

\endinput
