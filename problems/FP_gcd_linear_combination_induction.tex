\documentclass[problem]{mcs}

\begin{pcomments}
  \pcomment{FP_gcd_linear_combination_induction}
  \pcomment{ARM 3/17/13, minor edit 12/13/13}
\end{pcomments}

\pkeywords{
  gcd
  induction
  linear_combination
}

%%%%%%%%%%%%%%%%%%%%%%%%%%%%%%%%%%%%%%%%%%%%%%%%%%%%%%%%%%%%%%%%%%%%%
% Problem starts here
%%%%%%%%%%%%%%%%%%%%%%%%%%%%%%%%%%%%%%%%%%%%%%%%%%%%%%%%%%%%%%%%%%%%%

\begin{problem}
Prove by induction that the gcd of a finite set of integers is an
integer linear combination of the numbers in the set.  You may assume
that the gcd of two integers is an integer linear combination of them,
which was proved \inhandout{in the
  text}\inbook{Theorem~\bref{gcd_is_lin_thm}}.  You may also assume
the easily verified fact that
\begin{equation}\label{gcdAua}
\gcd(A \union B) = \gcd(\gcd(A),\gcd(B)),
\end{equation}  
for any finite, nonempty sets $A,B$ of integers.

Be sure to clearly state and label your Induction Hypothesis, Base
case(s), and Induction step.

\begin{solution}
We proceed by induction on $n$ with induction hypothesis
\[
P(n) \eqdef\ \text{the gcd of set of $n$ of integers is an
integer linear combination of the integers in the set.}
\]
Alternatively,
\[
P(n) \eqdef\ \forall a_1, a_2 \dots, a_n \in \integers\, \exists s_1, s_2 \dots, s_n\in \integers.\,
\gcd(a_1, a_2 \dots, a_n) =  s_1a_1 + s_2a_2 + \cdots + s_na_n.
\]

\inductioncase{Base case}: ($n = 1$).  Let $s_1 = 1$.

\inductioncase{Inductive step}: Now we assume $P(n)$ holds for some $n
\geq 1$ and prove $P(n + 1)$.

Let $A \eqdef \set{a_1, a_2, \dots, a_{n}}$.  So $P(n)$ implies that
\[
\gcd(A) = s_1a_1 + s_2a_2 + \cdots + s_na_n.
\]
Now
\begin{align*}
\gcd(a_1,a_2,\dots,a_{n+1})
  & = \gcd(\gcd(a_1,a_2,\dots,a_{n}), a_{n+1})
       & \text{(by~\eqref{gcdAua})}\\
  & = s \cdot \gcd(a_1,a_2,\dots,a_{n}) + t \cdot a_{n+1} \text{ for some } s,t \in \integers
        &  \text{(the two element case)}\\
  & = s\cdot (s_1a_1 + s_2a_2 + \cdots + s_na_n) + t \cdot a_{n+1}\\
  & = (ss_1)a_1 + (ss_2)a_2 + \cdots + (ss_n)a_n + t a_{n+1}.
\end{align*}
This shows that $\gcd(a_1,a_2,\dots,a_{n+1})$ is also a linear
combination of $a_1,a_2,\dots,a_{n+1}$, which proves $P(n+1)$ holds,
completing the inductive step.

By induction, the claim holds for all $n \geq 1$.
\end{solution}
  
\end{problem}

%%%%%%%%%%%%%%%%%%%%%%%%%%%%%%%%%%%%%%%%%%%%%%%%%%%%%%%%%%%%%%%%%%%%%
% Problem ends here
%%%%%%%%%%%%%%%%%%%%%%%%%%%%%%%%%%%%%%%%%%%%%%%%%%%%%%%%%%%%%%%%%%%%%

\endinput
