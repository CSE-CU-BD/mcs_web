\documentclass[problem]{mcs}

\begin{pcomments}
  \pcomment{TP_Fermats_Little_Theorem_F13}
  \pcomment{tweak of TP_Fermats_Little_Theorem, ..._S13}
  \pcomment{edited 10/8/13 by ARM}
\end{pcomments}

\begin{problem}
You should not need to do any calculation to answer the next two questions:

\bparts

\ppart What is $\rem{3^{99}}{97}$?

\begin{center}
\exambox{0.75in}{0.5in}{0.0in}
\end{center}

\examspace[2in]

\begin{solution}
$\mathbf{27}$.

The divisor 97 is prime.  Therefore,
\begin{align*}
3^{99} & = 3^{96} \cdot 3^3\\
      & = 1 \cdot 3^3 \inzmod{97}
           & \text{(Euler's Theorem since $\phi(97) = 96$)}\\
      & = 27,
\end{align*}
which implies that $\rem{3^{99}}{97} = 27$.

\iffalse
Since $\phi(p) = p-1$, Euler's Theorem says that $n^{p-1} \equiv 1
\pmod p$ for $n \not \equiv 0 \pmod p$.  (This is known as Fermat's
Little Theorem.)  So $\rem{n^{p-1}}{p} = 1$, when $p$ is prime and $n$
is not divisible by $p$.
\fi
\end{solution}

\ppart What is $\rem{96^{123456789}}{97}$?

\begin{center}
\exambox{0.75in}{0.5in}{0.0in}
\end{center}

\begin{solution}
\textbf{96}.

$96 \equiv -1 \pmod {97}$, so 96 to an odd power is also congruent to
-1 modulo 97 and therefore has the same remainder as 96 divided by 97.
\end{solution}

\eparts

\end{problem}

\endinput
