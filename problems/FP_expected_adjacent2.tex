\documentclass[problem]{mcs}

\begin{pcomments}
  \pcomment{FP_expected_adjacent2}
  \pcomment{variant of FP_expected_adjacent}
  \pcomment{F15.final-conflict2}
\end{pcomments}

\pkeywords{
  expectation
  linearity
}

%%%%%%%%%%%%%%%%%%%%%%%%%%%%%%%%%%%%%%%%%%%%%%%%%%%%%%%%%%%%%%%%%%%%%
% Problem starts here
%%%%%%%%%%%%%%%%%%%%%%%%%%%%%%%%%%%%%%%%%%%%%%%%%%%%%%%%%%%%%%%%%%%%%

\begin{problem}
There are six kinds of cards, three of each kind, for a total of eighteen cards.
The cards are randonly shuffled and laid out in a row,
for example,
\[
\fbox{1}\ \fbox{2}\ \fbox{5}\ \fbox{5}\ \fbox{5}\ \fbox{1}\ 
\fbox{4}\ \fbox{6}\ \fbox{2}\ \fbox{6} \fbox{6}\ \fbox{2}\
\fbox{1}\ \fbox{4}\  \fbox{3}\ \fbox{3}\ \fbox{3}\ \fbox{4}
\]
In this case, there are two adjacent triples of the same kind, the
three 3's and the three 5's.

\bparts

\ppart\label{pr456m} Derive a formula for the probability that the 4th, 5th, and 6th
consecutive cards will be the same kind---that is, all 1's or all 2's
or\dots all 6's?

\examspace[2.0in]

\begin{solution}
\[
\frac{2}{16\cdot 17}
\]

Given that a particular card is in the 4th position, two of the
remaining 17 cards match it.  So
\[
\pr{\text{card 5 matches card 4}} = \frac{2}{17}.
\]
Given particular matching cards in the 4th and 5th positions, only one
of the remaining 16 cards will match them, so
\[
\prcond{\text{card 6 matches card 4}}{\text{card 5 matches card 4}} =
\frac{1}{16}.
\]
Therefore,
\begin{align*}
\lefteqn{\pr{\text{4th, 5th and 6th cards match}}}\\
   &= \prcond{\text{card 6 matches card 4}}{\text{card 5 matches card 4}}\cdot \pr{\text{card 5 matches card 4}}\\
   & = \frac{2}{16\cdot 17}.
\end{align*}
\end{solution}

\ppart Let $p \eqdef \pr{\text{4th, 5th and 6th cards match}}$---that
is, $p$ is the correct answer to part~\eqref{pr456m}.  Write a simple
formula for the expected number of matching triples in terms of $p$.

\begin{solution}
$\mathbf{16p}$.

The probability of a matching triple starting at any given one of the
first 16 positions is $p$, and therefore the expected number is also
$p$.  Of course a matching triple cannot begin at the 17th or 18th
position.  So by linearity of expectation, the expected number of
matching triples is the sum of the expected number starting at each of
the first 16 positions.
\end{solution}

\eparts

\end{problem}

%%%%%%%%%%%%%%%%%%%%%%%%%%%%%%%%%%%%%%%%%%%%%%%%%%%%%%%%%%%%%%%%%%%%%
% Problem ends here
%%%%%%%%%%%%%%%%%%%%%%%%%%%%%%%%%%%%%%%%%%%%%%%%%%%%%%%%%%%%%%%%%%%%%

\endinput
