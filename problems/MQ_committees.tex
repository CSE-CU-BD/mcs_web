\documentclass[problem]{mcs}

\begin{pcomments}
  \pcomment{MQ_committees}
\end{pcomments}

\pkeywords{
}

%%%%%%%%%%%%%%%%%%%%%%%%%%%%%%%%%%%%%%%%%%%%%%%%%%%%%%%%%%%%%%%%%%%%%
% Problem starts here
%%%%%%%%%%%%%%%%%%%%%%%%%%%%%%%%%%%%%%%%%%%%%%%%%%%%%%%%%%%%%%%%%%%%%

\begin{problem}
Twenty people work at CantorCorp, a small, unsuccessful start-up.  Each
of these employees has a unique ID number.  Two
six-person committees are to be formed.  The first will work only to prove
the Continuum Hypothesis.  The second will work only to disprove it.  Each employee
will be a member of at most one committee, and the people on each committee are of
equal standing.  

Each possible way of assigning people to committees can therefore be
represented uniquely by the ordered pair $\paren{S_1,S_2}$, where
\begin{itemize}
\item $S_1$ contains the ID numbers of the employees in the first committee.
\item $S_2$ contains the ID numbers of the employees in the second committee.
\item $|S_1|=|S_2|=6$.
\item $S_1\cap S_2 = \emptyset$.
\end{itemize}
We call each such pair a ``setpair''.

\begin{problemparts}

\problempart
Let $D$ denote the set of all possible setpairs.  Find $|D|$.
\begin{solution}
There are $\displaystyle\binom{20}{6}$ ways to choose employees for the first committee, and thus $\displaystyle\binom{20}{6}$ 
possible choices for $S_1$.  For each of these, there are 14 people left over and so $\displaystyle\binom{14}{6}$ ways to assign
employees to the second committee, or $\displaystyle\binom{14}{6}$ ways to choose $S_2$.  By the Generalized Product Rule,
\[|D| = \binom{20}{6}\binom{14}{6}.\]
\end{solution}

\problempart
Two of the workers, Aleph and Beth, will be unhappy if they are put on the same committee.

Let $P$ denote the set of setpairs that correspond to having Aleph and Beth serve on the same committee.
Find $|P|$.
\begin{solution}
Either Aleph and Beth serve on the first committee or they serve on the second.  If the first, then
there are 18 employees left to choose from, and four spots left in the committee: there are $\displaystyle\binom{18}{4}$
ways to build the first committee (to choose $S_1$).  For each such selection of the first committee, there are 14 employees left for
the second committee.  Of these, six must be chosen.  There are $\displaystyle\binom{14}{6}$ ways to do this (ways to choose $S_2$).  
So altogether (by the Generalized Product Rule), there are $\displaystyle\binom{18}{4}\binom{14}{6}$ ways to choose the committees 
(or setpairs) where Aleph and Beth serve together on the first committee.  Similarly, there are  $\displaystyle\binom{18}{4}\binom{14}{6}$ ways
to choose the committees (or setpairs) where Aleph and Beth serve together on the second committee.  So the number of setpairs in which
Aleph and Beth serve on the same committee is just, by the Sum Rule, 
\[|P|=2\binom{18}{4}\binom{14}{6}\].
(It should be obvious that the disjointness requirement needed to apply the Sum Rule is met here.)
\end{solution}

\problempart
Beth will also be unhappy if she has to serve on a committee with \textbf{both} Ferdinand and Georg.

Let $Q$ denote the set of setpairs that correspond to having Beth, Ferdinand, and Georg all serve on the same committee.
Find $|Q|$.
\begin{solution}
By similar reasoning to that used to find $|P|$, obtain 
\[|Q|=2\binom{17}{3}\binom{14}{6}.\]
\end{solution}

\problempart
Find $|P\cap Q|$.
\begin{solution}
$P\cap Q$ is the set of all setpairs in which Aleph, Beth, Ferdinand, and Georg all serve on the same committee.  Clearly,
\[|P\cap Q|=2\binom{16}{2}\binom{14}{6}.\]
\end{solution}

\problempart
Let $S$ denote the set of \textbf{all} setpairs corresponding to arrangements in which there is at least one unhappy employee.
Express $S$ in terms of $P$ and $Q$ \textbf{only}.
\begin{solution}
\[S = P\cup Q.\]
\end{solution}

\problempart
Find $|S|$.
\begin{solution}
Applying inclusion/exclusion, have
\begin{align*}
|S|       &= |P \cup Q| \\         
          &= |P| + |Q| - |P\cap Q| \\
          &= 2\binom{18}{4}\binom{14}{6} + 2\binom{17}{3}\binom{14}{6} - 2\binom{16}{2}\binom{14}{6} \\
          & = 2\paren{\binom{18}{4}+\binom{17}{3}-\binom{16}{2}}\binom{14}{6}
\end{align*}
\end{solution}

\problempart
At last, how many ways are there to form the committees so that no employee is unhappy? 
\begin{solution}
Let $R$ denote the set of setpairs corresponding to arrangements in which no employee is unhappy.  Clearly, $D = R\cup S$ and 
$R\cap S = \emptyset$.  So apply the Sum Rule: $|D| = |R| + |S|$.  Hence:
\begin{align*}
|R| &= |D| - |S| \\
          &= \binom{20}{6}\binom{14}{6} - 2\binom{18}{4}\binom{14}{6} - 2\binom{17}{3}\binom{14}{6} + 2\binom{16}{2}\binom{14}{6} \\
          & = \paren{\binom{20}{6}-2\binom{18}{4}-2\binom{17}{3}+2\binom{16}{2}}\binom{14}{6}
\end{align*}
\end{solution}

\end{problemparts}

\end{problem}

%%%%%%%%%%%%%%%%%%%%%%%%%%%%%%%%%%%%%%%%%%%%%%%%%%%%%%%%%%%%%%%%%%%%%
% Problem ends here
%%%%%%%%%%%%%%%%%%%%%%%%%%%%%%%%%%%%%%%%%%%%%%%%%%%%%%%%%%%%%%%%%%%%%
