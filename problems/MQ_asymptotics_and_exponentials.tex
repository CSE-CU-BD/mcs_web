\documentclass[problem]{mcs}

\begin{pcomments}
  \pcomment{MQ_asymptotics_and_exponentials}
\end{pcomments}

\pkeywords{
  asymptotic
  exponent
  increasing
}

%%%%%%%%%%%%%%%%%%%%%%%%%%%%%%%%%%%%%%%%%%%%%%%%%%%%%%%%%%%%%%%%%%%%%
% Problem starts here
%%%%%%%%%%%%%%%%%%%%%%%%%%%%%%%%%%%%%%%%%%%%%%%%%%%%%%%%%%%%%%%%%%%%%

\begin{problem}

\iffalse Give an example of a pair of strictly increasing total
functions, $f:\nngint^+\to\nngint^+$ and
$g:\nngint^+\to\nngint^+$, such that $f\sim g$ but $3^f\neq
O\paren{3^g}$.  \fi

Give an example of a pair of strictly increasing total functions,
$f:\nngint^+\to\nngint^+$ and $g:\nngint^+\to\nngint^+$, that
satisfy $f\sim g$ but \textbf{not} $3^f=O\paren{3^g}$.
\begin{solution}
The pair
\begin{align*}
f(n)&=n^2+n\\
g(n)&=n^2
\end{align*}
satisfies these criteria.  Since $n^2$ is the term that dominates the
behavior of $n^2+n$ as $n$ grows large, it follows that $n^2+n\sim
n^2$.  (Applying the limit definition of asymptotic equality is a
simple way to verify this.)  Clearly, $3^{f(n)}=3^{n^2+n}=3^n
3^{n^2}$, while $3^{g(n)}=3^{n^2}$.  Thus $3^{f(n)}=3^n3^{g(n)}$.
From this, it is obvious that $3^f\neq O\paren{3^g}$.  (It is easy to
check that, in fact, $3^g=o(3^f)$.)
\end{solution}

\end{problem}

%%%%%%%%%%%%%%%%%%%%%%%%%%%%%%%%%%%%%%%%%%%%%%%%%%%%%%%%%%%%%%%%%%%%%
% Problem ends here
%%%%%%%%%%%%%%%%%%%%%%%%%%%%%%%%%%%%%%%%%%%%%%%%%%%%%%%%%%%%%%%%%%%%%

\endinput
