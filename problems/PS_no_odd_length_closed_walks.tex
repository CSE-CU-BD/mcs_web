\documentclass[problem]{mcs}

\begin{pcomments}
  \pcomment{PS_no_odd_length_closed_walks}
  \pcomment{rephrasing of PS_no_odd_length_cycles}
  \pcomment{from: S10?, S09.ps6, S07.ps5, S06.ps4 S04.quiz1, S04.cp5w}
\end{pcomments}

\pkeywords{
  closed_walk
  walk
  graph_coloring
  2-color
  bipartite
  trees
  spanning_trees
}

%%%%%%%%%%%%%%%%%%%%%%%%%%%%%%%%%%%%%%%%%%%%%%%%%%%%%%%%%%%%%%%%%%%%%
% Problem starts here
%%%%%%%%%%%%%%%%%%%%%%%%%%%%%%%%%%%%%%%%%%%%%%%%%%%%%%%%%%%%%%%%%%%%%

\begin{problem}\label{evenlength}
In this problem you will prove:
\begin{theorem*}
A graph $G$ is 2-colorable iff it contains no odd length closed walk.
\end{theorem*}

As usual with ``iff'' assertions, the proof splits into two proofs:
part~(a) asks you to prove that the left side of the ``iff'' implies the
right side.  The other problem parts prove that the right side implies the
left.

\bparts

\ppart Assume the left side and prove the right side.  Three to five
sentences should suffice.

\begin{solution}
  Assume $G$ is 2-colorable and select a 2-coloring of $G$.  Consider an
  arbitrary closed walk through successive vertices $v_1, v_2, \dots, v_k,
  v_1$.  Then the vertices $v_i$ must be one color for all even $i$ and
  the other color for all odd $i$.  (This is obvious, but could of course,
  be proved by induction.)  Since $v_1$ and $v_k$ must be colored
  differently, $k$ must be even.  Thus, the walk has even length.
\end{solution}

\ppart Now assume the right side.  As a first step toward proving the
left side, explain why we can focus on a single connected component
$H$ within $G$.

\begin{solution}
If we can 2-color every connected component of $G$, then we can
2-color all of $G$.  Thus, it suffices to show that an arbitrary connected
component $H$ of $G$ is 2-colorable.
\end{solution}

\ppart As a second step, explain how to 2-color any tree.

\begin{solution}
A 2-coloring of a tree can be defined by selecting any fixed
vertex $v$, and coloring a vertex one color if the (unique) path to
it from $v$ has odd length, and coloring it with the other color if the
path has even length.

To verify that adjacent vertices in the tree get different colors, let $e
\eqdef \edge{x}{y}$ be an edge in the tree.  There is a unique path from $v$
to $x$.  If this path traverses $e$, it must consist of a path from $v$ to
$y$ followed by the $e$ traversal to $x$.  If this path does not traverse
$e$, then it can be extended to a path to $y$ by adding a final traversal
of $e$.  In either case, the paths to these vertices from $v$ differ by a
single traversal of $e$, and so the lengths of the paths differ by 1; in
particular, one is of odd length and the other is of even length, so $x$
and $y$ are differently-colored.
\end{solution}

\ppart Choose any 2-coloring of a spanning tree, $T$, of $H$.  Prove that
$H$ is 2-colorable by showing that any edge \emph{not} in $T$ must also
connect different-colored vertices.

\begin{solution}
  Let $\edge{x}{y}$ be an edge not in $T$.  Since $T$ is a tree, there are
  unique paths in $T$ from $v$ to $x$ and from $v$ to $y$.  Exactly one of
  these two paths must have odd length---otherwise, these two paths
  together with the edge $\edge{x}{y}$ would form an odd length cycle and
  hence an odd length closed walk.  But this means $x$ and $y$ are colored
  differently.
\end{solution}

\eparts

\end{problem}

%%%%%%%%%%%%%%%%%%%%%%%%%%%%%%%%%%%%%%%%%%%%%%%%%%%%%%%%%%%%%%%%%%%%%
% Problem ends here
%%%%%%%%%%%%%%%%%%%%%%%%%%%%%%%%%%%%%%%%%%%%%%%%%%%%%%%%%%%%%%%%%%%%%
