\documentclass[problem]{mcs}

\begin{pcomments}
  \pcomment{FP_infinite_binary_sequences}
  \pcomment{ARM 12/12/11}
\end{pcomments}

\pkeywords{
  countable
  union
  infinite_sequence
  uncountable
}

%%%%%%%%%%%%%%%%%%%%%%%%%%%%%%%%%%%%%%%%%%%%%%%%%%%%%%%%%%%%%%%%%%%%%
% Problem starts here
%%%%%%%%%%%%%%%%%%%%%%%%%%%%%%%%%%%%%%%%%%%%%%%%%%%%%%%%%%%%%%%%%%%%%

\newcommand{\binw}{\set{0,1}^\omega}

\begin{problem}
Let $\binw$ be the uncountable set of infinite binary sequences, and
let $F_n \subset \binw$ be the set of infinite binary sequences
whose bits are all 0 after the $n$th bit.  That is, if
$\mathbf{s} \eqdef (s_0,s_1,s_2,\dots) \in \binw$, then
\[
\mathbf{s} \in F_n \QIFF \forall i >n.\, s_i = 0.
\]
For example, the sequence $\mathbf{t}$ that starts $001101$ with 0's
after that is in $F_5$, since by definition $t_i = 0$ for all $i > 5$.
In fact, $\mathbf{t}$ is by definition also in $F_6,F_7,\dots$.

\bparts

\ppart\label{Fnsize}  What is the size, $\card{F_n}$, of $F_n$?

\begin{center}
\exambox{0.5in}{0.5in}{-0.1in}
\end{center}

\begin{solution}
\[
2^{n+1}
\]

There are only $2^{n+1}$ possible initial sequences of $n+1$ bits for
sequences in $F_n$ and after that all of them are all 0's.
\end{solution}

\ppart\label{Fcountable} Explain why the set $F \subset \binw$ of sequences with only
finitely many 1's, is a countable set.  (You may assume without proof
any results from class about countabillity.)

\examspace[2.5in]

\begin{solution}
Note that
\[
F = \lgunion_0^\infty F_n.
\]
Now $F_n$ is finite and therefore is countable, and a countable union
of countable sets is countable (see
Problem~\bref{MQ_countable_union}), so $F$ is countable.

\end{solution}

\ppart Prove that the set of infinite binary sequences with
infinitely many 1's is uncountable. \hint Use parts~\eqref{Fnsize}
and~\eqref{Fcountable}; a direct proof by diagonalization is tricky.

\examspace[3in]

\begin{solution}
The set of sequences with infinitely many 1's is $\binw - F$.

Suppose for the sake of contradiction that $\binw - F$ was countable.
Since the union of two countable sets is countable, this implies
$(\binw - F) \union F = \binw$ is countable, contradicting the fact
that $\binw$ is uncountable.
\end{solution}

\eparts

\end{problem}

%%%%%%%%%%%%%%%%%%%%%%%%%%%%%%%%%%%%%%%%%%%%%%%%%%%%%%%%%%%%%%%%%%%%%
% Problem ends here
%%%%%%%%%%%%%%%%%%%%%%%%%%%%%%%%%%%%%%%%%%%%%%%%%%%%%%%%%%%%%%%%%%%%%

\endinput
