\documentclass[problem]{mcs}

\begin{pcomments}
  \pcomment{FP_infinite_binary_sequences}
  \pcomment{ARM 12/12/11, soln edited 5/15/14}
\end{pcomments}

\pkeywords{
  countable
  union
  infinite_sequence
  uncountable
}

%%%%%%%%%%%%%%%%%%%%%%%%%%%%%%%%%%%%%%%%%%%%%%%%%%%%%%%%%%%%%%%%%%%%%
% Problem starts here
%%%%%%%%%%%%%%%%%%%%%%%%%%%%%%%%%%%%%%%%%%%%%%%%%%%%%%%%%%%%%%%%%%%%%

%\newcommand{\binw}{\set{0,1}^\omega}

\begin{problem}
Let $\binw$ be the uncountable set of infinite binary sequences, and
let $F_n \subset \binw$ be the set of infinite binary sequences
whose bits are all 0 after the $n$th bit.  That is, if
$\mathbf{s} \eqdef (s_0,s_1,s_2,\dots) \in \binw$, then
\[
\mathbf{s} \in F_n \QIFF \forall i >n.\, s_i = 0.
\]
For example, the sequence $\mathbf{t}$ that starts $001101$ with 0's
after that is in $F_5$, since by definition $t_i = 0$ for all $i > 5$.
In fact, $\mathbf{t}$ is by definition also in $F_6,F_7,\dots$.

\bparts

\ppart\label{Fnsize}  What is the size $\card{F_n}$ of $F_n$?

\begin{center}
\exambox{0.5in}{0.5in}{-0.1in}
\end{center}

\begin{solution}
\[
2^{n+1}
\]

\begin{staffnotes}
1/2 credit for $2^n$.
\end{staffnotes}
There are only $2^{n+1}$ possible initial sequences of $n+1$ bits for
sequences in $F_n$ and after that all of them are all 0's.
\end{solution}

\ppart\label{Fcountable} Explain why the set $F \subset \binw$ of
sequences with only finitely many 1's, is a countable set.
\inhandout{(You may assume without proof any results from class about
  countability.)}

\examspace[2.5in]

\begin{solution}
Note that
\[
F = \lgunion_0^\infty F_n.
\]
Now $F_n$ is finite and therefore is countable, and a countable union
of countable sets is countable (Problem~\bref{MQ_countable_union}), so
$F$ is countable.

An alternative explanation is to notice that the mapping $f:F \to
\naturals$, where
\[
f(s) \eqdef \sum_{n \in \naturals} s_n2^n
\]
is a bijection.

$F$ can also be shown to be countable by defining a $[\leq 1 \text{
    out} \geq 1 \text{ out}]$ mapping (surjective function) from some
countable set to $F$ (Problem~\bref{CP_countable_from_surj}).  This
is easy using the fact that the set $\finbin$ of finite binary strings
is countable (Problem~\bref{TP_countable_strings}).  Namely, define
the function $g:\finbin \to F$, by the rule that $g(z)$ is the finite
sequence $z$ followed by an infinite sequence of 0's.
\end{solution}

\ppart Prove that the set of infinite binary sequences with infinitely
many 1's is uncountable. \hint Use parts~\eqref{Fnsize}
and~\eqref{Fcountable}.

\examspace[3in]

\begin{solution}
The set of sequences with infinitely many 1's is $\bar{F} \eqdef \binw
- F$.

Suppose for the sake of contradiction that $\bar{F}$ was countable.
Since the union of two countable sets is countable, this implies
$\bar{F} \union F = \binw$ is countable, contradicting the fact that
$\binw$ is uncountable.

There is also a simple diagonal argument that proves that $\bar{F}$ is
uncountable using a $30^o$ diagonal instead of the basic $45^o$
diagonal.  Supposing there was a list of the elements of $\bar{F}$,
define a ``diagonal'' sequence $s$ that differs from the $n$th
sequence in the list at position $\mathbf{2}n$.  This still ensures
that $s$ is not in the list, but it leaves all the odd numbered bits
of $s$ unspecified.  So we can define all the odd numbered bits of $s$
to be 1, guaranteeing that $s$ has infinitely many 1's.

\begin{staffnotes}
Consider why the usual $45^o$ diagonal argument does not work here.
If $s'$ is the diagonal sequence that differs from the $n$th sequence
in the list at position $n$, for all $n \in \naturals$, then $s'$ is
not in the list, \emph{but there is no guarantee that $s'$ has
  infinitely many 1's}.  So the diagonalization may not yield a
``missing'' sequence, and the proof breaks down.
\end{staffnotes}
\end{solution}

\eparts

\end{problem}

%%%%%%%%%%%%%%%%%%%%%%%%%%%%%%%%%%%%%%%%%%%%%%%%%%%%%%%%%%%%%%%%%%%%%
% Problem ends here
%%%%%%%%%%%%%%%%%%%%%%%%%%%%%%%%%%%%%%%%%%%%%%%%%%%%%%%%%%%%%%%%%%%%%

\endinput
