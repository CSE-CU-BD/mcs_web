\documentclass[problem]{mcs}

\begin{pcomments}
  \pcomment{FP_com_proof_parts}
  \pcomment{from: s11 final, based on CP_com_proof}
\end{pcomments}

\pkeywords{
  combinatorial
  bijection
  binomial
}

%%%%%%%%%%%%%%%%%%%%%%%%%%%%%%%%%%%%%%%%%%%%%%%%%%%%%%%%%%%%%%%%%%%%%
% Problem starts here
%%%%%%%%%%%%%%%%%%%%%%%%%%%%%%%%%%%%%%%%%%%%%%%%%%%%%%%%%%%%%%%%%%%%%

\begin{problem}

Let $S$ be the set of all length~$n$ sequences of letters $a$, $b$,
and exactly one $c$.

\bparts %\mbox{}

\ppart Show that the cardinality of $S$ is equal to $n2^{n-1}$.

\examspace[2in]

\begin{solution}
Let $P \eqdef [0,n) \cross \set{a,b}^{n-1}$.  On
the one hand, there is a bijection from $P$ to $S$ by mapping $(k,x)$ to
the word obtained by inserting a $c$ just after the $k$th letter in the
length~$n-1$ word, $x$, of $a$'s and $b$'s.  So
\begin{equation}\label{SPn2n-1}
\card{S} = \card{P}= n 2^{n-1}
\end{equation}
by the Product Rule.
\end{solution}

\ppart Show that the cardinality of $S$ is equal to
\[
\sum_{k=1}^n k \binom{n}{k}.
\]

\examspace[2in]

\begin{solution}
Every sequence in $S$ contains between 1 and $n$
entries not equal to $a$ since the $c$, at least, is not $a$.  The
mapping from a sequence in $S$ with exactly $k$ non-$a$ entries to a
pair consisting of the set of positions of the non-$a$ entries and the
position of the $c$ among these entries is a bijection, and the number
of such pairs is $\binom{n}{k}k$ by the Generalized Product Rule.
Thus, by the Sum Rule:
\[
\card{S} = \sum_{k=1}^n k \binom{n}{k}
\]
\end{solution}

\ppart What combinatorial identity now follows?

\begin{center}
\exambox{4.0in}{0.6in}{0.3in}
\end{center}

\begin{solution}
These parts prove the identity 
\[
n2^{n-1} = \sum_{k=1}^n k \binom{n}{k}
\]
by showing that each side is equal to the cardinality of the same set.
\end{solution}

\eparts
\end{problem}

%%%%%%%%%%%%%%%%%%%%%%%%%%%%%%%%%%%%%%%%%%%%%%%%%%%%%%%%%%%%%%%%%%%%%
% Problem ends here
%%%%%%%%%%%%%%%%%%%%%%%%%%%%%%%%%%%%%%%%%%%%%%%%%%%%%%%%%%%%%%%%%%%%%
\endinput
