\documentclass[problem]{mcs}

\begin{pcomments}
  \pcomment{CP_fibonacci_by_induction}
  \pcomment{By Ali Kazerani, proof of Binet's formula using strong
    induction}
  \pcomment{edits & format by ARM 1/15/12}
  \pcomment{S15.ps3}
\end{pcomments}

\pkeywords{
  induction
  fibonacci
  recurrence
  closed_form
  Binet's_formula
}

%%%%%%%%%%%%%%%%%%%%%%%%%%%%%%%%%%%%%%%%%%%%%%%%%%%%%%%%%%%%%%%%%%%%%
% Problem starts here
%%%%%%%%%%%%%%%%%%%%%%%%%%%%%%%%%%%%%%%%%%%%%%%%%%%%%%%%%%%%%%%%%%%%%

\begin{problem}
\inhandout{The Fibonacci numbers $F(0) F(1) F(2), \dots$ are defined as follows:
\[
F(n) \eqdef \begin{cases}
  0               & \mbox{if $n = 0$},\\
  1               & \mbox{if $n = 1$},\\
  F(n-1) + F(n-2) & \mbox{if $n >1$}.
\end{cases}
\]}
\inbook{The Fibonacci numbers are described in Section~\bref{sec:fib}.}

Prove, using strong induction, the following closed-form formula for
the Fibonacci numbers.\footnote{This mind-boggling formula is known as \emph{\idx{Binet's
    formula}}.  We'll explain in
  Chapter~\bref{generating_function_chap}, and again in
  Chapter~\bref{chap:recurrences}, how it comes about.}
\[
F(n)= \frac{p^n-q^n}{\sqrt{5}}
\]
where $p=\frac{1+\sqrt{5}}{2}$ and $q=\frac{1-\sqrt{5}}{2}$.

\hint Note that $p$ and $q$ are the roots of $x^2-x-1=0$, and so
$p^2=p+1$ and $q^2=q+1$.
 
\begin{solution}

\begin{proof}
We will proceed by strong induction on $n$.  Let the induction
hypothesis $P(n)$ be that the given closed-form formula holds at
$n$, that is,
\[
F(n)= \frac{p^n-q^n}{\sqrt{5}}.
\]

\inductioncase{Base case}($n=0$): $P(0)$ is true, since
\[
\frac{p^n-q^n}{\sqrt{5}}=\frac{p^0-q^0}{\sqrt{5}}=\frac{1-1}{\sqrt{5}}=0=F_0
\]

\inductioncase{Base Case} ($n=1$):  $P(1)$ is true, since
\[
\frac{p^n-q^n}{\sqrt{5}} = \frac{p^1-q^1}{\sqrt{5}} =
\frac{p-q}{\sqrt{5}} = \frac{\sqrt{5}}{\sqrt{5}} = 1 = F_1.
\]

\inductioncase{Inductive Step} ($n > 1$):
Since $0\leq n-1,n < n+1$, we may assume the strong induction
hypothesis that $P(n-1)$ and $P(n)$ are both true.  We will use this
to prove $P(n+1)$.

That is, we may assume
\begin{align}
F(n-1) & = \frac{p^{n-1}-q^{n-1}}{\sqrt{5}}\label{Fn-1pn-1}\\
F(n) & = \frac{p^n-q^n}{\sqrt{5}}\label{Fnpnqn}\\.
\end{align}
From the hint we have that $p^2 = p+1$, which implies that $p^2p^{n-1}
= (p+1)p^{n-1}$ and so
\begin{equation}\label{pn+1pnpn-1}
p^{n+1} = p^n + p^{n-1}.
\end{equation}
Likewise $q^2 = q+1$, and so
\begin{equation}\label{qn+1qnqn-1}
q^{n+1} = q^n + q^{n-1}
\end{equation}
Subtracting~\eqref{qn+1qnqn-1} from~\eqref{pn+1pnpn-1} gives
\[
p^{n+1}-q^{n+1} = p^n-q^n+p^{n-1}-q^{n-1}
\]
and dividing by $\sqrt{5}$ yields
\begin{align}
\frac{p^{n+1}-q^{n+1}}{\sqrt{5}}
 & = \frac{p^n-q^n}{\sqrt{5}}+\frac{p^{n-1}-q^{n-1}}{\sqrt{5}}\notag\\
 & = F_n + F_{n-1} & \text{(by~\eqref{Fn-1pn-1} and~\eqref{Fnpnqn})}\label{pqFF}
\end{align}

But $F(n+1) = F(n) + F(n-1)$ for $n > 1$ by definition,
so~\eqref{pqFF} implies
\[
F(n+1) = \frac{p^{n+1}-q^{n+1}}{\sqrt{5}}.
\]
That is, $P(n+1)$ is true in this case as well.

We conclude by strong induction that $P(n)$ holds for all $n \in
\nngint$.
\end{proof}
\end{solution}

\end{problem}

%%%%%%%%%%%%%%%%%%%%%%%%%%%%%%%%%%%%%%%%%%%%%%%%%%%%%%%%%%%%%%%%%%%%%
% Problem ends here
%%%%%%%%%%%%%%%%%%%%%%%%%%%%%%%%%%%%%%%%%%%%%%%%%%%%%%%%%%%%%%%%%%%%%

\endinput
