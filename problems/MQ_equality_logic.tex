\documentclass[problem]{mcs}

\begin{pcomments}
  \pcomment{MQ_equality_logic}
  \pcomment{ARM 9/20/15}
\end{pcomments}

\pkeywords{
  predicate
  equality
  logic
}

%%%%%%%%%%%%%%%%%%%%%%%%%%%%%%%%%%%%%%%%%%%%%%%%%%%%%%%%%%%%%%%%%%%%%
% Problem starts here
%%%%%%%%%%%%%%%%%%%%%%%%%%%%%%%%%%%%%%%%%%%%%%%%%%%%%%%%%%%%%%%%%%%%%

\begin{problem}
Predicate Formulas whose only predicate symbol is equality are called
``pure equality'' formulas.  For example,
\begin{equation}
\forall x\, \forall y.\ x = y\tag{1-element}
\end{equation}
is a pure equality formula.  Its meaning is that there is exactly one
element in the domain of discourse.\footnote{Remember, a domain of
  discourse is not allowed to be empty.}  Another such formula is
\begin{equation}
\exists a\, \exists b\, \forall x.\ x=a \QOR x = b. \tag{$\leq 2$-elements}
\end{equation}
Its meaning is that there are at most two elements in the domain of
discourse.

A formula that is not a pure equality formula is
\begin{equation}
x \leq y.\tag{\text{\textbf{not}-pure}}
\end{equation}
Formula~(\textbf{not}-pure) uses the less-than-or-equal
predicate $\leq$ which is \emph{not} allowed.\footnote{In fact,
  formula~(\textbf{not}-pure) only makes sense when the domain elements
  are ordered, while pure equality formulas make sense over every
  domain.}

\bparts

\ppart Describe a pure equality formula that means that there are
\emph{exactly} two elements in the domain of discourse.

\examspace[2.0in]

\begin{solution}
A simple answer using the above formulas is:
\[
\text{formula~($\leq 2$-elements)} \QAND \QNOT(\text{formula~(1-element)}).
\]

An alternative is to use a variant of~($\leq 2$-elements):
\[
\exists a,b.\, a \neq b \QAND \forall x.\ x=a \QOR x = b.
\]
\end{solution}

\ppart Describe a pure equality formula that means that there are
\emph{exactly} three elements in the domain of discourse.

\examspace[2.0in]

\begin{solution}
We can say there are at most three elements using
\begin{equation}
\exists a,b,c\, \forall x.\ x=a \QOR x = b \QOR x =c.\tag{$\leq 3$-elements}
\end{equation}
Then exactly three elements can be expressed with
\[
\text{formula~($\leq 3$-elements)} \QAND 
\QNOT(\text{formula~($\leq 2$-elements)}).
\]

An alternative is
\[
\exists a,b,c.\ a \neq b \QAND b \neq c \QAND a \neq c \QAND \forall
x.\ x=a \QOR x = b \QOR x = c.
\]

\end{solution}

\eparts

\end{problem}

\endinput
