\documentclass[problem]{mcs}

\begin{pcomments}
  \pcomment{CP_implies_relation_on_propositional_formulas}
  \pcomment{about the same as CP_images_and-propositional_formulas,
    CP_implied_by_relation_on_propositional_formulas}
  \pcomment{was used as MQ_implies_relation_on_propositonal_formulas}
  \pcomment{from: Megumi Ando, 9/19/09, reworked by ARM}
\end{pcomments}

\pkeywords{
  propositional formula
  implies
  valid
  relation
  codomain
  image
  inverse
  graph_of_relation
}

%%%%%%%%%%%%%%%%%%%%%%%%%%%%%%%%%%%%%%%%%%%%%%%%%%%%%%%%%%%%%%%%%%%%%
% Problem starts here
%%%%%%%%%%%%%%%%%%%%%%%%%%%%%%%%%%%%%%%%%%%%%%%%%%%%%%%%%%%%%%%%%%%%%

\begin{problem}
  Let $A$ be the following set of five propositional formulas shown below
  on the left, and let $C$ be the set of three propositional formulas on
  the right.  The ``implies'' binary relation, $I$, from $A$ to $C$
  is defined by the rule
\[
F \mrel{I} G \qiff [\text{the formula } (F \QIMPLIES G) \text{ is valid}].
\]
For example, $(P \QAND Q) \mrel{I} P$, because the formula $(P \QAND Q)$
does imply $P$.  Also, it is not true that $(P \QOR Q) \mrel{I} P$
since $(P \QOR Q) \QIMPLIES P$ is not valid.

  \bparts
  \ppart Fill in the arrows so the following figure describes the graph of
  the relation, $I$:

\[\begin{array}{lcr}
A & \hspace{1in} \text{arrows} \hspace{1in} & C\\
\hline
&\\
M\\
&\\
                                  && M \QAND (P \QIMPLIES M)\\
&\\
P \QAND Q\\
&\\
                                  && Q\\
&\\
P \QOR Q\\
&\\
                                  && \bar{P} \QOR \bar{Q}\\
&\\
\QNOT(P \QAND Q)\\
&\\
&\\
P \QXOR Q\\
&\\
\end{array}\]

 \begin{solution}
Four arrows for $I$:
\begin{align*}
M & \qiff & M \QAND (P \QIMPLIES M)\\
P \QAND Q & \qimplies & Q\\
\QNOT(P \QAND Q) & \qiff & \bar{P} \QOR \bar{Q}\\
P \QXOR Q & \qimplies & \bar{P} \QOR \bar{Q}
\end{align*}
So the ``implies'' relation, $I$, is a surjective function.  That means
its inverse, $\inv{I}$, is a total injection.
\end{solution}

\ppart Circle the properties below possessed by the
 relation $I$:
  \[
  \begin{array}{ccccc}
  \mbox{function~~} & 
  \mbox{~~total~~} &
  \mbox{~~injective~~} &
  \mbox{~~surjective~~} &
  \mbox{~~bijective} 
  \end{array}
  \]

  \ppart  Circle the properties below possessed by the relation $\inv{I}$:
  \[
  \begin{array}{ccccc}
  \mbox{function~~} & 
  \mbox{~~total~~} &
  \mbox{~~injective~~} &
  \mbox{~~surjective~~} &
  \mbox{~~bijective~}
  \end{array}
  \]

\iffalse
%THERE is a good problem idea here, but what's written is garbled
%and needs work.

Here are some two groups of expressions involving images and inverse
images under the relation $I$:
\begin{enumerate}
\item
${P,Q}I$

\begin{solution}
$\set{(R \QOR \bar{R})}$.

${P,Q}I$ is by definition equal to the expressions in $C$ that are either
implied by $P$ or implied by $Q$.  The only expression in $C$ implied by
$P$ is the valid expression $(R \QOR \bar{R})$; this is also the only
expression in $C$ implied by $Q$.
\end{solution}

\item $\set{(P \QOR Q)}I \intersect \set{\QNOT(P\ \QAND\ Q)}I$

\begin{solution}
\[\begin{array}{l}
(P \XOR Q)I\\
\set{(\bar{P} \QOR \bar{Q}), (R \QOR \bar{R})}
\end{array}\]

These are the expressions in $C$ implied by both $(P \QOR Q)$ and also by
$\QNOT(P\ \QAND\ Q)$, which is the same as being implied by $P \XOR Q$.

\end{solution}

\item \set{(P \QAND Q),\QNOT(P\ \QAND\ Q)}I

\begin{solution}
These are the expressions in $C$ implied by the formula $(P \QAND Q)$ or by
$\QNOT(P \QAND Q)$

\end{solution}
\end{enumerate}
\fi

\iffalse

\ppart If we change the codomain $C$ to include $\bar{P}$, does your
answer to part~eqref{Iprops} change?  Explain your answer.

\fi

  \eparts

\end{problem}

%%%%%%%%%%%%%%%%%%%%%%%%%%%%%%%%%%%%%%%%%%%%%%%%%%%%%%%%%%%%%%%%%%%%%
% Problem ends here
%%%%%%%%%%%%%%%%%%%%%%%%%%%%%%%%%%%%%%%%%%%%%%%%%%%%%%%%%%%%%%%%%%%%%

\endinput
