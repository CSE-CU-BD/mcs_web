\documentclass[problem]{mcs}

\begin{pcomments}
  \pcomment{MQ_reindeer_games_f15}
  \pcomment{from F05.q1}
  \pcomment{variant version is MQ_reindeer_games_without_paths}
  \pcomment{don't use with MQ_tennis_match_partial_order}
  
\end{pcomments}

\pkeywords{
  DAG's
  linear
  paths
  tournament
  chain
  antichain
  partial_orders
}

%%%%%%%%%%%%%%%%%%%%%%%%%%%%%%%%%%%%%%%%%%%%%%%%%%%%%%%%%%%%%%%%%%%%%
% Problem starts here
%%%%%%%%%%%%%%%%%%%%%%%%%%%%%%%%%%%%%%%%%%%%%%%%%%%%%%%%%%%%%%%%%%%%%

\begin{problem}
Each year, Santa's reindeer hold ``Reindeer Games'', from which
Rudolph is pointedly excluded.  The Games consist of a sequence of
matches, where one reindeer competes against another.  Draws are not
possible.

On Christmas Eve, Santa produces a rank list of all his competing
reindeer.  If reindeer $p$ lost a match to reindeer $q$, then $p$
appears below $q$ in Santa's ranking, but if he has any choice because
of unplayed matches, Santa can give higher rank to the reindeer he
likes better.  To prevent confusion, two reindeer may not play a match
if either outcome would lead to a cycle of reindeer, where each lost
to the next.

\iffalse
Here is an example of a cycle:
\begin{figure}[h]
\graphic[height=1.75in]{cycle}
\end{figure}
\fi

Though it is only beginning of November, the 2015 Reindeer Games have
already begun (punctuality is key at the north pole).  We can describe
the results of the matches played so far with a binary relation $L$
on the set of reindeer, where $p L q$ means that reindeer $p$ lost a
match to reindeer $q$.  Let $L^+$ be the corresponding positive-length
walk relation\footnote{Thus, reindeer $p$ is related to reindeer $q$
  by $L^+$ if $p$ lost to $q$ or if $p$ lost to a reindeer who lost to
  $q$ or if $p$ lost to a reindeer who lost to a reindeer who lost to
  $q$, etc.}.  Note that $L^+$ is a partial order, so we can regard a
match loser as ``smaller'' than the winner.

\bigskip

Below is a list of terms and a sequence of statements.  Add the
appropriate term to each statement.

%\examspace

\begin{center}
\textbf{Terms}

\begin{tabular}{lll}
 a strict partial order  & a weak partial order   & a linear order\\
 comparable elements     & incomparable elements  & a chain\\
 an antichain            & a maximal antichain    & a topological sort\\
 a minimum element       & a minimal element\\
 a maximum element       & a maximal element
\end{tabular}
\end{center}

\centerline{\textbf{Statements}}

\bparts

\ppart A reindeer who \textbf{so far} is unbeaten is \brule{3in} of
the partial order $L^+$. \begin{solution} a maximal element
\end{solution}

\ppart A reindeer who \textbf{so far} has lost every match is
\brule{3in} of the partial order $L^+$. \begin{solution} a minimal
  element
\end{solution}

\ppart Two reindeer can\emph{not} play a match if they are
\brule{3in} of $L^+$. \begin{solution}
comparable elements
\end{solution}

\ppart A reindeer who \emph{must} come first in Santa's current
ranking is \brule{3in} of $L^+$. \begin{solution} a maximum element
\end{solution}

\ppart A sequence of reindeer which \emph{must} appear in the same
order in Santa's rank list is
\brule{2in}. \begin{solution}
a chain
\end{solution}

\ppart A set of reindeer such that any two could still play a match
is
\brule{3in}. \begin{solution}
an antichain
\end{solution}

\ppart The fact that no reindeer loses a match to himself implies
that $L^+$ is
\brule{3in}.  \begin{solution}
a strict partial order
\end{solution}

\ppart Santa's final ranking of his reindeer on Christmas Eve must be
\brule{3in} of $L^+$.  \begin{solution}
a topological sort
\end{solution}

\ppart No more matches are possible if and only if $L^+$ is
\brule{3in}.  \begin{solution}
a linear order
\end{solution}

\ppart Suppose that Santa has 11 competing reindeer.  If no more
matches can be played, what is the smallest possible number of matches
already played? \examrule[0.3in]

\begin{solution}
\textbf{10}.

Call the reindeer $r_1, \dots, r_{11}$.  Then no more matches can be
played if there were 10 matches in which $r_1$ lost to $r_2$, $r_2$
lost to $r_3$, etc.

If fewer than ten matches have been played, then the corresponding
digraph would not be connected even ignoring edge directions\inbook{
  (by Theorem~\bref{th:connectivity})}.  So there would be two
nonempty sets of reindeer, $S$ and $T$, with no reindeer in $S$ having
played a match against a reindeer in $T$, in which case at least one
more match could be held between a reindeer in $S$ and a reindeer in
$T$.
\end{solution}

\eparts

\end{problem}

%%%%%%%%%%%%%%%%%%%%%%%%%%%%%%%%%%%%%%%%%%%%%%%%%%%%%%%%%%%%%%%%%%%%%
% Problem ends here
%%%%%%%%%%%%%%%%%%%%%%%%%%%%%%%%%%%%%%%%%%%%%%%%%%%%%%%%%%%%%%%%%%%%%

\endinput
