\documentclass[problem]{mcs}

\begin{pcomments}
  \pcomment{MQ_asymptotic_true_false}
  \pcomment{from S10}
\end{pcomments}

\pkeywords{
  asymptotic
}

%%%%%%%%%%%%%%%%%%%%%%%%%%%%%%%%%%%%%%%%%%%%%%%%%%%%%%%%%%%%%%%%%%%%%
% Problem starts here
%%%%%%%%%%%%%%%%%%%%%%%%%%%%%%%%%%%%%%%%%%%%%%%%%%%%%%%%%%%%%%%%%%%%%

\begin{problem}
Assign true or false for each statement and prove it.

\begin{itemize}

\item $\displaystyle n^2 \sim n^{2} + n$

\item $\displaystyle 3^n=O\paren{2^n}$

\item $\displaystyle n^{\sin (n\pi/2) + 1} = o\paren{n^2}$

%\item $\displaystyle \ln \left(\left(n^2\right)!\right) = O\left(n^2\ln(n)\right)$ 

\item $\displaystyle n = \Theta \paren{\frac{3n^3}{(n+1)(n-1)}}$

\end{itemize}

\begin{solution}The 1st and 4th statements are true.

\begin{itemize}

\item $\frac{n^{2} + n}{n^2} = \frac{n^2}{n^2} + \frac{n}{n^2} =
  1+\frac{1}{n}$, so as $n$ approaches infinity, the ratio approaches
  $1 + 0 = 1$.  Therefore the two expressions are similar.

\item $\lim_{n\to\infty}{\frac{3^n}{2^n}} = \lim_{n\to\infty}{(3/2)^n}
  = \infty$

\item The left side never exceeds $n^2$, but when $n =
  1,5,9,13,\dots$, the left side is equal to $n^2$, and so is not
  $o\paren{n^2}$.

\item
\[
\lim_{n\to\infty} \frac{n}{\frac{3n^3}{(n+1)(n-1)}} =
\lim_{n\to\infty} \frac{n(n+1)(n-1)}{3n^3} = \frac{1}{3},
\]
so $n = O\paren{\frac{3n^3}{(n+1)(n-1)}}$.  Similarly,
\[
\lim_{n\to\infty} \frac{\frac{3n^3}{(n+1)(n-1)}}{n} = 3,
\]
so $\frac{3n^3}{(n+1)(n-1)} = O(n)$.  Because the two expressions are
big-O of each other, $n = \Theta \paren{\frac{3n^3}{(n+1)(n-1)}}$.
\end{itemize}

\end{solution}

\end{problem}

%%%%%%%%%%%%%%%%%%%%%%%%%%%%%%%%%%%%%%%%%%%%%%%%%%%%%%%%%%%%%%%%%%%%%
% Problem ends here
%%%%%%%%%%%%%%%%%%%%%%%%%%%%%%%%%%%%%%%%%%%%%%%%%%%%%%%%%%%%%%%%%%%%%


\endinput
