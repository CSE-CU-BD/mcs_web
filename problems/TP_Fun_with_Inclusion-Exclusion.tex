\documentclass[problem]{mcs}

\begin{pcomments}
    \pcomment{TP_Fun_with_Inclusion-Exclusion}
    \pcomment{expanded by CH, Spring '14}
\end{pcomments}

%%%%%%%%%%%%%%%%%%%%%%%%%%%%%%%%%%%%%%%%%%%%%%%%%%%%%%%%%%%%%%%%%%%%%
% Problem starts here
%%%%%%%%%%%%%%%%%%%%%%%%%%%%%%%%%%%%%%%%%%%%%%%%%%%%%%%%%%%%%%%%%%%%%

\begin{problem}

\bparts
Suppose that we select a positive integer $n \leq 100$ at random.

\ppart\label{ppart:inc_ex}
What is the probability that $n$ is divisible by 5 or 7?

\begin{solution}
Let $A$ denote the event that $n$ is divisible
by 5 and $B$ denote the event that it is divisible by 7. By the Law of Inclusion-Exclusion,
\begin{align*}
\pr{A \union B}
    & = \pr{A} + \pr{B}
         - \pr{A \intersect B} \\
    & = 20/100 + 14/100 - 2/100 \\
    & = 32/100 = 0.32 .
\end{align*}
\end{solution}

\examspace[1.5in]

\ppart Use your answer to part~\ref{ppart:inc_ex} to argue why the following statement is \textbf{incorrect}:
\begin{quote}
Fix positive integers $p$ and $q$. Then, the events ``$p$ divides
$n$'' and ``$q$ divides $n$'' are independent iff $p$ and $q$ are
relatively prime.
\end{quote}

\begin{solution}
Set $p = 5$ and $q = 7$. Then,  as calculated above, $\pr{A} = 0.2$
and $\pr{B} = 0.14$. However, $p(A \intersect B) = 0.02 \neq p(A)
P(B)$. Therefore, $A$ and $B$ are not independent.
\end{solution}

\eparts

\end{problem}

%%%%%%%%%%%%%%%%%%%%%%%%%%%%%%%%%%%%%%%%%%%%%%%%%%%%%%%%%%%%%%%%%%%%%
% Problem starts here
%%%%%%%%%%%%%%%%%%%%%%%%%%%%%%%%%%%%%%%%%%%%%%%%%%%%%%%%%%%%%%%%%%%%%

\endinput
