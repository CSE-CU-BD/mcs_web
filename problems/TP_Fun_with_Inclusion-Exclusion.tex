\documentclass[problem]{mcs}

\begin{pcomments}
    \pcomment{TP_Fun_with_Inclusion-Exclusion}
    \pcomment{expanded by CH, Spring '14}
\end{pcomments}

\pkeywords{
  probability
  conditional_probability
  independence
  divides
  prime
}

%%%%%%%%%%%%%%%%%%%%%%%%%%%%%%%%%%%%%%%%%%%%%%%%%%%%%%%%%%%%%%%%%%%%%
% Problem starts here
%%%%%%%%%%%%%%%%%%%%%%%%%%%%%%%%%%%%%%%%%%%%%%%%%%%%%%%%%%%%%%%%%%%%%

\begin{problem}

\bparts
Suppose that we select a positive integer $n \leq 100$ at random.

\ppart What is the probability that $n$ is divisible by 5?\hfill \examrule
\examspace[0.4in]

\begin{solution}
\textbf{0.2}

\[
\pr{5 \divides n} = \floor{100/5}/100 = 0.2.
\]
\end{solution}

\ppart What is the probability that $n$ is divisible by 7?\hfill \examrule
\examspace[0.4in]

\begin{solution}
\textbf{0.14}

\[
\pr{5 \divides n} = \floor{100/7}/100 = 0.14.
\]
\end{solution}

\ppart Show that the events ``5 divides $n$'' and ``7 divides $n$''
are not independent.
\examspace[1.5in]

\begin{solution}
As calculated above, $\pr{5 \divides n} = 0.2$ and $\pr{7 \divides n}
= 0.14$.  However,
\[
\pr{\text{both}} = \floor{100/(5 \cdot 7)}/100 =  0.02 \neq 0.028 = 0.2 \cdot 0.14.
\]

Therefore, the events are not independent.
\end{solution}

\ppart\label{ppart:inc_ex}
What is the probability that $n$ is divisible by 5 or 7?\hfill \examrule

\begin{solution}
\textbf{0.32}

Let $A$ denote the event that $n$ is divisible by 5 and $B$ denote the
event that it is divisible by 7.  By the Law of Inclusion-Exclusion,
\begin{align*}
\pr{A \union B}
    & = \pr{A} + \pr{B}
         - \pr{A \intersect B} \\
    & = \frac{\floor{100/5}}{100} + \frac{\floor{100/7}}{100} - \frac{\floor{100/(5\cdot 7)}}{100} \\
    & = \frac{20 + 14 - 2}{100}  = 0.32 .
\end{align*}
\end{solution}

\eparts

\end{problem}

%%%%%%%%%%%%%%%%%%%%%%%%%%%%%%%%%%%%%%%%%%%%%%%%%%%%%%%%%%%%%%%%%%%%%
% Problem starts here
%%%%%%%%%%%%%%%%%%%%%%%%%%%%%%%%%%%%%%%%%%%%%%%%%%%%%%%%%%%%%%%%%%%%%

\endinput
