\documentclass[problem]{mcs}

\begin{pcomments}
  \pcomment{MQ_iff_cnf_words}
  \pcomment{Inspired by textbook problem 3.2.1}
  \pcomment{Edited by dganelin 9/15/17. ARM 9/16/17}
\end{pcomments}

\pkeywords{
  proposition
  DNF
  CNF
  iff
}

%%%%%%%%%%%%%%%%%%%%%%%%%%%%%%%%%%%%%%%%%%%%%%%%%%%%%%%%%%%%%%%%%%%%%
% Problem starts here
%%%%%%%%%%%%%%%%%%%%%%%%%%%%%%%%%%%%%%%%%%%%%%%%%%%%%%%%%%%%%%%%%%%%%
\begin{problem}

\bparts \mbox{}

\inbook{\ppart Write a CNF (Conjunctive Normal Form) formula for
\[
P \QIFF Q.
\]

%\examspace[0.5in]

\begin{solution}
\[
(p \QOR \bar{Q}) \QAND (\bar{Q} \QOR P)
\]
\end{solution}
}

\ppart Five of the 6.042 TA's---Akhil, Daniela, Maggie, Preksha and
Sibo---are deciding whether to go to the Propositional Party together
this weekend.  Let $A$ be the proposition that \textbf{A}khil will
go to the Propositional Party, $D$ that \textbf{D}aniela will go, and
likewise $M$ for \textbf{M}aggie, and $P$ for \textbf{P}reksha, and
$S$ for Sibo.

Using these propositions, translate the following sentence
directly into a Full\footnote{``Full'' means each $\QAND$ term contains
  all of $A, D, M, P, S$.} propositional
formula in Disjunctive Normal Form.
\begin{equation}\tag{*}
\text{All five TA's will go to the Propositional Party, or none of them will.}
\end{equation}
\examspace[1in]

\begin{solution}
\[
(\QAND\ A\ D\ M\ P\ S) \QOR (\QAND \bar{A}\ \bar{D}\ \bar{M}\ \bar{P}\ \bar{S})
\]
\end{solution}

\examspace[1in]

\ppart How many \QOR-of-literals subformulas are there in the Full
\textbf{Con}junctive Normal Form that expresses~(*)?

\begin{center}
\exambox{0.6in}{0.6in}{-0.1in}
\end{center}


\begin{solution}
$30 = 2^5 -2$.

This is the number of \False\ entries in the 32 row truth table
for~(*), since there are only two \True\ entries.
\end{solution}

\eparts

\end{problem}

%%%%%%%%%%%%%%%%%%%%%%%%%%%%%%%%%%%%%%%%%%%%%%%%%%%%%%%%%%%%%%%%%%%%%
% Problem ends here
%%%%%%%%%%%%%%%%%%%%%%%%%%%%%%%%%%%%%%%%%%%%%%%%%%%%%%%%%%%%%%%%%%%%%%
\endinput
