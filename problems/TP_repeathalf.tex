\documentclass[problem]{mcs}

\begin{pcomments}
  \pcomment{TP_repeathalf}
  \pcomment{subsumed by second part of MQ_infinite_repeat}
  \pcomment{final F13}
  \pcomment{by ARM 12/13/13}
\end{pcomments}

\pkeywords{
  expectation
  failure
  infinite
  geometric_series
}

%%%%%%%%%%%%%%%%%%%%%%%%%%%%%%%%%%%%%%%%%%%%%%%%%%%%%%%%%%%%%%%%%%%%%
% Problem starts here
%%%%%%%%%%%%%%%%%%%%%%%%%%%%%%%%%%%%%%%%%%%%%%%%%%%%%%%%%%%%%%%%%%%%%

\begin{problem}
You have a biased coin with nonzero probability $p<1$ of tossing a
Head.  You toss until a Head comes up and record the number of Tails
that preceded this first Head.  Then\inbook{, similar to the example
  in Section~\bref{infinite_expect_sec},} you keep tossing until you
get another Head preceded by at least half as many consecutive Tails.
Prove that the expected number of Heads you toss is finite.  (The
proof is slightly simpler if the number of Tails preceding the first
Head is an even number.  It is OK to assume this.)

\begin{solution}
Let the random variable $T$ be the length of your initial run of
tails.  Suppose $T$ takes the value $k$, that is, you flip $k$
consecutive Tails followed by a Head.  Then you try flipping
until you get another Head.  If the number of consecutive Tails before
the last Head is $< k/2$, you try the same thing again.

The probability that one \textbf{try} will result in $\geq k/2$ Tails
is $q^{\ceil{k/2}}$, where $q \eqdef 1-p$.  So thinking of finally
flipping enough Tails as a ``failure,'' the expected number of tries
is the mean time to failure, namely, $1/q^{\ceil{k/2}}$.

Letting $H$ be the number of Heads that appear at the end of a try, we
have
\begin{align*}
\expect{H}
    & = \sum_{k \in \nngint} \expcond{H}{T=k} \cdot \prob{T=k}\\
    & = \sum_{k \in \nngint} \frac{1}{q^{\ceil{k/2}}} \cdot q^k p\\
    & \leq \sum_{k \in \nngint} \frac{1}{q^{k/2}} \cdot q^k p\\
    & = p \sum_{k \in \nngint}  q^{k/2}\\
    & = p\sum_{k \in \nngint} \paren{\sqrt{q}}^k \\
    & = \frac{p}{1- \sqrt{q}} < \infty.
\end{align*}
\end{solution}

\end{problem}

%%%%%%%%%%%%%%%%%%%%%%%%%%%%%%%%%%%%%%%%%%%%%%%%%%%%%%%%%%%%%%%%%%%%%
% Problem ends here
%%%%%%%%%%%%%%%%%%%%%%%%%%%%%%%%%%%%%%%%%%%%%%%%%%%%%%%%%%%%%%%%%%%%%

\endinput
 
