\documentclass[problem]{mcs}

\begin{pcomments}
  \pcomment{MQ_coloring}
\end{pcomments}

\pkeywords{
}

%%%%%%%%%%%%%%%%%%%%%%%%%%%%%%%%%%%%%%%%%%%%%%%%%%%%%%%%%%%%%%%%%%%%%
% Problem starts here
%%%%%%%%%%%%%%%%%%%%%%%%%%%%%%%%%%%%%%%%%%%%%%%%%%%%%%%%%%%%%%%%%%%%%

\begin{problem}
% 

\begin{problemparts}

\problempart
Consider the graph shown in Figure~\ref{fig:to_color}.  Determine a valid coloring of the graph, 
using as few colors as possible. (Simply write your proposed color for each 
vertex next to that vertex.  You may use $r$ for red, $g$ for green, etc.)

\begin{figure}[h]
\graphic{MQ_coloring}
\caption{\label{fig:to_color}}
\end{figure}
\examspace[1in]
\begin{solution}
There are odd-length cycles in the graph, so at least three colors will be needed.
So assume that three colors are sufficient.  (If we encounter a contradiction under this assumption, 
we will need to use more colors.)  Start with the length-3 cycle $abda$.  
All of its vertices must be colored differently, so assign red to $a$, blue to $b$, and green to $d$.  The 
length-3 cycle $bdhb$ now forces $h$ to be colored red.  $f$ must now be colored green and $g$ must be 
colored blue.  The coloring is valid so far.  $c$ is adjacent to a blue vertex and a green vertex, and no 
others, it must be colored red.  Finally, $e$ is not adjacent to any other vertices, so it can be assigned 
any of the three colors.  Choosing red for $e$, the result is shown in Figure~\ref{fig:colored}.  
There is no pair of like-colored adjacent vertices, so this coloring is valid.   
\begin{figure}[h]
\graphic{MQ_coloring_sol}
\caption{A valid coloring.\label{fig:colored}}
\end{figure}
\end{solution}

\problempart
What is the chromatic number of the graph?
\examspace[1in]
\begin{solution}
Three.  As already discussed, fewer than three colors are insufficient, and three colors can be used to 
create a valid coloring.
\end{solution}

\problempart
Is it possible to increase the chromatic number of the graph by adding just one edge?  If yes, state which 
new edge would do the trick.  If no, explain why.
\examspace[2in]
\begin{solution}
Certainly.  There are a few possibilies.  For instance, adding $\edge{a}{c}$ creates a subgraph isomorphic 
to $K_4$.  Since $\chi(K_4)=4$, the resulting graph would have a chromatic number of at least 
four.  (Exaclty four, in fact.)
\end{solution}

\problempart
Is it possible to decrease the chromatic number of the graph by removing just one edge?  If yes, state 
which edge could be removed to do the trick.  If no, explain why.
\examspace[2in]
\begin{solution}
No.  No matter which edge is removed, the graph will still contain odd-length cycles as subgraphs.  The 
chromatic number of the graph would thus still be at least three.  (Exactly three, actually.) 
\end{solution}

\end{problemparts}
\end{problem}

\endinput
