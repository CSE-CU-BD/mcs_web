\documentclass[problem]{mcs}

\begin{pcomments}
  \pcomment{TP_recset_circular}
  \pcomment{ARM 3/8/18}
\end{pcomments}

\pkeywords{
  recursive
  set_theory
}

\begin{problem}
We know that every pure set is a recursive set
\recset\inbook{\ Theorem~\bref{thm_recursive_sets}}, so why we can't
dispense with the axioms of set theory (ZFC) and just use the
recursive definition to define sets?

\begin{solution}
The definition of \recset\ specifies how to form a recursive set from
\emph{a set} of elements that are recursive sets.  So we need to know
what sets are before we can understand what recursive sets are.  In
other words, trying to \emph{define} the class of all sets to be
\recset\ would be circular.
\end{solution}

\end{problem}

\endinput


