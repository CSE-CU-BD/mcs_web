\documentclass[problem]{mcs}

\begin{pcomments}
  \pcomment{CP_fast_exponentiation}
  \pcomment{from: S09.cp5t}
  \pcomment{DO NOT USE IN BOOK OR ON PSET: subsumed by the notes as of Feb '11 & video in S12}
\end{pcomments}

\pkeywords{
  state_machines
  termination
  partial_correctness
  invariant
  exponentiation
  algorithm
}

%%%%%%%%%%%%%%%%%%%%%%%%%%%%%%%%%%%%%%%%%%%%%%%%%%%%%%%%%%%%%%%%%%%%%
% Problem starts here
%%%%%%%%%%%%%%%%%%%%%%%%%%%%%%%%%%%%%%%%%%%%%%%%%%%%%%%%%%%%%%%%%%%%%

\begin{problem}
The most straightforward way to compute the $b$th power of a number $a$
is to multiply $a$ by itself $b$ times.  This of course requires $b-1$
multiplications.  There is another way to do it using considerably fewer
multiplications.  This algorithm is called \term{fast exponentiation}:

Given inputs $a \in \reals, b \in \nngint$,
initialize registers $x,y,z$ to $a,1,b$ respectively,
and repeat the following sequence of steps until termination:
\begin{itemize}\renewcommand{\itemsep}{0pt}
\item if $z = 0$ \textbf{return} $y$ and terminate
\item $r := \text{remainder}(z,2)$
\item $z := \quotient(z,2)$
\item if $r = 1$, then $y := xy$
\item $x := x^2$
\end{itemize}
We claim this algorithm always terminates and leaves $y = a^b$.

\bparts

\ppart Model this algorithm with a state machine, carefully defining the
states and transitions.

\begin{solution}
\begin{enumerate}
\item The set of states is $\reals \cross \reals \cross \nngint$,
\item The start state is $(a,1,b)$,
\item the transitions are defined by the rule
\begin{equation*}
(x,y,z) \rightarrow
\begin{cases}
(x^2, y, \quotient(z,2)) & \text{if $z$ is positive and even},\\
(x^2, xy, \quotient(z,2)) & \text{if $z$ is positive and odd}.
\end{cases}
\end{equation*}
\end{enumerate}

\end{solution}

\ppart Verify that the predicate $P((x,y,z)) \eqdef\ [yx^z = a^b]$ is a
preserved invariant.

\begin{solution}
We show that $P$ is preserved, namely, assuming $P((x,y,z))$, that is,
\begin{equation}\label{yxzd}
yx^z = a^b
\end{equation}
holds and $(x,y,z) \rightarrow (x_t,y_t,z_t)$ is a transition, then
$P((x_t,y_t,z_t))$, that is,
\[
y_tx_t^{z_t} = a^b
\]
holds.

We consider two cases:

If $z > 0$ and is even, then we have that $x_t = x^2, y_t = y, z_t =
\quotient(z,2)$.  Therefore,
\begin{align*}
y_tx_t^{z_t} &  = yx^{2 \cdot \quotient(z,2)}\\
           & = yx^{2 \cdot (z/2)}\\
           & = yx^z\\
          & = a^b & \mbox{(by~\eqref{yxzd})}
\end{align*}

If $z > 0$ and is odd, then we have that $x_t = x^2, y_t = xy, z_t =
\quotient(z,2)$. Therefore,
\begin{align*}
y_tx_t^{z_t} & = xyx^{2 \cdot \quotient(z,2)}\\
& = yx^{1+2 \cdot (z-1)/2}\\
& = yx^{1+(z-1)}\\
& = yx^z\\
& = a^b & \mbox{(by~\eqref{yxzd})}
\end{align*}

So in both cases, $P((x_t,y_t,z_t))$ holds, proving that $P$ is a
preserved invariant.
\end{solution}

\ppart Prove that the algorithm is partially correct: if it halts, it does
so with $y=a^b$.

\begin{solution}
  $P$ holds for the start state $(a,1,b)$ since $1\cdot a^b = a^b$.  So by
  the Invariant Theorem, $P$ holds for all reachable states.  But a
  terminal state must have $z = 0$, so if any terminal state $(x,y,0)$ is
  reachable, then $y = yx^0 = a^b$ as required.
\end{solution}

\ppart Prove that the algorithm terminates.

\begin{solution}
Just notice that $z$ is a natural-number-valued variable that
gets smaller at every transition.  So by the Well-Ordering Principle, when
this variable reaches its minimum value, the algorithm terminates.
\end{solution}

\ppart In fact, prove that it requires at most $2 \ceil{\log_2 (b+1)}$
multiplications for the Fast Exponentiation algorithm to compute $a^b$ for
$b>1$.

\begin{solution}
The value of $z$ is initially $b$ and gets at least halved at
every step.  So it can't be halved more than $\ceil{\log_2 (b+1)}$ times before
hitting zero. We need $(b+1)$ because for $b = 2^p$, a power of two,
it takes $(p+1)$ halves to get zero. Since each of the transitions
involves at most two multiplications, the total number of multiplications
until $z=0$ is at most $2\ceil{\log_2 (b+1)}$.
\end{solution}

\eparts
\end{problem}

%%%%%%%%%%%%%%%%%%%%%%%%%%%%%%%%%%%%%%%%%%%%%%%%%%%%%%%%%%%%%%%%%%%%%
% Problem ends here
%%%%%%%%%%%%%%%%%%%%%%%%%%%%%%%%%%%%%%%%%%%%%%%%%%%%%%%%%%%%%%%%%%%%%

\endinput
