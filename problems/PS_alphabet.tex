\documentclass[problem]{mcs}

\begin{pcomments}
  \pcomment{PS_alphabet}
  \pcomment{from: S08 ps9, revised by ARM 11/14/09}
  \pcomment{soln to 2n-into-pairs part revised 5/12/13 ARM}
  \pcomment{subsumes PS_misc_counting}
\end{pcomments}

\pkeywords{
  binomial_coefficient
  string
  digits
}

%%%%%%%%%%%%%%%%%%%%%%%%%%%%%%%%%%%%%%%%%%%%%%%%%%%%%%%%%%%%%%%%%%%%%
% Problem starts here
%%%%%%%%%%%%%%%%%%%%%%%%%%%%%%%%%%%%%%%%%%%%%%%%%%%%%%%%%%%%%%%%%%%%%

\begin{problem}
Answer the following questions with a number or a simple formula
involving factorials and binomial coefficients.  Briefly explain your answers.

\bparts

\ppart How many ways are there to order the 26 letters of the
alphabet so that no two of the vowels {\tt a}, {\tt e}, {\tt i},
{\tt o}, {\tt u} appear consecutively and the last letter in the
ordering is not a vowel?

\hint  Every vowel appears to the left of a consonant.

\begin{solution}
The constraint on where vowels can appear is equivalent to the
requirement that every vowel appears to the left of a consonant.  So given
a sequence of the 21 consonants, there are $\binom{21}{5}$ positions where
the 5 vowels can be placed.  After determining such a placement, we can
reorder the consonants and vowels in any order.  Thus, the number is:
\[
\binom{21}{5}\cdot 21! \cdot 5!.
\]
\end{solution}

%was commented out in S08

\ppart How many ways are there to order the 26 letters of the alphabet so
that there are \emph{at least two} consonants immediately following each
vowel?

\begin{solution}

The pattern of consonants and vowels in any permutation of the
26 letters of the alphabet can be indicated by a binary string with 5
ones indicating where the vowels occur and 21 zeros where the consonants
occur.  Patterns where every vowel has at least two consonants to its
right can be constructed by taking a sequence of 16 zeros and inserting
``10'' to the left of 5 of the 16 zeros.  There are $\binom{16}{5}$ ways to
do this.  For any such pattern, there are $5!$ ways to place the vowels
in the positions where ones occur and $21!$ ways to place the consonants
where the zeroes occur.  Thus, the final answer is:
\[
\binom{16}{5}\cdot 5! \cdot 21!.
\]

\end{solution}

%%%%%%%
\iffalse
\ppart In how many different ways can the letters in the name of the
popular 1980's band \texttt{BANANARAMA} be arranged?

\begin{solution}
There are 5 $A$'s, 2 $N$'s, 1 $B$, 1 $R$, and 1 $M$.
Therefore, by the Bookkeeper Rule, the number of arrangements is:
\[
\frac{10!}{5!\ 2!\ 1!\ 1!\ 1!}
\]
\end{solution}

\fi

\ppart In how many different ways can $2n$ students be paired
up?


\begin{solution}

\begin{equation}\label{2nfacover} 
\frac{(2n)!}{n! 2^n}.
\end{equation}

There are $(2n)!$ permutations of the $2n$ people.  A permutation can
be mapped to a pairing up of the $2n$ people by pairing consecutive
people in the permutation.  That is, one pair consists of the first
and second people, another pair of the third and fourth people,
through an $n$th pair of the $(2n-1)$st and $2n$th people in the
permutation.

Two permutations will map to the same set of pairs iff one permutation
can be changed into the other permuting the order of the consecutive
pairs or by switching the elements of a pair.  Since there are $n$
consecutive pairs, there are $n!$ ways to permute the pairs and $2^n$
ways to switch the order within pairs.  So the mapping from
permutations to sets of pairs is $n!2^n$.  Now the Division
Rule~\bref{division_rule_sec} implies that the number of ways to
divide $2n$ people into $n$ pairs is given by~\eqref{2nfacover}.
\end{solution}

\iffalse

\ppart How many simple graphs are there with $n$ vertices numbered $1,
\dots n$?

\begin{solution}There are $\binom{n}{2}$ potential edges, each of which may or
may not appear in a given graph.  Therefore, the number of graphs is:
\[
2^{\binom{n}{2}}
\]
\end{solution}
\fi

\ppart Two $n$-digit sequences of digits 0,1,\dots,9 are said to be of the
\emph{same type} if the digits of one are a permutation of the digits of
the other.  For $n=8$, for example, the sequences \texttt{03088929} and
\texttt{00238899} are the same type.  How many types of $n$-digit
sequences are there?

\begin{solution}
  The type of a string is determined by the numbers of occurrences of the
  9 different digits in the string.  So there is a bijection between types
  of strings and strings with $n$ 0's and nine 1's: the length of the
  block of 0's before the $i$th 1 equals the number of occurrences of the
  digit $i$, and the length of the final block of 0's equals the number of
  occurrences of the digit 9.  Therefore, the number of different types is
  $\binom{n+9}{9}$
\end{solution}

\eparts

\end{problem}


%%%%%%%%%%%%%%%%%%%%%%%%%%%%%%%%%%%%%%%%%%%%%%%%%%%%%%%%%%%%%%%%%%%%%
% Problem ends here
%%%%%%%%%%%%%%%%%%%%%%%%%%%%%%%%%%%%%%%%%%%%%%%%%%%%%%%%%%%%%%%%%%%%%

\endinput
