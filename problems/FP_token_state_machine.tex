\documentclass[problem]{mcs}

\begin{pcomments}
  \pcomment{FP_token_state_machine}
  \pcomment{F15.midterm2}
  \pcomment{author: Zoran Dzunic, edited ARM 10/15/15}
\end{pcomments}

\pkeywords{
  state_machine
  invariant
  preserved_invariant
  induction
  congruence
}

%%%%%%%%%%%%%%%%%%%%%%%%%%%%%%%%%%%%%%%%%%%%%%%%%%%%%%%%%%%%%%%%%%%%%
% Problem starts here
%%%%%%%%%%%%%%%%%%%%%%%%%%%%%%%%%%%%%%%%%%%%%%%%%%%%%%%%%%%%%%%%%%%%%

\begin{problem}
\emph{Token replacing-1-2} is a single player game using a set of
tokens, each colored black or white.  Except for color, the tokens are
indistinguishable.  %The game starts with one black token.  In each
move, a player can replace one black token with two white tokens, or
replace one white token with two black tokens.

We can model this game as a state machine whose states are pairs
$(n_b,n_w)$ where $n_b \geq 0$ equals the number of black tokens, and
$n_w \geq 0$ equals the number of white tokens.

\bparts
  
\ppart List the numbers of the following predicates that are preserved
invariants:

\begin{center}
\exambox{4in}{0.5in}{0in}
\end{center}
\begin{align}
n_b + n_w  & \not\equiv 2 \pmod{3}\\
n_w - n_b  & \equiv 2 \pmod{3}\\
n_b - n_w  & \equiv 2 \pmod{3}\\
n_b+n_w & > 5 \\
n_b+n_w & < 5
\end{align}

\begin{solution}
Predicates 2, 3, and 4 are preserved invariants.

Detailed explanation:
\begin{enumerate}
\item \false.  True for $(1, 0)$, but not for the state $(0, 2)$
  that it transitions to.

\item \true.
Let $(n_b, n_w)$ be an arbitrary state for which $n_w - n_b \equiv 2 \pmod{3}$.
The only two possible transitions are into state $(n'_b, n'_w) = (n_b - 1, n_w + 2)$
or state $(n'_b, n'_w) = (n_b + 2, n_w - 1)$.  Note that
$n'_w - n'_b = n_w - n_b \pm 3 \equiv n_w - n_b \pmod{3} \equiv 2 \pmod{3}$,
and thus the property is preserved.

\item \true.
Same reasoning as for predicate~(2).

\item \true.  Note that $n_b + n_w$ increases by 1 at each move.
  Thus, if $n_b + n_w > 5$, the same must hold for any state it can
  move to.

\item \false.  Predicate is true for state $(2, 2)$, but not for
  state $(1, 4)$ that it can move to.
\end{enumerate}
\end{solution}

\medskip Now assume the game starts with a single black token, that
is, the start state is $(1,0)$.

\ppart  List the numbers of the predicates above are true for
all reachable states:

\begin{center}
\exambox{4in}{0.5in}{0in}
\end{center}

\begin{solution}
Predicate 2 is true for all reachable states.

Detailed explanation:
\begin{enumerate}

\item \false.
Not true for state $(0, 2)$ that is reachable:
$(1, 0) \rightarrow (0, 2)$.

\item \true.
True in the start state and is also a preserved invariant,
and is therefore true for all reachable states by the invariant principle.

\item \false.
Not true in the start state.

\item \false.
Not true in the start state.

\item \false.  Note that $n_b + n_w$ increases by 1 at each move,
so it is always false after the fourth move of any kind.
\end{enumerate}

\end{solution}

\ppart
Define the predicate $T(n_b,n_w)$ by the rule:
\[
T(n_b,n_w) \eqdef\  [n_w - n_b  \equiv 2 \pmod{3}].
\]

We will now prove the following:
\begin{claim*}
If $T(n_b, n_w)$, then state $(n_b, n_w)$ is reachable.
\end{claim*}

Note that this claim is different from the claim that $T$ is a
preserved invariant.

The proof of the Claim will be by induction in $n$ using induction
hypothesis $P(n) \eqdef$
\[
 \forall (n_b, n_w).\,
 [(n_b + n_w = n) \QAND  T(n_b,n_w)]
     \QIMPLIES (n_b,n_w) \text{ is reachable}.
\]

The base cases will be when $n \leq 2$.  

\begin{itemize}

\item Assuming that the base cases have been verified, complete the
  \textbf{Inductive Step}.

\examspace[3.5in]

\begin{solution}
\begin{proof}
Assume that the induction hypothesis holds for some $n \ge 2$.
Suppose $n_b + n_w = n+1$ and $T(n_b,n_w)$ holds.  We want to show
that $(n_b, n_w)$ is reachable.

Since $n+1 \geq 3$, either $n_b \geq 2$ or $n_w \geq 2$.

In the case that $n_b \geq 2$, we have $n_b-2 \geq 0$, so
$(n_b-2,n_w+1)$ is a state. Also, $T(n_b-2,n_w+1)$ holds because
\[
(n_b-2)-(n_w+1) = n_b-n_w -3 \equiv n_b-n_w \equiv 2 \pmod{3}.
\]
Since
\[
(n_b-2)+(n_w+1) = n_b+n_w - 1 = n,
\]
we conclude by induction hypothesis $P(n)$ that $(n_b-2,n_w+1)$ is
reachable.  But $(n_b-2,n_w+1)$ transitions in one step to
$(n_b,n_w)$, which proves that $(n_b,n_w)$ is reachable.

The same argument applies in the case that $n_w \geq 2$.

We conclude that in any case $(n_b,n_w)$ is reachable, which completes
the induction step.
\end{proof}

To understand the difference between this problem and proving that $T$
is a preserved invariant, it may be helpful to think about the problem
of making different postage amounts, for example using 6 and 15 cent
stamps\inbook{ (Problem~\bref{FP_6_and_15_cent_stamps_by WOP})}.  An
easy invariant proof confirms the fact that every makeable postage
amount is divisible by three.  This preserved invariant for postage
corresponds to invariance of $T$ for reachable token machine states.

A very different fact is that \emph{every} (large enough) amount of
postage that is divisible by three is actually makeable from 6 and 15
cent stamps.  Postage \emph{makeability} for amounts divisible by
three corresponds to \emph{reachability} of token machine states
satisfying $T$.
\end{solution}

\item Now verify the \textbf{Base Cases}: $P(n)$ for $n \leq 2$.

\examspace[3.5in]

\begin{solution}
There are only six states with $n \leq 2$:
\[
(0,0), (1,0), (0,1), (1,1), (0,2), (2,0).
\]
Of these, only $T(1,0)$ and $T(0,2)$ hold. $(1,0)$ is reachable
since it is the start state, and $(0,2)$ is reachable in one step from
the start state.  So all the states with $n \leq 2$ and satisfying
property $T$ are reachable.
\end{solution}

\end{itemize}

\eparts

\end{problem}

%%%%%%%%%%%%%%%%%%%%%%%%%%%%%%%%%%%%%%%%%%%%%%%%%%%%%%%%%%%%%%%%%%%%%
% Problem ends here
%%%%%%%%%%%%%%%%%%%%%%%%%%%%%%%%%%%%%%%%%%%%%%%%%%%%%%%%%%%%%%%%%%%%%

\endinput
