\documentclass[problem]{mcs}

\begin{pcomments}
  \pcomment{PS_stable_matching_unlucky}
  \pcomment{subsumes FP_stable_matching_unlucky}
  \pcomment{complete, elegant rewrite by ARM 12/6/15}
  \pcomment{from: F08.ps4; F03.ps4; F00.ps5, F09, S10}
\end{pcomments}

\pkeywords{
  stable_matching
  state_machines
  termination
  partial_correctness
  invariant
}

\providecommand{\boys}{\text{Boys}}
\providecommand{\girls}{\text{Girls}}

%%%%%%%%%%%%%%%%%%%%%%%%%%%%%%%%%%%%%%%%%%%%%%%%%%%%%%%%%%%%%%%%%%%%%
% Problem starts here
%%%%%%%%%%%%%%%%%%%%%%%%%%%%%%%%%%%%%%%%%%%%%%%%%%%%%%%%%%%%%%%%%%%%%


\begin{problem}
  In a stable matching between an equal number of boys and girls
  produced by the Mating Ritual, call a person \term*{lucky} if they
  are matched up with someone in the top half of their preference
  list.  Prove that there must be at least one lucky person.

\hint The average number of times a boy gets rejected by girls.

\begin{solution}
Let $a$ be the average number of times a boy gets rejected by girls.
It's not possible for all the boys to be rejected an above average
number of times, so some boy gets rejected $\leq a$ times.

Since the total of number rejections is the same for the boys and the
girls, $a$ is also the average number of times a girl rejects boys.
It's not possible for all the girls to reject boys a below average
number of times, so some girl rejects $\geq a$ boys.

Let $n$ be the number of boys.  If $a \leq n/2$, then some boy gets
rejected at most $\floor{n/2}$ times, and therefore he is lucky.
Likewise, if $a \geq n/2$, then some girl rejects at least
$\ceil{n/2}$ boys, and therefore she is lucky.
\end{solution}

\end{problem}


%%%%%%%%%%%%%%%%%%%%%%%%%%%%%%%%%%%%%%%%%%%%%%%%%%%%%%%%%%%%%%%%%%%%%
% Problem ends here
%%%%%%%%%%%%%%%%%%%%%%%%%%%%%%%%%%%%%%%%%%%%%%%%%%%%%%%%%%%%%%%%%%%%%

\endinput

\iffalse
To prove this, define the following derived variables for the Mating Ritual:
\begin{description}
\item $q(B) = j$, where j is the rank of the girl that boy $B$ is courting. That is to say, boy $B$ is always courting the $j$th girl on his list.

\item $r(G)$ is the number of boys that girl $G$ has rejected.
\end{description}

\begin{problemparts}

\ppart\label{Ssame}
Let
\begin{equation}\label{Seqsumb}
S \eqdef \sum_{B \in \boys} q(B) - \sum_{G \in \girls} r(G).
\end{equation}
Show that $S$ remains the same from one day to the next in the Mating
Ritual.

\begin{solution}
  Suppose the Mating Ritual is under way.  For each rejection of a
  boy $B$ by a girl $G$ the values of both $r(G)$ and $q(B)$
  increase by 1, so their difference remains unchanged.  Hence the
  right-hand side of~\eqref{Seqsumb} will be the same tomorrow.
\end{solution}

\ppart Prove the Theorem above.                        %~\ref{luckyperson}
(You may assume for simplicity that $n$ is even.)

\hint A girl is sure to be lucky if she has rejected half the boys.

\begin{solution}
Note that on the first morning, the $n$ boys are courting their first
choice girl and the girls haven't rejected anyone yet.  So $S=n$.
By part~\eqref{Ssame}, $S$ remains equal to $n$ every day.

Now suppose to the contrary that no person is lucky.  Since boy $B$ is, by
definition, lucky iff $q(B) \leq \ceil{n/2}$, we have
\[
q(B) \geq \ceil{\frac{n}{2}} + 1
\]
for all boys $B$.

Now the rank of a girl's suitor will be higher than the ranks of any she
has rejected, so she will certainly be lucky if she has rejected at least
$\ceil{n/2}$ suitors, so
\[
r(G)  \leq \ceil{\frac{n}{2}} - 1,
\]
for all girls $G$.

It follows that $2n \leq \sum_{B \in \boys} q(B) - \sum_{G \in \girls}
r(G) = S$, contradicting the fact that $S=n$.
\end{solution}

\end{problemparts}
\fi
