%PS_stable_matching_unlucky.tex

\documentclass[problem]{mcs}

\begin{pcomments}
  \pcomment{from: F03.ps4; F00.ps5; revised by ARM 10/08/09}
\end{pcomments}

\pkeywords{
  stable_matching
  state_machines
  termination
  partial_correctness
  invariant
}

%%%%%%%%%%%%%%%%%%%%%%%%%%%%%%%%%%%%%%%%%%%%%%%%%%%%%%%%%%%%%%%%%%%%%
% Problem starts here
%%%%%%%%%%%%%%%%%%%%%%%%%%%%%%%%%%%%%%%%%%%%%%%%%%%%%%%%%%%%%%%%%%%%%

\begin{problem}
  In a stable matching between $n$ boys and girls produced by the Mating
  Ritual, call a person \term*{lucky} if they are matched up with one of
  their $\floor{n/2}$ top choices.  We will prove:
\begin{theorem}\label{luckyperson}
  There must be at least one lucky person.
\end{theorem}

To prove this, define the following derived variables for the Mating Ritual:
\begin{description}
\item $B_i = j$, when the $i$th boy is courting the $j$th girl
on his list.
\item $G_i$ is the number of boys that the $i$th girl has rejected.
\end{description}

\begin{problemparts}

\ppart\label{Ssame}
Let
\[
S \eqdef \sum_{i=1} ^n B_i - \sum_{i=1} ^n G_i.
\]
Show that $S$ remains the same from one day to the next in the Mating
Ritual.

\begin{solution}
  Suppose the Mating Ritual is under way.  If the $j$th girl has $m>1$ suitors
  then she will reject $m-1$ of them, so $G_j$ will increase by $m-1$.  But
  $B_j$ increases by 1 for each rejected suitor, $b_j$, so
  the $G_j - \sum_{b_k \in \text{suitors}(G_j)}B_k$ will be the same
  tomorrow.  Hence $S$ will also be the same tomorrow.
\end{solution}

\ppart Prove Theorem~\ref{luckyperson}.

\hint A girl is sure to be lucky if she has rejected at least half of the
boys.

\begin{solution}
Note that on the first morning, the $n$ boys are courting their first
choice girl and the girls haven't rejected anyone yet.  So $S=n$.
By part~\eqref{Ssame}, $S$ remains equal to $n$ every day.

Now suppose to the contrary that no person is lucky.  Since the $j$th boy
is, by definition, lucky iff $B_j \leq \floor{n/2}$, we have
\[
B_i \geq \floor{\frac{n}{2}} + 1
\]
for $1 \leq i \leq n$.

Now the rank of a girl's suitor will be higher than the ranks of any
she has rejected, so she will certainly be lucky if she has rejected
$\floor{n/2}$ suitors, so
\[
G_i \leq \floor{\frac{n}{2}} - 1,
\]
for $1 \leq i \leq n$.

It follows that $2n \leq \sum_{i=1} ^n B_i - \sum_{i=1} ^n G_i = S$, contradicting
the fact that $S=n$.

\end{solution}

\end{problemparts}

\end{problem}


%%%%%%%%%%%%%%%%%%%%%%%%%%%%%%%%%%%%%%%%%%%%%%%%%%%%%%%%%%%%%%%%%%%%%
% Problem ends here
%%%%%%%%%%%%%%%%%%%%%%%%%%%%%%%%%%%%%%%%%%%%%%%%%%%%%%%%%%%%%%%%%%%%%


\endinput
