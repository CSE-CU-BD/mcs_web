%PS_stable_matching_unlucky.tex

\documentclass[problem]{mcs}

\begin{pcomments}
  \pcomment{from: F03.ps4; F00.ps5}
\end{pcomments}

\pkeywords{
  stable_matching
  state_machines
  termination
  partial_correctness
  invariant
}

%%%%%%%%%%%%%%%%%%%%%%%%%%%%%%%%%%%%%%%%%%%%%%%%%%%%%%%%%%%%%%%%%%%%%
% Problem starts here
%%%%%%%%%%%%%%%%%%%%%%%%%%%%%%%%%%%%%%%%%%%%%%%%%%%%%%%%%%%%%%%%%%%%%

\begin{problem}
  Consider an instance of the matching problem with $n$ boys and $n$
  girls.  Call a person \iffalse \term\fi \emph{unlucky} if they are
  matched up with one of their $\floor{n/2}$ last choices.  In this
  problem, we will prove the following:

\begin{theorem*}
The matching algorithm from class never produces a
matching in which every person is unlucky.
\end{theorem*}

Fix an execution of the matching algorithm.  Define the variables 
$B_i, G_i, i \in \set{ 1, 2, \ldots, n}$ as follows:
\begin{description}
\item $B_i = j$ if the $i$th boy is currently courting the $j$th girl
on his list
\item $G_i$ is the number of boys that the $i$th girl has rejected.
\end{description}

\begin{problemparts}

\ppart
Show that
\[
\sum_{i=1} ^n B_i - \sum_{i=1} ^n G_i
\]
is preserved at each step of the matching algorithm.

\begin{solution}
Let $S = \sum_{i=1} ^n B_i - \sum_{i=1} ^n G_i$.  Suppose that at step
$t$, boy $b_i$ proposes to girl $g_j$.  There are three possible outcomes:

If $g_j$ has not had any previous proposals, she must accept.  In this
case, none of the $B_i$s and none of the $G_j$s change, so $S$ is preserved.

If $g_j$ rejects, then $G_j$ increases by 1, $B_i$ increases by 1, and
the other $B$s and $G$s remain the same.  $S$ is preserved again.

If $g_j$ accepts $b_i$ because she likes him better than her current
mate $b_{i'}$, then $G_j$ increases by 1, $B_{i'}$ increases by 1, and
the other $B$s and $G$s remain the same.  $S$ is preserved in this case
as well.
\end{solution}

\ppart
Formulate a preserved invariant that you can use to prove the theorem, and 
verify that your invariant is preserved.

\begin{solution}

Let $P : S = n$. From part (a) it follows that
all transitions preserve the truth value of $P$.  Therefore $P$ is an
invariant.

\end{solution}

\ppart Use your preserved invariant to prove the theorem.

\begin{solution}
We prove the statement by contradiction.  Consider a match M in which
every person is unlucky.  Then:

\begin{description}
\item For every $i \in \{ 1, \ldots, n \}$, 
$B_i \geq \ceil{\frac{n}{2}} + 1$, and
\item For every $i \in \{ 1, \ldots, n \}$,
$G_i \leq \floor{\frac{n}{2}} - 1$.
\end{description}

It follows that $\sum_{i=1} ^n B_i - \sum_{i=1} ^n G_i \geq 2n$.  This
violates the invariant $P$.

\end{solution}

\end{problemparts}

\end{problem}

\endinput
