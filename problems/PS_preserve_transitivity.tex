\documentclass[problem]{mcs}

\begin{pcomments}
  \pcomment{PS_preserve_transitivity}
  \pcomment{from: F09.ps3 revised by ARM 9/26/09 from symmetry problem S02.ps3}
\end{pcomments}

\pkeywords{
  relations
  relational_properties
  composition
  transitive
}

%%%%%%%%%%%%%%%%%%%%%%%%%%%%%%%%%%%%%%%%%%%%%%%%%%%%%%%%%%%%%%%%%%%%%
% Problem starts here
%%%%%%%%%%%%%%%%%%%%%%%%%%%%%%%%%%%%%%%%%%%%%%%%%%%%%%%%%%%%%%%%%%%%%

\begin{problem}
Let $R$ and $S$ be transitive binary relations on the same set, $A$.
Which of the following new relations must also be transitive?  For
each part, justify your answer with a brief argument if the new
relation is transitive and a counterexample if it is not.

\bparts
\ppart $R^{-1}$

\begin{solution}
  $R^{-1}$: \textbf{Yes.}  Because $a \mrel{R} b \mrel{R} c$ iff $c
  \mrel{\inv{R}} b \mrel{\inv{R}} a$.
\end{solution}

\ppart $R \intersect S$

\begin{solution}
  \textbf{Yes.} $a \mrel{R\intersect S} b \mrel{R \intersect S} c$ iff [$a
  \mrel{R} b \mrel{R} c$ and $a \mrel{S} b \mrel{S} c]$ implies [$a
  \mrel{R} c$ and $a \mrel{S} c]$ iff $a \mrel{R\intersect S} b$.
\end{solution}


\ppart $R \composition R$

\begin{solution}
\textbf{Yes.}  $a R^2 c$ means $a \mrel{R} b \mrel{R} c$ for some $b$, and
since $R$ is transitive, $a \mrel{R} c$.  But then $a \mrel{R^2} b
\mrel{R^2} c$ implies $a \mrel{R} b \mrel{R} c$, which implies $a \mrel{R^2} c$.
\end{solution}

\ppart $R \composition S$

\begin{solution}
  \textbf{No.}  Suppose $A = \set{1,2,3,4,5}$, $\graph{R} =
  \set{\diredge{1}{2}, \diredge{3}{4}}$ , and $\graph{S} =
  \set{\diredge{2}{3}, \diredge{4}{5}}$.  Then $\graph{S \compose R}
  = \set{\diredge{1}{3}, \diredge{3}{5}}$.  Now $R$ and $S$ are
  vacuously transitive, but $S \compose R$ is missing the
  $\diredge{1}{5}$ arrow, and so is not transitive.
\end{solution}

\eparts

\end{problem}

%%%%%%%%%%%%%%%%%%%%%%%%%%%%%%%%%%%%%%%%%%%%%%%%%%%%%%%%%%%%%%%%%%%%%
% Problem ends here
%%%%%%%%%%%%%%%%%%%%%%%%%%%%%%%%%%%%%%%%%%%%%%%%%%%%%%%%%%%%%%%%%%%%%

\endinput
