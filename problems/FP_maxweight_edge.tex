\documentclass[problem]{mcs}

\begin{pcomments}
  \pcomment{FP_maxweight_edge}
  \pcomment{by ARM 4/24/16}
\end{pcomments}

\pkeywords{
  tree
  spanning_tree
  cycle
  MST
  weight
  maximum
}

%%%%%%%%%%%%%%%%%%%%%%%%%%%%%%%%%%%%%%%%%%%%%%%%%%%%%%%%%%%%%%%%%%%%%
% Problem starts here
%%%%%%%%%%%%%%%%%%%%%%%%%%%%%%%%%%%%%%%%%%%%%%%%%%%%%%%%%%%%%%%%%%%%%

\begin{problem}
Let $G$ be a connected simple graph, $T$ be a spanning tree of $G$,
and $e$ be an edge of $G$.  \inhandout{(Note that part~\eqref{enGeT}
  is not needed to complete parts~\eqref{eiscyG}
  and~\eqref{GwtminT}.)}

\bparts

\ppart\label{enGeT} Prove that if $e$ is \emph{not} on a cycle in $G$,
then $e$ is an edge of $T$.

\examspace[3.0in]

\begin{solution}
If $e$ is not on a cycle in $G$, then $e$ is a cut edge by
Lemma~\bref{lem:cutiffcycle}.  Therefore, $G-e$ is not connected,
which implies that $T-e$ is not connected.  Since $T$ is connected,
$e$ must be an edge of $T$.
\end{solution}

\ppart\label{eiscyG} Prove that if $e$ \emph{is} on a cycle in $G$,
and $e$ is in $T$, then there is an edge $f \neq e$ such that $T - e +
f$ is also a spanning tree.

\examspace[3.0in]

\begin{solution}
Since $T$ is a tree, $T-e$ has two components.  Since $e$ is on a
cycle in $G$, deleting $e$ from the cycle leaves a path in $G$ between
the endpoints of $e$.  This path connects the two components of $T-e$
and therefore must contain an edge, $f$, other than $e$ that connects
the two components.  Then $T-e+f$ is a also a spanning tree because it
is connected and has the same number of edges as $T$.
\end{solution}

\ppart\label{GwtminT} Suppose $G$ is edge-weighted, the weight of $e$
is larger than the weights of all the other edges, $e$ is on a cycle
in $G$, and $e$ is an edge of $T$.  Conclude that $T$ is \emph{not} a
minimum weight spanning tree of $G$.

\begin{solution}
By part~\eqref{eiscyG}, $T-e+f$ is also a spanning tree and its weight
is less than $T$'s since the weight of $f$ is less than the weight of
$e$.

Altogether, we have now shown that $e$ is a member of $T$ iff $e$ is a
cut edge.
\end{solution}

\eparts
\end{problem}

%%%%%%%%%%%%%%%%%%%%%%%%%%%%%%%%%%%%%%%%%%%%%%%%%%%%%%%%%%%%%%%%%%%%%
% Problem ends here
%%%%%%%%%%%%%%%%%%%%%%%%%%%%%%%%%%%%%%%%%%%%%%%%%%%%%%%%%%%%%%%%%%%%%

\endinput
