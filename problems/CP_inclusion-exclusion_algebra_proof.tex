\documentclass[problem]{mcs}

\begin{pcomments}
  \pcomment{CP_inclusion-exclusion_algebra_proof}
  \pcomment{from: ARM new 11/15/09 based on S09 slides}
\end{pcomments}

\pkeywords{
  inclusion-exclusion
  product
  sum
  subset
  algebra
}

%%%%%%%%%%%%%%%%%%%%%%%%%%%%%%%%%%%%%%%%%%%%%%%%%%%%%%%%%%%%%%%%%%%%%
% Problem starts here
%%%%%%%%%%%%%%%%%%%%%%%%%%%%%%%%%%%%%%%%%%%%%%%%%%%%%%%%%%%%%%%%%%%%%

\begin{problem}
Let's develop a proof of the Inclusion-Exclusion formula using high school
algebra.

\bparts

\ppart Most high school students will get freaked by the following formula,
even though they actually know the rule it expresses.  How would you
explain it to them?

\begin{equation}\label{1-xprod}
\prod_{i=1}^n \paren{1-x_i} = \sum_{I \subseteq \set{1,\dots,n}} (-1)^{\card{I}}\prod_{j \in I}x_j.
\end{equation}
\hint Show them an example.

\begin{solution}
Let's do an example.  To ``multiply out''
\begin{equation}\label{x3prod}
(1-x_1)(1-x_2)(1-x_3),
\end{equation}
you would form \emph{\idx{monomial}} products by selecting some of the
$(-x_i)$'s to multiply together.  For example, selecting $(-x_i)$'s with
\begin{itemize}
\item $i \in \set{1,3}$ leads to the monomial $(-x_1)(-x_3) = (-1)^2x_1x_3
  = x_1x_3$,
%\item $i \in \set{2}$ leads to the monomial $x_2$,
\item $i \in \set{1,2,3}$ leads to the monomial $(-x_1)(-x_2)(-x_3)= (-1)^3x_1x_2x_3 = -x_1x_2x_3$, and
\item $i \in \emptyset$ leads (by convention) to the monomial $1$.
\end{itemize}
Then you sum up the monomials from \emph{all possible} selections to get
\[
(1-x_1)(1-x_2)(1-x_3) = 1 - x_1 - x_2 - x_3 + x_1x_2 + x_1x_3 + x_2x_3 - x_1x_2x_3.
\]

Now we can decipher~\eqref{1-xprod} as saying to do the same thing for
the product of $n$ different $(1-x_i)$'s:  for any selection of $(-x_i)$'s
with $i$ in some subset, $I \subseteq \set{1,\dots,n}$, multiply the
$(-x_i)$'s to get the monomial
\[
\prod_{i \in I} (-x_i) = \prod_{i \in I} (-1)^{\card{I}} x_i,
\]
and sum up all such monomials obtained by every possible selection, $I$,
to get the right hand side of equation~\eqref{1-xprod}.
\end{solution}

\eparts

For any set, $S$, let $M_S$ be the \emph{membership} function of $S$:
\[
M_S(x) = \begin{cases}
         1 & \text{if $x \in S$},\\
         0 & \text{if $x \notin S$}.
         \end{cases}
\]

Let $S_1,\dots,S_n$ be a sequence of finite sets, and abbreviate $M_{S_i}$
as $M_i$.  Let the \idx{domain of discourse}, $D$, be the union of the
$S_i$'s.  That is, we let
\[
D \eqdef \lgunion_{i=1}^n S_i,
\]
and take complements with respect to $D$, that is,
\[
\bar{T} \eqdef D - T,
\]
for $T \subseteq D$.

\bparts

\ppart\label{Mprops}
Verify that for $T \subseteq D$ and $I \subseteq \set{1,\dots n}$,
\begin{align}
M_{\bar{T}}  & = 1- M_T, \label{Tbar1-T}\\
M_{\paren{\lgintersect_{i \in I} S_i}} & = \prod_{i \in I}M_{S_i}, \label{intersectSs}\\
M_{\paren{\lgunion_{i \in I} S_i}} & = 1- \prod_{i\in I}(1-M_i). \label{1-Mprod}
\end{align}
(Note that~\eqref{intersectSs} holds when $I$ is empty because, by
convention, an empty product equals 1, and an empty intersection equals
the domain of discourse, $D$.)

\begin{solution}
To prove~\eqref{Tbar1-T}, we have for all $u \in D$,
\begin{align*}
M_{\bar{T}}(u) = 1 & \qiff u \in \bar{T} \qiff M_T(u) = 0 \qiff 1-M_T(u) = 1,\\
M_{\bar{T}}(u) = 0 & \qiff u \notin \bar{T} \qiff u \in T \qiff M_T(u) = 1 \qiff 1-M_T(u) = 0,\\
\end{align*}
so $M_{\bar{T}}(u) = 1-M_T(u)$.

Similarly, to prove~\eqref{intersectSs},
\[
M_{\paren{\lgintersect_{i \in I} S_i}}(u) = 1
   \qiff u \in \lgintersect_{i \in I} S_i
   \qiff \lgand_{i \in I} u \in S_i
   \qiff \lgand_{i \in I} [M_i(u) = 1] \qiff \paren{\prod_{i \in I} M_i(u)} = 1.
\]

Finally,~\eqref{1-Mprod} follows from~\eqref{Tbar1-T}
and~\eqref{intersectSs} by DeMorgan's Law.
\end{solution}

\ppart Use~\eqref{1-xprod} and~\eqref{1-Mprod} to prove
\begin{equation}\label{MUsum}
M_D = \sum_{\emptyset \neq I \subseteq \set{1,\dots,n}}
             (-1)^{\card{I}+1} \prod_{j \in I}M_j.
\end{equation}
\begin{solution}

\begin{align*}
M_D & = M_{\paren{\lgunion_{i=1}^n S_i}}\\
    & = 1 - \prod_{i=1}^n (1-M_i)
         &\text{by~\eqref{1-Mprod}}\\
    & = 1 - \sum_{I \subseteq \set{1,\dots,n}} (-1)^{\card{I}}\prod_{j \in I}M_j
        & \text{by~\eqref{1-xprod}}\\
    & = 1 - \paren{1 + \sum_{\emptyset \neq I \subseteq \set{1,\dots,n}}
             (-1)^{\card{I}} \prod_{j \in I}M_j}
         & \text{($\prod_{j \in \emptyset} M_j \eqdef 1$)}\\
    & = \sum_{\emptyset \neq I \subseteq \set{1,\dots,n}}
             (-1)^{\card{I}+1} \prod_{j \in I}M_j.
\end{align*}

\end{solution}

\ppart Prove that
\begin{equation}\label{}\label{cardTsum}
\card{T}  = \sum_{u \in D} M_T(u).\\
\end{equation}

\begin{solution}
\[
\sum_{u \in D} M_T(u) = \sum_{u \in T} M_T(u) + \sum_{u \in \bar{T}} M_T(u) 
= \paren{\sum_{u \in T} 1} +  \paren{\sum_{u \in \bar{T}} 0}
= \card{T}+0 = \card{T},
\]
\end{solution}

\ppart Now use
\iffalse~\eqref{intersectSs},~\eqref{MUsum} and~\eqref{cardTsum} \fi
the previous parts to prove
\begin{equation}\label{incexc-subsets}
\card{D}  = \sum_{\emptyset \neq I \subseteq \set{1,\dots,n}}
                  (-1)^{\card{I}+1} \card{\lgintersect_{i \in I} S_i}
\end{equation}

\begin{solution}
Summing both sides of~\eqref{MUsum} over $u \in D$, we have
\begin{align*}
\card{D} & = \sum_{u \in D} M_D(u)
              & \text{(by~\eqref{cardTsum})}\\
         &  \sum_{u \in D} \paren{\sum_{\emptyset \neq I \subseteq \set{1,\dots,n}}
             (-1)^{\card{I}+1} \prod_{j \in I}M_j(u)}
              & \text{(by~\eqref{MUsum})}\\
         & = \sum_{u \in D}
                 \paren{\sum_{\emptyset \neq I \subseteq \set{1,\dots,n}}
                       (-1)^{\card{I}+1} M_{\lgintersect_{i \in I} S_i}(u)}
             & \text{(by~\eqref{intersectSs})}\\
         & = \sum_{\emptyset \neq I \subseteq \set{1,\dots,n}}
                  (-1)^{\card{I}+1} \paren{\sum_{u \in D}
                             M_{\lgintersect_{i \in I} S_i}(u)}
              & \text{(reversing the order of sums)}\\
         & = \sum_{\emptyset \neq I \subseteq \set{1,\dots,n}}
                  (-1)^{\card{I}+1} \card{\lgintersect_{i \in I} S_i}
              & \text{(by~\eqref{cardTsum})}.
\end{align*}
\end{solution}

\ppart Finally, explain why~\eqref{incexc-subsets} immediately implies the
usual form of the Inclusion-Exclusion Principle:

\begin{equation}\label{incexc-n}
  \card{D} = \sum_{i=1}^n (-1)^{i+1}
             \sum_{\substack{I \subseteq \set{1,\dots,n}\\ \card{I}=i}}
                 \card{\lgintersect_{j \in I} S_j}.
\end{equation}

\begin{solution}
  We obtain~\eqref{incexc-n} from~\eqref{incexc-subsets} by breaking up
  the sum over nonempty subsets, $I \subseteq \set{1,\dots,n}$, into
  separate sums over all the subsets of size $i$, for $1 \leq i \leq n$.
\end{solution}

\eparts
\end{problem}


%%%%%%%%%%%%%%%%%%%%%%%%%%%%%%%%%%%%%%%%%%%%%%%%%%%%%%%%%%%%%%%%%%%%%
% Problem ends here
%%%%%%%%%%%%%%%%%%%%%%%%%%%%%%%%%%%%%%%%%%%%%%%%%%%%%%%%%%%%%%%%%%%%%

\endinput
