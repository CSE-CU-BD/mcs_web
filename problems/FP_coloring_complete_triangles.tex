\documentclass[problem]{mcs}

\begin{pcomments}
  \pcomment{FP_coloring_complete_triangles}
  \pcomment{variation of FP_coloring_triangles from F03 final}
  \pcomment{from S06.final}
  \pcomment{minor edit ARM 5/20/12}
  \pcomment{soln to last part need rewrite ala Drew Minnear}
\end{pcomments}

\pkeywords{
 coloring
 probability
 expectation
 variance
 graphs
 expectation
 independence
 Chebyshev
 law_of_large_numbers
 asymptotic_notation
 big_oh 
}

%%%%%%%%%%%%%%%%%%%%%%%%%%%%%%%%%%%%%%%%%%%%%%%%%%%%%%%%%%%%%%%%%%%%%
% Problem starts here
%%%%%%%%%%%%%%%%%%%%%%%%%%%%%%%%%%%%%%%%%%%%%%%%%%%%%%%%%%%%%%%%%%%%%

\begin{problem}
Let $K_n$ be the complete graph with $n$ vertices.  Each of the edges of
the graph will be randomly assigned one of the colors red, green, or blue.
The assignments of colors to edges are mutually independent, and the
probabilty of an edge being assigned red is $r$, blue is $b$, and green is
$g$ (so $r+b+g=1$).

A set of three vertices in the graph is called a \emph{triangle}.  A
triangle is \emph{monochromatic} if the three edges connecting the vertices
are all the same color.

\bparts 

\ppart Let $m$ be the probability that any given triangle, $T$, is
monochromatic.  Write a simple formula for $m$ in terms of $r,b,$ and $g$.

\begin{center}
\exambox{2.0in}{0.5in}{0in}
\end{center}

\examspace[0.5in]

\begin{solution}
$m = r^3+b^3 +g^3$
\end{solution}

\ppart Let $I_T$ be the indicator variable for whether $T$ is
monochromatic.  Write simple formulas in terms of $m,r,b,$ and $g$ for
$\expect{I_T}$ and $\variance{I_T}$.

\begin{align*}
\expect{I_T}   & = \text{\exambox{1.5in}{0.4in}{-0.2in}}\\
\variance{I_T} & = \text{\exambox{1.5in}{0.4in}{-0.2in}}
\end{align*}

\examspace[1.0in]

\begin{solution}
\begin{align*}
\expect{I_T} & = m,\\
\variance{I_T} & = m(1-m).
\end{align*}
\end{solution}

\eparts

\begin{center}
{\large \textbf{Now assume $r=b=g = \dfrac{1}{3}$.}}
\end{center}
Let $T$ and $U$ be distinct triangles.

\bparts
\ppart What is the probability that $T$ and $U$ are both monochromatic?

\begin{center}
\exambox{0.5in}{0.4in}{0in}
\end{center}

\begin{solution}
\[
\frac{1}{3^4}.
\]

If $T$ and $U$ do not share an edge, then the three edges of $T$
match, and independently, the three edges of $U$ must match, so both
match with probability $(1/3)^2~\cdot~(1/3)^2$.  If they do share an
edge, the the five edges among them must all match, which happens
with probability $(1/3)^4$ as well.
\end{solution}

\ppart Show that $I_T$ and $I_U$ are independent random variables.

\examspace[2.0in]
\begin{solution}
Since $I_T$ and $I_U$ are indicators for events, it suffices to verify that
\[
\prob{I_T = 1}\cdot \prob{I_U = 1} = \prob{I_T\cdot I_U =1}.
\]
There are two cases depending on whether $T$ and $U$ share an edge.  In each
case,
\[
\prob{I_T = 1}\cdot \prob{I_U = 1} = \paren{\frac{1}{3}}^2 \cdot \paren{\frac{1}{3}}^2 =
\paren{\frac{1}{3}}^4 = \prob{I_T\cdot I_U =1}.
\]
\end{solution}

\ppart\label{part:exMvarM} Let $M$ be the number of monochromatic
triangles.  Write simple formulas in terms of $n$ and $m$ for
$\expect{M}$ and $\variance{M}$.

\begin{align*}
\expect{M}   & = \text{\exambox{1.5in}{0.4in}{-0.2in}}\\
\variance{M} & = \text{\exambox{1.5in}{0.4in}{-0.2in}}
\end{align*}

\examspace[1.0in]

\begin{solution}
\begin{align}
\expect{M}   & = m \cdot \text{(\# triangles)}\notag\\
             & = m\binom{n}{3},\label{expMmsh}\\
\variance{M} & = \variance{I_T} \cdot \text{(\# triangles)}\notag\\
             & = m(1-m)\binom{n}{3} = (1-m)\expect{M}.\label{varMITsh}
\end{align}
\end{solution}

\ppart  Let $\mu \eqdef \expect{M}$.  Prove that
\[
\Prob{\abs{M - \mu} > \sqrt{\mu \log \mu}} = O \paren{\frac{1}{\log n}}
\]

\examspace[2.5in]

\begin{solution}
According to Chebyshev's Bound:
\[%\label{CMmucsig}
\prob{\abs{M - \mu} > c \sigma} \leq \frac{1}{c^2}
\]
So
\begin{align*}
\Prob{\abs{M - \mu} > \sqrt{\mu \log \mu}}
  & = \Prob{\abs{M - \mu} > \sqrt{\log \mu} \sqrt{\mu}}\\
  & \leq \Prob{\abs{M - \mu} > \sqrt{\log \mu} \sqrt{(1-m)\mu}}
       & (0 \leq 1- m \leq 1)\\
  & = \Prob{\abs{M - \mu} > \sqrt{\log \mu}\ \sigma}
       & \text{(by~\eqref{varMITsh})}\\
  & \leq \frac{1}{\log \mu}
       & \text{(by Chebyshev)}\\%~\eqref{CMmucsig}
  & = O \paren{\frac{1}{\log n}}
       & \text{(by~\eqref{expMmsh}).}
\end{align*}
The last step follows because according to~\eqref{expMmsh},
\[
\mu = m\binom{n}{3} = \Theta(n^3),
\]
so
\[
\frac{1}{\log \mu} = \Theta\paren{\frac{1}{\log n^3}} = \Theta\paren{\frac{1}{\log n}}.
\]
\end{solution}

\iffalse
\ppart Conclude that
\[
\lim_{n\to \infty} \prob{\abs{M - \mu} > \sqrt{\mu \log \mu}} = 0
\]

\begin{solution}

Since $1/(\log n) \to 0$ as $n \to \infty$, the $O()$ bound on the
probability goes to 0 which means an upper bound on the limit is 0.  Since
the probability is nonnegative, the limit must be exactly 0.

\end{solution}
\fi

\eparts
\end{problem}
