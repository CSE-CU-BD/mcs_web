\documentclass[problem]{mcs}

\begin{pcomments}
  \pcomment{from: F08 ps9 revised in terms of describing bijections.}
\end{pcomments}

\pkeywords{
  counting 
  bijections
}

%%%%%%%%%%%%%%%%%%%%%%%%%%%%%%%%%%%%%%%%%%%%%%%%%%%%%%%%%%%%%%%%%%%%%
% Problem starts here
%%%%%%%%%%%%%%%%%%%%%%%%%%%%%%%%%%%%%%%%%%%%%%%%%%%%%%%%%%%%%%%%%%%%%

% F08 ps9 revised in terms of describing bijections.

\begin{problem}[4]
In this problem we consider 5-card poker hands drawn from a standard 
52-card deck. For each part describe a bijection or $k$-to-1 
mapping between a set that can easily be counted using the simple 
rules described in the lecture notes and the set of hands matching 
the specification.  \emph{We are not looking for numerical answers.}

For instance, consider the set of 5-card hands containing all 4 suits. 
Each must have 2 cards of one suit.  We can describe a bijection mapping 
each such hand to a sequence $\paren{s, \set{r_1,r_2}, r_3, r_4, r_5}$ 
specifying
\begin{enumerate} 
\item the repeated suit $s$,
\item the ranks $\set{r_1,r_2}$ of the repeated suit, and
\item the ranks $r_3$, $r_4$ and $r_5$ of the non-repeating suits, listed 
in increasing suit order\footnote{Any order will suffice but one of the
more common would be $\clubsuit \prec \diamondsuit \prec \heartsuit \prec \spadesuit$}.
\end{enumerate}
Sequences of this form can be counted using the subset and generalized 
product rule.

\bparts

\ppart A single pair and no 3-of-a-kind or 4-of-a-kind.

\begin{solution}
There is a bijection with sequences of the form 
$\paren{r_{12}, \set{s_1,s_2}, \set{r_3,r_4,r_5}, s_3, s_4, s_5}$ 
specifying
\begin{enumerate} 
\item the rank of the pair $r_{12}$,
\item the suits of the pair $\set{s_1,s_2}$,
\item the non-repeating ranks $\set{r_3, r_4, r_5}$, and 
\item the suits of the non-repeating ranks where $s_3$, $s_4$ and $s_5$ are
the suits of the smallest, middle and largest ranked non-repeating card 
(respecively).
\end{enumerate}

Alternatively, there is also a 3!-to-1 mapping from sequences of the
form $\paren{r_{12}, \set{s_1,s_2}, r_3, s_3, r_4, s_4, r_5, s_5}$
specifying
\begin{enumerate} 
\item the rank of the pair $r_{12}$,
\item the suits of the pair $\set{s_1,s_2}$, and
\item the rank and suit of the 3rd, 4th and 5th card.
\end{enumerate}\end{solution}

\ppart Three or more aces.

\begin{solution}
There is a bijection between the hands with exactly 3 aces and
the set, $A_3$, of sequences of the form
$\paren{\set{s_1,s_2,s_3}, (r_4,s_4), (r_5,s_5)}$ 
specifying
\begin{enumerate} 
\item the suits the aces in the hand $\set{s_1,s_2,s_3}$,
\item the rank/suit $(r_4, s_4)$ of the lesser-valued non-ace, and 
\item the rank/suit $(r_5, s_5)$ of the greater-valued non-ace.
\end{enumerate}

Similarly, there is a bijection between the hands with exactly 4
aces and the set, $A_4$, of sequences of the form $\paren{r,s}$ 
specifying the rank $r$ and the suit $s$ of the non-ace.

$A_3$ and $A_4$ are disjoint. Since the set of hands with exactly 
3 aces and the set of hands with exactly 4 aces are disjoint as well,
we have described a bijection between the hands with three or more 
aces and the set $A_3 \union A_4$.\end{solution}

\eparts
\end{problem}

%%%%%%%%%%%%%%%%%%%%%%%%%%%%%%%%%%%%%%%%%%%%%%%%%%%%%%%%%%%%%%%%%%%%%
% Problem ends here
%%%%%%%%%%%%%%%%%%%%%%%%%%%%%%%%%%%%%%%%%%%%%%%%%%%%%%%%%%%%%%%%%%%%%

\endinput
