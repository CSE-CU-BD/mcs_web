\documentclass[problem]{mcs}

\begin{pcomments}
  \pcomment{MQ_multi_schedule_tasks}
  \pcomment{cut from PS_Brents_theorem and revised by ARM 3/21/10;
    edited 4/2/17}
\end{pcomments}

\pkeywords{
  partial_orders
  schedule
  incomparable
}

%%%%%%%%%%%%%%%%%%%%%%%%%%%%%%%%%%%%%%%%%%%%%%%%%%%%%%%%%%%%%%%%%%%%%
% Problem starts here
%%%%%%%%%%%%%%%%%%%%%%%%%%%%%%%%%%%%%%%%%%%%%%%%%%%%%%%%%%%%%%%%%%%%%

\begin{problem}
Suppose the vertices of a DAG represent tasks taking unit time to
complete, and the edges indicate prerequisites among the tasks.
Assume there is no bound on how many tasks may be performed in
parallel.

What is the smallest number of vertices (tasks) possible in a DAG for
which there is more than one minimum time schedule?  Carefully justify
your answer.

\begin{solution}
Three tasks.

With one task, there is only one possible schedule.  Two tasks with no
edge between them can both be completed in one step, and this is the
unique minimum time schedule.  Two tasks with an edge from one to the
other can only be scheduled in order, and this is the unique minimum
time schedule.

An example with two minimum time schedules is a DAG with three tasks,
where two of the tasks have an edge between them and the third task
has no incident edges.

The tasks connected by the edge have a unique minimum time schedule of
length two.  So any schedule for the three tasks that is also length
two will certainly be minimum time for the three.  But the third task
can be scheduled at the same time as either the first or the second of
the comparable tasks, giving two minimum time schedules for the three
tasks.
\end{solution}
\end{problem}

%%%%%%%%%%%%%%%%%%%%%%%%%%%%%%%%%%%%%%%%%%%%%%%%%%%%%%%%%%%%%%%%%%%%%
% Problem ends here
%%%%%%%%%%%%%%%%%%%%%%%%%%%%%%%%%%%%%%%%%%%%%%%%%%%%%%%%%%%%%%%%%%%%%


\endinput
