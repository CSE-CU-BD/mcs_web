\documentclass[problem]{mcs}

\begin{pcomments}
  \pcomment{MQ_multi_schedule_tasks}
  \pcomment{cut from PS_Brents_theorem and revised by ARM 3/21/10}
\end{pcomments}

\pkeywords{
  partial_orders
  schedule
  incomparable
}

%%%%%%%%%%%%%%%%%%%%%%%%%%%%%%%%%%%%%%%%%%%%%%%%%%%%%%%%%%%%%%%%%%%%%
% Problem starts here
%%%%%%%%%%%%%%%%%%%%%%%%%%%%%%%%%%%%%%%%%%%%%%%%%%%%%%%%%%%%%%%%%%%%%

\begin{problem}
What is the smallest number of partially ordered tasks for which there
can be more than one minimum time schedule with unlimited number of processors?  Explain.

\begin{solution}
Three tasks.

With one task, there is only one possible schedule.  Two tasks that
are incomparable can both be completed in one step, and this is the
unique minimum step schedule.  For two tasks that are comparable,
there is only one possible schedule, which therefore is the unique
minimum time schedule.

For an example with three tasks with two minimum time schedules, let
two of the tasks be comparable and the third task incomparable to the
other two.  The two comparable tasks have a unique minimum time
schedule that takes two steps.  So any schedule for the three tasks
that also takes only two steps will certainly be minimum time for the
three.  But the third task can be scheduled at the same time as either
the first or the second of the comparable tasks, giving two minimum
schedules for the three tasks.

\end{solution}
\end{problem}

%%%%%%%%%%%%%%%%%%%%%%%%%%%%%%%%%%%%%%%%%%%%%%%%%%%%%%%%%%%%%%%%%%%%%
% Problem ends here
%%%%%%%%%%%%%%%%%%%%%%%%%%%%%%%%%%%%%%%%%%%%%%%%%%%%%%%%%%%%%%%%%%%%%


\endinput
