\documentclass[problem]{mcs}

\begin{pcomments}
  \pcomment{MQ_multi_schedule_tasks}
  \pcomment{cut from PS_Brents_theorem and revised by ARM 3/21/10;
    edited 4/2/17}
\end{pcomments}

\pkeywords{
  partial_orders
  schedule
  incomparable
}

%%%%%%%%%%%%%%%%%%%%%%%%%%%%%%%%%%%%%%%%%%%%%%%%%%%%%%%%%%%%%%%%%%%%%
% Problem starts here
%%%%%%%%%%%%%%%%%%%%%%%%%%%%%%%%%%%%%%%%%%%%%%%%%%%%%%%%%%%%%%%%%%%%%

\begin{problem}
Suppose the vertices of a DAG represent tasks taking unit time to
complete, and the edges indicate prerequisites among the tasks.
Assume there is no bound on how many tasks may be performed in
parallel.

What is the smallest DAG for which there can be more than one minimum
time schedule?  Carefully justify your answer.

\begin{solution}
Three tasks.

With one task, there is only one possible schedule.  Two tasks with no
edge between them can both be completed in one step, and this is the
unique minimum step schedule.  Two tasks with an edge from one to the
other can only be scheduled in order, and this is the unique minimum
time schedule.

An example with two minimum time schedules is a DAG with three tasks
(vertices), where two of the tasks have an edge between them and the
third task has no incident edges.

The tasks connected by the edge have a unique minimum time schedule that
takes two steps.  So any schedule for the three tasks that also takes
only two steps will certainly be minimum time for the three.  But the
third task can be scheduled at the same time as either the first or
the second of the comparable tasks, giving two minimum schedules for
the three tasks.
\end{solution}
\end{problem}

%%%%%%%%%%%%%%%%%%%%%%%%%%%%%%%%%%%%%%%%%%%%%%%%%%%%%%%%%%%%%%%%%%%%%
% Problem ends here
%%%%%%%%%%%%%%%%%%%%%%%%%%%%%%%%%%%%%%%%%%%%%%%%%%%%%%%%%%%%%%%%%%%%%


\endinput
