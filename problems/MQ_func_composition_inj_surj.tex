\documentclass[problem]{mcs}

\begin{pcomments}
\pcomment{TP_surjR_injinvR}
\pcomment{small part of CP_surj_relation, other direction of TP_injR_surjinvR}
\pcomment{S13, miniq2}
\pcomment{ARM 3/21/13, revised 9/18/13}
\end{pcomments}

\pkeywords{
  relations
  functions
  injections
  surjections
}

%%%%%%%%%%%%%%%%%%%%%%%%%%%%%%%%%%%%%%%%%%%%%%%%%%%%%%%%%%%%%%%%%%%%%
% Problem starts here
% %%%%%%%%%%%%%%%%%%%%%%%%%%%%%%%%%%%%%%%%%%%%%%%%%%%%%%%%%%%%%%%%%%%%

\begin{problem}
Let $f:A \to B$ and $g:B \to C$ be a total injective functions $[=
  1\ \text{out}, \le 1\ \text{in}]$.  Prove that
\[
f^{-1} \circ g^{-1} \circ g \circ f = \ident{A},
\]
where $\ident{A}$ is the identity function on $A$.

\begin{solution}
There is a unique $f$-arrow from any element of $A$ to a unique
element of $B$, and a unique $g$-arrow from any element of $B$ to a
unique element of $C$.  Arrows of the composition $g \circ f$
correspond to following an $f$-arrow followed by the unique $g$-arrow
that starts where the $f$-arrow ends.  Further composing with the
inverse functions corresponds to following these successive $f$- and
$g$-arrows backwards.  Since the arrows beginning and ending at any
point are unique, going backwards implies finishing where the arrows
began.  That is, starting at any $a \in A$ and following the forward
and backward arrows in the composition leads back to $a$.  So the
composition defines the identity function on $A$.
\end{solution}
\end{problem}
%%%%%%%%%%%%%%%%%%%%%%%%%%%%%%%%%%%%%%%%%%%%%%%%%%%%%%%%%%%%%%%%%%%%%
% Problem ends here
%%%%%%%%%%%%%%%%%%%%%%%%%%%%%%%%%%%%%%%%%%%%%%%%%%%%%%%%%%%%%%%%%%%%%

\endinput


