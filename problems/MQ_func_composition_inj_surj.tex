\documentclass[problem]{mcs}

\begin{pcomments}
\pcomment{TP_surjR_injinvR}
\pcomment{small part of CP_surj_relation, other direction of TP_injR_surjinvR}
\pcomment{S13, miniq2}
\pcomment{ARM 3/21/13}
\end{pcomments}

\pkeywords{
  relations
  functions
  injections
  surjections
}

%%%%%%%%%%%%%%%%%%%%%%%%%%%%%%%%%%%%%%%%%%%%%%%%%%%%%%%%%%%%%%%%%%%%%
% Problem starts here
%%%%%%%%%%%%%%%%%%%%%%%%%%%%%%%%%%%%%%%%%%%%%%%%%%%%%%%%%%%%%%%%%%%%%

\begin{problem}

Let $f$ be a bijection that maps A to B and let $g$ be an injection that maps B to C. Let $a$ be an element of A.

True or False: $((f^{-1} \circ g^{-1} \circ g \circ f)(a) = a$

Please prove or disprove. You may use arrow terminology and arrow diagrams to explain.

\begin{solution}
True. Since both $f$ and $g$ are injective, they have unique inverses. In terms of arrows, there is only one arrow going into any element in the ranges of $f$ and $g$, so when we take the inverses of the functions and reverse the arrows, we can only get back to where we started from. 

\end{solution}
\end{problem}
%%%%%%%%%%%%%%%%%%%%%%%%%%%%%%%%%%%%%%%%%%%%%%%%%%%%%%%%%%%%%%%%%%%%%
% Problem ends here
%%%%%%%%%%%%%%%%%%%%%%%%%%%%%%%%%%%%%%%%%%%%%%%%%%%%%%%%%%%%%%%%%%%%%

\endinput


