\documentclass[problem]{mcs}

\begin{pcomments}
    \pcomment{TP_Counting_Subsets}
    \pcomment{Converted from counting-subsets.scm
              by scmtotex and dmj
              on Sun 13 Jun 2010 03:52:26 PM EDT}
    \pcomment{revised by drewe 2 Aug 2011}
\end{pcomments}

\begin{problem}

%% type: short-answer
%% title: Counting Subsets

Consider a 6 element set $X$ with elements $\{x_{1}, x_{2}, x_{3},
x_{4}, x_{5}, x_{6}\}$.

%%     You may answer with a formula such as \textbf{(3*7)^2/(3 + 7)!}

\bparts

\ppart 
How many subsets of $X$ contain $x_{1}$?

\begin{solution}
$2^5$

We showed previously that there is a bijection between the
subsets of $X$ and binary strings of length~6. In the binary string,
one of the positions is now fixed---namely, the first position must
be~1 as we are only interested in subsets that have~$x_{1}$. We still
have 2 choices for all the other positions and by the product rule, we
get that the number of subsets is~$2^5$.
\end{solution}

\ppart 

How many subsets of $X$ contain $x_{2}$ and $x_{3}$ but do not
contain~$x_{6}$?

\begin{solution}
$2^3$

Same logic---a bijection between the answer to this question and the
number of binary strings of length~6 with 3~positions fixed. There are
$2^3$ subsets.
\end{solution}

\eparts

\end{problem}

\endinput
