\documentclass[problem]{mcs}

\begin{pcomments}
  \pcomment{MQ_Nerditosis_2}
\end{pcomments}

\pkeywords{
  conditional_probability
  tree_diagram
}

%%%%%%%%%%%%%%%%%%%%%%%%%%%%%%%%%%%%%%%%%%%%%%%%%%%%%%%%%%%%%%%%%%%%%
% Problem starts here
%%%%%%%%%%%%%%%%%%%%%%%%%%%%%%%%%%%%%%%%%%%%%%%%%%%%%%%%%%%%%%%%%%%%%

% F09, S09, S07

\begin{problem}
There is a rare and deadly disease called Nerditosis which afflicts about 20 MIT students in 100.
One symptom is a compulsion to refer to everything— fields of study, classes, buildings,
etc.— using numbers. It's horrible. As victims enter their final, downward spiral, they're
awarded a degree from MIT. Year ago, Professor Meyer came up with a test to diagnose this tragic disease, but it's not perfect.

The facts of the test:

-If you have Nerditosis, there's a 10\% chance the test will say you do not.

-If you don't have it, there's a 40\% chance the test will say you do.

A random MIT student is tested for the disease. If the test is positive, then what is the probability that the person has the disease?

Hint: You may use the four-step method and a tree diagram.

%%%todo: update solution

\begin{solution}

Let A be the event that the person has the disease. Let B be the event that the test is positive.

Then 

\begin{equation*}
\prcond{A}{B} = \frac{\pr{A \cap B}}{\pr{B}} = \frac{.18}{.18+.32} = \frac{9}{25} = .36
\end{equation*}

\end{solution}
\end{problem}

%%%%%%%%%%%%%%%%%%%%%%%%%%%%%%%%%%%%%%%%%%%%%%%%%%%%%%%%%%%%%%%%%%%%%
% Problem ends here
%%%%%%%%%%%%%%%%%%%%%%%%%%%%%%%%%%%%%%%%%%%%%%%%%%%%%%%%%%%%%%%%%%%%%

\endinput
