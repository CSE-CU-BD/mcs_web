\documentclass[problem]{mcs}

\begin{pcomments}
  \pcomment{TP_above_average_fingers}
  \pcomment{from: S10.TP13}
  \pcomment{edited by ARM and Liz Denys 4/29/10}
  \pcomment{edited by drewe 11 July 2011}
\end{pcomments}

\pkeywords{
  average
  Markov
}

%%%%%%%%%%%%%%%%%%%%%%%%%%%%%%%%%%%%%%%%%%%%%%%%%%%%%%%%%%%%%%%%%%%%%
% Problem starts here
%%%%%%%%%%%%%%%%%%%%%%%%%%%%%%%%%%%%%%%%%%%%%%%%%%%%%%%%%%%%%%%%%%%%%

\begin{problem}
  The vast majority of people have an above average number of fingers.
  Which of the following statements explain why this is true?  Explain
  your reasoning.

\begin{enumerate}
\item Most people have a super secret extra bonus finger of which they are
  unaware.
\begin{solution}
No: We don't think so.
\end{solution}

\item A pedantic minority don't count their thumbs as fingers, while the
  majority of people do.

\begin{solution}
No: The eccentric opinions of pedants don't matter, since they won't
affect the way we count fingers.
\end{solution}

\item Polydactyly is rarer than amputation.

\begin{solution}
No: We know most people have ten fingers, so we need to argue that the
average number of figures per person is less than ten.  This is the
same as showing that the total number of missing fingers is larger
than the total number of extra fingers.

But we are only given that the number of \emph{people} with more than
ten fingers is less than the number of people with fewer than ten
fingers.  If amputees were usually missing only one finger (which
might be true), while polydactyls usually had at least two extra
fingers (which is true), and polydactyls were only slightly rarer than
finger amputees (which is not really true, but this is not given),
then the total number of missing fingers in the world's population
would be fewer than the total number of extra fingers.
\end{solution}

\item When you add up the total number of fingers among the world's
  population and then divide by the size of the population, you get a
  number less than ten.
\begin{solution}
No: This statement means that the average number of fingers is less
than ten.  Given the obvious fact that the vast majority of people
have the usual ten fingers, this just restates the claim that most
people have an above average number of fingers, does it does not
explain why they do..
\end{solution}

\item This follows from \idx{Markov's Theorem}, since no one has a
  negative number of fingers.

\begin{solution}
No: Markov's Theorem has no apparent relevance to the claim.  Markov's
Theorem simply states that the fraction of people who make up the
majority, multiplied by the above average number of fingers, namely 10,
cannot be greater than the average.  So if 99 percent of the world has 10
fingers, Markov's Theorem implies that the average number of fingers is at
least 9.9, which says nothing about how the the average compares to
10
\end{solution}

\item Missing fingers are more common than extra ones.

\begin{solution}
Yes: As long as the total number of missing fingers in the population is
  larger than the total number of extra ones---it doesn't matter
  \emph{how much} larger---the average number of fingers will be less
  than ten.

\end{solution}

\end{enumerate}

\end{problem}

%%%%%%%%%%%%%%%%%%%%%%%%%%%%%%%%%%%%%%%%%%%%%%%%%%%%%%%%%%%%%%%%%%%%%
% Problem ends here
%%%%%%%%%%%%%%%%%%%%%%%%%%%%%%%%%%%%%%%%%%%%%%%%%%%%%%%%%%%%%%%%%%%%%

\endinput
