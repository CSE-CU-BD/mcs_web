\documentclass[problem]{mcs}

\begin{pcomments}
  \pcomment{CP_probable_satisfiability}
  \pcomment{from: S07.cp13f, S06.ps10 from S04.cp14w}
  \pcomment{revised from PS_logic_and_probability}
  \pcomment{rewritten by ARM 5/17/17}
  \pcomment{CP_probable_satisfiability_nk is parameterized version}
\end{pcomments}

\pkeywords{
  random_variables
  satisfiability
  propositions
  disjunction
  expectation
}

%%%%%%%%%%%%%%%%%%%%%%%%%%%%%%%%%%%%%%%%%%%%%%%%%%%%%%%%%%%%%%%%%%%%%
% Problem starts here
%%%%%%%%%%%%%%%%%%%%%%%%%%%%%%%%%%%%%%%%%%%%%%%%%%%%%%%%%%%%%%%%%%%%%

\begin{staffnotes}
\textbf{S17 final 8pts: parts(a),(b) 2 pts each; part(c) 4pts}
\end{staffnotes}

\begin{problem}
A \emph{literal} is a propositional variable $P$ or its negation
$\bar{P}$, where as usual ``$\bar{P}$'' abbreviates ``$\QNOT(P)$.'' A
\emph{3-clause} is an $\QOR$ of three literals from three different
variables.  For example,
\[
P_1 \QOR P_2 \QOR \bar{P_3}
\]
is a 3-clause, but $P_1 \QOR \bar{P_1} \QOR P_2$ is not because $P_1$
appears twice.  A \emph{3-CNF} is a formula that is an $\QAND$ of
3-clauses.  For example,
\[
(P_1 \QOR P_2 \QOR \bar{P_3}) \QAND (\bar{P_1 } \QOR P_3 \QOR \bar{P_4}) \QAND
(P_2 \QOR P_3 \QOR \bar{P_4})
\]
is a 3-CNF.

Suppose that $G$ is a 3-CNF with seven 3-clauses.  Assign true/false
values to the variables in $G$ independently and with equal
probability.

\bparts

\ppart What is the probability that the $n$th clause is true?

% \hfill\examrule[0.7in]

\begin{center}
\exambox{0.6in}{0.5in}{-0.1in}
\end{center}

\examspace[0.7in]

\begin{solution}
A 3-clause is true unless all three of its literals are false.  So it
is true with probability:
\[
1 - \paren{\frac{1}{2}}^3 = \frac{7}{8}.
\]
\end{solution}

\ppart\label{e3cf} What is the expected number of true 3-clauses in $G$?

%\hfill\examrule{0.7in}

\begin{center}
\exambox{0.6in}{0.5in}{-0.1in}
\end{center}

\examspace[0.7in]

\begin{solution}
Let $T_n$ be the indicator variable for the truth of the $n$th
3-clause in $G$.  The number of true propositions is $T_1 + \cdots +
T_7$, so the expected number is:
%
\begin{align*}
\expect{T_1 + \dots + T_7}
    & = \expect{T_1} + \cdots + \expect{T_7} \\
    & = 7\frac{7}{8}\\
    & = \frac{49}{8}= 6 \frac{1}{8}.
\end{align*}
\end{solution}

\ppart Use the fact that the answer to part~\eqref{e3cf} is greater
than six to conclude $G$ must be satisfiable.

\examspace[4in]

\begin{solution}
A random variable cannot always be less than its expectation, so there
must be some truth assignment outcome for which
\[
T_1 + \cdots + T_7 \geq 6 \frac{1}{8}.
\]
Since the sum will always be an integer, it follows that $T_1 + \cdots
+ T_7 = 7$ for at least one truth assignment, which means that with this
assignment all seven 3-clauses in $G$ are true.
\end{solution}

\eparts

\end{problem}

\endinput
