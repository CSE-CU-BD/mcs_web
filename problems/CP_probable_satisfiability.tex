\documentclass[problem]{mcs}

\begin{pcomments}
  \pcomment{CP_probable_satisfiability}
  \pcomment{from: S07.cp13f, S06.ps10 from S04.cp14w}
  \pcomment{revised from PS_logic_and_probability}
  \pcomment{parameterized version is CP_probable_satisfiability_nk}
\end{pcomments}

\pkeywords{
  random_variables
  satisfiability
  propositions
  disjunction
  expectation
}

%%%%%%%%%%%%%%%%%%%%%%%%%%%%%%%%%%%%%%%%%%%%%%%%%%%%%%%%%%%%%%%%%%%%%
% Problem starts here
%%%%%%%%%%%%%%%%%%%%%%%%%%%%%%%%%%%%%%%%%%%%%%%%%%%%%%%%%%%%%%%%%%%%%

\begin{problem}
Here are seven propositions:
\[
\begin{array}{rcrcr}
x_1       & \QOR & x_3        & \QOR & \bar{x_7} \\
\bar{x_5} & \QOR & x_6        & \QOR & x_7 \\
x_2       & \QOR & \bar{x_4}  & \QOR & x_6 \\
\bar{x_4} & \QOR & x_5        & \QOR &  \bar{x_7} \\
x_3       & \QOR & \bar{x_5}  & \QOR & \bar{x_8} \\
x_9       & \QOR & \bar{x_8}  & \QOR & x_2 \\
\bar{x_3} & \QOR & x_9        & \QOR & x_4
\end{array}
\]

Note that:

\begin{enumerate}

\item\label{3literals} Each proposition is the disjunction ($\QOR$) of three
terms of the form $x_i$ or the form $\bar{x_i}$.

\item\label{alldiff} The variables in the three terms in each proposition are all
different.

\end{enumerate}

Suppose that we assign true/false values to the variables
$x_1, \dots, x_9$ independently and with equal probability.

\bparts

\ppart What is the expected number of true propositions?\hfill\examrule[0.7in]

\hint Let $T_i$ be an indicator for the event that the $i$-th proposition is
true.

\begin{solution}
Each proposition is true unless all three of its terms are
false.  Thus, each proposition is true with probability:
%
\[
1 - \paren{\frac{1}{2}}^3 = \frac{7}{8}
\]

Let $T_i$ be an indicator for the event that the $i$-th proposition is
true.  Then the number of true propositions is $T_1 +
\dots + T_7$ and the expected number is:
%
\begin{align*}
\expect{T_1 + \dots + T_7}
    & = \expect{T_1} + \dots + \expect{T_7} \\
    & = \frac{7}{8} + \dots + \frac{7}{8} \\
    & = \frac{49}{8}= 6 \frac{1}{8}
\end{align*}
\end{solution}

\ppart Use your answer to prove that for \emph{any} set of 7 propositions
satisfying the conditions~\ref{3literals}.\ and~\ref{alldiff}., there is
an assignment to the variables that makes all 7 of the propositions true.

\examspace[4in]

\begin{solution}
The calculation that the expected number of true propositions is
$6\frac{1}{8}$ only used the fact that there were 7 propositions
satisfying the conditions~\ref{3literals}.\ and~\ref{alldiff}.  So for
\emph{any} such set of 7 propositions, the expected number of true
propositions is the same.

But a random variable can not always be less than its expectation, so
there must be some assignment such that
\[
T_1 + \cdots T_7 \geq 6 \frac{1}{8},
\]
which implies that $T_1 + \cdots + T_7 = 7$ for at least one outcome.
This outcome is an assignment to the variables such that all of the
propositions are true.
\end{solution}

\eparts

\end{problem}

\endinput
