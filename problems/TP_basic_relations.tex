\documentclass[problem]{mcs}

\begin{pcomments}
  \pcomment{TP_basic_relations}
  \pcomment{overlaps/subsumes TP_basic_partial_orders}
  \pcomment{by ARM 10/17/11}
\end{pcomments}


\pkeywords{
  partial_orders
  weak_partial_order  
  strict_partial_order
  transitive
  reflexive
  antisymmetric
  maximal
  minimal
  equivalence
}

%%%%%%%%%%%%%%%%%%%%%%%%%%%%%%%%%%%%%%%%%%%%%%%%%%%%%%%%%%%%%%%%%%%%%
% Problem starts here
%%%%%%%%%%%%%%%%%%%%%%%%%%%%%%%%%%%%%%%%%%%%%%%%%%%%%%%%%%%%%%%%%%%%%

\begin{problem}
For each of the binary relations below, state whether it is a strict
partial order, a weak partial order, an equivalence relation or none
of these.  If it is a partial order, state whether it is a linear
order.  If it is none, indicate which of the axioms for partial order
and equivalence relations it violates.


\begin{staffnotes}
This problem took longer than expected for students to go through in
the class.  Parts (f) and (h) are the trickiest parts, usually where
students made mistakes.  Give hints as needed to get them through
faster.
\end{staffnotes}

\bparts

\ppart The superset relation, $\supseteq$ on the power set
$\power{\set{1, 2, 3, 4, 5}}$.

\begin{solution}
This is a weak partial order, but not a linear one.  For example, 
the sets of size 3 form an antichain.
\end{solution}

\ppart The relation between any two nonnegative integers, $a$, $b$ that 
$a \equiv b \pmod{8}$.

%The remainder of $a$ divided by 8 equals the remainder of $b$ divided by 8.

\begin{solution}
An equivalence relation.%
\iffalse
Violates antisymmetry: $8 \mrel{R} 16$ and $16 \mrel{R} 8$ but $8 \neq
16$.  It is transitive, though.
\fi
\end{solution}

\ppart The relation between propositional formulas, $G$, $H$, that $[G
\QIMPLIES H]$ is valid.

\begin{solution}
  Violates antisymmetry: $P$ and $\QNOT(\QNOT(P))$ imply each other but
  are different expressions.  It is transitive, though.
\end{solution}

\ppart The relation between propositional formulas, $G$, $H$, that $[G
\QIFF H]$ is valid.

\begin{solution}
An equivalence relation.
\end{solution}


\ppart The relation 'beats' on Rock, Paper and Scissor (for those who don't
know the game Rock, Paper, Scissors, Rock beats Scissors, Scissors beats
Paper and Paper beats Rock).

\begin{solution}
  Obviously violates transitivity.  Irreflexive since nothing beats
  itself, and antisymmetric.
\end{solution}

\ppart The \idx{empty relation} on the set of real numbers.
\begin{solution}
  It's vacuously asymmetric and transitive, so it's a strict partial
  order.  It's not linear.  It not an equivalence relation because
  it is not reflexive.
\end{solution}

\ppart The \idx{identity relation} on the set of integers.

\begin{solution}

  It's obviously reflexive, antisymmetric and transitive, so it's a
  weak partial order.  It's not linear.  It's also an equivalence
  relation since it is symmetric as well.
\end{solution}


\ppart The divisibility relation on the integers, $\integers$.

\begin{solution}
Not antisymmetric: 3 and -3 divide each other.  It is transitive and reflexive.
\end{solution}

\eparts

\end{problem}

%%%%%%%%%%%%%%%%%%%%%%%%%%%%%%%%%%%%%%%%%%%%%%%%%%%%%%%%%%%%%%%%%%%%%
% Problem ends here
%%%%%%%%%%%%%%%%%%%%%%%%%%%%%%%%%%%%%%%%%%%%%%%%%%%%%%%%%%%%%%%%%%%%%

\endinput
