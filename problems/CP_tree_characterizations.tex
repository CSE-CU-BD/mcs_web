\documentclass[problem]{mcs}

\begin{pcomments}
  \pcomment{CP_tree_characterizations}
  \pcomment{subsumes PS_tree_definition}
  \pcomment{from: S09.cp6t, S02 pset4-2b}
  \pcomment{was commented out in S09}
\end{pcomments}

\pkeywords{
  trees
  graphs
  connectivity
  cycles
  unique_paths
}

%%%%%%%%%%%%%%%%%%%%%%%%%%%%%%%%%%%%%%%%%%%%%%%%%%%%%%%%%%%%%%%%%%%%%
% Problem starts here
%%%%%%%%%%%%%%%%%%%%%%%%%%%%%%%%%%%%%%%%%%%%%%%%%%%%%%%%%%%%%%%%%%%%%
\begin{problem}
Prove that a graph is a tree iff it has a unique  path between any
two vertices.

\begin{staffnotes}
Students should be told \emph{not} to look up the proof in the text
until they try this on their own.
\end{staffnotes}

\begin{solution}
Theorem~\bref{th:treeprops} shows that in a tree there are unique
paths between any two vertices.

Conversely, suppose we have a graph, $G$, with unique paths.  Now $G$
is connected since there is a path between any two vertices.  So we
need only show that $G$ has no positive length cycles.  But if there
was a positive length cycle in $G$, there are two paths between any
two vertices on the cycle (going one way around the cycle or the other
way around), a violation of uniqueness.  So $G$ must cannot have any
positive length cycles.
\end{solution}
\end{problem}

%%%%%%%%%%%%%%%%%%%%%%%%%%%%%%%%%%%%%%%%%%%%%%%%%%%%%%%%%%%%%%%%%%%%%
% Problem ends here
%%%%%%%%%%%%%%%%%%%%%%%%%%%%%%%%%%%%%%%%%%%%%%%%%%%%%%%%%%%%%%%%%%%%%
