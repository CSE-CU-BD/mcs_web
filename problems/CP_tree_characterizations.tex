\documentclass[problem]{mcs}

\begin{pcomments}
  \pcomment{CP_tree_characterizations}
  \pcomment{mostly already covered in the text in Theorem~th:treeprops}
  \pcomment{subsumes PS_tree_definition}
  \pcomment{from: S09.cp6t, S02 pset4-2b}
  \pcomment{was commented out in S09}
\end{pcomments}

\pkeywords{
  trees
  graphs
  connectivity
  cycles
  unique_paths
}

%%%%%%%%%%%%%%%%%%%%%%%%%%%%%%%%%%%%%%%%%%%%%%%%%%%%%%%%%%%%%%%%%%%%%
% Problem starts here
%%%%%%%%%%%%%%%%%%%%%%%%%%%%%%%%%%%%%%%%%%%%%%%%%%%%%%%%%%%%%%%%%%%%%
\begin{problem}
Prove that a graph is a tree iff it has a unique path between every two
vertices.

\begin{staffnotes}
Students should be told \emph{not} to look up the proof in the text
until they try this on their own.
\end{staffnotes}

\begin{solution}
Theorem~\bref{th:treeprops}.\bref{treeprops:uniquepath} shows that in a tree
there are unique paths between any two vertices.

\begin{staffnotes}
Since a tree is connected, there is at least one
path between every pair of vertices.  To show paths are unique:

\textbf{first proof}: Suppose for the purposes of contradiction, that
there are two different paths between some pair of vertices.  Then
there are two distinct paths $\walkv{p} \neq \walkv{q}$ between two
vertices with minimum total length $\lnth{\walkv{p}}+
\lnth{\walkv{q}}$.  If these paths shared a vertex, $w$, other than at
the start and end of the paths, then the parts of $\walkv{p}$ and
$\walkv{q}$ from start to $w$, or the parts of $\walkv{p}$ and
$\walkv{q}$ from $w$ to the end, must be distinct paths between the
same vertices with total length less than $\lnth{\walkv{p}}+
\lnth{\walkv{q}}$, contradicting the minimality of this sum.
Therefore, $\walkv{p}$ and $\walkv{q}$ are distinct paths with no
vertices in common besides their endpoints, and so
\[
\merge{\walkv{p}}{\text{reverse}(\walkv{q})}
\]
is a cycle.

\textbf{second proof}:  Beginning at $u$, let $x$ be the first
vertex where the paths diverge, and let $y$ be the next
vertex they share.  (For example, see
Figure~\ref{fig:5L}.)  Then there are two paths from $x$
to~$y$ with no common edges, which defines a cycle.
This is a contradiction, since trees are acyclic.
Therefore, there is exactly one path between every pair
of vertices.


\begin{figure}

\graphic{unique-path}

\caption{If there are two paths between $u$ and~$v$, the graph must
  contain a cycle.}

\label{fig:5L}
\end{figure}

\end{staffnotes}

Conversely, suppose we have a graph, $G$, with unique paths.  Now $G$
is connected since there is a path between any two vertices.  So we
need only show that $G$ has no cycles.  But if there was a cycle in
$G$, there are two paths between any two vertices on the cycle (going
one way around the cycle or the other way around), a violation of
uniqueness.  So $G$ cannot have any cycles.
\end{solution}
\end{problem}

%%%%%%%%%%%%%%%%%%%%%%%%%%%%%%%%%%%%%%%%%%%%%%%%%%%%%%%%%%%%%%%%%%%%%
% Problem ends here
%%%%%%%%%%%%%%%%%%%%%%%%%%%%%%%%%%%%%%%%%%%%%%%%%%%%%%%%%%%%%%%%%%%%%

\endinput
