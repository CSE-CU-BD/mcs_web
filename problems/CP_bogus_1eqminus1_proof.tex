\documentclass[problem]{mcs}

\begin{pcomments}
  \pcomment{CP_bogus_1eqminus1_proof}
  \pcomment{ARM 01/24/15 from welcome slides}
\end{pcomments}

\pkeywords{
  bogus_proof
  square_root
  complex_number
}

%%%%%%%%%%%%%%%%%%%%%%%%%%%%%%%%%%%%%%%%%%%%%%%%%%%%%%%%%%%%%%%%%%%%%
% Problem starts here
%%%%%%%%%%%%%%%%%%%%%%%%%%%%%%%%%%%%%%%%%%%%%%%%%%%%%%%%%%%%%%%%%%%%%

\begin{problem}
What's going on here?!

\[
1 = \sqrt{1} = \sqrt{(-1)(-1)} = \sqrt{-1} \sqrt{-1} = \paren{\sqrt{-1}}^2 = -1.
\]

\begin{problemparts} 

\ppart Precisely identify and explain the mistake(s) in this
\emph{bogus} proof.

\begin{solution}
The familiar rule that $\sqrt{rs} = \sqrt{r}\sqrt{s}$ (see part~\eqref{sqrtrssqrt2}) only holds for
\emph{positive} real numbers $r,s$ (OK, it holds for zero too).  It is not
valid for negative real numbers, as the middle equality shows.
\end{solution}

\ppart Prove (correctly) that if $1=-1$, then $2=1$.

\begin{solution}
\begin{align*}
1          & = -1,\\
\frac{1}{2}& = - \frac{1}{2}, & \text{multiply both sides by $\frac{1}{2}$}\\
2          & = 1              & \text{add $\frac{3}{2}$ to both sides}.
\end{align*}

If you deduce something false starting from true premises, then your
reasoning must be mistaken.  On the other hand, this example points
that starting from a false premise, other false conclusions can be
deduced by \emph{correct} reasoning.

Indeed, basic rules of logic imply that if you assume something false,
you can prove \emph{anything}.  A story illustrating this fact is told
about the Nobel Prize winning logician/philosopher Bertrand Russell, who
supposedly was challenged by a socialite at a party to prove from $1 =
-1$ that he was the Pope.  Russell was famous for his quick wit, and he
is said to have reasoned as above that if $1 = -1$ then $2=1$.  He
went on to observe, ``Now I and the Pope are clearly two, but since
$2=1$, I and the Pope are one, that is, I am the Pope.''

\end{solution}

\ppart\label{sqrtrssqrt2} Every \emph{positive} real number, $r$, has
two square roots, one positive and the other negative.  The standard
convention is that the expression $\sqrt{r}$ refers to the
\emph{positive} square root of $r$.  Assuming familiar properties of
multiplication of real numbers, prove that for positive real numbers
$r$ and $s$,
\[
\sqrt{rs} = \sqrt{r} \sqrt{s}.
\]

\begin{solution}
Since $\sqrt{rs}$ refers to the positive square root of $rs$, we just
have to verify that the positive number $\sqrt{r} \sqrt{s}$ is a
square root of $rs$, that is, that
\[
\paren{\sqrt{r} \sqrt{s}}^2 = rs.
\]
This follows directly from the fact that parentheses can be ignored in
multiplications (in Math jargon, multiplication is
\emph{\idx{associative}}) and that multiplications can be reordered
(multiplication is \emph{\idx{commutative}}), so
\begin{align*}
\paren{\sqrt{r} \sqrt{s}}^2
   & = \paren{\sqrt{r} \sqrt{s}} \paren{\sqrt{r} \sqrt{s}}
           & \text{(def of squaring)}\\
   & = \sqrt{r} \sqrt{s} \sqrt{r} \sqrt{s} & \text{(ignoring parens)}\\
   & = \sqrt{r} \sqrt{r} \sqrt{s} \sqrt{s} & \text{(reordering)}\\
   & = \paren{\sqrt{r}}^2 \paren{\sqrt{s}}^2 & \text{(def of squaring)}\\
   & = rs & \text{(def of square root)}.
\end{align*}
\end{solution}

\end{problemparts}
\end{problem}

%%%%%%%%%%%%%%%%%%%%%%%%%%%%%%%%%%%%%%%%%%%%%%%%%%%%%%%%%%%%%%%%%%%%%
% Problem ends here
%%%%%%%%%%%%%%%%%%%%%%%%%%%%%%%%%%%%%%%%%%%%%%%%%%%%%%%%%%%%%%%%%%%%%

\endinput
