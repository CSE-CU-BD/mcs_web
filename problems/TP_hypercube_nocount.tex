\documentclass[problem]{mcs}

\begin{pcomments}
  \pcomment{TP_hypercube_nocount}
  \pcomment{renamed from MQ}
  \pcomment{shorter version of MQ_hypercube, omitting counting H_n edges}
  \pcomment{ARM 3/30/15: revised to drop $H_n$ refs}
  \pcomment{ARM 3/29/14: all the H_n stuff is irrelevant for what's
    left below; don't use}
\end{pcomments}

\pkeywords{
  tree
  spanning_tree
  hypercube
}

%%%%%%%%%%%%%%%%%%%%%%%%%%%%%%%%%%%%%%%%%%%%%%%%%%%%%%%%%%%%%%%%%%%%%
% Problem starts here
%%%%%%%%%%%%%%%%%%%%%%%%%%%%%%%%%%%%%%%%%%%%%%%%%%%%%%%%%%%%%%%%%%%%%

\begin{problem}
Let $H_3$ be the graph shown in Figure~\ref{fig:H3}.  Explain why it
is impossible to find two spanning trees of $H_3$ that have no edges
in common.
\begin{figure}[h]
\graphic[width=0.3\linewidth]{MQ_H3}
\caption{$H_3$\,.\label{fig:H3}}
\end{figure}
\examspace[3in]

%\hint Count vertices \& edges.

\begin{solution}
$H_3$ has 8 vertices, so any spanning tree must have $8-1=7$
  edges. But $H_3$ has only 12 edges, so any two sets of 7 edges must
  overlap.
\end{solution}

\end{problem}

\endinput
