\documentclass[problem]{mcs}

\begin{pcomments}
  \pcomment{MQ_count_double_deck}
  \pcomment{from: F03.Quiz2}
\end{pcomments}

\pkeywords{
  counting
  division_rule
  card
}

%%%%%%%%%%%%%%%%%%%%%%%%%%%%%%%%%%%%%%%%%%%%%%%%%%%%%%%%%%%%%%%%%%%%%
% Problem starts here
%%%%%%%%%%%%%%%%%%%%%%%%%%%%%%%%%%%%%%%%%%%%%%%%%%%%%%%%%%%%%%%%%%%%%

\begin{problem}

\begin{problemparts}

\problempart
  Suppose that two identical 52-card decks are mixed together.  Write a
  simple formula for the number of 104-card double-deck mixes that are
  possible.

\examspace[3in]

\begin{solution}
  If the cards were all distinct, there would be $104!$ ``distinct" mixes.
  Call two distinct mixes \emph{similar} if they differ only in that the
  positions of the two distinct copies of the same card are exchanged.  So
  two distinct mixes correspond to the same double-deck mix iff they are
  similar.  Therefore, the map from distinct mixes to double-deck mixes is
  $(2!)^{52}$ to one, and so by the division rule, there are
\[
\frac{104!}{(2!)^{52}}
\]
double-deck mixes.
\end{solution}

\problempart
Using integers in the interval [1,n], how many strictly increasing length-m sequences are there?

\begin{solution}

\[
\binom{n}{m}
\]

Strictly increasing length-m sequences are uniquely determined by its elements.  
This could also be done via bijection with $0^{a_1} 1 0^{a_2} 1$ \dots $0^{a_m} 1 0^k$ with $k$ chosen so the number of 0's is $n-m$.

\end{solution}

\end{problemparts}

\end{problem}

%%%%%%%%%%%%%%%%%%%%%%%%%%%%%%%%%%%%%%%%%%%%%%%%%%%%%%%%%%%%%%%%%%%%%
% Problem ends here
%%%%%%%%%%%%%%%%%%%%%%%%%%%%%%%%%%%%%%%%%%%%%%%%%%%%%%%%%%%%%%%%%%%%%

\endinput

