\documentclass[problem]{mcs}

\begin{pcomments}
    \pcomment{MQ_basic_set_formulas-afternoon}
    \pcomment{excerpted from TP_basic_set_formulas}
\end{pcomments}

\pkeywords{
  logic
  sets
  set_theory
  predicate
  formula
  union
}

\begin{problem}
A \emph{formula of \idx{set theory}} is a predicate formula that only
uses the predicate ``$x \in y$.''  The domain of discourse is the
collection of sets, and ``$x \in y$'' is interpreted to mean that $x$
is one of the elements in $y$, where $x$ and $y$ are variables that
range over sets.

For example, since $x$ and $y$ are the same set iff they have the same
members, here's how we can express $x = y$ with a
formula of set theory:
\[
\forall z.\, (z \in x\ \QIFF\ z \in y).
\]

Express the assertion that $x$ has exactly 2 elements, that is,
$\card{x} =2$ by a formula of set theory.

\begin{solution}
\[
\exists y,z.\, y \neq z \QAND\ \forall w.\, w \in x \QIFF (x = y\ \QOR\ w = z).
\]
\end{solution}

\end{problem}

\endinput
