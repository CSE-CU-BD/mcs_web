\documentclass[problem]{mcs}

\begin{pcomments}
  \pcomment{CP_division_rule_assign_groups}
  \pcomment{from: S07 cp10w}
\end{pcomments}

\pkeywords{
  division_rule
  counting
}

%%%%%%%%%%%%%%%%%%%%%%%%%%%%%%%%%%%%%%%%%%%%%%%%%%%%%%%%%%%%%%%%%%%%%
% Problem starts here
%%%%%%%%%%%%%%%%%%%%%%%%%%%%%%%%%%%%%%%%%%%%%%%%%%%%%%%%%%%%%%%%%%%%%

\begin{problem}

Your 6.006 tutorial has 12 students, who are supposed to break up into 4
groups of 3 students each.  Your TA has observed that the students waste
too much time trying to form balanced groups, so he decided to pre-assign
students to groups and email the group assignments to his students.

\bparts

\ppart Your TA has a list of the 12 students in front of him, so he
divides the list into consecutive groups of 3.  For example, if the list
is ABCDEFGHIJKL, the TA would define a sequence of four groups to be
$(\set{A,B,C}, \set{D,E,F}, \set{G,H,I}, \set{J,K,L})$.  This way of
forming groups defines a mapping from a list of twelve students to a
sequence of four groups.  This is a $k$-to-1 mapping for what $k$?

\begin{solution}
Two lists map to the same sequence of groups iff the first 3 students are
the same on both lists, and likewise for the second, third, and fourth
consecutive sublists of 3 students.  So for a given sequence of 4 groups,
the number of lists which map to it is
\[
(3!)^4
\]
because there are $3!$ ways to order the students in each of the
4 consecutive sublists.
\end{solution}

\ppart A group assignment specifies which students are in the same group,
but not any order in which the groups should be listed.  If we map a
sequence of 4 groups,
\[
(\set{A,B,C}, \set{D,E,F}, \set{G,H,I},\set{J,K,L}),
\]
into a group assignment
\[
\set{\set{A,B,C}, \set{D,E,F}, \set{G,H,I}, \set{J,K,L}},
\]
this mapping is $j$-to-1 for what $j$?

\begin{solution} $4!$.

Each of the $4!$ sequences of a particular set of four groups maps to
that set of groups.
\end{solution}

\ppart How many group assignments are possible?

\begin{solution}
\[
\frac{12!}{4!\cdot(3!)^4}=15400
\]
different assignments.

There are $12!$ possible lists of students, and we can map each list to an
assignment by first mapping the list to a sequence of four groups, and
then mapping the sequence to the assignment.  Since the first map is 
$(3!)^4$-to-1 and and the second is $4!$-to-1, the composite map is
$(3!)^4\cdot 4!$-to-1.  So by the Division Rule, $12! = \paren{(3!)^4\cdot 4!}A$
where $A$ is the number of assignments.
\end{solution}

\ppart In how many ways can $3n$ students be broken up into $n$ groups of
3?  
\begin{solution}
\[
\frac{(3n)!}{(3!)^n n!}.
\]
This follows simply by replacing ``12'' by ``$3n$'' and ``4'' by
``$n$''in the solution to the previous problem parts.
\end{solution}

\eparts

\end{problem}

\endinput
