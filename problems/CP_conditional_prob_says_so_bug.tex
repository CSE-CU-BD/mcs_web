\documentclass[problem]{mcs}

\begin{pcomments}
  \pcomment{CP_conditional_prob_says_so_bug}
  \pcomment{similar to MQ_voldemort_returns}
  \pcomment{from: F05.ps9 problem 2}
  \pcomment{rephrased with hint--ARM 12/2/11}
\end{pcomments}

\pkeywords{
  conditional_probability
  tree_diagram
  four-step_method
}

%%%%%%%%%%%%%%%%%%%%%%%%%%%%%%%%%%%%%%%%%%%%%%%%%%%%%%%%%%%%%%%%%%%%%
% Problem starts here
%%%%%%%%%%%%%%%%%%%%%%%%%%%%%%%%%%%%%%%%%%%%%%%%%%%%%%%%%%%%%%%%%%%%%

\begin{problem}
There are three prisoners in a maximum-security prison for fictional
villains: the Evil Wizard Voldemort, the Dark Lord Sauron, and Little
Bunny Foo-Foo.  The parole board has declared that it will release two
of the three, chosen uniformly at random, but has not yet released
their names.  Naturally, Sauron figures that he will be released to
his home in Mordor, where the shadows lie, with probability
$2/3$.

A guard offers to tell Sauron the name of one of the other prisoners
who will be released (either Voldemort or Foo-Foo).  If the guard has
a choice of naming either Voldemort or Foo-Foo (because both are to be
released), he names one of the two with equal probability.

Sauron knows the guard to be a truthful fellow.  However, Sauron
declines this offer.  He reasons that knowing what the guards says
will reduce his chances, so he is better off not knowing.  For
example, if the guard says, ``Little Bunny Foo-Foo will be released'',
then his own probability of release will drop to $1/2$ because he will
then know that either he or Voldemort will also be released, and these
two events are equally likely.

Dark Lord Sauron has made a typical mistake when reasoning about
conditional probability.  Using a tree diagram and the four-step
method, \textbf{explain his mistake}.  What is the probability that
Sauron is released given that the guard says Foo-Foo is released?

\hint Define the events $S$, $F$, and ``$F$'' as follows:
%
\begin{align*}
\text{``$F$''} & = \text{Guard says Foo-Foo is released} \\
F & = \text{Foo-Foo is released} \\
S & = \text{Sauron is released}
\end{align*}

\begin{solution}
Sauron's mistake can be explained as his confusing the two different
events $F$ and ``$F$''.  His observation that $\prcond{S}{F} = 1/2$ is
correct, but that's the wrong thing to calculate.  He should be
calculating $\prcond{S}{\text{``$F$''}}$.

To clarify the error and work out the proper probability, let's begin
by working out the sample space, noting events of interest, and
computing outcome probabilities:
%
\begin{center}
\begin{picture}(360,175)(0,-40)
%\put(0,-40){\dashbox(360,175){}} % bounding box
\put(0,60){\line(1,1){60}}
\put(0,60){\line(1,0){60}}
\put(0,60){\line(1,-1){60}}
\put(30,-10){\makebox(0,0){released}}
\put(60,120){\line(1,0){60}}
\put(60,60){\line(2,1){60}}
\put(60,60){\line(2,-1){60}}
\put(60,0){\line(1,0){60}}
\put(90,-25){\makebox(0,0){guard says}}
\put(11,90){\makebox(0,0){$F,S$}}
\put(40,68){\makebox(0,0){$F,V$}}
\put(11,30){\makebox(0,0){$V,S$}}
\put(52,96){\makebox(0,0){$1/3$}}
\put(40,50){\makebox(0,0){$1/3$}}
\put(52,24){\makebox(0,0){$1/3$}}
\put(90,128){\makebox(0,0){$F$}}
\put(90,90){\makebox(0,0){$F$}}
\put(90,30){\makebox(0,0){$V$}}
\put(90,-10){\makebox(0,0){$V$}}
\put(90,110){\makebox(0,0){$1$}}
\put(102,70){\makebox(0,0){$1/2$}}
\put(102,50){\makebox(0,0){$1/2$}}
\put(90,8){\makebox(0,0){$1$}}
\put(150,120){\makebox(0,0){$1/3$}}
\put(150,90){\makebox(0,0){$1/6$}}
\put(150,30){\makebox(0,0){$1/6$}}
\put(150,0){\makebox(0,0){$1/3$}}
\put(150,-20){\makebox(0,0){prob.}}
\put(210,120){\makebox(0,0){$\times$}}
\put(210,90){\makebox(0,0){$\times$}}
\put(210,30){\makebox(0,0){}}
\put(210,0){\makebox(0,0){}}
\put(210,-25){\makebox(0,0){\shortstack{guard says\\``Foo-foo''}}}
\put(270,120){\makebox(0,0){$\times$}}
\put(270,90){\makebox(0,0){$\times$}}
\put(270,30){\makebox(0,0){$\times$}}
\put(270,0){\makebox(0,0){}}
\put(270,-25){\makebox(0,0){\shortstack{Foo-foo\\released}}}
\put(330,120){\makebox(0,0){$\times$}}
\put(330,90){\makebox(0,0){}}
\put(330,30){\makebox(0,0){}}
\put(330,0){\makebox(0,0){$\times$}}
\put(330,-25){\makebox(0,0){\shortstack{Sauron\\released}}}
\end{picture}
\end{center}

The outcomes in each of these events are noted in the tree diagram.

The tree shows how the event $F$ (Foo-foo will be released) is
different from the event ``$F$'' (the guard \emph{says} Foo-foo will
be released).  In particular, the probability that Sauron is released,
given that Foo-foo is released, is indeed $1/2$:
%
\begin{align*}
\prcond{S}{F}
    & = \frac{\pr{S \cap F}}{\pr{F}} \\
    & = \frac{\frac{1}{3}}{\frac{1}{3}+\frac{1}{6}+\frac{1}{6}} \\
    & = \frac{1}{2}
\end{align*}
%
But the probability that Sauron is released given that the guard
\emph{actually says so} is still $2/3$:
%
\begin{align*}
\prcond{S}{\text{``$F$''}}
    & = \frac{\pr{S \cap \text{``$F$''}}}{\pr{\text{``$F$''}}} \\
    & = \frac{\frac{1}{3}}{\frac{1}{3} + \frac{1}{6}} \\
    & = \frac{2}{3}
\end{align*}
%
So Sauron's probability of release is unchanged by the guard's
statement.
\end{solution}

\end{problem}


%%%%%%%%%%%%%%%%%%%%%%%%%%%%%%%%%%%%%%%%%%%%%%%%%%%%%%%%%%%%%%%%%%%%%
% Problem ends here
%%%%%%%%%%%%%%%%%%%%%%%%%%%%%%%%%%%%%%%%%%%%%%%%%%%%%%%%%%%%%%%%%%%%%

\endinput
