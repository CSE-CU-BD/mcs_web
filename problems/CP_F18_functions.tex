\documentclass[problem]{mcs}

\begin{pcomments}
  \pcomment{CP_F18_functions}
  \pcomment{from: S09.cp5m}
\end{pcomments}

\pkeywords{
  recursive_data
  structural_induction
  functions
}

%%%%%%%%%%%%%%%%%%%%%%%%%%%%%%%%%%%%%%%%%%%%%%%%%%%%%%%%%%%%%%%%%%%%%
% Problem starts here
%%%%%%%%%%%%%%%%%%%%%%%%%%%%%%%%%%%%%%%%%%%%%%%%%%%%%%%%%%%%%%%%%%%%%


\begin{problem}
%
The Elementary 18.01 Functions ($\EF$'s) are the set of functions of
one real variable defined recursively as follows:

\textbf{Base cases:}
\begin{itemize}
\item The identity function, $\ide(x) \eqdef x$ is an $\EF$,
\item any constant function is an $\EF$,
\item the sine function is an $\EF$,
\end{itemize}

\textbf{Constructor cases:}

If $f,g$ are $\EF$'s, then so are
\begin{enumerate}
\item $f + g$, $fg$, $2^g$,\label{+-}
\item the inverse function $f^{(-1)}$,\label{inversefunc}
\item the composition $f \compose g$.\label{cmp}

\end{enumerate}

\bparts

\ppart\label{1over} Prove that the function $1/x$ is an $\EF$.

\textbf{Warning:} Don't confuse $1/x = x^{-1}$ with the inverse,
$\ide^{(-1)}$ of the identity function $\ide(x)$.  The inverse
$\ide^{(-1)}$ is equal to $\ide$.

\begin{solution}
$\log_2$ is the inverse of $2^x$ so $\log_2 x \in \EF$.  Therefore
  so is $c\cdot \log_2 x$ for any constant $c$, and hence $2^{c \log_2 x} =
  x^c \in \EF$.  Now let $c = -1$ to get $x^{-1} = 1/x \in
  \EF$.\footnote{There's a little problem here: since $\log_2 x$ is not
    real-valued for $x \leq 0$, the function $f(x) \eqdef 1/x$ constructed
    in this way is only defined for $x >0$.  To get an $\EF$ equal to
    $1/x$ defined for all $x \neq 0$, use $\paren{x/\abs{x}}\cdot
    f(\abs{x})$, where $\abs{x} = \sqrt{x^2}$.}
\end{solution}

\ppart Prove by Structural Induction on this definition that the
Elementary 18.01 Functions are \emph{closed under taking derivatives}.
That is, show that if $f(x)$ is an $\EF$, then so is $f^{\prime} \eqdef
df/dx$.  (Just work out 2 or 3 of the most interesting constructor cases;
you may skip the less interesting ones.)

\begin{solution}

\begin{proof}
By Structural Induction on def of $f \in \EF$.  The induction hypothesis
is the above statement to be shown.

\item[Base Cases:] We want to show that the derivatives of all the
  base case functions are in $\EF$.

  This is easy: for example, $d\, \ide(x)/dx = 1$ is a constant function,
  and so is in $\EF$. Similarly, $d\, \sin(x)/dx = \cos(x)$ which is
  also in $\EF$ since $\cos(x) = \sin(x+\pi/2) \in \EF$ by rules for
  constant functions, the identity function, sum, and composition with
  sine.

This proves that the induction hypothesis holds in the Base cases.

\item[Constructor Cases:] ($f^{(-1)}$).  Assume $f, df/dx \in \EF$ to prove
  $d\, f^{(-1)}(x)/dx \in \EF$.
Letting $y = f(x)$, so $x=f^{(-1)}(y)$, we know from Leibniz's rule in
calculus that
\begin{equation}\label{Leib}
df^{(-1)}(y)/dy = dx/dy = \frac{1}{dy/dx}.
\end{equation}
For example,
\[
d \sin^{(-1)}(y)/dy = 1/(d \sin(x)/dx) = 1/\cos(x) = 1/\cos(\sin^{(-1)}(y)).
\]
Stated as in ~(\ref{Leib}), this rule is easy to remember, but can easily
be misleading because of the variable switching between $x$ and $y$.  It's
more clearly stated using variable-free notation:
\begin{equation}\label{Leib2}
(f^{(-1)})' = \frac{1}{f'\compose f^{(-1)}}.
\end{equation}
Now, since $f' \in \EF$ (by assumption), so is $f^{(-1)}$ (by
constructor rule~\ref{inversefunc}), and therefore so is their
composition (by rule~\ref{cmp}), as well as the composition of $1/x$
with $f'\compose f^{(-1)}$ (by part~\eqref{1over}).  Hence the
righthand side of equation~(\ref{Leib2}) defines a function in $\EF$.

\item[Constructor Case:] ($f \compose g$).  Assume $f, g, df/dx, dg/dx \in \EF$
to prove $d(f \compose g)(x)/dx \in \EF$.

The Chain Rule states that
\[
\frac{d(f(g(x)))}{dx} = \frac{d f(g)}{dg}\cdot\frac{dg}{dx}.
\]
Stated more clearly in variable-free notation, this is
\[
(f \compose g)' = (f' \compose g) \cdot g'.
\]
The righthand side of this equation defines a function in $\EF$ by
constructor rules~\ref{cmp} and~\ref{+-}.

The Constructor cases~\ref{+-} are straighforward.  So we conclude
that the induction hypothesis holds in all Constructor cases.

This completes the proof by structural induction that the statement holds
for all $f \in \EF$.
\end{proof}

\end{solution}
\eparts
\end{problem}


%%%%%%%%%%%%%%%%%%%%%%%%%%%%%%%%%%%%%%%%%%%%%%%%%%%%%%%%%%%%%%%%%%%%%
% Problem ends here
%%%%%%%%%%%%%%%%%%%%%%%%%%%%%%%%%%%%%%%%%%%%%%%%%%%%%%%%%%%%%%%%%%%%%

\endinput

