\documentclass[problem]{mcs}

\begin{pcomments}
  \pcomment{CP_F18_functions}
  \pcomment{from: S09.cp5m}
\end{pcomments}

\pkeywords{
  recursive_data
  structural_induction
  functions
}

%%%%%%%%%%%%%%%%%%%%%%%%%%%%%%%%%%%%%%%%%%%%%%%%%%%%%%%%%%%%%%%%%%%%%
% Problem starts here
%%%%%%%%%%%%%%%%%%%%%%%%%%%%%%%%%%%%%%%%%%%%%%%%%%%%%%%%%%%%%%%%%%%%%


\begin{problem}
%
The Elementary 18.01 Functions ($\EF$'s) are the set of functions of
one real variable defined recursively as follows:

\inductioncase{Base cases:}
\begin{itemize}
\item The identity function $\ide(x) \eqdef x$ is an $\EF$,
\item any constant function is an $\EF$,
\item the sine function is an $\EF$,
\end{itemize}

\inductioncase{Constructor cases:}

If $f,g$ are $\EF$'s, then so are
\begin{enumerate}
\item $f + g$, $fg$, $2^g$,\label{+-}
\item the inverse function $\inv{f}$,\label{inversefunc}
\item the composition $f \compose g$.\label{cmp}

\end{enumerate}

\bparts

\ppart\label{1over} Prove that the function $1/x$ is an $\EF$.

\textbf{Warning:} Don't confuse $1/x = x^{-1}$ with the inverse
$\inv{\ide}$ of the identity function $\ide(x)$.  The inverse
$\inv{\ide}$ is equal to $\ide$.

\begin{solution}
$\log_2$ is the inverse of $2^x$ so $\log_2 x \in \EF$.  Therefore so
  is $c\cdot \log_2 x$ for any constant $c$, and hence $2^{c \log_2 x}
  = x^c \in \EF$.  Now let $c = -1$ to get $x^{-1} = 1/x \in
  \EF$.\footnote{There's a little problem here: since $\log_2 x$ is
    not real-valued for $x \leq 0$, the function $f(x) \eqdef 1/x$
    constructed in this way is only defined for $x >0$.  To get an
    $\EF$ equal to $1/x$ defined for all $x \neq 0$, use
    $x f(x^2)$.}
\end{solution}

\ppart Prove by Structural Induction on this definition that the
Elementary 18.01 Functions are \emph{closed under taking derivatives}.
That is, show that if $f(x)$ is an $\EF$, then so is $f^{\prime} \eqdef
df/dx$.  (Just work out 2 or 3 of the most interesting constructor cases;
you may skip the less interesting ones.)

\begin{solution}

\begin{proof}
By Structural Induction on def of $f \in \EF$.  The induction hypothesis
is
\[
P(f) \eqdef f' \in \EF.
\]

\inductioncase[Base Cases]: We want to show that the derivatives of all the
  base case functions are in $\EF$.

  This is easy: for example, $d\, \ide(x)/dx = 1$ is a constant function,
  and so is in $\EF$. Similarly, $d\, \sin(x)/dx = \cos(x)$ which is
  also in $\EF$ since $\cos(x) = \sin(x+\pi/2) \in \EF$ by rules for
  constant functions, the identity function, sum, and composition with
  sine.

This proves that $P(f)$ holds for each function $f$ in the Base cases.

\inductioncase{Constructor Cases}: We may assume by induction that $f,
g \in \EF$ and $P(f)$ and $P(g)$ holds, that is, $df/dx, dg/dx \in
\EF$.

\inductioncase{Constructor Case}: ($\inv{f}$). we need to prove that
$d\, \inv{f}(x)/dx \in \EF$.  Letting $y = f(x)$, so $x=\inv{f}(y)$,
we know from Leibniz's rule in calculus that
\begin{equation}\label{Leib}
d\inv{f}(y)/dy = dx/dy = \frac{1}{dy/dx}.
\end{equation}
For example,
\[
d \inv{\sin}(y)/dy = 1/(d \sin(x)/dx) = 1/\cos(x) = 1/\cos(\inv{\sin}(y)).
\]
Stated as in~\eqref{Leib}, this rule is easy to remember, but can easily
be misleading because of the variable switching between $x$ and $y$.  It's
more clearly stated using variable-free notation:
\begin{equation}\label{Leib2}
(\inv{f})' = \frac{1}{f'\compose \inv{f}}.
\end{equation}
Now, since $f' \in \EF$ by hypothesis, so is $\inv{f}$ (by constructor
rule~\ref{inversefunc}), and therefore so is their composition (by
rule~\ref{cmp}), as well as the composition of $1/x$ with $f'\compose
\inv{f}$ (by part~\eqref{1over}).  Hence the right-hand side of
equation~(\ref{Leib2}) defines a function in $\EF$.

\inductioncase{Constructor Case}: ($f \compose g$).
The Chain Rule states that
\[
\frac{d(f(g(x)))}{dx} = \frac{d f(g)}{dg}\cdot\frac{dg}{dx}.
\]
Stated more clearly in variable-free notation, this is
\[
(f \compose g)' = (f' \compose g) \cdot g'.
\]
The right-hand side of this equation defines a function in $\EF$ by
constructor rules~\ref{cmp}.\ and~\ref{+-}.

\inductioncase{Constructor Case}: ($f\cdot g$).
The product rule for derivatives gives
\[
(f\cdot g)' = f' \cdot g+ f\cdot g'.
\]
Since products and sums of functions in $\EF$ are also in $\EF$, we
conclude that the right hand side of the equation describes aa
function in $\EF$.

The remaining cases for Constructor rule~\ref{+-}.\ are
straighforward.  So we conclude that the induction hypothesis holds in
all Constructor cases.

This completes the proof by structural induction that the $P(f)$ holds
for all $f \in \EF$.
\end{proof}

\end{solution}
\eparts
\end{problem}


%%%%%%%%%%%%%%%%%%%%%%%%%%%%%%%%%%%%%%%%%%%%%%%%%%%%%%%%%%%%%%%%%%%%%
% Problem ends here
%%%%%%%%%%%%%%%%%%%%%%%%%%%%%%%%%%%%%%%%%%%%%%%%%%%%%%%%%%%%%%%%%%%%%

\endinput

