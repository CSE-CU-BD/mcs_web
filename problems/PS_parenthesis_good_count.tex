%PS_parenthesis_good_count
%subsumes PS_Structural_Induction_LISP

\documentclass[problem]{mcs}

\begin{pcomments}
  \pcomment{from: F04.ps5, S02.ps5}
  \pcomment{related to PS_parenthesis_count_well}
\end{pcomments}

\pkeywords{
  induction
  matching_parentheses
  recursive_data
  strings
  structural_induction  
  count
}

%%%%%%%%%%%%%%%%%%%%%%%%%%%%%%%%%%%%%%%%%%%%%%%%%%%%%%%%%%%%%%%%%%%%%
% Problem starts here
%%%%%%%%%%%%%%%%%%%%%%%%%%%%%%%%%%%%%%%%%%%%%%%%%%%%%%%%%%%%%%%%%%%%%
\begin{problem}
\iffalse
Suppose we start with a sequence of fully parenthesized arithmetic or
Scheme expressions, and then erase all the alphanumeric characters other
than parentheses.  The result is a string of parentheses which ``match
up''.

For example, erasing non-parenthesis characters from the expression
\texttt{(if (or x y) u v)} leaves the string \texttt{(())}.  Similarly,
erasing non-parentheses from the sequence of three arithmetic expressions
\[
\texttt{(x * x), (y + (3 * (z - 9))), ((x + y) * (u + w))}
\]
leaves the string
\[
\texttt{()((()))(()())}.
\]
On the other hand, even though the string \texttt{())(()} has an equal
number of left and right parentheses, it cannot be obtained by erasing in
this way---because the two middle parentheses don't match up.  Also, no
string of parentheses which begins with a right parenthesis or ends with a
left parenthesis can match up.
\fi

The set, $\RM$, of strings of matching parentheses, is defined recursively
as follows:

\begin{itemize}
\item The empty string, $\emptystring$, (consisting of no parentheses) is
  in $\RM$.

\item If $s$ and $t$ are in $\RM$, then the string $\texttt{(}s\texttt{)}t$
is in $\RM$.  That is, the new string starts with a left parenthesis which
is followed by the parenthesis characters in the string $s$, followed by a
right parenthesis, followed by the parenthesis characters in the string
$t$.

\end{itemize}

There is a simple test to determine whether a string of parentheses is in
$\RM$: starting with zero, read the string from left to right adding one
for each left parenthesis and -1 for each right parenthesis.  A string has
a \term{good count} when the count never goes negative and is back to
zero by the end of the string.  Let $\GC$ be the parenthesis strings with
good counts.

\begin{problemparts}
\problempart Prove that $\GC$ contains $\RM$ by structural induction on the
definition of $\RM$.

\begin{solution}
We prove by induction on the definition of $\RM$ (that is,
structural induction) that every element of $\RM$ counts well, so $\RM$ is
contained in $\GC$.  The induction hypothesis is
\[
P(s) \eqdef s \in \GC.
\]
\begin{proof}

{\bf Base Case}: $P(\emptystring)$ holds since the count of the
empty string ends when it starts at zero.

{\bf Inductive Step:} Assume $P(s)$ and $P(t)$ are true.  We need to show
that $P(\texttt{(}s\texttt{)}t)$ is true.

The count values for $\texttt{(}s\texttt{)}t$ start with 0.  Reading the
initial left parenthesis yields 1 as the next count value.  This 1 serves
as the start of a series of count values exactly equal to the count values
of $s$, with each value incremented by one.  Since $s \in \GC$ by
hypothesis, these incremented count values begin with 1, always stay
positive, and end with 1.  The right parenthesis immediately after $s$
reduces the ending count to 0.  This 0 serves as the start of the
remaining count values which are exactly the count values of $t$.  Since
$t \in \GC$, these remaining values never go negative and end at 0.
Hence the entire sequence of count values for $\texttt{(}s\texttt{)}t$
starts with 0, never goes negative, and ends with 0, which proves that
$\texttt{(}s\texttt{)}t \in \GC$.

\end{proof}

\end{solution}

\problempart Conversely, prove that $\RM$ contains $\GC$.

\begin{solution}
  We show that every string $r\in \GC$ is an $\RM$ by strong induction on
  the length of $r$.  The induction hypothesis is
\[
Q(n) \eqdef \forall r \in \GC.\, \lnth{r} \leq n \QIMPLIES r \in \RM.
\]
\begin{proof}

{\bf Base Case} $n = 0$: In this case there is only one string of length
$n$, namely the empty string, which is in $\RM$ by definition, proving
$Q(0)$.

{\bf Inductive Step:} Assume that $Q(k)$ is true for all $k\leq n$, we
need to prove that $Q(n+1)$ is also true.

So suppose $r$ is a length $n+1$ string that counts well.  We must prove
that $r \in \RM$.

Now since $r$ has a good count, it must start with a left parenthesis (or else
the count would immediately go negative).  Likewise, since the count for
$r$ returns to the value 0 by the end, $r$ must end with right
parenthesis.  So there must be a \emph{first} right parenthesis in $r$
after which the count returns to 0.  Let $s$ be the substring of $r$
between the initial left parenthesis and this right parenthesis.  So
\[
r = \texttt{(}s\texttt{)}t
\]
for some string $t$.

Since counts only change by one as each parenthesis character is read, and
the count for $r$ \emph{first} returns to 0 after the right parenthesis
following $s$, the count during $s$ must start and end with 1 and must
stay \emph{positive} in between.  But this implies that a count for $s$
alone, which would start with 0, would also end with 0 and stay
\emph{nonnegative} in between.  That is, $s$ by itself has a good count.
Since the length of $s \in \GC$ is less than the length of $r$, we
have by strong induction that $s \in \RM$.

Further, we know the count for $r$ returns to 0 after the right
parenthesis following $s$, and since $r \in \GC$, the count ends with 0
again and stays nonnegative in between.  But this implies that $t$ has a
good count, and since the length of $t$ is less than the length of $r$, we
have by strong induction that $t \in \RM$.  Now by the second case in the
definition of $\RM$, we conclude $r = \texttt{(}s\texttt{)}t \in \RM$.

\end{proof}

\end{solution}

\end{problemparts}
\end{problem}

%%%%%%%%%%%%%%%%%%%%%%%%%%%%%%%%%%%%%%%%%%%%%%%%%%%%%%%%%%%%%%%%%%%%%
% Problem ends here
%%%%%%%%%%%%%%%%%%%%%%%%%%%%%%%%%%%%%%%%%%%%%%%%%%%%%%%%%%%%%%%%%%%%%

\endinput
