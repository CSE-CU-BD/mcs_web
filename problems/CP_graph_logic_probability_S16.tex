\documentclass[problem]{mcs}

\begin{pcomments}
  \pcomment{CP_graph_logic_probability_S16}
  \pcomment{subsumes CP_graph_logic_probability, FP_graph_logic_probability}
  \pcomment{ARM 5/10/16}
  \pcomment{soln to last part by ARM 11/8/17}
\end{pcomments}

\pkeywords{
  graph
  logic
  probability
  independence
}

%%%%%%%%%%%%%%%%%%%%%%%%%%%%%%%%%%%%%%%%%%%%%%%%%%%%%%%%%%%%%%%%%%%%%
% Problem starts here
%%%%%%%%%%%%%%%%%%%%%%%%%%%%%%%%%%%%%%%%%%%%%%%%%%%%%%%%%%%%%%%%%%%%%

\begin{problem}
Let $G$ be a simple graph with $n$ vertices.  Let ``$A(u,v)$'' mean
that vertices $u$ and $v$ are adjacent, and let ``$W(u,v)$'' mean that
there is a length-two walk between $u$ and $v$.

\bparts

\ppart Explain why $W(u,u)$ holds iff $\exists v.\ A(u,v)$.

\examspace[0.5in]

\begin{solution}
A length two path from $u$ to $u$ must be a path that goes back and
forth on some edge incident to~$u$.
\end{solution}

\ppart Write a predicate-logic formula defining $W(u,v)$ in terms of
the predicate $A(.,.)$ when $u \neq v$.

\examspace[0.6in]

\begin{solution}
\[
W(u,v) \eqdef \exists t.\, A(u,t) \QAND\ A(t,v).
\]

This formula actually works even if $u=v$.
\end{solution}

\eparts

There are $e \eqdef \binom{n}{2}$ possible edges between the $n$
vertices of $G$.  Suppose the actual edges of $\edges{G}$ are chosen
with randomly from this set of $e$ possible edges.  Each edge is
chosen with probability $p$, and the choices are mutually independent.

\bparts

\ppart Write a simple formula in terms of $p, e$ and $k$ for
$\pr{\card{\edges{G}} = k}$.

\exambox{0.8in}{0.5in}{0in}

\examspace[0.3in]

\begin{solution}
\[
\binom{e}{k} p^k(1-p)^{e-k}.
\]

The number of chosen edges has a binomial distribution.
\end{solution}

\ppart Write a simple formula in terms of $p$ and $n$ for $\pr{W(u,u)}$.

\exambox{0.6in}{0.5in}{0in}


\examspace[0.3in]

\begin{solution}
\[
1- (1-p)^{n-1}.
\]

$\QNOT(W(u,u))$ iff none of the possible $n-1$ incident edges are
edges of $G$.  Since edges of $G$ are chosen independently and the
probability is $1-p$ that any given edge is in $\edges{G}$,
\[
\pr{\QNOT(W(u,u))} = (1-p)^{n-1}.
\]
\end{solution}

\eparts

Let $w$, $x$, $y$ and $z$ be four distinct vertices.

Because edges are chosen mutually independently, if the edges that one
event depends on are disjoint from the edges that another event
depends on, then the two events will be mutually independent.  For
example, the events
\[
A(w,y) \QAND A(y,x)
\]
and
\[
A(w,z) \QAND A(z,x)
\]
are independent since the first event dependes on $\set{\edge{w}{y},
  \edge{y}{x}}$, while the second event depends on $\set{\edge{w}{z},
  \edge{z}{x}}$.

%\examspace

\bparts

\ppart  Let
\begin{equation}\label{nottwopath}
r \eqdef \pr{\QNOT(W(w,x))},
\end{equation}
where $w$ and $x$ are distinct vertices.  Write a simple formula for
$r$ in terms of $n$ and $p$.

\hint Different length-two paths between $x$ and $y$ don't share any
edges.

\exambox{1.5in}{0.5in}{0in}

\examspace[0.5in]

\begin{solution}
\[
r = (1-p^2)^{n-2}.
\]

Let $Z \eqdef \vertices{G} - \set{w,x}$ be the set of $n-2$ vertices
other than $w$ and $x$.
\begin{align*}
\lefteqn{\pr{\QNOT(W(w,x))}}\\
  & = \pr{\forall z \in Z.\, \QNOT(A(w,z) \QAND\ A(z,x))}\\
  & = \prod_{z \in Z} \pr{\QNOT(A(w,z) \QAND\ A(z,x))}
          & \text{(length-two paths are independent)}\\
  & = \prod_{z \in Z} (1 - \pr{A(w,z) \QAND\ A(z,x)})\\
  & = \prod_{z \in Z} (1-\pr{A(w,z)}\cdot \pr{A(z,x)})
          & \text{(edges are independent)}\\
  & = \prod_{z \in Z} (1-p^2) \\
  & = (1-p^2)^{n-2}.
\end{align*}
\end{solution}

\ppart \label{3cycle} Vertices $x$ and $y$ being on a three-cycle can
be expressed simply as
\[
A(x,y) \QAND\ W(x,y).
\]
Write a simple expression in terms of $p$ and $r$ for the probability
that $x$ and $y$ lie on a three-cycle in $G$.

\exambox{0.6in}{0.5in}{0in}

\examspace[0.5in]

\begin{solution}
\[
p(1-r).
\]

$A(x,y)$ and $W(x,y)$ are independent since $W(x,y)$ does not depend
on whether $\edge{x}{y}$ is an actual edge of $G$.  Therefore,
\begin{align*} 
\pr{A(x,y) \QAND\ W(x,y)}
   & = \pr{A(x,y)} \cdot \pr{W(x,y)} \\
   & = p (1-r) .
\end{align*}
\end{solution}

\ppart\label{Wwxyz} Show that $W(w,x)$ and $W(y,z)$ may not be independent events.
\hint Just consider the case that $\vertices{G} = \set{w,x,y,z}$ and $p=1/2$.

\examspace[0.75in]

\begin{solution}
Following the hint, assume $\vertices{G} = \set{w,x,y,z}$.  Now we
observe that $W(w,x)$ depends only on the four edges $\edge{w}{y},
\edge{w}{z}, \edge{x}{y}, \edge{x}{z}$:
\[
W(w,x)\quad \QIFF\quad (A(w,y) \QAND A(x,y)) \QOR  (A(w,z) \QAND A(x,z))
\]
so by Inclusion-Exclusion,
\begin{align*}
\lefteqn{\pr{W(w,x)}}\\
& = \pr{A(w,y) \QAND A(x,y)} + \pr{A(w,z) \QAND A(x,z)}\\
& \qquad   - \pr{A(w,y) \QAND A(x,y) \QAND A(w,z) \QAND A(x,z)}\\
& = \paren{\frac{1}{2}}^2 + \paren{\frac{1}{2}}^2 - \paren{\frac{1}{2}}^4\\
& = \frac{7}{16}.
\end{align*}
By symmetry, likewise
\[
\pr{W(y,z)} = \frac{7}{16}.
\]

However,
\begin{align*}
\lefteqn{[W(w,x)  \QAND W(y,z)]}\\
   & \QIFF\ \text{at most one of the edges}\ \edge{w}{y}, \edge{w}{z}, \edge{x}{y}, \edge{x}{z}\ \text{is missing from}\ \vertices{G}.
\end{align*}
Now the probabilility that some particular edge are among these four
is the only one missing is $(1/2)^4$ and the probability that none of
missing is also equal to $(1/2)^4$, so by Inclusion-Exclusion,
\[
\pr{W(w,x) \QAND W(y,z)} = \binom{4}{3}\paren{\frac{1}{2}}^4 - \paren{\frac{1}{2}}^4 = \frac{5}{16}.
\]
In particular,
\[
\pr{W(w,x) \QAND W(y,z)} = \frac{5}{16} \neq \paren{\frac{7}{16}}^2 =  \pr{W(w,x)}\cdot \pr{W(y,z)},
\]
proving that $W(w,x)$ and $W(y,z)$ are not independent.
\end{solution}

\begin{staffnotes}
Attempted general solution by Sibo 11/5/17:

\begin{align*}
\lefteqn{\prcond{\QNOT(W(w,x))}{W(y,z)}} \\
	&= \prcond{\QNOT(W(w,x))}{\exists u \in \vertices{G} - \set{y,z}.\, A(y,u) \QAND\ A(u,z)}\\
	&= \pr{u \in \vertices{G}-\set{y,z,w,x}} \cdot \prcond{\QNOT(W(w,x))}{A(y,u) \QAND\ A(u,z)} \\
	&\quad + 
           \pr{u \in \set{w,x}} \cdot \prcond{\QNOT(W(w,x))}{A(y,u) \QAND\ A(u,z)}\\
	&= \frac{n-4}{n-2} \cdot (1-p^2)^{n-2} + \frac{2}{n-2} \cdot (1-p^2)^{n-4} \cdot (1-p)^2\\
	& \leq r.
\end{align*}

\end{staffnotes}

\eparts
\end{problem}

%%%%%%%%%%%%%%%%%%%%%%%%%%%%%%%%%%%%%%%%%%%%%%%%%%%%%%%%%%%%%%%%%%%%%
% Problem ends here
%%%%%%%%%%%%%%%%%%%%%%%%%%%%%%%%%%%%%%%%%%%%%%%%%%%%%%%%%%%%%%%%%%%%%

\endinput


% Exam version commented out for use as a class problem.
\iffalse
\ppart \inbook{Indicate}\inhandout{Circle} the mathematical formula that best expresses the
definition of $W(x,y)$.

\begin{itemize}
\item $W(x,y) \eqdef \exists z.\ A(x,z) \QAND\ A(y,z)$
\examspace[0.15in]
\item $W(x,y) \eqdef x \neq y \QAND\ \exists z.\ A(x,z) \QAND\ A(y,z)$
\examspace[0.15in]
\item $W(x,y) \eqdef \forall z.\ A(x,z) \QOR\ A(y,z)$
\examspace[0.15in]
\item $W(x,y) \eqdef \forall z.\ x \neq y \QIMPLIES\ [A(x,z) \QOR\ A(y,z)]$
\end{itemize}
\fi


\begin{staffnotes}
The more randomly chosen edges there are in $G$, the more likely
length two walks become.  Now $W(w,x)$ implies the existence of two
edges---which might be incident to $y$ or $z$---and this makes
$W(y,z)$ more likely given $W(w,x)$.

This intuitive argument implicitly relies on assumptions about
independent choice of the ``more'' edges and is accordingly not really
convincing.
\end{staffnotes}
