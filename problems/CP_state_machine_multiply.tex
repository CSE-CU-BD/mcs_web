\documentclass[problem]{mcs}

\begin{pcomments}
  \pcomment{CP_state_machine_multiply}
  \pcomment{S07.miniquiz-3-07}
  \pcomment{CP version of MQ_state_machine_multiply}
  \pcomment{adapted by ARM 2/21/11}
\end{pcomments}

\pkeywords{
  state_machines
  termination
  partial_correctness
  invariant
  preserved_invariant
  algorithm
  multiply
}

%%%%%%%%%%%%%%%%%%%%%%%%%%%%%%%%%%%%%%%%%%%%%%%%%%%%%%%%%%%%%%%%%%%%%
% Problem starts here
%%%%%%%%%%%%%%%%%%%%%%%%%%%%%%%%%%%%%%%%%%%%%%%%%%%%%%%%%%%%%%%%%%%%%

\begin{problem}
%Tests state machine understanding, invariant method, termination arguments.

Multiplying and dividing an integer $n$ by 2 only requires a one digit
left or right shift of the binary representation of $n$, which are
hardware-supported fast operations on most computers.  Here is a state
machine $R$ that computes the product of two nonnegative integers
$x$ and $y$ using just these shift operations, along with integer
addition:

\begin{align*}
\text{states} & \eqdef \naturals^3  \qquad\text{(triples of nonnegative integers)}\\
\text{start state} & \eqdef (x,y,0)\\
\text{transitions} & \eqdef
(r,s,a)\movesto
        \begin{cases}
        (2r,s/2,a) &  \text{for even $s>0$},\\
        (2r,(s-1)/2,a+r) &\text{for odd $s>0$}.
       \end{cases}
\end{align*}

\bparts

\ppart\label{rsa} Verify that
\begin{equation}\label{inv}
P((r,s,a)) \eqdef\quad [rs+a = xy]
\end{equation}
is a preserved invariant of $R$.  %How about $Q((r,s,a)) \eqdef \quad [r=r+1]$? :-)

\begin{solution}
%Q is a trivial preserved invariant since it is always false.  

To prove that $P$ is a preserved invariant, assume that $P((r,s,a))$
and $(r,s,a) \movesto (r',s',a')$.  We must prove that $P((r',s',a'))$
holds, that is
\begin{equation}\label{inv'}
r's'+a' = xy.
\end{equation}

There are two cases corresponding to the transition cases:

If $s>0$ is even, then we have that $r' = 2r, s' = s/2, a'=a$.
Therefore,
\begin{align*}
  r's' + a' = & 2r \cdot \frac{s}{2} + a\\
            = & rs+a\\
            = & xy & \text{(by~\eqref{inv})}.
\end{align*}

If $s>0$ is odd, we have $r' = 2r, s' = (s-1)/2, a = a+r$.  So:
\begin{align*}
  r's' + a'  = & 2r \cdot \frac{s-1}{2} + a+r\\
   = & r\cdot(s-1) + a + r\\
   = & rs+a\\
   = & xy & \text{(by~\eqref{inv})}.
\end{align*}
So in both cases,~\eqref{inv'} holds, proving that $P$ is indeed an
invariant.
\end{solution}

\ppart Prove that $R$ is partially correct: if $R$ reachs a final
state---a state from which no transition is possible---then $a =
xy$.

\begin{solution}
Clearly, $P$ holds for the start state because
\[
P((x,y,0)) \QIFF\ [xy+0 = xy].
\]
The final states are those of the form $(r,0,a)$.  By the Invariant
Principle, if $(r,0,a)$ is reachable, then $P((r,0,a))$ holds, that
is,
\[
a = r\cdot 0 + a = xy.
\]
\end{solution}

\examspace

\ppart Briefly explain why this state machine will terminate after a
number of transitions bounded by a small constant times the
\emph{length} of the binary representation of $y$.

%for all $x,y \in \naturals$.

\begin{solution}
  We claim that the termination condition, $s=0$, will occur after at
  most $1+\log_2 y$ transitions:  Each transition reduces the
  value of $s$ to $\le s/2$.  Hence, after at most $1+ \log_2 y$
  transitions, the final value of $s$ is at most $1/2^{1+ \log_2 y} =
  1/2y$ times its initial value $y$.  This means the value of $s$
  will be less than 1 and so must be 0 at this point if it wasn't 0
  earlier.
\end{solution}

\eparts
\end{problem}

\endinput
