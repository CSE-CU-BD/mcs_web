\documentclass[problem]{mcs}

\begin{pcomments}
  \pcomment{TP_grading_final_exam} 
  \pcomment{formerly PS_grading_final_exam} 
  \pcomment{from S06, pset10}
\end{pcomments}

\pkeywords{
  probability
  random_variable
  expectation
}

%%%%%%%%%%%%%%%%%%%%%%%%%%%%%%%%%%%%%%%%%%%%%%%%%%%%%%%%%%%%%%%%%%%%%
% Problem starts here
%%%%%%%%%%%%%%%%%%%%%%%%%%%%%%%%%%%%%%%%%%%%%%%%%%%%%%%%%%%%%%%%%%%%%
\begin{problem}
Each 6.042 final exam will be graded according to a rigorous procedure:
\begin{itemize}

\item With probability $\frac{4}{7}$ the exam is graded by a {\em TA},with
probability $\frac{2}{7}$ it is graded by a {\em lecturer}, and with
probability $\frac{1}{7}$, it is accidentally dropped behind the radiator
and arbitrarily given a score of 84.

\item {\em TAs} score an exam by scoring each problem individually and
then taking the sum.

\begin{itemize}
\item There are ten true/false questions worth 2 points each.  For each,
full credit is given with probability $\frac{3}{4}$, and no credit is
given with probability $\frac{1}{4}$.

\item There are four questions worth 15 points each.  For each, the score
is determined by rolling two fair dice, summing the results, and adding 3.

\item The single 20 point question is awarded either 12 or 18 points with
equal probability.
\end{itemize}

\item {\em Lecturers} score an exam by rolling a fair die twice,
multiplying the results, and then adding a ``general impression''score.

\begin{itemize}
\item With probability $\frac{4}{10}$, the general impression score is 40.
\item With probability $\frac{3}{10}$, the general impression score is 50.
\item With probability $\frac{3}{10}$, the general impression score is 60.
\end{itemize}
\end{itemize}
Assume all random choices during the grading process are independent.

\bparts

\ppart What is the expected score on an exam graded by a TA?

\begin{solution}
Let the random variable $T$ denote the score a TA would give.  By
linearity of expectation, the expected sum of the problem scores is the sum
of the expected problem scores.  Therefore, we have:
\begin{eqnarray*}
\expect{T}
& = & 10 \cdot \expect{\text{T/F score}} +  4 \cdot
         \expect{\text{15pt prob score}} +  \expect{\text{20pt prob score}} \\
& = & 10 \cdot \left(\frac{3}{4} \cdot 2 + \frac{1}{4} \cdot 0 \right) + 4
         \cdot \left( 2 \cdot \frac{7}{2} + 3 \right) + \left(\frac{1}{2}
         \cdot 12 + \frac{1}{2} \cdot 18 \right) \\
& = & 10 \cdot \frac{3}{2} + 4 \cdot 10 + 15 \\
& = &  70
\end{eqnarray*}
\end{solution}

\ppart What is the expected score on an exam graded by a lecturer?

\begin{solution}
Now we find the expected value of $L$, the score a lecturer would give.
Employing linearity again, we have:
\begin{eqnarray*}
\expect{L}
& = & \expect{\text{product of dice}} +
        \expect{\text{general impression}} \\
& = & \left(\frac{7}{2}\right)^2 + \left(  \frac{4}{10} \cdot 40 +
        \frac{3}{10} \cdot 50 +   \frac{3}{10} \cdot 60 \right) \\
& = & \frac{49}{4} + 49 \\
& = & 61 \frac{1}{4}
\end{eqnarray*}
\end{solution}

\ppart What is the expected score on a 6.042 final exam?

\begin{solution}
Let $X$ equal the true exam score.  The total expectation law implies:

\begin{eqnarray*}
\expect{X}
& = & \frac{4}{7} \cdot \expect{T} +  \frac{2}{7} \cdot 
        \expect{L} +  \frac{1}{7} \cdot 84 \\
& = & \frac{4}{7} \cdot 70 +  \frac{2}{7} \cdot \paren{\frac{49}{4} +
        49} +  \frac{1}{7} \cdot 84\\
& = & 40 + \frac{7}{2} + 14 + 12 \\
& = & 69 \frac{1}{2}
\end{eqnarray*}
\end{solution}

\eparts 
\end{problem}

%%%%%%%%%%%%%%%%%%%%%%%%%%%%%%%%%%%%%%%%%%%%%%%%%%%%%%%%%%%%%%%%%%%%%
% Problem ends here
%%%%%%%%%%%%%%%%%%%%%%%%%%%%%%%%%%%%%%%%%%%%%%%%%%%%%%%%%%%%%%%%%%%%%

\endinput
