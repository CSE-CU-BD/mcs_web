\documentclass[problem]{mcs}

\begin{pcomments}
    \pcomment{TP_Basic_Propositions}
    \pcomment{Converted from prob1.scm by scmtotex and dmj
              on Sat 12 Jun 2010 09:47:19 PM EDT}
    \pcomment{edited by drewe 11.7.2011}
\end{pcomments}

\begin{problem}

%% type: short-answer
%% title: Basic Propositions

%%   \textbf{A word on notation}:  In all of the HTML pages 
%%  (problems  and exercises) we
%%  are using the following notation for propositional logic:
%%  
%%  \begin{description}
%%  
%%  
%%  
%%  \item
%%   <tt>^</tt> (caret) for and,
%%  
%%  \item
%%   <tt>v</tt> (\textbf{lower case} v) for or,
%%  
%%  \item
%%   <tt>~</tt> for not,
%%  
%%  \item
%%   <tt>-></tt> (hyphen,  > ) for implication, and
%%  
%%  \item
%%   <tt><-></tt> (&lt;, hyphen,  > ) for biconditionals.
%%  
%%  \item
%%   Propositional variables are upper case letters (and
%%  possibly numbers).
%%  
%%  \item
%%   The symbols <tt>true</tt> and <tt>false</tt> are written out in  lower case.
%%  
%%  \end{description}

%%  When you are asked to enter an expression in propositional logic syntax,
%%  you should use these conventions as well.<hr>

Suppose you are taking a class, and that class has a textbook and a final exam.  Let the propositional variables $P$, $Q$, and $R$ have the following
meanings:
\begin{description}

\item
$P = \text{You get an A on the final exam.}$

\item
$Q = \text{You do every exercise in the book.}$

\item
$R = \text{You get an A in the class.}$

\end{description}

Write the following propositions using $P$, $Q$, and $R$ and logical
connectives.

\bparts

\ppart
You get an A in the class, but you do not do every 
exercise in the book.

\begin{solution}
$R \QAND \QNOT(Q)$
\end{solution}

\ppart
You get an A on the final, you do every exercise in the
book, and you get an A in the class.

\begin{solution}
$P \QAND Q \QAND R$
\end{solution}

\ppart
To get an A in the class, it is necessary for you to get
an A on the final.

\begin{solution}
$R \QIMPLIES P$
\end{solution}

\ppart
You get an A on the final, but you don't do every exercise
in this book; nevertheless, you get an A in this class.

\begin{solution}
$P \QAND \QNOT(Q) \QAND R$
\end{solution}

\eparts

\end{problem}

\endinput
