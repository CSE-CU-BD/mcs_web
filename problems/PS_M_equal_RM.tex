\documentclass[problem]{mcs}

\begin{pcomments}
  \pcomment{PS_M_equal_RM}
  \pcomment{by ARM 2/23/11}
\end{pcomments}

\pkeywords{
  string
  matched
  bracket
  structural_induction
  induction
  concatenation
  ambiguous
}

%%%%%%%%%%%%%%%%%%%%%%%%%%%%%%%%%%%%%%%%%%%%%%%%%%%%%%%%%%%%%%%%%%%%%
% Problem starts here
%%%%%%%%%%%%%%%%%%%%%%%%%%%%%%%%%%%%%%%%%%%%%%%%%%%%%%%%%%%%%%%%%%%%%

\begin{problem}
The set of strings, $\RM$, is recursively defined as follows:
\begin{itemize}

\item \textbf{Base case:} $\emptystring \in\RM$.

\item \textbf{Constructor case:} If $s,t \in\RM$, then
\[
\lefbrk s\, \rhtbrk t \in \RM.
\]
\end{itemize}

The set of strings, $M$, is recursively defined as follows:
\begin{itemize}

\item \textbf{Base case:} $\emptystring \in M$,

\item \textbf{Constructor cases:} if $s,t \in M$, then
  the strings $\lefbrk s\, \rhtbrk$ and $s\cdot t$ are also in $M$.
\end{itemize}

Prove by structural induction that

\bparts

\ppart  $M \subseteq \RM$

\begin{solution}

TBA

\end{solution}

\ppart  $\RM \subseteq M$

\begin{solution}

TBA

\end{solution}

\eparts

\end{problem}

%%%%%%%%%%%%%%%%%%%%%%%%%%%%%%%%%%%%%%%%%%%%%%%%%%%%%%%%%%%%%%%%%%%%%
% Problem ends here
%%%%%%%%%%%%%%%%%%%%%%%%%%%%%%%%%%%%%%%%%%%%%%%%%%%%%%%%%%%%%%%%%%%%%

\endinput
