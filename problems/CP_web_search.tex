\documentclass[problem]{mcs}

\begin{pcomments}
  \pcomment{CP_web_search}
  \pcomment{created by adamc for S12:cp3t}
\end{pcomments}

\pkeywords{
  relations
}

%%%%%%%%%%%%%%%%%%%%%%%%%%%%%%%%%%%%%%%%%%%%%%%%%%%%%%%%%%%%%%%%%%%%%
% Problem starts here
%%%%%%%%%%%%%%%%%%%%%%%%%%%%%%%%%%%%%%%%%%%%%%%%%%%%%%%%%%%%%%%%%%%%%

\begin{problem}

  The language of sets and relations may seem remote from the practical world of programming, but in fact there is a close connection to \emph{relational databases}, a very popular software application building block implemented by such software packages as MySQL.  This problem explores the connection by considering how to manipulate and analyze a large data set using operators over sets and relations.  Systems like MySQL are able to execute very similar high-level instructions efficiently on standard computer hardware, which helps programmers focus on high-level design.

  Consider a basic Web search engine, which stores information on Web pages and processes queries to find pages satisfying conditions provided by users.  At a high level, we can formalize the key information as:
  \begin{itemize}
  \item A set $P$ of \emph{pages} that the search engine knows about
  \item A binary relation $L$ (for \emph{link}) over pages, defined such that $p_1 L p_2$ iff page $p_1$ links to $p_2$
  \item A set $E$ of \emph{endorsers}, people who have recorded their opinions about which pages are high-quality
  \item A binary relation $R$ (for \emph{recommends}) between endorsers and pages, such that $e R p$ iff person $e$ has recommended page $p$
  \item A set $W$ of \emph{words} that may appear on pages
  \item A binary relation $M$ (for \emph{mentions}) between pages and words, where $p M w$ iff word $w$ appears on page $p$
  \end{itemize}

  Each part of this problem describes an intuitive, informal query over the data, and your job is to produce a single expression using the standard set and relation operators, such that the expression can be interpreted as answering the query correctly, for any data set.  Your answers should use only the set and relation symbols given above, in addition to terms standing for constant elements of $E$ or $W$, plus the following operators introduced in the text:
  \begin{itemize}
  \item Set union $\cup$
  \item Set intersection $\cap$
  \item Set difference $-$
  \item Relational image (e.g., $R(s)$ for some set $s$ or $R(v)$ for some specific element $v$)
  \item Relational inverse $^{-1}$
  \item And one extra: \emph{relational composition} $\circ$, where we define $x (R_1 \circ R_2) y$ iff there exists $z$ such that $x R_1 z$ and $z R_2 y$ (in other words, connect the arrows of $R_1$ with the arrows of $R_2$)
  \end{itemize}

  \bparts

  \ppart The set of pages containing the word ``logic''

  \begin{solution}
    $$M^{-1}(\textrm{``logic''})$$
  \end{solution}

  \ppart The set of pages containing the word ``logic'' but not the word ``predicate''

  \begin{solution}
    $$M^{-1}(\textrm{``logic''}) - M^{-1}(\textrm{``predicate''})$$
  \end{solution}

  \ppart The set of pages containing the word ``set'' that have been recommended by ``Meyer''

  \begin{solution}
    $$M^{-1}(\textrm{``set''}) \cap R(\textrm{``Meyer''})$$
  \end{solution}

  \ppart The set of endorsers who have recommended pages containing the word ``algebra''

  \begin{solution}
    $$R^{-1}(M^{-1}(\textrm{``algebra''}))$$
  \end{solution}

  \ppart The relation that relates endorser $e$ and word $w$ iff $e$ has recommended a page containing $w$

  \begin{solution}
    $$R \circ M$$
  \end{solution}

  \ppart The set of pages that have at least one incoming or outgoing link

  \begin{solution}
    $$L(P) \cup L^{-1}(P)$$
  \end{solution}

  \ppart The relation that relates word $w$ and page $p$ iff $w$ appears on a page that links to $p$

  \begin{solution}
    $$M^{-1} \circ L$$
  \end{solution}

  \ppart The relation that relates word $w$ and endorser $e$ iff $w$ appears on a page that links to a page that $e$ recommends

  \begin{solution}
    $$M^{-1} \circ L \circ R^{-1}$$
  \end{solution}

  \ppart The relation that relates pages $p_1$ and $p_2$ iff $p_2$ can be reached from $p_1$ by following a sequence of exactly 3 links

  \begin{solution}
    $$L \circ L \circ L$$
  \end{solution}

  \eparts

\end{problem}

%%%%%%%%%%%%%%%%%%%%%%%%%%%%%%%%%%%%%%%%%%%%%%%%%%%%%%%%%%%%%%%%%%%%%
% Problems end here
%%%%%%%%%%%%%%%%%%%%%%%%%%%%%%%%%%%%%%%%%%%%%%%%%%%%%%%%%%%%%%%%%%%%%

\endinput
