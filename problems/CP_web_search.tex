\documentclass[problem]{mcs}

\begin{pcomments}
  \pcomment{CP_web_search}
  \pcomment{created by adamc for S12:cp3t}
\end{pcomments}

\pkeywords{
  relations
  composition
  inverse
}

%%%%%%%%%%%%%%%%%%%%%%%%%%%%%%%%%%%%%%%%%%%%%%%%%%%%%%%%%%%%%%%%%%%%%
% Problem starts here
%%%%%%%%%%%%%%%%%%%%%%%%%%%%%%%%%%%%%%%%%%%%%%%%%%%%%%%%%%%%%%%%%%%%%

\begin{problem}

  The language of sets and relations may seem remote from the
  practical world of programming, but in fact there is a close
  connection to \emph{relational databases}, a very popular
  software application building block implemented by such software
  packages as MySQL.  This problem explores the connection by
  considering how to manipulate and analyze a large data set using
  operators over sets and relations.  Systems like MySQL are able to
  execute very similar high-level instructions efficiently on standard
  computer hardware, which helps programmers focus on high-level
  design.

  Consider a basic Web search engine, which stores information on Web
  pages and processes queries to find pages satisfying conditions
  provided by users.  At a high level, we can formalize the key
  information as:
  \begin{itemize}

  \item A set $P$ of \emph{pages} that the search engine knows about

  \item A binary relation $L$ (for \emph{link}) over pages, defined
    such that $p_1 \mrel{L} p_2$ iff page $p_1$ links to $p_2$

  \item A set $E$ of \emph{endorsers}, people who have recorded their
    opinions about which pages are high-quality

  \item A binary relation $R$ (for \emph{recommends}) between
    endorsers and pages, such that $e \mrel{R} p$ iff person $e$ has
    recommended page $p$

  \item A set $W$ of \emph{words} that may appear on pages

  \item A binary relation $M$ (for \emph{mentions}) between pages and
    words, where $p \mrel{M} w$ iff word $w$ appears on page $p$
  \end{itemize}

  Each part of this problem describes an intuitive, informal query
  over the data, and your job is to produce a single expression using
  the standard set and relation operators, such that the expression
  can be interpreted as answering the query correctly, for any data
  set.  Your answers should use only the set and relation symbols
  given above, in addition to terms standing for constant elements of
  $E$ or $W$, plus the following operators introduced in the text:
  \begin{itemize}
  \item set union $\union$.
  \item set intersection $\intersect$.
  \item set difference, $-$.

  \item relational image---for example, $R(A)$ for some set $A$, or
    $R(a)$ for some specific element $a$.

  \item relational inverse $\inv{}$.

  \item \dots and one extra: \emph{relational composition} which
    generalizes composition of functions
\[
a \mrel{(R \compose S)} c \eqdef\ \exists b \in B.\, (a \mrel{S} b)
\QAND (b \mrel{R} c).
\]
In other words, $a$ is related to $c$ in $R \compose S$ if starting at
$a$ you can follow an $S$ arrow to the start of an $R$ arrow and then
follow the $R$ arrow to get to $c$.\footnote{Note the reversal of $R$
  and $S$ in the definition; this is to make relational composition
  work like function composition.  For functions, $f \compose g$ means
  you apply $g$ first.  That is, if we let $h$ be $f \compose g$, then
  $h(x) = f(g(x))$.}

  \end{itemize}

  Here is one worked example to get you started:

  \begin{itemize}
    \item \textbf{Search description:} The set of pages containing the word ``logic''
    \item \textbf{Solution expression:} $\inv{M}(\textrm{``logic''})$
  \end{itemize}

  Find similar solutions for each of the following searches:

  \bparts

  \ppart The set of pages containing the word ``logic'' but not the word ``predicate''

  \begin{solution}
    $$\inv{M}(\textrm{``logic''}) - \inv{M}(\textrm{``predicate''})$$
  \end{solution}

  \ppart The set of pages containing the word ``set'' that have been recommended by ``Meyer''

  \begin{solution}
    $$\inv{M}(\textrm{``set''}) \intersect R(\textrm{``Meyer''})$$
  \end{solution}

  \ppart The set of endorsers who have recommended pages containing the word ``algebra''

  \begin{solution}
    $$\inv{R}(\inv{M}(\textrm{``algebra''}))$$
Alternatively,
\[
\inv{(M \compose R)}(\textrm{``algebra''})
\]
  \end{solution}

  \ppart The relation that relates endorser $e$ and word $w$ iff $e$ has recommended a page containing $w$

  \begin{solution}
    $$M \compose R$$
  \end{solution}

  \ppart The set of pages that have at least one incoming or outgoing link

  \begin{solution}
    $$L(P) \union \inv{L}(P)$$
  \end{solution}

  \ppart The relation that relates word $w$ and page $p$ iff $w$ appears on a page that links to $p$

  \begin{solution}
    $$L \compose \inv{M}$$
  \end{solution}

  \ppart The relation that relates word $w$ and endorser $e$ iff $w$
  appears on a page that links to a page that $e$ recommends

  \begin{solution}
    $$\inv{R} \compose L \compose \inv{M}$$
  \end{solution}

  \ppart The relation that relates pages $p_1$ and $p_2$ iff $p_2$ can
  be reached from $p_1$ by following a sequence of exactly 3 links

  \begin{solution}
    $$L \compose L \compose L$$
  \end{solution}

  \eparts

\end{problem}

%%%%%%%%%%%%%%%%%%%%%%%%%%%%%%%%%%%%%%%%%%%%%%%%%%%%%%%%%%%%%%%%%%%%%
% Problems end here
%%%%%%%%%%%%%%%%%%%%%%%%%%%%%%%%%%%%%%%%%%%%%%%%%%%%%%%%%%%%%%%%%%%%%

\endinput
