\documentclass[problem]{mcs}

\begin{pcomments}
  \pcomment{PS_relation_transitive_properties}
  \pcomment{from: S04.ps3 (by wing)}
\end{pcomments}

\pkeywords{
  partial_orders
}

%%%%%%%%%%%%%%%%%%%%%%%%%%%%%%%%%%%%%%%%%%%%%%%%%%%%%%%%%%%%%%%%%%%%%
% Problem starts here
%%%%%%%%%%%%%%%%%%%%%%%%%%%%%%%%%%%%%%%%%%%%%%%%%%%%%%%%%%%%%%%%%%%%%

\begin{problem} 
  
  Let $R$ and $S$ be relations on a set, $A$ and consider
  \begin{itemize}

  \item $R^{-1}$

  \begin{solution}
    $R^{-1}$ is transitive. Take arbitrary $a,b,c$ such  that $(a, b), (b,c) \in R^{-1}$. It holds that $(b,a), (c,b) \in R$.
    Because $R$ is transitive, $(c,a) \in R$. By definition of inverse, $(a,c) \in R^{-1}$. Therefore, $R^{-1}$ is transitive. 
  \end{solution}

  \item $R \intersect S$

  \begin{solution}
    Let $R' \eqdef R \intersect S$. $R'$ is transitive.

    Suppose $(a,b), (b,c) \in R \intersect S$. Thus, we have $(a,b), (b,c) \in R \And (a,b), (b,c) \in S$.
    Because $R$ and $S$ are transitive, we have $(a,c) \in R \And (a,c) \in S$.
    Therefore, we have $(a,c) \in R \intersect S$. This satisfies the definition of transitivity.
  \end{solution}
  
  \item $R \union S$ 

  \begin{solution}
    We give a counterexample showing that $R \cup S$ is not transitive.
    Let $R$ and $S$ be relations on the set
    $\set{1, 2, 3}$ where
    \begin{align*}
    R \eqdef & \set{(1,1) (2,2) (3,3) (1,2) (2,1)},\\
    S \eqdef & \set{(1,1) (2,2) (3,3) (2,3) (3,2)}.
    \end{align*}
    It's easy to check that $R$ and $S$ are both transitive.
    But $R\cup S$ is not transitive, because $(1,2),(2,3) \in R\cup S$
    and $(1,3) \notin R\cup S$.  Therefore $R\cup S$ is not transitive.
  \end{solution}

  \item $R \composition S$
  
  \begin{solution}
    We give a counterexample showing that $R \composition S$ is not transitive.
    Let $R$ and $S$ be relations on the set $\set{1,2,3,4,5}$ where
    \begin{align*}
      S \eqdef & \set{(1,4) (2,5)},\\
      R \eqdef & \set{(4,2) (5,3)}.
    \end{align*}

    We see $R$ and $S$ are transitive. Also, $(1,2),(2,3) \in R \composition S$, but $(1,3) \notin R \composition S$. Therefore
    $R \composition S$ is not transitive.
  \end{solution}

  \end{itemize}

  Suppose $R$ and $S$ are transitive.  Indicate which of the relations above
  will \emph{not} be transitive, and give counterexamples for demonstrating this.
  
\end{problem}

%%%%%%%%%%%%%%%%%%%%%%%%%%%%%%%%%%%%%%%%%%%%%%%%%%%%%%%%%%%%%%%%%%%%%
% Problem ends here
%%%%%%%%%%%%%%%%%%%%%%%%%%%%%%%%%%%%%%%%%%%%%%%%%%%%%%%%%%%%%%%%%%%%%
