\documentclass[problem]{mcs}

\begin{pcomments}
  \pcomment{FP_independent_intransitive}
  \pcomment{ARM 11/28/17}
\end{pcomments}

\pkeywords{
  independence
  event
  transitive
  reflexive
}

%%%%%%%%%%%%%%%%%%%%%%%%%%%%%%%%%%%%%%%%%%%%%%%%%%%%%%%%%%%%%%%%%%%%%
% Problem starts here
%%%%%%%%%%%%%%%%%%%%%%%%%%%%%%%%%%%%%%%%%%%%%%%%%%%%%%%%%%%%%%%%%%%%%

\begin{problem}

\bparts

\ppart\label{tstrr}
Show that any total, symmetric, transitive relation is reflexive.

\examspace[1.0in]

\begin{solution}
If $R$ is a total binary relation, then for every $A \in \domain{R}$, there is a
$B \in \domain{R}$ such that $A \mrel{R} B$.  Also, $A \mrel{R} B$ implies $B \mrel{R} A$
when $R$ is symmetric, and then $A \mrel{R} B$ and $B \mrel{R} A$ together imply
$A \mrel{R} A$ when $R$ is transitive.  That is, $R$ is reflexive.
\end{solution}

\ppart
Conclude that there are events $A,B,C$ such that $A$ is independent of $B$, $B$ is
independent of $C$, but $C$ is not independent of $A$.

\examspace[1.0in]

\begin{solution}
Independence is a total relation on the events in a sample space, since every event is
independent of the empty event.  Also, independence is symmetric.  Now if independence was
also transitive, we could conclude from part~\eqref{tstrr} that every event was
independent of itself, a contradiction.
\end{solution}

\eparts

\end{problem}

%%%%%%%%%%%%%%%%%%%%%%%%%%%%%%%%%%%%%%%%%%%%%%%%%%%%%%%%%%%%%%%%%%%%%
% Problem ends here
%%%%%%%%%%%%%%%%%%%%%%%%%%%%%%%%%%%%%%%%%%%%%%%%%%%%%%%%%%%%%%%%%%%%%

\endinput

