\documentclass[problem]{mcs}

\begin{pcomments}
  \pcomment{TP_strictPOs_are_DAGs_modifiedpartb}
  \pcomment{subsumes TP_strictPOs_are_DAGs}
  \pcomment{S17.ps5}
\end{pcomments}

\pkeywords{
  partial_orders
  strict
  DAG
}

%%%%%%%%%%%%%%%%%%%%%%%%%%%%%%%%%%%%%%%%%%%%%%%%%%%%%%%%%%%%%%%%%%%%%
% Problem starts here
%%%%%%%%%%%%%%%%%%%%%%%%%%%%%%%%%%%%%%%%%%%%%%%%%%%%%%%%%%%%%%%%%%%%%

\begin{problem}

\bparts

\ppart Prove that every strict partial order is a DAG.

\begin{solution}
If the strict partial was not a DAG, then it has a vertex $v$ that is
on a cycle.  So there is a positive length walk from $v$ to $v$, which
implies that $v$ is related to itself in the partial order.  This
contradicts irreflexivity, proving that the strict partial order must be a
DAG.
\end{solution}

\ppart What is the smallest possible size of a DAG that is not
transitive?  Prove it.

\begin{solution}

\textbf{3}.

An example would be a graph with a directed edge from $a$ to $b$, a
directed edge from $b$ to $c$, and no other edges.  This graph is not
transitive since there is no edge $\diredge{a}{c}$.

To show that three is the minimum-size example, note that a graph with
no edges between distinct vertices is vacuously transitive, so a
one-vertex DAG, which cannot have any edges, must be transitive.
Also, if a 2-vertex DAG has an edge from one vertex to the other, then
it cannot have an edge back; this means the hypothesis
$\text{If }\edge{a}{b} \QAND \edge{b}{c}\dots$ of the transitivity
condition will be false, so transitivity again hold vacuously in this case.
\end{solution}

\ppart Prove that the positive walk relation of a DAG a strict partial
order.

\begin{solution}
We know that if there is positive length walk from a vertex to itself,
then the vertex is on a cycle\inbook{
  (Lemma~\bref{shortestclosedwalk_lem})}.  So in a DAG, there will be
no positive length walk from a vertex to itself---that is, the
positive-length walk relation is \textbf{irreflexive}.

If there is a positive length walk from $u$ to $v$ and another from
$v$ to $w$, then the merge of the walks is a positive-length walk from
$u$ to $w$---that is, the positive-length walk relation is
\textbf{transitive}.

These two properties imply the positive-length walk relation is a
strict partial order.
\end{solution}

\eparts

\end{problem}


%%%%%%%%%%%%%%%%%%%%%%%%%%%%%%%%%%%%%%%%%%%%%%%%%%%%%%%%%%%%%%%%%%%%%
% Problem ends here
%%%%%%%%%%%%%%%%%%%%%%%%%%%%%%%%%%%%%%%%%%%%%%%%%%%%%%%%%%%%%%%%%%%%%

\endinput
