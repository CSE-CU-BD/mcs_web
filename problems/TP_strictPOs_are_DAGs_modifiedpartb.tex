\documentclass[problem]{mcs}

\begin{pcomments}
  \pcomment{TP_strictPOs_are_DAGs_modifiedpartb}
  \pcomment{was a Quick Question in the partial order chapter}
  \pcomment{minor edit ARM 10/27/15}
  \pcomment{modified part(b) for S17 Pset}
\end{pcomments}

\pkeywords{
  partial_orders
  strict
  DAG
}

%%%%%%%%%%%%%%%%%%%%%%%%%%%%%%%%%%%%%%%%%%%%%%%%%%%%%%%%%%%%%%%%%%%%%
% Problem starts here
%%%%%%%%%%%%%%%%%%%%%%%%%%%%%%%%%%%%%%%%%%%%%%%%%%%%%%%%%%%%%%%%%%%%%

\begin{problem}

\bparts

\ppart Prove that every strict partial order is a DAG.

\begin{solution}
Suppose that a digraph $G$ is a strict partial order, and assume for
the sake of contradiction, that $G$ is not a DAG.  So $G$ must have a
vertex $v$ that is on a cycle.  Since $G$ is transitive, there must be
is a self-loop from $v$ to $v$.  Hence $G$ is not irreflexive,
contradicting the assumption that $G$ is strict.
\end{solution}

\ppart What is the smallest possible size of a DAG that is not
transitive?  Prove it.

\begin{solution}

\textbf{3}.

An example would be a graph with a directed edge from $a$ to $b$, a
directed edge from $b$ to $c$, and no other edges.  This graph is not
transitive since there is no edge $\diredge{a}{c}$.

To show that three is the minimum-size example, note that a graph with
no edges between distinct vertices is vacuously transitive.  Also, if
a 2-vertex DAG has an edge from one vertex to the other, then it
cannot have an edge back.  This leaves four cases according to whether
or not there is a self-loop on each vertex.  In each case,
transitivity follows immediately.
\end{solution}

\ppart Prove that the positive walk relation of a DAG a strict partial
order.

\begin{solution}
We know that if there is positive length walk from a vertex to itself,
then the vertex is on a cycle\inbook{
  (Lemma~\bref{shortestclosedwalk_lem})}.  So in a DAG, there will be
no positive length walk from a vertex to itself---that is, the
positive-length walk relation is \textbf{irreflexive}.  If there is a
positive length walk from $u$ to $v$ and another from $v$ to $w$, then
the merge of the walks is a positive-length walk from $u$ to
$w$---that is, the positive-length walk relation is
\textbf{transitive}.  These two properties imply the positive-length
walk relation is a strict partial order.
\end{solution}

\eparts

\end{problem}


%%%%%%%%%%%%%%%%%%%%%%%%%%%%%%%%%%%%%%%%%%%%%%%%%%%%%%%%%%%%%%%%%%%%%
% Problem ends here
%%%%%%%%%%%%%%%%%%%%%%%%%%%%%%%%%%%%%%%%%%%%%%%%%%%%%%%%%%%%%%%%%%%%%

\endinput
