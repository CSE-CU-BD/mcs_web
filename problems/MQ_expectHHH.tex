\documentclass[problem]{mcs}

\begin{pcomments}
  \pcomment{MQ_expectHHH}
  \pcomment{shorter variation of CP_consecutive_coin_flips}
  \pcomment{ARM 5/5/12}
\end{pcomments}

\pkeywords{
  expectation
  total_expectation
  probability_tree
  tree_model
}

%%%%%%%%%%%%%%%%%%%%%%%%%%%%%%%%%%%%%%%%%%%%%%%%%%%%%%%%%%%%%%%%%%%%%
% Problem starts here
%%%%%%%%%%%%%%%%%%%%%%%%%%%%%%%%%%%%%%%%%%%%%%%%%%%%%%%%%%%%%%%%%%%%%

\begin{problem}

\bparts

\ppart A coin with probability $p$ of flipping Heads and probability
$q \eqdef 1-p$ of flipping tails is repeatedly flipped until three
consecutive Heads occur.  What is the expected number of flips?

\hint The outcome tree, $D$, for this setup can be described as
\[
D \eqdef T\cdot D + H\cdot B
\]
where
\begin{align*}
B \eqdef T\cdot D + H \cdot C\\
C \eqdef T\cdot D + H.
\end{align*}

\begin{solution}
The expected number of flips is 14.

Let $e(T)$ be the expected number of flips starting at the root of
a subtree $T$ of $D$.  By the Total Expectation Rule, we have

\begin{align*}
e(D) & = q(1+ e(D)) + p(1+e(B)) = 1 + qe(D) + pe(B),\\
e(B) & = q(1+ e(D)) + p(1+e(C)) = 1 + qe(D) + pe(C),\\
e(C) & = q(1+ e(D)) + p(1+0)    = 1 + qe(D).
\end{align*}
Substituting for $e(C)$, we get
\[
e(B) = 1 + qe(D) + p(1+ qe(D)) = 1 + p + qe(D)(1+p)
\]
and then substituting this expression for $e(B)$, we get
\begin{align*}
e(D) & = 1 + qe(D) +p(1+p+qe(D)(1+p))\\
     & = 1+p+p^2 + qe(D)(1+p+p^2)\\
     & = (1+p+p^2)(1+qe(D))\\
     & = \frac{1-p^3}{q}(1+qe(D))\\
     & = (1-p^3)\paren{\frac{1}{q}+e(D)}
\end{align*}
so
\[
e(D)p^3  = \frac{1-p^3}{q}
\]
and
\[
e(D) = \frac{1-p^3}{qp^3}.
\]

\end{solution}


\end{problem}


%%%%%%%%%%%%%%%%%%%%%%%%%%%%%%%%%%%%%%%%%%%%%%%%%%%%%%%%%%%%%%%%%%%%%
% Problem ends here
%%%%%%%%%%%%%%%%%%%%%%%%%%%%%%%%%%%%%%%%%%%%%%%%%%%%%%%%%%%%%%%%%%%%%
