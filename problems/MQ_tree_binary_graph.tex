\documentclass[problem]{mcs}

\begin{pcomments}
  \pcomment{MQ_tree_binary_graph}
  \pcomment{by Giu 3/16/13}
  \pcomment{partial revision by ARM 3/16/13.  Whole approach to tree
    coloring via bipartite seems doubtful: why is it OK to appeal to
    the lemma that bipartite implies 2-colorable but not to appeal
    directly to the fact from lecture that trees are 2-colorable?...or
    the book theorem that no odd-length cycles implies 2-colorable?
    Also, bipartite is harder to prove than 2-coloring directly.  And
    technically, bipartite graphs come with a vertex partition, so you
    have to rephrase awkwardly as `The edge set of every tree equals
    the edge set of a bipartite graph.'}
  \pcomment{Revision by zabel 4/9/18. Removed ``bipartite'' language in favor of ``2-colorable.'' Also fixed bug with 1-vertex trees.}
\end{pcomments}

\pkeywords{
  tree
  coloring
  bipartite
}

%%%%%%%%%%%%%%%%%%%%%%%%%%%%%%%%%%%%%%%%%%%%%%%%%%%%%%%%%%%%%%%%%%%%%
% Problem starts here
%%%%%%%%%%%%%%%%%%%%%%%%%%%%%%%%%%%%%%%%%%%%%%%%%%%%%%%%%%%%%%%%%%%%%

\begin{problem}
\bparts

\ppart\label{Bipartite} Prove by induction that every tree is $2$-colorable. Be sure to carefully label your induction hypothesis.

\examspace[5in]

\begin{solution}
  The induction hypothesis $P(n)$ is the predicate ``every tree with $n$ vertices is $2$-colorable.''

\inductioncase{Base case} ($n=1$): A $1$-vertex tree is trivially $2$-colorable. We don't even need both colors!

\inductioncase{Induction step}:  Suppose $n \ge 1$, and assume that all $n$-vertex trees are $2$-colorable. Let $T$ be an $(n+1)$-vertex tree; we must show $T$ is $2$-colorable.

Because $T$ has at least two vertices, $T$ has at least one leaf vertex, $v$. Let the edge $v$ connects to in $T$ be $e = \edge{v}{w}$. Removing $v$ and $e$ from $T$ results in a new tree $T'$ with only $n$ vertices, which by the induction hypothesis can be colored with two colors. Assign these same colors to the vertices of $T$, and additionally, whatever color $w$ was assigned, choose the \emph{other} color for vertex $v$. This is a valid $2$-coloring of $T$: all edges other than $e$ are colored the same as in a valid coloring of $T'$ and therefore have different colors on their endpoints, and vertex $v$ was colored precisely to make the coloring of edge $e$ valid as well.
\end{solution}

\ppart From part~(\ref{Bipartite}) it follows that \emph{almost} all trees $T$ have chromatic number $\chi(T) = 2$. What are the exceptions, and what are their chromatic numbers?

\hint Think small. Recall that a simple graph, by definition, has a nonempty set of vertices.

\begin{solution}

  For every tree $T$, we've shown that $\chi(T) \le 2$. If $T$ has at least one edge, then both colors must be used on the two endpoints of this edge, which proves that $\chi(T) = 2$. So the only trees with $\chi(T)\ne 2$ cannot have any edges. A tree with $n$ vertices (where $n$ must be $\ge 1$ by definition of simple graph) has precisely $n-1$ edges, so the only trees without edges have $n=1$ vertices. These $1$-vertex trees have chromatic number $1$, not $2$.
  
\end{solution}

\eparts

\end{problem}

\endinput
