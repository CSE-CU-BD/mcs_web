\documentclass[problem]{mcs}

\begin{pcomments}
  \pcomment{MQ_tree_plus_edge}
  \pcomment{small part of CP_min_weight_edge}
  \pcomment{by ARM 3/16/13}
\end{pcomments}

\pkeywords{
  spanning_tree
  cycle
  cut_edge
  connected
  acyclic
}

%%%%%%%%%%%%%%%%%%%%%%%%%%%%%%%%%%%%%%%%%%%%%%%%%%%%%%%%%%%%%%%%%%%%%
% Problem starts here
%%%%%%%%%%%%%%%%%%%%%%%%%%%%%%%%%%%%%%%%%%%%%%%%%%%%%%%%%%%%%%%%%%%%%

\begin{problem}

In this problem we will prove that for every tree, $\chi (T) = 2$.
\bparts

\ppart\label{Bipartite} Prove by induction that a tree is bipartite. 

\examspace[3in]

\begin{solution}
$|V(T) |= 1$ is true.

Let $f(n)$ be the statement that a tree with $n$ vertices is bipartite.

Step: Suppose that $f(n)$ is true, let's prove $f(n+1)$ is true. Remove a leaf from the tree with $n+1$ vertices, then by the induction hypothesis the remainingtree is bipartite. When we add the new vertex, it will be connected to only one other vertex $w$, add $v$ to the opposide side of $w$. Thus, the new division will be bipartite.

\end{solution}

\ppart Conclude that $\chi (T) = 2$..

\begin{solution}
Since $T$ is bipartite, therefore $\chi (T) = 2$.
\end{solution}

\eparts

\end{problem}

\endinput
