\documentclass[problem]{mcs}

\begin{pcomments}
  \pcomment{MQ_tree_binary_graph}
  \pcomment{by Giu 3/16/13}
  \pcomment{partial revision by ARM 3/16/13.  Whole approach to tree
    coloring via bipartite seems doubtful: why is it OK to appeal to
    the lemma that bipartite implies 2-colorable but not to appeal
    directly to the fact from lecture that trees are 2-colorable?...or
    the book theorem that no odd-length cycles implies 2-colorable?
    Also, bipartite is harder to prove than 2-coloring directly.  And
    technically, bipartite graphs come with a vertex partition, so you
    have to rephrase awkwardly as `The edge set of every tree equals
    the edge set of a bipartite graph.'}
\end{pcomments}

\pkeywords{
  tree
  coloring
  bipartite
}

%%%%%%%%%%%%%%%%%%%%%%%%%%%%%%%%%%%%%%%%%%%%%%%%%%%%%%%%%%%%%%%%%%%%%
% Problem starts here
%%%%%%%%%%%%%%%%%%%%%%%%%%%%%%%%%%%%%%%%%%%%%%%%%%%%%%%%%%%%%%%%%%%%%

\begin{problem}
In this problem we will prove that for every tree, $\chi (T) = 2$.
\bparts

\ppart\label{Bipartite} Prove by induction that a tree is bipartite. 

\examspace[3in]

\begin{solution}
The induction hypothesis is that every tree with $n$ vertices is bipartite.

\inductioncase{Base case} ($n=2$): A 2-vertex tree is trivially bipartite.

\inductioncase{Inductionc step}:  Suppose $n /geq 2$ and let $T$ be an $(n+1)$-vertex tree.

Remove a leaf $f$ from $T$.  This leaves a tree $T-f$ with $n$
vertices, which, by induction hypothesis is bipartite.  If
$\edge{v}{f}$ is the edge of $T$ incident $f$, then assign $f$ to be
in the left (right) hand vertex set if $v$ is in the right (left) hand
set.  This assigment show that $T$ is bipartite.

\end{solution}

\ppart Conclude that $\chi (T) = 2$.

\begin{solution}
Since $T$ is bipartite, therefore $\chi (T) = 2$.
\end{solution}

\eparts

\end{problem}

\endinput
