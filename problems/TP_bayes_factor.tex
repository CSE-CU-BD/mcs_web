\documentclass[problem]{mcs}

\begin{pcomments}
    \pcomment{TP_bayes_factor}
    \pcomment{ARM 12/01/15}
\end{pcomments}

\pkeywords{
  conditional_probability
  independence
  Bayes_Theorem
  Bayes-factor
}

\inhandout{\newcommand{\BF}[2]{\operatorname{Bayes-factor}(#1,#2)}}

\begin{problem}
A somewhat reliable allergy test has the following properties:

\begin{itemize}

\item If you are allergic, there is a 10\% chance that the test
  will say you are not.

\item If you are not allergic, there is a 5\% chance that the
  test will say you are.
\end{itemize}

\bparts

\ppart The test results are correct at what confidence level?

\begin{solution}
The confidence is the smaller of the probability of a correct allergic
diagnosis and a correct non-allergic diagnosis, namely,
\[
\min{1 - 1/10, 1 - 1/20} = 9/10.
\]
So the test is correct at the 90\% confidence level.
\end{solution}

\ppart What is the Bayes factor for being allergic when the test
diagnoses a person as allergic?

\begin{solution}
IF $H$ is the event of being allergic, and $E$ is the event of being
diagnosed as allergic, then
\begin{align*}
\BF{E}{H}
  & \eqdef \frac{\prcond{E}{H}}{\prcond{E}{\bar{H}}}\\
  & =  frac{9/10}{1/20} = 18.
\end{align*}
\end{solution}

\ppart What can you conclude about the odds of a random person being
allergic given that the test diagnoses them as allergic?

\begin{solution}
Your odds of being allergic are 18 times those of the general
population.  But without knowing the fraction of allergic people in
the population, we can't tell what would be the odds of being allergic
given an allergic diagnosis.
\end{solution}

\eparts
\medskip

Suppose that your doctor tells you that because the test diagnosed you
as allergic, and about 25\% of people are allergic, the odds are six
to one that you are allergic.

\bparts

\ppart How would your doctor calculate these odds of being allergic
based on what's known about the allergy test?

\begin{solution}
The odds of a random person being allergic are one to three.  The odds
of being allergic given that someone tests as allergic are 18 times as
large, namely, $18 \cdot 1/3 = 6$.  So the odds are six to one that an
allergic diagnosis is correct.
\end{solution}

\ppart Another doctor reviews your test results and medical
record and says your odds of being allergic are really much higher,
namely thirty-six to one.  Briefly explain how two conscientious
doctors could disagree so much.  Is there a way you could determine
your actual odds of being allergic?

\begin{solution}
The doctors could disagree because they are seeing you as a
representative of two different groups.

Your first doctor just views you as a member of the general
population, so he reports your six to one odds based on the one to
three odds that a random person is allergic.

Your second doctor might observe from your medical record that you had
measles as a child and that 2/3 of people who had childhood measles
are known to be allergic.  So viewing you as a representative of the
childhood measles population, your second doctor would recognize that
the odds of alergy in this population are two to one.  With the Bayes
factor of 18, the odds of really being allergic for someone who tested
as allergic and also had childhood measles becomes 18 times two to
one, namely thirty-six to one.

This disagreement illustrates that there are no ``true'' odds that you
are allergic.  Either you are allergic or you are not; there are no
odds about it.  The statements about ``your'' odds are really about
the odds for random people with characteristics like yours.  Attending
to different characteristics will lead to different odds.  In the end,
no one besides yourself is \emph{exactly} like you, and the only true
odds of your being allergic are either zero if you are not allegic or
infinity if you are allergic.
\end{solution}

\eparts

\end{problem}

\endinput
