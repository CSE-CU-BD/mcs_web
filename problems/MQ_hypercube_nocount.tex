\documentclass[problem]{mcs}

\begin{pcomments}
  \pcomment{MQ_hypercube_nocount}
  \pcomment{shorter version of MQ_hypercube w/o counting H_n edges}
  \pcomment{S11.MQ4}
  \pcomment{bimodal answer distribution: so added hint --ARM 4/16/11}
\end{pcomments}

\pkeywords{
  tree
  spanning_tree
  hypercube
}

%%%%%%%%%%%%%%%%%%%%%%%%%%%%%%%%%%%%%%%%%%%%%%%%%%%%%%%%%%%%%%%%%%%%%
% Problem starts here
%%%%%%%%%%%%%%%%%%%%%%%%%%%%%%%%%%%%%%%%%%%%%%%%%%%%%%%%%%%%%%%%%%%%%

\begin{problem}

The $n$-dimensional hypercube, $H_n$, is a simple graph whose vertices
are the binary strings of length $n$. Two vertices are adjacent if and
only if they differ in exactly one bit.  Consider for example $H_3$,
shown in Figure~\ref{fig:H3}.  (Here, vertices \texttt{111} and
\texttt{011} are adjacent because they differ only in the first bit,
while vertices \texttt{101} and \texttt{011} are not adjacent because
they differ in both the first and second bits.)

Explain why it is impossible to find two spanning trees of $H_3$ that
have no edges in common.
\begin{figure}[h]
\graphic[width=0.3\linewidth]{MQ_H3}
\caption{$H_3$\,.\label{fig:H3}}
\end{figure}
\examspace[3in]

%\hint Count vertices \& edges.

\begin{solution}
$H_3$ has 8 vertices, so any spanning tree must have $8-1=7$
  edges. But $H_3$ has only 12 edges, so any two sets of 7 edges must
  overlap.
\end{solution}

\end{problem}

\endinput
