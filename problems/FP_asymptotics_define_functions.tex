\documentclass[problem]{mcs}

\begin{pcomments}
  \pcomment{FP_asymptotics_define_functions}
  \pcomment{from: S09final.prob7, S07.MQ-2/28-1}
  \pcomment{Adapted by Jodyann F09}
  \pcomment{partc from FP_multiple_choice adapted by Tom Brown}
  \pcomment{first part edited by ARM 12/18/11}
  \pcomment{2nd part added ARM 10/30/13}
\end{pcomments}

\pkeywords{
  asymptotics
  little_oh
  big_oh
  Theta
  asymptotically_equal

  partial_order
  equivalence_relation
  implies
}

%%%%%%%%%%%%%%%%%%%%%%%%%%%%%%%%%%%%%%%%%%%%%%%%%%%%%%%%%%%%%%%%%%%%%
% Problem starts here
%%%%%%%%%%%%%%%%%%%%%%%%%%%%%%%%%%%%%%%%%%%%%%%%%%%%%%%%%%%%%%%%%%%%%

\begin{problem} %\textbf{Asymptotics}
\mbox{}

\begin{problemparts} 

\ppart\label{eswn} Indicate which of the following asymptotic
relations below on the set of nonnegative real-valued functions are
equivalence relations, (\textbf{E}), strict partial orders
(\textbf{S}), weak partial orders (\textbf{W}), or \emph{none} of the
above (\textbf{N}).

\begin{itemize}

\item $f \sim g$, the ``asymptotically equal'' relation. \hfill \examrule
\begin{solution}
\textbf{E}
\end{solution}

\item $f=o(g)$, the ``little Oh'' relation. \hfill \examrule

\begin{solution}
\textbf{S}
\end{solution}

\item $f=O(g)$, the ``big Oh'' relation. \hfill \examrule

\begin{solution}
\textbf{N} because it is neither symmetric nor antisymmetric.
\end{solution}

\item $f=\Theta(g)$, the ``Theta'' relation. \hfill \examrule

\begin{solution}
\textbf{E}
\end{solution}

\item $f=O(g) \QAND \QNOT(g=O(f))$. \hfill \examrule

\begin{solution}
\textbf{S}.
\end{solution}

\end{itemize}

\ppart Indicate the implications among the assertions in
part~\eqref{eswn}.  For example,
\[
f=o(g) \QIMPLIES f = O(g).
\]

\examspace[1.0in]

\begin{solution}
\begin{align*}
f\sim g & \QIMPLIES f = \Theta(g)\ \QIMPLIES\ f= O(g),\\
f = o(g) & \QIMPLIES\ f=O(g) \QAND \QNOT(g=O(f)).
\end{align*}
\end{solution}

\problempart
Define two functions $f, g$ that are incomparable under big Oh:
\[
f \neq O(g) \QAND g \neq O(f).
\]

\examspace[1.5in]

\begin{solution}
One example is,
\[
f(n) \eqdef \begin{cases}            
n & \text{if $n$ is odd},\\
0   & \text{if $n$ is even}, 
\end{cases}\qquad
g(n) \eqdef \begin{cases}
0 & \text{if $n$ is odd},\\
n   & \text{if $n$ is even}, 
\end{cases}
\]
which can also be described by the formulas
\[
f(n) \eqdef n\sin\paren{\frac{n\pi}{2}}, \qquad g(n) \eqdef n\cos\paren{\frac{n\pi}{2}}.
\]
\end{solution}


\end{problemparts}
\end{problem}


%%%%%%%%%%%%%%%%%%%%%%%%%%%%%%%%%%%%%%%%%%%%%%%%%%%%%%%%%%%%%%%%%%%%%
% Problem ends here
%%%%%%%%%%%%%%%%%%%%%%%%%%%%%%%%%%%%%%%%%%%%%%%%%%%%%%%%%%%%%%%%%%%%%

\endinput
