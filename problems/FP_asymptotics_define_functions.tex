\documentclass[problem]{mcs}

\begin{pcomments}
  \pcomment{FP_asymptotics_define_functions}
  \pcomment{from: S09final.prob7}
  \pcomment{Adapted by Jodyann F09}
  \pcomment{partc from FP_multiple_choice adapted by Tom Brown}
\end{pcomments}

\pkeywords{
  asymptotics
  little_oh
  big_oh
  Theta
  asymptotically_equal
}

%%%%%%%%%%%%%%%%%%%%%%%%%%%%%%%%%%%%%%%%%%%%%%%%%%%%%%%%%%%%%%%%%%%%%
% Problem starts here
%%%%%%%%%%%%%%%%%%%%%%%%%%%%%%%%%%%%%%%%%%%%%%%%%%%%%%%%%%%%%%%%%%%%%


\begin{problem} %\textbf{Asymptotics}
\mbox{ }

\begin{problemparts} 

\problempart[2]
Define two functions $f, g$ that are incomparable under big Oh:
\[
f \neq O(g) \QAND g \neq O(f).
\]

\examspace[1.5in]

\begin{solution}
One example is,
\[
f(n) \eqdef \begin{cases}            
n & \text{if $n$ is odd},\\
0   & \text{if $n$ is even}, 
\end{cases}\qquad
g(n) \eqdef \begin{cases}
0 & \text{if $n$ is odd},\\
n   & \text{if $n$ is even}, 
\end{cases}
\]
which can also be described by the formulas
\[
f(n) \eqdef n\sin\paren{\frac{n\pi}{2}}, \qquad g(n) \eqdef n\cos\paren{\frac{n\pi}{2}}.
\]
\end{solution}

% FROM: Spring-07 MQ-2/28-1
%
% COMMENTS: Change to ask if they are weak/strict partial/total orders?
%           f=O(g), f=o(g), f~g, and subsets 


\ppart[3] For each of the asymptotic relations below on the set of
nonnegative real-valued functions, indicate whether it is
\emph{transitive} but not a partial order (\textbf{Tr}), a \emph{total
order} (\textbf{Tot}), a \emph{strict partial order} that is not total
(\textbf{Str}), a \emph{weak partial order} that is not total (\textbf{Wk}),
or \emph{none} of the above (\textbf{Non}).

\begin{itemize}

\item $f \sim g$, the ``asymptotically Equal'' relation. \hfill \examrule

\item $f=o(g)$,  the ``little Oh'' relation. \hfill \examrule

\item $f=O(g)$,  the ``big Oh'' relation. \hfill \examrule

\item $f=\Theta(g)$, the ``Theta'' relation. \hfill \examrule

\end{itemize}

\begin{solution}

\begin{itemize}
\item Asymptotically Equal is \textbf{Tr}.
\item Lttle Oh is \textbf{Str},
\item Big Oh is \textbf{Tr} (as it is not antisymmetric),
\item Theta is \textbf{Tr}.

\end{itemize}

\end{solution}

\end{problemparts}
\end{problem}


%%%%%%%%%%%%%%%%%%%%%%%%%%%%%%%%%%%%%%%%%%%%%%%%%%%%%%%%%%%%%%%%%%%%%
% Problem ends here
%%%%%%%%%%%%%%%%%%%%%%%%%%%%%%%%%%%%%%%%%%%%%%%%%%%%%%%%%%%%%%%%%%%%%

\endinput
