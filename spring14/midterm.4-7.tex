\documentclass[quiz]{mcs}

\renewcommand{\exampreamble}{   % !! renew \exampreamble
  \begin{tabular}{l}
    \textbf{Circle your}\quad   \teaminfo
  \end{tabular}

  \begin{itemize}

  \item
   This exam is \textbf{closed book} except for a 2-sided cribsheet.
   Total time is 80 minutes. 

  \item
   Write your solutions in the space provided.  If you need more
   space, \textbf{write on the back} of the sheet containing the
   problem.

%   Please keep your entire answer to a problem on that problem's page.
   
   \item In answering the following questions, you may use without
     proof any of the results from class or text.

\iffalse
  \item
   GOOD LUCK!
\fi

  \end{itemize}}

\begin{document}

\midterm{April 7}


%%%%%%%%%%%%%%%%%%%%%%%%%%%%%%%%%%%%%%%%%%%%%%%%%%%%%%%%%%%%%%%%%%%%%
% Problems start here
%%%%%%%%%%%%%%%%%%%%%%%%%%%%%%%%%%%%%%%%%%%%%%%%%%%%%%%%%%%%%%%%%%%%

%%%%%% modular arithmetic
\examspace
\pinput[points = 20, title= \textbf{Modular arithmetic}]{FP_divisibility_by_11}

%%%%%% Euler's Theorem & RSA
\examspace
\pinput[points = 20, title= \textbf{Number theory and RSA}]{FP_RSA_TF}

%%%%%% DAGs
\examspace
\pinput[points = 20, title= \textbf{DAGs and Partial Orders}]{FP_graph_100_dilworth}

%%%%%% simple graphs
% \examspace
% \pinput[points = 20, title= \textbf{Simple graphs}]{FP_graphs_short_answer}
% \examspace
% \pinput[points = 20, title= \textbf{Simple graphs}]{FP_simple_graphs_TF}

\examspace
\pinput[points = 20, title= \textbf{Degree sequences}]{FP_degree_sequences}

\examspace
\pinput[points = 20, title= \textbf{Colorings}]{FP_graph_colorable}

%%%%%% matching
%good problem used in Fall '13.  OK to re-use now, but wouold be
%better to slightly perturb the particlar graphs used.

\examspace
\pinput[points = 20, title= \textbf{Matching}]{MQ_matching}

% \examspace
% \pinput[points = 20, title= \textbf{Matching}]{MQ_edge_constrained}

%\examspace
%\pinput[points = 20, title= \textbf{Bipartite matching.}]{FP_bipartite_matching_recitation}

%%%%%% coloring
%\examspace
%\pinput[points = 20, title= \textbf{3-coloring.}]{MQ_3color_XOR}
%%%%%% 


%%%%%%%%%sums
\iffalse
\examspace
\pinput[points = 20, title= \textbf{Convergent Series}]{MQ_Summation}
\fi

%%%%%% asymptotics
% \examspace
% \pinput[points = 20, title= \textbf{Asymptotics}]{FP_asymptotics_proof}

%%%%%%%%%%%%%%%%%%%%%%%%%%%%%%%%%%%%%%%%%%%%%%%%%%%%%%%%%%%%%%%%%%%%%
% Problems end here
%%%%%%%%%%%%%%%%%%%%%%%%%%%%%%%%%%%%%%%%%%%%%%%%%%%%%%%%%%%%%%%%%%%%%
\end{document}
