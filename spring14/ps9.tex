\documentclass[handout]{mcs}

\begin{document}

\renewcommand{\reading}{
\begin{itemize}
\item \emph{Counting},~\bref{division_rule_sec} through~\bref{inc-ex_sec}
\item \emph{Generating functions},~\bref{generating_function_chap} through~\bref{sec:partial-fraction}
\end{itemize}
}

\problemset{9}

%%%%%%%%%%%%%%%%%%%%%%%%%%%%%%%%%%%%%%%%%%%%%%%%%%%%%%%%%%%%%%%%%%%%%
% Problems start here
%%%%%%%%%%%%%%%%%%%%%%%%%%%%%%%%%%%%%%%%%%%%%%%%%%%%%%%%%%%%%%%%%%

% Counting: binomial theorem, repetitions

\begin{editingnotes}
PS_more_numbered_trees, 14.28, is not worth the long effort

PS_multinomial_theorem is easy for a pset and uses combinatorial proof
which we're not covering this term.
\end{editingnotes}

% Counting: pigeon hole principle, inclusion-exclusion
\pinput{PS_pigeon_hunting}

\pinput{CP_inclusion-exclusion_algebra_proof} % currently in cp10m as supplemental problem but I think that cp10m is pretty long so we can cut that?  ARM: YES, did it.

% Generating functions: infinite series, counting, partial fractions
\pinput{PS_Catalan_numbers_meyer_version} % appeared in ps10 spring12
\pinput{PS_gen_fcn_quotient_polynomials} % appeared in ps10 spring12

%%%%%%%%%%%%%%%%%%%%%%%%%%%%%%%%%%%%%%%%%%%%%%%%%%%%%%%%%%%%%%%%%%%%%
% Problems end here
%%%%%%%%%%%%%%%%%%%%%%%%%%%%%%%%%%%%%%%%%%%%%%%%%%%%%%%%%%%%%%%%%%%%%
\end{document}
