\documentclass[handout]{mcs}

\begin{document}

\renewcommand{\reading}{ \iffalse
  Part~\bref{part:proofs}{. \emph{Proofs: Introduction}};\fi
\begin{itemize}
\item Chapter~\bref{predicate_sec}{.\ \emph{Predicate Formulas}},
\item Chapter~\bref{data_chap}{.\ \emph{Mathematical Data
Types} through \bref{rel_sec}{.\ \emph{Binary Relations}.}}
\end{itemize}
These assigned readings do \textbf{not}
  include the Problem sections.  (Many of the problems in the text
  will appear as class or homework problems.)}

\problemset{2}

\begin{staffnotes}
Lectures covered: Predicate Formulas, Sets \& Relations.
\end{staffnotes}

\iffalse

\medskip

\textbf{Reminder}:

\begin{itemize}

\item Problems should be
  \href{https://stellar.mit.edu/S/course/6/sp14/6.042/courseMaterial/topics/topic2/syllabus/text3/text}
       {submitted electronically}, and each problem should begin with a \emph{Collaboration Statement}.
\iffalse
\href{http://courses.csail.mit.edu/6.042/spring14/submission.shtml#collab-state}
{\emph{collaboration statement}}
\fi

\item The class has a
  \href{http://piazza.com/mit/spring2014/6042j18062j/home} {Piazza
    forum}.  With Piazza you may post questions---both administrative
  and content related---to the entire class or to just the staff.  You
  are likely to get faster response through Piazza than from direct
  email to staff.

Students are require to post a question or comment on Piazza once before lecture on Friday, 
February 14; after that Piazza use is optional.
\end{itemize}
\fi 

%%%%%%%%%%%%%%%%%%%%%%%%%%%%%%%%%%%%%%%%%%%%%%%%%%%%%%%%%%%%%%%%%%%%%
% Problems start here
%%%%%%%%%%%%%%%%%%%%%%%%%%%%%%%%%%%%%%%%%%%%%%%%%%%%%%%%%%%%%%%%%%%%%

\pinput{PS_predicate_calculus_power_of_two}
\pinput{PS_emailed_exactly_2_others}
%\pinput{CP_count_relations}

%\begin{staffnotes}
%maybe used in S13: PS_size_n_set_formula}
% CH: checked, no
%\end{staffnotes}
\pinput{PS_size_n_set_formula}

\pinput{PS_set_union}  

%\pinput{PS_composition_to_bijection}
%\pinput{FP_logical_jections}
%\pinput{CP_web_search}
\end{document}
