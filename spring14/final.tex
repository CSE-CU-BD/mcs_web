\documentclass[quiz]{mcs}

\renewcommand{\exampreamble}{   % !! renew \exampreamble

  \begin{itemize}

  \item This exam is \textbf{closed book} except for three 2-sided
    crib sheets.  Total time is 3 hours.

  \item Remember to write your name on every page.

  \item Write your solution in the space provided.  If you need more space, use the back-side of the sheet containing the problem and clearly label it.

\iffalse Please keep your entire answer to a problem on that problem's page.\fi

  \item
   GOOD LUCK!
  \end{itemize}}

%COMMENT OUT TO ENABLE EXAM SPACING
\renewcommand{\examspace}[]{}

\begin{document}

\final

\begin{editingnotes}
TOPICS
\[\begin{tabular}{lr}
- Proof techniques (contradiction, cases, WOP)    &               OK\\
- Propositional logic and predicate formulas       &         CH\\
- Set theory and relations                          &        CH\\
- Partial orders and equivalence                     &       CH\\
- Induction, state machines                           &           OK\\
- Infinite cardinality                                 &          OK\\
- Number theory (RSA)                                   &         OK\\
- Graph theory (simple graphs, trees, coloring)          &        OK\\
    Also digraphs, matching, isomorphism            & subsumed by CH\\
- Sums and products, asymptotics         &                        OK\\
- Counting (with bijections, inclusion-exclusion)  &              OK\\
- Generating functions and linear recurrences       &             OK\\
- Probability (PDFs and CDFs, conditional probability, random &   OK\\
    variables, expectation)                                    &  OK\\
- More probability (markov, chebyshev, sampling, estimation and & OK\\
    confidence)                                               &   OK\\
- Weak law of large numbers\\
- Infinite expectation                                         &  OK\\
- Gamblers' ruin
\end{tabular}\]\end{editingnotes}
%%%%%%%%%%%%%%%%%%%%%%%%%%%%%%%%%%%%%%%%%%%%%%%%%%%%%%%%%%%%%%%%%%%%%
% Problems start here
%%%%%%%%%%%%%%%%%%%%%%%%%%%%%%%%%%%%%%%%%%%%%%%%%%%%%%%%%%%%%%%%%%%%


\pinput[points = 10, title = \textbf{Basics}]{exit-diagnostic-short-S14}

\pinput[points = 10, title = \textbf{GCD \&  Invariants}]{FP_binary_gcd}

\pinput[points = 10, title = \textbf{Propositional formulas, Matchings}]{MQ_implies_relation_on_propositional_formulas_modified}

\begin{editingnotes}
\examspace
pinput[points = 10, title = \textbf{GCD}]{TP\_ax=b\_congruence}

FP\_divisibility\_by\_9  suggested by Paul Yuan
\end{editingnotes}

\examspace
\begin{editingnotes}
\TBA{ARM edit}: add part(a): $a^k = 1 \inzmod{n} \QIMPLIES k \divides
\phi(n)\text{ for } k \in [1,\phi(n)]$

\end{editingnotes}
\pinput[points = 10, title = \textbf{$\Zmod{n}$, Fermat's Little Theorem}]{FP_Fermat_primes}

\examspace
\begin{editingnotes}
minor generalization of FP\_tree\_color\_induction from cp10m:
\end{editingnotes}
\pinput[points = 10, title = \textbf{Tree coloring}]{FP_tree_kcolor}
% combed by Ying

\begin{editingnotes}
good backup problem if FP\_tree\_kcolor not used:

pinput[points = 10, title = \textbf{Coloring and XOR}]{MQ\_3color\_XOR}
\end{editingnotes}

\examspace
\begin{editingnotes}
\TBA{ARM REVISE}:
union of countable is countable, finite 1's in countable, all inf bib
strings is uncountable.
\end{editingnotes}

\pinput[points = 10, title = \textbf{Binary sequences}]{FP_infinite_binary_sequences}

\examspace
\pinput[points = 10, title = \textbf{Bijections, Inclusion-Exclusion}]{FP_string_inclusion_exclusion}

\examspace
\begin{editingnotes}
\TBA{CH REVISE} drop partial fractions.
\end{editingnotes}
\pinput[points = 10, title = \textbf{Generating Functions}]{FP_dangerous_dan_gen_func_S14}

\examspace
\pinput[points = 10, title = \textbf{Simple Graphs and Asymptotics}]{FP_simple_graphs_asymptotics}

\examspace
\pinput[points = 10, title = \textbf{Combinatorics}]{PS_3_friends}

\begin{editingnotes}
pinput[points = 10, title = \textbf{Conditional Probability}]{MQ\_voldemort\_really\_returns}
\end{editingnotes}

\examspace
\pinput[points = 10, title = \textbf{Conditional Probability}]{MQ_conditional_prob_inequality}

\examspace
\pinput[points = 10, title = \textbf{Independence}]{FP_conditional_independence}

\examspace
\pinput[points = 10, title = \textbf{Conditional Expectation, Theta()}]{FP_infinite_repeat}

\examspace
\pinput[points = 10, title = \textbf{Sampling Concepts}]{FP_sampling_concepts_S14}

\examspace
\begin{editingnotes}
\TBA{CH REVISE}: to use Chebyshev to bound fraction above average of
set of nums given ``collection-variance'' of the set.
\end{editingnotes}
\pinput[points = 10, title = \textbf{Markov Bound}]
{FP_hot_cows_markov}

\begin{editingnotes}
Suggested by yingz: pinput[points = 10, title = \textbf{Chebyshev Bound}]{FP\_Chebyshev\_tight}
\end{editingnotes}

\examspace
\begin{editingnotes}
Problems suggested by YingZ, newly made for Spring'14 Final
Status: to be benchmarked, solutions to be completed
\end{editingnotes}
\pinput[points = 10, title = \textbf{Tree diagram, Chebyshev, Sampling}]
{FP_coffee_tea_soda_drinkers}


%%%%%%%%%%%%%%%%%%%%%%%%%%%%%%%%%%%%%%%%%%%%%%%%%%%%%%%%%%%%%%%%%%%%%
% Problems end here
%%%%%%%%%%%%%%%%%%%%%%%%%%%%%%%%%%%%%%%%%%%%%%%%%%%%%%%%%%%%%%%%%%%%%

\end{document}
