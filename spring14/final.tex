\documentclass[quiz]{mcs}

\renewcommand{\exampreamble}{   % !! renew \exampreamble

  \begin{itemize}

  \item This exam is \textbf{closed book} except for a three 2-sided
    crib sheets.  Total time is 3 hours.

  \item Write your solutions in the space provided, with your name on every
    page.  If you need more space, write on the back of the sheet
    containing the problem.

\iffalse Please keep your entire answer to a problem on that problem's page.\fi

  \item
   GOOD LUCK!
  \end{itemize}}

%COMMENT OUT TO ENABLE EXAM SPACING
\renewcommand{\examspace}[]{}

\begin{document}

\final

% TOPICS
% - Proof techniques (contradiction, cases, WOP)
% - Propositional logic and predicate formulas
% - Set theory and relations
% - Induction, state machines
% - Infinite cardinality
% - Number theory (RSA)
% - Graph theory (simple graphs, matching, digraphs, trees, partial orders)
% - Sums and products, asymptotics
% - Counting (with bijections, inclusion-exclusion)
% - Generating functions and linear recurrences
% - Probability (PDFs and CDFs, conditional probability, random variables, expectation)
% - More probability (deviation, chebyshev, sampling, weak law of large numbers, estimation and confidence)
% - Infinite expectation and random walks


%%%%%%%%%%%%%%%%%%%%%%%%%%%%%%%%%%%%%%%%%%%%%%%%%%%%%%%%%%%%%%%%%%%%%
% Problems start here
%%%%%%%%%%%%%%%%%%%%%%%%%%%%%%%%%%%%%%%%%%%%%%%%%%%%%%%%%%%%%%%%%%%%

\begin{staffnotes}
maybe just part (c), or just (a) \& (c)
\end{staffnotes}
\pinput[points = 10, title = \textbf{GCD \& Invariants}]{FP_binary_gcd}

\examspace
\pinput[points = 10, title = \textbf{GCD}]{TP_ax=b_congruence}

\begin{staffnotes}
FP\_divisibility\_by\_9  suggested by Paul Yuan
\end{staffnotes}

\examspace
\pinput[points = 10, title = \textbf{$\Zmod{n}$, Fermat's Little Theorem}]{FP_Fermat_primes}

\examspace
\begin{staffnotes}
minor generalization of FP\_tree\_color\_induction from cp10m:
\end{staffnotes}
\pinput[points = 10, title = \textbf{Tree coloring}]{FP_tree_kcolor}

\begin{staffnotes}
good backup problem if FP\_tree\_kcolor not used:

pinput[points = 10, title = \textbf{Coloring and XOR}]{MQ\_3color\_XOR}
\end{staffnotes}


\examspace
\pinput[points = 10, title = \textbf{Counting, Bijections, PIE}]{FP_bijection_counting}

\examspace
\pinput[points = 10, title = \textbf{Simple Graphs and Asymptotics}]{FP_simple_graphs_asymptotics}

\examspace
\pinput[points = 10, title = \textbf{Combinatorics}]{PS_3_friends}

\begin{staffnotes}
one of the next three:
\end{staffnotes}
\examspace
\pinput[points = 10, title = \textbf{Conditional Probability}]{MQ_voldemort_really_returns}

\examspace
\pinput[points = 10, title = \textbf{Conditional Probability}]{MQ_conditional_prob_inequality}

\examspace
\pinput[points = 10, title = \textbf{Conditional Independence}]{CP_conditional_independence}
\begin{staffnotes}
maybe just the easier part(a)
\end{staffnotes}

\examspace
\pinput[points = 10, title = \textbf{Conditional Expectation, Theta()}]{FP_infinite_repeat}

% Problems suggested by YingZ from the svn repository.
% Status: to be combed, to be differentiated from past, to be benchmarked 

\examspace
\pinput[points = 10, title = \textbf{Markov Bound}]
{FP_hot_cows_markov}

\examspace
\pinput[points = 10, title = \textbf{Chebyshev Bound}]{FP_Chebyshev_tight}

% Problems suggested by YingZ, newly made for Spring'14 Final
% Status: to be polished, to be benchmarked

\examspace
\pinput[points = 10, title = \textbf{Expectation, Sampling}]
{FP_coffee_tea_soda_drinkers}

%%%%%%%%%%%%%%%%%%%%%%%%%%%%%%%%%%%%%%%%%%%%%%%%%%%%%%%%%%%%%%%%%%%%%
% Problems end here
%%%%%%%%%%%%%%%%%%%%%%%%%%%%%%%%%%%%%%%%%%%%%%%%%%%%%%%%%%%%%%%%%%%%%

\end{document}
