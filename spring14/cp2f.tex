\documentclass[handout]{mcs}

\begin{document}

\inclassproblems{2, Fri.}

%%%%%%%%%%%%%%%%%%%%%%%%%%%%%%%%%%%%%%%%%%%%%%%%%%%%%%%%%%%%%%%%%%%%%
% Problems start here
%%%%%%%%%%%%%%%%%%%%%%%%%%%%%%%%%%%%%%%%%%%%%%%%%%%%%%%%%%%%%%%%%%%%%

\pinput{TP_truth_table_for_distributive_law}
\insolutions{\newpage}
\pinput{CP_file_system_functioning_normally}
%\pinput{CP_valid_vs_satisfiable}  %on ps1
\pinput{CP_differentiable_implies_continuous}
\pinput{CP_binary_adder_logic}

\instatements{\newpage}

And if you have time\dots

\begin{staffnotes}
A programming problem---easy and not specially important---to raise
the morale of students who are better at programming than at proofs.
\end{staffnotes}

\begin{center}
\textbf{Supplmental Problem}\footnote{There is no need to study supplmental
  problems when preparing for quizzes or exams.}
\end{center}

As a break from proofs, here's a short programming exercise related to
truth tables:
\pinput{PS_printout_binary_strings}

\medskip

There are much faster adder circuits than the simple ripple carry
adder in Problem 4.  If you have time and are interested, you can try
tackling the following description of an ingenious (half) adder
design.
\pinput{PS_faster_adder_logic}

%%%%%%%%%%%%%%%%%%%%%%%%%%%%%%%%%%%%%%%%%%%%%%%%%%%%%%%%%%%%%%%%%%%%%
% Problems end here
%%%%%%%%%%%%%%%%%%%%%%%%%%%%%%%%%%%%%%%%%%%%%%%%%%%%%%%%%%%%%%%%%%%%%

\iffalse

\instatements{\newpage}
%\section*{Appendix}
\section*{The Propositional Operations}

\[
\begin{array}{c|c}
P & \QNOT P \\ \hline
\true & \false \\
\false & \true \\
\end{array}
\]

\[
\begin{array}{cc|c}
P & Q & P \QAND Q \\ \hline
\true & \true & \true \\
\true & \false & \false \\
\false & \true & \false \\
\false & \false & \false
\end{array}
\]


\[
\begin{array}{cc|c}
P & Q & P \QOR Q \\ \hline
\true & \true & \true \\
\true & \false & \true \\
\false & \true & \true \\
\false & \false & \false
\end{array}
\]

\[
\begin{array}{cc|c}
P & Q & P \QXOR Q \\ \hline
\true & \true & \false \\
\true & \false & \true \\
\false & \true & \true \\
\false & \false & \false
\end{array}
\]

\[
\begin{array}{cc|c}
    P  &   Q    & P \QIMP Q \\ \hline
\true  & \true  & \true \\
\true  & \false & \false \\
\false & \true  & \true \\
\false & \false & \true  
\end{array}
\]

\[
\begin{array}{cc|c}
P & Q & P \QIFF Q \\ \hline
\true & \true & \true \\
\true & \false & \false \\
\false & \true & \false \\
\false & \false & \true
\end{array}
\]
\fi

\end{document}

