\documentclass[12pt,oneside]{article}
\usepackage{light}

\showsolutions

%%%%%%%%%%%%%%%%%%%%%%%%%%%%%%%%%%%%%%%%%%%%%%%%%%%%%%%%%%%%%%%%%%%%%%%%%%%%%%%

\begin{document}

\generic{Quiz 1}{October 15, 2008}

\instatements{

\textbf{Circle the name of your recitation instructor}:

\begin{center}
\begin{tabular}{lllllll}
Brooke & Jay  & Bill & Nick & Chieu & Ling & Spyros 
\end{tabular}
\end{center}

\begin{itemize}

\item This quiz is \textbf{closed book}, but you may have one $8.5
\times 11$'' sheet with notes in your own handwriting on both sides.

\item Calculators are not allowed.

\item You may assume all of the results presented in class.

\item Write your solutions in the space provided.  If you
need more space, write on the back of the sheet containing the
problem.  Please keep your entire answer to a problem on that
problem's page.

\item Be neat and write legibly.  You will be graded not only on the
correctness of your answers, but also on the clarity with which you
express them.

\item If you get stuck on a problem, move on to others. The problems are not arranged in order of difficulty.

\item The exam ends at 9:30 PM.

\item For this quiz, $\mathbb N$ is the set of nonnegative integers (including 0): $\mathbb N = \{0,1, \ldots, \}$.
\item GOOD LUCK!

\end{itemize}

\vspace{0.25in}

\begin{center}
{\large
\begin{tabular}{|c|c|c|c|}
\hline
Problem & Points & Grade & Grader \\ \hline \hline
1 & 12 & & \\ \hline
2 & 16 & & \\ \hline
3 & 20 & & \\ \hline
4 & 20 & & \\ \hline
5 & 12 & & \\ \hline
6 & 10 & & \\ \hline
7 & 10 & & \\ \hline
Total & 100 & & \\ \hline 
\end{tabular}
}
\end{center}

}

%%%%%%%%%%%%%%%%%%%%%%%%%%%%%%%%%%%%%%%%%%%%%%%%%%%%%%%%%%%%%%%%%%%%%%%%%%%%%%%

\instatements{\newpage}

\begin{problem}{20}
For $n\geq 1$, let $a_n$ be the largest odd divisor of $n$, and let $b_n = a_1+a_2+ \ldots +a_n$. 

\bparts

\ppart{2} Prove the proposition that $a_{2n+1}=2n+1$. 

\solution{
\begin{proof}
The largest odd divisor of an odd number is the odd number itself.
\end{proof}
}

\ppart{5} Prove the proposition that $a_{2n}=a_n$.

\solution{
\begin{proof} Let $d$ be the largest odd divisor of $2n$. Since $d$ is odd,
$d$ must divide $n$. So, $a_{2n}\leq a_n$. Since the largest odd divisor of $n$ also divides $2n$, $a_n\leq a_{2n}$. Combining both inequalities proves the
proposition.
%If a divisor $d$ of $2n$ does not divide $n$, then $2$ must divide $d$.
%The contrapositive states that if $d$ is an odd divisor of $2n$, then it
%divides $n$. So, the largest odd divisor of $2n$ is equal to the largest
%odd divisor of $n$. 
\end{proof}
}

\ppart{10} Prove the proposition that $b_n\geq (n^2+2)/3$. 

Hint: use strong induction and when you prove the inductive hypothesis for $n+1$, distinguish the two cases $n+1$ is even (that is, $n+1=2k$ with $k\geq 1$) and $n+1$ is odd (that is, $n+1=2k+1$ with $k\geq 1$).

\solution{
\begin{proof}
We use strong induction. Let $P(n)$, for $n\geq 1$, be the predicate $b_n\geq (n^2+2)/3$.

{\bf Base case:}  
$b_1=a_1=1\geq (1^2+2)/3$. 

{\bf Inductive step:} For the purposes of proving $P(n+1)$, assume $P(k)$ for
$1 \leq k \leq n$. So, we assume that $b_k\geq (k^2+2)/3$ is true for 
$1 \leq k \leq n$. 

We distinguish the two cases $n+1$ is even and $n+1$ is odd.

If $n+1 = 2k$ with $k\geq 1$, then
\begin{eqnarray*}
b_{n+1} &=& (a_1 + a_3 + \ldots + a_{2k-1}) + (a_2 + a_4 +\ldots +a_{2k}) \\
&=& 1 + 3+\ldots +(2k -1) + (a_1 + a_2 + \ldots +a_k) \\
&=& k^2 + b_k \\
&\geq &  k^2 + (k^2 + 2)/3 \mbox{ (by the inductive hypothesis)} \\
&=& ((2k)^2 + 2)/3 \\
&=& ((n+1)^2+2)/3.
\end{eqnarray*}

If instead $n+1 = 2k + 1$
with $k\geq 1$, then
\begin{eqnarray*}
b_{n+1} &=& (a_1 + a_3 +\ldots +a_{2k+1}) + (a_2 + a_4 +\ldots +a_{2k}) \\
&=& 1 + 3+\ldots +(2k + 1) + (a_1 + a_2 +\ldots +a_k) \\
&=& (k + 1)^2 + b_k \\
&\geq& (k + 1)^2 + (k^2 + 2)/3 \mbox{ (by the inductive hypothesis)} \\
&=& ((2k + 1)^2 + 2)/3 +(2k+2)/3 \\
&>& ((n+1)^2 + 2)/3.
\end{eqnarray*}

\end{proof}
}

\ppart{0} DISREGARD THIS PART? 
Determine for which $n$ the equality $b_n=(n^2+2)/3$ holds. 
You do not need to prove your answer.

\solution{
The previous derivations show that, for odd $n+1$, $b_{n+1}>((n+1)^2 + 2)/3$, and, for even $n+1=2k$, $b_{n+1}=((n+1)^2+2)/3$ if and only if $b_k=(k^2 + 2)/3$.
This leads us to believe 
$$P(n) = ``b_n=(n^2+2)/3 \leftrightarrow n \mbox{ is a power of } 2''$$
for $n\geq 1$. 

\begin{proof} We use strong induction.

{\bf Base case:}  
$b_1=1= (1^2+2)/3$ and $1=2^0$, so $P(1)$ is true.

{\bf Inductive step:}
For the purpose of proving $P(n+1)$,
 suppose that $P(k)$ is true for $1 \leq k \leq n$.
If $n+1$ is odd $b_{n+1}>((n+1)^2 + 2)/3$ and $P(n+1)$ holds.
If $n+1=2k$ is even with $k\geq 1$, then $b_{n+1}=((n+1)^2+2)/3$ if and only if $b_k=(k^2 + 2)/3$, that is, if and only if $k$ is a power of $2$ by the inductive hypothesis. This proves $P(n+1)$, that is, $b_{n+1}=((n+1)^2+2)/3$ if and only if $n+1$ is a power of $2$.
\end{proof}
}

\eparts

\end{problem}

%%%%%%%%%%%%%%%%%%%%%%%%%%%%%%%%%%%%%%%%%%%%%%%%%%%%%%%%%%%%%%%%%%%%%%%

\begin{problem}{15}
Define the sequence of numbers $A_i$, by

$A_0=2$ and

$A_{n+1}=A_n/2 + 1/A_n$, for $n\geq1$.

Prove that $A_n\leq \sqrt{2}+1/2^n$ for all $n\geq 0$. You may use the following result:

\begin{lemma*} For real numbers $x>0$, $x/2+1/x\geq \sqrt{2}$.
\end{lemma*}

\end{problem}

\solution{
\begin{proof}
We will use induction. For $n\geq 0$, 
let $P(n)$ be the predicate $A_n\leq \sqrt{2}+1/2^n$.

{\bf Base case:}  
$A_0=2\leq \sqrt{2}+1/2^0$ is true. 

{\bf Inductive step:}
Let $n\geq 0$ and suppose the inductive hypothesis $P(n)$, that is,
$A_n\leq \sqrt{2}+1/2^n$.
We need the following lemma.

\begin{lemma*} For real numbers $x>0$, $x/2+1/x\geq \sqrt{2}$.
\end{lemma*}

\begin{proof}
For real numbers $x>0$,
\begin{eqnarray*}
&& x/2+1/x\geq \sqrt{2} \\
&\leftrightarrow & x^2+2\geq 2\sqrt{2}\cdot x \\
&\leftrightarrow & x^2-2\sqrt{2} \cdot x+2 \geq 0 \\
&\leftrightarrow & (x-\sqrt{2})^2 \geq 0,
\end{eqnarray*}
which is true.
\end{proof}

 By using induction it is straightforward to prove that
$A_n>0$ for $n\geq 0$ (base case: $A_0=2>0$; inductive step: if $A_n>0$, then $A_{n+1}=A_n/2 + 1/A_n>0$). By the lemma,
$A_n\geq \sqrt{2}$ for $n\geq 0$.
Together with the induction hypothesis
this can be used in the following derivation:
\begin{eqnarray*}
A_{n+1} &=& A_n/2 + 1/A_n \\
&\leq & (\sqrt{2}+1/2^n)/2+1/\sqrt{2}\\
&=& \sqrt{2}+1/2^{n+1}.
\end{eqnarray*}

This completes the proof.
\end{proof}
}

%%%%%%%%%%%%%%%%%%%%%%%%%%%%%%%%%%%%%%%%%%%%%%%%%%%%%%%%%%%%%%%%%%%%%%%%

\begin{problem}{15}

An $n$-player \emph{tournament} consists of some set of $n\geq 2$
\emph{players}, and has the property that for every two players, $p \neq
q$, either $p$ beats $q$ or $q$ beats $p$, but not both.  
%\iffalse That is,
%player $p$ beats player $q$ iff $q$ does not beat $p$, for all players $p
%\neq q$.\fi

%A sequence of distinct players $p_1,p_2, \ldots, p_k$, such that each
%player beats the next one (that is, $p_i$ beats player $p_{i+1}$ for $1
%\leq i < k$) is called a {\em ranking} of these players.  

%If also player
%$p_k$ beats player $p_1$, the ranking is called a $k$-\emph{cycle}.

\begin{problemparts}

\ppart {10} Prove the proposition that if there exists a cycle of at least two nodes in the tournament graph, then there exists a cycle of three nodes.

\solution{Let $v_1\rightarrow v_2 \ldots \rightarrow v_h\rightarrow v_1$ be a smallest length cycle. Since the graph is a tournament, there do not exist cycles of length $2$. So, $h\geq 3$. If $v_1\rightarrow v_3$, then $v_1\rightarrow v_3\rightarrow v_4 \ldots \rightarrow v_h\rightarrow v_1$
is a shorter cycle of length $h-1$. This contradicts $h$ being the length of the shortest cycle. So, there is  no edge $v_1\rightarrow v_3$. Since the graph represents a tournament $v_3\rightarrow v_1$. So, $v_1\rightarrow v_2\rightarrow v_3\rightarrow v_1$ is
a cycle of length 3, therefore $h\leq 3$. So, $h=3$.
}

\ppart {5} A {\em consistent ranking} is a sequence $p_1,p_2, \ldots,
p_n$ of all $n$ players in the tournament such that each player beats all
the later players in the sequence (that is, $p_i$ beats $p_j$ iff $i < j$,
for $1 \leq i,j \leq n$). Prove by using the previous problem parts, that a tournament has no consistent ranking \textit{iff} some subset of three of its players has no consistent ranking.

\solution{
A tournament has no consistent ranking iff there exists a cycle.
There exists a cycle iff  
there exists a 3-cycle by part a. 
There exists a 3-cycle iff  some subset of three players has no
consistent ranking.
}
\end{problemparts}

\end{problem}

%%%%%%%%%%%%%%%%%%%%%%%%%%%%%%%%%%%%%%%%%%%%%%%%%%%%%%%%%%%%%%%%%%%%%%%

\begin{problem}{15}
 Prove the proposition that if a finite digraph has no cycle at all, then it has a node with no incoming edges.
\end{problem}

\solution{By contradiction. Suppose that there are no cycles and suppose that each node has at least one incoming edge. We use induction and prove that there exists a walk of any length. For $h\geq 0$, let $P(h)$ be the predicate that there exists a walk of length $h$.

\textbf{Base case} $h=0$: there exists a walk of length 0.

\textbf{Inductive step}:
Assume that $P(h)$ is true in order to show that $P(h+1)$ is true.
Let $v_1\rightarrow v_2 \rightarrow \ldots \rightarrow v_h$ be a walk of length $h$. Since $v_1$ has an incoming edge $v_0\rightarrow v_1$, $v_0\rightarrow \ldots \rightarrow v_h$ is a walk of length $h+1$. This proves $P(h+1)$.

If $n$ is the number of nodes in the digraph, then there exists a node in a walk of length $n+1$ that repeats itself. This shows that there exists a cycle.
} 

%%%%%%%%%%%%%%%%%%%%%%%%%%%%%%%%%%%%%%%%%%%%%%%%%%%%%%%%%%%%%%%%%%%%%%%%

\begin{problem}{20}
Suppose $m,n$ are relatively prime and let $s$ and $t$ be integers 
such that $sm+tn=1$. 

\bparts

\ppart{10} Prove that $(sm)^k\equiv sm \mbox{ (mod } mn)$ for integers $k\geq 1$.

\solution{By strong induction. For $k\geq 1$, let $P(k)$ be the predicate 
$(sm)^k\equiv sm \mbox{ (mod } mn)$.

\textbf{Base case} For $k=1$, $(sm)^1\equiv sm \mbox{ (mod } mn)$. 
For $k=2$, we derive
\begin{eqnarray*}
(sm)^2 &=& sm(1-tn) \\
 &=& sm -stmn \\
&\equiv & sm \mbox{ (mod } mn).
\end{eqnarray*}

\textbf{Inductive step}:
Let $k\geq 2$.
Assume that $P(i)$ is true for $1\leq i \leq k$ in order to show that $P(k+1)$ is true. We derive
\begin{eqnarray*}
(sm)^{k+1} &=& (sm)(sm)^k \\
&\equiv & (sm)(sm) \mbox{ (mod } mn) \mbox{ (by $P(k)$)}\\
&\equiv & sm \mbox{ (mod } mn) \mbox{ (by $P(2)$)}
\end{eqnarray*}
}

\ppart{10} For integers $a$ and $b$, prove that 
$(sma+tnb)^k\equiv sma^k+tnb^k \mbox{ (mod } mn)$ for $k\geq 1$. You may use part a in your solution.

\solution{By induction.  For $k\geq 1$, let $P(k)$ be the predicate 
$(sma+tnb)^k\equiv sma^k+tnb^k \mbox{ (mod } mn)$.

\textbf{Base case} For $k=1$, $P(k)$ holds true.

\textbf{Inductive step}: Let $k\geq 1$ and assume $P(k)$ in order to prove $P(k+1)$. We derive
\begin{eqnarray*}
(sma+tnb)^{k+1} &= & (sma+tnb)(sma+tnb)^k \\
&\equiv & (sma+tnb)(sma^k+tnb^k) \mbox{ (mod } mn) \mbox{ (by $P(k)$)}\\
&=& (sm)^2a^{k+1} + (tn)^2b^{k+1} + (tbsa^k+satb^k)mn \\
&\equiv & (sm)^2a^{k+1} + (tn)^2b^{k+1} \mbox{ (mod } mn) \mbox{ (by $P(k)$)}\\
&\equiv & sma^{k+1} + tnb^{k+1} \mbox{ (mod } mn) \mbox{ (by part a)}
\end{eqnarray*}
}

\eparts
\end{problem}

\end{document}
