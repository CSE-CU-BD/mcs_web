\documentclass[twoside,12pt]{article}
\usepackage{light}

%\hidesolutions
\showsolutions



\begin{document}

\problemset{1}{September 4, 2008}{Monday, September 8}

%%%%%%%%%%%%%%%%%%%%%%%%%%%%%%%%%%%%%%%%%%%%%%%%%%%%%%%%%%%%%%%%%%%%%%%%%%%%%%%

\begin{problem}{12} Let $t$ be a positive integer.
Prove the following statement by proving its contrapositive: if $r$ is
irrational, then $r^{1/t}$ is irrational.  (Be sure to
\textit{state} the contrapositive explicitly.)
\end{problem}

\solution{
\begin{proof}
We prove the given statement by proving its contrapositive: if $r^{1/t}$ is rational, then
$r$ is rational. 

By our assumption that $r^{1/t}$ is rational, there exists an
integer $a$ and a positive integer $b$ such that:
%
\begin{equation}
r^{1/t} = \frac{a}{b}. \label{rab}
\end{equation}
%
Since $t$ is a positive integer, we may raise both sides of (\ref{rab}) 
to the power $t$, which gives:
%
\begin{align*}
r & = \frac{a^t}{b^t}
\end{align*}
%
Since $a^t$ and $b^t$ are integers, this implies that $r$ is 
rational.  (Note that $b^t \neq 0$, since $b$ is positive.)

This proves the contrapositive, so the original statement is also true.
\end{proof}
}

%%%%%%%%%%%%%%%%%%%%%%%%%%%%%%%%%%%%%%%%%%%%%%%%%%%%%%%%%%%%%%%%%%%%%%%%%%%%%%%

%%%%%%%%%%%%%%%%%%%%%%%%%%%%%%%%%%%%%%%%%%%%%%%%%%%%%%%%%%%%%%%%%%%%%%%%%%%


\begin{problem}{24}
In this problem $x$, $y$, and $z$ always denote positive integers.
Suppose that $x^2+y^2=z^2$.

\bparts
\ppart{8} Prove the proposition: if $z$ is even, then both $x$ and $y$ are even.

\solution{
\begin{proof}
%Assume that $z$ is even. We need to prove that both $x$ and $y$ are even.

We prove the proposition by contradiction. Assume that $z$ is even and that it is not true that both $x$ and $y$ are even; that is, at least one of $x$ or $y$ is odd. We use case analysis and show that each case leads to a contradiction.

Consider the first case: $x$ is odd and $y$ is even. Then $x^2$ is odd and $y^2$ is even. This implies that $z^2=x^2+y^2$ is odd. Since $z$ is assumed to be even, $z^2$ too is even. This leads to a contradiction: $z^2$ cannot be odd and even at the same time. 

Consider the second case: $x$ is even and $y$ is odd. This is similar to the first case and can be argued the same way by switching $x$ and $y$.
%A similar argument shows that the case where $x$ is even and $y$ is odd cannot be possible either.

Consider the third case: $x$ is odd and $y$ is odd. Then $x=2a+1$ and $y=2b+1$ for some integers $a$ and $b$. We derive
\[
z^2=x^2+y^2=4(a^2+a+b^2+b)+2,
\]
showing that $z^2$ is not a multiple of $4$. Since $z$ is assumed to be even, $z^2$ is a multiple of $4$. This contradicts the previous derivation that states that $z^2$ is not a multiple of $4$.

Since all three cases lead to a contradiction, we must conclude that our assumption---that at least one of $x$ or $y$ is odd---is incorrect. Therefore its negation---that both $x$ and $y$ are even---is true.
\end{proof}
}

\ppart{8} Prove the proposition: $x$ and $y$ cannot both be odd.

\solution{
\begin{proof}
We prove the proposition by contradiction.

Suppose to the contrary that $x$ and $y$ are both odd. Then, $x^2$ and $y^2$ are odd as well, implying that $z^2=x^2+y^2$ is even. We apply the following lemma to conclude that $z$ is even:

\begin{lemma*} If $z^2$ is even, then $z$ is even.
\end{lemma*}

\begin{proof} The proof is by contradiction.
If $z$ is odd, then $z^2$ is odd, contradicting the hypothesis that $z^2$ is even. 
\end{proof}

From the proposition in part (a) of this problem, we obtain that both $x$ and $y$ must be even (since $z$ is even). This contradicts the initial assumption that $x$ and $y$ are both odd. So the initial assumption cannot be true. That is, $x$ and $y$ cannot both be odd.
\end{proof}
}

\ppart{8} Prove the proposition: if at least one of $x$, $y$, and $z$ is odd, then $z$ is odd.

\solution{
\begin{proof}

We prove the proposition by contradiction. 

Suppose that at least one of $x$, $y$, and $z$ is odd and that $z$ is even. Then, by the proposition in part (a) of this problem, both $x$ and $y$ are even as well. This contradicts the hypothesis that at least one of $x$, $y$, and $z$ is odd. So the contrary assumption that $z$ is even cannot be true. Therefore, $z$ must be odd.
\end{proof}
}

\eparts
\end{problem}

%%%%%%%%%%%%%%%%%%%%%%%%%%%%%%%%%%%%%%%%%%%%%%%%%%%%%%%%%%%%%%%%%%%%%%%%%%%

%%%%%%%%%%%%%%%%%%%%%%%%%%%%%%%%%%%%%%%%%%%%%%%%%%%%%%%%%%%%%%%%%%%%%%%%%%%%%%%

\begin{problem}{12}
A \textit{triangle} is a set of three people such that either every
pair has shaken hands or no pair has shaken hands.  Prove that among
every six people there is a triangle. 

\noindent(Suggestion: Initially, break the problem into two cases:
%
For any person $X$ in a given group of six people,
\begin{enumerate}
\item There exist at least three other people who shook hands with $X$.
\item There exist at least three other people who didn't shake hands with $X$.
\end{enumerate}
%
Why must exactly one of these conditions hold?)
\end{problem}

\solution{
\begin{proof}
We first prove the statement given in the suggestion, which we state as a lemma:

\begin{lemma*} For any person $X$ in a given group of six people, exactly one of these conditions holds:
\begin{enumerate}
\item There exist at least three other people who shook hands with $X$.
\item There exist at least three other people who didn't shake hands with $X$.
\end{enumerate}
\end{lemma*}

\begin{proof} We prove the lemma by case analysis. There are there possible cases: neither condition holds; both conditions hold; or exactly one of the two conditions holds.

\begin{itemize}
\item Neither condition holds: that is, at most two other people have shaken hands with $X$, and at most two other people have not shaken hands with $X$. So at most four other people have shaken hands or have not shaken hands with $X$. However, since the set of people who are distinct from $X$ consists of five individuals and each individual must either have shaken hands with $X$ or have not shaken hands with $X$, we have a contradiction.

\item Both conditions hold: that is, at least three other people have shaken hands with $X$, and at least three other people have not shaken hands with $X$. Since no individual can be counted in both groups, the union of these two sets has at least six members. This contradicts the fact that there are only five individuals distinct from $X$.
\end{itemize}

Since two of the three cases lead to contradiction, we conclude that the only remaining possibility---that exactly one of the two conditions holds---must be correct.
\end{proof}

We now prove the proposition directly using case analysis. 

Let $X$ denote one of the six people. By our lemma, there
are two possibilities:
%
\begin{enumerate}
\item There exist three other people who shook hands with $X$.  Now
there are two further possibilities:
%
\begin{enumerate}
\item Among these three, some pair shook hands.  Then these two and
$X$ form a triangle.
\item Among these three, no pair shook hands.  Then these three form a
triangle.
\end{enumerate}
%
\item Otherwise, there are exactly three people who did not shake hands with $X$.
%Thus, there exist three people who didn't shake hands with $X$.
Again, there are two further possibilities:
%
\begin{enumerate}
\item Among these three, every pair shook hands.  Then these three
form a triangle.
\item Among these three, some pair did not shake hands.  Then these two
and $X$ form a triangle.
\end{enumerate}
\end{enumerate}
\end{proof}
}

%%%%%%%%%%%%%%%%%%%%%%%%%%%%%%%%%%%%%%%%%%%%%%%%%%%%%%%%%%%%%%%%%%%%%%%%%%%%%%%

\begin{problem}{16}
\bparts

\ppart{8} Prove that
\begin{equation}\label{v}
\exists z.\; [P(z) \land Q(z)] \implies [\exists x.\; P(x) \land \exists
y.\; Q(y)]
\end{equation}
is valid.  (Use the proof in the subsection on Validity in Recitation 1
 Notes as
a guide to writing your own proof here.)

\solution{
\begin{proof}
Assume
\begin{equation}\label{hyp}
\exists z.\; [P(z) \land Q(z)].
\end{equation}
That is, $P(z) \land Q(z)$ holds for some element $z$ of the domain.
Let $c$ be this element; that is, we have $P(c) \land Q(c)$.

In particular, $P(c)$ holds by itself.  So we conclude (by Existential
Generalization) $\exists x\ P(x)$.  We conclude $\exists y\ Q(y)$
similarly.  Hence,
\begin{equation}\label{conc}
\exists x.\; P(x) \land \exists y.\; Q(y)
\end{equation}
holds.  

This shows that~(\ref{conc}) holds in any interpretation in
which~(\ref{hyp}) holds.  Therefore,~(\ref{hyp}) implies~(\ref{conc}) in
all interpretations; that is,~(\ref{v}) is valid.
\end{proof}
}

\ppart{8} Prove that the converse of~(\ref{v}) is not valid by describing a
counter model as in Recitation 1
 Notes.

\solution{
We describe a counter model in which,~(\ref{conc}) is true and~(\ref{hyp})
is false.  Namely, let the domain, $D$, be $\set{\pi, e}$, $P(x)$ mean
``$x = \pi$,'' and $Q(y)$ mean ``$y = e$.''  Then, $\exists x.\; P(x)$ is
true (let $x$ be $\pi$) and likewise $\exists y.\; Q(y)$ is true (let $y$
be $e$), so~(\ref{conc}) holds.

On the other hand, $Q(\pi)$ is not true, so $P(\pi) \land Q(\pi)$ is not
true.  Likewise $P(e) \land Q(e)$ is not true.  Since these are the only
elements of $D$, it is not true that there is an element $z$ of $D$
such that $P(z) \land Q(z)$.  Therefore,~(\ref{hyp}) is not true.
}

\eparts
\end{problem}

%%%%%%%%%%%%%%%%%%%%%%%%%%%%%%%%%%%%%%%%%%%%%%%%%%%%%%%%%%%%%%%%%%%%%%%



\begin{problem}{24}

The binary logical connectives $\wedge$ (\emph{and}), $\vee$ (\emph{or}), and $\implies$ (\emph{implies})
appear often in not only computer programs, but also everyday speech. 
In computer chip designs, however, it is considerably easier to construct these out of another operation, $\nand$, which is simpler to represent in a circuit.  Here is the truth table for $\nand$:
%
\[
\begin{array}{cc|c}
P & Q & P \nand Q \\ \hline
\true & \true & \false \\
\true & \false & \true \\
\false & \true & \true \\
\false & \false & \true
\end{array}
\]
%

%
\bparts

\ppart{12} For each of the following expressions, find an equivalent expression
using only $\nand$ and $\neg$ (\emph{not}), as well as grouping parentheses to specify the order in which the operations apply. You may use $A$, $B$, and the operators any number of times.

\bsubparts
\psubpart $A \wedge B$

\solution{Observe from the truth table that $(A \nand B)$ is equivalent to $\neg(A \land B)$. We negate both expressions to produce
\[
(A \land B) = \neg(A \nand B).
\]
}

\psubpart $A \vee B$

\solution{The negation of the negation of a statement is equivalent to the original statement. We derive the desired expression as follows, applying the identity $\neg(P \vee Q) = (\neg P) \wedge (\neg Q)$:
\begin{eqnarray*}
(A \vee B) & = & \neg(\neg(A \vee B))\\
	& = & \neg((\neg A) \wedge (\neg B)) \\
	& = & (\neg A) \nand (\neg B).
\end{eqnarray*}
}
%\solution{$(\neg A) \nand (\neg B)$}

\psubpart $A \implies B$

\solution{An implication is true whenever the antecedent is false or the consequent is true:
\begin{eqnarray*}
(A \implies B) & = & ((\neg A) \lor B) \\
	& = & \neg(A \land (\neg B)) \\
	& = & A \nand (\neg B).
\end{eqnarray*}
}
%\solution{$A \nand (\neg B)$}
\esubparts



\ppart{4} It is actually possible to express each of the above using only $\nand$, without needing to use $\neg$. Find an equivalent expression for $(\neg A)$ using only $\nand$ and grouping parentheses.

\solution{Observe from the truth table for $\nand$ that when $P$ and $Q$ are both $\true$, $(P \nand Q)$ is $\false$, and when $P$ and $Q$ are both $\false$, $(P \nand Q)$ is $\true$. Setting $P$ and $Q$ equal to each other, we see that the value of  $(P \nand P)$ is the negation of the value of $P$. So an equivalent expression for $(\neg A)$ is
\[(A \nand A).\]
}
%\solution{$A \nand A$}

\ppart{8} The constants $\true$ and $\false$ themselves may be expressed using only $\nand$. Construct an expression using an arbitrary statement $A$ and $\nand$ that evaluates to $\true$ regardless of whether $A$ is $\true$ or $\false$. Construct a second expression that always evaluates to $\false$. Do not use the constants $\true$ and $\false$ themselves in your statements.

\solution{Observe from the truth table for $\nand$ that only three combinations of $\true$ and $\false$ produce $\true$. 
\[
\begin{array}{cc|c}
P & Q & P \nand Q \\ \hline
\true & \false & \true \\
\false & \true & \true \\
\false & \false & \true
\end{array}
\]
Furthermore, observe that of these three, two are interchangeable simply by negating both $P$ and $Q$:
\[
\begin{array}{cc|c}
P & Q & P \nand Q \\ \hline
\true & \false & \true \\
\false & \true & \true \\
\end{array}
\]
}
We therefore have a method by which we can produce a $\true$ value regardless of the value of $P$: make sure that $Q$ is the negation of $P$. Then, as long as $Q$ and $P$ are different, the value of $P$ has no bearing on the $\true$ value of $(P \nand Q)$.

So given an arbitrary proposition $A$, an expression that always evaluates to $\true$ is $(A \nand (\neg A))$. By part (b), this is equal to
\[
(A \nand (A \nand A)).
\]

The negation of $\true$ is $\false$. So by substituting the previous expression for $A$ in part (b), we can construct an expression that always evaluates to $\false$:
\[
(A \nand (A \nand A)) \nand (A \nand (A \nand A)).
\]

%\solution{$\begin{array}{ll}
%	\true: & \left(\left(A \nand A\right) \nand A\right)  \\
%	\false: & \left(\left(A \nand A\right) \nand A\right) \nand \left(\left(A \nand A\right) \nand A\right)
%		\end{array}$}

\eparts


\end{problem}



 
%%%%%%%%%%%%%%%%%%%%%%%%%%%%%%%%%%%%%%%%%%%%%%%%%%%%%%%%%%%%%%%%

\begin{problem}{12}
Let $A$, $B$, and $C$ be sets such that $A \cap B \cap C = \emptyset$.
Prove that:
%
\[
A \cup B \cup C = (A - B) \cup (B - C) \cup (C - A)
\]
%
You are welcome to use a diagram to aid your own reasoning, but your
proof must be text.  (For guidance on structuring the proof, see pages
64-71 in Cupillari.)
\end{problem}

\solution{
\begin{proof}
We prove that the left side is contained in the right and
the right side is contained in the left.

First, we show that the left side is contained in the right.  Let $x$
be an element of $A \cup B \cup C$.  Then $x$ is an element of at
least one of the sets $A$, $B$, or $C$.  However, by the assumption $A
\cap B \cap C = \emptyset$, we know $x$ is also \textit{not} an
element of at least one of $A$, $B$, or $C$.  Now we use case
analysis to show that $x$ is in either $A - B$, $B - C$, or $C - A$:

\begin{itemize}

\item If $x \in A$, then there are two possibilities:

\begin{itemize}

\item If $x \not\in B$, then $x \in A - B$.

\item If $x \in B$, then $x \not\in C$, since $x$ cannot be in all of
$A$, $B$, and $C$.  Thus, $x \in B - C$.

\end{itemize}

\item If $x \not\in A$, then there are again two possibilities:

\begin{itemize}

\item If $x \in C$, then $x \in C - A$.

\item If $x \not\in C$, then $x \in B$, since $x$ is in at least one
of $A$, $B$, or $C$.  Therefore, $x \in B - C$.

\end{itemize}

\end{itemize}

\noindent Since $x$ is an element of either $A - B$, $B - C$, or $C -
A$, we know that $x$ is an element of $(A - B) \cup (B - C) \cup (C -
A)$.  Therefore, the set on the left is contained in the set on the
right.

Next, we show that the right side is contained in the left.  Let $x$
be an element of $(A - B) \cup (B - C) \cup (C - A)$.  Then $x$ is
element of either $A - B$, $B - C$, or $C - A$.  This implies that $x$
is an element of either $A$, $B$, or $C$, respectively.  Thus, $x$ is
an element of $A \cup B \cup C$.  Therefore, the set on the right is
contained in the set on the left.

Since the set on the right contains the set on the left and
vice versa, the two sets are equal.
\end{proof}
}

%%%%%%%%%%%%%%%%%%%%%%%%%%%%%%%%%%%%%%%%%%%%%%%%%%%%%%%%%%%%%%%%%%%%%%%%%%%%%%%

\end{document}
