\documentclass[12pt,twoside]{article}   
\usepackage{light}

\newcommand{\hint}[1]{({\it Hint: #1})}
\newcommand{\card}[1]{\left|#1\right|}
\newcommand{\union}{\cup}
\newcommand{\lgunion}{\bigcup}
\newcommand{\intersect}{\cap}
\newcommand{\lgintersect}{\bigcap}
\newcommand{\cross}{\times}

\hidesolutions
%\showsolutions

\begin{document}
\problemset{9}{November 2, 2010}{Monday, November 8}


%Pigeonhole
\begin{problem}{10}

\bparts
%S08_cp9f.3d
\ppart{5} Show that of any $n+1$ distinct numbers chosen from the
set $\{1,2,\ldots,2n\}$, at least 2 must be relatively prime.
\hint{$\gcd(k,k+1)=1$.}

\solution{Treat the $n+1$ numbers as the pigeons and the $n$ 
disjoint subsets of the form $\{2j-1,2j\}$ as the pigeonholes. The
pigeonhole principle implies that there must exist a pair of 
consecutive integers among the $n+1$ chosen which, as suggested
in the hint, must be relatively prime.}

\ppart{5} Show that any finite connected undirected graph with 
$n \geq 2$ vertices must have 2 vertices with the same degree. 

\solution{In a finite connected graph with $n \geq 2$ vertices,
the domain for the vertex degrees is the set $\{1,2,\ldots,n-1\}$ 
since each vertex can be adjacent to at most all of the remaining
$n-1$ vertices and the existence of a degree 0 vertex would violate
the assumption that the graph be connected.  Therefore, treating
the $n$ vertices as the pigeons and the $n-1$ possible degrees as 
the pigeonholes, the pigeonhole principle implies that there
must exist a pair of vertices with the same degree.}

\eparts
\end{problem}


%%%%%%%%%%%%%%%%%%%%%%%%%%%%%%%%%%%%%%%%%%%%%%%%%
% new problem
% Inclusion-Exclusion (with and without replacement)
\begin{problem}{10} {\bf Under Siege!}

Fearing retribution for the many long hours his students spent 
completing problem sets, Prof. Leighton decides to convert his 
office into a reinforced bunker. His only remaining task is to 
set the 10-digit numeric password on his door.  Knowing the 
students are a clever bunch, he is not going to pick any passwords 
containing the forbidden consecutive sequences "18062", "6042" 
or "35876" (his MIT extension).

%\bparts
%\ppart{10}\label{without-replacement}
How many 10-digit passwords can he pick that don't contain forbidden 
sequences if each number $0, 1, \ldots, 9$ can only be chosen once 
(i.e. without replacement)?

\solution{The number of passwords he can choose is the number of 
permutations of the 10 digits minus the number of passwords containing 
one or more of the forbidden words, which we will find using 
inclusion-exclusion.

There are 6 positions 18062 could appear and the remaining digits 
could be any permutation of the remaining 5 digits.  Therefore,
there are $6 \cdot 5!$ passwords containing 18062. Similarly, there 
are $7 \cdot 6!$ passwords containing 6042 and $6 \cdot 5!$ passwords 
containing 35876.

Each of the forbidden words contain the digit 6 and since he must 
choose each number exactly once, the only way two forbidden words 
can appear in the same password is if they overlap at 6.  The only 
case where this can happen is if the password contains 35876042 and
there are $3 \cdot 2!$ such passwords.

By inclusion-exclusion the total number of passwords not containing
any of the forbidden words is
\[
10! - (6 \cdot 5! + 7 \cdot 6! + 6 \cdot 5!) + 3 \cdot 2! = 3622326
\]
}

%\ppart{10} How many 10-digit passwords can he pick that don't contain
%forbidden sequences if each number $0, 1, \ldots, 9$ can be chosen any 
%number of times (i.e. with replacement)?
%
%\solution{The number of passwords he can choose is the number of 
%length 10 strings over the alphabet $0, 1, \ldots, 9$ minus the 
%number of passwords containing one or more of the forbidden words, 
%which we will again find using inclusion-exclusion.
%
%There are 6 positions 18062 could appear and the remaining digits 
%could be any 5-digit string.  Therefore, there are $6 \cdot 10^5$ 
%passwords containing 18062. Similarly, there are $7 \cdot 10^6$ 
%passwords containing 6042 and $6 \cdot 10^5$ passwords containing 
%35876.
%
%As in part \ref{without-replacement} the only forbidden words
%that can overlap are 6042 and 35876, however, it is now possible
%to have a password containing both 35876 and 18062, both 6042 and 
%18062, or even both 6042 and 35876 without overlap.  No 10-digit
%password can contain all 3 forbidden words.
%
%There are only 2 passwords that contain both 35876 and 18062:
%3587618062 and 1806235876.
%
%To count the passwords containing both 6042 and 18062, there are 2 
%possibilities: 6042 can either come before or after 18062.  For 
%each case there are 3 possible positions for the remaining digit: 
%the first digit, the last digit or between the two words.  There 
%are 10 values for the remaining digit.  Therefore, there are 
%$2 \cdot 3 \cdot 10$ passwords containing both 6042 and 18062.
%
%By similar reasoning there are $2 \cdot 3 \cdot 10$ passwords 
%containing both 6042 and 35876 {\it without overlap} and there 
%are $3 \cdot 10^2$ passwords containing 35876042.
%
%By inclusion-exclusion the total number of passwords not containing
%any of the forbidden words is
%\[
%10^10 - \left[2\cdot(6 \cdot 10^5) + 7 \cdot 10^6 \right] 
%      + \left[2 + 2\cdot(2 \cdot 3 \cdot 10) + 3 \cdot 10^2\right]
%= 10^10 - 8199578 = 9991800422.
%\]
%}
%
%\eparts
\end{problem}



\begin{problem}{50} Be sure to show your work to receive full credit. In this problem we assume a standard card deck of 52 cards.
\bparts

\ppart{4} How many 5-card hands have a single pair and 
no 3-of-a-kind or 4-of-a-kind?

\solution{
There is a bijection with sequence of the form:

\[
(\text{value of pair}, \text{suits of pair}, \text{value of other three cards}, \text{suits of other three cards})
\]

Thus, the number of hands with a single pair is:

\[
13 \cdot \binom{4}{2} \cdot \binom{12}{3} \cdot 4^{3} = 1,098,240
\]


Alternatively, there is also a 3!-to-1 mapping to the tuple:
\[
\begin{array}{l}
(\text{value of pair}, \text{suits of pair}, \\
\text{value 3rd card}, \text{suit 3rd card},
\text{value 4th card}, \text{suit 4th card},
\text{value 5th card}, \text{suit 5th card})
\end{array}
\]

Thus, the number of hands with a single pair is:

\[
\frac{13 \cdot \binom{4}{2} \cdot 12 \cdot 4 \cdot 11 \cdot 4 \cdot 10 \cdot 4}{3!} = 1,098,240
\]
}

\ppart{4} For fixed positive integers $n$ and $k$, how many nonnegative 
integer solutions $x_0,x_1,\ldots,x_k$ are there to the following equation?
\[
\sum_{i=0}^k x_i = n
\]

\solution{There is a bijection from the solutions of the equation
to the binary strings containing $n$ zeros and $k$ ones where
$x_0$ is the number of 0s preceding the first 1, $x_k$ is the 
number of 0s following the last 1 and $x_i$ is the number of 0s 
between the $i^{th}$ and $(i+1)^{th}$ 1 for $0 < i < k$.

\[
\binom{n+k}{k}
\]
}

\ppart{4} For fixed positive integers $n$ and $k$, how many nonnegative 
integer solutions $x_0,x_1,\ldots,x_k$ are there to the following equation? 
\[
\sum_{i=0}^k x_i \leq n
\]

\solution{There is a bijection from the solutions of
\begin{align*}
\sum_{i=0}^k x_i
 & \leq n & \\
 & = n - x_{k+1} & \text{(for some $x_{k+1} \geq 0$)}
\end{align*}
and the solutions of
\[
\sum_{i=0}^{k+1} x_i = n.
\] 

\[
\binom{n+k+1}{k+1}
\]
}

\ppart{4} How many simple undirected graphs are there with $n$ vertices?

\solution{There are $\binom{n}{2}$ potential edges, each of which may or
may not appear in a given graph.  Therefore, the number of graphs is:
\[
2^{\binom{n}{2}}
\]
}

\ppart{4} How many directed graphs are there with $n$ vertices (self loops allowed)?

\solution{There are $n^2$ potential edges, each of which may or
may not appear in a given graph.  Therefore, the number of graphs is:
\[
2^{n^2}
\]
}

\newcommand{\beats}{\rightarrow}

\ppart{4} How many tournament graphs are there with $n$ vertices?

\solution{There are no self-loops in a tournament graph and 
for each of the $\binom{n}{2}$ pairs of distinct vertices $a$ and $b$,
either $a \beats b$ or $b \beats a$ but not both. Therefore, the 
number of tournament graphs is:
\[
2^{\binom{n}{2}}
\]
}

\ppart{4} How many acyclic tournament graphs are there with $n$ vertices?

\solution{For any path from $x$ to $y$ in a tournament graph, an edge
$y \beats x$ would create a cycle.  Therefore in any acyclic tournament
graph, the existence of a path between distinct vertices $x$ and $y$ 
would require the edge $x \beats y$ also be in the graph.  That is, the
"beats" relation for such a graph would be transitive.  Since each 
pair of distinct players are comparable (either $x \beats y$ or 
$y \beats x$) we can uniquely rank the players $x_1 < x_2 < \cdots < x_n$.
There are $n!$ such rankings.
}

\ppart{4}
How many numbers are there that are in the range $[1..700]$ which are divisible by $2,5$ or $7$?
\solution{
Let $S$ be the set of all numbers in range $[1..700]$. Let $S_2$ be the numbers in this range divisible by $2$, $S_5$ be the numbers in this range divisible by $5$ and $S_7$ be the numbers in this range divisible by $7$. By inclusion-exclusion, the number of elements in $S$ divisible by $2$, $5$ or $7$ is
\begin{eqnarray*}
n &=& |S_2|+|S_5|+|S_7| - |S_2S_5|-|S_2S_7|-|S_5S_7|+|S_2S_5S_7|\\
&=& \frac{700}{2}+\frac{700}{5}+\frac{700}{7}-\frac{700}{2\cdot5}-\frac{700}{2\cdot7}-\frac{700}{5\cdot7}+\frac{700}{2\cdot5\cdot7}\\
&=& 350+140+100-70-50-20+10\\
&=& 460.
\end{eqnarray*}
}

\ppart{9}
In how many ways can you arrange $n$ books on $k$ bookshelf (assuming the order of books on a shelf matters?)
\solution{\[n!\cdot \binom{n+k-1}{k-1}\]}
\ppart{9}
How about if there has to be at least $1$ book at each bookshelf?
\solution{\[k!\cdot \binom{n}{k}\cdot (n-k)! \cdot \binom{n-1}{k-1}\]}

\eparts
\end{problem}

% S08 - cp11w
\begin{problem}{15}
Give a combinatorial proof of the following theorem:
\[
n 2^{n-1} = \sum_{k=1}^n k \binom{n}{k}
\]

\hint{Consider the set of all length-$n$ sequences of 0's, 1's and a
single *.}

\solution{ Let $P = \{0,\dots,n-1\} \times \{0,1\}^{n-1}$.  On
the one hand, there is a bijection from $P$ to $S$ by mapping $(k,x)$ to
the word obtained by inserting a * just after the $k$th bit in the
length-$n-1$ binary word, $x$.  So
\begin{equation}\label{cp11m.P}
\card{S} = \card{P}= n 2^{n-1}
\end{equation}
by the Product Rule.

On the other hand, every sequence in $S$ contains between 1 and $n$
nonzero entries since the $*$, at least, is nonzero.  The mapping from a 
sequence in $S$ with exactly $k$ nonzero entries to a pair consisting of
the set of positions of the nonzero entries and the position of the *
among these entries is a bijection, and the number of such pairs is
$\binom{n}{k}k$ by the Generalized Product Rule.
Thus, by the Sum Rule:
\[
\card{S} = \sum_{k=1}^n k \binom{n}{k}
\]
Equating this expression and the expression~\eqref{cp11m.P} for $\card{S}$
proves the theorem.}

\end{problem}

%%%%%%%%%%%%%%%%%%%%%%%%%%%%%%%%%%%%%%%%%%%%%

\begin{problem}{15}
At a congressional hearing, there are $2n$ members present. Exactly $n$ of them are Democrats and $n$ of them are Republicans. The members want to select a smaller subcommittee of size $n$ from within those present at the hearing. However, since the Democrats currently hold majority, they want there to be more Democrats then Republicans in the committee. In how many ways can you select such a committee?
\hint{Consider two cases: $n$ odd and $n$ even.}
\solution{First, look at the case when $n$ is odd. There are are
\[\binom{2n}{n}\]
ways to choose any subcommittee of size $n$. $n$ is odd, so the number of Democrats is always different than the number of Republicans in a committee. Each of these committees will therefore either have more Democrats or more Republicans. However, there is an equal number of Democrats and Republicans present at the hearing, so the number of committees with more Republicans in them should by symmetry equal to the number of committees with more Democrats in them. Consequently, the number of committees with more Democrats than Republicans is
\[\frac12\binom{2n}{n}.\]
If, however, $n$ is even, then $\binom{n}{\frac{n}2}\binom{n}{\frac{n}2}$ committees will have an equal number of Democrats and Republicans (select $\frac{n}{2}$ out of $n$ Democrats and $\frac{n}{2}$ out of $n$ Republicans). The number of committees where one party has a majority is therefore $\binom{2n}{n}-\binom{n}{\frac{n}2}\binom{n}{\frac{n}2}$. Again by symmetry, there must be 
\[\frac12\Big(\binom{2n}{n}-\binom{n}{\frac{n}2}\binom{n}{\frac{n}2}\Big)\]
committees with more Democrats than Republicans.}
\end{problem}

\end{document}

