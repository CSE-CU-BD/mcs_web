\documentclass[12pt,twoside]{article}
\usepackage{light}
%\usepackage{tikz}
%\usepackage{bbding, manfnt, gensymb}
%\usepackage{pstricks}
\usepackage{verbatim}

%\hidesolutions
\showsolutions

\newcommand{\hint}[1]{({\it Hint: #1})}
\newcommand{\card}[1]{\left|#1\right|}

\begin{document}
\problemset{7}{October 14, 2008}{Monday, October 20}

\begin{problem}{10}
\bparts
\ppart{10}
What is the product of the first $n$ odd powers of two: $\prod\limits_{k=1}^n2^{2k-1}$?   

\medskip
\solution{ 
\[
\Pi_{k=1}^n2^{2k-1} = 2^{\sum_{k=1}^n{2k-1}}
= 2^{2\sum_{k=1}^n k-\sum_{k=1}^n 1}
= 2^{n(n+1) - n}
= 2^{n^2}
\]
}
%\vspace*{2in}
\ppart{10} What is $\sum\limits_{k=a}^b k$?   

\medskip
\solution{
\[
\sum_{k=a}^b k = 
\sum_{k=1}^b k - \sum_{k=1}^{a-1} k = 
\frac{b(b+1)}{2} - \frac{a(a-1)}{2}
\]

Also, some students computed the same answer directly
in the form
\[
\frac{(a+b)(b-a+1)}{2}
\]
by noticing that the first and last terms sum
to $(a + b)$, the second and next to last terms also
sum to $(a + b)$, etc. This means that we are adding
the term $(a + b)$ a number of times equal to
$\frac{(b - a + 1)}{2}$ sinc there are $(b -a + 1)$ terms
and we take two for each $(a + b)$.

The most common mistake on this part was to sum
up to $a$ instead of $a-1$ using the first approach.
}
%\vspace*{2in}

\ppart{10}

Find upper and lower bounds for 
$$
\sum_{k=1}^n\frac{1}{\sqrt{k}}
$$
which differ by at most 1.    
\medskip
\solution{ 

\begin{eqnarray*}
\int_0^n \frac{1}{\sqrt{x+1}} \ dx \leq & \sum_{k=1}^n \frac{1}{\sqrt{k}} & \leq \int_0^n \frac{1}{\sqrt{x}}\ dx \\
\int_0^n \frac{1}{\sqrt{x+1}} \ dx \leq & \sum_{k=1}^n \frac{1}{\sqrt{k}} & \leq 1 + \int_1^n \frac{1}{\sqrt{x}}\ dx \\
2\sqrt{n+1} - 2 \leq & \sum_{k=1}^n \frac{1}{\sqrt{k}} & \leq 1 + 2\sqrt{n} - 2\\
2\sqrt{n+1} - 2 \leq & \sum_{k=1}^n \frac{1}{\sqrt{k}} & \leq 2\sqrt{n} - 1
\end{eqnarray*}

The most common mistake on this part of the problem was
having the wrong limits of integration (going from $1$ to $n$
instead of from $0$ to $n$). The more severe (and less common)
mistake was to switch the two functions for the upper and lower
bound.
}
\eparts

\end{problem}
\begin{problem}{10}
Suppose you deposit \$100 into your MIT Credit Union account
today, \$99 in one month from now, \$98 in two months from now, and so on.
Given that the interest rate is constantly 0.3\% per month, how long will
it take to save \$5,000?

\solution{
First note that you will certainly manage to have saved \$5,000
\emph{some} day, since, even without your earnings from the interest,
you will have $100+99+\cdots+1=100\times 101/2 = 5,050$ dollars after
99~months. 

But fewer months will be needed: After the first deposit you will have
\$100. After the second deposit, you will have \$$(100\times 1.003 +
99)$. After your third deposit, your saved money will be \$$\bigl(
(100\times 1.003 + 99)\times 1.003 + 98) =$ \$$\bigl( 100\times
(1.003)^2 + 99\times 1.003 + 98\bigr)$, and so on. So, after the $n$th 
deposit,  
$$
S_n = \sum_{i=0}^{n-1} (100 - i) (1.003)^{n-i-1}
$$
dollars will be in your account. Substituting $j=n-i-1$, we can
rewrite this as  
$$
\sum_{j=0}^{n-1} \bigl( 100 - (n-j-1)\bigr) (1.003)^{j};
$$
and then as 
$$
(101-n) 
\Bigl( \sum_{j=0}^{n-1} (1.003)^{j}   \Bigr)+
\Bigl( \sum_{j=0}^{n-1} j (1.003)^{j} \Bigr).
$$
Using the closed forms from Course Notes~6, we can finally write $S_n$
as 
\[
(101-n)
\Bigl( \frac{1-1.003^n}{1-1.003} \Bigr)+
\Bigl( \frac{1.003-n1.003^n+(n-1)1.003^{n+1}}{(1-1.003)^2} \Bigr).
\]
Solving 
$$
S_n \geq 5,000
$$ 
for $n$, we get $n\geq 67$. That is, you'll need more than 5.5~years
to save~\$5,000.
}
\end{problem}

\begin{problem} 
False Claim:

\begin{equation}\label{2n1}
2^n = O(1).
\end{equation}

Explain why the claim is false.  Then identify and explain the mistake in
the following bogus proof.

\proof The proof by induction on $n$ where the induction 
hypothesis, $P(n)$, is the assertion~\eqref{2n1}.

{\bf base case:}  $P(0)$ holds trivially.

{\bf inductive step:} We may assume $P(n)$, so there is a constant $c >0$
such that $2^n \leq c \cdot 1$.  Therefore,
\[
2^{n+1} = 2 \cdot 2^n \leq (2c) \cdot 1,
\]
which implies that $2^{n+1} = O(1)$.  That is, $P(n+1)$ holds, which
completes the proof of the inductive step.

We conclude by induction that $2^n = O(1)$ for all $n$.  That is, the
exponential function is bounded by a constant.


\solution{
A function is $O(1)$ iff it is bounded by a constant, and since
the function $2^n$ grows unboundedly with $n$, it is not $O(1)$.

The mistake in the bogus proof is in its misinterpretation of the
expression $2^n$ in assertion~\eqref{2n1}.  The intended interpration
of~\eqref{2n1} is
\begin{equation}\label{f=exp}
\text{Let $f$ be the function defined by the rule $f(n) = 2^n$.  Then
$f = O(1)$.}
\end{equation}
But the bogus proof treats~\eqref{2n1} as an assertion, $P(n)$, about $n$.
Namely, it misinterprets~\eqref{2n1} as meaning:
\begin{quote}
  Let $f_n$ be the constant function equal to $2^n$.  That is, $f_n(k)
  = 2^n$ for all $k \in \mathbb{N}$.  Then
\begin{equation}\label{fn=c}
f_n = O(1).
\end{equation}
\end{quote}
Now~\eqref{fn=c} is true since every constant function is $O(1)$, and the
bogus proof is an unnecessarily complicated, but \emph{correct}, proof that
that for each $n$, the constant function $f_n$ is $O(1)$.  But in the
last line, the bogus proof switches from the misinterpretation~\eqref{fn=c}
and claims to have proved~\eqref{f=exp}.

So you could say that the exact place where the proof goes wrong is in its
first line, where it defines $P(n)$ based on
misinterpretation~\eqref{fn=c}.  Alternatively, you could say that the
proof was a correct proof (of the misinterpretation), and its first mistake
was in its last line, when it switches from the misinterpretation to the
proper interpretation~\eqref{f=exp}.
}

\end{problem}

\begin{problem}{30}
An explorer is trying to reach the Holy Grail, which she believes is
located in a desert shrine $d$ days walk from the nearest
oasis.\footnote{She's right about the location, but doesn't realize
that the Holy Grail is actually just the Bene\u{s} network.}  In the
desert heat, the explorer must drink continuously.  She can carry at
most 1 gallon of water, which is enough for 1 day.  However, she is
free to create water caches out in the desert.

For example, if the shrine were $2/3$ of a day's walk into the desert,
then she could recover the Holy Grail with the following strategy.
She leaves the oasis with 1 gallon of water, travels $1/3$ day into
the desert, caches $1/3$ gallon, and then walks back to the oasis---
arriving just as her water supply runs out.  Then she picks up another
gallon of water at the oasis, walks $1/3$ day into the desert, tops
off her water supply by taking the $1/3$ gallon in her cache, walks
the remaining $1/3$ day to the shine, grabs the Holy Grail, and then
walks for $2/3$ of a day back to the oasis---again arriving with no
water to spare.

But what if the shrine were located farther away?

\bparts

\ppart{5} What is the most distant point that the explorer can reach and
return from if she takes only 1 gallon from the oasis.?

\solution{At best she can walk $1/2$ day into the
desert and then walk back.}

\ppart{5} What is the most distant point the explorer can reach and
return form if she takes only 2 gallons from the oasis?  No proof is
required; just do the best you can.

\solution{The explorer walks $1/4$ day into the
desert, drops $1/2$ gallon, then walks home.  Next, she walks $1/4$
day into the desert, picks up $1/4$ gallon from her cache, walks an
additional $1/2$ day out and back, then picks up another $1/4$ gallon
from her cache and walks home.  Thus, her maximum distance from the
oasis is $3/4$ of a day's walk.}

\ppart5 What about 3 gallons?  (Hint: First, try to establish a cache
of 2 gallons \textit{plus} enough water for the walk home as far into
the desert as possible.  Then use this cache as a springboard for your
solution to the previous part.)

\solution{Suppose the explorer makes three trips $1/6$ day
into the desert, dropping $2/3$ gallon off units each time.  On the
third trip, the cache has 2 gallons of water, and the explorer still
has $1/6$ gllon for the trip back home.  So, instead of returning
immediately, she uses the solution described above to advance another
$3/4$ day into the desert and then returns home.  Thus, she reaches
%
\[
\frac{1}{6} + \frac{1}{4} + \frac{1}{2} = \frac{11}{12}
\]
%
days' walk into the desert.}

\ppart5 How can the explorer go as far as possible is she withdraws $n$
gallons of water?  Express your answer in terms of the Harmonic number
$H_n$, defined by:
%
\[
H_n = \frac{1}{1} + \frac{1}{2} + \frac{1}{3} + \ldots \frac{1}{n}
\]

\solution{With $n$ gallons of water, the explorer can
reach a point $H_n / 2$ days into the desert.

Suppose she makes $n$ trips $1/(2n)$ days into the desert, dropping of
$(n-1)/n$ gallons each time.  Before she leaves the cache for the last
time, she has $n-1$ gallons plus enough for the walk home.  So she
applies her $(n-1)$-day strategy to go an additional $H_{n-1} / 2$
days into the desert and then returns home.  Her maximum distance from
the oasis is then:
%
\[
\frac{1}{2n} + \frac{H_{n-1}}{2} = \frac{H_n}{2}
\]
}

\ppart5 Use the fact that
%
\[
H_n \sim \ln n
\]
%
to approximate your previous answer in terms of logarithms.

\solution{An approximate answer is $\ln n / 2$.  }

\ppart5 Suppose that the shrine is $d = 10$ days walk into the desert.
Relying on your approximate answer, how many days must the explorer
travel to recover the Holy Grail?

\solution{
She can obtains the Grail when:
%
\[
\frac{H_n}{2} \approx \frac{\ln n}{2} \geq 10
\]
%
This requires about $n \geq e^{20} = 4.8 \cdot 10^8$ days.
}

\eparts

\end{problem}

\begin{problem}
This problem continues the study of the asymptotics of factorials.
\bparts
\ppart

Either prove or disprove each of the following statements.
\begin{itemize}
\item $n! = O((n+1)!)$
\item $n! = \Omega((n+1)!)$
\item $n! = \Theta((n+1)!)$
\item $n! = \omega((n+1)!)$
\item $n! = o((n+1)!)$
\end{itemize}

\solution{Observe that $n! = (n+1)!/(n+1)$, and thus $n! = o((n+1)!)$. Thus, $n! = O((n+1)!)$ as well, but the remaining statements are false.}

\ppart
Show that $n! = \omega \left (\frac{n}{3} \right )^{n+e}$.

\solution{By Stirling's formula:
\[
n! \sim \sqrt{2 \pi n} \left(\frac{n}{e}\right)^{n}
\]

On the other hand, note that $\left (\frac{n}{3} \right )^{n+e} = \left (\frac{n}{3} \right )^e \left (\frac{n}{3} \right )^n$. Dividing $n!$ by this quantity,
$$\frac{3^e\sqrt{2\pi}}{n^{e-1/2}} \cdot \left (\frac{3}{e} \right )^n,$$
we see that since $3 > e$, this expression goes to $\infty$. Thus, $n! = \omega \left (\frac{n}{3} \right )^{n+e}$.
}

\ppart {100}
$n!$ is  $\Omega(2^n)$

\eparts
\end{problem}
 
\begin{problem}
There is a bug on the edge of a 1-meter rug.  The bug wants to cross
to the other side of the rug.  It crawls at 1 cm per second.  However,
at the end of each second, a malicious first-grader named Mildred
Anderson \textit{stretches} the rug by 1 meter.  Assume that her
action is instantaneous and the rug stretches uniformly.  Thus, here's
what happens in the first few seconds:

\begin{itemize}

\item The bug walks 1 cm in the first second, so 99 cm remain ahead.

\item Mildred stretches the rug by 1 meter, which doubles its length.
So now there are 2 cm behind the bug and 198 cm ahead.

\item The bug walks another 1 cm in the next second, leaving 3 cm
behind and 197 cm ahead.

\item Then Mildred strikes, stretching the rug from 2 meters to 3
meters.  So there are now $3 \cdot (3 / 2) = 4.5$ cm behind the bug
and $197 \cdot (3/2) = 295.5$ cm ahead.

\item The bug walks another 1 cm in the third second, and so on.

\end{itemize}

Your job is to determine this poor bug's fate.

\bparts

\item During second $i$, what \textit{fraction} of the rug does the
bug cross?

\solution{During second $i$, the length of the rug is $100i$ cm and
the bug crosses 1 cm.  Therefore, the fraction that the bug crosses is
$1 / 100i$.}

\item Over the first $n$ seconds, what fraction of the rug does the
bug cross altogether?  Express your answer in terms of the Harmonic

number $H_n$.

\solution{The bug crosses $1/100$ of the rug in the first second,
$1/200$ in the second, $1/300$ in the third, and so forth.  Thus, over
the first $n$ seconds, the fraction crossed by the bug is:
%
\[
\sum_{k=1}^{n} \frac{1}{100k} = H_n / 100
\]
%
(This formula is valid only until the bug reaches the far side of the
rug.)}

\item Approximately how many seconds does the bug need to cross the
entire rug?

\solution{The bug arrives at the far side when the fraction it has
crossed reaches 1.  This occurs when $n$, the number of seconds
elapsed, is sufficiently large that $H_n / 100 \geq 1$.  Now $H_n$ is
approximately $\ln n$, so the bug arrives about when:
%
\begin{align*}
\frac{\ln n}{100} & \geq 1 \\
\ln n & \geq 100 \\
n & \geq e^{100} \approx 10^{43} \text{ seconds}
\end{align*}
}

\eparts

\end{problem}

\begin{problem}
Find closed-form expressions for the following.  Show your work.

\bparts

\ppart{10}

\[
\sum_{i=0}^n \frac{9^i - 7^i}{11^i}
\]

\solution{Split the expression into two geometric series and then
apply the formula for the sum of a geometric series.
%
\begin{align*}
\sum_{i=0}^n \frac{9^i - 7^i}{11^i}
    & = \sum_{i=0}^n \left(\frac{9}{11}\right)^i -
        \sum_{i=0}^n \left(\frac{7}{11}\right)^i \\
    & = \frac{1 - \left(\frac{9}{11}\right)^{n+1}}{1 - \frac{9}{11}}
          - \frac{1 - \left(\frac{7}{11}\right)^{n+1}}{1 - \frac{7}{11}} \\
    & = - \frac{11}{2} \cdot \left(\frac{9}{11}\right)^{n+1}
        + \frac{11}{4} \cdot \left(\frac{7}{11}\right)^{n+1}
        + \frac{11}{4}
\end{align*}
}

\ppart{10}

\[
\sum_{i=0}^n \sum_{j=0}^m 3^{i+j}
\]

\solution{
\begin{align*}
\sum_{i=0}^n \sum_{j=0}^m 3^{i+j}
    & = \sum_{i=0}^n \left(3^i \cdot \sum_{j=0}^m 3^j\right) \\
    & = \left(\sum_{j=0}^m 3^j\right) \cdot \left(\sum_{i=0}^n 3^i\right)\\
    & = \left(\frac{3^{m+1} - 1}{2}\right) \cdot \left(\frac{3^{n+1} - 1}{2}\right)
\end{align*}
}

\ppart{10}

\[
\sum_{i=1}^n \sum_{j=1}^n (i + j)
\]

\solution{
\begin{align*}
\sum_{i=1}^n \sum_{j=1}^n (i + j)
    & = \left (\sum_{i=1}^n \sum_{j=1}^n i \right ) + \left (\sum_{i=1}^n \sum_{j=1}^n j \right )\\
    & = \left (\sum_{i=1}^n ni \right ) + \left (\sum_{i=1}^n  \frac{n(n+1)}{2} \right )\\
    & = \frac{2n^2(n+1)}{2}\\
    & = n^2(n+1)
\end{align*}
}

\ppart{10}
\[
\prod_{i=1}^n \prod_{j=1}^n 2^i \cdot 3^j
\]

\solution{
\begin{align*}
\prod_{i=1}^n \prod_{j=1}^n 2^i \cdot 3^j
    & = \left (\prod_{i=1}^n 2^{ni} \right ) \left (\prod_{j=1}^n 3^{nj} \right )\\
    & = 2^{n\sum_{i=1}^n i}3^{n\sum_{j=1}^n j}\\
    & = 2^{n^2(n+1)/2}3^{n^2(n+1)/2}
\end{align*}
}

\ppart{10}
In addition to expressing the following in closed form, compute the $\sim$ value for it.
\[
\prod_{i=1}^n (2i-1)
\]

\solution{
\begin{align*}
\prod_{i=1}^n (2i-1) 
    & = \frac{(2n)!}{\prod_{i=1}^n (2i)}\\
    & = \frac{(2n)!}{2^n \prod_{i=1}^n i}\\
    & = \frac{(2n)!}{2^n n!}
\end{align*}
Using Stirling's formula, $(2n)! \sim \sqrt{4\pi n} \left (\frac{2n}{e} \right )^{2n}$ and $n! \sim \sqrt{2 \pi n} \left (\frac{n}{e} \right )^n$. Thus the $\sim$ value of this expression is $\sqrt{2} \left (\frac{2}{e} \right )^n n^n$.
}
\eparts
\end{problem}

\begin{problem}{20}
For each of the following six pairs of functions $f$ and $g$ (parts (a) through (f)), state which of these order-of-growth relations hold (more than one may hold, or none may hold):

\begin{align*}
 f = o(g) && f=O(g) && f=\omega(g) && f=\Omega(g) && f=\Theta(g) && f \sim g
\end{align*}

\begin{align*}
\textbf{(a)}&& f(n) &= n!  &g(n) & = (n+1)! \\
\textbf{(b)}&& f(n) &= \log_2 n &  g(n) &= \log_{10} n \\
\textbf{(c)}&& f(n) &= 2^n & g(n) &= 10^n\\
\textbf{(d)}&& f(n) &= 0 & g(n) &= 17\\
\textbf{(e)}&& f(n) &= 1+\cos\left(\frac{\pi n}{2}\right) & g(n) &= 1+\sin\left(\frac{\pi n}{2}\right)\\
\textbf{(f)}&& f(n) &= {1.0000000001}^n & g(n) &= n^{10000000000}\\
\end{align*}

\solution{
 \begin{itemize}
  \item $f(n) = n!$ and $g(n) = (n+1)!$:
  \begin{align*}
   	\lim_{n\to\infty} \left|\frac{f(n)}{g(n)}\right|
	&= \lim_{n\to\infty}\frac1{n+1} \\
	&= 0
  \end{align*}
	 So $f(n) = o(g(n))$ and $f(n) = O(g(n))$.

  \item $f(n) = \log_2 n$ and $g(n) = \log_{10} n$:
\begin{align*}
   	\lim_{n\to\infty} \left|\frac{f(n)}{g(n)}\right|
	&= \lim_{n\to\infty}\frac{\ln n / \ln 2}{\ln n / \ln 10} \\
	&= \frac{\ln 10}{\ln 2}
  \end{align*}
	 So $f(n) = \Omega(g(n))$ and $f(n) = O(g(n))$ and $f(n) = \Theta(g(n))$.

  \item $f(n) = 2^n$ and $g(n) = 10^n$:
\begin{align*}
   	\lim_{n\to\infty} \left|\frac{f(n)}{g(n)}\right|
	&= \lim_{n\to\infty} \frac{2^n}{10^n} \\
	&= \lim_{n\to\infty} (1/5)^n \\
	&= 0
  \end{align*}
	 So $f(n) = o(g(n))$ and $f(n) = O(g(n))$.

\item $f(n) = 0$ and $g(n) = 17$:
\begin{align*}
   	\lim_{n\to\infty} \left|\frac{f(n)}{g(n)}\right|
	&= \frac{0}{17} \\
	&= 0
  \end{align*}
	 So $f(n) = o(g(n))$ and $f(n) = O(g(n))$.

\item $f(n) = 1+\cos\left(\frac{\pi n}{2}\right)$ and $g(n) = 1+\sin\left(\frac{\pi n}{2}\right)$:

	For all $n \equiv 1$ (mod 4), $f(n)/g(n) = 0$, so $f(n) \not= \Omega(g(n))$. Likewise, for all $n \equiv 0$ (mod 4), 
$g(n)/f(n) = 0$, so $f(n) \not= O(g(n))$. Therefore, none of the relations hold.

\item $f(n) = {1.0000000001}^n$ and $g(n) = n^{10000000000}$:
\begin{align*}
   	\lim_{n\to\infty} \left|\frac{f(n)}{g(n)}\right|
	&= \lim_{n\to\infty} \frac{1.0000000001^n}{n^{10000000000}} \\
	&= \lim_{n\to\infty} \frac{1.0000000001^n \ln 1.0000000001}{10000000000n^{9999999999}} \\
	&= \lim_{n\to\infty} \frac{1.0000000001^n (\ln 1.0000000001)^{10000000000}}{10000000000!} \\
	&= \infty
  \end{align*}
	 So $f(n) = \omega(g(n))$ and $f(n) = \Omega(g(n))$.

	
 \end{itemize}

}


\end{problem}


\end{document}
