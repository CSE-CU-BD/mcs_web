\documentclass[12pt,twoside]{article}   
\usepackage{light}

\hidesolutions
%\showsolutions

\newcommand{\hint}[1]{({\it Hint: #1})}
\newcommand{\card}[1]{\left|#1\right|}

\begin{document}
\problemset{6}{October 13, 2010}{Monday, October 19}

%%%%%%%%%%%%%%%%%%%%%%%%%%%%%%%%%%%%%%%%%%%%%%%%%%%%%%%%%%%%%%%%%%%%%%%%%%%%%%%


%-----------------------------------------------------------------%
% source: Velleman 4.6, Exercise 3.
% topic: equivalence relations
% last used Fall00 Problem Set 3 (in ProblemRepository)

\begin{problem}{20}[15]
For each of the following, either prove that it is an equivalence relation and state its equivalence classes, or give an example of why it is not an equivalence relation. 

\bparts
\ppart{5}
$R_n := \{(x, y) \in \mathbb Z \times \mathbb Z \textrm{ s.t. } x \equiv y \pmod n \}$
\solution{
	It is an equivalence relation.  To prove this, we will show that $R_n$ is symmetric, transitive, and reflexive.
	\begin{itemize}
		\item
			\textbf{Reflexive}: $x \equiv x \pmod n$.  This is because $x = x + 0 \cdot n$.  
		\item 
			\textbf{Symmetric}: We want to show that $R_n(x, y) \Rightarrow R_n(y, x)$.  If $R_n(x, y)$, then there is some $c \in \mathbb Z$ such that $x = y + c \cdot n$.  But then,
				subtracting $c \cdot n$ from both sides, we have that $y = x + (-c) \cdot n$, so $y \equiv x \pmod n$.  So $R_n(y, x)$, and the symmetric property holds.
		
		\item
			\textbf{Transitivity}.  Suppose $R_n(x, y)$ and $R_n(y, z)$.  From the first statement, we know that there is some $c \in \mathbb Z$ such that $x = y + c\cdot n$.  From the second,
			we know that there is some $d \in \mathbb Z$ such that $y = z + d \cdot n$.  Substituting in this value of y, we see that $x = (z + d \cdot n) + c \cdot n = z + (d + c) \cdot n$.  The
			sum $c + d$ is an integer, so $R_n(x, z)$ holds.  
	\end{itemize}
	The equivalence classes are then the sets of numbers congruent to the numbers $\{0, 1, \ldots, n-1 \}$ modulo n.
}

\ppart{5}
$R := \{ (x, y) \in P \times P \textrm{ s.t. } x \textrm{ is taller than } y \}$
where $P$ is the set of all people in the world today.
\solution{
	This is not an equivalence relation, because the concept of symmetry is broken.  If $y$ is taller than $x$, then $x$ is not taller than $y$.
}

\ppart{5} 
$R := \{ (x, y) \in \mathbb Z \times \mathbb Z \textrm{ s.t. } gcd(x,y)=1 \}$
\solution{
	This is not an equivalence relation, because transitivity is broken.  Consider the case when $x = 3$, $y = 7$, and $z = 15$.  Then, $gcd(x,y)=1$ and $gcd(y,z)=1$, but $gcd(x,z)=3\neq 1$.
}

\ppart{5}
$R_G := $ the set of $(x, y) \in V \times V$ such that $V$ is the set of vertices of a graph $G$, and there is a path $x, v_1, \ldots, v_k, y$ from $x$ to $y$ along the edges of $G$.

\solution{
	This is an equivalence relation.  We will show this by proving that it obeys reflexivity, symmetry, and transitivity.
	\begin{itemize}
	\item
		\textbf{Reflexivity}: Any vertex is connected to itself.  
	\item
		\textbf{Symmetry}: If $R_G(x, y)$, then there is a path $x, v_1, \ldots, v_k, y$ from $x$ to $y$.  The reverse of this path is $y, v_k, \ldots, v_1, x$, and is a path from
		$y$ to $x$.  So $R_G(y, x)$.
	\item
		\textbf{Transitivity}:  Suppose $R_G(x, y)$ and $R_G(y, z)$.  Then, there is a path from $x$ to $y$: $x, v_1, \ldots, v_k, y$.  Furthermore, there is a path from $y$ to $z$: 
		$y, w_1, \ldots, w_l, z$.  But then, the concatenation of those two is a path $x, v_1, \ldots, v_k, y, w_1, \ldots, w_l, z$ from $x$ to $z$.  So $R_G(x, z)$.
	\end{itemize}
	
	Thus we have shown that $R_G$ is an equivalence relation on a graph $G$, and the equivalence classes are the connected components of $G$.
}

\eparts
\end{problem}


%%%%%%%%%%%%%%%%%%%%%%%%%%%%%%%%%%%%%%%%%%%%%%%%%%%%%%%%%%%%%%%%%%%%%%%
\begin{problem}{20}
Every function has some subset of these properties:
%
\begin{center}
injective \hspace{0.75in} surjective \hspace{0.75in} bijective
\end{center}
%
Determine the properties of the functions below, and briefly explain
your reasoning.

\bparts

\ppart{5} The function $f : \mathbb{R} \to \mathbb{R}$ defined by $f(x) = x \sin(x)$.
\solution{This function is surjective, because for every $y\in \mathbb R$ there is a $x \in \mathbb R$ such that $x\sin(x) = y.$
You can see that, because the function is continuous and for every positive $N$ there is an $x$ such that $f(x)>N$ and an $x'$ such that $f(x')<-N$.
The function is not injective, because there are values of $y$ which equal $f(x_1)=f(x_2)$ for some different values of $x_1,x_2$.
For example, $f(0)=f(\pi)$. Consequently, the function is not bijective either.}

\ppart{5} The function $f : \mathbb{R} \to \mathbb{R}$ defined by $f(x) = 99x^{99}$.
\solution{This function is surjective, because for every $y\in \mathbb R$ there is a $x \in \mathbb R$ such that $99x^{99} = y.$
You can see that, for the same reasons as the function in part a.
The function is also injective, because for every values of $y \in R$, there is exactly one $x \in R$ such that $y=f(x)$. This holds, since the function is strictly increasing.
Consequently, the function is also bijective.}

\ppart{5} The function $f : \mathbb{R} \to \mathbb{R}$ defined by $\tan^{-1}(x)$.
\solution{This function is not surjective, because for all $x \in R$, $-\frac\pi 2 \leq tan^{-1}(x) \leq \frac\pi 2$. Consequently, there are some numbers $y\in R$, for example $y=3$, such that no $x\in R$ for which $y=tan^{-1}(x)$ exists.
The function is also injective, because for every values of $y \in R$, there is exactly one $x \in R$ such that $y=f(x)$. This holds, since the function is strictly increasing. Finally, the funcion is not bijective, because it is not surjective.}

\ppart{5} The function $f : \mathbb N \to \mathbb N$ defined by $f(x) = \textrm{ the number of numbers that divide }x$.  For example, $f(6) = 4$ because $1, 2, 3, 6$ all divide $6$.
\emph{Note: We define here the set $\mathbb N$ to be the set of all positive integers ($1, 2, \ldots$)}.

\solution{We claim that $f$ is surjective but not injective.  To see that it is not injective, note that $f(6)=4=f(10)$.
  
However, we must now show that it is surjective.  Given number $n$, we know from the fundamental theorem of arithmetic that it has a unique prime factorization 
$p_1^{k_1} p_2^{k_2} \ldots p_m^{k_m}$.  Note that the numbers that divide $n$ are simply numbers of the form $p_1^{r_1}p_2^{r_2}\ldots p_m^{r_m}$, where $r_i \leq k_i$ for all
indices $i$.  Because the $p_i$ in this product are unique primes, every combination of choices of exponents $\{r_i\}$ will yield a different number that divides $n$.  So the total
number of numbers that divides $n$ is $\prod_{i=1}^m k_i$.  One easy way to see that $f(n)$ is surjective, then, is to consider $n=2^k$.  Then, for any integer $k$, $f(2^{k-1}) = k$.  So 
$f$ is surjective on the positive integers.}

\eparts

\end{problem}

%%%%%%%%%%%%%%%%%%%%%%%%%%%%%%%%%%%%%%%%%%%%%%%%%%%%%%%%%%%%%%%%%%%%%%%%%%%%%%

\begin{problem}{20}
In this problem we study partial orders (posets). Recall that a weak partial order $\preceq$ on a set $X$ is reflexive $(x \preceq x)$, anti-symmetric ($x \preceq y \wedge y \preceq x \rightarrow x = y$), and transitive ($x \preceq y \wedge y \preceq z \rightarrow x \preceq z$). Note that it may be the case that neither $x \preceq y$ nor $y \preceq x$. A chain is a list of {\it distinct} elements $x_1, \ldots, x_i$ in $X$ for which $x_1 \preceq x_2 \preceq \cdots \preceq x_i$. An antichain is a subset $S$ of $X$ such that for all distinct $x, y \in S$, neither $x \preceq y$ nor $y \preceq x$. 

The aim of this problem is to show that any sequence of $(n-1)(m-1) + 1$ integers either contains a non-decreasing subsequence of length $n$ or a decreasing subsequence of length $m$. Note that the given sequence may be out of order, so, for instance, it may have the form $1, 5, 3, 2, 4$ if $n = m = 3$. In this case the longest non-decreasing and longest decreasing subsequences have length $3$ (for instance, consider $1, 2, 4$ and $5, 3, 2$).

%\solution{Consider the set $S_1$ of all minimal elements $x \in X$. Then $S_1$ is an antichain since for distinct $x,y \in S_1$, neither $x \preceq y$ nor $y \preceq x$ since both $x$ and $y$ are minimal. Now let $S_2$ be the set of all minimal elements in $X \setminus S_1$. Then $S_2$ is also an antichain. We repeat this process, obtaining sets $S_1, \ldots, S_r$ for which, (1) for all $1 \preceq i \preceq r$, $S_i$ is an antichain, (2) $S_i \cap S_j = \emptyset$ for all $i \neq j$, and (3) $X = S_1 \cup S_2 \cup \cdots \cup S_r$. Thus, $S_1, \ldots, S_r$ are antichains that partition $X$.

%It remains to show that we can choose $r \geq n$. We know that $X$ contains a chain of $n$ elements, i.e., $n$ distinct elements $x_1 \preceq x_2 \preceq \cdots \preceq x_n$. Then, for every $i$, $S_i$ can contain at most one $x_j$, since if $x_j$ and $x_{j'}$ were in $S_i$ for some $j < j'$, by transitivity of a poset, $x_j \preceq x_{j'}$, contradicting the fact that $S_i$ is an antichain. It follows that $r \geq n$.}

%Now we show $r \leq n$. Consider any element $s_r \in S_r$. Then we must have $s_{r-1} \preceq s_r$ for some element $s_{r-1} \in S_{r-1}$, as otherwise we would have put $s_{r}$ in $S_{r-1}$. Similarly, there must be some element $s_{r-2} \in S_{r-2}$ for which $s_{r-2} \preceq s_{r-1}$, as otherwise we would have put $s_{r-1}$ in $S_{r-2}$. Since $S_i \cap S_j = \emptyset$ for $i \neq j$, we obtain a list of distinct elements $s_1 \preceq s_2 \preceq \cdots \preceq s_r$, so we obtain a chain of length $r$. But the longest chain in $X$ has length $n$, so $r \leq n$.}
\bparts

\ppart{7} Label the given sequence of $(n-1)(m-1)+1$ integers $a_1, a_2, \ldots, a_{(n-1)(m-1)+1}$. Show the following relation $\preceq$ on $\{1, 2, 3, \ldots, (n-1)(m-1)+1\}$ is a weak poset: $i \preceq j$ if and only if $i\leq j$ and $a_i \leq a_j$ (as integers).

\solution{We show reflexivity, anti-symmetry, and transitivity. Clearly $i \preceq i$ since $i \leq i$ and $a_i \leq a_i$, so $\preceq$ is reflexive. Next, suppose $i \preceq j$ and $j \preceq i$. Then $i \leq j \leq i$, so $i = j$, and $\preceq$ is anti-symmetric. Finally, suppose $i \preceq j$ and $j \preceq k$. Then $i \leq j$ and $j \leq k$, so $i \leq k$. Moreover, $a_i \leq a_j$ and $a_j \leq a_k$, so $a_i \leq a_k$. Thus, $\preceq$ is transitive.}

For the next part, we will need to use Dilworth's theorem, as covered in lecture. Recall that Dilworth's theorem states that if $(X, \preceq)$ is any poset whose longest chain has length $n$, then $X$ can be partitioned into $n$ disjoint antichains. 

\ppart{7} Show that in any sequence of $(n-1)(m-1) + 1$ integers, either there is a non-decreasing subsequence of length $n$ or a decreasing subsequence of length $m$. 

\solution{Consider the $\preceq$ relation on $\{1, 2, \ldots, (n-1)(m-1)+1\}$ defined above. The length of the longest non-decreasing subsequence of the given integers is just the length of the longest chain in this poset. If the longest chain has length at least $n$, we are done, so suppose the length of the longest chain is at most $c \leq n-1$. 

Then, by the first part we know that $\{1,2, \ldots, (n-1)(m-1)+1\}$ can be decomposed into $c$ disjoint antichains. Consider the indices $i_1 \leq i_2 \leq \cdots \leq i_s$ in any antichain $A$. Then it must be the case that $a_{i_1} > a_{i_2} > \cdots > a_{i_s}$, as otherwise we would have $a_{i_j} \leq a_{i_{j'}}$ for some $j < j'$, and thus $j \preceq j'$, and $A$ could not be an antichain. It follows that there is a decreasing subsequence of length at least $|A|$.

Since it is possible to partition $\{1, 2, \ldots, (n-1)(m-1)+1\}$ into $c \leq n-1$ disjoint antichains, one such antichain must have size at least
$$\frac{(n-1)(m-1)+1}{c} \geq \frac{(n-1)(m-1)+1}{n-1} \geq m-1 + \frac{1}{n-1} \geq m,$$which completes the proof.}

\ppart{6} Construct a sequence of $(n-1)(m-1)$ integers, for arbitrary $n$ and $m$, that has no non-decreasing subsequence of length $n$ and no decreasing subsequence of length $m$. Thus in general, the result you obtained in the previous part is best-possible.

\solution{Consider the set of integers $\{1, 2, \ldots, (n-1)(m-1)\}$. For each $1 \leq i \leq n-1$, define the decreasing subsequence of length $m-1$: $$B_i = i(m-1), \ldots, (i-1)(m-1)+1.$$
Then the $B_i$ partition $\{1, 2, \ldots, (n-1)(m-1)\}$. Consider the sequence $$S = B_1 \circ B_2 \circ \cdots \circ B_{n-1}.$$ Any non-decreasing subsequence of $S$ can contain at most one integer from any single $B_i$, since the $B_i$ are decreasing subsequences. Thus, the length of the longest non-decreasing subsequence is at most $n-1$. 

Any decreasing subsequence must be entirely contained in a single $B_i$, since for $j > i$, any integer in $B_j$ is larger than any integer in $B_i$. Thus, the length of the longest decreasing subsequence is at most $m-1$.}

\eparts

\end{problem}

%%%%%%%%%%%%%%%%%%%%%%%%%%%%%%%%%%%%%%%%%%%%%%%%%%%%%%%%%%%%%%%%%%


%%%%%%%%%%%%%%%%%%%%%%%%%%%%%%%%%%%%%%%%%%%%%%%%%
% From F07, ps6, problem 5

\begin{problem}{20}
Louis Reasoner figures that, wonderful as the Bene\u{s} network may be, the
butterfly network has a few advantages, namely: fewer switches, smaller
diameter, and an easy way to route packets through it.  So Louis designs an
$N$-input/output network he modestly calls a \textit{Reasoner-net} with the
aim of combining the best features of both the butterfly and Bene\u{s} nets:
\begin{quote}
The $i$th input switch in a Reasoner-net connects to two switches, $a_i$
and $b_i$, and likewise, the $j$th output switch has two switches, $y_j$
and $z_j$, connected to it.  Then the Reasoner-net has an $N$-input
Bene\u{s} network connected using the $a_i$ switches as input switches and
the $y_j$ switches as its output switches.  The Reasoner-net also has an
$N$-input butterfly net connected using the $b_i$ switches as inputs and<
the $z_j$ switches as outputs.
\end{quote}

In the Reasoner-net the minimum latency routing does not have minimum
congestion.  The \textit{latency for min-congestion} (LMC) of a net is the
best bound on latency achievable using routings that minimize congestion.
Likewise, the \textit{congestion for min-latency} (CML) is the best bound
on congestion achievable using routings that minimize latency.

Fill in the following chart for the Reasoner-net and briefly explain your
answers.

\[
\begin{array}{|c|c|c|c|c|c|}
\textbf{diameter} &
\textbf{switch size(s)} &
\textbf{\# switches} &
\textbf{congestion} &
\textbf{LMC} &
\textbf{CML}\\
\hline
&&&&&\\
\hline
\end{array}
\]

\solution{
\[
\begin{array}{|c|c|c|c|c|c|}
\textbf{diameter} &
\textbf{switch size(s)} &
\textbf{\# switches} &
\textbf{congestion} &
\textbf{LMC}&
\textbf{CML}\\
\hline \log N + 4 & 2 \times 2 & 3N (\log N + 1) & 1 & 2 \log N +
3 &
\sqrt{N}\\
\hline
\end{array}
\]

The diameter of a Reasoner-net is the smaller diameter of the two components plus 2 (to connect to switch to input/output). The diameter of the butterfly component is $\log N + 2$, while the diameter of the Bene\u{s} component is $2 \log N + 1$, so overall diameter is 2 + diameter of butterfly = $\log N + 4$.

The number of switches is the number of input and output switches in the
Reasoner-net, $4N$, plus the number of switches in its butterfly
component, $N (\log N + 1)$, and its Bene\u{s} component, $2N \log N$.

The congestion is the congestion of the better of the two component nets, which is the congestion of the Bene\u{s} component.

The LMC for the butterfly net equals its diameter, and likewise for the LMC
of the Bene\u{s} net.  So the LMC of the Reasoner-net is 2 plus the LMC of
the routing through the component with minimum congestion, namely, 2 plus
the diameter of the Bene\u{s} net.

The CML equals the congestion of the routing through the component
with minimum latency, namely, the congestion of the butterfly net.}
\end{problem}


%%%%%%%%%%%%%%%%%%%%%%%%%%%%%%%%%%%%%%%%%%%%%%%%%

%%%%%%%%%%%%%%%%%%%%%%%%%%%%%%%%%%%%%%%%%%%%%%%%%
% From S07, ps5, commented out

\begin{problem}{20}
Let $B_n$ denote the butterfly network with $N=2^n$ inputs and $N$
outputs, as defined in Notes 6.3.8. We will show that the congestion of $B_n$ is
exactly $\sqrt{N}$ when $n$ is even.

%Let $b$ be the binary number for vertex $v$. Now, let the set of
%input vertices that can reach $v$ be $S_v$ and the set of output
%vertices reachable from $v$ be $T_v$. The set $S_v$ consists of all
%input vertices whose binary numbers match $b$ in the last $n-i$
%bits, but the first $i$ bits can be arbitrary. There are $2^i$ ways
%to set these bits, so there are $2^i$ inputs in $S_v$. The set $T_v$
%consists of all output vertices whose binary numbers match $b$ in
%the first $i$ bits, but the last $n-i$ bits can be arbitrary. There
%are $2^{n-i}$ ways to set these bits, so there are $2^{n-i}$ outputs
%in $T_v$.

\emph{Hints:}
\begin{itemize}
\item For the butterfly network, there is a unique path from
each input to each output, so the congestion is the maximum number
of messages passing through a vertex for any matching of inputs to
outputs.
\item If $v$ is a vertex at level $i$ of the butterfly
network, there is a path from exactly $2^i$ input vertices to $v$
and a path from $v$ to exactly $2^{n-i}$ output vertices.\\
\item At which level of the butterfly network must the congestion be
worst? What is the congestion at the node whose binary
representation is all $0$s at that level of the network?
\end{itemize}

\bparts
\ppart{10} Show that the congestion of $B_n$ is at most $\sqrt{N}$ when $n$ is even.

\solution{
First we will show that the congestion is at most $\sqrt{N}$.

Let $v$ be an arbitrary vertex at some level $i$. Let $S_v$ be the
set of inputs that can reach vertex $v$. Let $T_v$ be the set of
outputs that are reachable from vertex $v$.

By the hint, we have $\card{S_v} = 2^i$ and $\card{T_v} = 2^{n-i}$.
The number of inputs in $S_v$ that are matched with outputs in $T_v$
is at most $\min \set{2^i,2^{n-i}}$.  To obtain an upper-bound on
the congestion of the network, we need to find the maximum value of
$\min \set{2^i,2^{n-i}}$, where the maximum is taken over all $i$.
The maximum value is achieved when $2^i$ and $2^{n-i}$ are as equal
as possible. Since $n$ is even, these two quantities are equal when
$i=n/2$, hence the maximum congestion is
\[
2^{n/2}=N^{1/2} = \sqrt{N}.
\]}

\ppart{10} Show that the congestion achieves $\sqrt{N}$ somewhere in the network and conclude that the congestion of $B_n$ is exactly $\sqrt{N}$ when $n$ is even.

\solution{
We concluded that the congestion of
$\sqrt{N}$ can be achieved only at a node at level $\frac{n}{2}$.
Consider the node at that level whose binary representation is all
$0$s. Any packet from the input in the form
$z\underbrace{0\ldots000}_{n/2\ bits}$ with destination
$\underbrace{000\ldots0}_{n/2\ bits}z'$, where $z$ and $z'$ are any
$\frac{n}{2}$-bit numbers, must pass through this node. In the worse case, all packets from  input in the form
$z\underbrace{0\ldots000}_{n/2\ bits}$ will have destination in the form
$\underbrace{000\ldots0}_{n/2\ bits}z'$.  But there
are $2^{n/2}=\sqrt{N}$ of such possible packets, giving the node load $\sqrt{N}$.
Therefore, we can conclude that the congestion of $B_n$ is exactly
$\sqrt{N}$ when $n$ is even.}

%In fact, the set of vertices with load $\sqrt{N}$ are those vertices
%$v$ at level $\frac{n}{2}$ with the last $\frac{n}{2}$ bits being
%the reversal of its first $\frac{n}{2}$ bits.
%
%As a sanity check, there are a total of $\sqrt{N}\ n$-bit numbers in
%the form $kk^r$. So there are $\sqrt{N}$ vertices at level
%$\frac{n}{2}$ with load $\sqrt{N}$, giving us a total of $N$
%packets.


\eparts

\end{problem}

\end{document}
