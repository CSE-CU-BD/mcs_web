\documentclass[twoside,12pt]{article}
\usepackage{light}

\showsolutions
%\hidesolutions


\newcommand{\mfigure}[3]{\centerline{\resizebox{#1}{#2}{\includegraphics{#3}}}}
\newcommand{\union}{\cup}
\newcommand{\intersect}{\cap}


\begin{document}

\problemset{5}{October 3, 2006}{Wednesday, October 11 at 8pm}

%%%%%%%%%%%%%%%%%%%%%%%%%%%%%%%%%%%%%%%%%%%%%%%%%%%%%%%%%%%%%%%%%%%%%%%%%%%%%%

\begin{problem}[5 points]
Prove that every tree with more than one node has at least 2 leaves.

\solution{Similarly to the proof for one leaf given in lecture, let
$p$ be the longest path in the tree. Since the tree is connected,
and has at least 2 nodes, $p$ contains at least two nodes. Since the
tree is acyclic, and $p$ is the longest path, the two nodes at the
beginning and end of $p$ must be of degree 1. Therefore, there are
at least two nodes of degree 1, i.e. leaves.}

\end{problem}

%%%%%%%%%%%%%%%%%%%%%%%%%%%%%%%%%%%%%%%%%%%%%%%%%%%%%%%%%%%%%%%%%%%%%%%%%%%%%%%

\begin{problem}[10 points]
Prove that any connected, $n$-node graph $G$ with $n-1$ edges is a
tree.

\solution{To show that $G$ is a tree, given that it is connected, we
merely need to show that it is acyclic. We can prove this by
contradiction. Suppose that there is a cycle in $G$. Then we can
remove an edge from this cycle and $G$ will still be connected.
Continue this process until $G$ no longer contains a cycle. Then we
are left with a connected, $n$ node, acyclic graph, with $<n-1$
edges. But we proved in lecture that a connected, $n$ node, acyclic
graph (which is a tree), has exactly $n-1$ edges, thus there is a
contradiction, $G$ must be acyclic, and is therefore a tree.}
\end{problem}

%%%%%%%%%%%%%%%%%%%%%%%%%%%%%%%%%%%%%%%%%%%%%%%%%%%%%%%%%%%%%%%%%%%%%%%%%%%%%%%

\begin{problem}[10 points]
Complete binary trees with $N$ inputs and $N$ outputs, where $N=2^n$
for some $n\geq 0$, were described in class and the notes.  In this
problem we consider complete \emph{ternary} trees with $N$ inputs
and $N$ outputs, where $N=3^n$. The following figure shows a ternary
tree with 3 inputs and 3 outputs (i.e $N=3$, and $n=1$).

\mfigure{!}{2in}{pset5-tentree}

\bparts \ppart Find a closed-form expression for the diameter of the
$N$-input, $N$-output ternary tree.

\solution{ The diameter is $2 \log_3 N + 2$, or $2n + 2$. }

\ppart Prove that your expression is correct using induction.
\textit{Hint:} Your induction hypothesis should prove an expression
for the length of a path from an input to the root node.

\solution{We proceed by induction on $n$.  Let $P(n)$ be the
proposition that, in a complete ternary tree with $N=3^n$ inputs and
outputs:
\begin{itemize}
\item the number of edges in the path from any input to the root node,
or from the root node to an output is $n + 1$, and
\item the diameter in the tree is $2n + 2$.
\end{itemize}

\textbf{Base case} ($n=0$): A complete ternary tree with $N=3^0=1$
input and $1$ output consists of those two input/output nodes plus a
single switch. The only path from the input to the root traverses
$1$ edge, which equals $n + 1 = 0 + 1$. Similarly, the only path
from the root to the output traverses $1$ edge. The only path from
the input to the output traverses 2 edges, which equals $2n + 2 = 0
+ 2$. Thus $P(0)$ is true.

\textbf{Inductive step.}  Now assume $P(n)$ for $n \geq 0$ in order
to prove $P(n+1)$.  A complete ternary tree $T_0$ with $3^{n+1}$
inputs and $3^{n+1}$ outputs is constructed from three complete
ternary trees $T_1, T_2, T_3$ with $3^n$ inputs and $3^n$ outputs by
connecting their root nodes to a single new root node $r_0$.

By the inductive hypothesis, the length of a path from an input in
$T_1$ to its root $r_1$ is $n + 1$.  This root node is connected to
the new root $r_0$ by two edges $(r_0,r_1)$ and $(r_1,r_0)$.  Thus
the path from an input in $T_1$ to $r_0$ traverses $(n + 1) + 1$
edges, as required.  This same argument applies to the trees $T_2$
and $T_3$.

If an input and an output belong to $T_1$ then, by the inductive
hypothesis, the path between them traverses $2n + 2$ edges. For an
input and output in different subtrees, say $T_1$ and $T_2$, the
shortest path between them consists of the path from the input to
$r_1$, the edges $(r_1,r_0)$ and $(r_0,r_2)$, and the path from
$r_2$ to the output. This path traverses $(n + 1) + 2 + (n + 1) =
2(n + 1) + 2$ edges, as required. Thus the maximum distance from any
input to any output is $2(n+1) + 2$. Since such a path between nodes
in different subtrees will always exist, as the tree is connected,
this will be the minimum diameter as well, and so the diameter must
be exactly $2(n+1) + 2$. This proves $P(n+1)$.

By the principle of induction $P(n)$ is true for all $n \geq 0$.}
\eparts
\end{problem}

%%%%%%%%%%%%%%%%%%%%%%%%%%%%%%%%%%%%%%%%%%%%%%%%%%%%%%%%%%%%%%%%%%%%%%%%%%%%%%%

\begin{problem}[20 points]
Let $B_n$ denote the butterfly network with $N=2^n$ inputs and $N$
outputs, as defined in lecture. Show that the congestion of $B_n$ is
exactly $\sqrt{N}$ when $n$ is even.

%Let $b$ be the binary number for vertex $v$. Now, let the set of
%input vertices that can reach $v$ be $S_v$ and the set of output
%vertices reachable from $v$ be $T_v$. The set $S_v$ consists of all
%input vertices whose binary numbers match $b$ in the last $n-i$
%bits, but the first $i$ bits can be arbitrary. There are $2^i$ ways
%to set these bits, so there are $2^i$ inputs in $S_v$. The set $T_v$
%consists of all output vertices whose binary numbers match $b$ in
%the first $i$ bits, but the last $n-i$ bits can be arbitrary. There
%are $2^{n-i}$ ways to set these bits, so there are $2^{n-i}$ outputs
%in $T_v$.

\emph{Hints:}
\begin{itemize}
\item For the butterfly network, there is a unique path from
each input to each output, so the congestion is the maximum number
of messages passing through a vertex for any matching of inputs to
outputs.
\item If $v$ is a vertex at level $i$ of the butterfly
network, there is a path from exactly $2^i$ input vertices to $v$
and a path from $v$ to exactly $2^{n-i}$ output vertices.\\
\item At which level of the butterfly network must the congestion be
worst? What is the congestion at the node whose binary
representation is all $0$s at that level of the network?
\end{itemize}

\solution{

First we will show that the congestion is at most $\sqrt{N}$.

Let $v$ be an arbitrary vertex at some level $i$. Let $S_v$ be the
set of inputs that can reach vertex $v$. Let $T_v$ be the set of
outputs that are reachable from vertex $v$.

By the hint, we have $\size{S_v} = 2^i$ and $\size{T_v} = 2^{n-i}$.
The number of inputs in $S_v$ that are matched with outputs in $T_v$
is at most $\min \set{2^i,2^{n-i}}$.  To obtain an upper-bound on
the congestion of the network, we need to find the maximum value of
$\min \set{2^i,2^{n-i}}$, where the maximum is taken over all $i$.
The maximum value is achieved when $2^i$ and $2^{n-i}$ are as equal
as possible. Since $n$ is even, these two quantities are equal when
$i=n/2$, hence the maximum congestion is
$$ 2^{n/2}=N^{1/2} = \sqrt{N}. $$

Now we need to show that the congestion achieves $\sqrt{N}$
somewhere in the network. We concluded that the congestion of
$\sqrt{N}$ can be achieved only at a node at level $\frac{n}{2}$.
Consider the node at that level whose binary representation is all
$0$s. Any packet from the input in the form
$z\underbrace{0\ldots000}_{n/2\ bits}$ with destination
$\underbrace{000\ldots0}_{n/2\ bits}z'$, where $z$ and $z'$ are any
$\frac{n}{2}$-bit numbers, must pass through this node. But there
are $2^{n/2}=\sqrt{N}$ of them, giving the node load $\sqrt{N}$.
Therefore, we can conclude that the congestion of $B_n$ is exactly
$\sqrt{N}$ when $n$ is even.

%In fact, the set of vertices with load $\sqrt{N}$ are those vertices
%$v$ at level $\frac{n}{2}$ with the last $\frac{n}{2}$ bits being
%the reversal of its first $\frac{n}{2}$ bits.
%
%As a sanity check, there are a total of $\sqrt{N}\ n$-bit numbers in
%the form $kk^r$. So there are $\sqrt{N}$ vertices at level
%$\frac{n}{2}$ with load $\sqrt{N}$, giving us a total of $N$
%packets.

}

\end{problem}


%%%%%%%%%%%%%%%%%%%%%%%%%%%%%%%%%%%%%%%%%%%%%%%%%%%%%%%%%%%%%%%%%%%%%%%%%%%%

\begin{problem}[25 points]
Two students from Podunk University have a neat idea with which they
intend to beat out all of the top search engines! Their new product,
based on a simple web search algorithm called {\em Doodle},  uses
the following ranking algorithm:
$$Doodlerank(x) = \sum_{y \rightarrow x} Doodlerank(y)$$
(the Doodleranks are required to be greater than or equal to 0).
This is much nicer than Pagerank, since it gets rid of that silly
weighting scheme!

\bparts

\ppart Describe the set of possible settings of Doodlerank's for the
nodes in the following graphs.
\begin{enumerate}
%\bparts
\item The directed path of length $n$.

\solution{ The first node in the path must get Doodlerank 0, and so,
by induction on the number of nodes along the path, all nodes must
get Doodlerank 0. }

\item The directed cycle of length $n$.

\solution{ All nodes must get the same Doodlerank. }
\end{enumerate}
%\eparts

%\ppart Given any node $x$, let $R(x)$ be the set of nodes that $x$
%can reach by following directed edges in the graph. Show that for
%any member $y$ of $R(x)$, $Doodlerank(y)$ is at least as big as
%$Doodlerank(x)$.
%
%\solution{ By induction on the length of the path from $x$ to $y$. }

\ppart Show that there are graphs in which each node can reach any
other node, but for which the only way to assign weights so that the
Doodlerank equations are satisfied is so that the Doodlerank weights
are all zero!

\solution{An example of such a graph:

\begin{figure}[htp]
\begin{center}
\includegraphics[width=55mm]{pset5-x1tox3.jpg}
\end{center}
\end{figure}


%If either $x_1$ or $x_3$ has Doodlerank strictly greater than 0,
%then, by the previous problem part, since there is a path from
%$x_1$ to $x_3$ and from both have to be strictly greater than 0.
%Then the Doodlerank of $x_2$ and $x_4$ which is $x_1+x_3$, must each be
%strictly greater than that of $x_1$ or $x_3$. But, the Doodlerank of
%$x_1$ must be equal to that that of $x_2$, and similarly, the
%Doodlerank of $x_3$ must be equal to that that of $x_4$, so we have
%a contradiction. }

If either $x_1$ or $x_3$ has Doodlerank strictly greater than 0,
then $x_2$ must have  Doodlerank strictly greater than 0.  Then
$x_1$ and $x_3$ must both have Doodlerank strictly greater than 0.
So, $x_2$, which has Doodlerank $x_1+x_3$ has Doodlerank strictly
greater than that of $x_1$ or $x_3$. But, the Doodlerank of $x_1$
must be equal to that that of $x_2$, and similarly, the Doodlerank
of $x_3$ must be equal to that that of $x_2$, so we have a
contradiction. }


\ppart Ok,  the Podunk students are finally convinced that they have
to use the weighting scheme from Google -- that is, the equations
must satisfy
$$Doodlerank(x) = \sum_{y \rightarrow x} \frac{Doodlerank(y)}{outdegree(y)}$$
However, the Podunk students want to make their fortune by skipping
the modification of the original graph, so that the sinks are not
made to point to any new universal nodes as in Pagerank.

Show that their scheme has a major problem. First show that if any
node $y$ is assigned Doodlerank 0, then any node $x$ such that $x$
can reach $y$ in a directed walk must also be assigned Doodlerank
$0$.

\solution{ By induction on the length of the path. Let $P(n)$ be the
predicate that any node that can reach a 0 Doodlerank node in $ \leq
n$ steps must have 0 Doodlerank. For the base case, $n=0$, the node
has 0 Doodlerank and we are done. Assume $P(n)$ is true, let's try
to prove $P(n+1)$. Let $x$ be any node that can reach the 0
Doodlerank node in $i$ steps where $i \leq n+1$.  Let $d$ be the
Doodlerank of $x$. If $i \leq n$, then we know that $d$ must be $0$
by the induction hypothesis.  If $i=n+1$, then let $y$ be the first
node on the shortest path from $x$ to the 0 Doodlerank node.  Since
$y$ can reach the $0$ Doodlerank node in $n$ steps, we know, by the
induction hypothesis, that its Doodlerank must be 0.   On the other
hand, $x$ contributes $d/outdegree(x)>0$ to the Doodlerank of $y$.
The only way this could be the case is if  $d=0$. Thus, $P(n+1)$ is
true and we have shown that any node which can reach a 0 Doodlerank
node must be assigned a Doodlerank of 0. }

\ppart In the next set of problem parts we are going to show that
any sink must be assigned Doodlerank 0. To do this, we are going to
remember the view of the Doodlerank equations as describing votes by
nodes for each of their neighbors.

\begin{enumerate}

\item
First show that any set of Doodleranks that satisfy the equations
must satisfy ``what goes in must come out'' -- that is, the
Doodlerank of a node $x$ must equal the sum of his votes for (or
Doodlerank contribution to) each of his neighbors.

\solution{ $Doodlerank(x) = \frac{Doodlerank(x)}{outdegree(x)} \cdot
outdegree(x)$ }

\item
Next show the same for any set of nodes $S$ -- that is, show that
the sum of the Doodleranks of the nodes in $S$ is equal to the
weighted sum of the votes of the nodes in $S$ applied towards nodes
in $V$.

\solution{ $\sum_{x \in S} Doodlerank(x) = \sum_{x \in S}
\frac{Doodlerank(x)}{outdegree(x)} \cdot outdegree(x)\\ = \sum_{\{x
\rightarrow y\}| x \in S, y \in V \}}
\frac{Doodlerank(x)}{outdegree(x)}  $ }

\item
Finally, show that any sink must be assigned Doodlerank 0. {\em
Hint:  Let $S$ be the set of sinks, and $T=V-S$ the set of nonsinks.
Write $\sum_{x \in T} Doodlerank(x)$ in terms of the weighted sum of
the votes of the nodes in the graph for nodes in $T$ and then in
terms of the weighted sum of the votes in the whole graph. Use this
to show that the weighted sum of the votes for nodes in $S$ must be
0.}

\solution{ On one hand, we have that
\begin{align*}
\sum_{x \in T} Doodlerank(x)  & = &
\sum_{x \in T} \sum_{\{y \rightarrow x| y \in V \}} \frac{Doodlerank(y)}{outdegree(y)} \\
&=&  \sum_{\{y \rightarrow x| x\in T, y \in V \}} \frac{Doodlerank(y)}{outdegree(y)} \\
\end{align*}
where the first equality is just by the Doodlerank equations.

On the other hand, we have that
\begin{align*}
\sum_{x \in T} Doodlerank(x)  & = &
\sum_{\{ x \rightarrow y | x \in T, y \in V\}}  \frac{Doodlerank(x)}{outdegree(x)} \\
& = & \sum_{\{ x \rightarrow y | x \in V, y \in V\}}  \frac{Doodlerank(x)}{outdegree(x)} \\
& = & \sum_{\{ y \rightarrow x | y \in V, x \in V\}}  \frac{Doodlerank(y)}{outdegree(y)} \\
\end{align*}
where the second equality follows since $\{ x \rightarrow y | x \in
T, y \in V\}$ describes the set of all edges in the graph, as there
are no edges directed out of $S$. The third equality follows by a
simple change of variables.

So,  putting these two together,
$$ \sum_{\{ y \rightarrow x | y \in V, x \in V\}}  \frac{Doodlerank(y)}{outdegree(y)}
=  \sum_{\{y \rightarrow x| x\in T, y \in V \}}
\frac{Doodlerank(y)}{outdegree(y)} $$

But,
$$ \sum_{\{ y \rightarrow x | y \in V, x \in V\}}  \frac{Doodlerank(y)}{outdegree(y)}
=  \sum_{\{y \rightarrow x| x\in T, y \in V \}}
\frac{Doodlerank(y)}{outdegree(y)} +  \sum_{\{y \rightarrow x| x\in
S, y \in V \}} \frac{Doodlerank(y)}{outdegree(y)} $$ so it must be
the case that
$$ \sum_{\{y \rightarrow x| x\in S, y \in V \}} \frac{Doodlerank(y)}{outdegree(y)} =0$$
Since all the Doodleranks are positive, this means that  for each
sink $x$, any node $y$ that has an edge to $x$ must have Doodlerank
$0$.  But then, by the Doodlerank equations, the Doodlerank of $x$
must also be $0$.

%& =& \sum_{x \sum_{x \rightarrow y} \frac{Doodlerank(x)}{outdegree(x)} \\
%& = & \sum_{x \in S} \sum_{x \rightarrow y} \frac{Doodlerank(x)}{outdegree(x)} \\
%& & + \sum_{x \notin S} \sum_{x \rightarrow y}
%\frac{Doodlerank(x)}{outdegree(x)} \\
%& =& \sum_{x \notin S} \sum_{x \rightarrow y}
%\frac{Doodlerank(x)}{outdegree(x)} \\
%&=& \sum_{x \notin S} Doodlerank(x) \\
%\end{align*}


%The first equality is by the definition of Doodlerank.
%The second equality comes from rearranging the terms so
%that edges directed out from $x$ are grouped together
%(rather than edges directed into $x$).  The third equality
%splits up the summation into those $x$ in $S$ and those $x$ not
%in $S$.  The fourth equality comes from noting that those
%$x$ in $S$ have no outedges, so the summation is empty and thus
%sums to 0.

%Finally, since the sum of the Doodleranks of those $x$ in $S$ must be 0,
%and Doodleranks are positive, it must be the case that each Doodelrank
%of an $x$ in $S$ is 0.
   }





\item Finally, conclude that any node which can reach a sink must also
be assigned a Doodlerank of 0.  So it's not too likely that Podunk
U. is going to be hitting up these students for contributions
anytime soon!

\solution{ We have shown that sinks must be assigned Doodlerank 0,
and that any node which can reach a Doodlerank 0 node must also be
assigned Doodlerank 0. Note that we just did a proof by induction in
which the base case was harder than the inductive step! }

\end{enumerate}

\eparts

\end{problem}
%%%%%%%%%%%%%%%%%%%%%%%%%%%%%%%%%%%%%%%%%%%%%%%%%%%%%%%%%%%%%%%%%%%%%%%%%%%%

\begin{problem}[20 points]

In "Die Hard: The Afterlife", the ghosts of Bruce and Sam have been
sent by the evil Simon on another mission to save  midtown
Manhattan. They have been told that there is a bomb on a street
corner that lies in Midtown Manhattan, which Simon defines as
extending from 41st Street to 59th Street and from 3rd Avenue to 9th
Avenue. Additionally, the code that they need to defuse the bomb is
on another street corner. Simon, in a good mood, also tosses them
two carrots:
\begin{itemize}
\item He will have a helicopter initially lower them to the street corner where the bomb is.
\item He promises that the code is placed only on a corner of a numbered
street and a numbered avenue, so they don't have to search Broadway.
\end{itemize}

The map of midtown Manhattan is an example of an $N \times M$
(undirected) grid.  In particular, midtown Manhattan is a $19 \times
7$ grid.

%INSERT FIGURE of 19 by 7 grid

Bruce and Sam need to check all $19 \cdot 7 = 133$ street corners
for the code.  Once they are at a corner, they don't need any
additional time to verify if the code is there.  Once they find the
code and return to the bomb, they can disarm it in 2 minutes (even,
or especially, as the timer ticks down to 0). Also, they can run one
block (in any of the four directions) in exactly 1 minute.
%(even though avenue blocks are longer than
%street blocks in NYC, it takes them the same amount of time for each
%kind).
They are given 135 minutes total in which to find the code and
disarm the bomb, which means that they need to return to the bomb,
code in hand, in 133 minutes.

Sam realizes that the map of NYC is actually a graph, and that they
need to use a cool new 6.042 concept: A {\em Hamiltonian cycle} is a
path that visits each vertex in a graph exactly once and ends at its
starting point (so it is a cycle). A graph is {\em Hamiltonian} if
it has a Hamiltonian cycle.

Hamiltonian graphs are really useful because you can visit each node
and return to the starting point by taking only $n$ steps, where $n$
is the number of nodes -- if a graph is not Hamiltonian, you would
need at least $n+1$ steps to visit each of the $n$ nodes and return
to the starting point.

In general, we don't know how to efficiently determine whether a
general graph is Hamiltonian or not. However, Sam is very excited
because he thinks that he can show that Midtown Manhattan is
Hamiltonian.  If it is, Bruce and Sam can save the day! Will they
make it?

\bparts

\ppart Show that they cannot do it -- that is, more generally,
show that if both $N$ and $M$ are odd, then the $N\times M$
grid is {\em not} Hamiltonian. {\em Hint:
First show that any $N \times M$ 2-dimensional undirected grid is
bipartite.}
\solution{ As per the hint, let us first show that any 2-dimensional grid 
is bipartite. For this, let us exhibit a coloring with 2 colors $\{0,1\}$. 
Indexing 
the vertices of the grid by their $(x,y)$-coordinates, color vertex 
$(i,j)$ with color $rem(i+j, 2)$. It is easy to see that this is a valid 
2-coloring.

Suppose the graph is Hamiltonian.
Now, since $N,M$ are both odd, there are an odd number of vertices in the 
graph. Thus the hamiltonian cycle in this graph is an odd cycle. However, 
since the graph is bipartite, this graph has no odd cycles. This is a 
contradiction: thus our supposition that the graph was Hamiltonian is 
wrong, and we are done.
}

\ppart
Suppose Simon defined Midtown in the more standard way
as extending from 40th Street to 59th Street and from 3rd Avenue to 9th
Avenue (that is
suppose Midtown Manhattan was a $20 \times 7$ grid),
and gave them another 7 minutes,
\begin{enumerate}
\item
Show that if either $N$ or $M$ is even, then the $N\times M$
grid is Hamiltonian.
{\em Hint: assuming $N$ is even (without loss of
generality) use induction on $N$. }
\solution{Suppose $N$ is even (wlog).

Assume the grid is laid out on the plane occupying the integer points 
between $(0,0)$ and $(N-1, M-1)$.

We will give the hamiltonian path explicitly by specifying the $k$'th 
vertex visited for each $k$ from $0$ to $NM$. Let $q = \lfloor 
\frac{k}{M-1} \rfloor$ and let $r = rem(k, M-1)$.

On step $k$:
\begin{itemize}
\item if $k \leq N(M-1)$ and $q$ is even, then visit vertex $(q,r+1)$.
\item if $k \leq N(M-1)$ and $q$ is odd, then visit vertex $(q, M-1-r)$.
\item if $k > N(M-1)$, then visit vertex $(0,N - (k-N(M-1)))$.
\end{itemize}

Checking that it is a Hamiltonian cycle is routine.


Drawing the above path on an actual grid is very enlightening.

}

\item Explain why your proof breaks down when $N$ and
$M$ are odd.
\solution{ The odd/even conditions on $q$ determine require $N$ to be even 
for the above sequence of vertices to actually give a cycle.
}
\item
Would they survive? Does it depend on where the bomb is placed?
\solution{Come on, of course! No it doesn't depend on where the bomb is 
placed.}
\end{enumerate}


\eparts


\end{problem}

%%%%%%%%%%%%%%%%%%%%%%%%%%%%%%%%%%%%%%%%%%%%%%%%%%%%%%%%%%%%%%%%%%%%%%%%

\begin{problem}[10 points]
Our friends at Podunk University, after their failure with
\emph{Doodle}, have instead decided to find fame and fortune by
solving a different problem. They now claim that they have
discovered a \emph{greedy algorithm} that solves the problem of
finding minimum-weight perfect matchings in (undirected) graphs
where a perfect matching exists. The greedy algorithm is as follows:
\begin{enumerate}
\item Add to the matching the minimum-weight edge in the graph which
is not incident to the same node as any of the edges already in the
matching.
\item Continue until there are $|V|/2$ edges in the matching,
where $|V|$ is the number of nodes in the graph.
\end{enumerate}

Show that this algorithm doesn't work by finding a counterexample,
i.e. a graph where there is a perfect matching, but this algorithm
fails to find the minimum-weight perfect matching.

\solution{One possible counterexample:\\
\begin{figure}[h!]
\begin{center}
\includegraphics[width=35mm]{pset5-counterexample.jpg}
\end{center}
\end{figure}\\

In this graph, the algorithm will select the edge $\{x_1, x_2\}$
first, and then be stuck with the edge $\{x_3, x_4\}$. The
minimum-weight perfect matching is, of course, $\{x_1, x_4\}$ and
$\{x_2, x_3\}$.

}

\end{problem}


\end{document}
