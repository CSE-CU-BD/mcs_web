\documentclass[12pt,twoside]{article}   
\usepackage{light}

\hidesolutions
%\showsolutions

%\newcommand{\new}{\forall^{(1)}}

% logic terms
%\newcommand{\true}{\mathbin{\mathtt{true}}}
%\newcommand{\false}{\mathbin{\mathtt{false}}}
%\newcommand{\nand}{\mathbin{\mathtt{nand}}}

% subparts
%\newcounter{problemsubpart}
%\renewcommand{\theproblemsubpart}{\roman{problemsubpart}}
%\newcommand{\bsubparts}{
%    \begin{list}{\textbf{(\roman{problemsubpart})}}{\usecounter{problemsubpart}}}
%\newcommand{\psubpart}
%  {\item%
%   \pdfbookmark[2]{(\theproblemsubpart)}{Problem\theproblemthm\theproblemsubpart}}
%\newcommand{\esubparts}{\end{list}}

%\renewcommand{\divides}{\mbox{ | }}


\newcommand{\hint}[1]{({\it Hint: #1})}
\newcommand{\card}[1]{\left|#1\right|}

\begin{document}
\problemset{3}{September 21, 2010}{Monday, September 27 at 7:00 PM}

% taken from: new problem
% comments: basic computations using number theory
\begin{problem}{16} {\bf Warmup Exercises}

For the following parts, a correct numerical answer will only earn 
credit if accompanied by it's derivation.  Show your work.

\bparts

\ppart{4}\label{pulverizer} Use the Pulverizer to find integers 
$s$ and $t$ such that $135 s + 59 t = \gcd(135,59)$.

\solution{
\[
\begin{array}{ccccrcl}
x & \quad & y & \quad & \rem(x,y) & = & x - q \cdot y \\ \hline
135 && 59 && 17  & = &   135 - 2 \cdot 59 \\
59 && 17 && 8   & = &   59 - 3 \cdot 17 \\
&&&&            & = &   59 - 3 \cdot (135 - 2 \cdot 59) \\
&&&&            & = &   -3 \cdot 135 + 7 \cdot 59 \\
17 && 8  && 1   & = &   17 - 2 \cdot 18 \\
&&&&            & = &   (135 - 2 \cdot 59) -
                               2 \cdot (-3 \cdot 135 + 7 \cdot 59) \\
&&&&            & = &   \fbox{$7 \cdot 135 - 16 \cdot 59$} \\
2  && 1  && 0
\end{array}
\]

{\it {\bf Exam tip:} Each time $rem(x,y)$ is calculated, 
substitutions are immediately made to then express it as a 
linear combination of 135 and 59 (using the remainders 
calculated on previous lines). Simplifying at each step 
leads to a much faster computation of $s$ and $t$.}
}


\ppart{4} Use the previous part to find the inverse of 59 modulo 135 
in the range $\{1,\ldots,134\}$.

\solution{119

From part \eqref{pulverizer}, $1 = 7 \cdot 135 - 16 \cdot 59$
and so $1 \equiv -16 \cdot 59 \pmod{135}$. Therefore -16 is \emph{an} 
inverse of 59. However, it is not the \emph{unique} inverse of
59 in the range $\{1,\ldots,134\}$, which is given by $\rem(-16,135)=119$.
One can easily check this by multiplication.
}


\ppart{4} Use Euler's theorem to find the inverse of 17 modulo 31 in
the range $\{1,\ldots,30\}$.

\solution{16

Since 31 is prime, Euler's theorem implies 
$17^{31-2} \cdot 17 \equiv 1 \pmod{31}$ and so $\rem(17^{31-2},31)$ is
the inverse of 17 in the range $\{1,\ldots,30\}$. Using the method
of repeated squaring,
%
\[
\begin{array}{lcl}
17^{2}  & =      & 289\\
        & =      & 9  \cdot 31 + 10\\
        & \equiv & 10\\ &&\\
17^{4}  & \equiv & 10^2\\
        & =      & 100\\
        & =      & 3  \cdot 31 + 7\\
        & \equiv & 7\\ &&\\
17^{8}  & \equiv & 7^2\\
        & =      & 49\\
        & =      & 31 + 18\\
        & \equiv & 18\\ &&\\
17^{16} & \equiv & 18^2\\
        & =      & 324\\
        & \equiv & 14\\ &&\\
17^{29} & =      & 17^{16} \cdot 17^{8} \cdot 17^{4} \cdot 17^{1}\\
        & \equiv & 14 \cdot 18 \cdot 7 \cdot 17\\
				& =      & (2 \cdot 7) \cdot (3 \cdot 6) \cdot 7 \cdot 17 \\
				& =      & (2 \cdot 17) \cdot (7 \cdot 6) \cdot (3 \cdot 7) \\
				& \equiv & 3 \cdot 11 \cdot 21 \\
				& \equiv & 2 \cdot 21 \\
				& =      & 42 \\
				& \equiv & \fbox{$11$}
\end{array}
\]
%
where the modulus for each of the congruences is 31.
}


\ppart{4} Find the remainder of $34^{82248}$ divided by $83$.
\hint{Euler's theorem.}

\solution{77

Since $34=2 \cdot 17$ and $83$ are relatively prime,
Euler's theroem implies that $34^{\phi(83)} \equiv 1 \pmod{83}$ where
%
\begin{align*}
\phi(83) & = 82 \\
\end{align*}
%

Now, notice that $82248 = 82 \cdot 1003 + 2$.  But then, this implies that 
%
\begin{align*}
34^{82248} & = 34^2 \cdot 34^{1003 \cdot 82} & \\
             & \equiv 34^2 \cdot 1^{1003} \pmod{83} & \text{(by Euler's Theorem)} \\
             & = 1156 & \\
             & \equiv 77 \pmod{83}&
\end{align*}
}

\eparts

\end{problem}

%%%%%%%%%%%%%%%%%%%%%%%%%%%%%%%%%%%%%%%%%%%%%%%%%%%%%%%%%

%new problem
\begin{problem}{16}

Prove the following statements, assuming all numbers are positive integers.

\bparts

\ppart{4} If $a \mid b$, then $\forall c$, $a \mid bc$

\solution{
	If $a \mid b$, then there is some positive integer $k$ such that $b = ak$.  But then, $bc = akc = a(kc)$, which is a multiple of $a$.
}

\ppart{4} If $a \mid b$ and $a \mid c$, then $a \mid sb + tc$.
\solution{
	If $a \mid b$, then there is some positive integer $j$ such that $b = aj$.  Similarly, there is some positive integer $k$ such that $c = ak$.  But then, we can rewrite the right side
	as $s(aj) + t(ak)$.  But we can rewrite this as $a(js) + a(kt) = a(js + kt)$, which is a multiple of $a$.
}

\ppart{4} $\forall c $, $a \mid b \Leftrightarrow ca \mid cb$
\solution{
	If $a \mid b$, then there is some positive integer $k$ such that $b = ak$.  But then, we can rewrite $cb = c(ak) = ca(k)$, which is a multiple of $ca$.  So the implication is true.
}

\ppart{4} $\gcd(ka, kb) = k \gcd(a, b)$
\solution{
Let $s, t$ be coefficients so that $s(ka) + t(kb) = \gcd(ka, kb)$.  We can factor out the $k$ so that $\gcd(ka, kb) = k(sa + tb)$.  We now argue that $sa + tb = \gcd(a,b)$.  Suppose it were not.  Then, there is some smaller positive linear combination of $a,b$ with coefficients $s'$ and $t'$ so that $s'a + t'b = \gcd(a,b)$.  But then, if we multiply this by $k$, we find that
$0<ks'a + kt'b = s'(ka) + t'(kb) < s(ka) + t(kb) = \gcd(ka, kb)$.  This is a contradiction with the definition of the $\gcd$, so $sa + tb = \gcd(a,b)$, and we can conclude that
$\gcd(ka, kb) = k\gcd(a, b)$.
}
\eparts

\end{problem}

%%%%%%%%%%%%%%%%%%%%%%%%%%%%%%%%%%%%%%%%%%%%%%%%%%%%%%%%%

%new problem

\begin{problem}{20} In this problem, we will investigate numbers which are squares modulo a prime number $p$.

\bparts

\ppart{5} An integer $n$ is a square modulo $p$ if there exists another integer $x$ such that $n \equiv x^2  \pmod p$. Prove that $x^2 \equiv y^2 \pmod p$ if and only if $x \equiv y \pmod p$ or $x \equiv -y \pmod p$.
\hint{$x^2-y^2 = (x+y)(x-y)$}

\solution{$x^2 \equiv y^2 \pmod p$ iff
$p \mid x^2-y^2$.  But $x^2-y^2 = (x-y)(x+y)$, and
since $p$ is a prime, this happens iff either $p \mid x-y$ or $p
\mid x+y$,  which is iff $x \equiv y \pmod p$ or $x \equiv -y \pmod p$.
}

\ppart{5} There is a simple test we can perform to see if a number $n$ is a square modulo $p$. It states that
\begin{theorem}[Euler's Criterion]
:
\begin{enumerate}
\item If $n$ is a square modulo $p$ then $n^{\frac{p-1}{2}} \equiv 1 \pmod p$.
\item If $n$ is not a square modulo $p$ then $n^{\frac{p-1}{2}} \equiv -1 \pmod p$.
\end{enumerate} 
\end{theorem}

Prove the first part of Euler's Criterion.
\hint{Use Fermat's theorem.}

\solution{If $n$ is a square modulo $p$, then there exists an $x$ such that $x^2 \equiv n (mod p)$. Consequently,
$$a^{\frac{p-1}{2}} \equiv x^{p-1} \equiv 1 \pmod p$$
by Fermat's theorem.}

\ppart{10}
Assume that $p \equiv 3 \pmod 4$ and $n \equiv x^2 \pmod p$. Given $n$ and $p$, find one possible value of $x$.
\hint{Write $p$ as $p=4k+3$ and use Euler's Criterion. You might have to multiply two sides of an equation by $n$ at one point.}

\solution{From Euler's Criterion:
$$n^{\frac{p-1}{2}}\equiv 1 \pmod p.$$
We can write $p=4k+3$, so $\frac{p-1}{2} = \frac{4k+3-1}{2}=k+1$.
As a result,
$n^{2k+1} \equiv 1 \pmod p$,
so
$n^{2k+2} \equiv n \pmod p$.
This can be rewritten as $\big(n^{k+1}\big)^2 \equiv n \pmod p$, so
$$n^{k+1} = n^{\frac{p-3}{4}+1}$$
is one possible value of $x$.
}

\eparts

\end{problem}

\instatements{\vspace{.3in}}


%%%%%%%%%%%%%%%%%%%%%%%%%%%%%%%%%%%%%%%%%%%%%%%%%%%%%%%%%%%%%%%%%%%%%%%%%%%%%%%

%spring07
\begin{problem}{10}
Prove that for any prime, $p$, and integer, $k\geq 1$,
\[
\phi(p^k) = p^k-p^{k-1},
\]
where $\phi$ is Euler's function.
\hint{Which numbers between 0 and $p^{k}-1$ \emph{are} divisible by $p$?
How many are there?}

\solution{The numbers in the interval from 0 to $p^{k}-1$ that are
divisible by $p$ are all those of the form $mp$.  For $mp$ to be in the
interval, $m$ can take any value from 0 to $p^{k-1}-1$ and no others, so
there are exactly $p^{k-1}$ numbers in the interval that are divisible by
$p$.  Now $\phi(p^{k})$ equals the number of remaining elements in the
interval, namely, $p^k -p^{k-1}$.}
\end{problem}


%%%%%%%%%%%%%%%%%%%%%%%%%%%%%%%%%%%%%%%%%%%%%%%%%%%%

%New problem, modified
\begin{problem}{18}
Here is a {\em very, very fun} game.  We start with two distinct,
positive integers written on a blackboard.  Call them $x$ and $y$.
You and I now take turns.  (I'll let you decide who goes first.)  On
each player's turn, he or she must write a new positive integer on the
board that is a common divisor of two numbers that are already there.
If a player can not play, then he or she loses.

For example, suppose that $12$ and $15$ are on the board initially.  Your
first play can be $3$ or $1$. Then I play $3$ or $1$, whichever one you did not play.  Then you can
not play, so you lose.

\bparts

\ppart{6} Show that every number on the board at the end of the game is either $x$, $y$, or a positive
divisor of $\gcd(x, y)$.

\solution{
  We use induction. Let $g = \gcd(x,y)$. Let our inductive hypothesis be $P(n)  = $ ``After $n$ moves, every number on the board is either $x$, $y$, or a positive divisor
  of $g$.''  For $n=0$, only $x$ and $y$ are on the board, so $P(0)$ holds. For the inductive case, after $n+1$ moves the numbers on the board are the same as the numbers after $n$ moves plus
  an additional positive integer $m$ which is a divisor of two numbers $a$ and $b$ which were already on the board.  We must show that $m$ is either $x$, $y$, or a positive divisor of $g$. We know $m$ cannot be equal $x$ or $y$ because it must be a new number, and we know $m$ is positive, so we have to show that $m|g$. We will consider two cases:
\begin{enumerate}
\item $a=x$ and $b=y$\\
In this case, $m|a$ and $m|b$, so $a=km$ and $b=lm$ for some integers $k$ and $l$. We know we can write $g$ as a linear combination of $a$ and $b$:
$$sa+tb=g.$$
Substituting the expressions for for $a$ and $b$, we obtain
$$skm+tlm=g,$$
which means $m(sk+tl)=g$, so $m|g$ and $P(n+1)$ holds.
\item $a\neq x$ or $b \neq y$\\
In this case, by inductive assumption $a|g$ or $b|g$. Assume that $a|g$. Then $m|a$ and $a|g$, so $m|g$. If on the other hand $a \not|g$ then $b|g$ and $m|b$, so $m|g$. Again, $P(n+1)$ holds.
\end{enumerate}
By induction, every number on the board at the end of the game is either $x$, $y$, or a positive divisor of $gcd(x,y)$. 
  }

\ppart{6} Show that every positive divisor of $\gcd(x, y)$  is on the board at the end of the game.

\solution{
Proof by contradiction. Assume there is a number $d$ such that $d|gcd(x,y)$ and $d$ is not on the board at the end of the game. Since $d|gcd(x,y)$ and $gcd(x,y)|x$ and $gcd(x,y)|y$, therefore $d|x$ and $d|y$. But $x$ and $y$ are on the board, so it is possible to add $d$ to the board. This means the game is not over yet! We have reached a contradiction, so $d$ must be on the board. This is true for all positive divisors of $gcd(x,y)$, so all of them must be on the board.
}

\ppart{6} Describe a strategy that lets you win this game every time.

\solution{

  We showed that $x$, $y$, and all positive divisors of $gcd(x,y)$ and only those numbers will be on the board at the end of the game. Let $D$ be the number of positive divisors of $gcd(x,y)$. If $x=gcd(x,y)$ or $y=gcd(x,y)$, then $D-1$ values will be added to the board before the game ends. Otherwise, $D$ values will be added. You can calculate the number of values to be placed and if this number is odd, decide to
  go first. Otherwise, decide to go second.

}

\eparts

\end{problem}


%%%%%%%%%%%%%%%%%%%%%%%%%%%%%%%%%%%%%%%%%%%%%%%%%%%%%%%%%%%%
\begin{problem}{20} 
In one of the previous problems, you calculated square roots of numbers modulo primes  equivalent to $3$ modulo $4$. In this problem you will prove that there are an infinite number of such primes!

\bparts

\ppart{6} As a warm-up, prove that there are an infinite number of prime numbers.\\
\hint{Suppose that the set $F$ of all prime numbers is finite, that is $F=\{p_1,p_2,\ldots,p_k\}$ and define $n=p_1p_2\ldots p_k +1$.}

\solution{By contradiction. Suppose that $F$ is finite. Let it be $F=\{p_1,p_2,\ldots,p_k\}$ and define $n=p_1p_2\ldots p_k +1$. For every $p \in F$, $$n \equiv 1 \pmod p.$$  Consequently, $\forall p \in F$, $p \not| n$. But the numbers in $F$ are all the prime numbers, so it must be that for all primes $p$, $p \not| n$. As a result, $n$ does not have a prime factor smaller than itself, so $n$ is a prime number! But $n$ is definitely larger than any number in $F$, so $n \notin F$. This is a contradiction. The initial assumption that $F$ is finite is false.
}

\ppart{2} Prove that if $p$ is an odd prime, then $p\equiv 1 \pmod 4$ or $p\equiv 3 \pmod 4$.

\solution{By the division theorem, there exist  integers $x$ and $r$ with $0\leq r\leq 3$ such that $p=4x+r$. If $2|r$, then $2|p$. Since $p$ is odd, $2 \not| r$. So, $r=1$ or $r=3$.
}

\ppart{6} Prove that if $n\equiv 3 \pmod 4$, then $n$ has a prime factor $p\equiv 3 \pmod 4$.

\solution{By contradiction. Suppose the contrary that $n\equiv 3 \pmod 4$ and that, for all   primes $p | n$,  $p\not\equiv 3 \pmod 4$. By part b, if prime  $p\not\equiv 3 \pmod 4$, then $p=2$ or $p\equiv 1 \pmod 4$. Since $n\equiv 3 \pmod 4$, $n$ is odd and $2\not| n$. So, by the fundamental theorem in arithmetic, $n$ is a product of primes $p$ with $p\equiv 1 \pmod 4$. This means that $n\equiv 1 \pmod 4$. This contradicts the original assumption that $n \equiv 3 \pmod 4$.
}

\ppart{6} Let $F$ be the set  of all primes $p$ such that $p\equiv 3$ (mod 4). 
Prove by contradiction that $F$ has an infinte number of primes.

\hint{Suppose that $F$ is finite, that is $F=\{p_1,p_2,\ldots,p_k\}$ and define $n=4p_1p_2\ldots p_k-1$. Prove that there exists a prime $p_i\in F$ such that $p_i|n$.}

\solution{By contradiction. Suppose that $F$ is finite. Let it be $F=\{p_1,p_2,\ldots,p_k\}$ and define $n=4p_1p_2\ldots p_k -1$. Notice that $F$ is not empty since $3\in F$. This shows that $n$
is at least $0$. By part c, $n=4x-1$ has a prime factor $p_i\in F$ such that $p_i|n$. So, $n\equiv 0 \pmod {p_i}$. Also, $n=4p_1p_2\ldots p_k-1 = yp_i-1$. This means $n \equiv -1 \pmod {p_i}$. This is a contradiction. The initial assumption that $F$ is finite is false.
}

\eparts

\end{problem}
\end{document}
