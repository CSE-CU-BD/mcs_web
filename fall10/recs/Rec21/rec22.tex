\documentclass[12pt]{article}
\usepackage{../../light}

\hidesolutions
%\showsolutions

\begin{document}

\newcommand{\prsub}[2]{\mathop{\textup{Pr}_{#1}}\nolimits\left(#2\right)}

\recitation{21}{December 1, 2010}

%%%%%%%%%%%%%%%%%%%%%%%%%%%%%%%%%%%%%%%%%%%%%%%%%%%%%%%%%%%%%%%%%%%%%%%%%%%%%%%


\insolutions{
\section{Conditional Expectation and Total Expectation}

There are conditional expectations, just as there are conditional
probabilities.  If $R$ is a random variable and $E$ is an event, then
the conditional expectation $\ex{R \mid E}$ is defined by:
%
\[
\ex{R \mid E} = \sum_{w \in S} R(w) \cdot \pr{w \mid E}
\]
%
For example, let $R$ be the number that comes up on a roll of a fair
die, and let $E$ be the event that the number is even.  Let's compute
$\ex{R \mid E}$, the expected value of a die roll, given that the
result is even.
%
\begin{align*}
\ex{R \mid E} 
    & = \sum_{w \in \set{1, \ldots, 6} } R(w) \cdot \pr{w \mid E} \\
    & = 1 \cdot 0 + 2 \cdot \frac{1}{3} + 3 \cdot 0 +
        4 \cdot \frac{1}{3} + 5 \cdot 0 + 6 \cdot \frac{1}{3} \\
    & = 4
\end{align*}

It helps to note that the conditional expectation, $\ex{R \mid
E}$ is simply the expectation of $R$ with respect to the probability
measure $\prsub{E}{}$ defined in PSet 10.  So it's linear:
\[
\ex{R_1+R_2 \mid E} = \ex{R_1 \mid E} + \ex{R_2 \mid E}.
\]

Conditional expectation is really useful for breaking down the calculation
of an expectation into cases.  The breakdown is justified by an analogue
to the Total Probability Theorem:

\begin{theorem}[Total Expectation]
Let $E_1, \ldots, E_n$ be events that partition the sample space and
all have nonzero probabilities.  If $R$ is a random variable, then:
%
\[
\ex{R} = \ex{R \mid E_1} \cdot \pr{E_1} + \cdots +
         \ex{R \mid E_n} \cdot \pr{E_n}
\]
\end{theorem}

For example, let $R$ be the number that comes up on a fair die and $E$
be the event that result is even, as before.  Then $\overline{E}$ is
the event that the result is odd.  So the Total Expectation theorem
says:
%
\[
\underbrace{\ex{R}}_{=\ 7/2} = \underbrace{\ex{R \mid E}}_{=\ 4} \cdot \underbrace{\pr{E}}_{=\ 1/2} + \underbrace{\ex{R \mid \overline{E}}}_{=\ ?} \cdot \underbrace{\pr{E}}_{=\ 1/2}
\]
%
The only quantity here that we don't already know is $\ex{R \mid
\overline{E}}$, which is the expected die roll, given that the result is
odd.  Solving this equation for this unknown, we conclude that $\ex{R \mid
\overline{E}} =3$.

To prove the Total Expectation Theorem, we begin with a Lemma.

\begin{lemma*}
Let $R$ be a random variable, $E$ be an event with positive probability,
and $I_E$ be the indicator variable for $E$.  Then
\begin{equation}\label{RE}
\ex{R \mid E}  = \frac{\ex{R \cdot I_E}}{\pr{E}}
\end{equation}
\end{lemma*}

\begin{proof}
Note that for any outcome, $s$, in the sample space,
\[
\pr{\set{s} \cap E} = \begin{cases} 0 & \text{if }I_E(s) = 0,\\
                                \pr{s} &  \text{if }I_E(s) = 1,
\end{cases}
\]
and so
\begin{equation}\label{sE}
\pr{\set{s}\cap E} = I_E(s) \cdot \pr{s}.
\end{equation}

Now,
\begin{align*}
\ex{R \mid E} & = \sum_{s \in S} R(s)\cdot \pr{\set{s} \mid E} & \text{(Def of
              $\ex{\cdot \mid E}$)}\\
              & = \sum_{s \in S} R(s)\cdot  \frac{\pr{\set{s} \cap E}}{\pr{E}} &
              \text{(Def of $\pr{\cdot \mid E}$)}\\
              & = \sum_{s \in S} R(s)\cdot 
              \frac{I_E(s) \cdot \pr{s}}{\pr{E}}  & \text{(by~\eqref{sE})}\\
             & = \frac{\sum_{s \in S} (R(s)\cdot I_E(s)) \cdot 
              \pr{s}}{\pr{E}}\\
             & = \frac{\ex{R\cdot I_E}}{\pr{E}} & \text{(Def of $\ex{R\cdot I_E}$)}
\end{align*}

\end{proof}

Now we prove the Total Expectation Theorem:

\begin{proof}
Since the $E_i$'s partition the sample space,
\begin{equation}\label{R=}
R = \sum_i R\cdot I_{E_i}
\end{equation}
for any random variable, $R$.  So
\begin{align*}
\ex{R} & = \ex{\sum_i R\cdot I_{E_i}}  & \text{(by~\eqref{R=})}\\
       & = \sum_i \ex{R\cdot I_{E_i}}  & \text{(linearity of $\ex{}$)}\\
       & = \sum_i \ex{R \mid E_i} \cdot \pr{E_i} & \text{(by~\eqref{RE})}
\end{align*}

\end{proof}

\iffalse

\begin{proof} % meta-Q disaster :-(
\begin{align*}
\ex{R}
  & = \sum_{w \in S} R(w) \cdot \pr{w} & \text{def. of expectation}\\
  & = \sum_{w \in S} R(w) \cdot \paren{\sum_{k=1}^n \pr{w \mid E_k}
  \cdot \pr{E_k}} & \text{Total Probability} \\ & = \sum_{w \in S}
  \sum_{k=1}^n R(w) \cdot \pr{w \mid E_k} \cdot \pr{E_k} & \text{pull
  $R(w)$ into inner sum} \\ & = \sum_{k=1}^n \sum_{w \in S} R(w) \cdot
  \pr{w \mid E_k} \cdot \pr{E_k} & \text{swap sums} \\ & =
  \sum_{k=1}^n \pr{E_k} \cdot \paren{\sum_{w \in S} R(w) \cdot \pr{w
  \mid E_k}} & \text{pull $\pr{E_k}$ out} \\ & = \sum_{k=1}^n \pr{E_k}
  \cdot \ex{R \mid E_k} & \text{def. of cond. expectation}
\end{align*}
\end{proof}
\fi


\newpage
}

%%%%%%%%%%%%%%%%%%%%%%%%%%%%%%%%%%%%%%%%%%%%%%%%%%%%%%%%%%%%%%%%%%%%%%%%%%%%%%%

\begin{problem}{}
Here's yet another fun 6.042 game!  You pick a number between 1 and 6.
Then you roll three fair, independent dice.
%
\begin{itemize}
\item If your number never comes up, then you lose a dollar.
\item If your number comes up once, then you win a dollar.
\item If your number comes up twice, then you win two dollars.
\item If your number comes up three times, you win \textit{four} dollars!
\end{itemize}
%
What is your expected payoff?  Is playing this game likely to be
profitable for you or not?

\solution{Let the random variable $R$ be the amount of money won or
lost by the player in a round.  We can compute the expected value of
$R$ as follows:
%
\begin{align*}
\ex{R}  & =    -1  \cdot \pr{\text{0 matches}} +
                     1 \cdot \pr{\text{1 match}} +
                     2  \cdot \pr{\text{2 matches}} +
                     4 \cdot \pr{\text{3 matches}} \\
        & =    -1 \cdot \left(\frac{5}{6}\right)^3 +
                1 \cdot 3\left(\frac{1}{6}\right)\left(\frac{5}{6}\right)^2+
                2 \cdot 3\left(\frac{1}{6}\right)^2\left(\frac{5}{6}\right)+
                4 \cdot \left(\frac{1}{6}\right)^3 \\
        & =    \frac{-125 + 75 + 30 + 4}{216} \\
        & =    \frac{-16}{216}
\end{align*}
%
You can expect to lose $16/216$ of a dollar (about 7.4 cents) in every
round.  This is a horrible game!}
\end{problem}

%%%%%%%%%%%%%%%%%%%%%%%%%%%%%%%%%%%%%%%%%%%%%%%%%%%%%%%%%%%%%%%%%%%%%%%%%%%%%%%

\newpage

\begin{problem}{}
The number of squares that a piece advances in one turn of the game
Monopoly is determined as follows:

\begin{itemize}

\item Roll two dice, take the sum of the numbers that come up, and
advance that number of squares.

\item If you roll {\em doubles} (that is, the same number comes up on
both dice), then you roll a second time, take the sum, and advance
that number of additional squares.

\item If you roll doubles a second time, then you roll a third time,
take the sum, and advance that number of additional squares.

\item However, as a special case, if you roll doubles a third time,
then you go to jail.  Regard this as advancing zero squares overall
for the turn.

\end{itemize}

\end{problem}

\bparts

\ppart{} What is the expected sum of two dice, given that the same
number comes up on both?

\solution[\vspace{2in}]{There are six equally-probable sums: 2, 4, 6, 8, 10, and 12.
Therefore, the expected sum is:
%
\[
\frac{1}{6} \cdot 2 + \frac{1}{6} \cdot 4 + \ldots + \frac{1}{6} \cdot 12 = 7
\]
}

\ppart{} What is the expected sum of two dice, given that different
numbers come up?  (Use your previous answer and the Total Expectation
Theorem.)

\solution[\vspace{3in}]{Let the random variables $D_1$ and $D_2$ be the
numbers that come up on the two dice.  Let $E$ be the event that they are
equal.  The Total Expectation Theorem says:
%
\begin{align*}
\ex{D_1 + D_2}
	& = 	\ex{D_1 + D_2 \mid E} \cdot \pr{E} + 
		\ex{D_2 + D_2 \mid \overline{E}} \cdot \pr{\overline{E}} \\
\end{align*}
%
Two dice are equal with probability $\pr{E} = 1 / 6$, the expected sum
of two independent dice is 7, and we just showed that $\ex{D_1 +
D_2 \mid E} = 7$.  Substituting in these quantities and solving the
equation, we find:
%
\begin{align*}
7	& = 	7 \cdot \frac{1}{6} + 
		\ex{D_2 + D_2 \mid \overline{E}} \cdot \frac{5}{6} \\
\ex{D_2 + D_2 \mid \overline{E}}
	& = 	7
\end{align*}
}

\instatements{\newpage}

\ppart{} To simplify the analysis, suppose that we always roll the dice
three times, but may ignore the second or third rolls if we didn't
previously get doubles.  Let the random variable $X_i$ be the sum of
the dice on the $i$-th roll, and let $E_i$ be the event that the
$i$-th roll is doubles.  Write the expected number of squares a piece
advances in these terms.

\solution[\vspace{3in}]{
From the total expectation formula, we get:
%
\begin{align*}
\ex{\text{advance}}
	& = 	\ex{X_1 \mid \overline{E_1}}
			\cdot \pr{\overline{E_1}} \\
	& \quad	+ \ex{X_1 + X_2 \mid E_1 \cap \overline{E_2}}
			\cdot \pr{E_1 \cap \overline{E_2}} \\
	& \quad	+ \ex{X_1 + X_2 + X_3 \mid E_1 \cap E_2 \cap
			  \overline{E_3}}
			\cdot \pr{E_1 \cap E_2 \cap \overline{E_3}} \\
	& \quad	+ \ex{0 \mid E_1 \cap E_2 \cap E_3}
			\cdot \pr{E_1 \cap E_2 \cap E_3} \\
\end{align*}
Then using linearity of (conditional) expectation, we refine this to
\begin{align*}
\lefteqn{\ex{\text{advance}}}\\
	& = 	\ex{X_1 \mid \overline{E_1}}
			\cdot \pr{\overline{E_1}} \\
	& \quad	+ \paren{\ex{X_1  \mid E_1 \cap \overline{E_2}}
                          + \ex{X_2 \mid E_1 \cap \overline{E_2}}}
			\cdot \pr{E_1 \cap \overline{E_2}} \\
	& \quad	+ \paren{\ex{X_1 \mid E_1 \cap E_2 \cap
			  \overline{E_3}} + \ex{X_2 \mid E_1 \cap E_2 \cap
			  \overline{E_3}} +
             \ex{X_3 \mid E_1 \cap E_2 \cap \overline{E_3}}}\\
        &          \qquad \cdot \pr{E_1 \cap E_2 \cap \overline{E_3}}\\
	& \quad	+ 0.
\end{align*}
Using mutual independence of the rolls, we simplify this to
\begin{align}
\lefteqn{\ex{\text{advance}}}\notag\\
	& = 	\ex{X_1 \mid \overline{E_1}}
			\cdot \pr{\overline{E_1}}\label{exa} \\
	& \quad	+ \paren{\ex{X_1  \mid E_1}
                          + \ex{X_2 \mid \overline{E_2}}}
			\cdot \pr{E_1} \cdot \pr{\overline{E_2}}\notag \\
	& \quad	+ \paren{\ex{X_1 \mid E_1} + \ex{X_2 \mid E_2} +
             \ex{X_3 \mid \overline{E_3}}}
			\cdot \pr{E_1} \cdot \pr{E_2} \cdot \pr{\overline{E_3}}\notag
\end{align}
}

\iffalse

ERIC: There's a bunch of defs and facts being used here that we haven't
mentioned explicitly.  We should remark that an RV, $R$, is independent of an
event, $E$, iff $R$ and $I_E$ are independent RV's.  Next we need that if
$R$ is independent of $E$, then $\ex{R \mid E} = \ex{R}$:
\begin{proof}
\begin{align*}
 ex{R \mid E} &  = \frac{\ex{R \cdot I_E}}{\pr{E}} & \text{(by~\eqref{RE})}\\
              & =  \frac{\ex{R} \cdot \ex{I_E}}{\pr{E}}
                        & \text{(independence of $R$ from $E$)}\\
              & =  \frac{\ex{R} \cdot \pr{E}}{\pr{E}}
                        & \text{(rule for Ex(indicator))}\\
              & = \ex{R}.
\end{align*}
\end{proof}
More generally, we need that $\ex{R \mid (E \cap F)} = \ex{R \mid F}$ when
$R$ and $F$ are both indep of $E$ -- or something like that I haven't had
time to check.
\fi

\ppart{} What is the expected number of squares that a piece advances in
Monopoly?

\solution[\vspace{3in}]{We plug the values
from parts (a) and (b) into equation~\eqref{exa}:
%
\begin{align*}
\ex{\text{advance}}
	& = 	7 \cdot \frac{5}{6}
		+ (7 + 7) \cdot \frac{1}{6} \cdot \frac{5}{6}
		+ (7 + 7 + 7) \cdot \frac{1}{6} \cdot \frac{1}{6} \cdot \frac{5}{6}\\
	& = 8\frac{19}{72}
\end{align*}
}

\eparts

\end{document}
