\documentclass[12pt]{article}
\usepackage{light}

\hidesolutions
%\showsolutions

\newcommand{\edge}[2]{#1\text{---}#2}
\newcommand{\mfigure}[3]{\bigskip\centerline{\resizebox{#1}{#2}{\includegraphics{#3}}}\bigskip}
\newcommand{\eqdef}{\mathbin{::=}}

\begin{document}

\recitation{7}{October 1, 2010}


\section{A Protocol for College Admission}

Next, we are going to talk about a generalization of the stable
marriage problem. Recall that we have some horses and we'd like to
pair them with stables so that there is no incentive for two horses to
swap stables. Oh wait, that's a different problem.

The problem we're going to talk about is a generalization of the one
done in lecture. In the new problem, there are $N$ students $s_1, s_2,
\ldots, s_N$ and $M$ universities $u_1, u_2, \ldots, u_M$. University
$u_i$ has $n_i$ slots for students, and we're guaranteed that
$\sum_{i=1}^M n_i = N$. Each student ranks all universities (no ties)
and each university ranks all students (no ties).

Design an algorithm to assign students to universities with the following properties
\begin{enumerate}
\item Every student is assigned to one university.
\item $\forall i$, $u_i$ gets assigned $n_i$ students.
\item There does not exist $s_i, s_j, u_k, u_{\ell}$ where $s_i$ is assigned to $u_k$, $s_j$ is assigned to $u_{\ell}$, $s_j$ prefers $u_k$ to $u_{\ell}$, and $u_k$ prefers $s_j$ to $s_i$. 
\item It is student-optimal. This means that of all possible assignments satisfying the first three properties, the students get their top choice of university amongst these assignments.
\end{enumerate}
The algorithm will be a slight modification of the mating algorithm given in lecture. For your convenience, we have provided a copy of the mating algorithm on the next page.
\newpage
\noindent
{\bf Each Day:}

\begin{itemize}
\item Morning:
\begin{itemize}
\item Each girl stands on her balcony
\item Each boy stands under the balcony of his favorite girl
whom he has not yet crossed off his list and serenades.
If there are no girls left on his list, he stays home and does 6.042 homework.
\end{itemize}
\item Afternoon:
\begin{itemize}
\item Girls who have at least one suitor say to their favorite
from among the suitors that day:  ``Maybe, come back tomorrow.''
\item To the others, they say ``No, I will never marry you!''
\end{itemize}
\item Evening:
\begin{itemize}
\item Any boy who hears ``No'' crosses that girl off his list.
\end{itemize}
\end{itemize}

\noindent
{\bf Termination Condition:}
If there is a day when every girl has at most one suitor, we stop
and each girl marries her current suitor (if any).

\newpage
\solution{

\noindent
{\bf Each Day:}

\begin{itemize}
\item Morning:
\begin{itemize}
\item Each university asks which students are interested in applying.
\item Each student applies to his/her favorite university that has not yet rejected him/her.
If there are no universities left on the student's list, the student takes some time off to think about life, the future, and the unchangeable past.
\end{itemize}
\item Afternoon:
\begin{itemize}
\item Universities $u_i$ do the following:
\item $u_i$ tells its favorite $n_i$ applicants ``Maybe, we are still processing your application.'' If $u_i$ has less than $n_i$ applicants, it tells all of its applicants this message.
\item If $u_i$ has more than $n_i$ applicants, it tells the remaining ones ``Sorry, there were a large number of very qualified students applying this year, yet we can only accept a very limited number. We regret to inform you that you were not accepted. Thank you for applying to our university.''
\end{itemize}
\item Evening:
\begin{itemize}
\item Any student who hears ``Sorry, $\ldots$'' from some university, crosses off that university from his/her list.
\end{itemize}
\end{itemize}
\noindent
{\bf Termination Condition:}
If there is a day when each university $u_i$ has at most $n_i$ applicants, we stop and each university accepts all of its applicants (if any).
}
\begin{enumerate}
\item Before we can say anything about our algorithm, we need to show that it terminates. Show that the algorithm terminates after $NM+1$ days. 

\solution{On each day, if the algorithm has not terminated, then some university $u_i$ has more than $n_i$ applicants. It follows that in the afternoon, at least one student $s_j$ hears ``Sorry, $\ldots$'', and thus in the evening $s_j$ crosses off $u_i$ from his/her list. As there are $N$ students and $M$ universities, it follows that the algorithm must terminate after $NM+1$ days, as otherwise there would be no university left for any student to cross off.}

\item Next, we will show that the four properties stated earlier are true of our algorithm. To start, let's show the following: if during some day a university $u_j$ has at least $n_j$ applicants, then when the algorithm terminates it accepts exactly $n_j$ students.

\solution{At this day, each of the students applying to $u_j$ has $u_j$ as their favorite university that has not yet rejected him/her. Therefore, if $u_j$ tells a student ``Maybe, $\ldots$'', that student will come back the next day. Since there are at least $n_j$ applicants, it follows that $u_j$ will tell its favorite $n_j$ ``Maybe, $\ldots$''. It follows by induction that every day after this day, $u_j$ will have at least $n_j$ applicants. Thus, this holds when the algorithm terminates. Since when the algorithm terminates there are at most $n_j$ applicants, it follows that exactly $n_j$ students are assigned to $u_j$.}

\item Next, show that every student is assigned to one university.

\solution{It is clear that no student can apply to more than one university at once since a student applies to at most one university on any given day, so this means the students can be assigned to at most one university. So we just need to show that each student is assigned to at least one university. We argue by contradiction. 

Suppose not, and let $s_j$ be a student not assigned to any
university. Then, since the algorithm terminates, and when the
algorithm terminates each university $u_i$ accepts at most $n_i$
students, it follows that some university $u_i$ accepts less than
$n_i$ students. By the previous problem, it follows that in every day,
$u_i$ had less than $n_i$ applicants. But then consider the day that
$s_j$ applied to $u_i$. Since there were less than $n_i$ applicants to
$u_i$ that day, it follows that $u_i$ would have told $s_j$ ``Maybe,
$\ldots$'' in that day, and thus in every future day. Thus, $s_j$
would be assigned $u_i$ when the algorithm terminates. This is a
contradiction.  }

\item Next, show that for all $i$, $u_i$ gets assigned $n_i$ students. 

\solution{Since the algorithm terminates, on some day each $u_i$ gets assigned at most $n_i$ students. Suppose some $u_i$ got assigned strictly less than $n_i$ students. Since $\sum_{i=1}^M u_i = N$, this means that some student is not assigned. This contradicts the previous problem.}

\item 

Before continuing, we need to establish the following
property. Suppose that on some day a university $u_j$ has at least
$n_j$ applicants. Define the {\it rank} of an applicant $s_i$ with
respect to a university $u_j$ as $s_i$'s location on $u_j$'s
preference list. So, for example, $u_j$'s favorite student has rank
$1$. Show that the rank of $u_j$'s least favorite applicant that it
says ``Maybe, $\ldots$'' to cannot decrease (e.g., going from 1000 to
1005 is decreasing) on any future day. Note that $u_j$'s least
favorite applicant might change from one day to the next.


\solution{

On the next day, there are two cases: $u_j$ either says ``Maybe,
$\ldots$'' to its least favorite applicant $s_i$ from the previous
day, or it says ``Sorry, $\ldots$'' to $s_i$. In the first case, this
means that all of the $n_j-1$ applicants $u_j$ liked more than $s_i$
on the previous day will also be told ``Maybe, $\ldots$'', and so
$s_i$ will again be $u_j$'s least favorite applicant it did not
reject. Thus, the rank of its least favorite applicant did not
decrease. In the second case, this means that there were at least
$n_j$ applicants that $u_j$ preferred to $s_i$, and thus the rank of
its least favorite applicant it said ``Maybe, $\ldots$'' to did not
decrease. As shown above, on any future day $u_j$ has at least $n_j$
applicants, and so by applying this analysis again, we conclude that
the rank of $u_j$'s least favorite applicant that it says ``Maybe,
$\ldots$'' to cannot decrease on any future day.  }

\item 

Next, show there does not exist $s_i, s_j, u_k,$ and $u_{\ell}$ where
$s_i$ is assigned to $u_k$, $s_j$ is assigned to $u_{\ell}$, $s_j$
prefers $u_k$ to $u_{\ell}$, and $u_k$ prefers $s_j$ to $s_i$. Note
that this is analogous to a ``rogue couple'' considered in lecture.

\solution{

Assume, towards a contradiction, that such an $s_i, s_j, u_k$ and
$u_{\ell}$ existed. Since $s_j$ prefers $u_k$ to $u_{\ell}$, but is
assigned to $u_{\ell}$, on some day $u_k$ told $s_j$ ``Sorry,
$\ldots$''. On that day, there must have been more than $n_k$
applicants to $u_k$. If $s_i$ was also an applicant to $u_k$ on that
day, then $s_i$ would have also been rejected since $u_k$ prefers
$s_j$ to $s_i$, and thus $s_i$ could not have been assigned to
$u_k$. On the other hand, if at any later day $s_i$ were to apply to
$u_k$, it would have been rejected since $s_i$'s rank is less than
$s_j$'s with respect to $u_k$, since by the previous problem we know
that the rank of the least favorite applicant that $u_k$ says ``Maybe,
$\ldots$'' to, cannot decrease. Thus, it is impossible for $s_i$ to be
assigned to $u_k$, which is a contradiction.  }

\item Finally, we show in a very precise sense that this algorithm is {\em student-optimal}. As in lecture, define the {\it realm of possibility} of a student to be the set of all universities $u$, for which there exists some assignment satisfying the first three properties above, in which the student is assigned to $u$. Of all universities in the realm of possibility of a student we say that the student's favorite is {\it optimal} for that student.

Show that each student is assigned to its optimal university.

\solution{We argue by contradiction. Consider the first (in time) student $s_i$ not assigned to its optimal university, and suppose for $s_i$ this university is $u_k$. Then on the date $s_i$ is rejected from $u_k$, there was another student $s_j$ which $u_k$ preferred to $s_i$. Since $u_k$ is in the realm of possibility of $s_i$, there is an assignment $M$ of students to universities assigning $s_i$ to $u_k$ with the properties above. Suppose $M$ assigns $s_j$ to $u_{\ell}$. Suppose $u_{opt}$ is $s_j$'s optimal university. Then, since $s_i$ was the first student not assigned its optimal university, $s_j$ prefers $u_k$ to $u_{opt}$, though $u_k$ may equal $u_{opt}$. On the other hand, $s_j$ prefers $u_{opt}$ to $u_{\ell}$, since $u_{opt}$ is its favorite university in its realm of possibility, and $u_{\ell}$ occurs in its realm of possibility. It follows that $s_j$ prefers $u_k$ to $u_{\ell}$. But then in the assignment $M$ we have found an $s_i, s_j, u_k$, and $u_{\ell}$ with $s_i$ assigned to $u_k$, $s_j$ assigned to $u_{\ell}$, $s_j$ prefers $u_k$ to $u_{\ell}$, and $u_k$ prefers $s_j$ to $s_i$. This is a contradiction to the property of $M$ established in the previous problem.
}
\end{enumerate}
\end{document}

