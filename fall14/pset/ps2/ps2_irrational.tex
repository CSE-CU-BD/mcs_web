\documentclass[twoside,12pt]{article}
\newcommand{\tab}{\hspace*{2em}}
\usepackage{light}
\usepackage{subfigure}
\usepackage{graphicx}
\usepackage{amsmath}
\usepackage{verbatim}

\usepackage{amsfonts}

\newcommand{\lr}{l_{right}}
\renewcommand{\ll}{l_{left}}

\newcommand{\hint}[1]{({\it Hint: #1})}
\newcommand{\card}[1]{\left|#1\right|}
\newcommand{\union}{\cup}
\newcommand{\lgunion}{\bigcup}
\newcommand{\intersect}{\cap}
\newcommand{\lgintersect}{\bigcap}
\newcommand{\cross}{\times}


%\hidesolutions
\showsolutions

\newlength{\strutheight}
\newcommand{\prob}[1]{\mathop{\textup{Pr}} \nolimits \left( #1 \right)}
\newcommand{\prsub}[2]{\mathop{\textup{Pr}_{#1}}\nolimits\left(#2\right)}
\newcommand{\prcond}[2]{%
  \ifinner \settoheight{\strutheight}{$#1 #2$}
  \else    \settoheight{\strutheight}{$\displaystyle#1 #2$} \fi %
  \mathop{\textup{Pr}}\nolimits\left(
    #1\,\left|\protect\rule{0cm}{\strutheight}\right.\,#2
  \right)}
\newcommand{\cE}{\mathcal{E}}
\renewcommand{\setminus}{-}
\renewcommand{\complement}[1]{\overline{#1}}

\providecommand{\abs}[1]{\lvert#1\rvert}

\begin{document}
\problemset{2}{September 9, 2014}{Monday, September 15}

%%%%%%%%%%%%%%%%%%%%%%%%%%%%%%%%%%%%%%%%%%%%%%%%%%%%%%%%%%%%%%%%%%%%%%%%%%%%%%%


\begin{problem}{30}
Can raising an irrational number $a$ to an irrational power $b$ result in a
rational number? Provide a proof that it can by considering $\sqrt{3}^{\sqrt{2}}$
and using proof by cases.
\solution{
This proof is by cases. We will consider $\sqrt{3}^{\sqrt{2}}$, and there are
two cases.

\begin{itemize}

\item $\sqrt{3}^{\sqrt{2}}$ is rational. In this case, since $\sqrt{3}$ is
irrational and $\sqrt{2}$ is irrational, we have found an $a$ and $b$ such that
both are irrational and $a^b$ is rational.

\item $\sqrt{3}^{\sqrt{2}}$ is irrational. In this case, notice that
$(\sqrt{3}^{\sqrt{2}})^{\sqrt{2}} = \sqrt{3}^{2} = 3$. Since $\sqrt{3}^{\sqrt{2}}$
and $\sqrt{2}$ are both irrational and $3$ is rational, we have found an
$a$ and $b$.

\end{itemize}

In all cases, we have found some irrational $a$ such that when raised to an
irrational power $b$, $a^b$ is rational.
}
\end{problem}


\end{document}
