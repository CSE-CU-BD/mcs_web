\documentclass[12pt,twoside]{article}   
\usepackage{light}
%\hidesolutions
\showsolutions

\begin{document}

\problemset{2}{September 14, 2010}{Monday, September 20 (7pm)}

%%%%%%%%%%%%%%%%%%%%%%%%%%%
%%%%% 3-chains

\begin{problem}{12}
Define a \textit{3-chain} to be a (not necessarily contiguous) subsequence of three integers, which is either monotonically increasing or monotonically decreasing.  We will show here that any sequence of five distinct integers will contain a \textit{3-chain}.  Write the sequence as $a_1, a_2, a_3, a_4, a_5$. Note that a monotonically increasing sequences is one in which each term is greater than or equal to the previous term. Similarly, a monotonically decreasing sequence is one in which each term is less than or equal to the previous term. Lastly, a subsequence is a sequence derived from the original sequence by deleting some elements without changing the location of the remaining elements.

\bparts

\ppart{4}  Assume that $a_1<a_2$. Show that if there is no \textit{3-chain} in our sequence, then $a_3$ must be less than $a_1$.  (Hint: consider $a_4$!)
\solution{
    We first assume that $a_1 < a_2$.  Now consider where $a_3$ must be placed.  If $a_3 > a_2$, then $a_1<a_2<a_3$, so a \textit{3-chain} is created. So $a_3 < a_2$.  Now, consider the case where $a_3 > a_1$ as well. In that case, there is no place for $a_4$ to go without creating a \textit{3-chain}.  If $a_4 < a_3$, then $a_4 < a_3 < a_2$, so a \textit{3-chain} is made.  If
    $a_4 > a_3$, then $a_1 < a_3 < a_4$, so a \textit{3-chain is formed as well.  So $a_3$ must be less than $a_1$.}
    } 

\ppart{2}  Using the previous part, show that if $a_1<a_2$ and there is no \textit{3-chain} in our sequence, then $a_3 < a_4 < a_2$.

\solution{
    From the previous part, we have that $a_3 < a_1 < a_2$.  In this case, $a_4$ cannot be less than $a_3$, because then $a_4 < a_3 < a_2$.  Similarly, 
    $a_4$ cannot be greater than $a_2$, because then $a_1 < a_2 < a_4$.  So we conclude that $a_3 < a_4 < a_2$.
}

\ppart{2}  Assuming that $a_1<a_2$ and $a_3<a_4<a_2$, show that any value of $a_5$ must result in a \textit{3-chain}.  
\solution{
    Consider how $a_5$ compares to the other numbers in the sequence.  If $a_5 < a_4$, then $a_5 < a_4 < a_2$, and a \textit{3-chain} is formed.  If $a_5 > a_4$, then $a_3<a_4<a_5$, so a 
    \textit{3-chain} is formed again.  And finally, $a_5$ cannot equal $a_4$, because it was stipulated that the numbers were all distinct.  So any value of $a_5$ must result in a        
    \textit{3-chain}.
}

\ppart{4}  Using the previous parts, prove by contradiction that any sequence of five distinct integers must contain a \textit{3-chain}.
\solution{
    From the previous parts, we see that assuming that $a_1 < a_2$, any choice of $a_5$ must result in a \textit{3-chain}.  However, the other case, that $a_1 > a_2$, results in the same 
    argument; one only needs to reverse the $>$ and $<$ signs.  So any choice of $a_5$ results in a \textit{3-chain} in that case as well.  What we have been doing is assuming that we 
    could construct such a five-integer sequence, and going through the steps to construct it.  Because the only results from such a construction contain \textit{3-chains}, we reach a 
    contradiction, and conclude that all sequences of five distinct integers must contain a \textit{3-chain}.
}

\eparts

\end{problem}


\begin{problem}{8}

Prove by either the Well Ordering Principle or induction that for all nonnegative integers, $n$: 
\begin{equation}\label{sum-to-n}
\sum_{i=0}^{n} i^3 = \left(\frac{n(n+1)}{2}\right)^2.
\end{equation}

\solution{
\noindent{\bf Proof by Well Ordering Principle} 
\begin{proof}
The proof is by contradiction and use of the Well Ordering Principle. Assume that the theorem is false. Then, some nonnegative integers serve as counterexamples to it.
Let�s collect these counterexamples in a set:
$C ::= \{ n \in \mathbb{N} | \sum_{i=0}^{n} i^3 \neq \left(\frac{n(n+1)}{2}\right)^2 \}$.

By our assumption that the theorem admits counterexamples, $C$ is a nonempty set
of nonnegative integers. So, by the Well Ordering Principle, $C$ has a minimum
element, call it $c$. That is, $c$ is the smallest counterexample to the theorem.

Since $c$ is the smallest counterexample, we know that equation \eqref{sum-to-n} is false for $n = c$ but true for all nonnegative integers $n < c$. But equation \eqref{sum-to-n} is true for $n = 0$ since $\sum_{i=0}^{0} i^3 = 0 = \left(\frac{0(0+1)}{2}\right)^2$. Hence $c > 0$. This means $c - 1$ is a nonnegative integer, and since it is less than $c$, equation \eqref{sum-to-n} is true for $c - 1$. That is,

\begin{equation}\label{sum-to-c-1}
\sum_{i=0}^{c-1} i^3 = \left(\frac{(c-1)c}{2}\right)^2.
\end{equation}

But then, adding $c^3$ to both sides of equation \eqref{sum-to-c-1} gives us
\[
\sum_{i=0}^{c} i^3
\]
on the left hand side. And the right hand side now equals
\begin{align*}
\left(\frac{(c-1)c}{2}\right)^2 + c^3 &= \frac{(c-1)^2c^2+4c^3}{2^2}\\
 &= \frac{c^2\left((c-1)^2+4c \right)}{2^2}\\
&= \frac{c^2\left(c^2-2c+1+4c \right)}{2^2}\\
&= \frac{c^2\left(c^2+2c+1 \right)}{2^2}\\
&= \frac{c^2(c+1)^2}{2^2}\\
&= \left(\frac{c(c+1)}{2}\right)^2.
\end{align*}

That is,
\[
\sum_{i=0}^{c} i^3 = \left(\frac{c(c+1)}{2}\right)^2,
\]
which means that equation \ref{sum-to-n} does hold for c, after all! This is a contradiction, and we are done.

\end{proof}

\noindent {\bf Proof by Induction} 
\begin{proof}
The proof is by induction on $n$.  Let $P(n)$ be the
proposition that equation \ref{sum-to-n} holds.

{\bf Base case:}  $P(0)$ is true because 

\[
\sum_{i=0}^{0} i^3 = 0 = \left(\frac{0(0+1)}{2}\right)^2.
\]

{\bf Inductive step:}  Assume $P(n)$ is true, that is $\sum_{i=0}^{n} i^3 = \left(\frac{n(n+1)}{2}\right)^2$.  Then we
can prove $P(n+1)$ is also true as follows:
\begin{align*}
\sum_{i=0}^{n+1} i^3 &=\sum_{i=0}^{n} i^3 + (n+1)^3\\
&= \left(\frac{n(n+1)}{2}\right)^2 + (n+1)^3 \\
&= \frac{n^2(n+1)^2+4(n+1)^3}{2^2} \\
&= \frac{(n+1)^2\left(n^2+4(n+1)\right)}{2^2} \\
 &= \frac{(n+1)^2\left(n^2+4n+4\right)}{2^2} \\
&= \frac{(n+1)^2(n+2)^2}{2^2} \\
&= \left(\frac{(n+1)(n+2)}{2}\right)^2
\end{align*}

The first step breaks up the sum. The second step uses the assumption $P(n)$. The rest of the steps are algebraic simplifications.

Thus, $P(0)$ is true and $P(n)$ implies $P(n+1)$ for all nonnegative integers.
Therefore, $P(n)$ is true for all nonegative integers by the principle of
induction.

\end{proof}

}
\end{problem}


\begin{problem}{20}
The following problem is fairly tough until you hear a certain
one-word clue. The solution is elegant but is slightly tricky, so don't hesitate to ask for hints!

During 6.042, the students are sitting in
an $n\times n$ grid. A sudden outbreak of beaver flu (a rare variant of bird flu that lasts forever; symptoms include yearning for problem sets and craving for ice cream study sessions) causes some students to get infected. Here is
an example where $n = 6$ and infected students are marked $\times$.

\[
\begin{array}{|c|c|c|c|c|c|}
\hline
\times& & & &\times& \\ \hline
 &\times& & & & \\ \hline
& &\times&\times& & \\ \hline
& & & & & \\ \hline
& &\times& & & \\ \hline
& & &\times& &\times \\ \hline
\end{array}
\]

\noindent Now the infection begins to spread every minute (in discrete time-steps). Two students are considered \textit{adjacent} if they
share an edge (i.e., front, back, left or right, but NOT diagonal); thus, each student is adjacent to 2, 3 or 4 others.  A
student is infected in the next time step if either

\begin{itemize}
\item the student was previously infected (since beaver flu lasts forever), or
\item the student is adjacent to \textit{at least two} already-infected students.
\end{itemize}

In the example, the infection spreads as shown below.
%
\[
\begin{array}{|c|c|c|c|c|c|}
\hline
\times& & & &\times& \\ \hline
 &\times& & & & \\ \hline
& &\times&\times& & \\ \hline
& & & & & \\ \hline
& &\times& & & \\ \hline
& & &\times& &\times \\ \hline
\end{array}
\Rightarrow
\begin{array}{|c|c|c|c|c|c|}
\hline
\times&\times& & &\times& \\ \hline
\times&\times&\times& & & \\ \hline
&\times&\times&\times& & \\ \hline
& &\times& & & \\ \hline
& &\times&\times& & \\ \hline
& &\times&\times&\times&\times \\ \hline
\end{array}
\Rightarrow
\begin{array}{|c|c|c|c|c|c|}
\hline
\times&\times&\times& &\times& \\ \hline
\times&\times&\times&\times& & \\ \hline
\times&\times&\times&\times& & \\ \hline
&\times&\times&\times& & \\ \hline
& &\times&\times&\times& \\ \hline
& &\times&\times&\times&\times \\ \hline
\end{array}
\]
%
In this example, over the next few time-steps, all the students in class become infected.

\begin{theorem*}
If fewer than $n$ students in class are initially infected, the whole class will never be completely infected.
\end{theorem*}

Prove this theorem.

{\em Hint: When one wants to understand how a system such as the above ``evolves" over time, it is usually a good strategy to (1) identify an appropriate  property of the system at the initial stage, and (2) prove, by induction on the number of time-steps, that the property is preserved at every time-step. So look for a property (of the set of infected students) that remains invariant as time proceeds.}

If you are stuck, ask your recitation instructor for the one-word clue and even more hints!

\solution{
\begin{proof}
Define the {\em perimeter} of an infected set of students to be the number of
edges with infection on exactly one side.  Let $I$ denote the
perimeter of the initially-infected set of students.

Now we use induction on the number of time steps to prove that the
perimeter of the infected region never increases.  Let $P(k)$ be the
proposition that after $k$ time steps, the perimeter of the infected
region is at most $I$.

{\bf Base case:} $P(0)$ is true by definition; the
perimeter of the infected region is at most $I$ after 0 time steps,
because $I$ is defined to be the perimeter of the initially-infected
region.

{\bf Inductive step:} Now we must show that $P(k)$ implies
$P(k+1)$ for all $k \geq 0$.  So assume that $P(k)$ is true, where $k \geq
0$; that is, the perimeter of the infected region is at most $I$ after $k$
steps.  The perimeter can only change at step $k + 1$ because some squares
are newly infected.  By the rules above, each newly-infected square is
adjacent to at least two previously-infected squares.  Thus, for each
newly-infected square, at least two edges are removed from the perimenter
of the infected region, and at most two edges are added to the perimeter.
Therefore, the perimeter of the infected region can not increase and is at
most $I$ after $k + 1$ steps as well.  This proves that $P(k)$ implies
$P(k+1)$ for all $k \geq 0$.

By the principle of induction, $P(k)$ is true for all $k \geq 0$.

If an $n \times n$ grid is completely infected, then the perimeter of
the infected region is $4n$.  Thus, the whole grid can become infected
only if the perimeter is initially at least $4n$.  Since each square
has perimeter 4, at least $n$ squares must be infected initially for the whole grid to be infected.
\end{proof}

The above proof shows that if initially $k$ students  are infected, then the perimeter of the infected region will never exceed $4k$. The largest number of students that can be contained within a region with perimeter $\leq 4k$ is equal to $k^2$, therefore, if $k$ students in class are initially infected, then at most $k^2$ students will eventually be infected. This feels intuitively true after having done the previous proof. However, to give a formal proof requires some case analysis (try it!).   
}

\end{problem}




%%%%%%%%%%%%%%%%%%%%%%%%%%%%%%%%%%%%%%%%%%%%%%%%%%%%%%%%%%%%%%%%%%%%%%%
%%flawed induction proof
\newcommand{\naturals}{\mathbb{N}}

\begin{problem}{10}
  Find the flaw in the following \emph{bogus} proof that $a^n = 1$ for all
  nonnegative integers $n$, whenever $a$ is a nonzero real number.

\begin{proof}
The \emph{bogus} proof is by induction on $n$, with hypothesis
\[
P(n)::= \forall k \leq n.\, a^k = 1,
\]
where $k$ is a nonnegative integer valued variable.

\textbf{Base Case:} $P(0)$ is equivalent to $a^0 =1$, which is true by
definition of $a^0$.  (By convention, this holds even if $a=0$.)

\textbf{Inductive Step:} By induction hypothesis, $a^k = 1$ for all $k \in
\mathbb{N}$ such that $k \leq n$.  But then
\[
a^{n+1} = \frac{a^n \cdot a^n}{a^{n-1}} = \frac{1 \cdot 1}{1} = 1,
\]
which implies that $P(n+1)$ holds.  It follows by induction that
$P(n)$ holds for all $n \in \mathbb{N}$, and in particular, $a^n = 1$
holds for all $n \in \naturals$.
\end{proof}

\solution{

{\em NOTE TO GRADERS}  A student may decide the flaw is in the base case because by extending the basecase to include $P\left(0\right)$ and $P\left(1\right)$ we could get a valid base case/induction step combination. If this is their argument, then mark it as correct. 

The more solid argument though is the following, since the inductive hypothesis was given in 'strong form': 

The flaw comes in the \emph{inductive step}, where we implicitly assume
$n\geq 1$ in order to talk about $a^{n-1}$ in the denominator
(otherwise the exponent is not a nonnegative integer, so we cannot
apply the inductive hypothesis).  The inductive step must work for all $n$ but in this case it does not. 
We checked the base case only for $n=0$, so we are not justified in assuming that $n\geq 1$ when we try
to prove the statement for $n+1$ in the inductive step.  And of course
the proposition first breaks precisely at $n=1$.}

\end{problem}



%%%%%%%%%%%%%%%%%%%%%%%%%%%%%%%%%%%%%%%%%%%%%%%%%%%%%%%%%%%%%%%%%%%%%%%%%%%%%%%
% strong induction
% source: Velleman section 6.4, problem 9
\begin{problem}{10}
Let the sequence $G_0, G_1, G_2,\ldots$ be defined recursively as
follows: $G_0 = 0$, $G_1 = 1$, and $G_n = 5 G_{n-1} - 6 G_{n-2}$, for
every $n \in \mathbb{N}, n \geq 2$.

Prove that for all $n \in \mathbb{N}$, $G_n = 3^n - 2^n$.

\solution{
\begin{proof}
The proof is by strong induction on $n$.  Let $P(n)$ be the proposition that $G_n = 3^n - 2^n$.


{\bf Base case:} 
$P(0)$ is true because $G_0 = 0$ and $3^0-2^0=1-1=0$.
$P(1)$ is true because $G_1 = 1$ and $3^1-2^1=3-2=1$.

{\bf Inductive step:} 
Let $n\geq 2$ and suppose that $P(k)$ is true for $0 \leq k \leq n$. That is, assume that for every $k$, $1 \leq k \leq n$, $G_k= 3^k - 2^k$. Then by definition of the sequence $G_{n+1} = 5 G_n - 6 G_{n-1}$. By our induction hypothesis, $G_n = 3^n - 2^n$ and $G_{n-1} = 3^{n-1} - 2^{n-1}$.Thus, $G_{n+1} = 5(3^n - 2^n) - 6(3^{n-1} - 2^{n-1})$. Simplifying we get,
\begin{align*}
G_{n+1} &= 5(3^n - 2^n) - 6(3^{n-1} - 2^{n-1})\\
&= 5(3\cdot 3^{n-1}-2\cdot 2^{n-1})- 6(3^{n-1} - 2^{n-1})\\
&= 15\cdot 3^{n-1}-10\cdot 2^{n-1})- 6\cdot 3^{n-1} +6\cdot 2^{n-1}\\
&= 9\cdot 3^{n-1}-4\cdot 2^{n-1}\\
&= 3^{n+1} - 2^{n+1}
\end{align*}
So, $G_{n+1} = 3^{n+1} - 2^{n+1}$, as needed for the inductive step.
      
This completes the proof. 
        
\end{proof}
}
\end{problem}

%%%%%%%%%%%%%%%%%%%%%%%%%%%
%%%% 15-puzzle
\begin{problem}{25}

In the 15-puzzle, there are 15 lettered tiles and a blank square
arranged in a $4 \times 4$ grid.  Any lettered tile adjacent to the
blank square can be slid into the blank.  For example, a sequence of
two moves is illustrated below:

\[
\begin{array}{|c|c|c|c|}
\hline
A & B & C & D \\ \hline
E & F & G & H \\ \hline
I & J & K & L \\ \hline
M & O & N &  \\ \hline
\end{array}
\quad \rightarrow \quad
\begin{array}{|c|c|c|c|}
\hline
A & B & C & D \\ \hline
E & F & G & H \\ \hline
I & J & K & L \\ \hline
M & O &   & \mathbf{N} \\ \hline
\end{array}
\quad \rightarrow \quad
\begin{array}{|c|c|c|c|}
\hline
A & B & C & D \\ \hline
E & F & G & H \\ \hline
I & J &   & L \\ \hline
M & O & \mathbf{K} & N \\ \hline
\end{array}
\]

In the leftmost configuration shown above, the O and N tiles are out
of order.   Using only legal moves, is it possible to  swap the
N and the O, while leaving all the other tiles in their original position
and the blank in the bottom right corner?
In this problem, you will prove the answer is ``no''.

\begin{theorem*}
No sequence of moves transforms the board below on the left into the
board below on the right.
%
\[
\begin{array}{|c|c|c|c|}
\hline
A & B & C & D \\ \hline
E & F & G & H \\ \hline
I & J & K & L \\ \hline
M & \mathbf{O} & \mathbf{N} &  \\ \hline
\end{array}
\hspace{1in}
\begin{array}{|c|c|c|c|}
\hline
A & B & C & D \\ \hline
E & F & G & H \\ \hline
I & J & K & L \\ \hline
M & \mathbf{N} & \mathbf{O} &  \\ \hline
\end{array}
\]
\end{theorem*}

\bparts

\ppart{3}
We define the ``order'' of the tiles in a board to be the sequence of tiles on the board reading from the top row to the bottom row and from left to right within a row.  For example, in the right board depicted in the above theorem, the order of the tiles is $A$, $B$, $C$, $D$, $E$, etc.

Can a row move change the order of the tiles?  Prove your answer.

\solution{
No.  A row move moves a tile from cell $i$ to cell $i + 1$ or vice versa.  This tile does not change its order with respect to any other tile.  Since no other tile moves, there is no change in the order of any of the other pairs of tiles.
}

\ppart{3}
How many pairs of tiles will have their relative order changed by a column move?  More formally, for how many pairs of letters $L_1$ and $L_2$ will $L_1$ appear earlier in the order of the tiles than $L_2$ before the column move and later in the order after the column move?  Prove your answer correct.

\solution{
A column move changes the relative order of exactly three pairs of tiles.
Sliding a tile down moves it after the next three tiles in the order.
Sliding a tile up moves it before the previous three tiles in the
order.  Either way, the relative order changes between the moved tile
and each of the three it crosses.
}

\ppart{3}
We define an \emph{inversion} to be a pair of letters $L_1$ and $L_2$ for which $L_1$ precedes $L_2$ in the alphabet, but $L_1$ appears after $L_2$ in the order of the tiles.  For example, consider the following configuration:
\[
\begin{array}{|c|c|c|c|}
\hline
A & B & C & E \\ \hline
D & H & G & F \\ \hline
I & J & K & L \\ \hline
M & N & O &  \\ \hline
\end{array}
\]
There are exactly four inversions in the above configuration: $E$ and $D$, $H$ and $G$, $H$ and $F$, and $G$ and $F$.

What effect does a row move have on the parity of the number of inversions?  Prove your answer.

\solution{
A row move never changes the parity of the number of inversions.  A row move does not change the order of the tiles, so it does not affect the total number of inversions.
}

\ppart{5}
What effect does a column move have on the parity of the number of
inversions?  Prove your answer.

\solution{
A column move always changes the parity of the number of inversions.
A column move changes the relative order
of exactly three pairs of tiles.  An inverted pair becomes
uninverted and vice versa.  Thus, one exchange flips the total number
of inversions to the opposite parity, a second exhange flips it back
to the original parity, and a third exchange flips it to the opposite
parity again.
}

\ppart{8}
The previous problem part 
implies that we must make an \textit{odd} number of column
moves in order to exchange just one pair of tiles (N and O, say).
But this is problematic, because each column move also knocks the
blank square up or down one row.  So after an \textit{odd} number of
column moves, the blank can not possibly be back in the last row,
where it belongs!  Now we can bundle up all these observations and
state an \emph{invariant}, a property of the puzzle that never changes, no
matter how you slide the tiles around.

\begin{lemma*}
In every configuration reachable from the position shown below, the
parity of the number of inversions is different from the parity of the
row containing the blank square.

\[
\begin{array}{l}
\textit{row 1} \\
\textit{row 2} \\
\textit{row 3} \\
\textit{row 4} \\
\end{array}
\qquad
\begin{array}{|c|c|c|c|}
\hline
A & B & C & D \\ \hline
E & F & G & H \\ \hline
I & J & K & L \\ \hline
M & O & N &  \\ \hline
\end{array}
\]
\end{lemma*}

Prove this lemma.

\solution{
\begin{proof}
The proof is by induction.  Let $P(n)$ be the proposition that after $n$
moves, the parity of the number of inversions is different from the parity
of the row containing the blank square.

{\bf Base case:} After zero moves, exactly one pair of
tiles is inverted (O and N), which is an odd number.  And the blank
square is in row 4, which is an even number.  Therefore, $P(0)$ is true.

{\bf Inductive step:} Now we must prove that $P(n)$
implies $P(n+1)$ for all $n \geq 0$.  So assume that $P(n)$ is true;
that is, after $n$ moves the parity of the number of inversions is
different from the parity of the row containing the blank square.
There are two cases:

\begin{enumerate}

\item Suppose move $n+1$ is a row move.  Then the parity of the total
number of inversions does not change.
 The parity of the row containing
the blank square does not change either, since the blank remains in
the same row.  Therefore, these two parities are different after $n+1$
moves as well, so $P(n+1)$ is true.

\item Suppose move $n+1$ is a column move.  Then the parity of the
total number of inversions changes.
 However, the parity of the row
containing the blank square also changes, since the blank moves up or
down one row.  Thus, the parities remain different after $n+1$ moves,
and so $P(n+1)$ is again true.

\end{enumerate}

\noindent Thus, $P(n)$ implies $P(n+1)$ for all $n \geq 0$.

By the principle of induction, $P(n)$ is true for all $n \geq 0$.
\end{proof}
}

\ppart{3}
Prove the theorem that we originally set out to prove.

\solution{
In the target configuration on the right, the total number of
inversions is zero, which is even, and the blank square is in row 4,
which is also even.  Therefore, by the lemma,
the target configuartion is unreachable.
}

\eparts

\end{problem}

%%%%%%%%%%%%%%%%%%%%%%%%%%%%%%
%%lings problem

\begin{problem}{15}
There are two types of creature on planet Char, Z-lings
and B-lings. Furthermore, every creature belongs to a particular
generation.  The creatures in each generation reproduce according to
certain rules and then die off.  The subsequent generation consists
entirely of their offspring.

The creatures of Char pair with a mate in order to reproduce.
First, as many Z-B pairs as possible are formed.  The remaining
creatures form Z-Z pairs or B-B pairs, depending on whether there is
an excess of Z-lings or of B-lings.  If there are an odd number of
creatures, then one in the majority species dies without reproducing.
The number and type of offspring is determined by the types of the
parents

\begin{itemize}
\item If both parents are Z-lings, then they have three Z-ling
offspring.

\item If both parents are B-lings, then they have two B-ling offspring
and one Z-ling offspring.

\item If there is one parent of each type, then they have one
offspring of each type.
\end{itemize}

There are 200 Z-lings and 800 B-lings in the first generation.  Use
induction to prove that the number of Z-lings will always be at most
twice the number of B-lings.

\emph{Hint: You may want to use a stronger hypothesis for the induction.}

\solution{
An induction proof with the hypothesis that the number of Z-lings is
always at most twice the number of B-lings will not go through. In
fact, you can check that if the number of Z-lings is larger than the
one of B-lings, then the fraction of B-lings will decrease at every step.
A stronger induction hypothesis is required.

\begin{proof}
We will prove that there are always at least as many B-lings as
Z-lings; the claim that the number of Z-lings is always at most twice
the number of B-lings is weaker and thus follows.

The proof is by induction on the generation number.  Let $P(n)$ be the
proposition that there are at least as many B-lings as Z-lings in
generation $n$.  In the base case, $P(1)$ is true because there are
800 B-lings and only 200 Z-lings in the first generation.

In the inductive step, for $n \geq 1$ assume that there are at least
as many B-lings as Z-lings in generation $n$ to prove that there are
at least as many B-lings as Z-lings in generation $n+1$.  Let $a$ be
the number of Z-lings in generation $n$, and let $b$ be the number of
B-lings.  Note that $b \geq a$ by the induction assumption.  Then
there are no Z-Z pairs formed, there are $a$ Z-B pairs formed, and
there are $\lfloor \frac{b - a}{2} \rfloor$ $B-B$ pairs formed.  As a
result, the number of Z-lings in generation $n+1$ is $a + \lfloor
\frac{b - a}{2} \rfloor$, and the number of B-lings is $a + 2 \lfloor
\frac{b - a}{2} \rfloor$.  Therefore, there are at least as many
B-lings as Z-lings in generation $n+1$.  Thus, for all $n \geq 1$,
$P(n)$ implies $P(n+1)$, and the claim is proved by induction.
\end{proof}
}

\end{problem}

%%%%%%%%%%%%%%%%%%%%%%%%%
\end{document}
