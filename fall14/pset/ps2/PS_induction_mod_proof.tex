\documentclass[problem]{mcs}

\begin{pcomments}
  \pcomment{PS_induction_mod_proof}
  \pcomment{from: F04.ps03.02}
  \pcomment{updated by Rich to use terms in the notes}
\end{pcomments}

\pkeywords{
  gcd
  linear_combinations
  divides
  number_theory
  common_divisor
  induction
}

%%%%%%%%%%%%%%%%%%%%%%%%%%%%%%%%%%%%%%%%%%%%%%%%%%%%%%%%%%%%%%%%%%%%%
% Problem starts here
%%%%%%%%%%%%%%%%%%%%%%%%%%%%%%%%%%%%%%%%%%%%%%%%%%%%%%%%%%%%%%%%%%%%%

\begin{problem}
  
  Use induction to prove the following statements:
  
  \bparts
  
  \ppart
  \[
  (\rem{a_1}{n}) \cdot (\rem{a_2}{n}) \cdots (\rem{a_k}{n}) \equiv a_1
  \cdot a_2 \cdots a_k \pmod{n}
  \]
  
  \hint You may use the following two facts:
  \begin{enumerate}
    \item If $a_1 \equiv b_1 \pmod{n}$ and $a_2 \equiv b_2 \pmod{n}$, then
      $a_1 a_2 \equiv b_1 b_2 \pmod{n}$.
    \item $(\rem{a}{n}) \equiv a \pmod{n}$
  \end{enumerate}
  
  \begin{solution}
    We proceed by induction on $k$ with the claim itself as the
    induction hypothesis.

    \noindent \inductioncase{Base case}: ($k=1$).  The claim holds for
    $k = 1$ by the second fact provided above.

    \noindent \inductioncase{Inductive step}: Now we assume that the claim holds
    for some $k \geq 1$ and prove that the claim holds for $k + 1$.
    Consider the expression:
    %
    \[
    (\rem{a_1}{n}) \cdot (\rem{a_2}{n}) \cdots (\rem{a_k}{n}) \cdot (\rem{a_{k+1}}{n})
    \]
    %
    By the induction assumption, the first $n$ terms are congruent to $a_1
    \cdot a_2 \cdots a_k$ modulo $n$.  By the second fact from above,
    $(\rem{a_{k+1}}{n})$ is congruent to $a_{k+1}$ modulo n.  Thus, by the
    first fact above, the whole product is congruent to
    %
    \[
    a_1 \cdot a_2 \cdots a_k \cdot a_{k+1}
    \]
    %
    modulo $n$.  Thus, the claim holds for $k + 1$.

    By the principle of induction, the claim holds for all $k \geq 1$.
  \end{solution}
  
  \ppart
  Let $p$ be a prime.  If $p \mid a_1 \cdot a_2 \cdots a_n$, then
  $p$ divides some $a_i$.

  \hint You may use the fact that if $p$ is a prime and $p
  \divides ab$, then $p \divides a$ or $p \divides b$.
  
  \begin{solution}
    We proceed by induction on $n$ with the claim itself as the
    induction hypothesis.

    \noindent \emph{Base case:} When $n = 1$, the claim asserts that if
    $p \divides a_1$, then $p \divides a_1$, which is trivially true.

    \noindent \emph{Inductive step:} Now we assume that the claim holds
    for some $n \geq 1$ and prove that it holds for $n + 1$.  Suppose that
    %
    \[
    p \divides a_1 \cdot a_2 \cdots a_n \cdot a_{n+1}
    \]
    %
    By grouping the first $n$ terms on the right and using the fact cited
    above from lecture, we know that either $p \divides (a_1 \cdot a_2 \cdots
    a_n)$ or $p \divides a_{n+1}$.  If the former case, the induction
    assumption implies that $p$ divides some some $a_i$ with $1 \leq i
    \leq n$.  Thus, in both cases, $p$ divides some $a_i$ as claimed.

    By induction, the claim holds for all $n \geq 1$.
  \end{solution}
  
  \eparts
  
\end{problem}

%%%%%%%%%%%%%%%%%%%%%%%%%%%%%%%%%%%%%%%%%%%%%%%%%%%%%%%%%%%%%%%%%%%%%
% Problem ends here
%%%%%%%%%%%%%%%%%%%%%%%%%%%%%%%%%%%%%%%%%%%%%%%%%%%%%%%%%%%%%%%%%%%%%

\endinput
