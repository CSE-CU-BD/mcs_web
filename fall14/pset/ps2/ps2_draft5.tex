\documentclass[twoside,12pt]{article}
\newcommand{\tab}{\hspace*{2em}}
\usepackage{light}
\usepackage{subfigure}
\usepackage{graphicx}
\usepackage{amsmath}
\usepackage{verbatim}

\usepackage{amsfonts}

\newcommand{\lr}{l_{right}}
\renewcommand{\ll}{l_{left}}

\newcommand{\hint}[1]{({\it Hint: #1})}
\newcommand{\card}[1]{\left|#1\right|}
\newcommand{\union}{\cup}
\newcommand{\lgunion}{\bigcup}
\newcommand{\intersect}{\cap}
\newcommand{\lgintersect}{\bigcap}
\newcommand{\cross}{\times}


\hidesolutions
%\showsolutions

\newlength{\strutheight}
\newcommand{\prob}[1]{\mathop{\textup{Pr}} \nolimits \left( #1 \right)}
\newcommand{\prsub}[2]{\mathop{\textup{Pr}_{#1}}\nolimits\left(#2\right)}
\newcommand{\prcond}[2]{%
  \ifinner \settoheight{\strutheight}{$#1 #2$}
  \else    \settoheight{\strutheight}{$\displaystyle#1 #2$} \fi %
  \mathop{\textup{Pr}}\nolimits\left(
    #1\,\left|\protect\rule{0cm}{\strutheight}\right.\,#2
  \right)}
\newcommand{\cE}{\mathcal{E}}
\renewcommand{\setminus}{-}
\renewcommand{\complement}[1]{\overline{#1}}

\providecommand{\abs}[1]{\lvert#1\rvert}

\begin{document}
\problemset{2}{September 9, 2014}{Monday, September 15}


%%%%%%%%%%%%%%%%%%%%%%%%%%%%%%%%%%%%%%%%%%%%%%%%%%%%
\noindent \textbf{Reading Assignment:}   Sections 2.5-2.7, 3.0-3.4, \& 3.5 (optional)
\\

\begin{problem}{6}
Can raising an irrational number $a$ to an irrational power $b$ result in a
rational number? Provide a proof that it can by considering $\sqrt{3}^{\sqrt{2}}$
and using proof by cases.
\solution{
This proof is by cases. We will consider $\sqrt{3}^{\sqrt{2}}$, and there are
two cases.

\begin{itemize}

\item $\sqrt{3}^{\sqrt{2}}$ is rational. In this case, since $\sqrt{3}$ is
irrational and $\sqrt{2}$ is irrational, we have found an $a$ and $b$ such that
both are irrational and $a^b$ is rational.

\item $\sqrt{3}^{\sqrt{2}}$ is irrational. In this case, notice that
$(\sqrt{3}^{\sqrt{2}})^{\sqrt{2}} = \sqrt{3}^{2} = 3$. Since $\sqrt{3}^{\sqrt{2}}$
and $\sqrt{2}$ are both irrational and $3$ is rational, we have found an
$a$ and $b$.

\end{itemize}

In all cases, we have found some irrational $a$ such that when raised to an
irrational power $b$, $a^b$ is rational.
}
\end{problem}


%%%%%%%%%%%%%%%%%%%%%%%%%%%%%%%%%%%%%%%%%%%%%%%%%%%%%%%%%%%%%%%%%%%%%%%%%%%%%%%
\begin{problem}{18}
The following problem is fairly tough until you hear a certain
one-word clue. The solution is elegant but is slightly tricky, so don't hesitate to ask for hints!

During 6.042, the students are sitting in
an $n\times n$ grid. A sudden outbreak of beaver flu (a rare variant of bird flu that lasts forever; symptoms include yearning for problem sets and craving for ice cream study sessions) causes some students to get infected. Here is
an example where $n = 6$ and infected students are marked $\times$.

\[
\begin{array}{|c|c|c|c|c|c|}
\hline
\times& & & &\times& \\ \hline
 &\times& & & & \\ \hline
& &\times&\times& & \\ \hline
& & & & & \\ \hline
& &\times& & & \\ \hline
& & &\times& &\times \\ \hline
\end{array}
\]

\noindent Now the infection begins to spread every minute (in discrete time-steps). Two students are considered \textit{adjacent} if they
share an edge (i.e., front, back, left or right, but NOT diagonal); thus, each student is adjacent to 2, 3 or 4 others.  A
student is infected in the next time step if either

\begin{itemize}
\item the student was previously infected (since beaver flu lasts forever), or
\item the student is adjacent to \textit{at least two} already-infected students.
\end{itemize}

In the example, the infection spreads as shown below.
%
\[
\begin{array}{|c|c|c|c|c|c|}
\hline
\times& & & &\times& \\ \hline
 &\times& & & & \\ \hline
& &\times&\times& & \\ \hline
& & & & & \\ \hline
& &\times& & & \\ \hline
& & &\times& &\times \\ \hline
\end{array}
\Rightarrow
\begin{array}{|c|c|c|c|c|c|}
\hline
\times&\times& & &\times& \\ \hline
\times&\times&\times& & & \\ \hline
&\times&\times&\times& & \\ \hline
& &\times& & & \\ \hline
& &\times&\times& & \\ \hline
& &\times&\times&\times&\times \\ \hline
\end{array}
\Rightarrow
\begin{array}{|c|c|c|c|c|c|}
\hline
\times&\times&\times& &\times& \\ \hline
\times&\times&\times&\times& & \\ \hline
\times&\times&\times&\times& & \\ \hline
&\times&\times&\times& & \\ \hline
& &\times&\times&\times& \\ \hline
& &\times&\times&\times&\times \\ \hline
\end{array}
\]
%
In this example, over the next few time-steps, all the students in class will become infected.

\begin{theorem*}
If fewer than $n$ students in class are initially infected, the whole class will never be completely infected.
\end{theorem*}

Prove this theorem.

\textit{ Hint:} To understand how a system such as the above ``evolves" over time, it is usually a good strategy to (1) identify an appropriate  property of the system at the initial stage, and (2) prove, by induction on the number of time-steps, that the property is preserved at every time-step. So look for a property (of the set of infected students) that remains invariant as time proceeds.

If you are stuck, ask your recitation instructor for the one-word clue and even more hints!

\solution{
\begin{proof}
Define the {\em perimeter} of an infected set of students to be the number of
edges with infection on exactly one side.  Let $I$ denote the
perimeter of the initially-infected set of students.

Now we use induction on the number of time steps to prove that the
perimeter of the infected region never increases.  Let $P(k)$ be the
proposition that after $k$ time steps, the perimeter of the infected
region is at most $I$.

{\bf Base case:} $P(0)$ is true by definition; the
perimeter of the infected region is at most $I$ after 0 time steps,
because $I$ is defined to be the perimeter of the initially-infected
region.

{\bf Inductive step:} Now we must show that $P(k)$ implies
$P(k+1)$ for all $k \geq 0$.  So assume that $P(k)$ is true, where $k \geq
0$; that is, the perimeter of the infected region is at most $I$ after $k$
steps.  The perimeter can only change at step $k + 1$ because some squares
are newly infected.  By the rules above, each newly-infected square is
adjacent to at least two previously-infected squares.  Thus, for each
newly-infected square, at least two edges are removed from the perimenter
of the infected region, and at most two edges are added to the perimeter.
Therefore, the perimeter of the infected region can not increase and is at
most $I$ after $k + 1$ steps as well.  This proves that $P(k)$ implies
$P(k+1)$ for all $k \geq 0$.

By the principle of induction, $P(k)$ is true for all $k \geq 0$.

If an $n \times n$ grid is completely infected, then the perimeter of
the infected region is $4n$.  Thus, the whole grid can become infected
only if the perimeter is initially at least $4n$.  Since each square
has perimeter 4, at least $n$ squares must be infected initially for the whole grid to be infected.
\end{proof}

The above proof shows that if initially $k$ students  are infected, then the perimeter of the infected region will never exceed $4k$. The largest number of students that can be contained within a region with perimeter $\leq 4k$ is equal to $k^2$, therefore, if $k$ students in class are initially infected, then at most $k^2$ students will eventually be infected. This feels intuitively true after having done the previous proof. However, to give a formal proof requires some case analysis (try it!).   
}

\end{problem}

%%%%%%%%%%%%%%%%%%%%%%%%%%%%%%%%%

\begin{problem}{20}

In the 15-puzzle, there are 15 lettered tiles and a blank square
arranged in a $4 \times 4$ grid.  Any lettered tile adjacent to the
blank square can be slid into the blank.  For example, a sequence of
two moves is illustrated below:

\[
\begin{array}{|c|c|c|c|}
\hline
A & B & C & D \\ \hline
E & F & G & H \\ \hline
I & J & K & L \\ \hline
M & O & N &  \\ \hline
\end{array}
\quad \rightarrow \quad
\begin{array}{|c|c|c|c|}
\hline
A & B & C & D \\ \hline
E & F & G & H \\ \hline
I & J & K & L \\ \hline
M & O &   & \mathbf{N} \\ \hline
\end{array}
\quad \rightarrow \quad
\begin{array}{|c|c|c|c|}
\hline
A & B & C & D \\ \hline
E & F & G & H \\ \hline
I & J &   & L \\ \hline
M & O & \mathbf{K} & N \\ \hline
\end{array}
\]

In the leftmost configuration shown above, the O and N tiles are out
of order.   Using only legal moves, is it possible to  swap the
N and the O, while leaving all the other tiles in their original position
and the blank in the bottom right corner?
In this problem, you will prove that the answer is ``no''.

\begin{theorem*}
No sequence of moves transforms the board below on the left into the
board below on the right.
%
\[
\begin{array}{|c|c|c|c|}
\hline
A & B & C & D \\ \hline
E & F & G & H \\ \hline
I & J & K & L \\ \hline
M & \mathbf{O} & \mathbf{N} &  \\ \hline
\end{array}
\hspace{1in}
\begin{array}{|c|c|c|c|}
\hline
A & B & C & D \\ \hline
E & F & G & H \\ \hline
I & J & K & L \\ \hline
M & \mathbf{N} & \mathbf{O} &  \\ \hline
\end{array}
\]
\end{theorem*}

\bparts

\ppart{2}
We define the ``order'' of the tiles in a board to be the sequence of tiles on the board reading from the top row to the bottom row and from left to right within a row.  For example, in the right board depicted in the above theorem, the order of the tiles is $A$, $B$, $C$, $D$, $E$, etc.

Can a row move change the order of the tiles?  Prove your answer.

\solution{
No.  A row move moves a tile from cell $i$ to cell $i + 1$ or vice versa.  This tile does not change its order with respect to any other tile.  Since no other tile moves, there is no change in the order of any of the other pairs of tiles.
}

\ppart{2}
How many pairs of tiles will have their relative order changed by a column move?  More formally, for how many pairs of letters $L_1$ and $L_2$ will $L_1$ appear earlier in the order of the tiles than $L_2$ before the column move and later in the order after the column move?  Prove your answer correct.

\solution{
A column move changes the relative order of exactly three pairs of tiles.
Sliding a tile down moves it after the next three tiles in the order.
Sliding a tile up moves it before the previous three tiles in the
order.  Either way, the relative order changes between the moved tile
and each of the three it crosses.
}

\ppart{2}
We define an \emph{inversion} to be a pair of letters $L_1$ and $L_2$ for which $L_1$ precedes $L_2$ in the alphabet, but $L_1$ appears after $L_2$ in the order of the tiles.  For example, consider the following configuration:
\[
\begin{array}{|c|c|c|c|}
\hline
A & B & C & E \\ \hline
D & H & G & F \\ \hline
I & J & K & L \\ \hline
M & N & O &  \\ \hline
\end{array}
\]
There are exactly four inversions in the above configuration: $E$ and $D$, $H$ and $G$, $H$ and $F$, and $G$ and $F$.

What effect does a row move have on the parity of the number of inversions?  Prove your answer.

\solution{
A row move never changes the parity of the number of inversions.  A row move does not change the order of the tiles, so it does not affect the total number of inversions.
}

\ppart{4}
What effect does a column move have on the parity of the number of
inversions?  Prove your answer.

\solution{
A column move always changes the parity of the number of inversions.
A column move changes the relative order
of exactly three pairs of tiles.  An inverted pair becomes
uninverted and vice versa.  Thus, one exchange flips the total number
of inversions to the opposite parity, a second exhange flips it back
to the original parity, and a third exchange flips it to the opposite
parity again.
}

\ppart{8}
The previous problem part 
implies that we must make an \textit{odd} number of column
moves in order to exchange just one pair of tiles (N and O, say).
But this is problematic, because each column move also knocks the
blank square up or down one row.  So after an \textit{odd} number of
column moves, the blank can not possibly be back in the last row,
where it belongs!  Now we can bundle up all these observations and
state an \emph{invariant}, a property of the puzzle that never changes, no
matter how you slide the tiles around.

\begin{lemma*}
In every configuration reachable from the position shown below, the
parity of the number of inversions is different from the parity of the
row containing the blank square.

\[
\begin{array}{l}
\textit{row 1} \\
\textit{row 2} \\
\textit{row 3} \\
\textit{row 4} \\
\end{array}
\qquad
\begin{array}{|c|c|c|c|}
\hline
A & B & C & D \\ \hline
E & F & G & H \\ \hline
I & J & K & L \\ \hline
M & O & N &  \\ \hline
\end{array}
\]
\end{lemma*}

Prove this lemma.

\solution{
\begin{proof}
The proof is by induction.  Let $P(n)$ be the proposition that after $n$
moves, the parity of the number of inversions is different from the parity
of the row containing the blank square.

{\bf Base case:} After zero moves, exactly one pair of
tiles is inverted (O and N), which is an odd number.  And the blank
square is in row 4, which is an even number.  Therefore, $P(0)$ is true.

{\bf Inductive step:} Now we must prove that $P(n)$
implies $P(n+1)$ for all $n \geq 0$.  So assume that $P(n)$ is true;
that is, after $n$ moves the parity of the number of inversions is
different from the parity of the row containing the blank square.
There are two cases:

\begin{enumerate}

\item Suppose move $n+1$ is a row move.  Then the parity of the total
number of inversions does not change.
 The parity of the row containing
the blank square does not change either, since the blank remains in
the same row.  Therefore, these two parities are different after $n+1$
moves as well, so $P(n+1)$ is true.

\item Suppose move $n+1$ is a column move.  Then the parity of the
total number of inversions changes.
 However, the parity of the row
containing the blank square also changes, since the blank moves up or
down one row.  Thus, the parities remain different after $n+1$ moves,
and so $P(n+1)$ is again true.

\end{enumerate}

\noindent Thus, $P(n)$ implies $P(n+1)$ for all $n \geq 0$.

By the principle of induction, $P(n)$ is true for all $n \geq 0$.  
\end{proof}
}

\ppart{2}
Prove the theorem that we originally set out to prove.

\solution{
In the target configuration on the right, the total number of
inversions is zero, which is even, and the blank square is in row 4,
which is also even.  Therefore, by the lemma,
the target configuartion is unreachable.
}

\eparts

\end{problem}

%%%%%%%%%%%%%%%%%%%%%%%%%%%%%%%%%%%%%%%%%%
\begin{problem}{14}\textit{The Well Ordering Principle (WOP)} states that ``every \textit{nonempty} set of 
\textit{nonnegative} integers has a \textit{smallest} element." (See Section 3.1 of the text \textit{Mathematics 
for Computer Science}.)  It captures a special property about nonnegative integers and can be extremely 
useful in proofs.  

\bparts
\ppart{4} 
Show that WOP can be proved when the principle of mathematical induction is taken as an 
axiom.  (\textit{Hint:} Begin by pretending that the well ordering principle were false and carry out induction on a 
nonempty set of nonnegative integers that has no least element.  The induction should lead to the conclusion
that the set is empty.)

\solution{
Let $S$ be a nonempty set of nonnegative integers that has no least element.  Let $P(n)$ be the 
proposition "$i \notin S, i = 0, 1, ..., n$."

\textbf{Base case:}   $P(0)$ is true, because if $0 \in S$, then $S$ has a least element, namely, $0$.

\textbf{Inductive step:}   Assume $P(n)$ is true; it means $0 \notin S, 1 \notin S, ..., n \notin S$.  If $n+1$ 
is not in $S$, then $P(n+1)$ is true; else, $n+1$ would be its least element. 

By the principle of mathematical induction, $n \notin S$ for all nonnegative integers $n$.  
Thus, $S = \emptyset$, a contradiction.
}

\ppart{5}
Prove using the Well Ordering Principle that for all nonnegative integers, $n$: 
\begin{equation}\label{sum-to-n}
\sum_{i=0}^{n} i^3 = \left(\frac{n(n+1)}{2}\right)^2.
\end{equation}

\solution{
\begin{proof}
The proof uses WOP. Assume that equation \eqref{sum-to-n} is false. Then, some nonnegative integers serve as counterexamples to it.
Let's collect these counterexamples in a set:
$C ::= \{ n \in \mathbb{N} | \sum_{i=0}^{n} i^3 \neq \left(\frac{n(n+1)}{2}\right)^2 \}$.

By our assumption that \eqref{sum-to-n} admits counterexamples, $C$ is a nonempty set
of nonnegative integers. So, by the Well Ordering Principle, $C$ has a minimum
element, call it $c$. That is, $c$ is the smallest counterexample to \eqref{sum-to-n}.

Since $c$ is the smallest counterexample, we know that equation \eqref{sum-to-n} is false for $n = c$, but it is true for all nonnegative integers $n < c$.  However, equation \eqref{sum-to-n} is true for $n = 0$ since $\sum_{i=0}^{0} i^3 = 0 = \left(\frac{0(0+1)}{2}\right)^2$. Hence, $c > 0$. This means $c - 1$ is a nonnegative integer, and since it is less than $c$, equation \eqref{sum-to-n} is true for $c - 1$. That is,

\begin{equation}\label{sum-to-c-1}
\sum_{i=0}^{c-1} i^3 = \left(\frac{(c-1)c}{2}\right)^2.
\end{equation}

But then, adding $c^3$ to both sides of equation \eqref{sum-to-c-1} gives us
\[
\sum_{i=0}^{c} i^3
\]
on the left hand side. And the right hand side now equals
\begin{align*}
\left(\frac{(c-1)c}{2}\right)^2 + c^3 &= \frac{(c-1)^2c^2+4c^3}{2^2}\\
 &= \frac{c^2\left((c-1)^2+4c \right)}{2^2}\\
&= \frac{c^2\left(c^2-2c+1+4c \right)}{2^2}\\
&= \frac{c^2\left(c^2+2c+1 \right)}{2^2}\\
&= \frac{c^2(c+1)^2}{2^2}\\
&= \left(\frac{c(c+1)}{2}\right)^2.
\end{align*}

That is,
\[
\sum_{i=0}^{c} i^3 = \left(\frac{c(c+1)}{2}\right)^2,
\]
which means that equation \ref{sum-to-n} does hold for c, after all! This is a contradiction, and we are done.

\end{proof}
}

\ppart{5}
Prove equation \eqref{sum-to-n} by induction.

\solution{ 
\begin{proof}
The proof is by induction on $n$.  Let $P(n)$ be the
proposition that equation \eqref{sum-to-n} holds.

{\bf Base case:}  $P(0)$ is true because 

\[
\sum_{i=0}^{0} i^3 = 0 = \left(\frac{0(0+1)}{2}\right)^2.
\]

{\bf Inductive step:}  Assume $P(n)$ is true, that is $\sum_{i=0}^{n} i^3 = \left(\frac{n(n+1)}{2}\right)^2$.  Then we
can prove $P(n+1)$ is also true as follows:
\begin{align*}
\sum_{i=0}^{n+1} i^3 &=\sum_{i=0}^{n} i^3 + (n+1)^3\\
&= \left(\frac{n(n+1)}{2}\right)^2 + (n+1)^3 \\
&= \frac{n^2(n+1)^2+4(n+1)^3}{2^2} \\
&= \frac{(n+1)^2\left(n^2+4(n+1)\right)}{2^2} \\
 &= \frac{(n+1)^2\left(n^2+4n+4\right)}{2^2} \\
&= \frac{(n+1)^2(n+2)^2}{2^2} \\
&= \left(\frac{(n+1)(n+2)}{2}\right)^2
\end{align*}

The first step breaks up the sum. The second step uses the assumption $P(n)$. The rest of the steps are algebraic simplifications.

Thus, $P(0)$ is true and $P(n)$ implies $P(n+1)$ for all nonnegative integers.
Therefore, $P(n)$ is true for all nonegative integers by the principle of
induction.

\end{proof}
}

\eparts

\end{problem}

%%%%%%%%%%%%%% Problem 7 from Fall2012 PS2
\begin{problem}{8}\textit{Euler's Conjecture} in 1769 was that there are no positive integer solutions to the 
equation
\begin{equation}
a^4 + b^4 + c^4 = d^4.
\end{equation}
Integer values for $a, b, c, d$ that do satisfy this equation were first discovered in 1986.  
So Euler guessed wrong, but it took more than two hundred years to prove it.

Now let's consider Moitra's equation, similar to Euler's but with some coefficients:
\begin{equation}\label{Moitra}
27a^4 + 9b^4 + 3c^4 = d^4
\end{equation}
Prove that Moitra's equation \eqref{Moitra} really does not have any positive integer solutions.

\textit{Hint:}   Consider the minimum value of $a$ among all possible solutions to \eqref{Moitra}.

\solution{
Suppose that there exists a solution.  Then there must be a solution in which $a$ has the smallest possible 
value.  We will show that, in this solution, $a, b, c, \text{and } d$ must all be multiples of 3.  However, we can then obtain 
another solution over the positive integers with a smaller $a$ by dividing $a, b, c, \text{and } d$ by 3.  This 
is a contradiction, and so no solution exists.  All that remains is to show that $a, b, c, \text{and } d$ must all 
be multiples of 3.  The left side of Moitra's equation is divisible by 3, so $d^4$ is a multiple of 3, so $d$ must be a multiple of 3.  Substituting 
$d=3d^{'}$ into Moitra's equation gives: 
\begin{equation}\label{M2}
27a^4 + 9b^4 + 3c^4 = 81d^{'4}
\end{equation}
Now, $3c^4$ must be a multiple of 9, since every other term is a multiple of 9.  This implies that $c^4$ is 
a multiple of 3 and so $c$ is also a multiple of 3.  Subsituting $c=3c^{'}$ into the previous equation gives:
\begin{equation}\label{M3}
27a^4 + 9b^4 + 243c^{'4} = 81d^{'4}
\end{equation}
Arguing in the same way, $9b^4$ must be a multiple of $27$, since every other term is.  Therefore, $b^4$ is 
a multiple of 3 and so $b$ is a multiple of 3.  Substituting $b=3b^{'}$ gives:
\begin{equation}\label{M4}
27a^4 + 729b^{'4} + 243c^{'4} = 81d^{'4}
\end{equation}
Finally, $27a^4$ must be a multiple of 81, $a^4$ must be a multiple of 3, and so $a$ must also be a multiple of 3.  Therefore, 
$a, b, c, \text{and } d$ must all be multiples of 3, as claimed.
}
\end{problem}


%%%%%%%%%%%%%%%%%%%%%%%%%%%%%%%%%%%%%%%%%%

\begin{problem}{18}
Nim is a game played between two players with three piles of stones. Players
alternate removing stones. A player picks a pile and removes any positive number
of stones. The goal is to be the last player to take a stone.
\bparts

\ppart{5} The winning strategy in Nim requires computing a Nim sum. A Nim sum
is defined to be the binary xor of the number of stones in each pile. Prove that
if the Nim sum is zero, then any move will result in Nim sum that is not zero.

\solution{
When a player removes stones from a pile, the binary representation must change
in at least one digit, otherwise the number of stones would be the same. At that
digit, the xor can no longer be zero since the bit has flipped. The Nim sum is
then non-zero.
}

\ppart{5} Prove that if the Nim sum is not zero that it is always possible to
make the Nim sum zero with one move.

\solution{
Let the Nim sum be $t$.  Let $d$ be the position of the most significant bit of
the Nim sum. One of the piles must have this bit non-zero; pick one of them. It
has $s$ stones remaining. Remove stones so that it has $s \oplus t$ stones
remaining, where $\oplus$ is the logic symbol for xor.  Such move is always possible because bit $d$ will be zero in $s \oplus
t$, so it will be less than $s$. This results in a Nim sum of zero because the
Nim sum of the other two piles stays the same, $s \oplus t$ and the third pile
has $s \oplus t$ stones left, so the Nim sum is zero.
}

\ppart{8} Using parts a and b, prove that if the game begins with a non-zero
Nim sum, then the first player has a winning strategy.

\solution{
The first player can remove stones such that the Nim sum is zero. Either this
results in no stones left or the second player must make a move that results
in a non-zero Nim sum. There must be at least one stone left because the Nim
sum is non-zero, so the second player has not won. This is the same situation
as the start except there are less stones. The result follows from induction
on the number of stones.
}

\eparts
\end{problem}

%%%%%%%%%%%%%%%%%%%%%%%%%%%%%%%%%%%%%%%%
\begin{problem}{8}
  A group of $n \ge 1$ people can be divided into teams, each
  containing either 4 or 7 people.  What are all the possible values of
  $n$?  Use induction to prove that your answer is correct.

\solution{
We begin by observing that the following numbers of people can be divided
into teams with 4 or 7 people per team:
\begin{align*}
4 & = 4 \\
7 & = 7 \\
8 & = 4 + 4 \\
11 & = 4 + 7 \\
12 & = 4 + 4 + 4 \\
14 & = 7 + 7 \\
15 & = 4 + 4 + 7 \\
16 & = 4 + 4 + 4 + 4 \\
18 & = 4 + 7 + 7 \\
19 & = 4 + 4 + 4 + 7 \\
20 & = 4 + 4 + 4 + 4 + 4 \\
21 & = 7 + 7 + 7,
\end{align*}
and these are the only numbers $\le 21$ that can be divided into such
teams.  Now we claim that every group of $n \ge 18$ people can be divided
into teams, each containing either 4 or 7 people.

\begin{proof}
  The proof is by strong induction on $n$.  Let $P(n)$ be the proposition
  that a group of $n \ge 18$ people can be divided into teams, with each
  containing either 4 or 7 people.

  \textbf{Base cases:} As shown above $P(18)$, $P(19)$, $P(20)$, and
  $P(21)$ are true.

\textbf{Inductive step:} For all $n \ge 21$, we assume that $P(18)$,
$P(19)$, $\dots$, $P(n)$ are true in order to prove that $P(n+1)$ is true.

Since $n+1 = (n-3) + 4$, a team of 4 people can be removed from the set of
$n+1$ people, leaving $n-3 \ge 18$ people.  By induction hypothesis, the
$n-3$ people can be further divided into disjoint teams with 4 or 7
people.  Since this divides the $n+1$ people into teams with 4 or 7, we
have shown that $P(n+1)$ is true.  It follows by strong induction that
$P(n)$ holds for all $n \ge 18$.

So all the possible values of $n$ are 4, 7, 8, 11, 12, 14, 15, 16, and
$\geq 18$.
\end {proof}
}
\end{problem}


%%%%%%%%%%%%%%%%%%%%%%%%%%%%%%%%%%%%%%%%%%%
\begin{problem} {8} Three pirates are considering an attack on the {\em Pi\~{n}ata}, a
Spanish galleon laden with $n \ge 20$ pieces of gold.  Redbeard
insists that his share of the treasure be a multiple of 5 gold pieces,
Bluebeard insists that his share be a multiple of 7 gold pieces, and
Blackbeard demands a multiple of 9 gold pieces.  

Furthermore, no pirate can tolerate treasure going to waste; if that is unavoidable,
the pirates will surely have a fatal quarrel.  For example, if there were 12 gold pieces aboard, 
then Redbeard could take 5, Bluebeard 7, and Blackbeard 0.  However, if there were 13 gold
pieces aboard, then the pirates would kill each other.

Can the pirates safely attack the {\em Pi\~{n}ata}?  Use strong
induction to prove that your answer is correct.

\solution{The pirates can safely attack the Pi\~{n}ata.  The
proof is by strong induction on $n$, the number of gold pieces aboard.

Let $P(n)$ be the proposition that $n$ is of the form $5x + 7y + 9z$,
where $x$, $y$, and $z$ are nonnegative integers.  First, we show that
$P(n)$ holds for all $n$ between 20 and 24.

\begin{eqnarray*}
20 & = & 5 \cdot 4 + 7 \cdot 0 + 9 \cdot 0 \\
21 & = & 5 \cdot 1 + 7 \cdot 1 + 9 \cdot 1 \\
22 & = & 5 \cdot 3 + 7 \cdot 1 + 9 \cdot 0 \\
23 & = & 5 \cdot 0 + 7 \cdot 2 + 9 \cdot 1 \\
24 & = & 5 \cdot 2 + 7 \cdot 2 + 9 \cdot 0
\end{eqnarray*}

Now we must show that $P(n)$ is true assuming that $P(20), \ldots, P(n
- 1)$ are all true, where $n \ge 25$.  Then, in particular, we know
that $P(n - 5)$ is true; that is, there is a way of dividing up $n -
5$ gold pieces among the three pirates with none going to waste.  We
can divide $n$ gold pieces in the same way, giving the 5 extras to
Redbeard.  This shows that $P(n)$ is true as well.  Therefore, $P(n)$
is true for all $n \ge 20$ by strong induction.
}
\end{problem}



\end{document}
