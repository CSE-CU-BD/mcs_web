\documentclass[twoside,12pt]{article}
\newcommand{\tab}{\hspace*{2em}}
\usepackage{light}
\usepackage{subfigure}
\usepackage{graphicx}
\usepackage{amsmath}
\usepackage{verbatim}

\usepackage{amsfonts}

\newcommand{\lr}{l_{right}}
\renewcommand{\ll}{l_{left}}

\newcommand{\card}[1]{\left|#1\right|}

\newcommand{\hint}[1]{({\it Hint: #1})}
\newcommand{\card}[1]{\left|#1\right|}
\newcommand{\union}{\cup}
\newcommand{\lgunion}{\bigcup}
\newcommand{\intersect}{\cap}
\newcommand{\lgintersect}{\bigcap}
\newcommand{\cross}{\times}


\hidesolutions
%\showsolutions

\newlength{\strutheight}
\newcommand{\prob}[1]{\mathop{\textup{Pr}} \nolimits \left( #1 \right)}
\newcommand{\prsub}[2]{\mathop{\textup{Pr}_{#1}}\nolimits\left(#2\right)}
\newcommand{\prcond}[2]{%
  \ifinner \settoheight{\strutheight}{$#1 #2$}
  \else    \settoheight{\strutheight}{$\displaystyle#1 #2$} \fi %
  \mathop{\textup{Pr}}\nolimits\left(
    #1\,\left|\protect\rule{0cm}{\strutheight}\right.\,#2
  \right)}
\newcommand{\comment}[1]{}
\newcommand{\cE}{\mathcal{E}}
\renewcommand{\setminus}{-}
\renewcommand{\complement}[1]{\overline{#1}}

\providecommand{\abs}[1]{\lvert#1\rvert}
\newcommand{\variance}[1]{(\Delta{#1})^2}

\begin{document}
\problemset{1}{September 4, 2014}{Monday, September 8}
%%%%%%%%%%%%%%%%%%%%%%%%%%%%%%%%%%%%%%%%%%%%%%%%%%%%%%%%%%%%%%%%%%%%%%%%%%%%%%%
\noindent \textbf{Reading Assignment:}   Chapter 1 \& Sections 2.0-2.4
\\

\begin{problem}{20} Suppose that $w^2 + x^2 + y^2 = z^2$, where $w,x,y,$ and $z$
always denote positive integers. (\textit{Hint:} It may be helpful to represent even
integers as $2i$ and odd integers as $2j+1$, where $i$ and $j$ are integers.)

Prove the proposition: $z$ is even if and only if $w, x,$ and $y$ are even. Do
this by considering all the cases of $w,x,y$ being odd or even.
\end{problem}

\solution{
\begin{proof}
As the problem suggests, we will build a truth table to figure out what $z$ is
in each case of $x,y,z$ being even.

\begin{itemize}

    \item $w,x,y$ are all even. In this case, we can write $w,x,y$ as $2i,2j,2k$,
        and the sum of squares is $4i^2 + 4j^2 + 4k^2 = 4(i^2 + j^2 + k^2)$. In
        this case, $z$ can be any even integer when $i^2 + j^2 + k^2 = l^2$ for
        some $l$.
    \item $w,x,y$ are all odd. In this case, each of their squares are odd, and the
        sum of three odd numbers is odd. However, if $z$ is even, then $z^2$ is
        also even. So it cannot be that $z^2 = w^2 + x^2 + y^2$.
    \item One of $w,x,y$ is odd. This case is the same as above. Assume $x$ is odd.
        The $x^2$ is odd, but $w^2$ and $y^2$ are even. So the sum is once again
        odd, which cannot equal the square of an even integer. So $z$ cannot be even
        in that case either. The same argument can be repeated for when $w$ and $y$
        are odd as well.
    \item Two of $w,x,y$ are odd. Assume that $w,x$ are odd and $y$ is even. Then
        we can write the sum as $w^2 + x^2 + y^2 = (2i + 1)^2 + (2j + 1)^2 + (2k)^2$.
        This can be rewritten as $4i^2 + 4i + 1 + 4j^2 + 4j + 1 + 4k^2 = 4(i^2 + i + j^2 + j) + 2$.
        However, if $z$ is even, then $z^2$ is a multiple of $4$. So $z$ cannot be an
        even integer in this case. The same argument can be used when $w,y$ are odd
        or when $x,z$ are odd.
\end{itemize}

\[
\begin{array}{ccc|c}
\text{w even} & \text{x even} & \text{y even} & \text{z even} \\ \hline
T & T & T & T \\
T & T & F & F \\
T & F & T & F \\
T & F & F & F \\
F & T & T & F \\
F & T & F & F \\
F & F & T & F \\
F & F & F & F
\end{array}
\]

\end{proof}
}

%%%%%%%%%%%%%%%%%%%%%%%%%%%%%%%%%%%%%%%%%%%%%%%%%%%%%%%%%%%%%%%%%%%%%%%%%%%%%%%
\begin{problem}{14}
The three logical expressions below are nearly equivalent.  However,
for one particular setting of $x$, $y$, and $z$, one of the
expressions is not like the others.  Write out \textit{complete truth
tables for all three}, and determine which one of these expressions
just doesn't belong.
%
\begin{align*}
1. & \qquad (x \rightarrow y) \wedge (y \rightarrow z) \wedge (z \rightarrow x) \\
2. & \qquad (\neg x) \wedge (\neg y) \wedge (\neg z) \\
3. & \qquad (\neg x \vee (y \wedge z)) \wedge (x \vee (\neg y \wedge \neg z))
\end{align*}

\solution{
\[
\begin{array}{c|c|c|c|c|c}
x & y & z & (1) & (2) & (3) \\ \hline
F & F & F & T & T & T \\
F & F & T & F & F & F \\
F & T & F & F & F & F \\
F & T & T & F & F & F \\
T & F & F & F & F & F \\
T & F & T & F & F & F \\
T & T & F & F & F & F \\
T & T & T & T & \textbf{F} & T
\end{array}
\]

Expression 2 just doesn't belong.
}

\end{problem}


%%%%%%%%%%%%%%%%%%%%%%%%%%%%%%%%%%%%%%%%%%%%%%%%%%%%%%%%%%%%%%%%%%%%%%%%%%%


\begin{problem}{14} A student is trying to prove that propositions $p$, $q$, and $r$ are
all true.  She proceeds as follows.  First, she proves three facts: $p
\rightarrow q$, $q \rightarrow r$, and $r
\rightarrow p$.  Then she concludes, ``thus obviously $p$, $q$, and
$r$ are all true.''  Let's first formalize her deduction as a logical
statement and then evaluate whether or not it is correct.


\bparts
\ppart{5} Using logic notation and the symbols $p$, $q$, and $r$,
write down the logical implication that she uses in her final step.

\solution{
\[
((p \rightarrow q) \wedge (q \rightarrow r) \wedge (r \rightarrow p))
\rightarrow (p \wedge q \wedge r)
\]
}

\ppart{5} Use a truth table to determine whether this logical
implication is a tautology (i.e., a universal truth in logic).
\solution{

\[
\begin{array}{ccc|c|c|c}
p & q & r &
((p \rightarrow q) \wedge (q \rightarrow r) \wedge (r \rightarrow p)) &
(p \wedge q \wedge r) &
\begin{array}{c}
\mbox{complete} \\
\mbox{expression}
\end{array} \\ \hline
T & T & T & T & T & T \\
T & T & F & F & F & T \\
T & F & T & F & F & T \\
T & F & F & F & F & T \\
F & T & T & F & F & T \\
F & T & F & F & F & T \\
F & F & T & F & F & T \\
F & F & F & T & F & \rightarrow F \leftarrow
\end{array}
\]

The truth table indicates that the implication she uses is {\em
not} a tautology.

}
\ppart{4} Is her proof that propositions $p$, $q$, and $r$ are all
true correct?  Briefly explain.

\solution{Her proof is incorrect; she makes use of a
false proposition.  (However, the first part of her argument is
sufficient to show that either all three statements are true {\em or}
all three statements are false.)
}

\eparts
\end{problem}


%%%%%%%%%%%%%%%%%%%%%%%%%%%%%%%%%%%%%%


\begin{problem}{24}
Translate the following statements into predicate logic.  For each,
specify the domain.  In addition to logic symbols, you
may build predicates using arithmetic, relational symbols, and
constants.  For example, the statement ``$n$ is an odd number'' could
be translated into $\exists m. (2m+1 = n)$, where the domain is $\mathbb{Z}$, 
the set of integers.  Another example, ``$p$ is a prime number,'' could be 
translated to 
\[
%(p > 1) \QAND\ \QNOT \paren{\exists m. \exists n. (m > 1 \QAND\ n > 1 \QAND\ mn = p)}
(p > 1) \text{   AND   } \text{   NOT} \paren{\exists m. \exists n. (m > 1 \text{   AND   } n > 1 \text{   AND   } mn = p)}
\]
Let $\text{prime}(p)$ be an abbreviation that you could use to denote the above formula 
in this problem.

\bparts

\ppart{4} (Lagrange's Four-Square Theorem) Every nonnegative integer is
expressible as the sum of four perfect squares.

\solution{The domain is $\mathbb{N}$, the natural numbers.

\[
\forall n. \exists w \exists x \exists y \exists z. (n = w^2 + x^2 + y^2 + z^2).
\]

}

\ppart{4} (Goldbach's Conjecture) Every even integer greater than two is
the sum of two primes.

\solution{
The domain is $\mathbb{N}$, the natural numbers.  The statement could be
translated as
\[
\forall n. 
%\paren{((n > 2) \QAND\ \exists m (n = 2m)) \QIMP\
%      \exists p \exists q (\text{prime}(p) \QAND\ \text{prime}(q) \QAND\ (n = p + q))}
\paren{((n > 2) \text{   AND   } \exists m (n = 2m)) \text{   IMPLIES   }
      \exists p \exists q (\text{prime}(p) \text{   AND   } \text{prime}(q) \text{   AND   } (n = p + q))}.
\]
Another translation is 
\[
\forall n. ((n > 2) \text{   AND   } \exists m (n = 2m)). \text{ } \exists p. \text{prime}(p). \text{ } \exists q. \text{prime}(q). \text{ }n = p + q.
\]
}

\ppart{4} The function $f : \mathbb{R} \mapsto \mathbb{R}
$ is continuous.

\solution{
The domain is $\mathbb{R}$, the real numbers.
\[
\forall a \forall x. \exists b. \forall y.
%\paren{(a > 0 \QAND\ b > 0 \QAND\ \abs{x-y} < b) \QIMP\ \abs{f(x) - f(y)} < a}
\paren{(a > 0 \text{   AND   } b > 0 \text{   AND   } \abs{x-y} < b) \text{   IMPLIES   } \abs{f(x) - f(y)} < a}.
\]
}

\ppart{4}
(Fermat's Last Theorem) There are no nontrivial solutions
to the equation:
\[
x^n + y^n = z^n
\]
over the nonnegative integers when $n > 2$.

\solution{
The domain is $\mathbb{N}$.
\[
\forall x \forall y \forall z \forall n.
%\paren{(x > 0 \QAND\ y > 0 \QAND\ z > 0 \QAND\ n > 2)
%    \QIMP\ \QNOT (x^n + y^n = z^n)}
\paren{(x > 0 \text{   AND   } y > 0 \text{   AND   } z > 0 \text{   AND   } n > 2)
\text{   IMPLIES   } \text{NOT} (x^n + y^n = z^n)}.
\]
}

\ppart{4}
There is no largest prime number.

\solution{
The domain is $\mathbb{Z}$, the integers.

\[
%\QNOT \paren{\exists p (\text{Prime}(p) \QAND\ (\forall q (\text{Prime}(q) \QIMP\ p \geq q)))}
\text{NOT} \paren{\exists p (\text{Prime}(p) \text{   AND   }(\forall q (\text{Prime}(q) \text{   IMPLIES   } p \geq q)))}.
\]
}

\ppart{4}
(Bertrand's Postulate) If $n > 1$, then there is always
at least one prime $p$ such that $n < p < 2n$.

\solution{
The domain is $\mathbb{Z}$.
\[
\forall n.
%\paren{ (n > 1) \QIMP\ (\exists p ( \text{Prime}(p)  \QAND\ (n < p) \QAND\ (p < 2n))) }
\paren{ (n > 1) \text{   IMPLIES   } (\exists p ( \text{Prime}(p)  \text{   AND   } (n < p) \text{   AND   } (p < 2n))) }.
\]

}

\eparts
\end{problem}


%%%%%%%%%%%%%%%%%%%%%%%%%%%%%%%%%%%%%%%%%%%%%%%%%%

\begin{problem}{16} You have a balance; by putting objects on each side of the balance, you 
can determine which side is heavier (or if both sides have the same weight).  

You also have a large collection of stones of known weight; there are stones that weigh 1 ounce, 
2 ounces, and so on all the way up to 40 ounces (and you have many stones of each weight).

You are given an object of unknown weight; all you know is that it weighs an integer number of 
ounces between 1 and 40, inclusive.  Your goal is to determine the weight of this object, by 
using as small a set of stones from your collection as possible (you may do as many weighings 
as you like with the stones you select).  Give a set of possible stones, and summarize the procedure 
that you will use to identify the weight of the object.  (You will get 8/16 points for using 6 stones; 
12/16 points for using just 5 stones; and 16/16 points for using only 4 stones).

\textit{Hint:}   To identify the unknown weight, you do not need to make every weight from 1 through 40 (e.g., 
making only the even weights also helps you identify the unknown weight should it be odd.)

\textit{Additional hint:}   When using the balance, you can place some stones on the same side as the object of unknown weight.

\solution{There are many possible sets of size 4, 5, and 6 that work.  As long as you explained your solution, you will receive 
credit commensurate to the number of stones you use.

We will show two possible ways to approach this problem, and then optimize these 
approaches with the same key insights to get from 6 stones to 4 stones.

Approach 1 involves building our set of stones "bottom up": that is, first ensuring that you can 
identify an unknown stone of weight 1, then ensuring that you can identify an unknown stone 
of weight 2, then ensuring that you can identify an unknown stone of weight 3 . . ., and finally 
ensuring that you can identify an unknown stone of weight 40.  Using this approach, one set of 
stones we could pick is {1, 2, 4, 8, 16, 32}, which is equivalent to the set $\{ 2^{0}, 2^{1}, 2^{2}, 2^{3}, 
2^{4}, 2^{5} \}$.  Using this set of stones, we can ``make" every weight from 1 to 40 (in fact, from 1 to 63): that is, 
to make a particular weight, we look at the binary representation of the weight, and use the stones 
that correspond to the binary representation.  For example, to make 13, whose binary representation is 1101, we should 
use the stones 1, 4, and 8. 

Approach 2 starts in the middle of the weight range for the unknown object, and we will refer to it as the ``binary search" approach: first, 
ensure that you can identify a stone in the middle, which will divide the weight range of unidentifiable object into two groups.  Then, we will pick a stone that can be used to identify a weight in the middle of all resulting groups, and repeat the process until we have no possible weights that cannot be identified.  Using this approach, the set of stones that we pick is $\{  20, 10, 5, 3, 2, 1 \}$.  Using this set of stones, we can again make every weight from 1 through 40: we can make 1 through 10 using the stones $\{  5, 3, 2, 1 \}$, and to 
make larger weights, we add either the 10oz stone or the 20oz stone (or both) to the stones we use to make 1 through 10.

Since both approaches can make every possible weight of the unknow object, we can incrementally test every possible weight to see what 
the unknown weight is.  Using the hint, we can modify the bottom up approach to only include the set of stones 
$\{  20, 4, 8, 16, 32 \}$.  Using these even-ounce stones, we can make every even weight from 1 through 40.  Likewise, we can use the hint to modify the binary search approach to only include the set of stones $\{  20, 10, 6, 4, 2 \}$: using the stones 2, 4, and 
6, we can make every even number from 1 through 10, and by also using the 10oz and 20oz stones, we can make every even number 
between 1 through 40.  

Using the additional hint, we can improve both approaches even more.  We can modify the (original) bottom up approach 
to only include the stones $ \{ 1, 3, 9, 37 \}$ (or $\{3^{0}, 3_{1}, 3^{2}, 3^{3} \}$), which is a set of stones that can create every 
number from 1 through 40.  For example, to create 2, we can put 3 on the side opposite the unknown weight, and 1 on the same side as 
the unknown weight.  We can use the stones 1 and 3 to create the numbers 1 through 4; then by adding the stone 9, create the numbers 5-9 (by putting 9 and ``substracting 0-4"); and create 10-13 (by putting 9 and adding 1-4).  Finally, by adding the stone 27, we can 
create the numbers 14-27 (by putting 27 and substracting 0-13") or create 28-40 (by putting 27 and adding 1-13).  
Since we can create every weight from 1 through 40, we can incrementally check through the possible weight range.  We can also modify the binary search approach using the additional hint to get $\{20, 12, 6, 2 \}$, which is the set that can create 
every even weight from 1 through 40.
}

\end{problem}

%%%%%%%%%%%%%%%%%%%%%%%%%%%%%%%%%%%%%%%%%

\begin{problem}{12}
A \textit{triangle} is a set of three people such that either every
pair has shaken hands or no pair has shaken hands.  

\bparts
\ppart{8}
Prove that among every six people there is a triangle. 

\noindent(Suggestion: Initially, break the problem into two cases:
%
for any person $X$ in a given group of six people,
\begin{enumerate}
\item there exist at least three other people who shook hands with $X$;
\item there exist at least three other people who didn't shake hands with $X$.
\end{enumerate}
%
Why must exactly one of these conditions hold?)

\solution{
\begin{proof}
We first prove the statement given in the suggestion, which we state as a lemma:

\begin{lemma*} For any person $X$ in a given group of six people, exactly one of these conditions holds:
\begin{enumerate}
\item there exist at least three other people who shook hands with $X$;
\item there exist at least three other people who didn't shake hands with $X$.
\end{enumerate}
\end{lemma*}

\begin{proof} We prove the lemma by case analysis. There are there possible cases: neither condition holds; both conditions hold; or exactly one of the two conditions holds.

\begin{itemize}
\item Neither condition holds: that is, at most two other people have shaken hands with $X$, and at most two other people have not shaken hands with $X$. So at most four other people have shaken hands or have not shaken hands with $X$. However, since the set of people who are distinct from $X$ consists of five individuals and each individual must either have shaken hands with $X$ or have not shaken hands with $X$, we have a contradiction.

\item Both conditions hold: that is, at least three other people have shaken hands with $X$, and at least three other people have not shaken hands with $X$. Since no individual can be counted in both groups, the union of these two sets has at least six members. This contradicts the fact that there are only five individuals distinct from $X$.
\end{itemize}

Since two of the three cases lead to contradiction, we conclude that the only remaining possibility---that exactly one of the two conditions holds---must be correct.
\end{proof}

We now prove the proposition directly using case analysis. 

Let $X$ denote one of the six people. By our lemma, there
are two possibilities.
%
\begin{enumerate}
\item There exist three other people who shook hands with $X$.  Now
there are two further possibilities:
%
\begin{enumerate}
\item Among these three, some pair shook hands.  Then these two and
$X$ form a triangle.
\item Among these three, no pair shook hands.  Then these three form a
triangle.
\end{enumerate}
%
\item Otherwise, there are exactly three people who did not shake hands with $X$.
%Thus, there exist three people who didn't shake hands with $X$.
Again, there are two further possibilities:
%
\begin{enumerate}
\item Among these three, every pair shook hands.  Then these three
form a triangle.
\item Among these three, some pair did not shake hands.  Then these two
and $X$ form a triangle.
\end{enumerate}
\end{enumerate}
\end{proof}
}

\ppart{4}
Also, is there always a triangle among every five people?

\solution{
No. One counter example is for persons A,B,C,D,E where

~~~ A has shaken hands with B and E;

~~~ B has shaken hands with A and C;

~~~ C has shaken hands with B and D;

~~~ D has shaken hands with C and E;

~~~ E has shaken hands with D and A.
}
\eparts

\end{problem}

%%%%%%%%%%%%%%%%%%%%%%%%%%%%%%%%%%%%%%%%%%%%%%%%%%%%%%%%%%%%%%%%%%%%%%%%%%%%%%%





\end{document}
