\documentclass[12pt]{article}
\usepackage{../light}

\hidesolutions
%\showsolutions

\newcommand{\abs}[1]{\left|#1\right|}

\begin{document}

\recitation{12}{October 17, 2014}
\newcommand{\bigbox}{\fbox{\vspace{0.5in} \hspace{0.75in}}}


\section{Asymptotic Notation}
Which of these symbols
%
\[
\Theta \qquad O \qquad \Omega \qquad o \qquad \omega
\]
%
can go in these boxes? (List all that apply.)

{\large
\begin{align*}
2 n + \log n \qquad & = \qquad \bigbox \ ( n ) \\
\insolutions{& \Theta, O, \Omega} \\
\log n \qquad & = \qquad \bigbox \ ( n ) \\
\insolutions{& O, o} \\
\sqrt{n} \qquad & = \qquad \bigbox \ ( \log^{300} n ) \\
\insolutions{& \Omega, \omega } \\
n 2^n \qquad & = \qquad \bigbox \ ( n ) \\
\insolutions{& \Omega, \omega } \\
n^7 \qquad & = \qquad \bigbox \ ( 1.01^n ) \\
\insolutions{& O, o}
\end{align*}
}


%%%%%%%%%%%%%%%%%%%%%%%%%%%%%%%%%%%%%%%%%%%%%%%%%%%%%%%%%%%%%%%%%%%%%%%%%%%%%%%

\newpage


%%%%%%%%%%%%%%%%%%%%%%%%%%%%%%%%%%%%%%%%%%%%%%%%%%%%%%%%%%%%%%%%%%%%%%%%%%%%%%%

%% added by Oscar. 
\section{Asymptotic Equivalence}

Suppose $f,g:\mathbb{Z}^+ \to \mathbb{Z}^+$ and $f \sim g$.

\begin{enumerate}
\item Prove that $2f \sim 2g$.

\solution{
\[
\frac{2f}{2g} = \frac{f}{g},
\]
so they have the same limit as $n \to infty$.
}

\item Prove that $f^2 \sim g^2$.

\solution{

\[
\lim_{n\to\infty} \frac{f(n)^2}{g(n)^2} = \lim_{n\to\infty} \frac{f(n)}{g(n)} \cdot \frac{f(n)}{g(n)}
 = \lim_{n\to\infty} \frac{f(n)}{g(n)} \cdot \lim_{n\to\infty} \frac{f(n)}{g(n)} = 1 \cdot 1 = 1.
\]

}

\item Give examples of $f$ and $g$ such that $2^f \not\sim 2^g$.

\solution{

\begin{align*}
f(n) & = n+1\\
g(n) & = n.
\end{align*}

Then $f \sim g$ since $\lim (n+1)/n = 1$, but $2^f= 2^{n+1} = 2\cdot 2^n =
2\cdot 2^g$ so
\[
\lim \frac{2^f}{2^g} = 2 \neq 1.
\]


}

\item Show that $\sim$ is an equivalence relation

\solution{ 
\begin{enumerate}

\item Reflexive: $f \sim f$ since $f(x)/f(x) = 1$ for all $x$ (assuming $f(x) \neq 0$), so $\lim_{x \to \infty} f(x)/f(x) = 1$

\item Symmetric: $f \sim g$ implies $g \sim f$ since if $\lim_{x \to \infty} f(x)/g(x)= 1$, then by the laws of limits $lim_{x \to \infty} g(x)/f(x) = 1$

\item Transitive: $f \sim g$ and $g \sim h$ implies $f\sim h$:  if $\lim_{x \to \infty} f(x)/g(x) = 1$, and $\lim_{x \to \infty} g(x)/h(x) = 1$, then multiplying the limits we get $$\lim_{x \to \infty}  f(x)/h(x)  = \lim_{x \to \infty} \dfrac{f(x)}{g(x)} \times \dfrac{g(x)}{h(x)} = 1$$
\end{enumerate}

}

\item Show that $\Theta$ is an equivalence relation

\solution{ 
\begin{enumerate}
\item Reflexive: $\lim_{x \to \infty} f(x)/f(x)  = 1 < \infty$, trivial.

\item Symmetric: If $f = \Theta(g)$, we wish to show $g = \Theta(f)$. From the definiton: $\lim_{x \to \infty} f(x)/g(x) = c$ for some non-zero finite constant c. Hence $\lim_{x \to \infty} g(x)/f(x) = 1/c$. Also a non-zero finite constant, so $g = \Theta(f)$.

\item Transitive: Want to show $f = \Theta(g), g=\Theta(h)$  then $f = \Theta(h)$. 
Let $\lim_{x \to \infty} f(x)/g(x) = c_1 $  and $\lim_{x \to \infty} g(x)/h(x) = c_2 $. Then $\lim_{x \to \infty} f(x)/h(x) = \lim_{x \to \infty} f(x)/g(x) \times g(x)/h(x)  = c_1\times c_2 $. Since both $c_1$ and $c_2$ are non-zero and finite, so is $c_1 \times c_2$.
\end{enumerate}
}

\end{enumerate}

\section{More Asymptotic Notation}

\begin{enumerate}

\item
Show that
\[
(an)^{b/n} \sim 1.
\]
where $a,b$ are positive constants and $\sim$ denotes asymptotic equality.
Hint $an=a2^{\log_2 n}$.

\solution{
\[
(an)^{b/n} = \paren{a^b}^{1/n} \cdot 2^{(b\log_2 n)/n} \to 1 \cdot 2^0 = 1,
\]
as $n \to \infty$.
}

\item
You may assume that if $f(n) \geq 1$ and $g(n) \geq 1$ for 
all $n$, then $f \sim g \implies f^\frac{1}{n} \sim g^\frac{1}{n}$. 
Show that
\[
\sqrt[n]{n!} = \Theta(n).
\]

\solution{
\begin{align*}
\sqrt[n]{n!} &
\sim \paren{(2 \pi n)^{\frac{1}{2}} \paren{\frac{n}{e}}^n}^{1/n}
& \text{(Stirling)}\\
& \sim (2 \pi n)^{\frac{1}{2n}} \frac{n}{e}\\
& \sim 1 \cdot  \frac{n}{e}  & \text{part (a)}\\
             & = \Theta(n)
\end{align*}
}

\end{enumerate}

\end{document}
