\documentclass[12pt]{article}
\usepackage{light}
\usepackage[stable]{footmisc}

\showsolutions
%\hidesolutions

\newtheorem{false-theorem}[theorem]{False Theorem}

\begin{document}

\recitation{4}{September 19, 2012}
%%%%%%%%%%%%%%%%%%%%%%%%%%%%%%%%%%%%%%%%%%%%%%%%%%%%%%%%%%%%%%%%%%%%%%%%%%%%%%%

\insolutions{

\section{The Pulverizer}
\label{sec:pulverizer}

We saw in lecture that the greatest common divisor (GCD) of two
numbers can be written as a linear combination of
them.\footnote{In fact, we proved that among all
positive linear combinations of the numbers their GCD is the
smallest.}  That is, no matter which pair of integers $a$ and $b$ we
are given, there is always a pair of integer coefficients $s$ and $t$
such that     
$$
\gcd(a, b)  =  s a + t b.
$$
However, the proof was \emph{nonconstructive}: it didn't suggest
a way for finding such $s$ and $t$. 

\noindent
That job is tackled by a mathematical tool that dates to sixth-century
India, where it was called {\em kuttak}, which means ``The
Pulverizer''.  Today, the Pulverizer is more commonly known as ``the
extended Euclidean GCD algorithm'', but that's lame.  We're sticking
with ``Pulverizer''. 

\noindent
Euclid's algorithm for finding the GCD of two numbers relies on
repeated application of the equation: 
\[
\gcd(a, b) = \gcd(b, \rem(a,b))
\]
which was proved in lecture (see the notes ``Number Theory~I'').  For 
example, we can compute the GCD of 259 and 70 as follows:
\[
\begin{array}{rclcl}
\gcd(259, 70)
    & = & \gcd(70, 49) & \quad & \text{since $\rem(259,70) = 49$}\\
    & = & \gcd(49, 21) && \text{since $\rem(70,49) = 21$} \\
    & = & \gcd(21, 7) && \text{since $\rem(49,21) = 7$} \\
    & = & \gcd(7, 0) && \text{since $\rem(21,7) = 0$} \\
    & = & 7.
\end{array}
\]
The Pulverizer goes through the same steps, but requires some extra
bookkeeping along the way: as we compute $\gcd(a, b)$, we keep track
of how to write each of the remainders (49, 21, and 7, in the example)
as a linear combination of $a$ and $b$ (this is worthwhile, because
our objective is to write the last nonzero remainder, which is the
GCD, as such a linear combination).  For our example, here is this
extra bookkeeping:
\[
\begin{array}{ccccrcl}
x & \quad & y & \quad & (\rem(x,y)) & = & x - q \cdot y \\ \hline
259 && 70 && 49 & = &   259 - 3 \cdot 70 \\
70 && 49 && 21  & = &   70 - 1 \cdot 49 \\
&&&&            & = &   70 - 1 \cdot (259 - 3 \cdot 70) \\
&&&&            & = &   -1 \cdot 259 + 4 \cdot 70 \\
49 && 21 && 7   & = &   49 - 2 \cdot 21 \\
&&&&            & = &   (259 - 3 \cdot 70) -
                                2 \cdot (-1 \cdot 259 + 4 \cdot 70) \\
&&&&            & = &   \fbox{$3 \cdot 259 - 11 \cdot 70$} \\
21 && 7 && 0
\end{array}
\]
We began by initializing two variables, $x = a$ and $y = b$.  In the
first two columns above, we carried out Euclid's algorithm.  At each
step, we computed $\rem(x,y)$, which can be written in the form $x - q
\cdot y$.  (Remember that the Division Algorithm says $x = q \cdot y +
r$, where $r$ is the remainder.  We get $r = x - q \cdot y$ by
rearranging terms.)  Then we replaced $x$ and $y$ in this equation
with equivalent linear combinations of $a$ and $b$, which we already
had computed.  After simplifying, we were left with a linear
combination of $a$ and $b$ that was equal to the remainder as desired.
The final solution is boxed.

\newpage
} % end insolutions

\section{Problem: The Pulverizer!}

There is a pond. Inside the pond there are $n$ pebbles, arranged in a
cycle. A frog is sitting on one of the pebbles. Whenever he jumps, he
lands exactly $k$ pebbles away in the clockwise direction, where
$0<k<n$. The frog's meal, a delicious worm, lies on the pebble right
next to his, in the clockwise direction.  

\textbf{(a)}
Describe a situation where the frog can't reach the worm.

\solution[\vspace*{3em}]{If $k\mid n$ (say $k=3$ and $n=6$), then no
number of jumps will lead the frog to the worm, as the frog will be
returning to his original pebble ad infinitum.}

\textbf{(b)}
In a situation where the frog can actually reach the worm, explain how
to use the Pulverizer to find how many jumps the frog will need.

\solution[\vspace*{5em}]{
Suppose the frog can reach the worm. When he actually reaches it, he
has jumped a number of times, say $j$, and he has travelled around the
cycle a number of times, call it $c$. Then, the distance that the frog
has covered is both $j\cdot k$ and $c\cdot n + 1$, so that 
$$
jk = cn + 1.
$$
But this means that $1$ can be written as a \emph{linear combination}
of $n$ and $k$:
$$
(-c)n + jk = 1.
$$
Since $1$ is positive, we conclude that it is a \emph{positive linear
combination} of $n$ and $k$. And since it is the smallest positive
integer, we also conclude that it is the \emph{smallest positive
linear combination} of $n$ and $k$. But we have proved in lecture that
the smallest positive linear combination of two integers is their
GCD. So, the GCD of $n$ and $k$ is 1:
$$
\gcd(n,k) = 1
$$
and we can use the Pulverizer to find $-c$ and $j$.}

\noindent\textbf{(c)}
Compute the number of jumps if $n=50$ and $k=21$. Anything strange? 
Can you fix it?

\solution{We go through the steps as described in
Section~1 (see the table below) to get that $1=8\cdot 50-19\cdot
21$,  or $1=-(-8)\cdot 50 + (-19)\cdot 21$. Hence, $c=-8$ and
$j=-19$, which makes little sense.  What does it mean for the
frog to make $-19$ jumps?

The point is that the Pulverizer is guaranteed to give us the integer 
coefficients of a linear combination that equals the GCD, but it
promises nothing about the signs of those coefficients (in
this case we wanted them to be $-$ and $+$). To get coefficients of the
desired sign, we have to think more.

One way to fix it is as explained in lecture. That is, subtract 21
from 8 and add 50 to $-19$.

Here is another way: We know $1=8\cdot 50-19\cdot 21$. Or, to obtain
meaningful signs for the numbers, $19\cdot 21 = 8\cdot 50 -1$. That
is, after $19$ jumps the frog will land $1$ pebble short of $8$ full
cycles. So, he will be right next to his original pebble, but in the
counter-clockwise direction. Given this, how can he reach the pebble
he is after?

Well, if he makes $19$ more jumps, he will land 2 pebbles away from
his original position in the counter-clockwise direction. Another $19$
jumps will lead him 3 pebbles away, and so on. After a total of $49$
sets of $19$ jumps, he will be $49$ pebbles away from its original
position in the counter-clockwise direction, which is of course
the worm's pebble. Then, the frog will have made $49*19=931$ jumps.

Here is the table produced by the Pulverizer:
\[
\begin{array}{ccccrcl}
x & \quad & y & \quad & (\rem(x,y)) & = & x - q \cdot y \\ \hline
 50 &&  21 &&  8 & = &    50 - 2 \cdot  21 \\
 21 &&   8 &&  5 & = &    21 - 2 \cdot  8 \\
&&&&             & = &    21 - 2 \cdot (50 - 2 \cdot  21) \\
&&&&             & = &   -2 \cdot 50 + 5 \cdot 21 \\
  8 &&   5 &&  3 & = &    8 - 1 \cdot 5  \\
&&&&             & = &   (50 - 2 \cdot  21) 
                         -1 \cdot (-2 \cdot 50 + 5 \cdot 21) \\
&&&&             & = &   3\cdot 50 -7 \cdot 21 \\
  5 &&   3 &&  2 & = &    5 - 1\cdot 3 \\
&&&&             & = &   (-2 \cdot 50 + 5 \cdot 21) 
                         -1 \cdot (3\cdot 50 -7 \cdot 21) \\
&&&&             & = &   -5\cdot 50 + 12 \cdot 21 \\
  3 &&   2 &&  1 & = &    3 - 1\cdot 2 \\
&&&&             & = &   (3\cdot 50 -7 \cdot 21) 
                         -1\cdot (-5\cdot 50 + 12 \cdot 21) \\
&&&&             & = &   \fbox{$8\cdot 50 - 19 \cdot 21$} \\
  2 &&   1 &&  0 &   & 
\end{array}
\]
}


\iffalse
The Pulverizer quickly solves many problems involving congruences.
For example, suppose that we want to find a multiplicative inverse of
$k$ modulo $n$, assuming such an inverse exists.  First, we use the
Pulverizer to find integers $s$ and $t$ such that:
\[
s k + t n =  1
\]

If this equation holds, then we know that:

\[
s k \equiv 1 \pmod{n}
\]

Therefore, $s$ is the multiplicative inverse of $k$.  Problem pulverized!
\fi


%%%%%%%%%%%%%%%%%%%%%%%%%%%%%%%%%%%%%%%%%%%%%%%%%%%%%%%%%%%%%%%%%%%%%%%%%%
\newpage
\section{Problem: The Fibonacci numbers. }

The Fibonacci numbers are defined as follows:
$$
F_0 = 0 \qquad 
F_1 = 1 \qquad 
F_n = F_{n-1} + F_{n-2} \quad \text{(for $n \geq 2$)}.
$$
Give an inductive proof that the Fibonacci numbers $F_n$ and $F_{n+1}$
are relatively prime for all $n \geq 0$.  

\solution{We use induction on $n$.  Let $P(n)$ be the proposition that $F_n$
and $F_{n+1}$ are relatively prime.

\noindent \textit{Base case:} $P(0)$ is true because $F_0 = 0$ and $F_1 = 1$
are relatively prime.

\noindent \textit{Inductive step:} We show that, for all $n\geq 0$,
$P(n)$ implies $P(n+1)$. So, fix some $n\geq 0$ and assume that $P(n)$
is true, that is, $F_n$ and $F_{n+1}$ are relatively prime.  We will
show that $F_{n+1}$ and $F_{n+2}$ are relatively prime as well.  We
will do this by contradiction.

Suppose $F_{n+1}$ and $F_{n+2}$ are not relatively prime. Then they
have a common divisor $d > 1$. But then $d$ also divides the linear
combination $F_{n+2} - F_{n+1}$, which actually equals $F_n$. Hence,
$d>1$ divides both $F_n$ and $F_{n+1}$. Which implies $F_n$, $F_{n+1}$
are not relatively prime, a contradiction to the inductive hypothesis.

Therefore, $F_{n+1}$ and $F_{n+2}$ are relatively prime. That is,
$P(n+1)$ is true.

The theorem follows by induction.}

%%%%%%%%%%%%%%%%%%%%%%%%%%%%%%%%%%%%%%%%%%%%%%%%%%%%%%%%%%%%%%%%%%%%%%%%%%

\newpage
\section{Extra Problem: The power of 3.\footnote{Try this if you have time!}}
Let $N$ be a number whose decimal expansion consists of $3^n$
identical digits.  Show by induction that $3^n \mid N$.  For example:
\[
3^2 \mid \underbrace{777777777}_{\text{$3^2 = 9$ digits}}
\]
Recall that $3$ divides a number iff it divides the sum of its digits.

\solution{We proceed by induction on $n$.  Let $P(n)$ be the
proposition that $3^n \mid N$, where the decimal expansion of $N$ 
consists of $3^n$ identical digits.

\noindent \textit{Base case.}  $P(0)$ is true because $3^0 = 1$
divides every number.

\noindent \textit{Inductive step.}  Now we show that, for all $n\geq
0$, $P(n)$ implies $P(n+1)$.  Fix any $n\geq 0$ and assume $P(n)$ is
true. Consider a number whose decimal expansion consists of $3^{n+1}$
copies of the digit $a$: 
\begin{align*}
\underbrace{aaaaaa \ldots aaaaaa}_{\text{$3^{n+1}$ digits}}
    & = \underbrace{aaa \ldots aaa}_{\text{$3^n$ digits}}
        \underbrace{aaa \ldots aaa}_{\text{$3^n$ digits}}
        \underbrace{aaa \ldots aaa}_{\text{$3^n$ digits}} \\
    & = \underbrace{aaa \ldots aaa}_{\text{$3^n$ digits}}
        \quad \cdot \quad
        1
        \underbrace{000 \ldots 001}_{\text{$3^n$ digits}}
        \underbrace{000 \ldots 001}_{\text{$3^n$ digits}} \\
\end{align*}
Now $3^n$ divides the first term by the assumption $P(n)$, and 3
divides the second term since the digits sum to 3.  Therefore, the
whole expression is divisible by $3^{n+1}$.  This proves $P(n+1)$.

By the principle of induction $P(n)$ is true for all $n \geq 0$.
} % end solution

\end{document}

\newpage

\section{Dressed to the nines.}

Give a proof by induction that $10^k \equiv 1 \pmod{9}$ for all
$k \geq 0$.  Why is a number written in decimal evenly divisible by 9
if and only if the sum of its digits is a multiple of 9?

\solution{The claim holds for $k = 0$, since $10^0 \equiv 1 \pmod{9}$.
Suppose the claim holds for some $k \geq 0$; that is, $10^k \equiv 1
\pmod{9}$.  Multiplying both sides by 10 gives $10^{k+1} \equiv 10
\equiv 1 \pmod{9}$.  So the claim holds for $k+1$ as well.

A number in decimal has the form:

\[
d_k \cdot 10^k + d_{k-1} \cdot 10^{k-1} + \ldots + d_1 \cdot 10 + d_0
\]

From the observation above, we know:

\begin{eqnarray*}
d_k \cdot 10^k + d_{k-1} \cdot 10^{k-1} + \ldots + d_1 \cdot 10 + d_0
    & \equiv & d_k + d_{k-1} + \ldots + d_1 + d_0 \pmod{9} \\
    \end{eqnarray*}

    This shows something stronger: the remainder when the original number
    is divided by 9 is equal to the remainder when the sum of the digits
    is divided by 9.  In particular, if one is zero, then so is the other.
    }


%%%%%%%%%%%%%%%%%%%%%%%%%%%%%%%%%%%%%%%%%%%%%%%%%%%%%%%%%%%%%%%%%%%%%%%%%%%%%%%



