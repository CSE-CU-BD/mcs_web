\documentclass[handout]{mcs}

\begin{document}

\renewcommand{\reading}{ \emph{For this pset}:
  Chapter 14.8-14.11 {\em Pigeonhole Principle, Inclusion-Exclusion},
  Chapter~15 {\em Generating Functions},

For Friday:  Chapter~\bref{monty_sec}{--\bref{probability_sets_sec}{
      Discrete Probability Spaces}}.
}

\problemset{10}

%%%%%%%%%%%%%%%%%%%%%%%%%%%%%%%%%%%%%%%%%%%%%%%%%%%%%%%%%%%%%%%%%%%%%
% Problems start here
%%%%%%%%%%%%%%%%%%%%%%%%%%%%%%%%%%%%%%%%%%%%%%%%%%%%%%%%%%%%%%%%%%%%%

\pinput{CP_7777}

%\pinput{PS_reverse_linear_recurrences}

\pinput{PS_gen_fcn_quotient_polynomials} %%%%%from: S09.cp12t

\pinput{PS_Catalan_numbers_meyer_version} %%%%revised 8/17/11 by ARM because earlier version assumed def of good-count that was commented out of recursive_data}

%\pinput{PS_gen_fcns_pennies_nickels_etc} %%%%%%%%%from: S08.ps8

% FALL11\pinput{PS_crazy_pet_lady}

% FALL11\pinput{PS_linear_recur_closed_form} %%%%%%from: F06.ps8 modified to use generating function

% FALL11\pinput{PS_four_door_random_or_not} %extends the monty hall problem to 4 doors

%\pinput{PS_random_poker_hand} %uses choose extensively to compute the probability of poker hands

%\pinput{PS_union_bound} %a neat and simple inequality

%%%%%%%%%%%%%%%%%%%%%%%%%%%%%%%%%%%%%%%%%%%%%%%%%%%%%%%%%%%%%%%%%%%%%
% Problems end here
%%%%%%%%%%%%%%%%%%%%%%%%%%%%%%%%%%%%%%%%%%%%%%%%%%%%%%%%%%%%%%%%%%%%%

\end{document}

