\documentclass[handout]{mcs}

\begin{document}

\renewcommand{\reading}{Part~\bref{part:proofs}{. \emph{Proofs:
      Introduction}}, Chapter~\bref{proofs_chap}{, \emph{What is a
      Proof?}}; Chapter~\bref{well_ordering_chap}{, \emph{The Well
      Ordering Principle}}; and Chapter~\bref{logicform_chap}{ through
    \bref{SAT_sec}{, covering \emph{Propositional Logic}}}.  These
  assigned readings \textbf{do not include the Problem sections}.  (Many
  of the problems in the text will appear as class or homework problems.)}

\problemset{1}

  \emph{Reminders}:
\begin{itemize}

\item 

  \href{http://courses.csail.mit.edu/6.042/spring12/courseinfo#comments}{Email
    comments} on the different reading assignments are due by 10PM the
  night before class on days indicated in the class calendar (also in
  online tutor problem set TP.2).  Make comments using the class
  \href{http://nb.csail.mit.edu}{NB annotation system}.  Reading
  Comments count for 5\% of the final grade.

\item Problems should be submitted separately following the pset
  \href{http://courses.csail.mit.edu/6.042/spring12/submission}{submission
    instructions}, and each problem should have a \emph{collaboration
    statement} at the beginning, with the requisite information
  written in or attached using the
  \href{http://courses.csail.mit.edu/6.042/spring12/submission_template.pdf}{collaboration
    statement template}.

 \end{itemize}

%%%%%%%%%%%%%%%%%%%%%%%%%%%%%%%%%%%%%%%%%%%%%%%%%%%%%%%%%%%%%%%%%%%%%
% Problems start here
%%%%%%%%%%%%%%%%%%%%%%%%%%%%%%%%%%%%%%%%%%%%%%%%%%%%%%%%%%%%%%%%%%%%%

\pinput{PS_log7_not_in_QZ}

%\pinput{PS_3_exponent_inquality}

%\pinput{PS_printout_binary_strings}

\pinput{CP_PorQorR_equiv}

\pinput{CP_valid_vs_satisfiable}

\pinput{PS_prime_polynomial_41}

\iffalse
\begin{center}
\large \textbf{Optional:}
\end{center}

\pinput{PS_faster_adder_logic}
\fi

%%%%%%%%%%%%%%%%%%%%%%%%%%%%%%%%%%%%%%%%%%%%%%%%%%%%%%%%%%%%%%%%%%%%%
% Problems end here
%%%%%%%%%%%%%%%%%%%%%%%%%%%%%%%%%%%%%%%%%%%%%%%%%%%%%%%%%%%%%%%%%%%%%
\end{document}
