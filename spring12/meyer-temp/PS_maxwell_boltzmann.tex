\problemdata                                                             
        {Physics models} % latex-friendly label for the prob.
        {Statistical Mechanics} % The topic(s) of the problem content.  
        {?} % Source (if known)                                          
        {S01 Tutorial 11, 3} % Usage (list ones you are aware of). 
        {Tina Wang, S02} % Last revision info (author, date).   
                                                                         
                                                                         
\begin{problem}           
Statistical Mechanics studies the behavior of large systems of
particles (such as a kettle of boiling water).  One of the most basic
physical systems is a {\em gas:} a collection of particles (electrons,
atoms, photons) floating around in a large container. 

One way to describe this system is to divide the container into lots
of tiny boxes.  
The ``state'' of the system says how the particles are distributed among
the tiny boxes. 
The system is equally likely to be in any one of its states (assuming
they all have the same energy, but discussing this would take us too
far afield).  
Many facts about the aggregate behavior of the system arise from
averaging over all its possible states.

It turns out that different kinds of particles exhibit different kinds
of aggregate behavior.
These differences can be explained by different assumptions about the 
system states.

The {\em Maxwell-Boltzmann\/} model assumes that the particles are 
{\em distinguishable}, say, numbered $1,\ldots$.
A state is described by where particle $1$ is, where particle $2$ is,
and so on. 
In the Maxwell-Boltzmann model, a single particle is equally
likely to be in any one of the tiny boxes.
Nature follows these rules when all the particles are
distinct (e.g., different atoms).

\begin{problemparts}

\problempart Suppose that there are $r$ particles in
the container and $n$ tiny boxes.
How many possible states are there under the Maxwell-Boltzmann model?

\solution{
Number of states: $n^r$
}

On the other hand, the {\em Bose-Einstein\/} model assumes that the
particles are {\em indistinguishable\/}.
A state is described by the {\em number\/} of particles appearing in
each tiny box. 
In the Bose-Einstein model, all the distributions of numbers of
particles in tiny boxes are equally likely.
Gases made up of photons obey this model.
% Photons are a kind of {\em Boson}
% and so is Helium 2 (the superconducting stuff).
% [[[ What is this last sentence saying?]]]

\problempart How many possible states are there under the Bose-Einstein model?

\solution{
Using the stars and bars model, we get that the number of possible states is ${n+r-1 \choose r}$.
}

The Maxwell-Boltzmann and Bose-Einstein models predict very different
aggregate system behavior.
For example, suppose $r/n = \lambda$ is the average number of particles per cell.  

\problempart Prove that in the Maxwell-Boltzmann model, the fraction of empty
cells is about
\[
e^{-\lambda}
\].

\solution{
The probability that a given cell is empty is just $(1 -
\frac{1}{n})^r$.
Using the approximation $(1 - \frac{1}{n}) \approx e^{-\frac{1}{n}}$
(valid for large $n$), 
this yields approximately $e^{-\frac{r}{n}} = e^{-\lambda}$.
This probability should be close to the fraction of empty cells.
}

\problempart Prove that under Bose--Einstein Statistics, the fraction of empty
cells is about
\[
\frac{1}{1+\lambda}
\]

\solution{
The probability that a given cell is empty is just the ratio of
the number of states in which that cell is empty to the total number
of states.
The total number of states is ${n+r-1 \choose r}$.
The number of states in which the given cell is empty is
the same as the number of states for a system with $n-1$ tiny boxes
and $r$ particles, namely, ${n+r-2 \choose r}$.
The ratio works out to $\frac{n-1}{n+r-1} = \frac{1 - \frac{1}{n}}{1 +
\frac{r}{n} - \frac{1}{n}}$.
Since we assume $n$ is very large, 
this is approximately $\frac{1}{1 + \frac{r}{n}}$, as needed.

When $\lambda$ is large, the first quantity is a lot less than the
second.  In other words, Maxwell-Boltzmann predicts a more spread-out
gas than you would actually find.
}

\end{problemparts}
\end{problem}
                                 
