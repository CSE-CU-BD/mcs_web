        \problemdata       % Takes 5 *mandatory* arguments
        {variance-6}             % latex-friendly label for the prob.
        {variance}               % The topic of the problem content.
        {Velleman}               % Source (if known)
        {S98 PS11-7; F97 PS11-8} % Usage (list ones you are aware of).
        {Theory Pig, S02}        % Last revision info (author, date).
% fall 01, pset 12, problem 4
\begin{problem}

Suppose we have a calculation that will require $n$~operations to
complete, and our computer has a mean time to failure of
$f$~operations where the failure process is memoryless.  When the
computer fails, it loses all state, so the calculation has to be
restarted from the beginning.

\begin{problemparts}

\problempart Louis Reasoner decides to just run the calculation over and
over until it eventually completes without failing. If the calculation
ever fails, Louis restarts the entire calculation. Give a lower bound
on the time that Louis can expect to wait for his code to complete
successfully on a typical gigahertz processor ($10^{12}$ operations per
second), with $n = 10^{13}$ and 
$f = 10^{12}$? 
%\hint $(1-p)^n \leq e^{-pn}$ for all $p \in [0,1]$ and
%$(1-p)^n \geq e^{-(p-p^2)n}$ for all $p \in [0,1/\sqrt{2}]$.

\hint $e^{-(p-p^2)n}\leq (1-p)^n \leq e^{-pn}$ for all $p \in [0,1/\sqrt{2}]$.

\begin{solution}
This part and the next were discussed in
\href{http://theory.lcs.mit.edu/classes/6.042/fall01/lectures/l13.pdf}{Notes
13.3}, where Wald's Theorem is used to show that
\begin{equation*}
\expect{T} = \frac{1}{p}(\frac{1}{(1-p)^n} - 1).
\end{equation*}

The hint now tells us that
  \begin{displaymath}
    \expect{T} \geq \frac{e^{pn}-1}{p} = 10^{12} \cdot (e^{10} - 1),
  \end{displaymath}
so the expected \emph{time} to complete the task is at least $e^{10}
- 1$ seconds, or roughly 6~hours. Since the operation should only
take 10~seconds in the absence of failures, this is quite a penalty.
\end{solution}

\problempart Alyssa P. Hacker decides to divide Louis's calculation into
10 equal-length parts, and has each part save its state when it
completes successfully. Saving state takes time~$s$. When a failure
occurs, Alyssa restarts the program from the latest saved state. How
long can Alyssa expect to wait for her code to complete successfully
on Louis's system? You can assume that $s < 10^{-4}$~seconds.

\begin{solution}
Now we have to do 10 calculations each of which can be
computed without regard to previous failures.  So by the previous
analysis, Alyssa expects the $i$th component to take
\[
    \expect{T_i} \geq \frac{e^{pn}-1}{p} = 10^{12} (e - 1)
\]
operations (note that $n = 10^{12}$). So, the expected number of
operations for the entire computation is
\[
\expect{\sum_{i=1}^{10} T_i} = 
\sum_{i=1}^{10} \expect{T_i} = 10^{13} (e - 1),
\]
giving an expected total computing time of roughly 17~seconds.  The 10
state saves take time $10 \cdot 10^{-4} = 10^{-3}$ seconds, which is
negligible.
\end{solution}

\problempart Alyssa tries to further optimize the expected total computing
time by dividing the calculation into 10\,000 parts instead of~10.
How long can Alyssa expect to wait for her code to complete
successfully? 

\hint $1+np\leq (1-p)^{-n}<1 + np + 2n^2p^2$ for all $np \in
[0,1], n>3$.

\begin{solution}
As before,
%\geq \frac{e^{pn}-1}{p} = \frac{1 + pn + O((pn)^2) - 1}{p}
\begin{eqnarray*} 
  \expect{T_i} &=& \frac{1}{p}\biggl(\frac{1}{(1-p)^n} - 1\biggr)\\
  &=& \frac{1 + pn + \delta(pn)^2 - 1}{p}\\
  &=& n(1 + \delta pn) 
\end{eqnarray*} 
with $0\leq\delta<2, n = 10^{13} / 10000 = 10^9$. So, the expected
number of operations for the entire computation, excluding state saves,
is roughly $10^{13}$ (since $p = 10^{-12}$ and $n = 10^9$, we ignore the
$\delta pn$ term).  This time the $10^{4}$ state saves are not
negligible, but only take one second, giving a total time of
11~seconds.
\end{solution}

\end{problemparts}
\end{problem}

\endinput
