%conflictone

\documentclass[quiz]{mcs}

\renewcommand{\examspace}{\textbf{SPACE}\newline}

\renewcommand{\exampreamble}{   % !! renew \exampreamble

  \begin{itemize}

  \item This exam is \textbf{closed book} except for a 4-sided
    crib sheet.  Total time is 3 hours.

  \item Write your solutions in the space provided with your name on every
    page.  If you need more space, write on the back of the sheet
    containing the problem.  Please keep your entire answer to a problem
    on that problem's page.

  \item
   GOOD LUCK!
  \end{itemize}}

\begin{document}

%\final
\conflictfinalone

%%%%%%%%%%%%%%%%%%%%%%%%%%%%%%%%%%%%%%%%%%%%%%%%%%%%%%%%%%%%%%%%%%%%%
% Problems start here
%%%%%%%%%%%%%%%%%%%%%%%%%%%%%%%%%%%%%%%%%%%%%%%%%%%%%%%%%%%%%%%%%%%%%

\examspace  

\begin{editingnotes}Graphs from S11, F11\end{editingnotes}

\pinput[points = 7, title =
  \textbf{graphs short answer fall11}]{FP_graphs_short_answer_fall11}

\examspace 
\begin{editingnotes}Number Theory  from S11, F11 \end{editingnotes}

\pinput[points = 9, title =
  \textbf{numbers short answer fall11}]{FP_numbers_short_answer_fall11}

\examspace 
\begin{editingnotes}
ADAMC: looks reasonable: Giuliano/true/false about primes and GCD's
\end{editingnotes}
\pinput{FP_number_short}

\examspace 
\begin{editingnotes}Partial orders from S11, rejected F11\end{editingnotes}

 \pinput[points = 10, title =
  \textbf{partial order short answer}]{FP_partial_order_short_answer}

\examspace  
\begin{editingnotes}Quantifiers \& Law of Large Numbers, from S11 rejected F11\end{editingnotes}

\pinput[points = 6, title =
 \textbf{FP large numbers quantifiers}]{FP_large_numbers_quantifiers}

\examspace
\begin{editingnotes} from S11 rej F11\end{editingnotes}

\pinput[points = 6, title =
 \textbf{binary relations on 01}]{FP_binary_relations_on_01}

\iffalse
\examspace
\begin{editingnotes}rej F11\end{editingnotes}
\pinput[points=6, title =
  \textbf{4 and 7 cent stamps by induction}]{FP_4_and_7_cent_stamps_by_induction}

\examspace
\begin{editingnotes}F11\end{editingnotes}
\pinput[points = 9, title=
  \textbf{structural induction lF18 composition}]
       {FP_structural_induction_lF18_composition}
\fi

\examspace
\begin{editingnotes}F11,NOT USED in F09: Logic formulas, Counting\end{editingnotes}

  \pinput[points = 5, title= \textbf{lining up fall11}]{FP_lining_up_fall11}

\iffalse
\examspace
\begin{editingnotes}F11\end{editingnotes}

\pinput[points = 6, title =
  \textbf{infinite binary sequences}]{FP_infinite_binary_sequences}
\fi

\examspace
\begin{editingnotes}F11

ADAMC: Ishaan looks reasonable:

Let $h(k) \eqdef \rem{k}{m}$ where $m = 2^p - 1$ and $k$ is a
character string interpreted in radix $2^p$.  Show that if we derive
string $x$ from string $y$ by permuting its characters, then $h(x) =
h(y)$.
\end{editingnotes}

\pinput[points = 6, title =
  \textbf{check factor by digits}]{FP_check_factor_by_digits}

\examspace  
\begin{editingnotes}Euler's Function rej F11\end{editingnotes}
  \pinput[points = 6, title = \textbf{modular powerful}]{FP_modular_powerful}

\examspace  
\begin{editingnotes}rej F11 Euler's Probability\end{editingnotes}
 \pinput[points = 5, title= \textbf{modular exponential}]{FP_modular_exponential}

\iffalse
\examspace
\begin{editingnotes}F11\end{editingnotes}

\pinput[points = 7, title =
  \textbf{directed graphs and probability}]{FP_directed_graphs_and_probability}
\fi

\examspace 
\begin{editingnotes}used S11\end{editingnotes}

\pinput[points=6, title =
  \textbf{bogus coloring proof}]{FP_bogus_coloring_proof}


\examspace
\begin{editingnotes}rej F11 GOOD \end{editingnotes}

\pinput[points = 6, title = \textbf{CP coloring}]{CP_coloring}

\examspace 

\begin{editingnotes}rej F11 Magic Trick Redux GOOD\end{editingnotes}

\pinput[points = 6, title =
  \textbf{magic trick 27 cards}]{FP_magic_trick_27_cards}

\iffalse
\examspace %Bipartite Average
\begin{editingnotes}rej F11 \end{editingnotes}

  \pinput[points = 3, title = \textbf{bipartite matching sex}]{FP_bipartite_matching_sex}
\fi


\examspace
\begin{editingnotes}F11 GOOD\end{editingnotes}

 \pinput[points = 5, title= \textbf{rogue pair}]{FP_rogue_pair}


\examspace %State Machine
\begin{editingnotes}rej NO \end{editingnotes}

  \pinput[points = 7, title=
    \textbf{santa state machine}]{FP_santa_state_machine}

\iffalse
\examspace
\begin{editingnotes}rej F11 Stable Matching\end{editingnotes}

  \pinput[points = 10, title= \textbf{marriage modify}]{FP_marriage_modify}
\fi

\examspace
\begin{editingnotes}F11\end{editingnotes}
  \pinput[points = 7, title=
    \textbf{asymptotics define functions}]{FP_asymptotics_define_functions}

\iffalse
\examspace 
\begin{editingnotes}F11 not used? perturbed F11 CP MAYBE\end{editingnotes}

  \pinput[points = 6, title = \textbf{big o}]{FP_big_o}
\fi

\examspace 
\begin{editingnotes}rej F11 inclusion-exclusion\end{editingnotes}

\pinput[points = 6, title =
MAYBE  \textbf{counting poker high cards}]{FP_counting_poker_high_cards}

\examspace  
\begin{editingnotes}rej F11 Counting Strings NO \end{editingnotes}

\pinput[points = 5, title= \textbf{string counting}]{FP_string_counting}

\examspace
\begin{editingnotes}rej F11 NO

ADAMC: John/FP\_counting\_given\_answers: This is a superset of a class
problem we used this semester.  Could make sense to pick some we used
in class and also some parts we omitted for class.  Good synthesis
problem that throws in some cardinality reasoning.
\end{editingnotes}

\pinput[points = 8, title= \textbf{Counting}]{FP_counting_given_answers}  

\iffalse
\examspace
\begin{editingnotes}rej F11\end{editingnotes}
\pinput[points = 6, title = \textbf{counting fall11}]{FP_counting_fall11}
\fi

\examspace
\begin{editingnotes}rej F11 Rich \#5 NO\end{editingnotes}

\pinput[points = 10, title =
   \textbf{string inclusion exclusion}]{FP_string_inclusion_exclusion}

\iffalse
\examspace 
\begin{editingnotes}
F11 GOOD Counting paths, part of FP\_more\_counting; S12 suggested
Keshav P

ADAMC: This is almost the same as one of our miniquiz problems.  That
might be a plus or a minus.
\end{editingnotes}

  \pinput[points = 8, title =
    \textbf{paths inclusion exclusion}]{FP_paths_inclusion_exclusion}

\examspace  %Combinatorial Proof
GOOD \pinput[points = 10, title =
  \textbf{combinatorial binomial}]{FP_combinatorial_binomial}
\fi

\examspace
\begin{editingnotes}F11\end{editingnotes}

\pinput[points = 6, title =
  \textbf{boat trip}]{FP_boat_trip}

\iffalse
\examspace
\begin{editingnotes}F11\end{editingnotes}

\pinput[points = 6, title = \textbf{red and blue goats}]
       {FP_red_and_blue_goats}

\exampsace %maybe modify to 4 towers
\pinput[points = 6, title =
  \textbf{towers of Sheboygan}]{FP_towers_of_Sheboygan}

\examspace  %boring Bayes'
\pinput[points=6, title =
  \textbf{college probability}]{FP_college_probability}

\examspace
\begin{editingnotes}midterm.S12 final.F11 Conditional Probability \end{editingnotes}

  \pinput[points = 7, title =
    \textbf{conditional prob inequality}]{FP_conditional_prob_inequality_fall11}
\fi

\examspace 
\begin{editingnotes}not used F11\end{editingnotes}
  \pinput[points = 7, title = \textbf{Probable Satisfiability}]{CP_probable_satisfiability_nk}

\iffalse
\examspace   
\begin{editingnotes}F11 Markov, Chebyshev Bounds\end{editingnotes}

  \pinput[points = 6, title= \textbf{gambling man}]{FP_gambling_man}
\fi

\examspace  
\begin{editingnotes}F11 Sampling \& Confidence\end{editingnotes}

  \pinput[points = 6, title= \textbf{random sampling}]{FP_random_sampling}

\examspace
\begin{editingnotes}not used F11\end{editingnotes}
\pinput[points = 6, title =
  \textbf{chebyshev hat check}]{FP_chebyshev_hat_check}

\examspace
\begin{editingnotes}rej F11 NO\end{editingnotes}

\pinput[points = 6, title =
  \textbf{coloring complete triangles}]{FP_coloring_complete_triangles}

\examspace
\begin{editingnotes}MEYER SUMMARY FINAL SUGGESTIONS\end{editingnotes}

\pinput{TP_mean_time_variance_given}

\pinput{FP_neighborhood_census}

\examspace
\begin{editingnotes}MQ9\end{editingnotes}
%\pinput{MQ_voldemort_returns}
\pinput{MQ_conditional_prob_inequality}
\pinput{CP_max_ranvar_n}
\pinput{MQ_expectHHH}

\examspace
\begin{editingnotes}Keshav P:\end{editingnotes}
\pinput{FP_structural_induction_arithmetic_expressions}

\begin{editingnotes}
ADAMC: Keshav P./many powers of 7: Peter suggested that this problem is too
hard to grade, but it at least seems simple to solve for prepared
students, and impossible for others. :)
\end{editingnotes}

Find the last digit of $7^{7^{\paren{7^7}}}$.


\begin{editingnotes}Boggs

ADAMC: The problem seems a bit easier than our solution for it
suggests, with the chance to use the equality notation defined as an
example.  Only the 2nd of the 3 parts seems interesting, though
perhaps one easier part as a warm-up would be OK.  This does seem to
be the right level of difficulty for final problems.  Perhaps a
variant of this format focusing on some topic from a later part of the
course, but still asking for first-order formulas, would be better.
\end{editingnotes}

\pinput{FP_logic_of_leq}

\examspace
\begin{editingnotes}Giuliano:\end{editingnotes}

\pinput{FP_lucas_induction}

\examspace
\documentclass[problem]{mcs}

\begin{pcomments}
\pcomment{FP_permutations_inc_exc}
\pcomment{final.S98}
\end{pcomments}

\pkeywords{
  permutation
  inclussion_exclusion
}

%%%%%%%%%%%%%%%%%%%%%%%%%%%%%%%%%%%%%%%%%%%%%%%%%%%%%%%%%%%%%%%%%%%%%
% Problem starts here
%%%%%%%%%%%%%%%%%%%%%%%%%%%%%%%%%%%%%%%%%%%%%%%%%%%%%%%%%%%%%%%%%%%%%

\begin{editingnotes}
ADAMC: Michaela/permutations of letters: Neat little problem that I
solved without too much trouble, which means it probably isn't too
hard, given my combinatorics skills. :) But I still do wonder that too
many students might fail to see the trick, which isn't quite like what
we've seen in class, unless I missed an alternate solution.
\end{editingnotes}

\begin{problem}Prove that there are 16,800 permutations of the letters
$\set{a,b,c,d,e,f,g,h}$ such that neither the letters $\set{a,b,c}$ nor
the letters $\set{d,e}$ occur in order.  For example, $fbgadceh$ is not
allowed because $d$ occurs before $e$, and $fagbecdh$ is not allowed
because $a$ occurs before $b$ and $b$ occurs before $c$, but $fbgaedch$ is
OK.

\begin{solution}
One way is to find three positions for $a$, $b$, and $c$, then two other
positions for $d$ and $e$, and count the number of ways $a$, $b$, and
$c$, can be out of order (namely, 5), the number of ways for $de$ (namely,
1), and the permutations of remaining 3 elements (namely, 3!):
\begin{eqnarray*}
\binom{8}{3}\binom{5}{2}3!5 & = &
      \frac{8*7*6}{3*2} \frac{5*4}{2} 3 * 2 * 5\\
& = & (8*7)(5*2)30 = 56*300 = 16800.
\end{eqnarray*}

Inclusion-exclusion is another way: let ALL be the $8!$ permutations of
all eight letters, let ABC be those permutations with $a,b,c$ appearing in
order, and DE those permutations with $d,e$ in order.  So
$\card{\mbox{ABC}} = P(8,5) ::= 8\cdot 7\cdot 6\cdot 5\cdot 4$, since a
permutation in ABC is uniquely determined by the 5 out of 8 positions of
the 5 letters besides $a,b,c$.  Likewise $\card{\mbox{DE}} = P(8,6)$.
Also, $\card{\mbox{ABC} \intersect \mbox{DE}}$ is determined by the
$P(8,3)$ positions of $f,g,h$ and then the $\binom{5}{2}$ places to put
$d$ and $e$ in order, that is $\card{\mbox{ABC} \intersect \mbox{DE}} =
P(8,3)\binom{5}{2}$.  Then the number of allowed permutations is
\begin{eqnarray*}
\card{\mbox{ALL}} - \card{\mbox{ABC} \union \mbox{DE}} & = & 8! -
\card{\mbox{ABC}} - \card{\mbox{DE}} + \card{\mbox{ABC} \intersect
\mbox{DE}}\\
& = & 8! - P(8,5) - P(8,6) + P(8,3)\binom{5}{2}\\
& = & 8*7*6[5! - 5*4 - 5*4*3 + 5*4/2]\\
& = & 8*7*6[120-20-60+10] = 8*7*6*50 = 16,800.
\end{eqnarray*}

\end{solution}
\end{problem}

%%%%%%%%%%%%%%%%%%%%%%%%%%%%%%%%%%%%%%%%%%%%%%%%%%%%%%%%%%%%%%%%%%%%%
% Problem ends here
%%%%%%%%%%%%%%%%%%%%%%%%%%%%%%%%%%%%%%%%%%%%%%%%%%%%%%%%%%%%%%%%%%%%%

\endinput


\iffalse
\examspace
\begin{editingnotes}
ADAMC: Shu Zheng: sums of kth powers: This looks like a reasonable
problem that only involves number theory.

Question 1 - Number Theory (from question 4 of Fall 2006 pset 3
\end{editingnotes}

\pinput{CP_Sk_equiv_-1_mod_p}
\fi

\examspace
\begin{editingnotes}
ADAMC: Shu/pigeonhole for numbers and graphs: This seems like a great
example of a synthesis question!

part(b) suggested By Shu
\end{editingnotes}

\pinput{PS_pigeon_hunting}

\examspace
Jayson:

1) Prove that the set of quadratic polynomials with integer
coefficients is countably infinite.

\examspace
\begin{editingnotes}Di Liu:

Week 8 Monday, Colorability of graphs with triangles: Problem 6,
Spring 07 final

ADAMC: Di/task scheduling: a CP version of this problem appeared on
ps6, so maybe we'd want to at least mix things up a bit.

Week 6 Wednesday, Task scheduling with DAGs: Problem 5, Spring 08 final
\end{editingnotes}

\pinput{MQ_tennis_match_partial_order}

\examspace
ADAMC email:

BIJECTIONS \& FINITE CARDINALITY

\examspace
\begin{editingnotes}S05.final\end{editingnotes}

\pinput{FP_sat_count_induction}

\pinput{FP_skywalker_prob_lin_recur}

\examspace
\begin{editingnotes}F05.final\end{editingnotes}

\pinput{FP_expected_black}

\examspace
\pinput{CP_variance_properties}

\examspace
\pinput{PS_probabilistic_proof}

\examspace
\pinput{TP_uncountable_stationary_distributions}

\end{document}
