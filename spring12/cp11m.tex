\documentclass[handout]{mcs}

\begin{document}

\inclassproblems{11, Mon.}

%%%%%%%%%%%%%%%%%%%%%%%%%%%%%%%%%%%%%%%%%%%%%%%%%%%%%%%%%%%%%%%%%%%%%
% Problems start here
%%%%%%%%%%%%%%%%%%%%%%%%%%%%%%%%%%%%%%%%%%%%%%%%%%%%%%%%%%%%%%%%%%%%%

%\pinput{CP_multinomial_fermat} %renamed from CP_multinomial_theorem_2

\pinput{CP_bag_of_donuts}
%\pinput{CP_gen_func_sum_of_squares} %renamed from CP_sum_of_squares_2
\pinput{PS_crazy_pet_lady}
%\pinput{CP_boat_trip}
\pinput{CP_nth_derivative_of_A}

\examspace
\section*{Appendix}
Let $[x^n]F(x)$ denote the coefficient of $x^n$ in the power series
for $F(x)$.  Then,
%\begin{equation}\label{1axk}
\[
[x^n]\paren{\frac{1}{\paren{1-\alpha x}^k}} = \binom{n+k-1}{n}\alpha^n.
\]
%\end{equation}

\subsection*{Partial Fractions}

Here's a particular case of the Partial Fraction Rule that should be
enough to illustrate the general Rule.  Let
\[
r(x) \eqdef \frac{p(x)}{(1-\alpha x)^2 (1-\beta x) (1-\gamma x)^3}
\]
where $\alpha, \beta, \gamma$ are distinct nonzero, complex numbers,
and $p(x)$ is a polynomial of degree less than the denominator,
namely, less than 6.  Then there are unique numbers
$a_1,a_2,b,c_1,c_2,c_3 \in \complexes$ such that
\[
r(x)
= \frac{a_1}{1-\alpha x} + \frac{a_2}{(1-\alpha x)^2}
+ \frac{b}{1-\beta x}
+ \frac{c_1}{1-\gamma x} + \frac{c_2}{(1-\gamma x)^2} + \frac{c_3}{(1-\gamma x)^3}
\]

\iffalse Explain why partial fractions together with~\eqref{1axk} imply
that there is a closed form expression (allowing binomial coefficients)
for $[x^n]\paren{R(x)/S(x)}$ for arbitrary polynomials $R(x),S(x)$.
\fi

%%%%%%%%%%%%%%%%%%%%%%%%%%%%%%%%%%%%%%%%%%%%%%%%%%%%%%%%%%%%%%%%%%%%%
% Problems end here
%%%%%%%%%%%%%%%%%%%%%%%%%%%%%%%%%%%%%%%%%%%%%%%%%%%%%%%%%%%%%%%%%%%%

\end{document}
