\documentclass[handout]{mcs}

\begin{document}

\renewcommand{\reading}{
  Chapter~\bref{induction_chap}{.
    on \emph{Induction}
  } and
  Chapter~\bref{recursive_data_chap}{.
    on \emph{Recursive Data Types}
  }
}.

\problemset{3}

%%%%%%%%%%%%%%%%%%%%%%%%%%%%%%%%%%%%%%%%%%%%%%%%%%%%%%%%%%%%%%%%%%%%%
% Problems start here
%%%%%%%%%%%%%%%%%%%%%%%%%%%%%%%%%%%%%%%%%%%%%%%%%%%%%%%%%%%%%%%%%%%%%

\pinput{CP_divisible_by_powers_of_3}

\pinput{CP_10_heads_and_100_tails}
%%state machine

\pinput{PS_card_shuffle_state_machine}
%%state machine

\pinput{CP_Zakim_bridge_state_machine}
%%state machine

\pinput{FP_structural_induction_rational_composition}
%%structural Induction

\pinput{PS_koch_snowflake}
%%structural induction

\pinput{PS_linear_combination_game}
%%structural induction

\pinput{FP_uncountable_infinite_sequences}
%%Uncountibility

\pinput{CP_N_to_N_diagonal_argument}
%%Uncountibility

%% I would also reccomend problems 4.7 and 4.8 from Sipser's Introduction to the Thoery of Computation.
% Let $\Beta$ be the set of all infinite sequences over {0,1}. Show that $\Beta$ is uncountable, using a proof by diagonalization.
%Let $T={(i,j,k)|i,j,k \in N}$. Show that $T$ is countable.
%%


%%\emph{PS_uncountable_infinite_sequences}
%%\pinput{PS_parenthesis_good_count}
%% this doesn't seem to exist in the problems repository anymore, we should figure out what happened to it...


%\pinput{MQ_uncountable_infinite_sequences}  not covered later chapter

%%%%%%%%%%%%%%%%%%%%%%%%%%%%%%%%%%%%%%%%%%%%%%%%%%%%%%%%%%%%%%%%%%%%%
% Problems end here
%%%%%%%%%%%%%%%%%%%%%%%%%%%%%%%%%%%%%%%%%%%%%%%%%%%%%%%%%%%%%%%%%%%%%
\end{document}
