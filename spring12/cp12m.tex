\documentclass[handout]{mcs}

\begin{document}

\inclassproblems{12, Mon.}

%%%%%%%%%%%%%%%%%%%%%%%%%%%%%%%%%%%%%%%%%%%%%%%%%%%%%%%%%%%%%%%%%%%%%
% Problems start here
%%%%%%%%%%%%%%%%%%%%%%%%%%%%%%%%%%%%%%%%%%%%%%%%%%%%%%%%%%%%%%%%%%%%%

\pinput{CP_conditional_prob_says_so_bug}
\pinput{CP_missing_card_probability}
\pinput{FP_conditional_prob_inequality}   %added for afternoon session
\pinput{CP_three_fair_coins}
\pinput{PS_conditional_aces}

\iffalse
\begin{staffnotes}
Students may finish early.  Here's an extra problem to offer
(part~(a) of~\bref{TP_indicator_independence}):

Prove that if $A$ and $B$ are independent events, then so are
$A$ and $\setcomp{B}$.
\begin{solution}
\begin{proof}
\begin{align*}
\pr{A \intersect \setcomp{B}}
  &  = \pr{A} - \pr{A \intersect B}
       & \text{(difference rule)}\\
  & =  \pr{A} - \pr{A} \cdot \pr{B}
       & \text{(independence of $A$ and $B$)}\\
  & =  \pr{A} (1 - \pr{B})\\
  & =  \pr{A} \cdot \pr{\setcomp{B}} & \text{(complement rule)}.
\end{align*}
\end{proof}
\end{solution}

\end{staffnotes}
\fi
%%%%%%%%%%%%%%%%%%%%%%%%%%%%%%%%%%%%%%%%%%%%%%%%%%%%%%%%%%%%%%%%%%%%%
% Problems end here
%%%%%%%%%%%%%%%%%%%%%%%%%%%%%%%%%%%%%%%%%%%%%%%%%%%%%%%%%%%%%%%%%%%%%
\end{document}
