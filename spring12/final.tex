\documentclass[quiz]{mcs}

%\renewcommand{\examspace}{\textbf{SPACE}\newline}

\renewcommand{\exampreamble}{   % !! renew \exampreamble

  \begin{itemize}

  \item This exam is \textbf{closed book} except for a 4-sided
    crib sheet.  Total time is 3 hours.

  \item Write your solutions in the space provided with your name on every
    page.  If you need more space, write on the back of the sheet
    containing the problem.  Please keep your entire answer to a problem
    on that problem's page.

  \item
   GOOD LUCK!
  \end{itemize}}

\begin{document}

\final
%\conflictfinalone

%%%%%%%%%%%%%%%%%%%%%%%%%%%%%%%%%%%%%%%%%%%%%%%%%%%%%%%%%%%%%%%%%%%%%
% Problems start here
%%%%%%%%%%%%%%%%%%%%%%%%%%%%%%%%%%%%%%%%%%%%%%%%%%%%%%%%%%%%%%%%%%%%%

\examspace 
\begin{editingnotes}
ADAMC: looks reasonable: Giuliano/true/false about primes and GCD's
\end{editingnotes}
\pinput[points = 8, title = 
\textbf{Numbers short answer}]{FP_number_short}

\examspace 
\begin{editingnotes}like partial orders from S11, rejected F11\end{editingnotes}

\pinput[points = 7, title = 
\textbf{relations short answer}]{FP_partial_order_short_answer_S12}

\examspace  
\begin{editingnotes}
Quantifiers \& Law of Large Numbers, from S11 rejected F11
\end{editingnotes}

\pinput[points = 6, title = \textbf{FP large numbers
    quantifiers}]{FP_large_numbers_quantifiers}

\examspace
\begin{editingnotes} from S11 rej F11\end{editingnotes}

\pinput[points = 9, title =
 \textbf{binary relations on 01}]{FP_binary_relations_on_01_S12}

\examspace  
\begin{editingnotes}rej F11 Euler's Probability\end{editingnotes}
 \pinput[points = 8, title= \textbf{modular exponential}]{FP_modular_exponential}

\examspace 
\begin{editingnotes}used S11\end{editingnotes}

\pinput[points=8, title =
  \textbf{bogus coloring proof}]{FP_bogus_coloring_proof}

\examspace 
\begin{editingnotes}State Machine rej NO \end{editingnotes}

\pinput[points = 6, title=
   \textbf{santa state machine}]{FP_santa_state_machine_S12}

\examspace
\begin{editingnotes}F11\end{editingnotes}

\pinput[points = 9, title =
  \textbf{boat trip}]{FP_boat_trip}

\examspace  
\begin{editingnotes}F11 Sampling \& Confidence\end{editingnotes}

  \pinput[points = 8, title= \textbf{random sampling}]{FP_random_sampling}

\examspace
\begin{editingnotes}rej F11 NO\end{editingnotes}

\pinput[points = 10, title =
  \textbf{coloring complete triangles}]{FP_coloring_complete_triangles}

\examspace
\pinput[points = 6, title =
  \textbf{structural induction congruence}]{FP_stuctural_ind_polynomials}

\examspace
\pinput[points = 8, title = \textbf{Counting Permutations}]{FP_permutations_inc_exc}

\begin{editingnotes}ADAMC email:

BIJECTIONS \& FINITE CARDINALITY
\end{editingnotes}

\examspace
\pinput[points = 4, title = \textbf{Countable Set}]{FP_countable_quadratics}

\examspace
\pinput[points = 4, title = \textbf{UnCountable Stationary
    Distributions}]{TP_uncountable_stationary_distributions}

\end{document}


\iffalse
  

\begin{editingnotes}Graphs from S11, F11\end{editingnotes}

\pinput[points = 7, title =
  \textbf{graphs short answer fall11}]{FP_graphs_short_answer_fall11}

 
\begin{editingnotes}Number Theory  from S11, F11 \end{editingnotes}

\pinput[points = 9, title =
  \textbf{numbers short answer fall11}]{FP_numbers_short_answer_fall11}


\begin{editingnotes}rej F11\end{editingnotes}
\pinput[points=6, title =
  \textbf{4 and 7 cent stamps by induction}]{FP_4_and_7_cent_stamps_by_induction}


\begin{editingnotes}F11\end{editingnotes}
\pinput[points = 9, title=
  \textbf{structural induction lF18 composition}]
       {FP_structural_induction_lF18_composition}


\begin{editingnotes}F11,NOT USED in F09: Logic formulas, Counting\end{editingnotes}

  \pinput[points = 5, title= \textbf{lining up fall11}]{FP_lining_up_fall11}


\begin{editingnotes}F11\end{editingnotes}

\pinput[points = 6, title =
  \textbf{infinite binary sequences}]{FP_infinite_binary_sequences}


\begin{editingnotes}F11\end{editingnotes}

\pinput[points = 7, title =
  \textbf{directed graphs and probability}]{FP_directed_graphs_and_probability}

  
\begin{editingnotes}Euler's Function rej F11\end{editingnotes}
 \pinput[points = 6, title = \textbf{modular powerful}]{FP_modular_powerful}


\begin{editingnotes}rej F11 GOOD \end{editingnotes}

\pinput[points = 6, title = \textbf{CP coloring}]{CP_coloring}

 %Bipartite Average
\begin{editingnotes}rej F11 \end{editingnotes}

  \pinput[points = 3, title = \textbf{bipartite matching sex}]{FP_bipartite_matching_sex}


\begin{editingnotes}F11 GOOD\end{editingnotes}

 \pinput[points = 5, title= \textbf{rogue pair}]{FP_rogue_pair}


\begin{editingnotes}F11\end{editingnotes}

\pinput[points = 6, title = \textbf{red and blue goats}]
       {FP_red_and_blue_goats}

\exampsace %maybe modify to 4 towers
\pinput[points = 6, title =
  \textbf{towers of Sheboygan}]{FP_towers_of_Sheboygan}

  %boring Bayes'
\pinput[points=6, title =
  \textbf{college probability}]{FP_college_probability}


\begin{editingnotes}midterm.S12 final.F11 Conditional Probability \end{editingnotes}

  \pinput[points = 7, title =
    \textbf{conditional prob inequality}]{FP_conditional_prob_inequality_fall11}


   
\begin{editingnotes}F11 Markov, Chebyshev Bounds\end{editingnotes}

  \pinput[points = 6, title= \textbf{gambling man}]{FP_gambling_man}


\begin{editingnotes}not used F11\end{editingnotes}
\pinput[points = 6, title =
  \textbf{chebyshev hat check}]{FP_chebyshev_hat_check}

\pinput[points = 6, title =
  \textbf{Mean Time Variance}]{TP_mean_time_variance_given}

\pinput[points = 6, title = \textbf{Conditional
    Inequality}]{MQ_conditional_prob_inequality}

\pinput[points = 6, title = \textbf{PDF of Max}]{CP_max_ranvar_n}


\begin{editingnotes}Boggs

ADAMC: The problem seems a bit easier than our solution for it
suggests, with the chance to use the equality notation defined as an
example.  Only the 2nd of the 3 parts seems interesting, though
perhaps one easier part as a warm-up would be OK.  This does seem to
be the right level of difficulty for final problems.  Perhaps a
variant of this format focusing on some topic from a later part of the
course, but still asking for first-order formulas, would be better.
\end{editingnotes}

\pinput[points = 6, title = \textbf{Logic of Inequalities}]{FP_logic_of_leq}


\begin{editingnotes}Giuliano:\end{editingnotes}

\pinput[points = 6, title = \textbf{Lucas Induction}]{FP_lucas_induction}


\begin{editingnotes}rej F11 Stable Matching\end{editingnotes}

  \pinput[points = 10, title= \textbf{marriage modify}]{FP_marriage_modify}


\begin{editingnotes}F11\end{editingnotes}
  \pinput[points = 7, title=
    \textbf{asymptotics define functions}]{FP_asymptotics_define_functions}

 
\begin{editingnotes}F11 not used? perturbed F11 CP MAYBE\end{editingnotes}

  \pinput[points = 6, title = \textbf{big o}]{FP_big_o}

 
\begin{editingnotes}
F11 GOOD Counting paths, part of FP\_more\_counting; S12 suggested
Keshav P

ADAMC: This is almost the same as one of our miniquiz problems.  That
might be a plus or a minus.
\end{editingnotes}

\pinput[points = 8, title =
    \textbf{paths inclusion exclusion}]{FP_paths_inclusion_exclusion}

  %Combinatorial Proof
GOOD \pinput[points = 10, title =
  \textbf{combinatorial binomial}]{FP_combinatorial_binomial}

  
\begin{editingnotes}rej F11 Counting Strings NO \end{editingnotes}

\pinput[points = 5, title= \textbf{string counting}]{FP_string_counting}


\begin{editingnotes}rej F11\end{editingnotes}
\pinput[points = 6, title = \textbf{counting fall11}]{FP_counting_fall11}


\begin{editingnotes}Di Liu:

Week 8 Monday, Colorability of graphs with triangles: Problem 6,
Spring 07 final

ADAMC: Di/task scheduling: a CP version of this problem appeared on
ps6, so maybe we'd want to at least mix things up a bit.

Week 6 Wednesday, Task scheduling with DAGs: Problem 5, Spring 08 final
\end{editingnotes}


\begin{editingnotes}
ADAMC: Shu Zheng: sums of kth powers: This looks like a reasonable
problem that only involves number theory.

Question 1 - Number Theory (from question 4 of Fall 2006 pset 3
\end{editingnotes}

\pinput{CP_Sk_equiv_-1_mod_p}


\begin{editingnotes}S05.final\end{editingnotes}

\pinput{FP_sat_count_induction}


\pinput{CP_variance_properties}


\pinput{PS_probabilistic_proof}

\begin{editingnotes}rej F11 Rich \#5 NO\end{editingnotes}

\pinput[points = 10, title =
   \textbf{string inclusion exclusion}]{FP_string_inclusion_exclusion}


\begin{editingnotes}
ADAMC: Shu/pigeonhole for numbers and graphs: This seems like a great
example of a synthesis question!

part(b) suggested By Shu
\end{editingnotes}

\pinput[points = 6, title = \textbf{Pigeon Hunting}]{PS_pigeon_hunting}

\pinput[points = 6, title = \textbf{Probability Recurrence}]{FP_skywalker_prob_lin_recur}

\begin{editingnotes}MQ9\end{editingnotes}
\pinput[points = 6, title = \textbf{Expected Run}]{MQ_expectHHH}

\begin{editingnotes}rej F11 inclusion-exclusion MAYBE\end{editingnotes}

\pinput[points = 6, title =
  \textbf{counting poker high cards}]{FP_counting_poker_high_cards}


\begin{editingnotes}rej F11 Magic Trick Redux GOOD\end{editingnotes}

\pinput[points = 6, title =
  \textbf{magic trick 27 cards}]{FP_magic_trick_27_cards}

\begin{editingnotes}not used F11\end{editingnotes}
  \pinput[points = 7, title = \textbf{Probable Satisfiability}]{CP_probable_satisfiability_nk}


\begin{editingnotes}F05.final\end{editingnotes}

\pinput[points = 6, title = \textbf{Expectation}]{FP_expected_black}

\begin{editingnotes}
ADAMC: Keshav P./many powers of 7: Peter suggested that this problem is too
hard to grade, but it at least seems simple to solve for prepared
students, and impossible for others. :)
\end{editingnotes}

\pinput[points = 6, title = \textbf{Super 7}]{CP_7777}

\begin{editingnotes}rej F11 NO

ADAMC: John/FP\_counting\_given\_answers: This is a superset of a class
problem we used this semester.  Could make sense to pick some we used
in class and also some parts we omitted for class.  Good synthesis
problem that throws in some cardinality reasoning.
\end{editingnotes}

\pinput[points = 8, title= \textbf{Counting}]{FP_counting_given_answers}  

\begin{editingnotes}F11

ADAMC: Ishaan looks reasonable:

Let $h(k) \eqdef \rem{k}{m}$ where $m = 2^p - 1$ and $k$ is a
character string interpreted in radix $2^p$.  Show that if we derive
string $x$ from string $y$ by permuting its characters, then $h(x) =
h(y)$.\
\end{editingnotes}

\pinput[points = 6, title =
  \textbf{check factor by digits}]{FP_check_factor_by_digits}

\pinput[points = 5, title = \textbf{Partial Order Concepts}]{MQ_tennis_match_partial_order}

\pinput[points = 6, title = \textbf{Conditional Census}]{FP_neighborhood_census}

\fi
