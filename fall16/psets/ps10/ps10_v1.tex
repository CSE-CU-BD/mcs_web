\documentclass[12pt,twoside]{article}   
\usepackage{light}

\newcommand{\card}[1]{\left|#1\right|}
\newcommand{\union}{\cup}
\newcommand{\lgunion}{\bigcup}
\newcommand{\intersect}{\cap}
\newcommand{\lgintersect}{\bigcap}
\newcommand{\cross}{\times}
\newcommand{\naturals}{\mathbb{N}}
\newcommand{\mfigure}[3]{\bigskip\centerline{\resizebox{#1}{#2}{\includegraphics{#3}}}\bigskip}
\newcommand{\eqdef}{\mathbin{::=}}
\newcommand{\heads}{H}
\newcommand{\tails}{T}

\hidesolutions
%\showsolutions

\newlength{\strutheight}
\newcommand{\prob}[1]{\mathop{\textup{Pr}} \nolimits \left( #1 \right)}
\newcommand{\prsub}[2]{\mathop{\textup{Pr}_{#1}}\nolimits\left(#2\right)}
\newcommand{\prcond}[2]{%
  \ifinner \settoheight{\strutheight}{$#1 #2$}
  \else    \settoheight{\strutheight}{$\displaystyle#1 #2$} \fi%
  \mathop{\textup{Pr}}\nolimits\left(
    #1\,\left|\protect\rule{0cm}{\strutheight}\right.\,#2
  \right)}
\newcommand{\comment}[1]{}
\newcommand{\cE}{\mathcal{E}}
\renewcommand{\setminus}{-}
\renewcommand{\complement}[1]{\overline{#1}}


\begin{document}
\problemset{10}{November 15, 2016}{Monday, November 21, 7:30pm}

\noindent \textbf{Reading Assignment:}   Sections  12.1-12.6, 14.1-14.5, 16.5


%%%%%%%%%%%%%%%%%%%%%%%%%%%%%%%%%%%%%%%%%%%%%%%%%%%%%%%%%%%%%%%%%%%%%%%%%%%%%
%%% Fall07 ps10 problem 1

\begin{problem}{30}
Generating functions are very useful for turning difficult combinatorial problems into simple algebra.  Solve the following combinatorial problems by using generating functions.
(\textit{Hint}: the product rule will be helpful in solving these problems).
\bparts
\ppart{4}  Consider rolling a pair of normal six-sided dice.  How many ways are there for the dice to sum to 7?
\ppart{6}  Bob has a basket with 4 apples and 5 bananas.  Suppose the apples are all distinguishable, and the bananas are all distinguishable.  How many ways are there for Bob to select 6 pieces of fruit such that he picks an even number of apples and at least two bananas?

\ppart{4}   Bob has a basket with 4 apples and 5 bananas.  Suppose the apples are now indistinguishable, and the bananas are indistinguishable.  How many ways are there for Bob to select 6 pieces of fruit such that he picks an even number of apples and at least two bananas?

\ppart{6}  Find the number of ways to collect 15 dollars from 20 people if each of the first 19 people can give a dollar or nothing, and the twentieth person can give either 1 dollar, 5 dollars, or nothing.

\ppart{6}  We have three pennies, four nickels, and two quarters. Find the generating function of the number of ways we can make change for $n$ cents.  Assume the coins are indistinguishable.

\ppart{4} We have three pennies, four nickels, and two quarters. Find the generating function of the number of ways we can make change for $n$ cents.  Assume the coins are distinguishable.

\eparts

\end{problem}

\begin{problem}{10}
Let $\mathcal{C}$ be the set of sequences formed by $\{a, b, c, d, 1, 2, 3\}$ such that the letters $\{a, b, c, d\}$ appear before the numbers $\{1, 2, 3\}$.  As an example, $abba12$ and $cdab321$ are sequences in  $\mathcal{C}$ but $a3b2c1$ is not a sequence in  $\mathcal{C}$.  Let $c_n$ be the number of sequences in  $\mathcal{C}$ of length $n$.  Let $C(x) = \displaystyle\sum\limits_{n=0}^\infty c_n x^n$. 

\bparts
\ppart{6} Determine an expression for $C(x)$.  
\ppart{4} Determine an explicit expression for $c_n$ (\textit{Hint}: use partial fraction decomposition on the generating function you find in part a).  
\eparts

\end{problem}

\begin{problem}{10}
Let $a_n = a_{n-1} + 2a_{n-2}$ for $n \in \mathbb{N}$ with $a_0 = 1, a_1 = 1$.  Use generating functions to find an explicit expression for $a_n$.  
\end{problem}

\begin{problem}{20}
For a given $n$, let $p_n$ be the number of ways of writing $n$ as a sum of 3 positive integers, where the order matters.  For example, we can write 5 as:
\begin{equation*}
1 + 1 + 3 \qquad 1 + 3 + 1 \qquad 3 + 1 + 1 \qquad 1 + 2 + 2 \qquad 2 + 1 + 2 \qquad 2 + 2 + 1
\end{equation*}
\bparts
\ppart{8} What is a formula for the generating function $P(x) =  \displaystyle\sum\limits_{n=0}^\infty p_n x^n$?

Now again, for a given $n$, consider the number of ways of writing $n$ as a sum of 3 positive integers as describe above.  Let $f_n$ be the sum of the product of each of the triples of numbers.  For example, for $n = 5$ again, we get 
\begin{equation*}
f_5 = 1 \cdot 1 \cdot 3 + 1 \cdot 3 \cdot 1+ 3 \cdot 1 \cdot 1 + 1 \cdot 2 \cdot 2 + 2 \cdot 1 \cdot 2 + 2 \cdot 2 \cdot 1
\end{equation*} 
%
Let $f_0 = 0$.  
\ppart{12} What is the formula for the generating function $F(x) =  \displaystyle\sum\limits_{n=0}^\infty f_n x^n$? (Your answer should be a closed formula, not a Taylor Series) 

\eparts
\end{problem}


%%%%%%%%%%%%%%%%%%%%%%%%%%%%%%%%%%%%%%%%%%%%%%%%%%%%%%%%%%%%%%%%%%%%%%%%


\begin{problem}{20}
In lecture we discussed the Birthday Paradox. Namely, we found that in a group of $m$ people with $N$ possible birthdays, if $m \ll N$, then:
\[
\pr{\text{all $m$ birthdays are different}} \sim e^{-\frac{m(m-1)}{2N}}
\]
To find the number of people, $m$, necessary for a half chance of a match, we set the probability to $1/2$ to get:
\[
m \sim \sqrt{(2\ln2)N} \approx 1.18\sqrt{N}
\]

For $N = 365$ days we found $m$ to be 23.

We could also run a different experiment. As we put on the board the birthdays of the people surveyed, we could ask the class if anyone has the same birthday. In this case, before we reached a match amongst the surveyed people, we would already have found other people in the rest of the class who have the same birthday as someone already surveyed. Let's investigate why this is.

\bparts
\ppart{10} Consider a group of $m$ people with $N$ possible birthdays amongst a larger class of $k$ people, such that $m \leq k$. Define $\pr{A}$ to be the probability that $m$ people all have different birthdays \textit{and} none of the other $k-m$ people have the same birthday as one of the $m$.

Show that, if $m \ll N$, then $\pr{A} \sim e^{\frac{m(m-2k)}{2N}}$. (Notice that the probability of no match is $e^{-\frac{m^2}{2N}}$ when $k$ is $m$, and it gets smaller as $k$ gets larger.)

\hspace{0.5in} \textit{Hints:} For $m \ll N$: $\frac{N!}{(N-m)!N^m} \sim e^{-\frac{m^2}{2N}}$, and $(1-\frac{m}{N}) \sim e^{-\frac{m}{N}}$.

\solution{
We know:
\[
\pr{A} = \frac{N(N-1)\ldots(N-m+1)\cdot(N-m)^{k-m}}{N^k}
\]

since there are $N$ choices for the first birthday, $N-1$ choices for the second birthday, etc., for the first $m$ birthdays, and $N-m$ choices for each of the remaining $k-m$ birthdays. There are total $N^k$ possible combinations of birthdays within the class.

\begin{align*}
\pr{A} &= \frac{N(N-1)\ldots(N-m+1)\cdot(N-m)^{k-m}}{N^k} \\
&= \frac{N!}{(N-m)!}\left(\frac{(N-m)^{k-m}}{N^k}\right) \\
&= \frac{N!}{(N-m)!N^m}\left(\frac{N-m}{N}\right)^{k-m} \\
&= \frac{N!}{(N-m)!N^m}\left(1-\frac{m}{N}\right)^{k-m} \\
&\sim e^{-\frac{m^2}{2N}} \cdot e^{-\frac{m}{N}(k-m)} & \text{(by the Hint)} \\
& = e^{\frac{m(m-2k)}{2N}}
\end{align*}
}

\ppart{10} Find the approximate number of people in the group, $m$, necessary for a half chance of a match (your answer will be in the form of a quadratic). Then simplify your answer to show that, as $k$ gets large  (such that $\sqrt{N} \ll k$), then $m \sim \frac{N\ln2}{k}$.

\hspace{0.5in} \textit{Hint:} For $x \ll 1$: $\sqrt{1-x} \sim (1-\frac{x}{2})$.

\solution{
Setting $\pr{A} = 1/2$, we get a solution for $m$:

\begin{align*}
1/2 &= e^{\frac{m(m-2k)}{2N}} \\
-2N\ln2 &= m^2 -2km  \\
0 &= m^2-2km + (2N\ln2) \\
m &= \frac{2k \pm \sqrt{(2k)^2 - 4(2N\ln2)}}{2}
\end{align*}

Simplifying the solution under the assumption of large $k$, we find:
\begin{align*}
m &= \frac{2k - \sqrt{4k^2 - 8N\ln2}}{2} & \text{(taking the lower positive root)} \\
&= k - k\sqrt{1 - \frac{2N\ln2}{k^2}} \\
&\sim k  - k \left(1-\frac{2N\ln2}{2k^2}\right) & \text{(by the Hint)} \\
&= \frac{N\ln2}{k}  
\end{align*}
}

\eparts

\end{problem}

%%%%%%%%%%%%%%%%%%%%%%%%%
%% fall 08, pset11

\instatements{\vspace{0.5in}}
\begin{problem}{10}
We're covering probability in 6.042 lecture one day, and you volunteer for one of Professor Leighton's demonstrations. He shows you a coin and says he'll bet you \$1 that the coin will come up heads. Now, you've been to lecture before and therefore suspect the coin is biased, such that the probability of a flip coming up heads, $\pr{H}$, is $p$ for $1/2 < p \leq 1$.

You call him out on this, and Professor Leighton offers you a deal. He'll allow you to come up with an algorithm using the biased coin to \textit{simulate} a fair coin, such that the probability you win and he loses, $\pr{W}$, is equal to the probability that he wins and you lose, $\pr{L}$. You come up with the following algorithm:

\begin{enumerate}
\item Flip the coin twice.
\item Based on the results:
	\begin{itemize}
	\item $TH \implies$ you win [$W$], and the game terminates.
	\item $HT \implies$ Professor Leighton wins [$L$], and the game terminates.
	\item $(HH \lor TT) \implies$ discard the result and flip again.
	\end{itemize}
\item If at the end of $N$ rounds nobody has won, declare a tie.
\end{enumerate}
As an example, for $N=3$, an outcome of $HT$ would mean the game ends early and you lose, $HHTH$ would mean the game ends early and you win, and $HHTTTT$ would mean you play the full $N$ rounds and result in a tie.

\bparts

\ppart{5}
Assume the flips are mutually independent. Show that $\pr{W} = \pr{L}$.

\solution{
The probability of you winning is equal to the probability that you win in the first round, plus the probability that nobody won in the first round times the probability that you win in the second round, plus the probability that nobody won in the first two round times the probability that you win in the third round, etc. The same goes for Professor Leighton. Hence:
\begin{align*}
\pr{W} &= \pr{TH} + \pr{HH \lor TT}\pr{TH} + \pr{HH \lor TT}^2\pr{TH} + \ldots \\
& = \pr{TH} \cdot \sum_{i=0}^N \pr{HH \lor TT}^i \\
& = \pr{HT} \cdot \sum_{i=0}^N \pr{HH \lor TT}^i\\
& = \pr{L}
\end{align*}
The middle step is possible because $\pr{TH} = (1-p)p = p(1-p) = \pr{HT}$.
}

\ppart{5}
Show that, if $p<1$, the probability of a tie goes to 0 as $N$ goes to infinity.

\solution{
The probability of a tie is just the probability that nobody won all $N$ rounds, namely:
\[
\pr{tie} = (\pr{HH \lor TT})^N = (\pr{HH} + \pr{TT})^N = (p^2 + (1-p)^2)^N
\]
So the limit as $N$ goes to infinity is 0, given that $p$ and therefore $p^2 + (1-p)^2$ are $< 1$.
}

\eparts

\end{problem}

%%%%%%%%%%%%%%%%%%%%%%%%%%%%%%%%%%%%%%%%%%%%%%%%%%%
\end{document}
