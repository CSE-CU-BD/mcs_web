\documentclass[12pt,twoside]{article}   
\usepackage{light}

\newcommand{\hint}[1]{({\it Hint: #1})}
\newcommand{\card}[1]{\left|#1\right|}
\newcommand{\union}{\cup}
\newcommand{\lgunion}{\bigcup}
\newcommand{\intersect}{\cap}
\newcommand{\lgintersect}{\bigcap}
\newcommand{\cross}{\times}

\hidesolutions
\showsolutions

\begin{document}
\problemset{9}{November 8, 2016}{Monday, November 14}

\noindent \textbf{Reading Assignment:}   Sections  7.2, 11.1-11.10


%Pigeonhole
\begin{problem}{10}

\bparts
%S08_cp9f.3d
\ppart{5} Show that of any $n+1$ distinct numbers chosen from the
set $\{1,2,\ldots,2n\}$, at least 2 must be relatively prime.
\hint{$\gcd(k,k+1)=1$.}

\solution{Treat the $n+1$ numbers as the pigeons and the $n$ 
disjoint subsets of the form $\{2j-1,2j\}$ as the pigeonholes. The
pigeonhole principle implies that there must exist a pair of 
consecutive integers among the $n+1$ chosen which, as suggested
in the hint, must be relatively prime.}

\ppart{5} Show that any finite connected undirected graph with 
$n \geq 2$ vertices must have 2 vertices with the same degree. 

\solution{In a finite connected graph with $n \geq 2$ vertices,
the domain for the vertex degrees is the set $\{1,2,\ldots,n-1\}$ 
since each vertex can be adjacent to at most all of the remaining
$n-1$ vertices and the existence of a degree 0 vertex would violate
the assumption that the graph be connected.  Therefore, treating
the $n$ vertices as the pigeons and the $n-1$ possible degrees as 
the pigeonholes, the pigeonhole principle implies that there
must exist a pair of vertices with the same degree.}

\eparts
\end{problem}



\begin{problem}{15}

Consider the 40 most popular cities on Earth.  Use the pigeonhole principle to show that there are two subsets of these cities that have exactly the same number of people.  Assume that there are $10^{10}$ people on the Earth.  

\end{problem}
\solution{
Let the pigeonholes be the positive integers mod $10^{10}$.  Let the pigeons be the number of people in each subset of the 40 cities.  There are $2^{40}$ subsets of our cities.  Now $2^{40}$ is much larger than $10^{10}$, and so at least two of the subsets of cities must fall into the same pigeonhole (that is they have the same number of people mod $10^{10}$).  But, note that the number of people in each subset must necessarily be less than the number of people in the world.  Hence the number of people in each subset is exactly one of the positive integers mod $10^{10}$.  Hence, we have that at least two subsets of the cities have the same number of people.
}



%%%%%%%%%%%%%%%%%%%%%%%%%%%%%%%%%%%%%%%%%%%%%%%%%
% new problem
% Inclusion-Exclusion (with and without replacement)
\begin{problem}{15}
In this problem, we will use the principle of inclusion-exclusion (PIE) to solve the problem of derangements.  

Suppose you attend a Halloween party with $n$ guests, where each guest dressed up as Harley-Quinn.  Each Harley-Quinn brought a bat as an accessory to the party, but didn't want to carry it through the night, and so all the bats were left stacked together in a single room.  Unfortunately, later on in the party, one of the party-goers bumped into the stack of bats causing them all to fall in a pile.  Thus, at the end of the party, each Harley-Quinn grabbed a random bat and left.  It turns out none of them got his/her own bat  back.  We will count the number of ways this can happen.

\bparts
\ppart{2} How many ways could the guests have picked up the bats at random?
\solution{
$n!$ as we are simply assigning one of the $n$ bats to each guests.  
}
\ppart{3} How many ways there are to choose 1 person to get their bat back and randomly assign everyone else?
\solution{
$\binom{n}{1} \cdot (n-1)! = \frac{n!}{1!}$ as we have $\binom{n}{1}$ choices for the person to get their bat back, and $(n-1)!$ ways to arrange the other guests and bats.  
}
\ppart{10} Use the principle of inclusion-exclusion to determine the number of ways in which no guest got their hat back.

\solution{
Following the reasoning in part b, we can use PIE to conclude that the number of ways in which at least one guest got their hat back will be:

\begin{equation*}
\begin{split}
\binom{n}{1} \cdot (n-1)! - \binom{n}{2} \cdot (n-2)! + \ldots (-1)^{n-1} \binom{n}{n} \cdot (n-n)! &= n! - \frac{n!}{2!} + \frac{n!}{3!} - \ldots + (-1)^{n-1} \frac{n!}{n!}
\end{split}
\end{equation*}

Therefore by subtracting this value from part a as we are counting the complement, the answer will be $n! - (n! - \frac{n!}{2!} + \frac{n!}{3!} - \ldots + (-1)^{n-1} \frac{n!}{n!})$.  
}
\eparts

%
%Fearing retribution for the many long hours his students spent 
%completing problem sets, Prof. Leighton decides to convert his 
%office into a reinforced bunker. His only remaining task is to 
%set the 10-digit numeric password on his door.  Knowing the 
%students are a clever bunch, he is not going to pick any passwords 
%containing the forbidden consecutive sequences "18062", "6042" 
%or "35876" (his MIT extension).
%
%%\bparts
%%\ppart{10}\label{without-replacement}
%How many 10-digit passwords can he pick that don't contain forbidden 
%sequences if each number $0, 1, \ldots, 9$ can only be chosen once 
%(i.e. without replacement)?
%
%\solution{The number of passwords he can choose is the number of 
%permutations of the 10 digits minus the number of passwords containing 
%one or more of the forbidden words, which we will find using 
%inclusion-exclusion.
%
%There are 6 positions 18062 could appear and the remaining digits 
%could be any permutation of the remaining 5 digits.  Therefore,
%there are $6 \cdot 5!$ passwords containing 18062. Similarly, there 
%are $7 \cdot 6!$ passwords containing 6042 and $6 \cdot 5!$ passwords 
%containing 35876.
%
%Each of the forbidden words contain the digit 6 and since he must 
%choose each number exactly once, the only way two forbidden words 
%can appear in the same password is if they overlap at 6.  The only 
%case where this can happen is if the password contains 35876042 and
%there are $3 \cdot 2!$ such passwords.
%
%By inclusion-exclusion the total number of passwords not containing
%any of the forbidden words is
%\[
%10! - (6 \cdot 5! + 7 \cdot 6! + 6 \cdot 5!) + 3 \cdot 2! = 3622326
%\]
%}
%
%%\ppart{10} How many 10-digit passwords can he pick that don't contain
%%forbidden sequences if each number $0, 1, \ldots, 9$ can be chosen any 
%%number of times (i.e. with replacement)?
%%
%%\solution{The number of passwords he can choose is the number of 
%%length 10 strings over the alphabet $0, 1, \ldots, 9$ minus the 
%%number of passwords containing one or more of the forbidden words, 
%%which we will again find using inclusion-exclusion.
%%
%%There are 6 positions 18062 could appear and the remaining digits 
%%could be any 5-digit string.  Therefore, there are $6 \cdot 10^5$ 
%%passwords containing 18062. Similarly, there are $7 \cdot 10^6$ 
%%passwords containing 6042 and $6 \cdot 10^5$ passwords containing 
%%35876.
%%
%%As in part \ref{without-replacement} the only forbidden words
%%that can overlap are 6042 and 35876, however, it is now possible
%%to have a password containing both 35876 and 18062, both 6042 and 
%%18062, or even both 6042 and 35876 without overlap.  No 10-digit
%%password can contain all 3 forbidden words.
%%
%%There are only 2 passwords that contain both 35876 and 18062:
%%3587618062 and 1806235876.
%%
%%To count the passwords containing both 6042 and 18062, there are 2 
%%possibilities: 6042 can either come before or after 18062.  For 
%%each case there are 3 possible positions for the remaining digit: 
%%the first digit, the last digit or between the two words.  There 
%%are 10 values for the remaining digit.  Therefore, there are 
%%$2 \cdot 3 \cdot 10$ passwords containing both 6042 and 18062.
%%
%%By similar reasoning there are $2 \cdot 3 \cdot 10$ passwords 
%%containing both 6042 and 35876 {\it without overlap} and there 
%%are $3 \cdot 10^2$ passwords containing 35876042.
%%
%%By inclusion-exclusion the total number of passwords not containing
%%any of the forbidden words is
%%\[
%%10^10 - \left[2\cdot(6 \cdot 10^5) + 7 \cdot 10^6 \right] 
%%      + \left[2 + 2\cdot(2 \cdot 3 \cdot 10) + 3 \cdot 10^2\right]
%%= 10^10 - 8199578 = 9991800422.
%%\]
%%}
%%
%%\eparts
\end{problem}



\begin{problem}{45} Be sure to show your work to receive full credit. In this problem we assume a standard card deck of 52 cards.
\bparts

\ppart{4} How many 5-card hands have a single pair and 
no 3-of-a-kind or 4-of-a-kind?

\solution{
There is a bijection with sequence of the form:

\[
(\text{value of pair}, \text{suits of pair}, \text{value of other three cards}, \text{suits of other three cards})
\]

Thus, the number of hands with a single pair is:

\[
13 \cdot \binom{4}{2} \cdot \binom{12}{3} \cdot 4^{3} = 1,098,240
\]


Alternatively, there is also a 3!-to-1 mapping to the tuple:
\[
\begin{array}{l}
(\text{value of pair}, \text{suits of pair}, \\
\text{value 3rd card}, \text{suit 3rd card},
\text{value 4th card}, \text{suit 4th card},
\text{value 5th card}, \text{suit 5th card})
\end{array}
\]

Thus, the number of hands with a single pair is:

\[
\frac{13 \cdot \binom{4}{2} \cdot 12 \cdot 4 \cdot 11 \cdot 4 \cdot 10 \cdot 4}{3!} = 1,098,240
\]
}

\ppart{4} For fixed positive integers $n$ and $k$, how many nonnegative 
integer solutions $x_0,x_1,\ldots,x_k$ are there to the following equation?
\[
\sum_{i=0}^k x_i = n
\]

\solution{There is a bijection from the solutions of the equation
to the binary strings containing $n$ zeros and $k$ ones where
$x_0$ is the number of 0s preceding the first 1, $x_k$ is the 
number of 0s following the last 1 and $x_i$ is the number of 0s 
between the $i^{th}$ and $(i+1)^{th}$ 1 for $0 < i < k$.

\[
\binom{n+k}{k}
\]
}

\ppart{4} For fixed positive integers $n$ and $k$, how many nonnegative 
integer solutions $x_0,x_1,\ldots,x_k$ are there to the following equation? 
\[
\sum_{i=0}^k x_i \leq n
\]

\solution{There is a bijection from the solutions of
\begin{align*}
\sum_{i=0}^k x_i
 & \leq n & \\
 & = n - x_{k+1} & \text{(for some $x_{k+1} \geq 0$)}
\end{align*}
and the solutions of
\[
\sum_{i=0}^{k+1} x_i = n.
\] 

\[
\binom{n+k+1}{k+1}
\]
}

\ppart{4} How many simple undirected graphs are there with $n$ vertices?

\solution{There are $\binom{n}{2}$ potential edges, each of which may or
may not appear in a given graph.  Therefore, the number of graphs is:
\[
2^{\binom{n}{2}}
\]
}

\ppart{4} How many directed graphs are there with $n$ vertices (self loops allowed)?

\solution{There are $n^2$ potential edges, each of which may or
may not appear in a given graph.  Therefore, the number of graphs is:
\[
2^{n^2}
\]
}

\newcommand{\beats}{\rightarrow}

\ppart{4} How many tournament graphs are there with $n$ vertices?

\solution{There are no self-loops in a tournament graph and 
for each of the $\binom{n}{2}$ pairs of distinct vertices $a$ and $b$,
either $a \beats b$ or $b \beats a$ but not both. Therefore, the 
number of tournament graphs is:
\[
2^{\binom{n}{2}}
\]
}

\ppart{4} How many acyclic tournament graphs are there with $n$ vertices?

\solution{For any path from $x$ to $y$ in a tournament graph, an edge
$y \beats x$ would create a cycle.  Therefore in any acyclic tournament
graph, the existence of a path between distinct vertices $x$ and $y$ 
would require the edge $x \beats y$ also be in the graph.  That is, the
"beats" relation for such a graph would be transitive.  Since each 
pair of distinct players are comparable (either $x \beats y$ or 
$y \beats x$) we can uniquely rank the players $x_1 < x_2 < \cdots < x_n$.
There are $n!$ such rankings.
}

\ppart{4}
How many numbers are there that are in the range $[1..700]$ which are divisible by $2,5$ or $7$?
\solution{
Let $S$ be the set of all numbers in range $[1..700]$. Let $S_2$ be the numbers in this range divisible by $2$, $S_5$ be the numbers in this range divisible by $5$ and $S_7$ be the numbers in this range divisible by $7$. By inclusion-exclusion, the number of elements in $S$ divisible by $2$, $5$ or $7$ is
\begin{eqnarray*}
n &=& |S_2|+|S_5|+|S_7| - |S_2S_5|-|S_2S_7|-|S_5S_7|+|S_2S_5S_7|\\
&=& \frac{700}{2}+\frac{700}{5}+\frac{700}{7}-\frac{700}{2\cdot5}-\frac{700}{2\cdot7}-\frac{700}{5\cdot7}+\frac{700}{2\cdot5\cdot7}\\
&=& 350+140+100-70-50-20+10\\
&=& 460.
\end{eqnarray*}
}

\ppart{4}
How many ways are there to list the digits $\{ 1, 1, 2, 2, 3, 4, 5 \}$ so that the $1$'s are always consecutive?

\solution{
Just treat the ones as a block $B$.  Then we just list the permutations of $\{B, 2, 2, 3, 4, 5\}$, which is $\frac{6!}{2!}$.  
}

\ppart{4} What is the coefficient of $x^3y^2z$ in the expansion of $(x + y + z)^6$?
\solution{
This is just $\frac{6!}{2! \cdot 3! \cdot 1!}$ by considering the number of ways of selecting 3 x's, 2 y's, and 1 z from the 6 factors (x + y + z) .  
}

\ppart{5} How many unique terms are there in the expansion of $(x + y + z)^6$?

\solution{
We want to essentially find non-negative integers $a, b, c$ such that $x^ay^bz^c$  is a term and so $a + b + c = 6$.  Hence we have a bijection from this problem with the problem of finding non-negative integers $a, b, c$ such that $a + b + c = 6$.  Hence, by following the solution in part b, there are $\binom{8}{2}$ terms.   
}

\eparts
\end{problem}

% S08 - cp11w
\begin{problem}{15}
Give a combinatorial proof of the following theorem:
\[
n 2^{n-1} = \sum_{k=1}^n k \binom{n}{k}
\]


\solution{Here is the \textbf{first solution}:

Consider the ways of forming a committee out of $n$ members and picking $1$ leaders.

The left hand side indicates picking a leader first out of the $n$ members ($\binom{n}{1}$ ways), and then picking a subset of the remaining $n-1$ members to form the committee ($2^{n-1}$ ways).  

The right hand side indicates picking a committee first, and then selecting a leader from the committee.  The committee can have any size from $1$ to $n$, since we have to have at least one member in the committee to be the leader.  Hence we sum from $k = 1$ to $n$ of the number of ways to select a $k$ person committee from $n$ people ($\binom{n}{k}$ ways), and then selecting a leader from the committee in $\binom{k}{1}$ ways.

Here is a \textbf{second solution}:

 Let $P = \{0,\dots,n-1\} \times \{0,1\}^{n-1}$.  On
the one hand, there is a bijection from $P$ to $S$ by mapping $(k,x)$ to
the word obtained by inserting a * just after the $k$th bit in the
length-$n-1$ binary word, $x$.  So
\begin{equation}\label{cp11m.P}
\card{S} = \card{P}= n 2^{n-1}
\end{equation}
by the Product Rule.

On the other hand, every sequence in $S$ contains between 1 and $n$
nonzero entries since the $*$, at least, is nonzero.  The mapping from a 
sequence in $S$ with exactly $k$ nonzero entries to a pair consisting of
the set of positions of the nonzero entries and the position of the *
among these entries is a bijection, and the number of such pairs is
$\binom{n}{k}k$ by the Generalized Product Rule.
Thus, by the Sum Rule:
\[
\card{S} = \sum_{k=1}^n k \binom{n}{k}
\]
Equating this expression and the expression~\eqref{cp11m.P} for $\card{S}$
proves the theorem.}

\end{problem}

%%%%%%%%%%%%%%%%%%%%%%%%%%%%%%%%%%%%%%%%%%%%%


\end{document}

