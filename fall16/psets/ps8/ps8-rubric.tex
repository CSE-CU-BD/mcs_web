\documentclass[12pt,twoside]{article}   
\usepackage{light}

\hidesolutions
\showsolutions

\begin{document}
\problemset{8}{November 2, 2016}{Tuesday, November 8}


\newcommand{\proofrubric}[3][Any correct proof.]
  {
  	\begin{center}
	\fbox{\begin{minipage}{35em}
	\textbf{Rubric}
	\par
	[#3pts] #1
		\begin{center}
		\textbf{or}
		\end{center}
	#2
	\end{minipage}}
	\end{center}
  }
  
  \newcommand{\proofrubriceach}[3][Any correct proof.]
  {
  	\begin{center}
	\fbox{\begin{minipage}{35em}
	\textbf{Rubric For Each}
	\par
	[#3pts] #1
		\begin{center}
		\textbf{or}
		\end{center}
	#2
	\end{minipage}}
	\end{center}
  }
  
  \newcommand{\rubric}[1]
  {
  	\begin{center}
	\fbox{\begin{minipage}{35em}
	\textbf{Rubric}
	\par
	#1
	\end{minipage}}
	\end{center}
  }

\begin{problem}{15}
This problem continues the study of the asymptotics of factorials.
\bparts
\ppart{5}

Either prove or disprove each of the following statements.
\begin{itemize}
\item $n! = O((n+1)!)$
\item $n! = \Omega((n+1)!)$
\item $n! = \Theta((n+1)!)$
\item $n! = \omega((n+1)!)$
\item $n! = o((n+1)!)$
\end{itemize}
\rubric{
[5pts] $n! = o((n+1)!)$ and $n! = O((n+1)!)$ \par
[-1pt] Each incorrect one
}
\solution{Observe that $n! = (n+1)!/(n+1)$, and thus $n! = o((n+1)!)$. Thus, $n! = O((n+1)!)$ as well, but the remaining statements are false.}
\rubric{
[2.5pts] $n! = o \left( \left (\frac{n}{2} \right )^{n+e} \right)$ \par
[2.5pts] Show work using Stirling's formula \par
[-1pt] Math errors
}
\ppart{5}
Is $n! = o \left( \left (\frac{n}{2} \right )^{n+e} \right)$ or is $n! = \omega \left( \left (\frac{n}{2} \right )^{n+e} \right)$? (Hint: Use Stirling's formula)

\solution{ We show that $n! = o \left( \left (\frac{n}{2} \right )^{n+e} \right)$.

By Stirling's formula:
\[
n! \sim \sqrt{2 \pi n} \left(\frac{n}{e}\right)^{n}
\]

Then, 
\begin{equation}
\begin{split}
\lim_{n\to\infty} \frac{n!}{ \left (\frac{n}{2} \right )^{n+e} } &= \lim_{n\to\infty} \sqrt{2 \pi n} \left(\frac{n}{e}\right)^{n}  \left (\frac{2}{n} \right )^{n+e}\\
&= \lim_{n\to\infty} \sqrt{2 \pi n} \left(\frac{2}{e}\right)^{n} \left( \frac{2}{n} \right) ^e \\
&= 0 \quad \text{Since $\frac{2}{e} < 1$}
\end{split}
\end{equation}

}

\ppart {5}
Show that $n! = \Omega(3^n)$
\proofrubric[Any correct solution with work shown]{
[2.5pts] Take the limit \par
[2.5pts] Show $<$ relation \par
[-1pt] Math errors\par
[-1pt] No work shown
}{5}
\solution{
We follow the definition of $\Omega$:

\begin{equation}
\begin{split}
\lim_{n\to\infty}  \frac{3^n}{n!} &= \lim_{n\to\infty} \frac{3^n}{1 \cdot 2 \cdot 3 \ldots \cdot n} \\
& < \lim_{n\to\infty} \frac{3^n}{1 \cdot 2 \cdot 3 \cdot 3 \ldots \cdot 3} \\
&= \lim_{n\to\infty} \frac{3^2}{1 \cdot 2} \\
&= \frac{9}{2} < \infty \\ 
\end{split}
\end{equation}

}
\eparts
\end{problem}



%%%%%%%%%%%%%%%%%%%%%%%%%%%%%%%%%%%%%%%%%%%%%%%%%%%%%%%%%%%%%%%%%%%%%%%%%%%%%%%
% Problem written Fall 2008

\begin{problem}{25}
Find $\Theta$ bounds for the following divide-and-conquer recurrences.
Assume $T(1) = 1$ in all cases.  Show your work.
\proofrubriceach[Any correct solution with work shown]{
[1pts] Use Akra-Bazzi \par
[2pts] Correct calculation of $p$ \par
[2pts] Correct $\Theta$ bounds\par
[-1pt] No work shown
}{5}
\begin{problemparts}

\ppart{5} $T(n) = 8T(\floor{n/2}) + n$

\ppart{5} $T(n) = 2T(\floor{n/8} + 1/n)+n$

\ppart{5} $T(n) = 7T(\floor{n/20}) + 2T(\floor{n/8}) + n$

\ppart{5} $T(n) = 2T(\floor{n/4}+1)+n^{1/2}$

\ppart{5} $T(n) = 3T(\floor{n/9+n^{1/9}}) + 1$

\end{problemparts}

\end{problem}

\solution{
We use the method of Akra-Bazzi for these problems.

\begin{enumerate}[(a)]

\item
We see that $a = 8$, $b = 1/2$, $h = \floor{n/2} - n/2$ so $p=3$ gives $ab^p = 1$.  
\[
T(n) = \Theta(n^{3}(1+ \int_1^n \frac{u}{u^{4}} du)) = \Theta(n^{3}(1+ \int_1^n u^{-3} du)) 
= \Theta(n^3).
\]

\item
$a_1 = 2$, $b_1 = 1/8$, $h_1(n) = \floor{n/8} - n/8 +  1/n$, 
$g(n) = n$, $p = 1/3$,
\begin{align*}
T(n) &= \Theta\left(n^{p}\left(1+ \int_1^n \frac{g(u)}{u^{p+1}} du \right)\right)\\
&= \Theta\left(n^{1/3}\left(1+\int_1^n \frac{u}{u^{4/3}}\, du \right)\right)\\
&= \Theta\left(n^{1/3} + n^{1/3} \int_1^n u^{-1/3} \, du\right)\\
&= \Theta(n^{1/3} + n^{1/3} \frac{3}{2}(n^{2/3}-1))\\
&= \Theta(n).
\end{align*}

\item
$a_1 = 7$, $b_1 = 1/20$,$a_2 = 2$, $b_2 = 1/8$, $h_1(n) = \floor{n/20} - n/20$, $h_2(n) = \floor{n/8} - n/8$, and $g(n) = n$.  Finally, note that although we do not know what $p$ is, we are guaranteed
that $p<1$.
\begin{align*}
T(n) = \Theta(n^p(1+ \int_1^n \frac{u}{u^{p+1}} du)) &= \Theta(n^p(1+ \int_1^n u^{-p} du))\\
&= \Theta(n^p + n^p \frac{1}{1-p}(n^{1-p}-1))\\
&= \Theta(n).
\end{align*}

\item
$a_1 = 2$, $b_1 = 1/4$, $h_1(n) = \floor{n/4}-n/4 + 1$,
 $g(n) = n^{1/2}$, $p = 1/2$,
\[
T(n) = \Theta(n^{1/2}(1+ \int_1^n \frac{u^{1/2}}{u^{3/2}} du)) =
\Theta(n^{1/2} \log n).
\]

\item
$a_1 = 3$, $b_1 = 1/9$, $h_1(n) = \floor{n/9+ n^{1/9}} - n/9$, 
$g(n) = 1$, $p = 1/2$,
\[
T(n) = \Theta(n^{1/2}(1+ \int_1^n \frac{1}{u^{3/2}} du)) =
\Theta(n^{1/2}).
\]

\end{enumerate}
}

\begin{problem}{20}
It is easy to misuse induction when working with asymptotic notation.

\textbf{False Claim}  If

$$T(1) = 1 \textrm{ and}$$
$$T(n) = 4T(n/2) + n$$

Then T(n) = O(n).

\textbf{False Proof}
We show this by induction.  Let $P(n)$ be the proposition that $T(n) = O(n)$.  

\textbf{Base Case}:
$P(1)$ is true because $T(1) = 1 = O(1)$. 

\textbf{Inductive Case}: 
For $n \geq 1$, assume that $P(n-1), \ldots, P(1)$ are true.  We then have that
$$T(n) = 4T(n/2) + n = 4O(n/2) + n = O(n)$$

And we are done.

\bparts
	\ppart{5} Identify the flaw in the above proof.  
	\rubric{
		[3pts] Flaw is in the predicate \par
		[2pts] Correct reasoning. $O(n)$ does not depend on $n$ but on the limit as $n$ approaches infinity.
	}
	\ppart{5} Using Akra-Bazzi theorem, find the correct asymptotic behavior of this recurrence.
	\rubric{
		[5pts] $\Theta(n^2)$ \par
		[-1pt] Math errors
	}{5}
	\ppart{10} We have now seen several recurrences of the form $T(n) = aT(\floor{n/b}) + n$.  Some of them give a runtime that is $O(n)$, and some don't.  Find the relationship
	between $a$ and $b$ that yields $T(n) = O(n)$, and prove that this is sufficient.
	\rubric{
		[5pts] $ab^p$ = 1 \par
		[5pts] When $a < b$, $p < 1$
	}
	
\eparts
\end{problem}
\solution{
\begin{enumerate}[(a)]
	\item 
		The flaw is that $P(n)$ is a predicate on $n$, whereas $O(n)$ is a statement not on $n$, but on the limit of $n$ as $n$ approaches infinity.  $T(n) = O(n)$ does not depend on the value of $n$ - it is either true or false.
	\item
		We have that $p=2$, so $T = \Theta( n^2 ( 1 + \int_{1}^{n}(u/u^3)du)) = \Theta(n^2)$.
	\item
		From analyzing the integral we can see that any case where $p < 1$ will give a linear solution, so having the condition $a < b$ is sufficient.
	
\end{enumerate}

}
\begin{problem}{15}
Define the sequence of numbers $A_i$ by

$A_0=2$

$A_{n+1}=A_n/2 + 1/A_n$ (for $n \geq 1$)

Prove that $A_n\leq \sqrt{2}+1/2^n$ for all $n\geq 0$.
\proofrubric{
[2pts] Proof by induction \par
[2pts] Hypothesis: $P(n)$ be the proposition that $A_n \leq \sqrt{2} + 1 / 2^n$ \par
[2pts] Base case: $A_0=2\leq \sqrt{2}+1/2^0$ \par
[4pts] Show that $A_n \geq \sqrt{2}$ \par
[5pts] Substitute in proper values to get $A_{n+1} = \sqrt{2} + 1/2^{n+1}$ \par
[-2pts] Math errors
}{15}
\solution{
\begin{proof}
The proof is by induction on $n$.  Let $P(n)$ be the proposition that $A_n \leq \sqrt{2} + 1 / 2^n$.

{\bf Base case:}  
$A_0=2\leq \sqrt{2}+1/2^0$ is true. 

{\bf Inductive step:}
Let $n\geq 0$ and assume the inductive hypothesis $A_n\leq \sqrt{2}+1/2^n$.
We need the following lemma.

\begin{lemma*} For real numbers $x>0$, $x/2+1/x\geq \sqrt{2}$.
\end{lemma*}

\begin{proof}
For real numbers $x>0$,
\begin{eqnarray*}
&& x/2+1/x\geq \sqrt{2} \\
&\Leftrightarrow & x^2+2\geq 2\sqrt{2}\cdot x \\
&\Leftrightarrow & x^2-2\sqrt{2} \cdot x+2 \geq 0 \\
&\Leftrightarrow & (x-\sqrt{2})^2 \geq 0,
\end{eqnarray*}
which is true.
\end{proof}

By using induction it is straightforward to prove that
$A_n>0$ for $n\geq 0$ (base case: $A_0=2>0$; inductive step: if $A_n>0$, then $A_{n+1}=A_n/2 + 1/A_n>0$). By the lemma,
$A_n\geq \sqrt{2}$ for $n\geq 0$.
Together with the inductive hypothesis
this can be used in the following derivation:
\begin{eqnarray*}
A_{n+1} &=& A_n/2 + 1/A_n \\
&\leq & (\sqrt{2}+1/2^n)/2+1/\sqrt{2}\\
&=& \sqrt{2}+1/2^{n+1}.
\end{eqnarray*}

This completes the proof.
\end{proof}
}

\end{problem}

%%%%%%%%%%%%%%%%%%%%%%%%%%%%%%%%%%%%%%%%%%%%%%%%%%%%%%%%%%%%%%%%%%%%%%%%%%%%%%%
% Based on problem from Fall 2006, pset 8 (I changed some of the numbers)

\begin{problem}{25}
Find closed-form solutions to the following linear recurrences.

\begin{problemparts}

\ppart{5} $x_n = 5x_{n-1} - 6x_{n-2} \quad (x_0 = 0, x_1 = 1)$
\rubric{
	[2pts] Find $r = 2, 3$\par
	[1pt] $x_n = A2^n + B3^n$ \par
	[2pts] $x_n = -2^n + 3^n$ \par
	[-1pt] Math errors
}
\solution{
The characteristic equation is just $r^2 - 5r + 6 = 0$, which has solutions $r = 2, 3$.

Hence $x_n = A2^n + B3^n$ for $n \in \mathbb{N}$.  Now $x_0 = 0$ implies $A + B = 0$, and $x_1 = 1$ implies, $2A + 3B = 1$.  Hence $B = 1, A = -1$.  

Thus, the complete solution is $x_n = -2^n + 3^n$ for $n \in \mathbb{N}$.

}

\ppart{10} $x_n = 4x_{n-1} - 4x_{n-2} \quad (x_0 = 0, x_1 = 2)$
\rubric{
	[2pts] Find $r = 2$\par
	[2pts] $x_n = A2^n + Bn2^n$ \par
	[2pts] Substitute in $x_0$ \par
	[2pts] Substitute in $x_1$ \par
	[2pts] $x_n = n2^n$ \par
	[-1pt] Math errors
}
\solution{
The characteristic equation is $r^2 -4r + 4 = 0$, which has one solution $r = 2$ of multiplicity $2$.

Hence $x_n = A2^n + Bn2^n$ for  $n \in \mathbb{N}$.  Now $x_0 = 0$ implies that $A = 0$, and $x_1 = 2$ implies that $B = 1$.  

Thus, the complete solution is $x_n = n2^n$ for $n \in \mathbb{N}$.
}

\ppart{10} 
$x_{n} = 4x_{n-1} - x_{n-2} - 6x_{n-3} \quad
	(x_0 = 3, x_1 = 4, x_2 = 14)$
\rubric{
	[3pts] Find $r = -1,2,3$\par
	[2pts] $x_n = A (-1)^n + B (2)^n + C (3)^n$ \par
	[1pt] Substitute in $x_0$ \par
	[1pt] Substitute in $x_1$ \par
	[1pt] Substitute in $x_2$ \par
	[2pts] $x_n  =  (-1)^{n} + 2^{n} + 3^{n}$ \par
	[-1pt] Math errors
}
\solution{
The characteristic equation is $r^3 - 4r^2 + r + 6 = 0$.

Generally, solving a cubic equation is a difficult problem.  However, we can find from inspection that the
roots are:
\begin{align*}
r_1 & = -1 \\
r_2 & = 2 \\
r_3 & = 3
\end{align*}

Therefore a general form for a solution is
\[
x_n = A (-1)^n + B (2)^n + C (3)^n.
\]

Substituting the initial conditions into this general form gives a
system of linear equations.
\begin{eqnarray*}
3 & = & A + B + C \\
4 & = & -A + 2B + 3C \\
14 & = & A + 4B + 9C
\end{eqnarray*}

The solution to this linear system is $A = 1$, $B = 1$, and
$C = 1$.  The complete solution to the recurrence is therefore
\[
x_n  =  (-1)^{n} + 2^{n} + 3^{n}.
\]
}


\eparts
\end{problem}

\begin{problem}{25}
In this problem, we will solve inhomogeneous linear recurrences.  For the following problems, use the technique learned in class.  That is, first solve the homogeneous linear recurrence, and guess the form of the particular solution.  Then add the particular solution and the homogeneous solution to get the general solution, and then use the boundary case to determine remaining constants.  
  {
  	\begin{center}
	\fbox{\begin{minipage}{35em}
	\textbf{Note}
	\par
	If the correct solution is presented with proper work shown, the student automatically receives full credit.
	\end{minipage}}
	\end{center}
  }
\bparts
\ppart{10} Find the solution to $x_{n} = 3x_{n-1} + n \quad (x_0 = 2)$.
\rubric{
	[2pts] Find root $r=3$ \par
	[2pts] Find general solution is $x_n = A3^n$ \par
	[1pt] Guess particular solution of the form $an + b$ \par
	[2pts] Solve $an + b = 3a(n-1) + 3b + n$ for $a = \frac{-1}{2}, b = \frac{-3}{4}$ \par
	[1pt] Plug in $x_0$ after finding homogeneous and inhomogeneous terms \par
	[2pts] $x_n = \frac{11\cdot 3^n - 2n - 3 }{4}$ \par
	[-1pt] Math errors
}
\solution{
First we find the general solution to the homogeneous recurrence, which is $x_n = 3x_{n-1}$.  The characteristic equation is $r - 3 = 0$ so $r = 3$, and $x_n = A3^n$.  

Now, we find the particular solution.  As the inhomogeneous term is $n$, we make a guess that the particular solution is of the form $an + b$. 

If this is true, then we must have:
\begin{equation}
\begin{split}
an + b &= 3a(n-1) + 3b + n \\
3an - an + n -3a + 3b -b  &= 0 \\
(2a + 1)n + 2b - 3a &= 0 \\
\end{split}
\end{equation}

Thus $2a +1 = 0$ and $2b -3a = 0$, and so $a = \frac{-1}{2}, b = \frac{-3}{4}$.
Hence the particular solution is of the form $\frac{-1}{2}n + \frac{-3}{4}$.

This means the general solution is of the form $A3^n + \frac{-1}{2}n + \frac{-3}{4}$.

Now substituting in $x_0 = 2$, we have that $A = \frac{11}{4}$.

Hence, the complete solution is:

$x_n = \frac{11\cdot 3^n - 2n - 3 }{4}$ for $n \in \mathbb{N}$.
}


\ppart{15} Find the solution to $x_{n} = -x_{n-1} + 2 x_{n-2} + n \quad (x_0 = 5, x_1 = -4/9)$.
\rubric{
	[2pts] Find roots $r=1, -2$ \par
	[2pts] Find general solution is $x_n = A3^n$ \par
	[3pts] Guess particular solution of the form $an + b$ and find that it does not work\par
	[1pt] Guess particular solution of the form $an^2 + bn + c$\par
	[3pts] Particular solution of $x_n = \frac{1}{6}n^2 - \frac{7}{18} n$ \par
	[2pts] Plug in initial conditions \par
	[2pts] $x_n =3+  2(-2)^{n} + \frac{1}{6}n^2 + \frac{7}{18} n $ \par
	[-2pts] Math errors
}
\solution{
First, we find the general solution to the homogeneous recurrence.  The
characteristic equation is $r^2 + r - 2 = 0$.  The roots of this
equation are $r_1 = 1$ and $r_2 = -2$.  Therefore, the general solution
to the homogenous recurrence is

\begin{eqnarray*}
x_n & = & A(-1)^n + B 2^n.
\end{eqnarray*}

Now we must find a particular solution to the recurrence, ignoring the
boundary conditions.  Since the inhomogenous term is linear, we guess
there is a linear solution, that is, a solution of the form $an + b$.  If
the solution is of this form, we must have
\[
an+b = -a(n-1) - b + 2a(n-2) + 2b + n
\]
Gathering up like terms, this simplifies to
\[
n(a + a - 2a - 1) + (b + a + b + 4a - 2b) = 0
\]
which implies that
\[
n = -5a
\]

But $a$ is a constant, so this cannot be so.  
So we make another guess, this time that there is a quadratic solution of
the form $an^2 + bn +c$.  If the solution is of this form, we must have
\[
an^2 + bn + c = -[a(n-1)^2 + b(n-1) + c] + 2[a(n-2)^2 + b(n-2) + c] + n
\]
which simplifies to
\[
n^2(a + a - 2a) + n(b + b - 2a + 8a - 2b - 1) + (c + a - b + c - 8a + 4b - 2c) = 0
\]

This simplifies to

\[
	n(6a - 1) + (-7a + 3b) = 0
\]
This last equation is satisfied only if the coefficient of $n$ and the
constant term are both zero, which implies $a = 1/6$ and $b = 7/18$.
Apparently, any value of $c$ gives a valid particular solution.  For
simplicity, we choose $c = 0$ and obtain the particular solution:
\[
x_n = \frac{1}{6}n^2 - \frac{7}{18} n.
\]

The complete solution to the recurrence is the homogenous solution
plus the particular solution:
\[
x_n = A(-1)^n + B 2^n + \frac{1}{6}n^2 - \frac{7}{18} n
\]
Substituting the initial conditions gives a system of linear equations:
\begin{eqnarray*}
5	& = &	A + B \\
-4/9	& = &	-A + 2 B - + 1/6 + 7/18
\end{eqnarray*}

The solution to this linear system is $A = 3$ and $B = 2$.
Therefore, the complete solution to the recurrence is
\[
x_n =3+  2(-2)^{n} + \frac{1}{6}n^2 + \frac{7}{18} n 
\]
}

\end{problemparts}

\end{problem}

\end{document}


