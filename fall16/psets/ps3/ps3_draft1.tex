\documentclass[12pt,twoside]{article}   
\usepackage{light}

\hidesolutions
\showsolutions

%\newcommand{\new}{\forall^{(1)}}

% logic terms
%\newcommand{\true}{\mathbin{\mathtt{true}}}
%\newcommand{\false}{\mathbin{\mathtt{false}}}
%\newcommand{\nand}{\mathbin{\mathtt{nand}}}

% subparts
%\newcounter{problemsubpart}
%\renewcommand{\theproblemsubpart}{\roman{problemsubpart}}
%\newcommand{\bsubparts}{
%    \begin{list}{\textbf{(\roman{problemsubpart})}}{\usecounter{problemsubpart}}}
%\newcommand{\psubpart}
%  {\item%
%   \pdfbookmark[2]{(\theproblemsubpart)}{Problem\theproblemthm\theproblemsubpart}}
%\newcommand{\esubparts}{\end{list}}

%\renewcommand{\divides}{\mbox{ | }}


\newcommand{\hint}[1]{({\it Hint: #1})}
\newcommand{\card}[1]{\left|#1\right|}

\begin{document}
\problemset{3}{September 20, 2016}{Monday, September 26}
\noindent \textbf{Reading Assignment:}   Sections 4.0-4.3, 4.5, 4.6
\\

% taken from: new problem
% comments: basic computations using number theory
\begin{problem}{18}
\bparts
\ppart{4} Use the Pulverizer to find integer values of $x, y$ that satisfy $71x + 50y = 1$.  What is the inverse of $71$ modulo $50$ (Write the inverse as a number in the set $\{ 1, 2, \ldots, 49 \}$?
\solution{
\[
\begin{array}{ccccrcl}
x & \quad & y & \quad & \rem(x,y) & = & x - q \cdot y \\ \hline
71 && 50 && 21  & = &   71 - 1 \cdot 50 \\
50 && 21 && 8   & = &   50 - 2 \cdot 21 \\
&&&&            & = &   50 - 2 \cdot (71 - 1 \cdot 50) \\
&&&&            & = &   -2 \cdot 71 + 3 \cdot 50 \\
21 && 8  && 5   & = &   21 - 2 \cdot 8 \\
&&&&            & = &   (71 - 1 \cdot 50) -
                               2 \cdot ( -2 \cdot 71 + 3 \cdot 50) \\
&&&&            & = &   5 \cdot 71 - 7 \cdot 50 \\
8 &&  5  && 3   & = &   8 - 5 \\
   &&&&            & = &   (-2 \cdot 71 + 3 \cdot 50) - (5 \cdot 71 - 7 \cdot 50) \\
&&&&            & = &   -7 \cdot 71 + 10 \cdot 50 \\
5 &&  3  && 2   & = &   5 - 3 \\
  &&&&            & = & ( 5 \cdot 71 - 7 \cdot 50 ) - (-7 \cdot 71 + 10 \cdot 50) \\
  &&&&            & = &   12 \cdot 71 - 17 \cdot 50 \\
3 &&  2  && 1   & = &   3 - 2 \\
  &&&&            & = & ( -7 \cdot 71 + 10 \cdot 50 ) - (12 \cdot 71 - 17 \cdot 50) \\
  &&&&            & = &  \fbox{$ -19 \cdot 71 + 27 \cdot 50$} \\
2  && 1  && 0
\end{array}
\]
Hence we have $x = -19, y = 27$.  Considering the equation modulo $50$, we have that $-19 \cdot 71 \equiv 1$ (mod $50$).  Thus the inverse of $71$ mod $50$ is $31$, as $31 \equiv -19$ (mod $50$).
}
\ppart{4} Use the Pulverizer to find integer values of $x, y$ that satisfy $43x + 64y = 1$.  What is the inverse of $64$ modulo $43$ (Write the inverse as a number in the set $\{1, 2, \ldots, 42 \}$?  
\solution{
\[
\begin{array}{ccccrcl}
x & \quad & y & \quad & \rem(x,y) & = & x - q \cdot y \\ \hline
64 && 43 && 21  & = &   64 - 1 \cdot 43 \\
43 && 21 && 1   & = &   43 - 2 \cdot 21 \\
&&&&            & = &   43 - 2 \cdot (64 - 43) \\
  &&&&            & = &  \fbox{$ -2 \cdot 64 + 3 \cdot 43$} \\
21  && 1  && 0
\end{array}
\]
Hence we have $x = -2, y = 3$.  Considering the equation modulo $43$, we have that $-2 \cdot 64 \equiv 1$ (mod $43$).  Thus the inverse of $64$ mod $43$ is $41$, as $41 \equiv -2$ (mod $43$).
}
\ppart{4} Prove that $2 \mid (n)(n+1)$ for all integers $n$. 
\solution{
We may solve this problem in cases on whether $n$ is even or odd.
\begin{enumerate}
\item If $n$ is even, then $2 | n$ so $2 | (n)(n+1)$.
\item If $n$ is odd, then let $n = 2k -1$ for some $k \in \mathbb{Z}$.  Then $n+1 = 2k$, so $2 | (n+1)$.  Thus, $2 | (n)(n+1)$.
\end{enumerate}
}
\ppart{6} Prove that $3! \mid (n)(n+1)(n+2)$ for all integers $n$.  
\solution{
From part c, we know that $2 | (n)(n+1)$ $\forall n \in \mathbb{Z}$.  Thus, we only need to show that $3 | (n)(n+1)(n+2)$.  We again solve this problem in cases.
\begin{enumerate}
\item Suppose $3 | n$.  Then $3 | (n)(n+1)(n+2)$.
\item Suppose $n$ leaves a remainder $1$ when divided by $3$, then let $n = 3k + 1$ for some $k \in \mathbb{Z}$.  Now $n + 2 = 3k + 1 + 2 =  3(k+1)$ and so $3 | n+2$.  Thus $3 | (n)(n+1)(n+2)$.
\item Suppose $n$ leaves a remainder $2$ when divided by $3$, then let $n = 3k + 2$ for some $k \in \mathbb{Z}$.  Now $n + 1 = 3k + 2 + 1 =  3(k+1)$ and so $3 | n+1$.  Thus $3 | (n)(n+1)(n+2)$.
\end{enumerate}
}

Although we won't ask you to prove it, this formula from parts c, d actually generalizes to $k! \mid (n)(n+1)\cdot \ldots \cdot (n+k-1)$.  As an extra challenge, see if you can prove it yourself.
\eparts

\end{problem}

\begin{problem}{20}
Prove the following statements about divisibility.
\bparts
\ppart{4} If $a \mid b$, then $\forall c$, $a \mid bc$

\solution{
	If $a \mid b$, then there is some positive integer $k$ such that $b = ak$.  But then, $bc = akc = a(kc)$, which is a multiple of $a$.
}

\ppart{4} If $a \mid b$ and $a \mid c$, then $a \mid sb + tc$.
\solution{
	If $a \mid b$, then there is some positive integer $j$ such that $b = aj$.  Similarly, there is some positive integer $k$ such that $c = ak$.  But then, we can rewrite the right side
	as $s(aj) + t(ak)$.  But we can rewrite this as $a(js) + a(kt) = a(js + kt)$, which is a multiple of $a$.
}

\ppart{4} $\forall c $, $a \mid b \Leftrightarrow ca \mid cb$
\solution{
	If $a \mid b$, then there is some positive integer $k$ such that $b = ak$.  But then, we can rewrite $cb = c(ak) = ca(k)$, from which it follows that $cb$ is a multiple of $ca$.  So the implication is true. Conversely, if $ca \mid cb$ then there is some positive integer $k$ such that $cb = cak$. We can cancel $c$ from both sides to conclude that $a \mid b$. 
}

\ppart{4} $\gcd(ka, kb) = k \gcd(a, b)$
\solution{
Let $s, t$ be coefficients so that $s(ka) + t(kb) = \gcd(ka, kb)$.  We can factor out the $k$ so that $\gcd(ka, kb) = k(sa + tb)$.  We now argue that $sa + tb = \gcd(a,b)$.  Suppose it were not.  Then, there is some smaller positive linear combination of $a,b$ with coefficients $s'$ and $t'$ so that $s'a + t'b = \gcd(a,b)$.  But then, if we multiply this by $k$, we find that
$0<ks'a + kt'b = s'(ka) + t'(kb) < s(ka) + t(kb) = \gcd(ka, kb)$.  This is a contradiction with the definition of the $\gcd$, so $sa + tb = \gcd(a,b)$, and we can conclude that
$\gcd(ka, kb) = k\gcd(a, b)$.
}
\ppart{4} Prove that for integers $a, b, c, d$ and $n \geq 1$, $a \equiv b$ (mod $n$), $c \equiv d$ (mod $n$) implies $ac \equiv bd$ (mod $n$).
\solution{
We want to show that $n \mid (ac -bd)$ and we know that $n \mid (a-b)$ and $n \mid (c-d)$.  Thus we consider $(ac - bd) = (ac - bc) + (bc - bd) = c(a-b) + b(c-d)$.  We have that $n \mid c(a-b) + b(c-d)$, and so the claim follows.  
}
\eparts
\end{problem}

\begin{problem}{22}
In this problem, we are going to walk through a proof of Wilson's theorem, which states the following:
\begin{theorem}[Wilson's Theorem]
If $p$ is a prime number, then $(p-1)! \equiv -1$ (mod $p$).
\end{theorem}
\bparts
\ppart{2}  Verify that Wilson's theorem holds for $p = 2, 3$.  
\solution{
For $p = 2$, we have that $(2-1)! = 1 \equiv -1$ (mod $2$).
For $p = 3$, we have that $(3-1)! = 2 \equiv -1$ (mod $3$).
}
\ppart{6} Prove the following theorem about the existence and uniqueness of modular inverses for prime modulos.
\begin{theorem}\label{inverses}
If $p$ is a prime, show that for all $a$, if gcd($a$, $p$) = $1$, then there exists some unique $b$ such that $ab \equiv 1$ (mod $p$) and $b \in \{1, 2, \ldots p-1 \}$.
\end{theorem}
There are two components to this proof (1) to show that such a $b$ exists and (2) that there is a unique $b$.

\textit{Hint}: To show that $b$ exists, consider that since gcd($a$, $p$) = $1$, there exist integers $b, c$ such that $ab + pc = 1$.  What happens if you consider this equation modulo $p$?
\solution{
Since, gcd($a$, $p$) = $1$, we know that there exist integers $b, c$ such that $ab + pc = 1$ (The Pulverizer helps us find these integers for given values of $a, p$).  Now if we consider the equation modulo $p$.
That is
\begin{equation*}
1 = ab + pc \equiv ab ~~ (\text{mod} ~~ p)
\end{equation*}
Now we find $b'$ such that $b' \equiv b$ (mod $p$) and $b' \in \{1, 2, \ldots p-1 \}$. Therefore, we can conclude that $b'$ is the inverse of $a$ modulo $p$.  So such an inverse exists.  Now we show that such an inverse is unique.  

Suppose that there are two integers $b, b'$ such that $ab \equiv ab' \equiv 1$ (mod $p$) with $b, b' \in \{1, 2, \ldots p-1 \}$.   Then we have that $p \mid ab - ab'$, and so $p \mid (b-b')$ since $p \nmid a$.  However, since $b, b' \in \{1, 2, \ldots p-1 \}$, this is only possible if $b = b'$, and hence the inverse is unique.  

}

\ppart{6} Let $p$ be a prime number.  Prove that for integer $a$, $a^2 \equiv 1$ (mod $p$) if and only if $a \equiv \pm 1$ (mod $p$).

\textit{Hint}: Consider $a^2 - 1 = (a + 1)(a-1)$.
\solution{
This follows almost directly from the hint.  If we have $a^2 \equiv 1$ (mod $p$), then we must have that $p \mid (a^2 -1)$.  So $p \mid (a+1)(a-1)$.  However, since $p$ is prime, we can conclude $p \mid (a+1)$ or $p \mid (a-1)$.  Hence $a \equiv \pm 1$ (mod $p$).

The other direction follows since if $a \equiv \pm 1$ (mod $p$), then we have that $p \mid (a+1)$ or $p \mid (a-1)$ and so $p \mid (a^2 - 1)$ as desired.  
}

\ppart{8} Prove Wilson's theorem using the above parts.  

{Hint}: Use theorem \ref{inverses} to pair up the integers in the expansion of $(p-1)!$ with their inverses.  Based on part c, which integers don't get paired?

\solution{
Consider the integers in the expansion of $(p-1)!$.  Each of these integers is in the set $\{1, 2, \ldots p-1 \}$, and so we can pair each integer with its unique inverse modulo $p$ as we proved in theorem \ref{inverses}.  Thus we will have pairs of integers $a, b \in \{1, 2, \ldots p-1 \}$ such that $ab \equiv 1$ (mod $p$).  However, we must account for the case where $a = b$.  This implies that $a^2 \equiv 1$ (mod $p$).  However, by part c, we know that there are only two such numbers in the set $\{1, 2, \ldots p-1 \}$ for which $a^2 \equiv 1$ (mod $p$).  Namely $a = 1, (p-1)$.  Hence we have that $(p-1)! \equiv 1 \cdot (p-1)$ (mod $p$).  This means that $(p-1)! \equiv -1$ (mod $p$) as desired.  
}

\eparts
\end{problem}

\begin{problem}{20}
The following parts can be solved using Fermat's little theorem, which states that for integers $a, p$ such that gcd($a, p$) = 1, $a^{p-1} \equiv 1$ (mod $p$).

\bparts
\ppart{2} Find $3^{31}$ (mod $7$).
\solution{
By a direct application of Fermat's little theorem, $3^6 \equiv 1$ (mod $7$).  Therefore, $3^{30} \equiv 1$ (mod $7$).  Thus $3^{31} \equiv 3^{30} \cdot 3 \equiv 1$ (mod $7$).
}
\ppart{4} Prove that $7 \mid n^6 - 1$ for all integers $n$ such that gcd($n, 7$) = 1.
\solution{
As gcd($n, 7$) = 1, we have that $n^6 \equiv 1$ (mod $7$) by Fermat's little theorem.  By definition, this means that $7 \mid n^6 -1$.
}
\ppart{6} Prove that $42 \mid n^7 - n$ for all integers $n$.
\solution{
We have that
 \begin{equation*} 
 n^7 - n = n (n^6 - 1) = n(n^3 + 1)(n^3 - 1) = n(n+1)(n-1)(n^2 + n +1)(n^2 - n +1)
 \end{equation*}
We can use part d of problem 1 on this problem set to conclude that $3! \mid n(n+1)(n-1)$. So we have that $6 \mid n^7 - n$ for all integers $n$.  Now we need to show that $7 \mid n^7 - n$ for all integers $n$.  Suppose that $7 \mid n$, then we are done.  Now if we assume that $7 \nmid n$, then gcd($7, n$) = 1.  Then we can use the previous part of this problem to conclude that $7 \mid n^6 -1$ and so $7 \mid n^7 -n$.   
}
\ppart{8} Prove that $\frac{n^5}{5} + \frac{n^3}{3} + \frac{7n}{15}$ is an integer $\forall n \in \mathbb{Z}$.
\solution{
We first take a common denominator.  
\begin{equation*}
\begin{split}
\frac{n^5}{5} + \frac{n^3}{3} + \frac{7n}{15} &= \frac{3n^5 + 5n^3 + 7n}{15} \\
&= \frac{n(3n^4 + 5n^2 + 7)}{15} \\
\end{split}
\end{equation*}
Now all we need to show is that $15 \mid n(3n^4 + 5n^2 + 7)$ for all integers $n$.  We first show that $3 \mid n(3n^4 + 5n^2 + 7)$.  If $3 \mid n$, then we are done.  Otherwise, gcd($n, 3$) = 1.  In this case, by Fermat's little theorem we have that $n^2 \equiv 1$ (mod $3$).  Thus we have that
\begin{equation*}
3n^4 + 5n^2 + 7 \equiv 2 + 1 \equiv 0 ~~ (\text{mod}~~ 3) 
\end{equation*}
Thus we have that $3 \mid n(3n^4 + 5n^2 + 7)$ for all integers $n$.

Now we show that $5 \mid n(3n^4 + 5n^2 + 7)$ for all integers $n$. 
Again if $5 \mid n$, then we are done.  Otherwise, gcd($n, 5$) = 1.  In this case, by Fermat's little theorem we have that $n^4 \equiv 1$ (mod $5$).  Thus we have that
\begin{equation*}
3n^4 + 5n^2 + 7 \equiv 3 + 2 \equiv 0 ~~ (\text{mod} ~~ 5) 
\end{equation*}
Thus we have that $5 \mid n(3n^4 + 5n^2 + 7)$ for all integers $n$.

Hence we have that $15 \mid n(3n^4 + 5n^2 + 7)$ for all integers $n$.
}
\eparts

\end{problem}


\begin{problem}{20}
  
  Prove that the greatest common divisor of three integers $a$, $b$, and
  $c$ is equal to their smallest positive linear combination; that is,
  the smallest positive value of $sa + tb + uc$, where $s$, $t$, and
  $u$ are integers.

%\hint  Let $m$ be the smallest positive linear combination of $a$, $b$, and
%    $c$.  Prove that $\gcd(a, b, c) \leq m$ and $m \leq \gcd(a, b, c)$.

  \solution{
% This is almost verbatim in the text, but it's a good proof to thoroughly understand, it teaches a useful trick for showing equality, and it will encourage people to read the textbook more. If it's too boring we can extend it to gcd(a_1, ..., a_n)

    Let $m$ be the smallest positive linear combination of $a$, $b$, and
    $c$.  We'll prove that $m = \gcd(a, b, c)$ by showing both $\gcd(a, b,
    c) \leq m$ and $m \leq \gcd(a, b, c)$.

    First, we show that $\gcd(a, b, c) \leq m$.  By the definition of
    common divisor, $\gcd(a, b, c)$ divides $a$, $b$, and $c$.  Therefore,
    for every triple of integers $s$, $t$, and $u$:
    %
    \[
    \gcd(a, b, c) \mid s a + t b + u c
    \]
    %
    Thus, in particular, $\gcd(a, b, c)$ divides $m$, and so $\gcd(a, b,
    c) \leq m$.

    Now we show that $m \leq \gcd(a, b, c)$.  We do this by showing
    that $m \mid a$.  Symmetric arguments show that $m \mid b$
    and $m \mid c$, which means that $m$ is a common divisor of
    $a$, $b$, and $c$.  Thus, $m$ must be less than or equal to the
    \emph{greatest} common divisor of $a$, $b$, and $c$.

    All that remains is to show that $m \mid a$.  By the division
    algorithm, there exists a quotient $q$ and remainder $r$ such 
\begin{align*}
    a & = q \cdot m + r & \text{(where $0 \leq r < m$)}
\end{align*}
    Now $m = s a + t b + u c$ for some integers $s$ and $t$.  Subtituting
    in for $m$ and rearranging terms gives:
    \begin{align*}
    a & = q \cdot (s a + t b + u c) + r \\
    r & = (1 - qs) a + (-qt) b + (-qu) c
    \end{align*}
    %
    We've just expressed $r$ as a linear combination of $a$, $b$, and $c$.
    However, $m$ is the \emph{smallest positive} linear combination and $0
    \leq r < m$.  The only possibility is that the remainder $r$ is not
    positive; that is, $r = 0$.  This implies $m \mid a$.
    
 }
  
\end{problem}


\begin{problem}{20} In this problem, we will investigate numbers which are squares modulo a prime number $p$.  These numbers are referred to quadratic residues of $p$.

\bparts

\ppart{5} An integer $n$ is a quadratic residue of $p$ if there exists another integer $x$ such that $n \equiv x^2  \pmod p$. Prove that $x^2 \equiv y^2 \pmod p$ if and only if $x \equiv y \pmod p$ or $x \equiv -y \pmod p$.
\hint{This is similar to problem 3c}

\solution{$x^2 \equiv y^2 \pmod p$ iff
$p \mid x^2-y^2$.  But $x^2-y^2 = (x-y)(x+y)$, and
since $p$ is a prime, this happens iff either $p \mid x-y$ or $p
\mid x+y$,  which is iff $x \equiv y \pmod p$ or $x \equiv -y \pmod p$.
}

\ppart{5} The following is a simple test we can perform to see if a number $n \not\equiv 0 \pmod p$ is a quadratic residue of $p$ for odd primes $p$. 
\begin{theorem}[Euler's Criterion]
:
\begin{enumerate}
\item $n$ is a quadratic residue of $p$ if and only if $n^{\frac{p-1}{2}} \equiv 1 \pmod p$.
\item $n$ is quadratic non-residue $p$ if and only if $n^{\frac{p-1}{2}} \equiv -1 \pmod p$.
\end{enumerate} 
\end{theorem}

This can be proved completely using Wilson's theorem and part a of this problem.  However for this part prove the following:
If $n$ is a quadratic residue of $p$, then $n^{\frac{p-1}{2}} \equiv 1 \pmod p$.

\solution{If $n$ is a quadratic residue $p$, then there exists an $a$ such that $a^2 \equiv n \pmod p$. Consequently,
$$n^{\frac{p-1}{2}} \equiv a^{p-1} \equiv 1 \pmod p$$
by Fermat's theorem.}

\ppart{10}
Assume that $p \equiv 3 \pmod 4$ and $n \equiv x^2 \pmod p$. Find one possible value for $x$, expressed as a function of $n$ and $p$. 
\hint{Write $p$ as $p=4k+3$ and use Euler's Criterion. You might have to multiply two sides of an equation by $n$ at one point.}

\solution{From Euler's Criterion:
$$n^{\frac{p-1}{2}}\equiv 1 \pmod p.$$
We can write $p=4k+3$, so $\frac{p-1}{2} = \frac{4k+3-1}{2}=k+1$.
As a result,
$n^{2k+1} \equiv 1 \pmod p$,
so
$n^{2k+2} \equiv n \pmod p$.
This can be rewritten as $\big(n^{k+1}\big)^2 \equiv n \pmod p$, so
$$n^{k+1} = n^{\frac{p-3}{4}+1}$$
is one possible value of $x$.
}

\eparts

\end{problem}




\end{document}
