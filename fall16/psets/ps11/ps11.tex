\documentclass[12pt,twoside]{article}   
\usepackage{light}

\newcommand{\card}[1]{\left|#1\right|}
\newcommand{\union}{\cup}
\newcommand{\lgunion}{\bigcup}
\newcommand{\intersect}{\cap}
\newcommand{\lgintersect}{\bigcap}
\newcommand{\cross}{\times}
\newcommand{\naturals}{\mathbb{N}}
\newcommand{\mfigure}[3]{\bigskip\centerline{\resizebox{#1}{#2}{\includegraphics{#3}}}\bigskip}
\newcommand{\eqdef}{\mathbin{::=}}
\newcommand{\heads}{H}
\newcommand{\tails}{T}

\hidesolutions
\showsolutions

\newlength{\strutheight}
\newcommand{\prob}[1]{\mathop{\textup{Pr}} \nolimits \left( #1 \right)}
\newcommand{\prsub}[2]{\mathop{\textup{Pr}_{#1}}\nolimits\left(#2\right)}
\newcommand{\prcond}[2]{%
  \ifinner \settoheight{\strutheight}{$#1 #2$}
  \else    \settoheight{\strutheight}{$\displaystyle#1 #2$} \fi%
  \mathop{\textup{Pr}}\nolimits\left(
    #1\,\left|\protect\rule{0cm}{\strutheight}\right.\,#2
  \right)}
\newcommand{\comment}[1]{}
\newcommand{\cE}{\mathcal{E}}
\renewcommand{\setminus}{-}
\renewcommand{\complement}[1]{\overline{#1}}


\begin{document}
\problemset{11}{November 22, 2016}{Tuesday, November 29, 7:30pm}
\noindent \textbf{Reading Assignment:}   Sections  12.1-12.6, 14.1-14.5
%%%%%%%%%%%%%%%%%%%%%%%%%%%%%%%%%%%%%%%%%%%%%%%%%%%%%%%%%%%%%%%%%%%%%%%%%%%%%
%%% Fall07 ps10 problem 1

\begin{problem}{20}
You are organizing a neighborhood census and instruct your census takers
to knock on doors and note the sex of any child that answers the knock.
Assume that there are two children in a household, that children are 
equally likely to be girls and boys, and that girls and boys are equally 
likely to open the door.

A sample space for this experiment has outcomes that are triples whose
first element is either \texttt{B} or \texttt{G} for the sex of the elder
child, likewise for the second element and the sex of the younger child,
and whose third coordinate is \texttt{E} or \texttt{Y} indicating whether
the \emph{e}lder child or \emph{y}ounger child opened the door.  For
example, $(\mathtt{B},\mathtt{G},\mathtt{Y})$ is the outcome that the elder
child is a boy, the younger child is a girl, and the girl opened the door.

\bparts

\ppart{5} Let \emph{T} be the event that the household has two girls,
and \emph{O} be the event that a girl opened the door.  List the outcomes
in \emph{T} and \emph{O}.

\solution{$T=\set{GGE,GGY}, O=\set{GGE,GGY,GBE,BGY}$}

\ppart{5} What is the probability $\prcond{T}{O}$, that both children are
girls, given that a girl opened the door?
\solution{1/2}

\ppart{10} Where is the mistake in the following argument for computing $\prcond{T}{O}$?

\begin{quote}
If a girl opens the door, then we know that there is at least one girl in
the household.  The probability that there is at least one girl is
\[
1 - \prob{\text{both children are boys}} = 1 - (1/2 \times 1/2) = 3/4.
\]
So,
\begin{eqnarray*}
\lefteqn{\prcond{T}{\text{there is at least one girl in the household}}}\\
& = & \frac{\prob{T \intersect \text{there is at least one girl in the household}}}
{\pr{\text{there is at least one girl in the household}}}\\
& = & \frac{\prob{T}}{\pr{\text{there is at least one girl in the household}}}\\
& = & (1/4) / (3/4) = 1/3.
\end{eqnarray*}
Therefore, given that a girl opened the door, the probability that there
are two girls in the household is \textup{1/3}.
\end{quote}

\solution{The argument is a correct proof that 
\[
\prcond{T}{\text{there is at least one girl in the household}} = 1/3.
\]
The problem is that the event, $H$, that the household has at least one girl,
namely,
\[
H = \set{\mathtt{GGE,GGY,GBE,GBY,BGE,BGY}},
\]
is not equal to the event, \emph{O}, that a girl opens the door.  These
two events differ:
\[
H-O = \set{\mathtt{BGE,GBY}},
\]
and their probabilities are different.  So the fallacy is in the final
conclusion where the value of $\prcond{T}{H}$ is taken to be the same as
the value $\prcond{T}{O}$.  Actually, $\prcond{T}{O} = 1/2$.  }

\eparts
\end{problem}

%%%%%%%%%%%%%%%%%%%%%%%%%%%%%%%%%%%%%%%%%%%%%%%%%%%%%%%%%%%%%%%%%%%%%%%%


%\begin{problem}{15}
%In lecture we discussed the Birthday Paradox. Namely, we found that in a group of $m$ people with $N$ possible birthdays, if $m \ll N$, then:
%\[
%\pr{\text{all $m$ birthdays are different}} \sim e^{-\frac{m(m-1)}{2N}}
%\]
%To find the number of people, $m$, necessary for a half chance of a match, we set the probability to $1/2$ to get:
%\[
%m \sim \sqrt{(2\ln2)N} \approx 1.18\sqrt{N}
%\]
%
%For $N = 365$ days we found $m$ to be 23.
%
%We could also run a different experiment. As we put on the board the birthdays of the people surveyed, we could ask the class if anyone has the same birthday. In this case, before we reached a match amongst the surveyed people, we would already have found other people in the rest of the class who have the same birthday as someone already surveyed. Let's investigate why this is.
%
%\bparts
%\ppart{5} Consider a group of $m$ people with $N$ possible birthdays amongst a larger class of $k$ people, such that $m \leq k$. Define $\pr{A}$ to be the probability that $m$ people all have different birthdays \textit{and} none of the other $k-m$ people have the same birthday as one of the $m$.
%
%Show that, if $m \ll N$, then $\pr{A} \sim e^{\frac{m(m-2k)}{2N}}$. (Notice that the probability of no match is $e^{-\frac{m^2}{2N}}$ when $k$ is $m$, and it gets smaller as $k$ gets larger.)
%
%\hspace{0.5in} \textit{Hints:} For $m \ll N$: $\frac{N!}{(N-m)!N^m} \sim e^{-\frac{m^2}{2N}}$, and $(1-\frac{m}{N}) \sim e^{-\frac{m}{N}}$.
%
%\solution{
%We know:
%\[
%\pr{A} = \frac{N(N-1)\ldots(N-m+1)\cdot(N-m)^{k-m}}{N^k}
%\]
%
%since there are $N$ choices for the first birthday, $N-1$ choices for the second birthday, etc., for the first $m$ birthdays, and $N-m$ choices for each of the remaining $k-m$ birthdays. There are total $N^k$ possible combinations of birthdays within the class.
%
%\begin{align*}
%\pr{A} &= \frac{N(N-1)\ldots(N-m+1)\cdot(N-m)^{k-m}}{N^k} \\
%&= \frac{N!}{(N-m)!}\left(\frac{(N-m)^{k-m}}{N^k}\right) \\
%&= \frac{N!}{(N-m)!N^m}\left(\frac{N-m}{N}\right)^{k-m} \\
%&= \frac{N!}{(N-m)!N^m}\left(1-\frac{m}{N}\right)^{k-m} \\
%&\sim e^{-\frac{m^2}{2N}} \cdot e^{-\frac{m}{N}(k-m)} & \text{(by the Hint)} \\
%& = e^{\frac{m(m-2k)}{2N}}
%\end{align*}
%}
%
%\ppart{10} Find the approximate number of people in the group, $m$, necessary for a half chance of a match (your answer will be in the form of a quadratic). Then simplify your answer to show that, as $k$ gets large  (such that $\sqrt{N} \ll k$), then $m \sim \frac{N\ln2}{k}$.
%
%\hspace{0.5in} \textit{Hint:} For $x \ll 1$: $\sqrt{1-x} \sim (1-\frac{x}{2})$.
%
%\solution{
%Setting $\pr{A} = 1/2$, we get a solution for $m$:
%
%\begin{align*}
%1/2 &= e^{\frac{m(m-2k)}{2N}} \\
%-2N\ln2 &= m^2 -2km  \\
%0 &= m^2-2km + (2N\ln2) \\
%m &= \frac{2k \pm \sqrt{(2k)^2 - 4(2N\ln2)}}{2}
%\end{align*}
%
%Simplifying the solution under the assumption of large $k$, we find:
%\begin{align*}
%m &= \frac{2k - \sqrt{4k^2 - 8N\ln2}}{2} & \text{(taking the lower positive root)} \\
%&= k - k\sqrt{1 - \frac{2N\ln2}{k^2}} \\
%&\sim k  - k \left(1-\frac{2N\ln2}{2k^2}\right) & \text{(by the Hint)} \\
%&= \frac{N\ln2}{k}  
%\end{align*}
%}
%
%\eparts
%
%\end{problem}

%%%%%%%%%%%%%%%%%%%%%%%%%%%%%%%%%%%%%%%%%%%%%%%%%%%%%%%%%%%%%%%%%%%%%%%%%%%
%\newpage

%%%%%%%%%%%%%%%%%%%%%%%%%%%%%%%%%%%%%%%%%%%%%%%%%%%%%%%%%%%%%%%%%%%%%%%%55


\begin{problem}{20}

\bparts

\ppart{7} Suppose you repeatedly flip a fair coin until you see the sequence
\texttt{HHT} or the sequence \texttt{TTH}.  What is the probability you
will see \texttt{HHT} first? 

\textit{Hint:}
Use a bijection argument.

\solution{
In this case the answer is $1/2$.  The proof is by a bijection
argument on the sample space.  Let $A$ denote the event that you see
\texttt{HHT} before \texttt{TTH}, and $B$ denote the event that you see
\texttt{TTH} before \texttt{HHT}.

We will define a bijection, $g$, between $A$ and $B$ so that the
probability of $g(w)$ is equal to the probability of $w$.  The bijection
is quite simple.  Given a sample point $w \in A$, define $g(w) = \bar{w}$,
where $\bar{w}$ is the outcome where every \texttt{H} is replaced by a
\texttt{T} and vice versa.  For example $g(\mathtt{HHT}) =
\overline{\mathtt{HHT}} = \mathtt{TTH}$.

To show that $g$ is a bijection, we first observe that $g:A \rightarrow
B$.  This follows from the fact that \texttt{HHT} precedes \texttt{TTH} in
$w$ iff $\overline{\mathtt{HHT}} = \mathtt{TTH}$ precedes
$\overline{\mathtt{TTH}} = \mathtt{HHT}$ in $\bar{w}$.  And $g$ is onto by
the same reasoning.  Since $g$ is clearly an injection, we can conclude
that it is a bijection.

Then we observe that $\prob{w} = \prob{g(w)}$ for any $w$.  This is because
$\prob{\heads} = \prob{\tails}$ and $g(w)$ has the same length as
$w$.  Hence,
\[
\prob{A} = \sum_{w \in A} \prob{w} = \sum_{w\in A} \prob{g(w)}
= \sum_{w' \in B} \prob{w'}  =  \prob{B}.
\]
The second equality is valid because $g$ preserves the probability, and
the third by the bijection property with $w' = g(w)$.  Note that the fact
that \texttt{H} and \texttt{T} are equally likely is critical in these
calculations; this analysis would fail for a biased coin.

Finally we have to show that $\prob{A \union B} = 1$.  This follows from the
fact that the only way never to throw either pattern is to throw all
\texttt{H}'s or all \texttt{T}'s after the first toss, and we know that
the probability of there being an unbounded number of tosses of only H or
only T is zero.  That is, $\prob{\overline{A \union B}} = 0$ and so $\prob{A
\union B} = 1$.  Since $A$ and $B$ are disjoint, this means that $\prob{A} +
\prob{B} = 1$ and hence
\[
\prob{A} = \frac{1}{2}.
\]
}

\ppart{7} What is the probability you see the sequence \texttt{HTT} before
you see the sequence \texttt{HHT}?

\textit{Hint:}
Try to find the probability that \texttt{HHT} comes before
\texttt{HTT} conditioning on whether you first toss an \texttt{H} or a
\texttt{T}. Somewhat surprisingly, the answer is not $1/2$.

\solution{
Let $A$ be the event that \texttt{HTT} appears before \texttt{HHT}, and let
$p \eqdef \prob{A}$.

Suppose our first toss is \texttt{T}.  Since neither of our patterns
starts with \texttt{T}, the probability that $A$ will occur from this
point on is still $p$.  That is, $\prcond{A}{\tails} = p$.

Suppose our first toss is \texttt{H}.  To find the probability that $A$
will now occur, that is, to find $q \eqdef \prcond{A}{\heads}$, we
consider different cases based on the subsequent throws.

Suppose the next toss is \texttt{H}, that is, the first two tosses are
\texttt{HH}.  Then neither pattern appears if we continue flipping
\texttt{H}, and when we eventually toss a \texttt{T}, the pattern
\texttt{HHT} will then have appeared first.  So in this case, event $A$
will never occur.  That is $\prcond{A}{\mathtt{HH}} = 0$.

Suppose the first two tosses are \texttt{HT}.  If we toss a \texttt{T}
again, then we have tossed \texttt{HTT}, so event $A$ has occurred.
If we next toss an \texttt{H}, then we have tossed \texttt{HTH}.  But this
puts us in the same situation we were in after rolling an \texttt{H} on
the first toss.  That is, $\prcond{A}{\mathtt{HTH}} = q$.

Summarizing this we have:
\begin{align*}
\pr{A} &=
\prcond{A}{\tails}\pr{\tails}+\prcond{A}{\heads}\pr{\heads}
& \text{(Law of Total Probability)}\\
p & = p\frac{1}{2} + q\frac{1}{2} & \text{so}\\
p & = q.%\label{peq}
\end{align*}

Continuing, we have
\begin{align}
\prcond{A}{\heads}
&=
\prcond{A}{\mathtt{HT}}\pr{\tails}+\prcond{A}{\mathtt{HH}}\pr{\heads}
    & \text{(Law of Total Probability)}\notag\\
q &= \prcond{A}{\mathtt{HT}}\frac{1}{2} + 0\cdot\frac{1}{2}\label{here}\\
\prcond{A}{\mathtt{HT}} & =
\prcond{A}{\mathtt{HTT}}\pr{\tails}+\prcond{A}{\mathtt{HTH}}\pr{\heads}
& \text{(Law of Total Probability)}\notag\\
\prcond{A}{\mathtt{HT}} & = 1\cdot\frac{1}{2} + q\frac{1}{2} \label{there}\\
q & = (\frac{1}{2} + \frac{q}{2})\frac{1}{2} & \text{by~\eqref{here} \&~\eqref{there}}\notag\\
q & = \frac{1}{3}.\notag
\end{align}

So \texttt{HTT} comes before \texttt{HHT} with probability
\[
p=q =\cfrac{1}{3}.
\]

These kind of events are have an amazing \emph{intransitivity}
property: if you pick \emph{any} pattern of three tosses such as
\texttt{HTT}, then I can pick a pattern of three tosses such as
\texttt{HHT}.  If we then bet on which pattern will appear first in a
series of tosses, the odds will be in my favor.  In particular, even
if you instead picked the ``better'' pattern \texttt{HHT}, there is
another pattern I can pick that has a more than even chance of
appearing before \texttt{HHT}.  Watch out for this intransitivity
phenomenon if somebody proposes that you bet real money on coin flips.
}

\ppart{6}
Suppose you flip three fair coins.
Define the following events:
\begin{itemize}
\item Let $A$ be the event that \emph{the first} coin is heads.
\item Let $B$ be the event that \emph{the second} coin is heads.
\item Let $C$ be the event that \emph{the third} coin is heads.
\item Let $D$ be the event that \emph{an even number of} coins are heads.
\end{itemize}

Use the four step method to determine the probability of each of $A,B,C,D$.

\solution{
The tree is a binary tree with depth 3 and 8 leaves.  The successive
levels branch to show whether or not the successive events $A,B,C$
occur.  By the definitions of the characteristics \emph{fair} and
\emph{independent}, each branch from a vertex is equally likely to be
followed.  So the probability space has, as outcomes, eight length-3
strings of $\heads$'s and $\tails$'s, each of which has probability
$(1/2)^3 = 1/8$.

  Each of the events $A,B,C,D$ are true in four of the outcomes
  and hence has probability 1/2.
}
\eparts
\end{problem}
%%%%%%%%%%%%%%%%%%%%%%%%%%%%%%%%%%%%
%\begin{problem}{20}
%
%  Professor Moitra has a deck of $52$ randomly shuffled playing cards, $26$ red, $26$ black.
%  He proposes the following game: he will continually draw a card off the top of the deck, turn
%  it face up so that you can see it and then put it aside. At any point while there
%  are still cards left in the deck, you may say ``stop" and he will flip over one last card.  If that next card turns
%  up black you win and otherwise you lose. Either way, the game ends.
%
%\bparts
%
%\ppart{4} Show that if you say ``stop" before you have seen any
%cards, you then have probability $1/2$ of winning the game.
%
%\solution{
%If we just record the sequence of black and red cards that
%  will be drawn, there are $\binom{51}{25}$ sequences with first card
%  black: $25$ positions for the black cards chosen from the $51$
%  remaining positions.  Since there are $\binom{52}{26}$ sequences in
%  all, the probability of winning on the first draw is
%  $\binom{51}{25}/\binom{52}{26} = 26/52 = 1/2$.
%}
%
%\ppart{4} Suppose you don't say ``stop" before the first card is flipped and it turns up red.
%Show that you then have a probability of winning the game that is
%greater than $1/2$.
%
%\solution{
%Suppose you take the next card after that.  There are
%  $\binom{50}{25}$ sequences that start with a red card and then a
%  black and there are $\binom{51}{26}$ sequences that start with a red
%  card. So then there is a $\binom{50}{25}/\binom{50}{26} = 26/51 >
%  1/2$ chance of winning.  Any optimum strategy would have to
%  guarantee a probability of winning as least as big as that.
%}
%
%\ppart{4} If there are $r$ red cards left in the deck and $b$ black
%cards, show that the probability of winning if you say ``stop" before the next card is flipped
%is $b/(r+b)$.
%
%\solution{
%The probability is $\binom{b+r-1}{b-1}/\binom{b+r}{b} =
%b/(r+b)$.}
%
%\ppart{8} Either,
%\begin{enumerate}
%\item come up with a strategy for this game that gives you a
%  probability of winning strictly greater than $1/2$ and prove that
%  the strategy works, or,
%\item come up with a proof that no such strategy can exist.
%\end{enumerate}
%
%\solution{
%There is no such strategy.  Let $S_{b,r}$ be a strategy that achieves
%the best probability of winning when starting with $b$ black cards and
%$r$ red cards.  The claim is that $\pr{\mbox{win by playing
%    $S_{b,r}$}} = b/(r+b)$ for all $b,r$ with at least $b>0$ or $r>0$.
%
%Clearly $\pr{\mbox{win by playing $S_{1,0}$}} = 1$ and $\pr{\mbox{win
%    by playing $S_{0,1}$}} = 0$.  We prove the rest of the claim by
%induction on $r+b$.  If the strategy $S_{b,r}$ is to take the next
%card, then $\pr{\mbox{win by playing $S_{b,r}$}} = b/(r+b)$ as
%claimed.  Suppose then that the strategy $S_{r,b}$ is to not take the
%first card, but to keep playing.  Then by the law of total
%probability,
%\begin{align*}
%\pr{\mbox{win by playing $S_{b,r}$}}
%   & = \prcond{\mbox{win by playing $S_{b,r}$}}{\mbox{first card is black}} \pr{\mbox{first
%      card is black}} +\\
%   &\qquad \prcond{\mbox{win by playing
%    $S_{b,r}$}}{\mbox{first card is red}} \pr{\mbox{first card is
%    red}}\\
%   & = \pr{\mbox{win by playing $S_{b-1,r}$}} (b/(r+b)) + \pr{\mbox{win
%    by playing $S_{b,r-1}$}}(r/(b+r)).
%\end{align*}
%By induction, this is
%\[
%\pr{\mbox{win by playing $S_{b,r}$}} = ((b-1)/(b-1+r))(b/(r+b)) +
%  (b/(b+r-1))(r/(b+r)) = b/(b+r),
%\]
%as claimed
%
%Why is
%\[
%\prcond{\mbox{win by playing $S_{b,r}$}}{\mbox{first card is
%    black}} = \pr{\mbox{win by playing $S_{b-1,r}$}}?
%\]
%\dots because if you have decided to see at least one more card, and
%that card is black, this means you are starting the game over again
%with $S_{b-1,r}$.
%}
%
%\eparts
%
%\end{problem}
%%%%%%%%%%%%%%%%%%%%%%%%%%%%%%%%%%%%%%%%%%%%%%%%%%%
\begin{problem}{20}
Suppose you have seven standard dice with faces numbered 1 to 6. Each die has a label corresponding to a letter of the alphabet (A through G). A
\emph{roll} is a sequence specifying a value for each die in alphabet
order.  For example, one roll is $(6,1,4,1,3,5,2)$
indicating that die A showed a 6, die B showed 1, die C showed 4, \dots.
%For the problems below, describe a bijection between the specified set
%of rolls and another set that is easily counted using the Product,
%Generalized Product, and similar rules.  Then write a simple
%arithmetic formula, possibly involving factorials and
%binomial coefficients, for the size of the set of rolls.  You do
%not need to prove that the correspondence between sets you describe is
%a bijection, and you do not need to simplify the expression you come
%up with.
%
%For example, let $A$ be the set of rolls where 4 dice come up showing
%the same number, and the other 3 dice also come up the same, but with
%a different number.  Let $R$ be the set of seven rainbow colors and
%$S \eqdef [1,6]$ be the set of dice values.
%
%Define $B \eqdef P_{S,2} \cross R_3$, where $P_{S,2}$ is the set of 
%2-permutations of $S$ and $R_3$ is the set of size-3 subsets of $R$.  
%Then define a bijection from $A$ to $B$ by
%mapping a roll in $A$ to the sequence in $B$ whose first element is
%a pair consisting of the number that came up 
%three times followed by the number that came up four times, and whose 
%second element is the set of colors of the three matching dice.
%
%For example, the roll
%\[
%(4,4,2,2,4,2,4) \in A
%\]
%maps to
%\[
%((2,4),\set{\text{yellow,green,indigo}}) \in B.
%\]
%
%Now by the Bijection rule $\card{A} = \card{B}$, and by the
%Generalized Product and Subset rules,
%\[
%\card{B} = 6 \cdot 5 \cdot \binom{7}{3}.
%\]

\bparts

\ppart{5} \label{66} What is the probability of a roll where \emph{exactly} two dice have
the value 3 and the remaining five dice all have different values?

Example: $(3, 2, 3, 1, 6, 4, 5)$ is a roll of this type, but $(1, 1, 2, 6,
3, 4, 5)$ and $(3, 3, 1, 2, 4, 6, 4)$ are not.

\solution{
As in the example, map a roll into an element of $B \eqdef R_2
\cross P_5$ where $P_5$ is the set of permutations of $\set{1,\dots,5}$.  A roll
maps to the pair whose first element is the set of colors of the two dice
with value 6, and whose second element is the sequence of values of the
remaining dice (in rainbow order).  So $(3, 2, 3, 1, 6, 4, 5)$ above maps
to $(\set{\text{A,C}}, (2,1,6,4,5))$.  By the Product rule,
\[
\card{B} = \binom{7}{2}\cdot 5!.
\]

The probability is
\[
\frac{\binom{7}{2}\cdot 5!}{6^7}
\]
}


\ppart{5} What is the probability of a roll where two dice have an even value and the
remaining five dice all have different values?

Example: $(4, 2, 4, 1, 3, 6, 5)$ is a roll of this type, but $(1, 1, 2, 6,
1, 4, 5)$ and $(6, 6, 1, 2, 4, 3, 4)$ are not.  

\solution{
Map a roll into a triple whose first element is in $S$,
indicating the value of the pair of matching dice, whose second element is the
set of colors of the two matching dice, and whose third element is the
sequence of the remaining five dice values (in rainbow order).

So $(4, 2, 4, 1, 3, 6, 5)$ above maps to $(4, \set{\text{A,C}},
(2,1,3,6,5))$.  Notice that the number of choices for the third element of
a triple is the number of permutations of the remaining five values,
namely $5!$.  This mapping is a bijection, so the number of such rolls
equals the number of such triples.
By the Generalized Product rule, the number of such triples is
\[
3 \cdot \binom{7}{2} \cdot 5!.
\]

The probability is
\[
\frac{3 \cdot \binom{7}{2} \cdot 5!}{6^7}
\]

Alternatively, we can define a map from rolls in this part to the
rolls in part~\eqref{66}, by replacing the value of the duplicated values
with 6's and replacing any 6 in the remaining values by the value of the
duplicated pair.  So the roll $(4, 2, 4, 1, 3, 6, 5)$ would map to the
roll $(6, 2, 6, 1, 3, 4, 5)$.  Now a type~\ref{66} roll, $r$, is mapped to
by exactly the rolls obtainable from $r$ by exchanging occurrences of 6's
and $i$'s, for $i = 1,\dots,6$.  So this map is 6-to-1, and by the
Division rule, the number of rolls here is 6 times the number of rolls in
part~\eqref{66}.

\iffalse

An useful variation of this idea is to map a roll into a triple in the set
$B \eqdef S cross R_2 \cross P_5$.  Again, the first element of a triple
in $B$ is the value of the matching dice, the second element is the set of
colors of the two matching dice, and the third element is the sequence of
\emph{rankings}\footnote{Given a set of $n$ numbers, the \emph{rank} of a
number in the set is its position when the numbers are listed in
increasing order.  So the rank of the smallest number is 1, the rank of
the second smallest is 2, \dots, and the rank of the largest is $n$.} of
the remaining five dice values (in rainbow order).

So now $(4, 2, 4, 1, 3, 6, 5)$ above maps to $(4, \set{\text{A,C}},
(2,1,3,5,4))$.  By the Product rule, $\card{B} = \binom{7}{2}\cdot 5!$.

The use of ranking let us define $B$ without having the possible values of
third elements in triples depend on the first elements.  This let us give
a simpler definition of the set, $B$, of triples corresponding to rolls.
With the simpler definition of $B$, we could calculate its size using the
Product rule instead of the Generalized Product rule.
\fi
}

\ppart{10} What is the probability of a roll where two dice have one value, two different
dice have a second value, and the remaining three dice a third value?

Example: $(6, 1, 2, 1, 2, 6, 6)$ is a roll of this type, but $(4, 4, 4, 4,
1,3,5)$ and $(5, 5, 5, 6, 6,1,2)$ are not.

\solution{
Map a roll of this kind into a 4-tuple whose first element is
the set of two numbers of the two pairs of matching dice, whose second
element is the set of two colors of the pair of matching dice with the
smaller number, whose third element is the set of two letters of the larger
of the matching pairs, and whose fourth element is the value of the
remaining three dice.  For example, the roll $(6, 1, 2, 1, 2, 6, 6)$ maps
to the triple
\[
(\set{1,2},\set{\text{B,D}},\set{\text{C,F}}, 6).
\]

There are $\binom{6}{2}$ possible first elements of a triple,
$\binom{7}{2}$ second elements, $\binom{5}{2}$ third elements since the
second set of two colors must be different from the first two, and 4 ways
to choose the value of the three dice since their value must differ from
the values of the two pairs.  So by the Generalized Product rule, there
are
\[
\binom{6}{2} \cdot \binom{7}{2} \cdot \binom{5}{2} \cdot 4
\]
possible rolls of this kind.

The probability is \[
\frac{\binom{6}{2} \cdot \binom{7}{2} \cdot \binom{5}{2} \cdot 4}{6^7}
\]
}

\eparts
\end{problem}
%%%%%%%%%%%%%%%%%%%%%%%%%%%%%%%%%%%%%%%%%%%%%%%%%%%
\end{document}
