\documentclass[problem]{mcs}

\begin{pcomments}
  \pcomment{PS_AkraB_multipart}
  \pcomment{formatted, but unedited from fall97 pset6}
\end{pcomments}

\pkeywords{
  recurrence
  Akra-Bazzi
}

%%%%%%%%%%%%%%%%%%%%%%%%%%%%%%%%%%%%%%%%%%%%%%%%%%%%%%%%%%%%%%%%%%%%%
% Problem starts here
%%%%%%%%%%%%%%%%%%%%%%%%%%%%%%%%%%%%%%%%%%%%%%%%%%%%%%%%%%%%%%%%%%%%%

\begin{problem}
Use the Akra-Bazzi formula to find $\Theta()$ asymptotic bounds
for the following divide-and-conquer recurrences.  For each
recurrence, $T(1)=1$ and $T(n)=\Theta(1)$ for all constant $n$.  State
the value of $p$ you get for each recurrence (which can be left in the
form of logs). Also, state the values of the $a_i, b_i,~ \mbox{and}~
h_i(n)$ for each recurrence.

\begin{enumerate}
\item $T(n) = 3T\left(\floor{n/3}\right) + n$.
\item $T(n) = 4T\left(\floor{n/3}\right) + n^2$.
\item $T(n) = 3T\left(\floor{n/4}\right) + n$.
\item $T(n)=T(\floor{n/4})+T(\floor{n/3})+n$.
\item $T(n)=T(\ceil{n/4})+T(\floor{3n/4})+n$.
\item $T(n) = 2T(\floor{n/4})+\sqrt{n}$.
\item $T(n) = 2T(\floor{n/4}+1)+\sqrt{n}$.
\item $T(n) = 2T(\floor{n/4+\sqrt{n}}) + 1$.
\item $T(n) = 3T(\ceil{n^{1/3}})+\log_3{n}$.  (For this problem, $T(2) = 1$.)
\item $T(n) = \sqrt{e}T(\floor{n^{1/e}})+\ln{n}$.    
\end{enumerate}

\begin{solution}
\begin{enumerate}

\item
$a_1 = 3$, $b_1 = 1/3$, $h_1(n) = \floor{n/3} - n/3$, $g(n) =
  n$, $p = 1$,
%
\[
T(n) = \Theta(n(1+ \int_1^n \frac{u}{u^2} du)) = \Theta(n \log n).
\]

\item
$a_1 = 4$, $b_1 = 1/3$, $h_1(n) = \floor{n/3} - n/3$, $g(n) =
  n^2$, $p = \log_3 4$,
\begin{align*}
T(n) & = \Theta(n^{\log_3 4}(1+ \int_1^n \frac{u^2}{u^{\log_3 4 + 1}} du))\\
     & = \Theta(n^{\log_3 4}(1+ \int_1^n u^{1 - \log_3 4} du))\\
     & = \Theta(n^2).
\end{align*}

\item
$a_1 = 3$, $b_1 = 1/4$, $h_1(n) = \floor{n/4} - n/4$, $g(n) =
  n$, $p = \frac{1}{2} \log_2 3$,
%
\[
T(n) = \Theta(n^{(\log_2 3)/2}(1+ \int_1^n \frac{u}{u^{(\log_2 3)/2
      + 1}} du)) = \Theta(n^{(\log_2 3)/2}(1+ \int_1^n u^{- (\log_2
    3)/2} du)) = \Theta(n).
\]

\item
$a_1 = 1$, $a_2 = 1$, $b_1 = 1/4$, $b_2 = 1/3$, $h_1(n) = \floor{n/4}
  - n/4$, $h_2(n) = \floor{n/3} - n/3$, $g(n) = n$, $p =
  0.5605$ (the exact value does not matter, it turns out that for this
  case, we only need to know that $p < 1$),
%
\[
T(n) = \Theta(n^p(1+ \int_1^n \frac{u}{u^{p+1}} du)) =
\Theta(n^p(1+ \int_1^n u^{-p} du)) = \Theta(n).
\]

\item
$a_1 = 1$, $a_2 = 1$, $b_1 = 1/4$, $b_2 = 3/4$, $h_1(n) = \lceil n/4
\rceil - n/4$, $h_2(n) = \lfloor 3n/4 \rfloor - 3n/4$, $g(n) = n$, $p
= 1$,
%
$$T(n) = \Theta(n(1+ \int_1^n \frac{u}{u^2} du)) =
\Theta(n \log n).$$

\item
$a_1 = 2$, $b_1 = 1/4$, $h_1(n) = \floor{n/4}-n/4$, 
$g(n) = \sqrt{n}$, $p = 1/2$,
%
$$T(n) = \Theta(n^{1/2}(1+ \int_1^n \frac{\sqrt{u}}{u^{3/2}} du)) =
\Theta(n^{1/2} \log n).$$

\item
$a_1 = 2$, $b_1 = 1/4$, $h_1(n) = \floor{n/4}-n/4 + 1$,
 $g(n) = \sqrt{n}$, $p = 1/2$,
%
$$T(n) = \Theta(n^{1/2}(1+ \int_1^n \frac{\sqrt{u}}{u^{3/2}} du)) =
\Theta(n^{1/2} \log n).$$

\item
$a_1 = 2$, $b_1 = 1/4$, $h_1(n) = \floor{n/4+ \sqrt{n}} - n/4$, 
$g(n) = 1$, $p = 1/2$,
%
$$T(n) = \Theta(n^{1/2}(1+ \int_1^n \frac{1}{u^{3/2}} du)) =
\Theta(n^{1/2}).$$

\item
%{\bf NOTE: For this part the initial condition should have been T(2) = 1}
We can turn this recurrence into a divide and conquer recurrence with a
change of variables. First we show that $T(n)$ is non-decreasing.

\begin{claim}
For all $n \geq 3$, $T(n) \geq T(n-1)$.
\end{claim}

\begin{proof}

By strong induction on $n$.

I.H.:  $P(n): T(n) \geq T(n-1)$.

Base:  $n=3$: $T(3) = 3T(\ceil{3^{1/3}}) + 1 = 3T(2) + 1 = 4 > 1 = T(1)$.

Ind. Step: Assume $P(k)$ is true for all $3 \leq k \leq n$. We will prove that
$P(n+1)$ is true.  

\begin{eqnarray*}
T(n+1) & =  &3T(\ceil{{(n+1)}^{1/3}}) + \log_3 (n+1)\\
& \geq &  3T(\ceil{n}) + \log n\\
& = & T(n)
\end{eqnarray*}

So $T(n+1) \geq T(n)$. 

\end{proof}

We will now solve the recurrence assuming $n$ is a power of $3$.
First, make the substitution $n  = 3^m$. The recurrence becomes 
$$T(3^m) = 3 T(\ceil{3^{m/3}}) + m$$.

Now define $S(m) = T(\ceil{3^m})$ for all integers $m \geq 0$. 
Note that since $T$ is non-decreasing, for all $m > 1$,  
$S(m+1) \geq S(m)$,
and hence $S$ is also a non-decreasing function.


Therefore, (remember $m$ is an integer),

\begin{eqnarray*}
S(m) & = & T(\ceil{3^m})\\
& = & T(3^m)\\
& = & 3T(\ceil{3^{m/3}}) + m\\
&\leq & 3T(3^{\ceil{m/3}}) + m\\
& \leq & 3S(\ceil{m/3}) + m
\end{eqnarray*}

That is $S(m) \leq 3S(\ceil{m/3}) + m$ 

We can then view this as $S(m) \leq 3S(m/3 + h(m)) + m$ where
$h(m) = \lceil m/3 \rceil - m/3 $. Clearly $h(m) < 1$ and hence
it is certainly $h(m) = O( m /(\log m)^2)$. 

Now to solve this we can apply the Akra-Bazzi theorem to

$$S(m) = 3S(\frac{m}{3}) + m,$$

for which $a_1 = 3$, $b_1 = 1/3$, $h_1(m) = 0$, $g(m) = m$, $p = 1$,

$$S(m) = \Theta(m(1+ \int_1^m \frac{u}{u^2}du)) = \Theta(m \log m),$$

so that

$$T(n) \leq T(3^{\ceil{\log n}}) = S(\ceil{\log n}) = O(\log n \log \log n).$$


Now we derive a lower bound in a similar fashion.

\begin{eqnarray*}
S(m) & = & T(\ceil{3^m})\\
&=& T(3^m)\\
&=&  3T(\ceil{3^{m/3}}) + m\\
&\geq & 3T(3^{\ceil{m/3}}) + m\\
&\geq & 3S(\ceil{m/3}) + m
\end{eqnarray*}

That is $S(m) \geq 3S(\ceil{m/3}) + m$.

Define a new recurrence $S'(m) = 3S'(\ceil{m/3}) + m$ with
$S'(0) = 1$.

\begin{claim*}
$S(m) \geq S'(m)$
\end{claim*}

\begin{proof}

By strong Induction on $m$.

I.H.: $P(m) \eqdef S(m) \geq S'(m)$.

Base: ($m=1$) $S(0) = S'(0) = 1$.

Ind. Step: Assume $P(k)$ is true for all $k \leq m$.
We will prove $P(m+1)$. 

\begin{eqnarray*}
S(m) &\geq& 3S(\ceil{m/3}) + m\\
&\geq& 3S'(\ceil{m/3}) + m\\
&\geq &S'(m)
\end{eqnarray*}

\end{proof}

Now we can apply Akra-Bazzi to solve the recurrence for
$S'$ in a similar way as for the upper bound to get 
$S'(m) = \Theta(m \log m)$.
From the claim above this implies that $S(m) = \Omega(m \log m)$.

Hence $T(n) =  \Omega (\log n \log \log n)$.
From the upper and lower bounds we conclude that 
$T(n) = \Theta( \log n \log \log n)$.

\item

This problem is similar to that of part 9.
We make the substitution $n = e^m$ and 
set $S(m) = T(e^m)$.  Then the recurrence becomes
\[
S(m) = \sqrt{e} S(\frac{m}{e}) + m,
\]
for which $a_1 = \sqrt{e}$, $b_1 = 1/e$, $h_1(m) = 0$, $g(m) = m$, $p = 1/2$,

We follow the steps in part 9 to conclude that,
\[
S(m) = \Theta(m^{1/2}(1+ \int_1^m \frac{u}{u^{3/2}}du)) = \Theta(m),
\]
so that
\[
T(n) = S(\log n) = \Theta(\log n).
\]

\end{enumerate}

\end{solution}


\end{problem}

%%%%%%%%%%%%%%%%%%%%%%%%%%%%%%%%%%%%%%%%%%%%%%%%%%%%%%%%%%%%%%%%%%%%%
% Problem ends here
%%%%%%%%%%%%%%%%%%%%%%%%%%%%%%%%%%%%%%%%%%%%%%%%%%%%%%%%%%%%%%%%%%%%%
\endinput
