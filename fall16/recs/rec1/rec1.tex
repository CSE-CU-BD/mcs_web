\documentclass[12pt]{article}
\usepackage{light}

% add not exists = forall not

\hidesolutions
\showsolutions

\newcommand{\card}[1]{\left|#1\right|}

\newcommand{\C}{\textit{incabal}}

\begin{document}

\recitation{1}{September 9, 2016}

%\section*{Deductive reasoning}
%
%As covered in lecture, a mathematical proof of a proposition is a chain of logical deductions, or \emph{inference rules} leading to the proposition from a base set of axioms. Chaining together logical deductions allows us to prove new propositions using previously proved one. If we know have proof of a proposition $P$, and we know that the proposition implies another proposition, $Q$, then we know that we have proof of $Q$. (Why is that?). This is known as \emph{modus ponens}. \\
%
%
%\textbf{Reminder:} Propositions may be true or false, and axioms are propositions that are accepted as true. \\
%
%
%\textbf{Example:}  Let a, b and c be integers. Prove that if $a$ divides $b$ and $a$ divides $c$ then $a$ also divides $b + c$
%\proof{
%First we assume “$a|b$ and $a|c$” (i.e. "$a$ divides $b$ and $a$ divides $c$") to be true
%
%Since a divides b, we know that there exists some x such that b can be written as $b = ax$
%
%Same thing for $c$
%
%$b + c$ can be written as $ax \times ay = a(x + y)$
%
%Since $a|b$ and $a|c$ leads to $a|(b + c)$ being true, we can conclude that $a$ divides $b$ and $a$ divides $c$ implies a divides $b + c$
%}
%
%
%Each step in a proof should be clear and “logical”; in particular, you should state what previously proved facts are used to derive each new conclusion.

\break

%%%%%%%%%%%%%%%%%%%%%%%%%%%%%%%%%%%%%%%%%%%%%%%%%%%%%%%%%%%%%%%%%%%%%%%%%%%%%%%
\section{Team Problem: Contrapositive}
Prove by truth table that an implication is equivalent to its contrapositive.
\solution{
\[\begin{array}{c c c c c c c}
x  &  y  &  x $\rightarrow$ y & $\neg$y & $\neg$x & $\neg$y $\rightarrow$ $\neg$x  & (x $\rightarrow$ y) $\longleftrightarrow$ ($\neg$y $\rightarrow$ $\neg$x) \\
\hline
T & T & T & F & F & T & T \\
T & F & F & T & F & F & T \\
F & T & T & F & T & T & T \\
F & F & T & T & T & T & T \\
\end{array}\]

In  every  row,  x $\rightarrow$ y  is  T  precisely  when $\neg$y $\rightarrow$ $\neg$x  is  T.  Thus,  we  conclude  that  an implication is equivalent to its contrapositive.
}

\break
\section{Team Problem:  A Mystery}

A certain cabal within the 6.042 course staff is plotting to make the
final exam \textit{ridiculously hard}.  (``Problem 1.  Prove that the
axioms of mathematics are complete and consistent.  Express your
answer in Mayan hieroglyphics.'')  The only way to stop their evil plan
is to determine exactly who is in the cabal.  The course staff
consists of nine people:
%
\[
%\set{\text{Spyros}, \text{Ling}, \text{Chieu}, \text{Nick},
%\text{Bill}, \text{Jay}, \text{Brooke}, \text{Marten}, \text{Rachel}}
\set{\text{Devin}, \text{Elizabeth}, \text{Emanuele}, \text{Hao}, \text{Henry}, \text{Hyungie}, \text{Michael}, \text{Patrick}, \text{Rachel}}
\]
% Spyros -> Devin
%Ling -> Michael
% Chieu->Elizabeth
%Marten->Patrick
%Nick->Emanuele
%Bill->Henry
%Brooke->Hyungie
%Jay->Hao
%
The cabal is a subset of these nine.  A membership roster has been
found and appears below, but it is deviously encrypted in logic
notation.  The predicate $\C$ indicates who is in the cabal; that is,
$\C(x)$ is true if and only if $x$ is a member.  Translate each
statement below into English and deduce who is in the cabal.

\begin{enumerate}[\upshape (i)]

\item\label{eee} $\exists x \ \exists y \ \exists z \
    (x \neq y \wedge
     x \neq z \wedge
     y \neq z \wedge
     \C(x) \wedge \C(y) \wedge \C(z))$

\solution[\vspace{0.3in}]{A direct English paraphrase would be ``There
exist people we'll call $x,y$, and $z$, who are all different, such that
$x,y$ and $z$ are each in the cabal.''  A better version would use the
fact that there's no need in this case to give names to the people.
Namely, a better paraphrase is, ``There are 3 different people in the
cabal.''  Perhaps a simpler way to say this is, ``The cabal is of size at
least 3.''}

\item\label{nYW} $\neg (\C(\text{Michael}) \wedge \C(\text{Henry}))$

\solution[\vspace{0.3in}]{Michael and Henry are not both in the cabal.
Equivalently: at least one of Michael and Henry is not in the cabal.}

\item\label{Fall} ($\C(\text{Hyungie}) \vee \C(\text{Emanuele})) \rightarrow \forall x \ \C(x)$

\solution[\vspace{0.3in}]{If either Hyungie or Emanuele is in the cabal, then everyone is.}

\item\label{YW} $\C(\text{Michael}) \rightarrow \C(\text{Henry})$

\solution[\vspace{0.3in}]{If Michael is in the cabal, then Henry is also.}

\item\label{CB} $\C(\text{Elizabeth}) \rightarrow \C(\text{Hyungie})$

\solution[\vspace{0.3in}]{If Elizabeth is in the cabal, then Hyungie is also.}

\item\label{SJnT}
$(\C(\text{Devin}) \vee \C(\text{Hao})) \rightarrow \neg \C(\text{Rachel})$

\solution[\vspace{0.3in}]{If either of Devin or Hao is in the cabal,
then Rachel is not.  Equivalently, if Rachel \emph{is} in the cabal, then neither
Devin nor Hao is.}

\item\label{SWnM}
$(\C(\text{Devin}) \vee \C(\text{Henry})) \rightarrow \neg \C(\text{Patrick})$

\solution[\vspace{0.3in}]{If either of Devin or Henry is in the cabal,
then Patrick is not.  Equivalently, if Patrick \emph{is} in the cabal, then
neither Devin nor Henry is.  }
\end{enumerate}

\insolutions{So much for the translations.  We now argue that the only
cabal satisfying all seven propositions above is one whose members are
exactly Devin, Henry, and Hao.

We first observe that by~\eqref{nYW}, there must be someone --- either
Michael or Henry --- who is not in the cabal.  But if either Hyungie
or Emanuele were in the cabal, then by~\eqref{Fall}, everyone would be.
So we conclude by contradiction that

\begin{equation}\label{nF}
\text{Hyungie and Emanuele are not in the cabal.}
\end{equation}

Now consider that~\eqref{CB} implies its contrapositive: if Hyungie is
not in the cabal, then neither is Elizabeth. Therefore, since Hyungie is
not in the cabal,

\begin{equation}
\text{Elizabeth is not in the cabal.}
\end{equation}

Next observe that if Michael were in the cabal, then by~\eqref{YW},
Henry would be too, contradicting~\eqref{nYW}.  So by again
contradiction, we conclude that
\begin{equation}\label{nC}
\text{Michael is not in the cabal.}
\end{equation}

Now suppose Rachel is in the cabal.  Then by~\eqref{SJnT}, Devin and
Hao are not. We already know Hyungie, Emanuele, Elizabeth, and Michael are not
in the cabal, leaving only three who could be --- Rachel, Patrick, and
Henry.  But by~\eqref{eee} the cabal must have at least three
members, so it follows that the cabal must consist of exactly these
three.  This proves:
\begin{lemma}\label{TMW}
\text{If Rachel is in the cabal, then Patrick and Henry are in the cabal.}
\end{lemma}

But by~\eqref{SWnM}, if Henry is the cabal, then Patrick is not.  That is, 
\begin{lemma}\label{WnM}
\text{Henry and Patrick cannot both be in the cabal.}
\end{lemma}
Now from Lemma~\ref{WnM} we conclude that the conclusion of
Lemma~\ref{TMW} is false.  So by contrapositive, the hypothesis of
Lemma~\ref{TMW} must also be false, namely,
\begin{equation}\label{nT}
\text{Rachel is not in the cabal.}
\end{equation}

Finally, suppose Patrick is in the cabal.  Then by~\eqref{SWnM},
Devin and Henry are not, and we already know Hyungie, Emanuele, Elizabeth,
Michael, and Rachel are not. So the cabal must consist of at most two
people (Patrick and Hao). This contradicts~\eqref{eee}, and we conclude
by contradiction that
\begin{equation}\label{nM}
\text{Patrick is not in the cabal.}
\end{equation}
So the only remaining people who could be in the cabal are Devin,
Henry, and Hao.  Since the cabal must have at least three members,
we conclude that
\begin{lemma}
The only possible cabal consists of Devin, Henry, and Hao.
\end{lemma}

But we're not done yet: we haven't shown that a cabal consisting of
Devin, Henry, and Hao actually does satisfy all seven conditions.
So let $\mathcal{A} =\set{\text{Devin}, \text{Henry},
  \text{Hao}}$, and let's quickly check that $\mathcal{A}$
satisfies~\eqref{eee}--\eqref{SWnM}:

\begin{itemize}

\item $\card{A} = 3$, so $A$ satisfies~\eqref{eee}.
\item Michael is not in $A$, so $A$ satisfies~\eqref{nYW} and~\eqref{YW}.
\item Neither Hyungie nor Emanuele is in $A$, so the hypothesis
  of~\eqref{Fall} is false, which means that $A$
  satisfies~\eqref{Fall}.
\item Elizabeth is not in $A$, so $A$ satisfies~\eqref{CB}.
\item Finally, Rachel and Patrick are not in $A$, so the conclusions of
both~\eqref{SJnT} and~\eqref{SWnM} are true, and so $A$
satisfies~\eqref{SJnT} and ~\eqref{SWnM}.

\end{itemize}

So now we have proved
\begin{proposition*}
$\set{\text{Devin}, \text{Henry}, \text{Hao}}$ is the \emph{unique} cabal
satisfying conditions~\eqref{eee}--\eqref{SWnM}.
\end{proposition*}}

\end{document}
