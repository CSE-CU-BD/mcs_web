\documentclass[12pt]{article}
\usepackage{light}

\showsolutions
%\hidesolutions

\newtheorem{false-theorem}[theorem]{False Theorem}

\begin{document}

\recitation{2}{September 14, 2016}
%%%%%%%%%%%%%%%%%%%%%%%%%%%%%%%%%%%%%%%%%%%%%%%%%%%%%%%%%%%%%%%%%%%%%%%%%%%%%%
\iftrue             % For showing teaching notes
\iffalse        % No teaching notes

\insolutions{
%[\stamp]
\section{Induction}

Recall the principle of induction:

\textbox{
\textbf{Principle of Induction.}  Let $P(n)$ be a predicate.  If

\begin{itemize}
\item $P(0)$ is true, and
\item for all $n \in \mathbb{N}$, $P(n)$ implies $P(n+1)$,
\end{itemize}

then $P(n)$ is true for all $n \in \mathbb{N}$.
}

We'll use induction to prove the following theorem:

\begin{theorem}
$\forall n \in \mathbb{N}. \text{   }3 \mid (n^3-n)$.
\end{theorem}

The above theorem means that $n^3-n$ is a multiple of 3.  For example, when $n=5, \\ 3 \mid (125 - 5)$. 

\begin{proof}
We use induction.  Let $P(n)$ be the proposition that:
\begin{equation}\label{1223}
3 \mid (n^3-n)
\end{equation}

Again, here is an example where the induction hypothesis is what we are trying to prove.

\noindent \textit{Base case $n=0$:} $P(0)$ is true, because $3 \mid (0^3-0)$ is true (i.e., 
3 divides 0).

\noindent \textit{Inductive step:} We must show that $P(n)$ implies
$P(n+1)$ for all $n \ge 0$.  So assume that $P(n)$ is true,
where $n$ denotes any nonnegative integer.  Then we examine
\begin{align*}
(n+1)^3 - (n+1) &= n^3+3n^2+3n+1-(n+1)\\
    & = n^3+3n^2+2n
\end{align*}

Question, is $n^3+3n^2+2n$ a multiple of 3?  Not clear by just looking at it.
Hmmmm...how can we show 3 divides it?

Hint, we can use the assumption that $3 \mid (n^3-n)$ here after all.  The whole point of 
using induction is to get some advantage by being able to assume that the induction hypothesis is true.

So let's massage the calculation of $n^3+3n^2+2n$ into one for $n^3-n+3n^2+3n$. 

Since $3 \mid (3n^2)$, $3 \mid 3n$, and $3 \mid (n^3-n)$ 
by the induction hypothesis, 
$3 \mid (n^3-n+3n^2+3n)$, which implies that $P(n+1)$ is true.

\end{proof}

This theorem is a special case of the more general result known as Fermat's Little Theorem, 
which is obtained by replacing 3 with any prime - useful in cryptography as we will see next week.
\\

Let's review the key steps to remember in an induction proof:
\begin{enumerate}
\item Say that the proof is by induction (i.e., write "proof by induction")
\item Define the induction hypothesis, a predicate $P$ defined on the
natural numbers (i.e., write "$P(n)$: ..." and state that it is your "induction hypothesis").
\item Handle the base case: verify that $P(0)$ is true.
\item Handle the inductive step: prove that $P(n)$ implies $P(n+1)$
for all integers $n \ge 0$ (i.e., write "Indutive step", assume $P(n)$ is true, and then verify that 
$P(n+1)$ is true for all $n \ge 0$).
\item Conclude that $P(n)$ is true for all $n \in \mathbb{N}$ by the
principle of induction.
\end{enumerate}

Technically speaking, we don't need to use 0 as the base case.   We can use 1 or any other integer, 
but the proof then will only hold from that point forward.  In general, we could verify $P(b)$ as the 
base case and carry out the inductive step for any $n \ge b$.

Let's write it out, using a base case $n = 2$: \\
%\begin{align*}
if      ~~~ ~$P(2)$
\\
and    ~ $P(2) \rightarrow P(3)$, $P(3) \rightarrow P(4)$, $P(4) \rightarrow P(5)$, ... 
\\
--------------------------------------------------------------------------------------------------
\\
then  ~ $\forall n \ge 2, P(n)$ is true. 
%\end{align*}

}
%%%%%%%%%%%%%%%%%%%%%%%%%%%%%%%%%%%%%%%%%%%%%%%%%%%%%%%%%%%%%%%%%%%%%%%%%%%%%%
\newpage
\fi

\section{Problem:  A Geometric Sum}

Perhaps you encountered this classic formula in school:
%
\[
1 + r + r^2 + r^3 + \ldots + r^n = \frac{1 - r^{n+1}}{1 - r}
\]
%
Use induction to prove that this formula is correct for all real values $r \neq 1$.

\bigskip

\noindent\textit{Prepare a complete, careful solution.  You'll be
passing your proof to another group for ``constructive criticism''!}

\solution{

{\bf Proof by Induction} 
\begin{proof}
We use induction.  Let $P(n)$ be the proposition that the following
equation holds for all $r \neq 1$:
%
\[
1 + r + r^2 + r^3 + \ldots + r^n = \frac{1 - r^{n+1}}{1 - r}
\]
%
\noindent \textit{Base case:} $P(0)$ is true, because both sides of
the equation are equal to 1.

\noindent \textit{Inductive step:} We must show that $P(n)$ implies
$P(n + 1)$ for all $n \in \mathbb{N}$.  So assume that $P(n)$ is true,
where $n$ denotes an arbitrary natural number.  We can reason as
follows:
%
\begin{align*}
1 + r + r^2 + r^3 + \ldots + r^n + r^{n+1}
  & = \frac{1 - r^{n+1}}{1 - r} + r^{n+1} \\
  & = \frac{1 - r^{n+1} + (1 - r) \cdot r^{n+1}}{1 - r} \\
  & = \frac{1 - r^{n+2}}{1 - r}
\end{align*}
%
The first equation follows from the assumption $P(n)$, and the
remaining steps are simplifications.  The derivation proves that $P(n+1)$ is
also true.  Therefore, $P(n)$ implies $P(n+1)$ for all $n \in
\mathbb{N}$.  By the principle of induction, $P(n)$ is true for all $n
\in \mathbb{N}$.
\end{proof}

\noindent \textbf{Note:} You may have encountered a different proof of
this formula.  We'll write down a sequence of equations and then
explain the reasoning.
%
\begin{align*}
S & = 1 + r + r^2 + r^3 + \ldots + r^n \\
r S & = r + r^2 + r^3 + \ldots + r^{n+1} \\
S - r S & = 1 - r^{n+1} \\
S & = \frac{1 - r^{n+1}}{1 - r}
\end{align*}
%
We define $S$ on the first line, multiply by $r$ to get the second
equation, subtract the second equation from the first to get the
third, and then solve for $S$.  This gives the formula above!

This argument is great!  It is a derivation of the formula rather than
just a verification.  But, at some level, we've only hidden the use of
induction, since the operations we're doing on $n$-term sums are
justified using --- you guessed it --- induction.}

%%%%%%%%%%%%%%%%%%%%%%%%%%%%%%%%%%%%%%
\newpage

\section{Problem: Fibonacci}
The Fibonacci sequence is defined by the equation $F_n = F_{n-1} + F_{n-2}$, with
$F_0 = 0$ and $F_1 = 1$. Prove that $\sum_{i=1}^n F_i = F_{n+2} - 1$.

\solution{
%\begin{proof}
The proof is by induction on $n$.

\emph{Base case:}   There are two things to check in the base case, for $n=0 \text{ and }  n=1$. The first thing is 
$\sum_{i=1}^1 F_i = 1 = F_{3} - 1$. The second is
$\sum_{i=1}^2 F_i = 1 + 1 = 2 = F_{4} - 1$.

\emph{Induction step:}   We must show that $\sum_{i=1}^n F_i = F_{n+2} - 1$
implies that $\sum_{i=1}^{n+1} F_i = F_{n+3} - 1$.
\begin{align*}
\sum_{i=1}^{n+1} F_i
  & = \sum_{i=1}^n F_i + F_{n+1} \\
  & = F_{n+2} + F_{n+1} - 1 \\
  & = F_{n+3} - 1
\end{align*}
Thus, we conclude that $\sum_{i=1}^n F_i = F_{n+2} - 1$.
%\end{proof}
}

%%%%%%%%%%%%%%%%%%%%%%%%%%%%%%%%%%%%%%
\newpage

\section{Problem: False Proof}
Find the error in this false proof.
\begin{theorem}
I can lift all the sand on the beach.
\end{theorem}
\begin{proof}
The proof is by induction.
$P(n):$ I can lift $n$ grains of sand.

\emph{Base case:}  $n=0$. Trivial! 

\emph{Induction step:}  Assume I can lift $n$ grains of sand. Then
surely I can lift $n+1$ grains of sand. Hence $P(n) \rightarrow P(n+1)$.

So $\forall n P(n)$ is true.

Let $M$ be the number of grains of sand on the beach, then $P(M) \rightarrow$
I can lift all the grains on the beach.
\end{proof}
\solution{
The inductive step does break down at some point.  
We are not sure exactly where, but there is that one grain of sand that when 
added will get you - you've heard the saying “the straw that broke the camel's
back”?  Same idea - keep on loading up - but eventually a tiny thing causes a breakdown. 
}

%%%%%%%%%%%%%%%%%%%%%%%%%%%%%%%%%%%%%%%%%%%%%%%%%%%%%%%%%%%%%%%%%%%%%%%%%%%%%%%


\newpage

\section{Problem: Surveyevor}

In a new reality TV series called {\em Surveyevor}, a group of
contestants is placed on a small island. Before the series begins,
each contestant agrees to have a small purple or red tattoo, in the
shape of an eye, applied to the middle of his or her forehead. In all,
there are $p \geq 1$ purple eyes and $r \geq 0$ red eyes.  However,
none of the contestants knows the color of his or her third eye, nor
how many total purple and red eyes there are. Furthermore, there are
no mirrors and no one is allowed to discuss the tattoos ever.
Therefore, everyone knows the colors of everyone else's third eye, but
not their own. Good thing, because a contestant who learns that he or
she has a purple eye must leave the island at the end of the show that
day, and is therefore no longer eligible to win the \$1 million cash
prize at the end of the show!

The contestants live in uneasy ignorance for several weeks. As time
goes on, however, most of them lose their fear of being exiled, adapt
to island living, and even make friends with one another. Things are
going quite well for the islanders, but as you might suppose, the
television audience grows bored, and the show's ratings plummet. When
the network threatens to cancel the series, the producer decides she
needs to do something, fast: on the next show, to the surprise of the
happy islanders, the producer herself appears and convenes a
meeting. Very loudly, she proclaims, ``I see that at least one person
here has a purple eye.''  Assuming that all the contestants are master
logicians, what happens?

\solution[\vspace{1in}]{All the purple-tattooed contestants leave the
island at the end of the $p$th day.}

\noindent Use induction to prove that your conclusion is correct.  We
suggest a hypothesis $P(n)$ that asserts all of the following are true
on day $n$:

\begin{enumerate}
\item If $p > n$, then \underline{\hspace{5in}}.
\item If $p = n$, then \underline{\hspace{5in}}.
\item If $p < n$, then \underline{\hspace{5in}}.
\end{enumerate}

(We leave the task of filling in the blanks to you.)

\solution{Note that a red-eyed islander shouldn't ever conclude that
she has a purple eye, since she doesn't, and we're assuming the
contestants always reason correctly from what they know (and that what
they know from the producer is also true).  So no red-eyed contestant
should ever leave the island.

\begin{theorem}
All the purple-eyed people leave the island on day $p$.
\end{theorem}

\begin{proof}
We use induction.  Let $P(n)$ be the proposition that all of the
following are true on day $n$:

\begin{enumerate}
\item If $p > n$, then all purple-eyed people survive the day.
\item If $p = n$, then all purple-eyed people leave the island.
\item If $p < n$, then all purple-eyed people are already gone.
\end{enumerate}

\noindent \textit{Base case:} We must verify that the three parts of
$P(n)$ hold on day $n = 1$.

\begin{enumerate}

\item Suppose $p > 1$.  Consider events on day 1 from the perspective
of a purple-eyed islander.  The producer says that someone has a
purple eye, and she can indeed see at least one other person with a
purple eye.  Therefore, the facts available to her are consistent with
her having either a purple or red eye.  So she survives the day.

\item Suppose $p = 1$.  The single purple-eyed islander sees no one
else with a purple eye, concludes that he must have a purple eye, and
leaves the island at the end of the show.  No one else leaves because
everyone else does see the purple-eyed islander, and they have no
reason at this point to think they too are purple-eyed.

\item This statement is vacuously true, because the if-part ($p < 1$)
is false; the problem statement says that $p \geq 1$.

\end{enumerate}

\noindent Therefore, $P(1)$ is true.

\noindent \textit{Inductive step:} Now suppose that $P(n)$ is true
where $n \geq 0$.  We must verify the three parts of $P(n+1)$.

\begin{enumerate}

\item Suppose $p > n + 1$.  Then $p > n$ so all the purple-eyed
contestants survived the preceding day by part 1 of $P(n)$.
Furthermore, each purple-eyed islander can see at least $n + 1 > n$
other purple-eyed people, so the observation that everyone survives is
consistent with she herself having either a purple or a red eye by
$P(n)$ as well.  Thus, each purple-eyed islander survives the day.

\item Suppose $p = n + 1$.  Then $p > n$, so all the purple-eyed
contestants survived the preceding day by part 1 of $P(n)$.  Thus, on
day $n + 1$ each purple-eyed islander knows $p > n$
%\textbf{[ARM: isn't this using strong induction?]}
, but sees only $n$ other people with a purple eye.  Thus, each
purple-eyed islander realizes that she has a purple eye and leaves the
island at the end of the show.

\item Suppose $p < n + 1$.  Then either $p = n$ (in which case all the
purple-eyed contestants left the island on day $n$ by part 2 of
$P(n)$) or else $p < n$ (in which case all the purple-eyed contestants
were already gone on day $n$ by part 3 of $P(n)$).  In either case,
all the purple-eyed people are already out of luck.

\end{enumerate}

\noindent Therefore $P(n)$ implies $P(n+1)$ for all $n \geq 0$.

By the principle of induction, $P(n)$ is true for all $n \geq 0$, and
the theorem follows.
\end{proof}
}

\iffalse
\noindent If there are at least two blue-eyed people, then the
explorer didn't tell the villagers anything that they didn't already
know, so why was his arrival significant?

\solution{The explorer changes what the villagers know about what
their compatriots are thinking.  For example, suppose that there are
only two villagers, $X$ and $Y$, both with blue eyes.  Before the
explorer arrives, both $X$ and $Y$ know that there is a villager with
blue eyes.  But neither knows that the \textit{other} realizes this.
In contrast, after the explorer makes his statement, each villager
knows that the other knows that someone has blue eyes.}
\fi

%%%%%%%%%%%%%%%%%%%%%%%%%%%%%%%%%%%%%%%%%%%%%%%%%%%%%%%%%%%%%%%%%%%%%%%%%%%%%%%

\end{document}

\remove{
\newpage
\section{Problem: A False Proof}

In lecture, we proved that:
%
\[
1 + 2 + 3 + \ldots + n = \frac{n(n+1)}{2}
\]
%
But now we're going to prove a \textit{contradictory} theorem!

\begin{false-theorem}
For all $n \geq 0$,
%
\[
2 + 3 + 4 + \ldots + n = \frac{n(n+1)}{2}
\]
\end{false-theorem}

\begin{proof}
We use induction.  Let $P(n)$ be the proposition that $2 + 3 + 4 +
\ldots + n = n(n+1)/2$.

\noindent \textit{Base case:} $P(0)$ is true, since both sides of the
equation are equal to zero.  (Recall that a sum with no terms is
zero.)

\noindent \textit{Inductive step:} Now we must show that $P(n)$
implies $P(n+1)$ for all $n \geq 0$.  So suppose that $P(n)$ is true;
that is, $2 + 3 + 4 + \ldots + n = n(n+1)/2$.  Then we can reason as
follows:
%
\begin{align*}
2 + 3 + 4 + \ldots + n + (n+1)
    & = \bigl[2 + 3 + 4 + \ldots + n\bigr] + (n+1) \\
    & = \frac{n(n+1)}{2} + (n+1) \\
    & = \frac{(n+1)(n+2)}{2}
\end{align*}
%
Above, we group some terms, use the assumption $P(n)$, and then
simplify.  This shows that $P(n)$ implies $P(n+1)$.  By the principle
of induction, $P(n)$ is true for all $n \in \mathbb{N}$.
\end{proof}

Where exactly is the error in this proof?

\noindent \textit{Discuss your explanation with your recitation
instructor.  We don't want you to conclude that there is something
wrong with induction proofs in general!}

\solution{The short answer is that we failed to prove $P(0) \implies
P(1)$, just as in the colored horses problem in lecture.  In fact,
once again, the error is rooted in the misleading nature of the
``$\ldots$'' notation.

More precisely, in the inductive step we are required to prove that
$P(n)$ implies $P(n+1)$ for all $n \geq 0$.  However, the argument
given above breaks down when $n = 0$.  Let's look more closely at the
first equation in the inductive step to see why:
%
\[
2 + 3 + 4 + \ldots + n + (n+1)
     = \bigl[2 + 3 + 4 + \ldots + n\bigr] + (n+1)
\]
%
This seems completely innocuous; after all, we've only grouped terms!
However, the left side contains \textit{no terms} when $n = 0$.  The
``$\ldots$'' is completely misleading in this case; 2, 3, 4, and
$n+1$ are actually \textit{not} in the sum.  This misimpression
becomes an error when we ``pull out'' the $(n+1)$ term on the right
side, disregarding the fact that no such term actually existed on the
left.  Thus, for $n = 0$, the equation we've just written down says:
%
\[
\underbrace{2 + 3 + 4 + \ldots + n + (n+1)}_{= 0}
     = \bigl[\underbrace{2 + 3 + 4 + \ldots + n}_{= 0}\bigr] +
       \underbrace{(n+1)}_{= 1}
\]
%
The assertion $0 = 0 + 1$ is false, and so we have not shown that
$P(0)$ implies $P(1)$.  There is no way to fix this problem and
correctly prove that $P(0)$ implies $P(1)$, because actually $P(0)$ is
true and $P(1)$ is false.

Thus, we've only established $P(0)$, $P(1) \implies P(2)$, $P(2)
\implies P(3)$, and so forth.  The induction argument falls apart
because of the missing link $P(0) \not\implies P(1)$.}

}
