\documentclass[12pt,twoside]{article}
\usepackage{fancyhdr}
\usepackage{light}
\usepackage{float}
\usepackage{subfigure}
\usepackage{enumitem}
\usepackage{graphicx}
\hidesolutions
\RequirePackage{cases}
\showsolutions


\newcommand{\Expect}[1]{\mathbb{E}[#1]}
\newcommand{\abs}[1]{|#1|}
\newcommand{\QAND}{\cap}

\newcommand{\eqdef}{:=}
\newcommand{\prcond}[2]{\Pr[#1 \mid #2]}
\newcommand{\prob}[1]{\Pr[#1]}
\renewcommand{\pr}[1]{\Pr \left[ #1 \right]}
\newcommand{\var}[1]{\mathrm{Var}(#1)}
\renewcommand{\ex}[1]{\mathop{\textup{Ex}}[#1]}
\newcommand{\inductioncase}[1]{\textbf{#1}}
\newcommand{\quizz}[2]{
    \vspace*{-2cm}
  \noindent \coursename \hfill #2 \newline
  \coursestaff \vspace{-1.5ex} \newline
  \mbox{} \hrulefill \mbox{}
%  \vspace{-0.15in}
  \begin{center}
    \ifthenelse{\boolean{showsolutions}}
               {\Large \textbf{Final Solutions}}
               {\Large \textbf{Final}}
  \end{center}
  \vspace{-.1in}
  \thispagestyle{plain}
  \pagestyle{myheadings}
  \thispagestyle{empty}
  \markboth{Quiz #1}{Quiz #1}
  %\coursecopyright
 }
  

\begin{document}



%Solutions are currently incorrect
\quizz{1}{12/22/2016}
\thispagestyle{empty}
\fancyhf{}
\renewcommand{\headrulewidth}{0pt}
\pagestyle{fancy}
\fancyhead[L]{Name: \rule{2in}{0.5pt}}

\instatements{

\begin{itemize}

\item  The exam is \textbf{closed book}, but you may have four $8.5''  \times 11''$ sheet with notes (either printed or in your own handwriting) on both sides.

\item Calculators and electronic devices (including cell phones) are not allowed.

\item You may assume all of the results presented in class. This does \textbf{not} include results demonstrated in practice quiz material.

\item Write your name on each page of the exam

 \item Please show your work. Partial credit cannot be given for a wrong answer if your work isn't shown.

 \item Write your solutions in the space provided. If you need more space, write on the back of the sheet containing the problem. Please keep your entire answer to a problem on that problem's page.

 \item Be neat and write legibly. You will be graded not only on the correctness of your answers, but also on the clarity with which you express them.

 \item  If you get stuck on a problem, move on to others. The problems are not arranged in order of difficulty.\\

 \textbf{NAME:} \rule{5in}{0.5pt}\\
 
 \textbf{TA:} \rule{5.34in}{0.5pt}\\
 
\centering
\scalebox{1.1}{
\begin{tabular}{|c|c|c|c|}
\hline
\textbf{Problem} & \textbf{Value} & \textbf{Score} & \textbf{Grader} \\\hline
1 & 10 & & \\\hline
2 & 10 & & \\\hline
3 & 15 & & \\\hline
4 & 20 & & \\\hline
5 & 10 & & \\\hline
6 & 20 & & \\\hline
7 & 10 & & \\\hline
8 & 15 & & \\\hline
9 & 10 & & \\\hline
10 & 10 & & \\\hline
11 & 10 & & \\\hline
\textbf{Total} & 140 & & \\\hline
\end{tabular}
}
\end{itemize}
}
%\instatements{\newpage}
\newpage

\begin{problem}{10}
Find a closed form for $\displaystyle\sum\limits_{i=1}^n \displaystyle\sum\limits_{j=i}^{m} \frac{i}{j}$.  Leave your answer in terms of $n, m$.

\textit{Hint:} Use $H_k$ to represent the $k$th harmonic number
\solution{
\begin{equation*}
\frac{n(n+1)}{2}H_m
\end{equation*}
}
\end{problem}

\newpage

\begin{problem}{10}
We define the sequence of numbers
\[
a_n = \begin{cases}
  1,   &  \text{for $n \leq 3$,}\\
  a_{n-1} + a_{n-2} + a_{n-3}+ a_{n-4},
      &  \text{for $n > 3$.}
 \end{cases}
\]

Use \emph{strong induction} to prove that $\text{remainder}(a_n,3) =
1$ for all $n\geq 0$.

\solution{
We use the fact that $\rem{m}{3} = 1$ iff $m \equiv 1 \pmod 3$.
Letting
\[
P(n)\eqdef\quad a_n \equiv 1 \pmod 3,
\]
we need only show by strong induction that $P(n)$ is true for all $n$.

\inductioncase{Base case} ($n\leq 3$): $a_n \eqdef 1$, so $a_n \equiv
1 \pmod 3$.

\inductioncase{Inductive step}: For $n > 3$, assume $P(k)$ for $0\leq k < n$ 
in order to prove $P(n)$.

In particular, we may assume that $a_k \equiv 1 \pmod 3$ for $k =
n-4,n-3,n-2,n-1$.  But $P(n)$ is equivalent to $a_{n-4} + a_{n-3} +
a_{n-2} + a_{n-1} = 1 \pmod 3$, so $P(n)$ is true because
\[
a_n = a_{n-4} + a_{n-3} + a_{n-2} + a_{n-1} \equiv (1+1+1+1) = 1 \pmod 3,
\]
}

\end{problem}

\newpage
\begin{problem}{15}
\bparts
\ppart{10} Find a solution to $x_n = 4x_{n-1} + n + 1$ with $x_0 = 2$.
\solution{
We begin by first solving the homogeneous linear recurrence $x_n = 4x_{n-1}$.  
We have that the characteristic polynomial is $r - 4 = 0$ and so $x_n = C_1 \cdot 4^n$ for some constant $C_1$.  

Now we solve for the particular solution.  As the inhomogeneous term is $n + 1$, we make a guess that the solution is of the form $An + B$ for constants $A, B$.  Substituting we find:

\begin{equation*}
\begin{split}
An + B &= 4 (An - A + B) + n + 1 \\
An + B &= 4An - 4A + 4B + n + 1 \\
3An -4A + 3B + n + 1 &= 0 \\
\end{split}
\end{equation*}
Hence we must have that $3A + 1 = 0$ and $-4A + 3B + 1 = 0$, implying that $A = \frac{-1}{3}$ and $B = \frac{-7}{9}$.

Substituting back, we find that our solution is $x_n = C_1 \cdot 4^n -\frac{1}{3} n - \frac{7}{9}$.  Now we are given that $x_0 = 2$ and so $C_1 = \frac{25}{9}$. 

Hence the solution is $x_n = \frac{25*4^n - 3n - 7}{9}$.

}
\ppart{5} Give an asymptotic expression for the following recurrence, in $\Theta$ notation:
\begin{equation*}
\begin{split}
T(n) &= 2T(\frac{n}{4}) + 3T(\frac{n}{6}) + n^3, \qquad T(1) = 0
\end{split}
\end{equation*}
\solution{
Using Akra-Bazzi, we see that $a_1 = 2, b_1 = \frac{1}{4}, a_2 = 3, b_2 = \frac{1}{6}$ and so $p = 1$.  
Hence, $T(n) = \Theta \left( n^2 \left( 1 + \int_{1}^{n} \frac{n^3}{n^3} dn \right) \right) = \Theta (n^3) $.
}
\eparts
\end{problem}

\newpage
\begin{problem}{20}
\bparts
\ppart{10} Suppose that we are flipping a fair coin $n$ times. What is 
the probability that there are exactly $k$ heads, where the heads
must be separated by at least 2 tails?

\solution{
  There are $\binom{n-2k+2}{k}$ number of ways to choose where to 
  place the heads (consider a bijection with having $k$ 1s 
  in an $n-2k+2$ bit binary string). Note that it is $n-2(k-1)$ 
  because the last head does not need to be followed by 2 tails.
  
  The total number of results is $2^n$, so the probability is:
  
  \[ \binom{n-2k+2}{k} \over {2^n} \]
}

\ppart{10} Give a combinatorial proof for this identity:
\[
\sum_{i=0}^n \binom{k+i}{k} = \binom{k+n+1}{k+1}
\]
\textit{Hint: }Let $S_i$ be the set of binary sequences with exactly $n$
zeroes, $k + 1$ ones, and a total of exactly $i$ occurrences of zeroes
appearing before the rightmost occurrence of a one.
\eparts
\end{problem}

\newpage

\begin{problem}{10}
Is there a bipartite graph with ordered degree sequence $3, 3, 3, 3, 3, 4, 4, 4$?
\textit{Hint: } The vertices of a bipartite graph can be divided into two subsets.  Consider the sum of degrees of the vertices in each subset.
\end{problem}

\solution{

This is not possible.  We know that the vertices of a bipartite graph can be divided into two subsets, $A$, $B$.  Now as there are only edges between $A$ and $B$ in the graph, the sum of degrees of vertices in subset $A$ must be the same as the sum of degrees of vertices in subset $B$.
}



\newpage

\begin{problem}{20}
The hat-check staff has had a long day serving at a party, and at the
end of the party they simply return the $n$ checked hats uniformly at random, such that the probability that any particular person gets their
own hat back is $1/n$.

Let $X_i$ be the \textit{indicator variable} for the $i$th person getting
their own hat back.  Let $S_n$ be the total number of people who get
their own hat back.

\bparts

\ppart{2} What is the expected number of people who get their own hat back?

\iffalse
\begin{center}
\exambox{1.5in}{0.5in}{0.0in}
\end{center}
\fi

\solution[\vspace{2in}]{
$S_n = \sum_{i=1}^n X_i$, so by \textit{linearity of expectation},
\[
\expect{S_n} = \sum_{i=1}^n \expect{X_i}.
\]
Since the probability a person gets their own hat back is $1/n$, therefore
$\pr{X_i=1} = 1/n$. Now, since $X_i$ is an
indicator, we have $\expect{X_i} = 1/n$.  By linearity of expectation,
\[
\expect{S_n} = \sum_{i=1}^n \expect{X_i} = n\cdot\frac{1}{n} = 1.
\]

}

\ppart{3} Write a simple formula for $\expect{X_i \cdot X_j}$ for $i \neq j$.

\textit{Hint: } What is $\Pr{\{X_j=1|X_i=1\}}$?

\solution[\vspace{2in}]{
We observed above that $\pr{X_i=1} = 1/n$.  Also, given that the
$i$th person got their own hat, each other person has an equal chance of
getting their own hat among the remaining $n-1$ hats.  So
\[
\prcond{X_j=1}{X_i=1} = \frac{1}{n-1},
\]
for $j \neq i$.  Therefore,
\[
\pr{X_i=1 \QAND X_j=1}
  = \prcond{X_j=1}{X_i=1}\cdot \pr{X_i=1} = \frac{1}{n(n-1)}.
\]
But $X_i=1 \QAND X_j=1$ iff $X_i X_j = 1$, so
\[
\expect{X_i X_j} = \pr{X_i X_j =1} = \pr{X_i=1 \QAND X_j=1},
\]
and hence
\[
\expect{X_i X_j} = \frac{1}{n(n-1)}.
\]
}

\ppart{5} Show that $\expect{(S_n)^2} = 2$.

\solution[\vspace{2in}]{

\begin{align*}
\expect{S_n^2}
     & =  \Expect{\sum_{i=1}^n (X_i)^2 + 2\sum_{1 \leq i < j \leq n} X_i X_j}
           & \text{(expanding the sum for $S_n$)}\\
     & =  \sum_{i=1}^n \expect{(X_i)^2} + 2\sum_{1\leq i<j \leq n} \expect{X_i X_j}
           & \text{(linearity of $\expect{\cdot}$)}\\
     & =  \sum_{i=1}^n \expect{X_i} + 2\sum_{1\leq i<j \leq n} \frac{1}{n(n-1)}
           & \text{(since $(X_i)^2 = X_i$)}\\
     & =  n \cdot \frac{1}{n} + 2\binom{n}{2} \frac{1}{n(n-1)} \\
     & =  1+1 = 2.
\end{align*}

}
\newpage
\ppart{5} What is the variance of $S_n$?

\iffalse
\begin{center}
\exambox{0.5in}{0.5in}{0in}
\end{center}
\fi

\solution[\vspace{3in}]{
\begin{equation}\label{snsn2-}
\variance{S_n} = \expect{(S_n)^2} - \expectsq{S_n} = 2 - 1^2 = 1.
\end{equation}
}

\ppart{5} Use the \textit{Chebyshev bound} to show that there is
at most a 1\% chance that more than 10 people get their own hat back.
\iffalse
Try to give an intuitive explanation of why the chance remains this
small regardless of $n$.
\fi

\solution[\vspace{3in}]{
\begin{align*}
\pr{S_n \geq 11} &= \pr{S_n - \expect{S_n} \geq 11 - \expect{S_n}} \\
                 &= \pr{S_n - \expect{S_n} \geq 10} & \text{(by \eqref{snsn2-})}\\
                 &\leq \pr{\abs{S_n - \expect{S_n}} \geq 10} \\
                 &\leq \frac{\variance{S_n}}{10^2} = .01  & \text{(by Chebyshev)}
\end{align*}
}

\eparts
\end{problem}

\newpage

\begin{problem}{10}
Find the generating function for the number of ways to pay any amount of money using only pennies, nickels, dimes, and quarters.
\solution{
A generating function for $x_1$ is just $(x + x^2 + x^3 + \ldots) = \frac{x}{1-x}$.
A generating function for $2x_2$ is just $(x^2 + x^4 + x^6 + \ldots) = \frac{x^2}{1-x^2}$. 
A generating function for $3x_3$ is just $(x^3 + x^6 + x^9 + \ldots) = \frac{x^3}{1-x^3}$. 
A generating function for $4x_4$ is just $(x^4 + x^8 + x^{12} + \ldots) = \frac{x^4}{1-x^4}$.
Hence a generating function for the number of solutions to the above equation is just $\frac{x \cdot x^2 \cdot x^3 \cdot x^4}{(1-x)(1-x^2)(1-x^3)(1-x^4)}$, and the number of solutions to the particular solution above is the coefficient of the $x^n$ term in the expansion of the generating function.    
}
\end{problem}

\newpage

\begin{problem}{15}
Vlad has been sick for the past few days and is curious to know which disease he is suffering from.  He knows that he has the flu with probability .3 and the common cold with probability .7.

If he has the flu, the conditional probability that the flu medicine will work and cure all symptoms is .4.  Similarly, if he has the common cold, then he will be free of symptoms by taking the common cold medicine with conditional probability .15. Vlad can only take medicine for one disease each day.

You don't have to reduce your answer for the following problems.

\bparts
\ppart{5}  Which disease should Vlad take medicine for on the first day in order to maximize the probability of curing his disease?

\solution{
We just compare the probabilities that he is cured taking the flu medicine and the cold medicine.  As $.4 \cdot .3 \geq .15 \cdot .7$, he should take the flu medicine first.  
}

\ppart{5}  Vlad took medicine for the flu on the first day but is still sick.  What is the probability that he has the flu?

\solution{
This is just $\frac{.6 \cdot .3}{.6 \cdot .3 + .7 \cdot 1}$.
}

\ppart{5}  Vlad flips a fair coin to determine which medicine to take the first day and gets better on the first day.  What is the probability that he took the flu medicine?


\solution{
This is just the conditional probability that Vlad looked took the flu medicine and was cured.  This is given by $\frac{.5 \cdot .3 \cdot .4}{.5 \cdot .3 \cdot .4 + .5 \cdot .6 \cdot .15}$.  
}

\eparts
\end{problem}

\newpage

\begin{problem}{10}
In a permutation of $n$ elements, a pair $(i, j)$ is called an inversion if and only if $i < j$ and $i$ comes after $j$.  For example, the permutation $31542$ in the case of $n = 5$ has five inversions: $(3, 1), (3, 2), (5, 4), (5, 2)$ and $(4, 2)$. What is the expected number of inversions in a uniform random permutation of the number $1, 2, \ldots n$?

\textit{Hint:} Use appropriate indicator variables and linearity of expectation.
\solution{
We use $\mathbb{I}(i, j)$ to be an indicator variable that the pair $(i, j)$ is an inversion.  Then if $T$ is the number of inversions, by linearity of expectation, the number of inversions $\mathbb{E}(T) = \displaystyle\sum\limits_{i=1}^{n-1} \displaystyle\sum\limits_{j=i}^{n}\mathbb{E}(\mathbb{I}(i, j))$.  Now the probability that $(i, j)$ is an inverted pair is just $\frac{1}{2}$ as we draw a uniform random permutation.  Hence $\mathbb{E}(\mathbb{I}(i, j)) = \frac{1}{2}$ and so the $\mathbb{E}(T) = \frac{1}{2} \binom{n}{2}$.
}
\end{problem}

\newpage

\begin{problem}{10}
Consider the following three random variables:
\begin{enumerate}
\item Let A be a binary random variable that is $1$ if a coin $C_1$ comes up heads and $0$ otherwise.
\item Let B be a binary random variable that is $1$ if a coin $C_2$ comes up heads and $0$ otherwise.
\item Let C be a binary random variable that is $1$ if both A and B are different values and $0$ otherwise.
\end{enumerate}
Assume that $C_1$ and $C_2$ are independent coins.
\bparts
\ppart{5} Are A, B, C mutually independent?
\solution{
No A, B, C are not mutually independent as fixing values for A, B, uniquely determine a value for variable C. 
}
\ppart{5} Are A, B, C pairwise independent?
\solution{
A, B, and C are pairwise independent.  We first see that A, B are independent.  Now we show that A, C are independent and by symmetry this will imply B, C are independent.  We have that $P(C = 1 | A = 1) = \frac{1}{2} = P(A = 1)$ as now C is uniquely determined by the value of B and so is $1$ with probability $\frac{1}{2}$ and $0$ with probability $\frac{1}{2}$.  Similarly, $P(C = 1 | A = 0) = \frac{1}{2} = P(A = 0)$.  Hence we have that C and A are independent events.  
}
\eparts
\end{problem}
\newpage

\begin{problem}{10}
Consider tossing a fair coin $C$ until one throws a heads.  Tossing $C$ results in heads with probability $\frac{1}{2}$.  Let $X$ be a random variable corresponding to the number of tosses needed until one throws a heads (so $X \geq 1$).  
\bparts
\ppart{5} Calculate $\mathbb{E}(X)$.
\solution{
We use a recursion to calculate $\mathbb{E}(X)$.  Namely, $\mathbb{E}(X) = \frac{1}{2} + \frac{1}{2}(\mathbb{E}(X) + 1)$. This implies that $\mathbb{E}(X) = 2$.
}
\ppart{10} Calculate $\mathbb{E}(X^3)$.

\solution{
We use the hint.  We already know $\mathbb{E}(X)$.  Now we just calculate $\mathbb{E}(X^2)$ again by using a recursion.  Namely, $\mathbb{E}(X^2) = \frac{1}{3} + \frac{2}{3}(\mathbb{E}((X + 1)^2))$.

Hence we have, $\mathbb{E}(X^2) = \frac{1}{3} + \frac{2}{3}(\mathbb{E}(X^2) + 2\mathbb{E}(X) + 1)$, and so $\mathbb{E}(X^2) = 15$.  

Thus $\mathbb{E}(X^2) - \mathbb{E}(X)^2 = 15 - 9 = 6$ is the variance.  
}
\eparts
\end{problem}

\newpage 

\begin{center}
(This space left blank internationally)
\end{center}

\newpage
\begin{center}
(This space left blank internationally)
\end{center}

\mbox{}

\newpage
\begin{center}
(This space left blank internationally)
\end{center}

\mbox{}

\end{document}
