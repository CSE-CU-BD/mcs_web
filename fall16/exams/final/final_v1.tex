\documentclass[12pt,twoside]{article}
\usepackage{fancyhdr}
\usepackage{light}
\usepackage{float}
\usepackage{subfigure}
\usepackage{enumitem}
\usepackage{graphicx}
\hidesolutions
\RequirePackage{cases}
\showsolutions
\begin{document}
\newcommand{\inductioncase}[1]{
\textbf{#1}
}
\newcommand{\quizz}[2]{
    \vspace*{-2cm}
  \noindent \coursename \hfill #2 \newline
  \coursestaff \vspace{-1.5ex} \newline
  \mbox{} \hrulefill \mbox{}
%  \vspace{-0.15in}
  \begin{center}
    \ifthenelse{\boolean{showsolutions}}
               {\Large \textbf{Final Solutions}}
               {\Large \textbf{Final}}
  \end{center}
  \vspace{-.1in}
  \thispagestyle{plain}
  \pagestyle{myheadings}
  \thispagestyle{empty}
  \markboth{Quiz #1}{Quiz #1}
  %\coursecopyright
  }
%Solutions are currently incorrect
\quizz{1}{12/22/2016}
\thispagestyle{empty}
\fancyhf{}
\renewcommand{\headrulewidth}{0pt}
\pagestyle{fancy}
\fancyhead[L]{Name: \rule{2in}{0.5pt}}

\instatements{

\begin{itemize}

\item  The exam is \textbf{closed book}, but you may have four $8.5''  \times 11''$ sheet with notes (either printed or in your own handwriting) on both sides.

\item Calculators and electronic devices (including cell phones) are not allowed.

\item You may assume all of the results presented in class. This does \textbf{not} include results demonstrated in practice quiz material.

\item Write your name on each page of the exam

 \item Please show your work. Partial credit cannot be given for a wrong answer if your work isn't shown.

 \item Write your solutions in the space provided. If you need more space, write on the back of the sheet containing the problem. Please keep your entire answer to a problem on that problem's page.

 \item Be neat and write legibly. You will be graded not only on the correctness of your answers, but also on the clarity with which you express them.

 \item  If you get stuck on a problem, move on to others. The problems are not arranged in order of difficulty.\\

 \textbf{NAME:} \rule{5in}{0.5pt}\\
 
 \textbf{TA:} \rule{5.34in}{0.5pt}\\
 
\centering
\scalebox{1.1}{
\begin{tabular}{|c|c|c|c|}
\hline
\textbf{Problem} & \textbf{Value} & \textbf{Score} & \textbf{Grader} \\\hline
1 & 10 & & \\\hline
2 & 10 & & \\\hline
3 & 10 & & \\\hline
4 & 15 & & \\\hline
5 & 20 & & \\\hline
6 & 10 & & \\\hline
7 & 15 & & \\\hline
8 & 10 & & \\\hline
9 & 10 & & \\\hline
10 & 10 & & \\\hline
11 & 15 & & \\\hline
\textbf{Total} & 135 & & \\\hline
\end{tabular}
}
\end{itemize}
}
%\instatements{\newpage}

\newpage
\begin{problem}{10}
Let $a_0 = a_1 = 1$, and let $a_{n+2} = a_{n+1} + 5a_{n}$ for $n \geq 0$.  Prove by strong induction that $a_n \leq 3^n$ for all $n \geq 0$.  You do not need to solve the recursion to do this.
\solution{
Base Case: $a_0 \leq 3^0 = 1$ \\
Inductive Hypothesis: Assume for $i = 0, 1, 2, \ldots n + 1$ that $a_i \leq 3^i$. \\
Inductive Step: $a_{n+2} = a_{n+1} + 5a_{n} \leq 3^{n+1} + 5\cdot 3^n = 8 \cdot 3^n \leq 9 \cdot 3^n = 3^{n+2}$ \\
Hence as the statement holds for $i = n+2$ given $i = 0,1, 2, \ldots n+1$, we conclude that it holds for all $n \geq 0$.
}
\end{problem}

\newpage
\begin{problem}{10}
The MIT Social Statistics Society (MIT $\text{S}^3$) is doing a study to see how social engineers are.  So they host a party with 15 students.  After the party, 8 students report not having met anyone at the party, 4 students report having met one other student each, 2 students report have met two other students each, and one student reports having met three other students.  Is this possible?  You must prove your answer. Assume that the ``met" relation is symmetric and anti-reflexive.

\solution{
This is not possible.  The total sum of degrees is $1 + 1 + 1 + 1 + 2 + 2 + 3 = 11$, which implies that there are $\frac{11}{2}$ edges, which is not possible. 
}
\end{problem}

\newpage
\begin{problem}{10}
Find a closed form for $\displaystyle\prod\limits_{i=1}^n \displaystyle\prod\limits_{j=i}^{n} 2^{(i-j)}$.
\solution{
The solution is:
\begin{equation*}
\begin{split}
\displaystyle\prod\limits_{i=1}^n 2^{(n-i + 1)i} 2^{-(i + \ldots + n)} &= \displaystyle\prod\limits_{i=1}^n 2^{(n-i + 1)i} 2^{-\frac{(n+i)(n-i + 1)}{2}}\\
&= 2 ^ {\left( \displaystyle\sum\limits_{i=1}^n \left( i(n- i+1) -\frac{(n+i)(n-i + 1)}{2}  \right) \right)} \\
&= 2^ { \left( \displaystyle\sum\limits_{i=1}^n \left( i(n- i+1) - \frac{n^2 - i^2}{2} - \frac{n+i}{2} \right) \right)} \\
&= 2^ { \left(\frac{n(n+1)^2}{2} -\frac{n(n+1)(2n+1)}{6} - \frac{n^3}{2} + \frac{n(n+1)(2n+1)}{12} - \frac{n^2}{2} - \frac{n(n+1)}{4} \right)}\\
&= 2^{\frac{n-n^3}{6} }\\
\end{split}
\end{equation*}
}
\end{problem}


\newpage
\begin{problem}{10}
\bparts
\ppart{7} Find a solution to $f_n = 4 f_{n-1} + 5 f_{n-2}$, with $f_0 =1 , f_1 = 1$.  
\solution{$f_i = (2/3) (-1)^i + (1/3) (5)^i$}

%\ppart{10} Find a solution to $x_n = 5x_{n-1} + n - 1$ with $x_0 = 0$.
%\solution{
%We begin by first solving the homogeneous linear recurrence $x_n = 5x_{n-1}$.  
%We have that the characteristic polynomial is $r - 5 = 0$ and so $x_n = C_1 \cdot 5^n$ for some constant $C_1$.  
%
%Now we solve for the particular solution.  As the inhomogeneous term is $n - 1$, we make a guess that the solution is of the form $An + B$ for constants $A, B$.  Substituting we find:
%
%\begin{equation*}
%\begin{split}
%An + B &= 5 (An - A + B) + n - 1 \\
%An + B &= 5An - 5A + 5B + n - 1 \\
%4An -5A + 4B + n - 1 &= 0 \\
%\end{split}
%\end{equation*}
%Hence we must have that $4A + 1 = 0$ and $-5A + 4B -1 = 0$, implying that $A = \frac{-1}{4}$ and $B = \frac{-1}{16}$.
%
%Substituting back, we find that our solution is $x_n = C_1 \cdot 5^n -\frac{1}{4} n - \frac{1}{16}$.  Now we are given that $x_0 = 0$ and so $C_1 = \frac{1}{16}$. 
%
%Hence the solution is $x_n = \frac{5^n - 4n - 1}{16}$.
%}

\newpage
\ppart{8} Give an asymptotic expression for the following recurrence, in $\Theta$ notation:
\begin{equation*}
\begin{split}
T(n) &= 8T(\frac{n}{4}) + 18T(\frac{n}{6}) + n^3, \qquad T(1) = 0
\end{split}
\end{equation*}
\solution{
Using Akra-Bazzi, we see that $a_1 = 8, b_1 = \frac{1}{4}, a_2 = 18, b_2 = \frac{1}{36}$ and so $p = 2$.  
Hence, $T(n) = \Theta \left( n^2 \left( 1 + \int_{1}^{n} \frac{n^3}{n^3} dn \right) \right) = \Theta (n^3) $.
}
\eparts
\end{problem}

\newpage
\begin{problem}{20}
\bparts
\ppart{6} A cashier wants to work 5 days a week, but he wants to have at least one of Saturday or Sunday off.  How many ways can he choose the days he will work? Your answer should be an integer.
\vspace{3.5in}
\newpage
\solution{
The cashier can choose to work on exactly one of Saturday or Sunday in $2 \cdot \binom{5}{4}$ ways, and the cashier can choose to work on neither Saturday or Sunday in $\binom{5}{5}$ ways.  Hence the total number of ways is just $11$.  
}

\ppart{6} How many permutations of $1, 2, 3, \dots n$ are there if $1$ must precede $2$ and $3$ must precede $4$ (for positive integers $n \geq 4$). Your answer should be in terms of $n$.  
\vspace{3.5in}
\solution{ 
Let us label both $1, 2$ as $A, A$ and $3, 4$ as $B, B$.  Then each permutation of the $n$ symbols $A, A, B, B, 5, 6, \ldots n$ corresponds to a permutation where $1$ precedes $2$ and $3$ precedes $4$.  Hence the number of ways to order these $n$ symbols is just $\frac{n!}{2 \cdot 2} = \frac{n!}{4}$. 
}
\newpage

\ppart{8} Let $a_1, a_2, \ldots a_k$ be positive integers with sum at most $n$ (with $k > 1$).  Use a combinatorial argument to show that  $a_1!a_2!\ldots a_k! < n!$.
\solution{
Let us first increase $a_k$ to $a_k'$ such that $a_1 + a_2 + \ldots + a_k' = n$.  Then $\frac{n!}{a_1! a_2!\ldots a_k'!}$ is just the number of ways to order $n$ items of which there are $a_1$ of them being the same, $a_2$ of them being the same, ... , $a_k'$ of them being the same.  As this is just an enumeration of items, we must have that it is a positive quantity.  Hence $\frac{n!}{a_1! a_2!\ldots a_k'!} \geq 1$.  Now this implies that $n! \geq a_1! a_2!\ldots a_k'! > a_1! a_2!\ldots a_k!$.  
}
\eparts
\end{problem}


\newpage
\begin{problem}{10}
Find the generating function of the number of solutions to 
\begin{equation*}
\begin{split}
x_1 + 2x_2 + 3x_3 + 4x_4 = n
\end{split}
\end{equation*}
where $x_1, x_2, x_3, x_4$ are positive integers. Express your answer as the inverse of a product of polynomials.
\solution{
A generating function for $x_1$ is just $(x + x^2 + x^3 + \ldots) = \frac{x}{1-x}$.
A generating function for $2x_2$ is just $(x^2 + x^4 + x^6 + \ldots) = \frac{x^2}{1-x^2}$. 
A generating function for $3x_3$ is just $(x^3 + x^6 + x^9 + \ldots) = \frac{x^3}{1-x^3}$. 
A generating function for $4x_4$ is just $(x^4 + x^8 + x^{12} + \ldots) = \frac{x^4}{1-x^4}$.
Hence a generating function for the number of solutions to the above equation is just $\frac{x \cdot x^2 \cdot x^3 \cdot x^4}{(1-x)(1-x^2)(1-x^3)(1-x^4)}$, and the number of solutions to the particular solution above is the coefficient of the $x^n$ term in the expansion of the generating function.    
}
\end{problem}

\newpage
\begin{problem}{15}
Vlad's tiger has wandered into one of two forests overnight.  It is in forest A with probability .4 and in forest B with probability .6.  

If his tiger is in forest A and Vlad spends a day searching for it in forest A, the conditional probability that he will find his tiger that day is .25.  Similarly, if his tiger is in forest B and he spends a day searching for it in forest B, then he will find his tiger that day with conditional probability .15.

You don't have to reduce your answer for the following problems.

\bparts
\ppart{5}  In which forest should Vlad look on the first day in order to maximize the probability of finding his tiger that day?
\solution{
We just compare the probabilities that he finds the tiger in forest A vs. in forest B.  As $.25 \cdot .4 \geq .15 \cdot .6$, he should look in forest A first.  
}
\vspace{3in}

\newpage
\ppart{5}  Vlad looked in forest A on the first day but didn't find his tiger.  What is the probability that the tiger is in forest A?
\solution{
We just find the conditional probability that the tiger is in forest A given that Vlad didn't find his tiger and that he searched in forest A.  This is just $\frac{.75 \cdot .4}{.75 \cdot .4 + .6 \cdot 1} = \frac{1}{3}$.
}

\newpage
\ppart{5}  Vlad flips a fair coin to determine where to look on the first day and finds his tiger on the first day.  What is the probability that he looked in forest A?
\vspace{3.5in}
\solution{
This is just the conditional probability that Vlad looked in forest A given that he found his tiger.  This is given by $\frac{.5 \cdot .4 \cdot .25}{.5 \cdot .4 \cdot .25 + .5 \cdot .6 \cdot .15} = \frac{10}{19}$.  
}

\eparts
\end{problem}

\newpage

\begin{problem}{10}
In a permutation of $n$ elements, a pair $(i, j)$ is called an inversion if and only if $i < j$ and $i$ comes after $j$.  For example, the permutation $31542$ in the case of $n = 5$ has five inversions: $(3, 1), (3, 2), (5, 4), (5, 2)$ and $(4, 2)$. What is the expected number of inversions in a uniform random permutation of the number $1, 2, \ldots n$?

\textit{Hint:} Use appropriate indicator variables and linearity of expectation.
\solution{
We use $\mathbb{I}(i, j)$ to be an indicator variable that the pair $(i, j)$ is an inversion.  Then if $T$ is the number of inversions, by linearity of expectation, the number of inversions $\mathbb{E}(T) = \displaystyle\sum\limits_{i=1}^{n-1} \displaystyle\sum\limits_{j=i}^{n}\mathbb{E}(\mathbb{I}(i, j))$.  Now the probability that $(i, j)$ is an inverted pair is just $\frac{1}{2}$ as we draw a uniform random permutation.  Hence $\mathbb{E}(\mathbb{I}(i, j)) = \frac{1}{2}$ and so the $\mathbb{E}(T) = \frac{1}{2} \binom{n}{2}$.
}
\end{problem}



\newpage
\begin{problem}{10}
Consider tossing a non-fair coin $C$ until one throws a heads.  Tossing $C$ results in heads with probability $\frac{1}{3}$.  Let $X$ be a random variable corresponding to the number of tosses needed until one throws a heads (so $X \geq 1$).  
\bparts
\ppart{5} Calculate $\mathbb{E}(X)$.
\vspace{3in}
\solution{
We use a recursion to calculate $\mathbb{E}(X)$.  Namely, $\mathbb{E}(X) = \frac{1}{3} + \frac{2}{3}(\mathbb{E}(X) + 1)$. This implies that $\mathbb{E}(X) = 3$.
}

\newpage
\ppart{5} Calculate the variance of $X$.
%\textit{Hint:} Use the fact that Var($X$) = $\mathbb{E}(X^2) - \mathbb{E}(X)^2$.
\solution{
We use the hint.  We already know $\mathbb{E}(X)$.  Now we just calculate $\mathbb{E}(X^2)$ again by using a recursion.  Namely, $\mathbb{E}(X^2) = \frac{1}{3} + \frac{2}{3}(\mathbb{E}((X + 1)^2))$.

Hence we have, $\mathbb{E}(X^2) = \frac{1}{3} + \frac{2}{3}(\mathbb{E}(X^2) + 2\mathbb{E}(X) + 1)$, and so $\mathbb{E}(X^2) = 15$.  

Thus $\mathbb{E}(X^2) - \mathbb{E}(X)^2 = 15 - 9 = 6$ is the variance.  
}
\eparts
\end{problem}

\newpage
\begin{problem}{10}
One raffle ticket is drawn randomly from a bowl containing four tickets numbered 1, 2, 3, and 4. Consider the following three random variables:
\begin{enumerate}
\item Let A be a binary random variable that is $1$ if a 1 or 2 is drawn and $0$ otherwise. 
\item Let B be a binary random variable that is $1$ if a 1 or 3 is drawn and $0$ otherwise.
\item Let C be a binary random variable that is $1$ if a 1 or 4 is drawn and $0$ otherwise.
\end{enumerate}
Assume that all raffle tickets have equal probability of being drawn
\bparts
\ppart{5} Are A, B, C mutually independent? Briefly justify your answer. 
\vspace{3in}
\solution{
No A, B, C are not mutually independent as fixing values for A, B, uniquely determine a value for variable C. 
}
\newpage
\ppart{5} Are A, B, C pairwise independent? Briefly justify your answer.
\solution{
A, B, and C are pairwise independent.  We first see that A, B are independent.  Now we show that A, C are independent and by symmetry this will imply B, C are independent.  We have that $P(C = 1 | A = 1) = \frac{1}{2} = P(A = 1)$ as now C is uniquely determined by the value of B and so is $1$ with probability $\frac{1}{2}$ and $0$ with probability $\frac{1}{2}$.  Similarly, $P(C = 1 | A = 0) = \frac{1}{2} = P(A = 0)$.  Hence we have that C and A are independent events.  
}
\eparts
\end{problem}

\newpage
\begin{problem}{15} Let $X$ be a random variable indicating the runtime of an algorithm on an input of size $n$. You know that $X \geq 0$ and that $\mathbb{E}(X) \leq 10n$.  
\bparts
\ppart{5} Give as good a bound as you can on the probability that $X \geq 20n$. 
\vspace{3.5in} 
\solution{
Using Markov's inequality, we find that $\mathbb{P}(X \geq 20n) \leq \frac{10n}{20n} = \frac{1}{2}$.
}

%\ppart{5} Show that with the given information, there is no stronger bound you can give, by giving an explicit random variable that satisfies the conditions given above and matches your bound.  
%\solution{
%Consider a random variable $T$ that is $20n$ with probability $\frac{1}{2}$ and is $0$ with probability $\frac{1}{2}$.  Then $\mathbb{E}(T) = 10n$ and $\mathbb{P}(T \geq 20n) = \frac{1}{2}$.
%}
\newpage
\ppart{5} Now suppose you are told that Var($X$)  $ \leq 10n$.  Use Chebyshev's inequality to bound the probability that $X \geq 20n$. 
\vspace{3.5in} 
\solution{
By Chebyshev's inequality, we have $\mathbb{P}(X \geq 20n) \leq \mathbb{P}( |X - 10n| \geq 10n) \leq \frac{\text{Var}(X)}{100n^2} \leq \frac{1}{10}$.
}

\newpage
\ppart{5} Now suppose there is another random variable $T$ indicating the runtime of a different algorithm on the same input of size $n$.  You are given that $\mathbb{E}(e^T) \leq e^{10n}$.  Write down as strong a bound as you can on the probability that $T \geq 20n$.  
\solution{
We use the fact that $\mathbb{P}(T \geq 20n) \leq \mathbb{P}(e^T \geq e^{20n}) \leq \frac{e^{10n}}{e^{20n}} = e^{-10n}$. 
}
\eparts
\end{problem}


\newpage 
\begin{center}
(This space left blank internationally)
\end{center}

\newpage
\begin{center}
(This space left blank internationally)
\end{center}

\mbox{}

\newpage
\begin{center}
(This space left blank internationally)
\end{center}

\mbox{}

\end{document}
%Find a closed form for $\displaystyle\sum\limits_{i=1}^n \displaystyle\sum\limits_{j=i}^{i + m} \frac{i(i + m)}{j(j+1)}$.  Leave your answer in terms of $n, m$.
%
%\textit{Hint:} Use partial fraction decomposition.
%\solution{
%\begin{equation*}
%\begin{split}
%\displaystyle\sum\limits_{i=1}^n \displaystyle\sum\limits_{j=i}^{i+m} \frac{i(i+m)}{j(j+1)} &= \displaystyle\sum\limits_{i=1}^n i(i+m) \displaystyle\sum\limits_{j=i}^{i+m} \frac{1}{j(j+1)} \\
%&= \displaystyle\sum\limits_{i=1}^n i(i+m)  \displaystyle\sum\limits_{j=i}^{i+m} \left( \frac{1}{j} - \frac{1}{j+1} \right) \\
%&= \displaystyle\sum\limits_{i=1}^n i(i+m) \left(\frac{1}{i} - \frac{1}{i+m} \right) \\
%&= \displaystyle\sum\limits_{i=1}^n \left(i + m - i \right) \\
%&= mn \\
%\end{split}
%\end{equation*}
%}