\documentclass[11pt]{article}
\usepackage{light}


\title{Graph Theory and Stable Matching}
\begin{document}
\maketitle

\section{Graph Theory}
\textbf{Relevant material: }
pset 4 -- problems 4, 5 and 6\\
 pset 5 -- problems 1 and 2\\
 pset 6 -- problems 1 and 2\\
   recitation 6\\
   Midterm practice -- problems 5 and 7\\
  2014 -- problem 3 and 7

\begin{itemize}
	\item A \textbf{graph} $G = (V, E)$ is a set of vertices and edges between them.
	\item The \textbf{degree} of a vertex is the number of edges incident to it.
	\item Two graphs $G_1$ and $G_2$ are \textbf{isomorphic} if there exists a bijection
	$f: V_1 \rightarrow V_2$ such that every edge in $E_1$ corresponds to exactly one
	edge in $E_2$.  That is, $G_1$ is isomorphic to $G_2$ if we can relabel $G_2$ with
	the vertex names of $G_1$, and find that the edges(relationships) between each pair
	of vertices are the same. 
	\item To disprove isomorphism, look for properties that are not maintained.  For example, if one graph has a vertex of degree $6$ and the other does not, they cannot
	be isomorphic.
	\item $G' = (V', E')$ is a \textbf{subgraph} of $G = (V, E)$ if $V' \subseteq V$ and
	$E' \subseteq E$. 
	
	\item A graph $G$ is \textbf{bipartite} if its vertices can be split into sets $V_1, V_2$
	such that vertices in $V_1$ only have edges going to $V_2$, and vice versa.
	\item A graph is \textbf{k-colorable} if there is a way to assign $k$ colors to the vertices of the graph such that if a vertex is colored $c$, it is not adjacent to any other vertices of color $c$.
	\item Bipartite $\Leftrightarrow$ 2-colorable $\Leftrightarrow$ the graph has no odd-length cycles.
	\item Vertices $v_1$ and $v_2$ are \textbf{connected} if there is a path in the graph
	from $v_1$ to $v_2$.
	\item A graph $G$ is connected if every vertex has a path to every other vertex.
	\item A \textbf{connected component} is a connected subgraph of a graph.
	\item A \textbf{closed walk} is a set of vertices $v_0v_1\ldots v_k$ where each
	$v_i$ is adjacent to $v_{i+1}$ and where $v_0 = v_k$.  If each of the $v_i$ are
	distinct, a closed walk is called a \textbf{cycle}.
	\item An \textbf{Euler walk} is a walk on a graph that traverses every edge exactly once.  An \textbf{Euler tour} is an Euler walk where the start and end nodes are the same.
	\item A connected graph has an Euler tour $\Leftrightarrow$ every vertex has even
	degree.
	\item A \textbf{Hamiltonian cycle} is a cycle that visits every node exactly once.
	\item An undirected graph $G$ has \textbf{width} $w$ if the vertices can be
arranged in a sequence
%
\[
v_1,\ v_2,\ v_3,\ \ldots,\ v_n
\]
%
such that each vertex $v_i$ is joined by an edge to at most $w$
preceding vertices.  (Vertex $v_j$ \textit{precedes} $v_i$ if $j <
i$.)

\end{itemize}
	
\section{Stable Matching}
\textbf{Relevant material: } pset 5 -- problems 3\\
 recitation 7\\
 Midterm practice -- problem 6\\
 2014 -- problem 6
 
 \begin{itemize}
 \item Let $G=(V,E)$ be a bipartite graph, with left vertex set $L$ and right vertex set $R$.
Recall that for a subset $S$ of the vertices, $N(S)$ is the set of vertices which are adjacent to some vertex in $S$:
\[ N(S) = \{ r \in V \mid \{r, s\} \in E \text{ for some } s \in S \}. \]

\textbf{Halls' theorem} says that if for every subset $S$ of $L$ we have $|N(S)| \geq |S|$, then there is a matching in $G$ that covers $L$.
\item The Mating Algorithm returns a matching where every boy is matched with his optimal mate.
\item The Mating Algorithm returns a matching where every girl is matched with her pessimal mate.
\item If the boy optimal matching is the same as the boy pessimal matching, there is only one stable matching
\end{itemize}


\end{document}