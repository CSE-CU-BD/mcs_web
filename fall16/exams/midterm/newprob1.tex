\documentclass[12pt,twoside]{article}
\usepackage{light}
\usepackage{subfigure}
\usepackage{enumitem}
\usepackage{graphicx}
\usepackage{amsmath}
\usepackage{amsfonts}


\hidesolutions
\showsolutions

\begin{document}

\begin{problem}{10}
\bparts
\ppart{5}
Prove or disprove $\neg (A \wedge B) \iff (\neg A \vee \neg B)$. \emph{Hint}: Use a truth table.
\solution{
\[                                                                                                                                                                                
\begin{array}{c|c|c|c|c|c|c}                                                                                                                                                   
A         & B          &A \wedge B  & \neg (A \wedge B) &\neg A  &\neg B  &\neg A \vee \neg B\\ \hline
\true     &\true       & \true      & \false            & \false & \false & \false \\
\true     &\false      & \false     & \true             & \false & \true  & \true \\                                                                                                         
\false    &\true       & \false     & \true             & \true  & \false & \true \\
\false    &\false      & \false     & \true             & \true  & \true  & \true \\
\end{array}
\]

Comparing the fourth and last colums, we see that the statement is true. \emph{Note:} In fact, this is one of \emph{DeMorgan's Laws}.
}

\ppart{5}
Translate the following statements from English into propositional logic or vice versa. 
\begin{enumerate}
\item If $n > 1$, then there is always at least one prime $p$ such that $n < p < 2n$.
\emph{Hint}: Let $\text{Prime}(p) := p \text{ is a prime}$
\solution{
The domain is $\mathbb{Z}$.
\begin{equation*}
\forall n. (n > 1) \Rightarrow (\exists p. \text{Prime}(p)  \wedge (n < p) \wedge (p < 2n))
\end{equation*}

\emph{Note:} This is known as \emph{Bertrand's Postulate}
}

\item The domain is $\mathbb{N}$. $\forall m \exists p>m. \text{Prime}(p) \wedge \text{Prime}(p+2)$
\solution{
There are infinitely many primes $p$ such that $p+2$ is also prime. \emph{Note:} This is known as the \emph{Twin Prime Conjecture}
}


\item Let $T$ be the set of TA's, $S$ be the set of students, and $G(x,y) := x \text{ grades } y\text{'s exam}$
\begin{equation*}
\exists t \in T \forall s \in S. G(t,s)
\end{equation*}
\solution{
One TA will grade all of the exams. \emph{Note:} This is not a famous law, theorem, or conjecture, but would be quite impressive.
}
\end{enumerate}
\eparts
\end{problem}


\end{document}