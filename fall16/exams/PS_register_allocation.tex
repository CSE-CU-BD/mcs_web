\documentclass[problem]{mcs}

\begin{pcomments}
  \pcomment{PS_register_allocation}
  \pcomment{subsumed by CP_ version}
  \pcomment{f01-ps4-4}
  \pcomment{revised from Rosen 7.8.21}

\end{pcomments}

\pkeywords{
  graphs
  graph_coloring
}

%%%%%%%%%%%%%%%%%%%%%%%%%%%%%%%%%%%%%%%%%%%%%%%%%%%%%%%%%%%%%%%%%%%%%
% Problem starts here
%%%%%%%%%%%%%%%%%%%%%%%%%%%%%%%%%%%%%%%%%%%%%%%%%%%%%%%%%%%%%%%%%%%%%

\begin{problem}
My computer program has seven variables, {\tt t,u,v,w,x,y,z}, and
computes in 6 steps. The steps in which each variable is used are as
follows: {\tt t:} steps 1 through 6; {\tt u:} step 2; {\tt v:} steps 2
through 4; {\tt w:} steps 1,3 and 5; {\tt x:} steps 1 and 6; {\tt y:}
steps 3 through 6; {\tt z:} steps 4 and 5.  Each variable will need to
occupy the same index register at each one of the steps in which it is
going to be used; however during the steps when the variable is not
being used the register may be used by some other variable. How many
such registers are needed for my program? Explain how you arrived at
the answer.

\begin{solution}
This problem shows an application of graph coloring to
register allocation in a processor.  Consider a simple graph $G$ having
the seven variables as its nodes, and an edge between two variables
iff there is at least one step of the program during which both
variables are going to be used.  Figure~\ref{fig:vars} contains the
graph we are talking about if we make the assumption that a variable
must always reside in a single register.

\begin{figure}
\begin{center}
\unitlength=0.035pt
\input{ps4-4a.latex}
\end{center}
\caption{}
\label{fig:vars}
\end{figure}

Clearly, two variables have to occupy different index registers iff
they are going to be used simultaneously sometime during the
progam execution, that is, iff they are connected in~$G$. 

To correctly and economically assign variables to registers means to
map each variable to one register in a way that does not cause
collisions (two simultaneously used variables mapped to the same
register) and uses as few registers as possible. But (if we think of
registers as colors) this is the same as to color $G$ in a way that
does not cause neighboring vertices of the same color and uses as few
colors as possible. So, we have reduced the original problem to the
problem of coloring~$G$.

Figure~\ref{fig:varscol} shows such a coloring, using 5 colors. To see
that this is the minimum one, just notice that variables $t$, $v$,
$w$, $y$, and $z$ form a 5-vertex subgraph which is complete, and
hence cannot be colored with less than~5 colors.

\begin{figure}
\begin{center}
\unitlength=0.035pt
\input{ps4-4b.latex}
\end{center}
\caption{}
\label{fig:varscol}
\end{figure}

However, in a real compiler, our assumption that every variable occupies
at most one register would be overly restrictive because a
variable is backed up into memory every time it is evicted from the
register set and then reloaded from memory into the register set when
it needs to be used again. When it is reloaded there is no reason why
the same register must be used. For example, say we had a variable $x$
which is used in steps 1 and then later used in step 6. Then $x$
could be stored in register $r1$ for step 1 and some other
register $r2$ for step 6.  We can still use coloring to solve the
problem by treating $x$ as two separate variables: $x_{1}$ and
$x_6$.

If we apply this logic to the problem, we can color the resulting graph with
only four colors by changing node $w$ to GREEN in steps 1 and 3 and ORANGE
in step 5.  We can see that four colors are still required, because
in step 3 variables $t$, $v$, $w$ and $y$ are all in use.

\end{solution}

\end{problem}

%%%%%%%%%%%%%%%%%%%%%%%%%%%%%%%%%%%%%%%%%%%%%%%%%%%%%%%%%%%%%%%%%%%%%
% Problem ends here
%%%%%%%%%%%%%%%%%%%%%%%%%%%%%%%%%%%%%%%%%%%%%%%%%%%%%%%%%%%%%%%%%%%%%

\endinput
