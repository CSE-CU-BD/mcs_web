\documentclass[handout]{mcs}

\begin{document}

\renewcommand{\reading}{
  Chapter~\bref{number_theory_chap}.\ ~\bref{modular_arithmeric_sec}.\ \emph{Modular Arithmetic}\ through~\bref{RSA_sec}.\ \emph{RSA Encryption}
  }

\problemset{5}

\begin{staffnotes}
Lectures covered: Number Theory: Modular Arithmetic, Number Theory: RSA Encryption
\end{staffnotes}

%%%%%%%%%%%%%%%%%%%%%%%%%%%%%%%%%%%%%%%%%%%%%%%%%%%%%%%%%%%%%%%%%%%%%
% Problems start here
%%%%%%%%%%%%%%%%%%%%%%%%%%%%%%%%%%%%%%%%%%%%%%%%%%%%%%%%%%%%%%%%%%%%%

%Modular Arithmetic

\pinput{PS_check_factor_by_digits}
\pinput{PS_chinese_remainder_general}
%\pinput{PS_congruent_modulo_1000}
\pinput{PS_self-inverse_mod_p}


%Euler's Function

\pinput{PS_Euler_function_multiplicativity}
\pinput{FP_Euler_theorem_calculation}


%RSA

%\pinput{PS_RSA_key_implies_factoring}
\pinput{TP_RSA_reversed}		%Used in a lot of previous psets
\pinput{PS_RSA_correctness}
%\pinput{PS_Rabin_cryptosystem}

% stephan: if you need more material, here is a problem part that i wrote but never used
\begin{problem}
Compute
\[
\rem{2498754270^{184638463} 9234673466^{844759364}}{22}.
\]

\hint $18^9 \equiv 8 \mod 22$

\begin{solution}
First, we replace the bases of the exponents with their remainders.
\[
\rem{12^{184638463} 18^{844759364}}{22}
\]
Now, let's examine the remainders of the first few powers of $12$.

\begin{align*}
  \rem{12}{22} &= 12 \\
  \text{rem}{12^2}{22} &= 12 \\
  \rem{12^3}{22} &= 12 \\
  \cdots
\end{align*}

The remainder is always $12$.  So we can now write
\[
\rem{12\cdot 18^{844759364}}{22}.
\]

Now let's look at the remainders of powers of $18$.
\begin{align*}
  \rem{18}{22} &= 18 \\
  \rem{18^2}{22} &= 16 \\
  \rem{18^3}{22} &= 2 \\
  \rem{18^4}{22} &= 14 \\
  \rem{18^5}{22} &= 10 \\
  \rem{18^6}{22} &= 4 \\
  \rem{18^7}{22} &= 6 \\
  \rem{18^8}{22} &= 20 \\
  \rem{18^9}{22} &= 8 \\
  \rem{18^{10}}{22} &= 12 \\
  \rem{18^{11}}{22} &= 18 \\
  \cdots
\end{align*}

So it repeats after $11$ steps.  This means we can keep subtracting
$11$ from the exponent until the computation becomes feasable.  In
other words, we can replace the exponent with its remainder mod 11.
\[
  \rem{12 \cdot 18^9}{22} &= \rem{(12)(8)}{22} = 8
\]

\end{solution}
\end{problem}

%%%%%%%%%%%%%%%%%%%%%%%%%%%%%%%%%%%%%%%%%%%%%%%%%%%%%%%%%%%%%%%%%%%%%
% Problems end here
%%%%%%%%%%%%%%%%%%%%%%%%%%%%%%%%%%%%%%%%%%%%%%%%%%%%%%%%%%%%%%%%%%%%%
\end{document}
