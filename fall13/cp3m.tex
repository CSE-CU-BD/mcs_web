\documentclass[handout]{mcs}

\begin{document}

\inclassproblems{3, Mon.}

%%%%%%%%%%%%%%%%%%%%%%%%%%%%%%%%%%%%%%%%%%%%%%%%%%%%%%%%%%%%%%%%%%%%%
% Problems start here
%%%%%%%%%%%%%%%%%%%%%%%%%%%%%%%%%%%%%%%%%%%%%%%%%%%%%%%%%%%%%%%%%%%%%

\pinput{CP_proving_basic_set_identity}
\pinput{CP_subset_take_away}

%\textbf{Supplemental problem:}

\pinput{CP_set_pairing}

\begin{staffnotes}
Tell students not to worry if they don't get through the next two
problems.  The first one about finite ordinals gives more practice
with the membership and containment relations, but we won't make
further use of the ordinals.  The second is more practice with
predicate formulas than learning about sets.
\end{staffnotes}

\textbf{Extra practice with set formulas}:

\pinput{TP_basic_set_formulas}

\instatements{\newpage}

\textbf{Supplemental problem:}

\pinput{CP_finite_ordinals}

%\pinput{PS_binary_relations_on_a_set}  renamed TP; stupid problem

%\pinput{CP_web_search}  on pset 3

%\pinput{CP_distributive-law-for-sets-by-WOP}  boring & cumbersome

%%%%%%%%%%%%%%%%%%%%%%%%%%%%%%%%%%%%%%%%%%%%%%%%%%%%%%%%%%%%%%%%%%%%%
% Problems end here
%%%%%%%%%%%%%%%%%%%%%%%%%%%%%%%%%%%%%%%%%%%%%%%%%%%%%%%%%%%%%%%%%%%%%
\end{document}

