\documentclass{article}

\usepackage{amsmath}
\usepackage{amsfonts}
\usepackage{amsthm}
\usepackage{amssymb}
\usepackage{enumerate}
\usepackage{fancyhdr}
\usepackage{graphicx}
\usepackage[margin=1.2 in]{geometry}

\pagestyle{fancy}

\begin{document}

\lhead{6.042 Fall 2013}
\chead{Midterm 1 Grading Rubric}
%Mostly written by Elizabeth S.

\textbf{This rubric is still slightly tentative and may be edited in the coming weekend.}
\section*{Problem 1}
Each subquestion (b through h) is worth 3 points, for a total of 21 points. 
\\1 point is taken off for an incorrect answer component- i.e. leaving out a correct answer or having an extra incorrect answer. If this would result in negative points, then 0 points are awarded. 
\\ \textbf{Examples:}
\\ $\bullet$ The correct answer for part (g) was ``3 and 4". An answer of ``4" would be worth 2 points. An answer of ``1, 4" would be worth 1 point.
\\ $\bullet$ The correct answer for part (h) was ``2". An answer of ``1, 2, 3, 4" would be no points. 
\\Scores are capped at 20.  

\section*{Problem 2}
Part (a) was worth 9 points and part (b) was worth 6 points.
\\Full points were awarded for correct numerical answers, regardless of work shown. 
\\Partial credit was awarded for correct work, such as stating $\phi(97) = 96$.
\\Minimal points were deducted for basic math errors, such as saying ``$3^3 = 81"$. 
\\2 points were taken off for an answer of -1 to part (b). The remainder must be in the range [0, 97). 
\\4 points were taken off for an answer that used the hint but got an answer other than -1 or 96.

\section*{Problem 3}
Part (a) is worth 2 points, (b) is worth 9 points, and (c) is worth 9 points. 
\\ $\bullet$ For part (b), if the induction hypothesis is \fbox{$s \in$ RecMatch $\implies s \in$ GoodCount}, the implication technically should not be there. However, full points were awarded. This decision was back-implemented on the second day of grading, so if your test has not been changed, talk to staff. 
\\3 points are taken off if there is a $\forall$, which indicates a lack of understanding how the induction hypothesis will be applied. 
\\3 points are deducted for trying to do the constructor case in the statement of the induction hypothesis. Thus, anything in terms of $[s] t$ loses these three points. 
\\ $\bullet$ For part (c), it is difficult to award partial credit. The answer must somehow mention taking the length of the strings- if there is no indication of what natural number $n$ we're inducting over, no points.
\\2 points were taken off for switching GoodCount and RecMatch but otherwise being correct.


\section*{Problem 4}
Part (a) is worth 5 points, (b) is worth 7 points, and (c) is worth 13 points. 
\\ $\bullet $ For part (a): A common mistake is stating the inductive hypothesis to be ``if p $\vert$ a*b then p$\vert$a or p$\vert$b"
\\ This hypothesis lacks $n$, making it impossible to induct on. 
\\We also cannot define the product of integers as $n$ and then attempt to induct on $n=1$, $n=2$, etc.
\\Due to unclear wording of the problem, full credit will be awarded to induction hypotheses using a product of two numbers e.g. \fbox{$P(n) ::= \forall a, n \in \mathbb{Z}. \forall p \in $primes$. p \vert a*n \implies p \vert a$ or $p \vert n$}  or \\ \fbox{$\forall n \in $primes$. n \vert a*b \implies n \vert a $ or $n\vert b$}. 
\\ $\bullet$ For part (b): Full credit is awarded for correctly checking the base case, even if the induction hypothesis is incorrect.
\\ $\bullet$ For part (c): It is impossible to complete the proof with an incorrect hypothesis, but up to half credit will be awarded for solid attempts demonstrating understanding of induction.


\section*{Problem 5}
Part (a) is worth 4 points, part (b) is worth 10 points, and part (c) is worth 6 points. 
\\ $\bullet$ For part (a), a basic math error (e.g. ``21+1+20+2= 42") was a 1 point deduction.
\\Stopping at (20, 21) rather than (22, 22) was a common mistake. Thus, an answer of 41 was awarded 2 out of 4 points.
\\ $\bullet$ Many people in part (b) got confused in proving $b-r \geq 0$. 
\\Something like ``Applying rule (i) gives b-(r+1) $\geq 0 \rightarrow b-r \geq 1 \rightarrow b-r \geq 0$" is backwards in logic. 
\\Our starting condition is $b \succ r$ and we need to show as a result that $b - (r+1) \geq 0$.  
\\An unsuccessful attempt at casework was awarded 4 of 10 points.  
\\ $\bullet$ For part (c), a correct answer stated that b-r was strictly decreasing, therefore after a finite number of steps, the condition b-r $\succ$ 0 would no longer be satisfied, and thus have reached a stop state.   
\\ $\bullet$ Other common errors: Assuming that the steps (i) through (iv) were applied in a cycle, the start state was 16 red balls and 10 blue balls, or for part (c), that we could choose which steps were applied, e.g. repeatedly applying step (i).  

\end{document}