\documentclass[quiz]{mcs}

\renewcommand{\exampreamble}{   % !! renew \exampreamble
  \begin{tabular}{l}
    \textbf{Circle your}\quad   \teaminfo
  \end{tabular}

  \begin{itemize}

  \item
   This exam is \textbf{closed book} except for a 2-sided cribsheet.
   Total time is 60 minutes. 

  \item
   Write your solutions in the space provided.  If you need more
   space, write on the back of the sheet containing the problem.

%   Please keep your entire answer to a problem on that problem's page.
   
   \item In answering the following questions, you may use without
     proof any of the results from class or text (unless explicitly
     instructed otherwise).

\iffalse
  \item
   GOOD LUCK!
\fi

  \end{itemize}}

\begin{document}

\midterm{November 6}

\begin{staffnotes}
RSA, Ch. 8.11--8.12; digraphs, posets; simple graphs: matching,
coloring, connectivity, trees; sums; asymptotics.
\end{staffnotes}

\examspace

%%%%%%%%%%%%%%%%%%%%%%%%%%%%%%%%%%%%%%%%%%%%%%%%%%%%%%%%%%%%%%%%%%%%%
% Problems start here
%%%%%%%%%%%%%%%%%%%%%%%%%%%%%%%%%%%%%%%%%%%%%%%%%%%%%%%%%%%%%%%%%%%%

\pinput[points = 20, title= \textbf{Bipartite Averages}]{FP_bipartite_matching_sex}

\examspace
\pinput[points = 15, title= \textbf{Asymptotic Partial Orders}]{FP_asymptotics_define_functions}

\examspace
\pinput[points = 15, title= \textbf{Overused bogus coloring induction}]{FP_bogus_coloring_proof}

\examspace
\pinput[points = 15, title= \textbf{Graphs multi-choice, needs work}]{FP_multiple_choice_unhidden}

\examspace
\pinput[points = 15, title= \textbf{Coloring F11.S12.finals?}]{CP_coloring}

\examspace
\pinput[points = 15, title= \textbf{tennis poset concepts}]{MQ_tennis_match_partial_order}

\begin{center}
\textbf{one of the following two:}
\end{center}

\examspace
\pinput[points = 15, title= \textbf{Bipartite Matching}]{MQ_matching}

%\examspace
\pinput[points = 15, title= \textbf{Bipartite Matching}]{MQ_matching_afternoon}

\examspace
\pinput[points = 15, title= \textbf{Partial Order}]{MQ_partial_order_on_123}

\examspace
\pinput[points = 10, title= \textbf{Big Oh}]{MQ_O_nsq_ln_n}

\iffalse

\insolutions{
\medskip

The next problem is a simpler variant of
  Problem~\bref{CP_erasable_strings} which appeared on the
  \href{http://courses.csail.mit.edu/6.042/fall13/cp4f.pdf}{ Friday,
    Week 4 class problems}.}

\pinput[points = 20, title = \textbf{Structural Induction}]{MQ_bracket_good_count}

\fi




%%%%%%%%%%%%%%%%%%%%%%%%%%%%%%%%%%%%%%%%%%%%%%%%%%%%%%%%%%%%%%%%%%%%%
% Problems end here
%%%%%%%%%%%%%%%%%%%%%%%%%%%%%%%%%%%%%%%%%%%%%%%%%%%%%%%%%%%%%%%%%%%%%
\end{document}
