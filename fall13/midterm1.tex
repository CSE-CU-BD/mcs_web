\documentclass[quiz]{mcs}

\renewcommand{\exampreamble}{   % !! renew \exampreamble
  \begin{tabular}{l}
    \textbf{Circle your}\quad   \teaminfo
  \end{tabular}

  \begin{itemize}

  \item
   This exam is \textbf{closed book} except for a 2-sided cribsheet.
   Total time is 60 minutes. 

  \item
   Write your solutions in the space provided.  If you need more
   space, write on the back of the sheet containing the problem.

%   Please keep your entire answer to a problem on that problem's page.
   
   \item In answering the following questions, you may use without
     proof any of the results from class or text (unless explicitly
     instructed otherwise).

\iffalse
  \item
   GOOD LUCK!
\fi

  \end{itemize}}

\begin{document}

\midterm{October 9}


%%%%%%%%%%%%%%%%%%%%%%%%%%%%%%%%%%%%%%%%%%%%%%%%%%%%%%%%%%%%%%%%%%%%%
% Problems start here
%%%%%%%%%%%%%%%%%%%%%%%%%%%%%%%%%%%%%%%%%%%%%%%%%%%%%%%%%%%%%%%%%%%%

\examspace

\insolutions{The following problem is a slight variant of a problem
  which appeared on the
  \href{http://courses.csail.mit.edu/6.042/spring13/midterm.pdf}{Spring13
    midterm} that was made available for review.}

\pinput[points = 20, title= \textbf{Predicates \& Relations}]{MQ_predicate_jections_F13}
%fixed typo in property 3: function [<= 1 out]


\examspace
\pinput[points = 15, title= \textbf{Remainder Arithmetic}]{TP_Fermats_Little_Theorem_F13}
%hint added for part b)

\examspace

\insolutions{
\medskip

The next problem is a simpler variant of
  Problem~\bref{CP_erasable_strings} which appeared on the
  \href{http://courses.csail.mit.edu/6.042/fall13/cp4f.pdf}{ Friday,
    Week 4 class problems}.}

\pinput[points = 20, title = \textbf{Structural Induction}]{MQ_bracket_good_count}

\examspace

\insolutions{
\medskip

The following problem is a small variation of
Problem~\bref{PS_bogus_prime_divides_integer_product} which appeared
on the \href{http://courses.csail.mit.edu/6.042/fall13/cp4m.pdf}{
  Monday, Week 4 class problems}.  The version actually given in class
used the word ``product'' instead of ``\emph{finite} product,'' which
confused many students who misinterpreted ``product'' to mean
``product of two factors.''}

\pinput[points = 25, title= \textbf{Induction}]{MQ_divide_product_induction}
%should have said ``finite'' product of integers

%\examspace
%\pinput[points = 25, title= \textbf{Induction}]{TP_gcd_linear_combination_induction}

%\examspace
%\pinput[points = 20, title= \textbf{Preserved Invariant}]{MQ_binary_gcd}

\examspace
\pinput[points = 20, title= \textbf{Preserved Invariant}]{MQ_red_blue_machine}

%\examspace
%\pinput[points = 20, title= \textbf{Preserved Invariant}]{MQ_fast_exponentiation}


%\examspace
%\pinput[points = 20, title= \textbf{Well Ordering}]{TP_divide_product_wop}


%%%%%%%%%%%%%%%%%%%%%%%%%%%%%%%%%%%%%%%%%%%%%%%%%%%%%%%%%%%%%%%%%%%%%
% Problems end here
%%%%%%%%%%%%%%%%%%%%%%%%%%%%%%%%%%%%%%%%%%%%%%%%%%%%%%%%%%%%%%%%%%%%%
\end{document}
