\documentclass{article}

\usepackage{amsmath}
\usepackage{amsfonts}
\usepackage{amsthm}
\usepackage{amssymb}
\usepackage{enumerate}
\usepackage{fancyhdr}
\usepackage{graphicx}
\usepackage[margin=1.2 in]{geometry}

\pagestyle{fancy}

\begin{document}

\lhead{6.042 Fall 2013}
\chead{Midterm 2 Grading Rubric}
%Mostly written by Elizabeth S.

\section*{Problem 1}
Worth 10 points. 
Students that indicated that they needed to find a bottleneck but did not actually find one were awarded 5 points unless they gave the wrong definition of bottleneck. Finding an incorrect bottleneck but having an otherwise correct method was also 5 points. 0-3 points for mentioning bottleneck but defining it incorrectly.

\section*{Problem 2}
Worth 10 points. \\
Little credit was awarded to attempts to prove that there does not exist $c, n_0$ such that for $n>n_0$, $f(n) < c\cdot g(n)$. \\
Students that had the correct idea of using limits generally got the entire problem correct. Up to half credit was awarded for correct set-ups that got algebraically confused. 


\section*{Problem 3}
Worth 15 points. \\
The simplest correct answer was the length 5 cycle with equal edge weights. Other possible graphs are correct. \\
7 points were deducted for a non-simple graph. \\
5 points were deducted if the student drew multiple MSTs but not the required 5. \\
1 point was deducted if no indication of the edge weights were given. \\
These deductions are additive. 

\section*{Problem 4}
Part (a) is worth 3 points, part (b) is worth 5 points, and part (c) is worth 7 points. \\
\textbf{Parallel Time:} 1 points was awarded for an answer of 5 and the indication that the student forgot the empty set or counted edges rather than nodes of the antichain. \\
\textbf{Maximum antichain:} Full credit was awarded for indicating all subsets of size 2 (or 3). \\
4 points were awarded for answer of just 10.  \\
2 points were awarded for an answer of all subsets of size n, where n was not 1, 2, or 3. \\
\textbf{Min = Max:} This question was asking for the proof of this statement specific to the problem. Simply stating that all the elements were incomparable and required their own processor was not enough, as it was key that all the elements occurred at the same time step. \\
Full points were awarded for describing a greedy strategy or saying that at each step you had to take the subsets of one size.\\
1 or 2 points were deducted if student had the correct idea but explained it unclearly. \\
3 points were awarded for stating that elements of an antichain are incomparable or for the definition of an antichain. 

\section*{Problem 5}
Part (a) is worth 5 points, part (b) is worth 10 points, and part (c) is worth 10 points. \\
\textbf{Counterexample:} No partial credit was awarded for part a. A common mistake was ignoring a condition of the claim. G must have a vertex with degree strictly less than 2, so a triangle wouldn't be a counterexample. All of the vertex degrees must be less than equal to two, so a triangle with one vertex connected to another point will not work. \\
Another common source of confusion is the definition of k-colorability. If something is 1-colorable, then it is also 2-colorable. This is different from $\chi$() which is the necessary and sufficient number of colors needed to color a graph. \\
\textbf{Bogus Line:} For part (b), partial credit was awarded for underlining the incorrect sentence but mentioning the crux of the problem, that $v$ may have degree 0. \\
\textbf{Correction:} For part (c), 6 points were awarded for circling one of the two correct answers. 3 points were deducted for each incorrect answer circled. e.g. circling 1, 2, and 4 would be 7 points. Circling 1 and 2 would be 3 points. 

\section*{Problem 6}
Part (a) is worth 5 points and part (b) is worth 20 points.  \\
\textbf{Example:} 3 points were awarded for a digraph that had multiple possible rankings but did not satisfy the tournament condition. \\
.\\
Part (b) was further broken down into 5 points for the induction hypothesis, 5 points for the base case, and 10 points for the inductive step. \\
\textbf{Induction Hypothesis}: No points were awarded for an induction hypothesis lacking $n$ or clearly incorrect. Some partial credit given to induction hypothesis that could be used to prove the claim but required extra steps. 2 points if they indicate some sort of strong induction.\\
\textbf{Base Case}: Mostly all or nothing. Varying credit was given based on the wording. Points were deducted when it seemed like the student was proving a base case using a specific tournament digraph with 2 vertices. \\
\textbf{Inductive Step}: Grading was strict on this one, since ample partial credit was given for illustrating understanding of induction in parts (a) and (b). \\ 
For partial credit, students that attempted to insert the new node in the middle of the previous ranking got 0-2 points depending on the development of the rest of their proof. Students who figured out the correct method of node insertion but lacked adequate justification received 5-9pts.  
All other solutions received 0pts.
\end{document}