%finalS17

\documentclass[quiz]{mcs}

%\renewcommand{\examspace}{}  %DISABLE EXAM SPACINGG

\renewcommand{\exampreamble}{
  \begin{itemize}

  \item
   This exam is \textbf{closed book} except for two 2-sided cribsheets.
   Total time is 180 minutes.

  \item
   Write your solutions in the space provided.  If you need more
   space, \textbf{write on the back} of the sheet containing the
   problem.

   \item In answering the following questions, you may use without
     proof any of the results from class or text.

  \end{itemize}}

\begin{document}

\final

%%%%%%%%%%%%%%%%%%%%%%%%%%%%%%%%%%%%%%%%%%%%%%%%%%%%%%%%%%%%%%%%%%%%%
% Problems start here
%%%%%%%%%%%%%%%%%%%%%%%%%%%%%%%%%%%%%%%%%%%%%%%%%%%%%%%%%%%%%%%%%%%%%

\begin{center}
{\LARGE \textbf{Short-Answer Questions}}
\end{center}

\emph{\large The following questions are short-answer.  The graders
  will not read explanations, so do not spend time including them.}

%\examspace
%\pinput[points = 12, title= \textbf{Relations, Probability}]
%{FP_probability_relations_short_answer}

%\examspace
\pinput[points = 8, title = \textbf{Simple Graphs \& Trees}]
{FP_simple_graphs_trees_short_answer_S17}

\examspace
\pinput[points = 4, title= \textbf{Quantifiers}]{FP_infinitely_often_quantifiers}

\examspace[0.7in]

\iffalse
\pinput[points = 3, title = \textbf{Random Walk, Stationary Distribution}]
       {FP_random_walk_multiple}

\examspace\fi

\pinput[points = 5, title = \textbf{Logical Injections}]{FP_logical_jections}

\examspace
\pinput[points = 7, title = \textbf{Number Theory}]
       {FP_numbers_short_answer_F15}

%% \examspace
%% \pinput[points = 10, title = \textbf{Relations and Predicates}]
%%        {FP_relation_properties_expressions_final_f13}

%\pinput[points = 10, title= \textbf{Cardinality}]{FP_card_bij}
%\examspace

%\pinput[points = 7, title = \textbf{GCD}]
%       {FP_gcd_TF}
%\examspace

\examspace
\pinput[points = 8, title= \textbf{Counting}]{FP_counting_given_answers}

%% \begin{staffnotes}
%%  conflicts with FP\_sampling\_wafers\_S13
%% \end{staffnotes}
%% \pinput[points = 9, title = \textbf{Sampling, True/False}]
%%        {TP_sampling_perturbed}

\examspace
\begin{center}
{\LARGE \textbf{Proof and Concept Questions}}
\end{center}

%\pinput[points = 12, title = \textbf{Structural Induction}]
%{FP_binary_tree_induction}
%\examspace

%\pinput[points = 6, title = \textbf{Structural Induction}]
%{FP_red_black_tree_induction}

\pinput[points = 10, title = \textbf{Structural Induction}]{FP_arith_trig_functions}

\examspace
\pinput[points = 8, title= \textbf{Probable Satisfiability}]{CP_probable_satisfiability}

%\pinput[points = 8, title = \textbf{Modular Sum of Digits}]
%       {FP_check_factor_by_digits}

%\examspace
%\pinput[points = 25, title = \textbf{Modular Arithmetic}]{FP_order_modn}

\examspace
\pinput[points = 10, title = \textbf{Connectivity,
    Induction}]{FP_connected_induction}

%\pinput[points = 6, title =
%  \textbf{Counting}]{FP_counting_poker_high_cards}

%elim from S12,final
%\examspace
%\pinput[points = 10, title = \textbf{Probability}]{PS_probabilistic_proof}

\examspace
\pinput[points = 8, title= \textbf{Conditional Probability}]
       {FP_conditional_prob_inequality}
\examspace

%\pinput[points = 10, title= \textbf{Expectation}]{MQ_infinite_repeat}

\examspace
\pinput[points = 8, title = \textbf{Expectation, Sums}]{PS_expected_gain}

%\pinput[points = 7, title = \textbf{Markov's Bound}]{FP_hot_cows_markov}

%\examspace
%\pinput[points = 12, title = \textbf{Variance}]{PS_suit_variance}

\examspace
\pinput[points = 10, title = \textbf{Chebyshev Bound}]{FP_hot_cows_chebyshevS15}

%\pinput[points = 10, title= \textbf{Random Walk}]{CP_simple_google_graph}

\examspace
\pinput[points = 8, title = \textbf{Random Walk, Expectation}]
       {FP_ran3p_graph}

\examspace
\pinput[points = 6, title =
  \textbf{Sampling \& Confidence}]{FP_sampling_wafers_S13}


%%%%%%%%%%%%%%%%%%%%%%%%%%%%%%%%%%%%%%%%%%%%%%%%%%%%%%%%%%%%%%%%%%%%%
% Problems end here
%%%%%%%%%%%%%%%%%%%%%%%%%%%%%%%%%%%%%%%%%%%%%%%%%%%%%%%%%%%%%%%%%%%%%
\end{document}

\iffalse

\begin{staffnotes}
too hard for final?  hard to grade?:
\end{staffnotes}

\pinput[points = 10, title= \textbf{Minimum Spanning Trees}]{CP_maxweight_edge}

\pinput{FP_m_envelopes_induction}  %cumbersome way to discuss binary represntation

%ps3
\pinput{PS_palindromes}   too many new given facts to understand
\pinput{FP_rogue_pair}

%ps4
\pinput{PS_Rabin_cryptosystem}  %not usable -- see comments in file
\pinput{PS_team_division}   %trite; lots of perturbations done already
\pinput{PS_Euler_theorem_not_rel_prime}  depends on Euler Thm which was not covered

\pinput[points = 10, title= \textbf{Cardinality}]{TP_cardinality_class}
\pinput[points = 10, title= \textbf{Cardinality}]{MQ_set_cardinality}
\pinput{FP_cardinality}  

\pinput{FP_structural_ind_polynomials}  %induction step so simple it is confusing.

\pinput{MQ_task_parallel_scheduling_v1}  %routine finger exercise


\begin{staffnotes}
OK
\end{staffnotes}
\pinput[points = 10, title= \textbf{Cardinality}]{FP_infinite_sequence_injection} 

\begin{staffnotes}
needs figures
\end{staffnotes}
\pinput[points = 10, title= \textbf{Partial Orders}]{TP_which_are_partial_orders}

\pinput[points = 10, title= \textbf{Partial Orders}]{FP_partial_order_short_answer}

\pinput[points = 10, title= \textbf{Partial Orders}]{TP_equivalence_relations_partial_orders}

\pinput[points = 10, title= \textbf{Matching}]{MQ_degree_constrained}

\begin{staffnotes}
\textbf{oy}
\end{staffnotes}
\pinput[points = 10, title= \textbf{Asymptotics}]{PS_asymptotics_table}

\begin{staffnotes}
repeat? if so, could perturb or skip
\end{staffnotes}
\pinput[points = 10, title= \textbf{Pigeonholes}]{PS_monochromatic_rectangle}

%\begin{staffnotes}
%\textbf{ARM; maybe:}
%\end{staffnotes}
%\pinput[points = 10, title= \textbf{Logic; Sets}]{PS_size_n_set_formula}

%\pinput[points = 10, title= \textbf{State Machines}]{MQ_red_blue_machine}

%% \begin{problem}
%% \TBA{Win-Lose Lemma for stable marriage?}
%% \end{problem}

%\pinput[points = 10, title= \textbf{Partial Orders}]{MQ_tennis_match_partial_order}

\fi
