%candidate probs final S17

\documentclass[quiz]{mcs}

\renewcommand{\examspace}{}  %DISABLE EXAM SPACING

\renewcommand{\exampreamble}{}

\begin{document}

\final

%%%%%%%%%%%%%%%%%%%%%%%%%%%%%%%%%%%%%%%%%%%%%%%%%%%%%%%%%%%%%%%%%%%%%
% Problems start here
%%%%%%%%%%%%%%%%%%%%%%%%%%%%%%%%%%%%%%%%%%%%%%%%%%%%%%%%%%%%%%%%%%%%%

\begin{problem}
\begin{staffnotes}
F01 final

overlaps (subsumed?) FP\_counting\_given\_answers
\end{staffnotes}

We are going to classify different counting problems by figuring out
which ones have the same formula.  Here are a set of six formula.

\begin{center}
\begin{tabular}{rl}
$n^m$:       &\brule{1in}\\
$m^n$:       &\brule{1in}\\
$n!/(n-m)!$:    &\brule{1in}\\  %P(n,m)
$\binom{n-1+m}{m}$:&\brule{1in}\\       %C(n-1+m,m)
$\binom{n-1+m}{n}$:&\brule{1in}\\       %C(n-1+m,n)$
$2^{mn}$:     &\brule{1in}
\end{tabular}
\end{center}

For each problem part below, write its label---(i),(ii),\dots---on the line
next to the corresponding formula above.

\begin{enumerate}[(i)]

\item Number of ways of putting $m$ indistinguishable balls into $n$
distinguishable urns.

\item Number of ways of putting $m$ distinguishable balls into $n$
distinguishable urns.

\item Number of ways of creating $m$ letter words from an alphabet of size $n$,
with no letter used more than once ($m \leq n$).

\item Number of ways of creating $m$ letter words from an alphabet of size $n$,
where any letter can be repeated any number of times.

\item Number of permutations of $m$ indistinguishable stars and
$n-1$ indistinguishable bars.

\item Number of matrices of size $\sqrt{n} \times \sqrt{n}$ with $m$
possible values for each entry.  (Assume $n$ is a perfect square.)

\item Number of possible subsets of A, where $|A|=nm$.

\item Number of possible functions from set $A$ to $B$ where $|A|=n$ and $|B|=m$.

\item Number of relations from set $A$ to $B$ where $|A|=n$ and $|B|=m$.

\item Number of injections from set $A$ to $B$ where $\card{A}=m$ and
$\card{B}=n$, and $m \leq n$.

\item Number of natural number solutions of the equation $x_1 + x_2 + x_3
\cdots x_n = m$, where $m$ is a non-negative integer.  (A solution is
a non-negative integer vector of values for $x_1,x_2,\cdots x_n$.)

\item Number of ways of lining up $n$ red balls and $m-1$ blue balls

%\iffalse

\item
Number of ways of putting $m$ indistinguishable balls into $n$
distinguishable urns. $C(n-1+m,m)$

\item Number of ways of putting $m$ distinguishable balls into $n$
distinguishable urns. $n^m$

\item Number of ways of creating $m$ letter words from an alphabet of size $n$,
with no letter used more than once ($m \leq n$). $P(n,m)$

\item Number of ways of creating $m$ letter words from an alphabet of size $n$,
where any letter can be repeated any number of times. $n^m$

\item Number of ways of permutations of $m$ indistinguishable stars and
$n-1$ indistinguishable bars. $C(n-1+m,m)$

\item Number of matrices of size $\sqrt{n} \times \sqrt{n}$ with
$m$ possible values for each entry. $m^n$

\item Number of possible subsets of A, where $|A|=nm$. $2^{mn}$

\item Number of possible functions from set $A$ to $B$ where $|A|=n$ and
$|B|=m$. $m^n$

\item Number of binary relations between set $A$ and set $B$ where $\card{A}=n$ and
$\card{B}=m$. $2^{mn}$

\item Number of injections from set $A$ to $B$ where $|A|=m$ and $|B|=n$,
and $m \leq n$. $P(n.m)$

\item Number of natural number solutions of the equation $x_1 + x_2 + x_3
\cdots x_n = m$, where $m$ is a non-negative integer.  (A solution is
a non-negative integer vector of values for $x_1,x_2,\cdots x_n$.)
$C(n-1+m,m)$

\item Number of ways of lining up $n$ red balls and $m-1$ blue balls. $C(n-1+m,n)$
%\fi

\end{enumerate}

\begin{solution}
\begin{center}
\begin{tabular}{rl}
$n^m$: & b d\\
$m^n$: & f h \\
$n!/(n-m)!$: & c j \\  %P(n,m)
$\binom{n-1+m}{m}$: & a e k \\  %C(n-1+m}{m}
$\binom{n-1+m}{n}$: & l \\  %C(n-1+m}{n}
$2^{mn}$: &  g i\\
\end{tabular}
\end{center}
\end{solution}

\end{problem}


\begin{problem}
\begin{staffnotes}
Graphs, Logic, Probability

F01 final
\end{staffnotes}

Suppose we construct a simple undirected graph $G$ with vertex set
$V=\set{1,2,\dots,n}$, as follows:

\begin{itemize}
\item
First, we color the nodes: For each $x\in V$, we flip a coin twice. 
If we get heads in both tosses, we paint $x$ blue; otherwise, we paint 
$x$ red. 

\item
Next, we draw the edges: For each unordered pair $\set{x,y}$ of
distinct vertices from $V$, we roll a die. If we get~$6$, we draw edge
$\edge{x}{y}$ connecting $x$ and $y$; otherwise, $x$ and $y$ remain
unconnected.
\end{itemize} 

The coin and the die are both fair; every toss and every roll occurs
independently from any other toss and any other roll. 

For a vertex $x\in V$, let $B(x)$ denote the property ``$x$ is blue'' and
$R(x)$ denote the property ``$x$ is red''. Similarly, for vertices $x,y\in
V$, let $E(x,y)$ denote the property ``there is an edge connecting $x$ and
$y$''. For example, the formula
\[
E(1,2) \QAND R(1)  \QAND \forall x.\, (x \neq 1) \QIMP B(x)
\]
denotes the property ``there is an edge connecting nodes $1$~and~$2$,
node~$1$ is red, and all other nodes are blue''.

Note that \emph{$E(x,x)$ is false} for every $x \in V$, because a simple
graph does not have self-loops.

\bparts
\ppart
For each one of the following properties, write down a formula that
denotes that property.  Only symbols from the list
\[
R\ B\ E\ =\ \neq\ \forall\ \exists\ \QAND\ \QOR\ \neg\ \QIMP\ \QIFF\ x\ y\ .\ ,\ ) \ (
\]
can appear in your formula.

\begin{enumerate}[(i)]

%% \iffalse

\item \label{item-first} ``all nodes are red'': 

\begin{solution}
$(\forall x)R(x)$
\end{solution}

\item
``all nodes are blue'': 

\begin{solution}
$\forall x.\, B(x)$\hrulefill
\end{solution}

\item ``not all nodes are blue'': 

\item ``at least one node is red'': 

\begin{solution}
Any of the two: $\neg(\forall x)B(x)$, $(\exists x)R(x)$
\end{solution}

\item ``at least two nodes are red'': 

\begin{solution}
$(\exists x)(\exists y)\bigl( x\neq y \QAND R(x) \QAND R(y) \bigr)$
\end{solution}

\examspace[0.75in]

\item ``the graph is complete'' (contains every possible edge)

\begin{solution}
$\forall x\forall y\, x\neq y \QIMP E(x,y)$
\end{solution}

\examspace[0.75in]

\item ``the graph is \emph{not} complete'': 

\begin{solution}
$(\exists x)(\exists y)\bigl( x\neq y \QAND \neg E(x,y) \bigr)$
\end{solution}\hrulefill

\item ``the graph is empty'': 

\begin{solution}
$(\forall x)(\forall y) \neg E(x,y)$
\end{solution}

\item \label{item-fourth}
``the set of edges in the graph is \emph{not} empty'': 

\begin{solution}
$\exists x\exists y\, E(x,y)$
\end{solution}

\examspace[0.75in]
%% \fi

\item ``there exist at least two blue adjacent vertices'':

\begin{solution}
$\exists x\exists y.\, E(x,y) \QAND B(x) \QAND B(y)$
\end{solution}
\examspace[0.75in]

\item ``every red vertex is adjacent to every blue vertex'': 

\begin{solution}
$\forall x\, \forall y.\ \paren{R(x) \QAND B(y) \QIMP E(x,y)}$
\end{solution}

\examspace[0.75in]

\item  \label{item-difficult} ``no two adjacent vertices have the same color'': 

\begin{solution}
$(\forall x)(\forall y)\bigl( 
	E(x,y) 
	\QIMP
	\bigl( R(x) \QAND B(y) \bigr)
	\QOR
	\bigl( B(x) \QAND R(y) \bigr)
\bigr)$, or

$\QNOT(\exists x \exists y.E(x,y) \QAND R(x) \QAND B(y)\ )$
\end{solution}

\examspace[0.75in]

\end{enumerate}

%% \iffalse

\ppart For each one of the properties i--iv above, write a simple
formula in $n$ for the probability that $G$ has the property.

\begin{enumerate}[(i)]

\item all nodes are red
\begin{solution}
\brule{0.5in}	$(\tfrac{3}{4})^n$
\end{solution}
 
%\item	% all nodes are blue
%\begin{solution}
% 	$(\tfrac{1}{4})^n$
%\end{solution}\hrulefill
%
%$\lim_{n\to\infty}p(n)=$\solnospace{0}\hrulefill

%\item	% at least one node is red
%$p(n)=$\solnospace{$1 - (\tfrac{1}{4})^n$}\hrulefill
%
%$\lim_{n\to\infty}p(n)=$ \solnospace{1}\hrulefill

\item at least 2 nodes are red
\begin{solution}
 	$1 - (\tfrac{1}{4})^n - n\tfrac{3}{4}(\tfrac{1}{4})^{n-1}
	=
	1 - \tfrac{3n+1}{4^n}$
\end{solution}\brule{0.5in}

\item  the graph is complete
\begin{solution}
	$(\tfrac{1}{6})^{n(n-1)/2}$
\end{solution}\brule{0.5in}

%\item	% the graph is not complete
%$p(n)=$\solnospace{$1 - (\tfrac{1}{6})^{n(n-1)/2}$}\hrulefill
%
%$\lim_{n\to\infty}p(n)=$\solnospace{1}\hrulefill

%\item	 the graph is empty
%$p(n)=$\$(\tfrac{5}{6})^{n(n-1)/2}$\hrulefill
%
%$\lim_{n\to\infty}p(n)=$\solnospace{0}\hrulefill

\item the graph is not empty
\begin{solution}
$1 - (\tfrac{5}{6})^{n(n-1)/2}$
\end{solution}\brule{0.5in}

%\fi
\end{enumerate}

\ppart For each one of the following logical formulas, write a simple
expression in $n$ for the probability that $G$ satisfies the formula.

\begin{enumerate}

\item
$\exists x\, B(x)$ \brule{0.5in}

\begin{solution}
$1 - (\tfrac{3}{4})^n$
\end{solution}

\item  $\forall x\, [B(x) \QOR R(x)]$ \brule{0.5in}

\begin{solution}
1
\end{solution}

\item  $\forall x\, B(x) \QOR \forall x\, R(x)$\brule{0.5in}

\begin{solution}
\[
\paren{\frac{1}{4}}^n + \paren{\frac{3}{4}}^n
\]
\end{solution}

\item $\forall x\, B(x) \QAND \forall x\forall y\, \neg E(x,y)$\brule{0.5in}

\begin{solution}
\[
\bigl(\frac{1}{4}\bigr)^n\bigl(\frac{5}{6}\bigr)^{n(n-1)/2}
\]
\end{solution}

\end{enumerate}

%% \iffalse

\ppart What is the the probability that $G$ has
property~\eqref{item-difficult} of part~(a)?  Your answer need not be
in closed form.

\begin{solution}
Fix any $k\in\set{0,\dots,n}$, and any $k$ vertices $x_1,\dots,x_k\in
V$. The probability that $x_1,\dots,x_k$ are blue, the rest of the
vertices are red and there is no edge between a blue and a red vertex
is 
\begin{equation} \label{probab-1}
(\tfrac{1}{4})^k 
(\tfrac{3}{4})^{n-k}
(\tfrac{5}{6})^{k(n-k)}.
\end{equation}
Since there are $\tbinom{n}{k}$ different ways to fix $x_1,\dots,x_k$,
there are exactly $\tbinom{n}{k}$ outcomes (graphs) supporting the
event that exactly $k$ vertices are blue, and there is no edge between
a blue and a red vertex, each with
probability~\eqref{probab-1}. Hence, the probability of this event is 
\begin{equation} \label{probab-2}
\tbinom{n}{k}
(\tfrac{1}{4})^k 
(\tfrac{3}{4})^{n-k}
(\tfrac{5}{6})^{k(n-k)}.
\end{equation}
Finally, the event that no edge connects a blue and a red vertex is
the union of the $n+1$ disjoint events that we get by varying $k$ over 
$\set{0,\dots,n}$, each of them having probability~\eqref{probab-2}. 
Hence, the coloring is consistent with probability 
\[
\sum_{k=0}^{n}
\tbinom{n}{k}
(\tfrac{1}{4})^k 
(\tfrac{3}{4})^{n-k}
(\tfrac{5}{6})^{k(n-k)}.
\]
\end{solution}

%% \fi

\eparts
\end{problem}


%%%%%%%%%%%%%%%%%%%%%%%%%%%%%%%%%%%%%%%%%%%%%%%%%%%%%%%%%%%%%%%%%%%%%
% Problems end here
%%%%%%%%%%%%%%%%%%%%%%%%%%%%%%%%%%%%%%%%%%%%%%%%%%%%%%%%%%%%%%%%%%%%%
\end{document}
