\documentclass[quiz]{mcs}

\renewcommand{\exampreamble}{   % !! renew \exampreamble
   \teaminfo

  \begin{itemize}

  \item
    This exam is \textbf{closed book} except for a 2-sided cribsheet.
   Total time is 90 minutes.

  \item
   Write your solutions in the space provided.  If you need more
   space, write on the back of the sheet containing the problem.

 % following blank page.
%   Please keep your entire answer to a problem on that problem's page.

   \item In answering the following questions, you may use without
     proof any of the results from class or text.
     \iffalse (unless explicitly instructed otherwise).\fi


\iffalse
  \item
   GOOD LUCK!
\fi
  \end{itemize}}

\begin{document}

\midterm{March 14}

%%%%%%%%%%%%%%%%%%%%%%%%%%%%%%%%%%%%%%%%%%%%%%%%%%%%%%%%%%%%%%%%%%%%%
% Problems start here
%%%%%%%%%%%%%%%%%%%%%%%%%%%%%%%%%%%%%%%%%%%%%%%%%%%%%%%%%%%%%%%%%%%%

\begin{staffnotes}
\begin{verbatim}
TOPICS:
Normal Forms, Predicate Formulas     Ch.3.4.1, 3.5-6 (omit 3.4.2)
Predicate Logic                      Ch.3.7
Sets \& Sequences                    Ch.4-4.2
Binary Relations                     Ch.4.3-5
Induction                            Ch.5-5.3
State Machines                       Ch.6-6.3
Stable matching                      Ch.6.4
Recursive Data, Structural Induction Ch.7-7.4
Recursive Games
\end{verbatim}
\end{staffnotes}

% Notes:
% Relevant semesters: {spring17, fall16, spring16, spring15}


\pinput[points = 9, title=
  {\textbf{Jections}}]{MQ_simpsurj}
\examspace

\pinput[points = 16, title=
  \textbf{Logical Formulas, Induction}]{FP_induction_plus_2_S17}
\examspace

\pinput[points = 15, title= \textbf{Induction}]
       {FP_tetris_recurrence_induction}
\examspace

\pinput[points = 20, title=
  \textbf{State Machines}]{FP_sort_cyclic_shift_three}
\examspace

\pinput[points = 15, title=
  \textbf{Stable Marriage}]{CP_mating_ritual_proof}
\examspace

\pinput[points = 25, title=
  \textbf{Propositional Formulas, Structural Induction}]
         {FP_OR_AND_recursive_multivar}
%\examspace

%%%%%%%%%%%%%%%%%%%%%%%%%%%%%%%%%%%%%%%%%%%%%%%%%%%%%%%%%%%%%%%%%%%%%
% Problems end here
%%%%%%%%%%%%%%%%%%%%%%%%%%%%%%%%%%%%%%%%%%%%%%%%%%%%%%%%%%%%%%%%%%%%%
\end{document}

\iffalse

% USED RECENTLY
\pinput[points = 10, title=
  \textbf{Predicate Logic [USED S15 MIDTERM]}]{PS_predicate_calculus_power_of_two}


%% USED RECENTLY
\pinput[points = 10, title=
 \textbf{Propositional Connectives [USED S15 MIDTERM]}]{CP_XOR_AND_formulas}

\pinput[points = 8, title=
  \textbf{Predicate Logic \& Relations}]{MQ_predicate_jections_S14}

%uninspiring
\pinput[points = 10, title=
  \textbf{Induction}]{FP_3_exponent_inequality_induction}

%Asks for special case of already known more general result.  Not good
%for exam.
\pinput[points = 10, title=
  \textbf{Induction}]{CP_courtyard_tiling_corner}

%alternate S17.ps2 problem
\pinput[points = 10, title=
  \textbf{Induction}]{FP_sat_count_induction}

\pinput[points = 10, title=
  \textbf{Preserved Invariant}]{MQ_red_blue_machine}

%light weight multiple choice
\pinput[points = 10, title=
  \textbf{State Machines}]{MQ_state_machine_invariant_afternoon}


%%% USED RECENTLY
\pinput[points = 10, title=
\textbf{State Machines [USED S15 MIDTERM]}]{MQ_state_machine_buckets_S15}

%%% USED RECENTLY
\pinput[points = 10, title=
\textbf{Stable Marriage [USED S16 MIDTERM]}]{FP_stable_marriage_S16}

%% Both these stable marriage problems offer a basis for a nice
%% problem or two, but only checking or forcing uniqueness involves
%% little understanding.  The invariants do test understanding, but
%% the ``true/false/unsure'' format tends to upset students.

\pinput[points = 10, title=
  \textbf{Stable Marriage}]{MQ_stable_matching_unique}

\pinput[points = 10, title=
  \textbf{Stable Marriage}]{FP_stable_marriage_S16}


%%  [USED S15 MIDTERM, S15 PS4]
\pinput[points = 10, title=  \textbf{Structural Induction}]
{FP_structural_induction_rational_composition}


%uses congruence not yet covered.  not a great problem anyway since
%constructor step just uses the fact that congruence is a congruence

\pinput[points = 10, title=
  \textbf{Structural Induction}]{FP_structural_ind_polynomials}

\pinput[points = 10, title=
  \textbf{Structural Induction}]{MQ_K_subs_erasable}

\pinput[points = 10, title= \textbf{Structural Induction}]
      {CP_structural_induction_rational_composition}

\fi
