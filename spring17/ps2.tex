\documentclass[handout]{mcs}

\begin{document}

\renewcommand{\reading}{
\begin{itemize}
\item Chapter~\bref{predicate_sec}{.\ \emph{Predicate Formulas}},
\item Chapter~\bref{data_chap}{.\ \emph{Mathematical Data
Types} through \bref{rel_sec}{.\ \emph{Binary Relations}}},
\item Chapter~\bref{induction_chap}{.\ \emph{Induction}}.

\end{itemize}
These assigned readings do \textbf{not}
  include the Problem sections.  (Many of the problems in the text
  will appear as class or homework problems.)}

\problemset{2}

\medskip

\textbf{Reminder}:

\begin{itemize}

\item
  \href{https://courses.csail.mit.edu/6.042/spring17/pset_instructions}
       {Instructions for PSet submission} are on the class
       \href{https://stellar.mit.edu/S/course/6/sp17/6.042/}{Stellar
         page}.  Remember that each problem should prefaced with a
       \href{http://courses.csail.mit.edu/6.042/spring17/pset_instructions.shtml#collab-state}
            {\emph{collaboration statement}}.
            \item The class has a
  \href{https://piazza.com/mit/spring2017/6042/home} {Piazza
    forum}.  With Piazza you may post questions---both administrative
  and content related---to the entire class or to just the staff.  You
  are likely to get faster response through Piazza than from direct
  email to staff.

  You should post a question or comment to Piazza at least once by the
  end of the second week of the class; after that Piazza use is
  optional.
\end{itemize}

\begin{staffnotes}
Lectures covered: Predicate Formulas, Predicate Logic, Sets and Sequences, Binary Relations, and Induction 
\end{staffnotes}


%%%%%%%%%%%%%%%%%%%%%%%%%%%%%%%%%%%%%%%%%%%%%%%%%%%%%%%%%%%%%%%%%%%%%
% Problems start here
%%%%%%%%%%%%%%%%%%%%%%%%%%%%%%%%%%%%%%%%%%%%%%%%%%%%%%%%%%%%%%%%%%%%%


 \pinput{PS_predicate_calculus_power_of_two}

 \pinput{PS_set_union}

 \pinput{MQ_predicates_sets_relations}

 \pinput{PS_fib_induction}


\end{document}
