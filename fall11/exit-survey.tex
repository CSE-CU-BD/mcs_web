\documentclass[handout]{mcs}

\makeatletter

\def\ps@myheadings{%
    \let\@oddfoot\@empty\let\@evenfoot\@empty
    \def\@evenhead{\thepage\hfil\slshape\leftmark}%
    \def\@oddhead{\rlap{\slshape\rightmark}\hfil\thepage\hfil \llap{Name:\brule{2in}}}%
    \let\@mkboth\@gobbletwo
    \let\sectionmark\@gobble
    \let\subsectionmark\@gobble
    }

\makeatother

\begin{document}

\paper{exit}{}{End-of-term Survey}

This survey asks for your feedback on how well 6.042J/18.062J helped
you learn and appreciate the subject, and
\iffalse
.  Comparing student self-assessments given in this survey to
student grades helps us determine how to improve the course.
We would also be grateful\fi for any improvements you care to suggest.
We also encourage you to complete an
\href{https://sixweb.mit.edu/student/evaluate/6.042-f2011}{HKN survey}
which asks a different set of questions.

\textbf{You may submit this form anonymously}.  (It would still be
helpful if you would indicate your TA and/or LA.)

\large{Your Name: (optional)} \brule{3in}

\large{Circle your:}
\begin{center}
  \begin{tabular}{rc}
TA: &  Ali  \qquad\qquad  Nick   \qquad\qquad Oscar   \qquad\qquad Oshani\\
%        \courseassistants\\
and /or LA: &
Becky \quad
Chinua  \quad
Henrique \quad
Kyle \quad
Radhika \quad
Subha \quad
Tigran
  \end{tabular}
\end{center}

\section*{Course Activities}

Please indicate with a digit how helpful the following features of
6.042 were to you in taking the subject, where digits from five (5) to
one (1) mean
\begin{center}
\textbf{5}  \textbf{very} \qquad
\textbf{4}  \textbf{moderately}\qquad
\textbf{3} \textbf{somewhat}\qquad
\textbf{2} \textbf{barely}\qquad
\textbf{1} \textbf{not}

  \textbf{helpful}
\end{center}

\iffalse
 How helpful have the following aspects of the course been in
achieving the subject outcomes for you personally:
\fi

\begin{center}

\begin{tabular}{| l | c |}
%| c |}
%                & \textbf{understanding the big ideas}
%                & \textbf{preparing for exams}
\hline
\hspace{1in} feature &  value\\  %\hspace{0.3in}
\hline  \hline
%   NB annotation system  & \\  \hline
   team problem-solving in class  & \\   \hline
   TA/LA in class        & \\  \hline
   lecture/presentations by Prof. Meyer  & \\  \hline
   Prof. Meyer during team-problem solving &\\ \hline
   reading lecture slides during class    & \\  \hline
   The Math for CS text          & \\  \hline
   doing problem sets          & \\  \hline
   online tutor problems & \\  \hline
%   6042-probs email      &\\ \hline
   reading lecture slides after class     & \\  \hline
   reviewing team problem solutions       & \\  \hline
   reviewing problem set solutions & \\  \hline
   weekly miniquizzes                   & \\  \hline
   staff outside class (office hours/email/\dots) & \\  \hline
%   staff email           &\\ \hline
   collaborating on psets                 & \\  \hline
   online grade reports                   & \\  \hline
\end{tabular}
\end{center}

For learning the material, 6.042 has overall been  \hfill \brule{0.5in}

For coming to appreciate the role of
Math in CS, 6.042 has overall been \hfill \brule{0.5in}

\newpage
\section*{Learning Outcomes}

Please indicate with a digit how thoroughly the following outcomes
were met for you in taking 6.042, where digits from five (5) to one
(1) mean the outcome was
\begin{center}
\textbf{5}  \textbf{thoroughly} \qquad
\textbf{4}  \textbf{adequately}\qquad
\textbf{3} \textbf{somewhat}\qquad
\textbf{2} \textbf{barely}\qquad
\textbf{1} \textbf{not}\\

\textbf{achieved for me personally}.
\end{center}

\iffalse

\subsection{Objectives}
On completion of 6.042, students will be able to
\begin{enumerate}
\item
\label{Basic Discrete Mathematics Concepts}
\textbf{reason mathematically about basic data types and structures} (such
as numbers, sets, graphs, and trees) used in computer algorithms and
systems; distinguish rigorous definitions and conclusions from merely
plausible ones; synthesize elementary proofs, especially proofs by
induction.\brule{0.5in}

\item
\label{Computational Processes} 
\textbf{model and analyze computational processes} using state machine
methods.\brule{0.5in}

\item \label{Discrete Probability} \textbf{apply principles of discrete
probability} to calculate probabilities and expectations of simple random
processes.\brule{0.5in}

\item 
\label{teams} 
\textbf{work in small teams} to accomplish all the objectives above.\brule{0.5in}
\end{enumerate}

\section*{Learning Outcomes}

In the indicated space next to each item, please enter a digit from five
(5) to one (1) where

\begin{center}
\begin{tabular}{rcll}
\hline
\textbf{5} & means &  ``this objective/outcome was & \textbf{thoroughly}
achieved for me personally.''\\
\textbf{4} & means &  ``\dots & \textbf{adequately} \dots''\\
\textbf{3} & means &  ``\dots & \textbf{somewhat} \dots''\\
\textbf{2} & means &  ``\dots & \textbf{barely} \dots''\\
\textbf{1} & means &  ``\dots & \textbf{not} \dots''\\
\hline
\end{tabular}
\end{center}
\fi

Upon completion of 6.042, students will be able to:
\begin{enumerate}

\item \label{basic} read and use \textbf{logical notation} in
  definitions and proofs. \hfill \brule{0.5in}

\item \label{logic} know the definition and use of logical
  concepts such as \textbf{validity and satisfiability} \hfill \brule{0.5in}

\item \label{sets} know the definitions and elementary properties of
  basic math concepts such as \textbf{sets, relations, and
    functions}. \hfill \brule{0.5in}

%(enter 5 ``thoroughly achieved, \dots, 1 ``not achieved'') \hfill \brule{0.5in}

\item \label{proofs} create elementary \textbf{mathematical
proofs} and identify fallacious \emph{reasoning} (not just fallacious
conclusions).  \hfill \brule{0.5in}

\item \label{well ordering principle} use the \textbf{well ordering
  principle} in simple proofs.  \hfill \brule{0.5in}

\item \label{induction} synthesize \textbf{induction hypotheses} and
  simple proofs using various forms of \textbf{induction}.  \hfill
  \brule{0.5in}

\item \label{invariants} apply the method of invariants
\iffalse and well-ordering\fi
to prove correctness and termination of \textbf{state
    machines} and processes.  \hfill \brule{0.5in}

\iffalse

\item \label{structural induction} find simple proofs using
  \textbf{structural induction}.  \hfill \brule{0.5in}
\fi

\item \label{arithmetic} prove elementary properties of \textbf{modular
arithmetic} and explain their applications in Computer Science,
for example in \textbf{cryptography}.\hfill \brule{0.5in}

\item \label{graphs} apply \textbf{graph theory} models of data
  structures to solve problems of scheduling and connectivity.  \hfill
  \brule{0.5in}

\item \label{asymptotics} compare asymptotic growth rates of
  elementary functions and explain properties \textbf{asymptotic
    relations} such as O() and o().

\item \label{counting} find \textbf{counting formulas} for numbers of
  outcomes of elementary combinatorial processes such as
  \textbf{permutations and combinations}. \hfill \brule{0.5in}


\item\label{generating functions} use \textbf{generating functions} to solve
  linear recurrences and elementary counting problems.
  \hfill \brule{0.5in}

\item \label{probability} calculate \textbf{probabilities} and
  discrete distributions for simple combinatorial processes.  \hfill \brule{0.5in}

\item \label{random_variables} define \textbf{random variables},
  calculate \textbf{expectations}, and explain their use
  in \textbf{sampling}; explain \textbf{confidence levels} \hfill
  \brule{0.5in}

\item \label{student teams} problem-solve in a \textbf{small team} with
fellow students.  \hfill \brule{0.5in}

\item \label{writing} write brief, clear explanations and solutions for
  class problems \hfill \brule{0.5in}
\end{enumerate}

\newpage
\section*{Further Comments}

How interested would you be in having another class in the
``lecture-review/team-problems'' style of 6.042?
\begin{center}
\begin{tabular}{ccccc}
enthusiastic &  interested &  somewhat interested  &  uninterested &  unwilling
\end{tabular}
\end{center}

\vspace{0.5in}
We would be pleased to hear any other comments or suggestions you may have
about the course:

\textbox{\hspace{7in}
\vspace{6in}}

\end{document}
