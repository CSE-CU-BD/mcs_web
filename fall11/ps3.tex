\documentclass[handout]{mcs}

%finite cardinality, mapping rule
%infinite cardinality
%diagonal argument, Russell paradox
%induction
%NO STATE MACHINES

\begin{document}

\renewcommand{\reading}
{
\emph{For this pset}: Chapter~\bref{set_theory_chap}{on Finite \& Infinite
  Cardinality}, Chapter~\bref{ordinary_induct_sec}{--\bref{versusWO}{on Induction}}

\emph{For lecture, Friday, Sept. 23}:
Chapter~\bref{state_machine_sec}{State Machines}
}

\problemset{3}

%%%%%%%%%%%%%%%%%%%%%%%%%%%%%%%%%%%%%%%%%%%%%%%%%%%%%%%%%%%%%%%%%%%%%
% Problems start here
%%%%%%%%%%%%%%%%%%%%%%%%%%%%%%%%%%%%%%%%%%%%%%%%%%%%%%%%%%%%%%%%%%%%%

%mapping rule

%Cantor set

\pinput{CP_mapping_rule}

\pinput{PS_unit_interval}

\pinput{PS_team_division} %: good basic induction}

%\pinput{PS_fib_induction}

\pinput{PS_sums_and_products_of_integers} %nice since induction is not on $n$

\pinput{CP_bogus_unique_prime_factors}

%\pinput{PS_ripple_carry_adder_correctness}

%%%%%%%%%%%%%%%%%%%%%%%%%%%%%%%%%%%%%%%%%%%%%%%%%%%%%%%%%%%%%%%%%%%%%
% Problems end here
%%%%%%%%%%%%%%%%%%%%%%%%%%%%%%%%%%%%%%%%%%%%%%%%%%%%%%%%%%%%%%%%%%%%%

\end{document}
