\documentclass[handout]{mcs}

\begin{document}

\renewcommand{\reading}{Part~\bref{part:proofs}{. \emph{Proofs:
      Introduction}}, Chapter~\bref{proofs_chap}{, \emph{What is a
      Proof?}}; Chapter~\bref{well_ordering_chap}{, \emph{The Well
      Ordering Principle}}; and Chapter~\bref{logicform_chap}{ through
    \bref{SAT_sec}{, covering \emph{Propositional Logic}}}.  These
  assigned readings \textbf{do not include the Problem sections}.  (Many
  of the problems in the text will appear as class or homework problems.)}

\problemset{1}

  \emph{Reminders}:
\begin{itemize}

\item The final deadline for an
  \href{http://courses.csail.mit.edu/6.042/fall11/courseinfo#comments}{email
    comment} on the reading using the class
  \href{http://www.piazza.com/mit/fall2011/6042j18062j}{piazza forum}
  is Thursday, Sept.\ 15, 10PM.  Reading Comments count for 5\% of the
  final grade.

\item Problems should be submitted separately following the pset
  \href{http://courses.csail.mit.edu/6.042/fall11/submission}{submission
    instructions}, and each problem should have a \emph{collaboration
    statement} at the beginning, with the requisite information
  written in or attached using the
  \href{http://courses.csail.mit.edu/6.042/fall11/submission_template.pdf}{collaboration
    statement template}.

 \end{itemize}

%%%%%%%%%%%%%%%%%%%%%%%%%%%%%%%%%%%%%%%%%%%%%%%%%%%%%%%%%%%%%%%%%%%%%
% Problems start here
%%%%%%%%%%%%%%%%%%%%%%%%%%%%%%%%%%%%%%%%%%%%%%%%%%%%%%%%%%%%%%%%%%%%%

\pinput{PS_log7_not_in_QZ}

\pinput{PS_3_exponent_inquality}

\pinput{PS_printout_binary_strings}

\pinput{CP_PorQorR_equiv}

\pinput{PS_prime_polynomial_41}

\begin{center}
\large \textbf{Optional:}
\end{center}

\pinput{PS_faster_adder_logic}

%%%%%%%%%%%%%%%%%%%%%%%%%%%%%%%%%%%%%%%%%%%%%%%%%%%%%%%%%%%%%%%%%%%%%
% Problems end here
%%%%%%%%%%%%%%%%%%%%%%%%%%%%%%%%%%%%%%%%%%%%%%%%%%%%%%%%%%%%%%%%%%%%%
\end{document}
