\documentclass[handout]{mcs}

\begin{document}

\inclassproblems{14, Fri.}

%%%%%%%%%%%%%%%%%%%%%%%%%%%%%%%%%%%%%%%%%%%%%%%%%%%%%%%%%%%%%%%%%%%%%
% Problems start here
%%%%%%%%%%%%%%%%%%%%%%%%%%%%%%%%%%%%%%%%%%%%%%%%%%%%%%%%%%%%%%%%%%%%%
\begin{staffnotes}
Variance \& Deviation from the Mean
\end{staffnotes}

\pinput{CP_cold_cows_markov}
\pinput{TP_markov_chebyshev_for_card_games}
\pinput{CP_pairwise_independent_theorem}

%\pinput{CP_chebyshev_hat_check} %ps11

\pinput{CP_chebyshev_tight}
%\pinput{TP_mean_time_variance_given}

%%%%%%%%%%%%%%%%%%%%%%%%%%%%%%%%%%%%%%%%%%%%%%%%%%%%%%%%%%%%%%%%%%%%%
% Problems end here
%%%%%%%%%%%%%%%%%%%%%%%%%%%%%%%%%%%%%%%%%%%%%%%%%%%%%%%%%%%%%%%%%%%%%

\inhandout{\newpage}
\section*{Pairwise Independent Sampling}

Let $R$ be a random variable, and $a$ a constant. Then
\begin{equation}\label{a2R}
\variance{a R} = a^2 \variance{R}.
\end{equation}

\begin{theorem*}[Pairwise Independent Sampling]
Let $G_1, \dots, G_n$ be pairwise independent variables with the same
mean, $\mu$, and deviation, $\sigma$.  Define
\[
S_n \eqdef \sum_{i=1}^n G_i.
\]
Then
\[
\pr{\abs{\frac{S_n}{n} - \mu} \geq x}
    \leq \frac{1}{n} \paren{\frac{\sigma}{x}}^2.
\]
\end{theorem*}

\begin{proof}
\begin{align*}
\Expect{\frac{S_n}{n}} & = \Expect{\frac{\sum_{i=1}^n G_i}{n}}
         & \text{(def of $S_n$)}\\
 & = \frac{\sum_{i=1}^n \expect{G_i}}{n} 
     & \text{(linearity of expectation)}\\
 & = \frac{\sum_{i=1}^n \mu}{n}\\
 & = \frac{n\mu}{n} = \mu.
\end{align*}

\begin{align}
\Variance{\frac{S_n}{n}} & =  \paren{\frac{1}{n}}^2 \variance{S_n}
          & \mbox{(by~\eqref{a2R})}\notag\\
 & =  \frac{1}{n^2} \Variance{\sum_{i=1}^n G_i} 
          & \text{(def of $S_n$)}\notag\\
 & =  \frac{1}{n^2} \sum_{i=1}^n \variance{G_i}
        & \text{(pairwise independent additivity)}\notag\\
 & =  \frac{1}{n^2}\cdot n\sigma^2 =  \frac{\sigma^2}{n}.\label{Snu}
\end{align}

This is enough to apply \idx{Chebyshev's Theorem} and conclude:
\begin{align*}
\Pr{\abs{\frac{S_n}{n} - \mu} \geq x} & \leq \frac{\Variance{S_n/n}}{x^2}.
       & \text{(Chebyshev's bound)}\\
    & = \frac{\sigma^2/n}{x^2} & \text{(by~\eqref{Snu})}\\
    & = \frac{1}{n} \paren{\frac{\sigma}{x}}^2.
\end{align*}

\end{proof}

\end{document}
