\documentclass[quiz]{mcs}

\begin{document}

\miniquiz{6, morning}

%sets, mapping rule
%no diagonal

%%%%%%%%%%%%%%%%%%%%%%%%%%%%%%%%%%%%%%%%%%%%%%%%%%%%%%%%%%%%%%%%%%%%%
% Problems start here
%%%%%%%%%%%%%%%%%%%%%%%%%%%%%%%%%%%%%%%%%%%%%%%%%%%%%%%%%%%%%%%%%%%%%

\pinput[points = 5]{MQ_Sk_equiv_-1_mod_p} % New; slightly modified repo class problem.
\examspace
\pinput[points = 5]{MQ_runs_of_composites} % New; modified repo class problem.
\examspace
\pinput[points = 5]{CP_proving_basic_gcd_properties} % Pick and choose, split between quiz sessions.
\examspace
\pinput[points = 5]{CP_proving_basic_congruence_properties} % Pick and choose, split between quiz sessions.
\examspace
\pinput[points = 5]{TP_GCDs_I} % Part (a) only; I'm happy to change the numbers in here if you like the problem.
\examspace
\pinput[points = 5]{CP_gcd_lcm} % Part (a) only; I'm happy to change the numbers in here if you like the problem.
\examspace
\pinput[points = 5]{PS_calculating_inverses} % I'm happy to change the numbers in here if you like the problem.  
                                             % Also, forcing students into long computations on a quiz seems silly.
\examspace

\pinput[points = 5]{MQ_Euler_function_of_100}
\examspace
\pinput[points = 5]{TP_Relative_Primality_6042}
\examspace
\pinput[points = 5]{FP_Euler_theorem_calculation}
\examspace
\pinput[points = 5]{TP_Eulers_Theorem}
\examspace
\pinput[points = 5]{MQ_RSA_reversed}
\examspace

%%\pinput[points = 5]{MQ_sets_to_membership_no_intro} %1
%%\examspace
%\pinput[points = 6]{MQ_implies_relation_on_propositional_formulas_modified}
%\examspace
%%\pinput[points = 5]{MQ_countable_union}
%\pinput[points = 4]{MQ_product_of_countables.tex} %2

%%%%%%%%%%%%%%%%%%%%%%%%%%%%%%%%%%%%%%%%%%%%%%%%%%%%%%%%%%%%%%%%%%%%%
% Problems end here
%%%%%%%%%%%%%%%%%%%%%%%%%%%%%%%%%%%%%%%%%%%%%%%%%%%%%%%%%%%%%%%%%%%%%
\end{document}
