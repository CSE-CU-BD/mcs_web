\documentclass[handout]{mcs}

\begin{document}

%%%%%%%%%%%%%%%%%%%%%%%%%%%%%%%%%%%%%%%%%%%%%%%%%%%%%%%%%%%%%%%%%%%%%
% Problems start here
%%%%%%%%%%%%%%%%%%%%%%%%%%%%%%%%%%%%%%%%%%%%%%%%%%%%%%%%%%%%%%%%%%%%%

%topics
state machines invariance, recursive data, GCD's

\newcommand{\fbt}{\text{FBT}}
\newcommand{\wlt}{\text{WLT}}
\newcommand{\tg}{\text{2PTG}}
\newcommand{\vg}{\text{VG}}


\large\textbf{Repo: PS\_bracket\_good\_count}
\pinput{PS_bracket_good_count}

\large\textbf{Repo: PS\_calculating\_inverses}
\pinput{PS_calculating_inverses}

%\large\textbf{Repo: PS\_congruent\_modulo\_1000}
%\pinput{PS_congruent_modulo_1000}

%\large\textbf{Repo: PS\_RSA\_correctness}
%\pinput{PS_RSA_correctness}

\large\textbf{Repo: CP\_binary\_trees}
\pinput{CP_binary_trees}

\large\textbf{S07PS3P1}
\begin{problem}

Let $m,n$ be integers, not both zero.  Define a set of integers,
$L_{m,n}$, recursively as follows:
\begin{itemize}
\item \textbf{Base cases}: $m,n \in L_{m,n}$.
\item \textbf{Constructor steps}:
If $j,k \in L_{m,n}$, then
\begin{enumerate}
\item $-j \in  L_{m,n}$,
\item $j+k \in L_{m,n}$.
\end{enumerate}
\end{itemize}
Let $L$ be an abbreviation for $L_{m,n}$ in the rest of this problem.

\begin{problemparts}

\problempart \label{Lmx+ny}  Show \emph{by structural induction} that
\[
L \subseteq \set{mx+ny \suchthat x,y \in \integers}.
\]

\solution{We need to prove that every number in $L$ is of the form $mx+ny$.
We do this by structural induction on the definition of $L$.

\textbf{Base cases}: The base cases are of the required form because
\begin{align*}
m & = m\cdot 1+ n \cdot 0,\\
n & = m\cdot 0+ n \cdot 1.
\end{align*}

\textbf{Constructor steps}: We must prove that $-j$ and $j+k$ are of the
required form for $j,k \in L$, where by structural induction hypothesis,
we may assume $j,k$ are of the required form.  But this follows
immediately since
\begin{align*}
-j  & = -(mx+ny) = m(-x)+ n(-y)\\
j+k & = (mx+ny) + (mx'+ny') = m(x+x')+n(y+y')
\end{align*}
This completes the structural induction.}

\problempart \label{mx+nyL}  Show that
\[
\set{mx+ny \suchthat x,y \in \integers} \subseteq L.
\]

\solution{ We must show that $mx+ny \in L$ for all integers $x,y$.  To
begin, suppose $j\in L$ and let
\[
P_j(k) \eqdef jk \in L.
\]
We prove that $P_j(k)$ holds for all $k \in \naturals$ by induction on $k$,
with induction hypothesis, $P_j$.

\textbf{Base case} $(k=0)$: $j0 = 0 = m + (-m) \in L$.

\textbf{Induction step}: Assume $P(k)$, so $jk \in L$.  But we are given
that $j \in L$, and $j(k+1) = jk+j \in L$ by the second recursive step in
the definition of $L$.  This proves $P(k+1)$, completing the induction
proof.

Now since $jk \in L$ implies $-jk \in L$ by the first recursive step in
the definition of $L$, we conclude
\begin{lemma*}
If $j\in L$, then $jx \in L$ for all $x \in \integers$.
\end{lemma*}
But $m,n \in L$ by definition of $L$, so the Lemma implies that $mx \in L$ for
$x \in \integers$, and also $ny \in L$ for all $y \in \integers$.  Now by
the second recursive step in the definition of $L$, we conclude that
$mx+ny \in L$ for all integers $x,y$.}

\problempart \label{common}

Conclude that any common divisor of $m$ and $n$ also divides every member
of $L$.

\solution{If $k \in L$, then by part~(\ref{Lmx+ny}), $k = mx+ny$.
Now if $d$ is a common divisor of $m$ and $n$, then it divides $mx$ and
$ny$, and hence divides their sum $mx+ny$.  So $d$ also divides $k$.}


\problempart Show that if $j,k \in L$ and $k\neq 0$, then the remainder of
$j$ divided by $k$ is in $L$.

\solution{Say $r$ is the remainder of $j$ divided by $k$.  That is, $j =
qk+r$ for some quotient $q \in \integers$ and remainder, $r$, where $0\leq
r < \card{k}$.

Now the Lemma in the solution to part~\ref{mx+nyL} implies that $(-q)k \in
L$, and then the second recursive step in the definition of $L$ implies
that $j + (-q)k \in L$.  That is, $r \in L$.}

\problempart \label{g}

Show that there is a positive integer $g \in L$ which divides every
member of $L$.  \hint The least positive integer in $L$.

\solution{At least one of the integers $m, -m, n, -n \in L$ must be
positive.  Hence by the Well Ordering Principle, there is a least positive
integer $g \in L$.

Suppose $j \in L$.  We must show that $g \divides j$.

We prove this by contradiction: if $g$ does not divide $j$, then the
remainder of $j$ divided by $g$ is a positive number smaller than $g$,
which by the previous part is in $L$, contradicting the fact that
$g$ is the smallest such integer.
}

\problempart Conclude that $\gcd(m,n) = mx+ny$ for some $x,y \in \integers$.

\solution{From part~(\ref{g}), we have a positive integer $g \in L$ that
divides every element of $L$.  In particular, $g$ is a common divisor of
$m,n$, since $m,n \in L$.

Further, $g =mx+ny$ for some $x,y \in \integers$ by part~(\ref{Lmx+ny}), and
any common divisor of $m$ and $n$ divides $g$, by part~(\ref{common}).
So $g$ is a common divisor that is divided by, and hence at least as large
as, any common divisors.  So $g$ must be the \emph{greatest} common divisor
of $m$ and $n$.}

\end{problemparts}

\end{problem}

\large\textbf{S07PS3P2}

\begin{problem}

We define full binary trees, $\fbt$'s, as a tagged recursive
data type with tags from some set of ``labels.''
\begin{definition*}

\textbf{Base case}: $\ang{l, \texttt{leaf}}$ is an $\fbt$, where $l$ is a
label.

\textbf{Constructor case}: If $B_1$ and $B_2$ are $\fbt$'s, then
$\ang{l,\ B_1,\ B_2}$ is an $\fbt$, where $l$ is a label.

\end{definition*}

The labels and leaf labels \emph{appearing} in an $\fbt$, $B$, are defined
recursively in the obvious way:

\begin{definition*}

\textbf{Base case}: If $B = \ang{l, \texttt{leaf}}$.  Then $l$
\emph{appears} in $B$ and is a \emph{leaf label} of $B$.

\textbf{Constructor case}: If $B=\ang{l, B_1,\ B_2}$ is an $\fbt$, then
the labels that \emph{appear} in $B$ are the ones that appear in $B_1$, or
in $B_2$; also, $l$ \emph{appears} in $B$.  The \emph{leaf labels} of $B$
are the union of the leaf labels of $B_1$ and the leaf labels of $B_2$.

\end{definition*}

The $\fbt$'s with \emph{unique} labels are also defined recursively:
\begin{definition*}

\textbf{Base case}: If $B = \ang{l,\texttt{leaf}}$.  Then $B$ has
\emph{unique labels}.

\textbf{Constructor case}: If $B=\ang{l, B_1,\ B_2}$ is an $\fbt$, then
$B$ has \emph{unique labels} iff $l$ does not appear in $B_1$ or $B_2$,
and no other label appears in both $B_1$ and $B_2$.

\end{definition*}

If $B$ is an $\fbt$, let $n_B$ be the number of labels appearing in $B$
and $f_B$ be the number of leaf labels of $B$.  Use structural induction
to prove that
\begin{equation}\label{lv}
f_B = \frac{{n_B}+1}{2}
\end{equation}
for all $\fbt$'s \emph{with unique labels}.  Also give a counterexample
for an $\fbt$ that does not have unique labels.  So your proof had better
use uniqueness of labels at some point; be sure to indicate where.

\solution{
\textbf{Base case}: If $B = \ang{l,\texttt{leaf}}$, then $f_B=n_B=1$ and
\[
1 = \frac{1+1}{2}
\]
proving equation~\eqref{lv} in this case.

\textbf{Constructor case}: If $B=\ang{l, B_1,\ B_2}$, has unique labels
then

\begin{align*}
f_B & =  \card{\text{leaf-labels}(B_1) \union \text{leaf-labels}(B_2)}
         & \text{(by def. of leaf labels)}\\
 & = f_{B_1} + f_{B_2} & \text{(no label appears in both $B_1$ and $B_2$)}\\
 & =  \frac{{n_{B_1}}+1}{2} + \frac{{n_{B_2}}+1}{2}
          & \text{(by structural induction hypothesis)}\\
 &= \frac{(1+ n_{B_1} + n_{B_2}) +1}{2}\\
 & = \frac{\card{\set{l} \union \text{labels}(B_1) \union \text{labels}(B_2)}+ 1}{2}
         &  \text{(uniqueness of labels)}\\
 & = \frac{n_B + 1}{2} & \text{(by def. of $n_B$)}.
\end{align*}
This proves~\eqref{lv} holds for $B$, completing the proof of the
Constructor case.  It follows by structural induction that ~\eqref{lv}
holds for all $\fbt$'s with unique labels.

A counterexample is the $\fbt$
\[
C \eqdef\quad \ang{0, \ang{1, \texttt{leaf}}, \ang{1, \texttt{leaf}}},
\]
where $n_C=2$ and $f_C = 1 \neq 1.5 = (2+1)/2 = (n_C + 1)/2$.
}

\end{problem}

\large\textbf{Repo: PS\_linear\_combination\_game}
\pinput{PS_linear_combination_game}


%%%%%%%%%%%%%%%%%%%%%%%%%%%%%%%%%%%%%%%%%%%%%%%%%%%%%%%%%%%%%%%%%%%%%
% Problems end here
%%%%%%%%%%%%%%%%%%%%%%%%%%%%%%%%%%%%%%%%%%%%%%%%%%%%%%%%%%%%%%%%%%%%%
\end{document}
