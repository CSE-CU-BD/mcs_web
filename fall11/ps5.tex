\documentclass[handout]{mcs}

\begin{document}

\renewcommand{\reading}
{
  \emph{For this pset}:
  Sections~\bref{Turing_sec}{--}\bref{mod_prime_sec}{ on Modular Arithmetic}, and
  Sections~\bref{arithmetic_modn_sec}{ on Euler's Theorem}.
  %Section~\bref{state_machine_sec}{ on State Machines}, 
  %Chapter~\bref{recursive_data_chap}{ on Recursive Data}, and 
  %Sections~\bref{divisibility_sec}{--}
  %  ~\bref{fundamental_theorem_sec}{ on Number Theory.}

\emph{For lecture, Friday, Oct. 14}:
Sections~\bref{RSA_sec}{--}\bref{SAT_RSA-sec}{ on The RSA crypto-system}.
%Section~\bref{Turing_sec}{--}~\bref{mod_prime_sec}{ on Modular Arithmetic.}
}

\problemset{5}


\emph{Reminders}:
\begin{itemize}
\item The deadline for a
  \href{http://courses.csail.mit.edu/6.042/fall11/courseinfo#comments}{Piazza
    comment} on the reading is Thursday, Oct. 13, 10PM.
\item Problems should be submitted separately following the pset
  \href{http://courses.csail.mit.edu/6.042/fall11/submission}{submission
    instructions}, and each problem should have a \emph{collaboration
    statement} at the beginning, with the requisite information
  written in or attached using the
  \href{http://courses.csail.mit.edu/6.042/fall11/submission_template.pdf}{collaboration
    statement template}.
\end{itemize}


%TOPICS: state machines invariance, recursive data, GCD's

%%%%%%%%%%%%%%%%%%%%%%%%%%%%%%%%%%%%%%%%%%%%%%%%%%%%%%%%%%%%%%%%%%%%%
% Problems start here
%%%%%%%%%%%%%%%%%%%%%%%%%%%%%%%%%%%%%%%%%%%%%%%%%%%%%%%%%%%%%%%%%%

% PSET PROBLEMS FOR WEEK 5 FRIDAY
\pinput{PS_calculating_inverses} % I can change the numbers if necessary.
\pinput{PS_self-inverse_mod_p} % Should remove all mention of the theorem name; move it to the solution
                               % so students don't just look up the proof.
\pinput{CP_polynomials_produce_multiples} % New to repo; hints and restructuring may be in order.

% PSET PROBLEMS FOR WEEK 6 WEDNESDAY
%cp6w + mq7m: \pinput{PS_Euler_theorem_calculation}
\pinput{CP_13th_roots}
\pinput{PS_Euler_function_multiplicativity}
\pinput{FP_modular_powerful}


%mq7m: \pinput{TP_Relative_Primality}

% PSET PROBLEMS FOR WEEK 6 FRIDAY
%cp6f: \pinput{PS_RSA_correctness}
\pinput{PS_RSA_key_implies_factoring}
\pinput{PS_Rabin_cryptosystem}

% CLASS PROBLEMS FOR WEEK 6 WEDNESDAY
%\pinput{PS_Euler_theorem_calculation}
%\pinput{PS_congruent_modulo_1000}
%\pinput{CP_chinese_remainder}
%\pinput{PS_Euler_function_multiplicativity}

% CLASS PROBLEMS FOR WEEK 6 FRIDAY
%\pinput{CP_RSA_between_tables}
%\pinput{CP_RSA_proving_correctness}

% MINIQUIZ PROBLEMS: MORNING
%\pinput[points = 5]{MQ_Euler_function_of_100}
%\pinput[points = 5]{TP_Relative_Primality_6042}
%\pinput[points = 5]{FP_Euler_theorem_calculation}
%\pinput[points = 5]{TP_Eulers_Theorem}

% MINIQUIZ PROBLEMS: AFTERNOON
%\pinput[points = 5]{MQ_Euler_function_of_6042}
%\pinput[points = 5]{TP_Relative_Primality}

% COPIED FROM spring11/ps5.tex
%%\pinput{CP_conquering_the_galaxy}
%%\pinput{PS_directed_Euler_circuits}
%\pinput{CP_binary_relations_on_01}
%\pinput{PS_top_sort_for_closure_of_DAG}
%\pinput{PS_Brents_theorem}
%%\pinput{CP_product_relation_properties}
%\pinput{PS_subsequences_partial_order_Dilworth_Lemma}
%%\pinput{PS_weak_partial_order_isomorphic_to_subset}

% COPIED FROM fall11/ps4.tex
%\pinput{CP_robot_invariant}
%\pinput{PS_bracket_good_count}
%\iffalse
%%NOT THIS WEEK
%\pinput{PS_calculating_inverses}
%\large\textbf{Repo: PS\_congruent\_modulo\_1000}
%\pinput{PS_congruent_modulo_1000}
%\large\textbf{Repo: PS\_RSA\_correctness}
%\pinput{PS_RSA_correctness}
%\fi
%%\large\textbf{Repo: CP\_binary\_trees}
%%\pinput{CP_binary_trees}
%\pinput{PS_linear_combination_by_structural_induction}
%\pinput{PS_labeled_binary_trees}
%%\large\textbf{Repo: PS\_linear\_combination\_game}
%%\pinput{PS_linear_combination_game}

%%%%%%%%%%%%%%%%%%%%%%%%%%%%%%%%%%%%%%%%%%%%%%%%%%%%%%%%%%%%%%%%%%%%%
% Problems end here
%%%%%%%%%%%%%%%%%%%%%%%%%%%%%%%%%%%%%%%%%%%%%%%%%%%%%%%%%%%%%%%%%%%%%
\end{document}
