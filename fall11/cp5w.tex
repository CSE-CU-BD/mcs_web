\documentclass[handout]{mcs}

\begin{document}

\inclassproblems{5, Wed.}

%%%%%%%%%%%%%%%%%%%%%%%%%%%%%%%%%%%%%%%%%%%%%%%%%%%%%%%%%%%%%%%%%%%%%
% Problems start here
%%%%%%%%%%%%%%%%%%%%%%%%%%%%%%%%%%%%%%%%%%%%%%%%%%%%%%%%%%%%%%%%%%%%%


\pinput{CP_gcd_lcm}

\pinput{CP_runs_of_composites} % New to repo.

\pinput{CP_proving_basic_gcd_properties}

\iffalse

\pinput{CP_perfect_numbers}

\pinput{CP_divisible_by_powers_of_3} % New to repo.

\large\textbf{This, from F05PS5P2, would be a nice final part for the gcd basic proofs problem.
gcd(km,kn) = k gcd(m,n) exists there already (with a different proof), so I 
didn't add it again.  That part of the problem is inactive, however.}

\begin{problem}
Prove that $\gcd(ka, kb) = k \cdot \gcd(a, b)$ for all $k > 0$.

\solution{The smallest positive value of:

\[
k \ \cdot\ (s \cdot a + t \cdot b)
\]

(which is equal to $s (k a) + t (k b) = \gcd(ka, kb)$) must be $k$
times the smallest positive value of:

\[
s \cdot a + t \cdot b
\]

(which is equal to $\gcd(a, b)$).
}
\end{problem}


\large\textbf{This one didn't strike me as terribly interesting.  Got it from F06Rec4P3.}

\begin{problem}
The Fibonacci numbers are defined as follows:
$$
F_0 = 0 \qquad
F_1 = 1 \qquad
F_n = F_{n-1} + F_{n-2} \quad \text{(for $n \geq 2$)}.
$$
Give an inductive proof that the Fibonacci numbers $F_n$ and $F_{n+1}$
are relatively prime for all $n \geq 0$.

\solution{We use induction on $n$.  Let $P(n)$ be the proposition that $F_n$
and $F_{n+1}$ are relatively prime.

\noindent \textit{Base case:} $P(0)$ is true because $F_0 = 0$ and $F_1 = 1$
are relatively prime.

\noindent \textit{Inductive step:} We show that, for all $n\geq 0$,
$P(n)$ implies $P(n+1)$. So, fix some $n\geq 0$ and assume that $P(n)$
is true, that is, $F_n$ and $F_{n+1}$ are relatively prime.  We will
show that $F_{n+1}$ and $F_{n+2}$ are relatively prime as well.  We
will do this by contradiction.

Suppose $F_{n+1}$ and $F_{n+2}$ are not relatively prime. Then they
have a common divisor $d > 1$. But then $d$ also divides the linear
combination $F_{n+2} - F_{n+1}$, which actually equals $F_n$. Hence,
$d>1$ divides both $F_n$ and $F_{n+1}$. Which implies $F_n$, $F_{n+1}$
are not relatively prime, a contradiction to the inductive hypothesis.

Therefore, $F_{n+1}$ and $F_{n+2}$ are relatively prime. That is,
$P(n+1)$ is true.

The theorem follows by induction.}

\end{problem}
\fi

% Problems end here
%%%%%%%%%%%%%%%%%%%%%%%%%%%%%%%%%%%%%%%%%%%%%%%%%%%%%%%%%%%%%%%%%%%%%

\end{document}

\endinput
