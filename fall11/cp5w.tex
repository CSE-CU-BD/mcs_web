\documentclass[handout]{mcs}

\begin{document}

\inclassproblems{5, Wed.}

%%%%%%%%%%%%%%%%%%%%%%%%%%%%%%%%%%%%%%%%%%%%%%%%%%%%%%%%%%%%%%%%%%%%%
% Problems start here
%%%%%%%%%%%%%%%%%%%%%%%%%%%%%%%%%%%%%%%%%%%%%%%%%%%%%%%%%%%%%%%%%%%%%

\large\textbf{F05PS5P2}

\begin{problem}
Prove that $\gcd(ka, kb) = k \cdot \gcd(a, b)$ for all $k > 0$.
\end{problem}

\solution{The smallest positive value of:

\[
k \ \cdot\ (s \cdot a + t \cdot b)
\]

(which is equal to $s (k a) + t (k b) = \gcd(ka, kb)$) must be $k$
times the smallest positive value of:

\[
s \cdot a + t \cdot b
\]

(which is equal to $\gcd(a, b)$).
}

\large\textbf{F05PS5P5}

\begin{problem}
Here is a long run of composite numbers:

\[
114, 115, 116, 117, 118, 119, 120, 121, 122, 123, 124, 125, 126
\]

Prove that there exist arbitrarily long runs of composite numbers.
Consider numbers a little bigger than $n!$ where $n! = n \cdot (n - 1)
\cdots 3 \cdot 2 \cdot 1$.


\solution{

  Let $k$ be some natural number such that $1 < k \leq n$.  We know $k
  | (n! + k)$ because $k \mid n!$ and $k \mid k$. Thus, the numbers
  $n!+2, n!+3, n!+4, \ldots, n!+n$ must all be composite.  This is a
  run of $n-1$ consecutive composite numbers. Because we can
  arbitrarily choose $n$, we know arbitrarily long runs of compisite
  numbers exist.}

\end{problem}

\large\textbf{Repo: CP\_gcd\_lcm}
\pinput{CP_gcd_lcm}

\large\textbf{Repo: CP\_perfect\numbers}
\pinput{CP_perfect_numbers}

\large\textbf{F07Rec8HP1}

\begin{problem}
\bparts

\ppart Use the Pulverizer (described in the Appendix) to find integers
$x,y$ such that
\[
x50 + y21 = \gcd(50,21).
\]

\solution{Here is the table produced by the Pulverizer:
\[
\begin{array}{ccccrcl}
x & \quad & y & \quad & \rem{x}{y} & = & x - q \cdot y \\ \hline
 50 &&  21 &&  8 & = &    50 - 2 \cdot  21 \\
 21 &&   8 &&  5 & = &    21 - 2 \cdot  8 \\
&&&&             & = &    21 - 2 \cdot (50 - 2 \cdot  21) \\
&&&&             & = &   -2 \cdot 50 + 5 \cdot 21 \\
  8 &&   5 &&  3 & = &    8 - 1 \cdot 5  \\
&&&&             & = &   (50 - 2 \cdot  21)
                         -1 \cdot (-2 \cdot 50 + 5 \cdot 21) \\
&&&&             & = &   3\cdot 50 -7 \cdot 21 \\
  5 &&   3 &&  2 & = &    5 - 1\cdot 3 \\
&&&&             & = &   (-2 \cdot 50 + 5 \cdot 21)
                         -1 \cdot (3\cdot 50 -7 \cdot 21) \\
&&&&             & = &   -5\cdot 50 + 12 \cdot 21 \\
  3 &&   2 &&  1 & = &    3 - 1\cdot 2 \\
&&&&             & = &   (3\cdot 50 -7 \cdot 21)
                         -1\cdot (-5\cdot 50 + 12 \cdot 21) \\
&&&&             & = &   \fbox{$8\cdot 50 - 19 \cdot 21$} \\
  2 &&   1 &&  0 &   &
\end{array}
\]
}

\ppart Now find integer $x',y'$ with $y'>0$ such that
\[
x'50 + y'21 = \gcd(50,21)
\]
\solution{
since $(x,y) = (8,-19)$ works, so does $(8 - 21n,-19+50n)$ for any $n \in
\integers$, so letting $n=1$, we have
\[
-13 \cdot 50 + 31 \cdot 21 = 1
\]
}

\eparts

\end{problem}

\large\textbf{F07Rec8HP5}

\begin{problem}
  You may remember from Week 1 Notes the polynomial $p(n) \eqdef n^2 + n +
  41$ whose values on nonnegative integers were all prime until $p(40)$.
  Well, $p$ didn't work, but are there any other polynomials whose values
  are always prime?  No way!  In fact, we'll prove a much stronger claim:

  Suppose $q$ is a polynomial with integer coefficients whose domain is
  restricted to be the nonnegative integers.  We'll say that $q$
  \emph{produces multiples} if, for every nonzero value in the range of $q$,
  there are infinitely many multiples of that value also in the range.

  For example, if $q$ produces multiples and $q(4) = 7$, then there are
  infinitely many different multiples of 7 in the range of $q$, and of
  course, except for 7 itself, none of these multiples is prime.

  We claim that if $q$ is not a constant function, then $q$ produces
  multiples.

  \bparts

  \ppart\label{jk} Prove that if $j \equiv k \pmod n$, then $q(j) \equiv
  q(k) \pmod n$.

  \hint The set, $A$, of polynomial functions with integer coefficients can be
  defined recursively:
\begin{itemize}
\item \textbf{Base cases}:
  \begin{itemize}

   \item the identity function, $i(x) \eqdef x$ is in $A$.

   \item for any integer, $k$, the constant function, $c(x) \eqdef k$ is in $A$.
  \end{itemize}

\item \textbf{Constructor cases}.  If $r,s \in A$, then $r+s$ and $r \cdot
  s \in A$.
\end{itemize}

\solution{
The proof is by structural induction on the definition of $A$.  The
hypothesis $P(q)$ is that
\[
\text{for all } k,n \in \naturals,\text{ if } j \equiv k \pmod n, \text{
then } q(j) \equiv q(k) \pmod n.
\]

\begin{itemize}
\item \textbf{Base cases}: $P(i)$ and $P(c)$ both hold trivially.

\item \textbf{Constructor cases}.  Suppose $P(r)$ and $P(t)$ hold, and let
  $t \eqdef r+s$.  To show $P(t)$, suppose $j \equiv k \pmod n$.  Since
  $P(r)$ holds, we have that $r(j) \equiv r(k) \pmod n$.  Likewise, $s(j)
  \equiv s(k) \pmod n$.  So
\[
r(j) + s(s) \equiv r(j) + s(k) \pmod n,
\]
that is, $t(j) \equiv t(k) \pmod n$.

The proof for $t \eqdef r \cdot s$ is the same.

\end{itemize}
}

\ppart Prove the claim.

\solution{ Suppose $q$ is a nonconstant polynomial.  Since $q$ produces
  multiples iff $-q$ produces multiples, we can assume without loss of
  generality that the leading coefficient of $p$ is positive.  Now we use
  the following familiar fact polynomials of positive degree whose leading
  coefficient is positive: once $x$ is above a certain bound, the function
  $p(x)$ is strictly increasing.  \footnote{We'll prove this and similar
    ``growth rate'' facts about polynomials and other functions next
    week.}

Now assume that $v > 0$ is the absolute value of some integer in the range
of $q$, that is, $0 < v = \abs{q(k)}$ for some $k \in \naturals$.
Therefore,
\[
q(k) \equiv 0 \pmod v.
\]

Since $q$ strictly increases once its argument is big enough, the sequence
\[
q(k),q(k+v),q(k+2v),q(k+3v),\dots
\]
will include an infinite number of values.

But by part~\eqref{jk}, each of the elements in the sequence is $\equiv 0
\pmod v$.  So the sequence includes an infinite number of multiples of
$v$, which proves that $q$ multiplies.}

\eparts

\end{problem}

\large\textbf{CP\_proving\_basic\_gcd\_properties}
\pinput{CP_proving_basic_gcd_properties}

\large\textbf{F06Rec4P3 -- use WOP instead.}

\begin{problem}
The Fibonacci numbers are defined as follows:
$$
F_0 = 0 \qquad
F_1 = 1 \qquad
F_n = F_{n-1} + F_{n-2} \quad \text{(for $n \geq 2$)}.
$$
Give an inductive proof that the Fibonacci numbers $F_n$ and $F_{n+1}$
are relatively prime for all $n \geq 0$.

\solution{We use induction on $n$.  Let $P(n)$ be the proposition that $F_n$
and $F_{n+1}$ are relatively prime.

\noindent \textit{Base case:} $P(0)$ is true because $F_0 = 0$ and $F_1 = 1$
are relatively prime.

\noindent \textit{Inductive step:} We show that, for all $n\geq 0$,
$P(n)$ implies $P(n+1)$. So, fix some $n\geq 0$ and assume that $P(n)$
is true, that is, $F_n$ and $F_{n+1}$ are relatively prime.  We will
show that $F_{n+1}$ and $F_{n+2}$ are relatively prime as well.  We
will do this by contradiction.

Suppose $F_{n+1}$ and $F_{n+2}$ are not relatively prime. Then they
have a common divisor $d > 1$. But then $d$ also divides the linear
combination $F_{n+2} - F_{n+1}$, which actually equals $F_n$. Hence,
$d>1$ divides both $F_n$ and $F_{n+1}$. Which implies $F_n$, $F_{n+1}$
are not relatively prime, a contradiction to the inductive hypothesis.

Therefore, $F_{n+1}$ and $F_{n+2}$ are relatively prime. That is,
$P(n+1)$ is true.

The theorem follows by induction.}

\end{problem}

\large\textbf{F08Rec4P4}
\begin{problem}
Let $N$ be a number whose decimal expansion consists of $3^n$
identical digits.  Show by induction that $3^n \mid N$.  For example:
\[
3^2 \mid \underbrace{777777777}_{\text{$3^2 = 9$ digits}}
\]
Recall that $3$ divides a number iff it divides the sum of its digits.

\solution{We proceed by induction on $n$.  Let $P(n)$ be the
proposition that $3^n \mid N$, where the decimal expansion of $N$
consists of $3^n$ identical digits.

\noindent \textit{Base case.}  $P(0)$ is true because $3^0 = 1$
divides every number.

\noindent \textit{Inductive step.}  Now we show that, for all $n\geq
0$, $P(n)$ implies $P(n+1)$.  Fix any $n\geq 0$ and assume $P(n)$ is
true. Consider a number whose decimal expansion consists of $3^{n+1}$
copies of the digit $a$:
\begin{align*}
\underbrace{aaaaaa \ldots aaaaaa}_{\text{$3^{n+1}$ digits}}
    & = \underbrace{aaa \ldots aaa}_{\text{$3^n$ digits}}
        \underbrace{aaa \ldots aaa}_{\text{$3^n$ digits}}
        \underbrace{aaa \ldots aaa}_{\text{$3^n$ digits}} \\
    & = \underbrace{aaa \ldots aaa}_{\text{$3^n$ digits}}
        \quad \cdot \quad
        1
        \underbrace{000 \ldots 001}_{\text{$3^n$ digits}}
        \underbrace{000 \ldots 001}_{\text{$3^n$ digits}} \\
\end{align*}
Now $3^n$ divides the first term by the assumption $P(n)$, and 3
divides the second term since the digits sum to 3.  Therefore, the
whole expression is divisible by $3^{n+1}$.  This proves $P(n+1)$.

By the principle of induction $P(n)$ is true for all $n \geq 0$.
} 
\end{problem}

% Problems end here
%%%%%%%%%%%%%%%%%%%%%%%%%%%%%%%%%%%%%%%%%%%%%%%%%%%%%%%%%%%%%%%%%%%%%

\end{document}

\endinput
