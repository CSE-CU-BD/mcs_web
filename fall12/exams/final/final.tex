\documentclass[12pt]{article}
\usepackage{amsmath}
\usepackage{light}
\usepackage{graphicx}
\usepackage{verbatim}

%\hidesolutions
\showsolutions

\newcommand{\prcond}[2]{\Pr[#1 \mid #2]}
\newcommand{\prob}[1]{\Pr[#1]}
\renewcommand{\pr}[1]{\Pr[#1]}
\renewcommand{\ex}[1]{\mathop{\textup{Ex}}[#1]}


\newcommand{\mfigure}[3]{\bigskip\centerline{\resizebox{#1}{#2}{\includegraphics{#3}}}\bigskip}
\newcommand{\naturals}{\mathbb N}
\newcommand{\eqdef}{:=}
\newcommand{\prsub}[2]{\mathop{\textup{Pr}_{#1}}\nolimits\left(#2\right)}

\newcommand{\edge}[2]{#1\text{---}#2}

\newcommand{\todo}[1]{\emph{TODO: #1}}
\newcommand{\remark}[1]{\emph{#1}}
\newcommand{\skipthis}[1]{}



\newcommand{\quizz}[1]{
    \vspace*{-2cm}
  \noindent \coursename \hfill #1 \newline
  \coursestaff \vspace{-1.5ex} \newline
  \mbox{} \hrulefill \mbox{}
%  \vspace{-0.15in}
  \begin{center}
    \ifthenelse{\boolean{showsolutions}}
               {\Large \textbf{Final Exam Solutions}}
               {\Large \textbf{Final Exam}}
  \end{center}
  \vspace{-.1in}
  \thispagestyle{plain}
  \pagestyle{myheadings}
  \thispagestyle{empty}
  \markboth{Quiz #1}{Quiz #1}
  %\coursecopyright
  }

\begin{document}

\quizz{12/17/2012}


\begin{itemize}
\item  The exam is \textbf{closed book}, but you may have three $8.5''  \times 11''$ sheet with notes (either printed or in your own handwriting) on both sides.

\item Calculators and electronic devices (including cell phones) are not allowed.

\item You may assume all of the results presented in class. This does \textbf{not} include results demonstrated in practice quiz material.

 \item Please show your work. Partial credit cannot be given for a wrong answer if your work isn't shown.

 \item Write your solutions in the space provided. If you need more space, write on the back of the sheet containing the problem. Please keep your entire answer to a problem on that problem's page.

 \item Be neat and write legibly. You will be graded not only on the correctness of your answers, but also on the clarity with which you express them.

 \item  If you get stuck on a problem, move on to others. The problems are not arranged in order of difficulty.\\


 \textbf{NAME:} \rule{5in}{0.5pt}\\
 
 \textbf{TA:} \rule{5.34in}{0.5pt}\\
 
\centering
\scalebox{1.3}{
\begin{tabular}{|c|c|c|c|}
\hline
\textbf{Problem} & \textbf{Value} & \textbf{Score} & \textbf{Grader} \\\hline
 1 & 10   & &\\ \hline
 2 & 12   & &\\ \hline
 3 & 12   & &\\ \hline
 4 & 10   & &\\ \hline
 5 & 15   & &\\ \hline
 6 & 15   & &\\ \hline
 7 & 11   & &\\ \hline
 8 & 14   & &\\ \hline
 9 & 16   & &\\ \hline
10 & 15   & &\\ \hline
11 & 15   & &\\ \hline
12 & 15   & &\\ \hline
\textbf{Total} & 160 & & \\\hline
\end{tabular}
}
\end{itemize}

\newpage

\begin{problem}{10} It is well-known that \emph{elitotis} is a common disease amongst Harvard students: in fact, 1 in 10 of their students have it. 
    Fortunately, MIT has developed a reliable test for the presence of elevated \emph{arrogentes} levels, which is helpful for testing for elitotis because:
    %test for the disease, though it's still in early stages of development:
%    have elProstate-specific antigen is produced by the prostate gland and its blood concentration (aka. \emph{PSA level}) is used to detect prostate cancer.
%
\begin{itemize}
%\item A man selected uniformly at random from the age group between 40 and 65 has prostate cancer with probability $1/10$.
\item a student with elitotis has elevated arrogentes levels with probability $4/5$, and
\item a student with no elitosis has elevated arrogentes levels with probability $1/3$.
\end{itemize}
%
What is the probability that a student selected uniformly at random from Harvard has elitosis, given that he or she has elevated arrogentes levels?

\end{problem}

\solution[\newpage]{

Let $C$ be the event that the randomly-selected student has elitosis, 
and let $A$ be the event that he/she has elevated arrogentes levels.
%A tree diagram is worked out below:
%
%\begin{center}
% need draw a tree diagram
%\includegraphics{}
%\end{center}
%
The probability that a student has elitosos, given that he/she has elevated arrogentes levels is:

%
\begin{align*}
\prcond{C}{A}
    & = \frac{\pr{C \cap A}}{\pr{A}} \\
	& = \frac{\prcond{A}{C}\pr{C}}{\pr{A \cap C}+ \pr{A \cap \bar{C}}} \\
		& = \frac{\prcond{A}{C}\pr{C}}{\prcond{A}{C}\pr{C}+ \prcond{A}{\bar{C}}\pr{\bar{C}}} \\
			& = \frac{(4/5) \times (1/10)}{(4/5) \times (1/10) + (1/3) \times (9/10)} \\
	   			 & = \frac{4}{4+15} \\
	       				 & = \frac{4}{19}.
						\end{align*}


}


\begin{problem}{12}
Amy, Bill, and Poor Pete play a game:

\begin{enumerate}
\item Each player puts \$2 on the table.
\item Each player secretly writes a number between 1 and 4.
\item They roll a fair, four-sided die with faces numbered 1, 2, 3, and 4.
\item The money on the table is divided among the players that guessed
correctly.  If no one guessed correctly, then everyone gets their
money back \textit{and Poor Pete is paid \$0.25 in ``service fees''}.
\end{enumerate}

Suppose that, Amy and Bill cheat by picking a pair of
\textit{distinct} numbers uniformly at random.

\bparts

\ppart{8} For each event listed below, indicate the probability of
the event (in the box on the left) and Poor Pete's profit (in the box on the right) if that event occurs.

\newcommand{\answerbox}{\fbox{\begin{minipage}{1in}\vspace{0.5 in}\hspace{1in}\end{minipage}}}
\newcommand{\answeredbox}[1]{\fbox{\begin{minipage}{1in}#1\hspace{1in}\end{minipage}}}

\begin{center}
\begin{tabular}{lcc}
\begin{minipage}{3in}
Pete guesses right AND \\
either Amy or Bill guesses right
\end{minipage} & \answerbox & \answerbox \\
\begin{minipage}{3in}
Pete guesses right AND \\
both Amy and Bill guess wrong
\end{minipage}& \answerbox & \answerbox \\
\begin{minipage}{3in}
Pete guesses wrong AND \\
either Amy or Bill guesses right
\end{minipage} & \answerbox & \answerbox \\
\begin{minipage}{3in}
Pete guesses wrong AND \\
both Amy and Bill guess wrong
\end{minipage} & \answerbox & \answerbox \\
\end{tabular}
\end{center}

\solution{
\begin{center}
\begin{tabular}{lcc}
\begin{minipage}{3in}
Pete guesses right AND \\
either Amy or Bill guesses right
\end{minipage} & \answeredbox{$1/8$} & \answeredbox{1} \\
\begin{minipage}{3in}
Pete guesses right AND \\
both Amy and Bill guess wrong
\end{minipage}& \answeredbox{$1/8$} & \answeredbox{$4$} \\
\begin{minipage}{3in}
Pete guesses wrong AND \\
either Amy or Bill guesses right
\end{minipage} & \answeredbox{$3/8$} & \answeredbox{$-2$} \\
\begin{minipage}{3in}
Pete guesses wrong AND \\
both Amy and Bill guess wrong
\end{minipage} & \answeredbox{$3/8$} & \answeredbox{$0.25$} \\
\end{tabular}
\end{center}
}

\ppart{4} What is Poor Pete's expected profit?

\solution[\newpage]{
    \[ \frac18\cdot1 + \frac18\cdot4 + \frac38\cdot(-2) + \frac38\cdot(0.25) = -\frac{1}{32}. \]
}
\eparts


\end{problem}


\begin{problem}{12}
Let $T$ be a positive integer. 
Consider the following recurrence equation:
    \begin{equation*}
        t(n) =  \frac{t(n+1)}{3} + \frac{2t(n-1)}{3} + 1 \text{ for } 1 \leq n \leq T - 1; \qquad t(0) = t(T) = 0.
\end{equation*}

Find a closed form solution for $t(n)$ for $0 \leq n \leq T$ as a function of $T$.

\solution[\newpage]{
The characteristic equation is  $x = \frac{x^2}{3} + \frac{2}{3}$. Thus we have that:

\[ x^2 - 3x + 2 = 0, \]
and so
\[ x = 1\mbox{ or } x=2 . \]

Thus the general homogenous solution is
\[ t_h(n) = a + b2^n. \]

We need to find some specific solution. A constant won't work, since it's already a soution to the homogenous problem. 
Thus, let's try $f(n) = pn + q$:

\[ pn + q  = \frac{pn + p + q}{3} + \frac{2pn -2p + 2q}{3}+ 1 \]
This has a solution $p=3$ (and for any value of $q$), so $t_p(n) = 3n$ is a particular solution.
Adding the homogenous solution, we obtain the general solution:
\[ t(n) = a + b 2^n + 3n. \]
Using $t(0) = 0$ and $t(T) = 0$, then:
\[ a  + b = 0 \mbox{ and } a + b2^T + 3T = 0. \]
Solving this, we obtain
\[ a = -b \mbox{ and } b(2^T -1 ) = -3T, \]
and hence
\[ a = - \frac{3T}{ 1-2^T } \mbox{ and } b = \frac{3T}{ 1-2^T }. \]
Thus
\[ t(n) = - \frac{3T}{1-2^T} + \frac{3T}{1-2^T}2^n + 3n. \]
}

\end{problem}

\begin{problem}{10}
Determine a closed form formula for the following sum (here, $n$ is a positive integer):
\[ \sum_{i=1}^n \sum_{j=i}^n \frac{1}{j}. \]
\end{problem}
\solution[\newpage]{Swapping the order of summation, we obtain
    \begin{align*}
        \sum_{i=1}^n \sum_{j=i}^n \frac{1}{j} &= \sum_{j=1}^n \sum_{i=1}^j \frac{1}{j}\\
                                              &= \sum_{j=1}^n \frac{1}{j} \sum_{i=1}^j 1\\
                                              &= \sum_{j=1}^n \frac{1}{j}\cdot j\\
                                              &= \sum_{j=1}^n 1\\
                                              &= n.
    \end{align*}
}


%\begin{problem}{10}
%    Prove that for all integers $n \geq 1$, 
%    \[ 4 \mid 3^{2n-1} + 1. \]
%\end{problem}
%\emph{Too easy?}
\begin{problem}{15}
    The \emph{Lucas numbers} $L_n$ are defined by the following recurrence:
%    For $n \geq 0$, let $F_n$ be the $n$'th Fibonacci number:
    \[ L_0 = 2, \quad L_1 = 1, \quad L_n = L_{n-1} + L_{n-2} \quad \text{for all } n \geq 2. \]
Prove by induction that for all $n \geq 1$, 
\[ \sum_{j=0}^n L_j = L_{n+2} - 1. \]
\end{problem}

\solution[\newpage]{
    Let $P(n)$ be the statement that $\sum_{j=0}^n L_j = L_{n+2} - 1$, for some fixed $n$.

    Base case: $\sum_{j=0}^1 L_1 = 2 + 1 = 3$. $L_3 -1= L_1 + L_2 -1= 2L_1 + L_0 -1= 3$. So $P(1)$ holds.
    (Alternatively: you could show that in fact $P(0)$ holds and use that as the base case.)

    Now assume for the purposes of induction that $P(n)$ holds. 
    Thus
    \[ \sum_{j=0}^n L_j = L_{n+2} - 1. \]
    Then
    \begin{align*}
        \sum_{j=0}^{n+1} L_j &= \sum_{j=0}^n L_j \;+\; L_{n+1}\\
                             &= (L_{n+2} - 1) \;+\; L_{n+1} \qquad \text{by inductive assumption}\\
                             &= L_{n+3} - 1 \qquad \text{by recurrence relation}.
    \end{align*}
    Hence $P(n+1)$ holds. Thus $P(n)$ holds for all $n \geq 1$ by induction.
}
%\emph{Alternative: Prove that $F_{3n}$ is even for every $n \geq 1$, where $F_n$ is the $n$'th Fibonacci number.}


\begin{problem}{15}

    Let $T$ be a tree with $n$ nodes, where $n$ is a positive integer.
Suppose we color each node of $T$ in a random way: each node is red with probability $1/3$, green with probability $1/3$ and blue with probability $1/3$, and the colors of distinct nodes are mutually independent.

    Find a formula for the probability that the resulting coloring is a proper coloring. 
    You do not need to give a full proof of your answer, but do include your reasoning. 

    \emph{(Hint: you may use the fact that any tree on at least $2$ nodes has at least one leaf.)}
%    You must give a proof that your formula is correct.

%    \emph{Answer is $(2/3)^{n-1}$. Argument: More than one way, but could e.g. do induction upon removing a leaf.}
\end{problem}
\solution[\newpage]{
The answer is $(2/3)^{n-1}$, as we will prove by induction.

Let $P(n)$ be the statement that a random coloring of any tree of $n$ nodes is a proper coloring with probability $(2/3)^{n-1}$.
Note that $P(1)$ holds; the coloring is always proper, and so the probability is $1$.

Now suppose $P(n)$ holds, and let $T$ be any tree with $n+1$ nodes. 
Let $v$ be any leaf of $T$, let $u$ be the node adjacent to $v$, and let $T'$ be the tree obtained by removing $v$.
Color $T$ randomly as prescribed. 
Since $T'$ has $n$ nodes, we know that the probability that the coloring restricted to $T'$ is proper is exactly $(2/3)^{n-1}$. 
Let us condition on this event; given that $T'$ is properly colored, what is the probability that $T$ is properly colored? 
This is just $2/3$; no matter the coloring of $T'$, we get a proper coloring of $T$ precisely if $v$ is colored a different color to $u$, which happens with probability $2/3$.

So the probability that $T$ is colored properly is
\begin{align*}
    \Pr[T \text{ properly colored}] &= \Pr[T \text{ properly colored} \mid T' \text{ properly colored}]\Pr[T' \text{ properly colored}]\\
                                    &= \frac23 \cdot (2/3)^{n-1}\\
                                    &= (2/3)^n.
\end{align*}
Hence $P(n+1)$ holds. Thus $P(n)$ holds for all $n \geq 1$ by induction.
}

\begin{comment}
\begin{problem}{15}

Prove that every connected graph $G = (V, E)$ with $|V| \geq 2$ contains at least two vertices $v_1, v_2$ such that $G-v_1$ is still connected, and $G-v_2$ is still connected.

\emph{A potential issue: one proof for this is simply: take any spanning tree of $G$ (we know this exists, from class). Any spanning tree has at least 2 leaves (in the book, so fair game). Removing any of these leaves, the graph must remain connected. 
Now that I write this down, maybe they wouldn't find this so easy to see anyway. So probably it's fine.}
\solution{

\textbf{} \\

\textbf{Proof:} \emph{(by strong induction)}

\textbf{Inductive Hypothesis:} $P(n)$: Every connected graph with $n$ vertices contains at least two vertices $v_1, v_2$ such that $G-v_1$ is still connected, and $G-v_2$ is still connected.

\textbf{Base Case:} $P(2)$: Both vertices of $G$ satisfy the condition, as a subgraph of one node is trivially connected.

\textbf{Inductive Step:} Show that $P(2) \wedge P(3) \wedge \cdots \wedge P(n) \Rightarrow P(n+1)$.

Assume for the purposes of induction that $P(2) \wedge P(3) \wedge \cdots \wedge P(n)$ holds.

\textbf{Case I:} $\forall v \in V$, $G-v$ is connected. Then $P(n+1)$ holds.

\textbf{Case II:} $\exists v$ such that $G-v$ is not connected. Then $G-v$ is composed of at least two connected components $G_1, G_2, \dots, G_k$. Let $G_x'$ be the graph formed by adding back $v$ and only those edges that are incident to both $v$ and some vertex in $G_x$. Then, each $G_x'$ is a connected graph of at least two vertices but not more than $n$ vertices.

By strong induction, each $G_x'$ contains at least one vertex $v_i \ne v$ such that $G_x' - v_i$ is connected. Then $G - v_i$ is also connected (since $v$ "bridges" all the other connected components). Since we have at least two connected components, then we indeed have at least two vertices such that removing each vertex from $G$ leaves the remainder of the graph still connected.

}


\end{problem}
\end{comment}

\begin{problem}{11}
You have twelve cards:
%
\[
\fbox{1}\ \fbox{1}\quad \fbox{2}\ \fbox{2}\quad \fbox{3}\ \fbox{3}\quad
\fbox{4}\ \fbox{4}\quad \fbox{5}\ \fbox{5}\quad \fbox{6}\ \fbox{6}
\]
%
You shuffle them well, and deal them in a row (so the ordering will be a uniformly random permutation).  For example, you might get:
%
\[
\fbox{1}\ \fbox{2}\ \fbox{3}\ \fbox{3}\ \fbox{4}\ \fbox{6}\
\fbox{1}\ \fbox{4}\ \fbox{5}\ \fbox{5}\ \fbox{2}\ \fbox{6}
\]
%
What is the expected number of adjacent pairs with the same value?  In
the example, there are two adjacent pairs with the same value, the 3's
and the 5's.

\end{problem}

\solution[\newpage]{
Let $I_j$ be the indicator of the event $A_j$ that the $j$'th card from the left is the same as the $(j+1)$'st card, for $1 \leq j \leq 11$.
Let $R$ be the r.v. for the number of adjacent pairs. Then
\[ R = \sum_{j=1}^{11} I_j; \]
thus by linearity of expectation, 
\[ \ex{R} = \sum_{j=1}^{11} \ex{I_j} = \sum_{j=1}^{11} \Pr[A_j]. \]

%Fix some $j \in \{1,2,\ldots, 11\}$; we want to determine $\Pr[A_j]$. 
%Imagine constructing the random permutation of the cards by first putting down a card of value $j$ in a uniformly random position, and then making a uniformly random choice from the remaining positions to place the the second card of value $j$. (After that all the remaining cards are placed randomly, but we don't care what happens to them.)
%Then 
%\medskip
%
%\emph{An alternative way of calculating $\Pr[A_j]$:}

Fix some $j \in \{1, 2, \ldots, 11\}$. Now we count the number of possible layouts for which cards $j$ and $j+1$ are the same. This is just 
\[ 6 \cdot \frac{10!}{(2!)^5}, \]
by the BOOKKEEPER lemma.
%TODO explanation
%since we can choose the number of the cards in spots $j$ and $j+1$, and then what remains is a word of length $10$ chosen 
On the other hand, the total number of possible layouts is
\[ \frac{12!}{(2!)^6}, \]
by similar reasoning. Thus
\[ \Pr[A_j] = \frac{610! / (2!)^5}{12!/(2!)^6} = \frac{2\cdot 6}{12 \cdot 11} = \frac{1}{11}. \]
Thus
\[ \ex{R} = 11 \cdot \frac1{11} = 1. \]
}

\begin{comment}
\begin{problem}{10}
Miss McGillicuddy never goes outside without a collection of pets. In particular:

\begin{itemize}
\item  She may or may not bring her alligator, Freddy.
\item She brings a number of songbirds, which always come in pairs.
\item She brings a positive number of chihuahuas and labradors leashed together in a line.
\end{itemize}


Let $p_n$ denote the number of different collections of $n$ pets that can accompany her, where we
regard chihuahuas and labradors leashed up in different orders as different collections, even if
there are the same number chihuahuas and labradors leashed in the line.

For example $p_2$ = 6 since there are 6 possible collections of 2 pets:
\begin{itemize}
\item 0 songbirds, 1 alligator, 1 chihuahua
\item 0 songbirds, 1 alligator, 1 labrador

\item 0 songbirds, 0 alligators, 2 labradors leashed in a line

\item 0 songbirds, 0 alligators, a labrador leashed behind a chihuahua

\item  0 songbirds, 0 alligators, a chihuahua leashed behind a labrador

\item 0 songbirds, 0 alligators, 2 chihuahuas leashed in a line
\end{itemize}


\bparts
 \ppart{5}
Let 
\[
P(x)= p_0 + p_1 x + p_2 x^2 + p_3  x^3 + p_4 x^4 _ ...
\]

be the generating function for the number of Miss McGillicuddy’s pet collections. Verify that
\[
P(x)= \frac{2x}{(1-x)(1-2x)}
\]
\solution[\newpage]{
$1+x$ is the generating function for her alligator, $\frac{1}{(1-x^2)}$ is the generating function for songbirds, and  $\frac{2x}{(1-2x)}$ is the generating function for her dogs. Multiplying these gives $P(x)$ as desired.
}
 \ppart{5}
Find Integers $A$ and $B$ which satisfy 
\[
P(x)= \frac{A}{(1-x)} + \frac{B}{(1-x)}
\]


\solution[\vspace{6cm}]{
$A=-2$, $B=2$
}
 \ppart{5}
Find a closed-form, simple formula for $p_n$ which is only in terms of $n$.

\solution{
$p_n=2^{n+1}-2$
}


\eparts

\end{problem}
\end{comment}


\begin{problem}{14}

T-Pain is planning an epic boat trip and he needs to decide what to bring with him.

\begin{itemize}

    \item He \emph{definitely} wants to bring burgers, but they only come in packs of 6.

\item He and his two friends can't decide whether they want to dress formally or
casually. He'll either bring 0 pairs of flip flops or 3 pairs.

\item He doesn't have very much room in his suitcase for towels, so he can
  bring at most 2 (and might not bring any!)

\item In order for the boat trip to be truly epic, he has to bring at least 1
nautical-themed pashmina afghan.

\end{itemize}

\bparts

\ppart{7} Let $g_n$ be the the number of different ways for T-Pain to bring
$n$ items (burgers, pairs of flip flops, towels, and/or afghans) on his
boat trip, satisfying the restrictions above.  Express the generating function $G(x) \eqdef
\sum_{n=0}^{\infty} g_nx^n$ as a quotient of polynomials.


\solution[\newpage]{

\begin{align*}
    G(x) &=\frac{x^6}{1-x^6}(1+x^3)(1+x+x^2) \frac{x}{1-x}\\
  &  = \frac{(1+x^3)(1+x+x^2)x^7}%
            {(1-x^3)(1+x^3)(1-x)}\\
  &  = \frac{(1+x+x^2)x^7}%
            {(1-x)(1+x+x^2)(1-x)}\\
  & = \frac{x^7}{(1-x)^2}
\end{align*}

}

\ppart{7} Let $H(x) \eqdef \sum_{n=0}^\infty h_nx^n$ be the generating function for the sequence $h_0, h_1, h_2, \ldots$ representing the number of ways T-Pain could write an epic book of $n$ chapters about his boat trip. It turns out that
\[ H(x) = \frac{3-2x}{(1-x)^2} - 2. \]
Using this information, determine a closed formula for $h_n$.

\solution[\newpage]{
    We have
    \[ H(x) = \frac{1}{(1-x)^2} + \frac{2}{1-x} - 2. \]
    Recall that 
    \[ \frac{1}{1-x} = 1 + x + x^2 \cdots = \sum_{n=0}^\infty x^n. \]
    Thus
\begin{align*}
\frac{1}{(1-x)^2} &= \frac{d}{dx}\left(\frac{1}{1-x}\right)\\
                  &= \sum_{n=0}^\infty\frac{d}{dx} x^n\\
                  &= \sum_{n=1}^\infty nx^{n-1}\\
                  &= \sum_{n=0}^{\infty}(n+1)x^n.
\end{align*}
Hence
\begin{align*}
    H(x) &= \sum_{n=0}^\infty (n+1)x^n + 2\sum_{n=1}^\infty x^n \;-\; 2\\
         &= 1 + \sum_{n=1}^\infty (n+3)x^n.
\end{align*}
So $h_n = n+3$ for $n \geq 1$, and $h_0 = 1$.
}

\eparts

\end{problem}


\begin{problem}{16}
    Clumsy Clarke is rather injury prone:
    \begin{itemize}
        \item Every time he enters his car, which he does $72$ times a month, he bumps his head with probability $1/6$. 
        \item Each time he enters his house, which he does $32$ times a month, he nicks his finger with probability $1/2$.
        \item Every time he does some shopping, which he does $25$ times a month, he drops a bag on his foot with probability $1/5$.
    \end{itemize}
    All of these events are mutually independent.

    %A gambler plays 120 hands of draw poker, 60 hands of black jack, and 20
 % hands of stud poker per day.  He wins a hand of draw poker
 % with probability $1/6$, a hand of black jack with probability $1/2$, and a
 % hand of stud poker with probability $1/5$.  Assume the outcomes of the card
 % games are mutually independent.

\bparts

\ppart{4} What is the expected number of injuries Clumsy Clarke experiences in a month?

\solution[\vspace{7cm}]
{
    Let $X$ be the number of injuries Clarke experiences.
    \[ \ex{X} = 72 / 6 + 32 / 2 + 25 / 5 = 33. \]
}

\ppart{4} What is the variance in the number of injuries Clarke has in a month?

\solution[\newpage]{
\[   \Var[X] =  72\cdot \tfrac16\cdot\tfrac56 + 32\cdot\tfrac12\cdot\tfrac12 + 25\cdot\tfrac15\cdot\tfrac45 = 22. \]
}

\ppart{4} What would the Markov bound be on the probability that Clarke has 100 or more injuries in a month?

  \solution[\vspace{8cm}]{
      \[ \pr{X \geq 100} \leq \frac{33}{100}. \]
  }

 \ppart{4} What would the Chebyshev bound be on the probability that Clarke has 100 or more injuries in a month?

  \solution[\newpage]{
  \[ \pr{X - 33 \geq 67} \leq \pr{|X-\ex{X}| \geq 67} \leq \frac{\Var[X]}{67^2} = \frac{22}{67^2}. \]
  }
\eparts
\end{problem}


\begin{problem}{15}
Alyssa, an industrious software developer, writes 100 lines of code each hour, with probability $p$ of making a mistake each hour (she never makes more than one mistake in an hour, though). Assuming Alyssa works $n$ hours in a day and the mistakes she makes in each hour are independent, give formulas for the following:

\bparts
\ppart{4} The probability of exactly $k$ mistakes in a day.

\solution[\vspace{7cm}]{Let $M$ be the random variable equal to the number of mistakes in
$n$ hours.  Then $M$ has binomial distribution with parameters $n,p$, so
\[
\pr{M=k} = \binom{n}{k}p^k(1-p)^{n-k}
\]
for $0\leq k \leq n$.}

\ppart{5} The probability of at least one mistake in a day.

\solution[\newpage]{\[
\prob{M>0}  = 1-\prob{\text{No Mistake}} = 1-(1-p)^n.
\]
}
%TODO this is harder and should be worth more points; earlier parts are v. easy, could be less.
\ppart{6} %The expected number of hours before the first mistake, or until the end of the work day (whichever comes first.
The expected number of hours until either the first mistake, or the end of the work day, whichever comes first. (Assume that if Alyssa makes a mistake in some hour, she makes it at the beginning of that hour).

\solution[\newpage]{Let $H$ be a random variable representing the number of hours
before the first mistake.  Calculating $\expect{H}$ is similar to finding
mean time to failure, except that we stop after $n$ hours.

\begin{eqnarray*}
\expect{T} &=& \sum_{i=0}^{\infty} \prob{H>i} \\
           &=& \sum_{i=0}^{n-1} \prob{H>i} \\
           &=& \sum_{i=0}^{n-1} (1-p)^i\\
           &=& \frac{1-(1-p)^n}{p}.
\end{eqnarray*}
}

\eparts

\end{problem}
%\begin{problem}
%You are trying to get to your hotel, but you've forgetten where it is! You only remember the general direction is north-east. So at each intersection, you randomly decide to go north with probability $1/3$, or east with probability $2/3$. 
%
%If the hotel is in fact 10 blocks north and 16 blocks east of you, what is the probability that you find your way back?
%\end{problem}
%\begin{problem}{15}
%    In a ``powerball'' lottery, there are two kinds of balls: there are $m$ white balls, numbered $1, 2, \ldots, m$, and there are $k$ red balls (numbered $1, 2, \ldots, k$). 
%
%    In a drawing, one red ball and $n$ distinct white balls ($n \leq m)$ are chosen; the order of the white balls doesn't matter. 
%    Players choose a lottery ticket by stating their guess for the $n$ white balls and the single red ball (the ``power ball'). 
%
%    \bparts
%    \ppart{5} What is the probability that a player wins?
%
%    \ppart{5} What is the probability that 
%    \eparts
%\end{problem}

%\begin{problem}{15}
%Call a number ``good'' if it contains each of the numbers 6, 0, 4 and 2 precisely once, and in that order (but not necessarily consecutively). So for example, $7690974219$ is a good number.
%
%\bparts
%\ppart{5} Suppose I choose a random 10-digit number, uniformly at random (note that a 10-digit number cannot start with a 0!). What is the probability that the number is good?
%
%\eparts
%\end{problem}



\begin{problem}{15}
    Consider the following game. You have the following grid:

    \begin{center}
    \begin{tabular}{|c|c|c|c|c|}
        \hline
        1 & 2 & 3 & 4 & 5\\
        \hline
        6 & 7 & 8 & 9 & 10\\
        \hline
        11 & 12 & 13 & 14 & 15\\
        \hline
        16 & 17 & 18 & 19 & 20\\
        \hline
        21 & 22 & 23 & 24 & 25\\
        \hline
    \end{tabular}
\end{center}

    There are 25 balls in a bucket, numbered from 1 to 25. 
    7 of these balls are randomly chosen from the bucket. You cross out the 7 numbers on your grid corresponding to the selected balls. 
    If you cross out an entire row, an entire column, or either of the diagonals, then you win! E.g., if the draw yields balls 2, 6, 7, 12, 15, 17 and 22, then you win.

    What is the probability of winning?
\end{problem}

\solution[\newpage]{
    Call a row, column, or diagonal of boxes a \emph{line}.
    Fix any line. The probability that we cross out all the squares in this line is
    \[ \frac{{20 \choose 2}}{{25 \choose 7}}. \]
    We can never cross out more than one line at the same time, so we just need to multiply by the number of possible lines. Thus the total probability is
    \[ 12 \cdot \frac{{20 \choose 2}}{{25 \choose 7}}. \]
}
%    A complete binary tree with $r$ levels is a tree with a single root node at level $1$, and such that every node at level $i$ for $1 \leq i < r$ has precisely 2 children; all nodes on level $r$ have no children. Here is a complete binary tree with $3$ levels:

 %   TODO
%
%    Let $T$ be a complete binary tree with $r$ levels. 
%Suppose we color each node of $T$ either red, blue, or green, in a random way: each node is independently red with probability $1/3$, green with probability $1/3$ and blue with probability $1/3$.
%
%    Prove that the probability that the resulting coloring is a proper coloring is 
%
%    \emph{Argument:
%
%        \emph{An easier problem would be to do this with $2$ colors. In this case, the problem could be: part a) show that $T$ has $2^{r}-1$ nodes, and b) determine the probability, which is then just $2/2^{2^r-1}$.}

\begin{problem}{15}
    %A rock-paper-scissors tournament is 
    Consider the following tennis tournament. There are $n$ players, and every pair of players play against each other once.
    Moreover, all the players are equally matched, and so the winner of each matchup is uniformly random; the game outcomes are also mutually independent.

    Call a player \emph{awesome} if they win all their games, and \emph{terrible} if they lose all of them.

    \bparts
    \ppart{5}  What is the probability that there will be an awesome player?

    \solution[\newpage]{
    Consider an arbitrary fixed player $i$, and let $E_i$ be the event that player $i$ is awesome. 
    Then $\pr{E_i} = 2^{-(n-1)}$, since to be awesome player $i$ must win all their $n-1$ games. 
    Now observe that there cannot be more than one awesome player (since between two players, whoever lost their match cannot be awesome).
    So all the events $E_i$ are disjoint; thus
    \[ \pr{E_1 \cup E_2 \cdots \cup E_n} = \sum_{i=1}^n \pr{E_i} = n2^{-(n-1)}. \]
    }
    \ppart{5} What is the probability that there will be both an awesome player and a terrible player?

    \solution[\vspace{12cm}]{
        Let $E_{i,j}$ be the probability that player $i$ is awesome and player $j$ is terrible, for $i,j \in \{1,2, \ldots, n\}$ and $i \neq j$. 
        Then
        \[ \pr{E_{i,j}} = 2^{-(2n-3)}, \]
        since player $i$ must win all her matches, and player $j$ must lose all their matches, thus fixing the outcome of precisely $2(n-1)-1 = 2n-3$ matches (the match between player $i$ and player $j$ must only be counted once).
        Again, the events $E_{i,j}$ are all disjoint, and so
        \[ \pr{\text{both an awesome and a terrible player}} = \sum_{i=1}^n \sum_{j \in \{1, 2, \ldots n\} : j \neq i} E_{i,j} = n(n-1) 2^{-(2n-3)}. \]
    }

    \ppart{5} What is the probability that there will be neither an awesome player nor a terrible player?

    \solution{
        The probability that there is a terrible player is the same as the probability that there is an awesome player. Thus, using the inclusion-exclusion principle,
    \begin{align*}
    \pr{\text{no awesome or terrible player}} &= 1 - 2n2^{-(n-1)} + n(n-1)2^{-(2n-3)}\\
                                              &= 1 - n2^{-(n-2)} + n(n-1)2^{-(2n-3)}.
    \end{align*}
}
    \eparts

%    \emph{This is another counting question really. One tricky point is that there can be simultaneously an awesome and useless player (but never more than one of each), so inclusion-exclusion is needed. So a bit tricky---but not so difficult.}
\end{problem}


\end{document}


