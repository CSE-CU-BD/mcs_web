\documentclass[handout]{mcs}

\begin{document}

\inclassproblems{7, Mon.}

%%%%%%%%%%%%%%%%%%%%%%%%%%%%%%%%%%%%%%%%%%%%%%%%%%%%%%%%%%%%%%%%%%%%%
% Problems start here
%%%%%%%%%%%%%%%%%%%%%%%%%%%%%%%%%%%%%%%%%%%%%%%%%%%%%%%%%%%%%%%%%%%%%

\begin{staffnotes}
Diagonal Arguments; Ch. 8.1.3; 8.2 optional, 8.3--4

In F17 we started with PS\_off\_diagonal\_arguments since it had a
builtin review of the diagonal argument that we discovered students
needed.  In the last session, coaches were instructed to intervene and
offer an explanation after 45min.
\end{staffnotes}

\insolutions{
It seemed that most students needed a review of diagonal arguments, so
Teams were instructed to start with Problem 3.  Almost all of the
teams in the early sessions spent the whole 1.5 hour session on this
problem.
}

\pinput{PS_A_to_B_diagonal_argument}

\pinput{TP_majorizing}

\pinput{PS_off_diagonal_arguments}

\begin{center}
\textbf{Supplemental (Optional)}
\end{center}

\pinput{CP_power_set_tower}

%\pinput{TP_countable_subsets_integers}

%\pinput{PS_add_countable_elements}

%\pinput{TP_countable_chain}

%\pinput{TP_finitely_discontinuous}

%\pinput{CP_computable_reducibility}

%% \begin{center}
%% \textbf{Supplemental (Optional)}
%% \end{center}

%% \pinput{CP_Schroeder_Bernstein_theorem}

%busy beaver

%program size?

%%%%%%%%%%%%%%%%%%%%%%%%%%%%%%%%%%%%%%%%%%%%%%%%%%%%%%%%%%%%%%%%%%%%%
% Problems end here
%%%%%%%%%%%%%%%%%%%%%%%%%%%%%%%%%%%%%%%%%%%%%%%%%%%%%%%%%%%%%%%%%%%%%

\end{document}
