\documentclass[handout]{mcs}

\begin{document}

%%%%%%%%%%%%%%%%%%%%%%%%%%%%%%%%%%%%%%%%%%%%%%%%%%%%%%%%%%%%%%%%%%%%%
% Problems start here
%%%%%%%%%%%%%%%%%%%%%%%%%%%%%%%%%%%%%%%%%%%%%%%%%%%%%%%%%%%%%%%%%%%%%

\documentclass[problem]{mcs}

\chapter{Infinite Cardinality, Number Theory I-II (GCD, Modular Arithmetic}


%% Siggi's Original Problems
\begin{enumerate}

\item If $A_n$ is countable for all $n \in \mathbb{N}$ prove that $$\cup_{n \in\mathbb{N}}A_n$$ is countable.

\item Prove that $pow(\mathbb{N})$ is uncountable.

\item Prove that the subset of a countable set is countable.

\item Sean wants to collect all pythagorean triples in a set. Show that this set is countable. (A pythagorean triple $(a,b,c) \in \mathbb{Z}^3$ is such that $a^2+b^2=c^2$).

\item Show that if $A$ is an infinite set and $C$ is countable then $A bij A\cup C$.

\item If $a=b \mod m$ and $d|m$, $d>0$ then $a=b \mod d$. 

\item Use CRT to find consecutive integers $a,b, (a+1=b)$ such that $4|b$ and $9|a$.

\item Prove that the $gcd(a,b)*lcm(a,b)=a*b$. Does this hold for three integers, i.e. $gcd(a,b,c)*lcm(a,b,c)=abc$?

\item Show that if $p$ is prime then ${p-1 \choose k}=(-1)^k (mod p)$ for $0\leq k \leq p-1$.

\item Show that if $p$ is prime then ${p \choose k}=0 (mod p)$ for $1\leq k \leq p-1$

\item In 1874 Iceland got its first constitution but what is $3^{1874} (\mod 6042)$?

\item What is the remainder of $2^{100}+3^{100}+4^{100}+5^{100}$ when divided by 7?

\item If $p$ is prime, prove that $(p-1)! = -1 (\mod p)$. [HINT: Think about inverses] (If this is too hard, they may take for granted that if $a = a^{-1} (\mod p)$ then $a = \pm 1 (\mod p)$.)

\item Prove that If $(p-1)! = -1 (mod p)$ then $p$ is prime.
 
\end{enumerate}
 
 % Problems more familiar to students
 %GCD, Modular Arithmetic, Euler's Theorem, Euclid's Algorithm
 \pinput{FP_Euler_theorem_calculation}
 \pinput{FP_GCD_algebra}
 \pinput{FP_multiple_choice} % Includes partial order, isomorphism
 \pinput{FP_pulverizer}
 \pinput{FP_fermats_theorem_inverse}
 
 
 
 
\chapter{DAGs and Scheduling, Partial Order, Equivalence Relations}
 % Digraphs
\pinput{PS_tangled_and_mangled_graphs}
\pinput{FP_digraph_neighbors_under_isomorphisms}
\pinput{FP_isomorphic_graphs}
\pinput{FP_digraph_triangle_inequality}
\pinput{FP_directed_graphs_and_probability} % Also has a bit of probability

% Partial orders
\pinput{TP_basic_partial_orders}
\pinput{FP_partial_order_short_answer}
\pinput{MQ_multi_schedule_tasks}
\pinput{TP_divisibility_partial_order}

% Equivalence relations
\pinput{TP_equivalence_relations_partial_orders}
\pinput{PS_equivalence_relation_operations_part_a} % Used for morning mini quiz, can be reused for afternoon final.
\pinput{PS_equivalence_relation_operations_part_b} % Used for afternoon mini quiz, can be reused for morning final.
 
 
 
 
\chapter{Linear Recurrences, Discrete Probability, Conditional Probability}
% Linear Recurrences
\pinput{FP_linear_recur_simplified}
\pinput{FP_skywalker_prob_lin_recur_gen_func} % Has a bit of probability

%Probability
\pinput{FP_probability}
\pinput{FP_red_and_blue_goats}
\pinput{FP_college_probability}
\pinput{FP_conditional_prob_inequality}
\pinput{TP_A_random_number}
\pinput{TP_conditional_probability}


\chapter{Sampling \& Confidence, Infinite Expectation}
% Sampling, Confidence
\pinput{FP_random_sampling}
%\pinput{FP_sampling_wafers} In F05 Final
\pinput{FP_sampling_concepts}

% Infinite Expectation, Random Walks, Gambler's Ruin
\pinput{FP_infinite_repeat}
\pinput{TP_Random_Walks}
\pinput{TP_Gambling_on_Coin_Flips}
\pinput{TP_Biased_Gamblers_Ruin}
\pinput{MQ_martingale}



%%%%%%%%%%%%%%%%%%%%%%%%%%%%%%%%%%%%%%%%%%%%%%%%%%%%%%%%%%%%%%%%%%%%%
% Problems end here
%%%%%%%%%%%%%%%%%%%%%%%%%%%%%%%%%%%%%%%%%%%%%%%%%%%%%%%%%%%%%%%%%%%%%
\end{document}
