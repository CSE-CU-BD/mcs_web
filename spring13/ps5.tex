\documentclass[handout]{mcs}

\begin{document}

\renewcommand{\reading}{Chapter~\bref{number_theory_chap}.\ \emph{Number
    Theory} through~\bref{Euler_sec}.\ \emph{Euler's Theorem}.}

\problemset{5}

\begin{staffnotes}
Lectures covered: Number theory I--III (not RSA)
\end{staffnotes}

%%%%%%%%%%%%%%%%%%%%%%%%%%%%%%%%%%%%%%%%%%%%%%%%%%%%%%%%%%%%%%%%%%%%%
% Problems start here
%%%%%%%%%%%%%%%%%%%%%%%%%%%%%%%%%%%%%%%%%%%%%%%%%%%%%%%%%%%%%%%%%%%%%

%% gcd
\pinput{PS_filling_buckets_with_water}
\pinput{PS_pulverizer_machine}

%% modular arithmetic
\pinput{PS_check_factor_by_digits}
\pinput{CP_pirate_treasure}
\pinput{CP_n5_last_digit}

%% Euler's function / Euler's theorem
%\pinput{CP_Euler_theorem_calculation}
\pinput{PS_Euler_function_multiplicativity}
\pinput{PS_order_divides_phi}
\pinput{PS_Euler_theorem_not_rel_prime}

%%%%%%%%%%%%%%%%%%%%%%%%%%%%%%%%%%%%%%%%%%%%%%%%%%%%%%%%%%%%%%%%%%%%%
% Problems end here
%%%%%%%%%%%%%%%%%%%%%%%%%%%%%%%%%%%%%%%%%%%%%%%%%%%%%%%%%%%%%%%%%%%%%
\end{document}

\endinput
