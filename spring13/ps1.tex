\documentclass[handout]{mcs}

\begin{document}

\renewcommand{\reading}{
Part~\bref{part:proofs}{. \emph{Proofs:
      Introduction}}; Chapter~\bref{proofs_chap}{.\ \emph{What is a
      Proof?}}; Chapter~\bref{well_ordering_chap}{.\ \emph{The Well
      Ordering Principle}}.  These
  assigned readings do \textbf{not} include the Problem sections.  (Many
  of the problems in the text will appear as class or homework problems.)}

\problemset{1}

\medskip

\textbf{Reminders}:
\begin{itemize}
\item 
  You are \textbf{required} to post some
  \href{http://courses.csail.mit.edu/6.042/spring13/courseinfo#comments}{comments}
  on the assigned reading using the class
  \href{http://nb.csail.mit.edu}{NB annotation system} by the end of
  the second week of class (Feb.\ 15).  After that commenting is
  encouraged but optional.

\item The class also has a
  \href{http://piazza.com/mit/spring2013/6042j18062j/home}{Piazza
    forum} structured to get you help fast and efficiently from
  classmates and staff.  With Piazza you may post questions---both
  administrative and content related---to the entire class or to just
  the staff.  You are likely to get faster response through Piazza
  than from direct email to staff.  Posting questions or comments to
  Piazza is optional.

\item Problems should be submitted separately following the pset
  \href{http://courses.csail.mit.edu/6.042/spring13/submission.shtml}
       {submission instructions}, and each problem should have an attached
  \href{http://courses.csail.mit.edu/6.042/spring13/submission_template.pdf}
       {\emph{collaboration statement}}.

 \end{itemize}

%%%%%%%%%%%%%%%%%%%%%%%%%%%%%%%%%%%%%%%%%%%%%%%%%%%%%%%%%%%%%%%%%%%%%
% Problems start here
%%%%%%%%%%%%%%%%%%%%%%%%%%%%%%%%%%%%%%%%%%%%%%%%%%%%%%%%%%%%%%%%%%%%%


%\pinput{PS_log2_of_3_irrational}

\pinput{PS_log7_not_in_QZ}

\pinput{PS_3_exponent_inequality}

%\pinput{PS_printout_binary_strings}

%\pinput{CP_PorQorR_equiv}

%\pinput{CP_truth_table_for_distributive_law}

%\pinput{CP_valid_vs_satisfiable}

\pinput{PS_prime_polynomial_41}

\iffalse
\begin{center}
\large \textbf{Optional:}
\end{center}

\pinput{PS_faster_adder_logic}
\fi

%%%%%%%%%%%%%%%%%%%%%%%%%%%%%%%%%%%%%%%%%%%%%%%%%%%%%%%%%%%%%%%%%%%%%
% Problems end here
%%%%%%%%%%%%%%%%%%%%%%%%%%%%%%%%%%%%%%%%%%%%%%%%%%%%%%%%%%%%%%%%%%%%%
\end{document}
