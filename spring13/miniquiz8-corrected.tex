\documentclass[quiz]{mcs}

\begin{document}

\renewcommand{\exampreamble}{   % !! renew \exampreamble
  \begin{tabular}{l}
    \textbf{Circle your}\quad   \teaminfo
  \end{tabular}

  \begin{itemize}

  \item
   This exam is \textbf{closed book} except for a 1-sided cribsheet.
   Total time is 20 minutes. 

  \item
   Write your solutions in the space provided.  If you need more
   space, write on the back of the sheet containing the problem.
%   Please keep your entire answer to a problem on that problem's page.

\iffalse
  \item
   GOOD LUCK!
\fi

  \end{itemize}}

\miniquiz{8}

%%%%%%%%%%%%%%%%%%%%%%%%%%%%%%%%%%%%%%%%%%%%%%%%%%%%%%%%%%%%%%%%%%%%%
% Problems start here
%%%%%%%%%%%%%%%%%%%%%%%%%%%%%%%%%%%%%%%%%%%%%%%%%%%%%%%%%%%%%%%%%%%%%
%\pinput[points = 1]{TP_count_simple_graphs}

%\pinput[points = 3]{MQ_count_double_deck}

\begin{staffnotes}
Grading at granulity of 1/2 points.  Total sum for all problems
should be rounded up.
\end{staffnotes}

\pinput[points = 1]{TP_count_birthday_pairs}

Problem 2 of the original exam was mistaken.  There are two correct
versions given as parts (a) and (b):

\begin{problem}

\bparts

\ppart 
The mapping from permutations of the subscripted letters
\iffalse
$\text{c}\,\text{a}_1\,\text{l}_1\,\text{l}_2\,\text{a}_2\,\text{b}\,\text{a}_3\,\text{s}\,\text{h}$
\fi
$\text{c},\text{a}_1,\text{l}_1,\text{l}_2,\text{a}_2,\text{b},\text{a}_3,\text{s},\text{h}$
  to permutations of the letters in the word ``callabash'' is
  $n$-to-one for $n=$\hfill \examrule[0.7in]

\begin{solution}
\[
3! 2!
\]
\end{solution}
\begin{staffnotes}
In repository as~\bref{TP_bookkeeper_erase_subs} TP\_bookkeeper\_erase\_subs.
\end{staffnotes}

\ppart 
Fill in the boxed entries in the multinomial coefficent below: the
number of permutations of the letters in the word ``callabash'' is
\[
\paren{\begin{array}{c}
\exambox{0.2in}{0.0in}{0.2in}\\
\\
\exambox{1.2in}{0.2in}{0.0in}
\end{array}}
\]

\begin{staffnotes}
As a 2 point problem: 1 point if bottom 1's are omitted; also 1 point
if ``9'' is replaced by ``8'' or ``10.''  No points if anything other
on bottom than one 2, one 3, and some 1's.
\end{staffnotes}

\begin{solution}
\[
\binom{9}{3,2,1,1,1,1}
\]
\end{solution}

\eparts
\end{problem}

\pinput[points = 2]{TP_bookkeeper_erase_subs}

\pinput[points = 1]{TP_pigeonhole_suits}

\pinput[points = 2]{TP_gen_func_coefficient}

%\pinput[points = 2]{TP_gen_func_notation}

\examspace

\begin{staffnotes}
Rubrics for these next problems should be developed by the graders,
with notes kept and final rubrics added to problem staff notes.
\end{staffnotes}

\pinput[points = 2]{TP_comb_proof_binom}

\pinput[points = 2]{TP_inc-exc-empty}


%%%%%%%%%%%%%%%%%%%%%%%%%%%%%%%%%%%%%%%%%%%%%%%%%%%%%%%%%%%%%%%%%%%%%
% Problems end here
%%%%%%%%%%%%%%%%%%%%%%%%%%%%%%%%%%%%%%%%%%%%%%%%%%%%%%%%%%%%%%%%%%%%%
\end{document}
