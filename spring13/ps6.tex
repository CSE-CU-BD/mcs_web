\documentclass[handout]{mcs}

\begin{document}

\renewcommand{\reading}{Section~\bref{RSA_sec}.\ \emph{RSA (Number
    Theory)}
  through~\bref{relation_compose_subsec}.\ \emph{Composition of
    Relations}.}

\problemset{6}

\begin{staffnotes}
(Sharon: Draft Version) Lectures covered: Number theory (RSA), Digraphs Walks, Paths
\end{staffnotes}

%%%%%%%%%%%%%%%%%%%%%%%%%%%%%%%%%%%%%%%%%%%%%%%%%%%%%%%%%%%%%%%%%%%%%
% Problems start here
%%%%%%%%%%%%%%%%%%%%%%%%%%%%%%%%%%%%%%%%%%%%%%%%%%%%%%%%%%%%%%%%%%%%%


%RSA
\pinput{PS_Rabin_cryptosystem}
%\pinput{PS_RSA_correctness}  %subsumed by next problem
%\pinput{CP_RSA_proving_correctness} %class problem cp6w
\pinput{PS_RSA_key_implies_factoring}
\pinput{PS_RSA_reversed}

%digraph
\pinput{PS_digraph_neighbors_under_isomorphisms}
\pinput{PS_directed_Euler_circuits}
\pinput{PS_finite_transitive_closure}
\pinput{PS_random_walk_strongly_connected}
\pinput{PS_shortest_directed_closed_walk}
\pinput{PS_top_sort_for_closure_of_DAG}
\pinput{PS_walk_relation_composition}

%%%%%%%%%%%%%%%%%%%%%%%%%%%%%%%%%%%%%%%%%%%%%%%%%%%%%%%%%%%%%%%%%%%%%
% Problems end here
%%%%%%%%%%%%%%%%%%%%%%%%%%%%%%%%%%%%%%%%%%%%%%%%%%%%%%%%%%%%%%%%%%%%%
\end{document}

\endinput
