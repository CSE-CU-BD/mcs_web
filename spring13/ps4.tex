\documentclass[handout]{mcs}

\begin{document}

\renewcommand{\reading}{
Chapter~\bref{induction_chap}{.\ \emph{State Machines}};
Chapter~\bref{recursive_data_chap}{.\ \emph{Recursive Data Types}};
Chapter~\bref{set_theory_chap}{.\ \emph{Infinite Sets}}
%Chapter~\bref{number_theory_chap}{.\ \emph{Number Theory} through
%  Section~\bref{Turing_sec}{.\ Alan Turing}}
}

\problemset{4}

%%%%%%%%%%%%%%%%%%%%%%%%%%%%%%%%%%%%%%%%%%%%%%%%%%%%%%%%%%%%%%%%%%%%%
% Problems start here
%%%%%%%%%%%%%%%%%%%%%%%%%%%%%%%%%%%%%%%%%%%%%%%%%%%%%%%%%%%%%%%%%%%%%
%State Machines

\pinput{PS_robot_on_2D_grid}

%\pinput{CP_robot_invariant} %used fall11, ps4

%Recursive Data and Structural Induction

\pinput{PS_bracket_good_count} 

\pinput{PS_linear_combination_by_structural_induction}

%Infinite sets

\pinput{PS_unit_interval}

%Uncountibility
%\pinput{FP_uncountable_infinite_sequences} %used in ocw spring10 with solutions

%Number Theory  %should be moved to week 5

%\pinput{PS_non_unique_factoring}  %used Spring12, ps4

%\pinput{PS_pulverizer_machine}


%%%%%%%%%%%%%%%%%%%%%%%%%%%%%%%%%%%%%%%%%%%%%%%%%%%%%%%%%%%%%%%%%%%%%
% Problems end here
%%%%%%%%%%%%%%%%%%%%%%%%%%%%%%%%%%%%%%%%%%%%%%%%%%%%%%%%%%%%%%%%%%%%%
\end{document}

\endinput
