\documentclass[handout]{mcs}

\begin{document}

\microquiz{Feb. 20}

%%%%%%%%%%%%%%%%%%%%%%%%%%%%%%%%%%%%%%%%%%%%%%%%%%%%%%%%%%%%%%%%%%%%%
% Problems start here
%%%%%%%%%%%%%%%%%%%%%%%%%%%%%%%%%%%%%%%%%%%%%%%%%%%%%%%%%%%%%%%%%%%%%
The \emph{\idx{power set}}, $\power(A)$, of a set $A$ is the
set consisting of all the subsets of $A$.

\begin{problem}
\bparts
\ppart
Write out the members of $\power(\set{1,2,3})$.

\ppart What is the size, $\card{\power(\set{1,2,3,4,5,6})}$, of $\power(\set{1,2,3,4,5,6})$?

\ppart What is $\card{\power(\set{1,2,3,4,5,6}) \union \power(\set{1,2,3,4,5})}$?

\ppart What is $\card{\power(\set{1,2,3,4,5,6}) \intersect \power(\set{1,2,3,4,5})}$?

\end{problemparts}

\begin{problem}
Define when a binary relation, $R$, is a \emph{function} in
terms of a $[\leq, =, \geq]$-1-arrow in/out property.
\end{problem}

\end{problem}
%%%%%%%%%%%%%%%%%%%%%%%%%%%%%%%%%%%%%%%%%%%%%%%%%%%%%%%%%%%%%%%%%%%%%
% Problems end here
%%%%%%%%%%%%%%%%%%%%%%%%%%%%%%%%%%%%%%%%%%%%%%%%%%%%%%%%%%%%%%%%%%%%%
\end{document}
