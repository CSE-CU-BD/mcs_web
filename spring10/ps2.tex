\documentclass[handout]{mcs}

\begin{document}

\renewcommand{\reading}{Chapters~\bref{sets_chap}{, \emph{Mathematical Data Types}};
  \bref{logic_chap}{, \emph{First-Order Logic}}.

These assigned readings do not include the Problem sections.  (Many of the
problems in the text will appear as class or homework problems.)

\emph{Reminder}: Comments on the reading using the \emph{NB online
  annotation system} are due at times indicated in the online tutor
problem set TP.2.  Reading Comments count for 5\% of the final grade.}

\problemset{2}

%%%%%%%%%%%%%%%%%%%%%%%%%%%%%%%%%%%%%%%%%%%%%%%%%%%%%%%%%%%%%%%%%%%%%
% Problems start here
%%%%%%%%%%%%%%%%%%%%%%%%%%%%%%%%%%%%%%%%%%%%%%%%%%%%%%%%%%%%%%%%%%%%%


\pinput{PS_logical_set_theory.tex}

\pinput{PS_disjoint_cartesian_products.tex}

\pinput{PS_relation_matricies.tex}

\pinput{PS_bogus_reflexive_proof}

\pinput{PS_which_are_partial_orders.tex}


%%%%%%%%%%%%%%%%%%%%%%%%%%%%%%%%%%%%%%%%%%%%%%%%%%%%%%%%%%%%%%%%%%%%%
% Problems end here
%%%%%%%%%%%%%%%%%%%%%%%%%%%%%%%%%%%%%%%%%%%%%%%%%%%%%%%%%%%%%%%%%%%%%
\end{document}