\documentclass[handout]{mcs}

\begin{document}

\renewcommand{\reading}{Chapters~\bref{sets_chap}{, \emph{Mathematical Data Types}};
  \bref{logic_chap}{, \emph{First-Order Logic}}.

These assigned readings do not include the Problem sections.  (Many of the
problems in the text will appear as class or homework problems.)

\emph{Reminder}: Comments on the reading using the \emph{NB online
  annotation system} are due at times indicated in the online tutor
problem set TP.2.  Reading Comments count for 5\% of the final grade.}

%TOPICS: Sets & Relations, Mapping Lemma & Finite Cardinality, Predicates & Quantifiers

\problemset{2}

%%%%%%%%%%%%%%%%%%%%%%%%%%%%%%%%%%%%%%%%%%%%%%%%%%%%%%%%%%%%%%%%%%%%%
% Problems start here
%%%%%%%%%%%%%%%%%%%%%%%%%%%%%%%%%%%%%%%%%%%%%%%%%%%%%%%%%%%%%%%%%%%%%

%NEEDS DEF OF RELATION MATRIX
\pinput{PS_relation_matrices.tex}

%more class prob level than pset; maybe escalate to express some ZF axiom?
\pinput{CP_logical_set_theory.tex}

%not every interesting, but usable, I guess. 
% Can we connect it to anything conceptual?
\pinput{PS_disjoint_cartesian_products.tex}

%Better as class problem: virtually a short answer prob too simple for PSet
\pinput{CP_bogus_reflexive_proof}

%PS3 coverage, but belongs in Tutor
%\pinput{TP_which_are_partial_orders.tex}

%OK, but isolated and abstract; can we connect it to something conceptual?
\pinput{PS_composition_to_bijection.tex}

%%%%%%%%%%%%%%%%%%%%%%%%%%%%%%%%%%%%%%%%%%%%%%%%%%%%%%%%%%%%%%%%%%%%%
% Problems end here
%%%%%%%%%%%%%%%%%%%%%%%%%%%%%%%%%%%%%%%%%%%%%%%%%%%%%%%%%%%%%%%%%%%%%
\end{document}
