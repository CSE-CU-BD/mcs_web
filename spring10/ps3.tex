\documentclass[handout]{mcs}

\begin{document}

\renewcommand{\reading}{Chapter~\bref{induction_chap},
  \emph{Induction}; Chapter~\bref{partial-order-chapter},
  \emph{Partial Orders}, \S\S 1--3.}

\problemset{3}

%%%%%%%%%%%%%%%%%%%%%%%%%%%%%%%%%%%%%%%%%%%%%%%%%%%%%%%%%%%%%%%%%%%%%
% Problems start here
%%%%%%%%%%%%%%%%%%%%%%%%%%%%%%%%%%%%%%%%%%%%%%%%%%%%%%%%%%%%%%%%%%%%%

\pinput{PS_team_division}

%\pinput{CP_flawed_induction_proof} already used in cp4m

%\pinput{PS_prime_divides_integer_product} %used in public S09

%\pinput{PS_periphery_length_game}  %easy and a lot to read.

%\pinput{PS_sums_and_products_of_integers} %nice since induction is
%not one n, but we don'r need it.

\pinput{PS_fib_induction}

% partial orders
%\pinput{PS_representing_PO_with_subset} used in public S09

\begin{problem}

\textbf{TBA: correctness of $n$-bit half or full ripple-carry adder
  from Problem~\bref{CP_binary_adder_logic}}
\end{problem}

\pinput{PS_relation_transitive_properties}

\pinput{PS_weak_partial_order_isomorphic_to_subset}

%\pinput{PS_strict_PO_irreflexive}  S09 cp3r

%%%%%%%%%%%%%%%%%%%%%%%%%%%%%%%%%%%%%%%%%%%%%%%%%%%%%%%%%%%%%%%%%%%%%
% Problems end here
%%%%%%%%%%%%%%%%%%%%%%%%%%%%%%%%%%%%%%%%%%%%%%%%%%%%%%%%%%%%%%%%%%%%%
\end{document}
