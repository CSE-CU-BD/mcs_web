\documentclass[handout]{mcs}

\begin{document}

\inclassproblems{2, Fri.}

%%%%%%%%%%%%%%%%%%%%%%%%%%%%%%%%%%%%%%%%%%%%%%%%%%%%%%%%%%%%%%%%%%%%%
% Problems start here
%%%%%%%%%%%%%%%%%%%%%%%%%%%%%%%%%%%%%%%%%%%%%%%%%%%%%%%%%%%%%%%%%%%%%

\pinput{TP_truth_table_for_distributive_law}
\insolutions{\newpage}

\pinput{CP_file_system_functioning_normally}
%\pinput{CP_valid_vs_satisfiable}  %on ps1

\begin{staffnotes}
After students have read the following problem, it's probably worth
having them review the definitions of valid and satisfiable, and seeing
if they remember some examples of formulas that are
\begin{itemize}
\item valid
\item satisfiable
\item unsatisfiable
\item neither valid nor unsatisfiable
\end{itemize}
Also have them review the key connections between these properties
\begin{itemize}
 \item satisfiable iff its negation is \emph{not} valid,
 \item valid iff its negation is unsatisfiable.
\end{itemize}
\end{staffnotes}

\pinput{CP_binary_adder_logic}

\pinput{CP_differentiable_implies_continuous}

%\instatements{\newpage}

%And if you have time\dots

\begin{staffnotes}
If students finish early, you might offer this programming problem,
motivated by truth-table construction.  The problem is easy and not
specially important, so it's fine to skip and use any extra time for
other review.
\pinput{PS_printout_binary_strings}
\end{staffnotes}

\iffalse

\begin{center}
\textbf{Supplmental Problem}\footnote{There is no need to study supplmental
  problems when preparing for quizzes or exams.}
\end{center}
\fi


%%%%%%%%%%%%%%%%%%%%%%%%%%%%%%%%%%%%%%%%%%%%%%%%%%%%%%%%%%%%%%%%%%%%%
% Problems end here
%%%%%%%%%%%%%%%%%%%%%%%%%%%%%%%%%%%%%%%%%%%%%%%%%%%%%%%%%%%%%%%%%%%%%

\iffalse

\instatements{\newpage}
%\section*{Appendix}
\section*{The Propositional Operations}

\[
\begin{array}{c|c}
P & \QNOT P \\ \hline
\true & \false \\
\false & \true \\
\end{array}
\]

\[
\begin{array}{cc|c}
P & Q & P \QAND Q \\ \hline
\true & \true & \true \\
\true & \false & \false \\
\false & \true & \false \\
\false & \false & \false
\end{array}
\]


\[
\begin{array}{cc|c}
P & Q & P \QOR Q \\ \hline
\true & \true & \true \\
\true & \false & \true \\
\false & \true & \true \\
\false & \false & \false
\end{array}
\]

\[
\begin{array}{cc|c}
P & Q & P \QXOR Q \\ \hline
\true & \true & \false \\
\true & \false & \true \\
\false & \true & \true \\
\false & \false & \false
\end{array}
\]

\[
\begin{array}{cc|c}
    P  &   Q    & P \QIMP Q \\ \hline
\true  & \true  & \true \\
\true  & \false & \false \\
\false & \true  & \true \\
\false & \false & \true  
\end{array}
\]

\[
\begin{array}{cc|c}
P & Q & P \QIFF Q \\ \hline
\true & \true & \true \\
\true & \false & \false \\
\false & \true & \false \\
\false & \false & \true
\end{array}
\]
\fi

\end{document}

