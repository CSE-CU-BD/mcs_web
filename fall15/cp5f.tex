\documentclass[handout]{mcs}

\begin{document}

\inclassproblems{5, Fri.}

%%%%%%%%%%%%%%%%%%%%%%%%%%%%%%%%%%%%%%%%%%%%%%%%%%%%%%%%%%%%%%%%%%%%%
% Problems start here
%%%%%%%%%%%%%%%%%%%%%%%%%%%%%%%%%%%%%%%%%%%%%%%%%%%%%%%%%%%%%%%%%%%%%

\begin{staffnotes}
Chapter~\bref{Turing_sec}.\ \emph{Turing}
through~\bref{sec:inverse}.\ \emph{Cancelling $\pmod{n}$}
\end{staffnotes}

%\pinput{CP_fast_exponentiation}
%\pinput{CP_calculating_inverses_fermat} %PS_calculating_inverses (b)

\pinput{CP_remainder_computation_practice}

%\Pinput{FP_inverse17mod29}

\pinput{CP_divisible_by_24}

\pinput{CP_chinese_remainder} %would like to use for ps5 -emshen

\pinput{CP_polynomials_produce_multiples} %long problem -emshen

%\pinput{CP_pirate_treasure}

%%%%%%%%%%%%%%%%%%%%%%%%%%%%%%%%%55
% Problems end here
%%%%%%%%%%%%%%%%%%%%%%%%%%%%%%%%%%%%%%%%%%%%%%%%%%%%%%%%%%%%%%%%%%%%%

\end{document}

\endinput

