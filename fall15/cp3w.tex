\documentclass[handout]{mcs}

\begin{document}

\inclassproblems{3, Wed.}

%%%%%%%%%%%%%%%%%%%%%%%%%%%%%%%%%%%%%%%%%%%%%%%%%%%%%%%%%%%%%%%%%%%%%
% Problems start here
%%%%%%%%%%%%%%%%%%%%%%%%%%%%%%%%%%%%%%%%%%%%%%%%%%%%%%%%%%%%%%%%%%%%%

%\pinput{CP_smallest_infinite_set}
\pinput{TP_inverse_relation_table}


\begin{staffnotes}
From Spring14: The next problem is the same as TP.3.6.  Offer it for
review, but skip if students report knowing it.
\end{staffnotes}

\pinput{TP_Images}

\pinput{CP_mapping_rule}
\pinput{CP_set_product_bijection}
\pinput{CP_surj_relation}   %rewrite soln to use arrows

%\pinput{PS_composition_to_bijection}
%\pinput{CP_rationals_are_countable}
%\pinput{CP_Schroeder_Bernstein_theorem}

%%%%%%%%%%%%%%%%%%%%%%%%%%%%%%%%%%%%%%%%%%%%%%%%%%%%%%%%%%%%%%%%%%%%%
% Problems end here
%%%%%%%%%%%%%%%%%%%%%%%%%%%%%%%%%%%%%%%%%%%%%%%%%%%%%%%%%%%%%%%%%%%%%
\end{document}

