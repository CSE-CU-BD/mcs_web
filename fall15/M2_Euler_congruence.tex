\documentclass[problem]{mcs}

\begin{pcomments}
    \pcomment{TP_Eulers_Theorem}
    \pcomment{Converted from euler-theorem.scm by scmtotex and dmj
              on Sat 12 Jun 2010 09:09:34 PM EDT}
    \pcomment{subsumed by MQ_modular_arithmetic}
\end{pcomments}

\begin{problem}

%% type: short-answer
%% title: Euler's Theorem

\bparts

\ppart
What is the value of $\phi(175)$, where $\phi$ is Euler's function?

\begin{solution}
\textbf{120}.

$175 = 5^{2} \cdot 7$, so $\phi(175)=(5^{2} -
5^{1})(7-1) = 20 \cdot 6 = 120$.
\end{solution}

\examspace[1.0in]

\ppart
What is the remainder of $22^{12001}$ divided by 175?

\begin{solution}
\textbf{22}.

\begin{align*}
22^{12001} &= 22^{(120 \cdot 100) +1} \\
           & = (22^{120})^{100}  \cdot  22 \\
           & \equiv  1^{100}  \cdot  22 \bmod 175\\
           & \equiv  22 \bmod 175.
\end{align*}
\end{solution}

\examspace[2.0in]

\ppart
Find the inverse of 22 modulo 175 in the interval $[1,174]$.

\begin{solution}
Note that $8 \cdot 22 - 175 = 1$. Therefore, $8$ is an inverse of $22$ modulo $175$.
\end{solution}

\examspace[2.0in]

\ppart
What is the remainder of $22^{11999}$ divided by 175?

\begin{solution}
\textbf{22}.

\begin{align*}
22^{11999} & \equiv 22^{11999} 22 8 \bmod 175 \\
           & \equiv 22^{12000} 8 \bmod 175 \\
           &= 22^{(120 \cdot 100)} 8 \bmod 175 \\
           & = (22^{120})^{100}  8 \bmod 175 \\
           & \equiv  1^{100}  8 \bmod 175\\
           & \equiv  8 \bmod 175.
\end{align*}
\end{solution}

\examspace[2.0in]

\eparts

\end{problem}

\endinput
