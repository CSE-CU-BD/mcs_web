%final-conflict

\documentclass[quiz]{mcs}

%\renewcommand{\examspace}[1][]{}

\renewcommand{\exampreamble}{   % !! renew \exampreamble

\begin{center}
{\large   \textbf{Circle your}\qquad   \teaminfo}
\end{center}

  \begin{itemize}

  \item
   This exam is \textbf{closed book} except for a 4-sided cribsheet.
   Total time is 180 minutes.

  \item

   Write your solutions in the space provided.  If you need more
   space, \textbf{write on the back} of the sheet containing the
   problem.

   \item We suggest that you move rapidly through the initial,
     short-answer part of the exam, returning to it after you complete
     the rest of the exam.

\iffalse
  \item
   GOOD LUCK!
\fi

  \end{itemize}}

\begin{document}

\conflictfinaltwo

%%%%%%%%%%%%%%%%%%%%%%%%%%%%%%%%%%%%%%%%%%%%%%%%%%%%%%%%%%%%%%%%%%%%%
% Problems start here
%%%%%%%%%%%%%%%%%%%%%%%%%%%%%%%%%%%%%%%%%%%%%%%%%%%%%%%%%%%%%%%%%%%%

\begin{center}
{\LARGE \textbf{Short-Answer Questions}}
\end{center}

In answering short-answer questions, do \emph{not} include
explanations---graders will not look at any.

\pinput[points = 16, title = \textbf{Number Theory}]
{FP_numbers_short_answer2_F15}

\pinput[points = 15, title= \textbf{Relations, Probability}]
{FP_probability_relations_short_answer2}

\pinput[points = 4, title= \textbf{Scheduling}]
{MQ_task_parallel_scheduling_v4}

\pinput[points = 18, title = \textbf{Simple Graphs \& Trees}]
{FP_simple_graphs_trees_short_answer2}

\pinput[points = 18, title= \textbf{Counting}]
{FP_counting_given_answers2}

\examspace

\begin{center}
{\LARGE \textbf{Proof Problems}}
\end{center}

\begin{staffnotes}
Typical error was using the red-label inequality as the induction
hypothesis and then trying to use it on black labels; they generally
got 3/14 points for this.  Only one person correctly carried through
(more or less) using only the red-label induction hypothesis sketched
at the end of the solution.  None of the nine students taking
final-conflict2 thought to use an induction hypothesis symmetric in
red and black labels as in the main solution.
\end{staffnotes}

\pinput[points = 14, title = \textbf{Structural Induction}]
{FP_red_black_tree_induction}
\examspace

\iffalse
\pinput [points = 20, title= \textbf{State Machine, Induction}]
{FP_token_state_machine2}
\examspace
\fi

\pinput[points = 14, title= \textbf{Stable Matching}]
{FP_stable_matching_unlucky}
\examspace

\pinput[points = 14, title= \textbf{Generating functions}]
{FP_boat_trip_fall11}
\examspace

\pinput[points = 18, title = \textbf{Counting, Conditional Probability}]
{FP_random_grid_walk}
\examspace

\pinput[points = 15, title= \textbf{Expectation}]
{FP_expected_adjacent2}
\examspace

\pinput[points = 14, title = \textbf{Expected Time}]
{MQ_expectHHH}
\examspace

\pinput[points = 20, title= \textbf{Deviation}]
{FP_expected_number_of_keys_deviation}

%%%%%%%%%%%%%%%%%%%%%%%%%%%%%%%%%%%%%%%%%%%%%%%%%%%%%%%%%%%%%%%%%%%%%
% Problems end here
%%%%%%%%%%%%%%%%%%%%%%%%%%%%%%%%%%%%%%%%%%%%%%%%%%%%%%%%%%%%%%%%%%%%%

\end{document}
