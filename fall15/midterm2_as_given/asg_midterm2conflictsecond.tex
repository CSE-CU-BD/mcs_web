\documentclass[quiz]{mcs}

\renewcommand{\exampreamble}{   % !! renew \exampreamble
     \textbf{Indicate your}\ \teaminfo

  \begin{itemize}

  \item
   This exam is \textbf{closed book} except for a 2-sided cribsheet.
   Total time is 90 minutes. 

  \item
   Write your solutions in the space provided.  If you need more
   space, write on the back of the sheet containing the problem.

%   Please keep your entire answer to a problem on that problem's page.
   
   \item In answering the following questions, you may use without
     proof any of the results from class or text.
     \iffalse (unless explicitly instructed otherwise).\fi

     \iffalse
   \item
     GOOD LUCK!
     \fi

\end{itemize}}

\begin{document}

\conflictmidtermsecond{October 15}

%%%%%%%%%%%%%%%%%%%%%%%%%%%%%%%%%%%%%%%%%%%%%%%%%%%%%%%%%%%%%%%%%%%%%
% Problems start here
%%%%%%%%%%%%%%%%%%%%%%%%%%%%%%%%%%%%%%%%%%%%%%%%%%%%%%%%%%%%%%%%%%%%

%\examspace

\iffalse
\begin{center}
{\Large DRAFT}
\end{center}
\fi

\begin{staffnotes}
\begin{center}
{\large Binary Relations, Mapping Lemma}
\end{center}
\end{staffnotes}

%\pinput[points = 7, title= \textbf{Binary Relations}]{TP_composition_of_jections}
%\begin{staffnotes}
%TP\_composition\_of\_jections. Proof takes too much explanation for a
%midterm.  Maybe convert to true/false of short/answer ``if $f$ is
%[in-out] and $g$ is [in-out], then $f \compose g$ is [in-out]
%\end{staffnotes}

%\pinput[points = 12, title= \textbf{Binary Relations}]{TP_total_inj_not_bij}
%\begin{staffnotes}
%TP\_total\_inj\_not\_bij.  Rewrite to use [arrows in-out].
%\end{staffnotes}
%
%\pinput[points = 12, title= \textbf{Mapping Rule}]{TP_doughnuts_to_binary_mapping_rule}

%{\color{red}
%Rewrite to use [arrows in-out].\\
%Add part (b) on Mapping Lemma.
%}

%\pinput[points = 15, title= \textbf{Binary Relations}]{M2_binary_relations}

\iffalse
\begin{staffnotes}

M2\_binary\_relations \textcolor{red}{NOT USABLE.}

Part(a) is about infinite sets, covered in Sets problem below.

Part(b) is unlike anything they have seen so far, and without such
exposure requires too much ingenuity.  Plus subscripts make it hard to
understand quickly

\end{staffnotes}
\fi

\iffalse
\inhandout{
%\examspace[0.6in]
\textbox{
\textboxheader{Arrow Properties}

\begin{definition*}
A binary relation, $R$ is
\begin{itemize}

\item is a \emph{function} when it has the $[\le 1\ \text{arrow
    \textbf{out}}]$ property.

\item is \emph{surjective} when it has the $[\ge 1\ \text{arrows
    \textbf{in}}]$ property.  That is, every point in the righthand,
     codomain column has at least one arrow pointing to it.

\item is \emph{total} when it has the $[\ge 1\ \text{arrows
       \textbf{out}}]$ property.

\item is \emph{injective} when it has the $[\le 1\ \text{arrow
    \textbf{in}}]$ property.

\item is \emph{bijective} when it has both the $[=1\ \text{arrow
    \textbf{out}}]$ and the $[=1\ \text{arrow \textbf{in}}]$ property.
\end{itemize}
\end{definition*}
}}
\fi

%% CUT OUT
%\pinput[points = 15, title= \textbf{Binary Relations}]{M2_function_composition_conflictsecond}
%
%\examspace

\begin{staffnotes}
\begin{center}
{\large Induction and State Machines}
\end{center}
\end{staffnotes}

\pinput[points = 25, title= \textbf{State Machines, Induction, Congruence}]{asg_TP_sub_one_with_two_of_opposite_color-conflictsecond}

\begin{staffnotes}
Substantially revised \& simplified.  May still be too hard w/o hint
(say that $P(n)$ is true for $n \leq 2$) since it tool a long time for
both ZD and ARM to write up a satifactory induction proof.
\end{staffnotes}

%%\examspace[4.0in]

%\begin{center}
%{\large State machines}
%\end{center}
%
%\pinput[points = 7, title= \textbf{State Machines}]{CP_10_heads_and_100_tails}
%
%\pinput[points = 7, title= \textbf{State Machines}]{MQ_state_machine_invariant_afternoon}
%
%\pinput[points = 7, title= \textbf{State Machines}]{MQ_state_machine_invariant_morning}
%
%\pinput[points = 7, title= \textbf{State Machines}]{CP_98_heads_and_4_tails}

\begin{staffnotes}
\begin{center}
{\large Stable Marriage}
\end{center}
\end{staffnotes}

%\pinput[points = 7, title= \textbf{Stable Marriage}]{MQ_stable_matching_unique_morning}

% CONFLICT
\pinput[points = 10, title= \textbf{Stable Marriage}]{asg_TP_Stable_Marriage_Invariants-conflict}

%\pinput[points = 10, title= \textbf{Stable Marriage}]{FP_stable_matching_unequal_invariants}

\examspace

%\examspace
\begin{staffnotes}
FP\_stable\_matching\_unequal\_invariants good variation of probs on cp4w.
\end{staffnotes}

\begin{staffnotes}
\begin{center}
{\large Recursive Data}
\end{center}
\end{staffnotes}

%\pinput[points = 7, title= \textbf{Recursive Data}]{MQ_ambiguous_recursive_def}

%\pinput[points = 7, title= \textbf{Recursive Data}]{MQ_ambiguous_recursive-def_morning}

%\pinput[points = 12, title= \textbf{Recursive Data}]{FP_structural_induction_rational_composition_S13}

% CONFLICT
\pinput[points = 15, title= \textbf{Recursive Data}]{asg_M2_rational_structural_induction}

\examspace

%\pinput[points = 7, title= \textbf{Recursive Data}]{FP_structural_ind_polynomials}

\begin{staffnotes}
\begin{center}
{\large Sets}
\end{center}
\end{staffnotes}

\pinput[points = 12, title= \textbf{Sets}]{asg_M2_sets_conflictsecond}

\examspace

%\pinput[points = 5, title= \textbf{Sets}]{TP_cardinality_class}
%
%{\color{red}
%Take one half for main exam and the other half conflict.
%}
%
%% maybe conflict
%%\pinput[points = 7, title= \textbf{Sets}]{FP_countable_quadratics}
%
%\pinput[points = 5, title= \textbf{Sets}]{TP_uncountable_example}
%
%{\color{red}
%Add as part (b) of the previous problems.
%}

%\begin{staffnotes}
%TP\_uncountable\_example: good problem for midterm 2
%\end{staffnotes}

\begin{staffnotes}
\begin{center}
{\large Number Theory: GCDs and Congruence}
\end{center}
\end{staffnotes}

%pinput[points = 7, title= \textbf{NT}]{CP_GCD_algebra}

%\pinput[points = 7, title= \textbf{NT}]{CP_relative_primality_under_remainder}

\pinput[points = 15, title= \textbf{GCD, Congruence}]{asg_M2_GCD_congruence_conflictsecond}

\examspace

%\pinput[points = 6, title= \textbf{NT}]{FP_GCD_algebra}
%
%\pinput[points = 4, title= \textbf{NT}]{TP_GCDs_II}
%\begin{staffnotes}
%TP\_GCDs\_II.  Lightweight variant of CP\_gcd\_lcm on cp5w.
%\end{staffnotes}
%
%{\color{red}
%Add a problem on finding an inverse using pulverizer.
%}

\begin{staffnotes}
\begin{center}
{\large Euler's theorem, NOT RSA}
\end{center}
\end{staffnotes}

\pinput[points = 13, title= \textbf{Euler, Congruence}]{asg_M2_Euler_congruence_conflictsecond}
%\pinput[points = 7, title= \textbf{Euler}]{TP_Eulers_Theorem}

%\pinput[points = 7, title= \textbf{Euler}]{TP_Fermats_Little_Theorem_F13}

%\pinput[points = 7, title= \textbf{Euler}]{TP_relative_primality_3780}

%\pinput[points = 7, title= \textbf{Euler}]{FP_modular_powerful}
%\begin{staffnotes}
%FP\_modular\_powerful Needs revision: change probability to ``fraction''
%\end{staffnotes}

%\pinput[points = 7, title= \textbf{Euler}]{FP_bogus_Fermat_theorem.spring12}


%%%%%%%%%%%%%%%%%%%%%%%%%%%%%%%%%%%%%%%%%%%%%%%%%%%%%%%%%%%%%%%%%%%%%
% Problems end here
%%%%%%%%%%%%%%%%%%%%%%%%%%%%%%%%%%%%%%%%%%%%%%%%%%%%%%%%%%%%%%%%%%%%%
\end{document}
