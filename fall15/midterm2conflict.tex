\documentclass[quiz]{mcs}

\renewcommand{\exampreamble}{   % !! renew \exampreamble
     \textbf{Indicate your}\ \teaminfo

  \begin{itemize}

  \item
   This exam is \textbf{closed book} except for a 2-sided cribsheet.
   Total time is 90 minutes. 

  \item
   Write your solutions in the space provided.  If you need more
   space, write on the back of the sheet containing the problem.

%   Please keep your entire answer to a problem on that problem's page.
   
   \item In answering the following questions, you may use without
     proof any of the results from class or text.
     \iffalse (unless explicitly instructed otherwise).\fi

     \iffalse
   \item
     GOOD LUCK!
     \fi

\end{itemize}}

\begin{document}

\conflictmidterm{October 15}

%%%%%%%%%%%%%%%%%%%%%%%%%%%%%%%%%%%%%%%%%%%%%%%%%%%%%%%%%%%%%%%%%%%%%
% Problems start here
%%%%%%%%%%%%%%%%%%%%%%%%%%%%%%%%%%%%%%%%%%%%%%%%%%%%%%%%%%%%%%%%%%%%

%\examspace

\iffalse
\begin{center}
{\Large DRAFT}
\end{center}
\fi

\begin{staffnotes}
\begin{center}
{\large Binary Relations, Mapping Lemma}
\end{center}
\end{staffnotes}

%% CUT OUT
%\pinput[points = 15, title= \textbf{Binary Relations}]{M2_function_composition_conflict}
%\examspace

\begin{staffnotes}
\begin{center}
{\large Induction and State Machines}
\end{center}
\end{staffnotes}

\pinput[points = 25, title= \textbf{State Machines, Induction, Congruence}]{FP_token_state_machine_conflict}

\examspace

\begin{staffnotes}
\begin{center}
{\large Stable Marriage}
\end{center}
\end{staffnotes}

%\pinput[points = 7, title= \textbf{Stable Marriage}]{MQ_stable_matching_unique_morning}

% CONFLICT
\pinput[points = 10, title= \textbf{Stable Marriage}]{midterm2_as_given/asg_TP_Stable_Marriage_Invariants}

%\pinput[points = 10, title= \textbf{Stable Marriage}]{FP_stable_matching_unequal_invariants}

\examspace

%\examspace
\begin{staffnotes}
FP\_stable\_matching\_unequal\_invariants good variation of probs on cp4w.
\end{staffnotes}

\begin{staffnotes}
\begin{center}
{\large Recursive Data}
\end{center}
\end{staffnotes}

%\pinput[points = 7, title= \textbf{Recursive Data}]{MQ_ambiguous_recursive_def}

%\pinput[points = 7, title= \textbf{Recursive Data}]{MQ_ambiguous_recursive-def_morning}

%\pinput[points = 12, title= \textbf{Recursive Data}]{FP_structural_induction_rational_composition_S13}

% CONFLICT
\pinput[points = 15, title= \textbf{Recursive Data}]{midterm2_as_given/asg_M2_rational_structural_induction}

\examspace

%\pinput[points = 7, title= \textbf{Recursive Data}]{FP_structural_ind_polynomials}

\begin{staffnotes}
\begin{center}
{\large Sets}
\end{center}
\end{staffnotes}

\pinput[points = 12, title= \textbf{Sets}]{midterm2_as_given/asg_M2_sets_conflict}

\examspace

%\pinput[points = 5, title= \textbf{Sets}]{TP_cardinality_class}
%
%{\color{red}
%Take one half for main exam and the other half conflict.
%}
%
%% maybe conflict
%%\pinput[points = 7, title= \textbf{Sets}]{FP_countable_quadratics}
%
%\pinput[points = 5, title= \textbf{Sets}]{TP_uncountable_example}
%
%{\color{red}
%Add as part (b) of the previous problems.
%}

%\begin{staffnotes}
%TP\_uncountable\_example: good problem for midterm 2
%\end{staffnotes}

\begin{staffnotes}
\begin{center}
{\large Number Theory: GCDs and Congruence}
\end{center}
\end{staffnotes}

%pinput[points = 7, title= \textbf{NT}]{CP_GCD_algebra}

%\pinput[points = 7, title= \textbf{NT}]{CP_relative_primality_under_remainder}

\pinput[points = 15, title= \textbf{GCD}]{midterm2_as_given/asg_M2_GCD_congruence_conflict}

\examspace

%\pinput[points = 6, title= \textbf{NT}]{FP_GCD_algebra}
%
%\pinput[points = 4, title= \textbf{NT}]{TP_GCDs_II}
%\begin{staffnotes}
%TP\_GCDs\_II.  Lightweight variant of CP\_gcd\_lcm on cp5w.
%\end{staffnotes}
%
%{\color{red}
%Add a problem on finding an inverse using pulverizer.
%}

\begin{staffnotes}
\begin{center}
{\large Euler's theorem, NOT RSA}
\end{center}
\end{staffnotes}

\pinput[points = 13, title= \textbf{Euler, Congruence}]{midterm2_as_given/asg_M2_Euler_congruence_conflict}
%\pinput[points = 7, title= \textbf{Euler}]{TP_Eulers_Theorem}

%\pinput[points = 7, title= \textbf{Euler}]{TP_Fermats_Little_Theorem_F13}

%\pinput[points = 7, title= \textbf{Euler}]{TP_relative_primality_3780}

%\pinput[points = 7, title= \textbf{Euler}]{FP_modular_powerful}
%\begin{staffnotes}
%FP\_modular\_powerful Needs revision: change probability to ``fraction''
%\end{staffnotes}

%\pinput[points = 7, title= \textbf{Euler}]{FP_bogus_Fermat_theorem.spring12}


%%%%%%%%%%%%%%%%%%%%%%%%%%%%%%%%%%%%%%%%%%%%%%%%%%%%%%%%%%%%%%%%%%%%%
% Problems end here
%%%%%%%%%%%%%%%%%%%%%%%%%%%%%%%%%%%%%%%%%%%%%%%%%%%%%%%%%%%%%%%%%%%%%
\end{document}
