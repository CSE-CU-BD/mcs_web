%final-ARM-dec15

\documentclass[quiz]{mcs}

%\renewcommand{\examspace}[1][]{}

\renewcommand{\exampreamble}{   % !! renew \exampreamble

\begin{center}
{\large   \textbf{Circle your}\qquad   \teaminfo}
\end{center}

  \begin{itemize}

  \item
   This exam is \textbf{closed book} except for a four-sided cribsheet.
   Total time is 180 minutes.

  \item

   Write your solutions in the space provided.  If you need more
   space, \textbf{write on the back} of the sheet containing the
   problem.

%   Please keep your entire answer to a problem on that problem's page.   
   \item In answering the following questions, you may use without
     proof any of the results from class or text.

   \item We suggest that you move rapidly through the initial,
     short-answer part of the exam, returning to it after you complete
     the rest of the exam.

\iffalse
  \item
   GOOD LUCK!
\fi

  \end{itemize}}

\begin{document}

\final

%%%%%%%%%%%%%%%%%%%%%%%%%%%%%%%%%%%%%%%%%%%%%%%%%%%%%%%%%%%%%%%%%%%%%
% Problems start here
%%%%%%%%%%%%%%%%%%%%%%%%%%%%%%%%%%%%%%%%%%%%%%%%%%%%%%%%%%%%%%%%%%%%
\examspace

\begin{center}
{\LARGE \textbf{Short-Answer Questions}}
\end{center}

In answering short-answer questions, do \emph{not} include explanations.

\pinput[points = 16, title = \textbf{Number Theory}]
{FP_numbers_short_answer_F15}

\pinput[points = 12, title= \textbf{Relations, Probability}]
{FP_probability_relations_short_answer}

\pinput[points = 4, title= \textbf{Scheduling}]
{MQ_task_parallel_scheduling_v3}

\pinput[points = 18, title = \textbf{Simple Graphs \& Trees}]
{FP_simple_graphs_trees_short_answer}

\pinput[points = 4, title = \textbf{Bipartite Matching}]
{TP_bottleneck}

\begin{staffnotes}
\begin{verbatim}
{FP_counting_poker_high_cards}
\end{verbatim}
was on S15.final.
\end{staffnotes}

\pinput[points = 16, title= \textbf{Counting}]
{FP_counting_given_answers}

\pinput[points = 6, title= \textbf{Expectation}]
{FP_expected_adjacent}

\examspace

\begin{center}
{\LARGE \textbf{Proof Problems}}
\end{center}

\pinput[points = 8, title= \textbf{Stable Matching}]
{TP_stable_wedding}

\examspace

\pinput[points = 12, title = \textbf{Structural Induction}]
{FP_binary_tree_induction}
\examspace

\pinput [points = 20, title= \textbf{State Machine, Induction}]
{FP_token_state_machine2}
\examspace

\pinput[points = 16, title= \textbf{Logic, Counting}]
{FP_sat_count_genfunc}
\examspace

\pinput[points = 12, title= \textbf{Conditional Probability}]
{FP_conditional_prob_inequality}
\examspace

\pinput[points = 16, title = \textbf{Expected Time}]
{MQ_expectHH_TT_F15}
\examspace

\pinput[points = 20, title= \textbf{Deviation}]
{FP_expected_number_of_keys_deviation}
%\examspace

\begin{staffnotes}
NOT COVERED:
\begin{itemize}
\item Countability
\end{itemize}
\end{staffnotes}

%%%%%%%%%%%%%%%%%%%%%%%%%%%%%%%%%%%%%%%%%%%%%%%%%%%%%%%%%%%%%%%%%%%%%
% Problems end here
%%%%%%%%%%%%%%%%%%%%%%%%%%%%%%%%%%%%%%%%%%%%%%%%%%%%%%%%%%%%%%%%%%%%%

\end{document}

\iffalse
\begin{staffnotes}
Jodie: This (very) shortened problem from a 6.046 pset from last year that I
think would be good for bipartite matching:

ARM: Too much story to wade through.  Also undesirable to use new
concept like Maximum Matching on a final exam.


BEGIN{problem}
Alyssa has recently acquired a large number of sentry turrets 1 from
the Aperture Science surplus store. She plants to use them for a
surprise attack on Bob's castle, which has a very simple shape - it is
just an $N \times M$ grid.  Some of the cells are lava-pits, so Alyssa
can only teleport turrets into some of the $N \times M$ cells.  A
turret located in a cell on row i at column j can (and will) attack
anything in the same row or on the same column.  Specifically, if two
turrets can see each other, they will attack each other.  Alyssa wants
to teleport the maximum number of turrets into the castle, while
avoiding ``friendly fire'' situations (when two turrets attack each
other).  Consider a bipartite graph derived from the setup above: each
row will be a vertex in the left subset and each column will be vertex
in the right subset.  Finish the construction of the graph in such a
way that the maximum matching in the resulting graph equals to the
maximum number of turrets that Alyssa can teleport on the grid, and
explain why it works.

Solution: In the left subset we have vertices for all of the rows and
in the right subset we have vertices for all of the columns.  An
intuitive way to connect the rows and columns is to add an edge where
they intersect. We will add an edge between a row and a column only if
their common cell is not a lava pit.  With maximum matching on this
graph, we will match certain rows with certain columns. If we match a
row and a column, this corresponds to teleporting a turret in the cell
of their common cell.  What does the constraints of the matching
problem tell us?  We can match a vertex from left side to at most one
vertex of the right side and we can match a vertex from the right side
with at most one vertex from the left side.  A match in our problem
corresponds to teleporting a turret, so we can teleport at most one
turret on every row and on every column.  Given this setup, the
problem that we want to solve is given a bipartite graph what is the
maximum number of vertices from the left side that we can match with
vertices from the right side.  The answer to this is just the maximum
matching in this graph.
end{problem}

Subsumed by 
\begin{verbatim}
MQ_expectHH_TT
\end{verbatim}
We know that \begin{verbatim}
CP_coin_flips
\end{verbatim} was used in a class problem.  However, we
think that it can be changed such that the coin flips used for winning
 are different, or longer sequences are needed to win.

\begin{verbatim}
FP_bogus_coloring_proof
\pinput[points = 16, title = \textbf{Graph Coloring \& Induction}]
{FP_bogus_coloring_proof}
\examspace
\end{verbatim}
was last problem on F15.cp8m, but still could re-use


\textbf{EXTRA}

\begin{verbatim}
CP_partial_order_short_answer
FP_partial_order_short_answer_f13
\pinput[points = 15, title= \textbf{psetshort11}]{CP_partial_order_short_answer}
\pinput[points = 15, title= \textbf{psetshort2}]{FP_partial_order_short_answer_f13}
\pinput[points = 15, title= \textbf{Multiple2}]{FP_multiple_choice_unhidden_fall13}


\pinput[points = 6, title= \textbf{Stable Matching}]
{FP_stable_matching_unlucky}

\pinput[points = 6, title= \textbf{Tree coloring}]
{FP_tree_color_induction}

\pinput[points = 8, title = \textbf{Logical Injections}]
{FP_logical_jections}

\pinput[points = 8, title = \textbf{Exponential mod $n$}]
{FP_Euler_theorem_calculation}

\pinput[points = 1, title= \textbf{Counting}]
{FP_more_counting_F15}

\pinput[points = 1, title= \textbf{Simple Graphs, Induction}]
{FP_cycles_components_induction}

\pinput[points = 10, title = \textbf{Magic Trick Redux}]
{FP_magic_trick_27_cards}

\pinput[points = 14, title = \textbf{Combinatorial proof}]
{FP_com_proof_parts}

\pinput[points = 12, title = \textbf{Expectation}]
{MQ_infinite_repeat}

\pinput[points = 1, title= \textbf{Probability of collision}]
{PS_ethernet}
\end{verbatim}

Would need revision since was already used as CP;
later parts depend heavily on earlier:

\begin{verbatim}
\pinput[points = 10, title = \textbf{Variance \& Deviation}]
{CP_chebyshev_hat_check}
\end{verbatim}

\begin{center}
{\large Week 11 -- Tomas}
\end{center}

\begin{verbatim}
\pinput[points = 1, title= \textbf{Generating functions}]
{FP_boat_trip_fall11}

\end{verbatim}

S11 final

\begin{verbatim}
\pinput[points = 15, title = \textbf{Graphs}]
{FP_graphs_short_answer}

\pinput[points = 10, title = \textbf{Partial orders}]
{FP_partial_order_short_answer}

\pinput[points = 10, title = \textbf{Big Oh}]
{MQ_big_oh_def}

\pinput[points = 10, title = \textbf{Counting passwords}]
{CP_inclusion-exclusion_passwords}

\pinput[points = 6, title = \textbf{Matching}]
{FP_bipartite_matching_sex}
\end{verbatim}

Fall 11 final

\begin{verbatim}
\pinput[points = 9, title = \textbf{numbers short answer}]
{FP_numbers_short_answer_fall11}

\pinput[points = 7, title = \textbf{asymptotics}]
{FP_asymptotics_define_functions}

\pinput[points = 6, title = \textbf{lining up fall11}]
{FP_lining_up_F15}

\pinput[points = 6, title = \textbf{probability}]
{FP_red_and_blue_goats_fall11}
\end{verbatim}

Spring 12

\begin{verbatim}
\pinput[points = 7, title = \textbf{Relations Short Answer}]
{FP_partial_order_short_answer_S12}

\pinput[points = 6, title = \textbf{Super 7}]
{CP_7777}
\end{verbatim}

Spring 13

\begin{verbatim}
\pinput[points = 8, title = \textbf{Graphs, Congruences}]
{FP_multiple_choice}
\end{verbatim}

Fall 13

\begin{verbatim}
\pinput[points = 9, title = \textbf{Counting Graphs \& Relations}]
{FP_counting_graphs_f13}
\end{verbatim}

\begin{center}
{\large Week 3 -- Tomas \& Mike}
\end{center}
\begin{verbatim}
FP_relation_properties_expressions_final_f13
\end{verbatim}
Nice problem, but I don't think we've had a prob like this before.

\begin{center}
{\large Mid 1 topics -- ARM }
\end{center}

\begin{verbatim}
\pinput[points = 6, title= \textbf{Well Ordering Principle}]
{FP_wop_nn1}
\end{verbatim}

\begin{center}
{\large Week 4 -- Tanya and Elizabeth S.}
\end{center}

\begin{verbatim}
\pinput[points = 6, title= \textbf{Structural Induction}]
{FP_structural_induction_arithmetic_expressions}
\end{verbatim}

Use one of these two stable marriage problems for conflict.  The first
is somehwhat harder.

\begin{verbatim}
FP_santa_state_machine
\end{verbatim}

\begin{center}
{\large Week 6 -- Tasha and Lisa}
\end{center}

\begin{verbatim}
\pinput[points = 6, title= \textbf{Euler's and RSA}]
{FP_RSA_TF_f15}

\pinput[points = 6, title= \textbf{Scheduling}]
{FP_chains_scheduling}

\pinput[points = 6, title= \textbf{DAGs}]
{MQ_minimum_DAG_positive_walk_relation} % (might want to combine with another short problem)

(aka the question we almost kept for midterm 3)
(or one of its variants - if I remember correctly, it has a few very similar versions)

\end{verbatim}

\begin{center}
{\large Midterm 2 -- Isabella and Xavier}
\end{center}

\begin{verbatim}
\pinput[points = 6, title= \textbf{Stable Marriage}]
{FP_stable_matching_unequal_invariants}

\end{verbatim}

\begin{center}
{\large Week 7 -- Julian and Harlin}
\end{center}
Here's a few problems that will likely need shortening/modification.

\begin{verbatim}
\pinput[points = 6, title= \textbf{Partial Order}]
{CP_minimal_maximal_elements}

\pinput[points = 6, title= \textbf{Digraphs}]
{CP_covering_edges}

CP_weak_partial_order_isomorphic_to_subset
\pinput[points = 6, title= \textbf{Partial Order}]
{CP_weak_partial_order_isomorphic_to_subset}
\end{verbatim}
isomorphic\_to\_subset too long, pedantic

\begin{center}
{\large Week 8 -- Annie and Jodie}
\end{center}
\begin{verbatim}
\pinput[points = 6, title= \textbf{Trees \& Paths}]
{TP_average_degree_of_tree_and_simple_path}

\pinput[points = 6, title= \textbf{Tree}]
{MQ_tree_plus_edge}

\pinput[points = 6, title= \textbf{Tree coloring}]
{FP_tree_kcolor}

\pinput[points = 6, title= \textbf{Harmonic Sum}]
{PS_bug_on_rug_harmonic_number}

\pinput[points = 6, title= \textbf{Connectedness}]
{CP_remove_connected}

\pinput[points = 6, title= \textbf{Coloring, induction}]
{PS_coloring_induction}

\pinput[points = 6, title= \textbf{Trees, Induction}]
{PS_tree_degree_sequence}
\end{verbatim}

\begin{verbatim}
TP_Summation
MQ_Summation_with_hint
\pinput[points = 6, title= \textbf{Coloring}]
{PS_graph_colorable}
\end{verbatim}

\begin{center}
{\large Week 9 -- Maria and Tanya}
\end{center}

For following with explanations required, so we would need to give
more than 6 points, maybe 10.
\begin{verbatim}
\pinput[points = 1, title= \textbf{Asymptotics}]
{MQ_asymptotic_true_false_makeup} 

\pinput[points = 1, title= \textbf{Asymptotics}]
{MQ_asymptotic_incomparable}

\pinput[points = 1, title= \textbf{Counting with Bijection}]
{FP_bijection_counting}

\pinput[points = 1, title= \textbf{Counting}]
{FP_toy_button_counting}

\end{verbatim}

\begin{center}
{\large Week 10 -- Parker}
\end{center}

\begin{verbatim}
\pinput[points = 1, title= \textbf{Generating Functions}]
{FP_dangerous_dan_gen_func_S14 }
\end{verbatim}

too tricky:
\begin{verbatim}
\pinput[points = 1, title= \textbf{Combinatorics}]
{PS_3_friends}
\end{verbatim}

\begin{center}
{\large Week 12 -- Jodie and Harlin}
\end{center}

\begin{verbatim}
\pinput[points = 1, title= \textbf{Independence}]
{TP_mutual_independent_pairs.tex}

\pinput[points = 1, title= \textbf{Graphs, Logic, Probability}]
{CP_graph_logic_probability}

\end{verbatim}

\begin{center}
{\large Week 13 -- Julian and Elizabeth S.}
\end{center}

\begin{verbatim}
\pinput[points = 1, title= \textbf{Expectation}]
{FP_expectation_dice}

\pinput[points = 1, title= \textbf{Expectation}]
{FP_expected_adjacent}

\end{verbatim}

too much story for final:
\begin{verbatim}
\pinput[points = 1, title= \textbf{Probability}]
{PS_testing_soldiers}
\end{verbatim}

From Mike:

\begin{verbatim}
\pinput[points = 1, title= \textbf{induction, with counting}]
{FP_hockey_stick_formula}

\pinput[points = 1, title= \textbf{Logic, Relations}]
{FP_logic_relations}

\pinput[points = 1, title= \textbf{relations, counting, graphs}]
{PS_counting_graphs}

\pinput[points = 1, title= \textbf{Probability, Propositional Logic}]
{FP_satisfy_implies_probability}
\end{verbatim}

Good problem that would require modification:
\begin{verbatim}
\pinput[points = 1, title= \textbf{Probability}]
{PS_fair_ruin_probability}
\end{verbatim}

Spring '10 final

\begin{verbatim}
\pinput[points = 10, title = \textbf{Asymptotic Bounds and Partial Orders}]
{FP_asymptotic_partial_order}

\pinput[points = 10, title = \textbf{Euler's Function}]
{FP_modular_powerful}

\pinput[points = 10, title = \textbf{Combinatorial Proof}]
{FP_combinatorial_binomial}
\end{verbatim}

More about counting than probability.
\begin{verbatim}
\pinput[points = 10, title = \textbf{Zoran probability}]
{FP_bus_row_seats_probability}

FP_red_black_tree_induction
FP_arith_trig_functions

\pinput[points = 1, title= \textbf{Variance}]
{FP_variance_dice_sum}

\pinput[points = 15, title = \textbf{Euler Theorem}]
{FP_Euler_theorem_calculation2}

\pinput[points = 10, title = \textbf{Conditional Probability}]
{FP_random_grid_walk}

\end{verbatim}

\end{staffnotes}

\fi
