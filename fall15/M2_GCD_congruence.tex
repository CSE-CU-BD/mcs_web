\documentclass[problem]{mcs}

\begin{pcomments}
  \pcomment{FP_GCD_algebra.tex + TP_GCDs_II.tex}
  \pcomment{prepared for midterm 2, Fall15}
  \pcomment{Zoran Dzunic 10/10/15}
\end{pcomments}

\pkeywords{
  GCD
  Euclidean algorithm
}

\begin{problem}

\bparts

\ppart

Prove that $\gcd(mb+ r, b) = \gcd(b, r)$ for all integers $m,r,b$.

\hint We proved a similar result in class when $b \neq 0$ and $r$ was
a remainder after division by $b$.

\begin{solution}
We know that if a number is a common divisor of numbers $x$ and $y$, then
it is a common divisor of any linear combination of $x$ and $y$.

So if $d$ is a common divisor of $mb+r$ and $b$, then it is a divisor of
$b$ and $1\cdot(mb+r) - m \cdot b = r$.

Conversely, if $d$ is a common divisor of $b$ and $r$, then it is a
common divisor of $m \cdot b+ 1 \cdot r$ and $b$.

Since the two pairs $(mb+ r, b)$ and $(b, r)$ have the \emph{same}
common divisors, they must have the same \emph{greatest} common
divisor.

\end{solution}

\examspace[3.0in]

\ppart
Let
\begin{align*}
    x & \eqdef 17^{88} \cdot 37^{2} \cdot 59^{1000}  \\
    y & \eqdef 19^{(9^{22})} \cdot 37^{12} \cdot 59^{29}.
\end{align*}

%Please give the prime factorization of your answer as a set of
%(\emph{prime exponent}) pairs.  For example, to write  
%the number $60 = 2^{2} \cdot 3 \cdot 5$, you can write
%\begin{equation*}
%(3 1) (2 2) (5 1)
%\end{equation*}

Express $\gcd(x,y)$ and $\lcm(x,y)$ as a product of prime powers.

\begin{solution}
$\gcd(x,y) = 37^{2}\cdot59^{29}$.

To get the GCD of two numbers: iterate over all primes that appear in
both factorizations; raise each of them to the \emph{smallest} of its
two exponents; then multiply the resulting powers.

$\lcm(x,y) = 17^{88} \cdot 37^{12} \cdot 59^{1000}\cdot19^{(9^{22})}$.

To get the LCM of two numbers: iterate over all primes that appear in
either factorization; raise each of them to the \emph{greatest} of its
two exponents; then multiply the resulting powers.

\end{solution}

\examspace[2.0in]

\iffalse
SUBSUMED BY EULER

\ppart
Find the remainders of $x$ and $y$ divided by 3.

\begin{solution}
\begin{align*}
x & = 17^{88} \cdot 31^{5} \cdot 37^{2} \cdot 59^{1000} \equiv (-1)^{88} \cdot 1^{5} \cdot 1^{2} \cdot (-1)^{1000} \equiv 1 \pmod{3} \\
y & = 19^{(9^{22})} \cdot 37^{12} \cdot 53^{3678} \cdot 59^{29} \equiv 1^{(9^{22})} \cdot 1^{12} \cdot 1^{3678} \cdot (-1)^{29} \equiv -1 \equiv 2 \pmod{3}
\end{align*}
\end{solution}

\examspace[2.0in]
\fi

\eparts

\end{problem}

\endinput
