\documentclass[problem]{mcs}

\begin{pcomments}
  \pcomment{TP_total_inj_not_bij + TP_doughnuts_to_binary_mapping_rule}
  \pcomment{prepared for midterm 2, Fall15}
  \pcomment{Zoran Dzunic 10/10/15}
\end{pcomments}

\pkeywords{
  binary relations
  mapping rule
}

\begin{problem}

\bparts

\ppart
Give an example of a relation $R$ that is a total injective
function from a set $A$ to itself but is not a bijection.

\examspace[2.0in]

\begin{solution}
Let $A = \naturals$ and $R$ be the successor function $R(n) = n+1$
(more precisely, $R(n) = \set{n+1}$).  There are many other possible
examples, but $A$ must be infinite.
\end{solution}

\ppart 

\iffalse
\ppart
Let $A$ be the set that represents all ways to select 12 doughnuts
when five varieties are available, i.e.,
\[
A = \set{(n_1, n_2, n_3, n_4, n_5) \in \naturals^5 \suchthat n_1 + n_2 + n_3 + n_4 + n_5 = 12} \, .
\]
Let $B$ be the set of all 16-bit sequences with exactly 4 ones, i.e.,
\[
B = \set{(b_1, b_2, \ldots, b_{16}) \in \set{0, 1}^{16} \suchthat \sum_{i=1}^{16} b_i = 4} \, .
\]
Show that $\card{A} = \card{B}$ using mapping rule (i.e., by finding a bijection from $A$ and $B$).

\examspace[4.0in]

\begin{solution}
%Let $f : A \rightarrow B$ be defined in the following way:
%\[
%f(n_1, n_2, n_3, n_4, n_5) = (b_1 = 0, \ldots, b_{n_1} = 0, b_{n_1 + 1} = 1, b_{n_1 + 2} = 0, \ldots, b_{n_1+ n_2 + 1} = 0, b_{n_1 + n_2 + 2} = 1, b_{n_1 + n_2 + 3} = 0, \ldots, b_{n_1 + n_2 + n_3 + 2} = 0, b_{n_1 + n_2 + n_3 + 3} = 1, b_{n_1 + n_2 + n_3 + 4} = 0, \ldots, b_{n_1 + n_2 + n_3 + n_4 + 3} = 0, \ldots, b_{n_1 + n_2 + n_3 + n_4 + 4} = 1, b_{n_1 + n_2 + n_3 + n_4 + 5} = 0, \ldots, b_{n_1 + n_2 + n_3 + n_4 + n_5 + 4} = 0) \, .
%\]
%In other words, 12 zeros are split into groups of $n_1, \ldots, n_5$ zeros, separated by ones.
Let $f : A \rightarrow B$ be defined in the following way. $f\left((n_1, n_2, n_3, n_4, n_5)\right)$ is obtained
by splitting twelve 0s into five groups of $n_1, \ldots, n_5$ zeros and inserting a single 1
in between groups. E.g., $f\left((3, 5, 0, 2, 2)\right) = (0, 0, 0, 1, 0, 0, 0, 0, 0, 1, 1, 0, 0, 1, 0, 0)$.
Note that $n_i = 0$ will result in 1 inserted before and after $i^{th}$ group to be next to each other.
%in the binary sequence.
An extreme example would be $f\left((0, 0, 0, 0, 12)\right) = (1, 1, 1, 1, 0, 0, 0, 0, 0, 0, 0, 0, 0, 0, 0, 0)$

It is easy to show that $f$ is a bijection from $A$ to $B$. $f$ is total ($\ge 1$ arrows out) by definition.
Now let us show that it is surjective ($\ge 1$ arrows in).
Let $b$ be an arbitrary element of $B$.
Let $(n_1, n_2, n_3, n_4, n_5)$ be such that
$n_1$ is the number of 0s before the first occurrence of 1,
$n_i$, $2 \le i \le 4$, is the number of 0s after $(i-1)^{th}$ and before $i^{th}$ occurrence of 1,
and $n_5$ is the number of 0s after the fourth occurrence of 1.
Then, $(n_1, n_2, n_3, n_4, n_5) \in A$ since $\sum_{i=1}^5 n_i$ is the total number of
zeros, which is 12. Also, $f(n_1, n_2, n_3, n_4, n_5) = b$ by definition of $f$.
Therefore, there is at least one arrow in for each $b \in B$, which concludes that
$f$ is surjective.
Since $f$ is both total and surjective, it is also bijective, and thus $\card{A} = \card{b}$
follows by mapping rule.

\end{solution}
\fi

\eparts

\end{problem}

\endinput

\iffalse
\inhandout{
%\examspace[0.6in]
\textbox{
\textboxheader{Arrow Properties}

\begin{definition*}
A binary relation, $R$ is
\begin{itemize}

\item is a \emph{function} when it has the $[\le 1\ \text{arrow
    \textbf{out}}]$ property.

\item is \emph{surjective} when it has the $[\ge 1\ \text{arrows
    \textbf{in}}]$ property.  That is, every point in the righthand,
     codomain column has at least one arrow pointing to it.

\item is \emph{total} when it has the $[\ge 1\ \text{arrows
       \textbf{out}}]$ property.

\item is \emph{injective} when it has the $[\le 1\ \text{arrow
    \textbf{in}}]$ property.

\item is \emph{bijective} when it has both the $[=1\ \text{arrow
    \textbf{out}}]$ and the $[=1\ \text{arrow \textbf{in}}]$ property.
\end{itemize}
\end{definition*}
}}\fi
