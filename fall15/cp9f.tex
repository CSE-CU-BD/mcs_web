\documentclass[handout]{mcs}

\begin{document}

\inclassproblems{9, Fri.}

\begin{staffnotes}
Counting Repetitions, Binomial Theorem: 14.3-14.7
\end{staffnotes}

%%%%%%%%%%%%%%%%%%%%%%%%%%%%%%%%%%%%%%%%%%%%%%%%%%%%%%%%%%%%%%%%%%%%%
% Problems start here
%%%%%%%%%%%%%%%%%%%%%%%%%%%%%%%%%%%%%%%%%%%%%%%%%%%%%%%%%%%%%%%%%%%%%

\pinput{CP_division_rule_assign_groups}
\pinput{CP_bookkeeper_tao}

\begin{staffnotes}
The following problem is really a review of the bijections from the
previous class, combined with bookkeeper formulas for the bijection
domain or codomain.  Encourage students to describe the bijections
explicitly, pointing out that mistakes in counting arguments often are
the result of a correspondence that is not a bijection.
\end{staffnotes}

\pinput{CP_nonadjacent_books_counting_sequel}

\pinput{CP_binom_coeff}
\pinput{CP_counting_practice}
%%%%%%%%%%%%%%%%%%%%%%%%%%%%%%%%%%%%%%%%%%%%%%%%%%%%%%%%%%%%%%%%%%%%%
% Problems end here
%%%%%%%%%%%%%%%%%%%%%%%%%%%%%%%%%%%%%%%%%%%%%%%%%%%%%%%%%%%%%%%%%%%%%


\end{document}


