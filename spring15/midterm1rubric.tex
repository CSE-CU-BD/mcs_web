\documentclass[12pt]{article}
\setlength{\parskip}{1pc}
\setlength{\parindent}{0pt}
\setlength{\topmargin}{-3pc}
\setlength{\textheight}{9in}
\setlength{\oddsidemargin}{0pc}
\setlength{\evensidemargin}{0pc}
\setlength{\textwidth}{6.8in}
\usepackage{color}
\begin{document}
\noindent
\textbf{6.042 Midterm 1 Grading Rubric, 2/26/15, 12PM}
%Written by Elizabeth Shen, 2/26/2015

\noindent
\emph{Question 1}\\
(Note that this problem is similar to CP1F \#2.)\\
Point breakdown:
\begin{itemize}
\item 1 point for making it clear that it's a proof by contradiction
\item 4 points for writing what it means for 7th root of 35 to be rational
	\begin{itemize}
	\item 2 for 7th root of 35=$\frac{a}{b} $
	\item 2 for a, b are relatively prime
	\end{itemize}
\item 5 points for getting to the point where we have that $35*b^7=a^7$
\item 7 points for getting to the point that 5 or 7 or 35 divides a
\item 8 points for getting that 5 or 7 or 35 divides b and so they are not relatively prime so contradiction
\end{itemize}
\noindent
Common \textcolor{red}{deductions}:
\begin{itemize}
\item 1 point for minor math errors (for example saying 3$\cdot$5=35)
\item 3 to 8 points for incorrect word ordering, such as ``divides" instead of ``is divided by"
\end{itemize}

\noindent
\emph{Question 2}\\
(Note that this problem was a clone of CP2W \#3.)\\
Point breakdown:
\begin{itemize}
\item 3 points for indicating some understanding of WOP (set of C, counterexamples, take the minimum element, etc) 
\item 3 points for having a metric defined to choose a minimum element.
\item 7 points for realizing that $c$ must be divisible by 3
\item 10 points for showing $(\frac{a}{3}, \frac{b}{3}, \frac{c}{3})$ is a solution
\item 2 points for a well-phrased conclusion mentioning the contradiction and that therefore there are no positive solutions to the equation.
\end{itemize}

\noindent
Common \textcolor{red}{deductions}:
\begin{itemize}
\item 1 point for using an unrelated variable in the proposition or counterexample set, such as ``$P(n) ::= 3a^4 + 9b^4 = c^4$ has no positive integer solutions''
\item 1 to 3 points for not concluding the proof, e.g. stopping at if $(a, b, c)$ is a solution, then $(\frac{a}{3}, \frac{b}{3}, \frac{c}{3})$ must be a solution. 
 
\end{itemize}

\noindent
\emph{Question 3}\\
Point breakdown: 4/3/4/4\\
Common \textcolor{red}{deductions}:
\begin{itemize}
\item 1 point for using a banned symbol (more if it trivializes the question)
\item $\frac{1}{2}$ point for missing $n=1$ in part c
\item 1 point for having an extra symbol in an otherwise correct predicate
\end{itemize}

\noindent
\emph{Question 4}\\
Point breakdown: 2/2/2/2/2\\
Note: In some cases, only 1 point was taken off if the student showed some work, e.g. if they said it was a surjection but the correct answer was bijection. \\

\noindent
\emph{Question 5}\\
(Note that this problem was a clone of CP4M \#3.)\\
Common \textcolor{red}{deductions}:
\begin{itemize}
\item 2-3 points for missing one of the base cases or having an error in one. 
\item 2 points for stating the strong inductive step incorrectly (Correct form is: assume for all $k\leq n$ that $S(k)$ is true; then we will show $S(n+1)$ is true) 
\item 5 points for treating S(n) as an equation rather than a proposition. 
\item 12-15 points for failing to show the inductive step (strong induction)
\item 15 points for a normal induction proof that failed at the inductive step
\item 15 points for using a base case/inductive step combination that skips some class sizes 

\end{itemize}



\end{document}
