\documentclass[handout]{mcs}

\begin{document}

\inclassproblems{3, Tue.}

%%%%%%%%%%%%%%%%%%%%%%%%%%%%%%%%%%%%%%%%%%%%%%%%%%%%%%%%%%%%%%%%%%%%%
% Problems start here
%%%%%%%%%%%%%%%%%%%%%%%%%%%%%%%%%%%%%%%%%%%%%%%%%%%%%%%%%%%%%%%%%%%%%%

\pinput{CP_domain_of_discourse}
\pinput{CP_assertions_about_binary_strings}
\pinput{PS_emailed_at_most_2_others}  %was PS_translate_sentence_into_predicate_formula
\pinput{CP_swapping_quantifiers}

\begin{center}
\textbf{Supplemental Problem}\footnote{There is no need to study supplemental
  problems when preparing for quizzes or exams.}
\end{center}

\begin{staffnotes}
This problem is a fun puzzle that provides more practice with
predicate formulas.  Tell students it's here for backup---we don't
really expect there will be time for it.
\end{staffnotes}

\pinput{PS_6042_staff_cabal}

\iffalse
\pinput{CP_variable_convention}
\pinput{PS_predicate_calculus_power_of_two}
\pinput{PS_6042_staff_cabal}
\pinput{FP_logic_of_leq}
\pinput{CP_logic_news_network}  %awkward
\pinput{PS_emailed_exactly_2_others}  %S13, PS2
\pinput{PS_express_in_predicate_form}
\pinput{PS_express_predicates_in_formal_logic_notation}
\pinput{PS_predicate_calculus_power_of_prime}  %S13, PS2
\pinput{PS_predicate_calculus_power_of_two}
\pinput{PS_rewrite_assertions_prime_goldbach_bertrand_fermat}
\pinput{PS_translate_to_predicate_logic}
\pinput{CP_a_season_for_every_purpose}   %covered in slides
\fi

% Problems end here
%%%%%%%%%%%%%%%%%%%%%%%%%%%%%%%%%%%%%%%%%%%%%%%%%%%%%%%%%%%%%%%%%%%%%
\end{document}
