\documentclass[handout]{mcs}

\begin{document}

\renewcommand{\reading}{Chapter~\bref{number_theory_chap}.\ \emph{Number Theory} through~\bref{Euler_sec}.\ \emph{Euler's Theorem}.}

\problemset{5}

\begin{staffnotes}
Lectures covered: Number Theory: GCD's, Ch. 8-8.5. Congruences
Ch. 8.6-8.9; $Z_n$, Euler's Theorem, Ch. 8.10
\end{staffnotes}


%%%%%%%%%%%%%%%%%%%%%%%%%%%%%%%%%%%%%%%%%%%%%%%%%%%%%%%%%%%%%%%%%%%%%
% Problems start here
%%%%%%%%%%%%%%%%%%%%%%%%%%%%%%%%%%%%%%%%%%%%%%%%%%%%%%%%%%%%%%%%%%%%%

%%%GCD%%%
%to be provided by Week 5 TAs
\pinput{PS_binary_pulverizer}
%\pinput{CP_gcd_TF}
%\pinput{CP_proving_basic_gcd_properties}

%%%Congruences%%%
%none of these were used in the past 3 semesters as pset problems- emshen
% \pinput{CP_proving_basic_congruence_properties} %should probably actually be used as a CP
%\pinput{CP_power_congruences} 
\pinput{PS_self-inverse_mod_p}

%\pinput{PS_fill_bucket_gcd}  NOT USED S14

%%%Euler's Theorem%%%
%\pinput{CP_7777}
%\pinput{CP_Sk_equiv_-1_mod_p}
\pinput{PS_Euler_function_multiplicativity}

\end{document}
