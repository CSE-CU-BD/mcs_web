\documentclass[handout]{mcs}

\begin{document}

\inclassproblems{8, Fri.}

%%%%%%%%%%%%%%%%%%%%%%%%%%%%%%%%%%%%%%%%%%%%%%%%%%%%%%%%%%%%%%%%%%%%%
% Problems start here
%%%%%%%%%%%%%%%%%%%%%%%%%%%%%%%%%%%%%%%%%%%%%%%%%%%%%%%%%%%%%%%%%%%%%

\pinput{CP_gcd_lcm}
\pinput{CP_proving_basic_congruence_properties}
\pinput{CP_multiples_of_9_and_11}
\pinput{CP_nonparallel_lines}

%%%%%%%%%%%%%%%%%%%%%%%%%%%%%%%%%%%%%%%%%%%%%%%%%%%%%%%%%%%%%%%%%%%%%
% Problems end here
%%%%%%%%%%%%%%%%%%%%%%%%%%%%%%%%%%%%%%%%%%%%%%%%%%%%%%%%%%%%%%%%%%%%%
\instatements{\newpage}

\section*{Appendix}

\iffalse
\begin{lemma}\label{mn}
Let $p_1,p_2,\dots,p_k$ be the unique factorization of an integer $n>1$
into primes.  That is, $p_1,p_2,\dots,p_k$ is a weakly increasing sequence
of primes and
\[
n = p_1 \cdot p_2\cdots p_k.
\]
If $m \divides n$ and $m>1$, then the unique factorization of
$m$ is a subsequence $p_{i_1},p_{i_2},\dots,p_{i_m}$ where $1 \leq i_1 < i_2 <
\cdots i_j \leq k$.
\end{lemma}
\fi

\begin{definition*}
$a \equiv b \pmod{n}$ iff $n \divides a-b$.
\end{definition*}
 
\begin{lemma}\label{facts} [Facts About Congruences]  The following  hold for 
$n \geq 1$:
%
\begin{enumerate}
\item $a \equiv a \pmod{n}$
\item $a \equiv b \pmod{n}$ implies $b \equiv a \pmod{n}$
\item $a \equiv b \pmod{n}$ and $b \equiv c \pmod{n}$ implies $a \equiv c \pmod{n}$
%\item $a \equiv b \pmod{n} \qiff \rem{a}{n} = \rem{b}{n}$
\item\label{cp8m.aran} $a \equiv \rem{a}{n} \pmod{n}$
\item $a \equiv b \pmod{n}$ implies $a + c \equiv b + c \pmod{n}$
\item $a \equiv b \pmod{n}$ implies $a c \equiv b c \pmod{n}$
\item $a \equiv b \pmod{n}$ and $c \equiv d \pmod{n}$ imply $a + c
\equiv b + d \pmod{n}$
\item $a \equiv b \pmod{n}$ and $c \equiv d \pmod{n}$ imply $a c
\equiv b d \pmod{n}$
\end{enumerate}
\end{lemma}

\begin{lemma}[Inverses mod $n$]
If $k$ and $n > 1$ are relatively prime, then there is a positive integer
$k^{-1} < n$ called the \emph{modulo $n$ inverse} of $k$, such that
\[
k \cdot k^{-1} \equiv 1 \pmod n.
\]

\begin{proof}
That integers $k$ and $n$ are relatively prime means that $\gcd(k,n)=1$.
But $\gcd(k,n)=1$ implies that $1= ak+bn$ for some integers $a,b$, and so
$1 \equiv ak \pmod n$.  So the positive integer less than $n$ that is
equivalent to $a \pmod n$ is $k^{-1}$, namely, $k^{-1} = \rem{a}{n}$.
\end{proof}
\end{lemma}

\end{document}