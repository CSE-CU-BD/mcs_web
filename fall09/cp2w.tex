\documentclass[handout]{mcs}

\begin{document}

\inclassproblems{2, Wed.}

%%%%%%%%%%%%%%%%%%%%%%%%%%%%%%%%%%%%%%%%%%%%%%%%%%%%%%%%%%%%%%%%%%%%%
% Problems start here
%%%%%%%%%%%%%%%%%%%%%%%%%%%%%%%%%%%%%%%%%%%%%%%%%%%%%%%%%%%%%%%%%%%%%

%S08 cp4m

\begin{problem}
  The proof below uses the Well Ordering Principle to prove that every
  amount of postage that can be paid exactly using only 6 cent and 15
  cent stamps, is divisible by 3.  That is, letting $S(n)$ mean that
  exactly $n$ cents postage can be paid using only 6 and 15 cent
  stamps, the proof shows that
%
\begin{equation}\tag{*}
S(n) \QIMPLIES [\text{n is divisable by 3, for all nonnegative integers $n$}].
\end{equation}
Fill in the missing portions (indicated by ``\dots'') of the following
proof of~(*).

\begin{quote}
Let $C$ be the set of \emph{counterexamples} to~(*), namely
\[
C \eqdef \set{n \suchthat \dots}
\]

\solution{ $n$ is a counterexample to~(*) if $n$ cents postage can be
  made and $n$ is not divisible by 3, so the predicate
\[
S(n)\text{ and } \QNOT(3 \divides n)
\]
defines the set, $C$, of counterexamples.
}

Assume for the purpose of obtaining a contradiction that $C$ is
nonempty.  Then by the WOP, there is a smallest number, $m \in C$.
This $m$ must be positive because\dots.

\solution{\dots $3 \divides 0$, so 0 is not a counterexample.}

But if $S(m)$ holds and $m$ is positive, then $S(m-6)$ or $S(m-15)$
must hold, because\dots.

\solution{\dots if $m>0$ cents postage is made from 6 and 15 cent
  stamps, at least one stamp must have been used, so removing this
  stamp will leave another amount of postage that can be made.  }

So suppose $S(m-6)$ holds.  Then $3 \divides (m-6)$, because\dots

\solution{\dots if $\QNOT(3 \divides (m-6))$, then $m-6$ would be
  a counterexample smaller than $m$, contradicting the minimality of
  $m$.}

But if $3 \divides (m-6)$, then obviously $3 \divides m$,
contradicting the fact that $m$ is a counterexample.

Next suppose $S(m-15)$ holds.  Then the proof for $m-6$ carries over
directly for $m-15$ to yield a contradiction in this case as well.
Since we get a contradiction in both cases, we conclude that\dots

\solution{\dots $C$ must be empty.  That is, there are no
  counterexamples to~(*), }

which proves that (*) holds.

\end{quote}
\end{problem}

\begin{problem}
Use the Well Ordering Principle to prove that
\begin{equation}\label{sum-of-sq}
\sum_{k=0}^n k^2 = \frac{n(n+1)(2n+1)}{6}.
\end{equation}
for all nonnegative integers, $n$.


\begin{solution}
The proof is by contradiction.

Suppose to the contrary that equation~\eqref{sum-of-sq} failed for some $n
\geq 0$.  Then by the WOP, there is a \emph{smallest} nonnegative integer,
$m$, such that~\eqref{sum-of-sq} does not hold when $n = m$.

But~\eqref{sum-of-sq} clearly holds when $n = 0$, which means that $m \geq
1$.  So $m-1$ is a nonegative, and since it is smaller than $m$,
equation~\eqref{sum-of-sq} must be true for $n = m-1$.  That is,
\begin{equation}\label{sum-to-m-1}
\sum_{k=0}^{m-1} k^2 = \frac{(m-1)((m-1) + 1)(2(m-1)+1)}{6}.
\end{equation}
Now add $m^2$ to both sides of equation~\eqref{sum-to-m-1}.
Then the left hand side equals
\[
\sum_{k=0}^{m} k^2
\]
and the right hand side equals
\[
\frac{(m-1)((m-1) + 1)(2(m-1)+1)}{6} + m^2 
\]
Now a little algebra\footnote{
\begin{align*}
\frac{(m-1)((m-1) + 1)(2(m-1)+1)}{6} + m^2 
= \frac{(m-1)m(2m-1)}{6} + m^2\\
 &  = \frac{(m^2-m)(2m-1)}{6} + m^2\\
 & = \frac{(2m^3-3m^2 +m)}{6} + \frac{6m^2}{6}\\
 &  = \frac{(2m^3 +3m^2 +m)}{6}
 &  = \frac{m(m+1)(2m+1)}{6}
\end{align*}}
shows that the right hand side equals
\[
\frac{m(m+1)(2m+1)}{6}.
\]
That is,
\[
\sum_{k=0}^{m} k^2 = \frac{m(m+1)(2m+1)}{6},
\]
contradicting the fact that equation~\eqref{sum-of-sq} does not hold for
$m$.

It follows that there is no smallest nonnegative integer for which
equation~\eqref{sum-of-sq} fails.  Hence~\eqref{sum-of-sq} must hod for
all nonnegative integers.

\end{solution}

\end{problem}

%S08 cp4m
\begin{problem}
We're interested in integer solutions to the equation
\begin{equation}\label{eq}
4a^3 + 2b^3 = c^3.
\end{equation}

\bparts

\ppart Use the Well-ordering Principle to prove that there is no integer
solution to~\eqref{eq} with $a>0$.

\solution{\hint Consider the smallest possible $a$.

The proof is by contradiction.

Let $S$ be the set of all positive integers, $a$, such that there exist
integers, $b$, and, $c$, that satisfy equation~\eqref{eq}.

Assume for the purpose of obtaining a contradiction that $S$ is nonempty.
Then $S$ contains a smallest element, $a_0>0$, by the Well-ordering
Principle.  By the definition of $S$, there exist corresponding integers,
$b_0$, and, $c_0$, such that:
\[ 
4a_0^3 + 2b_0^3 = c_0^3.
\] 
The left side of this equation is even, so $c_0^3$ is even, and therefore
$c_0$ is also even.  Thus, there exists an integer, $c_1$, such that $c_0
= 2 c_1$.  Now substituting $2c_1$ for $c_0$ in this equation and
then dividing both sides by 2 gives:
\begin{equation}\label{2a0}
2 a_0^3 + b_0^3 = 4 c_1^3,
\end{equation}
so
\[ 
b_0^3 = 2(2 c_1^3 -a_0^3).
\]
This implies that $b_0^3$ is even, so $b_0$ is even.  Thus, there exists an
integer, $b_1$, such that $b_0 = 2 b_1$.  Substituting $2b_1$ for $b_0$
in~\eqref{2a0} and dividing both sides by 2 gives:
\begin{equation}\label{a03} 
a_0^3 + 4 b_1^3 = 2 c_1^3 
\end{equation} 
From~\eqref{a03}, we conclude that $a_0^3$ is even, so $a_0$ is also even.
Thus, there exists an integer, $a_1$ such that $a_0 = 2 a_1$, where $a_1
>0$ since $a_0 >0$.  Substituting $2a_1$ for $a_0$ in~\eqref{a03} and
dividing by 2 one final time gives:
\[ 
4 a_1^3 + 2 b_1^3 = c_1^3 
\] 
So $a = a_1$, $b = b_1$, and $c = c_1$ is another solution to
the original equation~\eqref{eq}, and so $a_1$ is an element of $S$.  But
this is a contradiction, because $a_1 < a_0$ and $a_0$ was defined to be
the smallest element of $S$.  Therefore, our assumption was wrong, and the
original equation has no integer solutions with $a>0$.}

\ppart Show that the only integer solution to equation~\eqref{eq} is
$a=b=c=0$.

\solution{There are two cases left to consider: when $a<0$ or when $a=0$.

So let $S'$ be the set of all integers, $n>0$, such that there is an
integer solution to equation~\eqref{eq} with $a=-n$.  The previous
argument now goes through with $S'$ in place of $S$, proving that $S'$ is
empty.  So there is no solution with to equation~\eqref{eq} with $a<0$.

Finally, with $a=0$, to solve~\eqref{eq}, we need integers $b,c$ such that
\begin{equation}\label{2b}
2b^3=c^3.
\end{equation}
Now if $b=0$ in~\eqref{2b}, then obviously $c=0$, giving us the all-zero
solution.  On the other hand, if $b\neq 0$ in~\eqref{2b}, then $c/b$ would
be a rational cube root of 2, and we know there is none.  So the only
possibility is the all-zero solution.}

\eparts

\end{problem}



%%%%%%%%%%%%%%%%%%%%%%%%%%%%%%%%%%%%%%%%%%%%%%%%%%%%%%%%%%%%%%%%%%%%%
% Problems end here
%%%%%%%%%%%%%%%%%%%%%%%%%%%%%%%%%%%%%%%%%%%%%%%%%%%%%%%%%%%%%%%%%%%%%
\end{document}

