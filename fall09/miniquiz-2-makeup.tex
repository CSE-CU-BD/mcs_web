\documentclass[quiz]{mcs}

\begin{document}
\begin{center}
{\Large Morning}
\end{center}

\miniquiz{Miniquiz 2 Makeup}

%\section*{MORNING / AFTERNOON}                                                                    

%%%%%%%%%%%%%%%%%%%%%%%%%%%%%%%%%%%%%%%%%%%%%%%%%%%%%%%%%%%%%%%%%%%%%                  
% Problems start here                                                                  
%%%%%%%%%%%%%%%%%%%%%%%%%%%%%%%%%%%%%%%%%%%%%%%%%%%%%%%%%%%%%%%%%%%%%                  

% new problem - partial orders
% -------------------------------------------------------------------
\begin{problem}[7]

  \newcommand{\FLBS}{$\mathtt{0,1}^*$}
  \newcommand{\stringlnth}[1]{\mopt{length}(#1)}
 
  Define a strict partial order, $\prec$, on the set of positive length 
  binary strings based on their length.  Namely, for binary strings $s$ 
  and $t$,
  \[
  [s \prec t]\quad \eqdef\quad [\stringlnth{s} < \stringlnth{t}].
  \]
  You may assume without proof that $\prec$ is a strict partial order on
  this set of strings.

  \bparts

  \ppart[1] Describe the minimal element(s) of this partial order.  Is there
  a minimum element?

  \ppart[1] Give an example of 6 strings that form an antichain in this
  partial order.

  \ppart[3] Prove that $\prec$ is a well-founded partial order.

  \hint Apply the Well Ordering Principle to lengths.

  \ppart[2] Let $\preceq$ be a similarly defined reflexive relation on
  binary strings, namely,
  \[
  [s \preceq t]\quad \eqdef\quad [\stringlnth{s} \leq \stringlnth{t}].
  \]
  Explain why $\preceq$ is not a weak partial order on binary strings.
 
  \eparts
\end{problem}
\inhandout{\instatements{\newpage}}


%%%%%%%%%%%%%%%%%%%%%%


\begin{problem}[3]
For each of the binary relations below, indicate whether it is a
\textbf{F}unction, \textbf{A}symmetric, \textbf{TRAN}sitive, a total order
\textbf{TOrd}, or \textbf{N}one of the above.

\iffalse
 \textbf{R}eflexive,
\textbf{A}ntisymmetric, \textbf{TRAN}sitive, \textbf{TOT}al, or
\textbf{N}one of the above.
\fi

\iffalse

\begin{tabular}{l}
A \textbf{F}unction,\\
\textbf{A}symmetric,\\
\textbf{TRAN}sitive,\\
%a total relation \textbf{TRel},\\
%a strict \textbf{P}artial Order,\\
a total order \textbf{TOrd} or\\
\textbf{N}one of the above.
\end{tabular}
\fi


(More than one property may hold for some relations.)
\begin{enumerate}

\item
The relation on the set $\set{1,2,3,4}$ whose graph is

$\set{(1,1),(1,3),(3,1)}$ \hfill \brule{1.5in}

\item The relation on the set $\set{1,2,3,4}$ whose graph is

$\set{(1,2), (1,3), (1,4), (3,2), (2,4), (3,4)}$ \hfill \brule{1.5in}

\item The relation ``has the same name'' on people.   \hfill \brule{1.5in}

(People are considered to have the same name as themselves.)
\end{enumerate}

\end{problem}


%[Hide Quoted Text]

%[Relations: minimal, total, chain, antichain]
%Based off of TP.3.4 (smaller version of the set) and class problems 3T,

\begin{problem}[2]
Let the set $\set{2,6,9,12,18,27,36,96}$ be \iffalse
(weakly)\fi partially ordered by
the divides relation, $m$ is considered ``less that or equal to'' $n$ iff
$m$ divides $n$.  For this partial order

\begin{itemize}

\item what are the minimal elements? \hfill\brule{1.5in}

\item what are the maximal elements? \hfill\brule{1.5in}

\item give an example of a maximum-size chain. \hfill\brule{1.5in}

\item give an example of a maximum-size antichain. \hfill\brule{1.5in}

\end{itemize}

\end{problem}
\inhandout{\instatements{\newpage}}
%%%%%%%%%%%%%%%%%%%%%%%%%%%%%%%%%%%%%

\begin{problem}[4]

\bparts

\ppart[1] Express the sum of the first $n$ odd numbers, $1+3+5+\dots$,
without the dots by filling in the two missing parts in the following
$\sum$-expression:
\[
\underbrace{1+3+5+\dots}_{n \text{ terms}} = \sum_{i=0}^{(\qquad)} (\quad\qquad)
\]

\ppart[3] Prove by induction that this sum is $n^2$.

\eparts
\end{problem}



%%%%%%%%%%%%%%%%%%%%%%%%%%%%%%%%%%%%%%%%%%%%%%%%%%%%%%%%%%%%%%%%%%%%%                  
% Problems end here                                                                    
%%%%%%%%%%%%%%%%%%%%%%%%%%%%%%%%%%%%%%%%%%%%%%%%%%%%%%%%%%%%%%%%%%%%%                  
\end{document}

