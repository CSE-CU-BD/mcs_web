\documentclass[handout]{mcs}

\begin{document}

\inclassproblems{12, Mon.}

%%%%%%%%%%%%%%%%%%%%%%%%%%%%%%%%%%%%%%%%%%%%%%%%%%%%%%%%%%%%%%%%%%%%%
% Problems start here
%%%%%%%%%%%%%%%%%%%%%%%%%%%%%%%%%%%%%%%%%%%%%%%%%%%%%%%%%%%%%%%%%%%%%

\pinput{CP_multinomial_euler} %renamed from CP_multinomial_theorem_2
\pinput{CP_bag_of_donuts}
\pinput{CP_gen_func_sum_of_squares} %renamed from CP_sum_of_squares_2
\pinput{CP_nth_derivative_of_A}

%%%%%%%%%%%%%%%%%%%%%%%%%%%%%%%%%%%%%%%%%%%%%%%%%%%%%%%%%%%%%%%%%%%%%
% Problems end here
%%%%%%%%%%%%%%%%%%%%%%%%%%%%%%%%%%%%%%%%%%%%%%%%%%%%%%%%%%%%%%%%%%%%%

\section{Appendix}

\subsection{Multinomial Theorem}

\begin{definition}
For $n,k_1,\dots,k_m \in naturals$, such that $k_1+k_2+\cdots+k_m = n$,
define the \term{multinomial coefficient}
\[
\binom{n}{k_1, k_2, \dots, k_m} \eqdef \frac{n!}{k_1!\, k_2!\, \dots k_m!}.
\]
\end{definition}

\begin{theorem}[Multinomial Theorem]\label{ml}
For all $n \in \mathbb{N}$ and $z_1, \dots z_m \in \mathbb{R}$:
\[
(z_1 + z_2 + \cdots + z_m)^n =
   \sum_{\substack{k_1, \dots, k_m \in \mathbb{N} \\
                   k_1 + \cdots + k_m = n}}
   \binom{n}{k_1, k_2, \dots, k_m} z_1^{k_1} z_2^{k_2} \cdots z_m^{k_m} 
\]
\end{theorem}


\subsection{Convolution Rule}

Let
\begin{align*}
A(x) & = \sum_{n=0}^{\infty} a_n x^n, &
B(x) & = \sum_{n=0}^{\infty} b_n x^n, &
C(x) & = A(x) \cdot B(x) = \sum_{n=0}^{\infty} c_n x^n.
\end{align*}
Then
\[
c_n = a_0 b_n + a_1 b_{n-1} + a_2 b_{n-2} + \cdots + a_n b_0.
\]

\begin{mathrule}[Convolution Rule]
Let $A(x)$ be the generating function for selecting items from set
${\cal A}$, and let $B(x)$ be the generating function for selecting
items from set ${\cal B}$.  If ${\cal A}$ and ${\cal B}$ are disjoint,
then the generating function for selecting items from the union ${\cal
A} \cup {\cal B}$ is the product $A(x) \cdot B(x)$.
\end{mathrule}

By the Convolution Rule, or by Problem %~\ref{tay}, % FIXME
\[
\frac{1}{\paren{1-x}^k} = \sum_{n=0}^\infty \binom{n+k-1}{k-1}x^n.
\]

\subsection{Partial Fractions}

Here's a particular case of the Partial Fraction Rule that should be
enough to illustrate the general Rule:

Let $\alpha, \beta, \gamma$ be distinct complex numbers, and
let $p(x)$ be a polynomial of degree less than 6 $(= 3+1+2)$.  Then
there are unique numbers $A_1,A_2,B,C_1,C_2,C_3 \in \complexes$ such that

\[
\frac{p(x)}{(1-\alpha x)^2 (1-\beta x) (1-\gamma x)^3}
= \frac{A_1}{1-\alpha x} + \frac{A_2}{(1-\alpha x)^2}
+ \frac{B}{1-\beta x}
+ \frac{C_1}{1-\gamma x} + \frac{C_2}{(1-\gamma x)^2} + \frac{C_3}{(1-\gamma x)^3}
\]

\end{document}
