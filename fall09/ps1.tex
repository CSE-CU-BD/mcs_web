\documentclass[handout]{mcs}

\begin{document}

\renewcommand{\reading}{Notes Chapters 1--3.}

\problemset{1}

%%%%%%%%%%%%%%%%%%%%%%%%%%%%%%%%%%%%%%%%%%%%%%%%%%%%%%%%%%%%%%%%%%%%%
% Problems start here
%%%%%%%%%%%%%%%%%%%%%%%%%%%%%%%%%%%%%%%%%%%%%%%%%%%%%%%%%%%%%%%%%%%%%


\pinput{PS_log2-of-3-irrational}

\pinput{PS_prime-polynomial-41}

\pinput{PS_printout_binary_strings}

\begin{problem}
Prove that the propositional formulas
\[
P \QOR Q \QOR R
\]
and
\[
(P \QAND \QNOT Q) \QOR (Q \QAND \QNOT R) \QOR (R \QAND \QNOT P) \QOR (P \QAND Q \QAND R).
\]
are equivalent.

\begin{solution}
TBA
\end{solution}
\end{problem}

\pinput{PS_parallel-half-adder.tex}


%%%%%%%%%%%%%%%%%%%%%%%%%%%%%%%%%%%%%%%%%%%%%%%%%%%%%%%%%%%%%%%%%%%%%
% Problems end here
%%%%%%%%%%%%%%%%%%%%%%%%%%%%%%%%%%%%%%%%%%%%%%%%%%%%%%%%%%%%%%%%%%%%%
\end{document}
