\documentclass[handout]{mcs}

\begin{document}

\problemset{1}

%%%%%%%%%%%%%%%%%%%%%%%%%%%%%%%%%%%%%%%%%%%%%%%%%%%%%%%%%%%%%%%%%%%%%
% Problems start here
%%%%%%%%%%%%%%%%%%%%%%%%%%%%%%%%%%%%%%%%%%%%%%%%%%%%%%%%%%%%%%%%%%%%%


%unused from S09 cp2m
\begin{problem}
  You saw in a class problem on Monday, Week 2 how to design a simple
  half-adder circuit using propositional formulas.  That design was called
  ``ripple carry'' because the value of the carry from each digit combines
  directly with the following digit, and so the values of the carries
  ripple through all the successive inputs.  In particular, in an $n$-bit
  ripple carry half-adder, there is a path from the first input wire to
  the final carry out of the $n$th digit which goes through at least $n$
  gates,

Considerably faster adder circuits work by computing the values in later
columns for both a carry of 0 and a carry of 1, \emph{in parallel}.  Then
when the carry from the earlier columns finally arrives, the pre-computed
answer can be quickly selected.  We'll illustrate this idea by working out
the equations for an $n+1$-bit parallel half-adder.

Parallel half-adders are built out of parallel ``add1'' modules.  An
$n+1$-bit add1 module takes as input the $n+1$-bit binary representation,
$a_n \dots a_1 a_0$, of an integer, $s$, and produces as output the binary
representation, $c\,p_n\dots p_1\,p_0$, of $s+1$.

\bparts

\ppart A 1-bit add1 module just has input $a_0$.  Write propositional
formulas for its outputs $c$ and $p_0$.

\begin{solution}

\begin{align}
p_0 & = a_0\ \QXOR\ 1 = \QNOT(a_0)\\
c & = a_0.
\end{align}

\end{solution}

\ppart Explain how to build an $n+1$-bit parallel half-adder from an
$n+1$-bit add1 module by writing a propositional formula for the
half-adder output, $o_i$, using only the variables $a_i$, $p_i$, and $b$.

\begin{solution}

\[
o_i  = (b\ \QAND\ p_i)\ \QOR\ (\QNOT(b)\ \QAND\ a_i)
\]

\end{solution}

\eparts

We can build a double-size add1 module with $2(n+1)$ inputs using two
single-size add1 modules with $n+1$ inputs.  Suppose the inputs of the
double-size module are $a_{2n+1},\dots, a_1, a_0$ and the outputs are
$c,p_{2n+1},\dots, p_1,p_0$.  The setup is illustrated in
Figure~\ref{fig:add1}.

Namely, the first single size add1 module handles the first $n+1$ inputs.
The inputs to this module are the low-order $n+1$ input bits $a_n,\dots,
a_1, a_0$, and its outputs will serve as the first $n+1$ outputs $p_n,
\dots, p_1, p_0$ of the double-size module.  Let $c_{(1)}$ be the
remaining carry output from this module.

The inputs to the second single-size module are the higher-order $n+1$
input bits $a_{2n+1}, \dots, a_{n+2}, a_{n+1}$.  Call its first $n+1$
outputs $r_n, \dots, r_1, r_0$ and let $c_{(2)}$ be its carry.

\bparts

\ppart Write a formula for the carry, $c$, in terms of $c_{(1)}$ and
$c_{(2)}$.

\begin{solution}

\[
c = c_{(1)}\ \QAND\ c_{(2)}.
\]

\end{solution}

\ppart Complete the specification of the double-size module by writing
propositional formulas for the remaining outputs, $p_i$, for $n+1 \leq i
\leq 2n+1$.  The formula for $p_i$ should only involve the variables
$a_i$, $r_{i-(n+1)}$, and $c_{(1)}$.

\begin{solution}
 The $n+1$ high-order outputs of the double-size module are the
  same as the inputs if there is no carry from the low-order $n+1$
  outputs, and otherwise is the same as the outputs of the second
  single-size add1 module.  So
\begin{equation}\label{parallel-pi}
p_i = (\QNOT(c_{(1)})\ \QAND\ a_i)\ \QOR\ (c_{(1)}\ \QAND\ r_{i-(n+1)}).
\end{equation}
for $n+1 \leq i \leq 2n+1$.
\end{solution}

\ppart Parallel half-adders are exponentially faster than ripple-carry
half-adders.  Confirm this by determining the largest number of
propositional operations required to compute any one output bit of an
$n$-bit add module.  (You may assume $n$ is a power of 2.)

\begin{solution}
 The most operations for an output are those specified in
  formula~\eqref{parallel-pi}.  So it takes at most 4 additional
  operations to get any one double-size output bit from the single-size
  output bits that it depends on.  It takes $\log_2 n$ doublings to get to
  from 1-bit to $n$-bit modules, so the largest number of operations
  needed for any one output bit is $4 \log_2 n$.

This observation also shows that the \emph{total} number of operations
used in the parallel adder to calculate \emph{all} the output digits is
propositional to $ n \log_2 n$.  This is larger than the total for a
ripple-carry adder by a factor proportional to $\log_2 n$.
\end{solution}

\eparts

\end{problem}

\begin{problem}
  Show that no nonconstant polynomial, $p(x)$, with integer coefficients
  can map all nonnegative integers into prime numbers.

  \hint Let $c$ be the constant term in the polynomial.  Consider two
  cases: $c=0$ and $c \neq 0$.  In the second case, note that $p(cn)$ is a
  multiple of $c$ for all $n \in \integers$.  You may assume the familiar
  fact that any nonconstant polynomial, $p(x)$, grows unboundedly (in
  absolute value) as $x$ grows.
 
\begin{solution}
%from ARM email reply S07:
\begin{proof}
Suppose $p$ is a polynomial of degree $d$. So
\[
p(x) = \sum_{i=0}^d c_ix^i
\]
for some integer constants $c_i$.  Let $c \eqdef c_0$ be the constant term
of $p$.

The proof is by cases following the hint.  

\textbf{Case 1} ($c = 0$): Then all the terms in $p(x)$ are multiples of
$x$, so $p(2n)$ is always even.  Since $p$ is not constant, $\abs{p(2n)}$
grows unboundedly as $n$ increases.  But as soon as $\abs{p(2n)}$
grows bigger than 2, it won't be prime because it has 2 as a factor.

\textbf{Case 2} ($c \neq 0$): In this case, $p(cn)$ has $c$ as a factor
for all integers, $n$.  Because $p$ is not constant, $\abs{p(cn)}$
grows unboundedly as $n$ increases.  But as soon as $\abs{p(2n)}$
grows bigger than $c$, it won't be prime because it has $c$ as a factor.
\end{proof}

\end{solution}

\end{problem}

%%OTHER TOPICS:

%S09 ps1:
\begin{problem}
  Week Notes 1 described a conjecture made by the great Mathematician Euler
  in 1769: there are no positive integer solutions to the equation
\[
a^4 + b^4 + c^4 =  d^4.
\]
The Notes also gave integer values for $a,b,c,d$ that do satisfy this
equation, discovered by Noam Elkies at Harvard more than two hundred years
later.  So Euler guessed wrong.

Now let's consider Lehman's\footnote{Suggested by Eric Lehman, a former
  6.042 Lecturer.} equation, similar to Euler's but with some
coefficients:
\begin{eqnarray}\label{wc}
8 a^4 + 4 b^4 + 2 c^4 & = & d^4
\end{eqnarray}

Prove that Lehman's equation~\eqref{wc} really does not have any positive
integer solutions.

\hint Consider the minimum value of $a$ among all possible solutions
to~\eqref{wc}.

\solution{Suppose that there exists a solution.  Then there must be a
  solution in which $a$ has the smallest possible value.  We will show
  that, in this solution, $a$, $b$, $c$, and $d$ must all be even.
  However, we can then obtain another solution over the positive integers
  with a smaller $a$ by dividing $a$, $b$, $c$, and $d$ in half.  This is
  a contradiction, and so no solution exists.

All that remains is to show that $a$, $b$, $c$, and $d$ must all be even.
The left side of Lehman's equation is even, so $d^4$ is even, so $d$ must
be even.  Substituting $d = 2 d'$ into Lehman's equation gives:

\begin{eqnarray}
8 a^4 + 4 b^4 + 2 c^4 & = & 16 d'^4
\end{eqnarray}

Now $2 c^4$ must be a multiple of 4, since every other term is a
multiple of 4.  This implies that $c^4$ is even and so $c$ is also
even.  Substituting $c = 2 c'$ into the previous equation gives:

\begin{eqnarray}
8 a^4 + 4 b^4 + 32 c'^4 & = & 16 d'^4
\end{eqnarray}

Arguing in the same way, $4 b^4$ must be a mutliple of 8, since every
other term is.  Therefore, $b^4$ is even and so $b$ is even.
Substituting $b = 2 b'$ gives:

\begin{eqnarray}
8 a^4 + 64 b'^4 + 32 c'^4 & = & 16 d'^4
\end{eqnarray}

Finally, $8 a^4$ must be a multiple of 16, $a^4$ must be even, and so
$a$ must also be even.  Therefore, $a$, $b$, $c$, and $d$ must all be
even, as claimed.}
\end{problem}


\iffalse

\begin{problem}
Show that $\log_{7} n$ is either an integer or irrational, where $n$ is a
positive integer.  Use whatever familiar facts about integers and primes
you need, but explicitly state such facts.  (This problem will be graded
on the clarity and simplicity of your proof.  If you can't figure out how
to prove it, ask the staff for help and they'll tell you how.)

\solution{The statement to be proved is equivalent to the assertion that, for
all positive integers, $n$, if $\log_7 n$ is rational, then it is an
integer.

So we'll assume
\begin{equation}\label{ij}
\log_7 n = \frac{i}{j}
\end{equation}
for some integers, $i,j$.  Now if $i=0$, then $\log_7 n$ is the integer
0, so we can asssume $i>0$ and $n>1$.  Using the fact that $\log_7 x$
is positive for $x>1$, we can conclude that $j> 0$, also.

Now, raising $7$ to the power $\log_7 n$, we have from~\eqref{ij}
\[
n  = 7^{\log_7 n} = 7^{i/j}.
\]
Then, taking $j$th powers,
\begin{equation}\label{nj}
n^j =(7^{i/j})^j =  7^i.
\end{equation}
Since $i,j >0$, both sides of equation~\eqref{nj} are integers.  Also,
since the only prime dividing the righthand of~\eqref{nj} is 7,
\emph{the fact that integers factor uniquely into primes} implies that the
only prime factor of $n^j$, and hence the only prime factor of $n$, is 7.
This means that $n$ can only be a nonnegative power of 7, so $\log_7 n$
must be a nonnegative integer.}
\end{problem}

\fi

\begin{problem}
\bparts
\ppart Verify that $(P \xor Q)$ is equivalent to $\neg(P \iff Q)$.

\solution{
\[
\begin{array}{|c|c|c|}
\hline
\text{(}P & \xor  & Q\text{)}\\ \hline
\true     &\false      & \true    \\ \hline
\true     &\true       & \false   \\ \hline
\false    &\true       & \true    \\ \hline
\false    &\false      & \false   \\ \hline
\end{array}
\]

\[
\begin{array}{|c|ccc|}
\hline
\neg   &\text{(}P & \iff  & Q\text{)}\\ \hline
\false &\true     &\true  & \true    \\ \hline
\true  &\true     &\false & \false   \\ \hline
\true  &\false    &\false & \true    \\ \hline
\false &\false    &\true  & \false   \\ \hline
\end{array}
\]}

\ppart Verify that $P \Or Q \Or R$ is equivalent to
\[
(P \And \bar{Q}) \Or (Q \And \bar{R}) \Or (R \And \bar{P}) \Or (P \And Q \And R).
\]

\solution{TBA

The following table from cp2m might be useful to edit from:
\[
\begin{array}{|ccc|c|ccc|}
\hline
\text{[(}A  & \text{OR} & B\text{)} & \text{AND} & \text{(}A & \text{OR} & C\text{)]} \\  \hline
     \true  & \true     & \true     & \true      &   \true   & \true     & \true \\  \hline
     \true  & \true     & \true     & \true      &   \true   & \true     & \false\\  \hline
     \true  & \true     & \false    & \true      &   \true   & \true     & \true \\  \hline
     \true  & \true     & \false    & \true      &   \true   & \true     & \false\\  \hline
     \false & \true     & \true     & \true      &   \false  & \true     & \true \\  \hline
     \false & \true     & \true     & \false     &   \false  & \false    & \false\\  \hline
     \false & \false    & \false    & \false     &   \false  & \true     & \true \\  \hline
     \false & \false    & \false    & \false     &   \false  & \false    & \false\\  \hline
\end{array}
\]}

\eparts
\end{problem}

\begin{problem}
Describe a simple recursive procedure which, given a positive integer
argument, $n$, produces a truth table whose rows are all the assignments
of truth values to $n$ propositional variables.  For example, for $n=2$,
the table might look like:

\[\begin{array}{|c|c|}
\hline \true & \true\\
\true & \false\\
\false & \true\\
\false & \false\\
\hline
\end{array}\]

Your description can be in English, or a simple program in some familiar
language (say Scheme or Java), but if you do write a program, be sure to
include some sample output.

\solution{Start with an $n=1$ table, namely a one-column table whose
first row consists of a $\true$ entry and second row an $\false$ entry.
Build the $n+1$ table recursively by taking an $n$ table and attaching a
$\true$ at the beginning of every row, then taking another $n$ table and
attaching a $\false$ at the beginning of every row, and finally placing
the first table above the second table.

Here's a Scheme program that carries out this procedure:
\texttt{
\begin{tabbing}
(de\=fine (truth-values n)\\
   \> (if \= (= n 1) '((T) (F))\\
   \>    \> (let \=((table (truth-values (- n 1))))\\
              \>\>\> (ap\=pend\\
                \>\>\>\> (map (lambda (row) (cons 'T row)) table)\\
                \>\>\>\> (map (lambda (row) (cons 'F row)) table)))))\\
(truth-values 3)\\
;Value 17: \>\>\>((t t t) (t t f) (t f t) (t f f)\\
           \>\>\>\ (f t t) (f t f) (f f t) (f f f))
\end{tabbing}}
}
\end{problem}


%2 pred calc probs cut

\iffalse

\newcommand{\C}{C}

% Define the course staff here.

\newcommand{\Eric}{\text{Jeffrey }}
\newcommand{\Tom}{\text{Tina }}
\newcommand{\Albert}{\text{Albert }}
\newcommand{\Claire}{\text{Jessica }}
\newcommand{\Edmond}{\text{Jay }}
\newcommand{\Florent}{\text{Chuck }}
\newcommand{\Nick}{\text{Chiyoun }}

\begin{problem}
A certain cabal within the 6.042 course staff is plotting to make the
final exam \textit{ridiculously hard}.  (``Problem 1.  Prove the Poincare
Conjecture starting from the axioms of ZFC.  Express your answer in khipu
-- the knot language of the Incas.'')  The only way to stop their evil
plan is to determine exactly who is in the cabal.  The course staff
consists of seven people:
%
\[
\set{\Eric, \Tom, \Albert, \Claire, \Edmond, \Florent, \Nick }
\]
%
(\Florent is the course secretary.)  The cabal is a subset of these seven.
A membership roster has been found and appears below, but it is deviously
encrypted in logic notation.  The predicate $\C$ indicates who is in the
cabal; that is, $\C(x)$ is true if and only if $x$ is a member.  Translate
each statement below into English and deduce who is in the cabal.

\begin{enumerate}%[\upshape (i)]

\item\label{eee} $\exists x \ \exists y \ \exists z \
    (x \neq y \wedge
     x \neq z \wedge
     y \neq z \wedge
     \C(x) \wedge \C(y) \wedge \C(z))$

\solution{A direct English paraphrase would be ``There
exist people we'll call $x,y$, and $z$, who are all different, such that
$x,y$ and $z$ are each in the cabal.''  A better version would use the
fact that there's no need in this case to give names to the people.
Namely, a better paraphrase is ``There are 3 different people in the
cabal.''  Perhaps a simpler way to say this is: ``The cabal is of size at
least 3.''}

\item\label{nNC} $\neg (\C(\text{\Nick}) \wedge \C(\text{\Claire}))$

\solution{\Nick and \Claire are not both in the cabal.
Equivalently: at least one of \Nick and \Claire is not in the cabal.}

\item\label{Fall} $\C(\text{\Florent}) \rightarrow \forall x \ \C(x)$

\solution{If \Florent is in the cabal, then everyone is.}

\item\label{CN} $\C(\text{\Claire}) \rightarrow \C(\text{\Nick})$

\solution{If \Claire is in the cabal, then \Nick is also.}

\item\label{EAnT}
$(\C(\text{\Edmond}) \vee \C(\text{\Albert})) \rightarrow \neg \C(\text{\Tom})$

\solution{If either of \Edmond or \Albert is in the cabal,
then \Tom is not.  Equivalently, if \Tom \emph{is} in the cabal, the neither
\Albert nor \Edmond is.}

\item\label{ENnE}
$(\C(\text{\Edmond}) \vee \C(\text{\Nick})) \rightarrow \neg \C(\text{\Eric})$

\solution{If either of \Edmond or \Nick is in the cabal,
then \Eric is not.  Equivalently, if \Eric \emph{is} in the cabal, the
neither \Edmond nor \Nick is.  }
\end{enumerate}

\insolutions{So much for the translations.  We now argue that the only
cabal satisfying all six propositions above is one whose members are
exactly \Nick, \Edmond, and \Albert.

We first observe that by~\eqref{nNC}, there must be someone -- either \Nick
or \Claire -- who is not in the cabal.  But if Flo were in the cabal, then
by~\eqref{Fall}, everyone would be.  So we conclude by contradiction
that:

\begin{equation}\label{nF}
\text{\Florent is not in the cabal.}
\end{equation}

Next observe that if \Claire was in the cabal, then by~\eqref{CN}, \Nick would
be too, contradicting~\eqref{nNC}.  So by again contradiction, we conclude:
\begin{equation}\label{nC}
\text{\Claire is not in the cabal.}
\end{equation}

Now suppose \Tom is in the cabal.  Then by~\eqref{EAnT}, \Edmond and \Albert
are not, and we already know \Florent and \Claire are not, so only three remain
who could be in the cabal, namely, \Tom, \Nick, and \Eric.  But
by~\eqref{eee} the cabal must have at least three members, so it follows
that the cabal must consist of exactly these three.  This proves:
\begin{lemma}\label{TNE}
\text{If \Tom is in the cabal, then \Nick and \Eric are in the cabal.}
\end{lemma}

But by~\eqref{ENnE}, if \Nick is the cabal, then \Eric is not.  That is, 
\begin{lemma}\label{NnE}
\text{\Nick and \Eric cannot both be in the cabal.}
\end{lemma}
Now from Lemma~\ref{NnE} we conclude that the conclusion of
Lemma~\ref{TNE} is false.  So by contrapositive, the hypothesis of
Lemma~\ref{TNE} must also be false, namely,
\begin{equation}\label{nT}
\text{\Tom is not in the cabal.}
\end{equation}

Finally, suppose \Eric is in the cabal.  Then by~\eqref{ENnE}, \Edmond
and \Nick are not, and we already know \Florent, \Claire and \Tom are not. So
the cabel must consist of at most two people (\Albert and \Eric). This
contradicts~\eqref{eee}, and we conclude by contradiction that
\begin{equation}\label{nE}
\text{\Eric is not in the cabal.}
\end{equation}
So the only remaining people who could be in the cabal are \Albert, \Edmond,
and \Nick.  Since the cabal must have at least three members, we conclude
that
\begin{lemma}
The only possible cabal consists of \Albert, \Edmond, and \Nick.
\end{lemma}

But we're not done yet: we haven't shown that a cabal consisting of
\Albert, \Edmond, and \Nick actually does satisfy all six conditions.  So let
$\mathcal{A} =\set{\text{\Albert}, \text{\Edmond}, \text{\Nick}}$, and let's quickly
check that $\mathcal{A}$ satisfies~\eqref{eee}--\eqref{ENnE}:

\begin{itemize}

\item $\size{A} = 3$, so $A$ satisfies~\eqref{eee}.
\item \Claire is not in $A$, so $A$ satisfies~\eqref{nNC} and~\eqref{CN}.
\item \Florent is not in $A$, so the hypothesis of~\eqref{Fall} is false, which
means that $A$ satisfies~\eqref{Fall}.
\item Finally, \Tom and \Eric are not in $A$, so the conclusions of
both~\eqref{EAnT} and~\eqref{ENnE} are true, and so $A$
satisfies~\eqref{EAnT} and ~\eqref{ENnE}.

\end{itemize}

So now we have proved
\begin{proposition*}
$\set{\text{\Albert}, \text{\Edmond}, \text{\Nick}}$ is the \emph{unique} cabal
satisfying conditions~\eqref{eee}--\eqref{ENnE}.
\end{proposition*}}
\end{problem}
\fi


\iffalse

\begin{problem}
Let $A$, $B$, and $C$ be sets.
Prove that:
%
\[
A \cup B \cup C = (A - B) \cup (B - C) \cup (C - A) \cup (A \cap B \cap C).
\]
%
You are welcome to use a diagram to aid your own reasoning, but your
proof must be text.

\solution{
We prove that the left side is contained in the right side, and that
the right side is contained in the left side.

First, we show that the left side is contained in the right side.  Let $x$
be any element of $A \cup B \cup C$.  Then $x$ belongs to at least 
one of $A$, $B$, and $C$.  We distinguish two cases.
\begin{itemize}
\item $x$ belongs to all three sets: Then $x$ belongs to the
intersection $A\cap B\cap C$. 

\item $x$ does \emph{not} belong to all three sets: Then at least one
of $A$, $B$, $C$ does not contain $x$.  So overall, at least one set
contains $x$ and at least one set doesn't. We distinguish cases:
\begin{itemize}
\item If $A$ contains $x$, then one of $B$ and $C$ must not contain
it. 
\begin{itemize}
\item If $B$ does not contain it, then $x\in A-B$. 
\item If $B$ contains it, then $C$ does not, therefore $x\in B-C$.
\end{itemize}

\item If $A$ does \emph{not} contain $x$, then one of $B$ and $C$ must  
contain it. 

\begin{itemize}
\item If $C$ does, then $x\in C-A$. 
\item If $C$ does not contain it, then $B$ does, therefore $x\in B-C$. 
\end{itemize}
\end{itemize}
\end{itemize}
In all cases, we end up with $x$ being a member of one of 
$A - B$, $B - C$, $C - A$, or $A\cap B\cap C$.  Therefore, it belongs
to the right side. Hence, the set on the left is contained in the set
on the right.

Next, we show that the right side is contained in the left.  This is
easier. Let $x$ belong to the right side. Then it belongs to one of $A
- B$, $B - C$, $C - A$, or $A\cap B\cap C$.  In the first case, we
clearly know $x\in A$. In the second case, $x\in B$. In the third
case, $x\in C$. In the last case, $x\in A$ again. So, in all cases,
$x$ belongs to one of $A$, $B$, or $C$. So $x$ belongs to the left
side. Therefore, the set on the right is contained in the set on the
left.

Since each set is contained in the other, they are equal.}

\end{problem}


\begin{problem} %verbatim form Fall 05

\bparts
\ppart Give an example where the following result fails:

\begin{falsethm*}
For sets $A$, $B$, $C$, and $D$, let
\begin{align*}
L \eqdef (A \union C) \times (B \union D),\\
R \eqdef (A \times B) \union (C \times D).
\end{align*}
Then $L=R$.
\end{falsethm*}

\solution{
If $A=D=\emptyset$ and $B$ and $C$ are both nonempty, then $L = C \times B
\neq \emptyset$, but $R = \emptyset$.
}

\ppart Identify the mistake in the following proof of the False Theorem.

\begin{bogusproof}
Since $L$ and $R$ are both sets of pairs, it's sufficient to prove that
$(x,y) \in L \iff (x,y) \in R$ for all $x,y$.

The proof will be a chain of iff implications:
\begin{center}
\begin{tabular}{ll}
    & $(x, y) \in L$  \\
iff & $x \in A \union C$ and $y \in B \union D$ \\
iff & either $x \in A$ or $x \in C$, and either $y \in B$ or $y \in D$ \\
iff & ($x \in A$ and $y \in B$) or else ($x \in C$ and $y \in D$)  \\
iff & $(x,y) \in A \times B$, or $(x,y) \in C \times D$ \\
iff & $(x,y) \in (A \times B) \union (C \times D) $ \\
iff & $(x,y)\in R$.
\end{tabular}
\end{center}

\end{bogusproof}

\solution{The mistake is in the third ``iff.''  If [$x \in A$ or $x \in
C$, and either $y \in B$ or $y \in D$], it does not necessarily follow
that $(x,y) \in (A \times B) \union (C \times D)$.  It might be that
$(x,y)$ is in $A \times D$ instead.  This happens, for example, if $A =
\set{1}, B = \set{2}, C = \set{3}, D = \set{4}$, and $(x,y) = (1, 4)$.}

\ppart Fix the proof to show that $R \subseteq L$.

\solution{
Replacing the third ``iff'' by ``which will be true when,'' yields a
correct proof that $(x,y) \in L$ will be true when $(x,y) \in R$, which
implies that $R \subseteq L$.
}

\eparts

\end{problem}

%poor Russell problem cut
\fi

%%S09 ps1 ends here





%%%%%%%%%%%%%%%%%%%%%%%%%%%%%%%%%%%%%%%%%%%%%%%%%%%%%%%%%%%%%%%%%%%%%
% Problems end here
%%%%%%%%%%%%%%%%%%%%%%%%%%%%%%%%%%%%%%%%%%%%%%%%%%%%%%%%%%%%%%%%%%%%%
\end{document}
