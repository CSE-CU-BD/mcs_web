\documentclass[handout]{mcs}

\begin{document}
\paper{}{Oct. 21, 2009}{Appendix for Oct. 21 Miniquiz}

\section*{Appendix: Preserved Invariants}

\def\movesto{\mathrel{\longrightarrow}}

A \term{preserved invariant} of a state machine is a predicate, $P$, on states, such that whenever $P(q)$ is true of a state, $q$, and $q \movesto r$ for some state, $r$, then $P(r)$ holds.

\textbox{\begin{center}
{\Large The Invariant Principle}
\end{center}

{\large
\noindent If a preserved invariant of a state machine is true for the
start state,\\
then it is true for all reachable states.}}

\section*{Appendix: Derived Variables}

Let $\prec$ be a strict partial order on a set, $A$.  A derived variable
  $f : \text{states } \to A$ is \emph{strictly decreasing} iff
\[
q \movesto q' \text{  implies  } f(q') \prec f(q).
\]

Let $\preceq$ be a weak partial order on a set, $A$.  A derived variable
$f : Q \to A$ is \emph{weakly decreasing} iff
\[
q \movesto q' \text{  implies  } f(q') \preceq f(q).
\]

\section*{Appendix: Stable Matching}

\providecommand{\boys}{\text{the-Boys}}
\providecommand{\girls}{\text{the-Girls}}

A \term{Marriage Problem} consists of two disjoint sets of the same
  finite size, called \boys\ and \girls.  The members of \boys\ are called
  \emph{boys}, and members of \girls\ are called \emph{girls}.  For each
  boy, $B$, there is a strict total order, \term{$<_B$}, on \girls, and
  for each girl, $G$, there is a strict total order, \term{$<_G$}, on
  \boys.  If $G_1 <_B G_2$ we say $B$ \term*{prefers} girl $G_2$ to girl
  $G_1$.  Similarly, if $B_1 <_G B_2$ we say $G$ \term{prefers} boy $B_2$
  to boy $B_1$.

A \term{marriage assignment} or \term{perfect matching} is a bijection,
$w:\boys \to \girls$.  If $B \in \boys$, then $w(B)$ is called $B$'s
\emph{wife} in the assignment, and if $G \in \girls$, then $w^{-1}(G)$ is
called $G$'s \emph{husband}.  A \term{rogue couple} is a boy, $B$, and a
girl, $G$, such that $B$ prefers $G$ to his wife, and $G$ prefers $B$ to
her husband.  An assignment is \term{stable} if it has no rogue couples.
A \term{solution} to a marriage problem is a stable perfect matching.

The Mating Ritual marries every boy to his optimal spouse.

The Mating Ritual marries every girl to her pessimal spouse.

\section*{Appendix: Structural Induction}

\hyperdef{struct}{induction}{Structural} induction is a method for proving
some property, $P$, of all the elements of a recursively-defined data
type.  The proof consists of two steps:
\begin{itemize}
\item Prove $P$ for the base cases of the definition. 
\item Prove $P$ for the constructor cases of the definition, assuming that it
  is true for the component data items.  
\end{itemize}

\section*{Appendix: Simple Graphs}

A \term{simple graph}, $G$, consists of a nonempty set, $V$, called the
\term{vertices} of $G$, and a collection, $E$, of two-element subsets of
$V$.  The members of $E$ are called the \term{edges} of $G$.

Two vertices in a simple graph are said to be \term{adjacent} if they are
joined by an edge, and an edge is said to be \term{incident} to the
vertices it joins.  The number of edges incident to a vertex is called the
\term{degree} of the vertex; equivalently, the degree of a vertex is
equals the number of vertices adjacent to it.

\begin{enumerate}
\item Simple graphs do \emph{not} have edges going from a vertex back
  around to itself (called a \term{self-loop}).

\item There is at most one edge between two vertices of a simple graph.

\item Simple graphs have at least one vertex, though they might not have
any edges.
\end{enumerate}

\section*{Appendix: Degrees and Isomorphism}

The sum of the degrees of the vertices in a graph equals twice the number
of edges.

  If $G_1$ is a graph with vertices, $V_1$, and edges, $E_1$, and likewise
  for $G_2$, then $G_1$ is \term{isomorphic} to $G_2$ iff there exists a
  \textbf{bijection}, $f: V_1 \to V_2$, such that for every pair of
  vertices $u, v \in V_1$:

\section*{Appendix: Graph Connectedness}

  Two vertices in a graph are said to be \term{connected} if there is a
  path that begins at one and ends at the other.  By convention, every
  vertex is considered to be connected to itself by a path of length zero.

  Two vertices in a graph are $k$-\term{edge connected} if they remain
  connected in every subgraph obtained by deleting $k-1$ edges.  A graph
  with at least two vertices is $k$-edge connected\footnote{The
    corresponding definition of connectedness based on deleting vertices
    rather than edges is common in Graph Theory texts and is usually
    simply called ``$k$-connected'' rather than ``$k$-vertex connected.''}
  if every two of its vertices are $k$-edge connected.

\section*{Appendix: Trees}

Every tree has the following properties:

\begin{enumerate}
\item Any connected subgraph is a tree.
\item There is a unique simple path between every pair of vertices.
\item Adding an edge between two vertices creates a cycle.
\item Removing any edge disconnects the graph.
\item If it has at least two vertices, then it has at least two leaves.
\item The number of vertices is one larger than the number of edges.
\end{enumerate}

\end{document}
