\documentclass[handout]{mcs}

\begin{document}

\inclassproblems{12, Wed.}

%%%%%%%%%%%%%%%%%%%%%%%%%%%%%%%%%%%%%%%%%%%%%%%%%%%%%%%%%%%%%%%%%%%%%
% Problems start here
%%%%%%%%%%%%%%%%%%%%%%%%%%%%%%%%%%%%%%%%%%%%%%%%%%%%%%%%%%%%%%%%%%%%%

\pinput{CP_towers_of_Sheboygan}
\pinput{CP_gen_fcn_quotient_polynomials}

%%%%%%%%%%%%%%%%%%%%%%%%%%%%%%%%%%%%%%%%%%%%%%%%%%%%%%%%%%%%%%%%%%%%%
% Problems end here
%%%%%%%%%%%%%%%%%%%%%%%%%%%%%%%%%%%%%%%%%%%%%%%%%%%%%%%%%%%%%%%%%%%%%
\section{Appendix}

\subsection{Coefficients by Convolution}
\iffalse

Let
\begin{align*}
A(x) & = \sum_{n=0}^{\infty} a_n x^n, &
B(x) & = \sum_{n=0}^{\infty} b_n x^n, &
A(x) \cdot B(x) & = \sum_{n=0}^{\infty} c_n x^n.
\end{align*}
Then
\[
c_n = a_0 b_n + a_1 b_{n-1} + a_2 b_{n-2} + \cdots + a_n b_0.
\]

Letting $[x^n]F(x)$ denote the coefficient of $x^n$ in the power series
for $F(x)$, we can restate this as
\[
[x^n]\paren{A(x)\cdot B(x)} \eqdef \sum_{k=0}^n [x^k]A(x)\cdot [x^{n-k}]B(x)
\]

\begin{mathrule*}[Convolution Rule]
Let $A(x)$ be the generating function for selecting items from set
${\cal A}$, and let $B(x)$ be the generating function for selecting
items from set ${\cal B}$.  If ${\cal A}$ and ${\cal B}$ are disjoint,
then the generating function for selecting items from the union ${\cal
A} \cup {\cal B}$ is the product $A(x) \cdot B(x)$.
\end{mathrule*}
\fi

Letting $[x^n]F(x)$ denote the coefficient of $x^n$ in the power series
for $F(x)$, the Convolution Rule implies that
\[
[x^n]\paren{\frac{1}{\paren{1-x}^k}} = \binom{n+k-1}{k-1}.
\]
and hence
\begin{equation}\label{1axk}
[x^n]\paren{\frac{1}{\paren{1-\alpha x}^k}} = \binom{n+k-1}{k-1}\alpha^n.
\end{equation}


\subsection{Partial Fractions}

Here's a particular case of the Partial Fraction Rule that should be
enough to illustrate the general Rule.  Let $\alpha, \beta, \gamma$ be
distinct complex numbers, and let $p(x)$ be a polynomial of degree less
than 6 $(= 3+1+2)$.  Then there are unique numbers $a_1,a_2,b,c_1,c_2,c_3
\in \complexes$ such that

\[
\frac{p(x)}{(1-\alpha x)^2 (1-\beta x) (1-\gamma x)^3}
= \frac{a_1}{1-\alpha x} + \frac{a_2}{(1-\alpha x)^2}
+ \frac{b}{1-\beta x}
+ \frac{c_1}{1-\gamma x} + \frac{c_2}{(1-\gamma x)^2} + \frac{c_3}{(1-\gamma x)^3}
\]

Partial fractions together with~\eqref{1axk} imply that there is a closed
form expression for $[x^n]\paren{R(x)/S(x)}$ for arbitrary polynomials
$R(x),S(x)$.

\subsection{Finding a Generating Function for Fibonacci Numbers}
The Fibonacci numbers are defined by:
\begin{align*}
b_0 & \eqdef 0 \\
b_1 & \eqdef 1 \\
b_n & \eqdef b_{n-1} + b_{n-2} \qquad \text{(for $n \geq 2$)}
\end{align*}

Let $B$ be the generating function for the Fibonacci numbers, that is,
\[
B(x) \eqdef b_0 + b_1 x + b_2 x^2 + b_3 x^3 + b_4 x^4 + \cdots
\]
So we need to derive a generating function whose series has coefficients:
\begin{equation}\label{fibcoeff}
\ang{0,\ 1,\ b_1 + b_0,\ b_2 + b_1,\ b_3 + b_2,\ \dots}
\end{equation}
Now we observe that
\[
\begin{array}{ccccccccccl}
  & \langle & 0, & 1, & 0, & 0, & 0, & \dots & \rangle
    & \corresp & x \\
  & \langle & 0, & b_0, & b_1, & b_2, & b_3, & \dots & \rangle
    & \corresp & x B(x) \\
+ & \langle & 0, & 0, & b_0, & b_1, & b_2, & \dots & \rangle
    & \corresp & x^2 B(x) \\ \hline
  & \langle & 0, & 1 + b_0, & b_1 + b_0, & b_2 + b_1, & b_3 + b_2, & \dots & \rangle
    & \corresp & x + x B(x) + x^2 B(x) \\
\end{array}
\]
%
This last sequence is almost identical to the Fibonacci
sequence~\eqref{fibcoeff}.  The one blemish is that the second term is $1
+ b_0$ instead of simply 1.  But since $b_0 = 0$, the second term is ok.

So we have
\begin{align*}
B(x) & = x + x B(x) + x^2 B(x)\\
B(x) & = \frac{x}{1 - x - x^2}.
\end{align*}


\end{document}
