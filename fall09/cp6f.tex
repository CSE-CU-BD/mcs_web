%cp6m.tex (virtual Monday)

\documentclass[handout]{mcs}

\begin{document}

\inclassproblems{6, Fri.} %Virtual Monday

%%%%%%%%%%%%%%%%%%%%%%%%%%%%%%%%%%%%%%%%%%%%%%%%%%%%%%%%%%%%%%%%%%%%%
% Problems start here
%%%%%%%%%%%%%%%%%%%%%%%%%%%%%%%%%%%%%%%%%%%%%%%%%%%%%%%%%%%%%%%%%%%%%

\pinput{CP_23_high_priority_servers}
\pinput{CP_Mark_a_spanning_tree}b
\pinput{CP_Kn_is_very_connected}
\pinput{CP_tree_characterizations}
\pinput{CP_spanning_tree_proc}
\pinput{CP_n_dim_hypercube}


\appendix

\section{Product \& Lexicographic Partial Orders}

Let $R$ be a binary relation on a set, $A$, and $S$ be a binary
relation on a set, $B$.

%\begin{definition}\label{productrel}

The \term{product relation}, $R \cross S$, of $R$ and $S$ is the binary
relation, $T$, on $A \times B$ defined by the rule:
\[
(a,b) \mrel{T} (a',b') \qiff [a \mrel{R} a'\ \QAND\ b \mrel{S} b'],
\]
When $R$ and $S$ are partial orders, $T$ is also a partial order and is
called the \term{coordinatewise partial order} of $R$ and $S$.

The \term{lexicographic product} determined by $R$ and
$S$ is the relation, $L$, on $A \times B$ defined by the rule:
\[
(a,b) L (a',b') \qiff
    a \mrel{R} a'\ \QOR\ [a = a'\ \QAND\ b \mrel{S} b'].
\]


\section{Simple Graphs}

A \term{simple graph}, $G$, consists of a nonempty set, $V$, called the
\term{vertices} of $G$, and a collection, $E$, of two-element subsets of
$V$.  The members of $E$ are called the \term{edges} of $G$.

For vertices $u \neq v$, the edge whose elements are $u$ and $v$ is denoted
$\edge{u}{v}$; this edge is said to \term{be between} or \term{join} $u$
and $v$ and be \term{incident} to each of $u$ and $v$.

Vertices $u \neq v$ are \term{adjacent} iff $\edge{u}{v}$ is an edge of
$G$.

The number of edges incident to a vertex is called the \term{degree} of
the vertex; equivalently, the degree of a vertex is equals the number of
vertices adjacent to it.

Two graphs are \term{isomorphic} iff there is an edge-preserving bijection
between their vertices.  More formally, if for $i=1,2$, the graph $G_i$ has vertices,
$V_i$, and edges, $E_i$, then $G_1$ is
\term{isomorphic} to $G_2$ iff there exists a \textbf{bijection}, $f: V_1
\to V_2$, such that for every pair of vertices $u, v \in V_1$:
\[
\edge{u}{v} \in E_1 \qiff \edge{f(u)}{f(v)} \in E_2.
\]
The function $f$ is called an \term{isomorphism} between $G_1$ and $G_2$.

A \term{path} in a graph, $G$, is a sequence of $k \geq 0$ vertices
\[
v_0,\dots,v_k
\]
such that $\edge{v_i}{v_{i+1}}$ is an edge of $G$ for $0 \leq i < k$ .  The
path is said to \term{start} at $v_0$, and \term{end} at $v_k$,
and the \term{length} of the path is defined to be $k$.  An edge, $e$, is
\term{traversed $n$ times} by the path if there are $n$ different values of
$i$ such that edge $\edge{v_i}{v_{i+1}}$ is $e$.

The path is \term{simple} iff all the $v_i$'s are different, that is, $v_i
= v_j$ only if $i=j$.

Two vertices in a graph are \term{connected} iff there is a path that
begins with one of the vertices and ends with the other.  A graph is
\term{connected} iff every pair of vertices is connected.

\iffalse The \term{shortest} path between two vertices is always simple.\fi

\term{Cycles} are described by paths that begin and end with the same
vertex.  A \term{simple} cycle is a cycle that doesn't cross or backtrack
on itself.  More precisely, a simple cycle is a cycle that can be
described by a path of length at least three whose vertices are all
different except for the beginning and end vertices.

An \term{acyclic graph} is one without simple cycles.

A \term{connected component} of a graph is the set of all the vertices
connected to some single vertex.  So a graph is connected iff it has only
one connected component.

A graph is \term{$k$-edge-connected} iff only one connected component
remains when fewer than $k$ edges are deleted.  We use ``$k$-connected''
as an abbreviation for $k$-edge-connected.  (Note that in graph theory
texts ``$k$-connected'' usually means $k$-\emph{vertex}-connected.)

\section{Trees}

A \term{tree} is a connected acyclic graph.  The following are equivalent
for any graph, $T$:
\begin{itemize}

\item $T$ is a tree.

\item For any vertices $u,v$ of $T$, there is a \emph{unique} simple path
between $u$and $v$.

\item $T$ is connected and has one fewer edges than vertices.

\item Adding a new edge between two vertices of $T$ creates a simple
  cycle.

\item $T$ is connected and every edge is a cut edge.

\end{itemize}



%%%%%%%%%%%%%%%%%%%%%%%%%%%%%%%%%55
% Problems end here
%%%%%%%%%%%%%%%%%%%%%%%%%%%%%%%%%%%%%%%%%%%%%%%%%%%%%%%%%%%%%%%%%%%%%

\end{document}

\endinput
