\documentclass[handout]{mcs}

\begin{document}

\inclassproblems{6, Wed.}

%%%%%%%%%%%%%%%%%%%%%%%%%%%%%%%%%%%%%%%%%%%%%%%%%%%%%%%%%%%%%%%%%%%%%
% Problems start here
%%%%%%%%%%%%%%%%%%%%%%%%%%%%%%%%%%%%%%%%%%%%%%%%%%%%%%%%%%%%%%%%%%%%%
% (Simple graphs, degrees, isomorphism)

\pinput{CP_isomorphic_graphs}
\pinput{CP_pos_deg_but_not_connected.tex}
\pinput{CP_Handshaking_Theorem}
\pinput{PS_neighbors_under_isomorphism}
%\pinput{PS_graph_two_ends}



\inhandout{
\section*{Definitions}

A \term{simple graph}, $G$, consists of a nonempty set, $V$, called the
\term{vertices} of $G$, and a collection, $E$, of two-element subsets of
$V$.  The members of $E$ are called the \term{edges} of $G$.

For vertices $u \neq v$, the edge whose elements are $u$ and $v$ is denoted
$\edge{u}{v}$; this edge is said to \emph{be between} or \term{join} $u$
and $v$ and be \term{incident} to each of $u$ and $v$.

Vertices $u \neq v$ are \term{adjacent} iff $\edge{u}{v}$ is an edge of
$G$.

The number of edges incident to a vertex is called the \term{degree} of
the vertex; equivalently, the degree of a vertex is equals the number of
vertices adjacent to it.

Two graphs are \term{isomorphic} iff there is an edge-preserving bijection
between their vertices.  More formally, if for $i=1,2$, the graph $G_i$ has vertices,
$V_i$, and edges, $E_i$, then $G_1$ is
\term{isomorphic} to $G_2$ iff there exists a \textbf{bijection}, $f: V_1
\to V_2$, such that for every pair of vertices $u, v \in V_1$:
\[
\edge{u}{v} \in E_1 \qiff \edge{f(u)}{f(v)} \in E_2.
\]
The function $f$ is called an \term{isomorphism} between $G_1$ and $G_2$.

A \term{path} in a graph, $G$, is a sequence of $k \geq 0$ vertices
\[
v_0,\dots,v_k
\]
such that $\edge{v_i}{v_{i+1}}$ is an edge of $G$ for $0 \leq i < k$ .  The
path is said to \emph{start} at $v_0$, and \emph{end} at $v_k$.  \iffalse ,
and \emph{length} of the path is defined to be $k$.  An edge, $e$, is
\term{traversed $n$ times} by the path if there are $n$ different values of
$i$ such that edge $\edge{v_i}{v_{i+1}}$ is $e$.

The path is \emph{simple} iff all the $v_i$'s are different, that is, $v_i
= v_j$ only if $i=j$.
\fi

Two vertices in a graph are \term{connected} iff there is a path that
begins with one of the vertices and ends with the other.  A graph is
\term{connected} iff every pair of vertices is connected.

\iffalse The \emph{shortest} path between two vertices is always simple.

\emph{Cycles} are paths that begin and end with the same vertex.
\emph{Simple} cycles are cycles that don't cross themselves.\footnote{The
technical definition of simple cycle appears in Week 5 Notes.}

A \emph{connected component} of a graph is the set of all the vertices
connected to some single vertex.  So a graph is connected iff it has only
one connected component.

A \emph{cycle} is a path of length at least 3 that begins and ends with
the same vertex.  A \emph{simple cycle} is a cycle without repeated
vertices except for the beginning and end vertices.
\fi
}


% Problems end here
%%%%%%%%%%%%%%%%%%%%%%%%%%%%%%%%%%%%%%%%%%%%%%%%%%%%%%%%%%%%%%%%%%%%%

\end{document}

\endinput
