\documentclass[handout]{mcs}

\begin{document}

\inclassproblems{9, Mon.}

%%%%%%%%%%%%%%%%%%%%%%%%%%%%%%%%%%%%%%%%%%%%%%%%%%%%%%%%%%%%%%%%%%%%%
% Problems start here
%%%%%%%%%%%%%%%%%%%%%%%%%%%%%%%%%%%%%%%%%%%%%%%%%%%%%%%%%%%%%%%%%%%%%

\pinput{CP_RSA_between_tables}
\pinput{CP_RSA_proving_correctness}


\instatements{\newpage}

\section*{Appendix}

\subsection*{Inverses, Fermat, Euler}

\begin{lemma*}[Inverses mod $n$]
If $k$ and $n$ are relatively prime, then there is integer
$k'$ called the \emph{modulo $n$ inverse} of $k$, such that
\[
k \cdot k' \equiv 1 \pmod n.
\]
\end{lemma*}

\textbf{Remark:} If $\gcd(k,n) =1$, then $sk+tn = 1$ for some $s,t$, so we
can choose $k' \eqdef s$ in the previous Lemma.  So given $k$ and $n$, an
inverse $k'$ can be found efficiently using the Pulverizer.

\begin{theorem*}[Fermat's (Little) Theorem]
If $p$ is prime and $k$ is not a multiple of $p$, then
\[
k^{p-1} \equiv 1 \pmod{p}
\]
\end{theorem*}

\begin{definition*}
The value of \emph{Euler's totient function}, $\phi(n)$, is defined to be
the number of positive integers less than $n$ that are relatively prime to
$n$.
\end{definition*}

\begin{lemma*}[Euler Totient Function Equations]
\begin{align*}
\phi(p^k)& = p^k - p^{k-1} & \text{for prime, $p$, and $k>0$},\\
\phi(mn) & = \phi(m)\cdot \phi(n) &\text{ when } \gcd(m,n)=1.
\end{align*}
\end{lemma*}

\begin{theorem*}[Euler's Theorem]
If $k$ and $n$ are relatively prime, then
\[
k^{\phi(n)} \equiv 1 \pmod{n}
\]
\end{theorem*}

\begin{corollary*}
If $k$ and $n$ are relatively prime, then $k^{\phi(n)-1}$ is an inverse
modulo $n$ of $k$.
\end{corollary*}

\textbf{Remark:} Using fast exponentiation to compute $k^{\phi(n)-1}$ is
another efficient way to compute an inverse modulo $n$ of $k$.

\subsection*{The Pulverizer}

Euclid's algorithm for finding the GCD of two numbers relies on
repeated application of the equation: 
\[
\gcd(a, b) = \gcd(b, \rem{a}{b})
\]
For example, we can compute the GCD of 259 and 70 as follows:
\[
\begin{array}{rclcl}
\gcd(259, 70)
    & = & \gcd(70, 49) & \quad & \text{since $\rem{259}{70} = 49$}\\
    & = & \gcd(49, 21) && \text{since $\rem{70}{49} = 21$} \\
    & = & \gcd(21, 7) && \text{since $\rem{49}{21} = 7$} \\
    & = & \gcd(7, 0) && \text{since $\rem{21}{7} = 0$} \\
    & = & 7.
\end{array}
\]
The Pulverizer goes through the same steps, but requires some extra
bookkeeping along the way: as we compute $\gcd(a, b)$, we keep track
of how to write each of the remainders (49, 21, and 7, in the example)
as a linear combination of $a$ and $b$ (this is worthwhile, because
our objective is to write the last nonzero remainder, which is the
GCD, as such a linear combination).  For our example, here is this
extra bookkeeping:
\[
\begin{array}{ccccrcl}
x & \quad & y & \quad & \rem{x}{y} & = & x - q \cdot y \\ \hline
259 && 70 && 49 & = &   259 - 3 \cdot 70 \\
70 && 49 && 21  & = &   70 - 1 \cdot 49 \\
&&&&            & = &   70 - 1 \cdot (259 - 3 \cdot 70) \\
&&&&            & = &   -1 \cdot 259 + 4 \cdot 70 \\
49 && 21 && 7   & = &   49 - 2 \cdot 21 \\
&&&&            & = &   (259 - 3 \cdot 70) -
                                2 \cdot (-1 \cdot 259 + 4 \cdot 70) \\
&&&&            & = &   \fbox{$3 \cdot 259 - 11 \cdot 70$} \\
21 && 7 && 0
\end{array}
\]
We began by initializing two variables, $x = a$ and $y = b$.  In the
first two columns above, we carried out Euclid's algorithm.  At each
step, we computed $\rem{x}{y}$, which can be written in the form $x - q
\cdot y$.  (Remember that the Division Algorithm says $x = q \cdot y +
r$, where $r$ is the remainder.  We get $r = x - q \cdot y$ by
rearranging terms.)  Then we replaced $x$ and $y$ in this equation
with equivalent linear combinations of $a$ and $b$, which we already
had computed.  After simplifying, we were left with a linear
combination of $a$ and $b$ that was equal to the remainder as desired.
The final solution is boxed.




%%%%%%%%%%%%%%%%%%%%%%%%%%%%%%%%%%%%%%%%%%%%%%%%%%%%%%%%%%%%%%%%%%%%%
% Problems end here
%%%%%%%%%%%%%%%%%%%%%%%%%%%%%%%%%%%%%%%%%%%%%%%%%%%%%%%%%%%%%%%%%%%%%
\end{document}