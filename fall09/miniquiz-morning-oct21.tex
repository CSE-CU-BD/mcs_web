\documentclass[quiz]{mcs}

\begin{document}

\begin{center}
{\Large Morning}
\end{center}

\miniquiz{Oct. 21}


%%%%%%%%%%%%%%%%%%%%%%%%%%%%%%%%%%%%%%%%%%%%%%%%%%%%%%%%%%%%%%%%%%%%%
% Problems start here
%%%%%%%%%%%%%%%%%%%%%%%%%%%%%%%%%%%%%%%%%%%%%%%%%%%%%%%%%%%%%%%%%%%%%
\instatements{\newpage}
\pinput[points = 7]{MQ_stable_matching_unique_morning}

\instatements{\newpage}
\pinput[points = 8]{MQ_ambiguous_recursive-def}

\instatements{\newpage}
\pinput[points = 5]{CP_isomorphic_or_not_morning}

%%%%%%%%%%%%%%%%%%%%%%%%%%%%%%%%%%%%%%%%%%%%%%%%%%%%%%%%%%%%%%%%%%%%%
% Problems end here
%%%%%%%%%%%%%%%%%%%%%%%%%%%%%%%%%%%%%%%%%%%%%%%%%%%%%%%%%%%%%%%%%%%%%


%%%%%%%%%%%%%%%%%%%%%%%%%%%%%%%%%%%%%%%%%%%%%%%%%%%%%%%%%%%%%%%%%%%%%
% Appendix start here
%%%%%%%%%%%%%%%%%%%%%%%%%%%%%%%%%%%%%%%%%%%%%%%%%%%%%%%%%%%%%%%%%%%%%

\instatements{\newpage}
\section*{Appendix: Stable Matching}

\providecommand{\boys}{\text{the-Boys}}
\providecommand{\girls}{\text{the-Girls}}

A \term{Marriage Problem} consists of two disjoint sets of the same
  finite size, called \boys\ and \girls.  The members of \boys\ are called
  \emph{boys}, and members of \girls\ are called \emph{girls}.  For each
  boy, $B$, there is a strict total order, \term{$<_B$}, on \girls, and
  for each girl, $G$, there is a strict total order, \term{$<_G$}, on
  \boys.  If $G_1 <_B G_2$ we say $B$ \term*{prefers} girl $G_2$ to girl
  $G_1$.  Similarly, if $B_1 <_G B_2$ we say $G$ \term{prefers} boy $B_2$
  to boy $B_1$.

A \term{marriage assignment} or \term{perfect matching} is a bijection,
$w:\boys \to \girls$.  If $B \in \boys$, then $w(B)$ is called $B$'s
\emph{wife} in the assignment, and if $G \in \girls$, then $w^{-1}(G)$ is
called $G$'s \emph{husband}.  A \term{rogue couple} is a boy, $B$, and a
girl, $G$, such that $B$ prefers $G$ to his wife, and $G$ prefers $B$ to
her husband.  An assignment is \term{stable} if it has no rogue couples.
A \term{solution} to a marriage problem is a stable perfect matching.

The Mating Ritual marries every boy to his optimal spouse.

The Mating Ritual marries every girl to her pessimal spouse.

\section*{Appendix: Structural Induction}

\hyperdef{struct}{induction}{Structural} induction is a method for proving
some property, $P$, of all the elements of a recursively-defined data
type.  The proof consists of two steps:
\begin{itemize}
\item Prove $P$ for the base cases of the definition. 
\item Prove $P$ for the constructor cases of the definition, assuming that it
  is true for the component data items.  
\end{itemize}

\section*{Appendix: Ambiguous Recursive Definitions}

When a recursive definition of a data type
allows the same element to be constructed in more than one way, the
definition is said to be \emph{ambiguous}.

\section*{Appendix: Isomorphism}

  If $G_1$ is a graph with vertices, $V_1$, and edges, $E_1$, and likewise
  for $G_2$, then $G_1$ is \term{isomorphic} to $G_2$ iff there exists a
  \textbf{bijection}, $f: V_1 \to V_2$, such that for every pair of
  vertices $u, v \in V_1$:
\[
\edge{u}{v} \in E_1 \qiff \edge{f(u)}{f(v)} \in E_2.
\]
The function $f$ is called an \term{isomorphism} between $G_1$ and
$G_2$.

\end{document}
