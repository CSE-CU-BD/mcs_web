\documentclass[handout]{mcs}

\begin{document}

\inclassproblems{14, Mon.}

%%%%%%%%%%%%%%%%%%%%%%%%%%%%%%%%%%%%%%%%%%%%%%%%%%%%%%%%%%%%%%%%%%%%%
% Problems start here
%%%%%%%%%%%%%%%%%%%%%%%%%%%%%%%%%%%%%%%%%%%%%%%%%%%%%%%%%%%%%%%%%%%%%

\pinput{TP_markov_chebyshev_for_card_games}
\pinput{CP_cold_cows_markov}
%\pinput{CP_chebyshev_hat_check}
%\pinput{CP_chebyshev_tight}
\pinput{CP_pairwise_independent_theorem}
%\pinput{CP_infinite_variance}

%%%%%%%%%%%%%%%%%%%%%%%%%%%%%%%%%%%%%%%%%%%%%%%%%%%%%%%%%%%%%%%%%%%%%
% Problems end here
%%%%%%%%%%%%%%%%%%%%%%%%%%%%%%%%%%%%%%%%%%%%%%%%%%%%%%%%%%%%%%%%%%%%%

\instatements{
%\newpage
\section*{Appendix}

%\setcounter{secnumdepth}{0}

%%%%%%%%%%%%%%%%%%%%%%%%%% MARKOV %%%%%%%%%%%%%%%%%%%%%%%%%%%%%


\subsection*{Markov's Theorem}
If R is a nonnegative random variable, then for all $x > 0$
\[
\pr{R \geq x} \leq \frac{\expect{R}}{x}.
\]

\subsection*{Variance}

The \emph{variance}, $\variance{R}$, of a random variable, $R$, is:
\[
\variance{R} \eqdef \expect{(R - \expect{R})^2}.
\]
It is easy to show that
\[
\variance{R}= \expect{R^2} - \expectsq{R}.
\]

\textbf{[Variance of an indicator variable]}, $I$, with $\pr{I=1} = p$:
\[
\variance{I} = pq
\]
where $q \eqdef 1-p$.

\textbf{[Variance and constants]}
For constants, $a,b$,
\begin{equation}\label{a2v}
\variance{aR+b} = a^2\variance{R}.
\end{equation}

\textbf{[Variance Additivity]}
If $R_1,R_2,\dots, R_n$ are \emph{pairwise} independent variables, then
\[
\variance{R_1 + R_2 + \cdots + R_n} =  \variance{R_1} + \variance{R_2} +
\cdots + \variance{R_n}
\]

%\textbf{[Variance of an $(n,p)$-binomially distributed variable]} $=npq$.

\subsection*{Chebyshev' s Bound}
Let $R$ be a random variable, and let $x$ be a positive real number.
Then
\[
\pr{\abs{R - \expect{R}} \geq x} \leq \frac{\variance{R}}{x^2}.
\]

\subsection*{Pairwise Independent Sampling}

\begin{theorem*}
Let
\[
A_n \eqdef \frac{\sum_{i=1}^n R_i}{n}
\]
where $R_1, \dots, R_n$ are pairwise independent random variables with the
same mean, $\mu$, and deviation, $\sigma$.  Then
\begin{equation}\label{Sndev}
\pr{\abs{A_n - \mu} > x} \leq 
\paren{\frac{\sigma}{x}}^2 \cdot \frac{1}{n}.
\end{equation}
\end{theorem*}

\begin{proof}
By linearity of expectation,
\[
\expect{A_n} = \frac{\expect{\sum_{i=1}^n R_i}}{n}
   = \frac{\sum_{i=1}^n \expect{R_i}}{n}
   = \frac{n\mu}{n} = \mu.
\]

Since the $R_i$'s are pairwise independent, their variances will also add,
so
\begin{align*}
\variance{A_n}  = & \paren{\frac{1}{n}}^2 \variance{\sum_{i=1}^n R_i}
                        & \text{(by~\eqref{a2v})}\\
 = & \paren{\frac{1}{n}}^2 \sum_{i=1}^n \variance{R_i}
                        & \text{(additivity)}\\
 = & \paren{\frac{1}{n}}^2 n\sigma^2\\
 = & \frac{\sigma^2}{n}.
\end{align*}

Now letting $R$ be $A_n$ in Chebyshev's Bound
yields~\eqref{Sndev}, as required.

\end{proof}
}


\end{document}
