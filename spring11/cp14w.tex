\documentclass[handout]{mcs}

\begin{document}

\inclassproblems{14, Wed.}

%%%%%%%%%%%%%%%%%%%%%%%%%%%%%%%%%%%%%%%%%%%%%%%%%%%%%%%%%%%%%%%%%%%%%
% Problems start here
%%%%%%%%%%%%%%%%%%%%%%%%%%%%%%%%%%%%%%%%%%%%%%%%%%%%%%%%%%%%%%%%%%%%%

\pinput{CP_Fibonacci_and_bunnies}
\pinput{CP_towers_of_Sheboygan}

%%%%%%%%%%%%%%%%%%%%%%%%%%%%%%%%%%%%%%%%%%%%%%%%%%%%%%%%%%%%%%%%%%%%%
% Problems end here
%%%%%%%%%%%%%%%%%%%%%%%%%%%%%%%%%%%%%%%%%%%%%%%%%%%%%%%%%%%%%%%%%%%%%

\section*{Appendix}

Let $[x^n]F(x)$ denote the coefficient of $x^n$ in the power series
for $F(x)$.  Then,
\begin{equation}\label{1axk}
[x^n]\paren{\frac{1}{\paren{1-\alpha x}^k}} = \binom{n+k-1}{k-1}\alpha^n.
\end{equation}

\subsection*{Partial Fractions}

Here's a particular case of the Partial Fraction Rule that should be
enough to illustrate the general Rule.  Let
\[
r(x) \eqdef \frac{p(x)}{(1-\alpha x)^2 (1-\beta x) (1-\gamma x)^3}
\]
where $\alpha, \beta, \gamma$ are distinct complex numbers, and $p(x)$ is
a polynomial of degree less than the demoninator, namely, less than 6.
Then there are unique numbers $a_1,a_2,b,c_1,c_2,c_3 \in \complexes$ such
that
\[
r(x)
= \frac{a_1}{1-\alpha x} + \frac{a_2}{(1-\alpha x)^2}
+ \frac{b}{1-\beta x}
+ \frac{c_1}{1-\gamma x} + \frac{c_2}{(1-\gamma x)^2} + \frac{c_3}{(1-\gamma x)^3}
\]

\iffalse

Partial fractions together with~\eqref{1axk} imply that there is a closed
form expression for $[x^n]\paren{R(x)/S(x)}$ for arbitrary polynomials
$R(x),S(x)$.
\fi

\subsection*{Finding a Generating Function for Fibonacci Numbers}
The Fibonacci numbers are defined by:
\begin{align*}
f_0 & \eqdef 0 \\
f_1 & \eqdef 1 \\
f_n & \eqdef f_{n-1} + f_{n-2} \qquad \text{(for $n \geq 2$)}
\end{align*}

Let $F$ be the generating function for the Fibonacci numbers, that is,
\[
F(x) \eqdef f_0 + f_1 x + f_2 x^2 + f_3 x^3 + \cdots
\]
Now we have
\[
\begin{array}{rcrcrcrcrcr}
F(x)     & = & f_0 & + & f_1  x & + & f_2 x^2 & + & f_3 x^3 &  + \cdots.\\
-xF(x)   & = &     & - & f_0  x & - & f_1 x^2 & - & f_2 x^3 &  - \cdots.\\
-x^2F(x) & = &     &   &        & - & f_0 x^2 & - & f_1 x^3 &  - \cdots.\\
\hline
F(x)(1-x-x^2) 
         & = & f_0 & + & (f_1-f_0) 
                              x & + &   0 x^2 & + &   0 x^3 &  + \cdots.\\
         & = &  0  & + &   1  x.
\end{array}
\]
so
\[
F(x) = \frac{x}{1 - x - x^2}\, .
\]
\end{document}
