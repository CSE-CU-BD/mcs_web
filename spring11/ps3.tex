\documentclass[handout]{mcs}

%infinite cardinality
%induction
%state machines invariants

\begin{document}

\renewcommand{\reading}
{Chapter~\bref{mappingrule_sec}{--\bref{infinite_sec}{}},
Chapter~\bref{induction_chap}

Latest times for comments on different sections are indicated in the
online tutor problem set TP.4.}

\problemset{3}

%%%%%%%%%%%%%%%%%%%%%%%%%%%%%%%%%%%%%%%%%%%%%%%%%%%%%%%%%%%%%%%%%%%%%
% Problems start here
%%%%%%%%%%%%%%%%%%%%%%%%%%%%%%%%%%%%%%%%%%%%%%%%%%%%%%%%%%%%%%%%%%%%%

\pinput{CP_power_set_tower}
\pinput{PS_unit_interval}

\pinput{PS_team_division} %: good basic induction}

\pinput{PS_fib_induction}

\pinput{CP_bogus_unique_prime_factors}

\pinput{PS_ripple_carry_adder_correctness}

\pinput{PS_sums_and_products_of_integers} %nice since induction is not on $n$

%Cantor set

%induction, state machine invariants

%%%%%%%%%%%%%%%%%%%%%%%%%%%%%%%%%%%%%%%%%%%%%%%%%%%%%%%%%%%%%%%%%%%%%
% Problems end here
%%%%%%%%%%%%%%%%%%%%%%%%%%%%%%%%%%%%%%%%%%%%%%%%%%%%%%%%%%%%%%%%%%%%%

\end{document}
