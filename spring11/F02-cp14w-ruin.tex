\inclassproblems{14, Wed}

%F02 cp14w
%F01 pset10

\begin{problem}
A gambler has been studying the roulette wheel in a Las Vegas casino over
a long period of time, and his data confirms that the wheel is biased to
come up on a red number about 51\% of the time.  Since a bet on red pays
even money (bet \$1 and either lose it, or win and get back \$2), he
realizes the odds are in his favor.

So with high hopes and an initial stake of \$10,000, the gambler aims to
turn his stake into \$1,000,000.  He plans simply to continue making \$100
bets on red until either he is ahead the \$1,000,000, or he goes bankrupt.
He considers bankruptcy very unlikely because the odds are in his favor,
and the bets are small enough relative to his initial stake that he can
withstand a long run of losses.

\bparts

\ppart 
Based on this description, describe a reasonable probability space to
model this situation.

\solution{
An easy way to describe an appropriate sample space is to keep track of
the gambler's stake.  So sample points will be all the finite sequences of
nonnegative multiples of 100, that
\begin{itemize}
\item start at 10,000,
\item end at the first occurrence of either 0 or 1,000,000,
\item have successive integers in the sequence differ by exactly 100.
\end{itemize}
For example, the sample point describing two initial wins, followed by 102
losses and bankruptcy would be the length 105 sequence
\[
(10000, 10100,10200, 10100,10000,9900,9800,\dots,200,100,0).
\]
The sample point corresponding to his winning every bet until he reaches
his goal would the length 9,900 sequence
\[
(10000, 10100,10200, 10300,10400,\dots, 999800, 999900, 1000000).
\]
The probability of an outcome is defined to be $(0.51)^k(0.49)^j$ where
$k$ is the number of increases by 100, and $j$ is the number of decreases
by 100.  Note that either $j-k = 100$ for sample points that lead to
bankruptcy or $k-j = 9,900$ for those leading to \$1,000,000.}

\ppart Describe a positive constant, $\epsilon$, such that, at any point
during his play when the gambler has not gone bankrupt, there is a
probability of at least $\epsilon$ that he will reach his goal of
\$1,000,000 within the next 9,999 bets.

\solution{ $\epsilon \eqdef (0.51)^{9999}$ will do, since even if the
gambler's stake was down to \$100, there is still this much probability
that he will win the next 9,999 bets in a row and reach his goal of
\$1,000,000.}

\ppart Conclude that the probability that the gambler bets $n\geq 9,999$
or more times before the game ends is at most $r^{n-9999}$ for some
constant $r < 1$.

\solution{Let $r \eqdef (1-\epsilon)^{1/9999}$.  The reason this works is
that the probability of not reaching the goal of \$1,000,000 in the first
9999 bets is at most $1- \epsilon$, because there is at least an
$\epsilon$ chance the Gambler wins by the 9,999th bet.  Likewise, the
probability of not reaching the goal of \$1,000,000 in the second 9999
bets---given that he didn't in the first 9999---is also $\leq 1-
\epsilon$.  So the probability of not reaching the goal in the first
$2\cdot 9999$ bets is $\leq (1- \epsilon)^2$.  Repeating this argument for
the first $\floor{n/9999}$ blocks of 9999 bets, we conclude that the
probability of not reaching the goal of \$1,000,000 by the $n$th bet is
$\leq$ the probability of not reaching it by the $\floor{n/9999} \cdot
9999 \leq n\text{th}$ bet, which is $\leq (1- \epsilon)^{ \floor{n/9999}} = r^{
\floor{n/9999} \cdot 9999} \leq r^{n-9999}$.}

\iffalse

\ppart Conclude that the probability that he bets forever---without either
going bankrupt or reaching his \$1,000,000 goal---is zero.

\solution{By part~(b) the probability that the gambler does \emph{not}
reach his \$1,000,000 goal during his first 9,999 bets is at most
$1-\epsilon$.  The conditional probability that he reaches \$1,000,000
during his second sequence of 9,999 bets, given that he hasn't already
reached his goal or gone bankrupt, is at most $(1-\epsilon)^2$.  The
conditional probability that he reaches \$1,000,000 during his $n$th
sequence of 9,999 bets, given that he hasn't already reached his goal or
gone bankrupt, is at most $(1-\epsilon)^n$.

This implies that the probability that he bets more than $9999n$ times is
at most $(1-\epsilon)^n$, and so the probability that he bets forever is
at most $(1-\epsilon)^n$ for every $n$.  Since $(1-\epsilon)^n$ approaches
0 as $n \rightarrow \infty$.  The probability that the gambler bets
forever is at most 0.}
\fi

\eparts
\end{problem}

\textbf{NOTE:} We didn't get to any of the following three problems in
class.

\begin{problem}
The ``high hopes'' of the Gambler in Problem 1 overlook some difficulties
with his strategy.  Is he really very likely to win?  How long will it take?

\solution{He has a 98\% chance of winning: more than $1-
\paren{0.49/0.51}^{100}$ using the formula~(\ref{intended-profit})
bounding the probability of getting an intended profit in an unfair game.
To apply the formula, we reverse the roles of $T$ and $n$ to turn the
favorable game into an unfair one.  This turns into a game with
probability 0.49 of winning a bet, starting with a stake of 9900 betting
units of \$100 and aiming to win 100 units.  So the probability of a win
in this game is a most $\paren{0.49/0.51}^{100} \approx 0.018$; winning
this reversed game corresponds to going bankrupt in the original game, so
his probability of winning the real game is $\geq 1 - 0.18$.

It follows from the formula for expected number of bets that he can expect
to make about a half million bets before winning, however.
}

\end{problem}

\begin{problem}
Let $G$ be the amount won by the gambler when a Gambler's Ruin game ends.
Let $Q$ be the number of bets till the game ends.

The derivation of the formula for the expected number of bets,
$\expect{Q}$, in an unfair Gambler's Ruin game used the fact that
\begin{equation}\label{gw}
\expect{Q}\cdot\expect{\text{amount won win per bet}} = \expect{G}.
\end{equation}
Prove this equation. \hint Since the amount won per bet may be negative,
Wald's Theorem does not immediately apply.

\solution{Directly from Notes 13-14:

Let $G_{i}$ be the amount the gambler gains on the $i$th flip:
$G_{i}=1$ if the gambler wins the flip, $G_{i}=-1$ if the gambler loses
the flip, and $G_{i}=0$ if the game has ended before the $i$th flip.  So
the amount won by the gambler when the game ends is
\[
G = \sum_{i=1}^Q G_i.
\]

Now the random variable $G_i+1$ is nonnegative, and $\expcond{G_i+1}{Q\geq
i} = \expcond{G_i}{Q \geq i}+1 = \expect{\text{amount won win per bet}}+1$, so by
Wald's Theorem
\begin{equation}\label{G1}
\expect{\sum_{i=1}^Q (G_i+1)} = \expect{\text{amount won win per bet}}+1
\cdot \expect{Q}.
\end{equation}
But
\begin{eqnarray}
\expect{\sum_{i=1}^Q (G_i+1)} & = & \expect{\sum_{i=1}^Q G_i + \sum_{i=1}^Q 1}\notag\\
   & = & \expect{(\sum_{i=1}^Q G_i) + Q}\notag\\
   & = & \expect{\sum_{i=1}^Q G_i} + \expect{Q}\notag\\
   & = & \expect{G} + \expect{Q}\label{GQ}.
\end{eqnarray}
Now combining~(\ref{G1}) and~(\ref{GQ}) confirms the truth of~(\ref{gw}).}
\end{problem}

\begin{problem}
Prove that in an \emph{unbounded} fair game, where the Gambler plays until
he is broke no matter how much his stake increases in the meantime, the
Gambler is \emph{sure} to go broke, but the expected number of bets before
he goes broke is infinite.

\solution{
Directly from Notes 13-14:

\begin{lemma}\label{go broke}
If the gambler starts with one or more dollars and plays a fair game until
he is broke, then he will go broke with probability 1.
\end{lemma}

\begin{proof}
If the gambler has initial capital $n$ and goes broke in a game without
reaching a goal $T$, then he would also go broke if he were playing and
ignored the goal.  So the probability that he will lose if he keeps
playing without stopping at any goal $T$ must be at least as large as the
probability that he loses when he has a goal $T>n$.

But we know that in a fair game, the probability that he loses is $1 -
n/T$.  This number can be made arbitrarily close to 1 by choosing a
sufficiently large value of $T$.  Hence, the probability of his losing
while playing without any goal has a lower bound arbitrarily close to 1,
which means it must in fact be 1.
\end{proof}

So even if the gambler starts with a million dollars and plays a perfectly
fair game, he will eventually lose it all with probability 1.

\begin{lemma}\label{play forever}
If the gambler starts with one or more dollars and plays a fair game until
he goes broke, then his expected number of plays is infinite.
\end{lemma}

\begin{proof}
Consider the gambler's ruin game where the gambler starts with initial
capital $n$, and let $u_n$ be the expected number of bets for
the \emph{unbounded} game to end.  Also, choose any $T \geq n$, and as
above, let $e_n$ be the expected number of bets for the game to end when
the gambler's goal is $T$.

The unbounded game will have a larger expected number of bets compared to
the bounded game because, in addition to the possibility that the gambler
goes broke, in the bounded game there is also the possibility that the
game will end when the gambler reaches his goal, $T$.  That is,
\[
u_n \geq e_n.
\]
So by~(\ref{Tn}), 
\[
u_n \geq n(T-n).
\]
But $n \geq 1$, and $T$ can be any number greater than or equal to $n$, so
this lower bound on $u_n$ can be arbitrarily large.  This implies that
$u_n$ must be infinite.

Now by Lemma~\ref{go broke}, with probability 1, the unbounded game ends
when the gambler goes broke.  So the expected time for the unbounded game
to \emph{end} is the \emph{same} as the expected time for the gambler to
\emph{go broke}.  Therefore, the expected time to go broke is infinite.
\end{proof}

In particular, even if the gambler starts with just one dollar, his
expected number of plays before going broke is infinite!  Of course, this
does not mean that it is likely he will play for long.  For example, there
is a 50\% chance he will lose the very first bet and go broke right away.

Lemma~\ref{play forever} says that the gambler can ``expect'' to play
forever, while Lemma~\ref{go broke} says that with probability 1 he will
go broke.  These Lemmas sound contradictory, but our analysis showed that
they are not.
}
\end{problem}


\appendix
\section{Appendix}

\begin{theorem*}
In the Gambler's Ruin game with probability $p$ of winning
each individual bet, with initial capital, $n$, and goal, $T$,
\begin{align}
\pr{\text{the gambler is a winner in the fair game}} = & \frac{n}{T},
        \label{fairwin}\\
\pr{\text{the gambler is a winner a biased game}} = &
        \frac{(q/p)^n-1}{(q/p)^T -1}. \label{biaswin}
\end{align}
\begin{equation}\label{intended-profit}
\pr{\text{the gambler is a winner in an unfair game}} \leq (p/q)^{T-n}.
\end{equation}
Let $Q$ be the number of bets till the game ends.
\[
\expect{\text{$Q$ in an unfair game}} =
\frac{\pr{\text{gambler is a winner}}T-n}{2p-1}.
\]

\begin{equation}
\label{Tn}
\expect{\text{$Q$ in a fair game}} = n(T-n).
\end{equation}

\end{theorem*}

\end{document}



