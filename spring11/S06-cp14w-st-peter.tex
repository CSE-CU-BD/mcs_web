\documentclass[11pt]{article}   
\usepackage{latex-macros/course}

\begin{document}
\inclassproblems{14, Wed.}


\begin{problem} (Carried over from Friday, May 12.)

A gambler bets \$10 on ``red'' at a roulette table (the odds of red are
18/38 which slightly less than even) to win \$10.  If he wins, he gets
back twice the amount of his bet and he quits.  Otherwise, he doubles his
previous bet and continues.
\bparts

\ppart What is the expected number of bets the gambler makes before he
wins?

\solution{
mean time to fail $= 38/18 = 2\ \tfrac{1}{19}$.
}

\ppart What is his probability of winning?

\solution{
He is certain to win, since $\pr{> k \text{ bets}} = (20/38)^{k}$ which
goes to zero as $k$ goes to infinity.  More fully,
\[
\pr{\text{win}} \geq \pr{\text{win in $\leq k$ bets}}
= 1 - \pr{> k \text{ bets}}
\]
and this last expression goes to 1 as $k$ goes to infinity.}

\ppart What is his expected final profit (amount won minus amount lost)?

\solution{His final profit is always \$10 whenever he finally wins, and he
is certain to win, so \$10 is also his expected final profit.}

\ppart How can this be so despite the fact that the game is biased against
him?

\hint What is the expected size of his last bet?

\solution{
It sounds plausible that, since his expected number of bets is less than
three, the expected size of his bet would be less than the size of his
third bet, that is, $10\cdot 2^2 = 40$ dollars.  But this is a sloppy
---and wrong ---argument.  What it overlooks is that later bets, though
progressively less likely, grow tremendously, and so contribute heavily to
the expected bet size.

To get the answer, we go back to the definition of expected bets.  Let $B$
be the size of his last bet in dollars.  Now if he wins his $\$10$ final
profit on the $k$th bet, then $B=10\cdot 2^{k-1}$, so
\[
\pr{B=10\cdot 2^{k-1}} = (20/38)^{k-1}(18/38)
\]
So
\begin{align*}
\expect{B} & = \sum_{k \in \naturals^{+}} 10\cdot
      2^{k-1}\paren{(20/38)^{k-1}(18/38)}\\
  & = 10(18/38) \sum_{k \in \naturals^{+}}  2^{k-1}\paren{(20/38)^{k-1}}\\
  & =(90/19)\sum_{k \in \naturals^{+}} \paren{(40/38)^{k-1}}\\
  & > 5\sum_{k \in \naturals^{+}} 1 = \infty,
\end{align*}
so the gambler has to have an infinite bank account to win his certain
\$10!\dots and of course, his per cent profit on the money risked is
zero.}

\ppart Suppose the gambler has a large but finite amount, $L$, of dollars.
So he plays the doubling strategy until he wins \$10 or runs out of money.
Now what is his expected final profit?

\solution{
The gambler loses his whole stake if he loses $k$ individual spins of the
roulette wheel, where $k$ is the smallest integer such that the total
amount bet through the $k$th spin is at least $L$.  That is,
\begin{align*}
L & \leq  \sum_{i=1}^k 10\cdot2^{i-1}\\
 & = \sum_{i=0}^{k-1} 10\cdot2^{i}\\
 & = 10\frac{2^k - 1}{2-1} \leq 10\cdot 2^k
\end{align*}
Assuming for convenience that $L/10$ is a power of 2, then
\[
k = \log_2 (L/10).
\]

Now the probability of the gambler losing $k$ spins is $(10/19)^k$, so his
expected win is
\begin{align*}
10\cdot \paren{1-\paren{\frac{10}{19}}^k} - L\cdot\paren{\frac{10}{19}}^k
   & \leq 10 - L \paren{\frac{10}{19}}^k\\
   & =  10 - \frac{L}{\paren{19/10}^k}\\
   & =  10 - \frac{L}{\paren{2^{\log_2 (19/10)}}^k}\\
   & =  10 - \frac{L}{(L/10)^{\log_2 (19/10)}}\\
   & \leq  10 - \frac{L}{(L/10)^{0.926}}\\
   & = 10 - 10^{0.926} \cdot L^{0.074}
\end{align*}
which goes to $-\infty$ as $L$ grows.}

\ppart Suppose the casino inflates the value of the gambler's bets by
squaring.  That is, instead of having to bet $10\cdot 2^{k-1}$ dollars on
his $k$th bet, he need only bet $10\cdot \sqrt{2^{k-1}}$ to win an amount
equal to the sum of all his prior bets plus \$10.  Show that the expected
size of his last bet is now finite.

\solution{
Now
\[
\pr{B=10\cdot \sqrt{2^{k-1}}} = (20/38)^{k-1}(18/38)
\]
so
\begin{align*}
\expect{B} & = \sum_{k \in \naturals^{+}}10\cdot \sqrt{
      2^{k-1}}\paren{(20/38)^{k-1}(18/38)}\\
  & = 10(18/38) \sum_{k \in \naturals^{+}}  \sqrt{2}^{k-1}(20/38)^{k-1}\\
  & = (90/19)\sum_{k \in \naturals}(\sqrt{2}10/19)^{k}\\
  & < 5\sum_{k \in \naturals} (3/4)^{k} = 20.
\end{align*}

This contrived example illustrates a property of expectation which
shouldn't be overlooked: it is perfectly possible to have a random
variable, $B$, with infinite expectation, while $\sqrt{B}$ has finite
expectation!}

\eparts

\end{problem}

\end{document}
