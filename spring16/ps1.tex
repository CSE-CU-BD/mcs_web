\documentclass[handout]{mcs}

\begin{document}

\renewcommand{\reading}{
\begin{itemize}
\item Chapter~\bref{proofs_chap}{.\ \emph{What is a Proof?}},

\item  Chapter~\bref{well_ordering_chap}{.\ \emph{The Well Ordering
      Principle}} through~\bref{factor_sec}{;}
  (omit~\bref{well_ordering_sec}{.\ \emph{Well Ordered Sets}}),

\item  Chapter~\bref{logicform_chap}{.\ \emph{Logical Formulas}}
  through~\bref{SAT_sec}{.\ \emph{The SAT Problem}}.

\end{itemize}
These assigned readings do \textbf{not}
  include the Problem sections.  (Many of the problems in the text
  will appear as class or homework problems.)}

\problemset{1}

\medskip

\textbf{Reminder}:

\begin{itemize}

\item Problems should be
  \href{https://stellar.mit.edu/S/course/6/sp16/6.042/homework/index.html}
       {submitted electronically}, and each problem should begin with a
\href{http://courses.csail.mit.edu/6.042/spring16/pset_instructions.shtml#collab-state}
{\emph{collaboration statement}}.

\item The class has a
  \href{http://piazza.com/mit/spring2016/6042j18062j/home} {Piazza
    forum}.  With Piazza you may post questions---both administrative
  and content related---to the entire class or to just the staff.  You
  are likely to get faster response through Piazza than from direct
  email to staff.

  You should post a question or comment to Piazza at least once by the
  end of the second week of the class; after that Piazza use is
  optional.
\end{itemize}

\begin{staffnotes}
Lectures covered: Intro, Proof by Contradiction, WOP, Propositional Formulas
\end{staffnotes}


%%%%%%%%%%%%%%%%%%%%%%%%%%%%%%%%%%%%%%%%%%%%%%%%%%%%%%%%%%%%%%%%%%%%%
% Problems start here
%%%%%%%%%%%%%%%%%%%%%%%%%%%%%%%%%%%%%%%%%%%%%%%%%%%%%%%%%%%%%%%%%%%%%

%\pinput{CP_AMM_root_2_proof}

\pinput{CP_roots_of_polynomials}
 
    %%F15 & S15:

    %\pinput{MQ_log4_of_6_
    %\pinput{PS_3_exponent_inequality_wop}
    %\pinput{CP_valid_vs_satisfiable}
     %\pinput{PS_faster_adder_logic}

 %\pinput{CP_generalize_root_2_proof}                 on cp1f
 %\pinput{CP_irrational_raised_to_an_irrational}      on cp1f

 \pinput{MQ_prove_by_wop_odds}

 %\pinput{MQ_wop_proof_sumofodds}                     cp2m

 \pinput{PS_6_10_15_stamps_by_WOP}                      

 %\pinput{CP_binary_adder_logic}                      cp2w

 \pinput{MQ_truth_table_case_reasoning}

 \pinput{PS_find_dnf}

 %\pinput{PS_printout_binary_strings}

\end{document}
