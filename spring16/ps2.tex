\documentclass[handout]{mcs}

\begin{document}

\renewcommand{\reading}{
\begin{itemize}
 \item Chapter~\bref{prop_algebra_sec}{.\ \emph{The Algebra of Propositions}}
  through~\bref{SAT_sec},
\item Chapter~\bref{predicate_sec}{.\ \emph{Predicate Formulas}}.
\end{itemize}}

\problemset{2}

%%%%%%%%%%%%%%%%%%%%%%%%%%%%%%%%%%%%%%%%%%%%%%%%%%%%%%%%%%%%%%%%%%%%%
% Problems start here
%%%%%%%%%%%%%%%%%%%%%%%%%%%%%%%%%%%%%%%%%%%%%%%%%%%%%%%%%%%%%%%%%%%%%

\pinput{CP_sat_formulas_vs_circuits}

%\pinput{PS_equisatisfiable_3CNF}

\pinput{PS_equality_logic}

\pinput{CP_variable_convention}

\begin{staffnotes}
ADD PROBLEM explaining how CP\_variable\_convention leads to prenex
form:
* Use De Morgan to push negations down to atomic (no connectives) formulas,
* apply variable convention
* pull out all quantifier to front of formula in any order.
\end{staffnotes}

%\pinput{PS_sat_formulas_vs_circuits}  F15

%\pinput{PS_size_n_set_formula}   %S14 ps2

%\pinput{PS_set_union}            %S14 ps2

%\pinput{MQ_pair_predicate}

%\pinput{CP_set_DeMorgan}   F15
%\pinput{CP_powerset_union}  F15
%\pinput{PS_composition_to_bijection} F15
%\pinput{CP_web_search}  F15

%\pinput{TP_Quantifiers}

\end{document}
