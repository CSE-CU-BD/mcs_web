\documentclass[handout]{mcs}

\begin{document}

\inclassproblems{4, Fri.}

\begin{staffnotes}
Chapter~\bref{state_machine_chap}.\ \emph{State Machines}
through~\bref{partial_correct_sec}.\ \emph{Partial Correctness \&
  Termination}
\end{staffnotes}

%%%%%%%%%%%%%%%%%%%%%%%%%%%%%%%%%%%%%%%%%%%%%%%%%%%%%%%%%%%%%%%%%%%%%
% Problems start here
%%%%%%%%%%%%%%%%%%%%%%%%%%%%%%%%%%%%%%%%%%%%%%%%%%%%%%%%%%%%%%%%%%%%%

\begin{staffnotes}
Remind students of the Fast Exponentiation machine
(Section~\bref{fast_exp_subsec} and
\href{https://courses.csail.mit.edu/6.042/spring16/slidepdfs/state-machines.pdf}{slides}),
and point out that the binary-multiply machine of Problem 1 is
carrying out exactly the exponent updating of the Fast Exponentiation
machine.

A review problem on the Fast Exponentiation machine is given at the
end of these staff solutions for use as needed.
\end{staffnotes}


\pinput{CP_state_machine_multiply}

\pinput{CP_fifteen_puzzle}  %parity invariant

\pinput{CP_Zakim_bridge_state_machine} %uses derived variables

\textbf{Supplemental problem:}

\pinput{CP_beaver_flu} %uses weakly decreasing variable.

\begin{staffnotes}

\begin{center}
\textbf{Review of Fast Exponentiation machine}
\end{center}

\pinput{CP_fast_exponentiation}

\end{staffnotes}

%%%%%%%%%%%%%%%%%%%%%%%%%%%%%%%%%
% Problems end here
%%%%%%%%%%%%%%%%%%%%%%%%%%%%%%%%%

\end{document}

\endinput
