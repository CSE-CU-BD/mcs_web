\documentclass[handout]{mcs}

\begin{document}

\renewcommand{\reading}{
\begin{itemize}
\item Sections~\bref{sec:closed_products}.\ \emph{Products},
     \bref{asymptotic_sec}.\ \emph{Asymptotic Notation} (omit~\bref{doublesum_sec}).
\item Chapter~\bref{counting_chap}.\ \emph{Counting} through
  Section~\bref{poker_hands_sec}.\ \emph{Counting Practice}
\end{itemize}
}

\problemset{7}

%%%%%%%%%%%%%%%%%%%%%%%%%%%%%%%%%%%%%%%%%%%%%%%%%%%%%%%%%%%%%%%%%%%%%
% Problems start here
%%%%%%%%%%%%%%%%%%%%%%%%%%%%%%%%%%%%%%%%%%%%%%%%%%%%%%%%%%%%%%%%%%%%%

%%% Asymptotics problems %%% 
% I like PS_asymptotics_pairs.tex and PS_asymptotics_table.tex, but they were both used in fall15's pset. -EMS
 
\pinput{PS_Stirlings_and_log_n_factorial}
%\pinput{PS_sum_of_sixth_powers}  %ARM: covered on ps6

%\pinput{CP_theta_examples}

\pinput{PS_asymptotics_ordering}

%%% Counting with Bijections %%% -EMS

\pinput{PS_bijective_FLT_S16}

\pinput{PS_alphabet}

%%%% Counting Repetitions, Binomial Theorem
\pinput{CP_pizza_sale}

\end{document}
