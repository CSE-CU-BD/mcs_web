\documentclass[quiz]{mcs}

\renewcommand{\exampreamble}{   % !! renew \exampreamble
    \textbf{Indicate your}\ \teaminfo

  \begin{itemize}

  \item
   This exam is \textbf{closed book} except for a 2-sided cribsheet.
   Total time is 90 minutes. 

  \item
   Write your solutions in the space provided.  If you need more
   space, write on the back of the sheet containing the problem.

%   Please keep your entire answer to a problem on that problem's page.
   
   \item In answering the following questions, you may use without
     proof any of the results from class or text.
     \iffalse (unless explicitly instructed otherwise).\fi

   \item No explanation is required for short answer questions, but
     \textbf{part credit is more easily earned when an explanation is
       included}.

\iffalse
  \item
   GOOD LUCK!
\fi

  \end{itemize}}

%\renewcommand{\examspace}{}   %DRAFT

\begin{document}

\midterm{April 26}

%%%%%%%%%%%%%%%%%%%%%%%%%%%%%%%%%%%%%%%%%%%%%%%%%%%%%%%%%%%%%%%%%%%%%
% Problems start here
%%%%%%%%%%%%%%%%%%%%%%%%%%%%%%%%%%%%%%%%%%%%%%%%%%%%%%%%%%%%%%%%%%%%

%\pinput[points = 10, title = Bipartite Matching]{FP_matching}
%\examspace

%\pinput[points = 15, title = Graph Coloring]{FP_bogus_coloring_proof}  %from cp9w

%\pinput[points = 10, title = Graph Coloring]{FP_not_3_colors}  %excerpt from CP
%\examspace

%\pinput[points = 15, title = Trees]{FP_tree_degree_sequence_S16}
%\examspace

%\pinput[points = 10, title = Minumum Weight Spanning Trees]{FP_MST_min_out_S16}
%\examspace

%\pinput[points = 15, title = Weighted Graphs]{FP_maxweight_edge}
%\examspace

%\pinput[points = 15, title = Graph Coloring \& Counting]{FP_tree_kcolor}
%\examspace

\textbf{Remember, part credit is more easily earned when an explanation is
       included.}

\begin{staffnotes}
ARM used for Taylor Shaw Problem 1:\\
part(a) ``$40\cdot 41$'' gets 1 pt,\\
part(e) ``2'' get 0pts,\\
part(i) ``$\binom{41}{10}$'' gets 0pts.
\end{staffnotes}
\pinput[points = 20, title = Graphs \& Counting]{FP_simple_graphs_trees_short_answer}

\examspace

%\pinput[points = 15, title = Asymptotic Notation]{TP_exponentiate_asymptotic}
%\examspace

%\pinput[points = 15, title = Asymptotic Notation]{PS_asymptotics_table}
%\examspace

%\pinput[points = 10, title = little oh \& Stirling]{FP_stirling_little_oh}
%\examspace

\pinput[points = 15, title = Asymptotic Sum]{PS_sum_of_sixth_powers}
\examspace[5.0in]

\pinput[points = 15, title = Pigeonholing]{FP_15_pigeons}
\examspace

\begin{staffnotes}
ARM used for Taylor Shaw Problem 4\\
part(a) 7pts, part(b) 9pts, part(c) 4 pts.
\end{staffnotes}
\pinput[points = 20, title = Multinomial Coefficients \&
  Congruence]{CP_multinomial_fermat}
\examspace

\pinput[points = 10, title = Bipartite Matching]{TP_regular_matching}
\examspace[2.0in]

\pinput[points = 20, title = Card Counting]{FP_counting_poker_high_cards}

%%%%%%%%%%%%%%%%%%%%%%%%%%%%%%%%%%%%%%%%%%%%%%%%%%%%%%%%%%%%%%%%%%%%%
% Problems end here
%%%%%%%%%%%%%%%%%%%%%%%%%%%%%%%%%%%%%%%%%%%%%%%%%%%%%%%%%%%%%%%%%%%%%

\end{document}


