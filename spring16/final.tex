%final S16

\documentclass[quiz]{mcs}

%\renewcommand{\examspace}{}

\renewcommand{\exampreamble}{   % !! renew \exampreamble

\begin{center}
{\large   \textbf{Circle your}\qquad   \teaminfo}
\end{center}

  \begin{itemize}

  \item
   This exam is \textbf{closed book} except for two 2-sided cribsheets.
   Total time is 180 minutes.

  \item

   Write your solutions in the space provided.  If you need more
   space, \textbf{write on the back} of the sheet containing the
   problem.

%   Please keep your entire answer to a problem on that problem's page.   
   \item In answering the following questions, you may use without
     proof any of the results from class or text.

   \item Incorrect short answers are eligible for part credit when
     there is an explanation.

\iffalse
  \item
   GOOD LUCK!
\fi

  \end{itemize}}

\begin{document}

\final

%%%%%%%%%%%%%%%%%%%%%%%%%%%%%%%%%%%%%%%%%%%%%%%%%%%%%%%%%%%%%%%%%%%%%
% Problems start here
%%%%%%%%%%%%%%%%%%%%%%%%%%%%%%%%%%%%%%%%%%%%%%%%%%%%%%%%%%%%%%%%%%%%

\begin{staffnotes}
(a),(b),(c) 5 pts each
\end{staffnotes}
\pinput[points = 15, title= \textbf{Stable
    matching}]{FP_marriage_modify_S16}
%\examspace


\pinput[points = 10, title= \textbf{Modular Inverse}]{FP_353_factor}
\examspace


\begin{staffnotes}
(a) 6pts, (b) 9 pts; ans ``3'' on part(a) gets 4pts.
\end{staffnotes}

\pinput[points = 15, title=
  \textbf{Scheduling}]{MQ_task_parallel_scheduling_v5}
\examspace

\pinput[points = 15, title= \textbf{Probable
    Satisfiability)}]{FP_satisfy_implies_probability_alt}
%\examspace

\begin{staffnotes}
(a) 8pts, (b) 12pts
\end{staffnotes}

\pinput[points = 20, title= \textbf{Variance}]{FP_variance_dice_sum}
\examspace

\pinput[points = 20, title= \textbf{State Machines, Simple
    Graphs}]{MQ_graph_state_machine_f15}
\examspace

\pinput[points = 20, title= \textbf{Infinite
    Bijection}]{FP_infinite_sequence_injection}
\examspace


\begin{staffnotes}
(a) 8pts, 2pts each for the Not's and 1pt each for the V's;\\
(b) 5 pts, 2pts for Not, 1pt for the V's;\\
(c) 6pts, 2 each item\\
(d),(e) 3 pts each, -1pt for one mistake, -3pts for two or more mistakes
\end{staffnotes}

\pinput[points = 25, title = \textbf{Number Theory,
    Graphs}]{FP_multiple_choice_S16}
\examspace

\begin{staffnotes}
(a),(b): 4 pts each; (c),(d) 6pts each
\end{staffnotes}
\pinput[points = 20, title =
  \textbf{Counting}]{FP_string_inclusion_exclusion}
\examspace

\begin{staffnotes}
-5 for using 1/2 instead of p/q.
\end{staffnotes}
\pinput[points = 15, title = \textbf{Expectation}]
       {MQ_expectHHH}       %{MQ_expectHH_TT_S16}
\examspace

\begin{staffnotes}
(a) 3pts; (b) 4 pts; (c),(d)(e) 6pts each
\end{staffnotes}

\pinput[points = 25, title= \textbf{Chebyshev Bound}]{FP_chebyshev_pq}
\examspace

\pinput[points = 20, title= \textbf{Graphs, Logic,
    Probabililty}]{FP_graph_logic_probability}

\begin{staffnotes}
(a),(b) 15pts each
\end{staffnotes}
\pinput[points = 30, title=
  \textbf{Induction}]{FP_structural_linear_product}

%%%%%%%%%%%%%%%%%%%%%%%%%%%%%%%%%%%%%%%%%%%%%%%%%%%%%%%%%%%%%%%%%%%%%
% Problems end here
%%%%%%%%%%%%%%%%%%%%%%%%%%%%%%%%%%%%%%%%%%%%%%%%%%%%%%%%%%%%%%%%%%%%%
\end{document}



%\pinput[points = 20, title = \textbf{Independence}]{FP_flirt_dependence}
%\examspace

%% \begin{staffnotes}
%% Misha: move later; shorten.  ARM: use diff examples.    USED S15
%% \end{staffnotes}
%% \pinput[points = 5, title= \textbf{Set Formulas}]{FP_basic_set_formulas}
%% \examspace

%% \begin{staffnotes}
%% Marissa: remove part c obvs.  ARM: routine practice; nice to have more interesting
%% \end{staffnotes}
%% \pinput[points = 5, title= \textbf{Strong Induction \& GCD}]{FP_doublefib}
%% \examspace


%% ARM: Weak problem.  MAYBE REPLACE WITH RICHARD'S DYADIC RATIONALS PROBLEM
%% \pinput[points = 5, title= \textbf{Structural Induction}]{FP_structural_induction_polynomial}
%% \examspace


%% \begin{staffnotes}
%% ARM: part(b) OK, replace all others.  Marissa: (d)--not covered
%% \end{staffnotes}
%% \pinput[points = 5, title= \textbf{Number Theory}]{FP_numbers_short_answer_fall11}
%% \examspace

%% \begin{staffnotes}
%% Misha: long.  ARM: consider state machine updating graphs
%% \end{staffnotes}
%% \pinput[points = 5, title = \textbf{Invariants}]{TP_Chocolate_bars}
%% \examspace


%\pinput[points = 5, title= \textbf{Modular Arithmetic}]{FP_mods_new}
%% was CP & Fall15 final
%% \pinput[points = 5, title= \textbf{Scheduling \& DAGs}]{FP_chains_scheduling}
%% \examspace


%% uninteresting formal proof
%% \pinput[points = 5, title= \textbf{graphs, isomorphisms}]{FP_digraph_neighbors_under_isomorphisms}
%% \examspace


%\pinput[points = 5, title= \textbf{Big Oh, simple graphs}]{FP_simple_graphs_asymptotics}
%\examspace


%\pinput[points = 7, title =
%  \textbf{Markov's Bound}]{FP_hot_cows_markov}
%\examspace

%% \pinput[points = 10, title = \textbf{Tournament
%%     Probability}]{PS_probabilistic_proof}
%% \examspace


%\pinput[points = 6, title= \textbf{Uncountable
%   strings}]{FP_uncountable_ones}
%\examspace


%\pinput[points = 6, title= \textbf{Uncountable
%   strings}]{FP_uncountable_ones}
%\examspace


%\pinput[points = 8, title= \textbf{Expectation}]{FP_random_graphsS15}
%\examspace
   %ADAM, ARM

%% \pinput[points = 4, title = \textbf{Random Walk, Stationary Distribution}]
%%        {FP_random_walk_multiple}
%% \examspace

%% \pinput[points = 12, title = \textbf{Expectation}]
%%        {TP_repeathalf}
%% \examspace


%\pinput[points = 25, title = \textbf{Combinatorics}]{PS_3_friends}
%\examspace


%UNUSED
%\pinput[points = 10, title= \textbf{Graph Coloring}]{FP_chromatic_union}


%\pinput[points = 12, title= \textbf{Markov \& Chebyshev}]
%  {TP_markov_chebyshev_for_card_games}
%\examspace
  %, TASHA

%% \begin{problem}
%% Suppose that we build a graph on $n$ vertices as follows: for each
%% (unordered) pair of distinct vertices, we independently toss a fair
%% coin, and draw an edge between that pair of vertices if the coin lands
%% `heads'.

%% Justify your responses for all parts below.

%% \bparts

%% \ppart Given a vertex, what is the probability that the vertex has
%% degree exactly $3$?

%% \begin{solution}
%% This vertex is a member of $n-1$ vertex pairs, and this vertex has
%% degree $3$ precisely when exactly $3$ of these vertex pairs are
%% connected by edges. Considering the $2^{n-1}$ equally probable
%% outcomes for the edge status of these $n-1$ vertex pairs, there are
%% $\binom{n-1}{3}$ outcomes for which the vertex has degree $3$. Thus
%% the probability is
%% $$ 2^{-n+1} \binom{n-1}{3}. $$
%% \end{solution}

%% \ppart Let $D_i$ denote the degree of vertex $i$ as a random variable.
%% Which standard family of probability distributions does $D_i$ belong
%% to?

%% \begin{solution}
%% The degree follows a binomial distribution with parameters $n-1$
%% and $p = \frac12$.  The argument of the part above essentially shows
%% this; we could replace `$3$' with any other value in part (a) and
%% recover the pdf for a binomial distribution.
        
%% For a more formal argument, let $E_{ij}$ denote the indicator random
%% variable taking on value $1$ if $(i,j)$ is an edge, which is true with
%% probability $\frac12$; this is a Bernoulli random variable with
%% parameter $\frac12$, and the random variables $E_{ij}$ are
%% independent.  Fixing a vertex $i$, the degree of $i$ is $\sum_j
%% E_{ij}$, and a sum of $n-1$ independent Bernoulli variables with
%% parameter $\frac12$ is a binomial distribution as described.
%% \end{solution}

%% \ppart What is the expected number of vertices in the graph with
%% degree exactly $3$?

%% Partial credit will be given for answers written in terms of $p$,
%% where $p$ represents the correct answer to part (a).

%% \begin{solution}
%% Let $X_i$ be the indicator that vertex $i$ has degree $3$; then the
%% expectation of $X_i$ is the same as the probability that vertex $i$
%% has degree $3$, which we computed in part (a). The number of degree
%% $3$ vertices in the graph is $X = \sum_{i=1}^n X_i$. Its expectation
%% is
%% \[
%% \expect{X} = \expect{\sum_{i=1}^n X_i} = \sum_{i=1}^n \expect{X_i}
%% = n \cdot p = n \cdot 2^{-n+1} \binom{n-1}{3},
%% \]
%% by the linearity of expectation.
%% \end{solution}

%% \eparts
%% \end{problem}


%% \begin{problem}
%% Suppose we choose a random $9$-digit MIT ID number, taken uniformly
%% from the numbers 000000000 to 999999999.
    
%% Each answer should be expressed as a ratio of integers or in scientific notation.

%% \bparts

%% \ppart What is the expected number of total occurrences of the digit
%% sequence `6042' in the ID number?

%% \begin{solution}
%% For $1 \leq i \leq 6$, let $X_i$ be the indicator random variable with
%% value $1$ precisely when the $i$th digit of the ID is a $6$, the
%% $(i+1)$th digit is a $0$, the $(i+2)$th is a $4$, {\bf and} the
%% $(i+3)$th is a $2$.

%% Each $X_i$ is $1$ with probability $\frac{1}{10000}$. By the linearity
%% of expectation, the expected number of occurrences of `6042' is
%% \[\expect{X_1 + \ldots + X_6} = \expect{X_1} + \ldots + \expect{X_6} = \frac{6}{10000}.\]
%% \end{solution}

%% \ppart What is the expected total number of occurrences of `6042' or `6041'? That is,
%% $$ \expect{\text{\# occurrences of `6042'} + \text{\# occurrences of `6041'}}. $$

%% \begin{solution}
%% Again we use the linearity of expectation to rewrite this as
%% $$ \expect{\text{\# occurrences of `6042'}} + \expect{\text{\#
%%     occurrences of `6041'}}. $$ The first term is the answer to part
%% (a); the second term is the same, since we would not have answered
%% part (a) differently for the sequence `6041'. Thus the answer is
%% $\dfrac{12}{10000}$.
%% \end{solution}
   
%% \eparts
%% \end{problem}

%%
%%\pinput[points = 10, title= \textbf{Expectation)}]{FP_class_expectation}
%%\examspace


%% \begin{problem}
%% Identify each of the following asymptotic statements as true or false,
%% with brief justification.

%% \bparts

%% \ppart $4^x = O(3^x)$.

%% \begin{solution}
%% False. The ratio
%%     $$ \frac{4^x}{3^x} = \left(\frac43\right)^x $$ tends to $\infty$
%% as $x \to \infty$.
%% \end{solution}

%% \ppart $x^2 = O(x^3)$.

%% \begin{solution}
%%  True. We can multiply $x^3$ by a constant (say, $1$) such that it
%%  eventually forms an upper bound for $x^2$ (say, for all $x \geq 1$).
%% \end{solution}

%% \ppart $x^3 - x^2 = o(5x^3)$.

%% \begin{solution}
%%  False. The ratio $\dfrac{5x^3}{x^3 - x^2}$ tends to $5$, not to $\infty$, as $x \to \infty$.
%% \end{solution}

%% \ppart $\frac{1}{x} = \Theta(1)$.

%% \begin{solution}
%% False. $\frac{1}{x}$ is not bounded below by any \emph{positive} constant times $1$.
%% \end{solution}

%% \ppart $e^{\ln^2 x} = \Omega(x^{100})$.

%% \begin{solution}
%% True. We can write $e^{\ln^2 x} = x^{\ln x}$, which exceeds $x^{100}$
%% for all sufficiently (very!) large $x$.
%% \end{solution}

%% \eparts

%% \end{problem}

%% \begin{problem}
%% Alice decides to play the lottery.  She bought 10,000 tickets, each
%% with a probability of $\dfrac{1}{1,000,000}$ of winning a payout of
%% $\$ 1,000,000$, probability $\dfrac{1}{10}$ of paying out $\$ 10$, and
%% probability $1 - \dfrac{1}{10} - \dfrac{1}{1,000,000}$ of being worth
%% nothing.  If multiple tickets win, the payouts remain as above for
%% each ticket. The tickets are mutually independent.

%% \bparts

%% \ppart What is Alice's expected return? Express your answer as an
%% integer.

%% \begin{solution}
%% Each ticket has an expected return of $\frac{1}{1,000,000} \cdot
%% 1,000,000 + \frac{1}{10} \cdot 10 = 2$ dollars. By the linearity of
%% expectation, Alice's expected return is the sum of the expected
%% returns of $10,000$ such tickets, which gives a total of $\$ 20,000$.
%% \end{solution}

%% \ppart
%% Does your answer to part (a) depend on the tickets being mutually independent?
%% \begin{solution}
%% Part (a) uses only the linearity of expectation, so independence is not required.
%% \end{solution}

%% \ppart
%% What is the variance of Alice's return? Express your answer as an integer.

%% \begin{solution}
%% Because the payoffs $X_i$ of individual tickets are mutually
%% independent, the variance of the sum is the sum of the variances. So
%% we compute the variance of an individual ticket:
%% \begin{align*}
%% \variance{X_i}
%%   &= \expect{(X_i - \expect{X_i})^2} \\
%%   &= (-2)^2 \cdot \left( 1 - \frac{1}{10} - \frac{1}{1,000,000} \right)  + 999,998^2 \cdot \frac{1}{1,000,000} + 8^2 \cdot \frac{1}{10} \\
%%   &= 1,000,006.
%% \end{align*}

%% An arithmetically easier way to compute this is as follows:
%% \[\expect{X_i^2} = 1,000,000^2 \cdot \frac{1}{1,000,000} + 10^2 \cdot
%% \frac{1}{10} = 1,000,010,
%% \]
%% \[
%% \variance{X_i} = \expect{X_i^2} - (\expect{X_i})^2 = 1,000,010 - 4 =
%% 1,000,006.
%% \]

%% The total variance of $R = \sum_{i=1}^{100} X_i$ is then $10,000$
%% times this, for a final answer of $10,000,060,000$ (in units of
%% dollars squared).
%% \end{solution}

%% \ppart Does your answer to part (c) depend on the tickets being
%% mutually independent?

%% \begin{solution}
%% Part (c) uses the additivity of variance, so independence is required.
%% \end{solution}
        
%% \ppart
%% Give a Markov upper bound for the probability that Alice wins at least $\$ 100,000,000$. Your answer should be expressed as a ratio of integers.

%% \begin{solution}
%% Markov's Theorem states that, since $R$ is non-negative,
%% \[
%%   \pr{R \ge x} \le \frac{\expect{R}}{x}.
%% \]
%% In this case,
%%             $$\pr{R \ge 100,000,000} \le \frac{20,000}{100,000,000} = \frac{1}{5,000}.$$
%% \end{solution}
%% \ppart Give a Chebyshev upper bound for the probability that Alice
%% wins at least $\$ 100,000,000$.  Your answer should be expressed in
%% terms of basic arithmetic operations on integers, but need not be
%% simplified further.

%% \begin{solution}
%% \begin{align*}
%% \pr{R \geq 100,000,000} &= \pr{|R-20,000| \geq 99,980,000} \\ &=
%% \pr{|R - \expect{R}| \geq 99,980,000} \\
%% &\leq \frac{\variance{R}}{99,980,000^2} \\ &= \frac{10,000,060,000}{99,980,000^2}\\
%% &= \frac{1,000,006}{999,800^2} \\
%% &= \frac{1,000,006}{999,600,040,000} \\
%% &= \frac{500,003}{499,800,020,000}.
%% \end{align*}
%% \end{solution}

%% \eparts
%% \end{problem}

%% \begin{problem}
%% We play a simplified game of battleship. We are given a 4x4 board
%% \begin{center}
%% \begin{tabular}{| c | c | c | c |}
%% \hline & & & \\
%% \hline & & & \\
%% \hline & & & \\
%% \hline & & & \\
%% \hline
%% \end{tabular}
%% \end{center}
%% on which you have placed two pieces. Your destroyer is
%% $1\ \text{square} \times 2\ \text{squares}$
%% \begin{tabular}{| c | c |}
%% \hline 
%% \ & \\
%% \hline
%% \end{tabular}
%% and your submarine is $1 \text{ square} \times 3 \text{ squares}$
%% \begin{tabular}{| c | c | c |}
%% \hline 
%% \ & & \\
%% \hline
%% \end{tabular}.
%% The pieces lie entirely on the board, cannot overlap, and are arranged
%% either vertically or horizontally.

%% Your opponent picks 8 of the 16 squares uniformly at random and then
%% shoots at those 8 squares. A ship is sunk if all the squares it
%% occupies are shot at.

%% For this problem, you may leave your answer as the sum of expressions
%% that are products or ratios of integers, exponentials, factorials
%% and/or choose expressions.

%% \bparts
%% %\ppart{10} How many ways are there to place your pieces on the board?
%% %\solution[\newpage]{
%% %  First we count the number of ways to place the destroyer. If it is horizontal, there are
%% %  12 possible locations and by symmetry, there are 12 vertical locations, for a total of 24 locations.
%% %  Similarly, there are 16 locations for the submarine.
%% %  The total number of ways to place both pieces is the product of all locations for the destroyer and
%% %  the submarine, subtracting away the number of illegal overlapping positions.
%% %
%% %  Suppose the pieces are both horizontal or both vertical. There are 24 horizontal and 24 vertical
%% %  arrangements that give the overlap. Next suppose the submarine is horizontal and the destroyer is
%% %  vertical. If the submarine is in the top or bottom row, there are 4 positions for it and 3
%% %  overlapping positions for the destroyer for each. If the submarine is in a middle row, there are
%% %  4 positions for it and 6 overlapping positions for the destroyer for each. This adds to 12 + 24 = 36
%% %  overlapping positions. By symmetry, if the submarine is vertical and the destroyer is horizontal, there
%% %  are 36 overlapping positions.
%% %
%% %  The total number of ways to place the pieces on the board is $24 \cdot 16 - 8 - 8 - 36 - 36 = \fbox{264}$
%% %}

%% \ppart%{8}
%% What is the probability that both of your ships are sunk?
%% \begin{solution}
%%     There are $16}{8$ total choices for your opponent. There are $\binom{16 - 5}{8 - 5}$
%%     choices that pick all of your squares. So the probability that both of your ships 
%%     are sunk is
%%     $\frac{\binom{11}{3}}{\binom{16}{8}}$.
%% \end{solution}

%% \ppart %10

%% What is the probability of sinking the submarine but not the
%% destroyer?

%% \begin{solution}
%%  To sink the submarine (and perhaps also the destroyer), your opponent
%%  must choose the $3$ submarine squares and any $5$ of the $13$ others;
%%  so the probability of this event is
%%     $$ \frac{\binom{13}{5}}{\binom{16}{8}}. $$ To exclude the
%%  possibility of also sinking the destroyer, we subtract off our answer
%%  to part (a), and conclude with
%%     $$ \frac{\binom{13}{5} - \binom{11}{3}}{\binom{16}{8}}. $$
%% \end{solution}

%% \eparts

%% \end{problem}

   
%% \begin{staffnotes}
%% Richard thought confusingly worded
%% \end{staffnotes}
%% \pinput[points = 5, title= \textbf{Random Walks}]{FP_random_grid_walk}
%% \examspace


%% \pinput[points = 8, title = \textbf{Logical
%%     Injections}]{FP_logical_jections}
%% \examspace

%used S16 conflict midterm

%% \pinput[points = 10, title= \textbf{Simple
%%     Graphs}]{FP_degree_sequences}
%% \examspace

%% \pinput[points = 8, title= \textbf{Stable
%%     Marriage}]{FP_Stable_Marriage_Invariants}
%% \examspace


%% \pinput[points = 10, title= \textbf{Chebyshev's
%%     Bound}]{FP_hot_cows_chebyshevS15}
%% \examspace


%% \pinput[points = 10, title= \textbf{induction, graphs,
%%     ARM}]{FP_cycles_components_induction}
%% \examspace


%% \pinput[points = 20, title = \textbf{Independence}]{FP_conditional_independence}
%% \examspace

%% \pinput[points = 10, title= \textbf{Random Walks}]{FP_random_walk_examples}
%% \examspace


%% \pinput[points = 15, title = \textbf{Counting Graphs \& Relations}]
%%        {FP_counting_graphs_f13}
%% \examspace

%% \pinput[points = 10, title= \textbf{Asymptotics}]{FP_Oh_not_Theta}
%% \examspace

%% \pinput[points = 10, title= \textbf{Counting,
%%     Probability}]{FP_pair_probability}
%% \examspace

%% \pinput[points = 15, title = \textbf{Partial
%%     Orders}]{FP_partial_order_short_answer_f13}
%% \examspace

%% \pinput[points = 15, title = \textbf{Relations and Predicates}]
%%        {FP_relation_properties_expressions_final_f13}
%% \examspace

%% \pinput[points = 10, title= \textbf{Counting,
%%     Probability}]{FP_count_words}

