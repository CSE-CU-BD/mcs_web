\chapter{Propositions}\label{prop_chap}

\begin{definition*}
A \term{proposition} is a statement that is either true or false.
\end{definition*}

For example, both of the following statements are propositions.  The
first is true and the second is false.
\begin{proposition}
2 + 3 = 5.
\end{proposition}
\begin{proposition}
1 + 1 = 3.
\end{proposition}
Being true or false doesn't sound like much of a limitation, but it
does exclude statements such as, ``Wherefore art thou Romeo?'' and
``Give me an \emph{A}!''.

Unfortunately it is not always easy to decide if a proposition is true
or false, or even what the proposition means.  In part, this is
because the English language is riddled with ambiguities as in
\arm{delete ``consider''} the following statements:
%

\begin{itemize}
\item You may have cake, or you may have ice cream. \arm{bullets not numbers}
\item If pigs can fly, then you can understand the Chebyshev bound.
\item If you can solve any problem we come up with, then you get an
  \emph{A} for the course.
\item Every American has a dream.
\end{itemize}
%
What \emph{precisely} do these sentences mean?  Can you have both cake and
ice cream or must you choose just one dessert?  Since pigs can't fly, does
the second sentence say anything about your understanding the Chebyshev
bound?  If you can solve some problems we come up with, then does the
third sentence mean you can get an \emph{A} for the course?  And if you
can't solve a single one of the problems, does it mean you can't get an
\emph{A}?  Finally, does the last sentence imply that all Americans have
the same dream---say of owning a house---or might different Americans may
\arm{flesh out dreams}
have different dreams---say, Eric dreams of designing a ``killer''
software application, Tom of being a tennis champion, Albert of being able
to sing?

Some uncertainty is tolerable in normal conversation.  But when we need to
formulate ideas precisely---as in mathematics and programming---the
ambiguities inherent in everyday language can be a real problem.  We can't
hope to make an exact argument if we're not sure exactly what the
statements mean.  So before we start into mathematics, we need to
investigate the problem of how to talk about mathematics.

To get around the ambiguity of English, mathematicians have devised a
special \arm{why \emph{mini}-language?}language for talking about logical
relationships.  This language mostly uses ordinary English words and
phrases such as ``or'', ``implies'', and ``for all''.  But mathematicians
\arm{DELETED than those found in an ordinary dictionary.} give these words
precise and unambiguous definitions.
\begin{editingnotes}
\arm{REPLACED}: Without knowing these
  definitions, you might sometimes get the gist of statements in this
  language, but you would regularly get misled about what they really
  meant.
\end{editingnotes}
You should watch out for the pitfall of confusing ordinary language with
mathematical language that sounds ordinary but isn't.

Surprisingly, in the midst of learning the language of mathematics, we'll
come across the most important open problem in computer science---a
problem whose solution could change the world.

\section{Compound Propositions}

In English, we can modify, combine, and relate propositions with words
such as ``not'', ``and'', ``or'', ``implies'', and ``if-then''.
For example, we can combine three propositions into one like this:
%
\begin{center}
\textbf{If} all humans are mortal \textbf{and} all Greeks are human,
\textbf{then} all Greeks are mortal.
\end{center}

For the next while, we won't be much concerned with the internals of
propositions---whether they involve mathematics or Greek mortality---but
rather with how propositions are combined and related.  So we'll
frequently use variables such as $P$ and $Q$ in place of specific
propositions such as ``All humans are mortal'' and ``$2 + 3 = 5$''.  The
understanding is that these \term{propositional variables}\arm{CUT: like
  propositions}, can take on only the values \true~(true) and
\false~(false).  Propositional variables are also called \term{Boolean
  variables} after their inventor, the \arm{inserted 19th century}
nineteenth century mathematician George---you guessed it---Boole.

\subsection{\QNOT, \QAND, and \QOR}

\arm{operations} Mathematicians use the words $\QNOT$, $\QAND$, and $\QOR$
for operations that change or combine propositions.  The precise
mathematical meaning of these special words can be specified by
\term{truth tables}.  For example, if $P$ is a proposition,
then so is ``$\QNOT(P)$,'' and the truth value of the proposition
``$\QNOT(P)$'' is determined by the truth value of $P$ according to the
following truth table:
%
\[
\begin{array}{c|c}
P & \QNOT(P) \\ \hline
\true & \false \\
\false & \true \\
\end{array}
\]
%
The second row of the table indicates that when $P$ is false,
``$\QNOT(P)$'' is true, as you might expect.  The first row indicates that
when proposition $P$ is true, the proposition ``$\QNOT(P)$'' is false.

In general, a truth table indicates the true/false value of a proposition
\arm{REPLACED setting of the variables} for each possible set of truth
values for the variables.  For example, the truth table for the
proposition ``$P \QAND Q$'' has four lines, since there are four settings
of truth values for the two variables:
%
\[
\begin{array}{cc|c}
P & Q & P \QAND Q \\ \hline
\true & \true & \true \\
\true & \false & \false \\
\false & \true & \false \\
\false & \false & \false
\end{array}
\]
%
According to this table, the proposition ``$P \QAND Q$'' is true only when
$P$ and $Q$ are both true.  This is probably the way you ordinarily think
about the word ``and.''

There is a subtlety in the truth table for ``$P \QOR Q$'':
%
\[
\begin{array}{cc|c}
P & Q & P \QOR Q \\ \hline
\true & \true & \true \\
\true & \false & \true \\
\false & \true & \true \\
\false & \false & \false
\end{array}
\]
%
The third row of this table says that ``$P \QOR Q$'' is true even if
\emph{both} $P$ and $Q$ are true.  This isn't always the intended
meaning of ``or'' in everyday speech, but this is the standard definition
in mathematical writing.  So if a mathematician says, ``You may have cake,
or you may have ice cream,'' he means that you \emph{could} have both.

If you want to exclude the possibility of both having and eating, you
should combine them with the \term{exclusive-or} operation, $\QXOR$:
%
\[\begin{array}{cc|c}
P & Q & P \QXOR Q \\ \hline
\true & \true & \false \\
\true & \false & \true \\
\false & \true & \true \\
\false & \false & \false
\end{array}
\]
%

\subsection{Implies}

The combining operation with the least intuitive technical meaning is
``implies.''  Here is its truth table, with the lines labeled so we can
refer to them later.
%
\[
\begin{array}{cc|cr}
    P  &   Q    & \parbox[b]{13ex}{$P \QIMP Q$} \\ \hline
\true  & \true  & \true & \text{(tt)}\\
\true  & \false & \false  & \text{(tf)}\\
\false & \true  & \true  & \text{(ft)}\\
\false & \false & \true  & \text{(ff)}
\end{array}
\]
Let's experiment with this definition.  For example, is the following
proposition true or false?
\begin{center}
``If the Riemann Hypothesis is true, then $x^2 \geq 0$ for every real
number $x$.''
\end{center}
The Riemann Hypothesis is a famous unresolved conjecture in
mathematics---no one knows if it is true or false.  But that doesn't
prevent you from answering the question!  This proposition has the form $P
\QIMPLIES Q$ where the \emph{hypothesis},~$P$, is ``the Riemann Hypothesis
is true'' and the \emph{conclusion},~$Q$, is ``$x^2 \geq 0$ for every real
number $x$.''  Since the conclusion is definitely true, we're on either
line (tt) or line (ft) of the truth table.  Either way, the \QIMP\
proposition as a whole is \emph{true}!

One of our original examples demonstrates an even stranger side of
implications.
\begin{center}
``If pigs can fly, then you can understand the Chebyshev bound.''
\end{center}
Don't take this as an insult; we just need to figure out whether this
proposition is true or false.  Curiously, the answer has
\emph{nothing} to do with whether or not you can understand the
Chebyshev bound.  Pigs cannot fly, so we're on either line (ft) or
line (ff) of the truth table.  In both cases, the proposition is
\emph{true}!

In contrast, here's an example of a false implication:
%
\begin{center}
``If the moon shines white, then the moon is made of white cheddar.''
\end{center}
%
Yes, the moon shines white.  But, no, the moon is not made of white
cheddar cheese.  So we're on line (tf) of the truth table, and the
proposition is false.

The truth table for implications can be summarized in words as
follows:
%
\textbox{
An implication is true exactly when the if-part is false or the
then-part is true.
}
%
This sentence is worth remembering; a large fraction of all
mathematical statements are of the if-then form!

\subsection{If and only if}

Mathematicians commonly join propositions in one additional way that
doesn't arise in ordinary speech.  The proposition ``$P$ if and only
if $Q$'' asserts that $P$ and $Q$ have the same truth value, that is,
either both are true or both are false.
%
\[
\begin{array}{cc|c}
P & Q & P \QIFF Q \\ \hline
\true & \true & \true \\
\true & \false & \false \\
\false & \true & \false \\
\false & \false & \true
\end{array}
\]

For example, for every real number $x$, the following if-and-only-if
statement is true:
%
\begin{center}
$x^2 - 4 \geq 0 \qiff \abs{x} \geq 2.$
\end{center}
%
For some values of $x$, \emph{both} inequalities are true.  For
other values of $x$, \emph{neither} inequality is true .  In every
case, however, the \QIFF\ proposition as a whole is true.

\subsection{Notation}

Mathematicians have devised symbols to represent words like ``\QAND''
and ``\QNOT''.  The most commonly-used symbols are summarized in the
table below.
%
\begin{center}
\begin{tabular}{ll}
\textbf{English} & \textbf{Symbolic Notation} \\[1ex]
$\QNOT(P)$ & $\neg P$ \quad (alternatively, $\bar{P}$) \\
$P \QAND Q$ & $P \land Q$ \\
$P \QOR Q$ & $P \lor Q$ \\
$P \QIMP Q$ & $P \implies Q$ \\
if $P$ then $Q$ & $P \implies Q$ \\
$P \QIFF Q$ & $P \iff Q$
\end{tabular}
\end{center}
%
For example, using this notation, ``If $P \QAND \QNOT(Q)$, then $R$''
would be written:
%
\[
    (P \land \bar{Q}) \implies R
\]
This symbolic language is concise, but words such as ``\QOR'' and
``\QIMPLIES'' generally serve just as well as the symbols~$\lor$
and~$\implies$, and their meaning is easier to remember.  For the most
part, we will stick to using the words.\arm{, but you can
  feel free to use whichever convention is easiest for you.}

\subsection{Logically Equivalent Implications}\label{sec:logical_equivalence}

Do these two sentences say the same thing?
%
\begin{center}
If I am hungry, then I am grumpy. \\
If I am not grumpy, then I am not hungry.
\end{center}
%
We can settle the issue by recasting both sentences in terms of
propositional logic.
Let $P$ be the proposition ``I am hungry'', and let $Q$ be ``I am
grumpy''.  The first sentence says ``$P \QIMPLIES Q$'' and the second
says ``$\QNOT(Q) \QIMP \QNOT(P)$''.  We can compare these two
statements in a truth table:
%
\[
\begin{array}{c|c|c|c}
P & Q &
    P \QIMP Q &
    \QNOT(Q) \QIMP \QNOT(P) \\ \hline
\true & \true & \true & \true \\
\true & \false & \false & \false \\
\false & \true & \true & \true \\
\false & \false & \true & \true
\end{array}
\]
%
Statements which always have the same truth value are called
\term{logically equivalent}.  So sure enough, these two statements are
logically equivalent because the columns of truth values under each of
these statements are the same.  \arm{ADDED \QIFF} This is the same as
saying that each of these statements is true if and only if the other one
is, so the truth value of
\[
(P \QIMP Q)\ \QIFF\ (\QNOT(Q) \QIMP \QNOT(P))
\]
is \emph{always} true whatever the truth values of $P$ and $Q$ happen to
be.

The expression ``$\QNOT(Q) \QIMP \QNOT(P)$'' is called the
\term{contrapositive} of the implication ``$P \QIMP Q$.''  As we've
just shown, the two are just different ways of saying the same thing.

In contrast, the \term{converse} of ``$P \QIMP Q$'' is the statement
``$Q \QIMP P$''.  In terms of our example, the converse is:
%
\begin{center}
If I am grumpy, then I am hungry.
\end{center}
%
This sounds like a rather different contention, and a truth table
confirms this suspicion:
%
\[
\begin{array}{c|c|c|c}
P & Q &
    P \QIMP Q &
    Q \QIMP P \\ \hline
\true & \true & \true & \true \\
\true & \false & \false & \true \\
\false & \true & \true & \false \\
\false & \false & \true & \true
\end{array}
\]
%
\arm{inserted column mention} Since the column of truth values for $P
\QIMP Q$ is not the same as the column for $Q \QIMP P$, we have verified
that an implication is \emph{not} necessarily equivalent to its converse.

One final relationship: an implication and its converse together are
equivalent to an iff statement.  For example,
%
\begin{center}
If I am grumpy, then I am hungry, \QAND \\
if I am hungry, then I am grumpy.
\end{center}
%
are equivalent to the single statement:
%
\begin{center}
I am grumpy $\QIFF$ I am hungry.
\end{center}
%
Once again, we can verify this with a truth table.  We begin with a table
with just the truth values of $P$ and $Q$:
%
\[
\begin{array}{c|c|ccc|c}
P & Q & (P \QIMP Q) &\QAND & (Q  \QIMP  P) & P \QIFF Q \\
\hline
\true  &  \true\\
\true  &  \false\\
\false &  \true\\
\false &  \false
\end{array}
\]
These truth values are enough to fill in three more columns:
\[
\begin{array}{c|c|ccc|c}
P & Q & (P \QIMP Q) &\QAND & (Q  \QIMP  P) & P \QIFF Q \\
\hline
\true  &  \true  &\true  &&\true & \lgtrue \\
\true  &  \false &\false &&\true & \lgfalse\\
\false &  \true  &\true  &&\false& \lgfalse\\
\false &  \false &\true  &&\true & \lgtrue 
\end{array}
\]
Finally, now using the first two of the filled in columns, we can fill in
the fourth column:
\[
\begin{array}{c|c|ccc|c}
P & Q & (P \QIMP Q) &\QAND & (Q  \QIMP  P) & P \QIFF Q \\
\hline
\true  &  \true  &\true  &\lgtrue &\true & \lgtrue \\
\true  &  \false &\false &\lgfalse&\true & \lgfalse\\
\false &  \true  &\true  &\lgfalse&\false& \lgfalse\\
\false &  \false &\true  &\lgtrue &\true & \lgtrue
\end{array}
\]
So the fourth column giving the truth values of 
\[
(P \QIMP Q) \QAND (Q \QIMP P)
\]
is the same as the sixth column giving the truth values of $P \QIFF
Q$, which confirms that the \QAND\ of the implications is logically
equivalent to the \QIFF\ statement.

\begin{problems}
\practiceproblems
\pinput{TP_Basic_Propositions}

\classproblems
\pinput{CP_differentiable_implies_continuous}
\pinput{CP_truth_table_for_distributive_law}

\homeworkproblems
\pinput{PS_printout_binary_strings}
\end{problems}

\section{Propositional Logic in Computer Programs}

Propositions and logical connectives arise all the time in computer
programs.  For example, consider the following snippet, which could be
either C, C++, or Java:
%
\begin{tabbing}
\hspace{1in} \= \quad\quad \= \quad\quad \= \quad\quad \= \kill
\> \texttt{if ( x > 0 || (x <= 0 \&\& y > 100) )} \\
\> \> \vdots\\
\> \emph{(further instructions)}
\end{tabbing}
%
The symbol \texttt{||} denotes ``\QOR'', and the symbol \texttt{\&\&}
denotes ``\QAND''.  The \emph{further instructions} are carried out
only if the proposition following the word \texttt{if} is true.  On
closer inspection, this big expression is built from two simpler
propositions.  Let $A$ be the proposition that \texttt{x > 0}, and let
$B$ be the proposition that \texttt{y > 100}.  Then we can rewrite the
condition ``$A \QOR (\QNOT(A) \QAND B)$''.
%
A truth table reveals that this complicated expression is logically
equivalent to ``$A \QOR B$''.  We begin with a table with just the truth
values of $A$ and $B$:
\[
\begin{array}{cc|ccccc|c}
A      & B      & A  & \QOR  & (\QNOT(A)& \QAND & B) & A \QOR  B \\ \hline
\true  & \true \\
\true  & \false\\
\false & \true \\
\false & \false                       
\end{array}
\]
This is enough to fill in two more columns
\[
\begin{array}{cc|ccccc|c}
A      & B      & A & \QOR  & (\QNOT(A) & \QAND & B) & A \QOR  B \\ \hline
\true  & \true  &   &       &  \false   &       &    & \lgtrue \\
\true  & \false &   &       &  \false   &       &    & \lgtrue \\
\false & \true  &   &       &  \true    &       &    & \lgtrue \\
\false & \false &   &       &  \true    &       &    & \lgfalse\\
\end{array}
\]
Now we can fill in the \QAND\ column:
\[
\begin{array}{cc|ccccc|c}
A      & B      & A & \QOR  & (\QNOT(A) & \QAND   & B) & A \QOR  B \\ \hline
\true  & \true  &   &       &  \false   &  \false &    & \lgtrue \\
\true  & \false &   &       &  \false   &  \false &    & \lgtrue \\
\false & \true  &   &       &  \true    &  \true  &    & \lgtrue \\
\false & \false &   &       &  \true    &  \false &    & \lgfalse\\
\end{array}
\]
and then fill in the remaining column of values for the first \QOR\
expression:
\[
\begin{array}{cc|ccccc|c}
A      & B      & A & \QOR     &(\QNOT(A) & \QAND   & B) & A \QOR  B \\ \hline
\true  & \true  &   & \lgtrue  & \false &    \false &    & \lgtrue \\
\true  & \false &   & \lgtrue  & \false &    \false &    & \lgtrue \\
\false & \true  &   & \lgtrue  & \true  &    \true  &    & \lgtrue \\
\false & \false &   & \lgfalse & \true  &    \false &    & \lgfalse\\
\end{array}
\]
Since the (highlighted) columns of truth values of the two expressions are
the same, we can simplify the code snippet without changing the program's
behavior:
%
\begin{tabbing}
\hspace{1in} \= \quad\quad \= \quad\quad \= \quad\quad \= \kill
\> \texttt{if ( x > 0 || y > 100 )} \\
\> \> \vdots\\
\> \emph{(further instructions)}
\end{tabbing}

\arm{revised} Simplifying logical expressions has real practical
importance in computer science.  In programs, the kind of expression
simplification illustrated above can increase execution speed, in addition
to making a program easier to read and understand.  In hardware,
simplifying expressions can decrease the number of logic gates on a chip.
That's because digital circuits can be described by logical formulas (see
Problems~\ref{CP_binary_adder_logic} and~\ref{PS_faster_adder_logic}), and
minimizing the logical formulas corresponds to reducing the number of
gates in the circuit.
\begin{editingnotes}
  \arm{CUT} Chip designers face a similar challenge---instead of
  minimizing \texttt{\&\&} and \texttt{||} symbols in a program, their job
  is to minimize the number of analogous physical devices on a chip.
\end{editingnotes}
The payoff of gate minimization is potentially enormous: a chip with
fewer gates is smaller, consumes less power, has a lower defect rate,
and is cheaper to manufacture.

\begin{problems}
%\practiceproblems
\classproblems
\pinput{CP_file_system_functioning_normally}
\pinput{CP_binary_adder_logic}

\homeworkproblems
\pinput{PS_faster_adder_logic}

\end{problems}

\section{Standard forms for propositional formulas}
\begin{editingnotes}
DNF from True entries in truth table; CNF from false entries.   DNF from
algebra: distributivity, De Morgan, sort alphabetically.
\end{editingnotes}

\section{Predicates and Quantifiers}\label{pred_sec}

\subsection{Propositions with Infinitely Many Cases}

Truth values and equivalences of propositional formulas are easy to
determine using truth tables when there are only a few propositional
variables because there are only a few cases to check.  But truth tables
quickly reach unmanageable size when there are many variables.  For
example, for a proposition involving 20 propositional
variables there are more than a million cases (rows) to check.  
\begin{editingnotes}
\arm{CUT} \textcolor{red}{CUT by ARM}:
  Most of the examples of propositions that we have considered thus far
  have been straightforward in the sense that it has been relatively easy
  to determine if they are true or false.  At worse, there were only a few
  cases to check in a truth table.  But checking all the cases of a
  proposition is not going to work well when there are a very large number
  of cases,
\end{editingnotes}
 And case checking it is not going to work at all when there
are an infinite number of possible cases.  An example where there are an
infinite number of cases is the following proposition involving prime
numbers.\footnote{Reminder: a \term{prime} is an integer greater than~1
  whose only positive divisors are itself and 1.  For example, 2, 3, 5, 7,
  and~11 are primes, but 4, 6, and~9 are not.  A number greater than~1
  that is not prime is said to be \term{composite}.}
\begin{proposition}\label{41form}
For every nonnegative integer, $n$, the value of the
polynomal\footnote{The symbol \term{$\eqdef$} means ``equal by
  definition.''  It's always ok simply to write ``='' instead of
  $\eqdef$, but reminding the reader that an equality holds by
  definition can be helpful.}
\begin{equation}\label{pn41}
p(n) \eqdef  n^2 + n + 41
\end{equation}
is prime.
\end{proposition}
You probability can't tell right away whether Proposition~\ref{pn41}
is true or false.  Let's try checking it by computing the value of
$p(n)$ for several values of~$n$.  If any of these values is not
prime, then we will know that the proposition is false.

To start with, the value of $p(0)$ checks out OK, because $p(0) = 41$,
and 41 is prime.  Next, $p(1) = 43$ is also prime.  Continuing, $p(2)
= 47$, $p(3)=53$, \dots, and $p(20) = 461$ are all are prime.  So far,
so good; it is starting to look like $p(n)$ really might be prime all
the time.  In fact, continued checking shows that $p(n)$ is prime for
all $n \le 39$.  Now this may seem like persuasive evidence that
Proposition~\ref{41form} is true.  But no, things break at $n=40$.
Namely, $p(40) = 40^2 + 40 + 41 = 41 \cdot 41$, which is not prime.
So it's \emph{not} true that the expression is prime for \emph{all}
nonnegative integers, and thus the proposition is false!

Of course $p(n)$ is a contrived example with surprising behavior
chosen to make a point: jumping to a conclusion that \emph{all} values
come out OK because \emph{a lot} of values did, is dangerous.  This
point is not contrived: there are ``real'' propositions for which
checking lots of cases has turned out to be misleading.  Here's a
famous example:
\begin{proposition}\label{a4}
$a^4 + b^4 + c^4 = d^4$ has no solution when $a, b, c, d$ are positive
integers.
\end{proposition}
The great mathematician \idx{Euler} (pronounced ``oiler'') conjectured
in~1769 that this proposition was true.  It was checked by hand and
then by computers for many values of $a$, $b$, $c$, and~$d$ over the
next two centuries, and all the values checked out OK.  But ultimately
the proposition was proven false by a set of values worked out in~1987
by Noam \idx{Elkies}.  These values were $a = 95800, b = 217519, c =
414560, d = 422481$.  Think about how long it would take a computer to
search through all possible six digit values for $a,b,c$, and $d$.
No wonder it took 218 years to show the proposition is false!

Propositions that involve \emph{all} numbers are so common that there
is a special notation for them.  For example, Proposition~\ref{41form}
can also be written as
\begin{equation}\label{pn}
\forall n \in \naturals.\; p(n) \text{ is prime}.
\end{equation}
Here the symbol \term{$\forall$} is read ``for all''.  The symbol
\term{$\naturals$} stands for the set of \emph{nonnegative integers},
namely, 0, 1, 2, 3, \dots (ask your instructor for the complete list).
The symbol ``\term{$\in$}'' is read as ``is a member of,'' or
``belongs to,'' or simply as ``is in''.  The period after the
$\naturals$ is just a separator between phrases.

Euler's conjectured Proposition~\ref{a4} can now be written,
\[
\forall a \in \posints\, \forall b \in \posints\, \forall c \in
\posints\, \forall d \in \posints.\; a^4 + b^4 + c^4 \neq d^4.
\]
Here, \term{$\posints$} is a symbol for the positive integers.

Strings of $\forall$'s are usually abbreviated for easier reading, as
follows:
\[
\forall a, b, c, d \in \posints.\; a^4 + b^4 + c^4 \neq d^4.
\]

The following proposition is an even nastier example where value
checking is misleading.
\begin{proposition}\label{313elliptic}
$313 (x^3 + y^3) = z^3$ has no solution when $x, y, z\in\posints$.
\end{proposition}

Proposition~\ref{313elliptic} is also false, but the smallest values
for $x$, $y$, and~$z$ where it goes wrong have more than 1000 digits!
Even the world's largest computers would not be able to get that far
with brute force.

You might wonder why anyone would care whether or not there is a
positive integer solution to $313 (x^3 + y^3) = z^3$.  The reason is
that finding solutions to such equations is important in the field of
\term{elliptic curves}, which in turn is important in efforts to
\idx{factor} large integers, which in turns is important in cracking
commonly used \idx{cryptosystems} of the kind described in
Chapter~\ref{number_theory_chap}.  That's why mathematicians went to
the effort to find the solution with thousands of digits.

The overall message is that \textbf{you can't generally verify a claim
  about an infinite set by checking a finite number of its elements,
  no matter how large the finite number may be.}

Of course, not all propositions that have infinitely many cases to
check turn out to be false.  The following proposition, known as the
\term{\idx{Four-Color Theorem}}, happens to be true.
\begin{proposition}\label{4colorprop}
Every map can be colored with 4 colors so that adjacent\footnote{Two
  regions are adjacent only when they share a boundary segment of
  positive length.  They are not considered to be adjacent if their
  boundaries meet only at a few points.} regions have different
colors.
\end{proposition}
The proof of this proposition is difficult and took over a century to
perfect.  Along the way, many incorrect proofs were proposed,
including one that stood for 10 years in the late 19th century before
the mistake was found.  A laborious proof was finally found
in 1976 by mathematicians Appel and Haken, who used a complex computer
program to categorize the four-colorable maps; the program left a few
thousand maps uncategorized, and these were checked by hand by Haken
and his assistants---including his 15-year-old daughter.  There was a
lot of debate about whether this was a legitimate proof: the proof was
too big to be checked without a computer, and no one could guarantee
that the computer calculated correctly, nor did anyone have the energy
to recheck the four-colorings of the thousands of maps that were done by
hand.  Within the past decade, a mostly intelligible proof of the
Four-Color Theorem was found, though a computer is still needed to
check the colorability of several hundred special maps.\footnote{See
\href{http://www.math.gatech.edu/~thomas/FC/fourcolor.html}
{\texttt{http://www.math.gatech.edu/\~{}thomas/FC/fourcolor.html}}

The story of the Four-Color Proof is told in a well-reviewed
  popular (non-technical) book: ``Four Colors Suffice.  How the Map
  Problem was Solved.'' \emph{Robin Wilson}.  Princeton Univ. Press, 2003,
  276pp. ISBN 0-691-11533-8.}

There are also easy to state propositions whose truth is unknown such
as idx{Goldbach's conjecture}:
\begin{proposition}[Goldbach]
Every even integer~$n$ greater than~2 is the sum of two primes.
\end{proposition}
This conjecture has been intensely studied since~1742, and of course
has been checked by computer and found to be OK for a huge number of
values of~$n$---but that doesn't prove it.  To this day, no one knows
whether or not Goldbach's conjecture is true.

The preceding examples of propositions were about numbers, but the
issues they address show up as well for propositions about programs
and systems, for example.  The proposition that a program or system
does what it's supposed to do is pretty important to a computer
scientist, and the number of program behaviors is typically too large
thoroughly test.  Since bugs in programs are notoriously missed
despite the testing that is done, methods have been developed in the
past decade to \emph{prove} that programs will \emph{always} behave
correctly.  These methods have become so effective that leading
computer chip manufacturers commonly develop proofs that their designs
are correct.  For example, a notorious bug showed up in the 1990's in
the division circuit of an Intel chip.  Despite extensive advance
testing, the bug was not detected until the chip had been released and
began causing trouble for customers.  Nowadays such a bug would be
detected by procedures that prove chip correctness.

Developing mathematical methods to verify programs and systems remains
an active research area.  We'll consider some of these methods in
Chapter~\ref{chap:state-machines}.

\begin{problems}
\practiceproblems
\pinput{TP_Polynomial_for_the_Primes}

\homeworkproblems
\pinput{PS_prime_polynomial_41}

\iffalse
%FILL IN
\begin{editingnotes}
insert 4-coloring checker formula/circuit problem
\end{editingnotes}
\fi

\end{problems}

\subsection{Predicates}
A \term{predicate} is a proposition whose truth depends on the value of
one or more variables.  Most of the propositions above were defined in
terms of predicates.  For example,
%
\begin{center}
``$n$ is a perfect square''
\end{center}
%
is a predicate whose truth depends on the value of $n$.  The predicate is
true for $n = 4$ since four is a perfect square, but false for $n = 5$
since five is not a perfect square.

Like other propositions, predicates are often named with a letter.
Furthermore, a function-like notation is used to denote a predicate
supplied with specific variable values.  For example, we might name
our earlier predicate $P$:
%
\[
P(n) \eqdef \text{``$n$ is a perfect square''}
\]
%
Now $P(4)$ is true, and $P(5)$ is false.

This notation for predicates is confusingly similar to ordinary function
notation.  If $P$ is a predicate, then $P(n)$ is either \emph{true} or
\emph{false}, depending on the value of $n$.  On the other hand, if $p$
is an ordinary function, like $n^2 + n$, then $p(n)$ is a
\emph{numerical quantity}.  \textbf{Don't confuse these two!}

\subsection{Quantifiers}

There are a couple of assertions commonly made about a predicate: that it
is \emph{sometimes} true and that it is \emph{always} true.  For
example, the predicate
%
\[
\text{``$x^2 \geq 0$''}
\]
%
is always true when $x$ is a real number.  On the other hand, the
predicate
%
\[
\text{``$5x^2 - 7 = 0$''}
\]
%
is only sometimes true; specifically, when $x = \pm \sqrt{7/5}$.

There are several ways to express the notions of ``always true'' and
``sometimes true'' in English.  The table below gives some general
formats on the left and specific examples using those formats on the
right.  You can expect to see such phrases hundreds of times in
mathematical writing!
%
\begin{center}
\begin{tabular}{ll}
\multicolumn{2}{c}{\textbf{Always True}} \\[1ex]
For all $n$, $P(n)$ is true. & For all $x \in \reals$, $x^2 \geq 0$. \\
$P(n)$ is true for every $n$. & $x^2 \geq 0$ for every $x \in \reals$. \\[2ex]
\multicolumn{2}{c}{\textbf{Sometimes True}} \\[1ex]
There exists an $n$ such that $P(n)$ is true. & There exists an $x \in \reals$ such that $5x^2 - 7 = 0$.\\
$P(n)$ is true for some $n$. & $5x^2 - 7 = 0$ for some $x \in \reals$.\\
$P(n)$ is true for at least one $n$. & $5x^2-7=0$ for at least one $x \in \reals$.
\end{tabular}
\end{center}

All these sentences quantify how often the predicate is true.
Specifically, an assertion that a predicate is always true, is called
a \term{universally quantified} statement.  An assertion that a
predicate is sometimes true, is called an \term{existentially
  quantified} statement.

Sometimes English sentences are unclear about quantification:
%
\begin{center}
  ``If you can solve any problem we come up with, then you get an \emph{A}
  for the course.''
\end{center}
%
The phrase ``you can solve any problem we can come up with'' could
reasonably be interpreted as either a universal or existential
statement.  It might mean:
%
\begin{quote}
``You can solve \emph{every} problem we come up with,''
\end{quote}
or maybe
\begin{quote}
``You can solve \emph{at least one} problem we come up with.''
\end{quote}

In the preceding example, the quantified phrase appears inside a
larger if-then statement.  This is quite normal; quantified statements
are themselves propositions and can be combined with \QAND, \QOR,
\QIMPLIES, etc., just like any other proposition.

\subsection{More Notation}

There are symbols to represent universal and existential
quantification, just as there are symbols for ``$\QAND$'' ($\wedge$),
``$\QIMP$'' ($\implies$), and so forth.  In particular, to say
that a predicate, $P(x)$, is true for all values of $x$ in some set,
$D$, we write:
\begin{equation}\label{AxDPx}
\forall x \in D.\; P(x)\ .
\end{equation}
The \term{universal quantifier} symbol, $\forall$, is read ``for all,''
so this whole expression~\eqref{AxDPx} is read ``For all $x$ in $D$,
$P(x)$ is true.''  Remember that upside-down ``A'' stands for
``\textbf{A}ll.''

To say that a predicate $P(x)$ is true for at least one value of $x$
in $D$, we write:
\begin{equation}\label{ExDPx}
\exists x \in D.\; P(x)
\end{equation}
The \term{existential quantifier} symbol, \term{$\exists$}, is read
``there exists.''  So expression~\eqref{ExDPx} is read, ``There exists
an $x$ in $D$ such that $P(x)$ is true.''  Remember that backward
``E'' stands for ``\textbf{E}xists.''

The symbols $\forall$ and $\exists$ are always followed by a
variable---typically with an indication of the set the variable ranges
over---and then a predicate, as in the two examples above.

As an example, let $\probs$ be the set of problems we come up with,
$\solves(x)$ be the predicate ``You can solve problem $x$'', and $G$ be
the proposition, ``You get an \emph{A} for the course.''  Then the two
different interpretations of
%
\begin{quote}
  ``If you can solve any problem we come up with, then you get an \emph{A}
  for the course.''
\end{quote}
can be written as follows:
\[
(\forall x \in \probs.\; \solves(x)) \QIMP G,
\]
or maybe
\[
(\exists x \in \probs.\; \solves(x)) \QIMP G.
\]

\subsection{Mixing Quantifiers}

Many mathematical statements involve several quantifiers.  For
example, let's express \idx{Goldbach's Conjecture}, that every even
integer greater than 2 is the sum of two primes, being explicit about
the quantifiers, namely,
%
\begin{quote}
For every even integer $n$ greater than 2, there exist primes $p$ and
$q$ such that $n = p + q$.
\end{quote}
Let $\even$ be the set of even integers greater than 2, and let
$\primes$ be the set of primes.  Then this sentence
ould be expressed in logic notation as:
%
\begin{equation}\label{AEEGC}
\underbrace{\forall n \in \even.}_{\substack
    {\text{for every even} \\
     \text{integer $n > 2$}}}
\
\underbrace{\exists p \in \primes\ \exists q \in \primes.}_{\substack
    {\text{there exist primes} \\
     \text{$p$ and $q$ such that}}}
\ n = p + q.
\end{equation}
The proposition can also be written more simply as
\[
 \forall n \in \even.\, \exists p, q \in \primes. \; p + q = n.
\]

\subsection{Order of Quantifiers}

Swapping the order of different kinds of quantifiers (existential or
universal) usually changes the meaning of a proposition.  For example,
let's return to one of our initial, confusing statements:
\begin{center}
``Every American has a dream.''
\end{center}

This sentence is ambiguous because the order of quantifiers is
unclear.  Let $A$ be the set of Americans, let $D$ be the set of
dreams, and define the predicate $H(a, d)$ to be ``American $a$ has
dream $d$.''  Now the sentence could mean that there is a single dream
that every American shares:
\[
\exists\, d \in D.\, \forall a \in A.\; H(a, d)
\]
For example, it might be that every American shares the dream of owning
their own home.

Or it could mean that every American has a personal dream:
\[
\forall a \in A.\, \exists\, d \in D.\; H(a, d)
\]
For example, some Americans may dream of a peaceful retirement, while
others dream of continuing practicing their profession as long as they
live, and still others may dream of being so rich they needn't think at
all about work.

Swapping quantifiers in \idx{Goldbach's Conjecture}~\eqref{AEEGC}
creates a patently false statement:
\begin{falseclm*}
\begin{equation*}
\underbrace{\exists\, p, q \in \primes.}_{\substack
    {\text{there exist primes} \\
     \text{$p$ and $q$ such that}}}
\
\underbrace{\forall n \in \even.}_{\substack
    {\text{for every even} \\
     \text{integer $n > 2$}}}
\ n = p + q.
\end{equation*}
\end{falseclm*}
This claim says that every even number $\geq 2$ is the sum of \emph{the
  same} two primes, which is nonsensical: it implies that since $8$ is
the sum of the primes 3 and 5, the even number 4 is also the sum of 3
and 5, and so is 6, and so is every even integer.

\subsection{Variables Over One Domain}

When all the variables in a formula are understood to take values from the
same nonempty set, $D$, it's conventional to omit mention of $D$.  For
example, instead of $\forall x \in D\; \exists y \in D.\; Q(x,y)$ we'd
write $\forall x \exists y.\; Q(x,y)$.  The unnamed nonempty set that $x$
and $y$ range over is called the \term{domain of discourse}, or just plain
\term{domain}, of the formula.

It's easy to arrange for all the variables to range over one domain.
For example, \idx{Goldbach's Conjecture}~\eqref{AEEGC} could be
expressed with all variables ranging over the domain $\naturals$ as
\[
\forall n.\, (n \in \even)\ \QIMP\ (\exists p\, \exists q. \; p \in \primes \QAND
q \in \primes \QAND n = p + q).
\]

\subsection{Negating Quantifiers}

There is a simple relationship between the two kinds of quantifiers.  The
following two sentences mean the same thing:
%
\begin{quote}

It is not the case that everyone likes to snowboard.

There exists someone who does not like to snowboard.

\end{quote}
%
In terms of logic notation, this follows from a general property of
predicate formulas:
%
\[
\QNOT \paren{\forall x.\; P(x)}
\quad \text{is equivalent to} \quad
\exists x.\, \QNOT(P(x)).
\]
%
Similarly, these sentences mean the same thing:
%
\begin{quote}
There does not exist anyone who likes skiing over magma.

Everyone dislikes skiing over magma.
\end{quote}
%
We can express the equivalence in logic notation this way:
%
\begin{equation}\label{nE}
\QNOT\paren{\exists x.\, P(x)}\  \QIFF\  \forall x.\, \QNOT(P(x)).
\end{equation}
%
The general principle is that \emph{moving a ``not'' across a
quantifier changes the kind of quantifier.}


\begin{editingnotes}
\begin{staffnotes}
KEEP THIS CUT OUT:
\end{staffnotes}

Logicians have worked very hard to define strict rules for the use of
logic notation so that ideas can be expressed with absolute rigor.
It's all quite charming and clever.  However, the sad irony is that
applied mathematicians usually use their beloved notation as a crude
shorthand, breaking the rules and abusing the notation
willy-nilly---sort of like pounding nails with fine china.
\end{editingnotes}

\begin{problems}

\classproblems
\pinput{CP_logic_news_network}
\pinput{CP_assertions_about_binary_strings}
\pinput{CP_domain_of_discourse}
\pinput{CP_counter_model}

\homeworkproblems
\pinput{PS_express_in_predicate_form}
\pinput{PS_emailed_exactly_2_others}
\end{problems}



\section{Standard forms for quantified formulas}
\begin{editingnotes}
FILL IN

Explain prenex form: illustrate renaming bound variables by example.
Then claim (without proof) that variables can be renamed with fresh
variables yielding an equivalent formula satisfying the \term{variable
  convention}: no variable is quantified more than once, and no
quanitified var also occurs free.  Then get prenex form from DeMorgan
to push in negations and then pull out all quantifiers to the
beginning of the formula.  (These manipulations could be formulated as
a long example of recursive definition and proof by structural
induction.)
\end{editingnotes}


\section{Validity}\label{sec:validity}

A propositional formula is called \term{valid} when it evaluates to \true\
no matter what truth values are assigned to the individual propositional
variables.  For example, the propositional version of the \idx{Distributive Law}
is that $P \QAND (Q \QOR R)$ is equivalent to $(P \QAND Q) \QOR (P \QAND
R)$.  This is the same as saying that
\begin{equation}\label{eq:distributive_law}
[P \QAND (Q \QOR R)] \QIFF [(P \QAND Q) \QOR (P \QAND R)]
\end{equation}
is valid.  This can be verified by checking the truth table for $P
\QAND (Q \QOR R)$ and $(P \QAND Q) \QOR (P \QAND R)$:
\begin{equation*}
\begin{array}{c|c|c|ccc|ccc}
P & Q & R & P & \QAND & (Q \QOR R) & (P \QAND Q) &  \QOR & (P \QAND R) \\
\hline

\true & \true  & \true   && \lgtrue  &\true &\true & \lgtrue  &\true \\
\true & \true  & \false  && \lgtrue  &\true &\true & \lgtrue  &\false\\
\true & \false & \true   && \lgtrue  &\true &\false& \lgtrue  &\true \\
\true & \false & \false  && \lgfalse &\false&\false& \lgfalse &\false\\

\false & \true  & \true  && \lgfalse &\true &\false& \lgfalse &\false\\
\false & \true  & \false && \lgfalse &\true &\false& \lgfalse &\false\\
\false & \false & \true  && \lgfalse &\true &\false& \lgfalse &\false\\
\false & \false & \false && \lgfalse &\false&\false& \lgfalse &\false

\end{array}
\end{equation*}

The same idea extends to predicate formulas, but to be valid, a
formula now must evaluate to true no matter what values its variables
may take over any unspecified domain, and no matter what
interpretation a predicate variable may be given.  For example, we
already observed that the rule for negating a quantifier is captured
by the valid assertion~\eqref{nE}.

Another useful example of a valid assertion is
\begin{equation}\label{eaimpliesae}
\exists x \forall y.\; P(x,y) \QIMP \forall y \exists x.\; P(x,y).
\end{equation}

Here's an explanation why this is valid:

\begin{quote}
Let $D$ be the domain for the variables and $P_0$ be some
\idx{binary predicate}\footnote{That is, a predicate that depends on two variables.}
on $D$.  We need to show that if
\begin{equation}\label{exayp0}
\exists x \in D\; \forall y \in D.\; P_0(x,y)
\end{equation}
holds under this interpretation, then so does
\begin{equation}\label{ayexp0}
\forall y \in D\; \exists x \in D.\; P_0(x,y).
\end{equation}
So suppose~\eqref{exayp0} is true.  Then by definition of $\exists$, this
means that some element $d_0 \in D$ has the property that
\[
\forall y \in D.\, P_0(d_0, y).
\]
By definition of $\forall$, this means that
\[
P_0(d_0,d)
\]
is true for all $d \in D$.  So given any $d \in D$, there is an element in
$D$, namely, $d_0$, such that $P_0(d_0,d)$ is true.  But that's exactly
what~\eqref{ayexp0} means, so we've proved that~\eqref{ayexp0} holds under
this interpretation, as required.
\end{quote}

We hope this is helpful as an explanation, although purists would not
really want to call it a ``proof.''  The problem is that with something as basic
as~\eqref{eaimpliesae}, it's hard to see what more elementary axioms are
ok to use in proving it.  What the explanation above did was translate the
logical formula~\eqref{eaimpliesae} into English and then appeal to the
meaning, in English, of ``for all'' and ``there exists'' as justification.

In contrast to~\eqref{eaimpliesae}, the formula
\begin{equation}\label{aenotimplyea}
\forall y \exists x.\; P(x,y) \QIMP \exists x \forall y.\; P(x,y).
\end{equation}
is \emph{not} valid.  We can prove this by describing an
interpretation where the hypothesis, $\forall y \exists x.\; P(x,y)$, is
true but the conclusion, $\exists x \forall y.\; P(x,y)$, is not true.
For example, let the domain be the integers and $P(x,y)$ mean $x > y$.
Then the hypothesis would be true because, given a value, $n$, for $y$ we
could, for example, choose the value of $x$ to be $n+1$.  But under this
interpretation the conclusion asserts that there is an integer that is
bigger than all integers, which is certainly false.  An interpretation
like this which falsifies an assertion is called a \term{counter model} to
the assertion.

\section{Satisfiability}\label{SAT_sec}

A proposition is \textbf{satisfiable} if some setting of the variables
makes the proposition true.  For example, $P \QAND \bar{Q}$ is
satisfiable because the expression is true if $P$ is true or $Q$ is
false.  On the other hand, $P \QAND \bar{P}$ is not satisfiable because
the expression as a whole is false for both settings of $P$.  But
determining whether or not a more complicated proposition is satisfiable
is not so easy.  How about this one?
%
\[
(P \QOR Q \QOR R) \QAND (\bar P \QOR \bar Q)
                  \QAND (\bar P \QOR \bar R)
                  \QAND (\bar R \QOR \bar Q)
\]

The general problem of deciding whether a proposition is satisfiable
is called \term{SAT}.  One approach to SAT is to construct a truth
table and check whether or not a $\true$ ever appears.  But this
approach runs out of steam pretty quickly: a proposition with $n$
variables has a truth table with $2^n$ lines, so the effort required
to decide about a proposition grows \idx{exponentially} with the
number of variables.  For a proposition with just 30 variables, that's
already over a billion lines to check!

Is there a more \index{efficient solution} efficient solution
to SAT?  In particular, is there some, presumably very ingenious,
procedure that determines in a number of steps that grows
\emph{polynomially}---like $n^2$ or $n^{14}$---instead of
exponentially, whether any given proposition is satisfiable or not?
No one knows.  And an awful lot hangs on the answer.  An efficient
solution to SAT would immediately imply efficient solutions to many,
many other important problems involving packing, scheduling, routing,
and circuit verification, among other things.  This would be
wonderful, but there would also be worldwide chaos.  Decrypting coded
messages would also become an easy task (for most codes).  Online
financial transactions would be insecure and secret communications
could be read by everyone.

Recently there has been exciting progress on \term{sat-solvers} for
practical applications like digital circuit verification.  These
programs find satisfying assignments with amazing efficiency even for
formulas with millions of variables.  Unfortunately, it's hard to
predict which kind of formulas are amenable to sat-solver methods, and
for formulas that are \emph{un}satisfiable, sat-solvers generally take
exponential time to verify that no assignment works.

So no one has a good idea how to solve SAT in polynomial time, or how
to prove that it can't be done---researchers are completely stuck.
The problem of determining whether or not SAT has a polynomial time
solution is known as the ``\textbf{P} vs.\ \textbf{NP}'' problem.  It
is the outstanding unanswered question in theoretical computer
science.  It is also one of the
seven \href{http://www.claymath.org/millennium/}{Millenium Problems}:
the Clay Institute will award you \$1,000,000 if you solve
the \textbf{P} vs.\ \textbf{NP} problem.

\arm{ SAT vs VALID}
There is a close relation between satisfiability and validity, namely a
formula is satisfiable iff its negation is \emph{not} valid (see
Problem~\ref{CP_valid_vs_satisfiable}).  It follows that there is polynomial time
procedure for SAT iff there is there is polynomial time
procedure for validity.

\begin{problems}
\classproblems
\pinput{CP_valid_vs_satisfiable}
\end{problems}

\problemsection

\begin{editingnotes}
ARM, put this into a problem in the problem section at end of chapter:

In fact, it's not hard to show that \emph{no} polynomial
with integer coefficients can map all natural numbers into prime
numbers, unless it's a constant.
\end{editingnotes}

\endinput
