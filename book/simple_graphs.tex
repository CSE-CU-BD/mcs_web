\chapter{Graph Theory}\label{chap:graph_theory}

Informally, a graph is a bunch of dots and lines where the lines
connect some pairs of dots.  An example is shown in
Figure~\ref{fig:graph-example}.  The dots are called \emph{nodes} (or
\emph{vertices}) and the lines are called \emph{edges}.

\begin{figure}[h]

\graphic{graph-example}

\caption{An example of a graph with 9 nodes and 8 edges.}

\label{fig:graph-example}

\end{figure}

Graphs appear everywhere in computer science because they provide a handy
way to represent relationships between pairs of objects.  The objects
might be programs, people, cities, or web pages, and we connect two of
them with an edge if they are related in a certain way.  For example, an
edge between a pair of people might indicate that they like---or in
alternate scenarios that they don't like---each other.  An edge between a
pair of courses might indicate that they can't both be taken at the same
time.

In this chapter, we will focus on \emph{\idx{simple graph}s} where the
relationship denoted by an edge is \emph{\idx{symmetric}}, meaning that
the relationship is mutual and edges don't have a direction.  Then in
Chapter~\ref{chap:digraphs}, we consider \emph{\idx{directed graph}s}
where an edge denotes a one-way relationship, for example, where one web
page points to the other.\footnote{Two Stanford students became
  multibillionaires because of their analyses of such web-page graphs.  So
  pay attention to graph theory, and who knows what might happen!}

\section{Definitions}\label{degreessec}

\subsection{Simple Graphs}

\begin{definition}\label{graphdef}
\begin{editingnotes}
\arm{edited 7/29/10}
\end{editingnotes}
  A \term{simple graph}, $G$, consists of a nonempty set,~$\vertices{G}$,
  called the \term{vertices} of~$G$, and a set, $\edges{G}$, called the
  \term{edges} of $G$.  An element of $\vertices{G}$ is called a
  \term{vertex}.  A vertex is also called a \term{node}; the words
  ``vertex'' and ``node'' are used interchangeably.  An element of
  $\edges{G}$ must be a two-element subset of $\vertices{G}$ and is called
  an \term{edge}.  The vertices $u, v$ of edge $\set{u,v}$ are called the
  \term{endpoints} of the edge.
\end{definition}

For example, let $H$ be the graph pictured in
Figure~\ref{fig:graph-example}.  The vertices of $H$ correspond to the
nine dots in Figure~\ref{fig:graph-example}, that is,
\[
\vertices{H} =  \set{a, b, c, d, e, f, g, h, i}\, .
\]
The edges correspond to the eight lines, that is,
\[
\edges{H} =  \set{\, \set{a, b}, \set{a, c}, \set{b, d}, \set{c, d},
              \set{c, e}, \set{e, f}, \set{e, g}, \set{h, i} \,}.
\]
Mathematically, that's all there is to the graph $H$.
\begin{editingnotes}
  \arm{CUT}: So this $H$ has 9~nodes and 8~edges.
\end{editingnotes}

\begin{editingnotes}
\arm{a---b edge notation is back in}
\end{editingnotes}

There can be lots of two-element sets around, so to emphasize when a
two-element set is an edge, we'll use the notation $\edge{u}{v}$ for the
edge $\set{u,v}$.  For example, it's a bit clearer to describe the edges
of $H$ with this notation:
\[
\edges{H} =  \set{\, \edge{a}{b}, \edge{a}{c}, \edge{b}{d}, \edge{c}{d},
              \edge{c}{e}, \edge{e}{f}, \edge{e}{g}, \edge{h}{i} \,}.
\]
With this notation, edges $\edge{u}{v}$ and $\edge{v}{u}$ are actually the
same, since both are notations for the set $\set{u,v}$.

\begin{editingnotes}
\arm{CUT}: Note that $\edge{a}{b}$
  and $\edge{b}{a}$ are different descriptions of the same edge, since
  sets are unordered.
\end{editingnotes}

\begin{definition}
Two vertices in a simple graph are said to be \term{adjacent} if they
\begin{editingnotes}
\arm{CUT ``joined by''}
\end{editingnotes} are the endpoints of an edge, and an edge is
said to be \term{incident} to each of its endpoints.  The number of
edges incident to a vertex~$v$ is called the \term{degree} of the
vertex and is denoted by $\degr{v}$.  Equivalently, the degree of a
vertex is equals the number of vertices adjacent to it.
\end{definition}
For example, for the graph $H$ of Figure~\ref{fig:graph-example},
vertex~$a$ is adjacent to vertex~$b$, and $b$ is adjacent to~$d$.  The
edge $\edge{a}{c}$ is incident to vertices $a$ and~$c$.  Vertex~$h$
has degree~1, $d$ has degree~2, and $\degr{e} = 3$.  It is possible
for a vertex to have degree~0, in which case it is not adjacent to any
other vertices.  A simple graph, $G$, does not need to have any edges
at all, namely $\card{\edges{G}}$ could be zero,
\footnote{The size or \term{cardinality}, $\card{E}$, of a set,~$E$,
  is the number of elements in~$E$.}  which would implies that the
degree of every vertex is also zero.  But a simple graph must have at
least one vertex, that is, $\card{\vertices{G}}$ is required to be at
least one.

An arc going from a vertex back around to that same vertex, is called a
\term{self-loop}.  Self-loops aren't allowed in simple
graphs.\footnote{You might try to represent a self-loop going between a
  vertex, $v$, and itself as $\set{v, v}$, but this equals $\set{v}$, and
  wouldn't be an edge because an edge is defined to be set of \emph{two}
  vertices.}  In a more general class of graphs called \term{multigraphs}
there can be more than one edge with the same two endpoints, but this
doesn't happen in simple graphs since every edge is uniquely determined by
its two endpoints.
\begin{editingnotes} \arm{CUT edges between vertices $v$ and $w$ would
    equal the same set $\edge{v}{w}$.}
\end{editingnotes} 
Lastly, and most importantly, simple graphs do not
contain \emph{directed edges}, that is, edges, $\diredge{v}{w}$ that go
from a vertex $v$ to vertex $w$, but not backwards 
\begin{editingnotes}
\arm{CUT: the other way}
\end{editingnotes} from $w$ to $v$.  
\begin{editingnotes}
\arm{inserted}
\end{editingnotes} Graphs with directed
edges are the subject of Chapter~\ref{chap:digraphs}.

Sometimes graphs with no vertices, with self-loops, or with more than one
edge between the same two vertices are convenient to have, but we don't
need them, and sticking with simple graphs is simpler. \smiley

\emph{For the rest of this chapter we'll use ``graphs'' as an abbreviation
  for ``simple graphs.''}

\subsection{Some Common Graphs}

Some graphs come up so frequently that they have names.  A
\term{complete graph}, $K_n$, has $n$ vertices and an edge between
every two vertices, for a total of $n(n-1)/2$ edges.  For example,
a $K_5$ is shown in Figure~\ref{fig:K_5}.

\begin{figure}

\graphic{complete-graph}

\caption{$K_5$: a complete graph on 5 nodes.}
\label{fig:K_5}
\end{figure}

An \term{empty graph} has no edges at all.  For example, an empty
graph with 5 nodes is shown in Figure~\ref{fig:graph_empty_5}.

\begin{figure}

\graphic{empty-graph}

\caption{An empty graph with 5 nodes.}
\label{fig:graph_empty_5}
\end{figure}

An $n$-node graph containing $n - 1$ edges in sequence is known as
a \emph{line graph}~$L_n$.  More formally, $L_n$ has
\begin{equation*}
    \vertices{L_n} = \set{ v_1, v_2, \dots, v_n }
\end{equation*}
and
\begin{equation*}
    \edges{L_n} = \set{\, \edge{v_1}{v_2}, \edge{v_2}{v_3}, \dots,
    \edge{v_{n-1}}{v_n} \, }
\end{equation*}
For example, an $L_5$ is pictured in Figure~\ref{fig:graph_L_5}.

\begin{figure}

\graphic{path-graph}

\caption{$L_5$: a 5-node line graph.}

\label{fig:graph_L_5}

\end{figure}

There is also a one-way infinite line graph $L_{\infty}$ which can be
defined by letting the nonnegative integers $\naturals$ be the vertices
with edges $\edge{k,k+1}$ for all $k \in \naturals$.
$L_{\infty}$ is pictured in Figure~\ref{fig:graph_L_infty}.

If we add the edge $\edge{v_n}{v_1}$ to the line graph~$L_n$, we get a
graph called a \index{cycle!of length $n$}\term{length $n$ cycle}, \idx{$C_n$}.
Figure~\ref{fig:graph_C_5} shows a picture of length 5 cycle.

\begin{figure}

\graphic{cycle}

\caption{$C_5$: a 5-node cycle graph.}
\label{fig:graph_C_5}
\end{figure}

\subsection{Isomorphism}

Two graphs that look the same might actually be different in a formal
sense.  For example, the two graphs in Figure~\ref{fig:isomorphic-L4s}
\begin{editingnotes}
\arm{CUT}: Figure~\ref{fig:isomorphism} and use of
  $C_4$: bad example to have all vertices indistinguishable.  Insert a
  revised Figure ref{fig:isomorphic-L4s} of two 4-vertex
  line graphs with vertices abcd and 1234.
\end{editingnotes}
 are both line graphs with 4
  vertices, but one graph has vertex set $\set{a, b, c, d}$ while the other has vertex set $\set{1, 2, 3, 4}$.
\begin{figure}

\subfloat[]{%
    \graphic{isomorphism_a}
}
\qquad
\subfloat[]{%
    \graphic{isomorphism_b}
}

\caption{Isomorphic~$C_4$ graphs.}
\label{fig:isomorphism}
\end{figure}

\begin{editingnotes}
\arm{rephrased}
\end{editingnotes}
Strictly speaking, these graphs are different mathematical
objects, but this difference doesn't reflect the fact that the two graphs
can be pictured with exactly the \emph{same shape}: the
graphs are the same up to a ``relabeling'' of their vertices.  This idea
of ``same shape'' is elegantly captured by the notion of
\emph{isomorphism} between graphs.

For two graphs to be isomorphic, they first of all must have the same
number of vertices.  This makes it possible to have an exact
correspondence between their sets of vertices.  The technical word for an
``exact correspondence'' between two sets is \emph{bijection}:

\begin{definition}
  A \term{bijection} from a set $A$ to a set $B$ is a function, $f:A\to
  B$, such that for each element $a$ in set $A$ there is a unique element
  $f(a)$ in set $B$, and for each element $b$ in set $B$ there is a unique
  element $a$ in $A$ such that $b = f(a)$.
\end{definition}
A helpful way to think of a bijection is that if $a_1, a_2,\dots, a_n$ is
a list with no repeats of the elements in $A$, then $f(a_1), f(a_2),\dots,
f(a_n)$ must be a list with no repeats of all the elements in $B$.  We'll
focus on properties and further uses of bijections in
Chapter~\ref{chap:partial_orders}.

An exact correspondence between the vertices of two graphs that also
determines an exact correspondence between their edges is called an
\emph{isomorphism}:
\begin{definition}\label{simple-isomorphism}
An \term{isomorphism} between graphs $G$ and $H$ is a
  bijection $f$ from $\vertices{G}$ to
  $\vertices{H}$ such that for all $u, v \in \vertices{G}$:
\[
\edge{u}{v} \in \edges{G} \qiff \edge{f(u)}{f(v)} \in \edges{H}.
\]
Graphs $G$ and $H$ are \term{isomorphic} iff there exists an
isomorphism between them.
\end{definition}

An isomorphism between the two graphs shown in
Figure~\ref{fig:isomorphism} is easy to read off:
\[
\begin{array}{lll}
a \text{ corresponds to } 1 & \hspace{0.5in} & b \text{ corresponds to } 2 \\
d \text{ corresponds to } 4 & & c \text{ corresponds to } 3.
\end{array}
\]

\begin{editingnotes}
\arm{replaced:}
\begin{quote}
 You can check that there is an edge between two
  vertices in the graph on the left if and only if there is an edge
  between the two corresponding vertices in the graph on the right.
\end{quote}
by the following:
\end{editingnotes}

To see why this works, look at any edge in the first graph, say
$\edge{b}{c}$, and make sure that the vertices corresponding to $b$ and
$c$ are the endpoints of an edge in the second graph.  Namely, verify that
$\edge{2}{3}$ is an edge of the second graph; and it is.  Conversely, look
at any edge in the second graph, say $\edge{3}{4}$, and verify that the
corresponding vertices are the endpoints of an edge of the first
graph. Namely, verify that $\edge{c}{d}$ is an edge of the first graph;
and it is.  It's a good practice exercise to verify that every edge in
either of these graphs exactly corresponds in this way to an edge in the
other graph.

\begin{editingnotes}
\arm{ADDED:}
\end{editingnotes}
The definitions of bijection and isomorphism are meant to apply just as
well to infinite graphs as finite graphs, as do most of the results in the
rest of this chapter.  But graph theory focuses mostly on finite graphs,
and we will too.  So in the rest of this chapter we'll assume graphs are
finite, and you'll be spared the need to keep track of which results do
and which don't apply to infinite graphs.

\begin{editingnotes}
  So you won't have to worry about which results in this chapter
  apply to infinite as well as finite graphs.


\end{editingnotes}

\begin{editingnotes}
\arm{rephrased}
\end{editingnotes}

Two isomorphic graphs can be drawn to look the same, but
they don't have to be.  For example, two very different ways of drawing a
$C_5$ are shown Figure~\ref{fig:isomorphism-c5}.

\begin{figure}

\graphic{isomorphism-c5}
a
\caption{Two ways of drawing a length-5 cycle, $C_5$.}
\label{fig:isomorphism-c5}
\end{figure}

Isomorphism preserves the connection properties of a graph,
abstracting out what the vertices are called, what they are made out
of, or where they appear in a drawing of the graph.  More precisely, a
property of a graph is said to be \term{preserved under isomorphism}
if whenever $G$ has that property, every graph isomorphic to $G$ also
has that property.  For example, isomorphic graphs must have the same
number of vertices.  What's more, if $f$ is a graph isomorphism that
maps a vertex, $v$, of one graph to the vertex, $f(v)$, of an
isomorphic graph, then by definition of isomorphism, every vertex
adjacent to $v$ in the first graph will correspond to a vertex
adjacent to $f(v)$ in the isomorphic graph.  This means that $v$ and
$f(v)$ will have the same degree.  So if one graph has a vertex of
degree 4 and another does not, then they can't be isomorphic.  In
fact, they can't be isomorphic if the number of degree 4 vertices in
each of the graphs is not the same.

\begin{editingnotes}
\arm{ADDED:}
\end{editingnotes}
So far we've been careful to talk about ``\emph{a} $K_n$'' or
``\emph{an} $L_k$.''  From now on we'll drop the ``a'''s and ``an'''s
%articles
and just say, for example, that ``$K_n$'' has some property $P$.  This
means that $P$ is a graph property that is preserved under isomorphism,
and any graph isomorphic to $K_n$ has property $P$.

Looking for preserved properties can make it easy to determine that two
graphs are not isomorphic, or to guide the search for an
\begin{editingnotes}
\arm{rephrased}
\end{editingnotes}
isomorphism when there is one.  It's generally easy in practice to decide
whether two graphs are isomorphic.  However, no one has yet found a
procedure for determining whether two graphs are isomorphic that is
\emph{guaranteed} to run in \idx{polynomial time} on all pairs of
graphs.\footnote{A procedure runs in \emph{polynomial
    time} when it needs an amount of time of at most $ p(n)$, where $n$ is
  the total number of vertices and $p()$ is a fixed polynomial.}
\begin{editingnotes}
\arm{def of ptimes is rephrased, but shouldn't it be defined in an earlier chapter?}
\end{editingnotes}
Having such a procedure would be useful.  For example, it would make it
easy to search for a particular molecule in a database given the molecular
bonds.  On the other hand, knowing there is no such efficient procedure
would also be valuable: secure protocols for encryption and remote
authentication can be built on the hypothesis that graph isomorphism is
computationally exhausting.



\subsection{Subgraphs}

\begin{definition}\label{def:subgraph}
  A graph $G$ is said to be a \emph{subgraph} of a graph $H$ if
  $\vertices{G} \subseteq \vertices{H}$ and $\edges{G} \subseteq
  \edges{H}$.
\end{definition}

\begin{editingnotes}
\arm{rephrased}
\end{editingnotes}
For example, the one-edge graph $G$ where
\begin{equation*}
   \vertices{G} = \set{ g, h, i } \quad \text{and}\quad  \edges{G} =
   \set{\, \edge{h}{i} \, }
\end{equation*}
is a subgraph of the graph $H$ in Figure~\ref{fig:graph-example}.  On the
other hand, any graph containing an edge~$\edge{g}{h}$ will not be a
subgraph of $H$ because this edge is not in $\edges{H}$.  Another example
is an empty graph on $n$ nodes, which will be a subgraph of an~$L_n$ with
same set of nodes; similarly, $L_n$ is a subgraph of ~$C_n$, and ~$C_n$ is
a subgraph of ~$K_n$.

\begin{editingnotes}
\arm{CUT:}
Note that since a subgraph is itself a graph, the endpoints of any
edge in a subgraph must also be in the subgraph.  In other words if
$G'$ is a subgraph of some graph~$G$, and $\edge{v}{w} \in
\edges{G'}$, then it must be the case that $v$ and $w$ are vertices of
$G'$.
\end{editingnotes}

\subsection{Weighted Graphs}
\begin{editingnotes}
\arm{move these subsections}
This and the following subsec on adjacency matrices aren't needed until
subsec~\ref{num_walk_subsec}.  I suggest moving them there.

\arm{edited this whole next following subsection}
\end{editingnotes}

Sometimes we'll be interested in connections between nodes that have a
\emph{capacity} or \emph{weight}.  For example, we might be interested in
quantities such as the
\begin{itemize}

\item resistance of a wire between a pair of terminals, 

\item capacity of an Internet fiber between a pair of computers,

\item tension of a spring connecting a pair of devices in a dynamical system,

\item tension of a bond between a pair of atoms in a molecule,

\item distance of a highway between a pair of cities.

\end{itemize}
To model such cases, we associate with an edge a quantity called its
\emph{weight}.  More precisely,
\begin{definition}
  A \term{weight function} for a graph $G$ is a function $w: \edges{G} \to
  \reals$, where $w(e)$ is called the \term{weight of edge} $e$.
An \term{edge-weighted graph} consists of a simple graph along with
a weight function for the graph.
\end{definition}
We'll just say \term{weighted graph} when we mean
edge-weighted.\footnote{Vertex-weighted graphs can be defined similarly,
  but we won't need these.}
For example, Figure~\ref{fig:weighted_graph} shows a weighted graph
where the weight of edge $\edge{a}{b}$ is~5.

\begin{figure}

\graphic{Fig_5D}

\caption{A 4-node weighted graph where the edge~$\edge{a}{b}$ has
  weight~5.}
\label{fig:weighted_graph}
\end{figure}

\subsection{Adjacency Matrices}

\begin{editingnotes}
  \textcolor{red}{replaced by ARM with the next paragraph}: There are many
  ways to represent a graph.  We have already seen two ways: you can draw
  it, as in Figure~\ref{fig:weighted_graph} for example, or you can
  represent it with sets of vertices and edges.  Another common
  representation is with an adjacency matrix.
\end{editingnotes}

A useful way to specify a graph is with an \emph{adjacency matrix}.  The
adjacency matrix of an $n$-vertex graph $G$, is an $n \times n$ 0-1-valued
matrix whose $ij$th entry indicates whether there is an edge from the
$i$th to the $j$th vertex of $G$, where the vertices of $G$ are assumed to
be numbered from $1$ to $n$.  Sometimes it's more convenient to use the
vertices themselves to index the matrix, writing $A_{uv}$ instead of
$A_{ij}$ when $u$ and $v$ are the vertices numbered $i$ and $j$.  Several
formulas get simpler when we index the matrix entries this way.

\begin{definition}\label{def:adjacency_matrix}
The \term{adjacency matrix} for a graph~$G$ with
is the $n \by n$ matrix $A_G$  indexed by $\vertices{G}$ whose $uv$th entry is
\begin{equation*}
    (A_G)_{uv} \eqdef \begin{cases}
                1 & \text{if $\edge{u}{v} \in \edges{G}$}, \\
                0 & \text{otherwise.}
              \end{cases}
\end{equation*}
When $G$ is a weighted graph with weight function $w$, then the adjacency
matrix for~$G$ is the $n \by n$ matrix $A_G$ whose $uv$th entry is defined
to be:
\begin{equation*}
      (A_G)_{uv} \eqdef \begin{cases}
                w(\edge{u}{v}) & \text{if $\edge{u}{v} \in \edges{G}$}, \\
                \infty         & \text{otherwise.}
              \end{cases}
\end{equation*}
\end{definition}

For example, Figure~\ref{fig:adjacency_matrix}
\begin{editingnotes}
\arm{REVISED Fig needed}: weighted graph case 0's should be $\infty$'s.
\end{editingnotes}
displays the adjacency
matrices for the graphs shown in Figures~\ref{fig:isomorphism}(a)
and~\ref{fig:weighted_graph} where $v_1 = a$, $v_2 = b$, $v_3 = c$,
and $v_4 = d$.

\begin{figure}\redrawntrue
\normalbaselines
\subfloat[]{%
    $
       \begin{pmatrix}
           0 & 1 & 0 & 1 \\
           1 & 0 & 1 & 0 \\
           0 & 1 & 0 & 1 \\
           1 & 0 & 1 & 0
       \end{pmatrix}
   $
}
\qquad
\subfloat[]{%
   $
       \begin{pmatrix}
           \infty & 5 & \infty & \infty \\
           5 & \infty & 6 & \infty \\
           \infty & 6 & \infty & -3 \\
           \infty & \infty & -3 & \infty
       \end{pmatrix}
   $
}

\caption{Examples of adjacency matrices.  (a)~shows the adjacency
  matrix for the graph in Figure~\ref{fig:isomorphism}(a) and
  (b)~shows the adjacency matrix for the weighted graph in
  Figure~\ref{fig:weighted_graph}.  In each case, we  set $v_1
  = a$, $v_2 = b$, $v_3 = c$, and $v_4 = d$ to construct the matrix.}
\label{fig:adjacency_matrix}
\end{figure}

%% Simple Graphs Problems %%%%%%%%%%%%%%%%%%%%%%%%%%%%%%%%%%%%%%%%%%%%%%%%%%%%%
\begin{problems}
\classproblems
\pinput{CP_Handshaking_Lemma}
\pinput{CP_isomorphic_graphs}
% S09.cp6m.1
% S09.cp6m.3
% S09.cp6m.4

\homeworkproblems
\pinput{PS_choose_isomorphic_graphs}
\pinput{PS_neighbors_under_isomorphisms}
\pinput{PS_graph_two_ends}

\examproblems
\pinput{MQ_list_isomorphisms}
\end{problems}


\section{Matching Problems}\label{sexam}

For our first serious applications of graph theory we'll look at scenarios
where the nodes in a graph represent people and the edges represent a
relationship between pairs of people such as ``likes,'' ``marries,'' ``has
a common interest,'' and so on.  Now, you have good reason to wonder what
marriages have to do with computer science.  We'll explain how the
techniques we describe to find marriage compatible men and women are
also widely applied in practice to scenarios where applicants must be to
matched to jobs, packets matched to paths in a network, or Internet
traffic matched to web servers.  

To begin, we will show how graph theory can be used to debunk an urban
legend about sexual practices in America.  Yes, you read correctly.  So,
fasten your seat belt---who knew that math might actually be interesting!

\subsection{Sex in America}

On average, who has more opposite-gender partners: men or women?

Sexual demographics have been the subject of many studies.  In one of the
largest, researchers from the University of Chicago interviewed a random
sample of 2500 Americans over several years to try to get an answer to this
question.  Their study, published in 1994, and entitled \emph{The Social
  Organization of Sexuality} found that, on average, men have 74\% more
opposite-gender partners than women.

Other studies have found that the disparity is even larger.  In
particular, ABC News claimed that the average man has 20 partners over his
lifetime, and the average woman has 6, for a percentage disparity of
233\%.  The ABC News study, aired on Primetime Live in 2004, purported to
be one of the most scientific ever done, with only a 2.5\% margin of
error.  It was called ``American Sex Survey: A peek between the sheets.''
The promotion for the study is even better:
\begin{quote}
A ground breaking ABC News ``Primetime Live'' survey finds a range of
eye-popping sexual activities, fantasies and attitudes in this country,
confirming some conventional wisdom, exploding some myths---and venturing
where few scientific surveys have gone before.
\end{quote}
Probably that last part about going where few scientific surveys have gone
before is pretty accurate!

Yet again, in August, 2007, the N.Y. Times reported
%% \href{The-Myth-the-Math-the-Sex.pdf}{reported}
%% \iffalse
%% \href{http://www.nytimes.com/2007/08/12/weekinreview/12kolata.html?_r=1&n=Top/Reference/Times%20Topics/People/K/Kolata,%20Gina&oref=slogin}{reported}
%% \fi
on a study by the National Center for Health Statistics of the
U.S. Government showing that men had seven partners while women had
four.

Anyway, whose numbers do you think are more accurate, the University
of Chicago, ABC News, or the National Center for Health
Statistics?---don't answer; this is a setup question like ``When did
you stop beating your wife?''  Using a little graph theory, we will
now explain why none of these findings can be anywhere near the truth.

Let's model the question of heterosexual partners in graph theoretic
terms: we'll let $G$ be the graph whose vertices
are all the people in America.  Then we split $\vertices{G}$ into two separate
subsets: $M$, which contains all the males, and $F$, which contains
all the females.\footnote{For simplicity, we'll ignore the possibility
  of someone being both, or neither, a man and a woman.}  We'll put an
edge between a male and a female iff they have been sexual partners.
A possible subgraph of this graph is illustrated in
Figure~\ref{fig:partners} with males on the left and females on the
right.

\begin{figure}[htbp]

\graphic{sex-edges}

\caption{A possible subgraph of the sex partners graph.}
\label{fig:partners}
\end{figure}

Actually, $G$ is a pretty hard graph to figure out, let alone draw.  The
graph is \emph{enormous}: the US population is about 300 million, so
$\card{\vertices{G}} \approx 300M$.  In the United States, approximately
50.8\% of the population is female and 49.2\% is male, and so $\card{M}
\approx 147.6M$, and $\card{F} \approx 152.4M$.  And we don't even have
trustworthy estimates of how many edges there are, let alone exactly which
couples are adjacent.  But it turns out that we don't need to know any of
this to debunk the sex surveys---we just need to figure out the
relationship between the average number of partners per male and partners
per female.
\begin{editingnotes}
\arm{rephrased}
\end{editingnotes}
But you can see that every edge is incident
to exactly one $M$ vertex and one $F$ vertex (remember, we're only
considering male-female relationships); so the sum of the degrees of the
$M$ vertices equals the number of edges, and
\begin{editingnotes}
\arm{added likewise}
\end{editingnotes}
likewise
the sum of the degrees of the $F$ vertices equals the number of edges, so
the two sums are equal.  So the following equation is always true:
%
\[
\sum_{x \in M} \degr{x} = \sum_{y \in F} \degr{y}.
\]
%
If we divide both sides of this equation by the product of the sizes
of the two sets, $\card{M} \cdot \card{F}$, we obtain
%
\begin{equation}\label{eq:average-degree}
\left(\frac{\sum_{x \in M} \degr{x}}{\card{M}}\right) \cdot \frac{1}{\card{F}} =
\left(\frac{\sum_{y \in F} \degr{y}}{\card{F}}\right) \cdot \frac{1}{\card{M}}
\end{equation}
Notice that
\begin{equation*}
    \frac{\sum_{x \in M} \degr{x}}{\card{M}}
\end{equation*}
is simply the average degree of a node in~$M$.  This is the average
number of opposite-gender partners for a male in America.  Similarly,
\begin{equation*}
    \frac{\sum_{x \in F} \degr{x}}{\card{F}}
\end{equation*}
is the average degree of a node in~$F$, which is the average number of
opposite-gender partners for a female in America.  Hence,
Equation~\ref{eq:average-degree} implies that on average, an American
male has $\card{F}/\card{M}$ times as many opposite-gender partners as the
average American female.

From the Census Bureau reports, we know that there are slightly more
females than males in America; in particular $\card{F} / \card{M}$ is
about~1.035.  So we know that on average, males have 3.5\% more
opposite-gender partners than females.
\begin{editingnotes}
\arm{rephrased}
\end{editingnotes} It might seem
remarkable this statistic says nothing about any sex's promiscuity or
selectivity, but now you know why it doesn't.  All that the average
reflects is the ratio of men to women in the population.
\begin{editingnotes}
\arm{CUT}: Remarkably, promiscuity is completely irrelevant in this
  analysis.  That is because the ratio of the average number of
  partners is completely determined by the relative number of males
  and females.  Collectively, males and females have the same number
  of opposite gender partners, since it takes one of each set for
  every partnership, but there are fewer males, so they have a higher
  ratio.
\end{editingnotes}
This means that the University of Chicago, ABC, and the Federal
Government studies are way off.  After a huge effort, they gave a
totally wrong answer.

We don't have a definite explanation for why such surveys are consistently
wrong.  One hypothesis is that males exaggerate their number of
partners---or maybe females downplay theirs---but these explanations are
speculative.  Interestingly, the principal author of the National Center
for Health Statistics study reported that she knew the results had to be
wrong, but that was the data collected, and her job was to report it.

The same underlying issue has led to serious misinterpretations of other
survey data.  For example, a few years ago, the Boston Globe ran a story
on a survey of the study habits of students on Boston area campuses.
Their survey showed that on average, minority students tended to study
with non-minority students more than the other way around.  They went on
at great length to explain why this ``remarkable phenomenon'' might be
true.  But it's not remarkable at all---from the graph analysis above, you
can now see that all it says is that there are fewer minority students
than non-minority students, which is, of course, what ``minority'' means.

\subsubsection{The Handshaking Lemma}

The previous argument hinged on the connection between a sum of
degrees and the number edges.  There is a simple connection between
these quantities in any graph:
\begin{lemma}[The Handshaking Lemma]\label{sumedges}
The sum of the degrees of the vertices in a graph equals twice the
number of edges.
\end{lemma}

\begin{proof}
Every edge contributes two to the sum of the degrees, one for each of
its endpoints.
\end{proof}

Lemma~\ref{sumedges} is called the \term{Handshake Lemma} because if
we total up the number of people each person at a party shakes hands
with, the total will be twice the number of handshakes that occurred.

\subsection{Bipartite Matchings}\label{bipartitesec}

%\subsubsection*{Bipartite Graphs}\label{bipartitesubsec}

There were two kinds of vertices in the ``Sex in America''
graph---males and females, and edges only went between the two kinds.
Graphs like this come up so frequently that they have earned a special
name---they are called \emph{bipartite graphs}.

\begin{definition}
  A \term{bipartite graph} \index{graph!bipartite} is a graph whose vertices can be
  \idx{partition}ed\footnote{Partitioning a set means cutting it up into
    \emph{nonempty} pieces.  In this case, it means that $\leftbi{G}$ and
    $\rightbi{G}$ are nonempty, $\leftbi{G} \union \rightbi{G} =
    \vertices{G}$, and $\leftbi{G} \intersect \rightbi{G} =
    \emptyset$.} into two sets, $\leftbi{G}$ and
    $\rightbi{G}$, such that every edge is incident to a vertex in
    $\leftbi{G}$ and to a vertex in $\rightbi{G}$.
\end{definition}

\begin{editingnotes}
  \arm{so 1-vertex graphs are not bipartite, but all other empty graphs are.}
\end{editingnotes}

So every bipartite graph looks something like the graph in
Figure~\ref{fig:partners}.

\subsubsection{The Bipartite Matching Problem}

The bipartite matching problem is related to the sex-in-America
problem that we just studied; only now the goal is to get everyone
happily married.  As you might imagine, this is not possible for a
variety of reasons, not the least of which is the fact that there are
more women in America than men.  So, it is simply not possible to
marry every woman to a man so that every man is married at most once.

But what about getting a mate for every man so that every woman is married
at most once?  Is it possible to do this so that each man is paired with a
woman that he likes?  The answer, of course, depends on the bipartite graph
that represents who likes who, but the good news is that it is possible to
find natural properties of the who-likes-who graph that completely
determine the answer to this question.

In general, suppose that we have a set of men and an equal-sized or
larger set of women, and there is a graph with an edge between a man
and a woman if the man likes the woman.  In this scenario,
the ``likes'' relationship need not be symmetric, since for the time
being, we will only worry about finding a mate for each man that he
likes.\footnote{By the way, we do not mean to imply that marriage
  should or should not be of a heterosexual nature.  Nor do we mean to
  imply that men should get their choice instead of women.  It's just
  that with bipartite graphs, the edges only connected male nodes to
  female nodes and there are fewer men in America.  So please don't
  take offense.}  (Later, we will consider the ``likes'' relationship
from the female perspective as well.)  For example, we might obtain
the graph in Figure~\ref{fig:5J}.

\begin{figure}

\gnote{Tom: We couldn't figure out what was intended here: these edges
  are inconsistent with the text.}

\graphic{hall-graph}

\caption{A graph where an edge between a man and woman denotes that
  the man likes the woman.}

\label{fig:5J}

\end{figure}

In this problem, a \term{matching} will mean a way of assigning every
man to a woman so that different men are assigned to different women,
and a man is always assigned to a woman that he likes.  For example,
one possible matching for the men is shown in Figure~\ref{fig:5K}.

\begin{figure}

\gnote{Tom: We couldn't figure out what was intended here: these edges
  are inconsistent with the text.  David: Check line widths.}

\graphic{hall-graph-matched}

\caption{One possible matching for the men is shown with bold edges.
  For example, John is matched with Jane.}

\label{fig:5K}

\end{figure}

\subsubsection{The Matching Condition}

A famous result known as \idx{Hall's Matching Theorem} gives necessary
and sufficient conditions for the existence of a matching in a
bipartite graph.  It turns out to be a remarkably useful mathematical
tool.

We'll state and prove Hall's Theorem using man-likes-woman
terminology.  Define \emph{the set of women liked by a given set of
  men} to consist of all women liked by at least one of those men.
For example, the set of women liked by Tom and John in
Figure~\ref{fig:5J} consists of Martha, Sarah, and Mergatroid.  For us
to have any chance at all of matching up the men, the following
\term{matching condition} must hold:

\medskip

\noindent\text{\emph{The Matching Condition}: every subset of men
  likes at least as large a set of women.}

\medskip

For example, we can not find a matching if some set of 4~men like only
3~women.  Hall's Theorem says that this necessary condition is
actually sufficient; if the matching condition holds, then a matching
exists.

\begin{theorem}\label{thm:matching}
  A matching for a set~$M$ of men with a set~$W$ of women can be found if
  and only if the matching condition holds.
\end{theorem}

\begin{proof}
  First, let's suppose that a matching exists and show that the matching
  condition holds.  For any subset of men, each man likes at least the
  woman he is matched with and a woman is matched with at most one man.
  Therefore, every subset of men likes at least as large a set of women.
  Thus, the matching condition holds.

Next, let's suppose that the matching condition holds and show that a
matching exists.  We use strong induction on $\card{M}$, the number of
men, on the predicate:
\begin{align*}
    P(m) \eqdef & \text{for any set~$M$ of $m$ men, if the matching
      condition holds} \\
             & \text{for~$M$, then there is a matching for~$M$.}
\end{align*}

\inductioncase{Base case} ($\card{M}=1$): If $\card{M} = 1$, then the
matching condition implies that the lone man likes at least one woman,
and so a matching exists.

\inductioncase{Inductive Step:} We need to show that $P(m) \QIMPLIES
P(m + 1)$.  Suppose that $\card{M} = m + 1 \ge 2$.
\begin{description}

\item[Case 1:] Every \idx{proper subset}\footnote{A subset $A$ of~$B$ is
    \emph{proper} if $A \ne B$, see Section~\ref{subsubsec: subset}.} of
  men likes a \emph{strictly larger} set of women.  In this case, we have
  some latitude: we pair an arbitrary man with a woman he likes and send
  them both away.  The matching condition still holds for the remaining
  men and women since we have removed only one woman, so we can match the
  rest of the men by induction.

\item[Case 2:] Some proper subset of men $X \subset M$ likes an
  \emph{equal-size} set of women $Y \subset W$.  We match the men in
  $X$ with the women in $Y$ by induction and send them all away.  We
  can also match the rest of the men by induction if we show that the
  matching condition holds for the remaining men and women.  To check
  the matching condition for the remaining people, consider an
  arbitrary subset of the remaining men $X' \subseteq (M - X)$, and
  let $Y'$ be the set of remaining women that they like.  We must show
  that $\card{X'} \leq \card{Y'}$.  Originally, the combined set of
  men $X \cup X'$ liked the set of women $Y \cup Y'$.  So, by the
  matching condition, we know:
%
  \begin{equation*}
  \card{X \cup X'}  \leq  \card{Y \cup Y'}
  \end{equation*}
%
  We sent away $\card{X}$ men from the set on the left (leaving $X'$)
  and sent away an equal number of women from the set on the right
  (leaving $Y'$).  Therefore, it must be that $\card{X'} \leq
  \card{Y'}$ as claimed.
\end{description}

So in both cases, there is a matching for the men, which completes the
proof of the Inductive step.  The theorem follows by induction.
\end{proof}

The proof of Theorem~\ref{thm:matching} gives an algorithm for finding
a matching in a bipartite graph, albeit not a very efficient one.
However, efficient algorithms for finding a matching in a bipartite
graph do exist.  Thus, if a problem can be reduced to finding a
matching, the problem is essentially solved from a computational
perspective.

\subsubsection{A Formal Statement}

Let's restate Theorem~\ref{thm:matching} in abstract terms so that
you'll not always be condemned to saying, ``Now this group of men
likes at least as many women\dots''

\begin{definition}\label{def:5K}
\begin{editingnotes}
\textcolor{red}{edited ARM}:
\end{editingnotes}
\index{graph!matching} A \term{matching} in a graph $G$ is a set $M$ of
edges of $G$ such that no vertex is incident to more than one edge in $M$.
A matching is said to \term{cover} \index{edge cover} \index{set!covering}
a set, $S$, of vertices iff each vertex in $S$ is incident to an edge of
the matching.  A matching is said to be \emph{perfect} \index{perfect
  graph}\index{graph!perfect} if it covers $\vertices{G}$.  In any graph,
the set $N(S)$ of \term{neighbors} of some set $S$ of vertices is the set
of all vertices adjacent to some vertex in $S$.  That is,
\[
N(S) \eqdef \set{\,r \suchthat \edge{s}{r}\text{ is an edge of $G$ for
    some } s \in S\,}.
\]
$S$ is called a \term{bottleneck} if
\[
\card{S} > \card{N(S)}.
\]
\end{definition}

\begin{theorem}[\idx{Hall's Theorem}]\label{thm:halls}
  Let $G$ be a \idx{bipartite graph}.  There is matching in $G$ that
  covers $\leftbi{G}$ iff no subset of $\leftbi{G}$ is a bottleneck.
\end{theorem}

\subsubsection{An Easy Matching Condition}

The bipartite matching condition requires that \emph{every} subset of
men has a certain property.  In general, verifying that every subset
has some property, even if it's easy to check any particular subset
for the property, quickly becomes overwhelming because the number of
subsets of even relatively small sets is enormous---over a billion
subsets for a set of size 30.  However, there is a simple property of
vertex degrees in a bipartite graph that guarantees the existence of a
matching.  Namely, call a bipartite graph \term*{degree-constrained}
if vertex degrees on the left are at least as large as those on the
right.  More precisely,

\begin{editingnotes}
\arm{subsection revised}:
\end{editingnotes}

\begin{definition}\label{degree-constrained_def}
  A bipartite graph $G$ is \index{bipartite
    graph!degree-constrained}\term{degree-constrained} when $\degr{l} \geq
  \degr{r}$ for every $l \in \leftbi{G}$ and $r \in \rightbi{G}$.
\end{definition}

For example, the graph in Figure~\ref{fig:5J} is degree constrained
since every node on the left is adjacent to at least two nodes on the
right while every node on the right is incident to at most two nodes
on the left.

\begin{theorem}\label{lem:no-bottleneck}
  If $G$ is a degree-constrained bipartite graph, then there is a matching
  that covers~$\leftbi{G}$.
\end{theorem}

\begin{proof}
  The proof is by contradiction.  Suppose that $G$ is degree constrained
  but that there is no matching that covers~$\leftbi{G}$.  By
  Theorem~\ref{thm:halls}, this means that there must be a bottleneck $S
  \subseteq \leftbi{G}$.

  Let $d$ be a value such that $\degr{l} \ge d \ge \degr{r}$ for every $l
  \in L$ and $r \in R$.  Since every edge incident to some node in~$S$ is
  by definition incident to a node in $N(S)$, and every node in $N(S)$ is
  incident to at most $d$ edges, we know that
\[
d\card{N(S)} \geq \#\text{edges incident to a vertex in $S$}.
\]
Also, since every node in~$S$ is incident to at most $d$ edges
\[
\#\text{edges incident to a vertex in $S$} \geq d\card{S}.
\]
It follows that $d\card{N(S)} \ge d\card{S}$ and so
\begin{equation*}
    \card{N(S)} \ge \card{S}.
\end{equation*}
This means that $S$ is not a bottleneck, which is a contradiction.
Hence $G$ has a matching that covers~$\leftbi{G}$.
\end{proof}

Regular graphs are a large class of degree constrained graphs that often
arise in practice.  Hence, we can use Theorem~\ref{lem:no-bottleneck} to
prove that every regular bipartite graph has a perfect matching.  This
turns out to be a surprisingly useful result in computer science.

\begin{definition}\label{def:5P}
A graph is said to be \emph{regular} if every node has the same degree.
\end{definition}

\begin{theorem}\label{thm:5M}
Every regular bipartite graph has a perfect matching.
\end{theorem}

\begin{proof}
  Let $G$ be a regular bipartite graph.  Since regular graphs are
  degree-constrained, we know by Theorem~\ref{lem:no-bottleneck} that
  there must be a matching in~$G$ that covers~$\leftbi{G}$.  Such a
  matching is only possible when $\card{\leftbi{G}} \geq
  \card{\rightbi{G}}$.  But $G$ is also degree-constrained if the roles of
  $\leftbi{G}$ and $\rightbi{G}$ are switched, which implies that
  $\card{\rightbi{G}} \geq \card{\leftbi{G}}$ also.  That is, $\leftbi{G}$
  and $\rightbi{G}$ are the same size, and any matching covering
  $\leftbi{G}$ will also cover $\rightbi{G}$.  So every node in~$G$ is
  incident to an edge in the matching, and thus $G$ has a perfect
  matching.
\end{proof}

\subsection{The Stable Marriage Problem}
\label{stablemarriagesec}

We next consider a version of the bipartite matching problem where
there are an equal number of men and women, and where each person has
preferences about who they would like to marry.  In fact, we assume
that each man has a complete list of all the women ranked according
to his preferences, with no ties.  Likewise, each woman has a ranked
list of all of the men.

The preferences don't have to be symmetric.  That is, Jennifer might
like Brad best, but Brad doesn't necessarily like Jennifer best.  The
goal is to marry everyone: every man must marry exactly one woman and
vice-versa---no polygamy.  Moreover, we would like to find a matching
between men and women that is \emph{stable} in the sense that there is
no pair of people that prefer each other to their spouses.

For example, suppose \emph{every} man likes Angelina best, and every
woman likes Brad best, but Brad and Angelina are married to other
people, say Jennifer and Billy Bob.  Now \emph{Brad and Angelina
  prefer each other to their spouses}, which puts their marriages at
risk: pretty soon, they're likely to start spending late nights
together working on problem sets!

This unfortunate situation is illustrated in
Figure~\ref{fig:minWtMatch2}, where the digits ``1'' and ``2'' near a
man shows which of the two women he ranks first second, respectively,
and similarly for the women.

\begin{figure}

\graphic{minWtMatch2}

\caption{Preferences for four people.  Both men like Angelina best and
both women like Brad best.}
\label{fig:minWtMatch2}
\end{figure}

More generally, in any matching, a man and woman who are not married
to each other and who like each other better than their spouses, is
called a \emph{rogue couple}.  In the situation shown in
Figure~\ref{fig:minWtMatch2}, Brad and Angelina would be a rogue
couple.

Having a rogue couple is not a good thing, since it threatens the
stability of the marriages.  On the other hand, if there are no rogue
couples, then for any man and woman who are not married to each other,
at least one likes their spouse better than the other, and so they
won't be tempted to start an affair.

\begin{definition}
  A \term{stable matching} is a matching with no rogue couples.
\end{definition}

The question is, given everybody's preferences, how do you find a
stable set of marriages?  In the example consisting solely of the four
people in Figure~\ref{fig:minWtMatch2}, we could let Brad and Angelina
both have their first choices by marrying each other.  Now neither
Brad nor Angelina prefers anybody else to their spouse, so neither
will be in a rogue couple.  This leaves Jen not-so-happily married to
Billy Bob, but neither Jen nor Billy Bob can entice somebody else to
marry them, and so there is a stable matching.

Surprisingly, there always is a stable matching among a group of men
and women.  The surprise springs in part from considering the
apparently similar ``buddy'' matching problem.  That is, if people can
be paired off as buddies, regardless of gender, then a stable matching
\emph{may not} be possible.  For example, Figure~\ref{fig:buddy} shows
a situation with a love triangle and a fourth person who is everyone's
last choice.  In this figure Mergatroid's preferences aren't shown
because they don't even matter.  Let's see why there is no stable
matching.

\begin{figure}[htbp]

\graphic{loveTriangle}

\caption{Some preferences with no stable buddy matching.}
\label{fig:buddy}
\end{figure}

\begin{lemma}\label{lem:nostablematch}
There is no stable buddy matching among the four people in
Figure~\ref{fig:buddy}.
\end{lemma}

\begin{proof}
We'll prove this by contradiction.

Assume, for the purposes of contradiction, that there is a stable
matching.  Then there are two members of the love triangle that are
matched.  Since preferences in the triangle are symmetric, we may assume
in particular, that Robin and Alex are matched.  Then the other pair must
be Bobby-Joe matched with Mergatroid.

But then there is a rogue couple: Alex likes Bobby-Joe best, and Bobby-Joe
prefers Alex to his buddy Mergatroid.  That is, Alex and Bobby-Joe are a
rogue couple, contradicting the assumed stability of the matching.
\end{proof}

So getting a stable \emph{buddy} matching may not only be hard, it may
be impossible.  But when mens are only allowed to marry women, and
vice versa, then it turns out that a stable matching can always be
found.\footnote{Once again, we disclaim any political statement
  here---its just the way that the math works out.}

%%insert rest of story from gusfield book pp3--4??

\iffalse

\subsection{Failed attempts}

Let's find a stable matching in one possible situation, and hope to
translate our method to a general algorithm.  The table below shows the
preferences of each woman and man in decreasing order.

\begin{eqnarray*}
men & \quad & women \\
1 : C B E A D & \quad & A : 3 5 2 1 4 \\
2 : A B E C D & \quad & B : 5 2 1 4 3 \\
3 : D C B A E & \quad & C : 4 3 5 1 2 \\
4 : A C D B E & \quad & D : 1 2 3 4 5 \\
5 : A B D E C & \quad & E : 2 3 4 1 5
\end{eqnarray*}

How about we try a ``greedy'' strategy?\footnote{``Greedy'' is not any
moral judgment.  It refers to algorithms that work by always choosing the
next state that makes the largest immediate progress.}  We simply take
each man in turn and pack him off with his favorite among the women still
available.  This gives the following assignment.

\begin{eqnarray*}
1 \rightarrow C \\
2 \rightarrow A \\
3 \rightarrow D \\
4 \rightarrow B \\
5 \rightarrow E \\
\end{eqnarray*}

To determine whether this matching is stable, we have to check whether
there are any rogue couples.  Men 1, 2, and 3 all got their top pick
among the women; none would even think of running off.  Man~4 may be a
problem because he likes woman $A$ better than his mate, but she ranks him
dead last.  However, man~4 also likes woman $C$ better than his mate, and
she rates him above her own mate.  Therefore, man~4 and woman $C$ form a
rogue couple!  The marriages are not stable.  We could try to make ad hoc
repairs, but we're really trying to develop a general strategy.

Another approach would be to use induction.  Suppose we pair Man~1
with his favorite woman, $C$, try to show that neither of these two
will be involved in a rogue couple, and then solve the remaining
problem by induction.  Clearly Man~1 will never leave his top pick,
Woman $C$.  But the problem with this approach is that we \emph{can't}
be sure that Woman $C$ won't be in a rogue couple.  Woman $C$ might very
well dump Man~1---she might even rate him last!

This turns out to be a tricky problem.  The best approach is to use a
mating ritual that is reputed to have been popular in some mythic past.
\fi

\subsubsection{The Mating Ritual}

The procedure for finding a stable matching involves a \emph{Mating
Ritual} that takes place over several days.  The following events happen
each day:

\textbf{Morning}: Each woman stands on her balcony.  Each man stands
under the balcony of his favorite among the women on his list, and he
serenades her.  If a man has no women left on his list, he stays home
and does his math homework.

\textbf{Afternoon}: Each woman who has one or more suitors serenading
her, says to her favorite among them, ``We might get engaged.  Come
back tomorrow.''  To the other suitors, she says, ``No.  I will never
marry you!  Take a hike!''

\textbf{Evening}: Any man who is told by a woman to take a hike,
crosses that woman off his list.

\textbf{Termination condition}: When a day arrives in which every
woman has at most one suitor, the ritual ends with each woman marrying
her suitor, if she has one.

% Show example

There are a number of facts about this Mating Ritual that we would like to
prove:

\begin{itemize}
\item The Ritual eventually reaches the termination condition.
\item Everybody ends up married.
\item The resulting marriages are stable.
\end{itemize}


\subsubsection{There is a Marriage Day}

It's easy to see why the Mating Ritual has a terminal day when people
finally get married.  Every day on which the ritual hasn't terminated, at
least one man crosses a woman off his list.  (If the ritual hasn't
terminated, there must be some woman serenaded by at least two men, and at
least one of them will have to cross her off his list).  If we start with
$n$ men and $n$ women, then each of the $n$ men's lists initially has $n$
women on it, for a total of $n^2$ list entries.  Since no women ever gets
added to a list, the total number of entries on the lists decreases every
day that the Ritual continues, and so the Ritual can continue for at most
$n^2$ days.

\subsubsection{They All Live Happily Every After\dots}

We still have to prove that the Mating Ritual leaves everyone in a
stable marriage.  To do this, we note one very useful fact about the
Ritual: if a woman has a favorite suitor on some morning of the
Ritual, then that favorite suitor will still be serenading her the
next morning---because his list won't have changed.  So she is sure to
have today's favorite man among her suitors tomorrow.  That means she
will be able to choose a favorite suitor tomorrow who is at least as
desirable to her as today's favorite.  So day by day, her favorite
suitor can stay the same or get better, never worse.  This sounds like
an invariant, and it is.

\begin{definition}\label{def:P8}
Let $P$ be the predicate: For every woman, $w$, and every man, $m$, if
$w$ is crossed off $m$'s list, then $w$ has a suitor whom she prefers
over~$m$.
\end{definition}

\begin{lemma}\label{lem:5P}
$P$ is an invariant for The Mating Ritual.
\end{lemma}

\begin{proof}
By induction on the number of days.

\inductioncase{Base case}: In the beginning---that is, at the end of
day~0---every woman is on every list.  So no one has been crossed off, and
$P$ is vacuously true.

\inductioncase{Inductive Step}: Assume $P$ is true at the end of
day~$d$ and let $w$ be a woman that has been crossed off a man $m$'s
list by the end of day~$d + 1$.

\begin{description}

\item[Case 1:]
$w$ was crossed off $m$'s list on day $d + 1$.  Then, $w$ must have a
  suitor she prefers on day~$d+1$.

\item[Case 2:]
$w$ was crossed off $m$'s list prior to day~$d+1$.  Since $P$ is true
  at the end of day~$d$, this means that $w$ has a suitor she prefers
  to~$m$ on day~$d$.  She therefore has the same suitor or someone she
  prefers better at the end of day~$d + 1$.

\end{description}
In both cases, $P$ is true at the end of day~$d + 1$ and so $P$ must
be an invariant.
\end{proof}

With Lemma~\ref{lem:5P} in hand, we can now prove:

\begin{theorem}
Everyone is married by the Mating Ritual.
\end{theorem}

\begin{proof}
By contradiction. Assume that it is the last day of the Mating Ritual
and someone does not get married.  Since there are an equal number of
men and women, and since bigamy is not allowed, this means that at
least one man (call him Bob) and at least one woman do not get
married.

Since Bob is not married, he can't be serenading anybody and so his
list must be empty.  This means that Bob has crossed every woman off
his list and so, by invariant~$P$, every woman has a suitor whom she
prefers to Bob.  Since it is the last day and every woman still has a
suitor, this means that every woman gets married.  This is a
contradiction since we already argued that at least one woman is
\emph{not} married.  Hence our assumption must be false and so
everyone must be married.
\end{proof}

\begin{theorem}
The Mating Ritual produces a stable matching.
\end{theorem}

\begin{proof}
Let Brad and Jen be any man and woman, respectively, that are
\emph{not} married to each other on the last day of the Mating Ritual.
We will prove that Brad and Jen are not a rogue couple, and thus that
all marriages on the last day are stable.  There are two cases to consider.
\begin{description}

\item[Case 1:] Jen is not on Brad's list by the end.  Then by invariant
  $P$, we know that Jen has a suitor (and hence a husband) that she
  prefers to Brad.  So she's not going to run off with Brad---Brad and
  Jen cannot be a rogue couple.

\item[Case 2:] Jen is on Brad's list.  But since Brad is not married to
  Jen, he must be choosing to serenade his wife instead of Jen, so he
  must prefer his wife.  So he's not going to run off with Jen---once
  again, Brad and Jenn are not a rogue couple.
 \qedhere

\end{description}

\end{proof}


\subsubsection{\dots Especially the Men}

Who is favored by the Mating Ritual, the men or the women?  The women
\emph{seem} to have all the power: they stand on their balconies
choosing the finest among their suitors and spurning the rest.  What's
more, we know their suitors can only change for the better as the
Ritual progresses.  Similarly, a man keeps serenading the woman he
most prefers among those on his list until he must cross her off, at
which point he serenades the next most preferred woman on his list.  So
from the man's perspective, the woman he is serenading can only change
for the worse.  Sounds like a good deal for the women.

But it's not!  The fact is that from the beginning, the men are
serenading their first choice woman, and the desirability of the woman
being serenaded decreases only enough to ensure overall stability.
The Mating Ritual actually does as well as possible for all the men
and does the worst possible job for the women.

To explain all this we need some definitions.  Let's begin by
observing that while The Mating Ritual produces one stable matching,
there may be other stable matchings among the same set of men and
women.  For example, reversing the roles of men and women will often
yield a different stable matching among them.

But some spouses might be out of the question in all possible stable
matchings.  For example, given the preferences shown in
Figure~\ref{fig:minWtMatch2}, Brad is just not in the realm of
possibility for Jennifer, since if you ever pair them, Brad and
Angelina will form a rogue couple.

\begin{definition}
Given a set of preference lists for all men and women, one person is
in another person's \emph{realm of possible spouses} if there is a
stable matching in which the two people are married.  A person's
\term{optimal spouse} is their most preferred person within their
realm of possibility.  A person's \term{pessimal spouse} is their
least preferred person in their realm of possibility.
\end{definition}

Everybody has an optimal and a pessimal spouse, since we know there is at
least one stable matching, namely, the one produced by the Mating Ritual.
Now here is the shocking truth about the Mating Ritual:

\begin{theorem}\label{boyopt}
The Mating Ritual marries every man to his optimal spouse.
\end{theorem}

\begin{proof}
By contradiction.  Assume for the purpose of contradiction that some
man does not get his optimal spouse.  Then there must have been a day
when he crossed off his optimal spouse---otherwise he would still be
serenading (and would ultimately marry) her or some even more
desirable woman.

By the Well Ordering Principle, there must be a \emph{first} day when
a man (call him ``Keith'') crosses off his optimal spouse (call her
Nicole).
According to the rules of the Ritual, Keith crosses off Nicole because
Nicole has a preferred suitor (call him Tom), so
\begin{equation}
\text{Nicole prefers Tom to Keith.} \tag{$*$}
\end{equation}

Since this is the first day an optimal woman gets crossed off, we know
that Tom had not previously crossed off his optimal spouse, and so
\begin{equation}\tag{$**$}
\text{Tom ranks Nicole at least as high as his optimal spouse.}
\end{equation}
By the definition of an optimal spouse, there must be some stable set
of marriages in which Keith gets his optimal spouse, Nicole.  But then
the preferences given in~($*$) and~($**$) imply that Nicole and Tom
are a rogue couple within this supposedly stable set of marriages
(think about it).  This is a contradiction.
\end{proof}

\begin{theorem}
The Mating Ritual marries every woman to her pessimal spouse.
\end{theorem}

\begin{proof}
By contradiction.  Assume that the theorem is not true.  Hence there
must be a stable set of marriages~$\mathcal{M}$ where some woman (call
her Nicole) is married to a man (call him Tom) that she likes less
than her spouse in The Mating Ritual (call him Keith).  This means
that
\begin{equation}
\text{Nicole prefers Keith to Tom.} \tag{+}
\end{equation}

By Theorem~\ref{boyopt} and the fact that Nicole and Keith are married
in the Mating Ritual, we know that 
\begin{equation}\tag{++}
\text{Keith prefers Nicole to his spouse in~$\mathcal{M}$.}
\end{equation}
This means that Keith and Nicole form a rogue couple in~$\mathcal{M}$,
which contradicts the stability of~$\mathcal{M}$.
\end{proof}

\subsubsection{Applications}

The Mating Ritual was first announced in a paper by D. \idx{Gale} and
L.S. \idx{Shapley} in 1962, but ten years before the Gale-Shapley
paper was published, and unknown by them, a similar algorithm was
being used to assign residents to hospitals by the National Resident
Matching Program (NRMP)\footnote{Of course, there is no serenading
  going on in the hospitals---the preferences are submitted to a
  program and the whole process is carried out by a computer.}.  The
NRMP has, since the turn of the twentieth century, assigned each
year's pool of medical school graduates to hospital residencies
(formerly called ``internships'') with hospitals and graduates playing
the roles of men and women.  (In this case, there may be multiple
women married to one man, a scenario we consider in the problem
section at the end of the chapter.).  Before the Ritual-like algorithm
was adopted, there were chronic disruptions and awkward
countermeasures taken to preserve assignments of graduates to
residencies.  The Ritual resolved these problems so successfully, that
it was used essentially without change at least through
1989.\footnote{Much more about the Stable Marriage Problem can be
  found in the very readable mathematical monograph by Dan Gusfield
  and Robert W. Irving,
  \href{http://mitpress.mit.edu/catalog/item/default.asp?ttype=2&tid=7676}{The
    Stable Marriage Problem: Structure and Algorithms}, MIT Press,
  Cambridge, Massachusetts, 1989, 240 pp.}

The Internet infrastructure company, Akamai, also uses a variation of
the Mating Ritual to assign web traffic to its servers.  In the early
days, Akamai used other combinatorial optimization algorithms that got
to be too slow as the number of servers (over 65,000 in 2010) and
requests (over 800 billion per day) increased.  Akamai switched to a
Ritual-like approach since it is fast and can be run in a distributed
manner.  In this case, web requests correspond to women and web
servers correspond to men.  The web requests have preferences based on
latency and packet loss, and the web servers have preferences based on
cost of bandwidth and colocation.

Not surprisingly, the Mating Ritual is also used by at least one large
online dating agency.  Even here, there is no serenading going
on---everything is handled by computer.

\begin{problems}

\practiceproblems

\pinput{CP_mating_ritual_example}

\pinput{TP_mating_ritual_invariant}

\classproblems

\pinput{CP_mating_ritual_proof}

\pinput{CP_stable_matching_non_optimal}

\homeworkproblems

\pinput{PS_stable_matching_hospitals}

\pinput{PS_stable_matching_no_first_choice}

\pinput{PS_stable_matching_unlucky}

\begin{editingnotes}
\begin{problem*}
Add problem proving that the Mating Ritual need not proceed in
morning/afternoon/evening lock step: a woman can reject non-favorite
suitors one at a time and at any time, and a rejected man can change
the woman he serenades without waiting for the other men to change.
The proof uses the fact that single actions commute, so induction
proves that all executions are confluent---which implies all
executions end with the same man-optimal matching.  This lemma can be
cited in the planar graphs section to prove that the edges in an
embedding can be added in any order.
\end{problem*}
\end{editingnotes}

\end{problems}

\section{Coloring}\label{sec:coloring}

\begin{editingnotes}
\arm{reworded}:
\end{editingnotes}

In Section~\ref{sexam}, we used edges to indicate an affinity between a
pair of nodes.  But there are lots of situations where edges will
correspond to \emph{conflicts} between nodes.  Exam scheduling is a
typical example.

\subsection{An Exam Scheduling Problem}

Each term, the MIT Schedules Office must assign a time slot for each
final exam.  This is not easy, because some students are taking
several classes with finals, and (even at MIT) a student can take only
one test during a particular time slot.  The Schedules Office wants to
avoid all conflicts.  Of course, you can make such a schedule by
having every exam in a different slot, but then you would need
hundreds of slots for the hundreds of courses, and the exam period
would run all year!  So, the Schedules Office would also like to keep
exam period short.

The Schedules Office's problem is easy to describe as a graph.  There
will be a vertex for each course with a final exam, and two vertices
will be adjacent exactly when some student is taking both courses.
For example, suppose we need to schedule exams for 6.041, 6.042,
6.002, 6.003 and 6.170.  The scheduling graph might appear as in
Figure~\ref{fig:5R}.

\begin{figure}

\graphic{finals-subject-labels}

\caption{A scheduling graph for five exams.  Exams connected by an
  edge cannot be given at the same time.}

\label{fig:5R}

\end{figure}

6.002 and 6.042 cannot have an exam at the same time since there are
students in both courses, so there is an edge between their nodes.  On the
other hand, 6.042 and 6.170 can have an exam at the same time if they're
taught at the same time (which they sometimes are), since no student can
be enrolled in both (that is, no student \emph{should} be enrolled in both
when they have a timing conflict).

We next identify each time slot with a color.  For example, Monday
morning is red, Monday afternoon is blue, Tuesday morning is green,
etc.  Assigning an exam to a time slot is then equivalent to coloring
the corresponding vertex.  The main constraint is that \emph{adjacent
  vertices must get different colors}---otherwise, some student has
two exams at the same time.  Furthermore, in order to keep the exam
period short, we should try to color all the vertices using as
\emph{few different colors as possible}.  As shown in Figure~\ref{fig:5S},
three colors suffice for our example.

\begin{figure}

\graphic{finals-subject-colored}

\caption{A 3-coloring of the exam graph from Figure~\ref{fig:5R}.}

\label{fig:5S}

\end{figure}

The coloring in Figure~\ref{fig:5S} corresponds to giving one final on
Monday morning (red), two Monday afternoon (blue), and two Tuesday
morning (green).  Can we use fewer than three colors?  No! We can't
use only two colors since there is a triangle in the graph, and three
vertices in a triangle must all have different colors.

This is an example of a \term{graph coloring} problem:
\index{graph!coloring problem} given a graph $G$, assign colors to each
node such that adjacent nodes have different colors.  A color assignment
with this property is called a \term{valid coloring} \index{graph!valid
  coloring} of the graph---a ``\term{coloring},'' for short.  A graph $G$
is $k$-\term{colorable} if it has a coloring that uses at most $k$ colors.
\begin{definition}
  The minimum value of $k$ for which a graph, $G$, has a valid coloring is
  called its \term{chromatic number}, $\chi(G)$.
\end{definition}
\begin{editingnotes}
\arm{added:}
\end{editingnotes}
So $G$ is $k$-colorable iff $\chi(G) \leq k$.

In general, trying to figure out if you can color a graph with a fixed
number of colors can take a long time.  It's a classic example of a
problem for which no fast algorithms are known.  In fact, it is easy to
check if a coloring works, but it seems really hard to find it. (If you
figure out how, then you can get a \$1 million Clay prize.)


\subsection{Some Coloring Bounds}

There are some simple properties of graphs that give useful bounds on
colorability. 
\begin{editingnotes}
\arm{inserted:}
\end{editingnotes} The simplest property is being a \idx{cycle}: an even-length
closed cycle is 2-colorable, and since by definition it must have some
edges, it is not 1-colorable.  So
\[
\chi(C_{\text{even}}) = 2.
\]
On the other hand, an odd-length cycle requires 3 colors, that is,
\begin{equation}\label{Codd3}
\chi(C_{\text{odd}}) = 3.
\end{equation}
You should take a moment to think about why this equality holds.
\begin{editingnotes}
\arm{FTL proof yanked from Theorem~\ref{thm:2-colorable-equiv}}

Let $G$ be a 2-colorable graph and
\begin{equation*}
    \mathbf{w} \eqdef v_0, v_1, \dots, v_k
\end{equation*}
be any closed walk in~$G$.  So in any 2-coloring of~$G$, consecutive
vertices $v_i$ and $v_{i + 1}$ must be colored differently since
$\edge{v_i}{v_{i + 1}} \in \edges{G}$.  \arm{cut: for $0 \le i < k$}
Hence $v_0$, $v_2$, $v_4$, \dots, have one color and $v_1$, $v_3$,
$v_5$, \dots, have the other color.  Since $\mathbf{w}$ is a closed
walk, $v_k$ is the same node as~$v_0$, and so $k$ must be an even
number.  This means that $\mathbf{w}$ has even length.
\end{editingnotes}
Another simple example is a complete graph $K_n$:
\[
\chi(K_n) = n
\]
since no two vertices can have the same color.

Being bipartite is another property closely related to colorability.  If a
graph is bipartite, then you can color it with 2 colors usiing one color
for the nodes on the ``left'' and a second color for the nodes on the
``right.''  Conversely, graphs with chromatic number 2 are all bipartite
with all the vertices of one color on the ``left'' and those with the
other color on the right.  The graphs with chromatic number 1 are the
graphs with no edges---the \emph{\idx{empty graph}s}.  Empty graphs are
bipartite as long they have at least two vertices: a graph with only one
vertex is not bipartite because its vertex set cannot be partitioned into
two \emph{nonempty} subsets.  In short:
\begin{lemma}\label{2color-iff-bip}
A graph with more than one vertex is 2-colorable iff it is
bipartite.
\end{lemma}

Planarity is another property with important colorability consequences.
The famous 4-Color Theorem says that every planar graph is 4-colorable.
This is a hard result to prove, but we will come close in
Section~\ref{planar_graphs_sec} where we define planar graphs and prove
that they are 5-colorable.

The chromatic number of a graph can also be shown to be small if the
vertex degrees of the graph are small.  In particular, if we have an
upper bound on the degrees of all the vertices in a graph, then we can
easily find a coloring with only one more color than the degree bound.

\begin{theorem}\label{thm:k+1-colorable}
A graph with maximum degree at most $k$ is $(k+1)$-colorable.
\end{theorem}

Since $k$ is the only nonnegative integer valued variable mentioned in the
theorem, you might be tempted to try to prove this theorem using induction
on~$k$.  Unfortunately, this approach leads to disaster---we don't know of
any reasonable way to do this and expect it would ruin your week if you
tried it on a problem set.  When you encounter such a disaster using
induction on graphs, it is usually best to change what you are inducting
on.  In graphs, typical good choices for the induction parameter are $n$,
the number of nodes, or $e$, the number of edges.

\begin{proof}[Proof of Theorem~\ref{thm:k+1-colorable}]
We use induction on the number of vertices in the graph, which we
denote by $n$.  Let $P(n)$ be the proposition that an $n$-vertex graph
with maximum degree at most $k$ is $(k+1)$-colorable.

\inductioncase{Base case} ($n=1$): A 1-vertex graph has maximum degree
0 and is 1-colorable, so $P(1)$ is true.

\inductioncase{Inductive step}: Now assume that $P(n)$ is true, and
let $G$ be an $(n+1)$-vertex graph with maximum degree at most $k$.
Remove a vertex $v$ (and all edges incident to it), leaving an
$n$-vertex subgraph, $H$.  The maximum degree of $H$ is at most $k$,
and so $H$ is $(k+1)$-colorable by our assumption $P(n)$.  Now add
back vertex $v$.  We can assign $v$ a color (from the set of $k + 1$
colors) that is different from all its adjacent vertices, since there
are at most $k$ vertices adjacent to~$v$ and so at least one of the
$k+1$ colors is still available.  Therefore, $G$ is $(k+1)$-colorable.
This completes the inductive step, and the theorem follows by
induction.
\end{proof}

Sometimes $k+1$ colors is the best you can do.  For example, $\chi(K_n) = n$
and every node in $K_n$ has degree $k = n - 1$ and so this is an example where
Theorem~\ref{thm:k+1-colorable} gives the best possible bound.  By a
similar argument, we can show that Theorem~\ref{thm:k+1-colorable} gives
the best possible bound for \emph{any} graph with degree bounded by
$k$ that has $K_{k+1}$ as a subgraph.

But sometimes $k+1$ colors is far from the best that you can do.
For example, the $n$-node \emph{star graph} shown in
Figure~\ref{fig:5T} has maximum degree $n - 1$ but can be colored
using just 2 colors.

\begin{figure}

\graphic{star-graph}

\caption{A 7-node star graph.}

\label{fig:5T}

\end{figure}


\subsection{Why coloring?}

One reason coloring problems frequently arise in practice is because
scheduling conflicts are so common.  For example, at Akamai, a new
version of software is deployed over each of 65,000 servers every few
days.  The updates cannot be done at the same time since the servers
need to be taken down in order to deploy the software.  Also, the
servers cannot be handled one at a time, since it would take forever
to update them all (each one takes about an hour).  Moreover, certain
pairs of servers cannot be taken down at the same time since they have
common critical functions.  This problem was eventually solved by
making a 65,000-node conflict graph and coloring it with 8 colors---so
only 8 waves of install are needed!

Another example comes from the need to assign frequencies to radio
stations.  If two stations have an overlap in their broadcast area, they
can't be given the same frequency.  Frequencies are precious and
expensive, so you want to minimize the number handed out.  This amounts to
finding the minimum coloring for a graph whose vertices are the stations
and whose edges connect stations with overlapping areas.

Coloring also comes up in allocating registers for program variables.
While a variable is in use, its value needs to be saved in a register.
Registers can be reused for different variables but two variables need
different registers if they are referenced during overlapping
intervals of program execution.  So register allocation is the
coloring problem for a graph whose vertices are the variables:
vertices are adjacent if their intervals overlap, and the colors are
registers.  Once again, the goal is to minimize the number of colors
needed to color the graph.

Finally, there's the famous map coloring problem stated in
Proposition~\ref{4colorprop}.  The question is how many colors are needed
to color a map so that adjacent territories get different colors?  This is
the same as the number of colors needed to color a graph that can be drawn
in the plane without edges crossing.  A proof that four colors are enough
for \index{planar graphs} \emph{planar} graphs was acclaimed when it was
discovered about thirty years ago.  Implicit in that proof was a
4-coloring procedure that takes time proportional to the number of
vertices in the graph (countries in the map).  Surprisingly, it's another
of those million dollar prize questions to find an efficient procedure to
tell if a planar graph really \emph{needs} four colors or if three will
actually do the job.  (It is easy to tell if an \emph{arbitrary} graph is
2-colorable, as explained in Section~\ref{subsec:odd_cycles}.)  In
Section~\ref{planar_graphs_sec}, we'll develop enough planar graph theory
to present an easy proof that all planar graphs are 5-colorable.

%% Connectedness %%%%%%%%%%%%%%%%%%%%%%%%%%%%%%%%%%%%%%%%%%%%%%%%%%%%%%%%%%%%%%

\section{Getting from $u$ to~$v$ in a Graph}\label{sec:connectedness}
\begin{editingnotes}
\arm{major edits throughout this section}
\end{editingnotes}
When a graph serves as a map, with vertices representing local attractions
and edges representing trails between attractions, it's natural to look
for ways to \emph{walk} along the trails to get from one attraction to
another.  Sometimes it's nice to meander, but taking a walk that does not
cross itself would generally be the shorter route between attractions;
such walks that don't cross themselves are called \emph{paths}.  This
section contains precise definitions of walks and paths and their
properties, including how to find shortest walks between attractions.

\subsection{Paths and Walks}

\begin{definition}\label{def:undirected-path}
A \term{walk}\footnote{Some texts use the word \emph{path} for our
  definition of walk and the term \emph{simple path} for our
  definition of path.} in a graph, $G$, is a sequence of vertices
\begin{equation*}
\walkv{w} \eqdef v_0, v_1, \dots, v_k
\end{equation*}
and edges
\begin{equation*}
    \edge{v_0}{v_1}, \edge{v_1}{v_2}, \dots, \edge{v_{k - 1}}{v_k}
\end{equation*}
of $G$, where $0 \leq k$.
\begin{editingnotes}
  \arm{CUT}: ``such that $\edge{v_i}{v_{i+1}}
  \in \edges{G}$ for all $i$ where $0 \leq i < k$'' and changed $<$ to
  $\leq$ to allow length zero walks and paths
\end{editingnotes}
The walk is said to \index{walk!start of}\emph{start} at $v_0$ and to
\index{walk!end of}\emph{end} at $v_k$, and the \index{length of walk}
\index{walk!length of}\emph{length} $\abs{\walkv{w}}$ of the walk is
defined to be $k$.  An edge, $\edge{u}{v}$, is \term{traversed} $n$ times
by the walk if it occurs $n$ times in the sequence of edges.  A
\emph{path} is a walk in which no vertex appears more than once in the
sequence of vertices.
\end{definition}
\begin{editingnotes}
\arm{We use the edge sequence only to define ``traversal.''.  I still
  think we should drop it from the def of walk, adding a sentence to
  footnote below about its being in the def of walk in most graph
  theory texts.}
\end{editingnotes}

\begin{editingnotes}
  \arm{CUT:} ``different values of $i$ such that $\edge{v_i}{v_{i+1}} =
  \edge{u}{v}$'' This was needed for the def of walk as a sequence of
  vertices instead of edges.
\end{editingnotes}

\begin{editingnotes}
\arm{rephrased:}
\end{editingnotes}

The vertex sequence of a walk or path uniquely determines its edge
sequence, so we often refer to walks or paths as though they were just a
sequence of vertices.\footnote{In a graph with multiple edges, the vertex
  sequence of a walk wouldn't necessarily determine its edge sequence.}
For example, the graph in Figure~\ref{dg} has a length~6 path $a$, $b$,
$c$, $d$, $e$, $f$,~$g$.  This is the longest path in the graph.  Of
course, the graph has walks with arbitrarily large lengths, for example,
$a$, $b$, $a$, $b$, $a$, $b$,\dots.

\begin{figure}[htbp]

\graphic{distance-graph}

\caption{A graph containing a path $a$, $b$, $c$, $d$, $e$, $f$,~$g$
  of length~6.}
\label{dg}
\end{figure}

The length of a walk is the total number of consecutive edges it
traverses, which is \emph{one less} than its length as a sequence of
vertices.  For example, $a$, $b$, $c$, $d$, $e$, $f$,~$g$ is a sequence of
7~vertices but defines a path of length~6.

If you walk for a while, stop for a rest at some vertex $v$, and then
continue walking, you have broken a walk into two parts---the first
part $\walkv{f}$ that ends at $v$ and the rest $\walkv{r}$ that begins
at $v$ and continues to the end---and we'll refer to the whole walk as
the $v$-\emph{merge} of $\walkv{f}$ and $\walkv{r}$.  It's tempting
say the $v$-\emph{merge} is the \idx{concatentation} $\walkv{f}\,
\walkv{r}$ of the two walks, but that wouldn't quite be
right---$\walkv{f}\, \walkv{r}$ isn't even a walk, because $v$ appears
twice in a row where the walks meet.  Here's a precise definition:
\begin{definition}
If a walk $\walkv{f}$ ends at a vertex $v$ and a $\walkv{r}$ begins at
the same vertex $v$, then the \term{$v$-merge} of $\walkv{f}$ with
$\walkv{r}$, written,
\[
\catv{\walkv{f}}{v}{\walkv{r}},
\]
is the walk whose vertex sequence is the vertex sequence of
$\walkv{f}$ concatenated with the vertex sequence of $\walkv{r}$
without its initial $v$.  That is if
\begin{align*}
\walkv{r} & = v\,\vec{\alpha},
\end{align*}
for some finite sequence $\vec{\alpha}$ of vertices,
then
\[
\catv{\walkv{f}}{v}{\walkv{r}} \eqdef  \walkv{f}\alpha.
\]
\end{definition}
We'll get some useful mileage out of this walk notation. \smiley

\begin{editingnotes}
The sequences $\vec{\alpha}$ and $\vec{\beta}$ might be empty; otherwise
we could have said they were walks.
\end{editingnotes}

\subsection{Shortest Walks}
\begin{editingnotes}
\arm{retitled}
\end{editingnotes}

If you were trying to walk somewhere quickly, you'd know you were in
trouble if you came to the same place twice.  This is actually a basic
theorem of graph theory.

\begin{editingnotes}
  \arm{CUT:} Where there's a walk, there's a path.  This is sort of
  obvious, but it's easy enough to prove rigorously using the \idx{Well
    Ordering Principle}.
\end{editingnotes}

\begin{theorem}\label{simplepath}
The shortest walk between a pair of vertices is a path.
\end{theorem}

\begin{proof}
  If there is a walk from vertex $u$ to $v$, there must, by the
  Well-ordering Principle, be a minimum length walk $\walkv{w}$ from $u$
  to~$v$.  We claim $\walkv{w}$ is a path.

  To prove the claim, suppose to the contrary that $\walkv{w}$ is not a
  path, namely, some vertex $x$ occurs twice on this walk.  That is,
\[
\walkv{w} = \catv{\catv{\walkv{e}}{x}{\walkv{f}}}{x}{\walkv{g}}
\]
for some walks $\walkv{e}, \walkv{f}, \walkv{g}$ where the length of
$\walkv{f}$ is positive.  But then deleting $\walkv{f}$ yields is a
strictly shorter walk
\[
\catv{\walkv{e}}{x}{\walkv{g}}
\]
from $u$ to $v$, contradicting the minimality of $\walkv{w}$.
\end{proof}

\begin{editingnotes}
Distance properties below \arm{inserted by arm}
\end{editingnotes}

\begin{definition}
  The \index{distance!between vertices} \emph{distance} $\dstuv{u}{v}$
  between vertices $u,v$ in a graph is the length of a shortest walk
  from $u$ to $v$.
\end{definition}

\begin{lemma}\label{lem:tri-ineq} [The Triangle Inequality]
\[
\dstuv{u}{v} \leq \dstuv{u}{x} + \dstuv{x}{v}
\]
for all vertices $u,v,x$ with equality holding iff $x$ is on a shortest
path from $u$ to $v$.
\end{lemma}

\begin{editingnotes}
  Maybe leave all or part the following proof to a problem?
\end{editingnotes}

\begin{proof}
  To prove the inequality, suppose $\walkv{f}$ is a shortest path from $u$
  to $x$ and $\walkv{r}$ is a shortest path from $x$ to $v$.  Then
  $\catv{\walkv{f}}{x}{\walkv{r}}$ is a path of length $\dstuv{u}{x} +
  \dstuv{x}{v}$ from $u$ to $v$, so this sum is an upper bound on the
  length of the shortest path from $u$ to $v$.

\begin{editingnotes}
\arm{might come out shorter revised to be chain of iff's:}
\end{editingnotes}

  To prove the ``iff'' from left to right, suppose $\dstuv{u}{v} =
  \dstuv{u}{x} + \dstuv{x}{v}$.  Then taking a shortest path from $u$ to
  $x$ followed by a shortest path from $x$ to $w$ yields a path of whose
  length is $\dstuv{u}{x} + \dstuv{x}{v}$ which by assumption equals
  $\dstuv{u}{v}$. So this is a shortest path containing $x$.

  To prove the ``iff'' from right to left, suppose vertex $x$ is on a
  shortest path $\walkv{w}$ from $u$ to $v$, namely, $\walkv{w}$ is a
  shortest path of the form $\catv{\walkv{f}}{x}{\walkv{r}}$.  The path
  $\walkv{f}$ must be a shortest path from $u$ to $x$; otherwise replacing
  $\walkv{f}$ by a shorter path would yield a shorter path from $u$ to $v$
  than $\walkv{w}$.  Likewise from $\walkv{r}$ must be a shortest path
  from $x$ to $v$.  So $\dstuv{u}{v} = \abs{\walkv{w}} = \abs{\walkv{f}} +
  \abs{\walkv{r}} = \dstuv{u}{x} + \dstuv{x}{v}$.
  
\end{proof}

\subsection{Path Matrices}\label{num_walk_subsec}

Given a pair of nodes that are connected by a walk of length~$k$ in a
graph, there are often many walks that can be used to get from one
node to the other.  For example, there are 5 walks of length~3 that
start at~$v_1$ and end at~$v_4$ in the graph shown in
Figure~\ref{fig:5AD}.

\begin{figure}

\graphic{Fig_5AD}

\caption{A graph for which there are 5 walks of length~3 from $v_1$
  to~$v_4$.  The walks are $(v_1, v_2, v_1, v_4)$, $(v_1, v_3, v_1,
  v_4)$, $(v_1, v_4, v_1, v_4)$, $(v_1, v_2, v_3, v_4)$, and $(v_1,
  v_4, v_3, v_4)$.}
\label{fig:5AD}
\end{figure}

\subsubsection{Path Counting Matrices}

There is a surprising relationship between the number of length~$k$
walks between a pair of nodes in a graph ~$G$  and the $k$th power of
the adjacency matrix~$A_G$ for~$G$.  The relationship is captured in the
following theorem.
\begin{editingnotes}
  \arm{suggested reorg} The following subsections about path length would
  fit better in the Chapter~\ref{chap:digraphs} on digraphs, since
  undirectedness of edges and symmetry of $A_G$ is irrelevant.
\end{editingnotes}

\begin{editingnotes}
\arm{revised a lot:}
\end{editingnotes}

\begin{definition}
  The length $k$ \term{path counting matrix} for an $n$-vertex graph $G$
  is the $n \times n$ matrix $C$ such that
\begin{equation}\label{def:path_matrix}
C_{uv} \eqdef
\begin{cases} 1 & \text{if there is a length $k$ path from vertex
                     $u$ to vertex $v$},\\
              0 & \text{otherwise}.
\end{cases}
\end{equation}
\end{definition}

So the length $1$ path counting matrix for $G$ is precisely its adjacency matrix
$A_G$.

\begin{theorem}\label{thm:CkDm}
 If $C$ is length $k$ path counting matrix for a graph $G$,
  and $D$ is the length $m$ path counting matrix, then $CD$ is the length
  $k+m$ path counting matrix for $G$.
\end{theorem}

According to this theorem, the square $(A_G)^2$ of the adjacency matrix is
the length 2 path counting matrix for $G$.  Applying the theorem again to
$(A_G)^2A_G$, shows that the length 3 path counting matrix is $(A_G)^3$,
and more generally $(A_G)^k$ is the length $k$ path counting matrix.  In
other words, you can determine the number of length~$k$ walks between any
pair of vertices simply by computing the $k$th power of the adjacency
matrix!  That's pretty amazing.

For example, the first three powers of the adjacency matrix for the
graph in Figure~\ref{fig:5AD} are:
\begin{align*}
    A &= \begin{pmatrix}
            0 & 1 & 1 & 1 \\
            1 & 0 & 1 & 0 \\
            1 & 1 & 0 & 1 \\
            1 & 0 & 1 & 0
         \end{pmatrix} & % \\[\medskipamount]
  A^2 &= \begin{pmatrix}
            3 & 1 & 2 & 1 \\
            1 & 2 & 1 & 2 \\
            2 & 1 & 3 & 1 \\
            1 & 2 & 1 & 2
         \end{pmatrix} & % \\[\medskipamount]
  A^3 &= \begin{pmatrix}
            4 & 5 & 5 & 5 \\
            5 & 2 & 5 & 2 \\
            5 & 5 & 4 & 5 \\
            5 & 2 & 5 & 2
         \end{pmatrix}
\end{align*}

Sure enough, $(A^3)_{14}$ is $5$, which is the number of length~3 walks
from~$v_1$ to~$v_4$.  And $(A^3)_{24} = 2$, which is the number of
length~3 walks from $v_2$ to~$v_4$.  By proving the theorem, we'll
discover why it is true and thereby uncover the relationship between
matrix multiplication and numbers of walks.

\begin{editingnotes}
\arm{new proof replacing induction proof by ftl}
\end{editingnotes}

\begin{proof}[Proof of Theorem~\ref{thm:CkDm}]
  Any length $k+m$ path between vertices $u$ and $v$ begins with a
  length $k$ path starting at $u$ and ending at some vertex,
  $w$, followed by a length $m$ path starting at $w$ and ending at $v$.  So
  the number of length $k+m$ paths from $u$ to $v$ that go through
  $w$ at the $k$th step equals the number $P_{uw}$ of length $k$ paths
  from $u$ to $w$, times the number $Q_{wv}$ of length $m$ paths from $w$
  to $v$.  We can get the total number of length $k+m$ paths from $u$
  to $v$ by summing, over all possible vertices $w$, the number of such
  paths that go through $w$ at the $n$th step.  In other words,
\begin{equation}\label{ln+nuv}
\text{\# length $k+m$ paths from $u$ and $v$} =
              \sum_{w \in \vertices{G}} C_{uw}\cdot D_{wv}
\end{equation}
But the right hand side of~\eqref{ln+nuv} is precisely the definition of
$(CD)_{uv}$.  Thus, $CD$ is indeed the length $k+m$ path counting matrix.
\end{proof}



\subsubsection{Minimum Weight Path  Matrices}
The relation between powers of the adjacency matrix and numbers of
walks is cool (to us math nerds at least),
\begin{editingnotes}
\arm{math nerds? \smiley}
\end{editingnotes}
but a much more important
problem is finding \index{graph!shortest path} shortest paths between
pairs of nodes in a graph.  For example, when you drive home for vacation,
you generally want to take the shortest-time route.  It turns out that
shortest paths---even in weighted graphs---can be determined in pretty
much the same way that numbers of paths were counted using powers of the
connection matrix.

\begin{definition}\label{def:5H}
  The \index{path!weight of}\term{weight of a walk} \index{weighted graph,
    path weight} in a \idx{weighted graph} is the sum of the weights of
  the successive edges in the walk.
\end{definition}

\begin{editingnotes}
\arm{cut}
There is good news and bad news to report on this front.  The good
news is that it is not very hard to find a shortest path.  The bad
news is that you can't win one of those million dollar prizes for
doing it.

In fact, there are several good algorithms known for finding a shortest
path between nodes $u$ and $v$ in an $n$-node graph $G$.  The simplest to
explain (but not quite the fastest) is to compute \arm{revised to include
  stopping condition at $n$} the successive powers of $A_G$ one by one up
to the $n$th, watching for the first power at which the $uv$th entry is
nonzero.  That's because Theorem~\ref{thm:CkDm} implies that the length of
the shortest path, if any, between $u$ and~$v$ will be the smallest
value~$k$ for which $(A_G)_{uv}^k$ is nonzero, and if there is a shortest
path, its length will be $\leq n$.
\end{editingnotes}

\begin{definition}
  The \term{minimum weight matrix} for length $k$ walks in an $n$-vertex
  graph $G$ is the $n \times n$ matrix $W$ such that for $u,v \in \vertices{G}$,
\begin{equation}\label{def:path_matrix}
W_{uv} \eqdef
\begin{cases} w & \text{if $w$ is the minimum weight among length $k$
                            walks from $u$ to $v$},\\
              \infty & \text{if there is no length $k$ walk from $u$ to $v$}.
\end{cases}
\end{equation}
\end{definition}

So the minimum weight matrix for length $1$ walks in a weighted graph $G$
is precisely its adjacency matrix $A_G$.  Now for the purpose of finding
minimum weight walks, we'll modify the definition of matrix
multiplication, replacing multiplication of elements by addition, and the
sum of the products by the minimum.

\begin{definition}\label{def:minplus}
  The $\minplus$ product of two $n\times n$ matrices $W$ and $M$ with
  entries in $\reals\union \set{\infty}$ is the $n \times n$ matrix
  $(W\minplusop M)$ whose $ij$ entry is
\[
(W\minplusop M )_{ij} \eqdef \min \set{W_{ik} + M_{kj} \suchthat 1 \leq k \leq n}\, .
\]
\end{definition}

\begin{theorem}\label{thm:weightmatrix}
  If $W$ is the minimum weight matrix for length $k$ walks in a weighted
  graph $G$, and $M$ is the minimum weight matrix for length $m$ walks,
  then then $(W\minplusop M)$ is the minimum weight matrix for length $k+m$
  walks.
\end{theorem}

\begin{proof}
  The proof is virtually the same as the proof of Theorem~\ref{thm:CkDm}
  with multiplication of elements replaced by addition, and the sum of the
  multiplications by the minimum of the additions:

  Any length $k+m$ path between vertices $u$ and $v$ begins with a length
  $k$ path starting at $u$ and ending at some vertex, $x$, followed by a
  length $m$ path starting at $x$ and ending at $v$.  So the minimum
  weight of a length $k+m$ path from $u$ to $v$ that goes through $x$ at
  the $k$th step equals the minimum weight $W_{ux}$ of length $k$ walks
  from $u$ to $x$, plus the minimum weight $M_{xv}$ of length $m$ walks
  from $w$ to $v$.  So we can get the minimum weight of length $k+m$ walks
  from $u$ to $v$ by taking the minimum over all possible vertices $x$ of
  the minimum weight of such walks that go through $w$ at the $k$th step.
  In other words,
\begin{equation}\label{ln+nuv}
\text{min weight of a length $n+m$ path from $u$ to $v$} =
              \min_{x \in \vertices{G}} W_{ux}+M_{xv}\, .
\end{equation}
But the right hand side of~\eqref{ln+nuv} is precisely the definition of
$(W\minplusop M)_{uv}$.  Thus, $(W\minplusop M)$ is indeed the minimum weight
matrix for walks of length $k+m$.
\end{proof}

Now Theorem~\ref{thm:weightmatrix} implies that the $k$th $\minplus$ power
of $A_G$, that is,
\[
(A_G)^{k, \minplus} \eqdef \underbrace{\paren{A_G \minplusop \paren{A_G
      \minplusop \paren{\cdots \paren{A_G \minplusop A_G}} }}}_{k\ A_G\text{'s}},
\]
is the minimum weight matrix for the length $k$ walks.

This takes us most of the way, but we really want the minimum weight
regardless of the lengths of the walks.  To get this, we use the fact that
as long as all weights are \emph{nonnegative}, the minimum weight walk
between two vertices will be a path; this follows by the same reasoning
used for Theorem~\ref{simplepath}.  Since $n-1$ is the longest \emph{path}
can be in an $n$-node graph, we have an upper bound on the length of
minimum weight paths we have to look at.  We could now find the minimum
weight paths by computing all the $\minplus$ powers of $A_G$ up to the
$n-1$st.  But there is another trick that dramatically cuts the number of
$\minplus$ matrix multiplications.

Namely, for any graph $G$, let $G_0$ be the same as $G$ except that
self-loops of weight zero appear at every vertex.  So a path of length $k$
in $G$ can be extended to a path in $G_0$ with the same weight but with
any desired length $ \geq k$---just repeatedly follow weight zero
self-loops after the $k$th step.  This means that $(A_{G_0})^{k, \minplus}$
is the minimum weight matrix for walks of length \emph{less than or equal}
to $k$ in $G$.  So we can choose $k = n-1$ to get a matrix with the
actual minimum weights among all walks between vertices.

\begin{theorem}\label{thm:minweightmatrix}
Let $G$ be an $n$-vertex weighted graph with nonnegative weights, and let
$D_G$ be the adjacency matrix of $G$ with the diagonal entries set to 0.
Then $(D_G)^{n-1, \minplus}$ is the minimum weight path matrix for $G$, that
is,
\[
((D_G)^{n-1, \minplus})_{uv} = \text{the minimum weight of walks in $G$ from
 $u$ to $v$}\,.
\]
\end{theorem}
Now you can use the repeated squaring trick to compute $(D_G)^{n-1,
  \minplus}$ using about $\log n$ $\minplus$ matrix multiplications
instead of $n-2$ such multiplications.

\begin{editingnotes}
\arm{Good to add an example here}
\end{editingnotes}

\section{Connectivity}

\begin{definition}\label{def:connected-vertices}
  Two vertices in a graph are \term{connected} when there is a path
  that begins at one and ends at the other.  By convention, every
  vertex is connected to itself by a path of length zero.
\end{definition}

\begin{definition}\label{def:connected-graph}
A graph is \term{connected} when every pair of vertices are
connected.
\end{definition}

\subsection{Connected Components}

Being connected is usually a good property for a graph to have.  For
example, it could mean that it is possible to get from any node to any
other node, or that it is possible to communicate between any pair of
nodes, depending on the application.

But not all graphs are connected.  For example, the graph where nodes
represent cities and edges represent highways might be connected for
North American cities, but would surely not be connected if you also
included cities in Australia.  The same is true for communication
networks like the Internet---in order to be protected from viruses
that spread on the Internet, some government networks are completely
isolated from the Internet.

\begin{figure}[htbp]

\graphic{connectivity-graphs}

\caption{One graph with 3 connected components.}

\label{fig:3comp}
\end{figure}

Another example, is shown in Figure~\ref{fig:3comp}, which looks like a
picture of three graphs, but is intended to be a picture of \emph{one}
graph.  This graph consists of three pieces (subgraphs).  Each piece
by itself is connected, but there are no paths between vertices in
different pieces.  These connected pieces of a graph are called its
\term{connected components}.

\begin{definition}\label{def:connected-component}
A \emph{connected component} of a graph is a subgraph consisting of
some vertex and every node and edge that is connected to that vertex.
\end{definition}

So a graph is connected iff it has exactly one connected component.
At the other extreme, the empty graph on $n$ vertices has $n$
connected components.

\subsection{$k$-Connected Graphs}

If we think of a graph as modeling cables in a telephone network, or
oil pipelines, or electrical power lines, then we not only want
connectivity, but we want connectivity that survives component
failure.  So more generally we want to define how strongly two
vertices are connected.  Two vertices will be \term{$k$-edge
connected} when it takes at least $k$ ``edge-failures'' to disconnect
them..  More precisely:

\begin{definition}\label{def:k-connected}
Two vertices in a graph are $k$-\term{edge connected} when they remain
connected in every subgraph obtained by deleting $k-1$ edges.  A graph
is $k$-edge \index{connected!edge}\index{connected!$k$-edge}connected
when it is nonempty and every subgraph obtained by deleting
at most $k-1$ edges is connected.
\end{definition}
\iffalse
every two of its vertices are $k$-edge connected.
\fi

So two vertices or graphs are connected according to
Definition~\ref{def:connected-vertices} iff they are 1-edge connected
according to Definition~\ref{def:k-connected}.

There are other kinds of connectedness but edge-connectedness will be
enough for us From now on we'll drop the ``edge'' modifier and just
say ``\idx{connected}.''\footnote{There is An obvious definition of
  $k$-\idx{vertex connected}ness\index{connected!$k$-edge} based on
  deleting vertices rather than edges.  If you look at another graph
  theory text, be warned: texts usually use ``$k$-connected'' as
  shorthand for ``$k$-vertex connected.''}

For example, in the graph in Figure~\ref{dg}, vertices $c$ and~$e$ are
3 connected, $b$ and~$e$ are 2 connected, $g$ and $e$ are 1 connected,
and no vertices are 4 connected.  The graph as a whole is only 1
connected.  A complete graph, $K_n$, is $(n-1)$ connected.  Every
cycle is 2-connected.

The idea of a \emph{cut edge} is a useful way to explain 2-connectivity.
\begin{definition}
If two vertices are connected in a graph $G$, but not connected when
an edge $e$ is removed, then $e$ is called a \term{cut edge} of $G$.
\end{definition}
So $G$ is 2-connected iff it is nonempty, connected, and has no cut
edges.  The following Lemma is another immediate consequence of the
definition:
\begin{lemma}label{lem:cutiffcycle}
An edge is a cut edge iff it is not on a cycle.
\end{lemma}

More generally, if two vertices are connected by $k$ edge-disjoint
paths (that is, no two paths traverse the same edge), then they are
must $k$ connected, since at least one edge will have to be removed
from each of the paths.  A fundamental fact, whose ingenious proof we
omit, is \idx{Menger}'s theorem which confirms that the converse is
also true: if two vertices are $k$-connected, then there are $k$
edge-disjoint paths connecting them.  It even takes some ingenuity to
prove this for the case $k=2$.

\subsection{The Minimum Number of Edges in a Connected Graph}

The following theorem says that a graph with few edges must have many
connected components.
\begin{theorem}\label{th:connectivity}
Every graph with $v$ vertices and $e$ edges has at least $v - e$ connected
components.
\end{theorem}
Of course for Theorem~\ref{th:connectivity} to be of any use, there must
be fewer edges than vertices.

\begin{proof}
We use induction on the number of edges, $e$.  Let $P(e)$ be the
proposition that
\begin{quote}
for every $v$, every graph with $v$ vertices and $e$ edges has at least
$v-e$ connected components.
\end{quote}

\textbf{Base case:}($e=0$).  In a graph with 0 edges and $v$ vertices,
each vertex is itself a connected component, and so there are exactly $v =
v - 0$ connected components.  So $P(e)$ holds.

\textbf{Inductive step:} Now we assume that the induction hypothesis
holds for every $e$-edge graph in order to prove that it holds for
every $(e+1)$-edge graph, where $e \geq 0$.  Consider a graph, $G$,
with $e + 1$ edges and $v$ vertices.  We want to prove that $G$ has at
least $v - (e+1)$ connected components.  To do this, remove an
arbitrary edge $\edge{a}{b}$ and call the resulting graph $G'$.  By
the induction assumption, $G'$ has at least $v - e$ connected
components.  Now add back the edge $\edge{a}{b}$ to obtain the
original graph $G$.  If $a$ and $b$ were in the same connected
component of $G'$, then $G$ has the same connected components as $G'$,
so $G$ has at least $v -e > v - (e+1)$ components.  Otherwise, if $a$
and $b$ were in different connected components of $G'$, then these two
components are merged into one component in~$G$, but all other
components remain unchanged, reducing the number of components by 1.
Therefore, $G$ has at least $(v - e) - 1 = v - (e+1)$ connected
components.  So in either case, $P(e+1)$ holds.  This completes the
Inductive step.  The theorem now follows by induction.
\end{proof}

\begin{corollary}
\label{cor:n-1}
Every connected graph with $v$ vertices has at least $v - 1$ edges.
\end{corollary}

A couple of points about the proof of Theorem~\ref{th:connectivity}
are worth noticing.  First, we used induction on the number of edges
in the graph.  This is very common in proofs involving graphs, as is
induction on the number of vertices.  When you're presented with a
graph problem, these two approaches should be among the first you
consider.

The second point is more subtle.  Notice that in the inductive step,
we took an arbitrary $(n+1)$-edge graph, threw out an edge so that
we could apply the induction assumption, and then put the edge back.
You'll see this shrink-down, grow-back process very often in the
inductive steps of proofs related to graphs.  This might seem like
needless effort: why not start with an $n$-edge graph and add one
more to get an $(n+1)$-edge graph?  That would work fine in this
case, but opens the door to a nasty logical error called
\term{buildup error}.

\subsection{Build-Up Error}

\begin{falseclm*}
If every vertex in a graph has degree at least~1, then the graph is
connected.
\end{falseclm*}

There are many counterexamples; for example, see Figure~\ref{fig:5Z}.

\begin{figure}

\graphic{Fig_5Z}

\caption{A counterexample graph to the False Claim.}

\label{fig:5Z}
\end{figure}

\begin{bogusproof}
We use induction.  Let $P(n)$ be the proposition that if every vertex
in an $n$-vertex graph has degree at least~1, then the graph is
connected.

\inductioncase{Base case} ($n=1$): There is only one graph with a
single vertex and it has degree~0.  Therefore, $P(1)$ is vacuously true,
since the if-part is false.

\inductioncase{Inductive step}: We must show that $P(n)$ implies
$P(n+1)$ for all $n \ge 1$.  Consider an $n$-vertex graph in which
every vertex has degree at least~1.  By the assumption~$P(n)$, this
graph is connected; that is, there is a path between every pair of
vertices.  Now we add one more vertex~$x$ to obtain an $(n+1)$-vertex
graph as shown in Figure~\ref{fig:5Y}.

\begin{figure}

\graphic{false-connect-pic}

\caption{Adding a vertex~$x$ with degree at least~1 to a connected
  $n$-node graph.}

\label{fig:5Y}

\end{figure}

All that remains is to check that there is a path from $x$ to every
other vertex~$z$.  Since $x$ has degree at least one, there is an edge
from~$x$ to some other vertex; call it~$y$.  Thus, we can obtain a
path from~$x$ to~$z$ by adjoining the edge $\edge{x}{y}$ to the path
from~$y$ to~$z$.  This proves $P(n + 1)$.

By the principle of induction, $P(n)$ is true for all  $n \ge 1$,
which proves the theorem
\end{bogusproof}

Uh-oh\dots this proof looks fine!  Where is the bug?  It turns out
that the faulty assumption underlying this argument is that
\emph{every $(n + 1)$-vertex graph with minimum degree~1 can be
obtained from an $n$-vertex graph with minimum degree~1 by adding 1
more vertex}.  Instead of starting by considering an arbitrary $(n +
1)$-node graph, this proof only considered $(n + 1)$-node graphs
that you can make by starting with an $n$-node graph with minimum
degree~1.

The counterexample in Figure~\ref{fig:5Z} shows that this assumption
is false; there is no way to build the 4-vertex graph in
Figure~\ref{fig:5Z} from a 3-vertex graph with minimum degree~1.
Thus the first error in the proof is the statement ``This proves
$P(n + 1)$.''

This kind of flaw is known as ``build-up error.''  Usually, build-up
error arises from a faulty assumption that every size $n + 1$ graph
with some property can be ``built up'' from a size~$n$ graph with the
same property.  (This assumption is correct for some properties, but
incorrect for others---such as the one in the argument above.)

One way to avoid an accidental build-up error is to use a ``shrink
down, grow back'' process in the inductive step, namely, start with a
size $n+1$ graph, remove a vertex (or edge), apply the inductive
hypothesis $P(n)$ to the smaller graph, and then add back the vertex
(or edge) and argue that $P(n + 1)$ holds.  Let's see what would have
happened if we'd tried to prove the claim above by this method:

\inductioncase{Revised inductive step}: We must show that $P(n)$
implies $P(n + 1)$ for all $n \ge 1$.  Consider an $(n + 1)$-vertex
graph~$G$ in which every vertex has degree at least~1.  Remove an
arbitrary vertex~$v$, leaving an $n$-vertex graph~$G'$ in which every
vertex has degree\dots\ uh oh!

The reduced graph~$G'$ might contain a vertex of degree~0, making the
inductive hypothesis $P(n)$ inapplicable!  We are stuck---and
properly so, since the claim is false!

Always use shrink-down, grow-back arguments and you'll never fall into
this trap.

%S08, cp6m, S06 cp5f

%S06 cp5f

%% Connectedness Problems %%%%%%%%%%%%%%%%%%%%%%%%%%%%%%%%%%%%%%%%%%%%%%%%%%%%
\begin{problems}
\classproblems
\pinput{CP_n_dim_hypercube}
\pinput{CP_graph_maximal_connected}
\pinput{CP_Kn_is_very_connected}
\pinput{CP_pos_deg_but_not_connected}
\homeworkproblems
\pinput{PS_Euler_circuits}

% S09.cp6m.2
% S09.cp6t.1

\homeworkproblems
\pinput{PS_tangled_and_mangled_graphs}
\pinput{PS_circuit_graph_with_crossbars}
\end{problems}

\section{Around and Around We Go}

\subsection{Cycles and Closed Walks}

\begin{definition}
  A \emph{closed walk}\footnote{Some texts use the word \emph{cycle} for
    our definition of closed walk and \emph{simple cycle} for our
    definition of cycle.} in a graph~$G$ is a walk that begins and ends at
  the same vertex.  The \emph{length} of the closed walk is the number of
  consecutive edges it traverses.  A closed walk is said to be
  a \term{cycle} if it is of the form $\walkv{p}v$ where $\walkv{p}$ is a
  path that starts at $v$ and has length at least two .
\end{definition}

For example, $b$, $c$, $d$, $e$, $c$,~$b$ is a closed walk of
length~5 in the graph shown in Figure~\ref{dg}.  It is not a cycle
since it contains node~$c$ twice.  On the other hand, $c$, $d$,
$e$,~$c$ is a cycle of length~3 in this graph since every node
appears just once.

\begin{editingnotes}
\arm{revised:}
\end{editingnotes}

We really want to think of closed walks as having neither a start or end,
nor a direction (clockwise or counter-clockwise).  So we say a closed walk
$\walkv{w}$ that starts and ends with vertex $v$ and goes through vertex
$x \neq v$ is \emph{equivalent} to the corresponding closed walk that
starts and ends at $x$.  Namely, if
\[
\walkv{w} = \catv{\walkv{f}}{x}{\walkv{r}}
\]
% \[
% \walkv{w} = \catv{\catv{v}{v}{\catv{\walkv{f}}{x}{\walkv{r}}}}{v}{v}
% \]
for some walks $\walkv{f}$ and $\walkv{r}$, then
\[
\walkv{w} \text{ is equivalent to }
\catv{\walkv{r}}{v}{\walkv{f}}
\]
% \[
% \walkv{w} \text{ is equivalent to }
% \catv{\catv{x}{x}{\catv{\walkv{r}}{v}{\walkv{f}}}}{x}{x}.
% \]
Finally, the closed walk obtained by reversing the sequence of vertices in
$\walkv{w}$ is also equivalent $\walkv{w}$.

From now on, when there's no danger of confusion, we'll casually say
that walks are `the same,' when we really mean they are equivalent.
For example, $b$, $c$, $d$, $e$, $c$,~$b$ is the same as $c$, $d$,
$e$, $c$, $b$,~$c$ (just starting at node~$c$ instead of node~$b$) and
the same as $b$, $c$, $e$, $d$, $c$,~$b$ (just reversing the
direction).

\begin{editingnotes}
\arm{CUT:}
Cycles are similar to paths, except that the last node is the first
node and the notion of first and last does not matter.  Indeed, there
are many possible vertex orders that can be used to describe cycles
and closed walks, whereas walks and paths have a prescribed beginning,
end, and ordering.
\end{editingnotes}

\subsection{Odd Cycles and 2-Colorability}\label{subsec:odd_cycles}

We have already seen that determining the chromatic number of a graph is a
challenging problem.  There is one special case where this problem is very
easy, namely, when the graph is 2-colorable.

\begin{theorem}\label{thm:2-colorable-equiv}
The following graph properties are equivalent:

\begin{enumerate}

\item\label{has-odd-cycle}
The graph contains an odd length cycle.

\item\label{not-2-color}
The graph is not 2-colorable.

\item\label{has-odd-closed-walk}
The graph contains an odd length closed walk.

\end{enumerate}
\end{theorem}
In other words, if a graph has any one of the three properties above, then
it has all of the properties.

We will show the following implications among these properties:
\[
\text{\ref{has-odd-cycle}.} \QIMPLIES \text{\ref{not-2-color}.} \QIMPLIES
\text{\ref{has-odd-closed-walk}.} \QIMPLIES
\text{\ref{has-odd-cycle}}.
\]
So each of these properties implies the other two\footnote{Mutual
  implication follows from transivity of implication,
  Rule~\ref{rule:transitivity} in Section~\ref{sec:logical_deduction}},
 which means they all are equivalent.

\begin{description}

\item[\ref{has-odd-cycle} \QIMPLIES \ref{not-2-color}]
\begin{proof}
This follows from equation~\ref{Codd3}.
\end{proof}

\item[\ref{not-2-color} $\QIMPLIES$ \ref{has-odd-closed-walk}]

  If we prove this implication for connected graphs, then it will hold
  for an arbitrary graph because it will hold for each
  connected component.  So we can assume that $G$ is connected.
\begin{proof}

  Pick an arbitrary vertex $r$ of $G$.  Since $G$ is connected, for every
  node $u \in \vertices{G}$, there will be a walk $\walkv{w}_u$ starting
  at $u$ and ending at $r$.  Assign colors to vertices of $G$ as follows:
\[
\text{color}(u) = \begin{cases}
                   \text{black}, & \text{if $\abs{\walkv{w}_u}$ is even},\\
                   \text{white}, & \text{otherwise}.
\end{cases}
\]
Now since $G$ is not colorable, this can't be a valid coloring.  So there
must be an edge between two nodes $u$ and $v$ with the same color.  But in
that case
\[
\catv{\catv{\walkv{w}_u}{r}{\text{reverse}(\walkv{w}_v)}}{v}{vu}.
\]
is a closed walk starting and ending at $u$, and its length is
\[
\abs{\walkv{w}_u} + \abs{\walkv{w}_v}+ 1.
\]
This length is odd, since $\walkv{w}_u$ and $\walkv{w}_v$ are both even
length or are both odd length.
\end{proof}

\item[\ref{has-odd-closed-walk} $\QIMPLIES$ \ref{has-odd-cycle}]

\begin{proof}
  Since there is an odd length closed walk, the WOP implies there is a odd
  length closed walk $\walkv{w}$ of minimum length.  We claim $\walkv{w}$
  must be a cycle.  To show this, assume to the contrary that there is
  vertex $x$ that appears twice on the walk, so $\walkv{w}$ consists of a
  closed walk from $x$ to $x$ followed by another such walk.  That is,
\[
\walkv{w} = \catv{\walkv{f}}{x}{\walkv{r}}
\]
for some positive length walks $\walkv{f}$ and $\walkv{r}$ that begin and
end at $x$.  Since
\[
\abs{\walkv{w}} =  \abs{\walkv{f}}+ \abs{\walkv{r}}
\]
is odd, exactly one of $\walkv{f}$ and $\walkv{g}$ must have odd length,
and that one will be an odd length closed walk shorter than $\walkv{w}$, a
contradiction.

\end{proof}

This completes the proof of Theorem~\ref{thm:2-colorable-equiv}.

\begin{editingnotes}
\arm{Here's cleaned up version of FTL's direct proof that~\ref{not-2-color}
implies~\ref{has-odd-cycle}.  It's here for editing, but really should be
put in a problem (if it's used at all).}

\end{editingnotes}

But we're going to present an interesting direct proof that

\ref{not-2-color} $\QIMPLIES$~\ref{has-odd-cycle}

based on the triangle inequality, Lemma~\ref{lem:tri-ineq}.

\begin{proof}
  If we prove this implication for connected graphs, then it will hold
  for an arbitrary graph because it will hold for each connected
  component.  So we can assume that $G$ is connected.

  Pick an arbitrary vertex $r$ of $G$ and assign colors to vertices
  as follows:
\begin{equation}\label{colorurule}
\text{color}(u) \eqdef
             \begin{cases}
                   \text{black}, & \text{if $\dstuv{u}{r}$ is even},\\
                   \text{white}, & \text{otherwise}.
             \end{cases}
\end{equation}
Since $G$ is not supposed to be 2-colorable, this can't be a valid
coloring, that is, there must be an edge $\edge{u}{v}$ between two nodes
with the same color.

Because of the edge $\edge{u}{v}$, the distance between $u$ and $v$ is 1.
Because $G$ is connected, $\dstuv{u}{r}$ and $\dstuv{v}{r}$ are both
finite.  Therefore, the distance from $u$ to $r$ differs by at most 1 from
the distance from $v$ to $r$, by the triangle inequality.  But $u$ and $v$
have the same color, so their distances to $r$ can only differ by an even
number.  Of course, the only nonnegative that is most 1 and is even is 0.
So they can't differ at all:
\begin{equation}\label{dstur=vr}
\dstuv{u}{r}= \dstuv{v}{r}.
\end{equation}

Here's how we find the odd cycle: a shortest path from $u$ to $r$ and a
shortest path from $v$ to $r$ have the vertex $r$ in common, by
definition.  So by WOP, there must be a vertex $x$ on both paths whose
distance to $u$ is a minimum.  Now a shortest path from $u$ to $x$ and a
shortest path from $v$ to $x$ can \emph{only} have vertex $x$ in common,
since any other vertex they had in common would be closer to $u$ than $x$
was.  Therefore, a shortest path from $u$ to $x$ together with a shortest
path from $v$ to $x$ and the edge $\edge{u}{v}$ form a cycle of length
\begin{equation}\label{uxvx1}
\dstuv{u}{x}+  \dstuv{v}{x} +1.
\end{equation}

But since $x$ is on these shortest paths, we have
\begin{align*}
\dstuv{u}{r} & = \dstuv{u}{x} + \dstuv{x}{r}\\
\dstuv{v}{r} & = \dstuv{v}{x} + \dstuv{x}{r},
\end{align*}
by Lemma~\ref{lem:tri-ineq}, which along with equation~\eqref{dstur=vr},
implies
\[
\dstuv{v}{x} = \dstuv{u}{x}.
\]
So the the cycle length~\ref{uxvx1} equals the odd number $2\dstuv{u}{x}+1$ .
\end{proof}


\end{description}

Theorem~\ref{thm:2-colorable-equiv} turns out to be useful since bipartite
graphs come up fairly often in practice.  We'll see examples when we talk
about planar graphs in Section~\ref{planar_graphs_sec} and when we talk
about packet routing in communication networks in
Chapter~\ref{chap:digraphs}.

\subsection{Euler Tours}

Can you walk every hallway in the Museum of Fine Arts \emph{exactly
  once}?  If we represent hallways and intersections with edges and
vertices, then this reduces to a question about graphs.  For example,
could you visit every hallway exactly once in a museum with the
floor plan in Figure~\ref{fig:5BC}?

\begin{figure}

\gnote{Add a ``real'' floor-plan?}

\graphic{euler-tour}

\caption{A possible floor plan for a museum. Can you find a walk that
  traverses every edge exactly once?}

\label{fig:5BC}

\end{figure}

The entire field of graph theory began when Euler asked whether the seven
bridges of K\"onigsberg could all be traversed exactly once---essentially
the same question we asked about the Museum of Fine Arts.  In his honor,
an \term*{Euler walk} \index{Euler!walk} is a defined to be a walk that
traverses every edge in a graph exactly once.  Similarly, an \emph{Euler
  tour}\index{Euler!tour} is a closed Euler walk, that is an Euler walk that
  starts and finishes at the same vertex.  Graphs with Euler tours and
  Euler walks both have simple characterizations.
\begin{theorem}\label{thm:euler-tour}
A connected graph has an Euler tour if and only if every vertex has
even degree.
\end{theorem}

\begin{proof}
  We first show that if a graph has an Euler tour, then every vertex has
  even degree.  Assume that a graph $G$ has an Euler tour $v_0$, $v_1$,
  \dots, $v_k$ where $v_k = v_0$.  Since every edge is traversed once in
  the tour, $k = \card{\edges{G}}$ and the degree of a node~$u$ in~$G$ is
  twice the number of times that node~$u$ appears in the sequence $v_0$,
  $v_1$, \dots, $v_{k-1}$.  We multiply by two since if $u = v_i$ for
  some~$i$ where $0 < i < k$, then both $\edge{v_{i - 1}}{v_i}$ and
  $\edge{v_i}{v_{i + 1}}$ are edges incident to~$u$ in~$G$.  If $u = v_0 =
  v_k$, then both $\edge{v_{k - 1}}{v_k}$ and $\edge{v_0}{v_1}$ are edges
  incident to~$u$ in~$G$.  Hence, the degree of every node is even.

We next show that if the degree of every node is even in a graph
$G$, then there is an Euler tour.  Let
\[
\mathbf{w} \eqdef v_0, v_1, \dots, v_k
\]
be the longest walk in~$G$ that traverses \emph{no edge more than
  once}.\footnote{Did you notice that we are using a variation of the
  Well Ordering Principle here when we implicitly assume that a
  longest walk exists?  This is ok since the length of a walk where no
  edge is used more than once is at most~$\card{\edges{G}}$.}  The walk
$\mathbf{w}$ must traverse every edge incident to~$v_k$; otherwise the
walk could be extended and $\mathbf{w}$ would not be the longest walk
that traverses all edges at most once.  Moreover, it must be that $v_k
= v_0$ and that $\mathbf{w}$ is a closed walk, since otherwise $v_k$
would have odd degree in~$\mathbf{w}$, and hence in~$G$, which is not
possible by assumption.

We conclude the argument with a proof by contradiction.  Suppose that
$W$ is not an Euler tour.  Because $G$ is a connected graph, we can
find an edge not in $\mathbf{w}$ but incident to some vertex in $\mathbf{w}$.  Call this
edge $\edge{u}{v_i}$.  But then we can construct a walk $\mathbf{w}'$ that is
longer than~$\mathbf{w}$ but that still uses no edge more than once:
\[
   \mathbf{w}' ::= u, v_i, v_{i + 1}, \dots, v_k, v_1, v_2, \dots, v_i
\]
contradicts the definition of $\mathbf{w}$, so $\mathbf{w}$ must be an
Euler tour after all.
\end{proof}

It is not difficult to extend Theorem~\ref{thm:euler-tour} to prove that a
connected graph~$G$ has an Euler walk if and only if precisely 0 or~2
nodes in~$G$ have odd degree.  Hence, we can conclude that the graph
shown in Figure~\ref{fig:5BC} has an Euler walk but not an Euler tour
since the graph has precisely two nodes with odd degree.

Although the proof of Theorem~\ref{thm:euler-tour} does not explicitly
define a method for finding an Euler tour when one exists, it is not
hard to modify the proof to produce such a method.  The idea is to
grow a tour by continually splicing in closed walks until all the
edges are consumed.

\subsection{Hamiltonian Cycles}

Hamiltonian cycles are the unruly cousins of Euler tours.

\begin{definition}\label{def:hamiltonian-cycle}
A \emph{Hamiltonian cycle} in a graph~$G$ is a cycle that visits every
\emph{node} in~$G$ exactly once.  Similarly, a \emph{Hamiltonian} path
is a path in~$G$ that visits every node exactly once.
\end{definition}

Although Hamiltonian cycles sound similar to Euler tours---one visits
every node once while the other visits every edge once---finding a
Hamiltonian cycle can be a lot harder than finding an Euler tour.  The
same is true for Hamiltonian paths.  This is because no one has
discovered a simple characterization of all graphs with a Hamiltonian
cycle.  In fact, determining whether a graph has a Hamiltonian cycle
is the same category of problem as the SAT problem of
Section~\ref{SAT_sec} and the coloring problem in
Section~\ref{sec:coloring}: you get a million dollars for finding an
efficient way to determine when a graph has a Hamiltonian cycle---or
proving that no procedure works efficiently on all graphs.

\subsection{The Traveling Salesperson Problem}

As if the problem of finding a Hamiltonian cycle is not hard enough,
when the graph is weighted, we often want to find a Hamiltonian cycle
that has least possible weight.  This is a very famous optimization
problem known as the Traveling Salesperson Problem.

\begin{definition}
Given a weighted graph~$G$, the \emph{weight} of a cycle in~$G$ is
defined as the sum of the weights of the edges in the cycle.
\end{definition}

For example, consider the graph shown in Figure~\ref{fig:5AL} and
suppose that you would like to visit every node once and finish at the
node where you started.  Can you find  way to do this by traversing a
cycle with weight~15?

\begin{figure}

\graphic{Fig_5AL}

\caption{A weighted graph.  Can you find a cycle with weight~15 that
  visits every node exactly once?}

\label{fig:5AL}
\end{figure}

Needless to say, if you can figure out a fast procedure that finds the
optimal cycle for the traveling salesperson, let us know so that we
can win a million dollars.

\begin{problems}
\examproblems
\pinput{FP_bipartite_matching_sex}

\homeworkproblems
\pinput{PS_no_odd_length_closed_walks}
\end{problems}

\section{Trees}\label{trees-sec}

As we have just seen, finding good cycles in a graph can be trickier than
you might first think.  But what if a graph has no cycles at all?  Sounds
pretty dull.  But graphs without cycles, called \emph{acyclic graphs}, are
probably the most important graphs of all when it comes to computer
science.

\subsection{Definitions}

\begin{definition}\label{def:tree}
A connected acyclic graph is called a \emph{tree}.
\end{definition}


For example, Figure~\ref{fig:5H} shows an example of a 9-node tree.

\begin{figure}

\graphic{tree-example}

\caption{A 9-node tree.}
\label{fig:5H}
\end{figure}

The graph shown in Figure~\ref{fig:5I} is not a tree since it is not
connected, but it is a forest.  That's because, of course, it consists
of a collection of trees.

\begin{definition}\label{def:forest}
If every connected component of a graph~$G$ is a tree, then $G$ is a
\emph{forest}.
\end{definition}

\begin{figure}

\graphic{Fig_5I}

\caption{A 6-node forest consisting of 2 component trees.  Note that
  this 6-node graph is not itself a tree since it is not connected.}
\label{fig:5I}
\end{figure}

One of the first things you will notice about trees is that they tend
to have a lot of nodes with degree one.  Such nodes are called
\emph{leaves}.

\begin{definition}
A \emph{leaf} is a node with degree~1 in a tree (or forest).
\end{definition}

For example, the tree in Figure~\ref{fig:5H} has 5 leaves and the
forest in Figure~\ref{fig:5I} has 4 leaves.

Trees are a fundamental data structure in computer science.  For
example, information is often stored in tree-like data structures and
the execution of many recursive programs can be modeled as the
traversal of a tree.  In such cases, it is often useful to draw the
tree in a leveled fashion where the node in the top level is
identified as the \emph{root}, and where every edge joins a
\emph{parent} to a \emph{child}.  For example, we have redrawn the
tree from Figure~\ref{fig:5H} in this fashion in Figure~\ref{fig:5JJ}.
In this example, node~$d$ is a child of node~$e$ and a parents of
nodes $b$ and~$c$.

\begin{figure}

\graphic{Fig_5JJ}

\caption{The tree from Figure~\ref{fig:5H} redrawn in a leveled
  fashion, with node~$E$ as the root.}

\label{fig:5JJ}
\end{figure}

In the special case of \emph{ordered binary trees}, every node is the
parent of at most 2 children and the children are labeled as being a
left-child or a right-child.

\subsection{Properties}

Trees have many unique properties.  We have listed some of them in the
following theorem.

\begin{theorem}\label{th:treeprops}
Every tree has the following properties:

\begin{enumerate}

\item Every connected subgraph is a tree.\label{asub}

\item There is a unique simple path between every pair of vertices.

\item Adding an edge between nonadjacent nodes in a tree creates a
  graph with a cycle.

\item Removing any edge disconnects the graph.  That is, every edge is
  a cut edge.

\item If the tree has at least two vertices, then it has at least two
  leaves.

\item The number of vertices in a tree is one larger than the number
  of edges.

\end{enumerate}
\end{theorem}

\begin{editingnotes}
It would be more interesting and useful for spanning tree proofs below
to separate the implications from the equivalences (see
Lemma~\ref{lem:iffe=v-1}).  The equivalences are another chance to use
the cycle-of-implications proof organization.
\end{editingnotes}

\begin{proof}
\begin{enumerate}

\item A cycle in a subgraph is also a cycle in the whole graph, so any
  subgraph of an acyclic graph must also be acyclic.  If the subgraph
  is also connected, then by definition, it is a tree.

\item Since a tree is connected, there is at least one path between
  every pair of vertices.  Suppose for the purposes of contradiction,
  that there are two different paths between some pair of vertices
  $u$ and~$v$.  Beginning at $u$, let $x$ be the first vertex where
  the paths diverge, and let $y$ be the next vertex they share.  (For
  example, see Figure~\ref{fig:5L}.)  Then there are two paths from
  $x$ to~$y$ with no common edges, which defines a cycle.  This is a
  contradiction, since trees are acyclic.  Therefore, there is
  exactly one path between every pair of vertices.

\begin{figure}

\graphic{unique-path}

\caption{If there are two paths between $u$ and~$v$, the graph must
  contain a cycle.}

\label{fig:5L}
\end{figure}

\item An additional edge $\edge{u}{v}$ together with the unique path
  between $u$ and $v$ forms a cycle.

\item Suppose that we remove edge $\edge{u}{v}$.  Since the tree
  contained a unique path between $u$ and $v$, that path must have
  been $\edge{u}{v}$.  Therefore, when that edge is removed, no path
  remains, and so the graph is not connected.

\item
\begin{editingnotes}
\arm{rephrased}
\end{editingnotes}

  Since the tree has at least two vertices, the longest path in the
  tree will have different endpoints $u$ and $v$.  There cannot be an
  edge from an endpoint to any other vertex on the path, because that
  would form a cycle.  There also can't be an edge from an endpoint to
  any vertex not on the path, because that make a longer path.  So
  both endpoints must be leaves.

\item We use induction on the proposition
\[
P(n) \eqdef \text{there are $n - 1$ edges in any $n$-vertex tree}.
\]

\inductioncase{Base case} ($n = 1$): $P(1)$ is true since a tree with
1 node has 0 edges and $1 - 1 = 0$.

\inductioncase{Inductive step}: Now suppose that $P(n)$ is true and
consider an $(n+1)$-vertex tree, $T$.  Let $v$ be a leaf of the tree.
You can verify that deleting a vertex of degree 1 (and its incident
edge) from any connected graph leaves a connected subgraph.  So by
part~\ref{asub} of Theorem~\ref{th:treeprops}, deleting $v$ and its
incident edge gives a smaller tree, and this smaller tree $n - 1$
edges by induction.  If we re-attach the vertex, $v$, and its incident
edge, we find that $T$ has $ = (n + 1) - 1$ edges.  Hence, $P(n + 1)$
is true, and the induction proof is complete.  \qedhere

\end{enumerate}

\end{proof}

Various subsets of properties in Theorem~\ref{th:treeprops} provide
alternative characterizations of trees.  For example,

\begin{lemma}label{lem:iffe=v-1}
A graph $G$ is a tree iff it is connected and $\card{\vertices{G}} =
\card{\edges{G}} +1$.
\end{lemma}

We'll leave the proof as an exercise.
\begin{editingnotes}
\arm{make sure to add this proof problem.}
\end{editingnotes}

\subsection{Spanning Trees}

Trees are everywhere.  In fact, every connected graph contains a
subgraph that is a tree with the same vertices as the graph.  This is
a called a \term{spanning tree} for the graph.  For example,
Figure~\ref{fig:5LL} is a connected graph with a spanning tree
highlighted.

\begin{figure}

\graphic{spanning-tree}

\caption{A graph where the edges of a spanning tree have been
  thickened.}

\label{fig:5LL}

\end{figure}

\begin{theorem}
Every connected graph contains a spanning tree.
\end{theorem}

\begin{proof}
\begin{editingnotes}
\arm{rephrased as a DIRECT PROOF}
\end{editingnotes}
Suppose~$G$ is a connected graph.  Define a \term{spanning subgraph} of
$G$ to be a subgraph in which every pair of vertices of $G$ are connected.
Because the graph $G$ is connected, $G$ itself is a spanning subgraph.  So
by WOP, there must be a minimum-edge spanning subgraph $T$.

We claim $T$ is a spanning tree.  Since $T$ by is definition a spanning
subgraph, all we have to show is that $T$ is acyclic.  

So suppose to the contrary that $T$ contained a cycle $C$.  By
Lemma~\ref{cutiffcycle}, an edge $e$ of $C$ will not be a cut edge, so
removing it would leave a spanning subgraph that was smaller than $T$,
contradicting the minimality to $T$.
\end{proof}

\subsection{Minimum Weight Spanning Trees}

Spanning trees are interesting because they connect all the nodes of a
graph using the smallest possible number of edges.  For example the
spanning tree for the 6-node graph shown in Figure~\ref{fig:5LL} has 5
edges.

Spanning trees are very useful in practice, but in the real world, not
all spanning trees are equally desirable.  That's because, in
practice, there are often costs associated with the edges of the graph.

For example, suppose the nodes of a graph represent buildings or towns
and edges represent connections between buildings or towns.  The cost
to actually make a connection may vary a lot from one pair of
buildings or towns to another.  The cost might depend on distance or
topography.  For example, the cost to connect LA to~NY might be much
higher than that to connect NY to Boston.  Or the cost of a pipe
through Manhattan might be more than the cost of a pipe through a
cornfield.

In any case, we typically represent the cost to connect pairs of nodes
with a weighted edge, where the weight of the edge is its cost.  The
weight of a spanning tree is then just the sum of the weights of the
edges in the tree.  For example, the weight of the spanning tree shown
in Figure~\ref{fig:5KA} is~19.

\begin{figure}

\subfloat[]{%
    \graphic{Fig_5KA-a}
}
%
\qquad
%
\subfloat[]{%
    \graphic{Fig_5KA-b}
}

\caption{A spanning tree (a) with weight 19 for a graph~(b).}

\label{fig:5KA}

\end{figure}

The goal, of course, is to find the spanning tree with minimum weight,
called the min-weight spanning tree (MST for short).

\begin{definition}
The \emph{min-weight spanning tree} \textup(MST\textup) of an
edge-weighted graph~$G$ is the spanning tree of~$G$ with the smallest
possible sum of edge weights.
\end{definition}

Is the spanning tree shown in Figure~\ref{fig:5KA}(a) an MST of the
weighted graph shown in Figure~\ref{fig:5KA}(b)?  Actually, it is not,
since the tree shown in Figure~\ref{fig:5KB} is also a spanning tree
of the graph shown in Figure~\ref{fig:5KA}(b), and this spanning tree
has weight~17.

\begin{figure}

\graphic{Fig_5KB}

\caption{An MST with weight~17 for the graph in
  Figure~\ref{fig:5KA}(b).}
\label{fig:5KB}

\end{figure}

What about the tree shown in Figure~\ref{fig:5KB}?  Is it an MST?  It
seems to be, but how do we prove it?  In general, how do we find an
MST for a connected graph $G$?  We could try enumerating all subtrees
of $G$, but that approach would be hopeless for large graphs.

There actually are many good ways to find MST's based on an invariance
property of certain subgraphs of $G$.  In the rest of this section,
subgraphs and forests will always be forests and subgraphs of $G$ that
have the same set of vertices $G$.

A forest that is a subgraph of some MST of $G$ will be called a
\emph{pre-MST}.  For example, the empty graph with vertices
$\vertices{G}$ is guaranteed to be a pre-MST, and so is any MST of
$G$.

If $e$ is an edge of $G$ and $S$ is a subgraph, we'll write $S+e$ for
the subgraph with edges $\edges{S} \union \set{e}$.  If $F$ is a
pre-MST and $e$ is an new edge, that is $e \in \edges{G}-\edges{F}$,
then we'll say $e$ \emph{extends $F$} when $F+e$ is a pre-MST.  The
standard methods for finding MST's all start with the empty forest and
build up to an MST by adding one extending edge after another.  By
Lemma~\ref{iffe=v-1}}, the result after $\card{\vertices{G}}-1$
  extensions is guaranteed to be a connected pre-MST, which an MST.

So the problem of finding MST's reduces to the question of how to tell
if an edge is any good.  Here's how:

Let $F$ be a forest, and color the vertices in each connected
component of $F$ either all black or all white.  At least one
component of each color is required.  Call this a \emph{solid
  coloring} of $F$.  A \emph{gray edge} of a solid coloring is an edge
of $G$ with one black endpoint and one white endpoint.  Any path in
$G$ from a white vertex to a black vertex obviously must traverse a
gray vertex, so there is guaranteed to be at least one gray edge.  In
fact, there will have to be at least as many gray edges as there are
components with the same color.  And here's the punchline:

\begin{theorem}\label{lem:good-extension}
An edge is good for a pre-MST $F$ iff it is a minimum weight gray edge
in some coloring of the components of $F$.
\end{theorem}

Here are two possible algorithms based on Theorem~\ref{lem:good-extension}:

\begin{algorithm}\label{alg:MST1}

\begin{editingnotes}
Is this Kruskal or Prin?
\arm{rephrased}
\end{editingnotes}

  Grow a tree one edge at a time by adding a new edge that touches the
  tree, does not create a cycle, and has minimum weight among such
  edges.
\end{algorithm}
This is the algorithm that arises from coloring the growing tree white
and the vertices not in the tree black.  With this coloring, the edges
that can be added to the tree without creating a cycle are exactly
the gray edges.

\begin{algorithm}\label{alg:MST2}

\begin{editingnotes}
Then this would be Prin or Kruskal?
\arm{rephrased}
\end{editingnotes}

  Grow a forest one edge at a time by adding the minimum-weight new edge
  that does not create a cycle.
\end{algorithm}
A new edge that does not create a cycle must be an edge between two
components, so this is the algorithm that arises from taking the
minimum-weight new edge that does not create a cycle and assinging
different colors to the components it connects.

and the vertices not in the tree black.  With this coloring, the edges
that can be added to the tree without creating a cycle are exactly
the gray edges.


For example, in the weighted graph we have been considering, we might
run Algorithm~\ref{alg:MST1} as follows.  We would start by choosing
one of the weight~1 edges, since this is the smallest weight in the
graph.  Suppose we chose the weight~1 edge on the bottom of the
triangle of weight~1 edges in our graph.  This edge is incident to two
weight~1 edges, a weight~4 edge, a weight~7 edge, and a weight~3
edge.  We would then choose the incident edge of minimum weight.  In
this case, one of the two weight~1 edges.  At this point, we cannot
choose the third weight~1 edge since this would form a cycle, but we
can continue by choosing a weight~2 edge.  We might end up with the
spanning tree shown in Figure~\ref{fig:5KC}, which has weight~17, the
smallest we've seen so far.

\begin{figure}

\graphic{Fig_5KC}

\caption{A spanning tree found by Algorithm~\ref{alg:MST1}.}

\label{fig:5KC}

\end{figure}

Now suppose we instead ran Algorithm~\ref{alg:MST2} on our graph.  We
might again choose the weight~1 edge on the bottom of the triangle of
weight~1 edges in our graph.  Now, instead of choosing one of the
weight~1 edges it touches, we might choose the weight~1 edge on the
top of the graph.  Note that this edge still has minimum weight, and
does not cause us to form a cycle, so Algorithm~\ref{alg:MST2} can
choose it.  We would then choose one of the remaining weight~1 edges.
Note that neither causes us to form a cycle.  Continuing the
algorithm, we may end up with the same spanning tree in
Figure~\ref{fig:5KC}, though this need not always be the case.

It turns out that both algorithms work, but they might end up with
different MSTs.  The MST is not necessarily unique---indeed, if all
edges of an $n$-node graph have the same weight (${}=1$), then all
spanning trees have weight~$n - 1$.

These are examples of greedy approaches to optimization.  Sometimes it
works and sometimes it doesn't.  The good news is that it works to
find the MST\@.  In fact, both variations work.  It's a little easier
to prove it for Algorithm~\ref{alg:MST2}, so we'll do that one here.

\begin{theorem}\label{thm:MST2}
For any connected, weighted graph~$G$, Algorithm~\ref{alg:MST2}
produces an MST for~$G$.
\end{theorem}

\begin{proof}
The proof is a bit tricky.  We need to show the algorithm terminates,
namely, if we have selected fewer than $n - 1$ edges, then we can
always find an edge to add that does not create a cycle.  We also need
to show the algorithm creates a tree of minimum weight.

The key to doing all of this is to show that the algorithm never gets
stuck or goes in a bad direction by adding an edge that will keep us
from ultimately producing an MST\@.  The natural way to prove this is
to show that the set of edges selected at any point is contained in
some MST, in other words, we can always get to where we need to be.
We'll state this as a lemma.

\begin{lemma}\label{lemma:MST2}
  For any $m \ge 0$, let $S$ consist of the first $m$ edges selected by
  Algorithm~\ref{alg:MST2}.  Then there exists some MST~$T$ for~$G$ such
  that $S \subseteq \edges{T}$, that is, the set of edges that we are
  growing is always contained in some MST\@.
\end{lemma}

We'll prove this momentarily, but first let's see why it helps prove
the theorem.  Assume the lemma is true.  Then how do we know
Algorithm~\ref{alg:MST2} can always find an edge to add without
creating a cycle?  Well, as long as there are fewer than $n - 1$ edges
picked, there exists some edge in $\edges{T} - S$ and so there is an
edge that we can add to~$S$ without forming a cycle.  Next, how do we
know that we get an MST at the end?  Well, once $m = n - 1$, we know
that $S$ will
\begin{editingnotes}
\arm{inserted: the edges of}
\end{editingnotes}
be the edges of anMST\@.

Ok, so the theorem is an easy corollary of the lemma.  To prove the
lemma, we'll use induction on the number of edges chosen by the
algorithm so far.  This is very typical in proving that an algorithm
preserves some kind of invariant condition---induct on the number of
steps taken, namely, the number of edges added.

Our inductive hypothesis $P(m)$ is the following: for any $G$ and any
set~$S$ of $m$ edges initially selected by Algorithm~\ref{alg:MST2},
there exists an MST $T$ of~$G$ such that $S \subseteq \edges{T}$.

For the base case, we need to show $P(0)$.  In this case, $S =
\emptyset$, so $S \subseteq \edges{T}$ holds trivially for any MST
$T$.

For the inductive step, we assume $P(m)$ holds and show that it
implies $P(m + 1)$.  Let $e$ denote the $(m+1)$st edge selected by
Algorithm~\ref{alg:MST2}, and let $S$ denote the first $m$ edges
selected by Algorithm~\ref{alg:MST2}.  Let $T^*$ be the
MST such that $S \subseteq \edges{T^*}$, which exists by the inductive
hypothesis.  There are now two cases:
\begin{description}

\item[Case 1:] $e \in \edges{T^*}$, in which case $S \union \{e\}
  \subseteq \edges{T^*}$, and thus $P(m+1)$ holds.

\item[Case 2:]
$e \notin \edges{T^*}$, as illustrated in Figure~\ref{fig:5KD}.  Now we need
  to find a different MST that contains $S$ and~$e$.

\begin{figure}

\graphic{Fig_5KD}

\caption{The graph formed by adding $e$ to $\edges{T^*}$.  Edges of~$S$ are
  denoted with solid lines and edges of $\edges{T^*} - S$ are denoted with
  dashed lines.}

\label{fig:5KD}
\end{figure}

\end{description}

What happens when we add $e$ to~$\edges{T^*}$?  Since $T^*$ is a tree,
we get a cycle.  (Here we used part~3 of Theorem~\ref{th:treeprops}.)
Moreover, the cycle cannot only contains edges in~$S$, since $e$ was
chosen so that together with the edges in~$S$, it does not form a
cycle.  This implies that $\set{e} \union \edges{T^*}$ contains a
cycle that contains an edge $e'$ of $\edges{T^*} - S$.  For example,
such an $e'$ is shown in Figure~\ref{fig:5KD}.

Note that the weight of~$e$ is at most that of~$e'$.  This is because
Algorithm~\ref{alg:MST2} picks the minimum weight edge that does not
make a cycle with~$S$.  Since $e' \in \edges{T^*}$, edge $e'$ cannot
make a cycle with~$S$, and if the weight of~$e$ were greater than the
weight of~$e'$, Algorithm!\ref{alg:MST2} would no have selected~$e$
ahead of~$e'$.

Okay, we're almost done.  Now we'll make an MST that contains $S
\union \set{e}$.  Let $T^{**}$ be the graph with
\begin{align*}
\vertices{T^{**}} & \eqdef \vertices{T},\\
\edges{T^{**}} & \eqdef (\edges{T^*} - \set{e'}) \union \set{e}.
\end{align*}
That is, we swap $e$ and~$e'$ in~$\edges{T^*}$.

\begin{claim}\label{claim:MST2}
$T^{**}$ is an MST.
\end{claim}

\begin{proof}[Proof of claim]
We first show that $T^{**}$ is a spanning tree.  $T^{**}$ is acyclic
because it was produced by removing an edge from the only cycle in
$T^{*} \union \set{e}$.  $T^{**}$ is connected since the edge we
deleted from $\edges{T^*} \union \set{e}$ was on a cycle.  Since
$T^{**}$ contains all the nodes of~$G$, it must be a spanning tree
for~$G$.

Now let's look at the weight of~$T^{**}$.  Well, since the weight
of~$e$ was at most that of~$e'$, the weight of~$T^{**}$ is at most
that of~$T^*$, and thus $T^{**}$ is an MST for~$G$.
\end{proof}

Since $S \union \set{e} \subseteq \edges{T^{**}}$, we have shown that
  $P(m + 1)$ holds.  Thus, Algorithm~\ref{alg:MST2} must eventually
  produce an MST\@.  This will happens when it adds $n - 1$ edges to
  the subgraph it builds.
\end{proof}

So now we know for sure that the MST for our example graph has
weight~17 since it was produced by Algorithm~\ref{alg:MST2}.  And we
have a fast algorithm for finding a minimum-weight spanning tree for
any graph.

%% Trees Problems %%%%%%%%%%%%%%%%%%%%%%%%%%%%%%%%%%%%%%%%%%%%%%%%%%%%%%%%%%%%%
\begin{problems}
\classproblems
\pinput{CP_graph_edge_mark}
\pinput{CP_spanning_tree_proc}
\pinput{CP_tree_characterizations}
%\pinput{CP_23_high_priority_servers}

\homeworkproblems
\pinput{PS_average_degree_of_tree_and_simple_path}
% S09.cp6t.2
% S09.cp6t.3
\end{problems}


%% Coloring  Problems %%%%%%%%%%%%%%%%%%%%%%%%%%%%%%%%%%%%%%%%%%%%%%%%%%%
\begin{problems}
\classproblems
\pinput{CP_chromatic_number}

% S09.cp6r.1
% S09.cp6r.2

\homeworkproblems
\pinput{PS_TA_recitation_graph_coloring}
\pinput{PS_graph_colorable}
%\pinput{PS_simple_triangle_free_graph_coloring}

\examproblems
\pinput{FP_coloring_false_proof}

\end{problems}

%% Bipartite Matchings Problems %%%%%%%%%%%%%%%%%%%%%%%%%%%%%%%%%%%%%%%%%%%%%%%
% S09.cp6r.3
% S09.cp6r.4

\begin{problems}
\classproblems
\pinput{CP_student_clubs}
\pinput{CP_latin_squares}

\examproblems
\pinput{MQ_bipartite-matching_recitations}

\homeworkproblems
\pinput{PS_cards_in_4_rows_13_columns}
\pinput{PS_bipartite_matching_virtues}
\end{problems}

\section{Planar Graphs}\label{planar_graphs_sec}

\subsection{Drawing Graphs in the Plane}

Suppose there are three dog houses and three human houses, as shown in
Figure~\ref{fig:5DP}.  Can you find a route from each dog house to
each human house such that no route crosses any other route?

\begin{figure}

\graphic{dog-houses}

\caption{Three dog houses and and three human houses.  Is there a
  route from each dog house to each human house so that no pair of
  routes cross each other?}
\label{fig:5DP}
\end{figure}

For example, Figure~\ref{fig:5DC} \begin{editingnotes}
new figure needed
\end{editingnotes} shows how to route the first two dogs to all threes
houses and the third dog to two of the houses, all without any crossings.
But in this figure there is no way left to route the third dog to the last
house without crossing one of the other routes.  Is there a better routing
that does the job?

A similar question comes up about a little-known animal called a
\emph{quadrapus} that looks like an octopus with four stretchy arms
instead of eight.  If five quadrapi are resting on the sea floor, as shown
in Figure~\ref{fig:5DA}, can each quadrapus simultaneously shake hands
with every other in such a way that no arms cross?

\begin{figure}

\graphic{quadrapi}

\caption{Five quadrapi (4-armed creatures).}

\label{fig:5DA}

\end{figure}


\begin{editingnotes}
\textcolor{red}{rephrased by ARM 7/3/10:}
\end{editingnotes}

Both these puzzles can be understood as asking about drawing graphs in the
plane.  Replacing dogs and houses by nodes, the dog house puzzle can be
rephrased as asking whether there is a planar drawing of the graph with
six nodes and edges between each of the first three nodes and each of the
second three nodes.  This graph is called the \term{complete bipartite
  graph} \idx{$K_{3,3}$} and is shown in Figure~\ref{fig:nonplanar}.(a).
\begin{editingnotes}
Insert graphic and fix figure refs.
\end{editingnotes}
The quadrapi puzzle asks whether there is a planar drawing of the
\idx{complete graph} \idx{$K_5$} shown in
Figure~\ref{fig:nonplanar}.(b).

\begin{figure}

\subfloat[]{%
    \graphic{Fig_5_36a}
}
\qquad\qquad
\subfloat[]{%
    \graphic{Fig_5_36b}
}

\caption{$K_{3, 3}$ (a) and $K_5$ (b).  Can you redraw these graphs so
that no pairs of edges cross?}

\label{fig:nonplanar}

\end{figure}

In each case, the answer is, ``No---but almost!''  In fact, if you remove
an edge from either of these graphs, then the resulting graph \emph{can}
be redrawn in the plane so that no edges cross.  Figure~\ref{fig:5EF}
already showed how to do this for $K_{3,3}$ with one edge removed, and
Figure~\ref{fig:5DC} 
shows how to do this for $K_5$ with one edge removed.

\begin{figure}

\subfloat[]{
    \graphic{Fig_5DC_a}
}
\qquad
\subfloat[]{
    \graphic{Fig_5DC_b}
}

\caption{Planar drawings of (a) $K_{3, 3}$ without $\edge{u}{v}$, and
  (b) $K_5$ without $\edge{u}{v}$.}
\label{fig:5DC}
\end{figure}

Planar drawings have applications in circuit layout and are helpful in
displaying graphical data such as program flow charts, organizational
charts, and scheduling conflicts.  For these applications, the goal is
to draw the graph in the plane with as few edge crossings as possible.
(See the box on the following page for one such example.)

\begin{figure}[p]\redrawntrue
\textbox{\textboxtitle{Steve Wozniak and a Planar Circuit Design}

\noindent
When wires are arranged on a surface, like a circuit board or microchip,
crossings require troublesome three-dimensional structures.  When Steve
Wozniak designed the disk drive for the early Apple II computer, he
struggled mightily to achieve a nearly planar design:
%
\begin{quotation}
\noindent For two weeks, he worked late each night to make a satisfactory
design.  When he was finished, he found that if he moved a connector he
could cut down on feedthroughs, making the board more reliable.  To make
that move, however, he had to start over in his design.  This time it only
took twenty hours. He then saw another feedthrough that could be
eliminated, and again started over on his design.  ``The final design was
generally recognized by computer engineers as brilliant and was by
engineering aesthetics beautiful.  Woz later said, 'It's something you can
only do if you're the engineer and the PC board layout person yourself.
That was an artistic layout.  The board has virtually no
feedthroughs.'\,''\footnote{From apple2history.org which in turn quotes
\emph{Fire in the Valley} by Freiberger and Swaine.}
\end{quotation}
}
\end{figure}

\subsection{Definitions of Planar Graphs}\label{sec:recdef_planar}

We took the idea of a planar drawing for granted in the previous
section, but if we're going to \emph{prove} things about planar graphs, we
better have precise definitions.

\begin{definition}\label{def:planar_drawing}
A \term{drawing} of a graph assigns to each node a distinct point in
the plane and assigns to each edge a smooth curve in the plane whose
endpoints correspond to the nodes incident to the edge.  The drawing
is \index{planar drawing}\emph{planar} if none of the curves
cross themselves or other curves, namely, the only points that
appear more than once on any of the curves are the node points.  A
graph is \index{planar graph}\emph{planar} when it has a planar
drawing.
\end{definition}

Definition~\ref{def:planar_drawing} is precise but depends on
further concepts: ``smooth planar curves'' and ``points appearing more
than once'' on them.  We haven't defined these concepts---we just
showed the simple picture in Figure~\ref{fig:5DC} and hoped you would
get the idea.

Pictures can be a great way to get a new idea across, but it is generally
not a good idea to use a picture to replace precise mathematics.  Relying
solely on pictures can sometimes lead to disaster---or to bogus proofs,
anyway.  There is a long history of bogus proofs about planar graphs based
on misleading pictures.\footnote{The bogus proof of the
  4-Color Theorem for planar graphs is not the only example.  Mistakes
  creep in with statements like,
\begin{quote}
    As you can see from Figure~ABC, it must be that property~XYZ holds
    for all planar graphs.
\end{quote}}

The bad news is that to prove things about planar graphs using the planar
drawings of Definition~\ref{def:planar_drawing}, we'd have to take a chapter-long
excursion into continuous mathematics just to develop the needed concepts
from plane geometry and topology.
\begin{editingnotes} and this makes
  Definition~\ref{def:planar_drawing} troublesome.
\end{editingnotes}
The good news is that there is another way to define planar graphs that
uses only discrete mathematics.  In particular, we can define planar
graphs as a recursive data type.  In order to understand how it works, we
first need to understand the concept of a \emph{face} in a planar drawing.

\subsubsection{Faces}

In a planar drawing of a graph. the curves corresponding to the edges
divide up the plane into connected regions.  We call these regions the
\term{continuous faces}\footnote{Most texts drop the adjective
  \emph{continuous} from the definition of a face as a connected region.
  We need the adjective to distinguish continuous faces from the
  \emph{discrete} faces we're about to define.}  For example, the drawing
in Figure~\ref{fig:continuous-faces} has four continuous faces.  Face IV,
which extends off to infinity in all directions, is called the
\term{outside face}.

\begin{figure}

\graphic{continuous-faces}

\caption{A planar drawing with four continuous faces.}
\label{fig:continuous-faces}
\end{figure}

The vertices along the boundary of each continuous face in
Figure~\ref{fig:continuous-faces} form a cycle.  For example, labeling the
vertices as in Figure~\ref{fig:continuous-cycles}, the cycles for the face
boundaries are
\begin{equation}\label{eq:5DA}
abca \qquad abda \qquad bcdb \qquad acda.
\end{equation}
These four cycles correspond nicely to the four continuous faces in
Figure~\ref{fig:continuous-cycles}---so nicely, in fact, that we can
identify each of the faces in Figure~\ref{fig:continuous-cycles} by
its cycle.  For example, the cycle $abca$ identifies
face~III\@.  The cycles in list~\ref{eq:5DA} are called the
\emph{discrete faces} of the graph in
Figure~\ref{fig:continuous-cycles}.  We use the term ``discrete''
since cycles in a graph are a discrete data type---as opposed to a
region in the plane, which is a continuous data type.

\begin{figure}

\graphic{continuous-cycles}

\caption{The drawing with labeled vertices.}
\label{fig:continuous-cycles}
\end{figure}

Unfortunately, continuous faces in planar drawings are not always
bounded by cycles in the graph---things can get a little more
complicated.  For example, the planar drawing in
Figure~\ref{fig:bridge} has what we will call a \emph{bridge}, namely,
the edge $\edge{c}{e}$.  The sequence of vertices along the boundary
of the outer region of the drawing is
\[
abcefgecda.
\]
This sequence defines a closed walk, but does not define a cycle since
the walk traverses the bridge $\edge{c}{e}$ twice.

\begin{figure}

\graphic{edge-twice-same-face}

\caption{A planar drawing with a \emph{bridge}.}
\label{fig:bridge}
\end{figure}

The planar drawing in Figure~\ref{fig:dongle} illustrates another
complication.  This drawing has what we will call a \emph{dongle},
namely, the nodes $v$, $x$, $y$, and~$w$, and the edges incident to
them.  The sequence of vertices along the boundary
of the inner region is
\[
rstvxyxvwvtur.
\]
This sequence defines a closed walk, but once again does not define a
cycle, because the walk traverses \emph{every} edge of the dongle
twice---once ``coming'' and once ``going.''

\begin{figure}

\graphic{dongle-face}

\caption{A planar drawing with a \emph{dongle}.}
\label{fig:dongle}
\end{figure}

It turns out that bridges and dongles are the only complications, at least
for connected graphs.  In particular, every continuous face in a planar
drawing corresponds to a closed walk in the graph.  These closed walks
will be called the \emph{discrete faces} of the drawing, and we'll define
them next.

\subsubsection{A Recursive Definition for Planar Embeddings}

The association between the continuous faces of a planar drawing and
closed walks will allow us to characterize a planar drawing in terms
of the closed walks that bound the continuous faces.  In particular,
it leads us to the discrete data type of \term{planar embeddings}
that we can use in place of continuous planar drawings.  Namely,
we'll define a planar embedding recursively to be the set of
boundary-tracing closed walks that we could get by drawing one edge
after another.

\begin{definition}\label{def:embedding}\label{embeddingdef}
A \term{planar embedding} of a \emph{connected} graph consists of a
nonempty set of closed walks of the graph called the \term{discrete
  faces} of the embedding.  Planar embeddings are defined recursively
as follows:

\inductioncase{Base case}: If $G$ is a graph consisting of a single
vertex, $v$, then a planar embedding of $G$ has one discrete face,
namely, the length zero closed walk, $v$.

\inductioncase{Constructor case}: (split a face): Suppose $G$ is a
connected graph with a planar embedding, and suppose $a$ and $b$ are
distinct, nonadjacent vertices of $G$ that appear on some discrete
face, $\gamma$, of the planar embedding.  That is, $\gamma$ is a
closed walk of the form
\[
a \dots b \cdots a.
\]
Then the graph obtained by adding the edge $\edge{a}{b}$ to the edges of
$G$ has a planar embedding with the same discrete faces as $G$, except
that face $\gamma$ is replaced by the two discrete
faces\footnote{\label{C} There is a minor exception to this definition of
  embedding in the special case when $G$ is a line graph beginning with
  $a$ and ending with $b$.  In this case the cycles into which $\gamma$
  splits are actually the same.  That's because adding edge $\edge{a}{b}$
  creates a cycle that divides the plane into ``inner'' and ``outer''
  continuous faces that are both bordered by this cycle.  In order to
  maintain the correspondence between continuous faces and discrete faces
  in this case, we define the two discrete faces of the embedding to be
  two ``copies'' of this same cycle.}
\[
a\dots ba\quad \text{ and } \quad ab\cdots a,
\]
as illustrated in Figure~\ref{fig:face-splitting}.

\begin{figure}

\graphic{split-a-face}

\caption{The ``split a face'' case.}
\label{fig:face-splitting}
\end{figure}

\inductioncase{Constructor case} (add a bridge): Suppose $G$ and~$H$
are connected graphs with planar embeddings and disjoint sets of
vertices.  Let $a$ be a vertex on a discrete face, $\gamma$, in the
embedding of $G$.  That is, $\gamma$ is of the form
\[
a\dots a.
\]
Similarly, let $b$ be a vertex on a discrete face, $\delta$, in the
embedding of $H$.  So $\delta$ is of the form
\[
b\cdots b.
\]
Then the graph obtained by connecting $G$ and $H$ with a new edge,
$\edge{a}{b}$, has a planar embedding whose discrete faces are the union of
the discrete faces of $G$ and $H$, except that faces $\gamma$ and $\delta$
are replaced by one new face
\[
a\dots ab\cdots ba.
\]
This is illustrated in Figure~\ref{fig:add-bridge}, where the faces of
$G$ and $H$ are:
\[
G: \set{ axyza,\; axya,\; ayza }
    \qquad H: \set{ btuvwb,\; btvwb,\; tuvt },
\]
and after adding the bridge $\edge{a}{b}$, there is a
single connected graph with faces
\[
\set{ axyz{\color{blue}ab}tuvw{\color{blue}ba},\;
         axya,\; ayza,\; btvwb,\; tuvt }.
\]

\begin{figure}

\graphic{add-bridge}

\caption{The ``add a bridge'' case.}
\label{fig:add-bridge}
\end{figure}

\end{definition}

\subsubsection{Does It Work?}

Yes!  In general, a graph is planar if and only if each of its
connected components has a planar embedding as defined in
Definition~\ref{def:embedding}.  Unfortunately, proving this fact
requires a bunch of mathematics that we don't cover in this
text---stuff like geometry and topology.  Of course, that is why we
went to the trouble of including Definition~\ref{def:embedding}---we
don't want to deal with that stuff in this text and now that we have a
recursive definition for planar graphs, we won't need to.  That's the
good news.

The bad news is that Definition~\ref{def:embedding} looks a lot more
complicated than the intuitively simple notion of a drawing where
edges don't cross.  It seems like it would be easier to stick to the
simple notion and give proofs using pictures.  Perhaps so, but your
proofs are more likely to be complete and correct if you work from the
discrete Definition~\ref{def:embedding} instead of the continuous
Definition~\ref{def:planar_drawing}.

\subsubsection{Where Did the Outer Face Go?}

Every planar drawing has an immediately-recognizable outer face---its
the one that goes to infinity in all directions.  But where is the
outer face in a planar embedding?

There isn't one!  That's because there really isn't any need to
distinguish one.  In fact, a planar embedding could be drawn with any
given face on the outside.  An intuitive explanation of this is to
think of drawing the embedding on a \emph{sphere} instead of the
plane.  Then any face can be made the outside face by ``puncturing''
that face of the sphere, stretching the puncture hole to a circle
around the rest of the faces, and flattening the circular drawing onto
the plane.

So pictures that show different ``outside'' boundaries may actually be
illustrations of the same planar embedding.  For example, the two
embeddings shown in Figure~\ref{fig:5DE} are really the same.

\begin{figure}

\graphic{Fig_5DE}

\caption{Two illustrations of the same embedding.}
\label{fig:5DE}
\end{figure}

This is what justifies the ``add bridge'' case in
Definition~\ref{def:embedding}: whatever face is chosen in the
embeddings of each of the disjoint planar graphs, we can draw a
bridge between them without needing to cross any other edges in the
drawing, because we can assume the bridge connects two ``outer''
faces.

\subsection{Euler's Formula}

The value of the recursive definition is that it provides a powerful
technique for proving properties of planar graphs, namely, structural
induction.  For example, we will now use
Definition~\ref{def:embedding} and structural induction to establish
one of the most basic properties of a connected planar graph; namely,
the number of vertices and edges completely determines the number of
faces in every possible planar embedding of the graph.

\begin{theorem}[Euler's Formula\index{Euler!formula}]\label{thm:eulers_formula}
If a connected graph has a planar embedding, then
\begin{equation*}
    v - e + f = 2
\end{equation*}
where $v$ is the number of vertices, $e$ is the number of edges, and
$f$ is the number of faces.
\end{theorem}

For example, in Figure~\ref{fig:continuous-faces}, $v = 4$,
$e = 6$, and $f = 4$.  Sure enough, $4 - 6 + 4 = 2$, as Euler's
Formula claims.

\begin{proof}
The proof is by structural induction on the definition of planar
embeddings.  Let $P(\embed{E})$ be the proposition that $v - e + f = 2$ for an
embedding, $\embed{E}$.

\inductioncase{Base case} ($\embed{E}$ is the one-vertex planar
embedding):  By definition, $v=1$, $e=0$, and $f=1$, so $P(\embed{E})$
indeed holds.

\inductioncase{Constructor case} (split a face): Suppose $G$ is a
connected graph with a planar embedding, and suppose $a$ and $b$ are
distinct, nonadjacent vertices of $G$ that appear on some discrete
face, $\gamma= a \dots b \cdots a$, of the planar embedding.

Then the graph obtained by adding the edge $\edge{a}{b}$ to the edges of
$G$ has a planar embedding with one more face and one more edge than $G$.
So the quantity $v-e+f$ will remain the same for both graphs, and since by
structural induction this quantity is 2 for $G$'s embedding, it's also 2
for the embedding of $G$ with the added edge.  So $P$ holds for the
constructed embedding.

\inductioncase{Constructor case} (add bridge): Suppose $G$ and $H$ are
connected graphs with planar embeddings and disjoint sets of vertices.
Then connecting these two graphs with a bridge merges the two bridged
faces into a single face, and leaves all other faces unchanged.  So
the bridge operation yields a planar embedding of a connected graph
with $v_G +v_H$ vertices, $e_G + e_H +1$ edges, and $f_G + f_H - 1$
faces.  Since
\begin{align*}
\lefteqn{(v_G +v_H) - (e_G + e_H +1) + (f_G + f_H - 1)} \qquad\\
   & = (v_G  - e_G + f_G) + (v_H  - e_H  + f_H) -2\\
   & = (2)+(2)-2 \qquad \text{(by structural induction hypothesis)}\\
   & = 2,
\end{align*}
$v-e+f$ remains equal to~2 for the constructed embedding.  That is,
$P(e)$ also holds in this case.

This completes the proof of the constructor cases, and the theorem follows
by structural induction.
\end{proof}

\subsection{Bounding the Number of Edges in a Planar Graph}

Like Euler's formula, the following lemmas follow by structural induction
directly from Definition~\ref{def:embedding}.

\begin{lemma}\label{2e}
In a planar embedding of a connected graph, each edge is traversed once by
each of two different faces, or is traversed exactly twice by one face.
\end{lemma}

\begin{lemma}\label{3f}
  In a planar embedding of a connected graph with at least three vertices,
  each face is of length at least three.
\end{lemma}

Combining Lemmas~\ref{2e} and~\ref{3f} with Euler's Formula, we can
now prove that planar graphs have a limited number of edges:

\begin{theorem}\label{th:e3v}
  Suppose a connected planar graph has $v \geq 3$ vertices and $e$
  edges.  Then
\begin{equation}\label{eq:e3v}
    e \leq 3v-6.
\end{equation}
\end{theorem}

\begin{proof}
By definition, a connected graph is planar iff it has a planar embedding.
So suppose a connected graph with $v$ vertices and $e$ edges has a planar
embedding with $f$ faces.  By Lemma~\ref{2e}, every edge is traversed
exactly twice by the face boundaries.  So the sum of the lengths of the
face boundaries is exactly $2e$.  Also by Lemma~\ref{3f}, when $v \geq 3$,
each face boundary is of length at least three, so this sum is at least
$3f$.  This implies that
\begin{equation}\label{e3f}
3f \leq 2e.
\end{equation}
But $f = e-v+2$ by Euler's formula, and substituting into~\eqref{e3f} gives
\begin{align*}
3(e-v+2) & \leq 2e\\
e-3v + 6  & \leq 0\\
e & \leq 3v - 6 \qedhere
\end{align*}
\end{proof}

\subsection{Returning to $K_5$ and $K_{3,3}$}

Finally we have a simple way to answer the quadrapi question at the
begiing of this chapter: the five quadrapi can't all shake hands without
crossing.  The reason is that we know the quadrupi question is the same as
asking whether a complete graph $K_5$ is planar, and 
Theorem~\ref{th:e3v} has the immediate:
\begin{corollary}\label{k5not}
$K_5$ is not planar.
\end{corollary}
\begin{proof}
  $K_5$ is connected and has 5 vertices and 10 edges.  But since $10 > 3
  \cdot 5-6$, $K_5$ does not satisfy the inequality~\eqref{eq:e3v} that
  holds in all planar graphs.
\end{proof}

We can also use Euler's Formula to show that $K_{3, 3}$ is not
planar.  The proof is similar to that of Theorem~\ref{eq:e3v} except that
we use the additional fact that $K_{3, 3}$ is a bipartite graph.

\begin{editingnotes}
\textcolor{red}{CUT by FTL since proved in Theorem~\ref{thm:2-colorable-equiv}}

\begin{lemma*}\label{lem:5D5}
Every closed walk in a bipartite graph has even length.
\end{lemma*}

\begin{proof}
A bipartite graph $G$ is defined by the property that $\vertices{G}$
are partitioned into two sets $L$ and~$R$ where every edge
connects a node in~$L$ to a node in~$R$.  Hence, any closed walk
in~$G$ must alternate between a node in~$L$ followed by a node
in~$R$.  Since a closed walk ends on the same node it started with, it
must visit nodes in~$L$ equally often as it visits nodes in~$R$.
Hence it must have even length.
\end{proof}

\begin{corollary}\label{cor:5D6}
In a planar embedding of a connected \emph{bipartite} graph with at
least 3 vertices, each face has length at least~4.
\end{corollary}
\begin{proof}
  By Lemma~\ref{3f}, every face has length~3.  Since the graph is
  bipartite and since each face is a closed walk,
  Lemma~\ref{2color-iff-bip} and
  Theorem~\ref{thm:2-colorable-equiv}.\ref{has-odd-closed-walk} imply that
  no face can have length~3.  Hence, every face must actually have length
  at least~4.
\end{proof}
\end{editingnotes}

\begin{lemma}\label{lem:5D6}
In a planar embedding of a connected \idx{bipartite graph} with at
least 3 vertices, each face has length at least~4.
\end{lemma}

\begin{proof}
  By Lemma~\ref{3f}, every face of a planar embedding of the graph has
  length at least~3.  But by Lemma~\ref{2color-iff-bip} and
  Theorem~\ref{thm:2-colorable-equiv}.\ref{has-odd-closed-walk}, a
    bipartite graph can't have odd length closed walks.  Since the faces
    of a planar embedding are closed walks, there can't be any faces of
    length 3 in a bipartite embedding.  So every face must have length at
    least~4.
\end{proof}

\begin{theorem}\label{th:e2v}
Suppose a connected bipartite graph with $v \geq 3$ vertices and $e$ edges
is planar.  Then
\begin{equation}\label{eq:e2v}
    e \leq 2v-4.
\end{equation}
\end{theorem}

\begin{proof}
  Lemma~\ref{lem:5D6} implies that all the faces of an embedding of the
  graph have length at least 4.  Now arguing as in the proof of
  Theorem~\ref{th:e3v}, we find that the sum of the lengths of the face
  boundaries is exactly~$2e$ and at least~$4f$.  Hence,
\begin{equation}\label{4ele2e}
    4f \le 2e
\end{equation}
for any embedding of a planar bipartite graph.  By Euler's theorem,
$f=2-v+e$.  Substituting $2-v+e$ for $f$ in~\eqref{4ele2e}, we have
\[
4(2-v+e) \leq 2e,
\]
which simplies to~\eqref{eq:e2v}.
\end{proof}

\begin{corollary}\label{cor:K33-nonplanar} %\label{thm:K33-nonplanar}
$K_{3, 3}$ is not planar.
\end{corollary}

\begin{proof}
  $K_{3,3}$ is connected, bipartite and has 6 vertices and 9 edges.  But
  since $ 9 > 2 \cdot 6-4$, $K_{3.3}$ does not satisfy the
  inequality~\eqref{eq:e3v} that holds in all bipartite planar graphs.
\end{proof}

\subsection{Another Characterization for Planar Graphs}

We did not pick ~$K_5$ and~$K_{3, 3}$ as examples because of their
application to dog houses or quadrapi shaking hands.  We really picked
them because they provide another, famous, discrete characterizarion
of planar graphs:
\begin{theorem}[Kuratowski]\label{thm:kuratowski}
A graph is not planar if and only if it contains $K_5$ or~$K_{3, 3}$
as a minor.
\end{theorem}

\begin{definition}
  A \term{minor} of a graph~$G$ is a graph that can be obtained by
  repeatedly\footnote{The three operations can each be performed any
    number of times in any order.} deleting vertices, deleting edges,
  and \index{merging vertices}merging \emph{adjacent} vertices
  of~$G$.  \emph{Merging} two adjacent vertices, $n_1$ and~$n_2$ of a
  graph means deleting the two vertices and then replacing them by a
  new ``merged'' vertex, $m$, adjacent to all the vertices that were
  adjacent to either of~$n_1$ or~$n_2$, as illustrated in
  Figure~\ref{fig:merged}.
\end{definition}

\begin{figure}

\graphic{vertex-merge-arrows}

\caption{Merging adjacent vertices $n_1$ and $n_2$ into new vertex, $m$.}
\label{fig:merged}
\end{figure}

For example, Figure~\ref{fig:5DL} illustrates why $C_3$ is a minor of
the graph in Figure~\ref{fig:5DL}(a).  In fact $C_3$ is a minor of a
connected graph~$G$ if and only if $G$ is not a tree.

\begin{figure}

\gnote{Tom: Should we label $v_1$, $v_2$ and $v_3$ in all 6 graphs?}

\graphic{Fig_5DL}

\caption{One method by which the graph in~(a) can be reduced
  to~$C_3$~(f), thereby showing that $C_3$ is a minor of the graph.
  The steps are: merging the nodes incident to~$e_1$~(b),
  deleting~$v_1$ and all edges incident to it~(c), deleting $v_2$~(d),
deleting~$e_2$, and deleting $v_3$~(f).}

\label{fig:5DL}
\end{figure}


\begin{editingnotes}
\subsection*{Planar Subgraphs}

If you draw a graph in the plane by repeatedly adding edges that don't
cross, you clearly could add the edges in any other order and still wind
up with the same drawing.  This is so basic that we might presume that our
recursively defined planar embeddings have this property.  But that
wouldn't be fair: we really need to prove it.  After all, the recursive
definition of planar embedding was pretty technical---maybe we got it a
little bit wrong, with the result that our embeddings don't have this basic
draw-in-any-order property.

Now any ordering of edges can be obtained just by repeatedly switching the
order of successive edges, and if you think about the recursive definition
of embedding for a minute, you should realize that you can switch
\emph{any} pair of successive edges if you can just switch the last two.
So it all comes down to the following lemma.

\begin{lemma}\label{switch-edges} Suppose that,
  starting from some embeddings of planar graphs with disjoint sets of
  vertices, it is possible by two successive applications of constructor
  operations to add edges $e$ and then $f$ to obtain a planar embedding,
  $\embed{F}$.  Then starting from the same embeddings, it is also
  possible to obtain $\embed{F}$ by adding $f$ and then $e$ with two
  successive applications of constructor operations.
\end{lemma}

We'll leave the proof of Lemma~\ref{switch-edges} to
Problem~\ref{PS_planar_graph_construction_order}.

\begin{corollary}\label{permute-edges} Suppose that, starting from some
  embeddings of planar graphs with disjoint sets of vertices, it is
  possible to add a sequence of edges $e_0,e_1,\dots,e_n$ by successive
  applications of constructor operations to obtain a planar embedding,
  $\embed{F}$.  Then starting from the same embeddings, it is also
  possible to obtain $\embed{F}$ by applications of constructor operations
  that successively add any permutation\footnote{If $\pi:\set{0,1,\dots,n} \to
    \set{0,1,\dots,n}$ is a bijection, then the sequence
    $e_{\pi(0)},e_{\pi(1)},\dots,e_{\pi(n)}$ is called a \term{permutation} of
    the sequence $e_0,e_1,\dots,e_n$.} of the edges $e_0,e_1,\dots,e_n$.
\end{corollary}

\begin{corollary}\label{delete-edge}
Deleting an edge from a planar graph leaves a planar graph.

\begin{proof}
  By Corollary~\ref{permute-edges}, we may assume the deleted edge was the
  last one added in constructing an embedding of the graph.  So the
  embedding to which this last edge was added must be an embedding of the
  graph without that edge.
\end{proof}

\end{corollary}

Since we can delete a vertex by deleting all its incident edges,
Corollary~\ref{delete-edge} immediately implies

\begin{corollary}\label{delete-vertex}
Deleting a vertex from a planar graph, along with all its incident
edges of course, leaves another planar graph.
\end{corollary}

A \term{subgraph} of a graph, $G$, is any graph whose set of vertices is a
subset of the vertices of $G$ and whose set of edges is a subset of the
set of edges of $G$.  So we can summarize Corollaries~\ref{delete-edge}
and~\ref{delete-vertex} and their consequences in a Theorem.

\begin{theorem}\label{planar-subgraph}
  Any \index{planar subgraph}subgraph of a planar graph is planar.
\end{theorem}

\end{editingnotes}

The known proofs of Kuratowski's Theorem~\ref{thm:kuratowski} are a little
too long to include in an introductory text, so we won't prove it.  There
are two further facts about planarity that we will need in our proof
planar graphs are 5-colorable.

\begin{lemma}\label{planar-subgraph}
  Any \index{planar subgraph}subgraph of a planar graph is planar.
\end{lemma}

\begin{lemma}\label{mergelem}
Merging two adjacent vertices of a planar graph leaves another planar
graph.
\end{lemma}

Many authors take these lemmas for granted based on the continuous
Definition~\ref{def:planar_drawing} of planar graphs.  Of course they can
be proved properly from the recursive Definition~\ref{def:embedding} using
structural induction, but those proofs are better left to some homework
problems~\ref{PS_planar_graph_construction_order}.
\begin{editingnotes}
problem~\ref{PS_planar_graph_construction_order} needs a solution, maybe
an extension too.
\end{editingnotes}

\begin{editingnotes}
\arm{CUT: this are special cases of Lemma~\ref{planar-subgraph}.  Only
  purpose in memntioning them is if we were doing the proof.}

\begin{lemma}\label{lem:deleting_planar_edge}
Deleting an edge from a planar graph leaves another planar graph.
\end{lemma}

\begin{corollary}\label{delete-vertex}
Deleting a vertex from a planar graph, along with all its incident
edges, leaves another planar graph.
\end{corollary}
\end{editingnotes}

\subsection{Coloring Planar Graphs}

We've covered a lot of ground with planar graphs, but not nearly
enough to prove the famous 4-color theorem.  But we can get awfully
close.  Indeed, we have done almost enough work to prove that every
planar graph can be colored using only 5 colors.  We need only one
more lemma:
\begin{lemma}\label{lem:pg5}
Every planar graph has a vertex of degree at most five.
\end{lemma}

\begin{proof}
By contradiction.
If every vertex had degree at least~6, then the sum of the vertex
degrees is at least~$6v$, but since the sum of the vertex degrees
equals~$2e$, by the Handshake Lemma~\ref{sumedges}, we have $e
\ge 3v$ contradicting the fact that $e \le 3v - 6 < 3v$ by
Theorem~\ref{th:e3v}.
\end{proof}

\begin{theorem}
Every planar graph is five-colorable.
\end{theorem}

\begin{proof}
The proof will be by strong induction on the number, $v$, of vertices, with
induction hypothesis:
\begin{quote}
Every planar graph with $v$ vertices is five-colorable.
\end{quote}

\inductioncase{Base cases} ($v \leq 5$): immediate.

\inductioncase{Inductive case}: Suppose $G$ is a planar graph with
$v+1$ vertices.  We will describe a five-coloring of $G$.

First, choose a vertex, $g$, of $G$ with degree at most 5;
Lemma~\ref{lem:pg5} guarantees there will be such a vertex.
\begin{description}

\item[Case 1:] ($\degr{g}<5$): Deleting $g$ from $G$ leaves a graph,
$H$, that is planar by Lemma~\ref{planar-subgraph}, and, since $H$
has $v$ vertices, it is five-colorable by induction hypothesis.  Now
define a five coloring of $G$ as follows: use the five-coloring of $H$
for all the vertices besides $g$, and assign one of the five colors to
$g$ that is not the same as the color assigned to any of its
neighbors.  Since there are fewer than 5 neighbors, there will always
be such a color available for $g$.

\item[Case 2:] ($\degr{g}=5$): If the five neighbors of $g$ in $G$
  were all adjacent to each other, then these five vertices would form
  a nonplanar subgraph isomorphic to $K_5$, contradicting
  Lemma~\ref{planar-subgraph} (since $K_5$ is not planar).  So there
  must be two neighbors, $n_1$ and $n_2$, of $g$ that are not
  adjacent.  Now merge $n_1$ and $g$ into a new vertex,~$m$.  In this
  new graph, $n_2$ is adjacent to $m$, and the graph is planar by
  Lemma~\ref{mergelem}.  So we can then merge $m$ and $n_2$ into a
  another new vertex, $m'$, resulting in a new graph, $G'$, which by
  Lemma~\ref{mergelem} is also planar.  Since $G'$ has $v-1$
  vertices, it is five-colorable by the induction hypothesis.

  Now define a five coloring of $G$ as follows: use the five-coloring of $G'$
  for all the vertices besides $g$, $n_1$ and $n_2$.  Next assign the
  color of $m'$ in $G'$ to be the color of the neighbors $n_1$ and $n_2$.
  Since $n_1$ and $n_2$ are not adjacent in $G$, this defines a proper
  five-coloring of $G$ except for vertex $g$.  But since these two
  neighbors of $g$ have the same color, the neighbors of $g$ have been
  colored using fewer than five colors altogether.  So complete the
  five-coloring of $G$ by assigning one of the five colors to $g$ that is
  not the same as any of the colors assigned to its neighbors.
\end{description}

\end{proof}

\subsection{Classifying \idx{Polyhedra}}

The \idx{Pythagoreans} had two great mathematical secrets, the
irrationality of $\sqrt{2}$ and a geometric construct that we're about
to rediscover!

A \term{polyhedron} is a convex, three-dimensional region bounded by a
finite number of polygonal faces.  If the faces are identical regular
polygons and an equal number of polygons meet at each corner, then the
polyhedron is \index{regular polyhedron}\term*{regular}.  Three
examples of regular polyhedra are shown in Figure~\ref{fig:polyhedra}: the
tetrahedron, the cube, and the octahedron.

\begin{figure}

\subfloat[]{
    \graphic{Fig_5_47a}
}
\quad
\subfloat[]{
    \graphic{Fig_5_47b}
}
\quad
\subfloat[]{
    \graphic{Fig_5_47c}
}

\caption{The tetrahedron~(a), cube~(b), and octahedron~(c).}

\label{fig:polyhedra}
\end{figure}

We can determine how many more regular polyhedra there are by thinking
about planarity.  Suppose we took \emph{any} polyhedron and placed a
sphere inside it.  Then we could project the polyhedron face
boundaries onto the sphere, which would give an image that was a
planar graph embedded on the sphere, with the images of the corners of
the polyhedron corresponding to vertices of the graph.  We've already
observed that embeddings on a sphere are the same as embeddings on the
plane, so Euler's formula for planar graphs can help guide our search
for regular polyhedra.

For example, planar embeddings of the three polyhedra in
Figure~\ref{fig:5DP} are shown in Figure~\ref{fig:5DQ}.

\begin{figure}

\subfloat[]{
    \graphic{Fig_5_48a}
}
\quad
\subfloat[]{
    \graphic{Fig_5_48b}
}
\quad
\subfloat[]{
    \graphic{Fig_5_48c}
}

\caption{Planar embeddings of the tetrahedron~(a), cube~(b, and
  octahedron~(c).}

\label{fig:5DQ}

\end{figure}

Let $m$ be the number of faces that meet at each corner of a
polyhedron, and let $n$ be the number of edges on each face.  In the
corresponding planar graph, there are $m$ edges incident to each of
the $v$ vertices.  By the Handshake Lemma~\ref{sumedges}, we
know:
%
\[
m v = 2 e.
\]
%
Also, each face is bounded by $n$ edges.  Since each edge is on the
boundary of two faces, we have:
%
\[
n f = 2 e
\]
%
Solving for $v$ and $f$ in these equations and then substituting into
\idx{Euler's formula} gives:
\[
\frac{2e}{m} - e + \frac{2e}{n} = 2
\]
which simplifies to
\begin{equation}\label{1m1n}
\frac{1}{m} + \frac{1}{n} = \frac{1}{e} + \frac{1}{2}
\end{equation}
%
Equation~\ref{1m1n} places strong restrictions on the structure of a
polyhedron.  Every nondegenerate polygon has at least 3 sides, so $n
\geq 3$.  And at least 3 polygons must meet to form a corner, so $m
\geq 3$.  On the other hand, if either $n$ or $m$ were 6 or more, then
the left side of the equation could be at most $1/3 + 1/6 = 1/2$,
which is less than the right side.  Checking the finitely-many cases
that remain turns up only five solutions, as shown in
Figure~\ref{fig:5DR}.  For each valid combination of $n$ and $m$, we
can compute the associated number of vertices $v$, edges $e$, and
faces $f$.  And polyhedra with these properties do actually exist.
The largest polyhedron, the dodecahedron, was the other great
mathematical secret of the Pythagorean sect.

\begin{figure}\redrawntrue

\gnote{Add illustration of icosa- and dodecahedra?}

\[
\begin{array}{cc|ccc|l}
n & m & v  & e  &  f & \text{polyhedron} \\ \hline
3 & 3 & 4  & 6  &  4 & \text{tetrahedron} \\
4 & 3 & 8  & 12 &  6 & \text{cube} \\
3 & 4 & 6  & 12 &  8 & \text{octahedron} \\
3 & 5 & 12 & 30 & 20 & \text{icosahedron} \\
5 & 3 & 20 & 30 & 12 & \text{dodecahedron}
\end{array}
\]

\caption{The only possible regular polyhedra.}

\label{fig:5DR}

\end{figure}

The 5 polyhedra in Figure~\ref{fig:5DR} are the only possible regular
polyhedra.  So if you want to put more than 20 geocentric satellites
in orbit so that they \emph{uniformly} blanket the globe---tough luck!

%\section{Problems}
\problemsection

%% Planar Graphs Problems %%%%%%%%%%%%%%%%%%%%%%%%%%%%%%%%%%%%%%%%%%%%%%%%%%%%%
\begin{problems}

\examproblems
\pinput{MQ_planar_isomorphism}

\classproblems
\pinput{CP_planar_embedding_isomorphism}
\pinput{CP_K33_not_planar}
\pinput{CP_planar_structural_induction}

\homeworkproblems
\pinput{PS_triangle_free_planar_graphs}
\pinput{PS_planar_graph_construction_order}

%\pinput{CP0506_}
\end{problems}

%% Conclusion %%%%%%%%%%%%%%%%%%%%%%%%%%%%%%%%%%%%%%%%%%%%%%%%%%%%%%%%%%%%%%%%%
%\TBA{Add conclusion here...}

\endinput
