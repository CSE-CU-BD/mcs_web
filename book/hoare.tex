\hyperdef{while}{programs}{\chapter{Reasoning About \textbf{While} Programs}}\label{while_chap}

\iffalse
\def\movesto{\mathrel{\longrightarrow}}
\newcommand{\while}{\text{\textbf{while}}}
\newcommand{\docomm}{\text{\textbf{do}}}
\newcommand{\odcomm}{\text{\textbf{od}}}
\newcommand{\wloop}[2]{\while\ #1\ \docomm\ #2\ \odcomm}
%\newcommand{\wpr}{\while\text{ program}}
\newcommand{\assigned}{\mathbin{\mathtt{:=}}}
\newcommand{\condif}{\text{\textbf{if}}}
\newcommand{\condthen}{\text{\textbf{then}}}
\newcommand{\condelse}{\text{\textbf{else}}}
\newcommand{\condcomm}[3]{\condif\ #1\ \condthen\ #2\ \condelse\ #3}
\newcommand{\seqcomm}[2]{#1\mathbf{;}#2}
\newcommand{\Env}{\text{Env}}
\newcommand{\halt}{\text{\textbf{Done}}}
\newcommand{\state}[2]{\ang{#1,\, #2}}
\newcommand{\step}[4]{\state{#1}{#2} \movesto \state{#3}{#4}}
\newcommand{\haltswith}[3]{\state{#1}{#2}\movesto^* \state{\halt}{#3}}
\newcommand{\after}[3]{#1 \: \set{#2}\: #3}
\fi

Real programs and programming languages are often huge and complicated,
making them hard to model and even harder to reason about.  Still, making
programs ``reasonable'' is a crucial aspect of software engineering.  In
this section we'll illustrate what it means to have a clean mathematical
model of a simple programming language and reasoning principles that go
with it---if only real programming languages allowed for such simple,
accurate modeling.

\section{\textbf{While} Programs}

The programs we'll study are called ``\term{\while\ programs}.''  We
can define them as a recursive data type:
\begin{definition}\label{whiledef} \mbox{}

\textbf{base cases:}
\begin{itemize}

\item $x \assigned e$ is a \while\ program, called an \term{assignment
    statement}, where $x$ is a variable and $e$ is an \emph{expression}.

\item \term{$\halt$} is a \while\ program.

\end{itemize}

\textbf{constructor cases:}
If $C$ and $D$ are \while\ programs, and $T$ is a test, then the following are also
\while\ programs:
\begin{itemize}

\item $\seqcomm{C}{D}$
---called the \term{sequencing} of $C$ and $D$,

\item $\condcomm{T}{C}{D}$---called a \term{conditional} with
  \term{test} $T$ and \term{branches} $C$ and $D$,

\item $\wloop{T}{C}$---called a \term{while loop} with \term{test} $T$ and \term{body} $C$.

\end{itemize}
\end{definition}

For simplicity we'll stick to \while\ programs operating on integers.
So an \emph{expression} will be any of the familiar integer-valued
expressions involving integer constants and operations such as
addition, multiplication, or remainder.  As \emph{tests}, we'll allow
propositional formulas built from basic formulas of the form $e \leq
f$ where $e$ and $f$ are expressions.  For example, here is the
Euclidean algorithm for $\gcd(a,b)$ expressed as a \while\ program.
\begin{center}
\begin{tabbing}
XXX\=XXX\=XXX\=XXX\kill
$\mathtt{x} \assigned a$\textbf{;} \\
$\mathtt{y} \assigned b$\textbf{;} \\
\while\ $\mathtt{y} \neq 0$ \docomm\\
   \> $\mathtt{t} \assigned \mathtt{y}$\textbf{;}\\
   \> $\mathtt{y} \assigned \rem{\mathtt{x}}{\mathtt{y}}$\textbf{;}\\
   \> $\mathtt{x} \assigned \mathtt{t}$
\odcomm\\
\end{tabbing}
\end{center}

\section{The \textbf{While} Program State Machine}

A \while\ program acts as a pure command: it is run solely for its
side effects on stored data and it doesn't return a value.  The data
consists of integers stored as the values of variables, namely
environments:
\begin{definition}
An \term{environment} is a total function from variables to integers.
Let $\Env$ be set of all environments.
\end{definition}
So if $\rho$ is an environment and $x$ is a variable, then $\rho(x)$ is an
integer.  More generally, the environment determines the integer value of
each expression $e$ and the truth value of each test $T$.  We can think
of an expression, $e$ as defining a function $\mean{e}:\Env \to
\integers$, and refer to this function $\mean{e}$ as the \term{meaning}
of $e$.  Likewise, the meaning of a test will be a function from
environments to truth values.

It's standard in programming language theory to write $\mean{e}\rho$
as shorthand for $\mean{e}(\rho)$, that is, applying
the \emph{meaning} $\mean{e}$ of expression $e$ to environment
$\rho$.  For example, if $\rho(\mathtt{x}) =4$, and $\rho(\mathtt{y})
=-2$, then
\begin{equation}\label{x2y311}
\mean{\mathtt{x^2+y -3}}\rho = \rho(\mathtt{x})^2 + \rho(y) -3 = 11.
\end{equation}

\iffalse
(A variable is a special case of an expression, so $\mean{x}\rho \eqdef \rho(x)$.)
\fi

Executing a program causes a succession of changes to the
environment\footnote{More sophisticated programming models distinguish
the environment from a \term{store} which is affected by commands, but
this distinction is unnecessary for our purposes.}  which may continue
until the program halts.  Actually the only command which immediately
alters the environment is an assignment command.  Namely, the effect
of the command
\[
x \assigned e
\]
on an environment is that the value assigned to the variable $x$ is
changed to the value of $e$ in the original environment.  We can say
this precisely and concisely using the following notation:
$\patch{f}{a}{b}$ is a function that is the same as the function $f$
except that when applied to element $a$ its value is $b$.  Namely,
\begin{definition}
If $f:A \to B$ is a function and $a,b$ are arbitrary elements, define
  \[
\patch{f}{a}{b}
\]
to be the function $g$ such that
\begin{equation*}
g(u) = \begin{cases}
         b & \text{ if } u = a.\\
         f(u) & \text{otherwise.}
       \end{cases}
\end{equation*}
\end{definition}

Now we can specify the step-by-step execution of a \while\ program as a state
machine, where the states of the machine consist of a \while\ program paired with an
environment.  The transitions of this state machine are defined
recursively on the definition of \while\ programs.

\begin{definition}
The \term{transitions} $\step{C}{\rho}{D}{\rho'}$ of the \while\ program state
machine are defined as follows:

\inductioncase{Base case:}
\[
\step{x \assigned e}{\rho}{\halt}{\patch{\rho}{x}{\mean{e}\rho}}
\]

\inductioncase{Constructor cases:}
If $C$ and $D$ are \while\ programs, and $T$ is a test, then:

\begin{itemize}

\item if $\step{C}{\rho}{C'}{\rho'}$, then
\[
\step{\seqcomm{C}{D}}{\rho}{C';D}{\rho'}.
\]
Also,
\[
\step{\seqcomm{\halt}{D}}{\rho}{D}{\rho}.
\]

\item if $\mean{T}\rho = \true$, then
\[
\step{\condcomm{T}{C}{D}}{\rho}{C}{\rho},
\]
or if $\mean{T}\rho = \false$, then
\[
\step{\condcomm{T}{C}{D}}{\rho}{D}{\rho}.
\]

\item if $\mean{T}\rho = \true$, then
\[
\step{\wloop{T}{C}}{\rho}{\seqcomm{C}{\wloop{T}{C}}}{\rho}
\]
or if $\mean{T}\rho = \false$, then
\[
\step{\wloop{T}{C}}{\rho}{\halt}{\rho}.
\]
\end{itemize}
\end{definition}

Now \while\ programs are probably going to be the simplest kind of programs you
will ever see, it would be a mistake to underestimate them.  It turns
that \emph{every function on nonnegative integers that can be computed in any
programming language whatsoever} can also be computed by \while\ programs
(albeit more slowly).  We can't take the time to explain how such a sweeping
claim can be justified, but you can find out by taking a course in Computability
Theory. % such as 6.045 or 6.840.

\section{Denotational Semantics}

Denotational semantics for programs involves assigning abstract values
called their \emph{meaning} or \emph{semantics} to programs in a way
that captures their net computational behavior.

In particular, the net effect of starting a \while\ program in some
environment is reflected in the final environment when the program
halts.  So we can think of a \while\ program $C$ aas defining a
function $\mean{C}:\Env \to \Env$ from initial environments to
environments at halting.  The function $\mean{C}$ is called
the \term{meaning} of $C$.
\iffalse
$\mean{C}$ of a \while\ program, $C$ to be a partial function from $\Env$ to $\Env$
mapping an initial environment to the final halting environment.
\fi
This meaning function can be defined recursively following the
definition of the \while\ program recursive data type.

We'll need one new bit of notation in the definition.  For any set
$S$, function $f:S \to S$, and nonnegative integer $n$, let $f^{(n)}$
be the composition of $f$ with itself $n$ times.  Namely,
\begin{align*}
f^{(0)} & \eqdef \ident{S}\\
f^{(n+1)} & \eqdef f \compose f^{(n)},
\end{align*}
where ``$\compose$'' denotes functional composition.

\begin{definition}
The meaning function $\mean{C}$ of a \while\ program, $C$ is defined recursively with

\inductioncase{Base cases:}
\begin{itemize}

\item $\mean{x \assigned e}$ is the function from $\Env$ to $\Env$ defined
  by the rule:
\[
\mean{x \assigned e}\rho \eqdef \patch{\rho}{x}{\mean{e}\rho}.
\]

\item
\[
\mean{\halt} \eqdef \ident{\Env}
\]
where $\ident{\Env}$ is the identity function on $\Env$.  In other words,
$\mean{\halt}\rho \eqdef \rho$.

\end{itemize}

\inductioncase{Constructor cases:}
If $C$ and $D$ are \while\ programs, and $T$ is a test, then:
\begin{itemize}

\item
\[
\mean{\seqcomm{C}{D}} \eqdef \mean{D} \compose \mean{C}
\]
That is,
\[
\mean{\seqcomm{C}{D}}\rho \eqdef \mean{D}(\mean{C}\rho).
\]

\item
\[
\mean{\condcomm{T}{C}{D}}\rho
\eqdef
\begin{cases}
\mean{C}\rho & \text{if } \mean{T}\rho = \true,\\
\mean{D}\rho & \text{if } \mean{T}\rho = \false.
\end{cases}
\]

\item
\[
\mean{\wloop{T}{C}}\rho \eqdef \mean{C}^{(n)}\rho
\]
where $n$ is the least nonnegative integer such that
$\mean{T}(\mean{C}^{(n)}\rho) = \false$.  If there is no such $n$, then
$\mean{\wloop{T}{C}}\rho$ is undefined.
\end{itemize}

\end{definition}

The benefit of our denotational semantics of \while\ programs is that it allows
proving things about a \while\ program using induction on the structure of the
program.  This often turns out to be simpler than reasoning about the
step-by-step behavior of the program---say by using induction on the number of
steps in an execution.  Of course we have to be sure that the denotational
semantics is \term{sound}, that is, that it adequately captures the state
machine behavior.  In particular, using the notation $\movesto^*$ for the
transitive closure of the transition relation:

\begin{theorem}\label{sound-semantics}[Adequacy of \while\ program semantics]
\[
\haltswith{C}{\rho}{\rho'}\ \QIFF\ \mean{C}\rho = \rho'.
\]
\end{theorem}
Theorem~\ref{sound-semantics} can be proved in one direction by structural
induction on \while\ programs and in the other direction by induction on the
number of transitions in the state machine execution.  We leave the details to
Problem~\ref{PS_while_program_semantic_soundness_proof}.

\begin{problems}
\homeworkproblems
\pinput{PS_while_program_semantic_soundness_proof}
\end{problems}

\section{Logic of Programs}

A typical program specification describes the kind of inputs and
environments the program should handle, and then describes what should
result from an execution.  The specification of the inputs or initial
environment is called the \term{precondition} for program execution, and
the prescription of what the result of execution should be is called the
\emph{postcondition}.  So if $P$ is a logical formula expressing the precondition
for a program $C$ and likewise $Q$ expresses the postcondition, the
specification requires that
\begin{quote}
If $P$ holds when $C$ is started, then $Q$ will hold if and when $C$ halts.
\end{quote}
We'll express this requirement as a formula
\[
\after{P}{C}{Q}
\]
called a \term{partial correctness assertion}.

For example, if $E$ is \while\ program above for the Euclidean
algorithm, then the partial correctness of $E$ can be expressed as
\begin{equation}\label{eucpc}
\after{\paren{a, b \in \nngint\ \QAND\  \mathtt{x} \neq 0}}{E}
                   {\paren{\mathtt{x} = \gcd(a,b)}}.
\end{equation}

Remember that the value of a logical formula in an environment is a truth value.
For example, if $\rho(\mathtt{x}) =4$, and $\rho(\mathtt{y}) =-2$, then
by~\eqref{x2y311}, $\mean{\mathtt{x^2+y -3}}\rho = 11$, so
\begin{align*}
\mean{\exists \mathtt{z}.\, \mathtt{z} > 4 \QAND \mathtt{x}^2+\mathtt{y} -3=\mathtt{z}}\rho & = \true,\\
\mean{\exists \mathtt{z}.\, \mathtt{z} > 13 \QAND \mathtt{x}^2+\mathtt{y} -3=\mathtt{z}}\rho & = \false.
\end{align*}

\begin{definition}\label{def_afterPCQ}
For logical formulas $P$ and $Q$, and \while\ program $C$ the
partial correctness assertion
\[
\after{P}{C}{Q}
\]
is true proving that for all environments $\rho$
\[
\mean{P}\rho \QAND\ \mean{C}\rho \text{ is defined}\ \QIMPLIES\ \mean{Q}(\mean{C}\rho).
\]
\iffalse
 if $\mean{P}\rho$ is
true, and $\haltswith{C}{\rho}{\rho'}$ for some $\rho'$, then
$\mean{Q}\rho'$ is true.\fi
\end{definition}


In the 1970's, 
\begin{editingnotes}
check ref
\end{editingnotes}
Prof.\ Tony Hoare (now Sir Anthony) at Univ. Dublin formulated a set
of inference rules for proving partial correctness formulas.  These
rules are known as \term{Hoare Logic}.

The first rule captures the fact that strengthening the preconditions
and weakening the postconditions makes a partial correctness
specification easier to satisfy:

\Rule{P \ \QIMPLIES\  R,
     \quad \after{R}{C}{S},
     \quad S \ \QIMPLIES\  Q}{\after{P}{C}{Q}}

The rest of the logical rules follow the recursive definition of \while\ programs.
There are axioms for the base case commands:
\[\begin{array}{c}
  \after{P(x)}{x \assigned e}{P(e)},\\
  \after{P}{\halt}{P},
\end{array}\]
and proof rules for the constructor cases:

\begin{itemize}

\item
\Rule{\after{P}{C}{Q} \ \QAND\  \after{Q}{D}{R}}{\after{P}{\seqcomm{C}{D}}{R}}

\item
\Rule{\after{P\ \QAND\ T}{C}{Q}}
        {\after{P\ \QAND\ T}{\condcomm{T}{C}{D}}{Q\ \QAND\ T}}

\item
\Rule{\after{P\ \QAND\ T}{C}{P}}
     {\after{P}{\wloop{T}{C}}{P\ \QAND\ \QNOT(T)}}
\end{itemize}

\begin{example}
\textbf{Proof of partial correctness~\eqref{eucpc} for the Euclidean algorithm.}
\end{example}

\TBA{Brief discussion of ``relative completeness''.}

\endinput
