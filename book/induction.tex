\chapter{Induction}\label{induction_chap}
%\hyperdef{in}{duction}

\term{Induction} is a powerful method for showing a property is true
for all natural numbers.  Induction plays a central role in discrete
mathematics and computer science, and in fact, its use is a defining
characteristic of \emph{discrete} ---as opposed to \emph{continuous}
---mathematics.  This chapter introduces two versions of induction
---Ordinary and Strong ---and explains why work and how to use them in
proofs.  It also introduces the Invariance Principle, which is a
version of induction specially adapted for reasoning about
step-by-step processes.


\iffalse
we'll introduce induction and a variant called \term{strong
induction}.  Although these two version of methods look and feel
different, it turns out that they are equivalent in the sense that a
proof using any one of the methods can be automatically reformatted so
that it becomes a proof using any of the other methods.  The choice of
which method to use is up to you and typically depends on whichever
seems to be the easiest or most natural for the problem at hand.
\fi


\section{Ordinary Induction}\label{ordinary_induct_chap}

To understand how induction works, suppose there is a professor who
brings to class a bottomless bag of assorted miniature candy bars.
She offers to share the candy in the following way.  First, she lines
the students up in order.  Next she states two rules:

\begin{enumerate}
\item The student at the beginning of the line gets a candy bar.
\item If a student gets a candy bar, then the following student in line
  also gets a candy bar.
\end{enumerate}
%
Let's number the students by their order in line, starting the count with
0, as usual in computer science.  Now we can understand the second rule as
a short description of a whole sequence of statements:
%
\begin{itemize}
\item If student 0 gets a candy bar, then student 1 also gets one.
\item If student 1 gets a candy bar, then student 2 also gets one.
\item If student 2 gets a candy bar, then student 3 also gets one.

\hspace{1.2in} \vdots
\end{itemize}
%
Of course this sequence has a more concise mathematical description:
\begin{quote}
  If student $n$ gets a candy bar, then student $n+1$ gets a
  candy bar, for all nonnegative integers $n$.
\end{quote}
So suppose you are student 17.  By these rules, are you entitled to a
miniature candy bar?  Well, student 0 gets a candy bar by the first rule.
Therefore, by the second rule, student 1 also gets one, which means
student 2 gets one, which means student 3 gets one as well, and so on.  By
17 applications of the professor's second rule, you get your candy bar!
Of course the rules actually guarantee a candy bar to \emph{every}
student, no matter how far back in line they may be.

\subsection{A Rule for Ordinary Induction}

The reasoning that led us to conclude that every student gets a candy bar is 
essentially all there is to induction.
\textbox{ 
\textboxheader{The Principle of Induction.}

Let $P$ be a predicate on nonnegative integers.  If
%
\noindent \begin{itemize}
\item $P(0)$ is true, and
\item $P(n) \QIMPLIES P(n+1)$ for all nonnegative integers, $n$,
\end{itemize}
then
\begin{itemize}
\item $P(m)$ is true for all nonnegative integers, $m$.
\end{itemize}
}
\iffalse
So our claim that all the Professor's students get a candy bar was simply
an application of the Induction Rule with $P(n)$ defined to be the
predicate, ``student $n$ gets a candy bar.''
\fi

Since we're going to consider several useful variants of induction in
later sections, we'll refer to the induction method described above as
\term{ordinary induction} when we need to distinguish it.  Formulated as 
a proof rule, this would be
\begin{rul*} \textbf{Induction Rule}
\Rule{P(0), \quad \forall n \in \naturals.\, P(n) \QIMPLIES P(n+1)}
{\forall m \in \naturals.\, P(m)}
\end{rul*}

This general induction rule works for the same intuitive reason that all
the students get candy bars, and we hope the explanation using candy bars
makes it clear why the soundness of the ordinary induction can be taken
for granted.  In fact, the rule is so obvious that it's hard to see what
more basic principle could be used to justify it.\footnote{But see
Section~\ref{versusWO}.}  What's not so obvious is how much mileage 
we get by using it.

\subsection{A Familiar Example}

The formula~\eqref{sum-to-n-again} below for the sum of the nonnegative
integers up to $n$ is the kind of statement about all nonnegative integers
to which induction applies directly.  We already already proved it
(Theorem~\ref{sum_to_n_thm}) using the \idx{Well Ordering Principle}, but
now we'll prove it using induction.
\begin{theorem*}%\label{sum-to-n-again-theorem}
For all $n \in \naturals$,
\begin{equation}\label{sum-to-n-again}
1 + 2 + 3 + \cdots + n = \frac{n(n+1)}{2}
\end{equation}
\end{theorem*}

To use the Induction Principle to prove the Theorem, define predicate
$P(n)$ to be the equation~\eqref{sum-to-n-again}.  Now the theorem can
be restated as the claim that $P(n)$ is true for all $n \in
\naturals$.  This is great, because the induction principle lets us
reach precisely that conclusion, provided we establish two simpler
facts:
%
\begin{itemize}
\item $P(0)$ is true.
\item For all $n \in \naturals$, $P(n) \QIMPLIES P(n+1)$.
\end{itemize}

So now our job is reduced to proving these two statements.  The first
is true because $P(0)$ asserts that a sum of zero terms is equal to
$0(0+1)/2 = 0$, which is true by definition.

The second statement is more complicated.  But remember the basic
plan from Section~\ref{sec:prove_implies} for proving the validity of
any implication: \emph{assume} the statement on the left and then
\emph{prove} the statement on the right.  In this case, we assume
$P(n)$---namely, equation~\eqref{sum-to-n-again} ---in order to prove
$P(n+1)$, which is the equation
\begin{equation}\label{sum-to-n-again-Pn1}
1 + 2 + 3 + \cdots + n + (n+1) = \frac{(n+1)(n+2)}{2}.
\end{equation}
These two equations are quite similar; in fact, adding $(n+1)$ to both
sides of equation~\eqref{sum-to-n-again} and simplifying the right side 
gives the equation~\eqref{sum-to-n-again-Pn1}:
\begin{align*}
1 + 2 + 3 + \cdots + n + (n+1)
    & = \frac{n(n+1)}{2} + (n+1) \\
    & = \frac{(n+2)(n+1)}{2}
\end{align*}
Thus, if $P(n)$ is true, then so is $P(n+1)$.  This argument is valid for
every nonnegative integer $n$, so this establishes the second fact
required by the induction principle.  Therefore, the induction principle
says that the predicate $P(m)$ is true for all nonnegative integers, $m$,
so the theorem is proved.

\iffalse
In effect, we've just proved
that $P(0)$ implies $P(1)$, $P(1)$ implies $P(2)$, $P(2)$ implies
$P(3)$, etc., all in one fell swoop.
\fi

\subsection{A Template for Induction Proofs}
\label{templ-induct-proofs}

The proof of equation~\eqref{sum-to-n-again} was relatively simple,
but even the most complicated induction proof follows exactly the same
template.  There are five components:

\begin{enumerate}

\item \textbf{State that the proof uses induction.}  This immediately
  conveys the overall structure of the proof, which helps your reader
  follow your argument.

\item \textbf{Define an appropriate predicate $P(n)$.}  The predicate $P(n)$ is 
called the \term{induction hypothesis}.  The eventual conclusion of
the induction argument will be that $P(n)$ is true for all nonnegative
$n$.  Clearly stating the induction hypothesis is often the most
important part of an induction proof, and omitting it is the largest
source of confused proofs by students.

In the simplest cases, the induction hypothesis can be lifted straight
from the proposition you are trying to prove, as we did with
equation~\eqref{sum-to-n-again}.  Sometimes the induction hypothesis
will involve several variables, in which case you should indicate
which variable serves as $n$.

\item \textbf{Prove that $P(0)$ is true.}  This is usually easy, as in the
  example above.  This part of the proof is called the \term{base case}
  or \term{basis step}.\iffalse
  (Sometimes the base case will be $n=1$ or even
  some larger number, in which case the starting value of $n$ also should
  be stated.)\fi

\item \textbf{Prove that $P(n)$ implies $P(n+1)$ for every nonnegative
    integer $n$.}  This is called the \term{inductive step}.  The basic
  plan is always the same: assume that $P(n)$ is true and then use this
  assumption to prove that $P(n+1)$ is true.  These two statements should
  be fairly similar, but bridging the gap may require some ingenuity.
  Whatever argument you give must be valid for every nonnegative integer
  $n$, since the goal is to prove the implications $P(0) \rightarrow
  P(1)$, $P(1) \rightarrow P(2)$, $P(2) \rightarrow P(3)$, etc. all at
  once.

\item \textbf{Invoke induction.}  Given these facts, the induction
  principle allows you to conclude that $P(n)$ is true for all nonnegative
  $n$.  This is the logical capstone to the whole argument, but it is so
  standard that it's usual not to mention it explicitly.

\end{enumerate}

Always be sure to explicitly label the \emph{base case} and
the \emph{inductive step}.  It will make your proofs clearer, and it
will decrease the chance that you forget a key step (such as checking
the base case).

\subsection{A Clean Writeup}

The proof of the Theorem given above is perfectly valid; however, it
contains a lot of extraneous explanation that you won't usually see in
induction proofs.  The writeup below is closer to what you might see
in print and should be prepared to produce yourself.

\begin{proof}[Revised proof of the Theorem]
We use induction.  The induction hypothesis, $P(n)$, will be
equation~\eqref{sum-to-n-again}.

\textbf{Base case:} $P(0)$ is true, because both sides of
equation~\eqref{sum-to-n-again} equal zero when $n=0$.

\textbf{Inductive step:} Assume that $P(n)$ is true, where
$n$ is any nonnegative integer.  Then
\begin{align*}
1 + 2 + 3 + \cdots + n + (n+1)
    & = \frac{n(n+1)}{2} + (n+1) & \text{(by induction hypothesis)}\\
    & = \frac{(n+1)(n+2)}{2}  & \text{(by simple algebra)}
\end{align*}
which proves $P(n+1)$.

So it follows by induction that $P(n)$ is true for all nonnegative $n$.
\end{proof}

Induction was helpful for \emph{proving the correctness} of this
summation formula, but not helpful for \emph{discovering} it in the
first place.  Tricks and methods for finding such formulas will be
covered in Part~III of the text.  %add \ref to Part III


\iffalse
\subsection{Powers of Odd Numbers}

\begin{fact*}
The $n$th power of an odd number is odd, for all nonnegative integers, $n$.
\end{fact*}
The proof in Chapter~\ref{C01} that $\sqrt[n]{2}$ is irrational used this 
``obvious'' fact.  Instead of taking it for granted, we can prove this fact
by induction.
The proof will require a simple Lemma.
\begin{lemma*}
The product of two odd numbers is odd.
\end{lemma*}
To prove the Lemma, note that the odd numbers are, by definition, the
numbers of the form $2k+1$ where $k$ is an integer.  But
\[
(2k+1)(2k'+1) = 2(2kk' + k + k')+1,
\]
so the product of two odd numbers also has the form of an odd number,
which proves the Lemma.

Now we will prove the Fact using the induction hypothesis
\[
P(n) \eqdef \text{if $a$ is an odd integer, then so is $a^{n}$}.
\]

The base case $P(0)$ holds because $a^{0} =1$, and 1 is odd.

For the inductive step, suppose $n\geq 0$, $a$ is an odd number and $P(n)$
holds.  So $a^n$ is an odd number.  Therefore, $a^{n+1} = a^{n}a$ is a
product of odd numbers, and by the Lemma $a^{n+1}$ is also odd.  This
proves $P(n+1)$, and we conclude by induction that $P(n)$ holds for
nonnegative integers $n$.
\fi

\iffalse
An alternative proof of Lemma~\ref{finmin} that every partial order on a
nonempty finite set has a minimal element can be based on induction.  This
time there is no $n$ mentioned, so we better find one.

We'll use the induction hypothesis
\[
P(n) \eqdef \text{a strict partial order on a set of size $n$ has a minimal
  element}.
\]

As a base case, we'll use $n=1$.  Now $P(1)$ holds because in a
one-element partial order, the element is minimal (and maximal) by
definition.

For the inductive step, assume $P(n)$ holds and consider a strict partial
order, $R$, on a set, $A$, of size $n+1$ for $n \geq 1$.  We will prove
that $A$ has a minimal element.

Now $A$ has 2 or more elements, so pick one and call it $a_0$.  If $a_0$
is a minimal element, then we are done.  Otherwise, let $A'$ be the set $A
- \set{a_0}$ and $R'$ be the relation $R$ restricted to $A'$.

Now it's easy to check that $R'$ is a strict partial order on set $A'$
whose size is $n$.  So by induction, there is an $R'$-minimal element, $m
\in A'$.  We claim that $m$ is also a minimal element of $A$.

Now there is no element $a' \in A'$ such that $a'\,R\,m$, so to prove
$m$ is minimal in $A$,  as long as it is not true that $a_0\,R\, m$

This element $m$ will also be minimal in $A$ unless

Since $a_0$ is not minimal, there is an element $a_1 \in A'$ such that
$a_1\,R\,a_0$.

\fi

\subsection{A More Challenging Example}

\begin{figure}

\graphic{Fig_3-1}

\caption{A $2^n \times 2^n$ courtyard for $n = 3$.}
\label{fig:2nx2n-tile}
\end{figure}

During the development of MIT's famous Stata Center, as costs rose
further and further beyond budget, there were some radical fundraising
ideas.  One rumored plan was to install a big courtyard with
dimensions $2^n \times 2^n$ with one of the central
squares\footnote{In the special case $n = 0$, the whole courtyard
consists of a single central square; otherwise, there are four central
squares.} occupied by a statue of a wealthy potential donor ---who we
will refer to as ``Bill'', for the purposes of preserving anonymity.
The $n = 3$ case is shown in Figure~\ref{fig:2nx2n-tile}.

A complication was that the building's unconventional architect, Frank
Gehry, was alleged to require that only special L-shaped tiles (shown
in Figure~\ref{fig:Ltile}) be used for the courtyard.  For $n = 2$, a
courtyard meeting these constraints is shown in
Figure~\ref{fig:2Ltile}.  But what about for larger values of~$n$?  Is
there a way to tile a $2^n \times 2^n$ courtyard with L-shaped tiles
around a statue in the center?  Let's try to prove that this is so.

\begin{figure}

\graphic{Fig_3-2}

\caption{The special L-shaped tile.}
\label{fig:Ltile}
\end{figure}

\begin{figure}

\graphic{Fig_3-3}

\caption{A tiling using L-shaped tiles for $n = 2$ with Bill in a
  center square.}
\label{fig:2Ltile}
\end{figure}

\begin{theorem}\label{bill}
For all $n \geq 0$ there exists a tiling of a $2^n \times 2^n$
courtyard with Bill in a central square.
\end{theorem}

\begin{proof}
\emph{(doomed attempt)} The proof is by induction.  Let $P(n)$ be the
proposition that there exists a tiling of a $2^n \times 2^n$ courtyard
with Bill in the center.

\textbf{Base case:} $P(0)$ is true because Bill fills the whole courtyard.

\textbf{Inductive step:} Assume that there is a tiling of a
$2^n \times 2^n$ courtyard with Bill in the center for some $n \geq
0$.  We must prove that there is a way to tile a $2^{n+1} \times
2^{n+1}$ courtyard with Bill in the center \dots.
\end{proof}

Now we're in trouble!  The ability to tile a smaller courtyard with Bill
in the center isn't much help in tiling a larger courtyard with Bill in
the center.  We haven't figured out how to bridge the gap between $P(n)$
and $P(n+1)$.

So if we're going to prove Theorem~\ref{bill} by induction, we're going to
need some \emph{other} induction hypothesis than simply the statement
about $n$ that we're trying to prove.


%Hide after lecture:

\textbf{Maybe you can figure one a good induction hypothesis for
  tiling.  In class we'll present some hypotheses that do work.}

%end hide

\iffalse  %Unhide after lecture:

When this happens, your first fallback should be to look for a
\emph{stronger} induction hypothesis; that is, one which implies
your previous hypothesis.  For example, we could make $P(n)$ the
proposition that for \emph{every} location of Bill in a $2^n \times
2^n$ courtyard, there exists a tiling of the remainder.

This advice may sound bizarre: ``If you can't prove something, try to
prove something grander!''  But for induction arguments, this makes
sense.  In the inductive step, where you have to prove $P(n) \QIMPLIES
P(n+1)$, you're in better shape because you can \emph{assume} $P(n)$,
which is now a more powerful statement.  Let's see how this plays out
in the case of courtyard tiling.

\begin{proof}[Proof (successful attempt)]
The proof is by induction.  Let $P(n)$ be the proposition that for
every location of Bill in a $2^n \times 2^n$ courtyard, there exists a
tiling of the remainder.

\textbf{Base case:} $P(0)$ is true because Bill fills the
whole courtyard.

\textbf{Inductive step:} Assume that $P(n)$ is true for some
$n \geq 0$; that is, for every location of Bill in a $2^n \times 2^n$
courtyard, there exists a tiling of the remainder.  Divide the
$2^{n+1} \times 2^{n+1}$ courtyard into four quadrants, each $2^n
\times 2^n$.  One quadrant contains Bill (\textbf{B} in the diagram
below).  Place a temporary Bill (\textbf{X} in the diagram) in each of
the three central squares lying outside this quadrant as shown in
Figure~\ref{fig:stronger-bill}.

\begin{figure}

\graphic{Fig_3-4}

\caption{Using a stronger inductive hypothesis to prove
  Theorem~\ref{bill}.}
\label{fig:stronger-bill}
\end{figure}

Now we can tile each of the four quadrants by the induction
assumption.  Replacing the three temporary Bills with a single
L-shaped tile completes the job.  This proves that $P(n)$ implies
$P(n+1)$ for all $n \geq 0$.  Thus $P(m)$ is true for all $m \in
\naturals$, and the theorem follows as a special case where we put
Bill in a central square.
\end{proof}

This proof has two nice properties.  First, not only does the argument
guarantee that a tiling exists, but also it gives an algorithm for
finding such a tiling.  Second, we have a stronger result: if Bill
wanted a statue on the edge of the courtyard, away from the pigeons,
we could accommodate him!

Strengthening the induction hypothesis is often a good move when an
induction proof won't go through.  But keep in mind that the stronger
assertion must actually be \emph{true}; otherwise, there isn't much hope
of constructing a valid proof!  Sometimes finding just the right induction
hypothesis requires trial, error, and insight.  For example,
mathematicians spent almost twenty years trying to prove or disprove the
conjecture that ``Every planar graph is
5-choosable''\footnote{5-choosability is a slight generalization of
  5-colorability.  Although every planar graph is 4-colorable and
  therefore 5-colorable, not every planar graph is 4-choosable.  If this
  all sounds like nonsense, don't panic.  We'll discuss graphs, planarity,
  and coloring in Part~II of the text.}.  Then, in 1994, Carsten Thomassen
gave an induction proof simple enough to explain on a napkin.  The key
turned out to be finding an extremely clever induction hypothesis; with
that in hand, completing the argument was easy!

\fi  %end UnHide after lecture

\subsection{A Faulty Induction Proof}

If we have done a good job in writing this text, right about now you
should be thinking, ``Hey, this induction stuff isn't so hard after
all ---just show $P(0)$ is true and that $P(n)$ implies $P(n+1)$ for
any number~$n$.''  And, you would be right, although sometimes when
you start doing induction proofs on your own, you can run into
trouble.  For example, we will now attempt to ruin your day by using
induction to ``prove'' that all horses are the same color.  And just
when you thought it was safe to skip class and work on your robot
program instead.  Bummer!

\begin{falsethm*}
All horses are the same color.
\end{falsethm*}

Notice that no $n$ is mentioned in this assertion, so we're going to have
to reformulate it in a way that makes an $n$ explicit.  In particular,
we'll (falsely) prove that

\begin{falsethm}\label{horses}
In every set of $n \geq 1$ horses, all the horses are the same color.
\end{falsethm}

This a statement about all integers $n \geq 1$ rather $\geq 0$, so it's
natural to use a slight variation on induction: prove $P(1)$ in the base
case and then prove that $P(n)$ implies $P(n+1)$ for all $n \geq 1$ in the
inductive step.  This is a perfectly valid variant of induction and is
\emph{not} the problem with the proof below.

\begin{bogusproof}

The proof is by induction on $n$.  The induction hypothesis, $P(n)$,
will be
\begin{equation}\label{horsehyp}
\text{In every set of $n$ horses, all are the same color.}
\end{equation}

\textbf{Base case:} ($n=1$).  $P(1)$ is true, because in a set of horses
of size 1, there's only one horse, and this horse is definitely the same
color as itself.

\textbf{Inductive step:} Assume that $P(n)$ is true for some $n \geq 1$.
That is, assume that in every set of $n$ horses, all are the same color.
Now suppose we have a set of $n+1$ horses:
\[
h_1,\ h_2,\ \dots,\ h_n,\ h_{n+1}.
\]
We need to prove these $n+1$ horses are all the same color.

By our assumption, the first $n$ horses are the same color:
\[
\underbrace{h_1,\ h_2,\ \dots,\ h_n}_{\text{same color}}, h_{n+1}
\]
Also by our assumption, the last $n$ horses are the same color:
\[
h_1,\ \underbrace{h_2,\ \dots,\ h_n,\ h_{n+1}}_{\text{same color}}
\]
So $h_1$ is the same color as the remaining horses besides $h_{n+1}$
---that is, $h_2, \dots, h_n$.  Likewise, $h_{n+1}$ is the same
color as the remaining horses besides $h_1$ ---that is, $h_2, \dots,
h_n$, again.  Since $h_1$ and $h_{n+1}$ are the same color as $h_2,
\dots, h_n$, all $n+1$ horses must be the same color, and so $P(n+1)$
is true.  Thus, $P(n)$ implies $P(n+1)$.

By the principle of induction, $P(n)$ is true for all $n \geq 1$.
\end{bogusproof}
We've proved something false!  Is math broken?  Should we all become
poets?  No, this proof has a mistake.

%hide after lecture:

\textbf{See if you can figure it out before we take it up in class.}

%end hide

\iffalse %UNHIDE after lecture

The mistake in this argument is in the sentence that begins ``So $h_1$
is the same color as the remaining horses besides $h_{n+1}$ ---that is
$h_2, \dots, h_n, \dots$.''  The ``$\dots$'' notation in the
expression ``$h_1, h_2, \dots, h_n, h_{n+1}$'' creates the impression
that there are some remaining horses ---namely $h_2, \dots, h_n$
---besides $h_1$ and $h_{n+1}$.  However, this is not true when $n =
1$.  In that case, $h_1, h_2, \dots, h_n, h_{n+1}$ is just $h_1, h_2$
and \emph{there are no ``remaining'' horses} for $h_1$ to share a
color with.  And of course in this case $h_1$ and $h_2$ really don't
need to be the same color.

This mistake knocks a critical link out of our induction argument.  We
proved $P(1)$ and we \emph{correctly} proved $P(2) \implies P(3)$, $P(3)
\implies P(4)$, etc.  But we failed to prove $P(1) \implies P(2)$, and so
everything falls apart: we can not conclude that $P(2)$, $P(3)$, etc., are
true.  And, of course, these propositions are all false; there are
sets of $n$ horses of different colors for all $n \ge 2$.

\fi

%end unhide

Students sometimes explain that the mistake in the proof is because
$P(n)$ is false for $n \geq 2$, and the proof assumes something false,
namely, $P(n)$, in order to prove $P(n+1)$.  You should think about
how to explain to such a student why this explanation would get no
credit on a Math for Computer Science exam.

%% Ordinary Induction Problems %%%%%%%%%%%%%%%%%%%%%%%%%%%%%%%%%%%%%%%%%%%%%%%%
\begin{problems}
%\practiceproblems
\classproblems
\pinput{CP_cubic_series}
\pinput{CP_geometric_series_induction}
\pinput{CP_sum_of_inverse_squares_induction}
\pinput{CP_courtyard_tiling_corner}
\pinput{CP_flawed_induction_proof}
\pinput{CP_false_arithmetic_series_proof}
\homeworkproblems
\pinput{PS_sums_and_products_of_integers}
\pinput{PS_ripple_carry_adder_correctness}
\pinput{PS_periphery_length_game}
\end{problems}

%%%%%%%FTL
\section{State Machines}
State machines are a simple abstract model of step-by-step processes.
Since computer programs can be understood as defining step-by-step
computational processes, it's not surprising mthat state machines come
up regularly in computer science.  They also come up in many other
settings such as digital circuit design and modeling of probabilistic
processes.  This section introduces \term{Floyd's Invariance
  Principle} which is a version of induction tailored specifically for
proving propeorties of state machines.

\iffalse
You may already have seen them in a digital logic course,
a compiler course, or a probability course.
\fi

One of the most important uses of induction in computer science
involves proving one or more desirable properties continues to hold at
every-step in a process.  A property that is preserved through a
series of operations or steps is known as an \term{invariant}.
Examples of desirable invariants include properties such as a variable
never exceeding a certain value, the altitude of a plane never
dropping below 1,000 feet without the wingflaps
\iffalse and landing gear\fi
being deployed, and the temperature of a nuclear reactor never
exceeding the threshold for a meltdown.

\iffalse  %%FTL
In particular, we show that the proposition is true at the beginning
(this is the base case) and that if it is true after $t$ steps have
been taken, it will also be true after step~$t + 1$ (this is the
inductive step).  We can then use the induction principle to conclude
that the proposition is indeed an invariant, namely, that it will
always hold.
\fi

\subsection{States and Transitions}

Formally, a state machine is nothing more than a binary relation on a
set, except that the elements of the set are called ``states,'' the
relation is called the \term{transition relation}, and an arrow in the
graph of the transition relation is called a \term{transition}.  A
transition from state $q$ to state $r$ will be written $q \movesto r$.
The transition relation is also called the \term{state graph} of the
machine.  A state machine also comes equipped with a designated
\emph{start state}.

A simple example is a bounded counter, which counts from $0$ to $99$
and overflows at 100.  This state machine is pictured in
Figure~\ref{fig:counter}, with states pictured a circles, transitions
by arrows, and with start state 0 indicated by the double circle.
\begin{figure}
\graphic{counter}
\caption{\em State transitions for the 99-bounded counter.}
\label{fig:counter}
\end{figure}
To be precise, what the picture tells us is that this bounded counter machines has

\begin{align*}
\text{states} &  \eqdef \set{0, 1,\dots,99, \text{overflow}},\\
\text{start state}  & \eqdef 0,\\
\text{transitions} & \eqdef \set{n \movesto n+1 \suchthat 0 \le n < 99}
                     \union \set{99 \movesto \text{overflow},
                                 \text{overflow} \movesto \text{overflow}}.
\end{align*}
This machine isn't much use once it overflows, since it has no way to
get out of its overflow state.

State machines for digital circuits and string pattern matching
algorithms, for example, usually have only a finite number of states.
Machines that model continuing computations typically have an infinite
number of states.  For example, instead of the 99-bounded counter, we
could easily define an ``unbounded'' counter that just keeps counting up
without overflowing.  The unbounded counter has an infinite state set,
namely, the nonnegative integers, which makes its state diagram harder to
draw. \texttt{:-)}

States machines are often defined with labels on states and/or transitions
to indicate such things as input or output values, costs, capacities, or
probabilities.  Our state machines don't include any such labels because
they aren't needed for our purposes.  We do name states, as in
Figure~\ref{fig:counter}, so we can talk about them, but the names aren't
part of the state machine.

\subsection{Invariance for a Diagonally-Moving Robot}
Suppose we have a robot that moves on an infinite 2-dimensional
integer grid.  The \emph{state} of the robot at any time can be
specified by the integer coordinates $(x, y)$ of the robot's current
position.  The \emph{start state} is~$(0, 0)$ since it is given that
the robot starts at that position.  At each step, the robit may to a
diagonally adjacent grid point.  To be precise, robot's transitions
are:
\[
\set{(m,n)\movesto (m\pm 1, n\pm 1) \suchthat m,n \in \integers}.
\]
For example, after the first step, the robot could be in states $(1,
1)$, $(1, -1)$, $(-1, 1)$, or $(-1, -1)$.  After two steps, there are
9 possible states for the robot, including~$(0, 0)$.

Can the robot ever reach position~$(1, 0)$?

If you play around with the robot a bit, you'll probably notice that
the robot can only reach positions~$(m, n)$ for which $m + n$ is even,
which means, of course, that it can't reach $(1,0)$.  This all follows
because evenness of the sum of coordinates is preserved by
transitions.

This once, let's go through this preserved-property argument again
carefully highlighting where induction comes in.  Namely, define the
even-sum property of states to be:
\[
\text{Even-sum}((m,n)) \eqdef [m+n \text{ is even}].
\]
\begin{lemma}\label{even-sum-invar}
For any transition, $q \movesto r$, of the diagonally-moving robot, if
Even-sum($q$), then Even-sum($r$).
\end{lemma}
This lemma follows immediately from the definition of the robot's
transitions: $(m,n)\movesto (m\pm 1, n\pm 1)$.  After a transition,
the sum of coordinates changes by $(\pm 1) + (\pm 1)$, that is, by 0,
2, or -2.  Of course, adding 0, 2 or -2 to an even number gives an
even number.  So by a trivial induction on the number of transitions,
we can prove:
\begin{theorem}\label{th:diag-robot}
The sum of the coordinates of any state reachable by the
diagonally-moving robot is even.
\end{theorem}

\begin{proof}
The proof is induction on the number of transitions the robot has
made.  The induction hypothesis is
\[
P(n) \eqdef \text{if $q$ a state reachable in $n$ transitions, then
  Even-sum($q$)}.
\]

\textbf{base case}: $P(0)$ is true since the only state reachable in 0
transitions is the start state $(0, 0)$, and $0 + 0$ is even.

\textbf{inductive case} Assume that $P(n)$ is true, and let $r$ be any
state reachable in $n+1$ transitions. We need to prove that
Even-sum($r$) holds.

Now since $r$ is reachable in $n+1$ transitions, there must be a
state, $q$, reachable in $n$ transitions such that $q \movesto r$.  Now
Even-sum($q$) holds since $P(n)$ is assumed to be true, so by
Lemma~\ref{even-sum-invar}, Even-sum($r$) also holds.  This proves
that $P(n) \QIMPLIES P(n + 1)$ as required, completing the proof of
the inductive step.

We conclude by induction that for all $n \ge 0$, if $q$ is reachable
in $n$ transitions, then Even-sum($q$).  This implies that every
reachable state has the Even-sum property.

\end{proof}

\begin{corollary}\label{cor:diag-robot}
The robot can never reach position~$(1, 0)$.
\end{corollary}

\begin{proof}
By Theorem~\ref{th:diag-robot}, we know the robot can only reach
positions with coordinates that sum to an even number, and thus it
cannot reach position~$(1, 0)$.
\end{proof}

\iffalse
Since this was the first time we proved that a predicate was an
invariant, we were careful to go through all four cases in gory
detail.  As you become more experienced with such proofs, you will
likely become more brief as well.  Indeed, if we were going through
the proof again at a later point in the text, we might simply note
that the sum of the coordinates after step~$t + 1$ can be only $x +
y$, $x + y + 2$ or $x + y - 2$ and therefore that the sum is even.
%%%%%%%%%%%%FTL
\fi

\subsection{The Invariance Principle}
Using the Even-sum invariant to understand the diagonally-moving robot
is a simple example of a basic proof method called The Invariance
Principle.  The Principle summarizes how induction on the number of
steps to reach a state applies to invariants.  To formulate it
precisely, we need a definition of \term{reachability.}

\begin{definition}
The \term{reachable states} of a state machine, $M$, are defined
recursively as follows:
\begin{itemize}
\item the start state is reachable, and
\item if $p$ is a reachable state of $M$, and $p \movesto q$ is a
  transition of $M$, then $q$ is also a reachable state of $M$.
\end{itemize}
\end{definition}

\iffalse
A (possibly infinite) path through the state graph beginning at the
start state corresponds to a possible system behavior; such a path is
called an \term{execution} of the state machine.  A state is called
\term{reachable} if it appears in some execution.\fi

\begin{definition}
  A \term{preserved invariant} of a state machine is a predicate, $P$, on
  states, such that whenever $P(q)$ is true of a state, $q$, and $q
  \movesto r$ for some state, $r$, then $P(r)$ holds.
\end{definition}

\textbox{
\textboxheader{The Invariant Principle}
{\large 
\noindent If a preserved invariant of a state machine is true for the
start state,\\
then it is true for all reachable states.}}

The Invariant Principle is nothing more than the Induction Principle
reformulated in a convenient form for state machines.  Showing that a
predicate is true in the start state is the base case of the induction,
and showing that a predicate is a preserved invariant corresponds to the
inductive step.\footnote{Preserved invariants are commonly just called
  ``invariants'' in the literature on program correctness, but we decided
  to throw in the extra adjective to avoid confusion with other
  definitions.  For example, other texts (as well as another subject at
  MIT) use ``invariant'' to mean ``predicate true of all reachable
  states.''  Let's call this definition ``invariant-2.''  Now invariant-2
  seems like a reasonable definition, since unreachable states by
  definition don't matter, and all we want to show is that a desired
  property is invariant-2.  But this confuses the \emph{objective} of
  demonstrating that a property is invariant-2 with the \emph{method} of
  finding a \emph{preserved} invariant to \emph{show} that it is
  invariant-2.}

\graphic{Robert_Floyd_pic}

\textbox{
\textboxheader{Robert W. Floyd}

The Invariant Principle was formulated by Robert Floyd at Carnegie
Tech\footnote{The following year, Carnegie Tech was renamed
  Carnegie-Mellon Univ.} in 1967.  Floyd was already famous for work on
formal grammars that transformed the field of programming language
parsing; that was how he got to be a professor even though he never got a
Ph.D.  (He was admitted to a PhD program as a teenage prodigy, but flunked
out and never went back.)

In that same year, Albert R Meyer was appointed Assistant Professor in
the Carnegie Tech Computer Science Department where he first met Floyd.
Floyd and Meyer were the only theoreticians in the department, and they
were both delighted to talk about their shared interests.  After just a
few conversations, Floyd's new junior colleague decided that Floyd was the
smartest person he had ever met.

Naturally, one of the first things Floyd wanted to tell Meyer about was
his new, as yet unpublished, Invariant Principle.  Floyd explained the
result to Meyer, and Meyer wondered (privately) how someone as brilliant
as Floyd could be excited by such a trivial observation.  Floyd had to
show Meyer a bunch of examples before Meyer understood Floyd's excitement
---not at the truth of the utterly obvious Invariant Principle, but rather
at the insight that such a simple method could be so widely and easily
applied in verifying programs.

Floyd left for Stanford the following year.  He won the Turing award
---the ``Nobel prize'' of computer science--- in the late 1970's, in
recognition both of his work on grammars and on the foundations of
program verification.  He remained at Stanford from 1968 until his
death in September, 2001.  You can learn more about Floyd's life and
work by reading the
\href{http://courses.csail.mit.edu/6.042/spring10/floyd-eulogy-by-knuth.pdf}{eulogy}
written by his closest colleague, Don Knuth.

 \iffalse
   \href{http://oldwww.acm.org/pubs/membernet/stories/floyd.pdf}
{\texttt{http://oldwww.acm.org/pubs/membernet/stories/floyd.pdf}}.
\fi
}

\subsection{The Die Hard Example}
The movie \textit{Die Hard 3: With a Vengeance} includes an amusing
example of a state machine.  The lead characters played by Samuel
L. Jackson and Bruce Willis have to disarm a bomb planted by the
diabolical Simon Gruber:

\textbox{
\begin{list}{}{\itemsep=0in \leftmargin=0.25in \rightmargin=0.25in}

\item[\textbf{Simon:}] On the fountain, there should be 2 jugs, do you
see them?  A 5-gallon and a 3-gallon.  Fill one of the jugs with
exactly 4 gallons of water and place it on the scale and the timer
will stop.  You must be precise; one ounce more or less will result in
detonation.  If you're still alive in 5 minutes, we'll speak.

\item[\textbf{Bruce:}] Wait, wait a second. I don't get it. Do you get it?

\item[\textbf{Samuel:}] No.

\item[\textbf{Bruce:}] Get the jugs. Obviously, we can't fill the 3-gallon jug
with 4 gallons of water.

\item[\textbf{Samuel:}] Obviously.

\item[\textbf{Bruce:}] All right. I know, here we go. We fill the 3-gallon jug
exactly to the top, right?

\item[\textbf{Samuel:}] Uh-huh.

\item[\textbf{Bruce:}] Okay, now we pour this 3 gallons into the 5-gallon jug,
giving us exactly 3 gallons in the 5-gallon jug, right?

\item[\textbf{Samuel:}] Right, then what?

\item[\textbf{Bruce:}] All right. We take the 3-gallon jug and fill it a third
of the way...

\item[\textbf{Samuel:}] No!  He said, ``Be precise.''  Exactly 4
gallons.

\item[\textbf{Bruce:}] Sh - -.  Every cop within 50 miles is running his a - - off
and I'm out here playing kids games in the park.

\item[\textbf{Samuel:}] Hey, you want to focus on the problem at hand?

\end{list}
}

Fortunately, they find a solution in the nick of time.  You can work out
how.

\subsubsection{The Die Hard 3 State Machine}
The jug-filling scenario can be modeled with a state machine that keeps
track of the amount, $b$, of water in the big jug, and the ammount, $l$,
in the little jug.  With the 3 and a 5 gallon water jugs, the states
formally will be pairs, $(b,l)$ of real numbers such that $0 \leq b \leq
5, 0 \leq l \leq 3$.  (We can prove that the reachable values of $b$ and
$l$ will be nonnegative integers, but we won't assume this.)  The start
state is $(0,0)$, since both jugs start empty.

Since the amount of water in the jug must be known exactly, we will only
consider moves in which a jug gets completely filled or completely
emptied.  There are several kinds of transitions:
\begin{enumerate}

\item  Fill the little jug: $(b,l) \movesto (b,3)$ for $l < 3$.

\item  Fill the big jug: $(b,l) \movesto (5,l)$ for $b<5$.

\item  Empty the little jug: $(b,l) \movesto (b,0)$ for $l>0$.

\item  Empty the big jug: $(b,l) \movesto (0,l)$ for $b>0$.

\item  Pour from the little jug into the big jug: for $l>0$,
\begin{equation*}
(b,l) \movesto
\begin{cases}
(b+l, 0) & \text{if $b + l \le 5$,}\\
(5, l - (5 - b)) & \text{otherwise.}
\end{cases}
\end{equation*}

\item Pour from big jug into little jug: for $b>0$,
\begin{equation*}
(b,l) \movesto
\begin{cases}
(0, b+l) & \text{if $b + l \le 3$,}\\
(b - (3 -l), 3) & \text{otherwise.}
\end{cases}
\end{equation*}
\end{enumerate}

Note that in contrast to the 99-counter state machine, there is more than
one possible transition out of states in the Die Hard machine.  Machines
like the 99-counter with at most one transition out of each state are
called \emph{deterministic}.  The Die Hard machine is
\emph{nondeterministic} because some states have transitions to several
different states.

The Die Hard 3 bomb gets disarmed successfully because the state (4,3)
is reachable.

%\end{example}

%\subsubsection{Reachability and Preserved Invariants}

\subsubsection{Die Hard Once and For All}
The \emph{Die Hard} series is getting tired, so we propose a final
\emph{Die Hard Once and For All}.  Here Simon's brother returns to
avenge him, and he poses the same challenge, but with the 5 gallon jug
replaced by a 9 gallon one.  The state machine has the same
specification as in Die Hard 3, with all occurrences of ``5'' replaced
by ``9.''

Now reaching any state of the form $(4,l)$ is impossible.  We prove this
using the Invariant Principle.  Namely, we define the preserved invariant
predicate, $P((b,l))$, to be that $b$ and $l$ are nonnegative integer
multiples of 3.

To prove that $P$ is a preserved invariant of Die-Hard-Once-and-For-All
machine, we assume $P(q)$ holds for some state $q \eqdef (b,l)$ and that
$q \movesto r$.  We have to show that $P(r)$ holds.  The proof divides
into cases, according to which transition rule is used.

One case is a ``fill the little jug'' transition.  This means $r =
(b,3)$.  But $P(q)$ implies that $b$ is an integer multiple of 3, and
of course 3 is an integer multiple of 3, so $P(r)$ still holds.

Another case is a ``pour from big jug into little jug'' transition.
For the subcase when there isn't enough room in the little jug to hold
all the water, namely, when $b + l > 3$, we have $r = (b -( 3 -l), 3)$.
But $P(q)$ implies that $b$ and $l$ are integer multiples of 3, which
means $b -( 3 -l)$ is too, so in this case too, $P(r)$ holds.

We won't bother to crank out the remaining cases, which can all be checked
just as easily.  Now by the Invariant Principle, we conclude that every
reachable state satisifies $P$.  But since no state of the form $(4,l)$
satisifies $P$, we have proved rigorously that Bruce dies once and for
all!

By the way, notice that the state (1,0), which satisfies $\QNOT(P)$, has a
transition to (0,0), which satisfies $P$.  So the negation of a preserved
invariant may not be a preserved invariant.

\subsection{Fast Exponentiation}\label{fast_exp_subsec}

\subsubsection{Partial Correctness \& Termination}

Floyd distinguished two required properties to verify a program.  The
first property is called \term{partial correctness}; this is the property
that the final results, if any, of the process must satisfy system
requirements.

You might suppose that if a result was only partially correct, then it
might also be partially incorrect, but that's not what Floyd meant.  The
word ``partial'' comes from viewing a process that might not terminate as
computing a \emph{partial relation}.  Partial correctness means that
\emph{when there is a result}, it is correct, but the process might not
always produce a result, perhaps because it gets stuck in a loop.

The second correctness property called \emph{termination} is that the
process does always produce some final value.

Partial correctness can commonly be proved using the Invariant Principle.
Termination can commonly be proved using the Well Ordering Principle.
We'll illustrate this by verifying a Fast Exponentiation procedure.

\subsubsection{Exponentiating}\label{fast_exp_subsubsec}
The most straightforward way to compute the $b$th power of a number, $a$,
is to multiply $a$ by itself $b-1$ times.  There is another way to do it
using considerably fewer multiplications called \term{Fast
  Exponentiation}.  The register machine program below defines the fast
exponentiation algorithm.  The letters $x,y,z,r$ denote registers that
hold numbers. An \term{assignment statement} has the form ``$z := a$'' and
has the effect of setting the number in register $z$ to be the number $a$.

\textbox{
\textboxheader{A Fast Exponentiation Program}

Given inputs $a \in \reals, b \in \naturals$,
initialize registers $x,y,z$ to $a,1,b$ respectively,
and repeat the following sequence of steps until termination:
\begin{itemize}\renewcommand{\itemsep}{0pt}
\item if $z = 0$ \textbf{return} $y$ and terminate
\item $r := \text{remainder}(z,2)$
\item $z := \quotient(z,2)$
\item if $r = 1$, then $y := xy$
\item $x := x^2$
\end{itemize}
}

We claim this program always terminates and leaves $y = a^b$.

To begin, we'll model the behavior of the program with a state
machine:
\begin{enumerate}
\item $\text{states} \eqdef \reals \cross \reals \cross \naturals$,
\item $\text{start state} \eqdef (a,1,b)$,
\item transitions are defined by the rule
\begin{equation*}
(x,y,z) \movesto
\begin{cases}
(x^2, y, \quotient(z,2)) & \text{if $z$ is nonzero and even},\\
(x^2, xy, \quotient(z,2)) & \text{if $z$ is nonzero and odd}.
\end{cases}
\end{equation*}
\end{enumerate}

The preserved invariant, $P((x,y,z))$, will be
\begin{equation}\label{yxzd}
z \in \naturals \QAND yx^z = a^b.
\end{equation}

To prove that $P$ is preserved, assume $P((x,y,z))$ holds
and that $(x,y,z) \movesto (x_t,y_t,z_t)$.  We must prove that
$P((x_t,y_t,z_t))$ holds, that is,
\begin{equation}\label{ztytxt}
z_t \in \naturals \QAND y_tx_t^{z_t} = a^b.
\end{equation}

Since there is a transition from $(x,y,z)$, we have $z \neq 0$, and since
$z \in \naturals$ by~\eqref{yxzd}, we can consider just two cases:

If $z$ is even, then we have that $x_t = x^2, y_t = y, z_t =
\quotient(z,2)$.  Therefore, $z_t \in \naturals$ and
\begin{align*}
y_tx_t^{z_t} &  = yx^{2 \cdot \quotient(z,2)}\\
           & = yx^{2 \cdot (z/2)}\\
           & = yx^z\\
          & = a^b & \mbox{(by~\eqref{yxzd})}
\end{align*}

If $z$ is odd, then we have that $x_t = x^2, y_t = xy, z_t =
\quotient(z,2)$. Therefore, $z_t \in \naturals$ and
\begin{align*}
y_tx_t^{z_t} & = xyx^{2 \cdot \quotient(z,2)}\\
& = yx^{1+2 \cdot (z-1)/2}\\
& = yx^{1+(z-1)}\\
& = yx^z\\
& = a^b & \mbox{(by~\eqref{yxzd})}
\end{align*}

So in both cases,~\eqref{ztytxt} holds, proving that $P$ is a preserved
invariant.

Now it's easy to prove partial correctness, namely, if the Fast
Exponentiation program terminates, it does so with $a^b$ in register
$y$.  This works because obviously $1\cdot a^b = a^b$, which means
that the start state, $(a,1,b)$, satisifies $P$.  By the Invariant
Principle, $P$ holds for all reachable states.  But the program
only stops when $z = 0$, so if a terminated state, $(x,y,0)$ is
reachable, then $y = yx^0 = a^b$ as required.

Ok, it's partially correct, but what's fast about it?  The answer is
that the number of multiplications it performs to compute $a^b$ is
roughly the length of the binary representation of $b$.  That is, the
Fast Exponentiation program uses roughly $\log_2 b$ multiplications
compared to the naive approach of multiplying by $a$ a total of $b-1$
times.

More precisely, it requires at most $2 (\ceil{\log_2 b}+1)$
multiplications for the Fast Exponentiation algorithm to compute $a^b$ for
$b>1$.  The reason is that the number in register $z$ is initially $b$,
and gets at least halved with each transition.  So it can't be halved more
than $\ceil{\log_2 b}+1$ times before hitting zero and causing the
program to terminate.  \iffalse The $(b+1)$ comes in because for $b =
2^p$, a power of two, it takes $(p+1)$ halves to get zero.  \fi Since each
of the transitions involves at most two multiplications, the total number
of multiplications until $z=0$ is at most $2(\ceil{\log_2 b}+1)$ for $b
> 0$ (see Problem~\ref{PS_2logb_mults}).


\iffalse
The \index{GCD algorithm}\term{Euclidean algorithm} is a
three-thousand-year-old procedure to compute the greatest common divisor,
$\gcd(a,b)$ of integers $a$ and $b$.  We can represent this algorithm as a
state machine.  A state will be a pair of integers $(x,y)$ which we can
think of as integer registers in a register program.  The state
transitions are defined by the rule
\[
(x,y) \movesto (y, \remainder(x,y))
\]
for $y \neq 0$.  The algorithm terminates when no further transition is
possible, namely when $y=0$.  The final answer is in $x$.

We want to prove:
\begin{enumerate}
\item Starting from the state with $x = a$ and $y = b>0$, if we ever finish,
then we have the right answer.  That is, at termination, $x = \gcd(a,b)$.
This is a \emph{partial correctness} claim.

\item We do actually finish.  This is a process \emph{termination} claim.

\end{enumerate}

\paragraph{Partial Correctness of GCD}

First let's prove that if GCD gives an answer, it is a correct answer.
Specifically, let $d \eqdef \gcd(a,b)$.  We want to prove that \emph{if}
the procedure finishes in a state $(x,y)$, then $x = d$.

\begin{proof}
Define the state predicate
\[
P(x,y) \eqdef\ \ [\gcd(x,y) = d \text{ and } (x > 0 \text{ or } y > 0)].
\]

$P$ holds for the start state $(a,b)$, by definition of $d$ and the
requirement that $b$ is positive.  Also, the preserved invariance of
$P$ follows immediately from
\begin{lemma}\label{gcdlem}
For all $m,n \in \naturals$ such that $n \neq 0$,
\begin{equation}
\gcd(m,n) = \gcd(n,\remainder(m,n)).
\end{equation}
\end{lemma}

Lemma~\ref{gcdlem} is easy to prove: let $q$ be the quotient and $r$
be the remainder of $m$ divided by $n$.  Then $m = qn +r$ by
definition.  So any factor of both $r$ and $n$ will be a factor of
$m$, and similarly any factor of both $m$ and $n$ will be a factor of
$r$.  So $r,n$ and $m,n$ have the same common factors and therefore
the same gcd.  Now by the Invariant Principle, $P$ holds for all
reachable states.

Since the only rule for termination is that $y=0$, it follows that if
$(x,y)$ is a terminal state, then $y=0$.  If this terminal state is
reachable, then the preserved invariant holds for $(x,y)$.  This implies
that $\gcd(x,0) = d$ and that $x>0$.  We conclude that $x = \gcd(x,0) =
d$.
\end{proof}

\paragraph{Termination of GCD}

Now we turn to the second property, that the procedure must terminate.  To
prove this, notice that $y$ gets strictly smaller after any one
transition.  That's because the value of $y$ after the transition is the
remainder of $x$ divided by $y$, and this remainder is smaller than $y$ by
definition.  But the value of $y$ is always a nonnegative integer, so by the
Well Ordering Principle, it reaches a minimum value among all its values
at reachable states.  But there can't be a transition from a state where
$y$ has its minimum value, because the transition would decrease $y$ still
further.  So the reachable state where $y$ has its minimum value is a
state at which no further step is possible, that is, at which the
procedure terminates.

Note that this argument does not prove that the minimum value of $y$ is
zero, only that the minimum value occurs at termination.  But we already
noted that the only rule for termination is that $y=0$, so it follows that
the minimum value of $y$ must indeed be zero.

\subsubsection{The Extended Euclidean Algorithm}\label{ExtendedGCD}

An important fact about the $\gcd(a,b)$ is that it equals an integer
linear combination of $a$ and $b$, that is,
\begin{equation}\label{sa}
\gcd(a,b) = sa+ tb
\end{equation}
for some $s,t \in \integers$.  We'll see some nice proofs
of~\eqref{sa} later when we study Number Theory, but now we'll look
at an extension of the Euclidean Algorithm that efficiently, if
obscurely, produces the desired $s$ and $t$.  It is presented here
simply as another example of application of the Invariant Method
(plus, we'll need a procedure like this when we take up number theory
based cryptography in a couple of weeks).

\emph{Don't worry if you find this Extended Euclidean Algorithm hard to
  follow, and you can't imagine where it came from.  In fact, that's good,
  because this will illustrate an important point: given the right
  preserved invariant, you can verify programs you don't understand.}

In particular, given nonnegative integers $x$ and $y$, with $y>0$, we
claim the following procedure\footnote{This procedure is adapted from Aho,
  Hopcroft, and Ullman's text on algorithms.}  halts with registers
\texttt{S} and \texttt{T} containing integers $s$ and $t$
satisfying~\eqref{sa}.

Inputs: $a,b \in \naturals, b>0$.

Registers: \texttt{X,Y,S,T,U,V,Q}.

Extended Euclidean Algorithm:
\begin{center}
\begin{verbatim}
X := a; Y := b; S := 0; T := 1; U := 1; V := 0; 
loop:
if Y divides X, then halt
else
  Q := quotient(X,Y);
         ;;the following assignments in braces are SIMULTANEOUS
 {X := Y,
  Y := remainder(X,Y);
  U := S,
  V := T,
  S := U - Q * S,
  T := V - Q * T};
goto loop;
\end{verbatim}
\end{center}

Note that \texttt{X,Y} behave exactly as in the Euclidean GCD algorithm in
Section~\ref{euclid}, except that this extended procedure stops one step
sooner, ensuring that $\gcd(x,y)$ is in \texttt{Y} at the end.  So for all
inputs $x,y$, this procedure terminates for the same reason as the
Euclidean algorithm: the contents, $y$, of register \texttt{Y} is a
nonnegative integer-valued variable that strictly decreases each time
around the loop.

The following properties are preserved invariants that imply partial
correctness:
\begin{eqnarray}
\gcd(X,Y) &=& \gcd(a,b), \label{XY}\\
Sa+Tb &=& Y,\text{ and }\label{SaTb}\\
Ua+Vb &=& X. \label{uaVb}
\end{eqnarray}

To verify that these are preserved invariants, note that~\eqref{XY} is the
same one we observed for the Euclidean algorithm.  To check the other two
properties, let $x,y,s,t,u,v$ be the contents of registers
\texttt{X,Y,S,T,U,V} at the start of the loop and assume that all the
properties hold for these values.  We must prove that~\eqref{SaTb}
and~\eqref{uaVb} hold (we already know~\eqref{XY} does) for the new
contents $x',y',s',t',u',v'$ of these registers at the next time the loop
is started.

Now according to the procedure, $u'=s,v'=t,x'=y$, so~\eqref{uaVb} holds
for $u',v',x'$ because of~\eqref{SaTb} for $s,t,y$.  Also, 
\[
s'= u - qs,\quad t'= v - qt,\quad y' = x - qy
\]
where $q = \quotient(x,y)$,
so
\[
s'a+t'b = (u-qs)a + (v-qt)b =ua+vb - q(sa+tb) = x - qy = y',
\]
and therefore~\eqref{SaTb} holds for $s',t',y'$.

Also, it's easy to check that all three preserved invariants are true just
before the first time around the loop.  Namely, at the start:
\begin{align*}
X      =a, Y=b,S=0, T& =1 & \mbox{so}\\
Sa+Tb = 0a+1b=b& =Y & \mbox{confirming~\eqref{SaTb}.}
\end{align*}
Also,
\begin{align*}
U     & =1, V=0, & \mbox{so} \\
Ua+Vb & = 1a+0b=a =X & \mbox{confirming~\eqref{uaVb}.  }
\end{align*}
Now by the Invariant Principle, they are true at termination.  But at
termination, the contents, $Y$, of register \texttt{Y} divides the
contents, $X$, of register \texttt{X}, so preserved invariants~\eqref{XY}
and~\eqref{SaTb} imply
\[
\gcd(a,b) = \gcd(X,Y) = Y = Sa + Tb.
\]
So we have the gcd in register \texttt{Y} and the desired coefficients in
\texttt{S}, \texttt{T}.

Now we don't claim that this verification offers much insight.  In fact,
if you're not wondering how somebody came up with this concise program and
invariant, you:
\begin{itemize}

\item are blessed with an inspired intellect allowing you to see how this
  program and its invariant were devised,

\item have lost interest in the topic, or

\item haven't read this far.

\end{itemize}
If none of the above apply to you, we can offer some reassurance by
repeating that you're not expected to understand this program.

We've already observed that a preserved invariant is really just an
induction hypothesis.  As with induction, finding the right hypothesis
is usually the hard part.  We repeat:
\begin{quote}
  \textbf{Given the right preserved invariant, it can be easy to verify a
    program even if you don't understand it.}
\end{quote}
We expect that the Extended Euclidean Algorithm presented above
illustrates this point.
\fi
\iffalse

\subsection{Derived Variables}\label{derived_var_subsec}

The preceding termination proofs involved finding a nonnegative
integer-valued measure to assign to states.  We might call this measure
the ``size'' of the state.  We then showed that the size of a state
decreased with every state transition.  By the Well Ordering Principle,
the size can't decrease indefinitely, so when a minimum size state is
reached, there can't be any transitions possible: the process has
terminated.

\hyperdef{derived}{vars}{More} generally, the technique of assigning
values to states ---not necessarily nonnegative integers and not necessarily
decreasing under transitions--- is often useful in the analysis of
algorithms.  \emph{Potential functions} play a similar role in physics.
In the context of computational processes, such value assignments for
states are called \emph{derived variables}.

For example, for the Die Hard machines we could have introduced a derived
variable, $f: \text{states } \to \reals$, for the amount of water in both
buckets, by setting $f((a, b)) \eqdef a + b$.  Similarly, in the robot
problem, the position of the robot along the $x$-axis would be given by
the derived variable $x\text{-coord}$, where $x\text{-coord}((i, j))
\eqdef~i$.

We can formulate our general termination method as follows:

\begin{definition}
  Let $\prec$ be a strict partial order on a set, $A$.  A derived variable
  $f : \text{states } \to A$ is \emph{strictly decreasing} iff
\[
q \movesto q' \text{  implies  } f(q') \prec f(q).
\]
\end{definition}

We confirmed termination of the GCD and Extended GCD procedures by finding
derived variables, $y$ and \texttt{Y}, respectively, that were nonnegative
integer-valued and strictly decreasing.  We can summarize this approach to
proving termination as follows:
\begin{theorem}
\label{th:decr}
If $f$ is a strictly decreasing $\naturals$-valued derived variable of a
state machine, then the length of any execution starting at state $q$ is
at most $f(q)$.
\end{theorem}

Of course we could prove Theorem~\ref{th:decr} by induction on the value
of $f(q)$, but think about what it says: ``If you start counting down at
some nonnegative integer $f(q)$, then you can't count down more than
$f(q)$ times.''  Put this way, it's obvious.

\subsubsection{Weakly Decreasing Variables}

In addition being strictly decreasing, it will be useful to have derived
variables with some other, related properties.

\begin{definition}
Let $\preceq$ be a weak partial order on a set, $A$.  A derived variable
$f : Q \to A$ is \emph{weakly decreasing} iff
\[
q \movesto q' \text{  implies  } f(q') \preceq f(q).
\]

\emph{Strictly increasing} and \emph{weakly increasing} derived variables
are defined similarly.\footnote{Weakly increasing variables are often also
called \emph{nondecreasing}.  We will avoid this terminology to prevent
confusion between nondecreasing variables and variables with the much
weaker property of \emph{not} being a decreasing variable.}
\end{definition}
\fi

\begin{problems}
\practiceproblems
\pinput{TP_die_hard_machine}
\end{problems}

\begin{editingnotes}
\section*{Well-founded termination}
omitted

\iffalse

There are cases where it's easier to prove termination based on more
general partial orders than ``less-than'' on $\naturals$.  Termination is
guaranteed whenever there is a derived variable that strictly decreases with
respect to any well-founded partial order.

\iffalse
We now define some other useful flavors of derived variables taking values
over partial ordered sets.  We'll use the notational convention that when
$\prec$ denotes a strict partial order on some set, then $\preceq$ is the
corresponding \emph{weak} partial order
\[
a\preceq a' \ \eqdef\quad a \prec a' \lor a = a'.
\]
\fi

\begin{definition}
Let $\prec$ be a strict partial order on a set, $A$.  A derived variable
$f : Q \to A$ is \emph{strictly decreasing} with respect to $\prec$ iff
\[
q \movesto q' \text{ implies } f(q') \prec f(q).
\]
Also, $f$ is \emph{weakly decreasing} iff
\[
q \movesto q' \text{  implies  } f(q') \preceq f(q).
\]
where $\preceq$ is the weak partial order corresponding to $\prec$,
namely,
\[
[a_1 \preceq a_2] \eqdef [(a_1 \prec a_2) \text{ or } (a_1=a_2)].
\]

\emph{Strictly increasing} and \emph{weakly increasing} derived variables
are defined similarly.\footnote{Weakly increasing variables are often also
called \emph{nondecreasing}.  We will avoid this terminology to prevent
confusion between nondecreasing variables and variables with the much
weaker property of \emph{not} being a decreasing variable.}
\end{definition}

\begin{theorem}\label{well-founded-decreasing}
  If there exists a derived variable for a state machine that is strictly
  decreasing with respect to some well-founded partial order, then every
  execution terminates.
\end{theorem}

Theorem~\ref{well-founded-decreasing} follows immediately from the
\href{http://courses.csail.mit.edu/6.042/spring08/ln3.pdf#infinite.decreasing}
{observation in Notes 3} that a well-founded partial order has no infinite
decreasing sequences.

Note that the existence of a nonnegative integer-valued \emph{weakly}
decreasing derived variable does not guarantee that every execution
terminates.  That's because an infinite execution could proceed through
states in which a weakly decreasing variable remained constant.

\subsubsection{A Southeast Jumping Robot}

\iffalse Begin by defining the trivial ``pick how long'' game: P1 picks $n
\in \naturals$, the P2 and P1 alternate making forced moves.  The game
ends after $n$ forced moves; the last person to move wins.  So P1 strategy
is ``pick and even number.''  Insert here the discussion of ``terminates,
but no bound on number of steps...'' used below.

May also tell the ``guess a bigger number game''joke.
\fi

Here's a contrived but simple example of proving termination based on a
variable that is strictly decreasing over a well-founded order.  Let's
think about a robot positioned at an integer lattice-point in the
Northeast quadrant of the plane, that is, at $(x,y) \in \naturals^2$.

At every second when it is away from the origin, $(0,0)$, the robot must
make a move, which may be
\begin{itemize}

\item a unit distance West when it is not at the boundary of the Northeast
  quadrant (that is, $(x,y) \movesto (x-1,y)$ for $x>0$), or

\item a unit distance South combined with an arbitrary jump East (that is,
     $(x,y) \movesto (z,y-1)$ for $z\geq x$).

\end{itemize}
\begin{claim}\label{robotcl}
The robot will always get stuck at the origin.
\end{claim}

If we think of the robot as a nondeterministic state machine, then
Claim~\ref{robotcl} is a termination assertion.  The Claim may seem
obvious, but it really has a different character than termination based on
nonnegative integer-valued variables.  That's because, even knowing that
the robot is at position $(0,1)$, for example, there is no way to bound
the time it takes for the robot to get stuck.  It can delay getting stuck
for as many seconds as it wants by making its next move to a distant point
in the Far East.  This rules out proving termination using
Theorem~\ref{th:decr}.

So does Claim~\ref{robotcl} still seem obvious?

Well it is if you see the trick: if we reverse the coordinates, then every
robot move goes to a position that is smaller under lexicographic order.
More precisely, let $f:\naturals^2 \to \naturals^2$ be the derived variable
mapping a robot state ---its position $(x,y)$ ---to $(y,x) \in
\naturals^2$.  Now $(x,y)\movesto (x',y')$ is a legitimate robot move iff
$f((x',y')) \lex< f((x,y))$.  In particular, $f$ is a strictly
$\lex<$-decreasing derived variable, so
Theorem~\ref{well-founded-decreasing} proves that the robot always get
stuck as claimed.
\fi


\iffalse

We will prove that the robot always gets stuck at the origin by
generalizing the decreasing variable method, but with decreasing values
that are more general than nonnegative integers.  Namely, the traveling robot
can be modeled with a state machine with states of the form $((x,y),s,e)$
where
\begin{itemize}
\item $(x,y) \in \naturals^2$ is the robot's position,
\item $s$ is the number of moves South the robot took to get to this
position, and
\item $e \le 2s$ is the number of moves East the robot took to get to this
position. 
\end{itemize}

Now we define a derived variable $\vl:\text{States}\to \naturals^3$:
\[
\vl(((x,y),s,e)) \ \eqdef\quad (y,2s-e,x),
\]
and we order the values of states with the \emph{lexicographic} order,
$\lexle$, on $\naturals^3$:
\begin{equation}\label{lex3}
(k,l,m) \lexle (k',l',m') \ \eqdef\quad k < k' \text{ or } (k=k' \text{
and } l < l') \text{ or } (k=k' \text{ and } l = l' \text{ and } m \le m')
\end{equation}

Let's check that values are lexicographically decreasing.  Suppose the
robot is in state $((x,y),s,e)$.
\begin{itemize}
\item If the robot moves West it enters state $((x-1,y),s,e)$, and
\[
\vl(((x-1,y),s,e)) = (y,2s-e,x-1) \lex< (y,2s-e,x) = \vl(((x,y),s,e)),
\]
as required.


\item If the robot jumps East it enters a state $((z,y),s,e+1)$ for some
$z>x$.  Now
\[
\vl(((z,y),s,e+1)) = (y,2s-(e+1),z) = (y,2s-e-1,z),
\]
but since $2s-e-1 < 2s-e$, the rule~(\ref{lex3}) implies that
\[
\vl(((z,y),s,e+1)) = (y,2s-e-1,z)  \lex< (y,2s-e,x) = \vl(((x,y),s,e)),
\]
as required.

\item If the robot moves South it enters state $((x,y-1),s+1,e)$, and
\[
\vl(((x,y-1),s+1,e)) = (y-1,2(s+1)-e,x) \lex< (y,2s-e,x) = \vl(((x,y),s,e)),
\]
as required.

\end{itemize}

So indeed state-value is a decreasing variable under lexicographic order.
But since lexicographic order is well-founded, it is impossible for a
lexicographically-ordered value to be decreased an infinite number of
times.  That's just what we need to finish verifying Claim~\ref{robotcl}.
\fi

\end{editingnotes}


\begin{problems}
\homeworkproblems

\pinput{PS_filling_buckets_with_water}

\pinput{PS_linear_combination_game}

\pinput{PS_divide_using_3}

\pinput{PS_robot_on_2D_grid}

\pinput{PS_ant_on_grid}

\pinput{PS_card_shuffle_state_machine}

%\pinput{PS_top_sort_for_closure_of_DAG}

\classproblems

\pinput{CP_fifteen_puzzle}

%\pinput{CP_fast_exponentiation}

\pinput{CP_robot_invariant}

\pinput{CP_Zakim_bridge_no_derived_vars}

\pinput{CP_98_heads_and_4_tails}

%\pinput{CP_beaver_flu}

\pinput{CP_beaver_flu_using_invariant}

\end{problems}


\section{Strong Induction}\label{strong_ind_sec}

A useful variant of induction is called \term{Strong Induction}.  Strong
induction and ordinary induction are used for exactly the same thing:
proving that a predicate is true for all nonnegative integers.  Strong
induction is useful when a simple proof that the predicate holds for $n+1$
does not follow just from the fact that it holds at $n$, but from the fact
that it holds for other values $ \le n$.

\subsection{A Rule for Strong Induction}

\textbox{
\textboxheader{Principle of Strong Induction.}

Let $P$ be a predicate on  nonnegative integers.  If
\begin{itemize}
\item $P(0)$ is true, and
\item for all $n \in \naturals$, $P(0)$, $P(1)$, \dots, $P(n)$
\emph{together} imply $P(n+1)$,
\end{itemize}
then $P(m)$ is true for all $m \in \naturals$.
}

The only change from the ordinary induction principle is that strong
induction allows you to assume more stuff in the inductive step of
your proof!  In an ordinary induction argument, you assume that $P(n)$
is true and try to prove that $P(n+1)$ is also true.  In a strong
induction argument, you may assume that $P(0)$, $P(1)$, \dots, and
$P(n)$ are \emph{all} true when you go to prove $P(n+1)$.  These extra
assumptions can only make your job easier.  Hence the name:
\emph{strong} induction.

Formulated as a proof rule, strong induction is
\begin{rul*} \textbf{Strong Induction Rule}
\Rule{P(0), \quad \forall n \in \naturals. \;
    \bigl(P(0) \QAND P(1) \QAND \dots \QAND P(n) \bigr) \QIMPLIES P(n+1)}
{\forall m \in \naturals.\, P(m)}
\end{rul*}

Stated more succintly, the rule is
\begin{rul*}% \textbf{Strong Induction Rule}
\Rule{P(0), \quad [\forall k \le n \in \naturals.\, P(k)] \QIMPLIES P(n+1)}
{\forall m \in \naturals.\, P(m)}
\end{rul*}

The template for strong induction proofs is identical to the template
given in Section~\ref{templ-induct-proofs} for ordinary induction
except for two things:
\begin{itemize}

\item
you should state that your proof is by strong induction, and

\item
you can assume that $P(0)$, $P(1)$, \dots, $P(n)$ are all true instead
of only $P(n)$ during the inductive step.

\end{itemize}

\subsection{Products of Primes}

As a first example, we'll use strong induction to re-prove
Theorem~\ref{factor_into_primes} which we previously proved using \idx{Well
Ordering}.

\begin{theorem*}%\label{primprod}
Every integer greater than 1 is a product of primes.
\end{theorem*}

\begin{proof}

We will prove the Theorem by strong induction, letting the induction 
hypothesis, $P(n)$, be
\[
n \text{ is a product of primes}.
\]
So the Theorem will follow if we prove that $P(n)$ holds for all $n
\geq 2$.

\textbf{Base Case:} ($n=2$): $P(2)$ is true because $2$ is prime, so it is
a length one product of primes by convention.

\textbf{Inductive step:} Suppose that $n \geq 2$ and that $k$ is a product
of primes for every integer $k$ where $2 \leq k \le n$.  We must show that
$P(n+1)$ holds, namely, that $n+1$ is also a product of primes.  We argue
by cases:

If $n+1$ is itself prime, then it is a length one product of primes by
convention, and so $P(n+1)$ holds in this case.

Otherwise, $n + 1$ is not prime, which by definition means $n+1 = km$ for
some integers $k,m$ such that $2 \leq k,m \le n$.  Now by the strong
induction hypothesis, we know that $k$ is a product of primes.  Likewise,
$m$ is a product of primes.  By multiplying these products, it follows
immediately that $km = n$ is also a product of primes.  Therefore,
$P(n+1)$ holds in this case as well.

So $P(n+1)$ holds in any case, which completes the proof by strong
induction that $P(n)$ holds for all ~$n \ge 2$.

\end{proof}

\subsection{Making Change}

The country Inductia, whose unit of currency is the Strong, has coins
worth 3\sg\ (3 Strongs) and 5\sg.  Although the Inductians have some
trouble making small change like 4\sg\ or 7\sg, it turns out that they
can collect coins to make change for any number that is at least 8
Strongs.

Strong induction makes this easy to prove for $n+1 \ge 11$, because then
$(n+1)-3 \ge 8$, so by strong induction the Inductians can make change for
exactly $(n+1)-3$ Strongs, and then they can add a 3\sg\ coin to get
$(n+1)\sg$.  So the only thing to do is check that they can make change
for all the amounts from 8 to 10\sg, which is not too hard to do.

Here's a detailed writeup using the official format:

\begin{proof}

  We prove by strong induction that the Inductians can make change for any
  amount of at least 8\sg.  The induction hypothesis, $P(n)$ will be:
\begin{quote}
There is a collection of coins whose value is $n+8$ Strongs.
\end{quote}

We now proceed with the induction proof:

\textbf{Base case:} $P(0)$ is true because a 3\sg\ coin together with
a 5\sg\ coin makes 8\sg.

\textbf{Inductive step:}  We assume $P(k)$ holds for all $k \leq n$, and
prove that $P(n+1)$ holds.  We argue by cases:

\textbf{Case} ($n+1$ = 1): We have to make $(n+1) +8 =9$\sg.  We can do this using three 3\sg\ coins.

\textbf{Case} ($n+1$ = 2): We have to make $(n+1) +8 =10$\sg.  Use two
5\sg\ coins.

\textbf{Case} ($n+1 \geq 3$): Then $0 \leq n - 2 \leq n$, so by the
strong induction hypothesis, the Inductians can make change for $n-2$
Strong.  Now by adding a 3\sg\ coin, they can make change for
$(n+1)\sg$.

Since $n \ge 0$, we know that $n + 1 \ge 1$ and thus that the three cases
cover every possibility.  Since $P(n+1)$ is true in every case, we can
conclude by strong induction
that for all $n \ge 0$, the Inductians can make change for $n+8$
Strong.  That is, they can make change for any number of eight or more
Strong.
\end{proof}

\subsection{The Stacking Game}

Here is another exciting game that's surely about to sweep the
nation!

%\hyperdef{stack}{game}
You begin with a stack of $n$ boxes.  Then you
make a sequence of moves.  In each move, you divide one stack of boxes
into two nonempty stacks.  The game ends when you have $n$ stacks, each
containing a single box.  You earn points for each move; in particular, if
you divide one stack of height $a + b$ into two stacks with heights $a$
and $b$, then you score $ab$ points for that move.  Your overall score is
the sum of the points that you earn for each move.  What strategy should
you use to maximize your total score?

As an example, suppose that we begin with a stack of $n = 10$ boxes.
Then the game might proceed as shown in Figure~\ref{fig:stacking-10}.
Can you find a better strategy?
%
\begin{figure}\redrawntrue
\[
\begin{array}{cccccccccccl}
\multicolumn{10}{c}{\textbf{Stack Heights}} & \quad & \textbf{Score} \\
\underline{10}&&&&&&&&& && \\
5&\underline{5}&&&&&&&& && 25 \text{ points} \\
\underline{5}&3&2&&&&&&& && 6 \\
\underline{4}&3&2&1&&&&&& && 4 \\
2&\underline{3}&2&1&2&&&&& && 4 \\
\underline{2}&2&2&1&2&1&&&& && 2 \\
1&\underline{2}&2&1&2&1&1&&& && 1 \\
1&1&\underline{2}&1&2&1&1&1&& && 1 \\
1&1&1&1&\underline{2}&1&1&1&1& && 1 \\
1&1&1&1&1&1&1&1&1&1 && 1 \\ \hline
\multicolumn{10}{r}{\textbf{Total Score}} & = & 45 \text{ points}
\end{array}
\]
\caption{An example of the stacking game with $n = 10$ boxes.  On each
line, the underlined stack is divided in the next step.}
\label{fig:stacking-10}
\end{figure}

\subsubsection{Analyzing the Game}

%Hide in full version
You will see in class how to use strong induction to analyze this game of
blocks.
%end Hide

\iffalse  %unHide after Friday lecture:

Let's use strong induction to analyze the unstacking game.  We'll prove
that your score is determined entirely by the number of boxes ---your
strategy is irrelevant!

\begin{theorem}\label{stacking}
Every way of unstacking $n$ blocks gives a score of $n(n-1)/2$ points.
\end{theorem}

There are a couple technical points to notice in the proof:

\begin{itemize}

\item The template for a strong induction proof mirrors the one for
  ordinary induction.

\item As with ordinary induction, we have some freedom to adjust indices.
In this case, we prove $P(1)$ in the base case and prove that $P(1),
\dots, P(n)$ imply $P(n+1)$ for all $n \geq 1$ in the inductive step.

\end{itemize}

\begin{proof}
The proof is by strong induction.  Let $P(n)$ be the proposition that
every way of unstacking $n$ blocks gives a score of $n(n-1)/2$.

\textbf{Base case:} If $n = 1$, then there is only one
block.  No moves are possible, and so the total score for the game is
$1(1 - 1)/2 = 0$.  Therefore, $P(1)$ is true.

\textbf{Inductive step:} Now we must show that $P(1)$, \dots, $P(n)$ imply
$P(n+1)$ for all $n \geq 1$.  So assume that $P(1)$, \dots, $P(n)$ are all
true and that we have a stack of $n+1$ blocks.  The first move must split
this stack into substacks with positive sizes $a$ and $b$ where $a+b =
n+1$ and $0<a,b\leq n$.  Now the total score for the game is the sum of
points for this first move plus points obtained by unstacking the two
resulting substacks:
%
\begin{align*}
\text{total score}
    & = \text{(score for 1st move)} \\
    & \quad + \text{(score for unstacking $a$ blocks)} \\
    & \quad + \text{(score for unstacking $b$ blocks)} \\
    & = ab + \frac{a(a-1)}{2} + \frac{b(b-1)}{2} & \text{by $P(a)$ and $P(b)$}\\
    & = \frac{(a+b)^2-(a+b)}{2} = \frac{(a+b)((a+b)-1)}{2}\\
    & = \frac{(n+1)n}{2}
\end{align*}
%
This shows that $P(1)$, $P(2)$, \dots, $P(n)$ imply $P(n+1)$.

Therefore, the claim is true by strong induction.
\end{proof}
\fi
%end unHide

%% Strong Induction Problems %%%%%%%%%%%%%%%%%%%%%%%%%%%%%%%%%%%%%%%%%%%%%%%%%%
%\practiceproblems
%\pinput{}
\begin{problems}
\classproblems
\pinput{CP_bogus_unique_prime_factors}
\pinput{CP_box_unstacking} %not strong induction, but depends on stacking game
\homeworkproblems
\pinput{PS_team_division}
\pinput{PS_2logb_mults}
\pinput{PS_bogus_prime_divides_integer_product}
\pinput{PS_bogus_Fibonacci_induction}

\end{problems}

\section{Strong Induction vs.  Induction vs.  Well Ordering}
\label{versusWO}

Strong induction looks genuinely ``stronger'' than ordinary induction
---after all, you can assume a lot more when proving the induction step.
Since ordinary induction is a special case of strong induction, you might
wonder why anyone would bother with the ordinary induction.

But strong induction really isn't any stronger, because a simple text
manipulation program can automatically reformat any proof using strong
induction into a proof using ordinary induction ---just by decorating the
induction hypothesis with a universal quantifier in a standard way.
Still, it's worth distinguishing these two kinds of induction, since which
you use will signal whether the inductive step for $n+1$ follows directly
from the case for $n$ or requires cases smaller than $n$, and that is
generally good for your reader to know.

The template for the two kinds of induction rules look nothing like the
one for the\idx{Well Ordering Principle}, but this chapter included a
couple of examples where induction was used to prove something already
proved using Well Ordering.  In fact, this can always be done.  As the
examples may suggest, any Well Ordering proof can automatically be
reformatted it into an Induction proof.  So theoretically, no one need
bother with the Well Ordering Principle either.

But wait a minute ---it's equally easy go the other way, and automatically
reformat any Strong Induction proof into a Well Ordering proof.   The
three proof methods --Well Ordering, Induction, and Strong Induction, are
simply different formats for presenting the same mathematical reasoning!

So why three methods?  Well, sometimes induction proofs are clearer
because they don't require proof by contradiction.  Also, induction proofs
often provide recursive procedures that reduce handling large inputs to
handling smaller ones.  On the other hand, Well Ordering come out slightly
shorter and sometimes seem more natural (and less worrisome to beginners).

So which method should you use? ---whichever you find easier!  But
whichever method you choose, be sure to state the method up front to
help a reader follow your proof.

\begin{editingnotes}
Here's how to reformat an induction proof and into a Well
Ordering proof : suppose that we have a proof by induction with
hypothesis $P(n)$.  Then we start a Well Ordering proof by assuming the
set of counterexamples to $P$ is nonempty.  Then by Well Ordering there is
a smallest counterexample, $s$, that is, a smallest $s$ such that $P(s)$
is false.

Now we use the proof of $P(0)$ that was part of the Induction proof to
conclude that $s$ must be greater than 0.  Also since $s$ is the smallest
counterexample, we can conclude that $P(s-1)$ must be true.  At this point
we reuse the proof of the inductive step in the Induction proof, which
shows that since $P(s-1)$ true, then $P(s)$ is also true.  This
contradicts the assumption that $P(s)$ is false, so we have the
contradiction needed to complete the Well Ordering Proof that $P(n)$ holds
for all $n \in \naturals$.

\end{editingnotes}


\begin{editingnotes}
\begin{notesproblem}
Use strong induction to prove the Well Ordering Principle. \hint Prove
that if a set of nonnegative integers contains an integer, $n$, then it
has a smallest element.
\end{notesproblem}
\end{editingnotes}

\endinput
