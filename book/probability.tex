\chapter{Events and Probability Spaces}\label{probability_chap}

\section{Let's Make a Deal}\label{monty_sec}

In the September 9, 1990 issue of \emph{Parade} magazine, columnist
Marilyn vos Savant responded to this letter:

%  future:
%  Include the other problem that Marilyn addressed in the same issue.

\begin{quote}
\emph{Suppose you're on a game show, and you're given the
choice of three doors.  Behind one door is a car, behind the others,
goats.  You pick a door, say number 1, and the host, who knows what's
behind the doors, opens another door, say number 3, which has a goat.
He says to you, "Do you want to pick door number 2?"  Is it to your
advantage to switch your choice of doors?}
\begin{flushright}
\begin{tabular}{l}
Craig. F. Whitaker \\
Columbia, MD
\end{tabular}
\end{flushright}
\end{quote}

The letter describes a situation like one faced by contestants in the
1970's game show \emph{Let's Make a Deal}, hosted by Monty Hall and
Carol Merrill.  Marilyn replied that the contestant should indeed
switch.  She explained that if the car was behind either of the two
unpicked doors---which is twice as likely as the the car being behind
the picked door---the contestant wins by switching.  But she soon
received a torrent of letters, many from mathematicians, telling her
that she was wrong.  The problem became known as the \term{Monty Hall
  Problem} and it generated thousands of hours of heated debate.

This incident highlights a fact about probability: the subject uncovers
lots of examples where ordinary intuition leads to completely wrong
conclusions.  So until you've studied probabilities enough to have
refined your intuition, a way to avoid errors is to fall back on a
rigorous, systematic approach such as the Four Step Method that we
will describe shortly.  First, let's make sure we really understand
the setup for this problem.  This is always a good thing to do when
you are dealing with probability.

\subsection{Clarifying the Problem}

Craig's original letter to Marilyn vos Savant is a bit vague, so we
must make some assumptions in order to have any hope of modeling the
game formally.  For example, we will assume that:
\begin{enumerate}

\item The car is equally likely to be hidden behind each of the three
doors.

\item The player is equally likely to pick each of the three doors,
regardless of the car's location.

\item After the player picks a door, the host \emph{must} open a
different door with a goat behind it and offer the player the choice
of staying with the original door or switching.

\item If the host has a choice of which door to open, then he is
equally likely to select each of them.

\end{enumerate}
In making these assumptions, we're reading a lot into Craig
Whitaker's letter.  There are other plausible interpretations that
lead to different answers.  But let's accept these assumptions for now
and address the question, ``What is the probability that a player who
switches wins the car?''

\section{The Four Step Method}\label{4step_sec}

Every probability problem involves some sort of randomized experiment,
process, or game.  And each such problem involves two distinct
challenges:
%
\begin{enumerate}
\item How do we model the situation mathematically?
\item How do we solve the resulting mathematical problem?
\end{enumerate}
%
In this section, we introduce a four step approach to questions of the
form, ``What is the probability that\dots ?''  In this approach, we build
a probabilistic model step-by-step, formalizing the original question in
terms of that model.  Remarkably, the structured thinking that this
approach imposes provides simple solutions to many famously-confusing
problems.  For example, as you'll see, the four step method cuts through
the confusion surrounding the Monty Hall problem like a Ginsu knife.

\subsection{Step 1:  Find the Sample Space}

Our first objective is to identify all the possible outcomes of the
experiment.  A typical experiment involves several randomly-determined
quantities.  For example, the Monty Hall game involves three such
quantities:
%
\begin{enumerate}
\item The door concealing the car.
\item The door initially chosen by the player.
\item The door that the host opens to reveal a goat.
\end{enumerate}
%
Every possible combination of these randomly-determined quantities is
called an \term{outcome}.  The set of all possible outcomes is called
the \term{sample space} for the experiment.

A \term{tree diagram} is a graphical tool that can help us work
through the four step approach when the number of outcomes is not too
large or the problem is nicely structured.  In particular, we can use
a tree diagram to help understand the sample space of an experiment.
The first randomly-determined quantity in our experiment is the door
concealing the prize.  We represent this as a tree with three
branches, as shown in Figure~\ref{fig:14A1}.  In this diagram, the
doors are called $A$, $B$, and $C$ instead of 1, 2, and 3, because
we'll be adding a lot of other numbers to the picture later.

\begin{figure}

\graphic{Monty1}

\caption{The first level in a tree diagram for the Monty Hall
  Problem.  The branches correspond to the door behind which the car
  is located.}

\label{fig:14A1}

\end{figure}

For each possible location of the prize, the player could initially
choose any of the three doors.  We represent this in a second layer
added to the tree.  Then a third layer represents the possibilities of
the final step when the host opens a door to reveal a goat, as shown
in Figure~\ref{fig:14A2}.

\begin{figure}

\graphic{Monty2}

\caption{The full tree diagram for the Monty Hall Problem.  The second
level indicates the door initially chosen by the player.  The third
level indicates the door revealed by Monty Hall.}

\label{fig:14A2}

\end{figure}

Notice that the third layer reflects the fact that the host has either one
choice or two, depending on the position of the car and the door initially
selected by the player.  For example, if the prize is behind door A and
the player picks door B, then the host must open door C.  However, if the
prize is behind door A and the player picks door A, then the host could
open either door B or door C.

\begin{figure}

\graphic{Monty3}

\caption{The tree diagram for the Monty Hal Problem with the outcomes
  labeled for each path from root to leaf.  For example, outcome $(A,
  A, B)$ corresponds to the car being behind door~$A$, the player
  initially choosing door~$A$, and Monty Hall revealing the goat
  behind door~$B$.}

\label{fig:14A3}

\end{figure}

Now let's relate this picture to the terms we introduced earlier: the
leaves of the tree represent \emph{outcomes} of the experiment, and
the set of all leaves represents the \emph{sample space}.  Thus, for
this experiment, the sample space consists of 12 outcomes.  For
reference, we've labeled each outcome in Figure~\ref{fig:14A3} with a
triple of doors indicating:
%
\[
    (\text{door concealing prize}, \;
    \text{door initially chosen}, \;
     \text{door opened to reveal a goat}).
\]
%
In these terms, the sample space is the set
%
\[
\sspace = \left\{
\begin{array}{c@{\;}c@{\;}c@{\;}c@{\;}c@{\;}c}
(A, A, B), & (A, A, C), & (A, B, C), & (A, C, B), & (B, A, C), & (B, B, A), \\
(B, B, C), & (B, C, A), & (C, A, B), & (C, B, A), & (C, C, A), & (C, C, B)
\end{array}
\right\}
\]
%
The tree diagram has a broader interpretation as well: we can regard the
whole experiment as following a path from the root to a leaf, where the
branch taken at each stage is ``randomly'' determined.  Keep this
interpretation in mind; we'll use it again later.

\subsection{Step 2: Define Events of Interest}

Our objective is to answer questions of the form ``What is the
probability that \dots ?'', where, for example, the missing phrase
might be ``the player wins by switching'', ``the player initially
picked the door concealing the prize'', or ``the prize is behind door
C.''  Each of these phrases characterizes a set of outcomes. For
example, the outcomes specified by ``the prize is behind door $C$''
is:
%
\[
    \set{(C, A, B), (C, B, A), (C, C, A), (C, C, B)}.
\]
%
A set of outcomes is called an \term{event} and it is a subset of the
sample space.  So the event that the player initially picked the door
concealing the prize is the set:
%
\[
    \set{(A, A, B), (A, A, C), (B, B, A), (B, B, C), (C, C, A), (C, C, B)}.
\]
%
And what we're really after, the event that the player wins by
switching, is the set of outcomes:
\begin{align}
\lefteqn{[\text{switching-wins}]}\notag\\
   & \eqdef \set{(A, B, C), (A, C, B), (B, A, C),
  (B, C, A), (C, A, B), (C, B, A)}.\label{swwin-event}
\end{align}
These outcomes have check marks in Figure~\ref{fig:14A4}.
\begin{figure}

\graphic{Monty4}

\caption{The tree diagram for the Monty Hall Problem where the
  outcomes in the event where the player wins by switching are denoted
  with a check mark.}

\label{fig:14A4}

\end{figure}

Notice that exactly half of the outcomes are checked, meaning that the
player wins by switching in half of all outcomes.  You might be
tempted to conclude that a player who switches wins with probability
$1/2$.  \emph{This is wrong.}  The reason is that these outcomes are
not all equally likely, as we'll see shortly.

\subsection{Step 3: Determine Outcome Probabilities}

So far we've enumerated all the possible outcomes of the experiment.  Now
we must start assessing the likelihood of those outcomes.  In particular,
the goal of this step is to assign each outcome a probability, indicating
the fraction of the time this outcome is expected to occur.  The sum of
all outcome probabilities must be one, reflecting the fact that there
always is an outcome.

Ultimately, outcome probabilities are determined by the phenomenon
we're modeling and thus are not quantities that we can derive
mathematically.  However, mathematics can help us compute the
probability of every outcome \emph{based on fewer and more
elementary modeling decisions.}  In particular, we'll break the task
of determining outcome probabilities into two stages.

\subsubsection{Step 3a: Assign Edge Probabilities}

First, we record a probability on each \emph{edge} of the tree
diagram.  These edge-probabilities are determined by the assumptions
we made at the outset: that the prize is equally likely to be behind
each door, that the player is equally likely to pick each door, and
that the host is equally likely to reveal each goat, if he has a
choice.  Notice that when the host has no choice regarding which door
to open, the single branch is assigned probability 1.  For example,
see Figure~\ref{fig:14A6}.
\iffalse

\begin{figure}

\graphic{Monty5}

\caption{The tree diagram for the Monty Hall Problem where edge
  weights denote the probability of that branch being taken, given that
  we are at the parent of that branch.  For example, if the car is
  behind door~$A$, then there is a 1/3~chance that the player's
  initial selection is door~$B$.}

\label{fig:14A5}

\end{figure}
\fi

\subsubsection{Step 3b: Compute Outcome Probabilities}

Our next job is to convert edge probabilities into outcome
probabilities.  This is a purely mechanical process:
\begin{quote}
the probability of an outcome is equal to the product of the
edge-probabilities on the path from the root to that outcome.
\end{quote}
For example, the probability of the topmost outcome in
Figure~\ref{fig:14A6}, $(A, A, B)$, is
\[
\frac{1}{3} \cdot \frac{1}{3} \cdot \frac{1}{2} = \frac{1}{18}.
\]

There's an easy, intuitive justification for this rule.  As the steps in
an experiment progress randomly along a path from the root of the tree to
a leaf, the probabilities on the edges indicate how likely the path is to
proceed along each branch.  For example, a path starting at the root in
our example is equally likely to go down each of the three top-level
branches.

How likely is such a path to arrive at the topmost outcome, $(A, A,
B)$?  Well, there is a 1-in-3 chance that a path would follow the
$A$-branch at the top level, a 1-in-3 chance it would continue along
the $A$-branch at the second level, and 1-in-2 chance it would follow
the $B$-branch at the third level.  Thus, it seems that 1~path in~18
should arrive at the $(A, A, B)$ leaf, which is precisely the
probability we assign it.

We have illustrated all of the outcome probabilities in
Figure~\ref{fig:14A6}.

\begin{figure}

\graphic{Monty6}

\caption{The tree diagram for the Monty Hall Problem where edge
  weights denote the probability of that branch being taken given that
  we are at the parent of that branch.  For example, if the car is
  behind door~$A$, then there is a 1/3~chance that the player's
  initial selection is door~$B$.
  The rightmost column shows the outcome probabilities for the
  Monty Hall Problem.  Each outcome probability is simply the product
  of the probabilities on the path from the root to
  the outcome leaf.}

\label{fig:14A6}
\end{figure}

Specifying the probability of each outcome amounts to defining a
function that maps each outcome to a probability.  This function is
usually called $\pr{\cdot}$.  In these terms, we've just determined
that:
\begin{align*}
\pr{(A, A, B)} & = \frac{1}{18}, \\
\pr{(A, A, C)} & = \frac{1}{18}, \\
\pr{(A, B, C)} & = \frac{1}{9}, \\
               & \text{etc.}
\end{align*}

\subsection{Step 4: Compute Event Probabilities}

We now have a probability for each \emph{outcome}, but we want to
determine the probability of an \emph{event}.  The probability of an
event~$E$ is denoted by $\pr{E}$ and it is the sum of the
probabilities of the outcomes in~$E$.  For example, the probability of
the [switching wins] event~\eqref{swwin-event} is
\begin{align*}
\lefteqn{\pr{\text{switching wins}}}\\
    & = \pr{(A, B, C)} + \pr{(A, C, B)} + \pr{(B, A, C)} + \\
    & \qquad \pr{(B, C, A)} + \pr{(C, A, B)} + \pr{(C, B, A)} \\
    & = \frac{1}{9} + \frac{1}{9} + \frac{1}{9} +
        \frac{1}{9} + \frac{1}{9} + \frac{1}{9} \\
    & = \frac{2}{3}.
\end{align*}
It seems Marilyn's answer is correct!  A player who switches doors
wins the car with probability~$2/3$.  In contrast, a player who stays
with his or her original door wins with probability $1/3$, since
staying wins if and only if switching loses.

We're done with the problem!  We didn't need any appeals to intuition
or ingenious analogies.  In fact, no mathematics more difficult than
adding and multiplying fractions was required.  The only hard part was
resisting the temptation to leap to an ``intuitively obvious'' answer.

\subsection{An Alternative Interpretation of the Monty Hall Problem}

Was Marilyn really right?  Our analysis indicates that she was.  But a
more accurate conclusion is that her answer is correct \emph{provided
  we accept her interpretation of the question}.  There is an equally
plausible interpretation in which Marilyn's answer is wrong.  Notice
that Craig Whitaker's original letter does not say that the host is
\emph{required} to reveal a goat and offer the player the option to
switch, merely that he \emph{did} these things.  In fact, on the
\emph{Let's Make a Deal} show, Monty Hall sometimes simply opened the
door that the contestant picked initially.  Therefore, if he wanted
to, Monty could give the option of switching only to contestants who
picked the correct door initially.  In this case, switching never
works!

%% Monty Hall Problems %%%%%%%%%%%%%%%%%%%%%%%%%%%%%%%%%%%%%%%%%%%%%%%%%%%%%%%%

\begin{problems}
\practiceproblems
%\pinput{TP_A_random_number}  MISSING
\pinput{TP_Binomial_Random_Variable}
\pinput{TP_Flipping_coins}

\examproblems
\pinput{FP_probability}

\classproblems
\pinput{CP_a_baseball_series}
\pinput{CP_coin_flips}
\pinput{CP_simulating_fair_coin}

\homeworkproblems
\pinput{PS_four_door_random_or_not}
\pinput{PS_black_and_red_cards_revised}
\end{problems}

\section{Strange Dice}

The four-step method is surprisingly powerful.  Let's get some more
practice with it.  Imagine, if you will, the following scenario.

It's a typical Saturday night.  You're at your favorite pub,
contemplating the true meaning of infinite cardinalities, when a
burly-looking biker plops down on the stool next to you.  Just as you
are about to get your mind around~$\power(\power(\reals))$, biker dude
slaps three strange-looking dice on the bar and challenges you to a
\$100 wager.  His rules are simple.  Each player selects one die and
rolls it once.  The player with the lower value pays the other
player~\$100.

Naturally, you are skeptical, especially after you see that these are
not ordinary dice.  Each die has the usual six sides, but opposite
sides have the same number on them, and the numbers on the dice are
different, as shown in Figure~\ref{fig:14A7}.

\begin{figure}

\graphic{Fig_A7}

\caption{The strange dice.  The number of pips on
  each concealed face is the same as the number on the opposite face.
  For example, when you roll die~$A$, the probabilities of getting a
  2, 6, or~7 are each~$1/3$.}

\label{fig:14A7}

\end{figure}

Biker dude notices your hesitation, so he sweetens his offer: he will
pay you \$105 if you roll the higher number, but you only need pay him
\$100 if he rolls higher, \emph{and} he will let you pick a die first,
after which he will pick one of the other two.  The sweetened deal
sounds persuasive since it gives you a chance to pick what you think
is the best die, so you decide you will play.  But which of the dice
should you choose?  Die~$B$ is appealing because it has a~9, which is
a sure winner if it comes up.  Then again, die~$A$ has two fairly
large numbers and die~$C$ has an~8 and no really small values.

In the end, you choose die~$B$ because it has a~9, and then biker dude
selects die~$A$.  Let's see what the probability is that you will win.
(Of course, you probably should have done this before picking die~$B$
in the first place.)  Not surprisingly, we will use the four-step
method to compute this probability.

\subsection{Die~$A$ versus Die~$B$}

\paragraph{Step 1: Find the sample space.}

The tree diagram for this scenario is shown in Figure~\ref{fig:14A8}.
In particular, the sample space for this experiment are the nine pairs
of values that might be rolled with Die~$A$ and Die~$B$:

For this experiment, the sample space is a set of nine outcomes:
\begin{equation*}
\sspace = \set{\, (2, 1), \, (2, 5), \, (2, 9), \,
                  (6, 1), \, (6, 5), \, (6, 9), \,
                  (7, 1), \, (7, 5), \, (7, 9) \,}.
\end{equation*}


\iffalse
\footnote{Actually, the whole
  probability space is worked out in this one picture.  But pretend
  that each component sort of fades in---nyyrrroom!---as you read
  about the corresponding step below.}\fi

\begin{figure}

\graphic{Fig_A8}

\caption{The tree diagram for one roll of die~$A$ versus die~$B$.
  Die~$A$ wins with probability~$5/9$.}

\label{fig:14A8}

\end{figure}


\paragraph{Step 2: Define events of interest.}

We are interested in the event that the number on die~$A$ is greater
than the number on die~$B$.  This event is a set of five outcomes:
\begin{equation*}
    \set{\, (2, 1), \, (6, 1), \, (6, 5), \, (7, 1), \, (7, 5) \,}.
\end{equation*}
These outcomes are marked~$A$ in the tree diagram in
Figure~\ref{fig:14A8}.

\paragraph{Step 3: Determine outcome probabilities.}

To find outcome probabilities, we first assign probabilities to edges
in the tree diagram.  Each number on each die comes up with
probability~$1/3$, regardless of the value of the other die.
Therefore, we assign all edges probability~$1/3$.  The probability of
an outcome is the product of the probabilities on the corresponding
root-to-leaf path, which means that every outcome has
probability~$1/9$.  These probabilities are recorded on the right side
of the tree diagram in Figure~\ref{fig:14A8}.

\paragraph{Step 4: Compute event probabilities.}

The probability of an event is the sum of the probabilities of the
outcomes in that event.  In this case, all the outcome probabilities
are the same, so we say that the sample space is \emph{\idx{uniform}}.
Computing event probabilities for uniform sample spaces is
particularly easy since you just have to compute the number of
outcomes in the event.  In particular, for any event~$E$ in a uniform
sample space~$\sspace$,
\begin{equation}\label{eqn:14F1}
    \pr{E} = \frac{\card{E}}{\card{\sspace}}.
\end{equation}
In this case, $E$~is the event that die~$A$ beats die~$B$, so
$\card{E} = 5$, \ $\card{\sspace} = 9$, and
\begin{equation*}
    \pr{E} = 5/9.
\end{equation*}

This is bad news for you.  Die~$A$ beats die~$B$ more than half the
time and, not surprisingly, you just lost~\$100.

Biker dude consoles you on your ``bad luck'' and, given that he's a
sensitive guy beneath all that leather, he offers to go double or
nothing.\footnote{\term{Double or nothing} is slang for doing another
  wager after you have lost the first.  If you lose again, you will
  owe biker dude \emph{double} what you owed him before.  If you win,
  you will owe him \emph{nothing}; in fact, since he should pay you
  \$210 if he loses, you would come out \$10 ahead.}  Given that your
wallet only has \$25 in it, this sounds like a good plan.  Plus, you
figure that choosing die~$A$ will give \emph{you} the advantage.

So you choose~$A$, and then biker dude chooses~$C$.  Can you guess who
is more likely to win?  (Hint: it is generally not a good idea to
gamble with someone you don't know in a bar, especially when you are
gambling with strange dice.)

\subsection{Die~$A$ versus Die~$C$}

We can construct the three diagram and outcome probabilities as
before.  The result is shown in Figure~\ref{fig:14A9} and there is bad
news again.  Die~$C$ will beat die~$A$ with probability~$5/9$, and you
lose once again.

\begin{figure}

\graphic{Fig_A9}

\caption{The tree diagram for one roll of die~$C$ versus die~$A$.
  Die~$C$ wins with probability~$5/9$.}

\label{fig:14A9}

\end{figure}

You now owe the biker dude \$200 and he asks for his money.  You reply
that you need to go to the bathroom.

\subsection{Die~$B$ versus Die~$C$}

Being a sensitive guy, biker dude nods understandingly and offers yet
another wager.  This time, he'll let you have die~$C$.  He'll even let
you raise the wager to~\$200 so you can win your money back.

This is too good a deal to pass up.  You know that die~$C$ is likely
to beat die~$A$ and that die~$A$ is likely to beat die~$B$, and so
die~$C$ is \emph{surely} the best.  Whether biker dude picks $A$
or~$B$, the odds would be in your favor this time.  Biker dude must
really be a nice guy.

So you pick~$C$, and then biker dude picks~$B$.  Wait, how come you
haven't caught on yet and worked out the tree diagram before you took
this bet :-) ?  If you do it now, you'll see by the same reasoning as
before that $B$ beats~$C$ with probability~$5/9$.  But surely there is
a mistake!  How is it possible that
\begin{quote}

$C$ beats~$A$ with probability~$5/9$,

$A$ beats~$B$ with probability~$5/9$,

$B$ beats~$C$ with probability~$5/9$?
\end{quote}


\iffalse

The tree diagram and outcome probabilities for $B$ versus~$C$ are
shown in Figure~\ref{fig:14A10}.  The data there show that
\emph{die~$B$} wins with probability~$5/9$.

\begin{figure}

\graphic{Fig_A10}

\caption{The tree diagram for one roll of die~$B$ versus die~$C$.
  Die~$B$ wins with probability~$5/9$.}

\label{fig:14A10}

\end{figure}
\fi

The problem is not with the math, but with your intuition.  Since $A$
will beat~$B$ more often than not, and $B$ will beat~$C$ more often
than not, it \emph{seems} like $A$ ought to beat~$C$ more often than
not, that is, the ``beats more often'' relation ought to be
\emph{\idx{transitive}}.  But this intuitive idea is simply false:
whatever die you pick, biker dude can pick one of the others and be
likely to win.  So picking first is actually a big disadvantage, and
as a result, you now owe biker dude~\$400.

Just when you think matters can't get worse, biker dude offers you one
final wager for~\$1,000.  This time, instead of rolling each die once,
you will each roll your die twice, and your score is the sum of your
rolls, and he will even let you pick your die second, that is, after
he picks his.  Biker dude chooses die~$B$.  Now you know that die~$A$
will beat die~$B$ with probability~$5/9$ on one roll, so, jumping at
this chance to get ahead, you agree to play, and you pick
die~$A$.  After all, you figure that since a roll of die~$A$ beats a
roll of die~$B$ more often that not, two rolls of die~$A$ are even
more likely to beat two rolls of die~$B$, right?

Wrong! (Did we mention that playing strange gambling games with
strangers in a bar is a bad idea?)

\subsection{Rolling Twice}

If each player rolls twice, the tree diagram will have four levels and
$3^4 = 81$ outcomes.  This means that it will take a while to write
down the entire tree diagram.  But it's easy to write down the
first two levels as in Figure~\ref{fig:14A11}(a) and
then notice that the remaining two levels consist of nine identical
copies of the tree in Figure~\ref{fig:14A11}(b).

\begin{figure}

\graphic{Fig_A11}

\caption{Parts of the tree diagram for die~$B$ versus die~$A$ where
  each die is rolled twice.  The first two levels are shown in~(a).
  The last two levels consist of nine copies of the tree in~(b).}

\label{fig:14A11}

\end{figure}

The probability of each outcome is $(1/3)^4 = 1/81$ and so, once
again, we have a uniform probability space.  By
equation~\eqref{eqn:14F1}, this means that the probability that
$A$~wins is the number of outcomes where $A$ beats~$B$ divided by~81.

To compute the number of outcomes where $A$ beats~$B$, we observe that
the two rolls of die~$A$ result in nine equally likely 
outcomes in a sample space $\sspace_A$ in which the
two-roll sums take the values
\[
    (4, 8, 8, 9, 9, 12, 13, 13, 14).
\]
Likewise, two rolls of die~$B$ result in nine equally likely outcomes
in a sample space $\sspace_B$ in which the
two-roll sums take the values
\[
(2, 6, 6, 10, 10, 10, 14, 14, 18).
\]
We can treat the outcome of rolling both dice twice as a pair $(x,y) \in
\sspace_A \cross \sspace_B$, where $A$~wins iff the sum of the two
$A$-rolls of outcome $x$ is larger the sum of the two $B$-rolls of
outcome $y$.  If the $A$-sum is 4, there is only one~$y$ with a
smaller $B$-sum, namely, when the $B$-sum is 2.  If the $A$-sum is 8,
there are three~$y$'s with a smaller $B$-sum, namely, when the $B$-sum
is 2 or 6.  Continuing the count in this way, the number of pairs
$(x,y)$ for which the $A$-sum is larger than the $B$-sum is
\begin{equation*}
    1 + 3 + 3 + 3 + 3 + 6 + 6 + 6 + 6 = 37.
\end{equation*}
A similar count shows that there are 42~pairs for which $B$-sum is
larger than the $A$-sum, and there are two pairs where the sums are
equal, namely, when they both equal 14.  This means that $A$
\emph{loses} to~$B$ with probability $42/81 > 1/2$ and ties with
probability~$2/81$.  Die~$A$ wins with probability only~$37/81$.

How can it be that $A$~is more likely than~$B$ to win with one roll,
but $B$~is more likely to win with two rolls?  Well, why not?  The
only reason we'd think otherwise is our unreliable, untrained
intuition.  (Even the authors were surprised when they first learned
about this, but at least we didn't lose~\$1400 to biker dude. :-) )  In
fact, the die strength reverses no matter which two die we picked.  So
for one roll,
\begin{equation*}
    A \succ B \succ C \succ A,
\end{equation*}
but for two rolls,
\begin{equation*}
    A \prec B \prec C \prec A,
\end{equation*}
where we have used the symbols $\succ$ and~$\prec$ to denote which die
is more likely to result in the larger value.

\subsubsection{Even Stranger Dice}

The weird behavior of the three strange dice above generalizes in a
remarkable way.\footnote{\TBA{Reference Ron Graham paper.}}  The idea is
that you can find arbitrarily large sets of dice which will beat each
other in any desired pattern according to how many times the dice are
rolled.  The precise statement of this result involves several
alternations of universal and existential quantifiers, so it may take
a few readings to understand what it is saying:

\iffalse Now that we know that strange things can happen with strange
dice, it is natural, at least for mathematicians, to ask how strange
things can get.  It turns out that things can get very strange.  In
fact, mathematicians recently made the following discovery:\fi

\begin{theorem}\label{thm:14F2}
For any $n \ge 2$, there is a set of $n$~dice with the following
property: for \emph{any} $n$-node digraph with exactly one directed
edge between every two distinct nodes,\footnote{In other words, for
  every pair of nodes $u \neq v$, either $\diredge{u}{v}$ or
  $\diredge{v}{u}$, but not both, are edges of the graph.  Such graphs
  are called \emph{tournament graphs}, see
  Problem~\ref{CP_tournament_graphs}.} there is a number of rolls~$k$
such that the sum of $k$~rolls of the $i$th die is bigger than the sum
for the $j$th die with probability greater than~$1/2$ iff there is an
edge from the $i$th to the $j$th node in the graph.
\end{theorem}

For example, the eight possible relative strengths for $n =
3$ dice are shown in Figure~\ref{fig:14A13}.  

\begin{figure}

\graphic{Fig_A13}

\caption{All possible relative strengths for three dice $D_1$, $D_2$,
  and~$D_3$.  The edge $\diredge{D_i}{D_j}$ denotes that the sum of
  rolls for~$D_i$ is likely to be greater than the sum of rolls
  for~$D_j$.}

\label{fig:14A13}

\end{figure}

Our analysis for the dice in Figure~\ref{fig:14A7} showed that for
one roll, we have the relative strengths shown in
Figure~\ref{fig:14A13}(a), and for two rolls, we have the (reverse)
relative strengths shown in Figure~\ref{fig:14A13}(b).  If you are
prone to gambling with strangers in bars, it would be a good idea to
try figuring out what other relative strengths are possible for the
dice in Figure~\ref{fig:14A7} when using more rolls.

\section{Set Theory and Probability}\label{probability_sets_sec}

Let's abstract what we've just done with the Monty Hall and strange
dice examples into a general mathematical definition of sample spaces
and probability.

\subsection{Probability Spaces}

\begin{definition}\label{LN12:sampsp}
  A countable \term{sample space}~$\sspace$ is a nonempty countable
  set.\footnote{Yes, sample spaces can be infinite.  If you did not
    read Chapter~\ref{infinite_chap}, don't worry ---\emph{countable}
    just means that you can list the elements of the sample space as
    $\omega_0$, $\omega_1$, $\omega_2$, \dots.}  An element $\omega
  \in \sspace$ is called an \term{outcome}.  A subset of $\sspace$ is
  called an \term{event}.
\end{definition}

\begin{definition}\label{LN12:probsp}
 A \term{probability function} on a sample space~$\sspace$ is a total
 function $\operatorname{Pr}: \sspace\to \reals$ such that
\begin{itemize}
\item $\pr{\omega} \geq 0$ for all $\omega \in \sspace$, and
\item $\sum_{\omega \in \sspace} \pr{\omega} = 1$.
\end{itemize}
A sample space together with a probability function is called a
\term{probability space}.
For any event $E \subseteq \sspace$, the \index{probability of an event}
\emph{probability of $E$} is defined to be the sum of the probabilities of
the outcomes in $E$:
\[
    \pr{E} \eqdef \sum_{\omega \in E} \pr{\omega}.
\]
\end{definition}

In the previous examples there were only finitely many possible
outcomes, but we'll quickly come to examples that have a countably
infinite number of outcomes.

The study of probability is closely tied to set theory
because any set can be a sample space and any subset can be an event.
General probability theory deals with uncountable sets like the set of
real numbers, but we won't need these, and sticking to countable
sets lets us define the probability of events using sums instead of
integrals.  It also lets us avoid some distracting technical problems
in set theory like the Banach-Tarski ``paradox'' mentioned in
Chapter~\ref{infinite_chap}.

\subsection{Probability Rules from Set Theory}\label{sec:union_bound}

Most of the rules and identities that we have developed for finite
sets extend very naturally to probability.  

\iffalse We'll cover several examples in this section, but first let's
review some definitions that should already be familiar.\fi

An immediate consequence of the definition of event probability is
that for \emph{disjoint} events $E$ and~$F$,
\[
    \pr{E \union F} = \pr{E} + \pr{F}.
\]
This generalizes to a countable number of events, as follows.

\begin{rul}[\idx{Sum Rule}]
  If $\set{E_0,E_1,\dots}$ is collection of disjoint
  events, then
\[
    \Prob{\lgunion_{n\in\naturals}E_n} = \sum_{n\in\naturals} \pr{E_n}.
\]
\end{rul}

The Sum Rule lets us analyze a complicated event by breaking it down
into simpler cases.  For example, if the probability that a randomly
chosen MIT student is native to the United States is 60\%, to Canada
is 5\%, and to Mexico is 5\%, then the probability that a random MIT
student is native to North America is 70\%.

Another consequence of the Sum Rule is that $\pr{A} + \pr{\setcomp{A}} =
1$, which follows because $\pr{\sspace}=1$ and $\sspace$ is the union
of the disjoint sets $A$ and $\setcomp{A}$.  This equation often comes up
in the form:
\begin{equation}%\label{}
\pr{\setcomp{A}}  =  1 - \pr{A}. \tag{\idx{Complement Rule}}
\end{equation}
Sometimes the easiest way to compute the probability of an event is to compute
the probability of its complement and then apply this formula.

Some further basic facts about probability parallel facts about
cardinalities of finite sets.  In particular:
\begin{center}
\begin{tabular*}{\textwidth}{l@{\extracolsep{\fill}}lr@{}}
\hskip\parindent
&$\pr{B-A} = \pr{B} - \pr{A \intersect B}$,
    & (Difference Rule)\\
&$\pr{A \union B} = \pr{A} + \pr{B} - \pr{A \intersect B}$,
    & (Inclusion-Exclusion)\\
&$\pr{A \union B} \le \pr{A} + \pr{B}$,
    & (Boole's Inequality) \\
&If $A \subseteq B$, then $\prob{A} \leq \prob{B}$.
    & (Monotonicity Rule)
\end{tabular*}
\end{center}
The \idx{Difference Rule} follows from the Sum Rule because $B$ is the
union of the disjoint sets $B-A$ and $A \intersect B$.
\index{inclusion-exclusion for probabilities} Inclusion-Exclusion then
follows from the Sum and Difference Rules, because $A \union B$ is the
union of the disjoint sets $A$ and $B-A$.  \idx{Boole's inequality} is an
immediate consequence of Inclusion-Exclusion since probabilities are
nonnegative.  Monotonicity follows from the definition of event
probability and the fact that outcome probabilities are nonnegative.

The two-event Inclusion-Exclusion equation above generalizes to $n$ events
in the same way as the corresponding Inclusion-Exclusion rule for $n$
sets.  Boole's inequality also generalizes to
\begin{rul}[\idx{Union Bound}]
\begin{equation}
    \pr{E_1 \union \cdots \union E_n} \leq \pr{E_1} + \cdots + \pr{E_n}.
\end{equation}
\end{rul}
This simple Union Bound is useful in many calculations.  For example,
suppose that $E_i$ is the event that the $i$-th critical component in
a spacecraft fails.  Then $E_1 \union \cdots \union E_n$ is the event that
\emph{some} critical component fails.  If $\sum_{i = 1}^n \prob{E_i}$
is small, then the Union Bound can give an adequate upper bound on
this vital probability.

\subsection{Uniform Probability Spaces}

\begin{definition}\label{def:uniform_pspace}
A finite probability space, $\sspace$, is said to be \term{uniform} if
$\pr{\omega}$ is the same for every outcome $\omega \in \sspace$.
\end{definition}

As we saw in the strange dice problem, uniform sample spaces are
particularly easy to work with.  That's because for any event~$E
\subseteq \sspace$,
\begin{equation}\label{eqn:14G2}
    \pr{E} = \frac{\card{E}}{\card{\sspace}}.
\end{equation}
This means that once we know the cardinality of $E$ and~$\sspace$, we
can immediately obtain~$\pr{E}$.  That's great news because we
developed lots of tools for computing the cardinality of a set in
Part~\ref{part:counting}.

For example, suppose that you select five cards at random from a
standard deck of 52~cards.  What is the probability of having a full
house?  Normally, this question would take some effort to answer.  But
from the analysis in Section~\ref{sec:counting_full_houses}, we know
that
\begin{equation*}
    \card{\sspace} = \binom{52}{5}
\end{equation*}
and
\begin{equation*}
    \card{E} = 13 \cdot \binom{4}{3} \cdot 12 \cdot \binom{4}{2}
\end{equation*}
where $E$ is the event that we have a full house.  Since every
five-card hand is equally likely, we can apply
equation~\eqref{eqn:14G2} to find that
\begin{align*}
\pr{E}  &= \frac{13 \cdot 12 \cdot \binom{4}{3} \cdot \binom{4}{2}}
                {\binom{52}{5}} \\
        &= \frac{13 \cdot 12 \cdot 4 \cdot 6 \cdot 5 \cdot 4 \cdot 3 \cdot 2}
                {52 \cdot 51 \cdot 50 \cdot 49 \cdot 48} = \frac{18}{12495} \\
        &\approx \frac{1}{694}.
\end{align*}

\subsection{Infinite Probability Spaces}

\iffalse
General probability theory deals with uncountable sets like~$\reals$,
but in computer science, it is usually sufficient to restrict our
attention to countable probability spaces.  It's also a lot
easier ---infinite sample spaces are hard enough to work with without
having to deal with uncountable spaces.
\fi

Infinite probability spaces are fairly common.  For example, two
players take turns flipping a fair coin.  Whoever flips heads first is
declared the winner.  What is the probability that the first player
wins?  A tree diagram for this problem is shown in
Figure~\ref{fig:14A15}.

\begin{figure}

\graphic{infinite-tree1}

\caption{The tree diagram for the game where players take turns
  flipping a fair coin.  The first player to flip heads wins.}

\label{fig:14A15}

\end{figure}

The event that the first player wins contains an infinite number of
outcomes, but we can still sum their probabilities:
\begin{align*}
\pr{\text{first player wins}}
    & = \frac{1}{2} + \frac{1}{8} + \frac{1}{32} + \frac{1}{128} + \cdots \\
    & = \frac{1}{2} \sum_{n=0}^\infty \paren{\frac{1}{4}}^n \\
    & = \frac{1}{2}\paren{\frac{1}{1-1/4}} = \frac{2}{3}.
\end{align*}

Similarly, we can compute the probability that the second player wins:
\begin{align*}
\pr{\text{second player wins}}
     = \frac{1}{4} + \frac{1}{16} + \frac{1}{64} + \frac{1}{256}
                      + \cdots % \\
     = \frac{1}{3}.
\end{align*}

In this case, the sample space is the infinite set
\[
    \sspace \eqdef \set{\, \tails^n\heads \suchthat n \in \naturals \,},
\]
where $\tails^n$ stands for a length $n$ string of $\tails$'s.
The probability function is
\[
\pr{\tails^n\heads} \eqdef \frac{1}{2^{n+1}}.
\]
To verify that this is a probability space, we just have to check that
all the probabilities are nonnegative and that they sum to~1.
Nonnegativity is obvious, and applying the formula for the sum of a
geometric series, we find that
\begin{equation*}
\sum_{n \in \naturals} \pr{\tails^n\heads}
    = \sum_{n \in \naturals} \frac{1}{2^{n+1}} \\
    = 1.
\end{equation*}

Notice that this model does not have an outcome corresponding to the
possibility that both players keep flipping tails forever ---in the
diagram, flipping forever corresponds to following the infinite path
in the tree without ever reaching a leaf/outcome.  If leaving this
possibility out of the model bothers you, you're welcome to fix it by
adding another outcome, $\omega_{\text{forever}}$, to indicate that that's
what happened.  Of course since the probabililities of the other
outcomes already sum to 1, you have to define the probability of
$\omega_{\text{forever}}$ to be 0.  Now outcomes with probability zero will
have no impact on our calculations, so there's no harm in adding it in
if it makes you happier.  On the other hand, in countable probability
spaces it isn't necessary to have outcomes with probability zero, and
we will generally ignore them.

%% Set Theory and Probability Problems %%%%%%%%%%%%%%%%%%%%%%%%%%%%%%%%%%%%%%%%
\begin{problems}
\classproblems
\pinput{CP_system_component_failure}
\pinput{CP_proving_probability_rules}

\homeworkproblems
\pinput{PS_union_bound}
\pinput{PS_probabilistic_proof}

\end{problems}


\begin{editingnotes}
Split with chapter{Conditional Probability}\label{chap:cond_prob}
here.  Insert Monty Hall false conditioning examples from slides.
\end{editingnotes}

\section{Conditional Probability}\label{cond_prob_sec}

Suppose that we pick a random person in the world.  Everyone has an
equal chance of being selected.  Let $A$ be the event that the person
is an MIT student, and let $B$ be the event that the person lives in
Cambridge.  What are the probabilities of these events?  Intuitively,
we're picking a random point in the big ellipse shown in
Figure~\ref{fig:15B1} and asking how likely that point is to fall into
region $A$ or $B$.

\begin{figure}[h]

\graphic{cambridge-conditional}

\caption{Selecting a random person.  $A$ is the event that the person
  is an MIT student.  $B$ is the event that the person lives in
  Cambridge.}

\label{fig:15B1}

\end{figure}

The vast majority of people in the world neither live in Cambridge nor
are MIT students, so events $A$ and $B$ both have low probability.
But what about the probability that a person is an MIT student,
\emph{given} that the person lives in Cambridge?  This should be
much greater ---but what is it exactly?

What we're asking for is called a \term{conditional probability}; that
is, the probability that one event happens, given that some other
event definitely happens.  Questions about conditional probabilities
come up all the time:
%
\begin{itemize}
\item What is the probability that it will rain this afternoon, given
that it is cloudy this morning?
\item What is the probability that two rolled dice sum to 10, given
that both are odd?
\item What is the probability that I'll get four-of-a-kind in Texas No
Limit Hold 'Em Poker, given that I'm initially dealt two queens?
\end{itemize}

There is a special notation for conditional probabilities.  In
general, $\prcond{A}{B}$ denotes the probability of event $A$, given
that event $B$ happens.  So, in our example, $\prcond{A}{B}$ is the
probability that a random person is an MIT student, given that he or
she is a Cambridge resident.

How do we compute $\prcond{A}{B}$?  Since we are \emph{given} that the
person lives in Cambridge, we can forget about everyone in the world
who does not.  Thus, all outcomes outside event $B$ are irrelevant.
So, intuitively, $\prcond{A}{B}$ should be the fraction of Cambridge
residents that are also MIT students; that is, the answer should be
the probability that the person is in set $A \intersect B$ (the darkly
shaded region in Figure~\ref{fig:15B1}) divided by the probability
that the person is in set $B$ (the lightly shaded region).  This
motivates the definition of conditional probability:
\begin{definition}\label{LN12:prcond}
\[
\prcond{A}{B} \eqdef \frac{\pr{A \intersect B}}{\pr{B}}
\]
If $\pr{B} = 0$, then the conditional probability $\prcond{A}{B}$ is
undefined.
\end{definition}

Pure probability is often counterintuitive, but conditional
probability is even worse!  Conditioning can subtly alter
probabilities and produce unexpected results in randomized algorithms
and computer systems as well as in betting games.  Yet, the
mathematical definition of conditional probability given above is very
simple and should give you no trouble ---provided that you rely on
mathematical reasoning and not intuition.  The four-step method will
also be very helpful as we will see in the next examples.

\subsection{The Four-Step Method for Conditional
  Probability: The ``Halting Problem''}

The \emph{Halting Problem} was the first example of a property that
could not be tested by any program.  It was introduced by Alan Turing
in his seminal 1936 paper.  The problem is to determine whether a
Turing machine halts on a given \dots yadda yadda yadda \dots more
importantly, it was \emph{the name of the MIT EECS department's famed
  C-league hockey team}.

In a best-of-three tournament, the Halting Problem wins the first game
with probability $1/2$.  In subsequent games, their
probability of winning is determined by the outcome of the previous
game.  If the Halting Problem won the previous game, then they are
invigorated by victory and win the current game with probability
$2/3$.  If they lost the previous game, then they are
demoralized by defeat and win the current game with probability only
$1/3$.  What is the probability that the Halting Problem wins
the tournament, given that they win the first game?

This is a question about a conditional probability.  Let $A$ be the
event that the Halting Problem wins the tournament, and let $B$ be the
event that they win the first game.  Our goal is then to determine the
conditional probability $\prcond{A}{B}$.

We can tackle conditional probability questions just like ordinary
probability problems: using a tree diagram and the four step method.
A complete tree diagram is shown in Figure~\ref{fig:15B2}.

\begin{figure}[h]

\graphic{hockey}

\caption{The tree diagram for computing the probability that the
  ``Halting Problem'' wins two out of three games given that they won
  the first game.}

\label{fig:15B2}

\end{figure}

\paragraph{Step 1:  Find the Sample Space}

Each internal vertex in the tree diagram has two children, one
corresponding to a win for the Halting Problem (labeled~$W$) and one
corresponding to a loss (labeled~$L$).  The complete sample space is:
%
\[
    \sspace = \set{ WW, \, WLW,\, WLL,\, LWW,\, LWL,\, LL }.
\]

\paragraph{Step 2:  Define Events of Interest}

The event that the Halting Problem wins the whole tournament is:
%
\[
    T = \set{WW,\, WLW,\, LWW}.
\]
%
And the event that the Halting Problem wins the first game is:
%
\[
    F = \set{WW,\, WLW,\, WLL }.
\]
%
The outcomes in these events are indicated with check marks in the tree
diagram in Figure~\ref{fig:15B2}.

\paragraph{Step 3:  Determine Outcome Probabilities}

Next, we must assign a probability to each outcome.  We begin by
labeling edges as specified in the problem statement.  Specifically,
The Halting Problem has a $1/2$ chance of winning the first game, so
the two edges leaving the root are each assigned probability $1/2$.
Other edges are labeled $1/3$ or $2/3$ based on the outcome of the
preceding game.  We then find the probability of each outcome by
multiplying all probabilities along the corresponding root-to-leaf
path.  For example, the probability of outcome $WLL$ is:
%
\[
    \frac{1}{2} \cdot \frac{1}{3} \cdot \frac{2}{3} = \frac{1}{9}.
\]

\subsubsection*{Step 4: Compute Event Probabilities}

We can now compute the probability that The Halting Problem wins the
tournament, given that they win the first game:
%
\begingroup
\openup2pt
\begin{align*}
\prcond{A}{B}
    & = \frac{\pr{A \intersect B}}{\pr{B}} \\
    & = \frac{\pr{\set{WW, WLW}}}{\pr{\set{WW, WLW, WLL}}} \\
    & = \frac{1/3 + 1/18}{1/3 + 1/18 + 1/9} \\
    & = \frac{7}{9}.
\end{align*}
\endgroup
%
We're done!  If the Halting Problem wins the first game, then they win
the whole tournament with probability $7 / 9$.


\subsection{Why Tree Diagrams Work}\label{product_rule_subsec}

We've now settled into a routine of solving probability problems using
tree diagrams.  But we've left a big question unaddressed: what is the
mathematical justification behind those funny little pictures?  Why do
they work?

The answer involves conditional probabilities.  In fact, the
probabilities that we've been recording on the edges of tree diagrams
\emph{are} conditional probabilities.  For example, consider the
uppermost path in the tree diagram for the Halting Problem, which
corresponds to the outcome $WW$.  The first edge is labeled $1/2$,
which is the probability that the Halting Problem wins the first game.
The second edge is labeled $2 / 3$, which is the probability that the
Halting Problem wins the second game, \emph{given} that they won the
first ---that's a conditional probability!  More generally, on each
edge of a tree diagram, we record the probability that the experiment
proceeds along that path, given that it reaches the parent vertex.

So we've been using conditional probabilities all along.  But why can
we multiply edge probabilities to get outcome probabilities?  For
example, we concluded that:
%
\begin{equation*}
\pr{WW} = \frac{1}{2} \cdot \frac{2}{3}
	= \frac{1}{3}.
\end{equation*}
%
Why is this correct?

The answer goes back to Definition~\ref{LN12:prcond} of conditional probability
which could be written in a form called the \term{Product Rule} for
probabilities:
%
\begin{rul*}[Product Rule: 2 Events]
If $\pr{E_1} \neq 0$, then:
%
\[
    \pr{E_1 \intersect E_2} = \pr{E_1} \cdot \prcond{E_2}{E_1}.
\]
\end{rul*}
Multiplying edge probabilities in a tree diagram amounts to evaluating
the right side of this equation.  For example:
\begin{align*}
\lefteqn{\pr{\text{win first game} \intersect \text{win second game}}}
		\hspace{0.5in} \\[2pt]
	& = \pr{\text{win first game}} \cdot
            \prcond{\text{win second game}}{\text{win first game}} \\[2pt]
	& = \frac{1}{2} \cdot \frac{2}{3}.
\end{align*}
So the Product Rule is the formal justification for multiplying edge
probabilities to get outcome probabilities!  Of course to justify
multiplying edge probabilities along longer paths, we need a Product
Rule for $n$ events.

% \dmj{I need to have another go at formatting this equation.}
\begin{rul*}[Product Rule: $n$ Events]
\begin{align*}
\pr{E_1 \intersect E_2 \intersect \dots \intersect E_n}
   =& \pr{E_1}
        \cdot \prcond{E_2}{E_1}
        \cdot \prcond{E_3}{E_1 \intersect E_2}
        \cdots \\
    &\quad\cdot
        \prcond{E_n}{E_1 \intersect E_2 \intersect \dots
          \intersect E_{n - 1}}
\end{align*}
provided that %\prod_{i=1}^n \pr{E_i}{\lgintersect_{j=1}^i E_j} 
\begin{equation*}
    \pr{E_1 \intersect E_2 \intersect \cdots \intersect E_{n - 1}}
    \neq 0.
\end{equation*}
\end{rul*}
This rule follows by routine induction from the definition of
conditional probability.

\subsection{Medical Testing}\label{med_test-subsection}

\dmj{Honestly, is this the most dignified example you could come up
  with?}
\ftl{Good point.  Let's flag it to be changed.}
There is an unpleasant condition called \emph{BO} suffered by 10\% of the
population.  There are no prior symptoms; victims just suddenly start to
stink.  Fortunately, there is a test for latent \emph{BO} before things
start to smell.  The test is not perfect, however:
\begin{itemize}

\item If you have the condition, there is a 10\% chance that the test
  will say you do not have it.  These are called ``false negatives.''

\item If you do not have the condition, there is a 30\% chance that the test
will say you do.  These are ``false positives.''

\end{itemize}

Suppose a random person is tested for latent \emph{BO}.  If the test is
positive, then what is the probability that the person has the condition?

\subsubsection*{Step 1: Find the Sample Space}

The sample space is found with the tree diagram in
Figure~\ref{fig:15C1}.

\begin{figure}[h]

\graphic{BO}

\caption{The tree diagram for the BO problem.}

\label{fig:15C1}

\end{figure}

\subsubsection*{Step 2: Define Events of Interest}

Let $A$ be the event that the person has \emph{BO}.  Let $B$ be the
event that the test was positive.  The outcomes in each event are marked
in the tree diagram.  We want to find $\prcond{A}{B}$, the probability
that a person has \emph{BO}, given that the test was positive.

\subsubsection*{Step 3: Find Outcome Probabilities}

First, we assign probabilities to edges.  These probabilities are
drawn directly from the problem statement.  By the Product Rule, the
probability of an outcome is the product of the probabilities on the
corresponding root-to-leaf path.  All probabilities are shown in
Figure~\ref{fig:15C1}.

\subsubsection*{Step 4: Compute Event Probabilities}

From Definition~\ref{LN12:prcond}, we have
\begin{equation*}
\prcond{A}{B}	= \frac{\pr{A \intersect B}}{\pr{B}} %\\[2pt]
		= \frac{0.09}{0.09 + 0.27} %\\[2pt]
		= \frac{1}{4}.
\end{equation*}
%
So, if you test positive, then there is only a 25\% chance that you
have the condition!

This answer is initially surprising, but makes sense on reflection.
There are two ways you could test positive.  First, it could be that
you have the condition and the test is correct.  Second, it could be that you
are healthy and the test is incorrect.  The problem is that almost
everyone is healthy; therefore, most of the positive results arise
from incorrect tests of healthy people!

We can also compute the probability that the test is correct for a
random person.  This event consists of two outcomes.  The person could
have the condition and test positive (probability $0.09$), or the person
could be healthy and test negative (probability $0.63$).
Therefore, the test is correct with probability $0.09 + 0.63 = 0.72$.
This is a relief; the test is correct almost three-quarters of the
time.

But wait!  There is a simple way to make the test correct 90\% of the
time: always return a negative result!  This ``test'' gives the right
answer for all healthy people and the wrong answer only for the 10\%
that actually have the condition.  So a better strategy by this
measure is to completely ignore the test result!

There is a similar paradox in weather forecasting.  During winter,
almost all days in Boston are wet and overcast.  Predicting miserable
weather every day may be more accurate than really trying to get it
right!


\subsection{\emph{A Posteriori} Probabilities}\label{aposteriori_subsec}

If you think about it too much, the medical testing problem we just
considered could start to trouble you.  The concern would be that by
the time you take the test, you either have the BO condition or you
don't ---you just don't know which it is.  So you may wonder if a
statement like ``If you tested positive, then you have the condition
with probability~25\%'' makes sense.

In fact, such a statement does make sense.  It means that 25\% of the
people who test positive actually have the condition.  It is true that
any particular person has it or they don't, but a \emph{randomly
  selected} person among those who test positive will have the
condition with probability~25\%.

Anyway, if the medical testing example bothers you, you will
definitely be worried by the following examples, which go even further
down this path.

\subsection{The ``Halting Problem,'' in Reverse}

Suppose that we turn the hockey question around: what is the
probability that the Halting Problem won their first game, given that
they won the series?

This seems like an absurd question!  After all, if the Halting Problem
won the series, then the winner of the first game has already been
determined.  Therefore, who won the first game is a question of fact,
not a question of probability.  However, our mathematical theory of
probability contains no notion of one event preceding another---there
is no notion of time at all.  Therefore, from a mathematical
perspective, this is a perfectly valid question.  And this is also a
meaningful question from a practical perspective.  Suppose that you're
told that the Halting Problem won the series, but not told the results
of individual games.  Then, from your perspective, it makes perfect
sense to wonder how likely it is that The Halting Problem won the
first game.

A conditional probability $\prcond{B}{A}$ is called  \term{a
posteriori} if event $B$ precedes event $A$ in time.  Here are some
other examples of a posteriori probabilities:
%
\begin{itemize}
\item The probability it was cloudy this morning, given that it rained
in the afternoon.
\item The probability that I was initially dealt two queens in Texas
No Limit Hold 'Em poker, given that I eventually got four-of-a-kind.
\end{itemize}
%
Mathematically, a posteriori probabilities are \emph{no different}
from ordinary probabilities; the distinction is only at a higher,
philosophical level.  Our only reason for drawing attention to them is
to say, ``Don't let them rattle you.''

Let's return to the original problem.  The probability that the
Halting Problem won their first game, given that they won the series
is $\prcond{B}{A}$.  We can compute this using the definition of
conditional probability and the tree diagram in Figure~\ref{fig:15B2}:
%
\begin{align*}
\prcond{B}{A}  = \frac{\pr{B \intersect A}}{\pr{A}} %\\[2pt]
               = \frac{1/3 + 1/18}{1/3 + 1/18 + 1/9} %\\[2pt]
               = \frac{7}{9}.
\end{align*}

This answer is suspicious!  In the preceding section, we showed that
$\prcond{A}{B}$ was also $7/9$.  Could it be true that $\prcond{A}{B}
= \prcond{B}{A}$ in general?  Some reflection suggests this is
unlikely.  For example, the probability that I feel uneasy, given that
I was abducted by aliens, is pretty large.  But the probability that I
was abducted by aliens, given that I feel uneasy, is rather small.

Let's work out the general conditions under which $\prcond{A}{B} =
\prcond{B}{A}$.  By the definition of conditional probability, this
equation holds if an only if:
%
\[
\frac{\pr{A \intersect B}}{\pr{B}} = \frac{\pr{A \intersect B}}{\pr{A}}
\]
%
This equation, in turn, holds only if the denominators are equal or
the numerator is~0; namely if
%
\[
\pr{B} = \pr{A}
\hspace{0.25in} \text{or} \hspace{0.25in}
\pr{A \intersect B} = 0.
\]
%
The former condition holds in the hockey example; the probability that
the Halting Problem wins the series (event~$A$) is equal to the
probability that it wins the first game (event~$B$) since both
probabilities are~$1/2$.

In general, such pairs of probabilities are related by \idx{Bayes'
  Rule}:
%
\begin{theorem}[Bayes' Rule]
If $\pr{A}$ and $\pr{B}$ are nonzero, then:
%
\begin{equation}\label{bayesrule}
    \prcond{B}{A} = \frac{\prcond{A}{B} \cdot \pr{B}}{\pr{A}}
\end{equation}
\end{theorem}

\begin{proof}
When $\pr{A}$ and $\pr{B}$ are nonzero, we have
\[
\prcond{A}{B} \cdot \pr{B} = \prob{A \intersect B} = \prcond{B}{A} \cdot \pr{A}
\]
by definition of conditional probability.  Dividing by $\prob{A}$
gives~\eqref{bayesrule}.
\end{proof}

\iffalse

Next, let's look at a problem that even bothers us.

\subsection{A Coin Problem}

Suppose that someone hands you either a fair coin or a trick coin with
heads on both sides.  You flip the coin 100 times and see heads every
time.  What can you say about the probability that you flipped the
fair coin?  Remarkably, nothing!

In order to make sense out of this outrageous claim, let's formalize
the problem.  The sample space is worked out in the tree diagram shown
in Figure~\ref{fig:15C2}.  We do not know the probability~$p$ that you
were handed the fair coin initially---you were just given one coin or
the other.
%
\begin{figure}[h]

\graphic{trick-coin}

\caption{The tree diagram for the coin-flipping problem.}

\label{fig:15C2}

\end{figure}
%
Let $A$ be the event that you were handed the fair coin, and let $B$
be the event that you flipped 100 straight heads.  We're looking
for $\prcond{A}{B}$, the probability that you were handed the fair
coin, given that you flipped 100 heads.  The outcome probabilities are
worked out in Figure~\ref{fig:15C2}.  Plugging the results into the
definition of conditional probability gives:
%
\begin{align*}
\prcond{A}{B}	& = \frac{\pr{A \intersect B}}{\pr{B}} \\[2pt]
		& = \frac{p / 2^{100}}{1 - p + p / 2^{100}} \\[2pt]
		& = \frac{p}{2^{100} (1 - p) + p}.
\end{align*}
%
This expression is very small for moderate values of $p$ because of
the $2^{100}$ term in the denominator.  For example, if $p = 1/2$,
then the probability that you were given the fair coin is essentially
zero.

But we \emph{do not know} the probability $p$ that you were given
the fair coin.  And perhaps the value of $p$ is \emph{not} moderate;
in fact, maybe $p = 1 - 2^{-100}$.  Then there is nearly an even
chance that you have the fair coin, given that you flipped 100 heads.
In fact, maybe you were handed the fair coin with probability $p = 1$.
Then the probability that you were given the fair coin is, well,~1!

Of course, it is extremely unlikely that you would flip 100 straight
heads, but in this case, that is a given from the assumption of the
conditional probability.  And so if you really did see 100 straight
heads, it would be very tempting to also assume that $p$~is not close
to~1 and hence that you are very likely to have flipped the trick
coin.

We will encounter a very similar issue when we look at methods for
estimation by sampling in Section~\ref{sec:sampling}.
\fi

%\subsection{Conditional Identities}

\subsection{The Law of Total Probability}\label{sec:total_probability}

Breaking a probability calculation into cases simplifies many
problems.  The idea is to calculate the probability of an event $A$ by
splitting into two cases based on whether or not another event $E$
occurs.  That is, calculate the probability of $A\nobreak
\intersect\nobreak E$ and $A \intersect \setcomp{E}$.  By the Sum
Rule, the sum of these probabilities equals $\pr{A}$.  Expressing the
intersection probabilities as conditional probabilities yields:
\begin{rul}[Law of Total Probability, single event]\label{total_prob_Ebar}
If $\prob{E}$ and $\prob{\setcomp{E}}$~are nonzero, then
\[
\pr{A} = \prcond{A}{E} \cdot \pr{E} +
         \prcond{A}{\setcomp{E}} \cdot \pr{\setcomp{E}}.
\]
\end{rul}

For example, suppose we conduct the following experiment.  First, we
flip a fair coin.  If heads comes up, then we roll one die and take the
result.  If tails comes up, then we roll two dice and take the sum of
the two results.  What is the probability that this process yields a
2?  Let $E$ be the event that the coin comes up heads, and let $A$ be
the event that we get a 2 overall.  Assuming that the coin is fair,
$\pr{E} = \pr{\setcomp{E}} = 1/2$.  There are now two cases. If we
flip heads, then we roll a 2 on a single die with probability
$\prcond{A}{E} = 1/6$.  On the other hand, if we flip tails, then we
get a sum of 2 on two dice with probability
$\prcond{A}{\setcomp{E}} = 1/36$.  Therefore, the probability that
the whole process yields a 2 is
\[
\pr{A} = \frac{1}{2} \cdot \frac{1}{6} + \frac{1}{2} \cdot \frac{1}{36} =
  \frac{7}{72}.
\]

There is also a form of the rule to handle more than two cases.
\begin{rul}[Law of Total Probability]
If $E_1, \dots, E_n$ are disjoint events whose union is the whole
sample space, then:
\[
\pr{A} = \sum_{i=1}^{n} \prcond{A}{E_i} \cdot \pr{E_i}.
\]
\end{rul}

\subsection{Conditioning on a Single Event}\label{cond_ident_subsec}

The probability rules that we derived in Section~\ref{sec:union_bound}
extend to probabilities conditioned on the same event.  For example,
the Inclusion-Exclusion formula for two sets holds when all
probabilities are conditioned on an event $C$:
\[
\prcond{A \union B}{C} = \prcond{A}{C} + \prcond{B}{C} - \prcond{A \intersect B}{C}.
\]
This is easy to verify by plugging in the Definition~\ref{LN12:prcond}
of conditional
probability.\footnote{Problem~\ref{PS_conditional_space} explains why
  this and similar conditional identities follow on general principles
  from the corresponding unconditional identities.}

\iffalse
This follows from the fact that if $\pr{C} \neq 0$, then
\begin{align*}
\prcond{A \union B}{C}
    &= \frac{\pr{(A \union B) \intersect C}}{\pr{C}} \\[3pt]
    &= \frac{\pr{(A \intersect C) \union (B \intersect C)}}{\pr{C}} \\[3pt]
    &= \frac{\pr{A \intersect C} + \pr{B \intersect C}
             - \pr{A \intersect B \intersect C}}
            {\pr{C}} \\[3pt]
    &= \prcond{A}{C} + \prcond{B}{C} - \prcond{A \intersect B}{C}.
\end{align*}
\fi

It is important not to mix up events before and after the conditioning
bar.  For example, the following is \emph{not} a valid identity:
%
\begin{falseclm*}
\begin{equation}\label{LN12:fc}
\prcond{A}{B \union C} = \prcond{A}{B} + \prcond{A}{C} - \prcond{A}{B \intersect C}.
\end{equation}
\end{falseclm*}

A simple counter-example is to let $B$ and $C$ be events over a
uniform space with most of their outcomes in $A$, but not overlapping.
This ensures that $\prcond{A}{B}$ and $\prcond{A}{C}$ are both close
to 1.  For example,
\begin{align*}
B & \eqdef [0,9],\\
C & \eqdef [10,18] \union \set{0},\\
A & \eqdef [1,18],
\end{align*}
so
\[
\prcond{A}{B} = \frac{9}{10} = \prcond{A}{C}.
\]
Also, since 0 is the only outcome in $B \intersect C$ and $0 \notin
A$, we have
\[
\prcond{A}{B \intersect C} = 0
\]
So the right hand side of~\eqref{LN12:fc} is 1.8, while the left hand
side is a probability which can be at most 1 ---actually, it is 18/19.

\iffalse

A counterexample is shown in Figure~\ref{fig:15D2}.  In this case,
$\prcond{A}{B} = 1/2$, $\prcond{A}{C} = 1/2$, $\prcond{A}{B \intersect
  C} = 1$, and $\prcond{A}{B \union C} = 1/3$.  However, since
$1/3 \ne 1/2 + 1/2 - 1$, equation~\eqref{LN12:fc} does not hold.
%
\begin{figure}

\graphic{cx19}

\caption{A counterexample to equation~\eqref{LN12:fc}.  Event~$A$ is
  the dark-bordered rectangle, event~$B$ is the rectangle with
  vertical stripes, and event~$C$ is the rectangle with horizontal
  stripes.  $B \intersect C$ lies entirely within~$A$ while $B - C$
  and $C - B$ are entirely outside of~$A$.}

\label{fig:15D2}

\end{figure}
\fi


\subsection{Discrimination Lawsuit}\label{discrimination_subsec}

Several years ago there was a sex discrimination lawsuit against a
famous university.  A woman math professor was denied tenure,
allegedly because she was a woman.  She argued that in every one of
the university's 22 departments, the percentage of men candidates
granted tenure was greater than the percentage of women candidates
granted tenure.  This sounds very suspicious!

However, the university's lawyers argued that across the university as
a whole, the percentage of male candidates granted tenure was actually
\emph{lower} than the percentage for women candidates.  This suggests
that if there was any sex discrimination, then it was against men!
Surely, at least one party in the dispute must be lying.

Let's clarify the problem by expressing both arguments in terms of
conditional probabilities.  To simplify matters, suppose that there
are only two departments, EE and CS, and consider the experiment where
we pick a random candidate.  Define the following events:
%
\begin{itemize}
\item $A \eqdef$ the candidate is granted tenure,
\item $F_{EE} \eqdef$ the candidate is a woman in the EE department,
\item $F_{CS} \eqdef$ the candidate is a woman in the CS department,
\item $M_{EE} \eqdef$ the candidate is a man in the EE department,
\item $M_{CS} \eqdef$ the candidate is a man in the CS department.
\end{itemize}
Assume that all candidates are either men or women, and that no
candidate belongs to both departments.  That is, the events $F_{EE}$,
$F_{CS}$, $M_{EE}$, and $M_{CS}$ are all disjoint.

In these terms, the plaintiff is making the following argument:
%
\begin{align*}
\prcond{A}{F_{EE}} & < \prcond{A}{M_{EE}} \quad\text{and}\\
\prcond{A}{F_{CS}} & < \prcond{A}{M_{CS}}.
\end{align*}
That is, in both departments, the probability that a woman candidate
is granted tenure is less than the probability for a man.

The university retorts that \emph{overall}, a woman candidate is
\emph{more} likely to be granted tenure than a man; namely that
\[
    \prcond{A}{F_{EE} \union F_{CS}} > \prcond{A}{M_{EE} \union M_{CS}}.
\]

It is easy to believe that these two positions are contradictory, and
the phenomenon illustrated here is widely referred to as ``Simpson's
Paradox.''  But there is no contradiction or paradox, and in fact,
Table~\ref{fig:15D3} shows a set of candidate statistics for which the
assertions of both the plaintiff and the university hold.  In this
case, a higher percentage of men candidates were granted tenure in
each department, but overall a higher percentage of women candidates
were granted tenure!  How do we make sense of this?

\begin{table}

\begin{tabular}{crr}
CS & 0 women granted tenure, 1 candidates      &   0\% \\
   & 50 men granted tenure, 100 candidates     &  50\% \\
EE & 70 women granted tenure, 100 candidates   &  70\% \\
   & 1 man granted tenure, 1 candidates         & 100\% \\
\hline
Overall & 70 women granted tenure, 101 candidates & $\approx 70\%$ \\
        & 51 men granted tenure, 101 candidates   & $\approx 51\%$
\end{tabular}

\caption{A scenario where women are less likely to be granted tenure
  than men in each department, but more likely to be granted tenure
  overall.}

\label{fig:15D3}

\end{table}

With data like this showing that at the department level, women
candidates were less likely to be granted tenure than men, university
administrators would likely see an indication of bias against women,
and the departments would be directed to reexamine their tenure
procedures.

But suppose we replaced ``the candidate is a man/woman in the EE
department,'' by ``the candidate is a man/woman for whom a tenure
decision was made during an odd-numbered day of the month,'' and
likewise with CS and an even-numbered day of the month.  Since we
don't think the parity of a date is a cause for the outcome of a
tenure decision, we would ignore the ``coincidence'' that on both odd
and even dates, men are more frequently granted tenure.  Instead, we
would judge, based on the overall data showing women more likely to be
granted tenure, that gender bias against women was \emph{not} an issue
in the university.

The point is that it's the \emph{same data} that we interpret
differently based on our implicit causal beliefs.  It would be
circular to claim that the gender correlation observed in the data
corroborates our belief that there is discrimination, since our
interpretation of the data correlation \emph{depends} on our beliefs
about the causes of tenure decisions.\footnote{These issues are
  thoughtfully examined in \emph{Causality: Models, Reasoning and
    Inference}, Judea Pearl, Cambridge U. Press, 2001.}  This
illustrates a basic principle in statistics which people constantly
ignore: \emph{never assume that correlation implies causation}.

%% Conditional Probability Problems %%%%%%%%%%%%%%%%%%%%%%%%%%%%%%%%%%%%%%%%%%%

\begin{problems}
\practiceproblems
\pinput{TP_six_shooter_probability}

\classproblems
\pinput{CP_missing_card_probability}
\pinput{PS_conditional_aces}
\pinput{CP_conditional_prob_says_so_bug}
\pinput{FP_skywalker_prob_lin_recur_gen_func}
\pinput{FP_directed_graphs_and_probability}

\homeworkproblems
\pinput{PS_levitating_LAs}
\pinput{PS_conditional_probability_problem_errors}
\pinput{PS_coin_flip_sequences}
\pinput{PS_13_card_hand}
\pinput{PS_conditional_space}

\examproblems
\pinput{FP_monty_hall_variant}
\pinput{FP_conditional_prob_inequality}
\pinput{MQ_conditional_prob_inequality}
\pinput{FP_conditional_beaver_fever}
\pinput{FP_red_and_blue_goats}
\pinput{FP_neighborhood_census}
\pinput{MQ_voldemort_returns}
\end{problems}

\section{Independence}
Suppose that we flip two fair coins simultaneously on opposite sides
of a room.  Intuitively, the way one coin lands does not affect the
way the other coin lands.  The mathematical concept that captures
this intuition is called \term{independence}.
\begin{definition}\label{def:independence}
An event with probability 0 is defined to be independent of every
event (including itself).  If $\pr{B} \neq 0$, then
event $A$ is independent of event $B$ iff
\begin{equation}\label{eqn:independence}
    \prcond{A}{B} = \pr{A}.
\end{equation}
\end{definition}
In other words, $A$ and~$B$ are independent if knowing that $B$
happens does not alter the probability that $A$~happens, as is the
case with flipping two coins on opposite sides of a room.

\subsubsection{Potential Pitfall}

Students sometimes get the idea that disjoint events are independent.
The \emph{opposite} is true: if $A \intersect B = \emptyset$, then
knowing that $A$ happens means you know that $B$ does not happen.  So
disjoint events are \emph{never} independent---unless one of them has
probability zero.

\subsection{Alternative Formulation}

Sometimes it is useful to express independence in an alternate form
which follows immediately from Definition~\ref{def:independence}:

\begin{theorem}\label{thm:16A1}
$A$ is independent of~$B$ if and only if
\begin{equation}\label{eqn:15D3}
    \pr{A \intersect B} = \pr{A} \cdot \pr{B}.
\end{equation}
\end{theorem}

Notice that Theorem~\ref{thm:16A1} makes apparent the symmetry between
$A$ being independent of $B$ and $B$ being independent of $A$:
\begin{corollary}
$A$ is independent of $B$ iff $B$ is independent of $A$.
\end{corollary}


\iffalse

\begin{proof}
There are two cases to consider depending on whether or not $\prob{B} =
0$.
\begin{description}

\item[Case 1 $(\prob{B} = 0)$:]
If $\prob{B} = 0$, $A$ and~$B$ are independent by
Definition~\ref{def:independence}.  In addition,
equation~\eqref{eqn:15D3} holds since both sides are~0.  Hence, the
theorem is true in this case.

\item[Case 2 $(\prob{B} > 0)$:]
By Definition~\ref{LN12:prcond},
\begin{equation*}
    \prob{A \cap B} = \prcond{A}{B} \prob{B}.
\end{equation*}
So equation~\eqref{eqn:15D3} holds if
\begin{equation*}
    \prcond{A}{B} = \prob{A},
\end{equation*}
which, by Definition~\ref{def:independence}, is true iff $A$ and~$B$
are independent.  Hence, the theorem is true in this case as well.
\qedhere
\end{description}
\end{proof}
\fi

\subsection{Independence Is an Assumption}

Generally, independence is something that you \emph{assume} in
modeling a phenomenon.  For example, consider the experiment of
flipping two fair coins.  Let $A$~be the event that the first coin
comes up heads, and let $B$~be the event that the second coin is
heads.  If we assume that $A$ and~$B$ are independent, then the
probability that both coins come up heads is:
%
\begin{equation*}
\pr{A \intersect B}  = \pr{A} \cdot \pr{B} %\\[2pt]
               = \frac{1}{2} \cdot \frac{1}{2} %\\[2pt]
               = \frac{1}{4}.
\end{equation*}

In this example, the assumption of independence is reasonable.  The
result of one coin toss should have negligible impact on the outcome
of the other coin toss.  And if we were to repeat the experiment many
times, we would be likely to have~$A \cap B$ about~1/4 of the time.

There are, of course, many examples of events where assuming
independence is \emph{not} justified.  For example, let $C$~be the
event that tomorrow is cloudy and $R$ be the event that tomorrow is
rainy.  Perhaps $\pr{C} = 1/5$ and $\pr{R} = 1/10$ in Boston.  If
these events were independent, then we could conclude that the
probability of a rainy, cloudy day was quite small:
%
\begin{equation*}
\pr{R \intersect C} = \pr{R} \cdot \pr{C} % \\[2pt]
               = \frac{1}{5} \cdot \frac{1}{10} % \\[2pt]
               = \frac{1}{50}.
\end{equation*}
%
Unfortunately, these events are definitely not independent; in
particular, every rainy day is cloudy.  Thus, the probability of a
rainy, cloudy day is actually~$1/10$.

Deciding when to \emph{assume} that events are independent is a tricky
business.  In practice, there are strong motivations to assume
independence since many useful formulas (such as
equation~\eqref{eqn:15D3}) only hold if the events are independent.
But you need to be careful:
\iffalse
 lest you end up deriving false conclusions.
\fi
we'll describe several famous examples where (false) assumptions of
independence led to trouble.
\iffalse
 over the next several chapters
\fi
This problem gets even trickier when there are more than two events in
play.

\subsection{Mutual Independence}

%\subsection{Definition}

We have defined what it means for two events to be independent.  What
if there are more than two events?  For example, how can we say that
the flips of $n$~coins are all independent of one another?  A set of
events is said to be \term{mutually independent} if the probability
of each event in the set is the same no matter which of the other
events has occurred.  We could formalize this with conditional
probabilities as in Definition~\ref{def:independence}, but we'll jump
directly to the cleaner definition based on products of probabilities
as in Theorem~\ref{thm:16A1}:

\iffalse

\begin{definition}\label{def:mutual_independence}
A set of events~$E_1, E_2, \dots, E_n$, is \term{mutually independent}
if $\forall i \in [1, n]$ and $\forall S \subseteq [1, n] - \set{i}$,
either
\begin{equation*}
    \Prob{\bigcap_{j \in S} E_j} = 0
\quad
\text{or}
\quad
    \prob{E_i} = \prcond{E_i}{\bigcap_{j \in S} E_j}.
\end{equation*}
\end{definition}

\subsection{Alternative Formulation}

Just as Theorem~\ref{thm:16A1} provided an alternative definition of
independence for two events, there is an alternative definition for
mutual independence.

\fi

%\begin{theorem}\label{thm:16A2}

\begin{definition}\label{def:mutual_indep}
A set of events~$E_1, E_2, \dots, E_n$ is mutually independent iff
for all subsets $S \subseteq [1, n]$,
\begin{equation*}
    \Prob{\bigcap_{j \in S} E_j} = \prod_{j \in S} \prob{E_j}.
\end{equation*}
\end{definition}

%\end{theorem}
\iffalse
The proof of Theorem~\ref{thm:16A2} uses induction and reasoning
similar to the proof of Theorem~\ref{thm:16A1}.  We will not include
the details here.
\fi

Definition~\ref{def:mutual_indep} says that $E_1, E_2, \dots, E_n$~are
mutually independent if and only if all of the following equations
hold for all distinct $i$, $j$, $k$, and~$l$:
%
\begin{align*}
\pr{E_i \intersect E_j}
    & = \pr{E_i} \cdot \pr{E_j}
%    & \text{for all distinct $i$, $j$}
 \\
\pr{E_i \intersect E_j \intersect E_k}
    & = \pr{E_i} \cdot \pr{E_j} \cdot \pr{E_k}
%     & \text{for all distinct $i$, $j$, $k$}
 \\
\pr{E_i \intersect E_j \intersect E_k \intersect E_l}
    & = \pr{E_i} \cdot \pr{E_j} \cdot \pr{E_k} \cdot \pr{E_l}
%    & \text{for all distinct $i$, $j$, $k$, $l$}
 \\
    & \XasWideAsY{\vdots}{${}={}$} \\
\pr{E_1 \intersect \cdots \intersect E_n} & = \pr{E_1} \cdots \pr{E_n}.
\end{align*}

For example, if we toss $n$~fair coins, the tosses are mutually
independent iff for every subset of~$m$~coins, the probability that
every coin in the subset comes up heads is~$2^{-m}$.

\subsection{DNA Testing}

Assumptions about independence are routinely made in practice.
Frequently, such assumptions are quite reasonable.  Sometimes,
however, the reasonableness of an independence assumption is not so
clear, and the consequences of a faulty assumption can be severe.

For example, consider the following testimony from the O. J. Simpson
murder trial on May 15, 1995:
\begin{description}

\item[Mr. Clarke:] When you make these estimations of frequency---and
I believe you touched a little bit on a concept called independence?

\item[Dr. Cotton:] Yes, I did.

\item[Mr. Clarke:] And what is that again?

\item[Dr. Cotton:] It means whether or not you inherit one allele that
you have is not---does not affect the second allele that you might
get.  That is, if you inherit a band at 5,000 base pairs, that doesn't
mean you'll automatically or with some probability inherit one at
6,000.  What you inherit from one parent is what you inherit from the
other.

\item[Mr. Clarke:] Why is that important?

\item[Dr. Cotton:] Mathematically that's important because if that
were not the case, it would be improper to multiply the frequencies
between the different genetic locations.

\item[Mr. Clarke:] How do you---well, first of all, are these markers
independent that you've described in your testing in this case?

\end{description}

Presumably, this dialogue was as confusing to you as it was for the
jury.  Essentially, the jury was told that genetic markers in blood
found at the crime scene matched Simpson's.  Furthermore, they were
told that the probability that the markers would be found in a
randomly-selected person was at most 1 in 170 million.  This
astronomical figure was derived from statistics such as:
%
\begin{itemize}
\item 1 person in 100 has marker $A$.
\item 1 person in 50 marker $B$.
\item 1 person in 40 has marker $C$.
\item 1 person in 5 has marker $D$.
\item 1 person in 170 has marker $E$.
\end{itemize}
%
Then these numbers were multiplied to give the probability that a
randomly-selected person would have all five markers:
\begin{align*}
\pr{A \intersect B \intersect C \intersect D \intersect E}
    & = \pr{A} \cdot \pr{B} \cdot \pr{C} \cdot \pr{D} \cdot \pr{E}\\
    & = \frac{1}{100} \cdot \frac{1}{50} \cdot \frac{1}{40}
                     \cdot \frac{1}{5} \cdot \frac{1}{170}
     = \frac{1}{170{,}000{,}000}.
\end{align*}

\iffalse
\begin{align*}
\pr{A \intersect B \intersect C \intersect D \intersect E}
    & = \pr{A} \cdot \pr{B} \cdot \pr{C} \cdot \pr{D} \cdot \pr{E} \\[2pt]
    & = \frac{1}{100} \cdot \frac{1}{50} \cdot \frac{1}{40}
                      \cdot \frac{1}{5} \cdot \frac{1}{170} \\[2pt]
    & = \frac{1}{170{,}000{,}000}.
\end{align*}
\fi
%
The defense pointed out that this assumes that the markers appear
mutually independently.  Furthermore, all the statistics were based on
just a few hundred blood samples.  

After the trial, the jury was widely mocked for failing to
``understand'' the DNA evidence.  If you were a juror, would
\emph{you} accept the 1 in 170 million calculation?

\subsection{Pairwise Independence}

The definition of mutual independence seems awfully complicated
---there are so many subsets of events to consider!  Here's an example
that illustrates the subtlety of independence when more than two
events are involved.  Suppose that we flip three fair,
mutually-independent coins.  Define the following events:
%
\begin{itemize}
\item $A_1$ is the event that coin 1 matches coin 2.
\item $A_2$ is the event that coin 2 matches coin 3.
\item $A_3$ is the event that coin 3 matches coin 1.
\end{itemize}
%
Are $A_1$, $A_2$, $A_3$ mutually independent?

The sample space for this experiment is:
%
\[
    \set{HHH,\, HHT,\, HTH,\, HTT,\, THH,\, THT,\, TTH,\, TTT}.
\]
%
Every outcome has probability $(1/2)^3 = 1/8$ by our assumption that
the coins are mutually independent.

To see if events $A_1$, $A_2$, and $A_3$ are mutually independent, we
must check a sequence of equalities.  It will be helpful first to
compute the probability of each event $A_i$:
%
\begin{align*}
\pr{A_1} & = \pr{HHH} + \pr{HHT} + \pr{TTH} + \pr{TTT} \\[2pt]
         & = \frac{1}{8} + \frac{1}{8} + \frac{1}{8} + \frac{1}{8}%\\[2pt]
          = \frac{1}{2}.
\end{align*}
%
By symmetry, $\pr{A_2} = \pr{A_3} = 1/2$ as well.  Now we can begin
checking all the equalities required for mutual independence in
Definition~\ref{def:mutual_indep}:
\begin{align*}
\pr{A_1 \intersect A_2}
       & = \pr{HHH} + \pr{TTT}
         = \frac{1}{8} + \frac{1}{8}
         = \frac{1}{4}
         = \frac{1}{2} \cdot \frac{1}{2}\\
       & = \pr{A_1} \pr{A_2}.
\end{align*}

\iffalse
\begin{align*}
\pr{A_1 \intersect A_2}
	& = \pr{HHH} + \pr{TTT} \\[2pt]
        & = \frac{1}{8} + \frac{1}{8} \\[2pt]
        & = \frac{1}{4} \\[2pt]
        & = \frac{1}{2} \cdot \frac{1}{2}\\[2pt]
        & = \pr{A_1} \pr{A_2}.
\end{align*}\fi

By symmetry, $\pr{A_1 \intersect A_3} = \pr{A_1} \cdot \pr{A_3}$ and
$\pr{A_2 \intersect A_3} = \pr{A_2} \cdot \pr{A_3}$ must hold also.
Finally, we must check one last condition:

\begin{align*}
\pr{A_1 \intersect A_2 \intersect A_3}
        & = \pr{HHH} + \pr{TTT}
          = \frac{1}{8} + \frac{1}{8}
          = \frac{1}{4}\\
        & \textcolor{red}{\mathbf{\neq}} \frac{1}{8} = \pr{A_1} \pr{A_2} \pr{A_3}.
\end{align*}


\iffalse
\begin{align*}
\pr{A_1 \intersect A_2 \intersect A_3}      & = \pr{HHH} + \pr{TTT} \\[2pt]
                                & = \frac{1}{8} + \frac{1}{8} \\[2pt]
                                & = \frac{1}{4} \\[2pt]
                                & \neq \pr{A_1} \pr{A_2} \pr{A_3} = \frac{1}{8}.
\end{align*}
\fi
%
The three events $A_1$, $A_2$, and~$A_3$ are not mutually independent
even though any two of them are independent!  This not-quite mutual
independence seems weird at first, but it happens.  It even
generalizes:

\begin{definition}\label{kway_independent_events}
  A set $A_1$, $A_2$, \dots, of events is \term{$k$-way independent}
  iff every set of $k$ of these events is mutually independent.  The
  set is \term{pairwise independent} iff it is 2-way independent.
\end{definition}

So the sets $A_1$, $A_2$, $A_3$ above are pairwise independent, but
not mutually independent.  Pairwise independence is a much weaker
property than mutual independence.

For example, suppose that the prosecutors in the O.~J. Simpson trial
were wrong and markers $A$, $B$, $C$, $D$, and $E$ appear only
\emph{pairwise} independently.  Then the probability that a
randomly-selected person has all five markers is no more than:
%
\begin{align*}
\pr{A \intersect B \intersect C \intersect D \intersect E}
    & \leq \pr{A \intersect E} = \pr{A} \cdot \pr{E}\\
    & = \frac{1}{100} \cdot \frac{1}{170} = \frac{1}{17{,}000}.
\end{align*}
%
The first line uses the fact that $A \intersect B \intersect C \intersect
D \intersect E$ is a subset of $A \intersect E$.  (We picked out the $A$
and $E$ markers because they're the rarest.)  We use pairwise independence
on the second line.  Now the probability of a random match is 1 in
17,000 ---a far cry from 1 in 170 million!  And this is the strongest
conclusion we can reach assuming only pairwise independence.

On the other hand, the 1 in 17,000 bound that we get by assuming
pairwise independence is a lot better than the bound that we would
have if there were no independence at all.  For example, if the
markers are dependent, then it is possible that
\begin{quote}
everyone with marker~$E$ has marker~$A$,

everyone with marker~$A$ has marker~$B$,

everyone with marker~$B$ has marker~$C$, and

everyone with marker~$C$ has marker~$D$.
\end{quote}
In such a scenario, the probability of a match is
\begin{equation*}
    \pr{E} = \frac{1}{170}.
\end{equation*}

So a stronger independence assumption leads to a smaller bound on the
probability of a match.  The trick is to figure out what independence
assumption is reasonable.  Assuming that the markers are
\emph{mutually} independent may well \emph{not} be reasonable unless
you have examined hundreds of millions of blood samples.  Otherwise,
how would you know that marker~$D$ does not show up more frequently
whenever the other four markers are simultaneously present?

We will conclude our discussion of independence with a useful, and
somewhat famous, example known as the Birthday Principle.

\subsection{The Birthday Principle}\label{birthday_principle_sec}

There are 95 students in a class.  What is the probability that some
birthday is shared by two people?  Comparing 95 students to the 365
possible birthdays, you might guess the probability lies somewhere
around $1/4$ ---but you'd be wrong: the probability that there will be
two people in the class with matching birthdays is actually more than
$0.9999$.

To work this out, we'll assume that the probability that a randomly
chosen student has a given birthday is $1/d$, where $d= 365$ in this
case.  We'll also assume that a class is composed of $n$ randomly and
independently selected students, with $n=95$ in this case.  These
randomness assumptions are not really true, since more babies are born
at certain times of year, and students' class selections are typically
not independent of each other, but simplifying in this way gives us a
start on analyzing the problem.  More importantly, these assumptions
are justifiable in important computer science applications of birthday
matching.  For example, the birthday matching is a good model for
collisions between items randomly inserted into a hash table.  So we
won't worry about things like Spring procreation preferences that make
January birthdays more common, or about twins' preferences to take
classes together (or not).  \begin{editingnotes}
or that fact that a student
can't be selected twice in making up a class list.
\end{editingnotes}

Selecting a sequence of $n$ students for a class yields a sequence of
$n$ birthdays.  Under the assumptions above, the $d^n$ possible
birthday sequences are equally likely outcomes.  Let's examine the
consequences of this probability model by focussing on the $i$th and
$j$th elements in a birthday sequence, where $1 \leq i \neq j \leq n$.
It makes for a better story if we refer to the $i$th birthday as
``Alice's'' and the $j$th as ``Bob's.''

Now if Alice, Bob, Carol, and Don are four different people, then
whether Alice and Bob have matching birthdays is independent of
whether Carol and Don do.  What's more interesting is that whether
Alice and \emph{Carol} have the same birthday is independent of
whether Alice and Bob do.  This follows because Carol is as likely to
have the same birthday as Alice, independently of whatever birthdays
Alice and Bob happen to have; a formal proof of this claim appears in
Problem~\ref{PS_equal_birthdays}.  In short, the set of all events
that a couple has matching birthdays is \index{pairwise independent}
\emph{pairwise} independent, even for overlapping couples.  This will
be important in Chapter~\ref{deviation_chap} because pairwise
independence will be enough to justify some conclusions about the
expected number of matches.  However, these matching birthday events
are obviously \emph{not} even 3-way independent: if Alice and Bob
match, and also Alice and Carol match, then Bob and Carol will match.

\iffalse
We could justify all these assertions of independence using the four
step method, but it's pretty boring, and we'll skip it.
\fi

It turns out that as long as the number of students is noticeably
smaller than the number of possible birthdays, we can get a pretty
good estimate of the birthday matching probabilities by
\emph{pretending} that the matching events are mutually independent.
(An intuitive justification for this is that with only a small number
of matching pairs, it's likely that none of the pairs overlap.)  Then
the probability of \emph{no} matching birthdays would be the same as
the $r$th power of the probability that a couple does \emph{not} have
matching birthdays, where $r \eqdef \binom{n}{2}$ is the number of
couples.  That is, the probability of no matching birthdays would be
\begin{equation}\label{11dbinn2}
(1-1/d)^{\binom{n}{2}}.
\end{equation}
Using the fact that $1+x < e^x$ for all $x$,\footnote{This
  approximation is obtained by truncating the Taylor series $e^{-x} =
  1 - x + x^2/2!  - x^3/3! + \cdots$.  The approximation $e^{-x}
  \approx 1 - x$ is pretty accurate when $x$ is small.} we would conclude
that the probability of no matching birthdays is at most
\begin{equation}\label{bday-approx}
e^{-\binom{n}{2}/d}.
\end{equation}

The matching birthday problem fits in here so far as a nice example
illustrating pairwise and mutual independence, but it's actually not
hard to justify the bound~\eqref{bday-approx} without any pretence of
independence.  Namely, there are $d (d - 1) (d - 2) \cdots (d - (n -
1))$ length $n$ sequences of distinct birthdays.  So the probability
that everyone has a different birthday is:
\begin{align*}
\lefteqn{\frac{d (d - 1) (d - 2) \cdots (d - (n - 1))}{d^n}}\\
   & = \frac{d}{d} \cdot \frac{d-1}{d} \cdot \frac{d-2}{d} \cdots \frac{d - (n - 1)}{d}\\
   & = \paren{1 - \frac{0}{d}}
             \paren{1 - \frac{1}{d}}
             \paren{1 - \frac{2}{d}}
             \cdots
             \paren{1 - \frac{n - 1}{d}}\\
   & < e^0 \cdot e^{-1/d} \cdot e^{-2/d} \cdots e^{-(n-1)/d} 
             & \text{(since $1+x < e^x$)} \\
   & = e^{-\paren{\sum_{i=1}^{n-1} i/d}}\\
   & = e^{-\paren{n(n-1)/2d}}\\
   & = \text{the bound~\eqref{bday-approx}}.
\end{align*}

For $n=85$ and $d = 365$, the value of~\eqref{bday-approx} is less
than $1/17,000$, which means the probability of having some pair of
matching birthdays actually is more than $1 - 1/17,000 > 0.9999$.  So
it would be pretty astonishing if there were no pair of students in
the class with matching birthdays.

For $d \leq n^2/2$, the probability of no match turns out to be
asymptotically equal to the upper bound~\eqref{bday-approx}.  For $d =
n^2/2$ in particular, the probability of no match is asymptotically
equal to $1/e$.  This leads to a rule of thumb which is useful in many
contexts in computer science:

\textbox{
\begin{center}
\large The \index{birthday principle} Birthday Principle
\end{center}

If there are $d$ days in a year and $\sqrt{2d}$ people in a
room, then the probability that two share a birthday is about 
$1 - 1/e \approx 0.632$.
}

For example, the Birthday Principle says that if you have $\sqrt{2
  \cdot 365} \approx 27$ people in a room, then the probability that
two share a birthday is about $0.632$.  The actual probability is
about $0.626$, so the approximation is quite good.

Among other applications, it implies that to use a hash function that
maps $n$ items into a hash table of size $d$, you can expect many
collisions unless $n^2$ is a small fraction of $d$.  The Birthday
Principle also famously comes into play as the basis of ``birthday
attacks'' that crack certain cryptographic systems.


\begin{problems}
\practiceproblems
\pinput{TP_Binomial_Board_Breaking}
\pinput{TP_Practice_with_Bounds}
%\pinput{FP_random_sampling}

\examproblems
\pinput{FP_college_probability}
\pinput{FP_product_rule_and_independence}

\classproblems
\pinput{CP_mutual_independence}
\pinput{CP_three_fair_coins}

\homeworkproblems
\pinput{PS_bogus_discrimination_contradiction}
\pinput{FP_graph_logic_probability}

\end{problems}


\iffalse %ftl version

\subsection{The Birthday Paradox}\label{birthday_principle_sec}

Suppose that there are 100 students in a class.  What is the
probability that some birthday is shared by two people?  Comparing 100
students to the 365 possible birthdays, you might guess the
probability lies somewhere around~$1/3$---but you'd be wrong: the
probability that there will be two people in the class with matching
birthdays is actually~$0.999999692\dots$.  In other words, the
probability that all 100 birthdays are different is less than 1
in~3,000,000.

Why is this probability so small?  The answer involves a phenomenon
known as the \term{Birthday Paradox} (or the \term{Birthday
  Principle}), which is surprisingly important in computer science, as
we'll see later.

Before delving into the analysis, we'll need to make some modeling
assumptions:
\begin{itemize}

\item
For each student, all possible birthdays are equally likely.  The idea
underlying this assumption is that each student's birthday is
determined by a random process involving parents, fate, and, um, some
issues that we discussed earlier in the context of graph theory.
The assumption is not completely accurate, however; a disproportionate
number of babies are born in August and September, for example.

\item
Birthdays are mutually independent.  This isn't perfectly accurate
either.  For example, if there are twins in the class, then their
birthdays are surely not independent.

\end{itemize}
We'll stick with these assumptions, despite their limitations.  Part
of the reason is to simplify the analysis.  But the bigger reason is
that our conclusions will apply to many situations in computer science
where twins, leap days, and romantic holidays are not considerations.
After all, whether or not two items collide in a hash table really has
nothing to do with human reproductive preferences.  Also, in pursuit
of generality, let's switch from specific numbers to variables.  Let
$m$~be the number of people in the room, and let $N$~be the number of
days in a year.

We can solve this problem using the standard four-step method.
However, a tree diagram will be of little value because the sample
space is so enormous.  This time we'll have to proceed without the
visual aid!

\paragraph{Step 1: Find the Sample Space}

Let's number the people in the room from 1 to~$m$.  An outcome of the
experiment is a sequence $(b_1, \dots, b_m)$ where $b_i$~is the
birthday of the $i$th person.  The sample space is the set of all such
sequences:
\begin{equation*}
    \sspace = \{\, (b_1, \dots, b_m) \suchthat b_i \in \set{1, \dots
      N} \,\}.
\end{equation*}

\paragraph{Step 2: Define Events of Interest}

Our goal is to determine the probability of the event~$A$ in which
some pair of people have the same birthday.  This event is a little
awkward to study directly, however.  So we'll use a common trick,
which is to analyze the \term{complementary} event~$\setcomp{A}$, in
which all $m$~people have different birthdays:
\begin{equation*}
    \setcomp{A} = \set{\, (b_1, \dots, b_m) \in \sspace
                    \suchthat \text{all $b_i$ are distinct} \,}.
\end{equation*}
If we can compute $\pr{\setcomp{A}}$, then we can compute what
really want, $\pr{A}$, using the identity
\begin{equation*}
    \pr{A} + \pr{\setcomp{A}} = 1.
\end{equation*}

\paragraph{Step 3: Assign Outcome Probabilities}

We need to compute the probability that $m$~people have a particular
combination of birthdays ~$(b_1, \dots, b_m)$.  There are $N$~possible
birthdays and all of them are equally likely for each student.
Therefore, the probability that the $i$th person was born on day~$b_i$
is~$1/N$.  Since we're assuming that birthdays are mutually
independent, we can multiply probabilities.  Therefore, the
probability that the first person was born on day~$b_1$, the second
on~$b_2$, and so forth is~$(1/N)^m$.  This is the probability of every
outcome in the sample space, which means that the sample space is
uniform.  That's good news, because, as we have seen, it means that
the analysis will be simpler.

\paragraph{Step 4: Compute Event Probabilities}

We're interested in the probability of the event~$\setcomp{A}$ in
which everyone has a different birthday:
\begin{equation*}
    \setcomp{A} = \set{\, (b_1, \dots, b_n) \suchthat
                            \text{all $b_i$ are distinct} \,}.
\end{equation*}
This is a gigantic set.  In fact, there are $N$~choices for~$b_i$,
\ $N - 1$ choices for~$b_2$, and so forth.  Therefore, by the
Generalized Product Rule,
\begin{equation*}
\card{\setcomp{A}}
    = \frac{N!}{(N - m)!}
    = N (N - 1) (N - 2) \cdots (N - m + 1).
\end{equation*}
Since the sample space is uniform, we can conclude that
\begin{equation}\label{eqn:15E4}
\pr{\setcomp{A}}
    = \frac{\card{\setcomp{A}}}{N^m} \\
    = \frac{N!}{N^m (N - m)!}.
\end{equation}
We're done!

Or are we?  While correct, it would certainly be nicer to have a
closed-form expression for equation~\eqref{eqn:15E4}.  That means
finding an approximation for $N!$ and~$(N - m)!$.  But this is what we
learned how to do in Section~\ref{sec:closed_products}.  In fact, since
$N$ and~$N - m$ are each at least~100, we know from
Corollary~\ref{cor:9A2} that
\begin{equation*}
    \stirling{N} \quad \text{and} \quad \stirling*{N - m}
\end{equation*}
are excellent approximations (accurate to within~.09\%) of~$N!$ and~$(N
- m)!$, respectively.  Plugging these values into
equation~\eqref{eqn:15E4} means that (to within~.2\%)\footnote{If there
are two terms that can be off by~.09\%, then the ratio can be off by
at most a factor of~$(1.0009)^2 < 1.002$.}
\begingroup
\openup2\jot
\begin{align}
\prob{\setcomp{A}}
    &= \frac{ \stirling{N} }{ N^m \stirling*{N - m} } \notag\\
    &= \sqrt{\frac{N}{N - m}}
             \frac{ e^{N \ln(N) - N} }
                  { e^{m \ln(N)} e^{(N - m) \ln(N - m) - (N - m) } }
                  \notag\\
    &= \sqrt{\frac{N}{N - m}}
         e^{ (N - m)\ln(N) - (N - m) \ln(N - m) - m } \notag\\
    &= \sqrt{\frac{N}{N - m}}
         e^{ (N - m)\ln\paren{\frac{N}{N - m}} - m } \notag\\
    &= e^{ \paren{N - m + \frac{1}{2}}\ln\paren{\frac{N}{N - m}} - m }.
        \label{eqn:15E9}
\end{align}
\endgroup
We can now evaluate equation~\eqref{eqn:15E9} for $m = 100$ and $N =
365$ to find that the probability that all 100 birthdays are different
is\footnote{The possible .2\%~error is so small that
  it is lost in the \dots after 3.07.}
\begin{equation*}
    3.07\ldots \cdot 10^{-7}.
\end{equation*}

We can also plug in other values of~$m$ to find the number of people
so that the probability of a matching birthday will be about~$1/2$.
In particular, for $m = 23$ and $N = 365$, equation~\eqref{eqn:15E9}
reveals that the probability that all the birthdays differ is
0.49\dots.  So if you are in a room with 23 other people, the
probability that some pair of people share a birthday will be a little
better than~$1/2$.  It is because 23 seems like such a small number of
people for a match that the phenomenon is called the \term{Birthday
  Paradox}.

\subsection{Applications to Hashing}

Hashing is frequently used in computer science to map large strings of
data into short strings of data.  In a typical scenario, you have a
set of $m$~items and you would like to assign each item to a number
from 1 to~$N$ where no pair of items is assigned to the same number
and $N$~is as small as possible.  For example, the items might be
messages, addresses, or variables.  The numbers might represent
storage locations, devices, indices, or digital signatures.

If two items are assigned to the same number, then a \term{collision}
is said to occur.  Collisions are generally bad.  For example,
collisions can correspond to two variables being stored in the same
place or two messages being assigned the same digital signature.  Just
imagine if you were doing electronic banking and your digital
signature for a \$10~check were the same as your signature for a
\$10~million dollar check.  In fact, finding collisions is a common
technique in breaking cryptographic codes.\footnote{Such techniques
  are often referred to as \term{birthday attacks} because of the
  association of such attacks with the Birthday Paradox.}

In practice, the assignment of a number to an item is done using a
hash function
\begin{equation*}
    h: S \to [1, N],
\end{equation*}
where $S$~is the set of items and $m = \card{S}$.  Typically, the
values of~$h(S)$ are assigned randomly and are assumed to be equally
likely in~$[1, N]$ and mutually independent.

For efficiency purposes, it is generally desirable to make~$N$ as
small as necessary to accommodate the hashing of $m$~items without
collisions.  Ideally, $N$~would be only a little larger than~$m$.
Unfortunately, this is not possible for random hash functions.  To see
why, let's take a closer look at equation~\eqref{eqn:15E9}.

By Theorem~\ref{thm:stirling} and the derivation of
equation~\eqref{eqn:15E9}, we know that the probability that there are
no collisions for a random hash function is
\begin{equation}\label{eqn:16K}
    \sim e^{ \paren{N - m + \frac{1}{2}} \ln\paren{\frac{N}{N - m}} - m }.
\end{equation}
For any~$m$, we now need to find a value of~$N$ for which this
expression is at least~1/2.  That will tell us how big the hash table
needs to be in order to have at least a 50\%~chance of avoiding
collisions.  This means that we need to find a value of~$N$ for which
\begin{equation}\label{eqn:16P}
    \paren{N - m + \frac{1}{2}} \ln\paren{\frac{N}{N - m}} - m 
        \sim
    \ln\paren{\frac{1}{2}}.
\end{equation}

To simplify equation~\eqref{eqn:16P}, we need to get rid of the
$\ln\paren{\frac{N}{N - m}}$~term.  We can do this by using the Taylor
Series expansion for
\begin{equation*}
    \ln(1 - x) = -x - \frac{x^2}{2} - \frac{x^3}{3} - \cdots
\end{equation*}
to find that\footnote{This may not look like a simplification, but
  stick with us here.}
\begin{align*}
\ln\paren{\frac{N}{N - m}}
    &= - \ln \paren{\frac{N - m}{N}} \\
    &= - \ln \paren{1 - \frac{m}{N}} \\
    &= - \paren{ -\frac{m}{N} - \frac{m^2}{2N^2} - \frac{m^3}{3N^3} - \cdots }\\
    &= \frac{m}{N} + \frac{m^2}{2N^2} + \frac{m^3}{3N^3} + \cdots.
\end{align*}
Hence,
\begin{align}
\paren{N - m + \frac{1}{2}} \ln\paren{\frac{N}{N - m}} - m
    &= \paren{N - m + \frac{1}{2}}
        \paren{\frac{m}{N} + \frac{m^2}{2N^2} + \frac{m^3}{3N^3} + \cdots}
        - m \notag\\
    &= \paren{ m + \frac{m^2}{2N} + \frac{m^3}{3N^2} + \cdots }
            \notag\\
    &\phantom{=}\qquad - \paren{ \frac{m^2}{N} + \frac{m^3}{2N^2} +
          \frac{m^4}{3N^3} + \cdots }
            \notag\\
    &\phantom{=}\qquad + \frac{1}{2} \paren{\frac{m}{N} +
          \frac{m^2}{2N^2} + \frac{m^3}{3N^3} + \cdots} -m 
            \notag\\
    &= - \paren{ \frac{m^2}{2N} + \frac{m^3}{6N^2} + \frac{m^4}{12N^3}
            + \cdots} \notag\\
    &\phantom{=}\qquad
        + \frac{1}{2}\paren{ \frac{m}{N} + \frac{m^2}{2N^2} +
          \frac{m^3}{3N^3} + \cdots }.
    \label{eqn:16Q}
\end{align}

If $N$~grows faster than~$m^2$, then the value in
equation~\eqref{eqn:16Q} tends to~0 and equation~\eqref{eqn:16P}
cannot be satisfied.  If $N$~grows more slowly than~$m^2$, then the
value in equation~\eqref{eqn:16Q} diverges to negative infinity, and,
once again, equation~\eqref{eqn:16P} cannot be satisfied.  This suggests
that we should focus on the case where~$N = \Theta(m^2)$, when
equation~\eqref{eqn:16Q} simplifies to
\begin{equation*}
    \sim \frac{-m^2}{2N}
\end{equation*}
and equation~\eqref{eqn:16P} becomes
\begin{equation}\label{eqn:16R}
    \frac{-m^2}{2N} \sim \ln\paren{\frac{1}{2}}.
\end{equation}

equation~\eqref{eqn:16R} is satisfied when
\begin{equation}\label{eqn:16S}
    N \sim \frac{m^2}{2 \ln(2)}.
\end{equation}

In other words, $N$~needs to grow quadratically with~$m$ in order to
avoid collisions.  This unfortunate fact is known as the
\term{Birthday Principle} and it limits the efficiency of hashing in
practice ---either $N$~is quadratic in the number of items being hashed
or you need to be able to deal with collisions.

\fi

\endinput
