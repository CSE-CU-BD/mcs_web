\chapter{Events and Probability Spaces}\label{probability_chap}

\section{Let's Make a Deal}\label{monty_sec}

In the September 9, 1990 issue of \emph{Parade} magazine, columnist
Marilyn vos Savant responded to this letter:

%  future:
%  Include the other problem that Marilyn addressed in the same issue.

\begin{quote}
\emph{Suppose you're on a game show, and you're given the
choice of three doors.  Behind one door is a car, behind the others,
goats.  You pick a door, say number 1, and the host, who knows what's
behind the doors, opens another door, say number 3, which has a goat.
He says to you, ``Do you want to pick door number 2?''  Is it to your
advantage to switch your choice of doors?}
\begin{flushright}
\begin{tabular}{l}
Craig. F. Whitaker \\
Columbia, MD
\end{tabular}
\end{flushright}
\end{quote}

The letter describes a situation like one faced by contestants in the
1970's game show \emph{Let's Make a Deal}, hosted by Monty Hall and
Carol Merrill.  Marilyn replied that the contestant should indeed
switch.  She explained that if the car was behind either of the two
unpicked doors---which is twice as likely as the the car being behind
the picked door---the contestant wins by switching.  But she soon
received a torrent of letters, many from mathematicians, telling her
that she was wrong.  The problem became known as the \term{Monty Hall
  Problem} and it generated thousands of hours of heated debate.

This incident highlights a fact about probability: the subject uncovers
lots of examples where ordinary intuition leads to completely wrong
conclusions.  So until you've studied probabilities enough to have
refined your intuition, a way to avoid errors is to fall back on a
rigorous, systematic approach such as the Four Step Method that we
will describe shortly.  First, let's make sure we really understand
the setup for this problem.  This is always a good thing to do when
you are dealing with probability.

\subsection{Clarifying the Problem}

Craig's original letter to Marilyn vos Savant is a bit vague, so we
must make some assumptions in order to have any hope of modeling the
game formally.  For example, we will assume that:
\begin{enumerate}

\item The car is equally likely to be hidden behind each of the three
doors.

\item The player is equally likely to pick each of the three doors,
regardless of the car's location.

\item After the player picks a door, the host \emph{must} open a
different door with a goat behind it and offer the player the choice
of staying with the original door or switching.

\item If the host has a choice of which door to open, then he is
equally likely to select each of them.

\end{enumerate}
In making these assumptions, we're reading a lot into Craig
Whitaker's letter.  There are other plausible interpretations that
lead to different answers.  But let's accept these assumptions for now
and address the question, ``What is the probability that a player who
switches wins the car?''

\section{The Four Step Method}\label{4step_sec}

Every probability problem involves some sort of randomized experiment,
process, or game.  And each such problem involves two distinct
challenges:
%
\begin{enumerate}
\item How do we model the situation mathematically?
\item How do we solve the resulting mathematical problem?
\end{enumerate}
%
In this section, we introduce a four step approach to questions of the
form, ``What is the probability that\dots ?''  In this approach, we
build a probabilistic model step by step, formalizing the original
question in terms of that model.  Remarkably, this structured approach
provides simple solutions to many famously confusing problems.  For
example, as you'll see, the four step method cuts through the
confusion surrounding the Monty Hall problem like a Ginsu knife.

\subsection{Step 1:  Find the Sample Space}

Our first objective is to identify all the possible outcomes of the
experiment.  A typical experiment involves several randomly-determined
quantities.  For example, the Monty Hall game involves three such
quantities:
%
\begin{enumerate}
\item The door concealing the car.
\item The door initially chosen by the player.
\item The door that the host opens to reveal a goat.
\end{enumerate}
%
Every possible combination of these randomly-determined quantities is
called an \term{outcome}.  The set of all possible outcomes is called
the \term{sample space} for the experiment.

A \term{tree diagram} is a graphical tool that can help us work
through the four step approach when the number of outcomes is not too
large or the problem is nicely structured.  In particular, we can use
a tree diagram to help understand the sample space of an experiment.
The first randomly-determined quantity in our experiment is the door
concealing the prize.  We represent this as a tree with three
branches, as shown in Figure~\ref{fig:14A1}.  In this diagram, the
doors are called $A$, $B$, and $C$ instead of 1, 2, and 3, because
we'll be adding a lot of other numbers to the picture later.

\begin{figure}

\graphic{Monty1}

\caption{The first level in a tree diagram for the Monty Hall
  Problem.  The branches correspond to the door behind which the car
  is located.}

\label{fig:14A1}

\end{figure}

For each possible location of the prize, the player could initially
choose any of the three doors.  We represent this in a second layer
added to the tree.  Then a third layer represents the possibilities of
the final step when the host opens a door to reveal a goat, as shown
in Figure~\ref{fig:14A2}.

\begin{figure}

\graphic{Monty2}

\caption{The full tree diagram for the Monty Hall Problem.  The second
level indicates the door initially chosen by the player.  The third
level indicates the door revealed by Monty Hall.}

\label{fig:14A2}

\end{figure}

Notice that the third layer reflects the fact that the host has either one
choice or two, depending on the position of the car and the door initially
selected by the player.  For example, if the prize is behind door A and
the player picks door B, then the host must open door C.  However, if the
prize is behind door A and the player picks door A, then the host could
open either door B or door C.

\begin{figure}

\graphic{Monty3}

\caption{The tree diagram for the Monty Hall Problem with the outcomes
  labeled for each path from root to leaf.  For example, outcome $(A,
  A, B)$ corresponds to the car being behind door~$A$, the player
  initially choosing door~$A$, and Monty Hall revealing the goat
  behind door~$B$.}

\label{fig:14A3}

\end{figure}

Now let's relate this picture to the terms we introduced earlier: the
leaves of the tree represent \emph{outcomes} of the experiment, and
the set of all leaves represents the \emph{sample space}.  Thus, for
this experiment, the sample space consists of 12 outcomes.  For
reference, we've labeled each outcome in Figure~\ref{fig:14A3} with a
triple of doors indicating:
%
\[
    (\text{door concealing prize}, \;
    \text{door initially chosen}, \;
     \text{door opened to reveal a goat}).
\]
%
In these terms, the sample space is the set
%
\[
\sspace = \left\{
\begin{array}{c@{\;}c@{\;}c@{\;}c@{\;}c@{\;}c}
(A, A, B), & (A, A, C), & (A, B, C), & (A, C, B), & (B, A, C), & (B, B, A), \\
(B, B, C), & (B, C, A), & (C, A, B), & (C, B, A), & (C, C, A), & (C, C, B)
\end{array}
\right\}
\]
%
The tree diagram has a broader interpretation as well: we can regard the
whole experiment as following a path from the root to a leaf, where the
branch taken at each stage is ``randomly'' determined.  Keep this
interpretation in mind; we'll use it again later.

\subsection{Step 2: Define Events of Interest}

Our objective is to answer questions of the form ``What is the
probability that \dots ?'', where, for example, the missing phrase
might be ``the player wins by switching,'' ``the player initially
picked the door concealing the prize,'' or ``the prize is behind door
C.''

A set of outcomes is called an \term{event}.  Each of the preceding
phrases characterizes an event.  For example, the event
$[\text{prize is behind door $C$}]$ refers to the set:
%
\[
    \set{(C, A, B), (C, B, A), (C, C, A), (C, C, B)},
\]
%
and the event $[\text{prize is behind the door first picked
    by the player}]$ is:
%
\[
    \set{(A, A, B), (A, A, C), (B, B, A), (B, B, C), (C, C, A), (C, C, B)}.
\]
%
Here we're using square brackets around a property of outcomes as a
notation for the event whose outcomes are the ones that satisfy the
property.

What we're really after is the event $[\text{player wins by switching}]$:
\begin{equation}\label{swwin-event}
\set{(A, B, C), (A, C, B), (B, A, C), (B, C, A), (C, A, B), (C, B, A)}.
\end{equation}
The outcomes in this event are marked with checks in Figure~\ref{fig:14A4}.
\begin{figure}

\graphic{Monty4}

\caption{The tree diagram for the Monty Hall Problem, where the
  outcomes where the player wins by switching are denoted with a
  check mark.}

\label{fig:14A4}

\end{figure}

Notice that exactly half of the outcomes are checked, meaning that the
player wins by switching in half of all outcomes.  You might be
tempted to conclude that a player who switches wins with probability
$1/2$.  \emph{This is wrong.}  The reason is that these outcomes are
not all equally likely, as we'll see shortly.

\subsection{Step 3: Determine Outcome Probabilities}

So far we've enumerated all the possible outcomes of the experiment.
Now we must start assessing the likelihood of those outcomes.  In
particular, the goal of this step is to assign each outcome a
probability, indicating the fraction of the time this outcome is
expected to occur.  The sum of all the outcome probabilities must
equal one, reflecting the fact that there always must be an outcome.

Ultimately, outcome probabilities are determined by the phenomenon
we're modeling and thus are not quantities that we can derive
mathematically.  However, mathematics can help us compute the
probability of every outcome \emph{based on fewer and more
elementary modeling decisions.}  In particular, we'll break the task
of determining outcome probabilities into two stages.

\subsubsection{Step 3a: Assign Edge Probabilities}

First, we record a probability on each \emph{edge} of the tree
diagram.  These edge-probabilities are determined by the assumptions
we made at the outset: that the prize is equally likely to be behind
each door, that the player is equally likely to pick each door, and
that the host is equally likely to reveal each goat, if he has a
choice.  Notice that when the host has no choice regarding which door
to open, the single branch is assigned probability 1.  For example,
see Figure~\ref{fig:14A6}.
\iffalse

\begin{figure}

\graphic{Monty5}

\caption{The tree diagram for the Monty Hall Problem where edge
  weights denote the probability of that branch being taken, given that
  we are at the parent of that branch.  For example, if the car is
  behind door~$A$, then there is a 1/3~chance that the player's
  initial selection is door~$B$.}

\label{fig:14A5}

\end{figure}
\fi

\subsubsection{Step 3b: Compute Outcome Probabilities}

Our next job is to convert edge probabilities into outcome
probabilities.  This is a purely mechanical process:
\begin{quote}
calculate the probability of an outcome by multiplying the
edge-probabilities on the path from the root to that outcome.
\end{quote}
For example, the probability of the topmost outcome in
Figure~\ref{fig:14A6}, $(A, A, B)$, is
\begin{equation}\label{131312118}
\frac{1}{3} \cdot \frac{1}{3} \cdot \frac{1}{2} = \frac{1}{18}.
\end{equation}

We'll examine the official justification for this rule in
Section~\ref{product_rule_subsec}, but here's an easy, intuitive
justification: as the steps in an experiment progress randomly along a
path from the root of the tree to a leaf, the probabilities on the
edges indicate how likely the path is to proceed along each branch.
For example, a path starting at the root in our example is equally
likely to go down each of the three top-level branches.

How likely is such a path to arrive at the topmost outcome, $(A, A,
B)$?  Well, there is a 1-in-3 chance that a path would follow the
$A$-branch at the top level, a 1-in-3 chance it would continue along
the $A$-branch at the second level, and 1-in-2 chance it would follow
the $B$-branch at the third level.  Thus, there is half of a one third
of a one third chance, of arriving at the $(A, A, B)$ leaf.  That is,
the chance is $1/3 \cdot 1/3 \cdot 1/2 = 1/18$---the same product (in
reverse order) we arrived at in~\eqref{131312118}.

We have illustrated all of the outcome probabilities in
Figure~\ref{fig:14A6}.

\begin{figure}

\graphic{Monty6}

\caption{The tree diagram for the Monty Hall Problem where edge
  weights denote the probability of that branch being taken given that
  we are at the parent of that branch.  For example, if the car is
  behind door~$A$, then there is a 1/3~chance that the player's
  initial selection is door~$B$.
  The rightmost column shows the outcome probabilities for the
  Monty Hall Problem.  Each outcome probability is simply the product
  of the probabilities on the path from the root to
  the outcome leaf.}

\label{fig:14A6}
\end{figure}

Specifying the probability of each outcome amounts to defining a
function that maps each outcome to a probability.  This function is
usually called $\pr{\cdot}$.  In these terms, we've just determined
that:
\begin{align*}
\pr{(A, A, B)} & = \frac{1}{18}, \\
\pr{(A, A, C)} & = \frac{1}{18}, \\
\pr{(A, B, C)} & = \frac{1}{9}, \\
               & \text{etc.}
\end{align*}

\subsection{Step 4: Compute Event Probabilities}

We now have a probability for each \emph{outcome}, but we want to
determine the probability of an \emph{event}.  The probability of an
event~$E$ is denoted by $\pr{E}$, and it is the sum of the
probabilities of the outcomes in~$E$.  For example, the probability of
the [switching wins] event~\eqref{swwin-event} is
\begin{align*}
\lefteqn{\pr{\text{switching wins}}}\\
    & = \pr{(A, B, C)} + \pr{(A, C, B)} + \pr{(B, A, C)} + \\
    & \qquad \pr{(B, C, A)} + \pr{(C, A, B)} + \pr{(C, B, A)} \\
    & = \frac{1}{9} + \frac{1}{9} + \frac{1}{9} +
        \frac{1}{9} + \frac{1}{9} + \frac{1}{9} \\
    & = \frac{2}{3}.
\end{align*}
It seems Marilyn's answer is correct!  A player who switches doors
wins the car with probability~$2/3$.  In contrast, a player who stays
with his or her original door wins with probability $1/3$, since
staying wins if and only if switching loses.

We're done with the problem!  We didn't need any appeals to intuition
or ingenious analogies.  In fact, no mathematics more difficult than
adding and multiplying fractions was required.  The only hard part was
resisting the temptation to leap to an ``intuitively obvious'' answer.

\subsection{An Alternative Interpretation of the Monty Hall Problem}

Was Marilyn really right?  Our analysis indicates that she was.  But a
more accurate conclusion is that her answer is correct \emph{provided
  we accept her interpretation of the question}.  There is an equally
plausible interpretation in which Marilyn's answer is wrong.  Notice
that Craig Whitaker's original letter does not say that the host is
\emph{required} to reveal a goat and offer the player the option to
switch, merely that he \emph{did} these things.  In fact, on the
\emph{Let's Make a Deal} show, Monty Hall sometimes simply opened the
door that the contestant picked initially.  Therefore, if he wanted
to, Monty could give the option of switching only to contestants who
picked the correct door initially.  In this case, switching never
works!

%% Monty Hall Problems %%%%%%%%%%%%%%%%%%%%%%%%%%%%%%%%%%%%%%%%%%%%%%%%%%%%%%%%

\begin{problems}
\practiceproblems
%\pinput{TP_A_random_number}  MISSING
\pinput{TP_Binomial_Random_Variable}

\examproblems
\pinput{FP_probability}

\classproblems
\pinput{CP_a_baseball_series}
\pinput{CP_coin_flips}
%\pinput{CP_simulating_fair_coin}

\homeworkproblems
\pinput{PS_four_door_random_or_not}
\pinput{PS_immortal_probability}
\pinput{PS_black_and_red_cards_revised}

\end{problems}

\section{Strange Dice}\label{sec:strange-dice}

The four-step method is surprisingly powerful.  Let's get some more
practice with it.  Imagine, if you will, the following scenario.

It's a typical Saturday night.  You're at your favorite pub,
contemplating the true meaning of infinite cardinalities, when a
burly-looking biker plops down on the stool next to you.  Just as you
are about to get your mind around~$\power(\power(\reals))$, biker dude
slaps three strange-looking dice on the bar and challenges you to a
\$100 wager.  His rules are simple.  Each player selects one die and
rolls it once.  The player with the lower value pays the other
player~\$100.

Naturally, you are skeptical, especially after you see that these are
not ordinary dice.  Each die has the usual six sides, but opposite
sides have the same number on them, and the numbers on the dice are
different, as shown in Figure~\ref{fig:14A7}.

\begin{figure}

\graphic{Fig_A7}

\caption{The strange dice.  The number of pips on
  each concealed face is the same as the number on the opposite face.
  For example, when you roll die~$A$, the probabilities of getting a
  2, 6, or~7 are each~$1/3$.}

\label{fig:14A7}

\end{figure}

Biker dude notices your hesitation, so he sweetens his offer: he will
pay you \$105 if you roll the higher number, but you only need pay him
\$100 if he rolls higher, \emph{and} he will let you pick a die first,
after which he will pick one of the other two.  The sweetened deal
sounds persuasive since it gives you a chance to pick what you think
is the best die, so you decide you will play.  But which of the dice
should you choose?  Die~$B$ is appealing because it has a~9, which is
a sure winner if it comes up.  Then again, die~$A$ has two fairly
large numbers, and die~$C$ has an~8 and no really small values.

In the end, you choose die~$B$ because it has a~9, and then biker dude
selects die~$A$.  Let's see what the probability is that you will win.
(Of course, you probably should have done this before picking die~$B$
in the first place.)  Not surprisingly, we will use the four-step
method to compute this probability.

\subsection{Die~$A$ versus Die~$B$}

\paragraph{Step 1: Find the sample space.}

The tree diagram for this scenario is shown in Figure~\ref{fig:14A8}.
In particular, the sample space for this experiment are the nine pairs
of values that might be rolled with Die~$A$ and Die~$B$:

For this experiment, the sample space is a set of nine outcomes:
\begin{equation*}
\sspace = \set{\, (2, 1), \, (2, 5), \, (2, 9), \,
                  (6, 1), \, (6, 5), \, (6, 9), \,
                  (7, 1), \, (7, 5), \, (7, 9) \,}.
\end{equation*}


\iffalse
\footnote{Actually, the whole
  probability space is worked out in this one picture.  But pretend
  that each component sort of fades in---nyyrrroom!---as you read
  about the corresponding step below.}\fi

\begin{figure}

\graphic{Fig_A8}

\caption{The tree diagram for one roll of die~$A$ versus die~$B$.
  Die~$A$ wins with probability~$5/9$.}

\label{fig:14A8}

\end{figure}


\paragraph{Step 2: Define events of interest.}

We are interested in the event that the number on die~$A$ is greater
than the number on die~$B$.  This event is a set of five outcomes:
\begin{equation*}
    \set{\, (2, 1), \, (6, 1), \, (6, 5), \, (7, 1), \, (7, 5) \,}.
\end{equation*}
These outcomes are marked~$A$ in the tree diagram in
Figure~\ref{fig:14A8}.

\paragraph{Step 3: Determine outcome probabilities.}

To find outcome probabilities, we first assign probabilities to edges
in the tree diagram.  Each number on each die comes up with
probability~$1/3$, regardless of the value of the other die.
Therefore, we assign all edges probability~$1/3$.  The probability of
an outcome is the product of the probabilities on the corresponding
root-to-leaf path, which means that every outcome has
probability~$1/9$.  These probabilities are recorded on the right side
of the tree diagram in Figure~\ref{fig:14A8}.

\paragraph{Step 4: Compute event probabilities.}

The probability of an event is the sum of the probabilities of the
outcomes in that event.  In this case, all the outcome probabilities
are the same, so we say that the sample space is \emph{\idx{uniform}}.
Computing event probabilities for uniform sample spaces is
particularly easy since you just have to compute the number of
outcomes in the event.  In particular, for any event~$E$ in a uniform
sample space~$\sspace$,
\begin{equation}\label{eqn:14F1}
    \pr{E} = \frac{\card{E}}{\card{\sspace}}.
\end{equation}
In this case, $E$~is the event that die~$A$ beats die~$B$, so
$\card{E} = 5$, \ $\card{\sspace} = 9$, and
\begin{equation*}
    \pr{E} = 5/9.
\end{equation*}

This is bad news for you.  Die~$A$ beats die~$B$ more than half the
time and, not surprisingly, you just lost~\$100.

Biker dude consoles you on your ``bad luck'' and, given that he's a
sensitive guy beneath all that leather, he offers to go double or
nothing.\footnote{\term{Double or nothing} is slang for doing another
  wager after you have lost the first.  If you lose again, you will
  owe biker dude \emph{double} what you owed him before.  If you win,
  you will owe him \emph{nothing}; in fact, since he should pay you
  \$210 if he loses, you would come out \$10 ahead.}  Given that your
wallet only has \$25 in it, this sounds like a good plan.  Plus, you
figure that choosing die~$A$ will give \emph{you} the advantage.

So you choose~$A$, and then biker dude chooses~$C$.  Can you guess who
is more likely to win?  (Hint: it is generally not a good idea to
gamble with someone you don't know in a bar, especially when you are
gambling with strange dice.)

\subsection{Die~$A$ versus Die~$C$}

We can construct the tree diagram and outcome probabilities as
before.  The result is shown in Figure~\ref{fig:14A9}, and there is bad
news again.  Die~$C$ will beat die~$A$ with probability~$5/9$, and you
lose once again.

\begin{figure}

\graphic{Fig_A9}

\caption{The tree diagram for one roll of die~$C$ versus die~$A$.
  Die~$C$ wins with probability~$5/9$.}

\label{fig:14A9}

\end{figure}

You now owe the biker dude \$200 and he asks for his money.  You reply
that you need to go to the bathroom.

\subsection{Die~$B$ versus Die~$C$}

Being a sensitive guy, biker dude nods understandingly and offers yet
another wager.  This time, he'll let you have die~$C$.  He'll even let
you raise the wager to~\$200 so you can win your money back.

This is too good a deal to pass up.  You know that die~$C$ is likely
to beat die~$A$ and that die~$A$ is likely to beat die~$B$, and so
die~$C$ is \emph{surely} the best.  Whether biker dude picks $A$
or~$B$, the odds would be in your favor this time.  Biker dude must
really be a nice guy.

So you pick~$C$, and then biker dude picks~$B$.  Wait---how come you
haven't caught on yet and worked out the tree diagram before you took
this bet?  If you do it now, you'll see by the same reasoning as
before that $B$ beats~$C$ with probability~$5/9$.  But surely there is
a mistake!  How is it possible that
\begin{quote}

$C$ beats~$A$ with probability~$5/9$,

$A$ beats~$B$ with probability~$5/9$,

$B$ beats~$C$ with probability~$5/9$?
\end{quote}


\iffalse

The tree diagram and outcome probabilities for $B$ versus~$C$ are
shown in Figure~\ref{fig:14A10}.  The data there show that
\emph{die~$B$} wins with probability~$5/9$.

\begin{figure}

\graphic{Fig_A10}

\caption{The tree diagram for one roll of die~$B$ versus die~$C$.
  Die~$B$ wins with probability~$5/9$.}

\label{fig:14A10}

\end{figure}
\fi

The problem is not with the math, but with your intuition.  Since $A$
will beat~$B$ more often than not, and $B$ will beat~$C$ more often
than not, it \emph{seems} like $A$ ought to beat~$C$ more often than
not, that is, the ``beats more often'' relation ought to be
\emph{\idx{transitive}}.  But this intuitive idea is simply false:
whatever die you pick, biker dude can pick one of the others and be
likely to win.  So picking first is actually a disadvantage, and as a
result, you now owe biker dude~\$400.

Just when you think matters can't get worse, biker dude offers you one
final wager for~\$1,000.  This time, instead of rolling each die once,
you will each roll your die twice, and your score is the sum of your
rolls, and he will even let you pick your die second, that is, after
he picks his.  Biker dude chooses die~$B$.  Now you know that die~$A$
will beat die~$B$ with probability~$5/9$ on one roll, so, jumping at
this chance to get ahead, you agree to play, and you pick
die~$A$.  After all, you figure that since a roll of die~$A$ beats a
roll of die~$B$ more often that not, two rolls of die~$A$ are even
more likely to beat two rolls of die~$B$, right?

Wrong! (Did we mention that playing strange gambling games with
strangers in a bar is a bad idea?)

\subsection{Rolling Twice}

If each player rolls twice, the tree diagram will have four levels and
$3^4 = 81$ outcomes.  This means that it will take a while to write
down the entire tree diagram.  But it's easy to write down the
first two levels as in Figure~\ref{fig:14A11}(a) and
then notice that the remaining two levels consist of nine identical
copies of the tree in Figure~\ref{fig:14A11}(b).

\begin{figure}

\graphic{Fig_A11}

\caption{Parts of the tree diagram for die~$B$ versus die~$A$ where
  each die is rolled twice.  The first two levels are shown in~(a).
  The last two levels consist of nine copies of the tree in~(b).}

\label{fig:14A11}

\end{figure}

The probability of each outcome is $(1/3)^4 = 1/81$ and so, once
again, we have a uniform probability space.  By
equation~\eqref{eqn:14F1}, this means that the probability that
$A$~wins is the number of outcomes where $A$ beats~$B$ divided by~81.

To compute the number of outcomes where $A$ beats~$B$, we observe that
the two rolls of die~$A$ result in nine equally likely 
outcomes in a sample space $\sspace_A$ in which the
two-roll sums take the values
\[
    (4, 8, 8, 9, 9, 12, 13, 13, 14).
\]
Likewise, two rolls of die~$B$ result in nine equally likely outcomes
in a sample space $\sspace_B$ in which the
two-roll sums take the values
\[
(2, 6, 6, 10, 10, 10, 14, 14, 18).
\]
We can treat the outcome of rolling both dice twice as a pair $(x,y) \in
\sspace_A \cross \sspace_B$, where $A$~wins iff the sum of the two
$A$-rolls of outcome $x$ is larger the sum of the two $B$-rolls of
outcome $y$.  If the $A$-sum is 4, there is only one~$y$ with a
smaller $B$-sum, namely, when the $B$-sum is 2.  If the $A$-sum is 8,
there are three~$y$'s with a smaller $B$-sum, namely, when the $B$-sum
is 2 or 6.  Continuing the count in this way, the number of pairs
$(x,y)$ for which the $A$-sum is larger than the $B$-sum is
\begin{equation*}
    1 + 3 + 3 + 3 + 3 + 6 + 6 + 6 + 6 = 37.
\end{equation*}
A similar count shows that there are 42~pairs for which $B$-sum is
larger than the $A$-sum, and there are two pairs where the sums are
equal, namely, when they both equal 14.  This means that $A$
\emph{loses} to~$B$ with probability $42/81 > 1/2$ and ties with
probability~$2/81$.  Die~$A$ wins with probability only~$37/81$.

How can it be that $A$~is more likely than~$B$ to win with one roll,
but $B$~is more likely to win with two rolls?  Well, why not?  The
only reason we'd think otherwise is our unreliable, untrained
intuition.  (Even the authors were surprised when they first learned
about this, but at least they didn't lose~\$1400 to biker dude.)  In
fact, the die strength reverses no matter which two die we picked.  So
for one roll,
\begin{equation*}
    A \succ B \succ C \succ A,
\end{equation*}
but for two rolls,
\begin{equation*}
    A \prec B \prec C \prec A,
\end{equation*}
where we have used the symbols $\succ$ and~$\prec$ to denote which die
is more likely to result in the larger value.

\subsubsection{Even Stranger Dice}

The weird behavior of the three strange dice above generalizes in a
remarkable way.\footnote{\TBA{Reference Ron Graham paper.}}  The idea is
that you can find arbitrarily large sets of dice which will beat each
other in any desired pattern according to how many times the dice are
rolled.  The precise statement of this result involves several
alternations of universal and existential quantifiers, so it may take
a few readings to understand what it is saying:

\iffalse Now that we know that strange things can happen with strange
dice, it is natural, at least for mathematicians, to ask how strange
things can get.  It turns out that things can get very strange.  In
fact, mathematicians recently made the following discovery:\fi

\begin{theorem}\label{thm:14F2}
For any $n \ge 2$, there is a set of $n$~dice with the following
property: for \emph{any} $n$-node digraph with exactly one directed
edge between every two distinct nodes,\footnote{In other words, for
  every pair of nodes $u \neq v$, either $\diredge{u}{v}$ or
  $\diredge{v}{u}$, but not both, are edges of the graph.  Such graphs
  are called \emph{tournament graphs}, see
  Problem~\ref{CP_tournament_graphs}.} there is a number of rolls~$k$
such that the sum of $k$~rolls of the $i$th die is bigger than the sum
for the $j$th die with probability greater than~$1/2$ iff there is an
edge from the $i$th to the $j$th node in the graph.
\end{theorem}

For example, the eight possible relative strengths for $n =
3$ dice are shown in Figure~\ref{fig:14A13}.  

\begin{figure}

\graphic{Fig_A13}

\caption{All possible relative strengths for three dice $D_1$, $D_2$,
  and~$D_3$.  The edge $\diredge{D_i}{D_j}$ denotes that the sum of
  rolls for~$D_i$ is likely to be greater than the sum of rolls
  for~$D_j$.}

\label{fig:14A13}

\end{figure}

Our analysis for the dice in Figure~\ref{fig:14A7} showed that for
one roll, we have the relative strengths shown in
Figure~\ref{fig:14A13}(a), and for two rolls, we have the (reverse)
relative strengths shown in Figure~\ref{fig:14A13}(b).  If you are
prone to gambling with strangers in bars, it would be a good idea to
try figuring out what other relative strengths are possible for the
dice in Figure~\ref{fig:14A7} when using more rolls.


\section{The Birthday Principle}\label{birthday_principle_sec}

There are 95 students in a class.  What is the probability that some
birthday is shared by two people?  Comparing 95 students to the 365
possible birthdays, you might guess the probability lies somewhere
around $1/4$---but you'd be wrong: the probability that there will be
two people in the class with matching birthdays is actually more than
$0.9999$.

To work this out, we'll assume that the probability that a randomly
chosen student has a given birthday is $1/d$.  We'll also assume that
a class is composed of $n$ randomly and independently selected
students.  Of course $d= 365$ and $n=95$ in this case, but we're
interested in working things out in general.  These randomness
assumptions are not really true, since more babies are born at certain
times of year, and students' class selections are typically not
independent of each other, but simplifying in this way gives us a
start on analyzing the problem.  More importantly, these assumptions
are justifiable in important computer science applications of birthday
matching.  For example, birthday matching is a good model for
collisions between items randomly inserted into a hash table.  So we
won't worry about things like spring procreation preferences that make
January birthdays more common, or about twins' preferences to take
classes together (or not).  \begin{editingnotes} or that fact that a
  student can't be selected twice in making up a class list.
\end{editingnotes}

\subsection{Exact Formula for Match Probability}
\iffalse The matching birthday problem fits in here so far as a nice
example illustrating pairwise and mutual independence, but t's
actually not hard to justify the bound~\eqref{bday-approx} without any
pretence of independence.
\fi

There are $d^n$ sequences of $n$ birthdays, and under our assumptions,
these are equally likely.  There are $d (d - 1) (d - 2) \cdots (d - (n
- 1))$ length $n$ sequences of distinct birthdays.  That means the
probability that everyone has a different birthday is:\footnote{The
  fact that $1-x < e^{-x}$ for all $x>0$ follows by truncating the Taylor
  series $e^{-x} = 1 - x + x^2/2!  - x^3/3! + \cdots$.  The
  approximation $e^{-x} \approx 1 - x$ is pretty accurate when $x$ is
  small.}

\begin{align}
\lefteqn{\frac{d (d - 1) (d - 2) \cdots (d - (n - 1))}{d^n}}\notag\\
   & = \frac{d}{d} \cdot \frac{d-1}{d} \cdot \frac{d-2}{d} \cdots \frac{d - (n - 1)}{d}\\
   & = \paren{1 - \frac{0}{d}}
             \paren{1 - \frac{1}{d}}
             \paren{1 - \frac{2}{d}}
             \cdots
             \paren{1 - \frac{n - 1}{d}}\notag\\
   & < e^0 \cdot e^{-1/d} \cdot e^{-2/d} \cdots e^{-(n-1)/d} 
             & \text{(since $1+x < e^x$)}\notag\\
   & = e^{-\paren{\sum_{i=1}^{n-1} i/d}}\notag\\
   & = e^{-\paren{n(n-1)/2d}}\label{bday-approx}.
\end{align}

For $n=95$ and $d = 365$, the value of~\eqref{bday-approx} is less
than $1/200,000$, which means the probability of having some pair of
matching birthdays actually is more than $1 - 1/200,000 > 0.99999$.  So
it would be pretty astonishing if there were no pair of students in
the class with matching birthdays.

For $d \leq n^2/2$, the probability of no match turns out to be
asymptotically equal to the upper bound~\eqref{bday-approx}.  For $d =
n^2/2$ in particular, the probability of no match is asymptotically
equal to $1/e$.  This leads to a rule of thumb which is useful in many
contexts in computer science:

\textbox{
\textboxtitle{The \index{birthday principle} Birthday Principle}

If there are $d$ days in a year and $\sqrt{2d}$ people in a
room, then the probability that two share a birthday is about 
$1 - 1/e \approx 0.632$.
}

For example, the Birthday Principle says that if you have $\sqrt{2
  \cdot 365} \approx 27$ people in a room, then the probability that
two share a birthday is about $0.632$.  The actual probability is
about $0.626$, so the approximation is quite good.

Among other applications, it implies that to use a hash function that
maps $n$ items into a hash table of size $d$, you can expect many
collisions if $n^2$ is more than a small fraction of $d$.  The
Birthday Principle also famously comes into play as the basis of
``birthday attacks'' that crack certain cryptographic systems.


\iffalse %ftl version

\section{The Birthday Paradox}\label{birthday_principle_sec}

Suppose that there are 100 students in a class.  What is the
probability that some birthday is shared by two people?  Comparing 100
students to the 365 possible birthdays, you might guess the
probability lies somewhere around~$1/3$---but you'd be wrong: the
probability that there will be two people in the class with matching
birthdays is actually~$0.999999692\dots$.  In other words, the
probability that all 100 birthdays are different is less than 1
in~3,000,000.

Why is this probability so small?  The answer involves a phenomenon
known as the \term{Birthday Paradox} (or the \term{Birthday
  Principle}), which is surprisingly important in computer science, as
we'll see later.

Before delving into the analysis, we'll need to make some modeling
assumptions:
\begin{itemize}

\item
For each student, all possible birthdays are equally likely.  The idea
underlying this assumption is that each student's birthday is
determined by a random process involving parents, fate, and, um, some
issues that we discussed earlier in the context of graph theory.
The assumption is not completely accurate, however; a disproportionate
number of babies are born in August and September, for example.

\item
Birthdays are mutually independent.  This isn't perfectly accurate
either.  For example, if there are twins in the class, then their
birthdays are surely not independent.

\end{itemize}
We'll stick with these assumptions, despite their limitations.  Part
of the reason is to simplify the analysis.  But the bigger reason is
that our conclusions will apply to many situations in computer science
where twins, leap days, and romantic holidays are not considerations.
After all, whether or not two items collide in a hash table really has
nothing to do with human reproductive preferences.  Also, in pursuit
of generality, let's switch from specific numbers to variables.  Let
$m$~be the number of people in the room, and let $N$~be the number of
days in a year.

We can solve this problem using the standard four-step method.
However, a tree diagram will be of little value because the sample
space is so enormous.  This time we'll have to proceed without the
visual aid!

\paragraph{Step 1: Find the Sample Space}

Let's number the people in the room from 1 to~$m$.  An outcome of the
experiment is a sequence $(b_1, \dots, b_m)$ where $b_i$~is the
birthday of the $i$th person.  The sample space is the set of all such
sequences:
\begin{equation*}
    \sspace = \{\, (b_1, \dots, b_m) \suchthat b_i \in \set{1, \dots
      N} \,\}.
\end{equation*}

\paragraph{Step 2: Define Events of Interest}

Our goal is to determine the probability of the event~$A$ in which
some pair of people have the same birthday.  This event is a little
awkward to study directly, however.  So we'll use a common trick,
which is to analyze the \term{complementary} event~$\setcomp{A}$, in
which all $m$~people have different birthdays:
\begin{equation*}
    \setcomp{A} = \set{\, (b_1, \dots, b_m) \in \sspace
                    \suchthat \text{all $b_i$ are distinct} \,}.
\end{equation*}
If we can compute $\pr{\setcomp{A}}$, then we can compute what
really want, $\pr{A}$, using the identity
\begin{equation*}
    \pr{A} + \pr{\setcomp{A}} = 1.
\end{equation*}

\paragraph{Step 3: Assign Outcome Probabilities}

We need to compute the probability that $m$~people have a particular
combination of birthdays ~$(b_1, \dots, b_m)$.  There are $N$~possible
birthdays and all of them are equally likely for each student.
Therefore, the probability that the $i$th person was born on day~$b_i$
is~$1/N$.  Since we're assuming that birthdays are mutually
independent, we can multiply probabilities.  Therefore, the
probability that the first person was born on day~$b_1$, the second
on~$b_2$, and so forth is~$(1/N)^m$.  This is the probability of every
outcome in the sample space, which means that the sample space is
uniform.  That's good news, because, as we have seen, it means that
the analysis will be simpler.

\paragraph{Step 4: Compute Event Probabilities}

We're interested in the probability of the event~$\setcomp{A}$ in
which everyone has a different birthday:
\begin{equation*}
    \setcomp{A} = \set{\, (b_1, \dots, b_n) \suchthat
                            \text{all $b_i$ are distinct} \,}.
\end{equation*}
This is a gigantic set.  In fact, there are $N$~choices for~$b_i$,
\ $N - 1$ choices for~$b_2$, and so forth.  Therefore, by the
Generalized Product Rule,
\begin{equation*}
\card{\setcomp{A}}
    = \frac{N!}{(N - m)!}
    = N (N - 1) (N - 2) \cdots (N - m + 1).
\end{equation*}
Since the sample space is uniform, we can conclude that
\begin{equation}\label{eqn:15E4}
\pr{\setcomp{A}}
    = \frac{\card{\setcomp{A}}}{N^m} \\
    = \frac{N!}{N^m (N - m)!}.
\end{equation}
We're done!

Or are we?  While correct, it would certainly be nicer to have a
closed-form expression for equation~\eqref{eqn:15E4}.  That means
finding an approximation for $N!$ and~$(N - m)!$.  But this is what we
learned how to do in Section~\ref{sec:closed_products}.  In fact, since
$N$ and~$N - m$ are each at least~100, we know from
Corollary~\ref{cor:9A2} that
\begin{equation*}
    \stirling{N} \quad \text{and} \quad \stirling{N - m}
\end{equation*}
are excellent approximations (accurate to within~.09\%) of~$N!$ and~$(N
- m)!$, respectively.  Plugging these values into
equation~\eqref{eqn:15E4} means that (to within~.2\%)\footnote{If there
are two terms that can be off by~.09\%, then the ratio can be off by
at most a factor of~$(1.0009)^2 < 1.002$.}
\begingroup
\openup2\jot
\begin{align}
\prob{\setcomp{A}}
    &= \frac{ \stirling{N} }{ N^m \stirling*{N - m} } \notag\\
    &= \sqrt{\frac{N}{N - m}}
             \frac{ e^{N \ln(N) - N} }
                  { e^{m \ln(N)} e^{(N - m) \ln(N - m) - (N - m) } }
                  \notag\\
    &= \sqrt{\frac{N}{N - m}}
         e^{ (N - m)\ln(N) - (N - m) \ln(N - m) - m } \notag\\
    &= \sqrt{\frac{N}{N - m}}
         e^{ (N - m)\ln\paren{\frac{N}{N - m}} - m } \notag\\
    &= e^{ \paren{N - m + \frac{1}{2}}\ln\paren{\frac{N}{N - m}} - m }.
        \label{eqn:15E9}
\end{align}
\endgroup
We can now evaluate equation~\eqref{eqn:15E9} for $m = 100$ and $N =
365$ to find that the probability that all 100 birthdays are different
is\footnote{The possible .2\%~error is so small that
  it is lost in the \dots after 3.07.}
\begin{equation*}
    3.07\ldots \cdot 10^{-7}.
\end{equation*}

We can also plug in other values of~$m$ to find the number of people
so that the probability of a matching birthday will be about~$1/2$.
In particular, for $m = 23$ and $N = 365$, equation~\eqref{eqn:15E9}
reveals that the probability that all the birthdays differ is
0.49\dots.  So if you are in a room with 23 other people, the
probability that some pair of people share a birthday will be a little
better than~$1/2$.  It is because 23 seems like such a small number of
people for a match that the phenomenon is called the \term{Birthday
  Paradox}.

\subsection{Applications to Hashing}

Hashing is frequently used in computer science to map large strings of
data into short strings of data.  In a typical scenario, you have a
set of $m$~items and you would like to assign each item to a number
from 1 to~$N$ where no pair of items is assigned to the same number
and $N$~is as small as possible.  For example, the items might be
messages, addresses, or variables.  The numbers might represent
storage locations, devices, indices, or digital signatures.

If two items are assigned to the same number, then a \term{collision}
is said to occur.  Collisions are generally bad.  For example,
collisions can correspond to two variables being stored in the same
place or two messages being assigned the same digital signature.  Just
imagine if you were doing electronic banking and your digital
signature for a \$10~check were the same as your signature for a
\$10~million dollar check.  In fact, finding collisions is a common
technique in breaking cryptographic codes.\footnote{Such techniques
  are often referred to as \term{birthday attacks} because of the
  association of such attacks with the Birthday Paradox.}

In practice, the assignment of a number to an item is done using a
hash function
\begin{equation*}
    h: S \to [1, N],
\end{equation*}
where $S$~is the set of items and $m = \card{S}$.  Typically, the
values of~$h(S)$ are assigned randomly and are assumed to be equally
likely in~$[1, N]$ and mutually independent.

For efficiency purposes, it is generally desirable to make~$N$ as
small as necessary to accommodate the hashing of $m$~items without
collisions.  Ideally, $N$~would be only a little larger than~$m$.
Unfortunately, this is not possible for random hash functions.  To see
why, let's take a closer look at equation~\eqref{eqn:15E9}.

By Theorem~\ref{thm:stirling} and the derivation of
equation~\eqref{eqn:15E9}, we know that the probability that there are
no collisions for a random hash function is
\begin{equation}\label{eqn:16K}
    \sim e^{ \paren{N - m + \frac{1}{2}} \ln\paren{\frac{N}{N - m}} - m }.
\end{equation}
For any~$m$, we now need to find a value of~$N$ for which this
expression is at least~1/2.  That will tell us how big the hash table
needs to be in order to have at least a 50\%~chance of avoiding
collisions.  This means that we need to find a value of~$N$ for which
\begin{equation}\label{eqn:16P}
    \paren{N - m + \frac{1}{2}} \ln\paren{\frac{N}{N - m}} - m 
        \sim
    \ln\paren{\frac{1}{2}}.
\end{equation}

To simplify equation~\eqref{eqn:16P}, we need to get rid of the
$\ln\paren{\frac{N}{N - m}}$~term.  We can do this by using the Taylor
Series expansion for
\begin{equation*}
    \ln(1 - x) = -x - \frac{x^2}{2} - \frac{x^3}{3} - \cdots
\end{equation*}
to find that\footnote{This may not look like a simplification, but
  stick with us here.}
\begin{align*}
\ln\paren{\frac{N}{N - m}}
    &= - \ln \paren{\frac{N - m}{N}} \\
    &= - \ln \paren{1 - \frac{m}{N}} \\
    &= - \paren{ -\frac{m}{N} - \frac{m^2}{2N^2} - \frac{m^3}{3N^3} - \cdots }\\
    &= \frac{m}{N} + \frac{m^2}{2N^2} + \frac{m^3}{3N^3} + \cdots.
\end{align*}
Hence,
\begin{align}
\paren{N - m + \frac{1}{2}} \ln\paren{\frac{N}{N - m}} - m
    &= \paren{N - m + \frac{1}{2}}
        \paren{\frac{m}{N} + \frac{m^2}{2N^2} + \frac{m^3}{3N^3} + \cdots}
        - m \notag\\
    &= \paren{ m + \frac{m^2}{2N} + \frac{m^3}{3N^2} + \cdots }
            \notag\\
    &\phantom{=}\qquad - \paren{ \frac{m^2}{N} + \frac{m^3}{2N^2} +
          \frac{m^4}{3N^3} + \cdots }
            \notag\\
    &\phantom{=}\qquad + \frac{1}{2} \paren{\frac{m}{N} +
          \frac{m^2}{2N^2} + \frac{m^3}{3N^3} + \cdots} -m 
            \notag\\
    &= - \paren{ \frac{m^2}{2N} + \frac{m^3}{6N^2} + \frac{m^4}{12N^3}
            + \cdots} \notag\\
    &\phantom{=}\qquad
        + \frac{1}{2}\paren{ \frac{m}{N} + \frac{m^2}{2N^2} +
          \frac{m^3}{3N^3} + \cdots }.
    \label{eqn:16Q}
\end{align}

If $N$~grows faster than~$m^2$, then the value in
equation~\eqref{eqn:16Q} tends to~0 and equation~\eqref{eqn:16P}
cannot be satisfied.  If $N$~grows more slowly than~$m^2$, then the
value in equation~\eqref{eqn:16Q} diverges to negative infinity, and,
once again, equation~\eqref{eqn:16P} cannot be satisfied.  This suggests
that we should focus on the case where~$N = \Theta(m^2)$, when
equation~\eqref{eqn:16Q} simplifies to
\begin{equation*}
    \sim \frac{-m^2}{2N}
\end{equation*}
and equation~\eqref{eqn:16P} becomes
\begin{equation}\label{eqn:16R}
    \frac{-m^2}{2N} \sim \ln\paren{\frac{1}{2}}.
\end{equation}
Equation~\eqref{eqn:16R} is satisfied when
\begin{equation}\label{eqn:16S}
    N \sim \frac{m^2}{2 \ln(2)}.
\end{equation}

In other words, $N$~needs to grow quadratically with~$m$ in order to
avoid collisions.  This unfortunate fact is known as the
\term{Birthday Principle} and it limits the efficiency of hashing in
practice---either $N$~is quadratic in the number of items being hashed
or you need to be able to deal with collisions.

\fi



\section{Set Theory and Probability}\label{probability_sets_sec}

Let's abstract what we've just done into a general mathematical
definition of sample spaces and probability.

\subsection{Probability Spaces}

\begin{definition}\label{LN12:sampsp}
  A countable \term{sample space}~$\sspace$ is a nonempty countable
  set.\footnote{Yes, sample spaces can be infinite.  If you did not
    read Chapter~\ref{infinite_chap}, don't worry---\emph{countable}
    just means that you can list the elements of the sample space as
    $\omega_0$, $\omega_1$, $\omega_2$, \dots.}  An element $\omega
  \in \sspace$ is called an \term{outcome}.  A subset of $\sspace$ is
  called an \term{event}.
\end{definition}

\begin{definition}\label{LN12:probsp}
 A \term{probability function} on a sample space~$\sspace$ is a total
 function $\operatorname{Pr}: \sspace\to \reals$ such that
\begin{itemize}
\item $\pr{\omega} \geq 0$ for all $\omega \in \sspace$, and
\item $\sum_{\omega \in \sspace} \pr{\omega} = 1$.
\end{itemize}
A sample space together with a probability function is called a
\term{probability space}.
For any event $E \subseteq \sspace$, the \index{probability of an event}
\emph{probability of $E$} is defined to be the sum of the probabilities of
the outcomes in $E$:
\[
    \pr{E} \eqdef \sum_{\omega \in E} \pr{\omega}.
\]
\end{definition}

In the previous examples there were only finitely many possible
outcomes, but we'll quickly come to examples that have a countably
infinite number of outcomes.

The study of probability is closely tied to set theory
because any set can be a sample space and any subset can be an event.
General probability theory deals with uncountable sets like the set of
real numbers, but we won't need these, and sticking to countable
sets lets us define the probability of events using sums instead of
integrals.  It also lets us avoid some distracting technical problems
in set theory like the Banach-Tarski ``paradox'' mentioned in
Chapter~\ref{infinite_chap}.

\subsection{Probability Rules from Set Theory}\label{sec:union_bound}

Most of the rules and identities that we have developed for finite
sets extend very naturally to probability.  

\iffalse We'll cover several examples in this section, but first let's
review some definitions that should already be familiar.\fi

An immediate consequence of the definition of event probability is
that for \emph{disjoint} events $E$ and~$F$,
\[
    \pr{E \union F} = \pr{E} + \pr{F}.
\]
This generalizes to a countable number of events:
\begin{rul}[\idx{Sum Rule}]
  If $E_0,E_1,\dots,E_n,\dots$ are pairwise disjoint events, then
\[
    \Prob{\lgunion_{n\in\naturals}E_n} = \sum_{n\in\naturals} \pr{E_n}.
\]
\end{rul}

The Sum Rule lets us analyze a complicated event by breaking it down
into simpler cases.  For example, if the probability that a randomly
chosen MIT student is native to the United States is 60\%, to Canada
is 5\%, and to Mexico is 5\%, then the probability that a random MIT
student is native to one of these three countries
\begin{editingnotes} North America\end{editingnotes}
is 70\%.

Another consequence of the Sum Rule is that $\pr{A} + \pr{\setcomp{A}} =
1$, which follows because $\pr{\sspace}=1$ and $\sspace$ is the union
of the disjoint sets $A$ and $\setcomp{A}$.  This equation often comes up
in the form:
\begin{equation}%\label{}
\pr{\setcomp{A}}  =  1 - \pr{A}. \tag{\idx{Complement Rule}}
\end{equation}
Sometimes the easiest way to compute the probability of an event is to compute
the probability of its complement and then apply this formula.

Some further basic facts about probability parallel facts about
cardinalities of finite sets.  In particular:
\begin{center}
\begin{tabular*}{\textwidth}{l@{\extracolsep{\fill}}lr@{}}
\hskip\parindent
&$\pr{B-A} = \pr{B} - \pr{A \intersect B}$,
    & (Difference Rule)\\
&$\pr{A \union B} = \pr{A} + \pr{B} - \pr{A \intersect B}$,
    & (Inclusion-Exclusion)\\
&$\pr{A \union B} \le \pr{A} + \pr{B}$,
    & (Boole's Inequality) \\
&If $A \subseteq B$, then $\prob{A} \leq \prob{B}$.
    & (Monotonicity Rule)
\end{tabular*}
\end{center}
The \idx{Difference Rule} follows from the Sum Rule because $B$ is the
union of the disjoint sets $B-A$ and $A \intersect B$.
\index{inclusion-exclusion for probabilities} Inclusion-Exclusion then
follows from the Sum and Difference Rules, because $A \union B$ is the
union of the disjoint sets $A$ and $B-A$.  \idx{Boole's inequality} is an
immediate consequence of Inclusion-Exclusion since probabilities are
nonnegative.  Monotonicity follows from the definition of event
probability and the fact that outcome probabilities are nonnegative.

The two-event Inclusion-Exclusion equation above generalizes to any
finite set of events in the same way as the corresponding
Inclusion-Exclusion rule for $n$ sets.  Boole's inequality also
generalizes to both finite and countably infinite sets of events:
\begin{rul}[\idx{Union Bound}]
\begin{equation}
    \pr{E_1 \union \cdots \union E_n \union \cdots} \leq \pr{E_1} + \cdots + \pr{E_n} + \cdots.
\end{equation}
\end{rul}
The Union Bound is useful in many calculations.  For example, suppose
that $E_i$ is the event that the $i$-th critical component among $n$
components in a spacecraft fails.  Then $E_1 \union \cdots \union E_n$
is the event that \emph{some} critical component fails.  If $\sum_{i =
  1}^n \prob{E_i}$ is small, then the Union Bound can provide a
reassuringly small upper bound on this overall probability of critical
failure.

\subsection{Uniform Probability Spaces}

\begin{definition}\label{def:uniform_pspace}
A finite probability space, $\sspace$, is said to be \term{uniform} if
$\pr{\omega}$ is the same for every outcome $\omega \in \sspace$.
\end{definition}

As we saw in the strange dice problem, uniform sample spaces are
particularly easy to work with.  That's because for any event~$E
\subseteq \sspace$,
\begin{equation}\label{eqn:14G2}
    \pr{E} = \frac{\card{E}}{\card{\sspace}}.
\end{equation}
This means that once we know the cardinality of $E$ and~$\sspace$, we
can immediately obtain~$\pr{E}$.  That's great news because we
developed lots of tools for computing the cardinality of a set in
Part~\ref{part:counting}.

For example, suppose that you select five cards at random from a
standard deck of 52~cards.  What is the probability of having a full
house?  Normally, this question would take some effort to answer.  But
from the analysis in Section~\ref{sec:counting_full_houses}, we know
that
\begin{equation*}
    \card{\sspace} = \binom{52}{5}
\end{equation*}
and
\begin{equation*}
    \card{E} = 13 \cdot \binom{4}{3} \cdot 12 \cdot \binom{4}{2}
\end{equation*}
where $E$ is the event that we have a full house.  Since every
five-card hand is equally likely, we can apply
equation~\eqref{eqn:14G2} to find that
\begin{align*}
\pr{E}  &= \frac{13 \cdot 12 \cdot \binom{4}{3} \cdot \binom{4}{2}}
                {\binom{52}{5}} \\
        &= \frac{13 \cdot 12 \cdot 4 \cdot 6 \cdot 5 \cdot 4 \cdot 3 \cdot 2}
                {52 \cdot 51 \cdot 50 \cdot 49 \cdot 48} = \frac{18}{12495} \\
        &\approx \frac{1}{694}.
\end{align*}

\subsection{Infinite Probability Spaces}

\iffalse
General probability theory deals with uncountable sets like~$\reals$,
but in computer science, it is usually sufficient to restrict our
attention to countable probability spaces.  It's also a lot
easier---infinite sample spaces are hard enough to work with without
having to deal with uncountable spaces.
\fi

Infinite probability spaces are fairly common.  For example, two
players take turns flipping a fair coin.  Whoever flips heads first is
declared the winner.  What is the probability that the first player
wins?  A tree diagram for this problem is shown in
Figure~\ref{fig:14A15}.

\begin{figure}

\graphic{infinite-tree1}

\caption{The tree diagram for the game where players take turns
  flipping a fair coin.  The first player to flip heads wins.}

\label{fig:14A15}

\end{figure}

The event that the first player wins contains an infinite number of
outcomes, but we can still sum their probabilities:
\begin{align*}
\pr{\text{first player wins}}
    & = \frac{1}{2} + \frac{1}{8} + \frac{1}{32} + \frac{1}{128} + \cdots \\
    & = \frac{1}{2} \sum_{n=0}^\infty \paren{\frac{1}{4}}^n \\
    & = \frac{1}{2}\paren{\frac{1}{1-1/4}} = \frac{2}{3}.
\end{align*}

Similarly, we can compute the probability that the second player wins:
\begin{align*}
\pr{\text{second player wins}}
     = \frac{1}{4} + \frac{1}{16} + \frac{1}{64} + \frac{1}{256}
                      + \cdots % \\
     = \frac{1}{3}.
\end{align*}

In this case, the sample space is the infinite set
\[
    \sspace \eqdef \set{\, \tails^n\heads \suchthat n \in \naturals \,},
\]
where $\tails^n$ stands for a length $n$ string of $\tails$'s.
The probability function is
\[
\pr{\tails^n\heads} \eqdef \frac{1}{2^{n+1}}.
\]
To verify that this is a probability space, we just have to check that
all the probabilities are nonnegative and that they sum to~1.  The
given probabilities are all nonnegative, and applying the formula for
the sum of a geometric series, we find that
\begin{equation*}
\sum_{n \in \naturals} \pr{\tails^n\heads}
    = \sum_{n \in \naturals} \frac{1}{2^{n+1}} \\
    = 1.
\end{equation*}

Notice that this model does not have an outcome corresponding to the
possibility that both players keep flipping tails forever.  (In the
diagram, flipping forever corresponds to following the infinite path
in the tree without ever reaching a leaf/outcome.)  If leaving this
possibility out of the model bothers you, you're welcome to fix it by
adding another outcome, $\omega_{\text{forever}}$, to indicate that
that's what happened.  Of course since the probabililities of the
other outcomes already sum to 1, you have to define the probability of
$\omega_{\text{forever}}$ to be 0.  Now outcomes with probability zero
will have no impact on our calculations, so there's no harm in adding
it in if it makes you happier.  On the other hand, in countable
probability spaces it isn't necessary to have outcomes with
probability zero, and we will generally ignore them.

%% Set Theory and Probability Problems %%%%%%%%%%%%%%%%%%%%%%%%%%%%%%%%%%%%%%%%
\begin{problems}
\classproblems
\pinput{CP_system_component_failure}
\pinput{CP_proving_probability_rules}

\homeworkproblems
\pinput{PS_union_bound_infinite}
\pinput{PS_probabilistic_proof}

\homeworkproblems
\pinput{PS_prob_space_symmetry}

\end{problems}

\begin{editingnotes}
Insert Monty Hall false conditioning examples from slides.
\end{editingnotes}

\section{References}
\cite{Feller68v1},
\cite{GrinsteadS},
\cite{LawlerC1999},
\cite{Malkiel2003},
\cite{Mitzenmacher2005},
\cite{MotwaniR95}
\cite{Ross2001},
\cite{Ross2002},
\cite{Williams2001}
