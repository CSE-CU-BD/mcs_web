\chapter{First-Order Logic}

\newcommand{\solves}{\text{Solves}}
\newcommand{\probs}{\text{Probs}}
\newcommand{\even}{\text{Evens}}
\newcommand{\primes}{\text{Primes}}

\section{Quantifiers}

There are a couple of assertions commonly made about a predicate: that it
is \textit{sometimes} true and that it is \textit{always} true.  For
example, the predicate
%
\[
\text{``$x^2 \geq 0$''}
\]
%
is always true when $x$ is a real number.  On the other hand, the
predicate
%
\[
\text{``$5x^2 - 7 = 0$''}
\]
%
is only sometimes true; specifically, when $x = \pm \sqrt{7/5}$.

There are several ways to express the notions of ``always true'' and
``sometimes true'' in English.  The table below gives some general
formats on the left and specific examples using those formats on the
right.  You can expect to see such phrases hundreds of times in
mathematical writing!
%
\begin{center}
\begin{tabular}{ll}
\multicolumn{2}{c}{\textbf{Always True}} \\[1ex]
For all $n$, $P(n)$ is true. & For all $x$, $x^2 \geq 0$. \\
$P(n)$ is true for every $n$. & $x^2 \geq 0$ for every $x$. \\[2ex]
\multicolumn{2}{c}{\textbf{Sometimes True}} \\[1ex]
There exists an $n$ such that $P(n)$ is true. & There exists an $x$ such that $5x^2 - 7 = 0$.\\
$P(n)$ is true for some $n$. & $5x^2 - 7 = 0$ for some $x$.\\
$P(n)$ is true for at least one $n$. & $5x^2-7=0$ for at least one $x$.
\end{tabular}
\end{center}

All these sentences quantify how often the predicate is true.
Specifically, an assertion that a predicate is always true is called a
\term{universal} quantification, and an assertion that a predicate is
sometimes true is an \term{existential} quantification.  Sometimes the
English sentences are unclear with respect to quantification:
%
\begin{center}
  ``If you can solve any problem we come up with, then you get an \emph{A}
  for the course.''
\end{center}
%
The phrase ``you can solve any problem we can come up with'' could
reasonably be interpreted as either a universal or existential
quantification:
%
\begin{quote}
``you can solve \textit{every} problem we come up with,''
\end{quote}
or maybe
\begin{quote}
``you can solve \textit{at least one} problem we come up with.''
\end{quote}
%
In any case, notice that this quantified phrase appears inside a
larger if-then statement.  This is quite normal; quantified statements
are themselves propositions and can be combined with and, or, implies,
etc., just like any other proposition.

\subsection{More Cryptic Notation}

There are symbols to represent universal and existential
quantification, just as there are symbols for ``and'' ($\wedge$),
``implies'' ($\implies$), and so forth.  In particular, to say that a
predicate, $P$, is true for all values of $x$ in some set, $D$, one
writes:
%
\[
\forall x \in D.\; P(x)
\]
%
The symbol $\forall$ is read ``for all'', so this whole expression is
read ``for all $x$ in $D$, $P(x)$ is true''.  To say that a predicate
$P(x)$ is true for at least one value of $x$ in $D$, one writes:
%
\[
\exists x \in D.\; P(x)
\]
%
The backward-E is read ``there exists''.  So this expression would be
read, ``There exists an $x$ in $D$ such that $P(x)$ is true.''  The
symbols $\forall$ and $\exists$ are always followed by a variable
---usually with an indication of the set the variable ranges over ---and
then a predicate, as in the two examples above.

As an example, let $\probs$ be the set of problems we come up with,
$\solves(x)$ be the predicate ``You can solve problem $x$'', and $G$ be
the proposition, ``You get an \emph{A} for the course.''  Then the two
different interpretations of
%
\begin{quote}
``If you can solve any problem we come up with, then you get an \emph{A} for the course.''
\end{quote}
%
can be written as follows:
%
\[
(\forall x \in \probs.\; \solves(x)) \implies G,
\]
or maybe
\[
(\exists x \in \probs.\; \solves(x)) \implies G.
\]

\subsection{Mixing Quantifiers}

Many mathematical statements involve several quantifiers.  For
example, Goldbach's Conjecture states:
%
\begin{center}
``Every even integer greater than 2 is the sum of two primes.''
\end{center}
%
Let's write this more verbosely to make the use of quantification
clearer:
%
\begin{quote}
For every even integer $n$ greater than 2,
there exist primes $p$ and $q$ such that $n = p + q$.
\end{quote}
%
Let $\even$ be the set of even integers greater than 2, and let $\primes$ be the
set of primes.  Then we can write Goldbach's Conjecture in logic
notation as follows:
%
\[
\underbrace{\forall n \in \even}_{\substack
    {\text{for every even} \\
     \text{integer $n > 2$}}}
\
\underbrace{\exists p \in \primes\ \exists q \in \primes.}_{\substack
    {\text{there exist primes} \\
     \text{$p$ and $q$ such that}}}
\ n = p + q.
\]

\subsection{Order of Quantifiers}

Swapping the order of different kinds of quantifiers (existential or
universal) usually changes the meaning of a proposition.  For example,
let's return to one of our initial, confusing statements:
\begin{center}
``Every American has a dream.''
\end{center}

This sentence is ambiguous because the order of quantifiers is
unclear.  Let $A$ be the set of Americans, let $D$ be the set of
dreams, and define the predicate $H(a, d)$ to be ``American $a$ has
dream $d$.''.  Now the sentence could mean there is a single dream
that every American shares:
\[
\exists\, d \in D\; \forall a \in A.\; H(a, d)
\]
For example, it might be that every American shares the dream of owning
their own home.

Or it could mean that every American has a personal dream:
\[
\forall a \in A\; \exists\, d \in D.\; H(a, d)
\]
For example, some Americans may dream of a peaceful retirement, while
others dream of continuing practicing their profession as long as they
live, and still others may dream of being so rich they needn't think at
all about work.

Swapping quantifiers in Goldbach's Conjecture creates a patently false
statement that every even number $\geq 2$ is the sum of \emph{the same}
two primes:
\[
\underbrace{\exists\, p \in \primes\ \exists\, q \in \primes}_{\substack
    {\text{there exist primes} \\
     \text{$p$ and $q$ such that}}}
\
\underbrace{\forall n \in \even.}_{\substack
    {\text{for every even} \\
     \text{integer $n > 2$}}}
\ n = p + q.
\]

\subsubsection{Variables over One Domain}
When all the variables in a formula are understood to take values from the
same nonempty set, $D$, it's conventional to omit mention of $D$.  For
example, instead of $\forall x \in D\; \exists y \in D.\; Q(x,y)$ we'd write
$\forall x \exists y.\; Q(x,y)$.  The unnamed nonempty set that $x$ and
$y$ range over is called the \term{domain} of the formula.

It's easy to arrange for all the variables to range over one domain.  For
example, Goldbach's Conjecture could be expressed with all variables
ranging over the domain $\naturals$ as
\[
\forall n.\; n \in \even \implies (\exists\, p \exists\, q.\; p \in \primes \land
q \in \primes \land n = p + q).
\]

\subsection{Negating Quantifiers}

There is a simple relationship between the two kinds of quantifiers.  The
following two sentences mean the same thing:
%
\begin{quote}

It is not the case that everyone likes to snowboard.

There exists someone who does not like to snowboard.

\end{quote}
%
In terms of logic notation, this follows from a general property of
predicate formulas:
%
\[
\QNOT \forall x.\; P(x)
\hspace{0.1in} \text{is equivalent to} \hspace{0.1in}
\exists x.\; \QNOT P(x).
\]
%
Similarly, these sentences mean the same thing:
%
\begin{quote}
There does not exist anyone who likes skiing over magma.

Everyone dislikes skiing over magma.
\end{quote}
%
We can express the equivalence in logic notation this way:
%
\begin{equation}\label{nE}
(\QNOT \exists x.\; P(x))  \QIFF  \forall x.\; \QNOT P(x).
\end{equation}
%
The general principle is that \textit{moving a ``not'' across a
quantifier changes the kind of quantifier.}

\iffalse Logicians have worked very hard to define strict rules for the
use of logic notation so that ideas can be expressed with absolute rigor.
It's all quite charming and clever.  However, the sad irony is that
applied mathematicans usually use their beloved notation as a crude
shorthand, breaking the rules and abusing the notation willy-nilly ---sort
of like pounding nails with fine china.  \fi

\subsection{Validity}

A propositional formula is called \term{valid} when it evaluates to \true\
no matter what truth values are assigned to the individual propositional
variables.  For example, the propositional version of the Distributive Law
is that $P \conj (Q \disj R)$ is equivalent to $(P \conj Q) \disj (P \conj
R)$.  This is the same as saying that
\[
[P \conj (Q \disj R)] \iff [(P \conj Q) \disj (P \conj R)]
\]
is valid.

The same idea extends to predicate formulas, but to be valid, a
formula now must evaluate to true no matter what values its variables
may take over any unspecified domain, and no matter what
interpretation a predicate variable may be given.  For example, we
already observed that the rule for negating a quantifier is captured
by the valid assertion~\eqref{nE}.

Another useful example of a valid assertion is
\[
\exists x \forall y.\; P(x,y) \implies \forall y \exists x.\; P(x,y).
\]
We could prove this as follows:
\begin{proof}
Let $D$ be the domain for the variables and $P_0$ be some
binary predicate\footnote{That is, a predicate that depends on two variables.}
on $D$.  We need to show that if $\exists x \in D\; \forall y \in D.\;
P_0(x,y)$ holds under this interpretation, then so does $\forall y \in D\;
\exists x \in D.\; P_0(x,y)$.

So suppose $\exists x \in D\; \forall y \in D.\; P_0(x,y)$.  Then some
element $x_0 \in D$ has the property that $P_0(x_0, y)$ is true for all $y
\in D$.  So for every $y \in D$, there is some $x \in D$, namely $x_0$,
such that $P_0(x,y)$ is true.  That is, $\forall y \in D\exists x \in D.\;
P_0(x,y)$ holds; that is, $\forall y\; \exists x.\; P(x,y)$ holds under this
interpretation, as required.
\end{proof}

On the other hand,
\[
\forall y \exists x.\; P(x,y) \implies \exists x \forall y.\; P(x,y).
\]
is \emph{not} valid.  We can prove this simply by describing an
interpretation where the hypothesis, $\forall y \exists x.\; P(x,y)$, is
true but the conclusion, $\exists x \forall y.\; P(x,y)$, is not true.
For example, let the domain be the integers and $P(x,y)$ mean $x > y$.
Then the hypothesis would be true because, given a value, $n$, for $y$ we
could choose the value of $x$ to be $n+1$, for example.  But under this
interpretation the conclusion asserts that there is an integer that is
bigger than all integers, which is certainly false.  An interpetation like
this which falsifies an assertion is called a \emph{counter model} to the
assertion.

\section{The Logic of Sets}

\subsection{Russell's Paradox}

Reasoning naively about sets quickly leads to the following
contradiction ---known as Russell's Paradox:

\textbox{
\begin{quote}
Let $S$ be a variable ranging over all sets, and define
\[W \eqdef \set{S \suchthat S \not\in S}.\]
So by definition,
\[S \in W  \mbox{  iff  } S \not\in S,\]
for every set $S$.  In particular, we can let $S$ be $W$, and obtain
the contradictory result that
\[W \in W  \mbox{  iff  } W \not\in W.\]
\end{quote}}

This paradox revealed a fatal flaw in Frege's initial effort to axiomatize
set theory.  This was an astonishing blow to efforts to provide an
axiomatic foundation for Mathematics.

But a way out was clear at the time: \emph{we cannot assume that $W$ is a
set}.  So the step in the proof where we let $S$ be $W$ is invalid,
because $S$ ranges over sets, and $W$ is not a set.

But denying that $W$ is a set means we must reject the axiom that every
mathematically well-defined collection of elements is actually a set.

The problem faced by Logicians was how to axiomatize rules determining
which well-defined collections are sets.  Russell and his colleague
Whitehead immediately went to work on this problem and spent a dozen years
developing a huge new axiom system in an even huger monograph called
\emph{Principia Mathematica}.


\subsection{The ZFC Axioms for Sets}

It's generally agreed that, using some simple logical deduction rules,
essentially all of Mathematics can be derived from some axioms about sets
called the Axioms of Zermelo-Frankel Set Theory with Choice (ZFC).

We're \textit{not} going to be working with these axioms in this course,
but we thought you might like to see them --and while you're at it, get
some practice reading quantified formulas:
%

\begin{description}

\item[Extensionality.] Two sets are equal if they have the same members.
In formal logical notation, this would be stated as:
\[
(\forall z.\; (z \in x \QIFF z \in y)) \QIMPLIES x = y.
\]


\item[Pairing.] For any two sets $x$ and $y$, there is a set,
     $\set{x,y}$, with $x$ and $y$ as its only elements.

\item[Union.] The union of a collection of sets is also a set.
In formal logical notation, this would be stated as:
\[
\exists u \forall x.\; (\exists y.\; x \in y \qand y \in z) \QIFF x \in u.
\]


\item[Infinity.]  There is an infinite set.  Specifically, there is a
  nonempty set, $x$, such that for any set $y \in x$, the set $\set{y}$ is
  also a member of $x$

\item[Subset.] Given any set, $x$, and any proposition $P(y)$, there is a
  set containing precisely those elements $y \in x$ for which $P(y)$ holds.

\item[Power Set.]  All the subsets of a set form another set.

\item[Replacement.]  The image of a set under a function is a set.
In formal logical notation, this would be stated as:
%
\[
\forall w \exists y \forall z (\phi(w,z) \QIMPLIES z = y)
        \QIMPLIES \exists y \forall z (
            z \in y \iff \exists w (w \in x \QAND \phi(w,z)))
\]


\item[Foundation.] 
There cannot be an infinite sequence
\[
\cdots \in x_n \in \cdots \in x_1 \in x_0
\]
of sets each of which is a member of the previous one.
(This most technical of the axioms aims to capture the idea that sets are
built up successively from simpler sets.  In particular, this axiom
prevents a set from being a member of itself.)
\iffalse  %USE FOR WELL-FOUNDED POSETS
For every non-empty set, $x$, there is a set $y \in x$
  such that $x$ and $y$ have no elements in common.  
\fi

\item[Choice.]  We can choose one element from each set in a collection of
  nonempty sets.  More precisely, if $c$ is a set, and every element
  of $c$ is itself a set that is nonempty, then there is a ``choice''
  function, $g$, such that $g(y) \in y$ for every $y \in c$.

\begin{align*}
\exists y \forall z \forall w & ( (z \in w \QAND w \in x) \QIMPLIES\\
                              &\quad \exists v \exists u (\exists t &
                         &&((u \in w \QAND w \in t) \QAND & (u \in t \QAND t \in y))\\
&&& \QIFF u = v))
\end{align*}

\end{description}


\subsection{Avoiding Russell's Paradox}

The modern ZFC axioms for set theory are much simpler than the
Russell/Whitehead system and are close to Frege's original axioms.  They
specify that sets must be built up from ``simpler'' sets in certain
standard ways.  In particular, no set is ever a member of itself.  So the
modern resolution of Russell's paradox goes as follows: since $S \not \in
S$ for all sets $S$, it follows that $W$, defined above, contains every
set.  So $W$ can't be a set or it would be a member of itself.

\subsection{Power sets are strictly bigger}

It turns out that the ideas behind Russell's Paradox, which caused so much
trouble for the early efforts to formulate Set Theory, lead to a correct
and astonishing fact about infinite sets: they are \emph{not all the same
  size}.

In particular,
\begin{theorem}\label{powbig}
For any set, $A$, the power set, $\power(A)$, is strictly bigger than $A$.
\end{theorem}
\begin{proof}
  First of all, $\power(A)$ is as big as $A$: for example, the partial
  function $f:\power(A) \to A$, where $f(\set{a}) \eqdef a$ for $a \in A$
  and $f$ is only defined on one-element sets, is a surjection.

  To show that $\power(A)$ is strictly bigger than $A$, we have to show
  that if $g$ is a function from $A$ to $\power(A)$, then $g$ is not a
  surjection.  So, mimicking Russell's Paradox, define
  \[
  A_g \eqdef \set{a \in A \suchthat a \notin g(a)}.
  \]
  Now $A_g$ is a well-defined subset of $A$, which means it is a member of
  $\power(A)$.  But $A_g$ can't be in the range of $g$, because if it
  were, we would have
\[
A_g = g(a_0)
\]
for some $a_0 \in A$, so by definition of $A_g$,
\[
a \in g(a_0) \qiff a \in A_g \qiff a \notin g(a)
\]
for all $a \in A$.  Now letting $a = a_0$ yields the contradiction
\[
a_0 \in g(a_0) \qiff a_0 \notin g(a_0).
\]
So $g$ is not a surjection, because there is an element in the power set
of $A$, namely the set $A_g$, that is not in the range of $g$.
\end{proof}

\subsubsection{Larger Infinities}

There are lots of different sizes of infinite sets.  For example, starting
with the infinite set, $\naturals$, of nonnegative integers, we can build
the infinite sequence of sets
\[
\naturals,\ \power(\naturals),\ \power(\power(\naturals)),\
\power(\power(\power(\naturals))),\ \dots.
\]
By Theorem~\ref{powbig}, each of these sets is strictly bigger than all
the preceding ones.  But that's not all: the union of all the sets in the
sequence is strictly bigger than each set in the sequence.  In this way
you can keep going, building still bigger infinities.

\begin{notesproblem}{}
  Prove that the union of this sequence of sets is strictly bigger than
  each of the sets in the sequence.
\end{notesproblem}

So there is an endless variety of different size infinities.

\subsection{Does All This Really Work?}

So this is where mainstream mathematics stands today: there is a handful
of ZFC axioms from which virtually everything else in mathematics can be
logically derived.  This sounds like a rosy situation, but there are
several dark clouds, suggesting that the essence of truth in mathematics
is not completely resolved.


%
\begin{itemize}

\item The ZFC axioms weren't etched in stone by God.  Instead, they
were mostly made up by some guy named Zermelo.  Probably some days he
forgot his house keys.

\item While the ZFC axioms largely generate the mathematics everyone wants
  ---where $3 + 3 = 6$ and other basic facts are true ---they also imply
  some disturbing conclusions.  For example, the Banach-Tarski Theorem
  says that, as a consequence of the Axiom of Choice, a solid ball can be
  divided into six pieces and then the pieces can be rigidly rearranged to
  give \textit{two} solid balls, each the same size as the original!

\item No one knows whether the ZFC axioms are logically consistent; there
  is some possibility that one person might prove a proposition $P$ and
  another might prove the proposition $\QNOT P$.  Then Math would be
  broken.  This sounds like a crazy situation, but it has happened before.
  At the beginning of the 20th century, the logician Gotlob Frege made an
  initial attempt to axiomatize set theory using a few very plausible
  axioms.  Several mathematicians ---most famously Bertrand
  Russell\footnote{Bertrand Russell was a Mathematician/Logician at
    Cambridge University at the turn of the Twentieth Century.  He
    reported that when he felt too old to do Mathematics, he began to
    study and write about Philosophy, and when he was no longer smart
    enough to do Philosophy, he began writing about Politics.  He was
    jailed as a conscientious objector during World War I.  For his
    extensive philosophical and political writing, he won a Nobel Prize
    for Literature.}  ---discovered that Frege's axioms actually
  \textit{were} self-contradictory!

\item In the 1930's, G\"{o}del proved that, assuming that an axiom system
  like ZFC is consistent, then the system is not \emph{complete}: that is,
  there exist propositions that are true, but do not logically follow from
  the axioms.  As a matter of fact, the proposition that the axiom system
  is consistent (which is not too hard to express as a logical formula) is
  an example of a true proposition that cannot be proved.

  There seems to be no way out of this disturbing situation; simply adding
  more axioms does not eliminate the problem.

\item The nineteenth century Mathematician, Georg Cantor, who first
  developed the theory of infinite sizes (because he thought he needed it
  in his study of Fourier series) asked whether there is a set whose size
  is strictly between $\naturals$ and $\power(\naturals)$; he guessed not:

\textbf{Cantor's Continuum Hypothesis}: There is no set, $A$, such that
$\power(\naturals)$ is strictly bigger than $A$ and $A$ is strictly bigger
than $\naturals$.

The Continuum Hypothesis remains an open problem a century later.  Its
difficulty arises from one of the deepest results in modern Set Theory
---discovered in part by G\"odel in the 1930's and Paul Cohen in the
1960's ---namely, the ZFC axioms are not sufficient to settle the
Continuum Hypothesis: there are two collections of sets, each obeying the
laws of ZFC, and in one collection the Continuum Hypothesis is true, and
in the other it is false.  So settling the Continuum Hypothesis requires a
new understanding of what Sets should be to arrive at persuasive new
axioms that extend ZFC and are strong enough to determine the truth of the
Continuum Hypothesis one way or the other.

\end{itemize}

\subsection{Infinities in Computer Science}

If the romance of different infinities and continuum hypotheses doesn't
appeal to you, not knowing about them is not going to lower your
professional abilities as a Computer Scientist.  In fact, at the end of
the 19th century, the general Mathematical community also doubted the
relevance of what they called ``Cantor's paradise'' of unfamiliar sets of
arbitrary infinite size.

In fact, these abstract issues about infinite sets rarely come up in
mainstream Mathematics, and they don't come up at all in Computer Science,
where the focus is generally on ``countable,'' and often just finite,
sets.  In practice, only Logicians and Set Theorists have to worry about
collections that are too big to be sets.

But the proof that power sets are bigger gives the simplest form of what
is known as a ``diagonal argument.''  Diagonal arguments are used to prove
many fundamental results about the limitations of computation, such as the
undecidability of the Halting Problem for programs \iffalse
%INSERT BACK IF PS2 USES THIS PROBLEM

(a variation of which is given in
\href{http://courses.csail.mit.edu/6.042/spring09/ps2.pdf#unrecognizable.set}
{Pset 2, prob 5})
\fi
and the inherent, unavoidable, inefficiency (exponential
time or worse) of procedures for other computational problems.  So
Computer Scientists do need to study diagonal arguments in order to
understand the logical limits of computation.

\newpage
\section{Glossary of Symbols}
\begin{center}
\begin{tabular}{ll}
symbol &  meaning\\
\hline
$\eqdef$ & is defined to be\\
$\land$ & and\\
$\lor$ & or\\
$\implies$ & implies\\
$\neg$    & not\\
$\neg{P}$ & not $P$\\
$\bar{P}$ & not $P$\\
$\iff$    & iff\\
$\iff$    & equivalent\\
$\oplus$   & xor\\
$\exists$ & exists\\
$\forall$ & for all\\
$\in$   &  is a member of\\

\iffalse

$\subseteq$ & is a subset of\\
$\subset$ & is a proper subset of\\
$\union$  & set union\\
$\intersect$ & set intersection\\
$\bar{A}$ & complement of a set, $A$\\
$\power(A)$ & powerset of a set, $A$\\
$\emptyset$ & the empty set, $\set{}$\\
$\naturals$ & nonnegative integers \\
$\integers$ & integers\\
$\integers^+$ & positive integers\\
$\integers^-$ & negative integers\\
$\rationals$ & rational numbers\\
$\reals$ & real numbers\\
$\complexes$ & complex numbers\\
$\emptystring$ & the empty string/list
\fi

\end{tabular}
\end{center}

\endinput
