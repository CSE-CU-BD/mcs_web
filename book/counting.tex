\chapter{Cardinality Rules}\label{counting_chap}

\newcommand{\Jay}{Bob}
\newcommand{\Jer}{Ted}

\section{Counting One Thing by Counting Another}\label{bijection_counting_sec}

How do you count the number of people in a crowded room?  You could
count heads, since for each person there is exactly one head.
Alternatively, you could count ears and divide by two.  Of course, you
might have to adjust the calculation if someone lost an ear in a
pirate raid or someone was born with three ears.  The point here is
that you can often \emph{count one thing by counting another}, though
some fudge factors may be required.  This is a central theme of
counting, from the easiest problems to the hardest.  In fact, we've
already seen this technique used in Theorem~\ref{powset_fincard} where
the number of subsets of an $n$-element set was proved to be the same as
the number of length-$n$ bit-strings by describing a bijection
between the subsets and the bit-strings.

The most direct way to count one thing by counting another is to find
a bijection between them, since if there is a bijection between two
sets, then the sets have the same size.  This important fact is
commonly known as the \term{Bijection Rule}.  We've already seen it as
the \idx{Mapping Rules} bijective case~\eqref{bij_same_fincard}.

\subsection{The Bijection Rule}

\iffalse
\begin{rul}[Bijection Rule]\label{rul:bijection}
If there is a bijection $f: A \to B$ between $A$ and~$B$, then
$\card{A} = \card{B}$.
\end{rul}
\fi

The Bijection Rule acts as a magnifier of counting ability; if you
figure out the size of one set, then you can immediately determine the
sizes of many other sets via bijections.  For example, let's look at
the two sets mentioned at the beginning of Part~\ref{part:counting}:
%
\begin{align*}
A & = \text{all ways to select a dozen doughnuts when five varieties are available} \\
B & = \text{all 16-bit sequences with exactly 4 ones}
\end{align*}

An example of an element of set $A$ is:
%
\[
\underbrace{0\ 0}_{\text{chocolate}} \quad
\underbrace{}_{\text{lemon-filled}} \quad
\underbrace{0\ 0\ 0\ 0\ 0\ 0}_{\text{sugar}} \quad
\underbrace{0\ 0}_{\text{glazed}} \quad
\underbrace{0\ 0}_{\text{plain}}
\]
Here, we've depicted each doughnut with a $0$ and left a gap between
the different varieties.  Thus, the selection above contains two
chocolate doughnuts, no lemon-filled, six sugar, two glazed, and two
plain.  Now let's put a 1 into each of the four gaps:
\[
\underbrace{0\ 0}_{\text{chocolate}} \quad 1 \quad
\underbrace{}_{\text{lemon-filled}} \quad 1 \quad
\underbrace{0\ 0\ 0\ 0\ 0\ 0}_{\text{sugar}} \quad 1 \quad
\underbrace{0\ 0}_{\text{glazed}} \quad 1 \quad
\underbrace{0\ 0}_{\text{plain}}
\]
and close up the gaps:
\[
0011000000100100 \,.
\]
We've just formed a 16-bit number with exactly 4 ones ---an element of
$B$!

This example suggests a bijection from set $A$ to set $B$: map a dozen
doughnuts consisting of:
%
\[
\text{$c$ chocolate, $l$ lemon-filled, $s$ sugar, $g$ glazed, and $p$ plain}
\]
%
to the sequence:
%
\[
\underbrace{\ 0 \ldots 0\ }_{\text{$c$}} \quad 1 \quad
\underbrace{\ 0 \ldots 0\ }_{\text{$l$}} \quad 1 \quad
\underbrace{\ 0 \ldots 0\ }_{\text{$s$}} \quad 1 \quad
\underbrace{\ 0 \ldots 0\ }_{\text{$g$}} \quad 1 \quad
\underbrace{\ 0 \ldots 0\ }_{\text{$p$}}
\]

The resulting sequence always has 16 bits and exactly 4 ones, and thus
is an element of $B$.  Moreover, the mapping is a bijection; every
such bit sequence comes from exactly one order of a dozen doughnuts.
Therefore, $\card{A} = \card{B}$ by the Bijection Rule!

This example demonstrates the magnifying power of the bijection rule.
We managed to prove that two very different sets are actually the same
size ---even though we don't know exactly how big either one is.  But
as soon as we figure out the size of one set, we'll immediately know
the size of the other.

This particular bijection might seem frighteningly ingenious if you've
not seen it before.  But you'll use essentially this same argument
over and over, and soon you'll consider it routine.

\section{Counting Sequences}\label{sec:counting_sequences}

The Bijection Rule lets us count one thing by counting another.  This
suggests a general strategy: get really good at counting just a
\emph{few} things and then use bijections to count
\emph{everything else}.  This is the strategy we'll follow.  In
particular, we'll get really good at counting \emph{sequences}.  When
we want to determine the size of some other set $T$, we'll find a
bijection from $T$ to a set of sequences $S$.  Then we'll use our
super-ninja sequence-counting skills to determine $\card{S}$, which
immediately gives us $\card{T}$.  We'll need to hone this idea
somewhat as we go along, but that's pretty much the plan!

\subsection{The Product Rule}

The \term{Product Rule} gives the size of a product of sets.  Recall
that if $P_1, P_2, \ldots, P_n$ are sets, then
%
\[
P_1 \times P_2 \times \ldots \times P_n
\]
%
is the set of all sequences whose first term is drawn from $P_1$,
second term is drawn from $P_2$ and so forth.

\begin{rul}[Product Rule]
If $P_1, P_2, \ldots P_n$ are finite sets, then:
%
\begin{align*}
\card{P_1 \times P_2 \times \ldots \times P_n}
    & = \card{P_1} \cdot \card{P_2} \cdots \card{P_n}
\end{align*}
\end{rul}

For example, suppose a \emph{daily diet}
consists of a breakfast selected from set $B$, a lunch from set~$L$,
and a dinner from set~$D$ where:
%
\begin{align*}
B & = \set{\text{pancakes},
      	   \text{bacon and eggs},
           \text{bagel},
           \text{Doritos}} \\
L & = \set{\text{burger and fries},
           \text{garden salad},
           \text{Doritos}} \\
D & = \set{\text{macaroni},
           \text{pizza},
           \text{frozen burrito},
           \text{pasta},
           \text{Doritos}}
\end{align*}
%
Then $B \times L \times D$ is the set of all possible daily diets.
Here are some sample elements:
%
\begin{gather*}
(\text{pancakes}, \text{burger and fries}, \text{pizza}) \\
(\text{bacon and eggs}, \text{garden salad}, \text{pasta}) \\
(\text{Doritos}, \text{Doritos}, \text{frozen burrito})
\end{gather*}
%
The Product Rule tells us how many different daily diets are possible:
%
\begin{align*}
\card{B \times L \times D}
    & = \card{B} \cdot \card{L} \cdot \card{D} \\
    & = 4 \cdot 3 \cdot 5 \\
    & = 60.
\end{align*}

\subsection{Subsets of an $n$-element Set}\label{2nsubsets}

\iffalse

How many different subsets of an $n$-element set $X$ are there?  For
example, the set $X = \set{x_1, x_2, x_3}$ has eight different subsets:
%
\[
\begin{array}{cccc}
\emptyset & \set{x_1} & \set{x_2} & \set{x_1, x_2} \\
\set{x_3} & \set{x_1, x_3} & \set{x_2, x_3} & \set{x_1, x_2, x_3}.
\end{array}
\]

There is a natural bijection from subsets of $X$ to $n$-bit sequences.
Let $x_1, x_2, \ldots, x_n$ be the elements of $X$.  Then a particular
subset of $X$ maps to the sequence $(b_1, \ldots, b_n)$ where $b_i =
1$ if and only if $x_i$ is in that subset.  For example, if $n = 10$,
then the subset $\set{x_2, x_3, x_5, x_7, x_{10}}$ maps to a 10-bit
sequence as follows:
%
\[
\begin{array}{rrrrrrrrrrrrr}
\text{subset:} &
\{ &    & x_2, & x_3, &    & x_5, &   & x_7, &    &    & x_{10} & \} \\
\text{sequence:} &
(  & 0, &   1, &   1, & 0, &   1, & 0, &   1, & 0, & 0, &        1 & )
\end{array}
\]
\fi

The fact that there are $2^n$ subsets of an $n$-element set was proved
in Theorem~\ref{powset_fincard} by setting up a bijection between the
subsets and the length-$n$ bit-strings.  So the original problem about
subsets was tranformed into a question about sequences
---\emph{exactly according to plan!}  Now we can fill in the missing
explanation of why there are $2^n$ length-$n$ bit-strings:
\iffalse
Now if we answer the sequence
question, then we've solved our original problem as well.

But how many different $n$-bit sequences are there?  For example,
there are 8 different 3-bit sequences:
%
\[
\begin{array}{ccccccc}
(0,0,0) & \quad & (0,0,1) & \quad & (0,1,0) & \quad & (0,1,1) \\
(1,0,0) & \quad & (1,0,1) & \quad & (1,1,0) & \quad & (1,1,1)
\end{array}
\]

Well,\fi
we can write the set of all $n$-bit sequences as a product of
sets:
%
\[
\set{0,1}^n \eqdef \underbrace{\set{0,1} \times \set{0,1} \times
        \cdots \times \set{0,1}}_{\text{$n$ terms}}.
\]
%
Then Product Rule gives the answer:
%
\[
\card{\set{0,1}^n} = \card{\set{0,1}}^n  = 2^n.
\]


\iffalse
This means that the number of subsets of an $n$-element set $X$ is
also $2^n$.  We'll put this answer to use shortly.
\fi

\subsection{The Sum Rule}

Bart allocates his little sister Lisa a quota of 20 crabby days, 40
irritable days, and 60 generally surly days.  On how many days can
Lisa be out-of-sorts one way or another?  Let set $C$ be her crabby
days, $I$ be her irritable days, and $S$ be the generally surly.  In
these terms, the answer to the question is $\card{C \cup I \cup S}$.
Now assuming that she is permitted at most one bad quality each day,
the size of this union of sets is given by the \term{Sum Rule}:

\begin{rul}[Sum Rule]\label{rul:sum}
If $A_1, A_2, \ldots, A_n$ are \emph{disjoint} sets, then:
%
\[
\card{A_1 \cup A_2 \cup \ldots \cup A_n}
    = \card{A_1} + \card{A_2} + \ldots + \card{A_n}
\]
\end{rul}

Thus, according to Bart's budget, Lisa can be out-of-sorts for:
%
\begin{align*}
\card{C \cup I \cup S}
    & = \card{C} + \card{I} + \card{S} \\
    & = 20 + 40 + 60 \\
    & = 120 \text{ days}
\end{align*}

Notice that the Sum Rule holds only for a union of \emph{disjoint}
sets.  Finding the size of a union of overlapping sets is a more
complicated problem that we'll take up in Section~\ref{inc-ex_sec}.

\subsection{Counting Passwords}

Few counting problems can be solved with a single rule.  More often, a
solution is a flurry of sums, products, bijections, and other methods.

For solving problems involving passwords, telephone numbers, and
license plates, the sum and product rules are useful together.  For
example, on a certain computer system, a valid password is a sequence
of between six and eight symbols.  The first symbol must be a letter
(which can be lowercase or uppercase), and the remaining symbols must
be either letters or digits.  How many different passwords are
possible?

Let's define two sets, corresponding to valid symbols in the first and
subsequent positions in the password.
%
\begin{align*}
F & = \set{ a, b, \ldots, z, A, B, \ldots, Z } \\
S & = \set{ a, b, \ldots, z, A, B, \ldots, Z, 0, 1, \ldots, 9 }
\end{align*}
%
In these terms, the set of all possible passwords is:\footnote{The
  notation~$S^5$ means $S \cross S \cross S \cross S \cross S$.}
%
\[
(F \times S^5) \cup (F \times S^6) \cup (F \times S^7)
\]
%
Thus, the length-six passwords are in the set $F \times S^5$, the
length-seven passwords are in $F \times S^6$, and the length-eight
passwords are in $F \times S^7$.  Since these sets are disjoint, we
can apply the Sum Rule and count the total number of possible
passwords as follows:
%
\begin{align*}
\lefteqn{\card{(F \times S^5) \cup (F \times S^6) \cup (F \times S^7)}}
\qquad & \\
    & = \card{F \times S^5} + \card{F \times S^6} + \card{F \times S^7}
        & \text{Sum Rule} \\
    & = \card{F} \cdot \card{S}^5 +
          \card{F} \cdot \card{S}^6 +
          \card{F} \cdot \card{S}^7
        & \text{Product Rule} \\
    & = 52 \cdot 62^5 + 52 \cdot 62^6 + 52 \cdot 62^7 \\
    & \approx 1.8 \cdot 10^{14} \text{ different passwords}.
\end{align*}


%%%%%%%%%%%%%%%%%%%%%%%%%%%%%%%%%%%%%%%%%%%%%%%%%%%%%%%%%%%%%%%%%%%%%%%%%%%%%%%

\begin{problems}
\practiceproblems
\pinput{TP_Counting_Questions}
\pinput{TP_Counting_Answers}
\pinput{TP_Counting_Functions}
\pinput{TP_Counting_Subsets}

\classproblems
\pinput{CP_counting_license_plates}
\pinput{CP_nonadjacent_books}
\pinput{CP_inequality_string_bijections}
\pinput{CP_numbered_trees}
\pinput{CP_bijecting_bijections}
\end{problems}

\section{The Generalized Product Rule}\label{generalized_product_sec}
In how many ways can, say, a Nobel prize, a Japan prize, and a
Pulitzer prize be awarded to $n$ people?  This is easy to answer using
our strategy of translating the problem about awards into a problem
about sequences.  Let $P$ be the set of $n$ people taking the course.
Then there is a bijection from ways of awarding the three prizes to
the set $P^3 \eqdef P \times P \times P$.  In particular, the
assignment:
\begin{center}
``Barak wins a Nobel, George wins a Japan, and Bill wins a Pulitzer prize''
\end{center}
maps to the sequence $(\text{Barak}, \text{George}, \text{Bill})$.  By
the Product Rule, we have $\card{P^3}= \card{P}^3 = n^3$, so there are
$n^3$ ways to award the prizes to a class of $n$ people.  Notice that
$P^3$ includes triples like $(\text{Barak}, \text{Bill},
\text{Barak})$ where one person wins more than one prize.

But what if the three prizes must be awarded to \emph{different}
students?  As before, we could map the assignment to the triple
$(\text{Bill}, \text{George}, \text{Barak}) \in P^3$.  But this
function is \emph{no longer a bijection}.  For example, no valid
assignment maps to the triple $(\text{Barak}, \text{Bill},
\text{Barak})$ because now we're not allowing Barak to receive two
prizes.  However, there \emph{is} a bijection from prize assignments to
the set:
\[
S = \set{(x, y, z) \in P^3 \mid \text{$x$, $y$, and $z$ are different people}}
\]
This reduces the original problem to a problem of counting sequences.
Unfortunately, the Product Rule does not apply directly to counting
sequences of this type because the entries depend on one another; in
particular, they must all be different.  However, a slightly sharper
tool does the trick.

\textbox{
\textboxtitle{Prizes for \emph{truly exceptional} Coursework}

Given everyone's hard work on this material, the instructors
considered awarding some prizes for truly exceptional coursework.
Here are three possible prize categories:

\begin{description}

\item[Best Administrative Critique] We asserted that the quiz was
closed-book.  On the cover page, one strong candidate for this award
wrote, ``There is no book.''

\item[Awkward Question Award] ``Okay, the left sock, right sock, and
pants are in an antichain, but how ---even with assistance ---could I
put on all three at once?''

\item[Best Collaboration Statement] Inspired by a student who wrote
``I worked alone'' on Quiz 1.

\end{description}
}

\begin{rul}[\idx{Generalized Product Rule}]
Let $S$ be a set of length-$k$ sequences.  If there are:
%
\begin{itemize}
\item $n_1$ possible first entries,
\item $n_2$ possible second entries for each first entry,\\
\iffalse
\item $n_3$ possible third entries for each sequence of first and
second entries,\\
\fi
\vdots
\item $n_k$ possible $k$th entries for each sequence of first $k-1$
  entries,
\end{itemize}
%
then:
%
\[
\card{S} = n_1 \cdot n_2 \cdot n_3 \cdots n_k
\]
\end{rul}

In the awards example, $S$ consists of sequences $(x, y, z)$.  There
are $n$ ways to choose $x$, the recipient of prize \#1.  For each of
these, there are $n-1$ ways to choose $y$, the recipient of prize \#2,
since everyone except for person $x$ is eligible.  For each
combination of $x$ and $y$, there are $n-2$ ways to choose $z$, the
recipient of prize \#3, because everyone except for $x$ and $y$ is
eligible.  Thus, according to the Generalized Product Rule, there are
%
\[
\card{S} = n \cdot (n-1) \cdot (n-2)
\]
%
ways to award the 3 prizes to different people.

\subsection{Defective Dollar Bills}

A dollar bill is \emph{defective} if some digit appears
more than once in the 8-digit serial number.  If you check your
wallet, you'll be sad to discover that defective bills are
all-too-common.  In fact, how common are \emph{nondefective} bills?
Assuming that the digit portions of serial numbers all occur equally
often, we could answer this question by computing
%
\begin{equation}\label{eqn:11Q6}
% \text{fraction of nondefective bills}
\text{fraction of nondefective bills}
     = \frac{ \card{\set{\text{serial \#'s with all digits different}}} }
             { \card{\set{\text{serial numbers}}} }.
\end{equation}
%
Let's first consider the denominator.  Here there are no restrictions;
there are 10 possible first digits, 10 possible second digits, 10
third digits, and so on.  Thus, the total number of 8-digit serial
numbers is $10^8$ by the Product Rule.

Next, let's turn to the numerator.  Now we're not permitted to use any
digit twice.  So there are still 10 possible first digits, but only 9
possible second digits, 8 possible third digits, and so forth.  Thus, by
the Generalized Product Rule, there are
%
\begin{align*}
10 \cdot 9 \cdot 8 \cdot 7 \cdot 6 \cdot 5 \cdot 4 \cdot 3
     = \frac{10!}{2} % \\
     = 1{,}814{,}400
\end{align*}
%
serial numbers with all digits different.  Plugging these results into
Equation~\ref{eqn:11Q6}, we find:
%
\begin{align*}
\text{fraction of nondefective bills}
     = \frac{1{,}814{,}400}{100{,}000{,}000} %\\
     = 1.8144\%
\end{align*}

\subsection{A Chess Problem}

In how many different ways can we place a pawn ($\PAWN$), a knight
($\KNIGHT$), and a bishop ($\BISHOP$) on a chessboard so that no two
pieces share a row or a column?  A valid configuration is shown in
Figure~\ref{fig:11Q7}(a), and an invalid configuration is shown in
Figure~\ref{fig:11Q7}(b).

\begin{figure}\normalbaselines

\subfloat[valid]{\chessboard[setpieces={bb4, ne7, pf2}]}
\subfloat[invalid]{\chessboard[setpieces={bc3, pe6, nf3}]}

\caption{Two ways of placing a pawn~($\pawn$), a knight~($\knight$),
  and a bishop~($\bishop$) on a chessboard.  The configuration shown
  in~(b) is invalid because the bishop and the knight are in the same
  row.}

\label{fig:11Q7}

\end{figure}

First, we map this problem about chess pieces to a question about
sequences.  There is a bijection from configurations to sequences
%
\[
    (r_{\PAWN}, c_{\PAWN}, r_{\KNIGHT}, c_{\KNIGHT}, r_{\BISHOP}, c_{\BISHOP})
\]
%
where $r_{\PAWN}$, $r_{\KNIGHT}$, and $r_{\BISHOP}$ are distinct rows
and $c_{\PAWN}$, $c_{\KNIGHT}$, and $c_{\BISHOP}$ are distinct
columns.  In particular, $r_{\PAWN}$ is the pawn's row, $c_{\PAWN}$ is
the pawn's column, $r_{\KNIGHT}$ is the knight's row, etc.  Now we can
count the number of such sequences using the Generalized Product Rule:
\begin{itemize}\compactlist

\item $r_{\PAWN}$ is one of 8 rows

\item $c_{\PAWN}$ is one of 8 columns

\item $r_{\KNIGHT}$ is one of 7 rows (any one but $r_{\PAWN}$)

\item $c_{\KNIGHT}$ is one of 7 columns (any one but $c_{\PAWN}$)

\item $r_{\BISHOP}$ is one of 6 rows (any one but $r_{\PAWN}$ or
  $r_{\KNIGHT}$)

\item $c_{\BISHOP}$ is one of 6 columns (any one but $c_{\PAWN}$ or
  $c_{\KNIGHT}$)

\end{itemize}
Thus, the total number of configurations is $(8 \cdot 7 \cdot 6)^2$.

\subsection{Permutations}

A \term{permutation} of a set $S$ is a sequence that contains every
element of $S$ exactly once.  For example, here are all the
permutations of the set $\set{a, b, c}$:
%
\[
\begin{array}{ccc}
(a, b, c) & (a, c, b) & (b, a, c) \\
(b, c, a) & (c, a, b) & (c, b, a)
\end{array}
\]

How many permutations of an $n$-element set are there?  Well, there
are $n$ choices for the first element.  For each of these, there are
$n - 1$ remaining choices for the second element.  For every
combination of the first two elements, there are $n - 2$ ways to
choose the third element, and so forth.  Thus, there are a total of
%
\[
n \cdot (n-1) \cdot (n-2) \cdots 3 \cdot 2 \cdot 1 = n!
\]
%
permutations of an $n$-element set.  In particular, this formula says
that there are $3! = 6$ permutations of the 3-element set $\set{a, b,
c}$, which is the number we found above.

Permutations will come up again in this course approximately 1.6
bazillion times.  In fact, permutations are the reason why factorial
comes up so often and why we taught you Stirling's approximation:
%
\[
n! \sim \sqrt{2 \pi n} \paren{\frac{n}{e}}^n.
\]

\section{The Division Rule}\label{division_rule_sec}

Counting ears and dividing by two is a silly way to count the number of
people in a room, but this approach is representative of a powerful
counting principle.

A \term{$k$-to-1 function} maps exactly $k$ elements of the domain to
every element of the codomain.  For example, the function mapping each
ear to its owner is 2-to-1. Similarly, the function mapping each
finger to its owner is 10-to-1, and the function mapping each finger
and toe to its owner is 20-to-1.  The general rule is:
\begin{rul}[\idx{Division Rule}]
If $f : A \to B$ is $k$-to-1, then $\card{A} = k \cdot \card{B}$.
\end{rul}

For example, suppose $A$ is the set of ears in the room and $B$ is the set
of people.  There is a 2-to-1 mapping from ears to people, so by the
Division Rule, $\card{A} = 2 \cdot \card{B}$.  Equivalently, $\card{B} =
\card{A} / 2$, expressing what we knew all along: the number of people is
half the number of ears.  Unlikely as it may seem, many counting problems
are made much easier by initially counting every item multiple times and
then correcting the answer using the Division Rule.  Let's look at some
examples.

\subsection{Another Chess Problem}

In how many different ways can you place two identical rooks on a
chessboard so that they do not share a row or column?  A valid
configuration is shown in Figure~\ref{fig:11Q8}(a), and an invalid
configuration is shown in Figure~\ref{fig:11Q8}(b).

\begin{figure}\normalbaselines

\subfloat[valid]{\chessboard[setpieces={ra1, rh8}]}
\subfloat[invalid]{\chessboard[setpieces={rd1, rd6}]}

\caption{Two ways to place 2~rooks ($\rook$) on a chessboard.  The
  configuration in~(b) is invalid because the rooks are in the same
  column.}

\label{fig:11Q8}

\end{figure}

Let $A$ be the set of all sequences
%
\[
(r_1, c_1, r_2, c_2)
\]
%
where $r_1$ and $r_2$ are distinct rows and $c_1$ and $c_2$ are
distinct columns.  Let $B$ be the set of all valid rook
configurations.  There is a natural function $f$ from set $A$ to set
$B$; in particular, $f$ maps the sequence $(r_1, c_1, r_2, c_2)$ to a
configuration with one rook in row $r_1$, column $c_1$ and the other
rook in row $r_2$, column $c_2$.

But now there's a snag.  Consider the sequences:
%
\[
(1, 1, 8, 8) \qquad \text{ and } \qquad (8, 8, 1, 1)
\]
%
The first sequence maps to a configuration with a rook in the
lower-left corner and a rook in the upper-right corner.  The second
sequence maps to a configuration with a rook in the upper-right corner
and a rook in the lower-left corner.  The problem is that those are
two different ways of describing the \emph{same} configuration!  In
fact, this arrangement is shown in Figure~\ref{fig:11Q8}(a).

More generally, the function $f$ maps exactly two sequences to
\emph{every} board configuration; that is $f$ is a 2-to-1 function.
Thus, by the quotient rule, $\card{A} = 2 \cdot \card{B}$.
Rearranging terms gives:
%
\begin{align*}
\card{B}
     = \frac{\card{A}}{2} % \\
     = \frac{(8 \cdot 7)^2}{2}.
\end{align*}
%
On the second line, we've computed the size of $A$ using the General
Product Rule just as in the earlier chess problem.

\subsection{Knights of the Round Table}

In how many ways can King Arthur arrange to seat his $n$ different
knights at his round table?  Two seatings are considered to be the
same \emph{arrangement} if they yield the same sequence of knights
starting at knight number 1 and going clockwise around the table.  For
example, the following two seatings determine the same arrangement:

\begin{center}
\begin{picture}(230,80)
%\put(0,0){\dashbox(230,80){}}
\put(0,0){
\begin{picture}(80,80)(-40,-40)
%\put(-40,-40){\dashbox(80,80){}}
\put(0,0){\circle{36}}
\put(0,26){\makebox(0,0){$k_1$}}
\put(26,0){\makebox(0,0){$k_2$}}
\put(0,-26){\makebox(0,0){$k_3$}}
\put(-26,0){\makebox(0,0){$k_4$}}
\end{picture}}
\put(150,0){
\begin{picture}(80,80)(-40,-40)
%\put(-40,-40){\dashbox(80,80){}}
\put(0,0){\circle{36}}
\put(0,26){\makebox(0,0){$k_3$}}
\put(26,0){\makebox(0,0){$k_4$}}
\put(0,-26){\makebox(0,0){$k_1$}}
\put(-26,0){\makebox(0,0){$k_2$}}
\end{picture}}
\end{picture}
\end{center}

So a seating is determined by the sequence of knights going clockwise
around the table starting at the top seat.  This means seatings are
formally the same as the set, $A$, of all permutations of the knights.
An arrangement is determined by the sequence of knights going
clockwise around the table starting after knight number 1, so it is
formally the same as the set, $B$, of all permutations of knights 2
through $n$.  We can map each permutation in $A$ to an arrangement in
set $B$ by seating the first knight in the permutation at the top of
the table, putting the second knight to his left, the third knight to
the left of the second, and so forth all the way around the table.
For example:
%
\begin{center}
\begin{picture}(200,80)(-160,-40)
%\put(-160,-40){\dashbox(200,80){}} % bounding box
\put(-120,0){\makebox(0,0){$(k_2, k_4, k_1, k_3)$}}
\put(-60,0){\makebox(0,0){$\implies$}}
\put(0,0){\circle{36}}
\put(0,26){\makebox(0,0){$k_1$}}
\put(26,0){\makebox(0,0){$k_3$}}
\put(0,-26){\makebox(0,0){$k_2$}}
\put(-26,0){\makebox(0,0){$k_4$}}
\end{picture}
\end{center}
%
This mapping is actually an $n$-to-1 function from $A$ to $B$, since
all $n$ cyclic shifts of the original sequence map to the same seating
arrangement.  In the example, $n = 4$ different sequences map to the
same seating arrangement:
%
\begin{center}
\begin{picture}(200,80)(-160,-40)
% \put(-160,-40){\dashbox(200,80){}} % bounding box
\put(-120,24){\makebox(0,0){$(k_2, k_4, k_1, k_3)$}}
\put(-120,8){\makebox(0,0){$(k_4, k_1, k_3, k_2)$}}
\put(-120,-8){\makebox(0,0){$(k_1, k_3, k_2, k_4)$}}
\put(-120,-24){\makebox(0,0){$(k_3, k_2, k_4, k_1)$}}
\put(-60,0){\makebox(0,0){$\implies$}}
\put(0,0){\circle{36}}
\put(0,26){\makebox(0,0){$k_1$}}
\put(26,0){\makebox(0,0){$k_3$}}
\put(0,-26){\makebox(0,0){$k_2$}}
\put(-26,0){\makebox(0,0){$k_4$}}
\end{picture}
\end{center}
%
Therefore, by the division rule, the number of circular seating
arrangements is:
%
\begin{align*}
\card{B}
     = \frac{\card{A}}{n} %\\
     = \frac{n!}{n} %\\
     = (n-1)!
\end{align*}
%
Note that $\card{A} = n!$ since there are $n!$ permutations of $n$
knights.

\begin{problems}
\homeworkproblems
\pinput{PS_bijective_FLT}
\end{problems}

\section{Counting Subsets}\label{combinations_sec}

How many $k$-element subsets of an $n$-element set are there?  This
question arises all the time in various guises:

\begin{itemize}

\item In how many ways can I select 5 books from my collection of 100
to bring on vacation?

\item How many different 13-card Bridge hands can be dealt from a
52-card deck?

\item In how many ways can I select 5 toppings for my pizza if there
are 14 available toppings?

\end{itemize}

This number comes up so often that there is a special notation for it:
\[
\binom{n}{k} \eqdef \text{ the number of $k$-element subsets of an $n$-element set.}
\]
The expression $\binom{n}{k}$ is read ``$n$ choose $k$.''  Now we can
immediately express the answers to all three questions above:

\begin{itemize}

\item I can select 5 books from 100 in $\binom{100}{5}$ ways.

\item There are $\binom{52}{13}$ different Bridge hands.

\item There are $\binom{14}{5}$ different 5-topping pizzas, if 14
toppings are available.

\end{itemize}

\subsection{The Subset Rule}

We can derive a simple formula for the $n$ choose $k$ number using the
Division Rule.  We do this by mapping any permutation of an $n$-element
set $\set{a_1,\dots, a_n}$ into a $k$-element subset simply by taking the
first $k$ elements of the permutation.  That is, the permutation
$a_1a_2\dots a_n$ will map to the set $\set{a_1,a_2,\dots,a_k}$.

Notice that any other permutation with the same first $k$ elements
$a_1,\dots,a_k$ \emph{in any order} and the same remaining elements $n-k$
elements \emph{in any order} will also map to this set.  What's more, a
permutation can only map to $\set{a_1,a_2,\dots,a_k}$ if its first $k$
elements are the elements $a_1,\dots,a_k$ in some order.  Since there are
$k!$ possible permutations of the first $k$ elements and $(n-k)!$
permutations of the remaining elements, we conclude from the Product Rule
that exactly $k!(n-k)!$ permutations of the $n$-element set map to the
particular subset, $S$.  In other words, the mapping from permutations to
$k$-element subsets is $k!(n-k)!$-to-1.

But we know there are $n!$ permutations of an $n$-element set, so by the
Division Rule, we conclude that
\[
n!= k!(n-k)!\binom{n}{k}
\]
which proves:
\begin{rul}[Subset Rule]
\label{rule:subset}
The number of $k$-element subsets of an $n$-element set is
\begin{equation*}
    \binom{n}{k} = \frac{n!}{k!\ (n-k)!}.
\end{equation*}
\end{rul}

Notice that this works even for 0-element subsets: $n!/0!n! = 1$.  Here we
use the fact that $0!$ is a \emph{product} of 0 terms, which by
convention\footnote{We don't use it here, but a \emph{sum} of zero
  terms equals~0.}
equals~1.

\subsection{Bit Sequences}

How many $n$-bit sequences contain exactly $k$ ones?  We've already seen
the straightforward bijection between subsets of an $n$-element set and
$n$-bit sequences.  For example, here is a 3-element subset of $\set{x_1,
x_2, \ldots, x_8}$ and the associated 8-bit sequence:
%
\[
\begin{array}{rccccccccl}
\{ & x_1, &    &    & x_4, & x_5  &    &    &   & \} \\
(  &   1, & 0, & 0, &   1, &   1, & 0, & 0, & 0 & )
\end{array}
\]
Notice that this sequence has exactly 3 ones, each corresponding to an
element of the 3-element subset.  More generally, the $n$-bit sequences
corresponding to a $k$-element subset will have exactly $k$ ones.  So by
the Bijection Rule,
\begin{corollary*}
The number of $n$-bit sequences with exactly $k$ ones is $\dbinom{n}{k}$.
\end{corollary*}

\begin{problems}
  \practiceproblems
  \pinput{TP_6042_TEAL_Table}
  \pinput{TP_rose_selections}
  \pinput{MQ_unselected_book_counting}

  \classproblems
  \pinput{CP_division_rule_assign_groups}
  \pinput{CP_pizza_sale}
  \pinput{CP_generalized_product}
  \pinput{CP_nonadjacent_books_counting_sequel}

  \homeworkproblems
  \pinput{PS_counting_graphs}
  \pinput{PS_alphabet}
  \pinput{PS_5_card_poker}
  \pinput{PS_counting_colored_dice}

  \examproblems
  \pinput{MQ_count_double_deck}

\end{problems}


\section{Sequences with Repetitions}\label{bookkeeper_sec}

\begin{editingnotes}
consider switching subsecs \textbf{Sequences of Subsets} and \textbf{The Bookkeeper Rule}
\end{editingnotes}

\subsection{Sequences of Subsets}

Choosing a $k$-element subset of an $n$-element set is the same as
splitting the set into a pair of subsets: the first subset of size $k$ and
the second subset consisting of the remaining $n-k$ elements.  So the
Subset Rule can be understood as a rule for counting the number of such
splits into pairs of subsets.

We can generalize this to splits into more than two subsets.  Namely, let
$A$ be an $n$-element set and $k_1,k_2, \dots, k_m$ be nonnegative integers
whose sum is $n$.  A \term{$(k_1,k_2, \dots, k_m)$-split of $A$} is a
sequence
\[
(A_1, A_2,\dots,A_m)
\]
where the $A_i$ are disjoint subsets of $A$ and $\card{A_i} = k_i$ for
$i=1,\dots,m$.

To count the number of splits we take the same approach as for the
Subset Rule.  Namely, we map any permutation $a_1a_2\dots a_n$ of an
$n$-element set~$A$ into a $(k_1,k_2, \dots, k_m)$-split by letting
the 1st subset in the split be the first $k_1$ elements of the
permutation, the 2nd subset of the split be the next $k_2$ elements,
\dots, and the $m$th subset of the split be the final $k_m$ elements
of the permutation.  This map is a
\hbox{$k_1!\ k_2!\ \cdots\ k_m!$-to-1} function from the
$n!$~permutations to the $(k_1,k_2, \dots, k_m)$-splits of $A$, so
from the Division Rule we conclude the Subset Split Rule:

\begin{definition}
For $n,k_1,\dots,k_m \in \naturals$, such that $k_1+k_2+\cdots+k_m = n$,
define the \term{multinomial coefficient}
\[
\binom{n}{k_1, k_2, \dots, k_m} \eqdef \frac{n!}{k_1!\, k_2!\, \dots k_m!}.
\]
\end{definition}

\begin{rul}[Subset Split Rule]
The number of $(k_1,k_2, \dots, k_m)$-splits of an $n$-element set is
\[
\binom{n}{k_1,\dots,k_m}\,.
\]
\end{rul}


\subsection{The Bookkeeper Rule}

We can also generalize our count of $n$-bit sequences
with $k$ ones to counting sequences of $n$~letters over an alphabet
with more than two letters.  For example, how many sequences can be
formed by permuting the letters in the 10-letter word BOOKKEEPER?

Notice that there are 1 B, 2 O's, 2 K's, 3 E's, 1 P, and 1 R in
BOOKKEEPER.  This leads to a straightforward bijection between
permutations of BOOKKEEPER and (1,2,2,3,1,1)-splits of $\set{1, 2,
  \dots, 10}$.  Namely, map a permutation to the sequence of sets of
positions where each of the different letters occur.

For example, in the permutation BOOKKEEPER itself, the B is in the 1st
position, the O's occur in the 2nd and 3rd positions, K's in 4th and 5th,
the E's in the 6th, 7th and 9th, P in the 8th, and R is in the 10th
position. So BOOKKEEPER maps to
\[
(\set{1}, \set{2,3}, \set{4,5}, \set{6,7,9}, \set{8}, \set{10}).
\]
From this bijection and the Subset Split Rule, we conclude that the
number of ways to rearrange the letters in the word BOOKKEEPER is:
\[
\frac{\overbrace{10!}^{\text{total letters}}}{
\underbrace{1!}_{\text{B's}}
\underbrace{2!}_{\text{O's}}
\underbrace{2!}_{\text{K's}}
\underbrace{3!}_{\text{E's}}
\underbrace{1!}_{\text{P's}}
\underbrace{1!}_{\text{R's}}}
\]

This example generalizes directly to an exceptionally useful counting
principle which we will call the
\begin{rul}[Bookkeeper Rule]
Let $l_1, \ldots, l_m$ be distinct elements.  The number of sequences with
$k_1$ occurrences of $l_1$, and $k_2$ occurrences of $l_2$, \dots, and
$k_m$ occurrences of $l_m$ is
\[
\binom{k_1 + k_2 + \cdots + k_m}{k_1,\dots,k_m}\,.
\]
\end{rul}

For example, suppose you are planning a 20-mile walk, which should
include 5 northward miles, 5 eastward miles, 5 southward miles, and 5
westward miles.  How many different walks are possible?

There is a bijection between such walks and sequences with 5 N's, 5
E's, 5 S's, and 5 W's.  By the Bookkeeper Rule, the number of such
sequences is:
\[
    \frac{20!}{(5!)^4}.
\]

\begin{problems}
\practiceproblems
\pinput{TP_MISSISSIPPI}
\examproblems
\pinput{MQ_counting_robot_paths}
\end{problems}

\subsubsection{A Word about Words}

Someday you might refer to the Subset Split Rule or the Bookkeeper Rule
in front of a roomful of colleagues and discover that they're all staring
back at you blankly.  This is not because they're dumb, but rather because
we made up the name ``Bookkeeper Rule.''  However, the rule is excellent
and the name is apt, so we suggest that you play through: ``You know?  The
Bookkeeper Rule?  Don't you guys know \emph{anything???}''

The Bookkeeper Rule is sometimes called the ``formula for permutations
with indistinguishable objects.''  The size $k$ subsets of an $n$-element
set are sometimes called \term{$k$-combinations}.  Other similar-sounding
descriptions are ``combinations with repetition, permutations with
repetition, $r$-permutations, permutations with indistinguishable
objects,'' and so on.  However, the counting rules we've taught you are
sufficient to solve all these sorts of problems without knowing this
jargon, so we won't burden you with it.

\begin{problems}
\classproblems
\pinput{CP_bookkeeper_tao}
\end{problems}


\subsection{The Binomial Theorem}\label{binomial_theorem_sec}

Counting gives insight into one of the basic theorems of algebra.  A
\term{binomial} is a sum of two terms, such as $a + b$.  Now consider its
4th power, $(a + b)^4$.

If we multiply out this 4th power expression completely, we get
\[\begin{array}{rccccccccc}
(a + b)^4
   & = &    & aaaa & + & aaab & + & aaba & + & aabb \\
   &   &  + & abaa & + & abab & + & abba & + & abbb \\
   &   &  + & baaa & + & baab & + & baba & + & babb \\
   &   &  + & bbaa & + & bbab & + & bbba & + & bbbb
\end{array}\]
Notice that there is one term for every sequence of $a$'s and $b$'s.  So
there are $2^4$ terms, and the number of terms with $k$ copies of $b$ and
$n - k$ copies of $a$ is:
\[
\frac{n!}{k!\ (n-k)!} = \binom{n}{k}
\]
by the Bookkeeper Rule.  Hence, the coefficient of $a^{n-k} b^k$ is
$\binom{n}{k}$.  So for $n = 4$, this means:
\[
(a + b)^4 =
    \binom{4}{0} \cdot a^4 b^0 +
    \binom{4}{1} \cdot a^3 b^1 +
    \binom{4}{2} \cdot a^2 b^2 +
    \binom{4}{3} \cdot a^1 b^3 +
    \binom{4}{4} \cdot a^0 b^4
\]
In general, this reasoning gives the Binomial Theorem:

\begin{theorem}[\idx{Binomial Theorem}]
For all $n \in \mathbb{N}$ and $a, b \in \mathbb{R}$:
%
\[
(a + b)^n = \sum_{k=0}^n \binom{n}{k} a^{n-k} b^k
\]
\end{theorem}
The Binomial Theorem explains why the $n$ choose $k$ number is called
a \term{binomial coefficient}.

This reasoning about binomials extends nicely to \term{multinomials},
which are sums of two or more terms.  For example, suppose we wanted
the coefficient of
%
\[
b o^2 k^2 e^3 p r
\]
%
in the expansion of $(b + o + k + e + p + r)^{10}$.  Each term in this
expansion is a product of 10 variables where each variable is one of
$b$, $o$, $k$, $e$, $p$, or $r$.  Now, the coefficient of $b o^2 k^2
e^3 p r$ is the number of those terms with exactly 1 $b$, 2 $o$'s, 2
$k$'s, 3 $e$'s, 1 $p$, and 1 $r$.  And the number of such terms is
precisely the number of rearrangements of the word BOOKKEEPER:
\[
\binom{10}{1,2,2,3,1,1} = \frac{10!}{1!\ 2!\ 2!\ 3!\ 1!\ 1!}.
\]
This reasoning extends to a general theorem.

\begin{theorem}[Multinomial Theorem]\label{multinom-thm}
For all $n \in \mathbb{N}$,
\[
(z_1 + z_2 + \cdots + z_m)^n =
   \sum_{\substack{k_1, \dots, k_m \in \mathbb{N} \\
                   k_1 + \cdots + k_m = n}}
   \binom{n}{k_1, k_2, \dots, k_m} z_1^{k_1} z_2^{k_2} \cdots z_m^{k_m}.
\]
\end{theorem}
You'll be better off remembering the reasoning behind the Multinomial
Theorem rather than this cumbersome formal statement.

\begin{problems}

%\practiceproblems
%\pinput{TP_binomial_coeff}
%\pinput{TP_Binomial_Coefficients}

\classproblems
\pinput{CP_binom_coeff}
\pinput{CP_multinomial_fermat}

\homeworkproblems
\pinput{PS_more_numbered_trees}

\end{problems}

\section{Counting Practice: Poker Hands}\label{poker_hands_sec}

Five-Card Draw is a card game in which each player is initially dealt
a \emph{hand} consisting of 5~cards from a deck of
52~cards.\footnote{There are 52 cards in a standard deck.  Each card
  has a \emph{suit} and a \emph{rank}.  There are four suits:
%
\[
\spadesuit   \text{ (spades)} \qquad
\heartsuit   \text{ (hearts)} \qquad
\clubsuit    \text{ (clubs)} \qquad
\diamondsuit \text{ (diamonds)}
\]
%
And there are 13 ranks, listed here from lowest to highest:
%
\[
\stackrel{\text{Ace}}{A},\
2\ ,\ 3\ ,\ 4\ ,\ 5\ ,\ 6\ ,\ 7\ ,\ 8\ ,\ 9\ ,\
\stackrel{\text{Jack}}{J}\ ,\
\stackrel{\text{Queen}}{Q}\ ,\
\stackrel{\text{King}}{K}.
\]
%
Thus, for example, $8 \heartsuit$ is the 8 of hearts and $A
\spadesuit$ is the ace of spades.}  (Then the game gets complicated,
but let's not worry about that.)  The number of different hands in
Five-Card Draw is the number of 5-element subsets of a 52-element set,
which is
%
\[
\binom{52}{5} = 2,598,960.
\]
%
Let's get some counting practice by working out the number of hands
with various special properties.

\subsection{Hands with a Four-of-a-Kind}

A \emph{Four-of-a-Kind} is a set of four cards with the same rank.
How many different hands contain a Four-of-a-Kind?  Here are a couple
examples:
\begin{gather*}
\set{ 8 \spa, \, 8 \dia, \, Q \hea, \, 8 \hea, \, 8 \clu  } \\
\set{ A \clu, \, 2 \clu, \, 2 \hea, \, 2 \dia, \, 2 \spa  } \\
\end{gather*}
%
As usual, the first step is to map this question to a
sequence-counting problem.  A hand with a Four-of-a-Kind is completely
described by a sequence specifying:
%
\begin{enumerate}
\item The rank of the four cards.
\item The rank of the extra card.
\item The suit of the extra card.
\end{enumerate}
%
Thus, there is a bijection between hands with a Four-of-a-Kind and
sequences consisting of two distinct ranks followed by a suit.  For
example, the three hands above are associated with the following
sequences:
%
\begin{align*}
(8, Q, \hea)
    & \leftrightarrow
        \set{ \, 8 \spa, \, 8 \dia, \, 8 \hea, \, 8 \clu, \, Q \hea } \\
(2, A, \clu)
    & \leftrightarrow
        \set{ 2 \clu, \, 2 \hea, \, 2 \dia, \, 2 \spa, \, A \clu } \\
\end{align*}
%
Now we need only count the sequences.  There are 13 ways to choose the
first rank, 12 ways to choose the second rank, and 4 ways to choose
the suit.  Thus, by the Generalized Product Rule, there are $13 \cdot
12 \cdot 4 = 624$ hands with a Four-of-a-Kind.  This means that only 1
hand in about 4165 has a Four-of-a-Kind.  Not surprisingly,
Four-of-a-Kind is considered to be a very good poker hand!

\subsection{Hands with a Full House}\label{sec:counting_full_houses}

A \emph{Full House} is a hand with three cards of one rank and
two cards of another rank.  Here are some examples:
%
\begin{gather*}
\set{ 2 \spa, \, 2 \clu, \, 2 \dia, \, J \clu, \, J \dia } \\
\set{ 5 \dia, \, 5 \clu, \, 5 \hea, \, 7 \hea, \, 7 \clu }
\end{gather*}
%
Again, we shift to a problem about sequences.  There is a bijection
between Full Houses and sequences specifying:
%
\begin{enumerate}
\item The rank of the triple, which can be chosen in 13 ways.
\item The suits of the triple, which can be selected in $\binom{4}{3}$ ways.
\item The rank of the pair, which can be chosen in 12 ways.
\item The suits of the pair, which can be selected in $\binom{4}{2}$ ways.
\end{enumerate}
%
The example hands correspond to sequences as shown below:
\begin{align*}
(2, \set{\spa, \clu, \dia}, J, \set{\clu, \dia})
    & \leftrightarrow
    \set{ 2 \spa, \, 2 \clu, \, 2 \dia, \, J \clu, \, J \dia } \\
(5, \set{\dia, \clu, \hea}, 7, \set{\hea, \clu})
    & \leftrightarrow
    \set{ 5 \dia, \, 5 \clu, \, 5 \hea, \, 7 \hea, \, 7 \clu }
\end{align*}
%
By the Generalized Product Rule, the number of Full Houses is:
%
\[
    13 \cdot \binom{4}{3} \cdot 12 \cdot \binom{4}{2}.
\]
%
We're on a roll ---but we're about to hit a speed bump.

\subsection{Hands with Two Pairs}

How many hands have \emph{Two Pairs}; that is, two cards of one
rank, two cards of another rank, and one card of a third rank?
Here are examples:
%
\begin{gather*}
\set{ 3 \dia, \, 3 \spa, \, Q \dia, \, Q \hea, \, A \clu } \\
\set{ 9 \hea, \, 9 \dia, \, 5 \hea, \, 5 \clu, \, K \spa }
\end{gather*}
%
Each hand with Two Pairs is described by a sequence consisting of:
%
\begin{enumerate}
\item The rank of the first pair, which can be chosen in 13 ways.
\item The suits of the first pair, which can be selected $\binom{4}{2}$ ways.
\item The rank of the second pair, which can be chosen in 12 ways.
\item The suits of the second pair, which can be selected in $\binom{4}{2}$ ways.
\item The rank of the extra card, which can be chosen in 11 ways.
\item The suit of the extra card, which can be selected in $\binom{4}{1} = 4$ ways.
\end{enumerate}
%
Thus, it might appear that the number of hands with Two Pairs is:
%
\[
    13 \cdot \binom{4}{2} \cdot 12 \cdot \binom{4}{2} \cdot 11 \cdot 4.
\]
%
Wrong answer!  The problem is that there is \emph{not} a bijection
from such sequences to hands with Two Pairs.  This is actually a
2-to-1 mapping.  For example, here are the pairs of sequences that map
to the hands given above:
%
\[
\begin{array}{rcl}
(3, \set{\dia, \spa}, Q, \set{\dia, \hea}, A, \clu) & \searrow \\
 & & \set{ 3 \dia, \, 3 \spa, \, Q \dia, \, Q \hea, \, A \clu } \\
(Q, \set{\dia, \hea}, 3, \set{\dia, \spa}, A, \clu) & \nearrow \\
\\
(9, \set{\hea, \dia}, 5, \set{\hea, \clu}, K, \spa) & \searrow \\
& & \set{ 9 \hea, \, 9 \dia, \, 5 \hea, \, 5 \clu, \, K \spa } \\
(5, \set{\hea, \clu}, 9, \set{\hea, \dia}, K, \spa) & \nearrow \\
\end{array}
\]
%
The problem is that nothing distinguishes the first pair from the
second.  A pair of 5's and a pair of 9's is the same as a pair of 9's
and a pair of 5's.  We avoided this difficulty in counting Full Houses
because, for example, a pair of 6's and a triple of kings is different
from a pair of kings and a triple of 6's.

We ran into precisely this difficulty last time, when we went from
counting arrangements of \emph{different} pieces on a chessboard to
counting arrangements of two \emph{identical} rooks.  The solution
then was to apply the Division Rule, and we can do the same here.  In
this case, the Division rule says there are twice as many sequences
as hands, so the number of hands with Two Pairs is actually:
%
\[
\frac{13 \cdot \binom{4}{2} \cdot 12 \cdot \binom{4}{2} \cdot 11 \cdot 4}{2}.
\]

\subsubsection*{Another Approach}

The preceding example was disturbing!  One could easily overlook the
fact that the mapping was 2-to-1 on an exam, fail the course, and turn
to a life of crime.  You can make the world a safer place in two ways:

\begin{enumerate}

\item Whenever you use a mapping $f : A \to B$ to translate one counting
  problem to another, check that the same number elements in $A$ are
  mapped to each element in $B$.  If $k$ elements of $A$ map to each of
  element of $B$, then apply the Division Rule using the constant $k$.

\item As an extra check, try solving the same problem in a different
way.  Multiple approaches are often available ---and all had better
give the same answer!  (Sometimes different approaches give answers
that \emph{look} different, but turn out to be the same after some
algebra.)

\end{enumerate}

We already used the first method; let's try the second.  There is a
bijection between hands with two pairs and sequences that specify:
%
\begin{enumerate}
\item The ranks of the two pairs, which can be chosen in $\binom{13}{2}$ ways.
\item The suits of the lower-rank pair, which can be selected in $\binom{4}{2}$ ways.
\item The suits of the higher-rank pair, which can be selected in $\binom{4}{2}$ ways.
\item The rank of the extra card, which can be chosen in $11$ ways.
\item The suit of the extra card, which can be selected in $\binom{4}{1} = 4$ ways.
\end{enumerate}
%
For example, the following sequences and hands correspond:
%
\begin{align*}
(\set{3, Q}, \set{\dia, \spa}, \set{\dia, \hea}, A, \clu)
    & \leftrightarrow
        \set{ 3 \dia, \, 3 \spa, \, Q \dia, \, Q \hea, \, A \clu } \\
(\set{9, 5}, \set{\hea, \clu}, \set{\hea, \dia}, K, \spa)
    & \leftrightarrow
        \set{ 9 \hea, \, 9 \dia, \, 5 \hea, \, 5 \clu, \, K \spa }
\end{align*}
%
Thus, the number of hands with two pairs is:
%
\[
\binom{13}{2} \cdot \binom{4}{2} \cdot \binom{4}{2} \cdot 11 \cdot 4.
\]
%
This is the same answer we got before, though in a slightly different
form.

\subsection{Hands with Every Suit}

How many hands contain at least one card from every suit?  Here is an
example of such a hand:
%
\begin{equation*}
\set{ 7 \dia,\,  K \clu, \, 3 \dia, \, A \hea, \, 2 \spa }
\end{equation*}
%
Each such hand is described by a sequence that specifies:

\begin{enumerate}

\item The ranks of the diamond, the club, the heart, and the spade,
which can be selected in $13 \cdot 13 \cdot 13 \cdot 13 = 13^4$ ways.

\item The suit of the extra card, which can be selected in 4 ways.

\item The rank of the extra card, which can be selected in 12 ways.

\end{enumerate}

For example, the hand above is described by the sequence:
%
\[
(7, K, A, 2, \dia, 3) \leftrightarrow 
    \set{ 7 \dia, \, K \clu, \, A \hea, \, 2 \spa, \, 3 \dia  }.
\]
%
Are there other sequences that correspond to the same hand?  There is
one more!  We could equally well regard either the $3 \dia$ or the $7
\dia$ as the extra card, so this is actually a 2-to-1 mapping.  Here
are the two sequences corresponding to the example hand:
%
\[
\begin{array}{rcl}
(7, K, A, 2, \dia, 3) & \searrow & \\
 && \set{ 7 \dia, \, K \clu, \, A \hea, \, 2 \spa, \, 3 \dia } \\
(3, K, A, 2, \dia, 7) & \nearrow &
\end{array}
\]
%
Therefore, the number of hands with every suit is:
%
\[
\frac{13^4 \cdot 4 \cdot 12}{2}.
\]

\begin{problems}
\practiceproblems
\pinput{TP_Counting_Poker_Hands}

\classproblems
%\pinput{CP_more_counting}
\pinput{CP_counting_practice}


\examproblems
\pinput{FP_counting_given_answers}
\pinput{MQ_more_counting_practice}

\end{problems}


\section{The Pigeonhole Principle}\label{pigeon_hole_sec}

Here is an old puzzle:

\begin{quotation}
\noindent A drawer in a dark room contains red socks, green socks, and
blue socks.  How many socks must you withdraw to be sure that you have
a matching pair?
\end{quotation}

For example, picking out three socks is not enough; you might end up
with one red, one green, and one blue.  The solution relies on the

\textbox{
\textboxtitle{Pigeonhole Principle}
\begin{quote}
\emph{If there are more pigeons than holes they occupy, then at least two
  pigeons must be in the same hole.}
\end{quote}
}

What pigeons have to do with selecting footwear under poor lighting
conditions may not be immediately obvious, but if we let socks be
pigeons and the colors be three pigeonholes, then as soon as you pick
four socks, there are bound to be two in the same hole, that is, with
the same color.  So four socks are enough to ensure a matched
pair.  For example, one possible mapping of four socks to three colors
is shown in Figure~\ref{fig:11P1}.

\begin{figure}

\graphic{Fig_11P1}

\caption{One possible mapping of four socks to three colors.}

\label{fig:11P1}

\end{figure}

A rigorous statement of the Principle goes this way:
\begin{rul}[Pigeonhole Principle]
  If $\card{A} > \card{B}$, then for every total function $f : A \to
  B$, there exist two different elements of $A$ that are mapped by $f$
  to the same element of $B$.
\end{rul}
Stating the Principle this way may be less intuitive, but it should
now sound familiar: it is simply the contrapositive of the
\idx{Mapping Rules} injective case~\eqref{inj_le_fincard}.  Here, the
pigeons form set $A$, the pigeonholes are the set $B$, and $f$
describes which hole each pigeon occupies.

Mathematicians have come up with many ingenious applications for the
pigeonhole principle.  If there were a cookbook procedure for
generating such arguments, we'd give it to you.  Unfortunately, there
isn't one.  One helpful tip, though: when you try to solve a problem
with the pigeonhole principle, the key is to clearly identify three
things:

\begin{enumerate}

\item The set $A$ (the pigeons).

\item The set $B$ (the pigeonholes).

\item The function $f$ (the rule for assigning pigeons to pigeonholes).

\end{enumerate}


\subsection{Hairs on Heads}

There are a number of generalizations of the pigeonhole principle.
For example:
\begin{rul}[\idx{Generalized Pigeonhole Principle}]
  If $\card{A} > k \cdot \card{B}$, then every total function $f : A \to
  B$ maps at least $k+1$ different elements of $A$ to the same element of
  $B$.
\end{rul}

For example, if you pick two people at random, surely they are extremely
unlikely to have \emph{exactly} the same number of hairs on their heads.
However, in the remarkable city of Boston, Massachusetts there are
actually \emph{three} people who have exactly the same number of hairs!
Of course, there are many bald people in Boston, and they all have zero
hairs.  But we're talking about non-bald people; say a person is non-bald
if they have at least ten thousand hairs on their head.

Boston has about 500,000 non-bald people, and the number of hairs on a
person's head is at most 200,000.  Let $A$ be the set of non-bald people
in Boston, let $B = \set{10,000, 10,001, \dots, 200,000}$, and let $f$ map
a person to the number of hairs on his or her head.  Since $\card{A} > 2
\card{B}$, the Generalized Pigeonhole Principle implies that at least
three people have exactly the same number of hairs.  We don't know who
they are, but we know they exist!

\subsection{Subsets with the Same Sum}

For your reading pleasure, we have displayed ninety 25-digit numbers
in Figure~\ref{fig:11P3}.  Are there two different subsets of these
25-digit numbers that have the same sum? For example, maybe the sum of
the last ten numbers in the first column is equal to the sum of the
first eleven numbers in the second column?

\begin{figure}\normalbaselines\redrawntrue

\scriptsize

\begin{tabular}{rr}
0020480135385502964448038 &
3171004832173501394113017 \\
5763257331083479647409398 &
8247331000042995311646021 \\
0489445991866915676240992 &
3208234421597368647019265 \\
5800949123548989122628663 &
8496243997123475922766310 \\
1082662032430379651370981 &
3437254656355157864869113 \\
6042900801199280218026001 &
8518399140676002660747477 \\
1178480894769706178994993 &
3574883393058653923711365 \\
6116171789137737896701405 &
8543691283470191452333763 \\
1253127351683239693851327 &
3644909946040480189969149 \\
6144868973001582369723512 &
8675309258374137092461352 \\
1301505129234077811069011 &
3790044132737084094417246 \\
6247314593851169234746152 &
8694321112363996867296665 \\
1311567111143866433882194 &
3870332127437971355322815 \\
6814428944266874963488274 &
8772321203608477245851154 \\
1470029452721203587686214 &
4080505804577801451363100 \\
6870852945543886849147881 &
8791422161722582546341091 \\
1578271047286257499433886 &
4167283461025702348124920 \\
6914955508120950093732397 &
9062628024592126283973285 \\
1638243921852176243192354 &
4235996831123777788211249 \\
6949632451365987152423541 &
9137845566925526349897794 \\
1763580219131985963102365 &
4670939445749439042111220 \\
7128211143613619828415650 &
9153762966803189291934419 \\
1826227795601842231029694 &
4815379351865384279613427 \\
7173920083651862307925394 &
9270880194077636406984249 \\
1843971862675102037201420 &
4837052948212922604442190 \\
7215654874211755676220587 &
9324301480722103490379204 \\
2396951193722134526177237 &
5106389423855018550671530 \\
7256932847164391040233050 &
9436090832146695147140581 \\
2781394568268599801096354 &
5142368192004769218069910 \\
7332822657075235431620317 &
9475308159734538249013238 \\
2796605196713610405408019 &
5181234096130144084041856 \\
7426441829541573444964139 &
9492376623917486974923202 \\
2931016394761975263190347 &
5198267398125617994391348 \\
7632198126531809327186321 &
9511972558779880288252979 \\
2933458058294405155197296 &
5317592940316231219758372 \\
7712154432211912882310511 &
9602413424619187112552264 \\
3075514410490975920315348 &
5384358126771794128356947 \\
7858918664240262356610010 &
9631217114906129219461111 \\
8149436716871371161932035 &
3157693105325111284321993 \\
3111474985252793452860017 &
5439211712248901995423441 \\
7898156786763212963178679 &
9908189853102753335981319 \\
3145621587936120118438701 &
5610379826092838192760458 \\
8147591017037573337848616 &
9913237476341764299813987 \\
3148901255628881103198549 &
5632317555465228677676044 \\
5692168374637019617423712 &
8176063831682536571306791
\end{tabular}

\caption{Ninety 25-digit numbers.  Can you find two different subsets
  of these numbers that have the same sum?}

\label{fig:11P3}

\end{figure}

Finding two subsets with the same sum may seem like a silly puzzle,
but solving these sorts of problems turns out to be useful in diverse
applications such as finding good ways to fit packages into shipping
containers and decoding secret messages.

It turns out that it is hard to find different subsets with the same
sum, which is why this problem arises in cryptography.  But it is easy
to prove that two such subsets \emph{exist}.  That's where the
Pigeonhole Principle comes in.

Let $A$ be the collection of all subsets of the 90 numbers in the
list.  Now the sum of any subset of numbers is at most $90 \cdot
10^{25}$, since there are only 90 numbers and every 25-digit number is
less than $10^{25}$.  So let $B$ be the set of integers $\set{0, 1,
  \ldots, 90 \cdot 10^{25}}$, and let $f$ map each subset of numbers
(in $A$) to its sum (in~$B$).

We proved that an $n$-element set has $2^n$ different subsets in
Section~\ref{sec:counting_sequences}.  Therefore:
\begin{equation*}
\card{A} = 2^{90} \geq 1.237 \times 10^{27}
\end{equation*}
On the other hand:
%
\begin{equation*}
\card{B} = 90 \cdot 10^{25} + 1 \leq 0.901 \times 10^{27}.
\end{equation*}
%
Both quantities are enormous, but $\card{A}$ is a bit greater than
$\card{B}$.  This means that $f$ maps at least two elements of $A$ to
the same element of $B$.  In other words, by the Pigeonhole Principle,
two different subsets must have the same sum!

Notice that this proof gives no indication \emph{which} two sets of
numbers have the same sum.  This frustrating variety of argument is
called a \term{nonconstructive proof}.

\textbox{
\textboxtitle{The \$100 prize for two same-sum subsets}

To see if was possible to actually \emph{find} two different subsets
of the ninety 25-digit numbers with the same sum, we offered a \$100
prize to the first student who did it.  We didn't expect to have to
pay off this bet, but we underestimated the ingenuity and initiative
of the students.  One computer science major wrote a program that
cleverly searched only among a reasonably small set of ``plausible''
sets, sorted them by their sums, and actually found a couple with the
same sum.  He won the prize.  A few days later, a math major figured
out how to reformulate the sum problem as a ``\idx{lattice basis
  reduction}'' problem; then he found a software package implementing
an efficient basis reduction procedure, and using it, he very quickly
found lots of pairs of subsets with the same sum.  He didn't win the
prize, but he got a standing ovation from the class ---staff included.
}

%\begin{figure}[!p]\redrawntrue

%\begin{pagesidebar}[to \textheight]

\textbox{
\textboxtitle{The \$500 Prize for Sets with Distinct Subset Sums}

How can we construct a set of $n$ positive integers such that all its
subsets have \emph{distinct} sums?  One way is to use powers of two:
\[
\set{1, 2, 4, 8, 16}
\]
This approach is so natural that one suspects all other such sets must
involve larger numbers.  (For example, we could safely replace 16 by
17, but not by 15.)  Remarkably, there are examples involving
\emph{smaller} numbers.  Here is one:
\[
\set{6, 9, 11, 12, 13}
\]
One of the top mathematicians of the Twentieth Century, Paul
Erd\H{o}s, conjectured in 1931 that there are no such sets involving
\emph{significantly} smaller numbers.  More precisely, he conjectured
that the largest number in such a set must be greater than~$c 2^n$ for
some constant $c>0$.  He offered \$500 to anyone who could prove or
disprove his conjecture, but the problem remains unsolved.
}

\subsection{A Magic Trick}\label{cardmagic_sec}

\dmj{Need to standardize use of ``5'' vs.\ ``five'' and ``4''
  vs.\ ``four.''  Normally one would always spell out single-digit
  numbers.  Do you have a strong preference otherwise?}

A Magician sends an Assistant into the audience with a deck of 52
cards while the Magician looks away.

Five audience members each select one card from the deck.  The Assistant
then gathers up the five cards and holds up four of them so the Magician
can see them.  The Magician concentrates for a short time and then
correctly names the secret, fifth card!

Since we don't really believe the Magician can read minds, we know the
Assistant has somehow communicated the secret card to the Magician.  Real
Magicians and Assistants are not to be trusted, so we expect that the
Assistant would secretly signal the Magician with coded phrases or body
language, but for this trick they don't have to cheat.  In fact, the
Magician and Assistant could be kept out of sight of each other while some
audience member holds up the 4 cards designated by the Assistant for the
Magician to see.

Of course, without cheating, there is still an obvious way the Assistant
can communicate to the Magician: he can choose any of the $4! = 24$
permutations of the 4 cards as the order in which to hold up the cards.
However, this alone won't quite work: there are 48 cards remaining in the
deck, so the Assistant doesn't have enough choices of orders to indicate
exactly what the secret card is (though he could narrow it down to two
cards).

\subsection{The Secret}

The method the Assistant can use to communicate the fifth card exactly is
a nice application of what we know about counting and matching.

The Assistant has a second legitimate way to communicate: he can
choose \emph{which of the five cards to keep hidden}.  Of course, it's
not clear how the Magician could determine which of these five
possibilities the Assistant selected by looking at the four visible
cards, but there is a way, as we'll now explain.

The problem facing the Magician and Assistant is actually a bipartite
matching problem.  Each vertex on left will correspond to the
information available to the Assistant, namely, a \emph{set} of 5
cards.  So the set~$X$ of left hand vertices will have $\binom{52}{5}$
elements.  

Each vertex on right will correspond to the information available to
the Magician, namely, a \emph{sequence} of 4~distinct cards.  So the
set~$Y$ of right hand vertices will have $52\cdot 51 \cdot 50 \cdot
49$ elements.  When the audience selects a set of 5~cards, then the
Assistant must reveal a sequence of 4~cards from that hand.  This
constraint is represented by having an edge between a set of 5 cards
on the left and a sequence of 4 cards on the right precisely when
every card in the sequence is also in the set.  This specifies the
bipartite graph.  Some edges are shown in the diagram in
Figure~\ref{fig:11Q9}.

\begin{figure}

\graphic{Fig_Q9}

\caption{The bipartite graph where the nodes on the left correspond to
  \emph{sets} of 5~cards and the nodes on the right correspond to
  \emph{sequences} of 4~cards. There is an edge between a set and a
  sequence whenever all the cards in the sequence are contained in the
  set.}

\label{fig:11Q9}

\end{figure}

For example,
\begin{equation}\label{2dia6dia}
\set{ 8 \hea, K \spa, Q \spa, 2 \dia, 6 \dia }
\end{equation}
is an element of $X$ on the left.  If the audience selects this set of 5
cards, then there are many different 4-card sequences on the right in set
$Y$ that the Assistant could choose to reveal, including $(8 \hea, K \spa,
Q \spa, 2 \dia)$, $(K \spa, 8 \hea, Q \spa, 2 \dia)$, and $(K \spa, 8
\hea, 6 \dia, Q \spa)$.

What the Magician and his Assistant need to perform the trick is a
\emph{matching} for the $X$ vertices.  If they agree in advance on
some matching, then when the audience selects a set of 5 cards, the
Assistant reveals the matching sequence of 4 cards.  The Magician uses
the matching to find the audience's chosen set of 5 cards, and so he
can name the one not already revealed.

For example, suppose the Assistant and Magician agree on a matching
containing the two bold edges in Figure~\ref{fig:11Q9}.  If the
audience selects the set
\begin{equation}\label{8heaKspa}
\set{8 \hea, K \spa, Q \spa, 9 \clu, 6 \dia},
\end{equation}
then the Assistant reveals the corresponding sequence
\begin{equation}\label{Kspa8hea}
(K \spa, 8 \hea, 6 \dia, Q \spa).
\end{equation}
Using the matching, the Magician sees that the hand~\eqref{8heaKspa}
is matched to the sequence~\eqref{Kspa8hea}, so he can name the one
card in the corresponding set not already revealed, namely, the $9
\clu$.  Notice that the fact that the sets are \emph{matched}, that
is, that different sets are paired with \emph{distinct} sequences, is
essential.  For example, if the audience picked the previous
hand~\eqref{2dia6dia}, it would be possible for the Assistant to
reveal the same sequence~\eqref{Kspa8hea}, but he better not do that;
if he did, then the Magician would have no way to tell if the
remaining card was the $9 \clu$ or the $2 \dia$.

So how can we be sure the needed matching can be found?  The answer is
that each vertex on the left has degree $5 \cdot 4! = 120$, since
there are five ways to select the card kept secret and there are $4!$
permutations of the remaining 4 cards.  In addition, each vertex on
the right has degree 48, since there are 48 possibilities for the
fifth card.  So this graph is \emph{\idx{degree-constrained}}
according to Definition~\ref{degree-constrained_def}, and so
has a matching by Theorem~\ref{lem:no_bottleneck_degree_constrained}.

In fact, this reasoning shows that the Magician could still pull off
the trick if 120 cards were left instead of 48, that is, the trick
would work with a deck as large as 124 different cards ---without any
magic!

\subsection{The Real Secret}

But wait a minute!  It's all very well in principle to have the Magician
and his Assistant agree on a matching, but how are they supposed to
remember a matching with $\binom{52}{5} = 2,598,960$ edges?  For the trick
to work in practice, there has to be a way to match hands and card
sequences mentally and on the fly.

%We'll describe how in lecture\dots.  %hide after lecture

%\iffalse %unhide after lecture

We'll describe one approach.  As a running example, suppose that the
audience selects:
\[
10 \hea \quad 9 \dia \quad 3 \hea \quad Q \spa \quad J \dia.
\]

\begin{itemize}

\item The Assistant picks out two cards of the same suit.  In the
example, the assistant might choose the $3 \hea$ and $10 \hea$.  This
is always possible because of the Pigeonhole Principle ---there are
five cards and 4 suits so two cards must be in the same suit.

\item The Assistant locates the ranks of these two cards on the cycle
  shown in Figure~\ref{fig:11Q11}. For any two distinct ranks on this
  cycle, one is always between 1 and 6 hops clockwise from the other.
  For example, the $3 \hea$ is 6 hops clockwise from the $10 \hea$.


\begin{figure}

\graphic{Fig_Q11}

\caption{The 13 card ranks arranged in cyclic order.}

\label{fig:11Q11}

\end{figure}

\item The more counterclockwise of these two cards is revealed first,
and the other becomes the secret card.  Thus, in our example, the $10
\hea$ would be revealed, and the $3 \hea$ would be the secret card.
Therefore:

\begin{itemize}

\item The suit of the secret card is the same as the suit of the first
card revealed.

\item The rank of the secret card is between 1 and 6 hops clockwise
from the rank of the first card revealed.

\end{itemize}

\item All that remains is to communicate a number between 1 and 6.
The Magician and Assistant agree beforehand on an ordering of all the
cards in the deck from smallest to largest such as:
%
\[
A \clu\  A \dia\  A \hea\ A \spa\
2 \clu\  2 \dia\  2 \hea\ 2 \spa\
\dots\ K \hea\ K \spa
\]
%
The order in which the last three cards are revealed communicates the
number according to the following scheme:
%
\[
\begin{array}{r@{\,}ccc@{\,}ll}
(&\text{small},&\text{medium},&\text{large}&) & ${}= 1$ \\
(&\text{small},&\text{large},&\text{medium}&) & ${}= 2$ \\
(&\text{medium},&\text{small},&\text{large}&) & ${}= 3$ \\
(&\text{medium},&\text{large},&\text{small}&) & ${}= 4$ \\
(&\text{large},&\text{small},&\text{medium}&) & ${}= 5$ \\
(&\text{large},&\text{medium},&\text{small}&) & ${}= 6$
\end{array}
\]
%
In the example, the Assistant wants to send 6 and so reveals the
remaining three cards in large, medium, small order.  Here is the
complete sequence that the Magician sees:
%
\[
10 \hea \quad Q \spa \quad J \dia \quad 9 \dia
\]

\item The Magician starts with the first card, $10 \hea$, and hops 6
ranks clockwise to reach $3 \hea$, which is the secret card!

\end{itemize}

So that's how the trick can work with a standard deck of 52 cards.  On
the other hand, Hall's Theorem implies that the Magician and Assistant
can \emph{in principle} perform the trick with a deck of up to 124
cards.  It turns out that there is a method which they could actually
learn to use with a reasonable amount of practice for a 124-card deck,
but we won't explain it here.\footnote{See
  \href{http://courses.csail.mit.edu/6.042/spring11/Kleber-cardTrick.pdf}{\emph{The
      Best Card Trick}} by Michael Kleber for more information.}
%\fi  %unhide after lecture

\iffalse
Also, \emph{Using a Card Trick to Teach Discrete Mathematics}, Simonson,
Shai, Holm, Tara S., Primus: Problems, Resources, and Issues in
Mathematics Undergraduate Studies, Sep 2003.
\fi

\subsection{The Same Trick with Four Cards?}\label{4_card_trick_subsec}

Suppose that the audience selects only \emph{four} cards and the
Assistant reveals a sequence of \emph{three} to the Magician.  Can the
Magician determine the fourth card?

Let $X$ be all the sets of four cards that the audience might select,
and let $Y$ be all the sequences of three cards that the Assistant
might reveal.  Now, on one hand, we have
\[
\card{X} = \binom{52}{4} = 270,725
\]
by the Subset Rule.  On the other hand, we have
\[
\card{Y} = 52 \cdot 51 \cdot 50 = 132,600
\]
by the Generalized Product Rule.  Thus, by the Pigeonhole Principle, the
Assistant must reveal the \emph{same} sequence of three cards for at
least
\[
\ceil{\frac{270,725}{132,600}} = 3
\]
\emph{different} four-card hands.  This is bad news for the Magician:
if he sees that sequence of three, then there are at least three
possibilities for the fourth card which he cannot distinguish.  So there
is no legitimate way for the Assistant to communicate exactly what the
fourth card is!

\begin{editingnotes}

\arm{cut the following temporal spatial subsection by FTL since
done better in problem PS_magic_trick_4cards.}

\subsection{Never Say Never}

No sooner than we finished proving that the Magician can't pull off
the trick showing three cards out of four instead of four cards out of
five, a student showed us a way that it might be doable after all.
The idea is to place the three cards on a table one at a time instead
of revealing them all at once.  This provides the Magician with two
completely independent sequences of three cards: one for the
\emph{temporal} order in which the cards are placed on the table, and
one for the \emph{spatial} order in which they appear once pcr2laced.

For example, suppose the audience selects
\begin{equation*}
    10 \hea \quad 9 \dia \quad 3\hea \quad Q\spa
\end{equation*}
and the assistant decides to reveal
\begin{equation*}
    10 \hea \quad 9 \dia  \quad Q\spa.
\end{equation*}
The assistant might decide to reveal the~$Q\spa$ first, the~$10\hea$
second, and the $9\dia$ third, thereby production the \emph{temporal}
sequence
\begin{equation*}
    (Q\spa, 10\hea, 9\dia).
\end{equation*}
If the $Q\spa$ is placed in the middle position on the table, the
$10\hea$ is placed in the rightmost position on the table, and the
$9\dia$ is placed in the leftmost position on the table, the
\emph{spatial} sequence would be
\begin{equation*}
    (9\dia, Q\spa, 10\hea).
\end{equation*}

In this version of the card trick, $X$~consists of all sets of 4~cards
and $Y$~consists of all \emph{pairs} of sequences of the same 3~cards.  As
before, we can create a bipartite graph where an edge connects a
set~$S$ of 4~cards in~$X$ with a pair of sequences in~$Y$ if the
3~cards in the sequences are in~$S$.

The degree of every node in~$X$ is then
\begin{equation*}
    4 \cdot 3! \cdot 3! = 144
\end{equation*}
since there are 4~choices for which card is not revealed and
$3!$~orders for each sequence in the pair.

The degree of every node in~$Y$ is 49 since there are $52 - 3 = 49$
possible choices for the 4th card.  Since $144 \ge 49$, the graph is
\idx{degree-constrained}, and therefore has a matching for~$X$ by
Theorem~\ref{lem:no_bottleneck_degree_constrained}.
Hence, the magic trick \emph{is} doable with 4~cards ---the assistant
just has to convey more information.  

\arm{ASK FTL WHAT HE MEANS:}

Can you figure out a convenient
way to pull off the trick on the fly?

So what about the 3-card version?  Surely that is not doable.\dots
\end{editingnotes}

%%%%%%%%%%%%%%%%%
\begin{problems}

\practiceproblems
\pinput{TP_Pigeonhole_Principle}

\classproblems
\pinput{CP_pigeon_hole}
\pinput{CP_magic_trick_124_cards}
\pinput{CP_magic_trick_hide_2}


\homeworkproblems
\pinput{PS_pigeon_hunting}
\pinput{PS_pigeonhole-power_of_3}
\pinput{PS_monochromatic_rectangle}
\pinput{PS_magic_trick_4cards}
\end{problems}


%%%%%%%%%%%%%%%%%%%%%%%%%%%%%%%%%%%%%%%%%%%%%%%%%%%%%%%%%%%%%%%%%%%%%%%%%%%%%%%

\section{Inclusion-Exclusion}\label{inc-ex_sec}

How big is a union of sets?  For example, suppose there are 60 math
majors, 200 EECS majors, and 40 physics majors.  How many students are
there in these three departments?  Let $M$ be the set of math majors,
$E$ be the set of EECS majors, and $P$ be the set of physics majors.  In
these terms, we're asking for $\card{M \cup E \cup P}$.

The Sum Rule says that if $M$, $E$, and~$P$ are disjoint, then the sum
of their sizes is
%
\[
\card{M \cup E \cup P} = \card{M} + \card{E} + \card{P} .
\]
%
However, the sets $M$, $E$, and $P$ might \emph{not} be disjoint.  For
example, there might be a student majoring in both math and
physics.  Such a student would be counted twice on the right side of this
equation, once as an element of $M$ and once as an element of $P$.  Worse,
there might be a triple-major\footnote{\dots though not at MIT anymore.}
counted \emph{three} times on the right side!

Our most-complicated counting rule determines the size of a union of
sets that are not necessarily disjoint.  Before we state the rule,
let's build some intuition by considering some easier special cases:
unions of just two or three sets.

\subsection{Union of Two Sets}

For two sets, $S_1$ and $S_2$, the \term{Inclusion-Exclusion Rule} is that the
size of their union is:
\begin{equation}\label{IE2}
\card{S_1 \cup S_2} = \card{S_1} + \card{S_2} - \card{S_1 \cap S_2}
\end{equation}
Intuitively, each element of $S_1$ is accounted for in the first term,
and each element of $S_2$ is accounted for in the second term.
Elements in \emph{both} $S_1$ and $S_2$ are counted
\emph{twice} ---once in the first term and once in the second.  This
double-counting is corrected by the final term.

\subsection{Union of Three Sets}

So how many students are there in the math, EECS, and physics
departments?  In other words, what is $\card{M \cup E \cup P}$ if:
%
\begin{align*}
\card{M} & = 60 \\
\card{E} & = 200 \\
\card{P} & = 40.
\end{align*}
%
The size of a union of three sets is given by a more complicated
\idx{Inclusion-Exclusion} formula:
%
\begin{align*}
\card{S_1 \cup S_2 \cup S_3} & = \card{S_1} + \card{S_2} + \card{S_3} \\
  & \quad - \card{S_1 \cap S_2} - \card{S_1 \cap S_3} - \card{S_2 \cap S_3} \\
  & \quad + \card{S_1 \cap S_2 \cap S_3}.
\end{align*}
%
Remarkably, the expression on the right accounts for each element in the
union of $S_1$, $S_2$, and $S_3$ exactly once.  For example, suppose that
$x$ is an element of all three sets.  Then $x$ is counted three times (by
the $\card{S_1}$, $\card{S_2}$, and $\card{S_3}$ terms), subtracted off
three times (by the $\card{S_1 \cap S_2}$, $\card{S_1 \cap S_3}$, and
$\card{S_2 \cap S_3}$ terms), and then counted once more (by the
$\card{S_1 \cap S_2 \cap S_3}$ term).  The net effect is that $x$ is
counted just once.

If $x$~is in two sets (say, $S_1$ and~$S_2$), then $x$~is counted
twice (by the $\card{S_1}$ and $\card{S_2}$ terms) and subtracted once
(by the $\card{S_1 \intersect S_2}$ term).  In this case, $x$~does not
contribute to any of the other terms, since~$x \notin S_3$.

So we can't answer the original question without knowing the sizes of
the various intersections.  Let's suppose that there are:
%
\begin{center}
\begin{tabular}{cl}
4 & math - EECS double majors \\
3 & math - physics double majors \\
11 & EECS - physics double majors \\
2 & triple majors
\end{tabular}
\end{center}
%
Then $\card{M \cap E} = 4 + 2$, $\card{M \cap P} = 3 + 2$, $\card{E
\cap P} = 11 + 2$, and $\card{M \cap E \cap P} = 2$.  Plugging all this
into the formula gives:
%
\begin{align*}
\card{M \cup E \cup P}
    & = \card{M} + \card{E} + \card{P}
      - \card{M \cap E} - \card{M \cap P} - \card{E \cap P}
      + \card{M \cap E \cap P} \\
    & = 60 + 200 + 40 - 6 - 5 - 13 + 2 \\
    & = 278
\end{align*}

\subsection{Sequences with 42, 04, or 60}

In how many permutations of the set $\set{0, 1, 2, \dots, 9}$ do
either 4 and~2, 0 and~4, or 6 and~0 appear consecutively?  For
example, none of these pairs appears in:
%
\[
(7, 2, 9, 5, 4, 1, 3, 8, 0, 6).
\]
%
The 06 at the end doesn't count; we need 60.  On the other hand, both
04 and 60 appear consecutively in this permutation:
%
\[
(7, 2, 5, \underline{6}, \underline{0}, \underline{4}, 3, 8, 1, 9).
\]
%
Let $P_{42}$ be the set of all permutations in which 42 appears.
Define $P_{60}$ and $P_{04}$ similarly.  Thus, for example, the
permutation above is contained in both $P_{60}$ and $P_{04}$, but
not~$P_{42}$.  In these terms, we're looking for the size of the set
$P_{42} \cup P_{04} \cup P_{60}$.

First, we must determine the sizes of the individual sets, such as
$P_{60}$.  We can use a trick: group the 6 and 0 together as a single
symbol.  Then there is an immediate bijection between permutations of
$\set{0, 1, 2, \dots 9}$ containing 6 and 0 consecutively and
permutations of:
%
\[
\set{60, 1, 2, 3, 4, 5, 7, 8, 9}.
\]
%
For example, the following two sequences correspond:
%
\[
(7, 2, 5, \underline{6}, \underline{0}, 4, 3, 8, 1, 9)
\; \longleftrightarrow \;
(7, 2, 5, \underline{60}, 4, 3, 8, 1, 9).
\]
%
There are $9!$ permutations of the set containing 60, so
$\card{P_{60}} = 9!$ by the Bijection Rule.  Similarly, $\card{P_{04}}
= \card{P_{42}} = 9!$ as well.

Next, we must determine the sizes of the two-way intersections, such
as $P_{42} \cap P_{60}$.  Using the grouping trick again, there is a
bijection with permutations of the set:
%
\[
\set{42, 60, 1, 3, 5, 7, 8, 9}.
\]
%
Thus, $\card{P_{42} \cap P_{60}} = 8!$.  Similarly, $\card{P_{60} \cap
P_{04}} = 8!$ by a bijection with the set:
%
\[
\set{604, 1, 2, 3, 5, 7, 8, 9}.
\]
%
And $\card{P_{42} \cap P_{04}} = 8!$ as well by a similar argument.
Finally, note that $\card{P_{60} \cap P_{04} \cap P_{42}} = 7!$ by a
bijection with the set:
%
\[
\set{6042, 1, 3, 5, 7, 8, 9}.
\]

Plugging all this into the formula gives:
%
\begin{align*}
\card{P_{42} \cup P_{04} \cup P_{60}}
    & = 9! + 9! + 9! - 8! - 8! - 8! + 7!.
\end{align*}

\subsection{Union of $n$ Sets}

The size of a union of $n$ sets is given by the following rule.

\begin{rul}[\idx{Inclusion-Exclusion}]
\[
\card{S_1 \cup S_2 \cup \cdots \cup S_n} =
\]
%
\centerline{\begin{tabular}[t]{rl}
 & the sum of the sizes of the individual sets \\
\emph{minus} & the sizes of all two-way intersections \\
\emph{plus} & the sizes of all three-way intersections \\
\emph{minus} & the sizes of all four-way intersections \\
\emph{plus} & the sizes of all five-way intersections, etc.
\end{tabular}}
\end{rul}

The formulas for unions of two and three sets are special cases of this
general rule.

This way of expressing Inclusion-Exclusion is easy to understand and
nearly as precise as expressing it in mathematical symbols, but we'll need
the symbolic version below, so let's work on deciphering it now.

We already have a concise notation for the sum of sizes of the
individual sets, namely,
\[
\sum_{i=1}^n \card{S_i}.
\]
A ``two-way intersection'' is a set of the form $S_i \intersect S_j$ for
$i \neq j$.  We regard $S_j \intersect S_i$ as the same two-way
intersection as $S_i \intersect S_j$, so we can assume that $i < j$.  Now
we can express the sum of the sizes of the two-way intersections as
\[
\sum_{1\leq i < j \leq n} \card{S_i \intersect S_j}.
\]
Similarly, the sum of the sizes of the three-way intersections is
\[
\sum_{1\leq i < j < k \leq n} \card{S_i \intersect S_j \intersect S_k}.
\]
These sums have alternating signs in the Inclusion-Exclusion formula, with
the sum of the $k$-way intersections getting the sign $(-1)^{k-1}$.  This
finally leads to a symbolic version of the rule:

\begin{rul*}[Inclusion-Exclusion]
\begin{align*}
\Card{\lgunion_{i=1}^n S_i}
   = & \sum_{i=1}^n \card{S_i}\\
     & - \sum_{1\leq i < j \leq n} \card{S_i \intersect S_j}\\
     &  + \sum_{1\leq i < j < k \leq n} \card{S_i \intersect S_j
       \intersect S_k} + \cdots\\
     & + (-1)^{n-1} \Card{\lgintersect_{i=1}^n S_i}.
\end{align*}
\end{rul*}

While it's often handy express the rule in this way as a sum of sums,
it is not necessary to group the terms by how many sets are in the
intersections.  So another way to state the rule is:

\begin{rul*}[Inclusion-Exclusion-II]
\[
\Card{\lgunion_{i=1}^n S_i}
   =  \sum_{\emptyset \neq I \subseteq \set{1,\dots,n}} (-1)^{\card{I}+1} \Card{\lgintersect_{i \in I} S_i}
\]
\end{rul*}

A proof of these rules using just highschool algebra is given in
Problem~\ref{CP_inclusion-exclusion_algebra_proof}.


\begin{editingnotes}

\arm{Covered in Problem~\ref{PS_inclusion-exclusion_primes}}

\subsection{Counting Primes}

How many of the numbers $1, 2, \dots, 100$ are \idx{prime}?  One way to
answer this question is to test each number up to 100 for primality and
keep a count.  This requires considerable effort.  (Is 57 prime?  How
about 67?)

Another approach is to use the \idx{Inclusion-Exclusion Principle}.  This
requires one trick: to determine the number of primes, we will first count
the number of \emph{non-primes}.  By the Sum Rule, we can then find the
number of primes by subtraction from 100.  This trick of ``counting the
complement'' is a good one to remember.

\subsubsection{Reduction to a Union of Four Sets}

The set of non-primes in the range $1, \dots, 100$ consists of the set,
$C$, of composite numbers in this range: $4, 6, 8, 9, \dots, 99, 100$ and
the number 1, which is neither prime nor composite.  The main job is to
determine the size of the set $C$ of composite numbers.  For this purpose,
define $A_m$ to be the set of numbers in the range $m+1, \dots, 100$ that
are divisible by $m$:
\[
A_m \eqdef \set{x \leq 100 \suchthat x > m \text{ and } \paren{m \divides x}}
\]

For example, $A_2$ is all the even numbers from 4 to 100.  The following
Lemma will now allow us to compute the cardinality of $C$ by using
Inclusion-Exclusion for the union of four sets:

\begin{lemma}
\[
C = A_2 \cup A_3 \cup A_5 \cup A_7.
\]
\end{lemma}

\begin{proof}
We prove the two sets equal by showing that each contains the other.

To show that $A_2 \cup A_3 \cup A_5 \cup A_7 \subseteq C$, let $n$ be an
element of $A_2 \cup A_3 \cup A_5 \cup A_7$.  Then $n \in A_m$ for $m = 2,
3, 5$ or $7$.  This implies that $n$ is in the range $1, \dots, 100$ and
is composite because it has $m$ as a factor.  That is, $n \in C$.

Conversely, to show that $C \subseteq A_2 \cup A_3 \cup A_5 \cup A_7$, let
$n$ be an element of $C$.  Then $n$ is a composite number in the range $1,
\dots, 100$.  This means that $n$ has at least two prime factors.  Now if
both prime factors are $> 10$, then their product would be a number $>
100$ which divided $n$, contradicting the fact that $n<100$.  So $n$ must
have a prime factor $\leq 10$.  But 2, 3, 5, and 7 are the only primes
$\leq 10$.  This means that $n$ is an element of $A_2$, $A_3$, $A_5$, or
$A_7$, and so $n \in A_2 \cup A_3 \cup A_5 \cup A_7$.
\end{proof}

\subsubsection{Computing the Cardinality of the Union}

Now it's easy to find the cardinality of each set $A_m$: every $m$th
integer is divisible by $m$, so the number of integers in the range $1,
\dots, 100$ that are divisible by $m$ is simply $\floor{100/m}$.  So
\[
\card{A_m} = \floor{\frac{100}{m}} - 1,
\]
where the $-1$ arises because we defined $A_m$ to exclude $m$ itself.
This formula gives:

\begin{eqnarray*}
\card{A_2} & = \lfloor\frac{100}{2}\rfloor - 1 = & 49 \\
\card{A_3} & = \lfloor\frac{100}{3}\rfloor - 1 = & 32 \\
\card{A_5} & = \lfloor\frac{100}{5}\rfloor - 1 = & 19 \\
\card{A_7} & = \lfloor\frac{100}{7}\rfloor - 1 = & 13
\end{eqnarray*}

Notice that these sets $A_2$, $A_3$, $A_5,$ and $A_7$ are not disjoint.
For example, 6 is in both $A_2$ and $A_3$.  Since the sets intersect, we
must use the Inclusion-Exclusion Principle:

\begin{align*}
\card{C}
  = & \card{A_2 \cup A_3 \cup A_5 \cup A_7} \\
  = & \card{A_2} + \card{A_3} + \card{A_5} + \card{A_7} \\
    & - \card{A_2 \cap A_3} - \card{A_2 \cap A_5} - \card{A_2 \cap A_7}
             - \card{A_3 \cap A_5} - \card{A_3 \cap A_7} - \card{A_5 \cap A_7} \\
    & + \card{A_2 \cap A_3 \cap A_5} + \card{A_2 \cap A_3 \cap A_7}
             + \card{A_2 \cap A_5 \cap A_7} + \card{A_3 \cap A_5 \cap A_7} \\
    & - \card{A_2 \cap A_3 \cap A_5 \cap A_7}
\end{align*}

There are a lot of terms here!  Fortunately, all of them are easy to
evaluate.  For example, $\card{A_3 \cap A_7}$ is the number of multiples
of $3 \cdot 7 = 21$ in the range 1 to 100, which is $\floor{100/21} = 4$.
\iffalse (Note that there is no reason to subtract 1 as we did when
evaluating $\card{A_m}$ above.)\fi Substituting such values for all of the
terms above gives:
\begin{align*}
\card{C}  = & 49 + 32 + 19 + 13 \\
            & - 16 - 10 - 7 - 6 - 4 - 2 \\
            & + 3 + 2 + 1 + 0 \\
            & - 0 \\
          = & 74
\end{align*}

This calculation shows that there are 74 composite numbers in the
range 1 to 100.  Since the number 1 is neither composite nor prime,
there are $100 - 74 - 1 = 25$ primes in this range.

At this point it may seem that checking each number from 1 to 100 for
primality and keeping a count of primes might have been easier than using
Inclusion-Exclusion.  However, the Inclusion-Exclusion approach used here
is asymptotically faster as the range of numbers grows large.

\textcolor{blue}{The naive strategy requires $n$ runs of a primality
  test if the upper bound is $n$.  The Inclusion-Exclusion approach
  seems to require summing an immense number of terms, but fewer than
  $n$ of these are non-zero and the remaining zero terms obviously
  don't matter.}
\end{editingnotes}


\begin{editingnotes}
REVISE THIS WHOLE SECTION using secretary/characteristic functions as in ARM
Lecture and \ref{CP_inclusion-exclusion_algebra_proof}:

ppart Most high school students will get freaked by the following formula,
even though they actually know the rule it expresses.  How would you
explain it to them?

\begin{equation}\label{1-xprodedn}
\prod_{i=1}^n \paren{1-x_i} = \sum_{I \subseteq \set{1,\dots,n}} (-1)^{\card{I}}\prod_{j \in I}x_j.
\end{equation}
\hint Show them an example.

\begin{solution}
Let's do an example.  To ``multiply out''
\begin{equation}%\label{x3prod}
(1-x_1)(1-x_2)(1-x_3),
\end{equation}
you would form \emph{\idx{monomial}} products by selecting some of the
$(-x_i)$'s to multiply together.  For example, selecting $(-x_i)$'s with
\begin{itemize}
\item $i \in \set{1,3}$ leads to the monomial $(-x_1)(-x_3) = (-1)^2x_1x_3
  = x_1x_3$,
%\item $i \in \set{2}$ leads to the monomial $x_2$,
\item $i \in \set{1,2,3}$ leads to the monomial $(-x_1)(-x_2)(-x_3)= (-1)^3x_1x_2x_3 = -x_1x_2x_3$, and
\item $i \in \emptyset$ leads (by convention) to the monomial $1$.
\end{itemize}
Then you sum up the monomials from \emph{all possible} selections to get
\[
(1-x_1)(1-x_2)(1-x_3) = 1 - x_1 - x_2 - x_3 + x_1x_2 + x_1x_3 + x_2x_3 - x_1x_2x_3.
\]

Now we can decipher~\eqref{1-xprodedn} as saying to do the same thing for
the product of $n$ different $(1-x_i)$'s:  for any selection of $(-x_i)$'s
with $i$ in some subset, $I \subseteq \set{1,\dots,n}$, multiply the
$(-x_i)$'s to get the monomial
\[
\prod_{i \in I} (-x_i) = \prod_{i \in I} (-1)^{\card{I}} x_i,
\]
and sum up all such monomials obtained by every possible selection, $I$,
to get the right hand side of equation~\eqref{1-xprodedn}.
\end{solution}

\end{editingnotes}

\subsection{Computing Euler's Function}
As an example, let's use Inclusion-Exclusion to derive an explicit
formula~\eqref{inex-phi} for Euler's function, $\phi(n)$.  By
definition, $\phi(n)$ is the number of nonnegative integers less than
a positive integer $n$ that are relatively prime to $n$.  But the
set~$S$ of nonnegative integers less than~$n$ that are \emph{not}
relatively prime to $n$ will be easier to count.

Suppose the prime factorization of $n$ is $p_1^{e_1}\cdots p_m^{e_m}$
for distinct primes $p_i$.  This means that the integers in $S$ are
precisely the nonnegative integers less than $n$ that are divisible by at
least one of the $p_i$'s.  Letting $C_a$~be the set of nonnegative
integers less than $n$ that are divisible by $a$, we have
\[
S = \lgunion_{i=1}^m C_{p_i}.
\]

We'll be able to find the size of this union using Inclusion-Exclusion
because the intersections of the $C_p$'s are easy to count.  For example,
$C_p \intersect C_q \intersect C_r$ is the set of nonnegative integers less
than $n$ that are divisible by each of $p$, $q$ and $r$.  But since
the $p,q,r$ are distinct primes, being divisible by each of them is
the same as being divisible by their product.  Now observe that if $k$ is
a positive divisor of $n$, then exactly $n/k$ nonnegative integers less
than $n$ are divisible by $k$, namely, $0,k,2k,\dots,((n/k)-1)k$.  So
exactly $n/pqr$ nonnegative integers less than $n$ are divisible by
all three primes $p$, $q$, $r$.  In other words,
\[
\card{C_p \intersect C_q \intersect C_r} = \frac{n}{pqr}.
\]

Reasoning this way about all the intersections among the $C_p$'s and
applying Inclusion-Exclusion, we get
\begin{editingnotes}
Revise using:
\[
\prod_{i=1}^n (1-a_i) = \sum_{I\subset [1,n]} \prod_{j\in I} -a_j,
\]
and
\[
\card{\lgintersect_{j \in I} C_{p_j}} = \frac{n}{\prod_{j \in I} {p_j}},
\]
so
\begin{align*}
\prod_{i=1}^n \paren{1-\frac{1}{p_i}}
 & = \sum_{I\subset [1,n]} (-1)^{\card{I}}\prod_{j \in I} \frac{1}{p_j}\\
 & = \sum_{I\subset [1,n]} (-1)^{\card{I}}\frac{\card{\lgintersect_{j \in I} C_{p_j}}}{n}
\end{align*}
where the empty sum is 0, the empty product is 1, and the empty intersection of sets $A_1,\dots,A_n$ is
$\lgunion_{i=1}^n A_i$.
\end{editingnotes}

\begin{align*}
\card{S}
  & = \Card{\lgunion_{i=1}^m C_{p_i}}\\
  & = \sum_{i=1}^m \Card{C_{p_i}} - \sum_{1\leq i < j \leq m} \Card{C_{p_i} \intersect C_{p_j}}\\
  &\qquad  + \sum_{1\leq i < j < k \leq m} \Card{C_{p_i} \intersect C_{p_j} \intersect C_{p_k}} -
       \cdots + (-1)^{m-1} \Card{\lgintersect_{i=1}^m C_{p_i}}\\
  & = \sum_{i=1}^m \frac{n}{p_i} -
      \sum_{1\leq i < j \leq m} \frac{n}{p_ip_j}\\
  &\qquad  + \sum_{1\leq i < j < k \leq m} \frac{n}{p_ip_jp_k} -
       \cdots + (-1)^{m-1} \frac{n}{p_1p_2\cdots p_n}\\
  & = n\paren{\sum_{i=1}^m \frac{1}{p_i} -
      \sum_{1\leq i < j \leq m} \frac{1}{p_ip_j}
       + \sum_{1\leq i < j < k \leq m} \frac{1}{p_ip_jp_k} - \cdots
        + (-1)^{m-1} \frac{1}{p_1p_2\cdots p_n}}
\end{align*}
But $\phi(n)=n-\card{S}$ by definition, so
\begin{align}
  \phi(n) & = n\paren{1 - \sum_{i=1}^m \frac{1}{p_i} +  \sum_{1\leq i < j \leq m} \frac{1}{p_ip_j}
    - \sum_{1\leq i < j < k \leq m} \frac{1}{p_ip_jp_k} + \cdots
    + (-1)^m\frac{1}{p_1p_2\cdots p_n}}\notag\\
          &  = n \prod_{i=1}^m \paren{1 - \frac{1}{p_i}}.\label{inex-phi}
\end{align}

Yikes!  That was pretty hairy.  Are you getting tired of all that
nasty algebra?  If so, then good news is on the way.  In the next
section, we will show you how to prove some heavy-duty formulas
without using any algebra at all.  Just a few words and you are done.
No kidding.

\begin{problems}
\practiceproblems
\pinput{TP_Inclusion_Exclusion}
\pinput{MQ_sick_days}

\classproblems
\pinput{CP_inclusion-exclusion_passwords}
\pinput{CP_inclusion-exclusion_paths}
\pinput{CP_inclusion-exclusion_algebra_proof}

\homeworkproblems
\pinput{PS_path_counting}
\pinput{PS_derangement}
\pinput{PS_inclusion-exclusion_primes}

\examproblems
\pinput{FP_string_inclusion_exclusion}
\pinput{FP_count_lineup}

\end{problems}


\section{Combinatorial Proofs}\label{combinatorial_proof_sec}

Suppose you have $n$ different T-shirts, but only want to keep $k$.
You could equally well select the $k$ shirts you want to keep or
select the complementary set of $n - k$ shirts you want to throw out.
Thus, the number of ways to select $k$ shirts from among $n$ must be
equal to the number of ways to select $n - k$ shirts from among $n$.
Therefore:
%
\[
    \binom{n}{k} = \binom{n}{n-k}.
\]
%
This is easy to prove algebraically, since both sides are equal to:
%
\[
    \frac{n!}{k!\ (n-k)!}.
\]
%
But we didn't really have to resort to algebra; we just used counting
principles.

Hmmm.\dots

\subsection{Pascal's Identity}

\Jay, famed Math for Computer Science Teaching Assistant, has decided
to try out for the US Olympic boxing team.  After all, he's watched
all of the \emph{Rocky} movies and spent hours in front of a mirror
sneering, ``Yo, you wanna piece a' \emph{me}?!''  \Jay\ figures that $n$
people (including himself) are competing for spots on the team and
only $k$ will be selected.  As part of maneuvering for a spot on the
team, he needs to work out how many different teams are possible.
There are two cases to consider:

\begin{itemize}

\item \Jay\ \emph{is} selected for the team, and his $k - 1$
  teammates are selected from among the other $n - 1$ competitors.
  The number of different teams that can be formed in this way is:
%
\[
    \binom{n-1}{k-1}.
\]

\item \Jay\ is \emph{not} selected for the team, and all $k$ team
members are selected from among the other $n - 1$ competitors.  The
number of teams that can be formed this way is:
%
\[
    \binom{n - 1}{k}.
\]

\end{itemize}

All teams of the first type contain \Jay, and no team of the second
type does; therefore, the two sets of teams are disjoint.  Thus, by
the Sum Rule, the total number of possible Olympic boxing teams is:
%
\[
    \binom{n-1}{k-1} + \binom{n - 1}{k}.
\]

\Jer, equally-famed Teaching Assistant, thinks \Jay\ isn't so tough
and so he might as well also try out.  He reasons that $n$ people
(including himself) are trying out for $k$ spots.  Thus, the number of
ways to select the team is simply:
%
\[
    \binom{n}{k}.
\]

\Jer\ and \Jay\ each correctly counted the number of possible boxing
teams.  Thus, their answers must be equal.  So we know:
%
\begin{lemma}[Pascal's Identity]
\begin{equation}\label{pascal-ident}
    \binom{n}{k} = \binom{n-1}{k-1} + \binom{n - 1}{k}.
\end{equation}
\end{lemma}
%
This is called \term{Pascal's Identity}.  And we proved it
\emph{without any algebra!}  Instead, we relied purely on counting
techniques.


\subsection{Giving a Combinatorial Proof}

A \term{combinatorial proof} is an argument that establishes an
algebraic fact by relying on counting principles.  Many such proofs
follow the same basic outline:
%
\begin{enumerate}

\item Define a set $S$.

\item Show that $\card{S} = n$ by counting one way.

\item Show that $\card{S} = m$ by counting another way.

\item Conclude that $n = m$.

\end{enumerate}
%
In the preceding example, $S$ was the set of all possible Olympic boxing
teams.  \Jay\ computed
\[
\card{S} = \binom{n-1}{k-1} + \binom{n-1}{k}
\]
by counting one way, and \Jer\ computed
\[
\card{S} = \binom{n}{k}
\]
by counting another way.  Equating these two expressions gave Pascal's
Identity.

\subsubsection{Checking a Combinatorial Proof}

Combinatorial proofs are based on counting the same thing in different
ways.  This is fine when you've become practiced at different counting
methods, but when in doubt, you can fall back on bijections and
sequence counting to check such proofs.

For example, let's take a closer look at the combinatorial proof of
Pascal's Identity~\eqref{pascal-ident}.  In this case, the set $S$ of
things to be counted is the collection of all size-$k$ subsets of
integers in the interval $[1,n]$.

Now we've already counted $S$ one way, via the Bookkeeper Rule, and
found $\card{S} = \binom{n}{k}$.  The other ``way'' corresponds to
defining a bijection between $S$ and the disjoint union of two sets
$A$ and $B$ where,
\begin{align*}
A & \eqdef \set{(1,X) \suchthat X \subseteq [2,n] \QAND\ \card{X}=k-1}\\
B & \eqdef \set{(0,Y) \suchthat Y \subseteq [2,n] \QAND\ \card{Y}=k}.
\end{align*}
Clearly $A$ and $B$ are disjoint since the pairs in the two sets have
different first coordinates, so $\card{A \union B} = \card{A} +
\card{B}$.  Also,
\begin{align*}
\card{A} & = \text{\# specified sets $X$} =  \binom{n-1}{k-1},\\
\card{B} & = \text{\# specified sets $Y$} = \binom{n-1}{k}.
\end{align*}
Now finding a bijection $f:(A \union B) \to S$ will prove the
identity~\eqref{pascal-ident}.  In particular, we can define
\[
f(c) \eqdef \begin{cases} X \union \set{1} &\text{ if } c = (1,X),\\
                          Y  &\text{ if } c = (0,Y).
\end{cases}
\]
It should be obvious that $f$ is a bijection.

\subsection{A Colorful Combinatorial Proof} 

The set that gets counted in a combinatorial proof in different ways
is usually defined in terms of simple sequences or sets rather than an
elaborate story about Teaching Assistants.  Here is another colorful
example of a combinatorial argument.

\begin{theorem}
\label{th:comb-ex}
\[
\sum_{r=0}^n \binom{n}{r} \binom{2n}{n-r} = \binom{3n}{n}
\]
\end{theorem}

\begin{proof}
We give a combinatorial proof.  Let $S$ be all $n$-card hands that can
be dealt from a deck containing $n$ different red cards and $2n$
different black cards.  First, note that every $3n$-element set has
%
\[
\card{S} = \binom{3n}{n}
\]
%
$n$-element subsets.

From another perspective, the number of hands with exactly $r$ red
cards is
%
\[
\binom{n}{r} \binom{2n}{n - r}
\]
%
since there are $\binom{n}{r}$ ways to choose the $r$ red cards and
$\binom{2n}{n - r}$ ways to choose the $n - r$ black cards.  Since the
number of red cards can be anywhere from 0 to $n$, the total number of
$n$-card hands is:
%
\[
    \card{S} = \sum_{r=0}^n \binom{n}{r} \binom{2n}{n-r}.
\]
%
Equating these two expressions for $\card{S}$ proves the theorem.
\end{proof}

\subsubsection{Finding a Combinatorial Proof}

Combinatorial proofs are almost magical.  Theorem~\ref{th:comb-ex}
looks pretty scary, but we proved it without any algebraic
manipulations at all.  The key to constructing a combinatorial proof
is choosing the set $S$ properly, which can be tricky.  Generally, the
simpler side of the equation should provide some guidance.  For
example, the right side of Theorem~\ref{th:comb-ex} is
$\binom{3n}{n}$, which suggests that it will be helpful to choose~$S$
to be all $n$-element subsets of some $3n$-element set.

\begin{problems}
\classproblems
\pinput{CP_multinomial_theorem}
\pinput{CP_com_proof}
\pinput{CP_bit_string}


\homeworkproblems
\pinput{CP_combination_identity}
\pinput{PS_combinatorial_proof}
\pinput{PS_multinomial_theorem}
\pinput{CP_startup}    

\end{problems}


\endinput
