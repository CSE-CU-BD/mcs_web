\chapter{Number Theory}\label{number_theory_chap}

\emph{Number theory}\index{number theory} is the study of the
integers.  \emph{Why} anyone would want to study the integers may not
be obvious.  First of all, what's to know?  There's 0, there's 1, 2,
3, and so on, and, oh yeah, -1, -2, \dots.  Which one don't you
understand?  What practical value is there in it?

\iffalse Number theory is at the core of mathematics; even Ug the
Caveman surely had some grasp of the integers---at least the positive
ones.  In fact, the integers are so elementary that one might ask,
``What's to study?''  There's 0, there's 1, 2, 3 and so on, and
there's the negatives.  Which one don't you understand?  Doesn't math
become easy when we don't have to worry about nasty numbers like
$\sqrt{7}$, $1 / \pi$ and $i$?  We can even forget about fractions!
\fi

The mathematician \index{Hardy, G. H.}{G. H. Hardy} delighted 
at its impracticality.  He wrote:

 \begin{quotation}
 \noindent [Number theorists] may be justified in rejoicing that there
 is one science, at any rate, and that their own, whose very
 remoteness from ordinary human activities should keep it gentle and
 clean.
 \end{quotation}

Hardy was especially concerned that number theory not be used in
warfare; he was a pacifist.  You may applaud his sentiments, but he
got it wrong: number theory underlies modern cryptography, which is
what makes secure online communication possible.  Secure communication
is of course crucial in war---leaving poor Hardy spinning in
his grave.  It's also central to online commerce.  Every time you buy
a book from Amazon, use a certificate to access a web page, or use a
PayPal account, you are relying on number theoretic algorithms.

Number theory also provides an excellent environment for us to
practice and apply the proof techniques that we developed in previous
chapters.  We'll work out properties of \idx{greatest common divisor}s
(gcd's) and use them to prove that integers factor uniquely into
primes.  Then we'll introduce modular arithmetic and work out enough
of its properties to explain the \idx{RSA} public key crypto-system.

%~\ref{templates_chap} and~\ref{induction_chap}.

Since we'll be focusing on properties of the integers, we'll adopt the
default convention in this chapter that \emph{variables range over
  the set $\integers$ of integers}.

\section{Divisibility}\label{divisibility_sec}

The nature of number theory emerges as soon as we consider the \emph{divides}%
\index{divisibility|textbf} 
relation.
\begin{definition}\label{divides_def}
$a$ \emph{divides} $b$ (notation $a \divides b$) iff there is an integer $k$ such that
\[
ak = b.
\]
\end{definition}
The divides relation comes up so frequently that multiple synonyms for
it are used all the time.  The following phrases all say the same
thing:
\begin{itemize}
\item $a \divides b$,
\item $a$ divides $b$,
\item $a$ is a \emph{divisor} of $b$,
\item $a$ is a \emph{factor} of $b$,
\item $b$ is \emph{divisible} by $a$,
\item $b$ is a \emph{multiple} of $a$.
\end{itemize}
Some immediate consequences of Definition~\ref{divides_def} are that for
all $n$
\[
n  \divides 0,\qquad\qquad
n  \divides n, \text{ and}\qquad\qquad
\pm1  \divides n.
\]
Also,
\[
0 \divides n\ \QIMP\ n = 0.
\]

Dividing seems simple enough, but let's play with this definition.  The Pythagoreans, an
ancient sect of mathematical mystics, said that a number is \emph{perfect}%
\index{perfect number}
if it equals the sum of its positive 
integral divisors, excluding itself.  For example, $6 = 1 + 2 + 3$ 
and $28 = 1 + 2 + 4 + 7 + 14$ are perfect numbers.
On the other hand, 10 is not perfect because 
$1 + 2 + 5 = 8$, and 12 is not perfect because
$1 + 2 + 3 + 4 + 6 = 16$.  \idx{Euclid} characterized all the 
\emph{even} perfect numbers
around 300 BC (Problem~\ref{CP_perfect_numbers}).  But is there an \emph{odd} perfect
number?  More than two thousand years later, we still don't know!  
All numbers up to about
$10^{300}$ have been ruled out, but no one has proved that there isn't an odd perfect
number waiting just over the horizon.

So a half-page into number theory, we've strayed past the outer limits
of human knowledge.  This is pretty typical; number theory is full of
questions that are easy to pose, but incredibly difficult to answer.
We'll mention a few more such questions in later
sections.\footnote{\emph{Don't Panic}---we're going to stick to some
  relatively benign parts of number theory.  These super-hard unsolved
  problems rarely get put on problem sets.}

\subsection{Facts about Divisibility}

The following lemma collects some basic facts about divisibility.

\begin{lemma}\label{lem:div}\mbox{}
\begin{enumerate}
%\item If $a \divides b$, then $a \divides bc$ for all $c$.

\item\label{lem:divtrans} If $a \divides b$ and $b \divides c$, then $a \divides c$.

\item\label{lem:divsbtc} If $a \divides b$ and $a \divides c$, then $a \divides sb + tc$
  for all $s$ and $t$.

\item\label{lem:divcancel} For all $c \neq 0$, $a \divides b$ if and only if $ca \divides
  cb$.
\end{enumerate}
\end{lemma}

\begin{proof}
These facts all follow directly from Definition~\ref{divides_def}.  To
illustrate this, we'll prove just part~\ref{lem:divsbtc}:

Given that $a \divides b$, there is some $k_1 \in \integers$ such that $a k_1 = b$.
Likewise, $a k_2 = c$, so
\[
sb+tc= s(k_1a) + t(k_2a) = (sk_1+tk_2)a.
\]
Therefore $sb+tc = k_3a$ where $k_3 \eqdef (sk_1+tk_2)$, which means that
\[
a \divides sb+tc.
\]
\end{proof}

A number of the form $sb+tc$ is called an \emph{integer linear
  combination}%
\index{linear combination (integers)|textbf} 
of $b$ and $c$, or, since in this chapter we're only
talking about integers, just a \emph{linear combination}.  So
Lemma~\ref{lem:div}.\ref{lem:divsbtc} can be rephrased as
\begin{quote}
If $a$ divides $b$ and $c$, then $a$ divides every linear combination
of $b$ and~$c$.
\end{quote}
We'll be making good use of linear combinations, so let's get the
general definition on record:
\begin{definition}\label{linear_def}
An integer $n$ is a \emph{linear combination} of numbers $b_0,\dots,b_k$ iff
\[
n = s_0b_0+s_1b_1+\cdots+s_kb_k
\]
for some integers $s_0,\dots,s_k$.
\end{definition}

\subsection{When Divisibility Goes Bad}

As you learned in elementary school, if one number does \emph{not}
evenly divide another, you get a ``quotient'' and a ``remainder'' left
over.  More precisely:
\begin{theorem}\label{division_thm}[\idx{Division Theorem}]%
\footnote{This theorem is often called the ``Division Algorithm,'' but
  we prefer to call it a theorem since it does not actually describe
  a division procedure for computing the quotient and remainder.}  Let
$n$ and $d$ be integers such that $d \neq 0$.  Then there exists a unique
pair of integers $q$ and $r$, such that
\begin{equation}\label{nqdr}
n = q \cdot d + r \QAND\ 0 \leq r < \abs{d}.
\end{equation}
\end{theorem}
The number $q$ is called the \term{quotient} and the number $r$ is
called the \term{remainder} of $n$ divided by $d$.  We use the
notation $\qcnt{n}{d}$ for the quotient and $\rem{n}{d}$ for the
remainder.

The absolute value notation $\abs{d}$ used above is probably familiar
from introductory calculus, but for the record, let's define it.
\begin{definition}
For any real number $r$, the \term{absolute value} $\abs{r}$ of $r$
is:\footnote{The absolute value of $r$ could be defined as
  $\sqrt{r^2}$, which works because of the convention that square root
  notation always refers to the \emph{nonnegative} square root (see
  Problem~\ref{CP_bogus_1eqminus1_proof}).  Absolute value generalizes
  to complex numbers where it is called the \term{norm}.  For $a,b \in
  \reals$,
\[
\abs{a+bi} \eqdef \sqrt{a^2+b^2}.
\]}
\[
\abs{r} \eqdef \begin{cases}
                \ r & \text{if $r \geq 0$},\\
                -r & \text{if $r < 0$}.
\end{cases}
\]
\end{definition}

So by definition, the \emph{remainder $\rem{n}{d}$ is nonnegative}
regardless of the sign of $n$ and $d$.  For example, $\rem{-11}{7} =
3$, since $-11 = (-2) \cdot 7 + 3$.

``Remainder'' operations built into many programming languages can be
a source of confusion.  For example, the expression ``32~\%~5'' will
be familiar to programmers in Java, C, and C++; it evaluates to
$\rem{32}{5} =2$ in all three languages.  On the other hand, these and
other languages are inconsistent in how they treat remainders like
``32~\%~-5'' or ``-32~\%~5'' that involve negative numbers.  So don't
be distracted by your familiar programming language's behavior on
remainders, and stick to the mathematical convention that
\emph{remainders are nonnegative}.

The remainder on division by $d$ by definition is a number in the
(integer) \emph{interval} from 0 to $\abs{d}-1$.  Such integer
intervals come up so often that it is useful to have a simple notation
for them.  For $k \leq n \in \integers$,
\begin{align*}
\Zintvoo{k}{n} & \eqdef\quad \set{i \suchthat k < i < n},\\
\Zintvoc{k}{n} & \eqdef\quad (k,n) \union \set{n},\\ 
\Zintvco{k}{n} & \eqdef\quad \set{k} \union (k,n),\\
\Zintv{k}{n}   & \eqdef\quad \set{k} \union (k,n) \union \set{n} = \set{i \suchthat k \leq i \leq n}.
\end{align*}

\subsection{Die Hard}

\emph{Die Hard~3} is just a B-grade action movie, but we think it has
an inner message: everyone should learn at least a little number
theory.  In Section~\ref{diehard_machine}, we formalized a state
machine for the \idx{Die Hard} jug-filling problem using 3 and 5
gallon jugs, and also with 3 and 9 gallon jugs, and came to different
conclusions about bomb explosions.  What's going on in general?  For
example, how about getting 4 gallons from 12- and 18-gallon jugs,
getting 32 gallons with 899- and 1147-gallon jugs, or getting 3
gallons into a jug using just 21- and 26-gallon jugs?

\begin{editingnotes}
Unfortunately, Hollywood never lets go of a gimmick.  Although there
were no water jug tests in \emph{Die Hard~4: Live Free or Die Hard},
rumor has it that the jugs will return in future sequels:
\begin{description}

\item[Die Hard~5: Die Hardest] Bruce goes on vacation and---shockingly---happens into a
  terrorist plot.  To save the day, he must make 3~gallons using 21- and 26-gallon jugs.

\item[Die Hard~6: Die of Old Age] Bruce must save his assisted living facility from a
  criminal mastermind by forming 2 gallons with 899- and 1147-gallon jugs.

\item[Die Hard~7: Die Once and For All] Bruce has to make 4 gallons using 3- and 6-gallon
  jugs.

\end{description}
\end{editingnotes}

It would be nice if we could solve all these silly water jug questions
at once.  \iffalse In particular, how can one form $g$ gallons using
jugs with capacities $a$ and~$b$?\fi This is where number theory comes
in handy.

\subsubsection{A Water Jug Invariant}\label{jug_invar_subsubsec}
\index{invariant}

Suppose that we have water jugs with capacities $a$ and $b$ with $b
\geq a$.  Let's carry out some sample operations of the \idx{state machine}
and see what happens, assuming the $b$-jug is big enough:
\begin{align*}
(0,0) & \rightarrow (a,0) & \text{fill first jug} \\
& \rightarrow (0,a) & \text{pour first
    into second} \\
& \rightarrow (a, a) & \text{fill first jug} \\
& \rightarrow (2a-b, b)
  & \text{pour first into second (assuming $2a \geq b$)} \\
& \rightarrow (2a-b, 0) &
  \text{empty second jug} \\
& \rightarrow (0, 2a-b) & \text{pour first into second} \\
&
  \rightarrow (a, 2a-b) & \text{fill first} \\
& \rightarrow (3a-2b, b) & \text{pour first
    into second (assuming $3a \geq 2b$)}
\end{align*}
What leaps out is that at every step, the amount of water in each jug
is a linear combination of $a$ and $b$.  This is easy to prove by
induction on the number of transitions:
\begin{lemma}[Water Jugs]\label{lem:waterjugs}
In the \emph{Die Hard} state machine of
Section~\ref{diehard_machine} with jugs of sizes $a$ and $b$, the
amount of water in each jug is always a linear combination of $a$ and
$b$.
\end{lemma}

\begin{proof}
The induction hypothesis $P(n)$ is the proposition that after $n$
transitions, the amount of water in each jug is a linear combination
of $a$ and $b$.

\inductioncase{Base case} ($n = 0$): $P(0)$ is true, because both jugs
are initially empty, and $0 \cdot a + 0 \cdot b = 0$.

\inductioncase{Inductive step}: Suppose the machine is in state
$(x,y)$ after $n$ steps, that is, the little jug contains $x$ gallons
and the big one contains $y$ gallons.  There are two cases:

\begin{itemize}

\item If we fill a jug from the fountain or empty a jug into the
  fountain, then that jug is empty or full.  The amount in the other
  jug remains a linear combination of $a$ and $b$.  So $P(n+1)$ holds.

\item Otherwise, we pour water from one jug to another until one is
  empty or the other is full.  By our assumption, the amount $x$ and
  $y$ in each jug is a linear combination of $a$ and $b$ before we
  begin pouring.  After pouring, one jug is either empty (contains 0
  gallons) or full (contains $a$ or $b$ gallons).  Thus, the other jug
  contains either $x + y$, $x + y - a$ or $x + y - b$ gallons, all of
  which are linear combinations of $a$ and $b$ since $x$ and $y$ are.
  So $P(n+1)$ holds in this case as well.
\end{itemize}
Since $P(n+1)$ holds in any case, this proves the inductive step,
completing the proof by induction.
\end{proof}

So we have established that the jug problem has a preserved invariant,
namely, the amount of water in every jug is a linear combination of
the capacities of the jugs.  Lemma~\ref{lem:waterjugs} has an
important corollary:
\begin{corollary*}
In trying to get 4 gallons from 12- and 18-gallon jugs, and likewise
to get 32 gallons from 899- and 1147-gallon jugs,
\[
\textbf{Bruce will die!}
\]
\end{corollary*}

\begin{proof}
By the Water Jugs Lemma~\ref{lem:waterjugs}, with 12- and 18-gallon
jugs, the amount in any jug is a linear combination of 12 and 18.
This is always a multiple of 6 by
Lemma~\ref{lem:div}.\ref{lem:divsbtc}, so Bruce can't get 4 gallons.
Likewise, the amount in any jug using 899- and 1147-gallon jugs is a
multiple of 31, so he can't get 32 either.
\end{proof}

But the Water Jugs Lemma doesn't tell the complete story.  For
example, it leaves open the question of getting 3 gallons into a jug
using just 21- and 26-gallon jugs: the only positive factor of both 21
and 26 is 1, and of course 1 divides 3, so the Lemma neither rules out
nor confirms the possibility of getting 3 gallons.

A bigger issue is that we've just managed to recast a pretty
understandable question about water jugs into a technical question
about linear combinations.  This might not seem like a lot of
progress.  Fortunately, linear combinations are closely related to
something more familiar, greatest common divisors, and
will help us solve the general water jug problem.

\begin{problems}
\practiceproblems
\pinput{TP_linear-combs-combined}

\classproblems
\pinput{CP_perfect_numbers}

\end{problems}

\section{The Greatest Common Divisor}\label{sec:gcd}

A \emph{common divisor} of~$a$ and~$b$ is a number that divides them
both.  The \term{greatest common divisor} of $a$ and~$b$ is written%
\index{gcd|see{greatest common divisor}} 
$\gcd(a, b)$.  For example, $\gcd(18, 24) = 6$.

As long as $a$ and $b$ are not both 0, they will have a gcd.  The gcd
turns out to be very valuable for reasoning about the relationship
between $a$ and $b$ and for reasoning about integers in general.
We'll be making lots of use of gcd's in what follows.

Some immediate consequences of the definition of gcd are that for $n > 0$,
\[
\gcd(n, n) = n, \qquad \gcd(n, 1) = 1, \qquad  \gcd(n,0) = n,
\]
where the last equality follows from the fact that everything is a
divisor of 0.

\subsection{Euclid's Algorithm}\label{sec: Euclid}
The first thing to figure out is how to find gcd's.  A good way called \term{Euclid's
  algorithm} has been known for several thousand years.  It is based on the following
elementary observation.

\begin{lemma}\label{lem:gcdrem}
For $b \neq 0$,
\[
\gcd(a, b) = \gcd(b, \rem{a}{b}).
\]

\begin{proof}
By the Division Theorem~\ref{division_thm},
\begin{equation}\label{aqbrprf}
a = qb + r
\end{equation}
where $r = \rem{a}{b}$.  So $a$ is a linear combination of $b$ and
$r$, which implies that any divisor of $b$ and $r$ is a divisor of $a$
by Lemma~\ref{lem:div}.\ref{lem:divsbtc}.  Likewise, $r$ is a linear
combination $a-qb$ of $a$ and $b$, so any divisor of $a$ and $b$ is
a divisor of $r$.  This means that $a$ and $b$ have the same common
divisors as $b$ and $r$, and so they have the same \emph{greatest}
common divisor.
\end{proof}
\end{lemma}

Lemma~\ref{lem:gcdrem} is useful for quickly computing the greatest
common divisor of two numbers.  For example, we could compute the
greatest common divisor of 1147 and~899 by repeatedly applying it:
\begin{align*}
\gcd(1147, 899) &= \gcd(899, \underbrace{\rem{1147}{899}}_{{} = 248}) \\
&= \gcd\paren{248, \rem{899}{248} = 155} \\
&= \gcd\paren{155, \rem{248}{155} = 93} \\
&= \gcd\paren{ 93, \rem{155}{93} = 62}\\
&= \gcd\paren{ 62, \rem{93}{62} = 31} \\
&= \gcd\paren{ 31, \rem{62}{31} = 0} \\
&= 31
\end{align*}
This calculation that $\gcd(1147, 899) = 31$ was how we figured out
that with water jugs of sizes 1147 and 899, Bruce dies trying to get
32 gallons.

On the other hand, applying Euclid's algorithm to 26 and 21 gives
\[
\gcd(26, 21) = \gcd(21, 5) = \gcd(5, 1) = 1,
\]
so we can't use the reasoning above to rule out Bruce getting 3
gallons into the big jug.  As a matter of fact, because the gcd here
is 1, Bruce \emph{will} be able to get any number of gallons into the big jug
up to its capacity.  To explain this, we will need a little more
number theory.

\subsubsection{Euclid's Algorithm as a State Machine}
Euclid's algorithm can easily be formalized as a \idx{state
machine}.  The set of states is $\nngint^2$ and there is one
transition rule:
\begin{equation}\label{euclid_transition}
(x,y) \movesto (y, \rem{x}{y}),
\end{equation}
for $y>0$.  By Lemma~\ref{lem:gcdrem}, the gcd stays the same from one
state to the next.  That means the predicate
\[
\gcd(x,y) = \gcd(a,b)
\]
is a preserved invariant on the states $(x,y)$.  This preserved
invariant is, of course, true in the start state $(a,b)$.  So by the
Invariant Principle, if $y$ ever becomes $0$, the invariant will be
true and so
\[
x = \gcd(x,0) = \gcd(a,b).
\]
Namely, the value of $x$ will be the desired gcd.

What's more $x$ and therefore also $y$, gets to be 0 pretty fast.
To see why, note that starting from $(x,y)$, two transitions leads to
a state whose the first coordinate is $\rem{x}{y}$, which is at most
half the size of $x$.\footnote{In other words,
\begin{equation}\label{rxylx2}
\rem{x}{y} \le x/2 \qquad \text{for $0 < y \le x$}.
\end{equation}
This is immediate if $y \le x/2$, since the remainder of $x$ divided
by $y$ is less than $y$ by definition.  On the other hand, if $y >
x/2$, then $\rem{x}{y} = x - y < x/2$.}  Since $x$ starts off equal to
$a$ and gets halved or smaller every two steps, it will reach its
minimum value---which is $\gcd(a,b)$---after at most $2 \log a$
transitions.  After that, the algorithm takes at most one more
transition to terminate.  In other words, Euclid's algorithm
terminates after at most $1+2 \log a$ transitions.\footnote{A tighter
  analysis shows that at most $\log_\varphi(a)$ transitions are
  possible where $\varphi$ is the \idx{golden ratio} $(1 +
  \sqrt{5})/2$, see Problem~\ref{PS_gcd_termination}.}

\subsection{The Pulverizer}\label{sec:pulverizer}
We will get a lot of mileage out of the following key fact:
\begin{theorem}\label{gcd_is_lin_thm}
The greatest common divisor of $a$ and $b$ is a linear combination of
$a$ and~$b$.  That is,
\[
\gcd(a, b) = s a + t b,
\]
for some integers $s$ and $t$.\footnote{This result is often referred
  to as \emph{\idx{Bezout's lemma}}, which is a misattribution since
  it was first published in the West 150 years earlier by someone
  else, and was described a thousand years before that by Indian
  mathematicians Aryabhata and Bhaskara.}
\end{theorem}

We already know from Lemma~\ref{lem:div}.\ref{lem:divsbtc} that every
linear combination of $a$ and $b$ is divisible by any common factor of
$a$ and $b$, so it is certainly divisible by the greatest of these
common divisors.  Since any constant multiple of a linear combination
is also a linear combination, Theorem~\ref{gcd_is_lin_thm} implies
that any multiple of the gcd is a linear combination, giving:
\begin{corollary}\label{cor:lin-comb}
An integer is a linear combination of $a$ and $b$ iff it is a multiple
of $\gcd(a, b)$.
\end{corollary}

We'll prove Theorem~\ref{gcd_is_lin_thm} directly by explaining how to
find $s$ and $t$.  This job is tackled by a mathematical tool that
dates back to sixth-century India, where it was called \emph{kuttaka},
which means ``the Pulverizer.''  Today, the Pulverizer%
\index{Pulverizer|textbf} 
is more commonly known as the ``Extended Euclidean Gcd Algorithm,'' because it
is so close to Euclid's algorithm.

For example, following Euclid's algorithm, we can compute the gcd of
259 and~70 as follows:
\begin{align*}
\gcd(259, 70) & = \gcd(70, 49) & \quad & \text{since $\rem{259}{70} = 49$}\\
 & = \gcd(49, 21) && \text{since $\rem{70}{49} = 21$} \\
 & = \gcd(21, 7) && \text{since $\rem{49}{21} = 7$} \\
 & = \gcd(7, 0)
                && \text{since $\rem{21}{7} = 0$} \\
 & = 7.
\end{align*}
The Pulverizer goes through the same steps, but requires some extra
bookkeeping along the way: as we compute $\gcd(a, b)$, we keep track
of how to write each of the remainders (49, 21, and 7, in the example)
as a linear combination of $a$ and $b$.  This is worthwhile, because
our objective is to write the last nonzero remainder, which is the
$\gcd$, as such a linear combination.  For our example, here is this
extra bookkeeping:
\[
\begin{array}{ccccrcl}
x & \quad & y & \quad & (\rem{x}{y}) & = & x - q \cdot y \\
\hline
259 && 70 && 49 & = & a - 3 \cdot  b\\
 70 && 49 && 21 & = & b - 1 \cdot 49 \\
           &&&& & = & b - 1 \cdot (a - 3 \cdot b) \\
           &&&& & = & -1 \cdot a + 4 \cdot b \\
 49 && 21 && 7  & = & 49 - 2 \cdot 21\\
           &&&& & = & (a - 3 \cdot b) - 2 \cdot (-1 \cdot a + 4 \cdot b) \\
           &&&& & = & \fbox{$3 \cdot a - 11 \cdot b$} \\
 21 && 7 && 0
\end{array}
\]
We began by initializing two variables, $x = a$ and $y = b$.  In the
first two columns above, we carried out Euclid's algorithm.  At each
step, we computed $\rem{x}{y}$ which equals $x - \qcnt{x}{y} \cdot y$.
Then, in this linear combination of $x$ and $y$, we replaced $x$ and
$y$ by equivalent linear combinations of $a$ and $b$, which we already
had computed.  After simplifying, we were left with a linear
combination of $a$ and $b$ equal to $\rem{x}{y}$, as desired.  The
final solution is boxed.

This should make it pretty clear how and why the Pulverizer works.  If
you have doubts, you may work through
Problem~\ref{PS_pulverizer_machine}, where the Pulverizer is
formalized as a state machine and then verified using an invariant
that is an extension of the one used for Euclid's algorithm.

Since the Pulverizer requires only a little more computation than
Euclid's algorithm, you can ``pulverize'' very large numbers very
quickly by using this algorithm.  As we will soon see, its speed makes
the Pulverizer a very useful tool in the field of cryptography.

Now we can restate the Water Jugs Lemma~\ref{lem:waterjugs} in terms
of the greatest common divisor:
\begin{corollary}\label{cor:waterjugs}
Suppose that we have water jugs with capacities $a$ and $b$.  Then the
amount of water in each jug is always a multiple of $\gcd(a, b)$.
\end{corollary}

For example, there is no way to form 4 gallons using 3- and 6-gallon
jugs, because 4 is not a multiple of $\gcd(3, 6) = 3$.

\subsection{One Solution for All Water Jug Problems}\label{all_jugs_son_sec}

Corollary~\ref{cor:lin-comb} says that 3 can be written as a linear
combination of 21 and 26, since 3 is a multiple of $\gcd(21, 26) = 1$.
So the Pulverizer will give us integers $s$ and $t$ such that
\begin{equation}\label{3s21t26}
3 = s \cdot 21 + t \cdot 26
\end{equation}

The coefficient $s$ could be either positive or negative.
However, we can readily transform this linear combination into an
equivalent linear combination
\begin{equation}\label{3sprime21}
3 = s' \cdot 21 + t' \cdot 26
\end{equation}
where the coefficient $s'$ is positive.  The trick is to notice that
if in equation~\eqref{3s21t26} we increase $s$ by 26 and decrease $t$
by 21, then the value of the expression $s \cdot 21 + t \cdot 26$ is
unchanged overall.  Thus, by repeatedly increasing the value of $s$
(by 26 at a time) and decreasing the value of $t$ (by 21 at a time),
we get a linear combination $s' \cdot 21 + t' \cdot 26 = 3$ where the
coefficient $s'$ is positive.  (Of course $t'$ must then be negative;
otherwise, this expression would be much greater than 3.)

Now we can form 3 gallons using jugs with capacities 21 and~26: We
simply repeat the following steps $s'$ times:
\begin{enumerate}
\item Fill the 21-gallon jug.
\item Pour all the water in the 21-gallon jug into the 26-gallon jug.
  If at any time the 26-gallon jug becomes full, empty it out, and
  continue pouring the 21-gallon jug into the 26-gallon jug.
\end{enumerate}
At the end of this process, we must have emptied the 26-gallon
jug exactly $-t'$ times.  Here's why: we've taken $s' \cdot 21$
gallons of water from the fountain, and we've poured out some multiple
of 26 gallons.  If we emptied fewer than $-t'$ times, then
by~\eqref{3sprime21}, the big jug would be left with at least $3+26$
gallons, which is more than it can hold; if we emptied it more times,
the big jug would be left containing at most $3-26$ gallons, which is
nonsense.  But once we have emptied the 26-gallon jug exactly $-t'$
times, equation~\eqref{3sprime21} implies that there are exactly 3
gallons left.

Remarkably, we don't even need to know the coefficients $s'$ and $t'$
in order to use this strategy!  Instead of repeating the outer loop
$s'$ times, we could just repeat \emph{until we obtain 3 gallons},
since that must happen eventually.  Of course, we have to keep track
of the amounts in the two jugs so we know when we're done.  Here's the
solution using this approach starting with empty jugs, that is, at
$(0,0)$:
\[
\begin{array}{cccccccc}
\xrightarrow{\text{fill 21}} & (21,0)& \xrightarrow{\text{pour 21 into
    26}} & & & &&(0,21)\\
\xrightarrow{\text{fill 21}} & (21,21)&
\xrightarrow{\text{pour 21 to 26}} & (16,26)& \xrightarrow{\text{empty
    26}} & (16,0)& \xrightarrow{\text{pour 21 to 26}} &
(0,16)\\
\xrightarrow{\text{fill 21}} & (21,16)&
\xrightarrow{\text{pour 21 to 26}} & (11,26)& \xrightarrow{\text{empty
    26}} & (11,0)& \xrightarrow{\text{pour 21 to 26}} &
(0,11)\\
\xrightarrow{\text{fill 21}} & (21,11)&
\xrightarrow{\text{pour 21 to 26}} & (6,26)& \xrightarrow{\text{empty
    26}} & (6,0)& \xrightarrow{\text{pour 21 to 26}} &
(0,6)\\
\xrightarrow{\text{fill 21}} & (21,6)& \xrightarrow{\text{pour
    21 to 26}} & (1,26)& \xrightarrow{\text{empty 26}} & (1,0)&
\xrightarrow{\text{pour 21 to 26}} & (0,1)\\
\xrightarrow{\text{fill
    21}} & (21,1)& \xrightarrow{\text{pour 21 to 26}} &&&&&
(0,22)\\
\xrightarrow{\text{fill 21}} & (21,22)&
\xrightarrow{\text{pour 21 to 26}} & (17,26)& \xrightarrow{\text{empty
    26}} & (17,0)& \xrightarrow{\text{pour 21 to 26}} &
(0,17)\\
\xrightarrow{\text{fill 21}} & (21,17)&
\xrightarrow{\text{pour 21 to 26}} & (12,26)& \xrightarrow{\text{empty
    26}} & (12,0)& \xrightarrow{\text{pour 21 to 26}} &
(0,12)\\
\xrightarrow{\text{fill 21}} & (21,12)&
\xrightarrow{\text{pour 21 to 26}} & (7,26)& \xrightarrow{\text{empty
    26}} & (7,0)& \xrightarrow{\text{pour 21 to 26}} &
(0,7)\\
\xrightarrow{\text{fill 21}} & (21,7)& \xrightarrow{\text{pour
    21 to 26}} & (2,26)& \xrightarrow{\text{empty 26}} & (2,0)&
\xrightarrow{\text{pour 21 to 26}} & (0,2)\\
\xrightarrow{\text{fill
    21}} & (21,2)& \xrightarrow{\text{pour 21 to 26}} &
&&&&(0,23)\\
\xrightarrow{\text{fill 21}} & (21,23)&
\xrightarrow{\text{pour 21 to 26}} & (18,26)& \xrightarrow{\text{empty
    26}} & (18,0)& \xrightarrow{\text{pour 21 to 26}} &
(0,18)\\
\xrightarrow{\text{fill 21}} & (21,18)&
\xrightarrow{\text{pour 21 to 26}} & (13,26)& \xrightarrow{\text{empty
    26}} & (13,0)& \xrightarrow{\text{pour 21 to 26}} &
(0,13)\\
\xrightarrow{\text{fill 21}} & (21,13)&
\xrightarrow{\text{pour 21 to 26}} & (8,26)& \xrightarrow{\text{empty
    26}} & (8,0)& \xrightarrow{\text{pour 21 to 26}} &
(0,8)\\
\xrightarrow{\text{fill 21}} & (21,8)& \xrightarrow{\text{pour
    21 to 26}} & (3,26)& \xrightarrow{\text{empty 26}} & (3,0)&
\xrightarrow{\text{pour 21 to 26}} & (0,3)
\end{array}
\]

The same approach works regardless of the jug capacities and even
regardless of the amount we're trying to produce!  Simply repeat these
two steps until the desired amount of water is obtained:
\begin{enumerate}
\item Fill the smaller jug.

\item Pour all the water in the smaller jug into the larger jug.  If
  at any time the larger jug becomes full, empty it out, and continue
  pouring the smaller jug into the larger jug.
\end{enumerate}
By the same reasoning as before, this method eventually generates
every multiple---up to the size of the larger jug---of the greatest
common divisor of the jug capacities, all the quantities we
can possibly produce.  No ingenuity is needed at all!

So now we have the complete water jug story:
\begin{theorem}\label{th:waterjugs}
Suppose that we have water jugs with capacities $a$ and $b$.  For any
$c \in \Zintv{0}{a}$, it is possible to get $c$ gallons in the size $a$ jug
iff $c$ is a multiple of $\gcd(a, b)$.
\end{theorem}

\subsection{Properties of the \idx{Greatest Common Divisor}}

It can help to have some basic $\gcd$ facts on hand:

\begin{lemma}\label{lem:gcd-hold} \mbox{  }

\begin{enumerate}[a)]
\item\label{gcd2} $\gcd(k a, k b) = k \cdot \gcd(a, b)$ for all $k > 0$.
\item\label{gcd1} $(d \divides a \QAND d \divides b)\ \QIFF\ d \divides \gcd(a, b)$.
\item\label{gcd3} If $\gcd(a, b) = 1$ and $\gcd(a, c) = 1$, then $\gcd(a, bc) = 1$.
\item\label{gcd4} If $a \divides b c$ and $\gcd(a, b) = 1$, then $a \divides c$.
%\item\label{gcd5} $\gcd(a, b) = \gcd(b, \rem{a}{b})$. already in lem:gcdrem
\end{enumerate}
\end{lemma}

Showing how all these facts follow from Theorem~\ref{gcd_is_lin_thm}
that gcd is a linear combination is a good exercise
(Problem~\ref{CP_proving_basic_gcd_properties}).

These properties are also simple consequences of the fact that
integers factor into primes in a unique way
(Theorem~\ref{thm:unique_factor}).  But we'll need some of these facts
to prove unique factorization in
Section~\ref{fundamental_theorem_sec}, so proving them by appeal to
unique factorization would be circular.

\begin{problems}

\practiceproblems
\pinput{TP_GCDs_I}
\pinput{TP_GCDs_II}
%\pinput{TP_gcd_linear_combination_wop}
\pinput{TP_ax=b_congruence}
\pinput{TP_gcd_power}  %uncover after S16 mid3

\classproblems
\pinput{CP_GCD_algebra}
%\pinput{CP_use_the_pulverizer}
\pinput{MQ_use_the_pulverizer_morning}
\pinput{MQ_pulverizer_linear_combination}
\pinput{CP_gcd_TF}
\pinput{CP_proving_basic_gcd_properties}

\homeworkproblems
\pinput{PS_linear_combination_game}
\pinput{PS_pulverizer_machine}
\pinput{PS_gcd_termination}
\pinput{PS_filling_buckets_with_water}
\pinput{PS_binary_gcd}
\pinput{PS_binary_pulverizer}
\pinput{PS_gcd_union}

\examproblems
\pinput{FP_GCD_algebra}
%\pinput{FP_gcd_linear_combination_induction}
\pinput{MQ_state_machine_buckets_S15}
\pinput{FP_gcd_associative}
\end{problems}

\section{Prime Mysteries}

Some of the greatest mysteries and insights in number theory concern
properties of prime numbers:
\begin{definition}
A \term{prime} is a number greater than~1 that is divisible only by
itself and~1.  A number other than 0, 1, and $-1$ that is not a prime
is called \emph{composite}\index{composite|see{prime}}.\footnote{So 0, 1, 
and $-1$ are the only integers that are neither prime nor composite.}
\end{definition}

Here are three famous mysteries:

%\floatingtextbox{ \textboxtitle{Famous Conjectures about Primes}

\begin{description}

\item[\term{Twin Prime Conjecture}] There are infinitely many primes
  $p$ such that $p + 2$ is also a prime.

  In 1966, Chen showed that there are infinitely many primes $p$ such
  that $p + 2$ is the product of at most two primes.  So the
  conjecture is known to be \emph{almost} true!

\item[\emph{Conjectured Inefficiency of Factoring}\index{factoring}] Given the
  product of two large primes $n = pq$, there is no efficient
  procedure to recover the primes $p$ and $q$.  That is, no
  \emph{\idx{polynomial time}} procedure (see Section~\ref{SAT_sec})
  is guaranteed to find $p$ and $q$ in a number of steps bounded by a
  polynomial in the length of the binary representation of $n$ (not
  $n$ itself).  The length of the binary representation at most
  $1+\log_2 n$.

  The best algorithm known is the ``number field sieve,'' which runs
  in time proportional to:
  \[
  e^{1.9(\ln n)^{1/3} (\ln\ln n)^{2/3}}.
  \]
  This number grows more rapidly than any polynomial in $\log n$ and
  is infeasible when $n$ has 300 digits or more.

  Efficient factoring is a mystery of particular importance in
  computer science, as we'll explain later in this chapter.

\item[\emph{\idx{Goldbach's Conjecture}}] We've already mentioned Goldbach's
  Conjecture~\ref{Goldbach} several times: every even integer greater
  than two is equal to the sum of two primes.  For example, $4 = 2 +
  2$, $6 = 3 + 3$, $8 = 3 + 5$, etc.

  In 1939, Schnirelman proved that every even number can be written as
  the sum of not more than 300,000 primes, which was a start.  Today,
  we know that every even number is the sum of at most 6 primes.
\end{description}

Primes show up erratically in the sequence of integers.  In fact,
their distribution seems almost random:
\[
2, 3, 5, 7, 11, 13, 17, 19, 23, 29, 31, 37, 41, 43, \dots.
\]
One of the great insights about primes is that their density among the
integers has a precise limit.  Namely, let $\pi(n)$ denote the number
of primes up to $n$:

\begin{definition}\label{def:prime<x}
\[
  \pi(n) \eqdef \card{\set{p \in \Zintv{2}{n} \suchthat p \text{ is
        prime}}}.
\]
\end{definition}

For example, $\pi(1) = 0, \pi(2) = 1$ and $\pi(10) = 4$, because 2, 3,
5, and 7 are the primes less than or equal to 10.  Step by step,
$\pi$ grows erratically according to the erratic spacing between
successive primes, but its overall growth rate is known to smooth out
to be the same as the growth of the function $n/\ln n$:

\begin{theorem}[Prime Number Theorem]%
\index{prime!Prime Number Theorem}
\[
\lim_{n\to\infty} \frac{\pi(n)}{n/\ln n} = 1.
\]
\end{theorem}

Thus, primes gradually taper off.  As a rule of thumb, about 1 integer
out of every $\ln n$ in the vicinity of $n$ is a prime.

%\begin{editingnotes} The accent on Vallee screwed up the hyphens in
%the entire pdf file!!!  \end{editingnotes}

The Prime Number Theorem was conjectured by Legendre in 1798 and
proved a century later by de la Vall\'{e}e Poussin and Hadamard in 1896.
However, after his death, a notebook of \idx{Gauss} was found to contain the
same conjecture, which he apparently made in 1791 at age 15.  (You
have to feel sorry for all the otherwise ``great'' mathematicians who
had the misfortune of being contemporaries of Gauss.)

A proof of the Prime Number Theorem is beyond the scope of this text,
but there is a manageable proof (see
Problem~\ref{PS_Chebyshev_prime_bound}) of a related result that is
sufficient for our applications:
\begin{theorem}[Chebyshev's Theorem on Prime Density]
For $n >1$,
\[
\pi(n) > \frac{n}{3 \ln n}.
\]
\end{theorem}

\floatingtextbox{  \textboxheader{A Prime for Google}

In late 2004 a billboard appeared in various locations around the
country:

{\Large
\[
\left\{
\begin{array}{c}
\text{first 10-digit prime found}\\
\text{in consecutive digits of
  $e$}
\end{array}
\right\} \textbf{. com}
\]
}

Substituting the correct number for the expression in curly-braces
produced the URL for a Google employment page.  The idea was that
Google was interested in hiring the sort of people that could and
would solve such a problem.

How hard is this problem?  Would you have to look through thousands or
millions or billions of digits of $e$ to find a 10-digit prime?  The
rule of thumb derived from the Prime Number Theorem says that among
10-digit numbers, about 1 in
\[
\ln 10^{10} \approx 23
\]
is prime.  This suggests that the problem isn't really so hard!  Sure
enough, the first 10-digit prime in consecutive digits of $e$ appears
quite early:
\begin{align*}
e = & 2.718281828459045235360287471352662497757247093699959574966 \\
& 96762772407663035354759457138217852516642\textcolor{blue}{\mathbf{7427466391}}9320030\\
& 599218174135966290435729003342952605956307381323286279434\dots
\end{align*}

%\begin{editingnotes} Checked against
%http://www.gutenberg.org/1/2/127/ --dmj, \end{editingnotes}
}

\begin{problems}
\homeworkproblems
\pinput{PS_Chebyshev_prime_bound}
\end{problems}

\section{The Fundamental Theorem of Arithmetic}\label{fundamental_theorem_sec}

There is an important fact about primes that you probably already
know: every positive integer number has a \emph{unique} prime
factorization.  So every positive integer can be built up from primes
in \emph{exactly one way}.  These quirky prime numbers are the
building blocks for the integers.

Since the value of a product of numbers is the same if the numbers
appear in a different order, there usually isn't a unique way to
express a number as a product of primes.  For example, there are three
ways to write 12 as a product of primes:
\[
12 = 2 \cdot 2 \cdot 3 = 2 \cdot 3 \cdot 2 = 3 \cdot 2 \cdot 2.
\]
What's unique about the prime factorization of 12 is that any product
of primes equal to 12 will have exactly one 3 and two 2's.  This means
that if we \emph{sort} the primes by size, then the product really
will be unique.

Let's state this more carefully.  A sequence of numbers is
\emph{\idx{weakly decreasing}} when each number in the sequence is at
least as big as the numbers after it.  Note that a sequence of just
one number as well as a sequence of no numbers---the empty
sequence---is weakly decreasing by this definition.

\begin{theorem}\label{thm:unique_factor}[Fundamental Theorem of Arithmetic]
Every positive integer is a product of a \emph{unique} weakly
decreasing sequence of primes.
\end{theorem}
\index{Unique Factorization Theorem|textbf}
\index{Fundamental Theorem of Arithmetic|see{Unique Factorization Theorem}}

For example, 75237393 is the product of the weakly decreasing sequence
of primes
\[
23, 17, 17, 11, 7, 7, 7, 3,
\]
and no other weakly decreasing sequence of primes will give
75237393.\footnote{The ``product'' of just one number is defined to be
  that number, and the product of no numbers is by convention defined
  to be 1.  So each prime $p$ is uniquely the product of the primes
  in the length-one sequence consisting solely of $p$, and 1, which you will 
  remember is not a prime, is uniquely the product of the
  empty sequence.}

Notice that the theorem would be false if 1 were considered a prime;
for example, $15$ could be written as $5 \cdot 3$, or $5 \cdot 3 \cdot
1$, or $5 \cdot 3 \cdot 1 \cdot 1$, \dots.

There is a certain wonder in unique factorization, especially in view
of the prime number mysteries we've already mentioned.  It's a mistake
to take it for granted, even if you've known it since you were in a
crib.  In fact, unique factorization actually fails for many
integer-like sets of numbers, such as the complex numbers of the
form $n + m\sqrt{-5}$ for $m,n \in \integers$ (see
Problem~\ref{PS_non_unique_factoring}).

The Fundamental Theorem is also called the \emph{Unique Factorization
  Theorem}, which is a more descriptive and less pretentious,
name---but we really want to get your attention to the importance and
non-obviousness of unique factorization.

\subsection{Proving Unique Factorization}

The Fundamental Theorem is not hard to prove, but we'll need a couple
of preliminary facts.

\begin{lemma}
\label{lem:prime-divides}
If $p$ is a prime and $p \divides ab$, then $p \divides a$ or $p
\divides b$.
\end{lemma}

Lemma~\ref{lem:prime-divides} follows immediately from Unique
Factorization: the primes in the product $ab$ are exactly the primes
from $a$ and from $b$.  But proving the lemma this way would be
cheating: we're going to need this lemma to prove Unique
Factorization, so it would be circular to assume it.  Instead, we'll
use the properties of gcd's and linear combinations to give an easy,
noncircular way to prove Lemma~\ref{lem:prime-divides}.

\begin{proof}
One case is if $\gcd(a, p) = p$.  Then the claim holds, because $a$ is
a multiple of $p$.

Otherwise, $\gcd(a, p) \neq p$.  In this case $\gcd(a, p)$ must be 1,
since 1 and $p$ are the only positive divisors of $p$.  Now $\gcd(a,
p)$ is a linear combination of $a$ and $p$, so we have $1=sa+tp$ for
some $s,t$.  Then $b =s(ab)+ (tb)p$, that is, $b$ is a linear
combination of $ab$ and $p$.  Since $p$ divides both $ab$ and $p$, it
also divides their linear combination $b$.
\end{proof}

A routine induction argument extends this statement to:\iffalse the
fact we assumed last time:\fi

\begin{lemma}
\label{lem:prime-divides-ind}
Let $p$ be a prime.  If $p \divides a_1 a_2 \cdots a_n$, then $p$
divides some $a_i$.
\end{lemma}

Now we're ready to prove the Fundamental Theorem of Arithmetic.
\begin{proof}
Theorem~\ref{factor_into_primes} showed, using the Well Ordering
Principle, that every positive integer can be expressed as a product
of primes.  So we just have to prove this expression is unique.  We
will use Well Ordering to prove this too.

The proof is by contradiction: assume, contrary to the claim, that
there exist positive integers that can be written as products of
primes in more than one way.  By the Well Ordering Principle, there is
a smallest integer with this property.  Call this integer $n$, and let
\begin{align*}
n & = p_1 \cdot p_2 \cdots p_j, \\
& = q_1 \cdot q_2 \cdots q_k,
\end{align*}
where both products are in weakly decreasing order and $p_1 \le q_1$.

If $q_1 = p_1$, then $n/q_1$ would also be the product of different
weakly decreasing sequences of primes, namely,
\begin{align*}
 p_2 \cdots p_j, \\
q_2 \cdots q_k.
\end{align*}
Since $n/q_1 < n$, this can't be true, so we conclude that $p_1 <
q_1$.

Since the $p_i$'s are weakly decreasing, all the $p_i$'s are less than
$q_1$.  But
\[
q_1\divides n= p_1 \cdot p_2 \cdots p_j,
\]
so Lemma~\ref{lem:prime-divides-ind} implies that $q_1$ divides one of
the $p_i$'s, which contradicts the fact that $q_1$ is bigger than all
them.
\end{proof}

\begin{problems}
\practiceproblems
\pinput{TP_divide_product_induction}

\classproblems
\pinput{CP_gcd_lcm}

\homeworkproblems
\pinput{PS_non_unique_factoring}

\end{problems}

\section{Alan \idx{Turing}}\label{Turing_sec}

\begin{figure}\redrawntrue
\graphic[width=2in]{turing}
\caption{Alan Turing}
\label{fig:Turing}
\end{figure}

The man pictured in Figure~\ref{fig:Turing} is Alan Turing, the most
important figure in the history of computer science.  For decades, his
fascinating life story was shrouded by government secrecy, societal
taboo, and even his own deceptions.

At age 24, Turing wrote a paper entitled \emph{On Computable Numbers,
  with an Application to the Entscheidungsproblem}.  The crux of the
paper was an elegant way to model a computer in mathematical terms.
This was a breakthrough, because it allowed the tools of mathematics
to be brought to bear on questions of computation.  For example, with
his model in hand, Turing immediately proved that there exist problems
that no computer can solve---no matter how ingenious the programmer.
Turing's paper is all the more remarkable because he wrote it in 1936,
a full decade before any electronic computer actually existed.

The word ``Entscheidungsproblem'' in the title refers to one of the 28
mathematical problems posed by David Hilbert in 1900 as challenges to
mathematicians of the 20th century.  Turing knocked that one off in
the same paper.  And perhaps you've heard of the ``Church-Turing
thesis''?  Same paper.  So Turing was a brilliant guy who generated
lots of amazing ideas.  But this lecture is about one of Turing's
less-amazing ideas.  It involved codes.  It involved number theory.
And it was sort of stupid.

%\subsection{Turing's Code}

Let's look back to the fall of 1937.  Nazi Germany was rearming under
Adolf Hitler, world-shattering war looked imminent, and---like us
---Alan Turing was pondering the usefulness of number theory.  He
foresaw that preserving military secrets would be vital in the coming
conflict and proposed a way \emph{to encrypt communications using
  number theory}.  This is an idea that has ricocheted up to our own
time.  Today, number theory is the basis for numerous public-key
cryptosystems, digital signature schemes, cryptographic hash
functions, and electronic payment systems.  Furthermore, military
funding agencies are among the biggest investors in cryptographic
research.  Sorry, Hardy!

Soon after devising his code, Turing disappeared from public
view, and half a century would pass before the world learned the full
story of where he'd gone and what he did there.  We'll come back to
Turing's life in a little while; for now, let's investigate the code
Turing left behind.  The details are uncertain, since he never
formally published the idea, so we'll consider a couple of
possibilities.

\subsection{Turing's Code (Version 1.0)}\label{probable_primes}

The first challenge is to translate a text message into an integer so
we can perform mathematical operations on it.  This step is not
intended to make a message harder to read, so the details are not too
important.  Here is one approach: replace each letter of the message
with two digits ($A = 01$, $B = 02$, $C = 03$, etc.) and string all
the digits together to form one huge number.  For example, the message
``victory'' could be translated this way:
\begin{center}
\begin{tabular}{ccccccccc}
   &v & i & c & t & o & r & y \\
$\rightarrow$ & 22 & 09 & 03 & 20 &
  15 & 18 & 25
\end{tabular}
\end{center}
Turing's code requires the message to be a prime number, so we
may need to pad the result with some more digits to make a prime.  The
Prime Number Theorem indicates that padding with relatively few digits
will work.  In this case, appending the digits 13 gives the number
2209032015182513, which is prime.

Here is how the encryption process works.  In the description below,
$m$ is the unencoded message (which we want to keep secret), $\widehat{m}$ is
the encrypted message (which the Nazis may intercept), and $k$ is the
key.

\begin{description}

\item[Beforehand] The sender and receiver agree on a \emph{secret
  key}, which is a large prime~$k$.

\item[Encryption] The sender encrypts the message $m$ by computing:
\[
\widehat{m} = m \cdot k
\]

\item[Decryption] The receiver decrypts $\widehat{m}$ by computing:
\[
\frac{\widehat{m}}{k} = m.
\]

\iffalse = \frac{m \cdot k}{k} \fi

\end{description}

For example, suppose that the secret key is the prime number $k =
22801763489$ and the message $m$ is ``victory.''  Then the encrypted
message is:
\begin{align*}
\widehat{m} & = m \cdot k \\
& = 2209032015182513 \cdot 22801763489 \\
& =
50369825549820718594667857
\end{align*}

There are a couple of basic questions to ask about Turing's code.

\begin{enumerate}

\item How can the sender and receiver ensure that $m$ and $k$ are
  prime numbers, as required?

The general problem of determining whether a large number is prime or
composite has been studied for centuries, and tests for primes that
worked well in practice were known even in Turing's time.  In the past
few decades, very fast primality tests have been found as
described in the text box below.

\floatingtextbox{ \textboxheader{Primality Testing}

It's easy to see that an integer $n$ is prime iff it is not divisible
by any number from 2 to $\floor{\sqrt{n}}$ (see
Problem~\ref{TP_squareroot_size_factor}).  Of course this naive way to
test if $n$ is prime takes more than $\sqrt{n}$ steps, which is
exponential in the \emph{size} of $n$ measured by the number of digits
in the decimal or binary representation of $n$.  Through the early
1970's, no prime testing procedure was known that would never blow up
like this.

In 1974, Volker Strassen invented a simple, fast \emph{probabilistic}
primality test.  Strassens's test gives the right answer when applied
to any prime number, but has some probability of giving a wrong answer
on a nonprime number.  However, the probability of a wrong answer on
any given number is so tiny that relying on the answer is the best bet
you'll ever make.

Still, the theoretical possibility of a wrong answer was
intellectually bothersome---even if the probability of being wrong was
a lot less than the probability of an undetectable computer hardware
error leading to a wrong answer.  Finally in 2002, in a breakthrough
paper beginning with a quote from \idx{Gauss} emphasizing the
importance and antiquity of primality testing, Manindra Agrawal,
Neeraj Kayal, and Nitin Saxena presented an amazing, thirteen line
description of a polynomial time primality test.

This definitively places primality testing way below the exponential
effort apparently needed for SAT and similar problems.  \iffalse In
particular, the Agrawal \emph{et al.} test is guaranteed to give the
correct answer about primality of any number $n$ in about $(\log
n)^{12}$ steps, that is, a number of steps bounded by a twelfth degree
polynomial in the length (in bits) of the input $n$.  \fi The
polynomial bound on the Agrawal \emph{et al.} test had degree 12, and
subsequent research has reduced the degree to 5, but this is still too
large to be practical, and probabilistic primality tests remain the
method used in practice today.  It's plausible that the degree bound
can be reduced a bit more, but matching the speed of the known
probabilistic tests remains a daunting challenge.}

\item Is Turing's code secure?

The Nazis see only the encrypted message $\widehat{m} = m \cdot k$, so
recovering the original message $m$ requires factoring $\widehat{m}$.  Despite
immense efforts, no really efficient factoring algorithm has ever been
found.  It appears to be a fundamentally difficult problem.  So,
although a breakthrough someday can't be ruled out, the conjecture
that there is no efficient way to factor is widely accepted.  In
effect, Turing's code puts to practical use his discovery that there
are limits to the power of computation.  Thus, provided $m$ and $k$
are sufficiently large, the Nazis seem to be out of luck!

\end{enumerate}

This all sounds promising, but there is a major flaw in Turing's code.

\subsection{Breaking Turing's Code (Version 1.0)}

Let's consider what happens when the sender transmits a \emph{second}
message using Turing's code and the same key.  This gives the Nazis
two encrypted messages to look at:
\[
\widehat{m_1} = m_1 \cdot k
\hspace{0.75in} \text{and} \hspace{0.75in} \widehat{m_2} = m_2 \cdot k
\]
The greatest common divisor of the two encrypted messages, $\widehat{m_1}$ and
$\widehat{m_2}$, is the secret key $k$.  And, as we've seen, the $\gcd$ of two
numbers can be computed very efficiently.  So after the second message
is sent, the Nazis can recover the secret key and read \emph{every}
message!

A mathematician as brilliant as Turing is not likely to have
overlooked such a glaring problem, and we can guess that he had a
slightly different system in mind, one based on \emph{modular}
arithmetic.

\section{Modular Arithmetic}\label{modular_arithmeric_sec}
\index{modular arithmetic}
\index{congruence|seealso{modular arithemtic}}

\begin{editingnotes}
Congruence is a weak form of equality.
\end{editingnotes}

On the first page of his masterpiece on number theory,
\emph{Disquisitiones Arithmeticae}, \idx{Gauss} introduced the notion
of ``\term{congruence}\index{modular arithmetic!congruence|textbf}.''  Now, 
Gauss is another guy who managed to
cough up a half-decent idea every now and then, so let's take a look
at this one.  Gauss said that \emph{$a$ is congruent to $b$
  modulo $n$}\index{modulo $n$|see{modular arithmetic}} iff $n \divides (a - b)$.
This is written
\[
a \equiv b \pmod{n}.
\]
For example:
\[
29 \equiv 15 \pmod{7} \quad\text{ because } 7 \divides (29 - 15).
\]

\emph{It's not useful to allow a modulus $n \leq 1$, and so we will
  assume from now on that moduli are greater than 1.}

There is a close connection between congruences and remainders:
\begin{lemma}[Remainder]\label{lem:conrem}
\[
a \equiv b \pmod{n} \qiff \rem{a}{n} = \rem{b}{n}.
\]
\end{lemma}

\begin{proof}
By the Division Theorem~\ref{division_thm}, there exist unique pairs
of integers $q_1, r_1$ and $q_2, r_2$ such that:
\begin{align*}
a & = q_1 n + r_1\\
b & = q_2 n + r_2,
\end{align*}
where $r_1,r_2 \in \Zintvco{0}{n}$.  Subtracting the second equation from the
  first gives:
\begin{align*}
a - b & = (q_1 - q_2) n + (r_1 - r_2),
\end{align*}
where $r_1 - r_2$ is in the interval $(-n,n)$.  Now $a \equiv b
\pmod{n}$ if and only if $n$ divides the left-hand side of this equation.
This is true if and only if $n$ divides the right-hand side, which holds if
and only if $r_1 - r_2$ is a multiple of $n$.  But the only multiple
of $n$ in $(-n,n)$ is 0, so $r_1 - r_2$ must in fact equal 0, that is,
when $r_1 \eqdef \rem{a}{n} = r_2 \eqdef \rem{b}{n}$.
\end{proof}

So we can also see that
\[
29 \equiv 15 \pmod{7} \quad\text{ because } \rem{29}{7} = 1 =
\rem{15}{7}.
\]
Notice that even though ``(mod 7)'' appears on the end, the $\equiv$
symbol isn't any more strongly associated with the 15 than with the
29.  It would probably be clearer to write $29 \equiv_{\text{mod } 7}
15$, for example, but the notation with the modulus at the end is
firmly entrenched, and we'll just live with it.

The Remainder Lemma~\ref{lem:conrem} explains why the congruence
relation has properties like an equality relation.  In particular, the
following properties\footnote{Binary relations with these properties
  are called \emph{\idx{equivalence relation}s}, see
  Section~\ref{equiv_rel_sec}.}  follow immediately:
\begin{lemma}\label{mod_equiv_rel_lem} \mbox{}
\begin{align}
                  & a \equiv a \pmod{n}\tag{reflexivity}\\
a \equiv b\ \QIFF\ & b \equiv a \pmod{n} \tag{symmetry}\\
(a \equiv b\ \QAND  b \equiv c)\ \QIMPLIES\
                  & a \equiv c \pmod{n} \tag{transitivity}
\end{align}
\end{lemma}

We'll make frequent use of another immediate corollary of the
Remainder Lemma~\ref{lem:conrem}:
\begin{corollary}\label{aran}
\[
a \equiv \rem{a}{n} \pmod{n}
\]
\end{corollary}

Still another way to think about congruence modulo $n$ is that it
\emph{defines a partition of the integers into $n$ sets so that
  congruent numbers are all in the same set}.  For example, suppose
that we're working modulo 3.  Then we can partition the integers into
3 sets as follows:
\[
\begin{array}{cccccccccc}
\{ & \dots, & -6, & -3, & 0, & 3, & 6, & 9, & \dots & \} \\
\{ &
\dots, & -5, & -2, & 1, & 4, & 7, & 10, & \dots & \} \\
\{ & \dots, &
-4, & -1, & 2, & 5, & 8, & 11, & \dots & \}
\end{array}
\]
according to whether their remainders on division by 3 are 0, 1, or 2.
The upshot is that when arithmetic is done modulo $n$, there are really
only $n$ different kinds of numbers to worry about, because there are
only $n$ possible remainders.  In this sense, modular arithmetic is a
simplification of ordinary arithmetic.\iffalse and thus is a good
reasoning tool.\fi

The next most useful fact about congruences is that they are
\emph{preserved} by addition and multiplication:

\begin{lemma}[Congruence]\label{mod_congruence_lem}  If
$a \equiv b \pmod{n}$ and $c \equiv d \pmod{n}$, then
\begin{align}
a + c & \equiv b + d \pmod{n},\label{mod_congruence_lem+}\\
a c   & \equiv b d   \pmod{n}.\label{mod_congruence_lem*}
\end{align}
\end{lemma}

\begin{proof}
Let's start with~\ref{mod_congruence_lem+}.  Since $a \equiv b
\pmod{n}$, we have by definition that $n \divides (b-a) =(b+c)-(a+c)$,
so
\[
a+c \equiv b+c \pmod{n}.
\]
Since $c \equiv d \pmod{n}$, the same reasoning leads to
\[
b + c \equiv b + d \pmod{n}.
\]
Now transitivity (Lemma~\ref{mod_equiv_rel_lem}) gives
\[
a + c \equiv b + d \pmod{n}.
\]
 
The proof for~\ref{mod_congruence_lem*} is virtually identical, using
the fact that if $n$ divides $(b-a)$, then it certainly also divides
$(bc-ac)$.
\end{proof}

\begin{problems}
\practiceproblems
\pinput{MQ_congruent_mod_product}

\classproblems
\pinput{CP_divisible_by_24}
\pinput{CP_polynomials_produce_multiples}
\end{problems}

\section{Remainder Arithmetic}\label{remainder_arithmetic_sec}
\index{modular arithmetic|textbf}
\index{remainder arithmetic|see{modular arithmetic}}

The Congruence Lemma~\ref{lem:conrem} says that two numbers are
congruent iff their remainders are equal, so we can understand
congruences by working out arithmetic with remainders.  And if all we
want is the remainder modulo $n$ of a series of additions,
multiplications, subtractions applied to some numbers, we can take
remainders at every step so that the entire computation only involves
number in the range $\Zintvco{0}{n}$.

\textbox{

\textboxtitle{General Principle of Remainder Arithmetic}

To find the remainder on division by $n$ of the result of a series of
additions and multiplications, applied to some integers
\begin{itemize}

\item replace each integer operand by its remainder on division by
  $n$,

\item keep each result of an addition or multiplication in the range
  $\Zintvco{0}{n}$ by immediately replacing any result outside that range by
  its remainder on division by $n$.
\end{itemize}
}

For example, suppose we want to find
\begin{equation}\label{ex44427}
\rem{(44427^{3456789} + 15555858^{5555})403^{6666666}}{36}.
\end{equation}
This looks really daunting if you think about computing these large
powers and then taking remainders.  For example, the decimal
representation of $44427^{3456789}$ has about 20 million digits, so we
certainly don't want to go that route.  But remembering that integer
exponents specify a series of multiplications, we follow the General
Principle and replace the numbers being multiplied by their
remainders.  Since $\rem{44427}{36} = 3, \rem{15555858}{36} = 6$, and
$\rem{403}{36} = 7$, we find that~\eqref{ex44427} equals the remainder
on division by 36 of
\begin{equation}\label{ex367}
(3^{3456789} + 6^{5555})7^{6666666}.
\end{equation}
That's a little better, but $3^{3456789}$ has about a million digits
in its decimal representation, so we still don't want to compute that.
But let's look at the remainders of the first few powers of 3:
\begin{align*}
\rem{3}{36} & = 3\\
\rem{3^2}{36} & = 9\\
\rem{3^3}{36} & = 27\\
\rem{3^4}{36} & = 9.
\end{align*}

We got a repeat of the second step, $\rem{3^2}{36}$ after just two
more steps.  This means means that starting at $3^2$, the sequence of
remainders of successive powers of 3 will keep repeating every 2
steps.  So a product of an odd number of at least three 3's will have
the same remainder on division by 36 as a product of just three 3's.
Therefore,
\[
\rem{3^{3456789}}{36} = \rem{3^3}{36} = 27.
\]
What a win!

Powers of 6 are even easier because $\rem{6^2}{36} = 0$, so 0's keep
repeating after the second step.  Powers of 7 repeat after six
steps, but on the fifth step you get a 1, that is $\rem{7^6}{36}
=1$, so~\eqref{ex367} successively simplifies to be the remainders
of the following terms:
\begin{align*}
\lefteqn{(3^{3456789} + 6^{5555})7^{6666666}}\\
   & (3^3 + 6^2 \cdot 6^{5553})(7^6)^{1111111}\\
   & (3^3 + 0 \cdot 6^{5553})1^{1111111}\\
   & = 27.
\end{align*}

Notice that \emph{it would be a disastrous blunder to replace an
  exponent by its remainder}.  The general principle applies to
numbers that are \emph{operands} of plus and times, whereas the
exponent is a number that controls how many multiplications to
perform.  Watch out for this.

\subsection{The ring $\Zmod{n}$}\label{subsec:ringZn}

It's time to be more precise about the general principle and why it
works.  To begin, let's introduce the notation $+_n$ for doing an
addition and then immediately taking a remainder on division by $n$,
as specified by the general principle; likewise for multiplying:
\begin{align*}
i +_n j & \eqdef \rem{i+j}{n},\\
i \cdot_n j & \eqdef \rem{ij}{n}.
\end{align*}

Now the General Principle is simply the repeated application of the
following lemma.
\begin{lemma}\label{rem-morphism}
\begin{align}
\rem{i+j}{n} = \rem{i}{n} +_n \rem{j}{n},\label{+rhom}\\
\rem{ij}{n} = \rem{i}{n} \cdot_n \rem{j}{n}.\label{.rhom}
\end{align}
\end{lemma}

\begin{proof}
By Corollary~\ref{aran}, $i \equiv \rem{i}{n}$ and $j \equiv
\rem{j}{n}$, so by the Congruence Lemma~\ref{mod_congruence_lem}
\[
i + j \equiv \rem{i}{n} + \rem{j}{n} \pmod{n}.
\]
By Corollary~\ref{aran} again, the remainders on each side of this
congruence are equal, which immediately gives~\eqref{+rhom}.  An
identical proof applies to~\eqref{.rhom}.
\end{proof}

The set of integers in the range $\Zintvco{0}{n}$ together with the
operations $+_n$ and $\cdot_n$ is referred to as $\Zmod{n}$, the%
\index{commutative ring!ring of intergers modulo $n$|textbf}%
\index{ring|see{commutative ring}} \emph{ring of integers modulo $n$}.
As a consequence of Lemma~\ref{rem-morphism}, the familiar rules of
arithmetic hold in $\Zmod{n}$, for example:
\[
(i \cdot_n j) \cdot_n k = i \cdot_n (j \cdot_n k).
\]

These subscript-$n$'s on arithmetic operations really clog things up,
so instead we'll just write ``($\Zmod{n}$)'' on the side to get a
simpler looking equation:
\[
(i \cdot j) \cdot k = i \cdot (j \cdot k) \ \inzmod{n}.
\]

\begin{editingnotes}
From here on, replace $+_n$ by ``$+ \dots \inzmod{n}$ where
possible; likewise for $\cdot_n$.
\end{editingnotes}

In particular, all of the following equalities\footnote{A set with
  addition and multiplication operations that satisfy these equalities
  is known as a \term{commutative ring}.  In addition to $\Zmod{n}$,
  the integers, rationals, reals, and polynomials with integer
  coefficients are all examples of commutative rings.  On the other
  hand, the set $\set{\true, \false}$ of truth values with $\QOR$ for
  addition and $\QAND$ for multiplication is \emph{not} a commutative
  ring because it fails to satisfy one of these equalities.  The $n
  \times n$ matrices of integers are not a commutative ring because
  they fail to satisfy another one of these equalities.} are true in
$\Zmod{n}$:
\begin{align*}
(i \cdot j) \cdot k & = i \cdot (j \cdot k) 
       & \text{(associativity of $\cdot$)},\\
        (i + j) + k & = i + (j + k)  
       & \text{(associativity of $+$)},\\
           1 \cdot k  & = k  
       & \text{(identity for $\cdot$)},\\
              0 + k  & = k  
       & \text{(identity for $+$)},\\
           k + (-k)  & = 0  
       & \text{(inverse for $+$)},\\
                i + j & = j + i  
       & \text{(commutativity of $+$)}\\
    i \cdot (j + k) & = (i \cdot j) + (i \cdot k)  
       & \text{(distributivity)},\\
            i \cdot j & = j \cdot i  
       & \text{(commutativity of $\cdot$)}\\
\end{align*}

\begin{editingnotes}
Boring verification of example equation above---maybe put in a problem:

Letting $i'$ abbreviate $\rem{i}{n}$, and remembering that $i = i'$
for $i \in \Zintvco{0}{n}$, we have for $i,j,k \in \Zintvco{0}{n}$:
\begin{align*}
i \cdot_n (j +_n k) & = i' \cdot_n (j' +_n k')\\
   & = i' \cdot_n (j' + k')' & \text{(def of $+_n$)}\\
   & = i (j' + k') & \text{(by~\eqref{.rhom})}\\  %(ij)' = i' ._n j'
   & = i j' + i k'\\
   & = ij + ik\\
   & = (ij)' +_n (ik)' & \text{(by~\eqref{+rhom})}\\  %(i+j)' = i' +_n j'
   & = (i' \cdot_ n j') +_n (i' \cdot_ n k') & \text{(by~\eqref{.rhom})}\\
   & = (i \cdot_ n j) +_n (i \cdot_ n k).
\end{align*}
\end{editingnotes}

Associativity implies the familiar fact that it's safe to omit the
parentheses in products:
\[
k_1\ \cdot\ k_2\ \cdot \cdots \cdot\ k_m
\]
comes out the same in $\Zmod{n}$ no matter how it is parenthesized.

The overall theme is that remainder arithmetic is a lot like ordinary
arithmetic.  But there are a couple of exceptions we're about to
examine.

\begin{problems}
\practiceproblems
\pinput{TP_divisibility_and_congruence}
\pinput{TP_3_mod_21}  %uncover after S16 mid3

\homeworkproblems
\pinput{PS_eval_cong_aexp}
\pinput{PS_ring_theory}
\pinput{PS_reverse_linear_recurrences}

\classproblems
\pinput{CP_remainder_computation_practice}
\pinput{CP_proving_basic_congruence_properties}
\pinput{CP_multiples_of_9_and_11}
\pinput{CP_13th_roots}
\pinput{CP_pirate_treasure}

\examproblems
\pinput{FP_check_factor_by_digits_proof}
\pinput{FP_recurrence_strong_induction}

\end{problems}

\section{\index{Turing's code}Turing's Code (Version 2.0)}

In 1940, France had fallen before Hitler's army, and Britain stood
alone against the Nazis in western Europe.  British resistance
depended on a steady flow of supplies brought across the north
Atlantic from the United States by convoys of ships.  These convoys
were engaged in a cat-and-mouse game with German ``U-boats''
---submarines---which prowled the Atlantic, trying to sink supply
ships and starve Britain into submission.  The outcome of this
struggle pivoted on a balance of information: could the Germans locate
convoys better than the Allies could locate U-boats, or vice versa?

Germany lost.

A critical reason behind Germany's loss was not made public until
1974: Germany's naval code, \emph{Enigma}, had been broken by the Polish Cipher
  Bureau,\footnote{See
  \href{http://www.bletchleypark.org.uk/content/hist/history/polish.rhtm}
       {http://en.wikipedia.org/wiki/Polish\_Cipher\_Bureau}.}  and
the secret had been turned over to the British a few weeks before the
Nazi invasion of Poland in 1939.  Throughout much of the war, the
Allies were able to route convoys around German submarines by
listening in to German communications.  The British government didn't
explain \emph{how} Enigma was broken until 1996.  When the story was
finally released (by the US), it revealed that Alan Turing had joined
the secret British codebreaking effort at Bletchley Park in 1939,
where he became the lead developer of methods for rapid, bulk
decryption of German Enigma messages.  Turing's Enigma deciphering was
an invaluable contribution to the Allied victory over Hitler.

Governments are always tight-lipped about cryptography, but the
half-century of official silence about Turing's role in breaking
Enigma and saving Britain may be related to some disturbing events
after the war---more on that later.  Let's get back to number theory
and consider an alternative interpretation of Turing's code.  Perhaps
we had the basic idea right (multiply the message by the key), but
erred in using \emph{conventional} arithmetic instead of
\emph{modular} arithmetic.  Maybe this is what Turing meant:
\begin{description}

\item[Beforehand] The sender and receiver agree on a large number $n$,
  which may be made public.  (This will be the modulus for all our
  arithmetic.)  As in Version 1.0, they also agree that some prime
  number $k < n$ will be the secret key.

\item[Encryption] As in Version 1.0, the message $m$ should be another
  prime in $\Zintvco{0}{n}$.  The sender encrypts the message $m$ to produce
    $\widehat{m}$ by computing $mk$, but this time modulo $n$:
\begin{equation}
\widehat{m} \eqdef m \cdot k \inzmod{n}\label{eq:turing-code} 
\end{equation}

\item[Decryption] (Uh-oh.)

\end{description}

The decryption step is a problem.  We might hope to decrypt in the
same way as before by dividing the encrypted message $\widehat{m}$ by the key
$k$.  The difficulty is that $\widehat{m}$ is the \emph{remainder} when $mk$
is divided by $n$.  So dividing $\widehat{m}$ by $k$ might not even give us an
integer!

This decoding difficulty can be overcome with a better understanding
of when it is ok to divide by $k$ in modular arithmetic.

\begin{problems}
\examproblems
\pinput{FP_structural_induction_polynomial}
%\pinput{FP_structural_ind_polynomials}  %same prob in terms of aexps
\end{problems}

\section{Multiplicative Inverses and Cancelling}\label{sec:inverse}
\index{inverse (multiplicative)}

%Arithmetic with a Prime Modulus}\label{mod_prime_sec}
\begin{editingnotes}
Edit for consistency: canceled vs. cancelled.
\end{editingnotes}

The \term{multiplicative inverse} of a number $x$ is another number
$x^{-1}$ such that
\[
x^{-1} \cdot x  = 1.
\]
From now on, when we say ``inverse,'' we mean \emph{multiplicative}
(not relational) inverse.

For example, over the rational numbers, $1 / 3$ is, of course, an
inverse of 3, since,
\[
\frac{1}{3} \cdot 3 = 1.
\]
In fact, with the sole exception of 0, every rational number $n/m$ has
an inverse, namely, $m/n$.  On the other hand, over the integers, only
1 and -1 have inverses.  Over the ring
$\Zmod{n}$, things get a little more complicated.  For example, 2
  is a multiplicative inverse of 8 in $\Zmod{15}$, since
\[
2 \cdot 8 = 1 \inzmod{15}.
\]

On the other hand, 3 does not have a multiplicative inverse in
$\Zmod{15}$.  We can prove this by contradiction: suppose there was an
inverse $j$ for 3, that is
\[
1 = 3 \cdot j \inzmod{15}.
\]
Then multiplying both sides of this equality by 5 leads directly to
the contradiction $5 = 0$:
\begin{align*}
5 & = 5 \cdot (3 \cdot j)\\
  & = (5 \cdot 3) \cdot j\\
  & = 0 \cdot j = 0 \inzmod{15}.
\end{align*}
So there can't be any such inverse $j$.

So some numbers have inverses modulo 15 and others don't.  This may seem
a little unsettling at first, but there's a simple explanation of
what's going on.

\subsection{Relative Primality}

Integers that have no prime factor in common are called
\emph{relatively prime}\index{prime!relatively prime}.\footnote{Other texts call them
  \emph{coprime}.}  This is the same as having no common divisor
(prime or not) greater than~1.  It's also equivalent to saying
$\gcd(a, b) = 1$.

For example, 8 and 15 are relatively prime, since $\gcd(8, 15) = 1$.
On the other hand, 3 and 15 are not relatively prime, since $\gcd(3,
15) = 3 \neq 1$.  This turns out to explain why 8 has an inverse over
$\Zmod{15}$ and 3 does not.

\begin{lemma}\label{lem:inverse-arb} If $k \in \Zintvco{0}{n}$ is relatively prime to
$n$, then $k$ has an inverse in $\Zmod{n}$.
\end{lemma}

\begin{proof}
If $k$ is relatively prime to $n$, then $\gcd(n, k) = 1$ by definition
of gcd.  This means we can use the Pulverizer from
section~\ref{sec:pulverizer} to find a linear combination of $n$ and
$k$ equal to 1:
\[
s n + t k = 1.
\]
So applying the General Principle of Remainder Arithmetic
(Lemma~\ref{rem-morphism}), we get
\[
(\rem{s}{n} \cdot \rem{n}{n}) + (\rem{t}{n} \cdot \rem{k}{n}) = 1 \inzmod{n}.
\]
But $\rem{n}{n} = 0$, and $\rem{k}{n} = k$ since $k \in \Zintvco{0}{n}$, so we
  get
\[
\rem{t}{n} \cdot k = 1 \inzmod{n}.
\]
Thus, $\rem{t}{n}$ is a multiplicative inverse of $k$.
\end{proof}

By the way, it's nice to know that when they exist, inverses are
unique.  That is,
\begin{lemma}\label{uniq-inv}
If $i$ and $j$ are both inverses of $k$ in $\Zmod{n}$, then $i=j$.
\end{lemma}

\begin{proof}
\[
i = i \cdot 1 = i \cdot (k \cdot j) = (i \cdot k) \cdot j = 1 \cdot j = j \inzmod{n}.
\]
\end{proof}

So the proof of Lemma~\ref{lem:inverse-arb} shows that for any $k$
relatively prime to $n$, the inverse of $k$ in $\Zmod{n}$ is simply
the remainder of a coefficient we can easily find using the
Pulverizer.

Working with a prime modulus is attractive here because, like the
rational and real numbers, when $p$ is prime, every nonzero number has
an inverse in $\Zmod{p}$.  But arithmetic modulo a composite is really
only a little more painful than working modulo a prime---though you
may think this is like the doctor saying, ``This is only going to hurt
a little,'' before he jams a big needle in your arm.

\subsection{Cancellation}

Another sense in which real numbers are nice is that it's ok to cancel
common factors.  In other words, if we know that $t r = t s$ for real
numbers $r,s,t$, then as long as $t \neq 0$, we can cancel the $t$'s
and conclude that $r = s$.  In general, cancellation is \emph{not}
valid in $\Zmod{n}$.  For example,
\begin{equation}\label{3103515}
3 \cdot 10 = 3 \cdot 5 \inzmod{15},
\end{equation}
but cancelling the 3's leads to the absurd conclusion that 10 equals
5.

The fact that multiplicative terms cannot be cancelled is the most
significant way in which $\Zmod{n}$ arithmetic differs from ordinary
integer arithmetic.

\begin{definition}
A number $k$ is \emph{cancellable} in $\Zmod{n}$ iff
\[
k \cdot a = k \cdot b  \qimplies a = b  \inzmod{n}
\]
for all $a,b \in \Zintvco{0}{n}$.
\end{definition}

If a number is relatively prime to 15, it can be cancelled by
multiplying by its inverse.  So cancelling works for numbers that have
inverses:
\begin{lemma}\label{lem:cancellation-arb}
If $k$ has an inverse in $\Zmod{n}$, then it is cancellable.
\end{lemma}

But 3 is not relatively prime to 15, and that's why it is not
cancellable.  More generally, if $k$ is not relatively prime to $n$,
then we can show it isn't cancellable in $\Zmod{n}$ in the same way we
showed that 3 is not cancellable in~\eqref{3103515}.

\iffalse
Namely, suppose $\gcd(k,n) = m > 1$.  So $k/m$ and $n/m$ are
positive integers, and we have
\begin{align*}
          (n/m) \cdot k & = n \cdot (k/m),\\
\rem{(n/m) \cdot k}{n} & = \rem{n \cdot (k/m)}{n},\\
        (n/m) \cdot k & = 0 = 0 \cdot k \inzmod{n}.
\end{align*}
Now $k$ can't be cancelled or we would reach the false conclusion that
$n/m = 0$.
\fi

To summarize, we have
\begin{theorem}\label{thm:mod_inverses}
The following are equivalent for $k \in \Zintvco{0}{n}$:
\begin{align*}
& \gcd(k, n) = 1,\\
& k \text{ has an inverse in $\Zmod{n}$},\\
& k \text{ is cancellable in $\Zmod{n}$}.
\end{align*}
\end{theorem}

\subsection{Decrypting (Version 2.0)}

Multiplicative inverses are the key to decryption in Turing's code.
Specifically, we can recover the original message by multiplying the
encoded message by the $\Zmod{n}$-inverse $j$ of the key:
\[
\widehat{m} \cdot j = (m \cdot k) \cdot j = m \cdot (k \cdot j) = m \cdot 1 = m \inzmod{n}.
\]
So all we need to decrypt the message is to find an inverse of the
secret key $k$, which will be easy using the Pulverizer---providing
$k$ has an inverse.  But $k$ is positive and less than the modulus
$n$, so one simple way to ensure that $k$ is relatively prime to the
modulus is to have $n$ be a prime number.

\subsection{Breaking \index{Turing's code}Turing's Code (Version 2.0)}

The Germans didn't bother to encrypt their weather reports with the
highly-secure Enigma system.  After all, so what if the Allies learned
that there was rain off the south coast of Iceland?  But amazingly,
this practice provided the British with a critical edge in the
Atlantic naval battle during 1941.

The problem was that some of those weather reports had originally been
transmitted using Enigma from U-boats out in the Atlantic.  Thus, the
British obtained both unencrypted reports and the same reports
encrypted with Enigma.  By comparing the two, the British were able to
determine which key the Germans were using that day and could read all
other Enigma-encoded traffic.  Today, this would be called a
\emph{known-plaintext attack}.

Let's see how a known-plaintext attack would work against Turing's
code.  Suppose that the Nazis know both the plain text $m$ and its

\[
\widehat{m} = m \cdot k \inzmod{n},
\]
and since $m$ is positive and less than the prime $n$, the Nazis can
use the Pulverizer to find the $\Zmod{n}$-inverse $j$ of $m$.  Now
\[
j \cdot \widehat{m} = j \cdot (m \cdot k) = (j \cdot m) \cdot k
= 1 \cdot k = k \inzmod{n}.
\]
So by computing $j \cdot \widehat{m} = k \inzmod{n}$, the Nazis get the
secret key and can then decrypt any message!

This is a huge vulnerability, so Turing's hypothetical Version 2.0
code has no practical value.  Fortunately, Turing got better at
cryptography after devising this code; his subsequent deciphering of
Enigma messages surely saved thousands of lives, if not the whole of
Britain.

%  I could insert a bit about public-key cryptography here as introduction to the
%  recitation.

\subsection{Turing Postscript}

A few years after the war, \idx{Turing}'s home was robbed.  Detectives soon
determined that a former homosexual lover of Turing's had conspired in
the robbery.  So they arrested him---that is, they arrested Alan
Turing---because at that time in Britain, homosexuality was a crime punishable by up
to two years in prison.  Turing was sentenced to a
hormonal ``treatment'' for his homosexuality: he was given estrogen
injections.  He began to develop breasts.

Three years later, Alan Turing, the founder of computer science,
was dead.  His mother explained what happened in a biography of her
own son.  Despite her repeated warnings, Turing carried out chemistry
experiments in his own home.  Apparently, her worst fear was realized:
by working with potassium cyanide while eating an apple, he poisoned
himself.

However, Turing remained a puzzle to the very end.  His mother was a
devout woman who considered suicide a sin.  And, other biographers
have pointed out, Turing had previously discussed committing suicide
by eating a poisoned apple.  Evidently, Alan Turing, who founded
computer science and saved his country, took his own life in the end,
and in just such a way that his mother could believe it was an
accident.

Turing's last project before he disappeared from public view in 1939
involved the construction of an elaborate mechanical device to test a
mathematical conjecture called the Riemann Hypothesis.  This
conjecture first appeared in a sketchy paper by Bernhard Riemann in
1859 and is now one of the most famous unsolved problems in
mathematics.

\floatingtextbox{\textboxheader{The Riemann Hypothesis}

The formula for the sum of an infinite geometric series says:
\[
1 + x + x^2 + x^3 + \cdots = \frac{1}{1-x}.
\]
Substituting $x = \frac{1}{2^s}$, $x = \frac{1}{3^s}$, $x = \frac{1}{5^s}$, and so on for
each prime number gives a sequence of equations:
\begin{align*}
1 + \frac{1}{2^s} + \frac{1}{2^{2s}} + \frac{1}{2^{3s}} + \cdots & = \frac{1}{1 - 1 / 2^s}
\\
1 + \frac{1}{3^s} + \frac{1}{3^{2s}} + \frac{1}{3^{3s}} + \cdots & = \frac{1}{1 - 1 /
  3^s} \\
1 + \frac{1}{5^s} + \frac{1}{5^{2s}} + \frac{1}{5^{3s}} + \cdots & = \frac{1}{1 -
  1 / 5^s} \\
& \vdots
\end{align*}
Multiplying together all the left-hand sides and all the right-hand sides gives:
\[
\sum_{n=1}^{\infty} \frac{1}{n^s} = \prod_{p \in \text{primes}} \paren{\frac{1}{1 - 1 /
    p^s}}.
\]
The sum on the left is obtained by multiplying out all the infinite
series and applying the Fundamental Theorem of Arithmetic.  For
example, the term $1 / 300^s$ in the sum is obtained by multiplying $1
/ 2^{2s}$ from the first equation by $1 / 3^s$ in the second and $1 /
5^{2s}$ in the third.  Riemann noted that every prime appears in the
expression on the right.  So he proposed to learn about the primes by
studying the equivalent, but simpler expression on the left.  In
particular, he regarded $s$ as a complex number and the left side as a
function $\zeta(s)$.  Riemann found that the distribution of primes
is related to values of $s$ for which $\zeta(s) = 0$, which led to his
famous conjecture:

\begin{definition}\label{RiemannHyp}
  The \term{Riemann Hypothesis}: Every nontrivial zero of the zeta
  function $\zeta(s)$ lies on the line $s = 1/2 + c i$ in the complex
  plane.
\end{definition}
A proof would immediately imply, among other things, a strong form of
the Prime Number Theorem.\index{prime!Prime Number Theorem}

Researchers continue to work intensely to settle this conjecture, as
they have for over a century.  It is another of the
\href{http://www.claymath.org/millennium/}{Millennium Problems} whose
solver will earn \$1,000,000 from the Clay Institute.}

\begin{problems}
\practiceproblems
\pinput{MQ_explain_pulverizer}
\pinput{TP_multiplicative_inverses}
\pinput{TP_Inverse_with_Linear_Combinations}
\pinput{MQ_inverse_by_pulverizer}

\classproblems
\pinput{CP_nonparallel_lines}

\end{problems}

\section{Euler's Theorem}\label{Euler_sec}

The RSA cryptosystem examined in the next section, and other current
schemes for encoding secret messages, involve computing remainders of
numbers raised to large powers.  A basic fact about remainders of
powers follows from a theorem due to Euler about congruences.

\begin{definition}
For $n>0$, define\footnote{Since 0 is not relatively prime to
  anything, $\phi(n)$ could equivalently be defined using the
  interval $\Zintvoo{0}{n}$ instead of $\Zintvco{0}{n}$.}
\[
\phi(n) \eqdef \text{the number of integers in $\Zintvco{0}{n}$, that are relatively prime
  to~$n$.}
\]
This function $\phi$ is known as \idx{Euler's $\phi$
  function}.\footnote{Some texts call it Euler's \term{totient
    function}.}
\end{definition}

<For example, $\phi(7) = 6$ because all 6 positive numbers in $\Zintvco{0}{7}$
  are relatively prime to the prime number 7.  Only 0 is not
  relatively prime to 7.   Also, $\phi(12) = 4$ since 1, 5, 7,
  and~11 are the only numbers in~$\Zintvco{0}{12}$ that are relatively prime
    to~12.

More generally, if $p$ is prime, then $\phi(p) = p - 1$ since every
positive number in $\Zintvco{0}{p}$ is relatively prime to $p$.  When $n$ is
  composite, however, the $\phi$ function gets a little complicated.
  We'll get back to it in the next section.

Euler's Theorem is traditionally stated in terms of congruence:
\begin{theorem*}[\term{Euler's Theorem}]
If $n$ and $k$ are relatively prime, then
\begin{equation}\label{cong:euler}
k^{\phi(n)} \equiv 1 \pmod{n}.
\end{equation}
\end{theorem*}

Things get simpler when we rephrase Euler's Theorem in terms of
$\Zmod{n}$.

\begin{definition}
Let $\relpr{n}$  be the integers in $\Zintvoo{0}{n}$, that are relatively
  prime to~$n$:\footnote{Some other texts use the notation $n^*$ for
  $\relpr{n}$.}
\begin{equation}\label{def:relpr}
\relpr{n} \eqdef \set{k \in \Zintvoo{0}{n}\ \suchthat\ \gcd(k,n) = 1}.
\end{equation}
\end{definition}
Consequently,
\[
\phi(n) = \Card{\relpr{n}}.
\]

\begin{theorem}[Euler's Theorem for $\Zmod{n}$\index{Euler's Theorem!for $\Zmod{n}$}]\label{thmZn:euler}
For all $k \in \relpr{n}$,
\begin{equation}\label{eqZn:euler}
    k^{\phi(n)} = 1 \inzmod{n}.
\end{equation}
\end{theorem}

Theorem~\ref{thmZn:euler} will follow from two very easy lemmas.

Let's start by observing that $\relpr{n}$ is closed under
multiplication in $\Zmod{n}$:
\begin{lemma}\label{relprimgroup}
If $j,k \in \relpr{n}$, then $j \cdot_n k \in \relpr{n}$.
\end{lemma}
There are lots of easy ways to prove this (see
Problem~\ref{MQ_relprime_closed}).

\begin{editingnotes}
\begin{definition}
Define the \index{order over $\Zmod{n}$}{\term{order} of $k$
    over $\Zmod{n}$} to be
\[
\ordmod{k}{n} \eqdef \min \set{m \geq 0 \suchthat k^m = 1}.
\]
If no power of $k$ equals 1 in $\Zmod{n}$, then $\ordmod{k}{n} \eqdef
\infty$.
\end{definition}

\begin{lemma}\label{relprime_order}
Every element of $\relpr{n}$ has finite order.

\begin{proof}
Suppose $k \in \relpr{n}$.  We need to show is that some power of $k$
equals 1.

But since $\relpr{n}$ has fewer than $n$ elements, some number must
occur twice in the list
\[
k^1,\ k^2,\ \dots,\ k^n \inzmod{n}.
\] 
That is,
\begin{equation}\label{kni+m}
k^{i+m} = k^{i} \inzmod{n}
\end{equation}
for some $m >0$ and $i \in \Zintvco{0}{n}$.  But $k$ is cancellable over
  $\Zmod{n}$, so we can cancel the first $i$ of the $k$'s on both
  sides of~\eqref{kni+m} to get
\[
k^m =1 \inzmod{n}.
\]
\end{proof}
\end{lemma}
\end{editingnotes}

\begin{definition}\label{def:kS}
For any element $k$ and subset $S$ of $\Zmod{n}$, let
\[
kS \eqdef \set{k \cdot_n s\suchthat s \in S}.
\]
\end{definition}

\begin{lemma}\label{lem:cardks} %\label{lem:permutes-arb}
If $k \in \relpr{n}$ and $S \subseteq \Zmod{n}$, then
\[
\card{kS} = \card{S}.
\]
\end{lemma}

\begin{proof}
Since $k \in \relpr{n}$, by Theorem~\ref{thm:mod_inverses} it is
cancellable.  Therefore,
\[
[ks=kt \inzmod{n}] \qimplies s=t.
\]
So mulitplying by $k$ in $\Zmod{n}$ maps all the elements of $S$ to
distinct elements of $kS$, which implies $S$ and $kS$ are the same
size.
\end{proof}

\begin{corollary}\label{cor:krelprn}
If $k \in \relpr{n}$, then
\[
k\relpr{n} = \relpr{n}.
\]
\end{corollary}

\begin{proof}
A product of elements in $\relpr{n}$ remains in $\relpr{n}$ by
Lemma~\ref{relprimgroup}.  So if $k \in \relpr{n}$, then $k\relpr{n}
\subseteq \relpr{n}$.  But by Lemma~\ref{lem:cardks}, $k\relpr{n}$ and
$\relpr{n}$ are the same size, so they must be equal.
\end{proof}

Now we can complete the proof of Euler's Theorem~\ref{thmZn:euler} for $\Zmod{n}$):
\begin{proof}
Let
\[
P \eqdef k_1 \cdot k_2 \cdots k_{\phi(n)} \inzmod{n}
\]
be the product in $\Zmod{n}$ of all the numbers in $\relpr{n}$.  Let
\[
Q \eqdef (k\cdot k_1) \cdot (k\cdot k_2) \cdots (k\cdot k_{\phi(n)}) \inzmod{n}
\]
for some $k \in \relpr{n}$.  Factoring out $k$'s immediately gives
\[
Q= k^{\phi(n)}P \inzmod{n}.
\]
But $Q$ is the same as the product of the numbers in $k\relpr{n}$, and
$k\relpr{n} = \relpr{n}$, so we realize
\iffalse by Corollary\ref{cor:krelprn} \fi
that $Q$ is the product of the same numbers as $P$, just in a different
order.  Altogether, we have
\[
P = Q = k^{\phi(n)}P \inzmod{n}.
\]
Furthermore, $P \in \relpr{n}$ by Lemma~\ref{relprimgroup},
and so it can be cancelled from both sides of this equality, giving
\[
1 = k^{\phi(n)} \inzmod{n}.
\]
\end{proof}


\iffalse
Suppose $n>1$ and $k$ is relatively prime to $n$.
Let $k_1, \dots, k_r$ be all the integers in the interval
$\Zintvco{0}{n}$ that are relatively prime to $n$.  Then the sequence of
  remainders on division by $n$ of:
\[
k_1 \cdot k,\quad
k_2 \cdot k,\quad
k_3 \cdot k, \dots,\quad
k_r \cdot k
\]
is a permutation of the sequence:
\[
k_1,\quad k_2, \dots,\quad k_r.
\]
\end{lemma}

\begin{proof}
We will show that the remainders of the numbers in the first sequence
are all distinct and are equal to some member of the sequence of
$k_j$'s.  Since the two sequences have the same length, the first must
be a permutation of the second.

First, we show that the remainders in the first sequence are all
distinct.  Suppose that $\rem{k_i k}{n} = \rem{k_j k}{n}$.  This is
equivalent to $k_i k \equiv k_j k \pmod{n}$, which implies $k_i \equiv
k_j \pmod{n}$ by Lemma~\ref{lem:cancellation-arb}.  This, in turn,
means that $k_i = k_j$ since both are in $\Zintvco{0}{n}$.  Thus, none
of the remainder terms in the first sequence is equal to any other
remainder term.

Next, we show that each remainder in the first sequence equals one of
the $k_i$.  By assumption, $k_i$ and $k$ are relatively prime to $n$,
and therefore so is $k_ik$ by Unique Factorization.  Hence,
\begin{align*}
\gcd(n, \rem{k_i k}{n}) & = \gcd(k_i k, n)
            & \text{(Lemma~\ref{lem:gcdrem})}\\
      & = 1.
\end{align*}
Since $\rem{k_i k}{n}$ is in $\Zintvco{0}{n}$ by the definition of remainder,
  and since it is relatively prime to $n$, it must be equal to one of
  the~$k_i$'s.
\end{proof}

Let $k_1, \dots, k_r$ denote all integers relatively prime to $n$
where $k_i \in \Zintvoo{0}{n}$.  Then $r = \phi(n)$, by the definition of
  $phi$.  Now
\begin{align*}
\lefteqn{k_1 \cdot k_2 \cdots k_r} \hspace{0.25in} \\
  & = \rem{k_1 \cdot k}{n} \cdot \rem{k_2 \cdot k}{n} \cdots \rem{k_r \cdot k}{n}
      & \text{(by Lemma~\ref{lem:permutes-arb})}\\
  & \equiv (k_1 \cdot k) \cdot (k_2 \cdot k) \cdot \cdots (k_r \cdot k) \pmod{n}
      & \text{(by Cor~\ref{aran})}\\
  & \equiv (k_1 \cdot k_2 \cdots k_r) \cdot k^r \pmod{n}
      & \text{(rearranging terms)}
\end{align*}

By Lemma~\ref{lem:cancellation-arb}, each of the terms $k_i$ can be
cancelled, proving the claim.
\fi

\medskip
Euler's theorem offers another way to find inverses modulo
$n$: if~$k$ is relatively prime to~$n$, then~$k^{\phi(n)-1}$ is a
$\Zmod{n}$-inverse of~$k$, and we can compute this power of $k$
efficiently using fast exponentiation.  However, this approach
requires computing $\phi(n)$.  In the next section, we'll show that
computing $\phi(n)$ is easy \emph{if} we know the prime factorization
of~$n$.  But we know that finding the factors of~$n$ is generally hard
to do when $n$~is large, and so the Pulverizer remains the best
approach to computing inverses modulo~$n$.

\subsubsection{Fermat's Little Theorem}

For the record, we mention a famous special case of Euler's Theorem
that was known to Fermat a century
earlier.

\begin{corollary}[\term{Fermat's Little Theorem}]\label{fermat_little}
Suppose $p$ is a prime and $k$ is not a multiple of $p$.  Then
\[
k^{p-1} \equiv 1 \pmod{p}.
\]
\end{corollary}

\subsection{Computing Euler's $\phi$ Function}

RSA works using arithmetic modulo the product of two large primes, so we begin with an
elementary explanation of how to compute $\phi(pq)$ for primes $p$ and $q$:

\begin{lemma}\label{phi_pq}    %{cor:H7}
\[
\phi(pq) = (p-1) (q-1)
\]
for primes $p\neq q$.
\end{lemma}

\begin{proof}
Since $p$ and $q$ are prime, any number that is not relatively prime
to $pq$ must be a multiple of~$p$ or a multiple of~$q$.  Among the
$pq$ numbers in $\Zintvco{0}{pq}$, there are precisely $q$ multiples
of~$p$ and $p$ multiples of~$q$.  Since $p$ and~$q$ are relatively
prime, the only number in~$\Zintvco{0}{pq}$ that is a multiple of both
$p$ and~$q$ is 0.  Hence, there are $p + q - 1$ numbers
in~$\Zintvco{0}{pq}$ that are \emph{not} relatively prime to~$n$.  This
means that
\begin{align*}
    \phi(pq) & = pq - (p + q - 1) \\
& = (p - 1) (q - 1),
\end{align*}
as claimed.\footnote{This proof previews a kind of counting argument that we will explore
  more fully in Part~\ref{part:counting}.}
\end{proof}

The following theorem provides a way to calculate $\phi(n)$ for arbitrary $n$.
\begin{theorem}\label{th:phi}\mbox{}
\begin{enumerate}
\item[(a)] If $p$ is a prime, then $\phi(p^k) = p^k - p^{k-1}$ for $k \geq 1$.
\item[(b)] If $a$ and $b$ are relatively prime, then $\phi(ab) = \phi(a)\phi(b)$.
\end{enumerate}
\end{theorem}

Here's an example of using Theorem~\ref{th:phi} to compute $\phi(300)$:
\begin{align*}
\phi(300) & = \phi(2^2 \cdot 3 \cdot 5^2)\\
& = \phi(2^2) \cdot \phi(3) \cdot \phi(5^2) &
\text{(by Theorem~\ref{th:phi}.(b))}\\
& = (2^2 - 2^1) (3^1 - 3^0) (5^2 - 5^1) & \text{(by
  Theorem~\ref{th:phi}.(a))}\\
& = 80.
\end{align*}

Note that Lemma~\ref{phi_pq} also follows as a special case of
Theorem~\ref{th:phi}.(b), since we know that $\phi(p) = p-1$ for any
prime $p$.

To prove Theorem~\ref{th:phi}.(a), notice that every $p$th number among the $p^k$ numbers
in $\Zintvco{0}{p^{k}}$ is divisible by $p$, and only these are divisible by $p$.  So $1/p$ of
  these numbers are divisible by $p$ and the remaining ones are not.  That is,
\[
\phi(p^{k}) = p^k - (1/p)p^k = p^k -p^{k-1}.
\]
We'll leave a proof of Theorem~\ref{th:phi}.(b) to
Problem~\ref{PS_Euler_function_multiplicativity}.

As a consequence of Theorem~\ref{th:phi}, we have
\begin{corollary}\label{cor:phi}
For any number~$n$, if $p_1$, $p_2$, \dots, $p_j$ are the (distinct)
prime factors of~$n$, then
\begin{equation*}
    \phi(n) = n \paren{1 - \frac{1}{p_1}} \paren{1 - \frac{1}{p_2}} \cdots \paren{1 -
      \frac{1}{p_j}}.
 \end{equation*}
\end{corollary}
We'll give another proof of Corollary~\ref{cor:phi} based on rules for
counting in Section~\ref{computephi_sec}.

\iffalse are all those of the form $mp$.  For $mp$ to be in the interval, $m$ can take any
value from 0 to $p^{k-1}-1$ and no others, so there are exactly $p^{k-1}$ numbers in the
interval that are divisible by $p$.  Now $\phi(p^{k})$ equals the number of remaining
elements in the interval, namely, $p^k -p^{k-1}$.  \fi

\begin{problems}
\practiceproblems
\pinput{TP_relprime_inverse}
\pinput{TP_Fermats_Little_Theorem_S13}
\pinput{MQ_modular_arithmetic}
\pinput{TP_relative_primality_6042}
\pinput{TP_relative_primality_3780}
\pinput{FP_modular_exponential}

\classproblems
\pinput{CP_Euler_theorem_calculation}
\pinput{CP_7777}
\pinput{CP_n5_last_digit}
\pinput{CP_calculating_inverses_fermat}
\pinput{FP_Euler_function}
\pinput{CP_chinese_remainder}
\pinput{FP_Fermat_primes}

\homeworkproblems
\pinput{PS_squares_modp}
\pinput{PS_Euler_function_multiplicativity}
\pinput{PS_order_divides_phi}
\pinput{CP_chinese_remainder_general}
\pinput{PS_Euler_theorem_not_rel_prime}
%draft: CP_fermat_probable_prime
\pinput{PS_generalized_stamps}
 
\examproblems
\pinput{FP_Euler_theorem_calculation}
\pinput{MQ_relprime_closed}
\pinput{FP_numbers_short_answer}
\pinput{PS_congruent_modulo_1000}
\pinput{FP_modular_powerful}
\pinput{FP_bogus_Fermat_theorem}
\pinput{FP_number_short}
\pinput{MQ_phi_even}
\pinput{MQ_Euler_function_of_6042}
\pinput{MQ_Sk_equiv_-1_mod_p}
\end{problems}

\section{RSA Public Key Encryption}\label{RSA_sec}

Turing's code did not work as he hoped.  However, his essential
idea---using number theory as the basis for cryptography---succeeded
spectacularly in the decades after his death.

In 1977, \index{Rivest, Ronald}{Ronald Rivest}, \index{Shamir,
  Adi}{Adi Shamir}, and \index{Adleman, Leonard}{Leonard Adleman} at
MIT proposed a highly secure cryptosystem, called \textbf{\idx{RSA}},
based on number theory.  The purpose of the RSA scheme is to transmit
secret messages over public communication channels.  As with Turing's
codes, the messages transmitted are nonnegative integers
of some fixed size.

Moreover, RSA has a major advantage over traditional codes:
the sender and receiver of an encrypted message need not meet beforehand
to agree on a secret key.  Rather, the receiver has both a
\emph{private key}, which they guard closely, and a \emph{public key},
which they distribute as widely as possible.  A sender wishing to
transmit a secret message to the receiver encrypts their message using
the receiver's widely-distributed public key.  The receiver can then
decrypt the received message using their closely held private key.
The use of such a \term{public key cryptography} system allows you and
Amazon, for example, to engage in a secure transaction without meeting
up beforehand in a dark alley to exchange a key.

Interestingly, RSA does not operate modulo a prime, as Turing's
hypothetical Version 2.0 may have, but rather modulo the product of
\emph{two} large primes---typically primes that are hundreds of
digits long.  Also, instead of encrypting by multiplication with a
secret key, RSA exponentiates to a secret power---which is why Euler's
Theorem is central to understanding RSA.

The scheme for RSA public key encryption appears in the box.

\begin{figure}[p]\redrawntrue
\textbox{
\begin{minipage}{\textwidth}
\textboxtitle{The RSA Cryptosystem}

A \textbf{Receiver} who wants to be able to receive secret numerical
messages creates a \emph{private key}, which they keep secret, and a
\emph{public key}, which they make publicly available.  Anyone with the
public key can then be a \textbf{Sender} who can publicly send secret
messages to the \textbf{Receiver}---even if they have never
communicated or shared any information besides the public key.

Here is how they do it:
\begin{description}

\item[Beforehand] The \textbf{Receiver} creates a public key and a private key as follows.

\begin{enumerate}

\item Generate two distinct primes, $p$ and $q$.  These are used to generate the private
  key, and they must be kept hidden.  (In current practice, $p$ and $q$ are chosen to be
  hundreds of digits long.)

\item Let $n \eqdef pq$.

\item Select an integer $e \in \Zintvco{0}{n}$ such that $\gcd(e, (p-1)(q-1)) = 1$.\\
The \emph{public key} is the pair $(e, n)$.  This should be distributed widely.

\item Let the \emph{private key} $d \in \Zintvco{0}{n}$ be the inverse of $e$
  in the ring $\Zmod{(p-1)(q-1)}$.  This private key can be found using
  the Pulverizer.  The private key $d$ should be kept hidden!

\end{enumerate}

\item[Encoding]

\iffalse

Given a message~$m$, the sender first checks that $\gcd(m, n) =
1$.\footnote{It would be very bad if $\gcd(m, n)$ equals $p$ or $q$
  since then it would be easy for someone to use the encoded message
  to compute the private key If $\gcd(m, n) = n$, then the encoded
  message would be~0, which is fairly useless.  For very large values
  of~$n$, it is extremely unlikely that $\gcd(m, n) \ne 1$.  If this
  does happen, you should get a new set of keys or, at the very least,
  add some bits to~$m$ so that the resulting message is relatively
  prime to~$n$.}
\fi

To transmit a message $m \in \Zintvco{0}{n}$ to \textbf{Receiver}, a
  \textbf{Sender} uses the public key to encrypt $m$ into a numerical
  message
\[
\widehat{m} \eqdef m^e \inzmod{n}. %\powermod{m}{e}{n}.
\]
The \textbf{Sender} can then publicly transmit $\widehat{m}$ to the \textbf{Receiver}.

\item[Decoding] The \textbf{Receiver} decrypts message $\widehat{m}$ back to message $m$ using the
  private key:
\[
m = \widehat{m}^d \inzmod{n}. %\powermod{\widehat{m}}{d}{n}.
\]
\end{description}

\end{minipage}
}
\end{figure}

If the message $m$ is relatively prime to $n$, then a simple
application of Euler's Theorem implies that this way of decoding the
encrypted message indeed reproduces the original unencrypted message.
In fact, the decoding always works---even in (the highly unlikely)
case that $m$ is not relatively prime to $n$.  The details are worked
out in Problem~\ref{CP_RSA_proving_correctness}.

Why is RSA thought to be secure?  It would be easy to figure out the
private key~$d$ if you knew $p$ and~$q$---you could do it the same
way the \textbf{Receiver} does using the Pulverizer.  But assuming the
conjecture that it is hopelessly hard to factor a number that is the
product of two primes with hundreds of digits, an effort to factor $n$
is not going to break RSA.

Could there be another approach to reverse engineer the private key
$d$ from the public key that did not involve factoring $n$?  Not
really.  It turns out that given just the private and the public keys,
it is easy to factor $n$\footnote{In practice, for this reason, the
  public and private keys should be randomly chosen so that neither is
  ``too small.''}  (a proof of this is sketched in
Problem~\ref{PS_RSA_key_implies_factoring}).  So if we are confident
that factoring is hopelessly hard, then we can be equally confident
that finding the private key just from the public key will be
hopeless.

But even if we are confident that an RSA private key won't be found,
this doesn't rule out the possibility of decoding RSA messages in a
way that sidesteps the private key.  It is an important unproven
conjecture in cryptography that \emph{any} way of cracking RSA---not
just by finding the secret key---would imply the ability to factor.
This would be a much stronger theoretical assurance of RSA security
than is presently known.

But the real reason for confidence is that RSA has withstood all
attacks by the world's most sophisticated cryptographers for nearly 40
years.  Despite decades of these attacks, no significant weakness has
been found.  That's why the mathematical, financial, and intelligence
communities are betting the family jewels on the security of RSA
encryption.

You can hope that with more studying of number theory, you will be the
first to figure out how to do factoring quickly and, among other
things, break RSA.  But be further warned that even Gauss worked on
factoring for years without a lot to show for his efforts---and if you
do figure it out, you might wind up meeting some humorless fellows
working for a Federal agency in charge of security\dots.

\begin{problems}
\practiceproblems
\pinput{MQ_factoring_cracks_RSA}
\pinput{MQ_RSA_collision_probability_200}
\pinput{TP_RSA_lost_private_key}

\classproblems
\pinput{CP_RSA_between_tables}
\pinput{CP_findphi}
\pinput{CP_RSA_proving_correctness}

\homeworkproblems
\pinput{PS_Rabin_cryptosystem}
\pinput{PS_RSA_key_implies_factoring}

\examproblems
\pinput{MQ_RSA_reversed}

\end{problems}

\section{What has SAT got to do with it?}\label{SAT_RSA_sec}
So why does society, or at least everybody's secret codes, fall apart
if there is an efficient test for satisfiability%
\index{satisfiability!SAT} 
(SAT), as we claimed
in Section~\ref{SAT_sec}?  To explain this, remember that RSA can be
managed computationally because multiplication of two primes is fast,
but factoring a product of two primes seems to be overwhelmingly
demanding.

Let's begin with the observation from
Section~\ref{propositions_in_programs_sec} that a digital circuit can
be described by a bunch of propositional formulas of about the same
total size as the circuit.  So testing circuits for satisfiability is
equivalent to the SAT problem for propositional formulas (see
Problem~\ref{CP_sat_formulas_vs_circuits}).

Now designing digital multiplication circuits is completely routine.
We can easily build a digital ``product checker'' circuit out of
\QAND, \QOR, and \QNOT\ gates with 1 output wire and $4n$ digital
input wires.  The first $n$ inputs are for the binary representation
of an integer $i$, the next $n$ inputs for the binary representation
of an integer $j$, and the remaining $2n$ inputs for the binary
representation of an integer $k$.  The output of the circuit is 1 iff
$ij=k$ and $i,j>1$.  A straightforward design for such a product
checker uses proportional to $n^2$ gates.

Now here's how to factor any number $m$ with a length $2n$ binary
representation using a SAT solver.  First, fix the last $2n$ digital
inputs---the ones for the binary representation of $k$---so that $k$
equals $m$.

Next, set the first of the $n$ digital inputs for the representation
of $i$ to be 1.  Do a SAT test to see if there is a satisfying
assignment of values for the remaining $2n-1$ inputs used for the $i$
and $j$ representations.  That is, see if the remaining inputs for $i$
and $j$ can be filled in to cause the circuit to give output 1.  If
there is such an assignment, fix the first $i$-input to be 1,
otherwise fix it to be 0.  So now we have set the first $i$-input
equal to the first digit of the binary representations of an $i$ such
that $ij=m$.

Now do the same thing to fix the second of the $n$ digital inputs for
the representation of $i$, and then third, proceeding in this way
through all the $n$ inputs for the number $i$.  At this point, we have
the complete $n$-bit binary representation of an $i>1$ such $ij=m$ for
some $j>1$.  In other words, we have found an integer $i$ that is a
factor of $m$.  We can now find $j$ by dividing $m$ by $i$.

So after $n$ SAT tests, we have factored $m$.  This means that if SAT
for digital circuits with $4n$ inputs and about $n^2$ gates could be
determined by a procedure taking a number of steps bounded above by a
degree $d$ polynomial in $n$, then $2n$ digit numbers can be factored
in $n$ times this many steps, that is, with a number of steps bounded
by a polynomial of degree $d+1$ in $n$.  So if SAT could be solved
in polynomial time, then so could factoring, and consequently RSA
would be ``easy'' to break.

\begin{editingnotes}
\TBA{Add some problems}
\end{editingnotes}

\iffalse

So multiplication is, or at least seems to be, an example of a
``one-way function.''  A function $f$ mapping length-$n$ bit-strings
to length-$n$ bit-strings for $n > 0$ is a \term{one-way function}
when $f(x)$ is easy to compute (polynomial in $n$ number of steps) for
length-$n$ strings $u$ but going the other way, that is, finding any
element in $f^{-1}(y)$ is infeasible (exponential in $n$ number of
steps) for length-$n$ strings $v$.

Password security is also usually managed with one-way functions.
Keeping a file around with people's actual passwords is a bad risk, so
instead of keeping, say Alice's password $x$ in a file next to
Alice's name, we just store $f(x)$.  When Alice logs in with password
$x$, it's easy to look up and compute $f(x)$ to verify her password.
But if someone steals the password file, all they have is Alice's
$f(x)$, and that doesn't let them find $x$.  So they can't pretend to
have Alice's password.

\fi

\begin{editingnotes}
NOTES FOR FURTHER REVISIONS:
Add summary of what's easy: exponentiating, gcd's, inverses, finding
primes (density argument plus probabilistic prime testing)

What assumed hard: factoring

Work out Prob~\ref{PS_RSA_key_implies_factoring} that explains RSA security claim: finding
key implies factoring.

Complete the chebychev bound on prime density problem.

Explain simple probabilistic fermat test (assume Carmichael nums are sparse).
\end{editingnotes}


\section{References}

 \cite{BachS1996},
 \cite{Shoup2005}

\endinput


\iffalse
\begin{editingnotes}
\subsection{\idx{Linear Combinations} and the \idx{GCD}}

The theorem below relates the greatest common divisor to linear
combinations.  This theorem is \emph{very} useful; take the time to
understand it and then remember it!

\begin{theorem} \label{th:gcd} The greatest common divisor of $a$ and $b$ is equal to
the smallest positive linear combination of $a$ and $b$.
\end{theorem}

For example, the greatest common divisor of 52 and 44 is 4.  And, sure
enough, 4 is a linear combination of 52 and 44:
\[
6 \cdot 52 + (-7) \cdot 44 = 4
\]
Furthermore, no linear combination of 52 and 44 is equal to a smaller
positive integer.

\begin{proof}[Proof of Theorem~\ref{th:gcd}]
By the Well Ordering Principle, there is a smallest positive linear
combination of $a$ and $b$; call it $m$.  We'll prove that $m =
\gcd(a, b)$ by showing both $\gcd(a, b) \leq m$ and $m \leq \gcd(a,
b)$.

First, we show that $\gcd(a, b) \leq m$.  Now any common divisor of
$a$ and $b$---that is, any $c$ such that $c \divides a$ and $c
\divides b$---will divide both $sa$ and $tb$, and therefore also
$sa+tb$ for any $s$ and~$t$.  The $\gcd(a, b)$ is by definition a
common divisor of $a$ and $b$, so
\begin{equation}\label{gcdabdivlin} \gcd(a, b)
\divides s a + t b
\end{equation}
for every $s$ and $t$.  In particular, $\gcd(a, b) \divides m$, which
implies that $\gcd(a, b) \leq m$.

Now, we show that $m \leq \gcd(a, b)$.  We do this by showing that $m
\divides a$.  A symmetric argument shows that $m \divides b$, which
means that $m$ is a common divisor of $a$ and $b$.  Thus, $m$ must be
less than or equal to the \emph{greatest} common divisor of $a$ and
$b$.

All that remains is to show that $m \divides a$.  By the Division
Theorem, there exists a quotient $q$ and remainder $r$ such that:
\[
a = q \cdot m + r \hspace{1in} \text{(where $0 \leq r < m$)} \] Recall
that $m = s a + t b$ for some integers $s$ and $t$.  Substituting in
for $m$ gives:
\begin{align*} a & = q \cdot (s a + t b) + r,
\qquad \text{so} \\
r & = (1 - qs) a + (-qt) b.
\end{align*}
We've just expressed $r$ as a linear combination of $a$ and $b$.
However, $m$ is the \emph{smallest positive} linear combination and $0
\leq r < m$.  The only possibility is that the remainder $r$ is not
positive; that is, $r = 0$.  This implies $m \divides a$.  \end{proof}

\begin{corollary}\label{cor:lin-comb-edit}
An integer is linear combination of $a$ and $b$ iff it is a multiple
of $\gcd(a, b)$.
\end{corollary}

\begin{proof} By~\eqref{gcdabdivlin}, every linear combination of $a$ and $b$ is a
multiple of $\gcd(a, b)$.  Conversely, since $\gcd(a, b)$ is a linear
combination of $a$ and $b$, every multiple of $\gcd(a, b)$ is as
well.  \end{proof}

Now we can restate the water jugs lemma in terms of the greatest
common divisor:

\begin{corollary} \label{cor:waterjugs_note} Suppose that we have water jugs with
capacities $a$ and $b$.  Then the amount of water in each jug is
always a multiple of $\gcd(a, b)$.
\end{corollary}

For example, there is no way to form 4 gallons using 3- and 6-gallon
jugs, because 4 is not a multiple of $\gcd(3, 6) = 3$.

\end{editingnotes}
\fi
