\chapter{Number Theory}\label{number_theory_chap}
\term{Number theory} is the study of the integers.  \emph{Why} anyone
would want to study the integers is not immediately obvious.  First of
all, what's to know?  There's 0, there's 1, 2, 3, and so on, and, oh yeah,
-1, -2, \dots.   Which one don't you understand?  Second, what practical
value is there in it?

\iffalse
Number theory is right at the core of mathematics; even Ug the Caveman
surely had some grasp of the integers---at least the positive ones.
In fact, the integers are so elementary that one might ask, ``What's
to study?''  There's 0, there's 1, 2, 3 and so on, and there's the
negatives.  Which one don't you understand?  Doesn't math become easy
when we don't have to worry about nasty numbers like $\sqrt{7}$, $1 /
\pi$, and $i$?  We can even forget about fractions!
\fi

The mathematician G. H. \idx{Hardy} expressed pleasure in its
impracticality when he wrote:
%
 \begin{quotation}
 \noindent [Number theorists] may be justified in rejoicing that there
 is one science, at any rate, and that their own, whose very remoteness
 from ordinary human activities should keep it gentle and clean.
 \end{quotation}
%

 Hardy was specially concerned that number theory not be used in
 warfare; he was a pacifist.  You may applaud his sentiments, but he
 got it wrong: Number Theory underlies modern cryptography, which is
 what makes secure online communication possible.  Secure
 communication is of course crucial in war---which may leave poor
 Hardy spinning in his grave.  It's also central to online commerce.
 Every time you buy a book from Amazon, use a certificate to access a
 web page, or use a PayPal account, you are relying on number
 theoretic algorithms.

Number theory also provides an excellent environment for us to
practice and apply the proof techniques that we developed in previous
Chapters.  We'll work out properties of \idx{greatest common divisors}
(gcd's) and use them to prove that integers factor uniquely into
primes.  Then we'll introduce modular arithmetic and work out enough
of its properties to explain the \idx{RSA public key crypto-system}.

%~\ref{templates_chap} and~\ref{induction_chap}.

Since we'll be focusing on properties of the integers, we'll adopt
the default convention in this chapter that \emph{variables range over
the set, $\integers$, of integers}.

%% Divisibility %%%%%%%%%%%%%%%%%%%%%%%%%%%%%%%%%%%%%%%%%%%%%%%%%%%%%%%%%%%%%%%

\section{Divisibility}\label{divisibility_sec}

The nature of number theory emerges as soon as we consider the
\term{divides} relation, where
\begin{definition}\label{divides_def}
\[
a \divides b \eqdef\quad [ ak = b \text{ for some } k].
\]
\end{definition}
The divides relation comes up so frequently that multiple synonyms for
it are used all the time.  The following phrases all say the same
thing:
\begin{itemize}
\item $a \divides b$,
\item $a$ divides $b$,
\item $a$ is a \term{divisor} of $b$,
\item $a$ is a \term{factor} of $b$,
\item $b$ is \term{divisible} by $a$,
\item $b$ is a \term{multiple} of $a$.
\end{itemize}
Some immediate consequences of Definition~\ref{divides_def} are that
$n \divides 0$, $n \divides n,$ and $1 \divides n$ for all $n \neq 0$.

Dividing seems simple enough, but let's play with this definition.
The Pythagoreans, an ancient sect of mathematical mystics, said that a
number is \index{perfect number}\term*{perfect} if it equals the sum
of its positive integral divisors, excluding itself.  For example, $6
= 1 + 2 + 3$ and $28 = 1 + 2 + 4 + 7 + 14$ are perfect numbers.  On
the other hand, 10 is not perfect because $1 + 2 + 5 = 8$, and 12 is
not perfect because $1 + 2 + 3 + 4 + 6 = 16$.  \idx{Euclid}
characterized all the \emph{even} perfect numbers around 300 BC.  But
is there an \emph{odd} perfect number?  More than two thousand years
later, we still don't know!  All numbers up to about $10^{300}$ have
been ruled out, but no one has proved that there isn't an odd perfect
number waiting just over the horizon.

So a half-page into number theory, we've strayed past the outer limits
of human knowledge!  This is pretty typical; number theory is full of
questions that are easy to pose, but incredibly difficult to
answer.\footnote{\emph{Don't Panic}---we're going to stick to some
relatively benign parts of number theory.  These super-hard unsolved
problems rarely get put on problem sets.}

Some of the greatest insights and mysteries in number theory concern
properties of \term{prime} \index{prime number}{numbers}:
\begin{definition}
A \emph{prime} is a number greater than~1 that is divisible only by
itself and~1.
\end{definition}
Several such problems are included in the box on the following page.
Interestingly, we'll see that computer scientists have found ways to
turn some of these difficulties to their advantage.

\floatingtextbox{
\textboxtitle{Famous Conjectures in Number Theory}

\begin{description}

\item[\term{Goldbach Conjecture}] Every even integer greater than two
  is equal to the sum of two primes.  For example, $4 = 2 + 2$, $6 = 3
  + 3$, $8 = 3 + 5$, etc.  The conjecture holds for all numbers up to
  $10^{16}$.  In 1939 Schnirelman proved that every even number can be
  written as the sum of not more than 300,000 primes, which was a
  start.  Today, we know that every even number is the sum of at most
  6 primes.

\item[\term{Twin Prime Conjecture}] There are infinitely many primes $p$ such
that $p + 2$ is also a prime.  In 1966 Chen showed that there are
infinitely many primes $p$ such that $p + 2$ is the product of at most
two primes.  So the conjecture is known to be \emph{almost} true!

\item[\term{Primality Testing}] There is an efficient way to determine
  whether a number is prime.  A naive search for factors of an integer
  $n$ takes a number of steps proportional to $\sqrt{n}$, which is
  exponential in the \emph{size} of $n$ in decimal or binary notation.
  All known procedures for prime checking blew up like this on various
  inputs.  Finally in 2002, an amazingly simple, new method was
  discovered by \idx{Agrawal}, \idx{Kayal}, and \idx{Saxena}, which
  showed that prime testing only required a polynomial number of
  steps.  Their paper began with a quote from \idx{Gauss} emphasizing
  the importance and antiquity of the problem even in his time---two
  centuries ago.  So prime testing is definitely not in the category
  of infeasible problems requiring an exponentially growing number of
  steps in bad cases.

\item[\term{Factoring}] Given the product of two large primes $n = pq$,
  there is no efficient way to recover the primes $p$ and $q$.  The
  best known algorithm is the ``number field sieve'', which runs in
  time proportional to:
%
\[
e^{1.9(\ln n)^{1/3} (\ln\ln n)^{2/3}}
\]
%
This is infeasible when $n$ has 300 digits or more.

\item[\term{Fermat's Last Theorem}] There are no positive integers $x$,
$y$, and $z$ such that
%
\[
x^n + y^n = z^n
\]
%
for some integer $n > 2$.  In a book he was reading around 1630,
Fermat claimed to have a proof but not enough space in the margin to
write it down.  Wiles finally gave a proof of the theorem in 1994,
after seven years of working in secrecy and isolation in his attic.
His proof did not fit in any margin.

\end{description}
}

\subsection{Facts about Divisibility}

The following lemma collects some basic facts about divisibility.

\begin{lemma}\label{lem:div}\mbox{}
\begin{enumerate}
%\item If $a \divides b$, then $a \divides bc$ for all $c$.

\item\label{lem:divtrans} If $a \divides b$ and $b \divides c$, then $a \divides c$.

\item\label{lem:divsbtc} If $a \divides b$ and $a \divides c$,
then $a \divides sb + tc$ for all $s$ and $t$.

\item\label{lem:divcancel} For all $c \neq 0$, $a \divides b$ if and only if $ca \divides cb$.
\end{enumerate}
\end{lemma}

\begin{proof}
These facts all follow directly from Definition~\ref{divides_def}, and
we'll just prove part~\ref{lem:divsbtc}.\ for practice:

Given that $a \divides b$, there is some $k_1 \in \integers$ such that
$a k_1 = b$.  Likewise, $a k_2 = c$, so
\[
sb+tc= s(k_1a) + t(k_2a) = (sk_1+tk_2)a.
\]
Therefore $sb+tc = k_3a$ where $k_3 \eqdef (sk_1+tk_2)$, which means
that
\[
a \divides sb+tc.
\]
\end{proof}

A number of the form $sb+tc$ is called an \term{integer linear
  combination} of $b$ and $c$, or a plain \emph{linear combination},
since in this chapter we're only talking integers.  So
Lemma~\ref{lem:div}.\ref{lem:divsbtc} can be rephrased as
\begin{quote}
If $a$ divides $b$ and $c$, then $a$ divides every linear combination
of $b$ and~$c$.
\end{quote}
We'll be making good use of linear combinations, so let's get the general
definition on record:
\begin{definition}\label{linear_def}
An integer $n$ is a \term{linear combination} of numbers
$b_0,\dots,b_n$ iff
\[
n = s_0b_0+s_1b_1+\cdots+s_nb_n
\]
for some integers $s_0,\dots,s_n$.
\end{definition}

\subsection{When Divisibility Goes Bad}

As you learned in elementary school, if one number does \emph{not}
evenly divide another, you get a ``quotient'' and a ``remainder'' left
over.  More precisely:
\begin{theorem}\label{division_thm}[\idx{Division Theorem}]%
\footnote{This theorem is often called the ``Division Algorithm,'' even
though it is not what we would call an algorithm.  We will take this
familiar result for granted without proof.}  Let $n$ and $d$ be
integers such that $d > 0$.  Then there exists a unique pair of
integers $q$ and $r$, such that
\begin{equation}\label{nqdr}
n = q \cdot d + r \QAND\ 0  \leq r < d.
\end{equation}
\end{theorem}
The number $q$ is called the \term{quotient} and the number $r$ is
called the \term{remainder} of $n$ divided by $d$.  We use the
notation $\qcnt{n}{d}$ for the quotient and $\rem{n}{d}$ for the
remainder.

For example, $\qcnt{2716}{10} = 271$ and $\rem{2716}{10} = 6$, since
$2716 = 271 \cdot 10 + 6$.  Similarly, $\rem{-11}{7} = 3$, since $-11
= (-2) \cdot 7 + 3$.  There is a remainder operator built into many
programming languages.  For example, ``32~\%~5'' will be familiar as
remainder notation to programmers in in Java, C, and C++; it evaluates
to $\qcnt{32}{5} =2$ in all three languages.  On the other hand, these
languages treat quotients involving negative numbers
idiosyncratically, so if you program in one those languages, remember
to stick to the definition according to the Division
Theorem~\ref{division_thm}.

The remainder on division by $n$ is a number in interval from 0 to
$n-1$.  Such intervals come up so often that it is useful to have a
simple notation for them.
\begin{align*}
(k, n) & \eqdef\quad \set{i \suchthat k < i < n}\\
[k, n) & \eqdef\quad (k,n) \union \set{k}\\
(k, n] & \eqdef\quad (k,n) \union \set{n}\\
[k,n]  & \eqdef\quad (k,n) \union \set{k,n}
\end{align*}

\subsection{Die Hard}

\emph{Die Hard~3} is just a B-grade action movie, but we think it has
an inner message: everyone should learn at least a little number
theory.  In Section~\ref{diehard_machine}, we formalized a state
machine for the \idx{Die Hard} jug-filling problem using 3 and 5
gallon jugs, and also with 3 and 9 gallon jugs, and came to different
conclusions about bomb explosions.  What's going on in general?  For
example, how about getting 4 gallons from 12- and 18-gallon jugs,
getting 32 gallons with 899- and 1147-gallon jugs, or getting 3
gallons into a jug using just 21- and 26-gallon jugs?

\iffalse
Unfortunately, Hollywood never lets go of a gimmick.  Although there
were no water jug tests in \emph{Die Hard~4: Live Free or Die Hard},
rumor has it that the
jugs will return in future sequels:
\begin{description}

\item[Die Hard~5: Die Hardest]
Bruce goes on vacation and---shockingly---happens into a terrorist
plot.  To save the day, he must make 3~gallons using 21- and 26-gallon
jugs.

\item[Die Hard~6: Die of Old Age]
Bruce must save his assisted living facility from a criminal
mastermind by forming 2 gallons with 899- and 1147-gallon jugs.

\item[Die Hard~7: Die Once and For All]
Bruce has to make 4 gallons using 3- and 6-gallon jugs.

\end{description}\fi

It would be nice if we could solve all these silly water jug questions
at once.  \iffalse In particular, how can one form $g$ gallons using
jugs with capacities $a$ and~$b$?\fi This is where number theory comes
in handy.

\subsubsection{Finding an \idx{Invariant} Property}\label{jug_invar_subsubsec}

Suppose that we have water jugs with capacities $a$ and $b$ with $b
\geq a$.  Let's carry out some sample operations of the state machine
and see what happens, assuming the $b$-jug is big enough:
%
\begin{align*}
(0,0)
& \rightarrow (a,0) & \text{fill first jug} \\
& \rightarrow (0,a) & \text{pour first into second} \\
& \rightarrow (a, a) & \text{fill first jug} \\
& \rightarrow (2a-b, b) & \text{pour first into second (assuming $2a \geq b$)} \\
& \rightarrow (2a-b, 0) & \text{empty second jug} \\
& \rightarrow (0, 2a-b) & \text{pour first into second} \\
& \rightarrow (a, 2a-b) & \text{fill first} \\
& \rightarrow (3a-2b, b) & \text{pour first into second (assuming $3a \geq 2b$)}
\end{align*}
What leaps out is that at every step, the amount of water in each jug
is of a linear combination of $a$ and $b$.  This is easy to prove by
induction on the number of transitions:
\begin{lemma}[Water Jugs]\label{lem:waterjugs}
In the \emph{\idx{Die Hard}} state machine of
Section~\ref{diehard_machine} with jugs of sizes $a$ and $b$, the
amount of water in each jug is always a linear combination of $a$ and
$b$.\end{lemma}

\begin{proof}
The induction hypothesis, $P(n)$, is the proposition that after $n$
transitions, the amount of water in each jug is a linear combination
of $a$ and $b$.

\inductioncase{Base case} ($n = 0$):  $P(0)$ is true, because both jugs are
initially empty, and $0 \cdot a + 0 \cdot b = 0$.

\inductioncase{Inductive step}:  Suppose the machine is in state
$(x,y)$ after $n$ steps, that is, the little jug contains $x$ gallons
and the big one contains $y$ gallons.
There are two cases:

\begin{itemize}

\item If we fill a jug from the fountain or empty a jug into the
fountain, then that jug is empty or full.  The amount in the other jug
remains a linear combination of $a$ and $b$.  So $P(n+1)$ holds.

\item Otherwise, we pour water from one jug to another until one is
  empty or the other is full.  By our assumption, the amount $x$ and
  $y$ in each jug is a linear combination of $a$ and $b$ before we
  begin pouring.  After pouring, one jug is either empty (contains 0
  gallons) or full (contains $a$ or $b$ gallons).  Thus, the other jug
  contains either $x + y$ gallons, $x + y - a$, or $x + y - b$
  gallons, all of which are linear combinations of $a$ and $b$ since
  $x$ and $y$ are.  So $P(n+1)$ holds in this case as well.
\end{itemize}
Since $P(n+1)$ holds in any case, this proves the inductive step,
completing the proof by induction.
\end{proof}

So we have established that the jug problem has an invariant property,
namely that the amount of water in every jug is always a linear
combination of the capacities of the jugs.
Lemma~\ref{lem:waterjugs} has an important corollary:
\begin{corollary*}
Getting 4 gallons from 12- and 18-gallon jugs, and likewise
getting 32 gallons from 899- and 1147-gallon jugs,
\begin{center}
\textbf{Bruce dies!}
\end{center}
\end{corollary*}

\begin{proof}
By the Water Jugs Lemma~\ref{lem:waterjugs}, with 12- and 18-gallon
jugs, the amount in any jug is a linear combination of 12 and 18.
This is always a multiple of 6 by
Lemma~\ref{lem:div}.\ref{lem:divsbtc}, so Bruce can't get 4 gallons.
Likewise, the amount in any jug using 899- and 1147-gallon jugs is a
multiple of 31, so he can't get 32 either.
\end{proof}

But the Water Jugs Lemma isn't very satisfying.  One problem is that
it leaves the question of getting 3 gallons into a jug using just 21-
and 26-gallon jugs unresolved, since the only positive factor of both
21 and 26 is 1, and of course 1 divides 3.  A bigger problem is that
we've just managed to recast a pretty understandable question about
water jugs into a complicated question about linear combinations.
This might not seem like a lot of progress.  Fortunately, linear
combinations are closely related to something more familiar, namely
greatest common divisors, and these will help us solve the general
water jug problem.

%% Problems %%%%%%%%%%%%%%%%%%%%%%%%%%%%%%%%%%%%%%%%%%%%%%%%%%%%%%%%%%%%%%%%%%%
\begin{problems}
\practiceproblems
\pinput{TP_linear-combs-combined}

\classproblems
\pinput{CP_perfect_numbers}

\end{problems}

%% The Greatest Common Divisor %%%%%%%%%%%%%%%%%%%%%%%%%%%%%%%%%%%%%%%%%%%%%%%%
\section{The Greatest Common Divisor}
\label{sec:gcd}

A \term{common divisor} of~$a$ and~$b$ is a number that divides them
both.  The \emph{greatest common divisor (gcd)} of $a$ and~$b$ is
written $\gcd(a, b)$.  For example, $\gcd(18, 24) = 6$.  The gcd turns
out to be a very valuable piece of information about the relationship
between $a$ and $b$ and for reasoning about integers in general.  So
we'll be making lots of arguments about gcd's in what follows.

%% \subsection{\idx{Linear Combinations} and the \idx{GCD}}

%% The theorem below relates the greatest common divisor to linear
%% combinations.  This theorem is \emph{very} useful; take the time to
%% understand it and then remember it!

%% \begin{theorem}
%% \label{th:gcd}
%% The greatest common divisor of $a$ and $b$ is equal to the smallest
%% positive linear combination of $a$ and $b$.
%% \end{theorem}

%% For example, the greatest common divisor of 52 and 44 is 4.  And, sure
%% enough, 4 is a linear combination of 52 and 44:
%% %
%% \[
%% 6 \cdot 52 + (-7) \cdot 44  =  4
%% \]
%% %
%% Furthermore, no linear combination of 52 and 44 is equal to a smaller
%% positive integer.

%% \begin{proof}[Proof of Theorem~\ref{th:gcd}]
%% By the Well Ordering Principle, there is a smallest positive linear
%% combination of $a$ and $b$; call it $m$.  We'll prove that $m = \gcd(a,
%% b)$ by showing both $\gcd(a, b) \leq m$ and $m \leq \gcd(a, b)$.

%% First, we show that $\gcd(a, b) \leq m$.  Now any common divisor of
%% $a$ and $b$---that is, any $c$ such that $c \divides a$ and
%% $c \divides b$---will divide both $sa$ and $tb$, and therefore also
%% $sa+tb$ for any $s$ and~$t$.  The $\gcd(a, b)$ is by
%% definition a common divisor of $a$ and $b$, so
%% %
%% \begin{equation}\label{gcdabdivlin}
%% \gcd(a, b) \divides s a + t b
%% \end{equation}
%% for every $s$ and $t$.
%% %
%% In particular, $\gcd(a, b) \divides m$, which implies that $\gcd(a, b)
%% \leq m$.

%% Now, we show that $m \leq \gcd(a, b)$.  We do this by showing that $m
%% \divides a$.  A symmetric argument shows that $m \divides b$, which means
%% that $m$ is a common divisor of $a$ and $b$.  Thus, $m$ must be less than
%% or equal to the \emph{greatest} common divisor of $a$ and $b$.

%% All that remains is to show that $m \divides a$.  By the Division
%% Algorithm, there exists a quotient $q$ and remainder $r$ such that:
%% %
%% \[
%% a = q \cdot m + r \hspace{1in} \text{(where $0 \leq r < m$)}
%% \]
%% %
%% Recall that $m = s a + t b$ for some integers $s$ and $t$.
%% Substituting in for $m$ gives:
%% %
%% \begin{align*}
%% a & = q \cdot (s a + t b) + r, \qquad \text{so} \\
%% r & = (1 - qs) a + (-qt) b.
%% \end{align*}
%% %
%% We've just expressed $r$ as a linear combination of $a$ and $b$.
%% However, $m$ is the \emph{smallest positive} linear combination and
%% $0 \leq r < m$.  The only possibility is that the remainder $r$ is not
%% positive; that is, $r = 0$.  This implies $m \divides a$.
%% \end{proof}

%% \begin{corollary}\label{cor:lin-comb}
%% An integer is linear combination of $a$ and $b$ iff it is a multiple
%% of $\gcd(a, b)$.
%% \end{corollary}

%% \begin{proof}
%% By~\eqref{gcdabdivlin}, every linear combination of $a$ and $b$ is a
%% multiple of $\gcd(a, b)$.  Conversely, since $\gcd(a, b)$ is a linear
%% combination of $a$ and $b$, every multiple of $\gcd(a, b)$ is as well.
%% \end{proof}

%% Now we can restate the water jugs lemma in terms of the greatest
%% common divisor:
%% \begin{corollary}
%% \label{cor:waterjugs}
%% Suppose that we have water jugs with capacities $a$ and $b$.  Then the
%% amount of water in each jug is always a multiple of $\gcd(a, b)$.
%% \end{corollary}

%% For example, there is no way to form 4 gallons using 3- and 6-gallon
%% jugs, because 4 is not a multiple of $\gcd(3, 6) = 3$.

%% \subsection{Properties of the \idx{Greatest Common Divisor}}

%% We'll often make use of some basic $\gcd$ facts:

%% \begin{lemma} The following statements about the greatest common divisor hold:
%% \label{lem:gcd}
%% %
%% \begin{enumerate}
%% \item Every common divisor of $a$ and $b$ divides $\gcd(a, b)$.
%% \item\label{gcd2} $\gcd(k a, k b) = k \cdot \gcd(a, b)$ for all $k > 0$.
%% \item\label{gcd3} If $\gcd(a, b) = 1$ and $\gcd(a, c) = 1$, then $\gcd(a, bc) =
%% 1$.
%% \item\label{gcd4} If $a \divides b c$ and $\gcd(a, b) = 1$, then $a \divides c$.
%% \item\label{gcd5} $\gcd(a, b) = \gcd(b, \rem{a}{b})$.
%% \end{enumerate}
%% \end{lemma}

%% Here's the trick to proving these statements: translate the $\gcd$
%% world to the linear combination world using Theorem~\ref{th:gcd},
%% argue about linear combinations, and then translate back using
%% Theorem~\ref{th:gcd} again.

%% \begin{proof}
%% We prove only parts~\ref{gcd3}.\ and~\ref{gcd4}.

%% \textbf{Proof of~\ref{gcd3}}.  The assumptions together with Theorem~\ref{th:gcd} imply
%% that there exist integers $s$, $t$, $u$, and $v$ such that:
%% %
%% \begin{align*}
%% s a + t b & = 1 \\
%% u a + v c & = 1
%% \end{align*}
%% %
%% Multiplying these two equations gives:
%% \[
%% (s a + t b)(u a + v c) = 1
%% \]
%% %
%% The left side can be rewritten as $a \cdot (a s u + b t u + c s v) + b c
%% (t v)$.  This is a linear combination of $a$ and $b c$ that is equal to 1,
%% so $\gcd(a, bc) = 1$ by Theorem~\ref{th:gcd}.

%% \textbf{Proof of~\ref{gcd4}}.  Theorem~\ref{th:gcd} says that $\gcd(ac, bc)$ is equal to a
%% linear combination of $ac$ and $bc$.  Now $a \divides ac$ trivially
%% and $a
%% \divides bc$ by assumption.  Therefore, $a$ divides \emph{every} linear
%% combination of $ac$ and $bc$.  In particular, $a$ divides $\gcd(ac, bc) =
%% c \cdot \gcd(a, b) = c\cdot 1 = c$.  The first equality uses part~\ref{gcd2}.\ of
%% this lemma, and the second uses the assumption that $\gcd(a, b) = 1$.
%% \end{proof}

\subsection{Euclid's Algorithm}
The first thing to figure out is how to find gcd's.  A good way called
\term{Euclid's Algorithm} has been known for several thousand years.
It is based on the following elementary observation.

\begin{lemma}\label{lem:gcd}
\[
\gcd(a, b) = \gcd(b, \rem{a}{b}).
\]

\begin{proof}
By the Division Theorem~\ref{division_thm}, 
\begin{equation}\label{aqbrprf}
a = qb + r
\end{equation}
where $r = \rem{a}{b}$.  So $a$ is a linear combination of $b$ and
$r$, which implies that any divisor of $b$ and $r$ is a divisor of $a$
by Lemma~\ref{lem:div}.\ref{lem:divsbtc}.  Likewise, $r$ is a linear
combination, $a-qb$, of $a$ and $b$, so any divisor of $a$ and $b$ is
a divisor of $r$.  This means that $a$ and $b$ have the same common
divisors as $b$ and $r$, and so they have the same \emph{greatest}
common divisor.
\end{proof}
\end{lemma}

Lemma~\ref{lem:gcd} is useful for quickly computing the greatest
common divisor of two numbers.  For example, we could compute the
greatest common divisor of 1147 and~899 by repeatedly applying it:
\begin{align*}
\gcd(1147, 899)
    &= \gcd\bigl(899, \underbrace{\rem{1147}{899}}_{{} = 248}\bigr) \\
    &= \gcd\bigl(248, \underbrace{\rem{899}{248}}_{{} = 155}\bigr) \\
    &= \gcd\bigl(155, \underbrace{\rem{248}{155}}_{{} = 93}\bigr) \\
    &= \gcd\bigl(93,  \underbrace{\rem{155}{93}}_{{} = 62}\bigr) \\
    &= \gcd\bigl(62,  \underbrace{\rem{93}{62}}_{{} = 31}\bigr) \\
    &= \gcd\bigl(31,  \underbrace{\rem{62}{31}}_{{} = 0}\bigr) \\
    &= \gcd(31, 0) \\
    &= 31
\end{align*}
The last equation might look wrong, but 31 is a divisor of both 31
and~0 since every integer divides~0.  This calculation that
$\gcd(1147, 899) = 31$ was how we figured out that with water jugs of
sizes 1147 and 899, Bruce dies trying to get 32 gallons.

Euclid's algorithm can easily be formalized as a state machine.  The
set of states is $\naturals^2$ and there is one transition rule:
\begin{equation}\label{euclid_transition}
(x,y) \movesto (y, \rem{x}{y}),
\end{equation}
for $y>0$.  By Lemma~\ref{lem:gcd}, the gcd stays the same from one
state to the next.  That means the predicate
\[
\gcd(x,y) = \gcd(a,b)
\]
is a preserved invariant on the states $(x,y)$.  This preserved
invariant is, of course, true in the start state $(a,b)$.  So by the
Invariant Principle, if $y$ ever becomes $0$, the invariant will be
true and so
\[
x = \gcd(x,0) = \gcd(a,b).
\]
Namely, the value of $x$ will be the desired gcd.

What's more, $x$, and therefore also $y$, gets to be 0 pretty fast.
To see why, note that after two transitions~\eqref{euclid_transition},
the first coordinate of the state is $\rem{x}{y}$.  But
\begin{equation}\label{rxylx2}
\rem{x}{y} \le x/2 \qquad \text{for $0 < y \le x$}.
\end{equation}
This is immediate if $y \le x/2$ since the $\rem{x}{y}<y$ by
definition.  On the other hand, if $y > x/2$, then $\rem{x}{y} =
x - y < x/2$.  So $x$ gets halved or smaller every two steps, which
implies that after at most $2 \log a$ transitions, $x$ will reach its
minimum possible value, and at most one more transition will be
possible.  It follows that Euclid's algorithm terminates after at most
$1+2 \log a$ transitions.
\footnote{A tighter analysis shows that at most $\log_\varphi(a)$
  transitions are possible where $\varphi$ is the \term{golden ratio}
  $(1 + \sqrt{5})/2$ , see Problem~\ref{PS_gcd_termination}.}

But applying Euclid's algorithm to 26 and 21 gives
\[
\gcd(26, 21) = \gcd(21, 5) = \gcd(5, 1) = 1,
\]
which is why we left the 21- and 26-gallon jug problem unresolved.  To
resolve the matter, we will need more number theory.

\subsection{The Pulverizer}\label{sec:pulverizer}
We will get a lot of mileage out of the following key fact:
\begin{theorem}\label{gcd_is_lin_thm}
The greatest common divisor of $a$ and $b$ is a linear combination of
$a$ and~$b$.  That is,
\[
\gcd(a, b)  =  s a + t b,
\]
for some integers $s$ and $t$.
\end{theorem}

We already know from Lemma~\ref{lem:div}.\ref{lem:divsbtc} that every
linear combination of $a$ and $b$ is divisible by any common factor of
$a$ and $b$, so it is certainly divisible by the greatest of these
common divisors.  Since any constant multiple of a linear combination
is also a linear combination, Theorem~\ref{gcd_is_lin_thm} implies that
any multiple of the gcd is a linear combination.  So we have the
immediate corollary:
\begin{corollary}\label{cor:lin-comb}
An integer is a linear combination of $a$ and $b$ iff it is a multiple of
$\gcd(a, b)$.
\end{corollary}

We'll prove Theorem~\ref{gcd_is_lin_thm} directly by explaining how to
find $s$ and $t$.  This job is tackled by a mathematical tool that
dates to sixth-century India, where it was called \emph{kuttak}, which
means ``The Pulverizer''.  Today, the Pulverizer is more commonly
known as ``the extended Euclidean GCD algorithm'', because it is so
close to Euclid's Algorithm.

For example, following Euclid's Algorithm, we can compute the GCD of
259 and~70 as follows:
\begin{align*}
\gcd(259, 70)
    & = \gcd(70, 49) & \quad & \text{since $\rem{259}{70} = 49$}\\
    & = \gcd(49, 21) && \text{since $\rem{70}{49} = 21$} \\
    & = \gcd(21, 7) && \text{since $\rem{49}{21} = 7$} \\
    & = \gcd(7, 0) && \text{since $\rem{21}{7} = 0$} \\
    & = 7.
\end{align*}
The Pulverizer goes through the same steps, but requires some extra
bookkeeping along the way: as we compute $\gcd(a, b)$, we keep track
of how to write each of the remainders (49, 21, and 7, in the example)
as a linear combination of $a$ and $b$.  This is worthwhile, because
our objective is to write the last nonzero remainder, which is the
GCD, as such a linear combination.  For our example, here is this
extra bookkeeping:
\[
\begin{array}{ccccrcl}
x & \quad & y & \quad & (\rem{x}{y}) & = & x - q \cdot y \\ \hline
259 && 70 && 49 & = &   259 - 3 \cdot 70 \\
70 && 49 && 21  & = &   70 - 1 \cdot 49 \\
&&&&            & = &   70 - 1 \cdot (259 - 3 \cdot 70) \\
&&&&            & = &   -1 \cdot 259 + 4 \cdot 70 \\
49 && 21 && 7   & = &   49 - 2 \cdot 21 \\
&&&&            & = &   (259 - 3 \cdot 70) -
                                2 \cdot (-1 \cdot 259 + 4 \cdot 70) \\
&&&&            & = &   \fbox{$3 \cdot 259 - 11 \cdot 70$} \\
21 && 7 && 0
\end{array}
\]
We began by initializing two variables, $x = a$ and $y = b$.  In the
first two columns above, we carried out Euclid's algorithm.  At each
step, we computed $\rem{x}{y}$, which can be written in the form $x - q
\cdot y$.  (Remember that the Division Algorithm says $x = q \cdot y +
r$, where $r$ is the remainder.  We get $r = x - q \cdot y$ by
rearranging terms.)  Then we replaced $x$ and $y$ in this equation
with equivalent linear combinations of $a$ and $b$, which we already
had computed.  After simplifying, we were left with a linear
combination of $a$ and $b$ that was equal to the remainder as desired.
The final solution is boxed.

This should make it pretty clear how and why the Pulverizer works.
Anyone who has doubts can work out
Problem~\ref{PS_pulverizer_machine}, where the Pulverizer is
formalized as a state machine and then verified using an invariant
that is an extension of the one used for Euclid's algorithm.

Since the Pulverizer requires only a little more computation than
Euclid's algorithm, you can ``pulverize'' very large numbers very
quickly by using this algorithm.  As we will soon see, its speed makes
the Pulverizer a very useful tool in the field of cryptography.

Now we can restate the Water Jugs Lemma~~\ref{lem:waterjugs} in terms
of the greatest common divisor:
\begin{corollary}\label{cor:waterjugs}
Suppose that we have water jugs with capacities $a$ and $b$.  Then the
amount of water in each jug is always a multiple of $\gcd(a, b)$.
\end{corollary}
For example, there is no way to form 4 gallons using 3- and 6-gallon
jugs, because 4 is not a multiple of $\gcd(3, 6) = 3$.

\subsection{One Solution for All Water Jug Problems}\label{all_jugs_son_sec}

Corollary~\ref{cor:lin-comb} says that 3 can be written as a linear
combination of 21 and 26, since 3 is a multiple of $\gcd(21, 26) = 1$.
So the Pulverizer will give us integers $s$ and $t$ such that
\begin{equation}\label{3s21t26}
3 = s \cdot 21 + t \cdot 26
\end{equation}

Now the coefficient $s$ could be either positive or negative.
However, we can readily transform this linear combination into an
equivalent linear combination
\begin{equation}\label{3sprime21}
3 = s' \cdot 21 + t' \cdot 26
\end{equation}
where the coefficient $s'$ is positive.  The trick is to notice that
if in equation~\eqref{3s21t26} we increase $s$ by 26 and decrease $t$
by 21, then the value of the expression $s \cdot 21 + t \cdot 26$ is
unchanged overall.  Thus, by repeatedly increasing the value of $s$
(by 26 at a time) and decreasing the value of $t$ (by 21 at a time),
we get a linear combination $s' \cdot 21 + t' \cdot 26 = 3$ where the
coefficient $s'$ is positive.  Of course $t'$ must then be negative;
otherwise, this expression would be much greater than 3.

Now we can form 3 gallons using jugs with capacities 21 and~26:  We
simply repeat the following steps $s'$ times:
\begin{enumerate}
\item Fill the 21-gallon jug.
\item Pour all the water in the 21-gallon jug into the 26-gallon jug.
If at any time the 26-gallon jug becomes full, empty it out, and
continue pouring the 21-gallon jug into the 26-gallon jug.
\end{enumerate}
At the end of this process, we must have have emptied the 26-gallon
jug exactly $-t'$ times.  Here's why: we've taken $s' \cdot 21$
gallons of water from the fountain, and we've poured out some multiple
of 26 gallons.  If we emptied fewer than $-t'$ times, then
by~\eqref{3sprime21}, the big jug would be left with at least $3+26$
gallons, which is more than it can hold; if we emptied it more times,
the big jug would be left containing at most $3-26$ gallons, which is
nonsense.  But once we have emptied the 26-gallon jug exactly
$-t'$ times, equation~\eqref{3sprime21} implies that there are
exactly 3 gallons left.

Remarkably, we don't even need to know the coefficients $s'$ and $t'$
in order to use this strategy!  Instead of repeating the outer loop
$s'$ times, we could just repeat \emph{until we obtain 3 gallons},
since that must happen eventually.  Of course, we have to keep track
of the amounts in the two jugs so we know when we're done.  Here's the
solution that approach gives:
\[
\begin{array}{ccccccccc}
\lefteqn{(0,0)  \xrightarrow{\text{fill 21}}}\\
& (21,0)& \xrightarrow{\text{pour 21 into 26}} & (0,21)\\
& \xrightarrow{\text{fill 21}} & (21,21)& \xrightarrow{\text{pour 21 to 26}} & (16,26)& \xrightarrow{\text{empty 26}} & (16,0)& \xrightarrow{\text{pour 21 to 26}} & (0,16)\\
& \xrightarrow{\text{fill 21}} & (21,16)& \xrightarrow{\text{pour 21 to 26}} & (11,26)& \xrightarrow{\text{empty 26}} & (11,0)& \xrightarrow{\text{pour 21 to 26}} & (0,11)\\
& \xrightarrow{\text{fill 21}} & (21,11)& \xrightarrow{\text{pour 21 to 26}} & (6,26)& \xrightarrow{\text{empty 26}} & (6,0)& \xrightarrow{\text{pour 21 to 26}} & (0,6)\\
& \xrightarrow{\text{fill 21}} & (21,6)& \xrightarrow{\text{pour 21 to 26}} & (1,26)& \xrightarrow{\text{empty 26}} & (1,0)& \xrightarrow{\text{pour 21 to 26}} & (0,1)\\
& \xrightarrow{\text{fill 21}} & (21,1)& \xrightarrow{\text{pour 21 to 26}} & (0,22)\\
& \xrightarrow{\text{fill 21}} & (21,22)& \xrightarrow{\text{pour 21 to 26}} & (17,26)& \xrightarrow{\text{empty 26}} & (17,0)& \xrightarrow{\text{pour 21 to 26}} & (0,17)\\
& \xrightarrow{\text{fill 21}} & (21,17)& \xrightarrow{\text{pour 21 to 26}} & (12,26)& \xrightarrow{\text{empty 26}} & (12,0)& \xrightarrow{\text{pour 21 to 26}} & (0,12)\\
& \xrightarrow{\text{fill 21}} & (21,12)& \xrightarrow{\text{pour 21 to 26}} & (7,26)& \xrightarrow{\text{empty 26}} & (7,0)& \xrightarrow{\text{pour 21 to 26}} & (0,7)\\
& \xrightarrow{\text{fill 21}} & (21,7)& \xrightarrow{\text{pour 21 to 26}} & (2,26)& \xrightarrow{\text{empty 26}} & (2,0)& \xrightarrow{\text{pour 21 to 26}} & (0,2)\\
& \xrightarrow{\text{fill 21}} & (21,2)& \xrightarrow{\text{pour 21 to 26}} & (0,23)\\
& \xrightarrow{\text{fill 21}} & (21,23)& \xrightarrow{\text{pour 21 to 26}} & (18,26)& \xrightarrow{\text{empty 26}} & (18,0)& \xrightarrow{\text{pour 21 to 26}} & (0,18)\\
& \xrightarrow{\text{fill 21}} & (21,18)& \xrightarrow{\text{pour 21 to 26}} & (13,26)& \xrightarrow{\text{empty 26}} & (13,0)& \xrightarrow{\text{pour 21 to 26}} & (0,13)\\
& \xrightarrow{\text{fill 21}} & (21,13)& \xrightarrow{\text{pour 21 to 26}} & (8,26)& \xrightarrow{\text{empty 26}} & (8,0)& \xrightarrow{\text{pour 21 to 26}} & (0,8)\\
& \xrightarrow{\text{fill 21}} & (21,8)& \xrightarrow{\text{pour 21 to 26}} & (3,26)& \xrightarrow{\text{empty 26}} & (3,0)& \xrightarrow{\text{pour 21 to 26}} & (0,3)
\end{array}
\]

The same approach works regardless of the jug capacities and even
regardless the amount we're trying to produce!  Simply repeat these
two steps until the desired amount of water is obtained:
\begin{enumerate}
\item Fill the smaller jug.

\item Pour all the water in the smaller jug into the larger jug.
If at any time the larger jug becomes full, empty it out, and continue
pouring the smaller jug into the larger jug.
\end{enumerate}
By the same reasoning as before, this method eventually generates
every multiple ---up to the size of the larger jug ---of the greatest
common divisor of the jug capacities, namely, all the quantities we
can possibly produce.  No ingenuity is needed at all!

%% Problems %%%%%%%%%%%%%%%%%%%%%%%%%%%%%%%%%%%%%%%%%%%%%%%%%%%%%%%%%%%%%%%%%%%

\begin{problems}

\classproblems
\pinput{CP_use_the_pulverizer}
\pinput{CP_proving_basic_gcd_properties}

\homeworkproblems
%\pinput{PS_filling_buckets_with_water}
\pinput{PS_pulverizer_machine}
\pinput{PS_gcd_termination}

\end{problems}


%% The Fundamental Theorem of Arithmetic %%%%%%%%%%%%%%%%%%%%%%%%%%%%%%%%%%%%%%
\section{The Fundamental Theorem of Arithmetic}\label{fundamental_theorem_sec}

We now have almost enough tools to prove something that you probably
already know, namely, that every number has a unique prime
factorization.  

Let's state this more carefully.  A sequence of numbers is
\emph{\idx{weakly decreasing}} when each number in the sequence is
$\ge$ the numbers after it.  Note that a sequence of just one number
as well as a sequence of no numbers ---the empty sequence ---is weakly
decreasing by this definition.

\begin{theorem}\label{thm:unique_factor}[\idx{Fundamental Theorem of Arithmetic}]
Every positive integer is a product of a \emph{unique} weakly decreasing
  sequence of primes.
\end{theorem}

The Fundamental Theorem is also called the \term{Unique Factorization
  Theorem}, which is both a more descriptive and less pretentious name
---but hey, we really want to get your attention to the importance and
non-obviousness of unique factorization.

Notice that the theorem would be false if 1 were considered a prime;
for example, $15$ could be written as $5 \cdot 3$, or $5 \cdot 3 \cdot
1$, or $5 \cdot 3 \cdot 1 \cdot 1$, \dots.

There is a certain wonder in the Fundamental Theorem, even if you've
known it since you were in a crib.  Primes show up erratically in the
sequence of integers.  In fact, their distribution seems almost
random:
\[
2, 3, 5, 7, 11, 13, 17, 19, 23, 29, 31, 37, 41, 43, \dots
\]
Basic questions about this sequence have stumped humanity for
centuries.  And yet we know that every natural number can be built up
from primes in \emph{exactly one way}.  These quirky numbers are the
building blocks for the integers.

The Fundamental Theorem is not hard to prove, but we'll need a couple
of preliminary facts.

\floatingtextbox{
\textboxtitle{The Prime Number Theorem}

Let $\pi(x)$ denote the number of primes less than or equal to $x$.
For example, $\pi(10) = 4$ because 2, 3, 5, and 7 are the primes less
than or equal to 10.  Primes are very irregularly distributed, so the
growth of $\pi$ is similarly erratic.  However, the Prime Number
Theorem gives an approximate answer:
%
\[
\lim_{x\to\infty} \frac{\pi(x)}{x/\ln x} = 1
\]
%
Thus, primes gradually taper off.  As a rule of thumb, about 1 integer
out of every $\ln x$ in the vicinity of $x$ is a prime.

% The accent on Vallee screwed up the hyphens in the entire pdf file!!!

The Prime Number Theorem was conjectured by Legendre in 1798 and
proved a century later by de la Vallee Poussin and Hadamard in
1896.  However, after his death, a notebook of Gauss was found to
contain the same conjecture, which he apparently made in 1791 at age
15.  (You sort of have to feel sorry for all the otherwise ``great''
mathematicians who had the misfortune of being contemporaries of
Gauss.)

In late 2004 a billboard appeared in various locations around the
country:
%
{\Large
\[
\left\{
\begin{array}{c}
\text{first 10-digit prime found} \\
\text{in consecutive digits of $e$}
\end{array}
\right\}\textbf{. com}
\]
}
%
Substituting the correct number for the expression in curly-braces
produced the URL for a Google employment page.  The idea was that
Google was interested in hiring the sort of people that could and
would solve such a problem.

How hard is this problem?  Would you have to look through thousands or
millions or billions of digits of $e$ to find a 10-digit prime?  The
rule of thumb derived from the Prime Number Theorem says that among
10-digit numbers, about 1 in
%
\[
\ln 10^{10} \approx 23
\]
%
is prime.  This suggests that the problem isn't really so hard!  Sure
enough, the first 10-digit prime in consecutive digits of $e$ appears
quite early:
%
% Checked against http://www.gutenberg.org/1/2/127/ --dmj,
%
\begin{align*}
e = & 2.718281828459045235360287471352662497757247093699959574966 \\
    & 96762772407663035354759457138217852516642\textcolor{blue}{\mathbf{7427466391}}9320030 \\
    & 599218174135966290435729003342952605956307381323286279434\dots
\end{align*}
}

\begin{lemma}
\label{lem:prime-divides}
If $p$ is a prime and $p \divides ab$, then $p \divides a$ or $p \divides b$.
\end{lemma}

\begin{proof}
One case is if $\gcd(a, p) = p$.  Then the claim holds, because $a$ is
a multiple of $p$.

Otherwise, $\gcd(a, p) \neq p$.  In this case $\gcd(a, p)$ must be 1,
since 1 and $p$ are the only positive divisors of $p$.  Since $\gcd(a,
p)$ is a linear combination of $a$ and $p$, we have $1=sa+tp$ for some
$s,t$.  Then $b =s(ab)+ (tb)p$, that is, $b$ is a linear combination
of $ab$ and $p$.  Since $p$ divides both $ab$ and $p$, it also divides
their linear combination $b$.
\end{proof}

A routine induction argument extends this statement to:\iffalse the fact
we assumed last time:\fi

\begin{lemma}
\label{lem:prime-divides-ind}
Let $p$ be a prime.  If $p \divides a_1 a_2 \cdots a_n$, then $p$ divides
some $a_i$.
\end{lemma}

Now we're ready to prove the Fundamental Theorem of Arithmetic.
\begin{proof}
Theorem~\ref{factor_into_primes} showed, using the Well Ordering
Principle, that every positive integer can be expressed as a product
of primes.  So we just have to prove this expression is unique.
We will use Well Ordering to prove this too.

The proof is by contradiction: assume, contrary to the claim, that there
exist positive integers that can be written as products of primes in more
than one way.  By the Well Ordering Principle, there is a smallest integer
with this property.  Call this integer $n$, and let
%
\begin{align*}
n & = p_1 \cdot p_2 \cdots p_j, \\
  & = q_1 \cdot q_2 \cdots q_k,
\end{align*}
where both products are in weakly decreasing order and $p_1 \le q_1$.

If $q_1 = p_1$, then $n/q_1$ would also be the
product of different weakly decreasing sequences of primes, namely,
\begin{align*}
p_2 \cdots p_j, \\
q_2 \cdots q_k.
\end{align*}
Since $n/q_1 < n$, this can't be true, so we conclude that $p_1 <
q_1$.

Since the $p_i$'s are weakly decreasing, all the $p_i$'s are less than
$q_1$.  But $q_1 \divides n= p_1 \cdot p_2 \cdots p_j$, so
Lemma~\ref{lem:prime-divides-ind} implies that $q_1$ divides one of
the $p_i$'s, which contradicts the fact that $q_1$ is bigger than all
them.
\end{proof}

%% Problems %%%%%%%%%%%%%%%%%%%%%%%%%%%%%%%%%%%%%%%%%%%%%%%%%%%%%%%%%%%%%%%%%%%
\begin{problems}
\classproblems
\pinput{CP_gcd_lcm}
\end{problems}

\section{Alan \idx{Turing}}\label{Turing_sec}

\begin{figure}\redrawntrue
\graphic[width=2in]{turing}
\caption{Alan Turing}
\label{fig:Turing}
\end{figure}

The man pictured in Figure~\ref{fig:Turing} is Alan Turing, the most
important figure in the history of computer science.  For decades, his
fascinating life story was shrouded by government secrecy, societal
taboo, and even his own deceptions.

At age 24, Turing wrote a paper entitled \emph{On Computable Numbers,
with an Application to the Entscheidungsproblem}.  The crux of the
paper was an elegant way to model a computer in mathematical terms.
This was a breakthrough, because it allowed the tools of mathematics
to be brought to bear on questions of computation.  For example, with
his model in hand, Turing immediately proved that there exist problems
that no computer can solve---no matter how ingenious the programmer.
Turing's paper is all the more remarkable because he wrote it in 1936,
a full decade before any electronic computer actually existed.

The word ``Entscheidungsproblem'' in the title refers to one of the 28
mathematical problems posed by David Hilbert in 1900 as challenges to
mathematicians of the 20th century.  Turing knocked that one off in the
same paper.  And perhaps you've heard of the ``\idx{Church-Turing
  thesis}''?  Same paper.  So Turing was obviously a brilliant guy who
generated lots of amazing ideas.  But this lecture is about one of
Turing's less-amazing ideas.  It involved codes.  It involved number
theory.  And it was sort of stupid.

%\subsection{Turing's Code}

Let's look back to the fall of 1937.  Nazi Germany was rearming under
Adolf Hitler, world-shattering war looked imminent, and---like
us---Alan Turing was pondering the usefulness of number theory.  He
foresaw that preserving military secrets would be vital in the coming
conflict and proposed a way \emph{to encrypt communications using
number theory}.  This is an idea that has ricocheted up to our own
time.  Today, number theory is the basis for numerous public-key
cryptosystems, digital signature schemes, cryptographic hash
functions, and electronic payment systems.  Furthermore, military
funding agencies are among the biggest investors in cryptographic
research.  Sorry \idx{Hardy}!

Soon after devising his code, \idx{Turing} disappeared from public view,
and half a century would pass before the world learned the full story of
where he'd gone and what he did there.  We'll come back to Turing's life
in a little while; for now, let's investigate the code Turing left behind.
The details are uncertain, since he never formally published the idea, so
we'll consider a couple of possibilities.

\subsection{Turing's Code (Version 1.0)}

The first challenge is to translate a text message into an integer so
we can perform mathematical operations on it.  This step is not
intended to make a message harder to read, so the details are not too
important.  Here is one approach: replace each letter of the message
with two digits ($A = 01$, $B = 02$, $C = 03$, etc.) and string all
the digits together to form one huge number.  For example, the message
``victory'' could be translated this way:
%
\begin{center}
\begin{tabular}{ccccccccc}
   &v &  i &  c &  t & o & r & y \\
$\rightarrow$ & 22 & 09 & 03 & 20 & 15 & 18 & 25
\end{tabular}
\end{center}
%
\idx{Turing's code} requires the message to be a prime number, so we may
need to pad the result with a few more digits to make a prime.  In
this case, appending the digits 13 gives the number 2209032015182513,
which is prime.

Here is how the encryption process works.  In the description
below, $m$ is the unencoded message (which we want to keep secret),
$m^*$ is the encrypted message (which the Nazis may intercept), and
$k$ is the key.

\begin{description}

\item[Beforehand] The sender and receiver agree on a secret key, which
is a large prime~$k$.

\item[Encryption] The sender encrypts the message $m$ by computing:
\[
m^* = m \cdot k
\]

\item[Decryption] The receiver decrypts $m^*$ by computing:
\[
\frac{m^*}{k}
\iffalse
= \frac{m \cdot k}{k}
\fi
  = m.
\]

\end{description}

For example, suppose that the secret key is the prime number $k =
22801763489$ and the message $m$ is ``victory''.  Then the encrypted
message is:
%
\begin{align*}
m^* & = m \cdot k \\
   & = 2209032015182513 \cdot 22801763489 \\
   & = 50369825549820718594667857
\end{align*}

There are a couple of questions that one might naturally ask about Turing's
code.

\begin{enumerate}

\item How can the sender and receiver ensure that $m$ and $k$ are
prime numbers, as required?

The general problem of determining whether a large number is prime or
composite has been studied for centuries, and reasonably good primality
tests were known even in Turing's time.  In 2002, Manindra Agrawal, Neeraj
Kayal, and Nitin Saxena announced a primality test that is guaranteed to
work on a number $n$ in about $(\log n)^{12}$ steps, that is, a number of
steps bounded by a twelfth degree polynomial in the length (in bits) of
the input, $n$.  This definitively places primality testing way below the
problems of exponential difficulty.  Amazingly, the description of their
breakthrough algorithm was only thirteen lines long!

Of course, a twelfth degree polynomial grows pretty fast, so the
Agrawal, \emph{et al.}\ procedure is of no practical use.  Still, good
ideas have a way of breeding more good ideas, so there's certainly
hope that further improvements will lead to a procedure that is useful in
practice.  But the truth is, there's no practical need to improve it,
since very efficient \emph{probabilistic} procedures for prime-testing
have been known since the early 1970's.  These procedures have some
probability of giving a wrong answer, but their probability of being
wrong is so tiny that relying on their answers is the best bet you'll
ever make.

\item Is Turing's code secure?

The Nazis see only the encrypted message $m^* = m \cdot k$, so
recovering the original message $m$ requires factoring $m^*$.  Despite
immense efforts, no really efficient factoring algorithm has ever been
found.  It appears to be a fundamentally difficult problem, though a
breakthrough someday is not impossible.  In effect, Turing's code puts
to practical use his discovery that there are limits to the power of
computation.  Thus, provided $m$ and $k$ are sufficiently large, the
Nazis seem to be out of luck!

\end{enumerate}

This all sounds promising, but there is a major flaw in Turing's code.

\subsection{Breaking Turing's Code}

Let's consider what happens when the sender transmits a
\emph{second} message using Turing's code and the same key.  This
gives the Nazis two encrypted messages to look at:
%
\[
m_1^* = m_1 \cdot k
\hspace{0.75in} \text{and} \hspace{0.75in}
m_2^* = m_2 \cdot k
\]
%
The greatest common divisor of the two encrypted messages, $m_1^*$ and
$m_2^*$, is the secret key $k$.  And, as we've seen, the GCD of two
numbers can be computed very efficiently.  So after the second message is
sent, the Nazis can recover the secret key and read \emph{every}
message!

A mathematician as brilliant as Turing is not likely to have
overlooked such a glaring problem, and we can guess that he had a
slightly different system in mind, one based on \emph{modular}
arithmetic.


%% Modular Arithmetic %%%%%%%%%%%%%%%%%%%%%%%%%%%%%%%%%%%%%%%%%%%%%%%%%%%%%%%%%
\section{Modular Arithmetic}
\label{modular_arithmeric_sec}

% Congruence is a weak form of equality.

On the first page of his masterpiece on number theory, \emph{Disquisitiones
  Arithmeticae}, \idx{Gauss} introduced the notion of
``\idx{congruence}''.  Now, Gauss is another guy who managed to cough up a
half-decent idea every now and then, so let's take a look at this one.
Gauss said that $a$ is \term{congruent} to $b$ \term{modulo} $n$ iff $n
\divides (a - b)$.  This is written \index{$\equiv \pmod{n}$}
\[
a \equiv b \pmod{n}.
\]
For example:
%
\[
29 \equiv 15 \pmod{7}  \quad\text{ because }  7 \divides (29 - 15).
\]

There is a close connection between congruences and remainders:
\begin{lemma}[Remainder]
\label{lem:conrem}
\[
a \equiv b \pmod{n} \qiff \rem{a}{n} = \rem{b}{n}.
\]
\end{lemma}

\begin{proof}
By the Division Theorem~\ref{division_thm}, there exist unique pairs
of integers $q_1, r_1$ and $q_2, r_2$ such that:
%
\begin{align*}
a & = q_1 n + r_1\\
b & = q_2 n + r_2,
\end{align*}
where $r_1,r_2 \in [0,n)$.
Subtracting the second equation from the first gives:
\begin{align*}
a - b & = (q_1 - q_2) n + (r_1 - r_2),
\end{align*}
where $r_1 - r_2$ is in the interval $(-n,n)$.  Now $a \equiv b
\pmod{n}$ if and only if $n$ divides the left side of this equation.
This is true if and only if $n$ divides the right side, which holds if
and only if $r_1 - r_2$ is a multiple of $n$.  Given the bounds on
$r_1 - r_2$, this happens precisely when $r_1 = r_2$, that is, when
$\rem{a}{n} = \rem{b}{n}$.
\end{proof}

So we can also see that
\[
29 \equiv 15 \pmod{7} \quad\text{ because } \rem{29}{7} = 1 = \rem{15}{7}.
\]
Notice that even though ``(mod 7)'' appears on the end, the
$\equiv$ symbol isn't any more strongly associated with the 15
than with the 29.  It would really be clearer to write $29
\equiv_7 15$ for example, but the notation with the modulus at
the end is firmly entrenched and we'll stick to it.

The Remainder Lemma~\ref{lem:conrem} explains why the congruence
relation has properties like an equality relation.  In particular, the
following properties follow immediately:
\begin{lemma}\label{mod_equiv_rel_lem} \mbox{}
\begin{align}
                  & a \equiv a \pmod{n}\tag{reflexivity}\\
a \equiv b  \qiff & b \equiv a \pmod{n} \tag{symmetry}\\
(a \equiv b \text{ and }  b \equiv c) \qimplies & a \equiv c \pmod{n}
\tag{transitivity}
\end{align}
\end{lemma}


We'll make frequent use of another immediate Corollary of
the Remainder Lemma~\ref{lem:conrem}:
\begin{corollary}\label{aran}
\[
a \equiv \rem{a}{n} \pmod{n}
\]
\end{corollary}

Still another way to think about congruence modulo $n$ is that it
\emph{defines a partition of the integers into $n$ sets so that congruent
numbers are all in the same set}.  For example, suppose that we're working
modulo 3.  Then we can partition the integers into 3 sets as follows:
%
\[
\begin{array}{cccccccccc}
\{ & \dots, & -6, & -3, & 0, & 3, & 6, & 9, & \dots & \} \\
\{ & \dots, & -5, & -2, & 1, & 4, & 7, & 10, & \dots & \} \\
\{ & \dots, & -4, & -1, & 2, & 5, & 8, & 11, & \dots & \}
\end{array}
\]
according to whether their remainders on division by 3 are 0, 1, or 2.
The upshot is that when arithmetic is done modulo $n$ there are really
only $n$ different kinds of numbers to worry about, because there are only
$n$ possible remainders.  In this sense, modular arithmetic is a
simplification of ordinary arithmetic.\iffalse
 and thus is a good reasoning tool.\fi


The next most useful fact about congruences is that they are
\term{preserved} by addition and multiplication:

\begin{lemma}\label{mod_congruence_lem} For $n \geq 1$, if
$a \equiv b \pmod{n}$ and $c \equiv d \pmod{n}$, then
\begin{enumerate}
\item $a + c \equiv b + d \pmod{n}$,
\item $a c \equiv b d \pmod{n}$.
\end{enumerate}
\end{lemma}

\begin{proof}
We have that $n$ divides $(b-a)$ which is equal to $(b+c)-(a+c)$, so
\[
a+c \equiv b+c \pmod{n}.
\]
Also, $n$ divides $(d-c)$, so by the same reasoning
\[
b + c \equiv b + d \pmod{n}.
\]
Combining these according to Lemma~\ref{mod_equiv_rel_lem}, we get
\[
a + c  \equiv b + d \pmod{n}.
\]
 
The proof for multiplication is virtually identical, using the fact
that if $n$ divides $(b-a)$, then it obviously divides $(bc-ac)$ as
well.
\end{proof}

The overall theme is that \emph{congruences work a lot like arithmetic
  equations}, though there are a couple of exceptions we're about to
examine.

\subsection{\index{Turing's code}Turing's Code (Version 2.0)}

In 1940, France had fallen before Hitler's army, and Britain stood alone
against the Nazis in western Europe.  British resistance depended on a
steady flow of supplies brought across the north Atlantic from the United
States by convoys of ships.  These convoys were engaged in a cat-and-mouse
game with German ``U-boats''---submarines---which prowled the Atlantic,
trying to sink supply ships and starve Britain into submission.  The
outcome of this struggle pivoted on a balance of information: could the
Germans locate convoys better than the Allies could locate U-boats or vice
versa?

Germany lost.

But a critical reason behind Germany's loss was made public only in
1974: Germany's naval code, \term{Enigma}, had been broken by the
\href{http://en.wikipedia.org/wiki/Polish_Cipher_Bureau}{Polish Cipher
  Bureau}\footnote{See
  \url{http://en.wikipedia.org/wiki/Polish\_Cipher\_Bureau}.} and the
secret had been turned over to the British a few weeks before the Nazi
invasion of Poland in 1939.  Throughout much of the war, the Allies
were able to route convoys around German submarines by listening in to
German communications.  The British government didn't explain
\emph{how} Enigma was broken until 1996.  When it was finally released
(by the US), the story revealed that Alan Turing had joined the secret
British codebreaking effort at Bletchley Park in 1939, where he became
the lead developer of methods for rapid, bulk decryption of German
Enigma messages.  Turing's Enigma deciphering was an invaluable
contribution to the Allied victory over Hitler.

Governments are always tight-lipped about cryptography, but the
half-century of official silence about Turing's role in breaking
Enigma and saving Britain may be related to some disturbing events
after the war.  More on that later.  Let's get back to number theory
and consider an alternative interpretation of Turing's code.  Perhaps
we had the basic idea right (multiply the message by the key), but
erred in using \emph{conventional} arithmetic instead of
\emph{modular} arithmetic.  Maybe this is what Turing meant:
%
\begin{description}

\item[Beforehand] The sender and receiver agree on a large prime $p$,
which may be made public.  (This will be the modulus for all our
arithmetic.)  They also agree on a secret key $k \in [1, p)$.

\item[Encryption] The message $m$ can be any integer in 
$[0, p)$; in particular, the message is no longer
required to be a prime.  The sender encrypts the message $m$ to
produce $m^*$ by computing:
%
\begin{equation}
m^* = \rem{mk}{p} \label{eq:turing-code}
\end{equation}

\item[Decryption] (Uh-oh.)

\end{description}

The decryption step is a problem.  We might hope to decrypt in the
same way as before: by dividing the encrypted message $m^*$ by the key
$k$.  The difficulty is that $m^*$ is the \emph{remainder} when $mk$
is divided by $p$.  So dividing $m^*$ by $k$ might not even give us an
integer!

This decoding difficulty can be overcome with a better understanding
of arithmetic modulo a prime.

%% Problems %%%%%%%%%%%%%%%%%%%%%%%%%%%%%%%%%%%%%%%%%%%%%%%%%%%%%%%%%%%%%%%%%%%
\begin{problems}
  \homeworkproblems
  \pinput{PS_eval_cong_aexp}

  \classproblems
  \pinput{CP_proving_basic_congruence_properties}
  \pinput{CP_multiples_of_9_and_11}
  \pinput{CP_13th_roots}
\end{problems}

%% Arithmetic with a Prime Modulus %%%%%%%%%%%%%%%%%%%%%%%%%%%%%%%%%%%%%%%%%%%%
\section{Arithmetic with a Prime Modulus}\label{mod_prime_sec}

\subsection{\idx{Multiplicative Inverses}}
\label{sec:prime}

The \term{multiplicative inverse} of a number $x$ is another number
$x^{-1}$ such that:
%
\[
x \cdot x^{-1} = 1
\]
Generally, multiplicative inverses exist over the real numbers.  For
example, the multiplicative inverse of 3 is $1 / 3$ since:
%
\[
3 \cdot \frac{1}{3} = 1
\]
%
The sole exception is that 0 does not have an inverse.
On the other hand, over the integers, only 1 and -1 have inverses.

Surprisingly, when we're working \emph{modulo a prime number},
\emph{every} number that is not congruent to 0 has a multiplicative
inverse.  For example, if we're working modulo 5, then 3 is a
multiplicative inverse of 7, since:
%
\[
7 \cdot 3 \equiv 1 \pmod{5}
\]
%
(All numbers congruent to 3 modulo 5 are also multiplicative inverses
of 7; for example, $7 \cdot 8 \equiv 1 \pmod{5}$ as well.)  The only
exception is that numbers congruent to 0 modulo 5 (that is, the
multiples of 5) do not have inverses, much as 0 does not have an
inverse over the real numbers.  Let's prove this.

\begin{lemma}
\label{lem:inverses}
If $p$ is prime and $k$ is not a multiple of $p$, then $k$ has a
multiplicative inverse modulo~$p$.
\end{lemma}

\begin{proof}
Since $p$ is prime, it has only two divisors: 1 and $p$.  And since
$k$ is not a multiple of $p$, we must have $\gcd(p, k) = 1$.
Therefore, there is a linear combination of $p$ and $k$ equal to 1:
\[
s p + t k = 1,
\]
and therefore
\[
s p + t k \equiv 1 \pmod{p}.
\]
But $p \equiv 0 \pmod{p}$, so
\[
tk \equiv 0 + tk \equiv sp +tk \equiv 1 \pmod{p}.
\]
Thus, $t$ is a multiplicative inverse of $k$.
\end{proof}

Multiplicative inverses are the key to decryption in Turing's code.
Specifically, we can recover the original message by multiplying the
encoded message by the \emph{inverse} of the key:
\begin{align*}
m^* \cdot k^{-1}
    & = \rem{mk}{p} \cdot k^{-1}
         & \text{(the def.~\eqref{eq:turing-code} of $m^*$)}\\
    & \equiv (mk)k^{-1} \pmod{p} & \text{(by Cor.~\ref{aran})}\\
    & \equiv m \pmod{p}.
\end{align*}

This shows that $m^* k^{-1}$ is congruent to the original message $m$.
Since $m$ was in $[0,p)$, we can recover it exactly by taking a
remainder:
%
\[
m = \rem{m^* k^{-1}}{p}.
\]
%
So all we need to decrypt the message is to find a value of~$k^{-1}$.
From the proof of Lemma~\ref{lem:inverses}, we know that $t$ is such a
value, where $sp + tk = 1$.  Finding $t$ is easy using the Pulverizer.

\subsection{\idx{Cancellation}}

Another sense in which real numbers are nice is that one can cancel
multiplicative terms.  In other words, if we know that $m_1 k = m_2 k$,
then we can cancel the $k$'s and conclude that $m_1 = m_2$, provided $k
\neq 0$.  In general, cancellation is \emph{not} valid in modular
arithmetic.  For example,
%
\[
2 \cdot 3 \equiv 4 \cdot 3 \pmod{6},
\]
%
but canceling the 3's leads to the \emph{false} conclusion
that \textcolor{red}{$2 \equiv 4 \pmod{6}$}.  The fact that
multiplicative terms cannot be canceled is the most significant
sense in which congruences differ from ordinary equations.  However,
this difference goes away if we're working modulo a
\emph{prime}; then cancellation is valid.

\begin{lemma}
\label{lem:cancel}
Suppose $p$ is a prime and $k$ is not a multiple of $p$.  Then
%
\[
ak \equiv bk \pmod{p}\quad \QIMPLIES\quad a \equiv b \pmod{p}.
\]
\end{lemma}

\begin{proof}
Multiply both sides of the congruence by $k^{-1}$.
\end{proof}

We can use this lemma to get a bit more insight into how Turing's code
works.  In particular, the encryption operation in Turing's code
\emph{permutes the set of possible messages}.  This is stated more
precisely in the following corollary.

\begin{corollary}
\label{cor:prime-permutes}
Suppose $p$ is a prime and $k$ is not a multiple of $p$.  Then the
sequence of remainders on division by $p$ of the sequence:
\[
%\rem{(0 \cdot k)}{p},\quad
1 \cdot k,\quad
2 \cdot k,\quad
 \dots,\quad
(p-1) \cdot k
\]
is a permutation\footnote{A \term{permutation} of a sequence of elements
is a reordering of the elements.} of the sequence:
\[
%0,\quad
1,\quad 2,\quad \dots,\quad (p - 1).
\]
%This remains true if the first term is deleted from each sequence.
\end{corollary}

\begin{proof}
The sequence of remainders contains $p-1$ numbers.  Since $i \cdot k$
is not divisible by $p$ for $i=1,\dots p-1$, all these remainders are
in $[1,p)$ by the definition of remainder.  Furthermore, the
  remainders are all different: no two numbers in $[1,p)$ are
    congruent modulo $p$, and by Lemma~\ref{lem:cancel}, $i \cdot k
    \equiv j \cdot k \pmod{p}$ if and only if $i \equiv j \pmod{p}$.
    Thus, the sequence of remainders must contain \emph{all} of
    $[1,p)$ in some order.
%The claim remains true if the first terms are deleted, because both
%sequences begin with 0.
\end{proof}

For example, suppose $p = 5$ and $k = 3$.  Then the sequence:
%
\[
%\underbrace{\rem{(0 \cdot 3)}{5}}_{=0},\quad
\underbrace{\rem{(1 \cdot 3)}{5}}_{=3},\quad
\underbrace{\rem{(2 \cdot 3)}{5}}_{=1},\quad
\underbrace{\rem{(3 \cdot 3)}{5}}_{=4},\quad
\underbrace{\rem{(4 \cdot 3)}{5}}_{=2}
\]
%
is a permutation of
% 0,
1, 2, 3, 4.  As long as the Nazis don't know the secret key
$k$, they don't know how the set of possible messages are permuted by the
process of encryption and thus they can't read encoded messages.

%%%%%%%%%%%%%%%%%%%%%%%%%%%%%%%%%%%%%%%%%%%%%%%%%%%%%%%%%%%%%%%%%%%%%%%%%%%%%%%

\subsection{Fermat's Little Theorem}

An alternative approach to finding the inverse of the secret key~$k$
in Turing's code is to rely on Fermat's Little Theorem, which is much
easier than his famous Last Theorem.  \iffalse ---and more useful.\fi

\begin{theorem}[\idx{Fermat's Little Theorem}]\label{fermat_little}
Suppose $p$ is a prime and $k$ is not a multiple of $p$.  Then:
%
\[
k^{p-1} \equiv 1 \pmod{p}
\]
\end{theorem}

\begin{proof}
We reason as follows:
\begin{align*}
(p-1)! &\eqdef 1 \cdot 2 \cdots (p-1)\\
       & = \rem{k}{p} \cdot \rem{2k}{p} \cdots \rem{(p-1)k}{p}
           & \text{(by Cor~\ref{cor:prime-permutes})}\\
       & \equiv k \cdot 2k \cdots (p-1) k \pmod{p}
           & \text{(by Cor~\ref{aran})}\\
       & \equiv (p-1)! \cdot k^{p-1} \pmod{p}
           & \text{(rearranging terms)}\\
\end{align*}

Now $(p - 1)!$ is not a multiple of $p$ because the prime
factorizations of $1, 2, \dots$, $(p - 1)$ contain only primes smaller
than $p$.  So by Lemma~\ref{lem:cancel}, we can cancel $(p - 1)!$ from
the first and last expressions, which proves the claim.
\end{proof}

Here is how we can find inverses using Fermat's Theorem.  Suppose $p$
is a prime and $k$ is not a multiple of $p$.  Then, by Fermat's
Theorem, we know that:
%
\[
k^{p-2} \cdot k \equiv 1 \pmod{p}
\]
Therefore, $k^{p-2}$ must be a multiplicative inverse of $k$.  For
example, suppose that we want the multiplicative inverse of 6 modulo
17.  Then we need to compute $\rem{6^{15}}{17}$, which we can do using
the fast exponentiation procedure of Section~\ref{fast_exp_subsubsec},
with all the arithemetic done modulo 17.  Namely,
\begin{align*}
(6,1,15) \movesto (36,6,7) \equiv (2,6,7) \movesto (4,12,3)\\
\movesto (16,14,1) \movesto (256, 224, 0) \equiv (1, 3, 0).
\end{align*}
where the $\equiv$'s are modulo 17.  Therefore, $6^{15} \equiv 3
\pmod{17}$.  Sure enough, 3 is the multiplicative inverse of 6 modulo
17 since
\[
3 \cdot 6 = 18 \equiv 1 \pmod{17}.
\]

In general, if we were working modulo a prime $p$, finding a
multiplicative inverse by trying every value in $[1, p)$ would require
  about $p$ operations.  However, this approach, like the Pulverizer,
  requires only about $\log p$ transition, which is far better when
  $p$ is large.

\subsection{Breaking \index{Turing's code}Turing's Code---Again}

The Germans didn't bother to encrypt their weather reports with the
highly-secure Enigma system.  After all, so what if the Allies learned
that there was rain off the south coast of Iceland?  But, amazingly, this
practice provided the British with a critical edge in the Atlantic naval
battle during 1941.

The problem was that some of those weather reports had originally been
transmitted using Enigma from U-boats out in the Atlantic.  Thus, the
British obtained both unencrypted reports and the same reports
encrypted with Enigma.  By comparing the two, the British were able to
determine which key the Germans were using that day and could read all
other Enigma-encoded traffic.  Today, this would be called a
\term{known-plaintext attack}.

Let's see how a known-plaintext attack would work against Turing's
code.  Suppose that the Nazis know both $m$ and $m^*$ where:
%
\[
m^* \equiv mk \pmod{p}
\]
%
Now they can compute:
%
\begin{align*}
m^{p-2} \cdot m^*
  & = m^{p-2} \cdot \rem{mk}{p}
                & \text{(def.~\eqref{eq:turing-code} of $m^*$)}\\
  & \equiv m^{p-2} \cdot mk \pmod{p} & \text{(by Cor~\ref{aran})}\\
  & \equiv m^{p-1} \cdot k \pmod{p}\\ % & \text{(simplification)}\\
  & \equiv k \pmod{p} & \text{(Fermat's Theorem)}
\end{align*}
%
Now the Nazis have the secret key $k$ and can decrypt any message!

This is a huge vulnerability, so Turing's code has no practical value.
Fortunately, Turing got better at cryptography after devising this
code; his subsequent deciphering of Enigma messages surely saved
thousands of lives, if not the whole of Britain.

%  I could insert a bit about public-key cryptography here as introduction to
%  the recitation.

\subsection{Turing Postscript}

A few years after the war, Turing's home was robbed.  Detectives soon
determined that a former homosexual lover of Turing's had conspired in
the robbery.  So they arrested him---that is, they arrested Alan
Turing---because homosexuality was a British crime punishable by up
to two years in prison at that time.  Turing was sentenced to a
hormonal ``treatment'' for his homosexuality: he was given estrogen
injections.  He began to develop breasts.

Three years later, Alan \idx{Turing}, the founder of computer science, was
dead.  His mother explained what happened in a biography of her own
son.  Despite her repeated warnings, Turing carried out chemistry
experiments in his own home.  Apparently, her worst fear was realized:
by working with potassium cyanide while eating an apple, he poisoned
himself.

However, Turing remained a puzzle to the very end.  His mother was a
devoutly religious woman who considered suicide a sin.  And, other
biographers have pointed out, Turing had previously discussed
committing suicide by eating a poisoned apple.  Evidently, Alan
Turing, who founded computer science and saved his country, took his
own life in the end, and in just such a way that his mother could
believe it was an accident.

Turing's last project before he disappeared from public view in 1939
involved the construction of an elaborate mechanical device to test a
mathematical conjecture called the Riemann Hypothesis.  This conjecture
first appeared in a sketchy paper by Bernhard Riemann in 1859 and is now
one of the most famous unsolved problem in mathematics.

\floatingtextbox{
\textboxtitle{The \idx{Riemann Hypothesis}}

The formula for the sum of an infinite geometric series says:
\[
1 + x + x^2 + x^3 + \cdots = \frac{1}{1-x}
\]
Substituting $x = \frac{1}{2^s}$, $x = \frac{1}{3^s}$,
$x = \frac{1}{5^s}$, and so on for each prime
number gives a sequence of equations:
%
\begin{align*}
1 + \frac{1}{2^s} + \frac{1}{2^{2s}} + \frac{1}{2^{3s}} + \cdots
    & = \frac{1}{1 - 1 / 2^s} \\
1 + \frac{1}{3^s} + \frac{1}{3^{2s}} + \frac{1}{3^{3s}} + \cdots
    & = \frac{1}{1 - 1 / 3^s} \\
1 + \frac{1}{5^s} + \frac{1}{5^{2s}} + \frac{1}{5^{3s}} + \cdots
    & = \frac{1}{1 - 1 / 5^s} \\
    & \text{etc.}
\end{align*}
%
Multiplying together all the left sides and all the right sides gives:
%
\[
\sum_{n=1}^{\infty} \frac{1}{n^s} = \prod_{p \in \text{primes}} \paren{\frac{1}{1 - 1 / p^s}}
\]
%
The sum on the left is obtained by multiplying out all the infinite
series and applying the Fundamental Theorem of Arithmetic.  For
example, the term $1 / 300^s$ in the sum is obtained by multiplying $1
/ 2^{2s}$ from the first equation by $1 / 3^s$ in the second and $1 /
5^{2s}$ in the third.  Riemann noted that every prime appears in the
expression on the right.  So he proposed to learn about the primes by
studying the equivalent, but simpler expression on the left.  In
particular, he regarded $s$ as a complex number and the left side as a
function, $\zeta(s)$.  Riemann found that the distribution of primes
is related to values of $s$ for which $\zeta(s) = 0$, which led to his
famous conjecture:
\begin{definition}\label{RiemannHyp}
  \term{The Riemann Hypothesis}: Every nontrivial zero of the zeta
  function $\zeta(s)$ lies on the line $s = 1/2 + c i$ in the complex
  plane.
\end{definition}
A proof would immediately imply, among other things, a strong form of
the \idx{Prime Number Theorem}.

Researchers continue to work intensely to settle this conjecture, as
they have for over a century.  It is another of the
\href{http://www.claymath.org/millennium/}{Millennium Problems} whose solver will earn
\$1,000,000 from the Clay Institute.}

%% Problems %%%%%%%%%%%%%%%%%%%%%%%%%%%%%%%%%%%%%%%%%%%%%%%%%%%%%%%%%%%%%%%%%%%
\begin{problems}
\classproblems
\pinput{CP_nonparallel_lines}
\pinput{CP_Sk_equiv_-1_mod_p}
%\pinput{CP_calculating_inverses_fermat}

\homeworkproblems
\pinput{PS_calculating_inverses}
\end{problems}

%% Arithmetic with an Arbitrary Modulus %%%%%%%%%%%%%%%%%%%%%%%%%%%%%%%%%%%%%%%
\section{Arithmetic with an Arbitrary Modulus}\label{arithmetic_modn_sec}

Turing's code did not work as he hoped.  However, his essential
idea---using number theory as the basis for cryptography---succeeded
spectacularly in the decades after his death.

In 1977, Ronald \idx{Rivest}, Adi \idx{Shamir}, and
Leonard \idx{Adleman} at MIT proposed a highly secure cryptosystem
(called \textbf{\idx{RSA}}) based on number theory.  Despite decades
of attack, no significant weakness has been found.  Moreover, RSA has
a major advantage over traditional codes: the sender and receiver of
an encrypted message need not meet beforehand to agree on a secret
key.  Rather, the receiver has both a \term{secret key}, which she
guards closely, and a \term{public key}, which she distributes as
widely as possible.  The sender then encrypts his message using her
widely-distributed public key.  Then she decrypts the received message
using her closely-held private key.  The use of such a \term{public
key cryptography} system allows you and Amazon, for example, to engage
in a secure transaction without meeting up beforehand in a dark alley
to exchange a key.

Interestingly, RSA does not operate modulo a prime, as Turing's scheme
may have, but rather modulo the product of \emph{two} large primes.
Thus, we'll need to know a bit about how arithmetic works modulo a
composite number in order to understand RSA.  Arithmetic modulo an
arbitrary positive integer is really only a little more painful than
working modulo a prime---though you may think this is like
the doctor saying, ``This is only going to hurt a little,'' before he
jams a big needle in your arm.

\subsection{Relative Primality}

Integers that have no prime factor in common are called
\term{relatively prime}.  This is the same as having no common divisor
(prime or not) greater than~1.  It is also equivalent to saying
$\gcd(a, b) = 1$.

For example, 8 and 15 are relatively prime, since $\gcd(8, 15) = 1$.
Note that, except for multiples of $p$, every integer is relatively
prime to a prime number $p$.

Next we'll need to generalize what we know about arithmetic modulo a
prime to work modulo an arbitrary positive integer $n$.  The basic
theme is that arithmetic modulo $n$ may be complicated, but the
integers \emph{relatively prime} to $n$ keep the nice properties of
having inverses and being cancellable.  For example,
\begin{lemma}
\label{lem:inverse-arb}
Let $n$ be a positive integer.  If $k$ is \emph{relatively prime} to
$n$, then there exists an integer $k^{-1}$ such that:
\[
k \cdot k^{-1} \equiv 1 \pmod{n}.
\]
\end{lemma}
An inverse for any $k$ relatively prime to $n$ is simply the
coefficient of $k$ in the linear combination of $k$ and $n$ that
equals 1, exactly as in the proof of Lemma~\ref{lem:inverses}.

\iffalse
\begin{proof}
There exist integers $s$ and $t$ such that $s k + t n = \gcd(k, n) =
1$ by Theorem~\ref{th:gcd}.  Rearranging terms gives $tn = 1 - sk$,
which implies that $n \divides 1 - sk$ and $sk \equiv 1 \pmod{n}$.  Define
$k^{-1}$ to be $s$.
\end{proof}
\fi

As a consequence of this lemma, we can cancel a multiplicative term
from both sides of a congruence if that term is relatively prime to
the modulus:
\begin{corollary}
\label{cor:cancellation-arb}
Suppose $n$ is a positive integer and $k$ is relatively prime to $n$.  Then
\[
a k \equiv b k \pmod{n} \qimplies a \equiv b \pmod{n}.
\]
\end{corollary}

This holds because we can multiply both sides of the first congruence
by $k^{-1}$ and simplify to obtain the second.

The following lemma is a simple generalization of
Corollary~\ref{cor:prime-permutes} with much the same proof.
\begin{lemma}
\label{lem:permutes-arb}
Suppose $n$ is a positive integer and $k$ is relatively prime to $n$.
Let $k_1, \dots, k_r$ be all the integers in the interval
$[1,n)$ that are relatively prime to $n$.  Then the sequence of
  remainders on division by $n$ of:
%
\[
k_1 \cdot k,\quad
k_2 \cdot k,\quad
k_3 \cdot k, \dots,\quad
k_r \cdot k
\]
is a permutation of the sequence:
\[
k_1,\quad k_2, \dots,\quad k_r.
\]
\end{lemma}

\begin{proof}
We will show that the remainders in the first sequence are all
distinct and are equal to some member of the sequence of $k_j$'s.
Since the two sequences have the same length, the first must be a
permutation of the second.

First, we show that the remainders in the first sequence are all
distinct.  Suppose that $\rem{k_i k}{n} = \rem{k_j k}{n}$.  This is
equivalent to $k_i k \equiv k_j k \pmod{n}$, which implies $k_i \equiv
k_j \pmod{n}$ by Corollary~\ref{cor:cancellation-arb}.  This, in turn,
means that $k_i = k_j$ since both are in $[1,n)$.  Thus, none
of the remainder terms in the first sequence is equal to any other
remainder term.

Next, we show that each remainder in the first sequence equals one of
the $k_i$.  By assumption, $k_i$ and $k$ are relatively prime to $n$,
and therefore so is $k_ik$ by Unique Factorization.  Hence,
\begin{align*}
\gcd(n, \rem{k_i k}{n}) & = \gcd(k_i k, n)
            & \text{(Lemma~\ref{lem:gcd})}\\
      & = 1.
\end{align*}
Since $\rem{k_i k}{n}$ is in $[0, n)$ by the definition of remainder,
  and since it is relatively prime to $n$, it must be equal to one of
  the~$k_i$'s.
\end{proof}

\subsection{Euler's Theorem}

RSA relies heavily on a generalization of Fermat's Theorem known as
Euler's Theorem.  For both theorems, the exponent of~$k$ needed to
produce an inverse of~$k$ modulo~$n$ depends on the number,
\term{$\phi(n)$}, of integers in $[0, n)$, that are relatively
    prime to~$n$.  This function $\phi$ is known as \index{Euler's
      $\phi$ function}{Euler's $\phi$} or \term{totient function}.
    For example, $\phi(7) = 6$ since 1, 2, 3, 4, 5, and~6 are all
    relatively prime to~7.  Similarly, $\phi(12) = 4$ since 1, 5, 7,
    and~11 are the only numbers in~$[0, 12)$ that are relatively prime
    to~12.

\iffalse \footnote{Recall that $\gcd(n, n) = n$ and so $n$ is never
    relatively prime to itself.}
\fi

If $n$ is prime, then $\phi(n) = n - 1$ since every positive number
less than a prime number is relatively prime to that prime.  When $n$
is composite, however, the $\phi$ function gets a little complicated.
We'll get back to it in the next section.

We can now prove Euler's Theorem:

\begin{theorem}[\idx{Euler's Theorem}]
Suppose $n$ is a positive integer and $k$ is relatively prime to $n$.
Then
\begin{equation*}
    k^{\phi(n)} \equiv 1 \pmod{n}
\end{equation*}
\end{theorem}

\begin{proof}
Let $k_1, \dots, k_r$ denote all integers relatively prime to $n$
where $ k_i\in [0, n)$.  Then $r = \phi(n)$, by the definition of the
  function $\phi$.  The remainder of the proof mirrors the proof of
  Fermat's Theorem.  In particular,
\begin{align*}
\lefteqn{k_1 \cdot k_2 \cdots k_r} \hspace{0.25in} \\
& =
\rem{k_1 \cdot k}{n} \cdot
\rem{k_2 \cdot k}{n} \cdots
\rem{k_r \cdot k}{n} & \text{(by Lemma~\ref{lem:permutes-arb})}
\\
& \equiv
(k_1 \cdot k) \cdot
(k_2 \cdot k) \cdot
\cdots
(k_r \cdot k) \pmod{n} & \text{(by Cor~\ref{aran})}
\\
& \equiv
(k_1 \cdot k_2 \cdots k_r) \cdot k^r \pmod{n} & \text{(rearranging terms)}
\end{align*}

By Lemma~\ref{cor:cancellation-arb}, each of the terms $k_i$ can be
cancelled, proving the claim.
\end{proof}

We can find multiplicative inverses using Euler's theorem as we did
with Fermat's theorem: if~$k$ is relatively prime to~$n$,
then~$k^{\phi(n)-1}$ is a multiplicative inverse of~$k$ modulo~$n$.
However, this approach requires computing $\phi(n)$.  In the next
section, we'll show that computing $\phi(n)$ is easy \emph{if} we know
the prime factorization of~$n$.  Unfortunately, finding the factors
of~$n$ can be hard to do when $n$~is large, and so the Pulverizer is
generally the best approach to computing inverses modulo~$n$.

\subsection{Computing Euler's $\phi$ Function}

RSA works using arithmetic modulo the product of two large primes, so
we begin with an elementary explanation of how to compute $\phi(pq)$
for primes $p$ and $q$:

\begin{lemma}\label{phi_pq}    %{cor:H7}
\[
\phi(pq) = (p-1) (q-1)
\]
for primes $p\neq q$.
\end{lemma}

\begin{proof}
Since $p$ and $q$ are prime, any number that is not relatively prime
to $pq$ must be a multiple of~$p$ or a multiple of~$q$.  Among the
$pq$ numbers in $[0, pq)$, there are precisely $q$ multiples of~$p$
and $p$ multiples of~$q$.  Since $p$ and~$q$ are relatively prime, the
only number in~$[0, pq)$ that is a multiple of both $p$ and~$q$ is
0.  Hence, there are $p + q - 1$ numbers in~$[0, pq)$ that are
\emph{not} relatively prime to~$n$.  This means that
\begin{align*}
    \phi(pq) & = pq - (p + q - 1) \\
            & = (p - 1) (q - 1),
\end{align*}
as claimed.\footnote{This proof provides a brief preview of the
  kinds of counting arguments that we will explore more fully in
  Part~III.}  %\ref{part:counting}
\end{proof}

The following theorem provides a way to calculate $\phi(n)$ for
arbitrary $n$.
\begin{theorem}\label{th:phi}\mbox{}
\begin{enumerate}
\item[(a)] If $p$ is a prime, then $\phi(p^k) = p^k - p^{k-1}$ for $k \geq 1$.
\item[(b)] If $a$ and $b$ are relatively prime, then $\phi(ab) = \phi(a)\phi(b)$.
\end{enumerate}
\end{theorem}

Here's an example of using Theorem~\ref{th:phi} to compute $\phi(300)$:
\begin{align*}
\phi(300)
    & = \phi(2^2 \cdot 3 \cdot 5^2)\\
    & = \phi(2^2) \cdot \phi(3) \cdot \phi(5^2)
            & \text{(by Theorem~\ref{th:phi}.(a))}\\
    & = (2^2 - 2^1) (3^1 - 3^0) (5^2 - 5^1) 
            & \text{(by Theorem~\ref{th:phi}.(b))}\\
    & = 80.
\end{align*}
\iffalse
We factor 300 in the first step, use part (1) of Theorem~\ref{th:phi}
twice in the second step, use part (2) in the third step, and then
simplify.
\fi

To prove Theorem~\ref{th:phi}.(a), notice that every $p$th number
among the $p^k$ numbers in $[0, p^{k})$ is
divisible by $p$, and only these are divisible by $p$.  So $1/p$ of
these numbers are divisible by $p$ and the remaining ones are not.  That is,
\[
\phi(p^{k}) = p^k - (1/p)p^k = p^k -p^{k-1}.
\]
We'll leave a proof of Theorem~\ref{th:phi}.(b) to
Problem~\ref{PS_Euler_function_multiplicativity}.

As a consequence of Theorem~\ref{th:phi}, we have
\begin{corollary}\label{cor:phi}
For any number~$n$, if $p_1$, $p_2$, \dots, $p_j$ are the (distinct)
prime factors of~$n$, then
\begin{equation*}
    \phi(n) =  n \paren{1 - \frac{1}{p_1}}
                 \paren{1 - \frac{1}{p_2}}
                 \cdots
                 \paren{1 - \frac{1}{p_j}}.
 \end{equation*}
\end{corollary}
We'll give another proof of Corollary~\ref{cor:phi} in a few weeks
based on rules for counting.

\iffalse
are all those of the form
$mp$.  For $mp$ to be in the interval, $m$ can take any value from 0 to
$p^{k-1}-1$ and no others, so there are exactly $p^{k-1}$ numbers in the
interval that are divisible by $p$.  Now $\phi(p^{k})$ equals the number
of remaining elements in the interval, namely, $p^k -p^{k-1}$.
\fi

\begin{problems}
\practiceproblems
\pinput{MQ_modular_arithmetic}
\pinput{MQ_inverse_by_pulverizer}

\classproblems
\pinput{CP_chinese_remainder}

\homeworkproblems
\pinput{PS_Euler_function_multiplicativity}
\pinput{PS_chinese_remainder_general}

\examproblems
\pinput{PS_Euler_theorem_calculation}
\pinput{PS_congruent_modulo_1000}
\end{problems}


\section{The RSA Algorithm}\label{RSA_sec}
Finally, we are ready to see how the \term{RSA public key encryption
  scheme} works.  The details are in the box on the next page.

\begin{figure}[p]\redrawntrue
\textbox{
\begin{minipage}{\textwidth}
\textboxheader{The RSA Cryptosystem}

\begin{description}

\item[Beforehand] The receiver creates a public key and a secret key
as follows.

\begin{enumerate}

\item Generate two distinct primes, $p$ and $q$.  Since they can be
used to generate the secret key, they must be kept hidden.

\item Let $n = pq$.

\item Select an integer $e$ such that $\gcd(e, (p-1)(q-1)) = 1$.\\ The
\emph{public key} is the pair $(e, n)$.  This should be distributed
widely.

\item Compute $d$ such that $de \equiv 1 \pmod{(p-1)(q-1)}$.  This
can be done using the Pulverizer.\\ The
\emph{secret key} is the pair $(d, n)$.  This should be kept hidden!

\end{enumerate}

\item[Encoding]
Given a message~$m$, the sender first checks that $\gcd(m, n) =
1$.

\iffalse
\footnote{It would be very bad if $\gcd(m, n)$ equals $p$ or $q$
since then it would be easy for someone to use the encoded message to
compute the secret key  If $\gcd(m, n) = n$, then the encoded message
would be~0, which is fairly useless.  For very large values of~$n$, it
is extremely unlikely that $\gcd(m, n) \ne 1$.  If this does happen,
you should get a new set of keys or, at the very least, add some bits
to~$m$ so that the resulting message is relatively prime to~$n$.}
\fi

The sender then encrypts message $m$ to produce $m^*$ using
the public key:
\[
m^* = \rem{m^e}{n}.
\]

\item[Decoding] The receiver decrypts message $m^*$ back to message $m$
using the secret key:
\[
m = \rem{(m^*)^d}{n}.
\]

\end{description}
\end{minipage}
}
\end{figure}

It is not immediately clear from the description of the RSA
cryptosystem that the decoding of the encrypted message is, in fact,
the original unencrypted message.

%hide after lecture

We'll work that out in class.

%hide after lecture

%unhide after lecture

\iffalse
In order to check that this is the
case, we need to show that the decryption
$\rem{(m^*)^d}{n}$ is indeed equal to the sender's message~$m$.  Since
$m^* = \rem{m^e}{n}$, \ $m^*$ is congruent to~$m^e$ modulo~$n$ by
Corollary~\ref{aran}.  That is,
\begin{equation*}
    m^* \equiv m^e \pmod n.
\end{equation*}
By raising both sides to the power~$d$, we obtain the congruence
\begin{equation}\label{eq:RSAx1}
    (m^*)^d \equiv m^{ed} \pmod n.
\end{equation}
The encryption exponent~$e$ and the decryption exponent~$d$ are chosen
such that $de \equiv 1 \pmod{(p - 1)(q - 1)}$.  So, there exists an
integer~$r$ such that $ed = 1 + r(p - 1)(q - 1)$.  By substituting $1
+ r(p - 1)(q - 1)$ for~$ed$ in Equation~\ref{eq:RSAx1}, we obtain
\begin{equation}\label{eq:RSAx2}
    (m^*)^d \equiv m \cdot m^{r(p - 1)(q - 1)} \pmod n.
\end{equation}

By Euler's Theorem and the assumption that $\gcd(m, n) = 1$, we know
that
\begin{equation*}
    m^{\phi(n)} \equiv 1 \pmod n.
\end{equation*}
From Corollary~\ref{cor:H7}, we know that $\phi(n) = (p - 1)(q - 1)$.
Hence,
\begin{align*}
(m^*)^d  &= m \cdot m^{r(p-1)(q-1)} \pmod{n} \\
        &= m \cdot 1^{r} \pmod{n} \\
        &= m \pmod{n}.
\end{align*}
Hence, the decryption process indeed reproduces the original
message~$m$.
\fi

%end unhide after lecture

Is it hard for someone without the secret key to decrypt the message?
No one knows for sure but it is generally believed that if $n$ is a
very large number (say, with a thousand digits), then it is difficult
to reverse engineer~$d$ from $e$ and~$n$.  Of course, it is easy to
compute~$d$ if you know $p$ and~$q$ ---do it the same way the receiver
does using the Pulverizer ---but it is not known how to factor~$n$
into $p$ and~$q$ in any reasonable amount of time when $n$~is very
large.  Maybe with a little more studying of number theory, you will
be the first to figure out how to do it.  Although, we should warn you
that Gauss worked on it for years without a lot to show for his
efforts.  And if you do figure it out, you might wind up meeting some
serious-looking fellows who work for a Federal agency\dots.

\section{What has SAT got to do with it?}\label{SAT_RSA-sec}
So why does the world, or at least the world's secret codes, fall
apart if there is an efficient test for satisfiability?  To explain
this, remember that RSA can be managed computationally because
multiplication of two primes is fast, but factoring a product of two
primes seems to be overwhelmingly demanding.

Now designing digital multiplication circuits is completely routine.
This means we can easily build a digital circuit out of \QAND, \QOR,
and \QNOT\ gates that can take two input strings $u,v$ of length $n$,
and a third input string, $z$, of length~$2n$, and ``check'' if $z$
represents the product of the numbers represented by $u$ and $v$.
That is, it gives output 1 if $z$ represents the product of $u$ and
$v$, and gives output 0 otherwise.

Now here's how to factor any number with a length $2n$ representation
using a SAT solver.  Fix the $z$ input to be the representation of the
number to be factored.  Set the first digit of the $u$ input to 1, and
do a SAT test to see if there is a satisfying assignment of values for
the remaining bits of $u$ and $v$.  That is, see if the remaining bits
of $u$ and $v$ can be filled in to cause the circuit to give output 1.
If there is such an assignment, fix the first bit of $u$ to 1,
otherwise fix the first bit of $u$ to be 0.  Now do the same thing to
fix the second bit of $u$ and then third, proceeding in this way
through all the bits of $u$ and then of $v$.  The result is that after
$2n$ SAT tests, we have found an assignment of values for $u$ and $v$
that makes the circuit give output 1.  So $u$ and $v$ represent
factors of the number represented by $z$.  This means that if SAT
could be done in time bounded by a degree $d$ polynomial in $n$, then
$2n$ digit numbers can be factored in time bounded by a polynomial in
$n$ of degree $d+1$.  In sum, if SAT was easy, then so is factoring,
and so RSA would be easy to break.

\begin{editingnotes}
The above glosses over the diff between circuit SAT and propositional
formula SAT.  Maybe add a problem showing how formula SAT can also
solve circuit SAT.
\end{editingnotes}

\iffalse

So multiplication is, or
at least seems to be, an example of a ``one-way function.''  A
function $f$ mapping length-$n$ bit-strings to length-$n$ bit-strings
for $n > 0$ is a \term{one-way function} when $f(x)$ is easy to
compute (polynomial in $n$ number of steps) for length-$n$ strings,
$u$, but going the other way, that is, finding any element in
$f^{-1}(y)$ is infeasible (exponential in $n$ number of steps) for
length-$n$ strings, $v$.

Password security is also usually managed with one-way functions.
Keeping a file around with people's actual passwords is a bad risk, so
instead of keeping, say Alice's password, $u$, in a file next to
Alice's name, we just store $f(x)$.  When Alice logs in with password
$u$, it's easy to look up and compute $f(x)$ to verify her password.
But if someone steals the password file, all they have is Alice's
$f(x)$, and that doesn't let them find $u$.  So they can't pretend to
have Alice's password.

The fact that $f$ is easy to compute on strings of length $n$ implies
that there is a digital circuit of size polynomial in $n$ that will
carry out the computation of $f(x)$ for any length-$n$ input $u$.

as inputs any strings $u$ and $v$ of length~$n$
\fi

%% Problems %%%%%%%%%%%%%%%%%%%%%%%%%%%%%%%%%%%%%%%%%%%%%%%%%%%%%%%%%%%%%%%%%%%

\begin{problems}
%\examproblems

\classproblems
\pinput{CP_RSA_between_tables}
\pinput{CP_RSA_proving_correctness}
\pinput{PS_Rabin_cryptosystem}
\pinput{PS_RSA_key_implies_factoring}
\end{problems}


\begin{editingnotes}
Notes for further revisions:

drop separate mod a prime section: it's easy to do inverse iff rel
prime in general, and the Fermat proof is not really any easier than
Euler. So just do, Euler, get Fermat as corollary.

Simplify Euler proof to get rid of remainders, just using the fact
that no two in the same list are congruent (as in 2011 slides5f).

Add summary of what's easy: exponentiating, gcd's, inverses, finding
primes (elementary density argument plus fermat test probabilistic
prime testing)

What assumed hard: factoring

Work out Prob~\ref{PS_RSA_key_implies_factoring} that explains RSA security
claim: finding key implies factoring.

Add chebychev bound on prime density or similar bound (Nair, Shoup) with
elementary proof.  Explain simple probabilistic fermat test.
\end{editingnotes}

\endinput
