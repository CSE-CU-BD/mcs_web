\hyperdef{sets}{informal}{\chapter{Sets and Relations}}\label{sets_chap}

\hyperdef{logic}{sets}{\section{The Logic of Sets}}

\subsection{\idx{Russell's Paradox}}

Reasoning naively about sets turns out to be risky.  In fact, one of the
earliest attempts to come up with precise axioms for sets by a late
nineteenth century logican named Gotlob \term{Frege} was shot down by a three
line argument known as \emph{Russell's Paradox}:\footnote{Bertrand \term{Russell}
  was a mathematician/logician at Cambridge University at the turn of the
  Twentieth Century.  He reported that when he felt too old to do
  mathematics, he began to study and write about philosophy, and when he
  was no longer smart enough to do philosophy, he began writing about
  politics.  He was jailed as a conscientious objector during World War I.
  For his extensive philosophical and political writing, he won a Nobel
  Prize for Literature.}  This was an astonishing blow to efforts to
provide an axiomatic foundation for mathematics.

\textbox{
\begin{quote}
Let $S$ be a variable ranging over all sets, and define
\[W \eqdef \set{S \suchthat S \not\in S}.\]
So by definition,
\[S \in W  \mbox{  iff  } S \not\in S,\]
for every set $S$.  In particular, we can let $S$ be $W$, and obtain
the contradictory result that
\[W \in W  \mbox{  iff  } W \not\in W.\]
\end{quote}}

A way out of the paradox was clear to Russell and others at the time:
\emph{it's unjustified to assume that $W$ is a set}.  So the step in the
proof where we let $S$ be $W$ has no justification, because $S$ ranges
over sets, and $W$ may not be a set.  In fact, the paradox implies that
$W$ had better not be a set!

But denying that $W$ is a set means we must \emph{reject} the very natural
axiom that every mathematically well-defined collection of elements is
actually a set.  So the problem faced by Frege, Russell and their
colleagues was how to specify \emph{which} well-defined collections are
sets.  Russell and his fellow Cambridge University colleague Whitehead
immediately went to work on this problem.  They spent a dozen years
developing a huge new axiom system in an even huger monograph called
\emph{Principia Mathematica}.


\subsection{The ZFC Axioms for Sets}
It's generally agreed that, using some simple logical deduction rules,
essentially all of mathematics can be derived from some axioms about sets
called the Axioms of \idx{Zermelo-Frankel Set Theory} with Choice (\idx{ZFC}).

We're \emph{not} going to be working with these axioms in this course,
but we thought you might like to see them --and while you're at it, get
some practice reading quantified formulas:

\begin{description}

\item[\term{Extensionality}.] Two sets are equal if they have the same members.
In formal logical notation, this would be stated as:
\[
(\forall z.\; (z \in x \QIFF z \in y)) \QIMPLIES x = y.
\]

\item[\term{Pairing}.] For any two sets $x$ and $y$, there is a set,
     $\set{x,y}$, with $x$ and $y$ as its only elements:
\[
\forall x,y.\; \exists u.\; \forall z.\;
[z \in u \QIFF (z = x \QOR z = y)]
\]

\item[\index{Union axiom}Union.] The union, $u$, of a collection, $z$, of sets is also a set:
\[
\forall z.\, \exists u \forall x.\; (\exists y.\; x \in y \QAND y \in z) \QIFF x \in u.
\]

\item[\index{Infinity axiom}Infinity.]  There is an infinite set.
  Specifically, there is a nonempty set, $x$, such that for any set $y \in
  x$, the set $\set{y}$ is also a member of $x$.

\begin{editingnotes}

\item[Subset.] Given any set, $x$, and any proposition $P(y)$, there is a
  set containing precisely those elements $y \in x$ for which $P(y)$ holds.

\end{editingnotes}

\item[\index{Power Set axiom}Power Set.]  All the subsets of a set form another set:
\[
\forall x.\; \exists p.\; \forall u.\: u \subseteq x \QIFF u \in p.
\]

\item[\index{Replacement axiom}Replacement.]  Suppose a formula, $\phi$,
  of set theory defines the graph of a function, that is,
\[
\forall x, y, z.\, [\phi(x,y) \QAND \phi(x,z)] \QIMPLIES y = z.
\]
Then the image of any set, $s$, under that function is also a set, $t$.  Namely,
\[
\forall s\, \exists t\, \forall y.\, [\exists x.\, \phi(x,y) \QIFF y \in t].
\]


\item[\term{Foundation}.] 
There cannot be an infinite sequence
\[
\cdots \in x_n \in \cdots \in x_1 \in x_0
\]
of sets each of which is a member of the previous one.  This is equivalent
to saying every nonempty set has a ``member-minimal'' element.  Namely, define
\[
\text{member-minimal}(m, x) \eqdef [m \in x \QAND \forall y \in x.\, y \notin m].
\]
Then the Foundation axiom is
\[
\forall x.\ x \neq \emptyset\ \QIMPLIES\ \exists m.\, \text{member-minimal}(m, x).
\]

\begin{editingnotes}
If well-founded posets are defined, then rephrase Foundation as
\emph{The $\in$ relation on sets is well-founded.}
\end{editingnotes}

\item[\index{Choice axiom}Choice.]  Given a set, $s$, whose members are nonempty sets no two
  of which have any element in common, then there is a set, $c$,
  consisting of exactly one element from each set in $s$.


\iffalse
\begin{tabbing}
$\exists y \, \forall z \, \forall w \,
 \biggl( ($\=$z \in w \,\QAND\, w \in x) \; \QIMPLIES $\\
\> $\exists v \, \exists u \, \Bigl(\exists t \, \bigr((u \in w \, \QAND \, w \in t)$\=$\;\QAND\; (u \in t \,\QAND\, t \in y)\bigl) $\\
\> \> $\QIFF\; u = v\Bigr) \biggr)$
\end{tabbing}
\fi


\begin{editingnotes}

\[\begin{array}{rlll}
\exists y \forall z \forall w & ( (z \in w \QAND w \in x) \QIMPLIES\\
                              &\quad \exists v \exists u (\exists t
                                           ((u \in w \QAND & w \in t)
                                                              & \QAND (u \in t \QAND t \in y))\\
                                                            &&& \QIFF u = v))
\end{array}\]

\end{editingnotes}

\end{description}


\subsection{Avoiding \idx{Russell's Paradox}}

These modern ZFC axioms for set theory are much simpler than the system
Russell and Whitehead first came up with to avoid paradox.  In fact, the
ZFC axioms are as simple and intuitive as Frege's original axioms, with
one technical addition: the Foundation axiom.  Foundation captures the
intuitive idea that sets must be built up from ``simpler'' sets in certain
standard ways.  And in particular, Foundation implies that no set is ever
a member of itself.  So the modern resolution of Russell's paradox goes as
follows: since $S \not \in S$ for all sets $S$, it follows that $W$,
defined above, contains every set.  This means $W$ can't be a set ---or it
would be a member of itself.

\subsection{Does All This Really Work?}\label{setsreallywork}

So this is where mainstream mathematics stands today: there is a handful
of ZFC axioms from which virtually everything else in mathematics can be
logically derived.  This sounds like a rosy situation, but there are
several dark clouds, suggesting that the essence of truth in mathematics
is not completely resolved.

%
\begin{itemize}

\item The \idx{ZFC axioms} weren't etched in stone by God.  Instead, they were
  mostly made up by some guy named Zermelo.  Probably some days he
  forgot his house keys.

  So maybe \idx{Zermelo}, just like \idx{Frege}, didn't get his axioms
  right and will be shot down by some successor to \idx{Russell} who
  will use Zermelo's axioms to prove a proposition $P$ and its
  negation $\QNOT P$.  Then math would be broken.  This sounds crazy,
  but after all, it has happened before.

  In fact, while there is broad agreement that the ZFC axioms are capable
  of proving all of standard mathematics, the axioms have some further
  consequences that sound paradoxical.  For example, the \idx{Banach-Tarski}
  Theorem says that, as a consequence of the \idx{Axiom of Choice}, a solid ball
  can be divided into six pieces and then the pieces can be rigidly
  rearranged to give \emph{two} solid balls, each the same size as the
  original!

\item Georg \term{Cantor} was a contemporary of \idx{Frege} and
  \idx{Russell} who first developed the theory of infinite sizes (because
  he thought he needed it in his study of Fourier series).  Cantor raised
  the question whether there is a set whose size is strictly between the
  ``smallest\footnote{See Problem~\ref{CP_smallest_infinite_set}}''
  infinite set, $\naturals$, and $\power(\naturals)$; he guessed not:

  \textbf{Cantor's \term{Continuum Hypothesis}}: There is no set, $A$,
  such that $\power(\naturals)$ is \idx{strictly bigger} than $A$ and $A$
  is strictly bigger than $\naturals$.

  The Continuum Hypothesis remains an open problem a century later.  Its
  difficulty arises from one of the deepest results in modern Set Theory
  ---discovered in part by \idx{G\"odel} in the 1930's and Paul
  \idx{Cohen} in the 1960's ---namely, the ZFC axioms are not sufficient
  to settle the Continuum Hypothesis: there are two collections of sets,
  each obeying the laws of \idx{ZFC}, and in one collection the Continuum
  Hypothesis is true, and in the other it is false.  So settling the
  Continuum Hypothesis requires a new understanding of what Sets should be
  to arrive at persuasive new axioms that extend ZFC and are strong enough
  to determine the truth of the Continuum Hypothesis one way or the other.

\item But even if we use more or different axioms about sets, there are
  some unavoidable problems.  In the 1930's, \idx{G\"odel} proved that,
  assuming that an axiom system like \idx{ZFC} is consistent ---meaning
  you can't prove both $P$ and $\QNOT(P)$ for any proposition, $P$ ---then
  the very proposition that the system is consistent (which is not too
  hard to express as a logical formula) cannot be proved in the system.
  In other words, no \idx{consistent} system is strong enough to verify
  itself.
  
\end{itemize}

\subsection{Large Infinities in Computer Science}

If the romance of different size infinities and continuum hypotheses
doesn't appeal to you, not knowing about them is not going to lower your
professional abilities as a computer scientist.  These abstract issues
about infinite sets rarely come up in mainstream mathematics, and they
don't come up at all in computer science, where the focus is generally on
``\idx{countable},'' and often just finite, sets.  In practice, only
logicians and set theorists have to worry about collections that are too
big to be sets.  In fact, at the end of the 19th century, the general
mathematical community doubted the relevance of what they called
``\idx{Cantor's paradise}'' of unfamiliar sets of arbitrary infinite size.

But the proof that power sets are bigger gives the simplest form of what
is known as a ``\idx{diagonal argument}.''  Diagonal arguments are used to
prove many fundamental results about the limitations of computation, such
as the undecidability of the \idx{Halting Problem} for programs (see
Problem~\ref{CP_recognizable_sets}) and the inherent, unavoidable,
\idx{inefficiency} (\idx{exponential time} or worse) of procedures for
other computational problems.  So computer scientists do need to study
diagonal arguments in order to understand the logical limits of
computation.

\begin{problems}
%\practiceproblems
%\pinput{}
\classproblems
\pinput{CP_power_set_tower}
\pinput{CP_recognizable_sets}
\pinput{CP_undescribable_language}

\homeworkproblems
\pinput{PS_uncountable_infinite_sequences}
\pinput{PS_N_to_A_diagonal_argument}
\end{problems}

\section{Sequences}

Sets provide one way to group a collection of objects.  Another way is
in a \term{sequence}, which is a list of objects called \term{terms}
or \term{components}.  Short sequences are commonly described by
listing the elements between parentheses; for example, $(a, b, c)$ is
a sequence with three terms.

While both sets and sequences perform a gathering role, there are
several differences.
\begin{itemize}

\item The elements of a set are required to be distinct, but terms in a
sequence can be the same.  Thus, $(a, b, a)$ is a valid sequence of length
three, but $\set{a, b, a}$ is a set with two elements ---not three.

\item The terms in a sequence have a specified order, but the elements
of a set do not.  For example, $(a, b, c)$ and $(a, c, b)$ are
different sequences, but $\set{a, b, c}$ and $\set{a, c, b}$ are the
same set.

\item Texts differ on notation for the \term{empty sequence}; we use
  \term{$\lambda$} for the empty sequence.
\end{itemize}

The product operation is one link between sets and sequences.  A
\term{product of sets}, $S_1 \times S_2 \times \cdots \times S_n$, is a
new set consisting of all sequences where the first component is drawn
from $S_1$, the second from $S_2$, and so forth.  For example, $\naturals
\times \set{a,b}$ is the set of all pairs whose first element is a
nonnegative integer and whose second element is an $a$ or a $b$:
\[
\naturals \times \set{a,b}
    = \set{(0,a), (0,b), (1,a), (1,b), (2,a), (2, b), \dots}
\]
A product of $n$ copies of a set $S$ is denoted $S^n$.  For example,
$\set{0, 1}^3$ is the set of all $3$-bit sequences:
\[
\set{0, 1}^3 = \set{ (0,0,0), (0,0,1), (0,1,0), (0,1,1),
                     (1,0,0), (1,0,1), (1,1,0), (1,1,1) }
\]

\section{Functions}\label{funcsubsec}

A \term{function} assigns an element of one set, called the
\term{domain}, to elements of another set, called the \term{codomain}.
The notation
\[
f: A \to B
\]
indicates that $f$ is a function with domain, $A$, and codomain, $B$.  The
familiar notation ``$f(a) = b$'' indicates that $f$ assigns the element $b
\in B$ to $a$.  Here $b$ would be called the \term*{value} of $f$ at
\term*{argument} $a$.

Functions are often defined by formulas as in:
\[
f_1(x) \eqdef \frac{1}{x^2}
\]
where $x$ is a real-valued variable, or
\[
f_2(y,z) \eqdef y\mathtt{10}yz
\]
where $y$ and $z$ range over binary strings, or
\[
f_3(x, n) \eqdef \text{ the pair } (n, x)
\]
where $n$ ranges over the nonnegative integers.

A function with a finite domain could be specified by a table that shows
the value of the function at each element of the domain.  For example, a function
$f_4(P,Q)$ where $P$ and $Q$ are propositional variables is specified by:
\[\begin{array}{|cc|c|}
\hline
P & Q & f_4(P,Q)\\
\hline \true & \true & \true\\
\hline \true & \false & \false\\
\hline \false & \true & \true\\
\hline \false & \false & \true\\
\hline
\end{array}\]
Notice that $f_4$ could also have been described by a formula:
\[
f_4(P,Q)  \eqdef [P \QIMPLIES Q].
\]

A function might also be defined by a procedure for computing its value at
any element of its domain, or by some other kind of specification.  For
example, define $f_5(y)$ to be the length of a left to right search of the
bits in the binary string $y$ until a \texttt{1} appears, so
\begin{eqnarray*}
f_5(0010) & = &  3,\\
f_5(100)  & = & 1,\\
f_5(0000) & \text{is} & \text{undefined}.
\end{eqnarray*}

Notice that $f_5$ does not assign a value to any string of just \texttt{0}'s.
This illustrates an important fact about functions: they need not assign a
value to every element in the domain.  In fact this came up in our first
example $f_1(x)=1/x^2$, which does not assign a value to $0$.  So in
general, functions may be \term{partial functions}, meaning that there may be domain
elements for which the function is not defined.  If a function is defined
on every element of its domain, it is called a \term{total function}.

It's often useful to find the set of values a function takes when applied
to the elements in \emph{a set} of arguments.  So if $f:A \to B$, and $S$
is a subset of $A$, we define $f(S)$ to be the set of all the values that
$f$ takes when it is applied to elements of $S$.  That is,
\[
f(S) \eqdef \set{b \in B \suchthat f(s) = b \text{ for some } s
  \in S}.
\]
For example, if we let $[r,s]$ denote the interval from $r$ to $s$ on the
real line, then $f_1([1,2]) = [1/4,1]$.

For another example, let's take the ``search for a \texttt{1}''
function, $f_5$.  If we let $X$ be the set of binary words which
start with an even number of \texttt{0}'s followed by a
\texttt{1}, then $f_5(X)$ would be the odd nonnegative integers.

Applying $f$ to a set, $S$, of arguments is referred to as
\hyperdef{mapping}{pointwise}{``applying $f$ \idx{pointwise} to $S$''}, and the
set $f(S)$ is referred to as the \term{image} of $S$ under
$f$.\footnote{There is a picky distinction between the function $f$ which
  applies to elements of $A$ and the function which applies $f$ pointwise
  to subsets of $A$, because the domain of $f$ is $A$, while the domain of
  pointwise-$f$ is $\power(A)$.  It is usually clear from context whether
  $f$ or pointwise-$f$ is meant, so there is no harm in overloading the
  symbol $f$ in this way.}  The set of values that arise from applying $f$
to all possible arguments is called the \term{range} of $f$.  That is,
\[
\range{f} \eqdef f(\domain{f}).
\]
Some authors refer to the codomain as the range of a function, but they
shouldn't.  The distinction between the range and codomain will be
important in Sections~\ref{surj_sec} and~\ref{mappingrule_sec} when we
relate sizes of sets to properties of functions between
them.

\subsection{Function Composition}\label{func_compose_subsec}

Doing things step by step is a universal idea.  Taking a walk is a literal
example, but so is cooking from a recipe, executing a computer program,
evaluating a formula, and recovering from substance abuse.

Abstractly, taking a step amounts to applying a function, and going step
by step corresponds to applying functions one after the other.  This is
captured by the operation of \term{composing} functions.  Composing the
functions $f$ and $g$ means that first $f$ applied is to some argument,
$x$, to produce $f(x)$, and then $g$ is applied to that result to produce
$g(f(x))$.

\begin{definition}\label{func_compose_def}
  For functions $f:A \to B$ and $g:B \to C$, the \term{composition},
  $g \compose f$, of $g$ with $f$ is defined to be the function
  $h:A \to C$ defined by the rule:
\begin{displaymath}
h(x) \eqdef (g \compose f)(x) \eqdef g(f(x)),
\end{displaymath}
for all $x \in A$.
\end{definition}

Function composition is familiar as a basic concept from elementary
calculus, and it plays an equally basic role in discrete mathematics.

\section{Relations}\label{rel_sec}

\term{Relations} are another fundamental mathematical data type.  Equality
and ``less-than'' are very familiar examples of mathematical relations.
These are called \term{binary relations} because they apply to a pair
$(a,b)$ of objects; the equality relation holds for the pair when $a=b$,
and less-than holds when $a$ and $b$ are real numbers and $a < b$.

In this section we'll define some basic vocabulary and properties of binary
relations.

\hyperdef{func}{rel}{\subsection{Binary Relations and Functions}}
Binary relations are far more general than equality or less-than.
Here's the official definition:
\begin{definition}\label{reldef}
A \term{binary relation}, $R$, consists of a set, $A$, called
the \term{domain} of $R$, a set, $B$, called the \term{codomain} of $R$, and
a subset of $A \cross B$ called the \term{graph of $R$}.
\end{definition}

Notice that Definition~\ref{reldef} is exactly the same as the definition
in Section~\ref{funcsubsec} of a {\emph{function}}, except that it doesn't
require the functional condition that, for each domain element, $a$, there
is \emph{at most} one pair in the graph whose first coordinate is $a$.  So
a function is a special case of a binary relation.

A relation whose domain is $A$ and codomain is $B$ is said to be
``between $A$ and $B$'', or ``from $A$ to $B$.''  When the domain and
codomain are the same set, $A$, we simply say the \index{relation on a
set} relation is ``on $A$.''  It's common to use \idx{infix notation}
``$a \mrel{R} b$'' to mean that the pair $(a,b)$ is in the graph of
$R$.

For example, we can define an ``in-charge of'' relation, $T$, for MIT in
Spring '10 to have domain equal to the set, $F$, of names of the faculty
and codomain equal to all the set, $N$, of subject numbers in the current
catalogue.  The graph of $T$ contains precisely the pairs of the form
\[
(\ang{\text{instructor-name}}, \ang{\text{subject-num}})
\]
such that the faculty member named $\ang{\text{instructor-name}}$ is
in charge of the subject with number $\ang{\text{subject-num}}$ in Spring '10.
So $\graph{T}$ contains pairs like
\[\begin{array}{ll}
(\texttt{A. R. Meyer}, & \texttt{6.042}),\\
(\texttt{A. R. Meyer}, & \texttt{18.062}),\\
(\texttt{A. R. Meyer}, & \texttt{6.844}),\\
(\texttt{T. Leighton}, & \texttt{6.042}),\\
(\texttt{T. Leighton}, & \texttt{18.062}),\\
(\texttt{G, Freeman}, & \texttt{6.011}),\\
(\texttt{G. Freeman}, & \texttt{6.881})\\
(\texttt{G. Freeman}, & \texttt{6.882})\\
(\texttt{G. Freeman}, & \texttt{6.UAT})\\
(\texttt{T. Eng},      & \texttt{6.UAT})\\
(\texttt{J. Guttag},  & \texttt{6.00})\\
\qquad \vdots
\end{array}\]

This is a surprisingly complicated relation: Meyer is in charge of
subjects with three numbers.  Leighton is also in charge of subjects with
two of these three numbers ---because the same subject, Mathematics for
Computer Science, has two numbers: 6.042 and 18.062, and Meyer and
Leighton are co-in-charge of the subject.  Freeman is in-charge of even
more subjects numbers (around 20), since as Department Education Officer,
he is in charge of whole blocks of special subject numbers.  Some
subjects, like 6.844 and 6.00 have only one person in-charge.  Some
faculty, like Guttag, are in charge of only one subject number, and no one
else is co-in-charge of his subject, 6.00.

Some subjects in the codomain, $N$, do not appear in the list ---that is,
they are not an element of any of the pairs in the graph of $T$; these are
the Fall term only subjects.  Similarly, there are faculty in the domain,
$F$, who do not appear in the list because all their in-charge subjects
are Fall term only.

\subsection{Relational Images}
The idea of the image of a set under a function extends directly to
relations.

\begin{definition}
The \term{image} of a set, $Y$, under a relation, $R$, written $R(Y)$, is the
set of elements of the codomain, $B$, of $R$ that are related to some element in
$Y$, namely,
\[
R(Y) \eqdef \set{b \in B \suchthat yRb \text{ for some } y \in Y}.
\]
\end{definition}

For example, to find the subject numbers that Meyer is in
charge of in Spring '09, we can look for all the pairs of the form
\[
(\text{A. Meyer}, \ang{\text{subject-number}})
\]
in the graph of the teaching relation, $T$, and then just list the
right hand sides of these pairs.  These righthand sides are exactly the image
$T(\text{A. Meyer})$, which happens to be $\set{6.042, 18.062, 6.844}$.
Similarly, to find the subject numbers that either Freeman or Eng are
in charge of, we can collect all the pairs in $T$ of the form
\[
(\text{G. Freeman}, \ang{\text{subject-number}})
\]
or
\[
(\text{T. Eng}, \ang{\text{subject-number}});
\]
and list their right hand sides.  These right hand sides are exactly the image
$T(\set{\text{G. Freeman}, \text{T. Eng}}$.  So the partial list
of pairs in $T$ given above implies that
\[
\set{\texttt{6.011, 6.881, 6.882, 6.UAT}} \subseteq T(\set{\text{G. Freeman}, \text{T. Eng}}.
\]
Finally, since the domain, $F$, is the set of all in-charge faculty,
$T(F)$ is exactly the set of \emph{all} Spring '09 subjects being
taught.

\subsection{Inverse Relations and Images}

\begin{definition}
The \term{inverse}, $\inv{R}$ of a relation $R: A \to B$ is the
relation from $B$ to $A$ defined by the rule
\[
b \inv{R} a \QIFF a \mrel{R} B.
\]
The image of a set under the relation, $\inv{R}$, is called
the \term{inverse image} of the set.  That is, the inverse image of a
set, $X$, under the relation, $R$, is $\inv{R}{X}$.
\end{definition}

Continuing with the in-charge example above, we can find the faculty
in charge of 6.UAT in Spring '10 can be found by taking the pairs of
the form
\[
(\ang{\text{instructor-name}}, 6.UAT)
\]
in the graph of the teaching relation, $T$, and then just listing the
left hand sides of these pairs; these turn out to be just Eng and
Freeman.  These left hand sides are exactly the inverse image of
$\set{\text{6.UAT}}$ under $T$.

Now let $D$ be the set of introductory course 6 subject numbers.
These are the subject numbers that start with \emph{6.0}\,.  Now we
can likewise find out all the instructors who were in-charge of
introductory course 6 subjects in Spring '09, by taking all the pairs
of the form $(\ang{\text{instructor-name}}, 6.0\dots)$ and list the
left hand sides of these pairs.  These left hand sides are exactly the
inverse image of of $D$ under $T$.  From the part of the
graph of $T$ shown above, we can see that
\[
\set{\text{Meyer, Leighton, Freeman, Guttag}} \subseteq \inv{T}(D).
\]
That is, Meyer, Leighton, Freeman, and Guttag were among the
instructors in charge of introductory subjects in Spring '10.
Finally, the inverse image under $T$ of the set, $N$, of all subject
numbers is the set of all instructors who were in charge of a Spring
'09 subject.

It gets interesting when we write composite expressions mixing images,
inverse images and set operations.  For example, $T(\inv{T}(D))$ is
the set of Spring '09 subjects that have an instructor in charge who
also is in in charge of an introductory subject.  So $T(\inv{T}(D)) -
D$ are the advanced subjects with someone in-charge who is also
in-charge of an introductory subject.  Similarly, $\inv{T}(D) \intersect
\inv{T}(N-D)$ is the set of faculty in charge of both an introductory \emph{and}
an advanced subject in Spring '09.

\subsection{Surjective and Injective Relations}\label{surj_sec}

There are a few properties of relations that will be useful when we take
up the topic of counting because they imply certain relations between the
\emph{sizes} of domains and codomains.  We say a binary relation $R : A
\to B$ is:

\begin{itemize}

\item \term{surjective} when every element of $B$ is mapped to \textit{at
least once}; more concisely, $R$ is surjective iff $R(A)=B$.

\item \term{total} when every element of $A$ is assigned to some element of
  $B$; more concisely, $R$ is total iff $A=\inv{R}(B)$.

\item \term{injective} if every element of $B$ is mapped to \textit{at
most once}, and

\item \term{bijective} if $R$ is total, surjective, and injective
  \emph{function}.\footnote{These words ``surjective,''
``injective,'' and ``bijective'' are not very memorable.
Some authors use the possibly more memorable phrases \term{onto} for surjective,
\emph{one-to-one} for injective, and \emph{exact correspondence} for bijective.}

\end{itemize}

Note that this definition of $R$ being total agrees with the
definition in Section~\ref{funcsubsec} when $R$ is a function.

If $R$ is a binary relation from $A$ to $B$, we define $R(A)$ to to be the
\emph{range} of $R$.  So a relation is surjective iff its range equals its
codomain.  Again, in the case that $R$ is a function, these definitions of
``range'' and ``total'' agree with the definitions in
Section~\ref{funcsubsec}.

\subsection{Relation Diagrams}
We can explain all these properties of a relation $R:A \to B$ in terms of
a diagram where all the elements of the domain, $A$, appear in one column
(a very long one if $A$ is infinite) and all the elements of the codomain,
$B$, appear in another column, and we draw an arrow from a point $a$ in
the first column to a point $b$ in the second column when $a$ is related
to $b$ by $R$.  For example, here are diagrams for two functions:

\begin{center}
\begin{tabular}{ccc}

\unitlength = 2pt
\begin{picture}(50,60)(-10,-5)
\thinlines
\put(-5,50){\makebox(0,0){$A$}}
  \put(35,50){\makebox(0,0){$B$}}
\put(-5,40){\makebox(0,0){a}}
  \put(0,40){\vector(1,0){28}}
  \put(35,40){\makebox(0,0){1}}
\put(-5,30){\makebox(0,0){b}}
  \put(0,30){\vector(3,-1){28}}
  \put(35,30){\makebox(0,0){2}}
\put(-5,20){\makebox(0,0){c}}
  \put(0,20){\vector(3,-1){28}}
  \put(35,20){\makebox(0,0){3}}
\put(-5,10){\makebox(0,0){d}}
  \put(0,10){\vector(3,2){28}}
  \put(35,10){\makebox(0,0){4}}
\put(-5,0){\makebox(0,0){e}}
  \put(0,0){\vector(3,2){28}}
\end{picture}

& \hspace{0.5in} &

\unitlength = 2pt
\begin{picture}(50,60)(-10,-5)
\thinlines
\put(-5,50){\makebox(0,0){$A$}}
  \put(35,50){\makebox(0,0){$B$}}
\put(-5,40){\makebox(0,0){a}}
  \put(0,40){\vector(1,0){28}}
  \put(35,40){\makebox(0,0){1}}
\put(-5,30){\makebox(0,0){b}}
  \put(0,30){\vector(3,-1){28}}
  \put(35,30){\makebox(0,0){2}}
\put(-5,20){\makebox(0,0){c}}
  \put(0,20){\vector(3,-2){28}}
  \put(35,20){\makebox(0,0){3}}
\put(-5,10){\makebox(0,0){d}}
  \put(0,10){\vector(3,2){28}}
  \put(35,10){\makebox(0,0){4}}
\put(35,0){\makebox(0,0){5}}
\end{picture}

\end{tabular}
\end{center}

Here is what the definitions say about such pictures:
\begin{itemize}

\item ``$R$ is a function'' means that every point in the domain column,
  $A$, has \emph{at most one arrow out of it}.

\item ``$R$ is total'' means that \emph{every} point in the $A$ column has
  \emph{at least one arrow out of it}.  So if $R$ is a function, being
  total really means every point in the $A$ column has
  \emph{exactly one arrow out of it}.

\item ``$R$ is surjective'' means that \emph{every} point in the codomain
  column, $B$, has \emph{at least one arrow into it}.

\item ``$R$ is injective'' means that every point in the codomain column,
  $B$, has \emph{at most one arrow into it}.

\item ``$R$ is bijective'' means that \emph{every} point in the $A$ column
      has exactly one arrow out of it, and \emph{every} point in the $B$ column
      has exactly one arrow into it.

\end{itemize}

So in the diagrams above, the relation on the left is a total, surjective
function (every element in the $A$ column has exactly one arrow out, and
every element in the $B$ column has at least one arrow in), but not
injective (element 3 has two arrows going into it).  The relation on the
right is a total, injective function (every element in the $A$ column has
exactly one arrow out, and every element in the $B$ column has at most one
arrow in), but not surjective (element 4 has no arrow going into it).

%Define $\inv{R}$ and explain with reversed arrows.  Deduce that that
%$R$ is total iff $\inv{R} is surjective, $R$ is function iff
%$\inv{R}$ is injective.

Notice that the arrows in a diagram for $R$ precisely correspond to the
pairs in the graph of $R$.  But $\graph{R}$ does not determine by itself
whether $R$ is total or surjective; we also need to know what the domain
is to determine if $R$ is total, and we need to know the codomain to tell
if it's surjective.
\begin{example}
  The function defined by the formula $1/x^2$ is total if its domain is
  $\reals^+$ but partial if its domain is some set of real numbers
  including 0.  It is bijective if its domain and codomain are both
  $\reals^+$, but neither injective nor surjective if its domain and
  codomain are both $\reals$.
\end{example}


\hyperdef{mapping}{rule}{\section{Cardinality}}\label{mappingrule_sec}

The relational properties in Section~\ref{rel_sec} are useful in
figuring out the relative sizes of domains and codomains.

If $A$ is a finite set, we let $\card{A}$ be the number of elements in
$A$.  A finite set may have no elements (the empty set), or one element,
or two elements,\dots or any nonnegative integer number of elements.

Now suppose $R:A \to B$ is a function.  Then every arrow in the diagram
for $R$ comes from exactly one element of $A$, so the number of arrows is
at most the number of elements in $A$.  That is, if $R$ is a function,
then
\[
\card{A} \geq \#\text{arrows}.
\]
Similarly, if $R$ is surjective, then every element of $B$ has an arrow
into it, so there must be at least as many arrows in the diagram as the
size of $B$.  That is,
\[
\#\text{arrows} \geq \card{B}.
\]
Combining these inequalities implies that if $R$ is a surjective
function, then $\card{A} \geq \card{B}$.  In short, if we write
$A \surj B$ to mean that there is a surjective function from $A$ to
$B$, then we've just proved a lemma: if $A \surj B$, then
$\card{A} \geq \card{B}$.  The following definition and lemma lists
this statement and three similar rules relating domain and codomain
size to relational properties.

\begin{definition}\label{bigger}
  Let $A,B$ be (not necessarily finite) sets.  Then
  \begin{enumerate}
  \item $A$ \term{$\surj$} $B$ iff there is a surjective \emph{function} from $A$ to $B$.  

  \item $A$ \term{$\inj$} $B$ iff there is a total injective \emph{relation} from $A$ to $B$.

  \item $A$ \term{$\bij$} $B$ iff there is a bijection from $A$ to $B$.  

  \item $A$ \term{$\strict$} $B$ iff $A \surj B$, but not $B \surj A$.  

  \end{enumerate}
\end{definition}


\begin{lemma}\label{mapruldef}
\hyperdef{mapping-rule}{lemma}{[Mapping Rules]} \mbox{}
Let $A$ and $B$ be finite sets.

\begin{enumerate}

\item\label{mapping-sur} If $A \surj B$, then $\card{A} \geq \card{B}$.

\item\label{mapping-inj} If $A \inj B$, then $\card{A} \leq \card{B}$.

\item\label{mapping-bij} If $R \bij B$, then $\card{A} = \card{B}$.

\item\label{mapping-strict} If $R \strict B$, then $\card{A} > \card{B}$.

\end{enumerate}

\end{lemma}

Mapping rule~\ref{mapping-inj} can be explained by the same kind of
``arrow reasoning'' we used for rule~\ref{mapping-sur}.
Rules~\ref{mapping-bij} and ~\ref{mapping-strict} are immediate
consequences of these first two mapping rules.

\subsection{The sizes of infinite sets}

Mapping Rule~\ref{mapping-sur} has a converse:
if the size of a finite set, $A$, is greater than or equal to the size of
another finite set, $B$, then it's always possible to define a
surjective function from $A$ to $B$.  In fact, the surjection can be a
total function.  To see how this works, suppose for example that
\begin{align*}
A & =\set{a_0,a_1,a_2,a_3,a_4,a_5}\\
B & =\set{b_0,b_1,b_2,b_3}.
\end{align*}
Then define a total function $f:A\to B$ by the rules
\[
f(a_0) \eqdef b_0,\  f(a_1) \eqdef b_1,\  f(a_2) \eqdef b_2,\  f(a_3)=
f(a_4)=f(a_5) \eqdef b_3.
\]

\begin{editingnotes}

\[
f(a_i) \eqdef b_{\min(i,3)},
\]
for $i=0, \dots, 5$.  Since $5 \geq 3$, this $f$ is a surjection.
\end{editingnotes}

In fact, if $A$ and $B$ are finite sets of the same size, then we could also
define a bijection from $A$ to $B$ by this method.

In short, we have figured out if $A$ and $B$ are finite sets, then
$\card{A} \geq \card{B}$ \emph{if and only if} $A \surj B$, and similar
iff's hold for all the other Mapping Rules:
\begin{lemma}\label{finbig}
For \emph{finite} sets, $A,B$,
\begin{align*}
\card{A} \geq \card{B} & \qiff A \surj B,\\
\card{A} \leq \card{B} & \qiff A \inj B,\\
\card{A} = \card{B} & \qiff A \bij B,\\
\card{A} > \card{B} & \qiff A \strict B.
\end{align*}
\end{lemma}

This lemma suggests a way to generalize size comparisons to infinite sets,
namely, we can think of the relation $\surj$ as an ``\emph{at least as big
  as}'' relation between sets, even if they are infinite.  Similarly, the
relation $\bij$ can be regarded as a ``\term{same size}'' relation between
(possibly infinite) sets, and $\strict$ can be thought of as a
``\term{strictly bigger} than'' relation between sets.

\textcolor{red}{\textbf{Warning}}: We haven't, and won't, define what the
``size'' of an infinite is.  The definition of infinite ``sizes'' is
cumbersome and technical, and we can get by just fine without it.  All we
need are the ``as big as'' and ``same size'' relations, $\surj$ and
$\bij$, between sets.

But there's something else to \textcolor{red}{watch out for}.  We've
referred to $\surj$ as an ``as big as'' relation and $\bij$ as a ``same
size'' relation on sets.  Of course most of the ``as big as'' and ``same
size'' properties of $\surj$ and $\bij$ on finite sets do carry over to
infinite sets, but \emph{some important ones don't} ---as we're about to
show.  So you have to be careful: don't assume that $\surj$ has any
particular ``as big as'' property on \emph{infinite} sets until it's been
proved.

Let's begin with some familiar properties of the ``as big as'' and ``same
size'' relations on finite sets that do carry over exactly to infinite
sets:
\begin{lemma}\label{translem}
For any sets, $A,B,C$,
\begin{enumerate}

\item \label{bigtrans}
$A \surj  B \text{ and } B \surj C, \qimplies  A \surj C$.

\item \label{sametrans} $A \bij B \text{ and } B \bij C, \qimplies A \bij C$.

\item\label{sameABA}
$A \bij B \qimplies B \bij A$.
\end{enumerate}
\end{lemma}

Lemma~\ref{translem}.\ref{bigtrans} and~\ref{translem}.\ref{sametrans}
follow immediately from the fact that compositions of surjections are
surjections, and likewise for bijections, and
Lemma~\ref{translem}.\ref{sameABA} follows from the fact that the
inverse of a bijection is a bijection.  We'll leave a proof of these
facts to Problem~\ref{CP_surj_relation}.

Another familiar property of finite sets carries over to infinite sets,
but this time it's not so obvious:
\begin{theorem} [\idx{Schr\"oder-Bernstein}] For any sets $A,B$, if $A \surj B$
  and $B \surj A$, then $A \bij B$.
\end{theorem}

That is, the Schr\"oder-Bernstein Theorem says that if $A$ is at least as
big as $B$ and conversely, $B$ is at least as big as $A$, then $A$ is the
same size as $B$.  Phrased this way, you might be tempted to take this
theorem for granted, but that would be a mistake.  For infinite sets $A$
and $B$, the Schr\"oder-Bernstein Theorem is actually pretty technical.
Just because there is a surjective function $f:A\to B$ ---which need not
be a bijection ---and a surjective function $g:B \to A$ ---which also need
not be a bijection ---it's not at all clear that there must be a bijection
$e:A \to B$.  The idea is to construct $e$ from parts of both $f$ and $g$.
We'll leave the actual construction to
Problem~\ref{CP_Cantor_Schroeder_Bernstein_theorem}.

\subsubsection{Infinity is different}

A basic property of finite sets that does \emph{not} carry over to
infinite sets is that adding something new makes a set bigger.  That is,
if $A$ is a finite set and $b \notin A$, then $\card{A \union \set{b}} =
\card{A}+1$, and so $A$ and $A \union \set{b}$ are not the same size.  But
if $A$ is infinite, then these two sets \emph{are} the same size!

\begin{lemma}\label{AUb}
  Let $A$ be a set and $b \notin A$.  Then $A$ is infinite iff $A \bij A
  \union \set{b}$.
\end{lemma}
\begin{proof}
  Since $A$ is \emph{not} the same size as $A \union \set{b}$ when $A$ is
  finite, we only have to show that $A \union \set{b}$ \emph{is} the same
  size as $A$ when $A$ is infinite.

That is, we have to find a bijection between $A \union \set{b}$ and $A$
when $A$ is infinite.  Here's how: since $A$ is infinite, it certainly has
at least one element; call it $a_0$.  But since $A$ is infinite, it has at
least two elements, and one of them must not be equal to $a_0$; call this
new element $a_1$.  But since $A$ is infinite, it has at least three
elements, one of which must not equal $a_0$ or $a_1$; call this new
element $a_2$.  Continuing in the way, we conclude that there is an
infinite sequence $a_0,a_1,a_2,\dots,a_n,\dots$ of different elements of
$A$.  Now it's easy to define a bijection $e: A \union \set{b} \to A$:
\begin{align*}
e(b) & \eqdef a_0,\\
e(a_n) & \eqdef a_{n+1}  &\text{ for } n \in \naturals,\\
e(a) & \eqdef a & \text{ for } a \in A - \set{b,a_0,a_1,\dots}.
\end{align*}
\end{proof}

A set, $C$, is \term{countable} iff its elements can be listed in order,
that is, the distinct elements is $A$ are precisely
\[
c_0, c_1, \dots, c_n, \dots.
\]
This means that if we defined a function, $f$, on the nonnegative integers
by the rule that $f(i) \eqdef c_i$, then $f$ would be a bijection from
$\naturals$ to $C$.  More formally,

\begin{definition}
  A set, $C$, is \term{countably infinite} iff $\naturals \bij C$.  A set
  is \term{countable} iff it is finite or countably infinite.
\end{definition}

A small modification\footnote{See Problem~\ref{CP_smallest_infinite_set}}
of the proof of Lemma~\ref{AUb} shows that countably infinite sets are
the ``smallest'' infinite sets, namely, if $A$ is a countably infinite
set, then $A \surj \naturals$.

Since adding one new element to an infinite set doesn't change its
size, it's obvious that neither will adding any \emph{finite} number
of elements.  It's a common mistake to think that this proves that you
can throw in countably infinitely many new elements.  But just because
it's ok to do something any finite number of times doesn't make it OK
to do an infinite number of times.  For example, starting from 3, you
can add 1 any finite number of times and the result will be some
integer greater than or equal to 3.  But if you add add 1 a countably
infinite number of times, you don't get an integer at all.

It turns out you really can add a countably infinite number of new
elements to a countable set and still wind up with just a countably
infinite set, but another argument is needed to prove this:

\begin{lemma}\label{countable-union}
If $A$ and $B$ are countable sets, then so is $A \union B$.
\end{lemma}

\begin{proof}
Suppose the list of distinct elements of $A$ is $a_0,a_1,\dots$ and the
list of $B$ is $b_0,b_1, \dots$.  Then a list of all the elements in $A
\union B$ is just
\begin{equation}\label{a0b0list}
a_0,b_0,a_1,b_1, \dots a_n,b_n, \dots.
\end{equation}
Of course this list will contain duplicates if $A$ and $B$ have elements
in common, but then deleting all but the first occurrences of each element in
list~\eqref{a0b0list} leaves a list of all the distinct elements of $A$
and $B$.
\end{proof}

\subsection{\idx{Power sets} are \idx{strictly bigger}}

It turns out that the ideas behind \idx{Russell's Paradox}, which caused
so much trouble for the early efforts to formulate Set Theory, lead to a
correct and astonishing fact about infinite sets: they are \emph{not all
  the same size}.

In particular,
\begin{theorem}\label{powbig}
For any set, $A$, the \idx{power set}, $\power(A)$, is \idx{strictly bigger} than $A$.
\end{theorem}
\begin{proof}
  First of all, $\power(A)$ is as big as $A$: for example, the partial
  function $f:\power(A) \to A$, where $f(\set{a}) \eqdef a$ for $a \in A$
  and $f$ is only defined on one-element sets, is a surjection.

  To show that $\power(A)$ is strictly bigger than $A$, we have to show
  that if $g$ is a function from $A$ to $\power(A)$, then $g$ is not a
  surjection.  So, mimicking Russell's Paradox, define
  \[
  A_g \eqdef \set{a \in A \suchthat a \notin g(a)}.
  \]
  Now $A_g$ is a well-defined subset of $A$, which means it is a member of
  $\power(A)$.  But $A_g$ can't be in the range of $g$, because if it
  were, we would have
\[
A_g = g(a_0)
\]
for some $a_0 \in A$, so by definition of $A_g$,
\[
a \in g(a_0) \qiff a \in A_g \qiff a \notin g(a)
\]
for all $a \in A$.  Now letting $a = a_0$ yields the contradiction
\[
a_0 \in g(a_0) \qiff a_0 \notin g(a_0).
\]
So $g$ is not a surjection, because there is an element in the power set
of $A$, namely the set $A_g$, that is not in the range of $g$.
\end{proof}

\subsubsection{Larger Infinities}

There are lots of different sizes of infinite sets.  For example, starting
with the infinite set, $\naturals$, of nonnegative integers, we can build
the infinite sequence of sets
\[
\naturals,\ \power(\naturals),\ \power(\power(\naturals)),\
\power(\power(\power(\naturals))),\ \dots.
\]
By Theorem~\ref{powbig}, each of these sets is strictly bigger than all
the preceding ones.  But that's not all: the union of all the sets in the
sequence is strictly bigger than each set in the sequence
(see Problem~\ref{CP_power_set_tower}).  In this way you can keep going,
building still bigger infinities.

So there is an endless variety of different size infinities.

\subsection{Infinities in Computer Science}

We've run into a lot of computer science students who wonder why they
should care about infinite sets: any data set in a computer memory is
limited by the size of memory, and since the universe appears to have
finite size, there is a limit on the possible size of computer memory.

\iffalse need to learn all this abstract theory of infinite sets, and this
is a good question.  \fi

The problem with this argument is that universe-size bounds on data
items are so big and uncertain (the universe seems to be getting
bigger all the time), that it's simply not helpful to make use of
possible bounds.  For example, by this argument the physical sciences
shouldn't assume that measurements might yield arbitrary real numbers,
because there can only be a finite number of finite measurements in a
universe of finite lifetime.  What do you think scientific theories
would look like without using the infinite set of real numbers?

Similary, in computer science, it simply isn't plausible that writing a
program to add nonnegative integers with up to as many digits as, say, the
stars in the sky (billions of galaxies each with billions of stars), would
be any different than writing a program that would add any two integers
no matter how many digits they had.

That's why basic programming data types like integers or strings, for
example, can be defined without imposing any bound on the sizes of
data items.  Each datum of type \idx{\texttt{string}} has only a
finite number of letters, but there are an infinite number of data items
of type \texttt{string}.  When we then consider string procedures of
type \idx{\texttt{string-->string}}, not only are there
an infinite number of such procedures, but each procedure generally
behaves differently on different inputs, so that a single
\texttt{string-->string} procedure may embody an infinite number of
behaviors.

In short, an educated computer scientist can't get around having to
understand infinite sets.

\begin{problems}

\classproblems
\pinput{CP_surj_relation}

\pinput{CP_smallest_infinite_set}

\pinput{CP_mapping_rule}

\pinput{CP_set_product_bijection}

\pinput{CP_rationals_are_countable}

\pinput{CP_Cantor_Schroeder_Bernstein_theorem}

\homeworkproblems

\pinput{PS_composition-of-jections}
\pinput{PS_unit_interval}

\end{problems}

\endinput
