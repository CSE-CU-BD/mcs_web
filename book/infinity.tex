\chapter{Infinity}

\begin{editingnotes}
ADD INTRO about practice reasoning about infinity.
\end{editingnotes}

\section{Surjective and Injective Relations}\label{surj_sec}

There are a few properties of relations that will be useful when we take
up the topic of counting because they imply certain relations between the
\emph{sizes} of domains and codomains.  We say a binary relation $R : A
\to B$ is:

\begin{itemize}

\item \term{total} when every element of $A$ is assigned to some element of
  $B$; more concisely, $R$ is total iff $A=RB$.

\item \term{surjective} when every element of $B$ is mapped to \textit{at
least once}\footnote{
The names ``surjective'' and ``injective'' are unmemorable and
nondescriptive.  Some authors use the term \term{onto} for surjective and
\emph{one-to-one} for injective, which are shorter but arguably no more
memorable.}; more concisely, $R$ is surjective iff $AR=B$.

\item \term{injective} if every element of $B$ is mapped to \textit{at
most once}, and

\item \term{bijective} if $R$ is total, surjective, and injective
  \emph{function}.

\end{itemize}

Note that this definition of $R$ being total agrees with the definition in
Section~\ref{funcsubsec} when $R$ is a function.

If $R$ is a binary relation from $A$ to $B$, we define $AR$ to to be the
\emph{range} of $R$.  So a relation is surjective iff its range equals its
codomain.  Again, in the case that $R$ is a function, these definitions of
``range'' and ``total'' agree with the definitions in
Section~\ref{funcsubsec}.

\subsection{Relation Diagrams}
We can explain all these properties of a relation $R:A \to B$ in terms of
a diagram where all the elements of the domain, $A$, appear in one column
(a very long one if $A$ is infinite) and all the elements of the codomain,
$B$, appear in another column, and we draw an arrow from a point $a$ in
the first column to a point $b$ in the second column when $a$ is related
to $b$ by $R$.  For example, here are diagrams for two functions:

\begin{center}
\begin{tabular}{ccc}

\unitlength = 2pt
\begin{picture}(50,60)(-10,-5)
\thinlines
\put(-5,50){\makebox(0,0){$A$}}
  \put(35,50){\makebox(0,0){$B$}}
\put(-5,40){\makebox(0,0){a}}
  \put(0,40){\vector(1,0){28}}
  \put(35,40){\makebox(0,0){1}}
\put(-5,30){\makebox(0,0){b}}
  \put(0,30){\vector(3,-1){28}}
  \put(35,30){\makebox(0,0){2}}
\put(-5,20){\makebox(0,0){c}}
  \put(0,20){\vector(3,-1){28}}
  \put(35,20){\makebox(0,0){3}}
\put(-5,10){\makebox(0,0){d}}
  \put(0,10){\vector(3,2){28}}
  \put(35,10){\makebox(0,0){4}}
\put(-5,0){\makebox(0,0){e}}
  \put(0,0){\vector(3,2){28}}
\end{picture}

& \hspace{0.5in} &

\unitlength = 2pt
\begin{picture}(50,60)(-10,-5)
\thinlines
\put(-5,50){\makebox(0,0){$A$}}
  \put(35,50){\makebox(0,0){$B$}}
\put(-5,40){\makebox(0,0){a}}
  \put(0,40){\vector(1,0){28}}
  \put(35,40){\makebox(0,0){1}}
\put(-5,30){\makebox(0,0){b}}
  \put(0,30){\vector(3,-1){28}}
  \put(35,30){\makebox(0,0){2}}
\put(-5,20){\makebox(0,0){c}}
  \put(0,20){\vector(3,-2){28}}
  \put(35,20){\makebox(0,0){3}}
\put(-5,10){\makebox(0,0){d}}
  \put(0,10){\vector(3,2){28}}
  \put(35,10){\makebox(0,0){4}}
\put(35,0){\makebox(0,0){5}}
\end{picture}

\end{tabular}
\end{center}

Here is what the definitions say about such pictures:
\begin{itemize}

\item ``$R$ is a function'' means that every point in the domain column,
  $A$, has \emph{at most one arrow out of it}.

\item ``$R$ is total'' means that \emph{every} point in the $A$ column has
  \emph{at least one arrow out of it}.  So if $R$ is a function, being
  total really means every point in the $A$ column has
  \emph{exactly one arrow out of it}.

\item ``$R$ is surjective'' means that \emph{every} point in the codomain
  column, $B$, has \emph{at least one arrow into it}.

\item ``$R$ is injective'' means that every point in the codomain column,
  $B$, has \emph{at most one arrow into it}.

\item ``$R$ is bijective'' means that \emph{every} point in the $A$ column
      has exactly one arrow out of it, and \emph{every} point in the $B$ column
      has exactly one arrow into it.

\end{itemize}

So in the diagrams above, the relation on the left is a total, surjective
function (every element in the $A$ column has exactly one arrow out, and
every element in the $B$ column has at least one arrow in), but not
injective (element 3 has two arrows going into it).  The relation on the
right is a total, injective function (every element in the $A$ column has
exactly one arrow out, and every element in the $B$ column has at most one
arrow in), but not surjective (element 4 has no arrow going into it).

%Define $\inv{R}$ and explain with reversed arrows.  Deduce that that
%$R$ is total iff $\inv{R} is surjective, $R$ is function iff
%$\inv{R}$ is injective.

Notice that the arrows in a diagram for $R$ precisely correspond to the
pairs in the graph of $R$.  But $\graph{R}$ does not determine by itself
whether $R$ is total or surjective; we also need to know what the domain
is to determine if $R$ is total, and we need to know the codomain to tell
if it's surjective.
\begin{example}
  The function defined by the formula $1/x^2$ is total if its domain is
  $\reals^+$ but partial if its domain is some set of real numbers
  including 0.  It is bijective if its domain and codomain are both
  $\reals^+$, but neither injective nor surjective if its domain and
  codomain are both $\reals$.
\end{example}


\hyperdef{mapping}{rule}{\section{The Mapping Rule}}\label{mappingrule_sec}

The relational properties above are useful in figuring out the relative
sizes of domains and codomains.

If $A$ is a finite set, we let $\card{A}$ be the number of elements in
$A$.  A finite set may have no elements (the empty set), or one element,
or two elements,\dots or any nonnegative integer number of elements.

Now suppose $R:A \to B$ is a function.  Then every arrow in the diagram
for $R$ comes from exactly one element of $A$, so the number of arrows is
at most the number of elements in $A$.  That is, if $R$ is a function,
then
\[
\card{A} \geq \#\text{arrows}.
\]
Similarly, if $R$ is surjective, then every element of $B$ has an arrow
into it, so there must be at least as many arrows in the diagram as the
size of $B$.  That is,
\[
\#\text{arrows} \geq \card{B}.
\]
Combining these inequalities implies that if $R$ is a surjective function,
then $\card{A} \geq \card{B}$.  In short, if we write $A \surj B$ to mean
that there is a surjective function from $A$ to $B$, then we've just
proved a lemma: if $A \surj B$, then $\card{A} \geq \card{B}$.  The
following definition and lemma lists include this statement and three
similar rules relating domain and codomain size to relational properties.

\begin{definition}\label{bigger}
  Let $A,B$ be (not necessarily finite) sets.  Then
  \begin{enumerate}
  \item $A$ \term{$\surj$} $B$ iff there is a surjective \emph{function} from $A$ to $B$.  

  \item $A$ \term{$\inj$} $B$ iff there is a total injective \emph{relation} from $A$ to $B$.

  \item $A$ \term{$\bij$} $B$ iff there is a bijection from $A$ to $B$.  

  \item $A$ \term{$\strict$} $B$ iff $A \surj B$, but not $B \surj A$.  

  \end{enumerate}
\end{definition}


\begin{lemma}\label{mapruldef}
\hyperdef{mapping-rule}{lemma}{[Mapping Rules]} \mbox{}
Let $A$ and $B$ be finite sets.

\begin{enumerate}

\item\label{mapping-sur} If $A \surj B$, then $\card{A} \geq \card{B}$.

\item\label{mapping-inj} If $A \inj B$, then $\card{A} \leq \card{B}$.

\item\label{mapping-bij} If $R \bij B$, then $\card{A} = \card{B}$.

\item\label{mapping-strict} If $R \strict B$, then $\card{A} > \card{B}$.

\end{enumerate}

\end{lemma}

Mapping rule~\ref{mapping-inj} can be explained by the same kind of
``arrow reasoning'' we used for rule~\ref{mapping-sur}.
Rules~\ref{mapping-bij} and ~\ref{mapping-strict} are immediate
consequences of these first two mapping rules.

\section{The sizes of infinite sets}

Mapping Rule~\ref{mapping-sur} has a converse:
if the size of a finite set, $A$, is greater than or equal to the size of
another finite set, $B$, then it's always possible to define a
surjective function from $A$ to $B$.  In fact, the surjection can be a
total function.  To see how this works, suppose for example that
\begin{align*}
A & =\set{a_0,a_1,a_2,a_3,a_4,a_5}\\
B & =\set{b_0,b_1,b_2,b_3}.
\end{align*}
Then define a total function $f:A\to B$ by the rules
\[
f(a_0) \eqdef b_0,\  f(a_1) \eqdef b_1,\  f(a_2) \eqdef b_2,\  f(a_3)=
f(a_4)=f(a_5) \eqdef b_3.
\]

\begin{editingnotes}

\[
f(a_i) \eqdef b_{\min(i,3)},
\]
for $i=0, \dots, 5$.  Since $5 \geq 3$, this $f$ is a surjection.

\end{editingnotes}
In fact, if $A$ and $B$ are finite sets of the same size, then we could also
define a bijection from $A$ to $B$ by this method.

In short, we have figured out if $A$ and $B$ are finite sets, then
$\card{A} \geq \card{B}$ \emph{if and only if} $A \surj B$, and similar
iff's hold for all the other Mapping Rules:
\begin{lemma}\label{finbig}
For \emph{finite} sets, $A,B$,
\begin{align*}
\card{A} \geq \card{B} & \qiff A \surj B,\\
\card{A} \leq \card{B} & \qiff A \inj B,\\
\card{A} = \card{B} & \qiff A \bij B,\\
\card{A} > \card{B} & \qiff A \strict B.
\end{align*}
\end{lemma}

This lemma suggests a way to generalize size comparisons to infinite sets,
namely, we can think of the relation $\surj$ as an ``\emph{at least as big
  as}'' relation between sets, even if they are infinite.  Similarly, the
relation $\bij$ can be regarded as a ``\term{same size}'' relation between
(possibly infinite) sets, and $\strict$ can be thought of as a
``\term{strictly bigger} than'' relation between sets.

\textcolor{red}{\textbf{Warning}}: We haven't, and won't, define what the
``size'' of an infinite is.  The definition of infinite ``sizes'' is
cumbersome and technical, and we can get by just fine without it.  All we
need are the ``as big as'' and ``same size'' relations, $\surj$ and
$\bij$, between sets.

But there's something else to \textcolor{red}{watch out for}.  We've
referred to $\surj$ as an ``as big as'' relation and $\bij$ as a ``same
size'' relation on sets.  Of course most of the ``as big as'' and ``same
size'' properties of $\surj$ and $\bij$ on finite sets do carry over to
infinite sets, but \emph{some important ones don't} ---as we're about to
show.  So you have to be careful: don't assume that $\surj$ has any
particular ``as big as'' property on \emph{infinite} sets until it's been
proved.

Let's begin with some familiar properties of the ``as big as'' and ``same
size'' relations on finite sets that do carry over exactly to infinite
sets:
\begin{lemma}\label{translem}
For any sets, $A,B,C$,
\begin{enumerate}

\item \label{bigtrans}
$A \surj  B \text{ and } B \surj C, \qimplies  A \surj C$.

\item \label{sametrans} $A \bij B \text{ and } B \bij C, \qimplies A \bij C$.

\item\label{sameABA}
$A \bij B \qimplies B \bij A$.
\end{enumerate}
\end{lemma}

Lemma~\ref{translem}.\ref{bigtrans} and~\ref{translem}.\ref{sametrans}
follow immediately from the fact that compositions of surjections are
surjections, and likewise for bijections, and
Lemma~~\ref{translem}.\ref{sameABA} follows from the fact that the inverse
of a bijection is a bijection.  We'll leave a proof of these facts to
Problem~\ref{CP_surj_relation}.

Another familiar property of finite sets carries over to infinite sets,
but this time it's not so obvious:
\begin{theorem} [\idx{Schr\"oder-Bernstein}] For any sets $A,B$, if $A \surj B$
  and $B \surj A$, then $A \bij B$.
\end{theorem}

That is, the Schr\"oder-Bernstein Theorem says that if $A$ is at least as
big as $B$ and conversely, $B$ is at least as big as $A$, then $A$ is the
same size as $B$.  Phrased this way, you might be tempted to take this
theorem for granted, but that would be a mistake.  For infinite sets $A$
and $B$, the Schr\"oder-Bernstein Theorem is actually pretty technical.
Just because there is a surjective function $f:A\to B$ ---which need not
be a bijection ---and a surjective function $g:B \to A$ ---which also need
not be a bijection ---it's not at all clear that there must be a bijection
$e:A \to B$.  The idea is to construct $e$ from parts of both $f$ and $g$.
We'll leave the actual construction to
Problem~\ref{CP_Cantor_Schroeder_Bernstein_theorem}.

\subsubsection{Infinity is different}

A basic property of finite sets that does \emph{not} carry over to
infinite sets is that adding something new makes a set bigger.  That is,
if $A$ is a finite set and $b \notin A$, then $\card{A \union \set{b}} =
\card{A}+1$, and so $A$ and $A \union \set{b}$ are not the same size.  But
if $A$ is infinite, then these two sets \emph{are} the same size!

\begin{lemma}\label{AUb}
  Let $A$ be a set and $b \notin A$.  Then $A$ is infinite iff $A \bij A
  \union \set{b}$.
\end{lemma}
\begin{proof}
  Since $A$ is \emph{not} the same size as $A \union \set{b}$ when $A$ is
  finite, we only have to show that $A \union \set{b}$ \emph{is} the same
  size as $A$ when $A$ is infinite.

That is, we have to find a bijection between $A \union \set{b}$ and $A$
when $A$ is infinite.  Here's how: since $A$ is infinite, it certainly has
at least one element; call it $a_0$.  But since $A$ is infinite, it has at
least two elements, and one of them must not be equal to $a_0$; call this
new element $a_1$.  But since $A$ is infinite, it has at least three
elements, one of which must not equal $a_0$ or $a_1$; call this new
element $a_2$.  Continuing in the way, we conclude that there is an
infinite sequence $a_0,a_1,a_2,\dots,a_n,\dots$ of different elements of
$A$.  Now it's easy to define a bijection $e: A \union \set{b} \to A$:
\begin{align*}
e(b) & \eqdef a_0,\\
e(a_n) & \eqdef a_{n+1}  &\text{ for } n \in \naturals,\\
e(a) & \eqdef a & \text{ for } a \in A - \set{b,a_0,a_1,\dots}.
\end{align*}
\end{proof}

A set, $C$, is \term{countable} iff its elements can be listed in order,
that is, the distinct elements is $A$ are precisely
\[
c_0, c_1, \dots, c_n, \dots.
\]
This means that if we defined a function, $f$, on the nonnegative integers
by the rule that $f(i) \eqdef c_i$, then $f$ would be a bijection from
$\naturals$ to $C$.  More formally,

\begin{definition}
  A set, $C$, is \term{countably infinite} iff $\naturals \bij C$.  A set
  is \term{countable} iff it is finite or countably infinite.
\end{definition}

A small modification\footnote{See Problem~\ref{CP_smallest_infinite_set}}
of the proof of Lemma~\ref{AUb} shows that countably infinite sets are
the ``smallest'' infinite sets, namely, if $A$ is a countably infinite
set, then $A \surj \naturals$.

Since adding one new element to an infinite set doesn't change its
size, it's obvious that neither will adding any \emph{finite} number
of elements.  It's a common mistake to think that this proves that you
can throw in countably infinitely many new elements.  But just because
it's ok to do something any finite number of times doesn't make it OK
to do an infinite number of times.  For example, starting from 3, you
can add 1 any finite number of times and the result will be some
integer greater than or equal to 3.  But if you add add 1 a countably
infinite number of times, you don't get an integer at all.

It turns out you really can add a countably infinite number of new
elements to a countable set and still wind up with just a countably
infinite set, but another argument is needed to prove this:

\begin{lemma}\label{countable-union}
If $A$ and $B$ are countable sets, then so is $A \union B$.
\end{lemma}

\begin{proof}
Suppose the list of distinct elements of $A$ is $a_0,a_1,\dots$ and the
list of $B$ is $b_0,b_1, \dots$.  Then a list of all the elements in $A
\union B$ is just
\begin{equation}\label{a0b0list}
a_0,b_0,a_1,b_1, \dots a_n,b_n, \dots.
\end{equation}
Of course this list will contain duplicates if $A$ and $B$ have elements
in common, but then deleting all but the first occurrences of each element in
list~\eqref{a0b0list} leaves a list of all the distinct elements of $A$
and $B$.
\end{proof}

\subsection{Infinities in Computer Science}

We've run into a lot of computer science students who wonder why they
should care about infinite sets: any data set in a computer memory is
limited by the size of memory, and since the universe appears to have
finite size, there is a limit on the possible size of computer memory.

\iffalse need to learn all this abstract theory of infinite sets, and this
is a good question.  \fi

The problem with this argument is that universe-size bounds on data
items are so big and uncertain (the universe seems to be getting
bigger all the time), that it's simply not helpful to make use of
possible bounds.  For example, by this argument the physical sciences
shouldn't assume that measurements might yield arbitrary real numbers,
because there can only be a finite number of finite measurements in a
universe of finite lifetime.  What do you think scientific theories
would look like without using the infinite set of real numbers?

Similary, in computer science, it simply isn't plausible that writing a
program to add nonnegative integers with up to as many digits as, say, the
stars in the sky (billions of galaxies each with billions of stars), would
be any different than writing a program that would add any two integers
no matter how many digits they had.

That's why basic programming data types like integers or strings, for
example, can be defined without imposing any bound on the sizes of
data items.  Each datum of type \idx{\texttt{string}} has only a
finite number of letters, but there are an infinite number of data items
of type \texttt{string}.  When we then consider string procedures of
type \idx{\texttt{string-->string}}, not only are there
an infinite number of such procedures, but each procedure generally
behaves differently on different inputs, so that a single
\texttt{string-->string} procedure may embody an infinite number of
behaviors.

In short, an educated computer scientist can't get around having to
understand infinite sets.

\begin{problems}

\classproblems
\pinput{CP_surj_relation}

\pinput{CP_smallest_infinite_set}

\pinput{CP_mapping_rule}

\pinput{CP_set_product_bijection}

\pinput{CP_rationals_are_countable}

\pinput{CP_Cantor_Schroeder_Bernstein_theorem}

\homeworkproblems

\pinput{PS_unit_interval}

\end{problems}

\endinput
