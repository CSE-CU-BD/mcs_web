\chapter{The Well Ordering Principle}\label{well_ordering_chap}

\textbox{
\centerline{Every \emph{nonempty} set of \emph{nonnegative integers} has a
\emph{smallest} element.}
}

This statement is known as The \term{Well Ordering Principle}.  Do you
believe it?  Seems sort of obvious, right?  But notice how tight it is: it
requires a \emph{nonempty} set ---it's false for the empty set which has
\emph{no} smallest element because it has no elements at all!  And it
requires a set of \emph{nonnegative} integers ---it's false for the set of
\emph{negative} integers and also false for some sets of nonnegative
\emph{rationals} ---for example, the set of positive rationals.  So, the
Well Ordering Principle captures something special about the nonnegative
integers.

\hyperdef{well}{ordering}{\section{Well Ordering Proofs}}

While the Well Ordering Principle may seem obvious, \iffalse it looks
nothing like the induction axiom, and\fi it's hard to see offhand why it
is useful.  But in fact, it provides one of the most important proof rules
in discrete mathematics.  \iffalse We'll explain this after we introduce a
template for well ordering principle proofs resembling the template in
Section~\ref{templ-induct-proofs} for a proof by strong induction.\fi

In fact, looking back, we took the Well Ordering Principle for granted in
proving that $\sqrt{2}$ is irrational.  That proof assumed that for any
positive integers $m$ and $n$, the fraction $m/n$ can be written in
\term{lowest terms}, that is, in the form $m'/n'$ where $m'$ and $n'$
are positive integers with no common factors.  How do we know this is
always possible?

Suppose to the contrary that there were $m,n \in \integers^+$ such that the
fraction $m/n$ cannot be written in lowest terms.  Now let $C$ be the set
of positive integers that are numerators of such fractions.  Then $m \in
C$, so $C$ is nonempty.  Therefore, by Well Ordering, there must be a
smallest integer, $m_0 \in C$.  So by definition of $C$, there is an
integer $n_0 > 0$ such that
\[
\text{the fraction } \frac{m_0}{n_0} \text{ cannot be written in lowest
terms.}
\]
This means that $m_0$ and $n_0$ must have a common factor, $p>1$.  But
\[
\frac{m_0/p}{n_0/p} = \frac{m_0}{n_0},
\]
so any way of expressing the left hand fraction in lowest terms would also
work for $m_0/n_0$, which implies
\[
\text{the fraction } \frac{m_0/p}{n_0/p} \text{ cannot be in written in
lowest terms either.}
\]
So by definition of $C$, the numerator, $m_0/p$, is in $C$.  But $m_0/p <
m_0$, which contradicts the fact that $m_0$ is the smallest element of $C$.

Since the assumption that $C$ is nonempty leads to a contradiction, it
follows that $C$ must be empty.  That is, that there are no numerators of
fractions that can't be written in lowest terms, and hence there are no
such fractions at all.

We've been using the Well Ordering Principle on the sly from early on!

%\begin{problems}
%\practiceproblems
%\classproblems
%\homeworkproblems
%\end{problems}

\section{Template for Well Ordering Proofs}

More generally, there is a standard way to use Well Ordering to prove that
some property, $P(n)$ holds for every nonnegative integer, $n$.  Here is a
standard way to organize such a well ordering proof:

\textbox{To prove that ``$P(n)$ is true for all $n\in \naturals$'' using
the Well Ordering Principle:
\begin{itemize}

\item Define the set, $C$, of \emph{counterexamples} to $P$ being
  true.  Namely, define\footnote{The notation $\set{n \suchthat P(n)}$
    means ``the set of all elements $n$, for which $P(n)$ is true.}
\[
C \eqdef \set{n\in\naturals \suchthat P(n) \text{ is false}}.
\]

\item Assume for proof by contradiction that $C$ is nonempty.

\item By the Well Ordering Principle, there will be a smallest
      element, $n$, in $C$.

\item Reach a contradiction (somehow) ---often by showing how to use $n$
to find another member of $C$ that is smaller than $n$.  (This is the
open-ended part of the proof task.)

\item Conclude that $C$ must be empty, that is, no counterexamples exist.
QED

\end{itemize}
}


\begin{problems}
%\practiceproblems

\classproblems
\pinput{CP_6_and_15_cent_stamps}
\pinput{PS_Lehmans_equation}

\homeworkproblems
\pinput{PS_postage_by_WOP}
\end{problems}

\section{Summing the Integers}
Let's use this this template to prove %Theorem~\ref{th:sum-to-n}. 

\begin{theorem*}  %\label{sum-to-n}
\begin{equation}\label{sum1n}
1 + 2 + 3 + \cdots + n = n(n+1)/2
\end{equation}
for all nonnegative integers, $n$.
\end{theorem*}

First, we better address of a couple of ambiguous special
cases before they trip us up:
%
\begin{itemize}
%
\item If $n = 1$, then there is only one term in the summation, and so $1
  + 2 + 3 + \cdots + n$ is just the term 1.  Don't be misled by the
  appearance of 2 and 3 and the suggestion that $1$ and $n$ are distinct
  terms!
%
\item If $n \leq 0$, then there are no terms at all in the summation.  By
convention, the sum in this case is 0.
%
\end{itemize}
%
So while the dots notation is convenient, you have to watch out for these
special cases where the notation is misleading!  (In fact, whenever you
see the dots, you should be on the lookout to be sure you understand the
pattern, watching out for the beginning and the end.)

We could have eliminated the need for guessing by rewriting the left side
of~\eqref{sum1n} with \term{summation notation}:
\[
\sum_{i=1}^n i
\qquad \text{or} \qquad
\sum_{1 \leq i \leq n} i.
\]
Both of these expressions denote the sum of all values taken by the
expression to the right of the sigma as the variable, $i$, ranges from 1
to $n$.  Both expressions make it clear what~\eqref{sum1n} means when
$n=1$.  The second expression makes it clear that when $n=0$, there are no
terms in the sum, though you still have to know the convention that a sum
of no numbers equals 0 (the \emph{product} of no numbers is 1, by the
way).

OK, back to the proof:

\begin{proof}
By contradiction.  Assume that the theorem is
\emph{false}.  Then, some nonnegative integers serve as
\emph{counterexamples} to it. Let's collect them in a set: 
\[
C \eqdef \set{n\in\naturals \suchthat 
        1 + 2 + 3 + \cdots + n \neq \frac{n(n+1)}{2}}.
\]
By our assumption that the theorem admits counterexamples, $C$ is a
nonempty set of nonnegative integers.  So, by the Well Ordering Principle,
$C$ has a minimum element, call it~$c$.  That is, $c$ is the
\emph{\idx{smallest counterexample}} to the theorem.

Since $c$ is the smallest counterexample, we know that~\eqref{sum1n} is
false for $n=c$ but true for all nonnegative integers $n<c$.
But~\eqref{sum1n} is true for $n=0$, so $c > 0$.  This means $c-1$ is a
nonnegative integer, and since it is less than $c$, equation~\eqref{sum1n}
is true for $c-1$.  That is,
\[
        1 + 2 + 3 + \cdots + (c-1) = \frac{(c-1)c}{2}.
\]
But then, adding $c$ to both sides we get
\[
1 + 2 + 3 + \cdots + (c-1) + c 
        = \frac{(c-1)c}{2} + c
        = \frac{c^2 - c + 2c}{2} 
        = \frac{c(c+1)}{2},
\]
which means that~\eqref{sum1n} does hold for $c$, after all!  This is a
contradiction, and we are done.
\end{proof}

\begin{problems}
%\practiceproblems
\classproblems
\pinput{CP_sum_of_squares}
%\homeworkproblems
\end{problems}

\section{Factoring into Primes}

We've previously taken for granted the \term{Prime Factorization
  Theorem} that every integer greater than one has a
unique\footnote{\dots unique up to the order in which the prime
  factors appear} expression as a product of prime numbers.  This is
another of those familiar mathematical facts which are not really
obvious.  We'll prove the uniqueness of prime factorization in a later
chapter, but well ordering gives an easy proof that every integer
greater than one can be expressed as \emph{some} product of primes.

\begin{theorem}\label{factor_into_primes}
Every natural number can be factored as a product of primes.
\end{theorem}
\begin{proof}
The proof is by Well Ordering.

Let $C$ be the set of all integers greater than one that cannot be
factored as a product of primes.  We assume $C$ is not empty and derive a
contradiction.

If $C$ is not empty, there is a least element, $n \in C$, by Well
Ordering.  The $n$ can't be prime, because a prime by itself is considered
a (length one) product of primes and no such products are in $C$.

So $n$ must be a product of two integers $a$ and $b$ where $1<a,b<n$.
Since $a$ and $b$ are smaller than the smallest element in $C$, we know
that $a,b \notin C$.  In other words, $a$ can be written as a product of
primes $p_1p_2\cdots p_k$ and $b$ as a product of primes $q_1\cdots q_l$.
Therefore, $n=p_1\cdots p_k q_1 \cdots q_l$ can be written as a product of
primes, contradicting the claim that $n \in C$.  Our assumption that $C
\neq \emptyset$ must therefore be false.
\end{proof}

\endinput
