\chapter{Sums and Asymptotics}\label{asymptotics_chap}

Sums and products arise regularly in the analysis of algorithms,
financial applications, physical problems, and probabilistic systems.
For example, we have already encountered the sum $1 + 2 + 4 + \dots +
N$ when counting the number of nodes in a complete binary tree with
$N$ inputs.  Although such a sum can be represented compactly using
the sigma notation
\begin{equation}
    \sum_{i = 0}^{\log N} 2^i,
\end{equation}
it is a lot easier and more helpful to express the sum by
its \term{closed form} value
\begin{equation*}
    2 N - 1.
\end{equation*}

By \term{closed form}, we mean an expression that does not make use of
summation or product symbols or otherwise need those handy (but
sometimes troublesome) dots\dots.  Expressions in closed form are
usually easier to evaluate (it doesn't get much simpler than $2 N -
1$, for example) and it is usually easier to get a feel for their
magnitude than expressions involving large sums and products.

But how do you find a closed form for a sum or product?  Well, it's
part math and part art.  And it is the subject of this chapter.

We will start the chapter with a motivating example involving
annuities.  Figuring out the value of the annuity will involve a
large and nasty-looking sum.  We will then describe several methods
for finding closed forms for all sorts of sums, including the annuity
sums.  In some cases, a closed form for a sum may not exist and so we
will provide a general method for finding good upper and lower bounds
on the sum (which are closed form, of course).

The methods we develop for sums will also work for products since you
can convert any product into a sum by taking a logarithm of the
product.  As an example, we will use this approach to find a good
closed-form approximation to
\begin{equation*}
    n! ::= 1 \cdot 2 \cdot 3 \cdots n.
\end{equation*}

We conclude the chapter with a discussion of asymptotic notation.
Asymptotic notation is often used to bound the error terms when there
is no exact closed form expression for a sum or product.  It also
provides a convenient way to express the growth rate or order of
magnitude of a sum or product.


\section{The Value of an Annuity}\label{annuity_sec}

Would you prefer a million dollars today or \$50,000 a year for the
rest of your life?  On the one hand, instant gratification is nice.
On the other hand, the \emph{total dollars} received at \$50K per year
is much larger if you live long enough.

Formally, this is a question about the value of an annuity.  An
\term{annuity} is a financial instrument that pays out a fixed amount of
money at the beginning of every year for some specified number of years.
In particular, an $n$-year, $m$-payment annuity pays $m$ dollars at the
start of each year for $n$ years.  In some cases, $n$ is finite, but not
always.  Examples include lottery payouts, student loans, and home
mortgages.  There are even Wall Street people who specialize in trading
annuities.

A key question is, ``What is an annuity worth?''  For example,
lotteries often pay out jackpots over many years.  Intuitively,
\$50,000 a year for 20 years ought to be worth less than a million
dollars right now.  If you had all the cash right away, you could
invest it and begin collecting interest.  But what if the choice were
between \$50,000 a year for 20 years and a \emph{half} million
dollars today?  Now it is not clear which option is better.

\subsection{The Future Value of Money}

In order to answer such questions, we need to know what a dollar paid out
in the future is worth today.  To model this, let's assume that money can
be invested at a fixed annual interest rate $p$.  We'll assume an 8\%
rate\footnote{U.S. interest rates have dropped steadily for several years,
  and ordinary bank deposits now earn around 1.5\%.  But just a few years
  ago the rate was 8\%; this rate makes some of our examples a little more
  dramatic.  The rate has been as high as 17\% in the past thirty
  years.}  for
the rest of the discussion.

Here is why the interest rate $p$ matters.  Ten dollars invested today
at interest rate $p$ will become $(1+p)\cdot 10 = 10.80$ dollars in a
year, $(1+p)^2\cdot 10 \approx 11.66$ dollars in two years, and so
forth.  Looked at another way, ten dollars paid out a year from now is
only really worth $1/(1+p) \cdot 10 \approx 9.26$ dollars today.  The
reason is that if we had the \$9.26 today, we could invest it and
would have \$10.00 in a year anyway.  Therefore, $p$ determines the
value of money paid out in the future.

So for an $n$-year, $m$-payment annuity, the first payment of $m$ dollars
is truly worth $m$ dollars.  But the second payment a year later is worth
only $m/(1+p)$ dollars.  Similarly, the third payment is worth
$m/(1+p)^2$, and the $n$-th payment is worth only $m/(1+p)^{n-1}$.  The
total value, $V$, of the annuity is equal to the sum of the payment
values.  This gives:
\begin{align}
  V & = \sum_{i=1}^n \frac{m}{(1+p)^{i-1}}\notag\\
  & = m \cdot \sum_{j=0}^{n-1} \paren{\frac{1}{1+p}}^j
          && \text{(substitute $j = i-1$)}\notag\\
  & = m \cdot \sum_{j=0}^{n-1} x^j
          && \text{(substitute $x = 1/(1+p)$)}.\label{jn-1xsum}
\end{align}

The goal of the preceding substitutions was to get the summation into
a simple special form so that we can solve it with a general formula.
In particular, the terms of the sum
\begin{equation*}
    \sum_{j = 0}^{n - 1} x^j = 1 + x + x^2 + x^3 + \cdots + x^{n - 1}
\end{equation*}
form a \term{geometric series}, which means that the ratio of
consecutive terms is always the same and it is a positive value less
than one.  In this case, the ratio is always $0 < x < 1$ since we
assumed that~$p > 0$.  It turns out that there is a nice closed-form
expression for any geometric series, namely
\begin{equation}\label{geometric-sum-n-1}
    \sum_{i = 0}^{n - 1} x^i = \frac{1 - x^n}{1 - x}.
\end{equation}

Equation~\ref{geometric-sum-n-1} can be verified by induction, but, as
is often the case, the proof by induction gives no hint about how the
formula was found in the first place.  So we'll take this opportunity
to \dmj{``to describe a way you could figure it out''?}explain a way
that you could use to figure it out for yourself called the
\term{Perturbation Method}.

\subsection{The Perturbation Method}\label{sec:perturbation}

Given a sum that has a nice structure, it is often useful to
``perturb'' the sum so that we can somehow combine the sum with the
perturbation to get something much simpler.  For example, suppose
\begin{equation*}
    S = 1 + x + x^2 + \dots + x^{n - 1}.
\end{equation*}
An example of a perturbation would be
\begin{equation*}
    xS = x + x^2 + \dots + x^n.
\end{equation*}

The difference between $S$ and~$xS$ is not so great, and so if we were
to subtract~$xS$ from~$S$, there would be massive cancellation:
\begin{equation*}
\begin{series}
      S & = & 1 & + & x & + & x^2 & + & x^3 & + &\cdots & + & x^{n-1} \cr
    -xS & = &   & - & x & - & x^2 & - & x^3 & - &\cdots & - & x^{n-1} & - x^{n}.\cr
\end{series}
\end{equation*}
The result of the subtraction is
\begin{equation*}
    S-xS = 1 - x^n.
\end{equation*}
Solving for~$S$ gives the desired closed-form expression in
Equation~\ref{geometric-sum-n-1}:
\begin{equation*}
    S = \frac{1 - x^n}{1 - x}.
\end{equation*}
We'll see more examples of this method when we introduce
\emph{generating functions} in Chapter~\ref{generating_function_chap}.

\subsection{A Closed Form for the Annuity Value}

Using Equation~\ref{geometric-sum-n-1}, we can derive a simple formula
for~$V$, the \idx{value of an annuity} that pays $m$ dollars at the
start of each year for $n$ years.
\begin{align}
  V & = m \left( \frac{1 - x^n}{1-x} \right)
      && \text{(by Equations \ref{jn-1xsum} and \ref{geometric-sum-n-1})}\label{Vmfrac1x}\\
  & = m \left( \frac{1 + p - \paren{1/(1+p)}^{n-1}}{p} \right)
      && \text{(substituting $x = 1/(1+p)$)}.\label{annval1p}
\end{align}
Equation~\ref{annval1p} is much easier to use than a summation with
dozens of terms.  For example, what is the real value of a winning
lottery ticket that pays \$50,000 per year for 20~years?  Plugging in
$m = \text{\$50,000}$, $n = 20$, and $p = 0.08$ gives $V \approx
\text{\$530,180}$.  So because payments are deferred, the million
dollar lottery is really only worth about a half million dollars!
This is a good trick for the lottery advertisers.

\subsection{Infinite Geometric Series}

The question we began with was whether you would prefer a million
dollars today or \$50,000 a year for the rest of your life.  Of
course, this depends on how long you live, so optimistically assume
that the second option is to receive \$50,000 a year \emph{forever}.
This sounds like infinite money!  But we can compute the value of an
annuity with an infinite number of payments by taking the limit of our
geometric sum in Equation~\ref{geometric-sum-n-1} as $n$ tends to
infinity.
\begin{theorem}\label{th:series}
If $\abs{x} < 1$, then
\[
\sum_{i=0}^\infty x^i = \frac{1}{1-x}.
\]
\end{theorem}

\begin{proof}
\begin{align*}
\sum_{i=0}^\infty x^i
   & \eqdef  \lim_{n \rightarrow \infty} \sum_{i=0}^{n-1} x^i \\
   & = \lim_{n \rightarrow \infty} \frac{1 - x^n}{1-x}
        & & \text{(by Equation~\ref{geometric-sum-n-1})}\\
   & = \frac{1}{1-x}.
\end{align*}
The final line follows from that fact that $\lim_{n \rightarrow \infty}
x^n =0$ when $\abs{x} < 1$.
\end{proof}

In our annuity problem, $x=1/(1+p) < 1$, so Theorem~\ref{th:series}
applies, and we get
\begin{align*}
V & = m \cdot \sum_{j=0}^{\infty} x^j & \text{(by Equation~\ref{jn-1xsum})}\\
  &= m\cdot \frac{1}{1-x} & \text{(by Theorem~\ref{th:series})}\\
  &= m\cdot \frac{1 + p}{p} &(x = 1/(1+p)).
\end{align*}
Plugging in $m = \text{\$50,000}$ and $p = 0.08$, we see that the
value~$V$ is only \$675,000.  Amazingly, a million dollars today is
worth much more than \$50,000 paid every year forever!  Then again, if
we had a million dollars today in the bank earning 8\% interest, we
could take out and spend \$80,000 a year forever.  So on second
thought, this answer really isn't so amazing.

\subsection{Examples}

Equation~\ref{geometric-sum-n-1} and Theorem~\ref{th:series} are
incredibly useful in computer science.  In fact, we already used
Equation~\ref{geometric-sum-n-1} implicitly when we claimed in
Chapter~\ref{chap:digraphs} than an $N$-input complete binary tree
has
\begin{equation*}
    1 + 2 + 4 + \dots + N = 2 N - 1
\end{equation*}
nodes.  Here are some other common sums that can be put into closed
form using Equation~\ref{geometric-sum-n-1} and
Theorem~\ref{th:series}:
\begingroup
\openup3pt
\begin{alignat}{3}
1 + 1/2 + 1/4 + \cdots
    & = \sum_{i=0}^\infty \paren{\frac{1}{2}}^i
    &%\;\;
    & = \cfrac{1}{1-(1/2)} = 2 \label{is2}
\\
0.99999\dots
    & = 0.9 \sum_{i=0}^{\infty} \paren{\frac{1}{10}}^i
    &
    & = 0.9 \paren{\cfrac{1}{1-1/10}}
      = 0.9 \paren{\cfrac{10}{9}}
      = 1 \label{is1}
\\
1 - 1/2 + 1/4 - \cdots
    & = \sum_{i=0}^\infty \paren{\frac{-1}{2}}^i
    &
    & = \cfrac{1}{1-(-1/2)}
      = \frac{2}{3} \label{is23}
\\
1 + 2 + 4 + \cdots + 2^{n-1}
    & = \sum_{i=0}^{n-1} 2^i
    &
    & = \cfrac{1 - 2^n}{1 - 2}
      = 2^n - 1 \label{is2n1}
\\
1 + 3 + 9 + \cdots + 3^{n-1}
    & = \sum_{i=0}^{n-1} 3^i
    &
    & = \cfrac{1 - 3^n}{1 - 3}
      = \cfrac{3^n - 1}{2} \label{is3n1}
\end{alignat}
\endgroup

\dmj{Tom: The spacing around the =s above looks fine to me. The entire
  display is a cramped because of its width and unbalanced because of
  the varying widths of the second column.  It would help a little if
  we could either remove 9.7 completely or at least move it to a
  separate display, maybe with a note about why we're bothering to
  mention it.}

If the terms in a geometric sum grow smaller, as in
Equation~\ref{is2}, then the sum is said to be \emph{geometrically
  decreasing}.  If the terms in a geometric sum grow progressively
larger, as in Equations \ref{is2n1} and~\ref{is3n1}, then the sum is
said to be \emph{geometrically increasing}.  In either case, the sum
is usually approximately equal to the term in the sum with the
greatest absolute value.  For example, in Equations~(\ref{is2})
and~(\ref{is23}), the largest term is equal to 1 and the sums are 2
and 2/3, both relatively close to~1.  In Equation~(\ref{is2n1}), the
sum is about twice the largest term.  In Equation~(\ref{is3n1}), the
largest term is $3^{n-1}$ and the sum is $(3^n-1)/2$, which is only
about a factor of $1.5$ greater.  You can see why this rule of thumb
works by looking carefully at Equation~\ref{geometric-sum-n-1} and
Theorem~\ref{th:series}.

\subsection{Variations of Geometric Sums}

We now know all about geometric sums---if you have one, life is easy.
But in practice one often encounters sums that cannot be transformed
by simple variable substitutions to the form $\sum x^i$.

A non-obvious, but useful way to obtain new summation formulas from
old is by differentiating or integrating with respect to $x$.  As an
example, consider the following sum:
\[
\sum_{i=1}^{n-1} i x^i = x + 2 x^2 + 3 x^3 + \cdots + (n - 1) x^{n - 1}
\]
This is not a geometric sum, since the ratio between successive terms
is not fixed, and so our formula for the sum of a geometric sum cannot
be directly applied.  But suppose that we differentiate
Equation~\ref{geometric-sum-n-1}:
\begin{equation}\label{eqn:9A}
\frac{d}{dx} \paren{ \sum_{i = 0}^{n - 1} x^i }
    = \frac{d}{dx} \paren{ \frac{1 - x^n}{1 - x} }.
\end{equation}
The left-hand side of Equation~\ref{eqn:9A} is simply
\begin{equation*}
\sum_{i = 0}^{n - 1} \frac{d}{dx} (x^i)
    = \sum_{i = 0}^{n - 1} i x^{i - 1}.
\end{equation*}
The right-hand side of Equation~\ref{eqn:9A} is
\begin{align*}
\frac{ -n x^{n - 1} (1 - x) - (-1) (1 - x^n) }{ (1 - x)^2 }
    &= \frac{ -n x^{n - 1} + n x^n + 1 - x^n }{ (1 - x)^2 } \\
    &= \frac{1 - n x^{n - 1} + (n - 1) x^n}{ (1 - x)^2 }.
\end{align*}
Hence, Equation~\ref{eqn:9A} means that
\begin{equation*}
\sum_{i = 0}^{n - 1} i x^{i - 1}
    = \frac{1 - n x^{n - 1} + (n - 1) x^n}{ (1 - x)^2 }.
\end{equation*}

Often, differentiating or integrating messes up the exponent of $x$ in
every term.  In this case, we now have a formula for a sum of the form
$\sum i x^{i-1}$, but we want a formula for the series $\sum i x^i$.
The solution is simple: multiply by $x$.  This gives:
\begin{equation}\label{eqn:G1}
    \sum_{i=1}^{n - 1} i x^i = \frac{ x - n x^n + (n - 1) x^{n+1}}{(1 - x)^2}
\end{equation}
and we have the desired closed-form expression for our
sum\footnote{Since we could easily have made a mistake in the
  calculation, it is always a good idea to go back and validate a
  formula obtained this way with a proof by induction.}.  It's a
little complicated looking, but it's easier to work with than the sum.

Notice that if $\abs{x} < 1$, then this series converges to a finite
value even if there are infinitely many terms.  Taking the limit of
Equation~\ref{eqn:G1} as $n$ tends infinity gives the following
theorem:
\begin{theorem}\label{th:inf_ixi}
If $\abs{x} < 1$, then
\[
    \sum_{i=1}^\infty i x^i = \frac{x}{(1-x)^2}.
\]
\end{theorem}

As a consequence, suppose that there is an annuity that pays
$im$~dollars at the \emph{end} of each year~$i$ forever.  For example,
if $m = \text{\$50,000}$, then the payouts are \$50,000 and then
\$100,000 and then \$150,000 and so on.  It is hard to believe
that the value of this annuity is finite!  But we can use
Theorem~\ref{th:inf_ixi} to compute the value:
\begin{align*}
V & = \sum_{i=1}^\infty \frac{im}{(1+p)^i} \\
  & = m \cdot \frac{1/(1+p)}{(1 - \frac{1}{1+p})^2} \\
  & = m \cdot \frac{1+p}{p^2}.
\end{align*}
The second line follows by an application of Theorem~\ref{th:inf_ixi}.
The third line is obtained by multiplying the numerator and
denominator by $(1+p)^2$.

For example, if $m = \text{\$50,000}$, and $p = 0.08$ as usual, then
the value of the annuity is $V = \text{\$8,437,500}$.  Even though the
payments increase every year, the increase is only additive with time;
by contrast, dollars paid out in the future decrease in value
exponentially with time.  The geometric decrease swamps out the
additive increase.  Payments in the distant future are almost
worthless, so the value of the annuity is finite.

The important thing to remember is the trick of taking the derivative
(or integral) of a summation formula.  Of course, this technique
requires one to compute nasty derivatives correctly, but this is at
least theoretically possible!

\section{Power Sums}

In Chapter~\ref{induction_chap}, we verified the formula
\begin{equation}\label{eqn:G26}
    \sum_{i = 1}^n i = \frac{n (n + 1)}{2}.
\end{equation}
But the source of this formula is still a mystery.  Sure, we can prove
it is true using well ordering or induction, but where did the
expression on the right come from in the first place?  Even more
inexplicable is the closed form expression for the sum of consecutive
squares:
\begin{equation}\label{eqn:G27}
    \sum_{i = 1}^n i^2 = \frac{(2n+1) (n+1) n}{6}.
\end{equation}

It turns out that there is a way to derive these expressions, but
before we explain it, we thought it would be fun\footnote{Remember
  that we are mathematicians, so our definition of ``fun'' may be
  different than yours.} to show you how Gauss proved
Equation~\ref{eqn:G26} when he was a young boy.\footnote{We suspect
  that Gauss was probably not an ordinary boy.}

Gauss's idea is related to the perturbation method we used in
Section~\ref{sec:perturbation}.  Let
\begin{equation*}
    S = \sum_{i = 1}^n i.
\end{equation*}
Then we can write the sum in two orders:
\begin{equation*}
    \begin{series}
        S & = & 1 & + & 2       & + & \dots & + & (n - 1) & + & n,\cr
        S & = & n & + & (n - 1) & + & \dots & + & 2       & + & 1.
    \end{series}
\end{equation*}
Adding these two equations gives
\begin{align*}
    2S  & = (n + 1) + (n + 1) + \dots + (n + 1) + (n + 1) \\
        & = n (n + 1).
\end{align*}
Hence,
\begin{equation*}
    S = \frac{n (n + 1)}{2}.
\end{equation*}
Not bad for a young child.  Looks like Gauss had some potential.\dots

Unfortunately, the same trick does not work for summing consecutive
squares.  However, we can observe that the result might be a
third-degree polynomial in~$n$, since the sum contains $n$~terms that
average out to a value that grows quadratically in~$n$.  So we might
guess that
\begin{equation*}
    \sum_{i=1}^n i^2 = an^3 + bn^2 + cn + d.
\end{equation*}
If the guess is correct, then we can determine the parameters $a$,
$b$, $c$, and $d$ by plugging in a few values for $n$.  Each such
value gives a linear equation in $a$, $b$, $c$, and $d$.  If we plug
in enough values, we may get a linear system with a unique solution.
Applying this method to our example gives:
\begin{align*}
n = 0 & \qimplies  0 = d \\
n = 1 & \qimplies  1 = a + b + c + d \\
n = 2 & \qimplies  5 = 8a + 4b + 2c + d \\
n = 3 & \qimplies  14 = 27a + 9b + 3c + d.
\end{align*}
Solving this system gives the solution $a = 1/3$, $b = 1/2$, $c =
1/6$, $d = 0$.  Therefore, \emph{if} our initial guess at the form of
the solution was correct, then the summation is equal to $n^3/3 +
n^2/2 + n/6$, which matches Equation~\ref{eqn:G27}.

The point is that if the desired formula turns out to be a polynomial,
then once you get an estimate of the \emph{degree} of the polynomial,
all the coefficients of the polynomial can be found automatically.

\textbf{Be careful!}  This method let's you discover formulas, but it
doesn't guarantee they are right!  After obtaining a formula by this
method, it's important to go back and \emph{prove} it using induction
or some other method, because if the initial guess at the solution was
not of the right form, then the resulting formula will be completely
wrong!\footnote{Alternatively, you can use the method based on
  generating functions described in
  Chapter~\ref{generating_function_chap}, which does not require any
  guessing at all.}

\section{Approximating Sums}

Unfortunately, it is not always possible to find a closed-form
expression for a sum.  For example, consider the sum
\begin{equation*}
    S = \sum_{i = 1}^n \sqrt{i}.
\end{equation*}
No closed form expression is known for~$S$.

In such cases, we need to resort to approximations for~$S$ if we want
to have a closed form.  The good news is that there is a general
method to find closed-form upper and lower bounds that work for most
any sum.  Even better, the method is simple and easy to remember.  It
works by replacing the sum by an integral and then adding either the
first or last term in the sum.

\begin{theorem}\label{thm:9G3}
Let $f: \reals^+ \to \reals^+$ be a nondecreasing\footnote{A
  function~$f$ is \term{nondecreasing} if $f(x) \ge f(y)$ whenever $x
  \ge y$.  It is \term{nonincreasing} if $f(x) \le f(y)$ whenever $x
  \ge y$.} continuous function and let
\begin{equation*}
    S = \sum_{i = 1}^n f(i)
\end{equation*}
and
\begin{equation*}
    I = \int_1^n f(x)\, dx.
\end{equation*}
Then
\begin{equation*}
    I + f(1) \le S \le I + f(n).
\end{equation*}
Similarly, if $f$ is nonincreasing, then
\begin{equation*}
    I + f(n) \le S \le I + f(1).
\end{equation*}
\end{theorem}

\begin{proof}
Let $f: \reals^+ \to \reals^+$  be a nondecreasing function.  For
example, $f(x) = \sqrt{x}$ is such a function.

Consider the graph shown in Figure~\ref{fig:9G4}.  The value of
\begin{equation*}
    S = \sum_{i = 1}^n f(i)
\end{equation*}
is represented by the shaded area in this figure.  This is because the
$i$th rectangle in the figure (counting from left to right) has
width~1 and height~$f(i)$.

\begin{figure}

\graphic{Fig_G4}

\caption{The area of the $i$th rectangle is~$f(i)$.  The shaded region
has area $\sum_{i = 1}^n f(i)$.}

\label{fig:9G4}

\end{figure}

The value of
\begin{equation*}
    I = \int_1^n f(x) \, dx
\end{equation*}
is the shaded area under the curve of~$f(x)$ from 1 to~$n$ shown in
Figure~\ref{fig:9G5}.

\begin{figure}

\graphic{Fig_G5}

\caption{The shaded area under the curve of~$f(x)$ from 1 to~$n$
  (shown in bold) is $I = \int_1^n f(x)\, dx$.}

\label{fig:9G5}

\end{figure}

Comparing the shaded regions in Figures \ref{fig:9G4}
and~\ref{fig:9G5}, we see that $S$ is at least $I$~plus the area of
the leftmost rectangle.  Hence,
\begin{equation}\label{eqn:9G7}
    S \ge I + f(1)
\end{equation}
This is the lower bound for~$S$.  We next derive the upper bound.

Figure~\ref{fig:9G6} shows the curve of~$f(x)$ from 1 to~$n$ shifted
left by~1.  This is the same as the curve~$f(x + 1)$ from 0 to~$n - 1$
and it has the same area~$I$.

\begin{figure}

\graphic{Fig_G6}

\caption{The shaded area under the curve of~$f(x + 1)$ from 0 to~$n -
  1$ is~$I$, the same as the area under the curve of~$f(x)$ from 1
  to~$n$.  This curve is the same as the curve in Figure~\ref{fig:9G5}
  except that has been shifted left by~1.}

\label{fig:9G6}

\end{figure}

Comparing the shaded regions in Figures \ref{fig:9G4}
and~\ref{fig:9G6}, we see that $S$~is at most $I$~plus the area of the
rightmost rectangle.  Hence,
\begin{equation}\label{eqn:9G8}
    S \le I + f(n).
\end{equation}
Combining Equations \ref{eqn:9G7} and~\ref{eqn:9G8}, we find that
\begin{equation*}
    I + f(1) \le S \le I + f(n),
\end{equation*}
for any nondecreasing function~$f$, as claimed

The argument for the case when $f$is nonincreasing is very similar.
The analogous graphs to those shown in
Figures~\ref{fig:9G4}--\ref{fig:9G6} are provided in
Figure~\ref{fig:9G9}.  As you can see by comparing the shaded regions
in Figures \ref{fig:9G9}(a) and~\ref{fig:9G9}(b),
\begin{equation*}
    S \le I + f(1).
\end{equation*}
Similarly, comparing the shaded regions in Figures \ref{fig:9G9}(a)
and~\ref{fig:9G9}(c) reveals that
\begin{equation*}
    S \ge I + f(n).
\end{equation*}
Hence, if $f$~is nonincreasing,
\begin{equation*}
    I + f(n) \le S \le I + f(1).
\end{equation*}
as claimed.
\end{proof}

\begin{figure}

\subfloat[]{\graphic{Fig_G9-a}}

\subfloat[]{\graphic{Fig_G9-b}}

\subfloat[]{\graphic{Fig_G9-c}}

\caption{The area of the shaded region in~(a) is $S = \sum_{i = 1}^n
  f(i)$.  The area in the shaded regions in (b) and~(c) is $I =
  \int_1^n f(x)\,dx$.}

\label{fig:9G9}

\end{figure}

Theorem~\ref{thm:9G3} provides good bounds for most sums.  At worst,
the bounds will be off by the largest term in the sum.  For example,
we can use Theorem~\ref{thm:9G3} to bound the sum
\begin{equation*}
    S = \sum_{i = 1}^n \sqrt{i}
\end{equation*}
as follows.

We begin by computing
\begin{align*}
    I   &= \int_1^n \sqrt{x} \, dx \\
        &= \left. \frac{x^{3/2}}{3/2} \; \right|_1^n \\
        &= \frac{2}{3} (n^{3/2} - 1).
\end{align*}
We then apply Theorem~\ref{thm:9G3} to conclude that
\begin{equation*}
    \frac{2}{3} (n^{3/2} - 1) + 1
    \; \le \; S
    \; \le \; \frac{2}{3} (n^{3/2} - 1) + \sqrt{n}
\end{equation*}
and thus that
\begin{equation*}
    \frac{2}{3} n^{3/2} + \frac{1}{3}
    \; \le \; S
    \; \le \; \frac{2}{3} n^{3/2} + \sqrt{n} - \frac{2}{3}.
\end{equation*}
In other words, the sum is very close to~$\frac{2}{3} n^{3/2}$.

We'll be using Theorem~\ref{thm:9G3} extensively going forward.  At
the end of this chapter, we will also introduce some notation that
expresses phrases like ``the sum is very close to'' in a more precise
mathematical manner.  But first, we'll see how Theorem~\ref{thm:9G3}
can be used to resolve a classic paradox in structural engineering.

\begin{problems}

\classproblems
\pinput{CP_neat_trick_for_geometric_sum}

\homeworkproblems
\pinput{PS_MIT_Harvard_degree_value}
\pinput{PS_credit_union}

\end{problems}


\section{Hanging Out Over the Edge}\label{book_stacking_sec}

Suppose that you have $n$ identical blocks\footnote{We will assume
  that the blocks are rectangular, uniformly weighted and of
  length~1.}  and that you stack them one on top of the next on a
table as shown in Figure~\ref{fig:9G11}.  Is there some value of~$n$
for which it is possible to arrange the stack so that one of the
blocks hangs out completely over the edge of the table without having
the stack fall over?  (You are not allowed to use glue or otherwise
hold the stack in position.)

\begin{figure}

\graphic{Fig_G11}

\caption{A stack of 5 identical blocks on a table.  The top block is
  hanging out over the edge of the table, but if you try stacking the
  blocks this way, the stack will fall over.}

\label{fig:9G11}

\end{figure}

Most people's first response to this question---sometimes also their
second and third responses---is ``No. No block will ever get
completely past the edge of the table.''  But in fact, if $n$~is large
enough, you can get the top block to stick out as far as you want: one
block-length, two block-lengths, any number of block-lengths!

\subsection{Stability}\label{sec:stability}

A stack of blocks is said to be \emph{stable} if it will not fall over
of its own accord.  For example, the stack illustrated in
Figure~\ref{fig:9G11} is not stable because the top block is sure to
fall over.  This is because the center or mass of the top block is
hanging out over air.

In general, a stack of $n$~blocks will be stable if and only if the
center of mass of the top $i$~blocks sits over the $(i + 1)$st block
for $i = 1$, 2, \dots, $n - 1$, and over the table for~$i = n$.

We define the \term{overhang} of a stable stack to be the distance
between the edge of the table and the rightmost end of the rightmost
block in the stack.  Our goal is thus to maximize the overhang of a
stable stack.

For example, the maximum possible overhang for a single block
is~$1/2$.  That is because the center of mass of a single block is in
the middle of the block (which is distance~$1/2$ from the right edge
of the block).  If we were to place the block so that its right edge
is more than~$1/2$ from the edge of the table, the center of mass
would be over air and the block would tip over.  But we can place the
block so the center of mass is at the edge of the table, thereby
achieving overhang~$1/2$.  This position is illustrated in
Figure~\ref{fig:one-stable-block}.

\begin{figure}

\graphic{Bookstack-1}

\caption{One block can overhang half a block length.}

\label{fig:one-stable-block}

\end{figure}

In general, the overhang of a stack of blocks is maximized by sliding
the entire stack rightward until its center of mass is at the edge of
the table.  The overhang will then be equal to the distance between
the center of mass of the stack and the rightmost edge of the
rightmost block.  We call this distance the \term{spread} of the
stack.  Note that the spread does not depend on the location of the
stack on the table---it is purely a property of the blocks in the
stack.  Of course, as we just observed, the maximum possible overhang
is equal to the maximum possible spread.  This relationship is
illustrated in Figure~\ref{fig:overhang}.

\begin{figure}

\graphic{Bookstack-2}

\caption{The overhang is maximized by maximizing the spread and then
  placing the stack so that the center of mass is at the edge of the
  table.}

\label{fig:overhang}

\end{figure}

\subsection{A Recursive Solution}

Our goal is to find a formula for the maximum possible spread~$S_n$
that is achievable with a stable stack of $n$~blocks.

We already know that $S_1 = 1/2$ since the right edge of a single
block with length~1 is always distance~$1/2$ from its center of mass.
Let's see if we can use a recursive approach to determine~$S_n$ for
all~$n$.  This means that we need to find a formula for~$S_n$ in terms
of~$S_i$ where~$i < n$.

Suppose we have a stable stack~$\mathcal{S}$ of $n$~blocks with
maximum possible spread~$S_n$.  There are two cases to consider
depending on where the rightmost block is in the stack.

\subparagraph{Case 1:}

\emph{The rightmost block in~$\mathcal{S}$ is the bottom block.}
Since the center of mass of the top $n - 1$~blocks must be over the
bottom block for stability, the spread is maximized by having the
center of mass of the top $n - 1$~blocks be directly over the
\emph{left} edge of the bottom block.  In this case the center of mass
of~$\mathcal{S}$ is\footnote{The center of mass of a stack of blocks
  is the average of the centers of mass of the individual blocks.}
\begin{equation*}
\frac{ (n - 1) \cdot 1 + (1) \cdot \frac{1}{2} }{ n }
    = 1 - \frac{1}{2n}
\end{equation*}
to the left of the right edge of the bottom block and so the spread
for~$\mathcal{S}$ is
\begin{equation}\label{eqn:9G15}
    1 - \frac{1}{2n}.
\end{equation}
For example, see Figure~\ref{fig:9G14}.

\begin{figure}

\graphic{Fig_G14}

\caption{The scenario where the bottom block is the rightmost block.
  In this case, the spread is maximized by having the center of mass
  of the top $n-1$~blocks be directly over the left edge of the bottom
block.}

\label{fig:9G14}

\end{figure}

In fact, the scenario just described is easily achieved by arranging
the blocks as shown in Figure~\ref{fig:9G15}, in which case we have
the spread given by Equation~\ref{eqn:9G15}.  For example, the spread
is $3/4$ for 2~blocks, $5/6$ for 3~blocks, $7/8$ for 4~blocks, etc.

\begin{figure}

\graphic{Fig_G15}

\caption{A method for achieving spread (and hence overhang)~$1 - 1/2n$
  with $n$~blocks, where the bottom block is the rightmost block.}

\label{fig:9G15}

\end{figure}

Can we do any better?  The best spread in Case~1 is always less
than~1, which means that we cannot get a block fully out over the edge
of the table in this scenario.  Maybe our intuition was right that we
can't do better.  Before we jump to any false conclusions, however,
let's see what happens in the other case.

\subparagraph{Case 2:}

\emph{The rightmost block in~$\mathcal{S}$ is among the top $n -
  1$~blocks.}  In this case, the spread is maximized by placing the
top $n - 1$~blocks so that their center of mass is directly over the
\emph{right} end of the bottom block.  This means that the center of
mass for~$\mathcal{S}$ is at location
\begin{equation*}
    \frac{(n - 1) \cdot C + 1 \cdot \paren{C - \frac{1}{2}}}{n}
    = C - \frac{1}{2n}
\end{equation*}
where $C$~is the location of the center of mass of the top $n -
1$~blocks.  In other words, the center of mass of~$\mathcal{S}$
is~$1/2n$ to the left of the center of mass of the top $n - 1$~blocks.
(The difference is due to the effect of the bottom block, whose center
of mass is $1/2$ unit to the left of~$C$.)  This means that the spread
of~$\mathcal{S}$ is $1/2n$ greater than the spread of the top $n -
1$~blocks (because we are in the case where the rightmost block is
among the top $n - 1$~blocks.)

Since the rightmost block is among the top $n - 1$ blocks, the spread
for~$\mathcal{S}$ is maximized by maximizing the spread for the top $n
- 1$~blocks.  Hence the maximum spread for~$\mathcal{S}$ in this case
is
\begin{equation}\label{eqn:9G16}
    S_{n - 1} + \frac{1}{2n}
\end{equation}
where $S_{n - 1}$~is the maximum possible spread for $n - 1$~blocks
(using any strategy).

We are now almost done.  There are only two cases to consider when
designing a stack with maximum spread and we have analyzed both of
them.  This means that we can combine Equation~\ref{eqn:9G15} from
Case~1 with Equation~\ref{eqn:9G16} from Case~2 to conclude that
\begin{equation}\label{eqn:9G17}
    S_n = \max \left\{ 1 - \frac{1}{2n}, \; S_{n - 1} + \frac{1}{2n} \right\}
\end{equation}
for any $n > 1$.

Uh-oh. This looks complicated.  Maybe we are not almost done after
all!

Equation~\ref{eqn:9G17} is an example of a \term{recurrence}.  We will
describe numerous techniques for solving recurrences in
Chapter~\ref{chap:recurrences}, but, fortunately,
Equation~\ref{eqn:9G17} is simple enough that we can solve it without
waiting for all the hardware in Chapter~\ref{chap:recurrences}.

One of the first things to do when you have a recurrence is to get a
feel for it by computing the first few terms.  This often gives clues
about a way to solve the recurrence, as it will in this case.

We already know that $S_1 = 1/2$.  What about~$S_2$?  From
Equation~\ref{eqn:9G17}, we find that
\begin{align*}
S_2
    &= \max \left\{ 1 - \frac{1}{4}, \; \frac{1}{2} + \frac{1}{4} \right \} \\
    &= 3/4.
\end{align*}
Both cases give the same spread, albeit by different approaches.  For
example, see Figure~\ref{fig:9G20}.

\begin{figure}

\subfloat[]{\graphic{Fig_G20-a}}
\qquad
\subfloat[]{\graphic{Fig_G20-b}}

\caption{Two ways to achieve spread (and hence overhang)~$3/4$ with $n
  = 2$ blocks.  The first way~(a) is from Case~1 and the second~(b) is
  from Case~2.}

\label{fig:9G20}

\end{figure}

That was easy enough.  What about~$S_3$?
\begin{align*}
S_3 &= \max\left\{ 1 - \frac{1}{6}, \; \frac{3}{4} + \frac{1}{6} \right\} \\
    &= \max\left\{ \frac{5}{6}, \; \frac{11}{12} \right\} \\
    &= \frac{11}{12}.
\end{align*}
As we can see, the method provided by Case~2 is the best.  Let's check
$n = 4$.
\begin{align}
S_4 &= \max\left\{ 1 - \frac{1}{8}, \; \frac{11}{12} + \frac{1}{8} \right\}
    \notag \\
    &= \frac{25}{24}. \label{eqn:9G21}
\end{align}

Wow!  This is a breakthrough---for two reasons.  First,
Equation~\ref{eqn:9G21} tells us that by using only 4~blocks, we can
make a stack so that one of the blocks is hanging out completely over
the edge of the table.  The two ways to do this are shown in
Figure~\ref{fig:9G22}.

\begin{figure}

\subfloat[]{\graphic{Fig_G22-a}}
\qquad
\subfloat[]{\graphic{Fig_G22-b}}

\caption{The two ways to achieve spread (and overhang)~$25/24$.  The
  method in~(a) uses Case~1 for the top 2~blocks and Case~2 for the
  others.  The method in~(b) uses Case~2 for every block that is added
to the stack.}

\label{fig:9G22}

\end{figure}

The second reason that Equation~\ref{eqn:9G21} is important is that we
now know that $S_4 > 1$, which means that we no longer have to worry
about Case~1 for $n > 4$ since Case~1 never achieves spread greater
than~1.  Moreover, even for~$n \le 4$, we have now seen that the
spread achieved by Case~1 never exceeds the spread achieved by Case~2,
and they can be equal only for $n = 1$ and $n = 2$.  This means that
\begin{equation}\label{eqn:9G23}
    S_n = S_{n - 1} + \frac{1}{2n}
\end{equation}
for all $n > 1$ since we have shown that the best spread can always be
achieved using Case~2.

The recurrence in Equation~\ref{eqn:9G23} is much easier to solve than
the one we started with in Equation~\ref{eqn:9G17}.  We can solve it
by expanding the equation as follows:
\begin{align*}
S_n &= S_{n - 1} + \frac{1}{2n} \\
    &= S_{n - 2} + \frac{1}{2(n - 1)} + \frac{1}{2n} \\
    &= S_{n - 3} + \frac{1}{2(n - 2)} + \frac{1}{2(n - 1)} + \frac{1}{2n}
\end{align*}
and so on.  This suggests that
\begin{equation}\label{eqn:9G24}
    S_n = \sum_{i = 1}^n \frac{1}{2i},
\end{equation}
which is, indeed, the case.

Equation~\ref{eqn:9G24} can be verified by induction.  The base case
when $n = 1$ is true since we know that $S_1 = 1/2$.  The inductive
step follows from Equation~\ref{eqn:9G23}.

So we now know the maximum possible spread and hence the maximum
possible overhang for any stable stack of books.  Are we done?  Not
quite.  Although we know that $S_4 > 1$, we still don't know how big
the sum~$\sum_{i = 1}^n \frac{1}{2i}$ can get.

It turns out that $S_n$~is very close to a famous sum known as the
$n$th Harmonic number~$H_n$.

\subsection{Harmonic Numbers}

\begin{definition}
The \emph{$n$th} \term{Harmonic number} is
\[
    H_n \eqdef \sum_{i=1}^n \frac{1}{i}.
\]
\end{definition}
So Equation~\ref{eqn:9G24} means that
\begin{equation}\label{eqn:9G25}
    S_n = \frac{H_n}{2}.
\end{equation}

The first few Harmonic numbers are easy to compute.  For example,
\begin{equation*}
    H_4 = 1 + \frac{1}{2} + \frac{1}{3} + \frac{1}{4} = \frac{25}{12}.
\end{equation*}
There is good news and bad news about Harmonic numbers.  The bad news
is that there is no closed-form expression known for the Harmonic
numbers.  The good news is that we can use Theorem~\ref{thm:9G3} to
get close upper and lower bounds on~$H_n$.  In particular, since
\begin{align*}
\int_1^n \frac{1}{x} \, dx
    &= \ln(x) \; \Bigr|_1^n \\
    &= \ln(n),
\end{align*}
Theorem~\ref{thm:9G3} means that
\begin{equation}\label{eqn:9G30}
    \ln(n) + \frac{1}{n} \le H_n \le \ln(n) + 1.
\end{equation}
In other words, the $n$th Harmonic number is very close to~$\ln(n)$.

Because the Harmonic numbers frequently arise in practice,
mathematicians have worked hard to get even better approximations for
them.  In fact, it is now known that
\begin{equation}\label{eqn:9K2}
    H_n = \ln(n) + \gamma + \frac{1}{2n} + \frac{1}{12n^2} +
        \frac{\epsilon(n)}{120n^4}
\end{equation}
Here $\gamma$ is a value $0.577215664\dots$ called \term{Euler's
  constant}, and $\epsilon(n)$ is between 0 and 1 for all $n$.  We
will not prove this formula.

We are now finally done with our analysis of the block stacking
problem.  Plugging the value of~$H_n$ into Equation~\ref{eqn:9G25}, we
find that the maximum overhang for $n$~blocks is very close
to~$\frac{1}{2} \ln(n)$.  Since $\ln(n)$ grows to infinity as
$n$~increases, this means that if we are given enough blocks (in
theory anyway), we can get a block to hang out arbitrarily far over
the edge of the table.  Of course, the number of blocks we need will
grow as an exponential function of the overhang, so it will probably
take you a long time to achieve an overhang of 2 or~3, never mind an
overhang of~100.

\subsection{Asymptotic Equality}\label{sec:asymptotic_equality}

For cases like Equation~\ref{eqn:9K2} where we understand the growth
of a function like~$H_n$ up to some (unimportant) error terms, we use
a special notation,~$\sim$, to denote the leading term of the
function.  For example, we say that $H_n \sim \ln(n)$ to indicate that
the leading term of $H_n$ is $\ln(n)$.  More precisely:
\begin{definition}\label{def:sim}
  For functions $f,g: \reals \to \reals$, we say $f$ is \index{$\sim$
    (asymptotic equality)}\term{asymptotically equal} to $g$, in symbols,
\[
f(x) \sim g(x)
\]
iff
\[
\lim_{x \rightarrow \infty} f(x)/g(x) = 1.
\]
\end{definition}

Although it is tempting to write $H_n \sim \ln(n) + \gamma$ to indicate
the two leading terms, this is not really right.  According to
Definition~\ref{def:sim}, $H_n \sim \ln(n) + c$ where $c$ is \emph{any
  constant}.  The correct way to indicate that $\gamma$ is the
second-largest term is $H_n - \ln(n) \sim \gamma$.

The reason that the $\sim$ notation is useful is that often we do not care
about lower order terms.  For example, if $n = 100$, then we can compute
$H(n)$ to great precision using only the two leading terms:
\[
\abs{H_n - \ln(n) - \gamma} \leq \abs{\frac{1}{200} - \frac{1}{120000} +
\frac{1}{120 \cdot 100^4}} < \frac{1}{200}.
\]

We will spend a lot more time talking about asymptotic notation at the
end of the chapter.  But for now, let's get back to sums.

\begin{problems}
\classproblems
\pinput{CP_holy_grail}
\pinput{CP_harmonic_number_divergence}

\homeworkproblems
\pinput{PS_bug_on_rug_harmonic_number}

\end{problems}


\section{Double Trouble}

Sometimes we have to evaluate sums of sums, otherwise known as
\term{double summations}.  This sounds hairy, and sometimes it is.
But usually, it is straightforward---you just evaluate the inner sum,
replace it with a closed form, and then evaluate the outer sum (which
no longer has a summation inside it).  For example,\footnote{Ok, so
  maybe this one is a little hairy, but it is also fairly
  straightforward.  Wait till you see the next one!}
\begingroup
\openup3pt
\begin{align*}
\sum_{n=0}^{\infty} \paren{y^n \sum_{i=0}^n x^i}
 & = \sum_{n=0}^{\infty} \paren{y^n \frac{1-x^{n+1}}{1-x}}
     & \text{Equation~\ref{geometric-sum-n-1}}\\
%
 & = \paren{\frac{1}{1-x}} \sum_{n=0}^{\infty} y^n
     - \paren{\frac{1}{1-x}} \sum_{n=0}^{\infty} y^nx^{n+1} \\
%
 & = \frac{1}{(1-x)(1-y)}
    - \paren{\frac{x}{1 - x}} \sum_{n=0}^{\infty} \paren{xy}^n
      & \text{Theorem~\ref{th:series}}\\
%
 & = \frac{1}{(1-x)(1-y)} - \frac{x}{(1-x)(1-xy)}
      & \text{Theorem~\ref{th:series}}\\
%
  & = \frac{\paren{1-xy} - x(1-y)}{(1-x)(1-y)(1-xy)}\\
%
  & = \frac{1-x}{(1-x)(1-y)(1-xy)}\\
%
  & = \frac{1}{(1-y)(1-xy)}.
\end{align*}
\endgroup

When there's no obvious closed form for the inner sum, a special trick
that is often useful is to try \emph{exchanging the order of
  summation.}  For example, suppose we want to compute the sum of the
first $n$~Harmonic numbers
\begin{equation}\label{eqn:9B}
    \sum_{k=1}^n H_k = \sum_{k=1}^n \sum_{j=1}^k \frac{1}{j}
\end{equation}
For intuition about this sum, we can apply Theorem~\ref{thm:9G3} to
Equation~\ref{eqn:9G30} to conclude that the sum is close to
\begin{align*}
\int_{1}^n \ln(x) \, dx
    &=  x \ln(x) - x \; \Bigr|_1^n \\
    &= n \ln(n) - n + 1.
\end{align*}

Now let's look for an exact answer.  If we think about the pairs
$(k,j)$ over which we are summing, they form a triangle:
\[
\begin{array}{cc|ccccccc}
 &  & j &   &   &   &   &       &   \\
 &  & 1 & 2 & 3 & 4 & 5 & \dots & n \\
\hline
k & 1 & 1\\
  & 2 &1&1/2\\
  & 3 &1&1/2&1/3\\
  & 4 &1&1/2&1/3&1/4\\
  &   &\dots\\
  & n &1&1/2&&\dots&&&1/n
\end{array}
\]
The summation in Equation~\ref{eqn:9B} is summing each row and then
adding the row sums.  Instead, we can sum the columns and then add the
column sums.  Inspecting the table we see that this double sum can be
written as
\begingroup
\openup3pt
\begin{align}
\sum_{k=1}^n H_k &= \sum_{k=1}^n \sum_{j=1}^k \frac{1}{j}\notag\\
&= \sum_{j=1}^n \sum_{k=j}^n \frac{1}{j}\notag\\
&= \sum_{j=1}^n \frac{1}{j} \sum_{k=j}^n 1\notag\\
&= \sum_{j=1}^n \frac1j (n-j+1)\notag\\
&= \sum_{j=1}^n \frac{n+1}j - \sum_{j=1}^n \frac{j}{j}\notag\\
&= (n+1)\sum_{j=1}^n \frac1j - \sum_{j=1}^n 1\notag\\
&= (n+1)H_n-n.\label{sHk}
\end{align}
\endgroup

\section{Products}

We've covered several techniques for finding closed forms for sums but
no methods for dealing with products.  Fortunately, we do not need to
develop an entirely new set of tools when we encounter a product such
as
\begin{equation}\label{eqn:9P1}
    n! \eqdef \prod_{i = 1}^n i.
\end{equation}
That's because we can convert any product into a sum by taking a
logarithm.  For example, if
\begin{equation*}
    P = \prod_{i  = 1}^n f(i),
\end{equation*}
then
\begin{equation*}
    \ln(P) = \sum_{i = 1}^n \ln(f(i)).
\end{equation*}
We can then apply our summing tools to find a closed form (or
approximate closed form) for~$\ln(P)$ and then exponentiate at the end
to undo the logarithm.

For example, let's see how this works for the factorial
function~$n!$  We start by taking the logarithm:
\begin{align*}
\ln (n!)
       & =  \ln(1 \cdot 2 \cdot 3 \cdots (n-1) \cdot n) \\
       & =  \ln(1) + \ln(2) + \ln(3) + \cdots + \ln(n-1) + \ln(n) \\
       & =  \sum_{i=1}^n \ln(i).
\end{align*}

Unfortunately, no closed form for this sum is known.  However, we can
apply Theorem~\ref{thm:9G3} to find good closed-form bounds on the
sum.  To do this, we first compute
\begin{align*}
\int_1^n \ln(x) \, dx
    &= x \ln(x) - x \Bigr|_1^n \\
    &= n \ln(n) - n + 1.
\end{align*}
Plugging into Theorem~\ref{thm:9G3}, this means that
\begin{equation*}
    n \ln(n) - n + 1
    \;\le\; \sum_{i = 1}^n \ln(i)
    \;\le\; n \ln(n) - n + 1 + \ln(n).
\end{equation*}
Exponentiating then gives
\begin{equation}\label{eqn:9Q1}
    \frac{n^n}{e^{n - 1}} \;\le\; n! \;\le\; \frac{n^{n + 1}}{e^{n - 1}}.
\end{equation}
This means that $n!$ is within a factor of~$n$ of~$n^n/e^{n - 1}$.

\subsection{Stirling's Formula}

$n!$ is probably the most commonly used product in discrete
mathematics, and so mathematicians have put in the effort to find much
better closed-form bounds on its value.  The most useful bounds are
given in Theorem~\ref{thm:stirling}.

\begin{theorem}[Stirling's Formula]\label{thm:stirling}
For all $n \ge 1$,
\begin{equation*}
    n! = \sqrt{2 \pi n} \paren{\frac{n}{e}}^n e^{\epsilon(n)}
\end{equation*}
where
\begin{equation*}
    \frac{1}{12 n + 1} \le \epsilon(n) \le \frac{1}{2n}.
\end{equation*}
\end{theorem}

Theorem~\ref{thm:stirling} can be proved by induction on~$n$, but the
details are a bit painful (even for us) and so we will not go through
them here.

There are several important things to notice about Stirling's
Formula.  First, $\epsilon(n)$ is always positive.  This means that
\begin{equation*}
    n! > \sqrt{2\pi n} \paren{\frac{n}{e}}^n
\end{equation*}
for all~$n \in \naturals^+$.

Second, $\epsilon(n)$~tends to zero as $n$~gets large.  This means
that\footnote{The $\sim$ notation was defined in
  Section~\ref{sec:asymptotic_equality}.}
\begin{equation}\label{eqn:9.28}
    n! \sim \sqrt{2\pi n} \paren{\frac{n}{e}}^n,
\end{equation}
which is rather surprising.  After all, who would expect both $\pi$
and~$e$ to show up in a closed-form expression that is asymptotically
equal to~$n!$?

Third, $\epsilon(n)$~is small even for small values of~$n$.  This
means that Stirling's Formula provides good approximations for~$n!$
for most all values of~$n$.  For example, if we use
\begin{equation*}
    \stirling{n}
\end{equation*}
as the approximation for~$n!$, as many people, do, we are guaranteed
to be within a factor of
\begin{equation*}
    e^{\epsilon(n)} \le e^{\frac{1}{12n}}
\end{equation*}
of the correct value.  For $n \ge 10$, this means we will be
within~1\% of the correct value.  For~$n \ge 100$, the error will be
less than~0.1\%.

If we needed an even closer approximation for~$n!$, then we could use
either
\begin{equation*}
    \stirling{n} e^{1/12n}
\end{equation*}
or
\begin{equation*}
    \stirling{n} e^{1/12n + 1}
\end{equation*}
depending on whether we wanted an upper bound or a lower bound,
respectively.  By Theorem~\ref{thm:stirling}, we know that both bounds
will be within a factor of
\begin{equation*}
    e^{ \frac{1}{12n} - \frac{1}{12n + 1} } = e^{\frac{1}{144n^2 + 12n }}
\end{equation*}
of the correct value.  For~$n \ge 10$, this means that we will be
within~0.01\% of the correct value.  For~$n \ge 100$, the error will
be less than~0.0001\%.

For quick future reference, these facts are summarized in
Figure~\ref{fig:9A1}.

\begin{figure}\redrawntrue

\renewcommand{\arraystretch}{1.5}

\begin{tabular}{l|lll}

\multicolumn{1}{l}{\emph{Approximation}}
    & $n \ge 10$ & $n \ge 100$ & $n \ge 1000$ \\
\cline{2-4}
$\stirling{n}$
    & ${}\le{}$1\%  & ${}\le{}$0.1\%    & ${}\le{}$0.01\%\\

$\stirling{n} e^{1/12n}$
    & ${}\le{}$0.01\%  & ${}\le{}$0.0001\%    & ${}\le{}$0.000001\%
\end{tabular}

\caption{Error bounds on common approximations for~$n!$ from
  Theorem~\ref{thm:stirling}.  For example, if~$n \ge 100$, then
  $\protect\stirling{n}$ approximates~$n!$ to within~0.1\%.}

\label{fig:9A1}

\end{figure}

\section{Asymptotic Notation}\label{asymptotic_sec}

Asymptotic notation is a shorthand used to give a quick measure of the
behavior of a function $f(n)$ as $n$ grows large.  For example, the
asymptotic notation~\idx{$\sim$} of Definition~\ref{def:sim} is a
binary relation indicating that two functions grow at the \emph{same}
rate.  There is also a binary relation indicating that one function
grows at a significantly \emph{slower} rate than another.


\subsection{\index{o(), little oh}Little Oh}

\begin{definition}
  For functions $f,g: \reals \to \reals$, with $g$ nonnegative, we say
  $f$ is \index{o(), asymptotically smaller}\term{asymptotically
    smaller} than~$g$, in symbols,
\[
f(x) = o(g(x)),
\]
iff
\[
\lim_{x \rightarrow \infty} f(x)/g(x) = 0.
\]
\end{definition}

For example, $1000x^{1.9} = o(x^2)$, because $1000x^{1.9}/x^2 =
1000/x^{0.1}$ and since $x^{0.1}$ goes to infinity with $x$ and 1000 is
constant, we have $\lim_{x \rightarrow \infty} 1000x^{1.9}/x^2 = 0$.
This argument generalizes directly to yield
\begin{lemma}\label{xaoxb}
$x^a = o(x^b)$ for all nonnegative constants $a<b$.
\end{lemma}

Using the familiar fact that  $\log x < x$ for all $x >1$, we can prove
\begin{lemma}\label{logxxe}
$\log x = o(x^{\epsilon})$ for all $\epsilon >0$.
\end{lemma}

\begin{proof}
Choose $\epsilon > \delta > 0$ and let $x = z^\delta$ in the inequality
$\log x < x$.  This implies
\begin{equation}\label{zdd}
\log z  <  z^{\delta}/\delta
 =  o(z^{\epsilon})\qquad \text{by Lemma~\ref{xaoxb}}.
\end{equation}
\end{proof}

\begin{corollary}\label{xbax}
$x^b = o(a^x)$ for any $a,b \in \reals$ with $a>1$.
\end{corollary}

Lemma~\ref{logxxe} and Corollary~\ref{xbax} can also be proved using
l'H\^opital's Rule or the McLaurin Series for $\log x$ and $e^x$.
Proofs can be found in most calculus texts.

\subsection{\index{O(), big oh}Big Oh}

Big Oh is the most frequently used asymptotic notation.  It is used to
give an upper bound on the growth of a function, such as the running
time of an algorithm.
\begin{definition}
Given nonnegative functions $f, g : \reals \to \reals$, we
say that
\[
f = O(g)
\]
iff
\[
\limsup_{x \rightarrow \infty} f(x)/g(x) < \infty.
\]
\end{definition}
This definition\footnote{We can't simply use the limit as
$x \rightarrow \infty$ in the definition of $O()$, because if
$f(x)/g(x)$ oscillates between, say, 3 and 5 as $x$ grows, then $f =
O(g)$ because $f
\leq 5g$, but $\lim_{x \rightarrow \infty} f(x)/g(x)$ does not exist.
So instead of limit, we use the technical notion of $\limsup$.  In
this oscillating case, $\limsup_{x \rightarrow \infty} f(x)/g(x) = 5$.

The precise definition of $\limsup$ is
\[
\limsup_{x \rightarrow \infty} h(x) \eqdef \lim_{x \rightarrow \infty}
\text{lub}_{y \geq x} h(y),
\]
where ``lub'' abbreviates ``least upper bound.''} makes it clear that
\begin{lemma}\label{osimO}
If $f = o(g)$ or $f \sim g$, then $f = O(g)$.
\end{lemma}
\begin{proof}
$\lim f/g=0$ or $\lim f/g=1$ implies $\lim f/g<\infty$.
\end{proof}

It is easy to see that the converse of Lemma~\ref{osimO} is not true.  For
example, $2x = O(x)$, but $2x \not\sim x$ and $2x \neq o(x)$.

The usual formulation of Big Oh spells out the definition of $\limsup$
without mentioning it.  Namely, here is an equivalent definition:
\begin{definition}\label{def:O}
Given functions $f, g : \reals \to \reals$, we say that
\[
f = O(g)
\]
iff there exists a constant $c \geq 0$ and an $x_0$ such that for all $x \geq
x_0$, $\abs{f(x)} \leq c g(x)$.
\end{definition}

This definition is rather complicated, but the idea is simple: $f(x) =
O(g(x))$ means $f(x)$ is less than or equal to $g(x)$, except that we're
willing to ignore a constant factor, namely, $c$, and to allow exceptions for
small $x$, namely, $x < x_0$.

We observe,
\begin{lemma}
If $f = o(g)$, then it is \emph{not} true that $g = O(f)$.
\end{lemma}
\begin{proof}
\[
\lim_{x \rightarrow \infty} \frac{g(x)}{f(x)} =
 \frac{1}{\lim_{x \rightarrow \infty} f(x)/g(x)} =
 \frac{1}{0} = \infty,
\]
so $g \neq O(f)$.

\iffalse
We will prove the equivalent contrapositive, i.e., that
if $g=O(f)$ then it is not true that $f=o(g)$.  If
$g=O(f)$ then there
exists a constant $c \geq 0$ and an $x_0$ such that for all $x \geq
x_0$, $\abs{g(x)} \leq c f(x)$.
Then for all $x \geq x_0$, we have that $f(x)/|g(x)| = f(x)/g(x) >c>0$
(the first equality uses the nonnegativity of $g$).
Thus $\lim_{x \rightarrow \infty} f(x)/g(x) >0$ and so it is not true that
$f=o(g)$.
\fi

\end{proof}

\begin{proposition}
$100x^2 = O(x^2)$.
\end{proposition}

\begin{proof}
Choose $c = 100$ and $x_0 = 1$.  Then the proposition holds, since for all
$x \geq 1$, $\abs{100x^2} \leq 100 x^2$.
\end{proof}

\begin{proposition}\label{x2O}
$x^2 + 100x + 10 = O(x^2)$.
\end{proposition}

\begin{proof}
$(x^2 + 100x + 10)/x^2 = 1 + 100/x + 10/x^2$ and so its limit as $x$
approaches infinity is $1 + 0 + 0 = 1$.  So in fact, $x^2 + 100x + 10 \sim
x^2$, and therefore $x^2 + 100x + 10 = O(x^2)$.  Indeed, it's conversely
true that $x^2= O(x^2 + 100x + 10)$.
\end{proof}

Proposition~\ref{x2O} generalizes to an arbitrary polynomial:
\begin{proposition}
    $a_k x^k + a_{k-1} x^{k-1} + \cdots + a_1x + a_0 = O(x^k)$.
\end{proposition}
We'll omit the routine proof.

Big Oh notation is especially useful when describing the running time
of an algorithm.  For example, the usual algorithm for multiplying $n
\times n$ matrices uses \dmj{How about ``a number of operations
  proportional to~$n^3$''?  The current wording is extremely
  awkward.}proportional to $n^3$ operations in the worst case.  This
fact can be expressed concisely by saying that the running time is
$O(n^3)$.  So this asymptotic notation allows the speed of the
algorithm to be discussed without reference to constant factors or
lower-order terms that might be machine specific.  It turns out that
there is another, ingenious \idx{matrix multiplication} procedure that
uses $O(n^{2.55})$ operations.  This procedure will therefore be much
more efficient on large enough matrices.  Unfortunately, the
$O(n^{2.55})$-operation multiplication procedure is almost never used
in practice because it happens to be less efficient than the usual
$O(n^3)$ procedure on matrices of practical size.\footnote{It is even
  conceivable that there is an $O(n^2)$ matrix multiplication
  procedure, but none is known.}

\subsection{\index{$\Omega$@big omega}Omega}

Suppose you want to make a statement of the form ``the running time of
the algorithm is a least\dots''.  Can you say it is ``at
least~$O(n^2)$''?  No!  This statement is meaningless since big-oh can
only be used for \emph{upper} bounds.  For lower bounds, we use a
different symbol, called ``big-Omega.''

\begin{definition}\label{def:Omega}
Given functions $f, g : \reals \to \reals$, we say that
\begin{equation*}
    f = \Omega(g)
\end{equation*}
iff there exists a constant $c > 0$ and an $x_0$ such that for all $x
\ge x_0$, we have $f(x) \ge c |g(x)|$.
\end{definition}

In other words, $f(x) = \Omega(g(x))$ means that $f(x)$~is greater
than or equal to~$g(x)$, except that we are willing to ignore a
constant factor and to allow exceptions for small~$x$.

If all this sounds a lot like big-Oh, only in reverse, that's because
big-Omega is the opposite of big-Oh.  More precisely,

\begin{theorem}\label{thm:9S2}
$f(x) = O(g(x))$ if and only if $g(x) = \Omega(f(x))$.
\end{theorem}

\begin{proof}
\begin{align*}
\lefteqn{f(x) = O(g(x))} \quad&\\
   & \qiff \exists c > 0, x_0. \; \forall x \ge x_0. \;
            |f(x)| \le c g(x) 
        && \text{(Definition~\ref{def:O})} \\
    & \qiff \exists c > 0, x_0. \; \forall x \ge x_0. \;
            g(x) \ge \frac{1}{c} |f(x)| \\
    & \qiff \exists c' > 0, x_0. \; \forall x \ge x_0. \;
            g(x) \ge c' |f(x)|
        && \text{(set $c' = 1/c$)} \\
    & \qiff g(x) = \Omega(f(x))
        && \text{(Definition~\ref{def:Omega})}
\qedhere
\end{align*}
\end{proof}

For example, $x^2 = \Omega(x)$, \ $2^x = \Omega(x^2)$, and $x/100 =
\Omega(100 x + \sqrt{x})$.

So if the running time of your algorithm on inputs of size~$n$
is~$T(n)$, and you want to say it is at least quadratic, say
\begin{equation*}
    T(n) = \Omega(n^2).
\end{equation*}

\subsubsection{\index{$\omega$@little omega}Little Omega}

There is also a symbol called little-omega, analogous to little-oh, to
denote that one function grows strictly faster than another function.

\begin{definition}\label{def:omega}
For functions $f, g: \reals \to \reals$ with $f$~nonnegative, we say
that
\begin{equation*}
    f(x) = \omega(g(x))
\end{equation*}
iff
\begin{equation*}
    \lim_{x \to \infty} \frac{g(x)}{f(x)} = 0.
\end{equation*}
In other words,
\begin{equation*}
    f(x) = \omega(g(x))
\end{equation*}
iff
\begin{equation*}
    g(x) = o(f(x)).
\end{equation*}
\end{definition}

For example, $x^{1.5} = \omega(x)$ and $\sqrt{x} = \omega(\ln^2(x))$.

The little-omega symbol is not as widely used as the other asymptotic
symbols we have been discussing.


\subsection{\index{$\Theta()$}Theta}

Sometimes we want to specify that a running time~$T(n)$ is precisely
quadratic up to constant factors (both upper bound \emph{and} lower
bound).  We could do this by saying that $T(n) = O(n^2)$ and $T(n) =
\Omega(n^2)$, but rather than say both, mathematicians have devised
yet another symbol, $\Theta$, to do the job.

\begin{definition}\label{def:Theta}
\[
    f = \Theta(g)
    \qiff
    f=O(g) \text{ and } g=O(f).
\]
\end{definition}

The statement $f = \Theta(g)$ can be paraphrased intuitively as
``$f$ and $g$ are equal to within a constant factor.''  Indeed, by
Theorem~\ref{thm:9S2}, we know that
\begin{equation*}
    f = \Theta(g) \qiff \text{$f = O(g)$ and $f = \Omega(g)$}.
\end{equation*}

The Theta notation allows us to highlight growth rates and allow
suppression of distracting factors and low-order terms.  For example,
if the running time of an algorithm is
\[
T(n) = 10n^3 - 20n^2 + 1,
\]
then we can more simply write
\[
T(n) = \Theta(n^3).
\]
In this case, we would say that \emph{$T$ is of order $n^3$} or that
\emph{$T(n)$ grows cubically}, which is probably what we really want
to know.  Another such example is
\[
{{\pi^23^{x-7} + \frac{(2.7x^{113} + x^9- 86)^4}{\sqrt{x}} - 1.08^{3x}}} =
\Theta(3^x).
\]

Just knowing that the running time of an algorithm is $\Theta(n^3)$,
for example, is useful, because if $n$ doubles we can predict that the
running time will \emph{by and large}\footnote{Since $\Theta(n^3)$
  only implies that the running time, $T(n)$, is between $cn^3$ and
  $dn^3$ for constants $0<c<d$, the time $T(2n)$ could regularly
  exceed $T(n)$ by a factor as large as $8d/c$.  The factor is sure to
  be close to 8 for all large $n$ only if $T(n) \sim n^3$.} increase
by a factor of at most $8$ for large $n$.  In this way, Theta notation
preserves information about the scalability of an algorithm or system.
Scalability is, of course, a big issue in the design of algorithms and
systems.


\subsection{Pitfalls with Asymptotic Notation}

There is a long list of ways to make mistakes with asymptotic
notation.  This section presents some of the ways that Big Oh notation
can lead to ruin and despair.  With minimal effort, you can cause just
as much chaos with the other symbols.

\subsubsection{The Exponential Fiasco}

Sometimes relationships involving Big Oh are not so obvious.  For
example, one might guess that $4^x = O(2^x)$ since 4 is only a
constant factor larger than 2.  This reasoning is incorrect, however;
\dmj{maybe ``$4^x$ actually grows''?}actually $4^x$ grows as the square
of~$2^x$.

\subsubsection{Constant Confusion}

Every constant is $O(1)$.  For example, $17 = O(1)$.  This is true because
if we let $f(x) = 17$ and $g(x) = 1$, then there exists a $c > 0$ and an
$x_0$ such that $\abs{f(x)} \leq c g(x)$.  In particular, we could choose
$c$ = 17 and $x_0 = 1$, since $\abs{17} \leq 17 \cdot 1$ for all $x \geq
1$.  We can construct a false theorem that exploits this fact.

\begin{falsethm}
\[
\sum_{i=1}^n i = O(n)
\]
\end{falsethm}

\begin{bogusproof}
Define $f(n) = \sum_{i=1}^n i = 1 + 2 + 3 + \cdots + n$.  Since we
have shown that every constant $i$ is $O(1)$, $f(n) = O(1) + O(1) +
\cdots + O(1) = O(n)$.
\end{bogusproof}

Of course in reality $\sum_{i=1}^n i = n(n+1)/2 \neq O(n)$.

The error stems from confusion over what is meant in the statement $i
= O(1)$.  For any \emph{constant} $i\in \naturals$ it is true that $i
= O(1)$.  More precisely, if $f$ is any constant function, then $f =
O(1)$.  But in this False Theorem, $i$ is not constant---it ranges
over a set of values 0, 1,\dots, $n$ that depends on~$n$.

And anyway, we should not be adding $O(1)$'s as though they were numbers.
We never even defined what $O(g)$ means by itself; it should only be used
in the context ``$f = O(g)$'' to describe a relation between functions $f$
and $g$.

\subsubsection{Lower Bound Blunder}

Sometimes people incorrectly use Big Oh in the context of a lower
bound.  For example, they might say, ``The running time, $T(n)$, is at
least $O(n^2)$,'' when they probably mean\footnote{This can also be
  correctly expressed as $n^2 = O(T(n))$, but such notation is rare.}
``$T(n) = \Omega(n^2)$.''

\subsubsection{Equality Blunder}

The notation $f = O(g)$ is too firmly entrenched to avoid, but the use of
``='' is really regrettable.  For example, if $f = O(g)$, it seems quite
reasonable to write $O(g) = f$.  But doing so might tempt us to the
following blunder: because $2n = O(n)$, we can say $O(n) = 2n$.  But $n =
O(n)$, so we conclude that $n = O(n) = 2n$, and therefore $n = 2n$.  To
avoid such nonsense, we will never write ``$O(f) = g$.''

Similarly, you will often see statements like
\begin{equation*}
    H_n = \ln(n) + \gamma + O\paren{\frac{1}{n}}
\end{equation*}
or
\begin{equation*}
    n! = (1 + o(1)) \sqrt{2\pi n} \paren{\frac{n}{e}}^n.
\end{equation*}
In such cases, the true meaning is
\begin{equation*}
    H_n = \ln(n) + \gamma + f(n)
\end{equation*}
for some~$f(n)$ where $f(n) = O(1/n)$, and
\begin{equation*}
    n! = (1 + g(n)) \sqrt{2\pi n} \paren{\frac{n}{e}}^n
\end{equation*}
where $g(n) = o(1)$.  These transgressions are OK as long as you (and
you reader) know what you mean.

\problemsection

\begin{problems}
\practiceproblems
\pinput{MQ_asymptotics_table}

\homeworkproblems
\pinput{PS_little_oh_properties}
\pinput{PS_asymptotics_table}
\pinput{PS_Stirlings_and_log_n_factorial}

\classproblems
\pinput{CP_Theta_log_n_factorial}
%\pinput{CP_Stirlings_and_log_n_factorial}
\pinput{CP_big_oh_practice}
%\pinput{CP_asymptotic_equality_pitfall}
\pinput{CP_false_asymptotics_proof}
\pinput{FP_Oh_not_Theta}
\end{problems}

\endinput
