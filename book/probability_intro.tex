\part{Probability}
\label{part:probability}

\partintro

Probability is one of the most important disciplines in all of the
sciences.  It is also one of the least well understood.

Probability is especially important in computer science---it arises in
virtually every branch of the field.  In algorithm design and game
theory, for example, \term{randomized} algorithms and strategies
(those that use a random number generator as a key input for decision
making) frequently outperform deterministic algorithms and
strategies.  In information theory and signal processing, an
understanding of randomness is critical for filtering out noise and
compressing data.  In cryptography and digital rights management,
probability is crucial for achieving security.  The list of examples
is long.

Given the impact that probability has on computer science, it seems
strange that probability should be so misunderstood by so many.
Perhaps the trouble is that basic human intuition is wrong as often as
it is right when it comes to problems involving random events.  As a
consequence, many students develop a fear of probability.  Indeed, we
have witnessed many graduate oral exams where a student will solve the
most horrendous calculation, only to then be tripped up by the simplest
probability question.  Indeed, even some faculty will start squirming
if you ask them a question that starts ``What is the probability
that\dots?''

Our goal in the remaining chapters is to equip you with the tools that
will enable you to solve basic problems involving probability easily
and confidently.

Chapter~\ref{probability_chap} introduces the basic definitions and an
elementary 4-step process that can be used to determine the
probability that a specified event occurs.  We illustrate the method
on two famous problems where your intuition will probably fail you.
The key concepts of Conditional probability and independence are
introduced, along with examples of their use, and regrettable misuse,
in practice: the probability you have a disease given that a
diagnostic test says you do, and the probability that a suspect is
guilty given that his blood type matches the blood found at the scene
of the crime.

Random variables provide a more quantitative way to measure random
events and We study them in Chapter~\ref{ran_var_chap}.  For example,
instead of determining the probability that it will rain, we may want
to determine \emph{how much} or \emph{how long} it is likely to rain.
The fundamental concept of the \term{expected value} of a random
variable is introduced and some of its key properties are developed.

%In Chapter~\ref{chap:deviations},

In the following chapeter, we examine the probability that a random
variable deviates significantly from its expected value.  Probability
of deviation provides the theoretical basis for estimation by sampling
which is fundamental in science, engineering, and human affairs.  It
is also especially important in engineering practice, where things are
generally fine if they are going as expected, and you would like to be
assured that the probability of an unexpected event is very low.

%Chapter~\ref{ran_process_chap}

A final chapter applies the previously probabilitic tools to solve
problems involving more complex random processes.  You will see why
you will probably never get very far ahead at the casino and how two
Stanford graduate students became billionaires by combining graph
theory and probability theory to design a better search engine for the
web.

\endinput
