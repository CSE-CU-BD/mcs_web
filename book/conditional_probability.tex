\chapter{Conditional Probability}\label{cond_prob_sec}\label{chap:cond_prob}

Suppose that we pick a random person in the world.  Everyone has an
equal chance of being selected.  Let $A$ be the event that the person
is an MIT student, and let $B$ be the event that the person lives in
Cambridge.  What are the probabilities of these events?  Intuitively,
we're picking a random point in the big ellipse shown in
Figure~\ref{fig:15B1} and asking how likely that point is to fall into
region $A$ or $B$.

\begin{figure}[h]

\graphic{cambridge-conditional}

\caption{Selecting a random person.  $A$ is the event that the person
  is an MIT student.  $B$ is the event that the person lives in
  Cambridge.}

\label{fig:15B1}

\end{figure}

The vast majority of people in the world neither live in Cambridge nor
are MIT students, so events $A$ and $B$ both have low probability.
But what about the probability that a person is an MIT student,
\emph{given} that the person lives in Cambridge?  This should be
much greater---but what is it exactly?

What we're asking for is called a \term{conditional
  probability}---that is, the probability that one event happens,
given that some other event definitely happens.  Questions about
conditional probabilities come up all the time:
%
\begin{itemize}
\item What is the probability that it will rain this afternoon, given
that it is cloudy this morning?
\item What is the probability that two rolled dice sum to 10, given
that both are odd?
\item What is the probability that I'll get four-of-a-kind in Texas No
Limit Hold 'Em Poker, given that I'm initially dealt two queens?
\end{itemize}

There is a special notation for conditional probabilities.  In
general, $\prcond{A}{B}$ denotes the probability of event $A$, given
that event $B$ happens.  So, in our example, $\prcond{A}{B}$ is the
probability that a random person is an MIT student, given that he or
she is a Cambridge resident.

How do we compute $\prcond{A}{B}$?  Since we are \emph{given} that the
person lives in Cambridge, we can forget about everyone in the world
who does not.  Thus, all outcomes outside event $B$ are irrelevant.
So, intuitively, $\prcond{A}{B}$ should be the fraction of Cambridge
residents that are also MIT students; that is, the answer should be
the probability that the person is in set $A \intersect B$ (the darkly
shaded region in Figure~\ref{fig:15B1}) divided by the probability
that the person is in set $B$ (the lightly shaded region).  This
motivates the definition of conditional probability:
\begin{definition}\label{LN12:prcond}
\[
\prcond{A}{B} \eqdef \frac{\pr{A \intersect B}}{\pr{B}}
\]
If $\pr{B} = 0$, then the conditional probability $\prcond{A}{B}$ is
undefined.
\end{definition}

Pure probability is often counterintuitive, but conditional
probability is even worse!  Conditioning can subtly alter
probabilities and produce unexpected results in randomized algorithms
and computer systems as well as in betting games.  Yet, the
mathematical definition of conditional probability given above is very
simple and should give you no trouble---provided that you rely on
mathematical reasoning and not intuition.  The four-step method will
also be very helpful as we will see in the next examples.

\subsection{The Four-Step Method for Conditional
  Probability: The ``Halting Problem''}

The \emph{Halting Problem} was the first example of a property that
could not be tested by any program.  It was introduced by Alan Turing
in his seminal 1936 paper.  The problem is to determine whether a
Turing machine halts on a given \dots yadda yadda yadda \dots more
importantly, it was \emph{the name of the MIT EECS department's famed
  C-league hockey team}.

In a best-of-three tournament, the Halting Problem wins the first game
with probability $1/2$.  In subsequent games, their
probability of winning is determined by the outcome of the previous
game.  If the Halting Problem won the previous game, then they are
invigorated by victory and win the current game with probability
$2/3$.  If they lost the previous game, then they are
demoralized by defeat and win the current game with probability only
$1/3$.  What is the probability that the Halting Problem wins
the tournament, given that they win the first game?

This is a question about a conditional probability.  Let $A$ be the
event that the Halting Problem wins the tournament, and let $B$ be the
event that they win the first game.  Our goal is then to determine the
conditional probability $\prcond{A}{B}$.

We can tackle conditional probability questions just like ordinary
probability problems: using a tree diagram and the four step method.
A complete tree diagram is shown in Figure~\ref{fig:15B2}.

\begin{figure}[h]

\graphic{hockey}

\caption{The tree diagram for computing the probability that the
  ``Halting Problem'' wins two out of three games given that they won
  the first game.}

\label{fig:15B2}

\end{figure}

\paragraph{Step 1:  Find the Sample Space}

Each internal vertex in the tree diagram has two children, one
corresponding to a win for the Halting Problem (labeled~$W$) and one
corresponding to a loss (labeled~$L$).  The complete sample space is:
%
\[
    \sspace = \set{ WW, \, WLW,\, WLL,\, LWW,\, LWL,\, LL }.
\]

\paragraph{Step 2:  Define Events of Interest}

The event that the Halting Problem wins the whole tournament is:
%
\[
    T = \set{WW,\, WLW,\, LWW}.
\]
%
And the event that the Halting Problem wins the first game is:
%
\[
    F = \set{WW,\, WLW,\, WLL }.
\]
%
The outcomes in these events are indicated with check marks in the tree
diagram in Figure~\ref{fig:15B2}.

\paragraph{Step 3:  Determine Outcome Probabilities}

Next, we must assign a probability to each outcome.  We begin by
labeling edges as specified in the problem statement.  Specifically,
The Halting Problem has a $1/2$ chance of winning the first game, so
the two edges leaving the root are each assigned probability $1/2$.
Other edges are labeled $1/3$ or $2/3$ based on the outcome of the
preceding game.  We then find the probability of each outcome by
multiplying all probabilities along the corresponding root-to-leaf
path.  For example, the probability of outcome $WLL$ is:
%
\[
    \frac{1}{2} \cdot \frac{1}{3} \cdot \frac{2}{3} = \frac{1}{9}.
\]

\subsubsection*{Step 4: Compute Event Probabilities}

We can now compute the probability that The Halting Problem wins the
tournament, given that they win the first game:
%
\begingroup
\openup2pt
\begin{align*}
\prcond{A}{B}
    & = \frac{\pr{A \intersect B}}{\pr{B}} \\
    & = \frac{\pr{\set{WW, WLW}}}{\pr{\set{WW, WLW, WLL}}} \\
    & = \frac{1/3 + 1/18}{1/3 + 1/18 + 1/9} \\
    & = \frac{7}{9}.
\end{align*}
\endgroup
%
We're done!  If the Halting Problem wins the first game, then they win
the whole tournament with probability $7 / 9$.


\subsection{Why Tree Diagrams Work}\label{product_rule_subsec}

We've now settled into a routine of solving probability problems using
tree diagrams.  But we've left a big question unaddressed: what is the
mathematical justification behind those funny little pictures?  Why do
they work?

The answer involves conditional probabilities.  In fact, the
probabilities that we've been recording on the edges of tree diagrams
\emph{are} conditional probabilities.  For example, consider the
uppermost path in the tree diagram for the Halting Problem, which
corresponds to the outcome $WW$.  The first edge is labeled $1/2$,
which is the probability that the Halting Problem wins the first game.
The second edge is labeled $2 / 3$, which is the probability that the
Halting Problem wins the second game, \emph{given} that they won the
first---that's a conditional probability!  More generally, on each
edge of a tree diagram, we record the probability that the experiment
proceeds along that path, given that it reaches the parent vertex.

So we've been using conditional probabilities all along.  But why can
we multiply edge probabilities to get outcome probabilities?  For
example, we concluded that:
%
\begin{equation*}
\pr{WW} = \frac{1}{2} \cdot \frac{2}{3}
	= \frac{1}{3}.
\end{equation*}
%
Why is this correct?

The answer goes back to Definition~\ref{LN12:prcond} of conditional probability
which could be written in a form called the \term{Product Rule} for
probabilities:
%
\begin{rul*}[Product Rule: 2 Events]
If $\pr{E_1} \neq 0$, then:
%
\[
    \pr{E_1 \intersect E_2} = \pr{E_1} \cdot \prcond{E_2}{E_1}.
\]
\end{rul*}
Multiplying edge probabilities in a tree diagram amounts to evaluating
the right side of this equation.  For example:
\begin{align*}
\lefteqn{\pr{\text{win first game} \intersect \text{win second game}}}
		\hspace{0.5in} \\[2pt]
	& = \pr{\text{win first game}} \cdot
            \prcond{\text{win second game}}{\text{win first game}} \\[2pt]
	& = \frac{1}{2} \cdot \frac{2}{3}.
\end{align*}
So the Product Rule is the formal justification for multiplying edge
probabilities to get outcome probabilities.  Of course, to justify
multiplying edge probabilities along longer paths, we need a Product
Rule for $n$ events.

% \dmj{I need to have another go at formatting this equation.}
\begin{rul*}[Product Rule: $n$ Events]
\begin{align*}
\pr{E_1 \intersect E_2 \intersect \dots \intersect E_n}
   =& \pr{E_1}
        \cdot \prcond{E_2}{E_1}
        \cdot \prcond{E_3}{E_1 \intersect E_2}
        \cdots \\
    &\quad\cdot
        \prcond{E_n}{E_1 \intersect E_2 \intersect \dots
          \intersect E_{n - 1}}
\end{align*}
provided that %\prod_{i=1}^n \pr{E_i}{\lgintersect_{j=1}^i E_j} 
\begin{equation*}
    \pr{E_1 \intersect E_2 \intersect \cdots \intersect E_{n - 1}}
    \neq 0.
\end{equation*}
\end{rul*}
This rule follows by routine induction from the definition of
conditional probability.

\subsection{Medical Testing}\label{med_test-subsection}

\dmj{Honestly, is this the most dignified example you could come up
  with?}
\ftl{Good point.  Let's flag it to be changed.}
There is an unpleasant condition called \emph{BO} suffered by 10\% of the
population.  There are no prior symptoms; victims just suddenly start to
stink.  Fortunately, there is a test for latent \emph{BO} before things
start to smell.  The test is not perfect, however:
\begin{itemize}

\item If you have the condition, there is a 10\% chance that the test
  will say you do not have it.  These are called ``false negatives.''

\item If you do not have the condition, there is a 30\% chance that the test
will say you do.  These are ``false positives.''

\end{itemize}

Suppose a random person is tested for latent \emph{BO}.  If the test is
positive, then what is the probability that the person has the condition?

\subsubsection*{Step 1: Find the Sample Space}

The sample space is found with the tree diagram in
Figure~\ref{fig:15C1}.

\begin{figure}[h]

\graphic{BO}

\caption{The tree diagram for the BO problem.}

\label{fig:15C1}

\end{figure}

\subsubsection*{Step 2: Define Events of Interest}

Let $A$ be the event that the person has \emph{BO}.  Let $B$ be the
event that the test was positive.  The outcomes in each event are marked
in the tree diagram.  We want to find $\prcond{A}{B}$, the probability
that a person has \emph{BO}, given that the test was positive.

\subsubsection*{Step 3: Find Outcome Probabilities}

First, we assign probabilities to edges.  These probabilities are
drawn directly from the problem statement.  By the Product Rule, the
probability of an outcome is the product of the probabilities on the
corresponding root-to-leaf path.  All probabilities are shown in
Figure~\ref{fig:15C1}.

\subsubsection*{Step 4: Compute Event Probabilities}

From Definition~\ref{LN12:prcond}, we have
\begin{equation*}
\prcond{A}{B}	= \frac{\pr{A \intersect B}}{\pr{B}} %\\[2pt]
		= \frac{0.09}{0.09 + 0.27} %\\[2pt]
		= \frac{1}{4}.
\end{equation*}
%
So, if you test positive, then there is only a 25\% chance that you
have the condition!

This answer is initially surprising, but makes sense on reflection.
There are two ways you could test positive.  First, it could be that
you have the condition and the test is correct.  Second, it could be that you
are healthy and the test is incorrect.  The problem is that almost
everyone is healthy; therefore, most of the positive results arise
from incorrect tests of healthy people!

We can also compute the probability that the test is correct for a
random person.  This event consists of two outcomes.  The person could
have the condition and test positive (probability $0.09$), or the person
could be healthy and test negative (probability $0.63$).
Therefore, the test is correct with probability $0.09 + 0.63 = 0.72$.
This is a relief; the test is correct almost three-quarters of the
time.

But wait!  There is a simple way to make the test correct 90\% of the
time: always return a negative result!  This ``test'' gives the right
answer for all healthy people and the wrong answer only for the 10\%
that actually have the condition.  So a better strategy by this
measure is to completely ignore the test result!

There is a similar paradox in weather forecasting.  During winter,
almost all days in Boston are wet and overcast.  Predicting miserable
weather every day may be more accurate than really trying to get it
right!


\subsection{\emph{A Posteriori} Probabilities}\label{aposteriori_subsec}

If you think about it too much, the medical testing problem we just
considered could start to trouble you.  The concern would be that by
the time you take the test, you either have the BO condition or you
don't---you just don't know which it is.  So you may wonder if a
statement like ``If you tested positive, then you have the condition
with probability~25\%'' makes sense.

In fact, such a statement does make sense.  It means that 25\% of the
people who test positive actually have the condition.  It is true that
any particular person has it or they don't, but a \emph{randomly
  selected} person among those who test positive will have the
condition with probability~25\%.

Anyway, if the medical testing example bothers you, you will
definitely be worried by the following examples, which go even further
down this path.

\subsection{The ``Halting Problem,'' in Reverse}

Suppose that we turn the hockey question around: what is the
probability that the Halting Problem won their first game, given that
they won the series?

This seems like an absurd question!  After all, if the Halting Problem
won the series, then the winner of the first game has already been
determined.  Therefore, who won the first game is a question of fact,
not a question of probability.  However, our mathematical theory of
probability contains no notion of one event preceding another---there
is no notion of time at all.  Therefore, from a mathematical
perspective, this is a perfectly valid question.  And this is also a
meaningful question from a practical perspective.  Suppose that you're
told that the Halting Problem won the series, but not told the results
of individual games.  Then, from your perspective, it makes perfect
sense to wonder how likely it is that The Halting Problem won the
first game.

A conditional probability $\prcond{B}{A}$ is called  \term{a
posteriori} if event $B$ precedes event $A$ in time.  Here are some
other examples of a posteriori probabilities:
%
\begin{itemize}
\item The probability it was cloudy this morning, given that it rained
in the afternoon.
\item The probability that I was initially dealt two queens in Texas
No Limit Hold 'Em poker, given that I eventually got four-of-a-kind.
\end{itemize}
%
Mathematically, a posteriori probabilities are \emph{no different}
from ordinary probabilities; the distinction is only at a higher,
philosophical level.  Our only reason for drawing attention to them is
to say, ``Don't let them rattle you.''

Let's return to the original problem.  The probability that the
Halting Problem won their first game, given that they won the series
is $\prcond{B}{A}$.  We can compute this using the definition of
conditional probability and the tree diagram in Figure~\ref{fig:15B2}:
%
\begin{align*}
\prcond{B}{A}  = \frac{\pr{B \intersect A}}{\pr{A}} %\\[2pt]
               = \frac{1/3 + 1/18}{1/3 + 1/18 + 1/9} %\\[2pt]
               = \frac{7}{9}.
\end{align*}

This answer is suspicious!  In the preceding section, we showed that
$\prcond{A}{B}$ was also $7/9$.  Could it be true that $\prcond{A}{B}
= \prcond{B}{A}$ in general?  Some reflection suggests this is
unlikely.  For example, the probability that I feel uneasy, given that
I was abducted by aliens, is pretty large.  But the probability that I
was abducted by aliens, given that I feel uneasy, is rather small.

Let's work out the general conditions under which $\prcond{A}{B} =
\prcond{B}{A}$.  By the definition of conditional probability, this
equation holds if an only if:
%
\[
\frac{\pr{A \intersect B}}{\pr{B}} = \frac{\pr{A \intersect B}}{\pr{A}}
\]
%
This equation, in turn, holds only if the denominators are equal or
the numerator is~0; namely if
%
\[
\pr{B} = \pr{A}
\hspace{0.25in} \text{or} \hspace{0.25in}
\pr{A \intersect B} = 0.
\]
%
The former condition holds in the hockey example; the probability that
the Halting Problem wins the series (event~$A$) is equal to the
probability that it wins the first game (event~$B$) since both
probabilities are~$1/2$.

In general, such pairs of probabilities are related by \idx{Bayes'
  Rule}:
%
\begin{theorem}[Bayes' Rule]
If $\pr{A}$ and $\pr{B}$ are nonzero, then:
%
\begin{equation}\label{bayesrule}
    \prcond{B}{A} = \frac{\prcond{A}{B} \cdot \pr{B}}{\pr{A}}
\end{equation}
\end{theorem}

\begin{proof}
When $\pr{A}$ and $\pr{B}$ are nonzero, we have
\[
\prcond{A}{B} \cdot \pr{B} = \prob{A \intersect B} = \prcond{B}{A} \cdot \pr{A}
\]
by definition of conditional probability.  Dividing by $\prob{A}$
gives~\eqref{bayesrule}.
\end{proof}

\iffalse

Next, let's look at a problem that even bothers us.

\subsection{A Coin Problem}

Suppose that someone hands you either a fair coin or a trick coin with
heads on both sides.  You flip the coin 100 times and see heads every
time.  What can you say about the probability that you flipped the
fair coin?  Remarkably, nothing!

In order to make sense out of this outrageous claim, let's formalize
the problem.  The sample space is worked out in the tree diagram shown
in Figure~\ref{fig:15C2}.  We do not know the probability~$p$ that you
were handed the fair coin initially---you were just given one coin or
the other.
%
\begin{figure}[h]

\graphic{trick-coin}

\caption{The tree diagram for the coin-flipping problem.}

\label{fig:15C2}

\end{figure}
%
Let $A$ be the event that you were handed the fair coin, and let $B$
be the event that you flipped 100 straight heads.  We're looking
for $\prcond{A}{B}$, the probability that you were handed the fair
coin, given that you flipped 100 heads.  The outcome probabilities are
worked out in Figure~\ref{fig:15C2}.  Plugging the results into the
definition of conditional probability gives:
%
\begin{align*}
\prcond{A}{B}	& = \frac{\pr{A \intersect B}}{\pr{B}} \\[2pt]
		& = \frac{p / 2^{100}}{1 - p + p / 2^{100}} \\[2pt]
		& = \frac{p}{2^{100} (1 - p) + p}.
\end{align*}
%
This expression is very small for moderate values of $p$ because of
the $2^{100}$ term in the denominator.  For example, if $p = 1/2$,
then the probability that you were given the fair coin is essentially
zero.

But we \emph{do not know} the probability $p$ that you were given
the fair coin.  And perhaps the value of $p$ is \emph{not} moderate;
in fact, maybe $p = 1 - 2^{-100}$.  Then there is nearly an even
chance that you have the fair coin, given that you flipped 100 heads.
In fact, maybe you were handed the fair coin with probability $p = 1$.
Then the probability that you were given the fair coin is, well,~1!

Of course, it is extremely unlikely that you would flip 100 straight
heads, but in this case, that is a given from the assumption of the
conditional probability.  And so if you really did see 100 straight
heads, it would be very tempting to also assume that $p$~is not close
to~1 and hence that you are very likely to have flipped the trick
coin.

We will encounter a very similar issue when we look at methods for
estimation by sampling in Section~\ref{sec:sampling}.
\fi

%\subsection{Conditional Identities}

\subsection{The Law of Total Probability}\label{sec:total_probability}

Breaking a probability calculation into cases simplifies many
problems.  The idea is to calculate the probability of an event $A$ by
splitting into two cases based on whether or not another event $E$
occurs.  That is, calculate the probability of $A\nobreak
\intersect\nobreak E$ and $A \intersect \setcomp{E}$.  By the Sum
Rule, the sum of these probabilities equals $\pr{A}$.  Expressing the
intersection probabilities as conditional probabilities yields:
\begin{rul}[Law of Total Probability, single event]\label{total_prob_Ebar}
If $\prob{E}$ and $\prob{\setcomp{E}}$~are nonzero, then
\[
\pr{A} = \prcond{A}{E} \cdot \pr{E} +
         \prcond{A}{\setcomp{E}} \cdot \pr{\setcomp{E}}.
\]
\end{rul}

For example, suppose we conduct the following experiment.  First, we
flip a fair coin.  If heads comes up, then we roll one die and take the
result.  If tails comes up, then we roll two dice and take the sum of
the two results.  What is the probability that this process yields a
2?  Let $E$ be the event that the coin comes up heads, and let $A$ be
the event that we get a 2 overall.  Assuming that the coin is fair,
$\pr{E} = \pr{\setcomp{E}} = 1/2$.  There are now two cases. If we
flip heads, then we roll a 2 on a single die with probability
$\prcond{A}{E} = 1/6$.  On the other hand, if we flip tails, then we
get a sum of 2 on two dice with probability
$\prcond{A}{\setcomp{E}} = 1/36$.  Therefore, the probability that
the whole process yields a 2 is
\[
\pr{A} = \frac{1}{2} \cdot \frac{1}{6} + \frac{1}{2} \cdot \frac{1}{36} =
  \frac{7}{72}.
\]

There is also a form of the rule to handle more than two cases.
\begin{rul}[Law of Total Probability]
If $E_1, \dots, E_n$ are disjoint events whose union is the whole
sample space, then:
\[
\pr{A} = \sum_{i=1}^{n} \prcond{A}{E_i} \cdot \pr{E_i}.
\]
\end{rul}

\subsection{Conditioning on a Single Event}\label{cond_ident_subsec}

The probability rules that we derived in Section~\ref{sec:union_bound}
extend to probabilities conditioned on the same event.  For example,
the Inclusion-Exclusion formula for two sets holds when all
probabilities are conditioned on an event $C$:
\[
\prcond{A \union B}{C} = \prcond{A}{C} + \prcond{B}{C} - \prcond{A \intersect B}{C}.
\]
This is easy to verify by plugging in the Definition~\ref{LN12:prcond}
of conditional
probability.\footnote{Problem~\ref{PS_conditional_space} explains why
  this and similar conditional identities follow on general principles
  from the corresponding unconditional identities.}

\iffalse
This follows from the fact that if $\pr{C} \neq 0$, then
\begin{align*}
\prcond{A \union B}{C}
    &= \frac{\pr{(A \union B) \intersect C}}{\pr{C}} \\[3pt]
    &= \frac{\pr{(A \intersect C) \union (B \intersect C)}}{\pr{C}} \\[3pt]
    &= \frac{\pr{A \intersect C} + \pr{B \intersect C}
             - \pr{A \intersect B \intersect C}}
            {\pr{C}} \\[3pt]
    &= \prcond{A}{C} + \prcond{B}{C} - \prcond{A \intersect B}{C}.
\end{align*}
\fi

It is important not to mix up events before and after the conditioning
bar.  For example, the following is \emph{not} a valid identity:
%
\begin{falseclm*}
\begin{equation}\label{LN12:fc}
\prcond{A}{B \union C} = \prcond{A}{B} + \prcond{A}{C} - \prcond{A}{B \intersect C}.
\end{equation}
\end{falseclm*}

A simple counter-example is to let $B$ and $C$ be events over a
uniform space with most of their outcomes in $A$, but not overlapping.
This ensures that $\prcond{A}{B}$ and $\prcond{A}{C}$ are both close
to 1.  For example,
\begin{align*}
B & \eqdef [0,9],\\
C & \eqdef [10,18] \union \set{0},\\
A & \eqdef [1,18],
\end{align*}
so
\[
\prcond{A}{B} = \frac{9}{10} = \prcond{A}{C}.
\]
Also, since 0 is the only outcome in $B \intersect C$ and $0 \notin
A$, we have
\[
\prcond{A}{B \intersect C} = 0
\]
So the right hand side of~\eqref{LN12:fc} is 1.8, while the left hand
side is a probability which can be at most 1---actually, it is 18/19.

\iffalse

A counterexample is shown in Figure~\ref{fig:15D2}.  In this case,
$\prcond{A}{B} = 1/2$, $\prcond{A}{C} = 1/2$, $\prcond{A}{B \intersect
  C} = 1$, and $\prcond{A}{B \union C} = 1/3$.  However, since
$1/3 \ne 1/2 + 1/2 - 1$, equation~\eqref{LN12:fc} does not hold.
%
\begin{figure}

\graphic{cx19}

\caption{A counterexample to equation~\eqref{LN12:fc}.  Event~$A$ is
  the dark-bordered rectangle, event~$B$ is the rectangle with
  vertical stripes, and event~$C$ is the rectangle with horizontal
  stripes.  $B \intersect C$ lies entirely within~$A$ while $B - C$
  and $C - B$ are entirely outside of~$A$.}

\label{fig:15D2}

\end{figure}
\fi


\subsection{Discrimination Lawsuit}\label{discrimination_subsec}

Several years ago there was a sex discrimination lawsuit against a
famous university.  A woman math professor was denied tenure,
allegedly because she was a woman.  She argued that in every one of
the university's 22 departments, the percentage of men candidates
granted tenure was greater than the percentage of women candidates
granted tenure.  This sounds very suspicious!

However, the university's lawyers argued that across the university as
a whole, the percentage of male candidates granted tenure was actually
\emph{lower} than the percentage for women candidates.  This suggests
that if there was any sex discrimination, then it was against men!
Surely, at least one party in the dispute must be lying.

Let's clarify the problem by expressing both arguments in terms of
conditional probabilities.  To simplify matters, suppose that there
are only two departments, EE and CS, and consider the experiment where
we pick a random candidate.  Define the following events:
%
\begin{itemize}
\item $A \eqdef$ the candidate is granted tenure,
\item $F_{EE} \eqdef$ the candidate is a woman in the EE department,
\item $F_{CS} \eqdef$ the candidate is a woman in the CS department,
\item $M_{EE} \eqdef$ the candidate is a man in the EE department,
\item $M_{CS} \eqdef$ the candidate is a man in the CS department.
\end{itemize}
Assume that all candidates are either men or women, and that no
candidate belongs to both departments.  That is, the events $F_{EE}$,
$F_{CS}$, $M_{EE}$, and $M_{CS}$ are all disjoint.

In these terms, the plaintiff is making the following argument:
%
\begin{align*}
\prcond{A}{F_{EE}} & < \prcond{A}{M_{EE}} \quad\text{and}\\
\prcond{A}{F_{CS}} & < \prcond{A}{M_{CS}}.
\end{align*}
That is, in both departments, the probability that a woman candidate
is granted tenure is less than the probability for a man.

The university retorts that \emph{overall}, a woman candidate is
\emph{more} likely to be granted tenure than a man; namely that
\[
    \prcond{A}{F_{EE} \union F_{CS}} > \prcond{A}{M_{EE} \union M_{CS}}.
\]

It is easy to believe that these two positions are contradictory, and
the phenomenon illustrated here is widely referred to as ``Simpson's
Paradox.''  But there is no contradiction or paradox, and in fact,
Table~\ref{fig:15D3} shows a set of candidate statistics for which the
assertions of both the plaintiff and the university hold.  In this
case, a higher percentage of men candidates were granted tenure in
each department, but overall a higher percentage of women candidates
were granted tenure!  How do we make sense of this?

\begin{table}

\begin{tabular}{crr}
CS & 0 women granted tenure, 1 candidates      &   0\% \\
   & 50 men granted tenure, 100 candidates     &  50\% \\
EE & 70 women granted tenure, 100 candidates   &  70\% \\
   & 1 man granted tenure, 1 candidates         & 100\% \\
\hline
Overall & 70 women granted tenure, 101 candidates & $\approx 70\%$ \\
        & 51 men granted tenure, 101 candidates   & $\approx 51\%$
\end{tabular}

\caption{A scenario where women are less likely to be granted tenure
  than men in each department, but more likely to be granted tenure
  overall.}

\label{fig:15D3}

\end{table}

With data like this showing that at the department level, women
candidates were less likely to be granted tenure than men, university
administrators would likely see an indication of bias against women,
and the departments would be directed to reexamine their tenure
procedures.

But suppose we replaced ``the candidate is a man/woman in the EE
department,'' by ``the candidate is a man/woman for whom a tenure
decision was made during an odd-numbered day of the month,'' and
likewise with CS and an even-numbered day of the month.  Since we
don't think the parity of a date is a cause for the outcome of a
tenure decision, we would ignore the ``coincidence'' that on both odd
and even dates, men are more frequently granted tenure.  Instead, we
would judge, based on the overall data showing women more likely to be
granted tenure, that gender bias against women was \emph{not} an issue
in the university.

The point is that it's the \emph{same data} that we interpret
differently based on our implicit causal beliefs.  It would be
circular to claim that the gender correlation observed in the data
corroborates our belief that there is discrimination, since our
interpretation of the data correlation \emph{depends} on our beliefs
about the causes of tenure decisions.\footnote{These issues are
  thoughtfully examined in \emph{Causality: Models, Reasoning and
    Inference}, Judea Pearl, Cambridge U. Press, 2001.}  This
illustrates a basic principle in statistics which people constantly
ignore: \emph{never assume that correlation implies causation}.

%% Conditional Probability Problems %%%%%%%%%%%%%%%%%%%%%%%%%%%%%%%%%%%%%%%%%%%

\begin{problems}
\practiceproblems
\pinput{TP_six_shooter_probability}

\classproblems
\pinput{CP_missing_card_probability}
\pinput{PS_conditional_aces}
\pinput{CP_conditional_prob_says_so_bug}
\pinput{FP_skywalker_prob_lin_recur_gen_func}
\pinput{FP_directed_graphs_and_probability}

\homeworkproblems
\pinput{PS_levitating_LAs}
\pinput{PS_conditional_probability_problem_errors}
\pinput{PS_coin_flip_sequences}
\pinput{PS_13_card_hand}
\pinput{PS_conditional_space}

\examproblems
\pinput{FP_monty_hall_variant}
\pinput{FP_conditional_prob_inequality}
\pinput{MQ_conditional_prob_inequality}
\pinput{FP_conditional_beaver_fever}
\pinput{FP_red_and_blue_goats}
\pinput{FP_neighborhood_census}
\pinput{MQ_voldemort_returns}
\end{problems}

\section{Independence}
Suppose that we flip two fair coins simultaneously on opposite sides
of a room.  Intuitively, the way one coin lands does not affect the
way the other coin lands.  The mathematical concept that captures
this intuition is called \term{independence}.
\begin{definition}\label{def:independence}
An event with probability 0 is defined to be independent of every
event (including itself).  If $\pr{B} \neq 0$, then
event $A$ is independent of event $B$ iff
\begin{equation}\label{eqn:independence}
    \prcond{A}{B} = \pr{A}.
\end{equation}
\end{definition}
In other words, $A$ and~$B$ are independent if knowing that $B$
happens does not alter the probability that $A$~happens, as is the
case with flipping two coins on opposite sides of a room.

\subsubsection{Potential Pitfall}

Students sometimes get the idea that disjoint events are independent.
The \emph{opposite} is true: if $A \intersect B = \emptyset$, then
knowing that $A$ happens means you know that $B$ does not happen.  So
disjoint events are \emph{never} independent---unless one of them has
probability zero.

\subsection{Alternative Formulation}

Sometimes it is useful to express independence in an alternate form
which follows immediately from Definition~\ref{def:independence}:

\begin{theorem}\label{thm:16A1}
$A$ is independent of~$B$ if and only if
\begin{equation}\label{eqn:15D3}
    \pr{A \intersect B} = \pr{A} \cdot \pr{B}.
\end{equation}
\end{theorem}

Notice that Theorem~\ref{thm:16A1} makes apparent the symmetry between
$A$ being independent of $B$ and $B$ being independent of $A$:
\begin{corollary}
$A$ is independent of $B$ iff $B$ is independent of $A$.
\end{corollary}


\iffalse

\begin{proof}
There are two cases to consider depending on whether or not $\prob{B} =
0$.
\begin{description}

\item[Case 1 $(\prob{B} = 0)$:]
If $\prob{B} = 0$, $A$ and~$B$ are independent by
Definition~\ref{def:independence}.  In addition,
equation~\eqref{eqn:15D3} holds since both sides are~0.  Hence, the
theorem is true in this case.

\item[Case 2 $(\prob{B} > 0)$:]
By Definition~\ref{LN12:prcond},
\begin{equation*}
    \prob{A \cap B} = \prcond{A}{B} \prob{B}.
\end{equation*}
So equation~\eqref{eqn:15D3} holds if
\begin{equation*}
    \prcond{A}{B} = \prob{A},
\end{equation*}
which, by Definition~\ref{def:independence}, is true iff $A$ and~$B$
are independent.  Hence, the theorem is true in this case as well.
\qedhere
\end{description}
\end{proof}
\fi

\subsection{Independence Is an Assumption}

Generally, independence is something that you \emph{assume} in
modeling a phenomenon.  For example, consider the experiment of
flipping two fair coins.  Let $A$~be the event that the first coin
comes up heads, and let $B$~be the event that the second coin is
heads.  If we assume that $A$ and~$B$ are independent, then the
probability that both coins come up heads is:
%
\begin{equation*}
\pr{A \intersect B}  = \pr{A} \cdot \pr{B} %\\[2pt]
               = \frac{1}{2} \cdot \frac{1}{2} %\\[2pt]
               = \frac{1}{4}.
\end{equation*}

In this example, the assumption of independence is reasonable.  The
result of one coin toss should have negligible impact on the outcome
of the other coin toss.  And if we were to repeat the experiment many
times, we would be likely to have~$A \cap B$ about~1/4 of the time.

There are, of course, many examples of events where assuming
independence is \emph{not} justified.  For example, let $C$~be the
event that tomorrow is cloudy and $R$ be the event that tomorrow is
rainy.  Perhaps $\pr{C} = 1/5$ and $\pr{R} = 1/10$ in Boston.  If
these events were independent, then we could conclude that the
probability of a rainy, cloudy day was quite small:
%
\begin{equation*}
\pr{R \intersect C} = \pr{R} \cdot \pr{C} % \\[2pt]
               = \frac{1}{5} \cdot \frac{1}{10} % \\[2pt]
               = \frac{1}{50}.
\end{equation*}
%
Unfortunately, these events are definitely not independent; in
particular, every rainy day is cloudy.  Thus, the probability of a
rainy, cloudy day is actually~$1/10$.

Deciding when to \emph{assume} that events are independent is a tricky
business.  In practice, there are strong motivations to assume
independence since many useful formulas (such as
equation~\eqref{eqn:15D3}) only hold if the events are independent.
But you need to be careful:
\iffalse
 lest you end up deriving false conclusions.
\fi
we'll describe several famous examples where (false) assumptions of
independence led to trouble.
\iffalse
 over the next several chapters
\fi
This problem gets even trickier when there are more than two events in
play.

\subsection{Mutual Independence}

%\subsection{Definition}

We have defined what it means for two events to be independent.  What
if there are more than two events?  For example, how can we say that
the flips of $n$~coins are all independent of one another?  A set of
events is said to be \term{mutually independent} if the probability
of each event in the set is the same no matter which of the other
events has occurred.  We could formalize this with conditional
probabilities as in Definition~\ref{def:independence}, but we'll jump
directly to the cleaner definition based on products of probabilities
as in Theorem~\ref{thm:16A1}:

\iffalse

\begin{definition}\label{def:mutual_independence}
A set of events~$E_1, E_2, \dots, E_n$, is \term{mutually independent}
if $\forall i \in [1, n]$ and $\forall S \subseteq [1, n] - \set{i}$,
either
\begin{equation*}
    \Prob{\bigcap_{j \in S} E_j} = 0
\quad
\text{or}
\quad
    \prob{E_i} = \prcond{E_i}{\bigcap_{j \in S} E_j}.
\end{equation*}
\end{definition}

\subsection{Alternative Formulation}

Just as Theorem~\ref{thm:16A1} provided an alternative definition of
independence for two events, there is an alternative definition for
mutual independence.

\fi

%\begin{theorem}\label{thm:16A2}

\begin{definition}\label{def:mutual_indep}
A set of events~$E_1, E_2, \dots, E_n$ is mutually independent iff
for all subsets $S \subseteq [1, n]$,
\begin{equation*}
    \Prob{\bigcap_{j \in S} E_j} = \prod_{j \in S} \prob{E_j}.
\end{equation*}
\end{definition}

%\end{theorem}
\iffalse
The proof of Theorem~\ref{thm:16A2} uses induction and reasoning
similar to the proof of Theorem~\ref{thm:16A1}.  We will not include
the details here.
\fi

Definition~\ref{def:mutual_indep} says that $E_1, E_2, \dots, E_n$~are
mutually independent if and only if all of the following equations
hold for all distinct $i$, $j$, $k$, and~$l$:
%
\begin{align*}
\pr{E_i \intersect E_j}
    & = \pr{E_i} \cdot \pr{E_j}
%    & \text{for all distinct $i$, $j$}
 \\
\pr{E_i \intersect E_j \intersect E_k}
    & = \pr{E_i} \cdot \pr{E_j} \cdot \pr{E_k}
%     & \text{for all distinct $i$, $j$, $k$}
 \\
\pr{E_i \intersect E_j \intersect E_k \intersect E_l}
    & = \pr{E_i} \cdot \pr{E_j} \cdot \pr{E_k} \cdot \pr{E_l}
%    & \text{for all distinct $i$, $j$, $k$, $l$}
 \\
    & \XasWideAsY{\vdots}{${}={}$} \\
\pr{E_1 \intersect \cdots \intersect E_n} & = \pr{E_1} \cdots \pr{E_n}.
\end{align*}

For example, if we toss $n$~fair coins, the tosses are mutually
independent iff for every subset of~$m$~coins, the probability that
every coin in the subset comes up heads is~$2^{-m}$.

\subsection{DNA Testing}

Assumptions about independence are routinely made in practice.
Frequently, such assumptions are quite reasonable.  Sometimes,
however, the reasonableness of an independence assumption is not so
clear, and the consequences of a faulty assumption can be severe.

For example, consider the following testimony from the O. J. Simpson
murder trial on May 15, 1995:
\begin{description}

\item[Mr. Clarke:] When you make these estimations of frequency---and
I believe you touched a little bit on a concept called independence?

\item[Dr. Cotton:] Yes, I did.

\item[Mr. Clarke:] And what is that again?

\item[Dr. Cotton:] It means whether or not you inherit one allele that
you have is not---does not affect the second allele that you might
get.  That is, if you inherit a band at 5,000 base pairs, that doesn't
mean you'll automatically or with some probability inherit one at
6,000.  What you inherit from one parent is what you inherit from the
other.

\item[Mr. Clarke:] Why is that important?

\item[Dr. Cotton:] Mathematically that's important because if that
were not the case, it would be improper to multiply the frequencies
between the different genetic locations.

\item[Mr. Clarke:] How do you---well, first of all, are these markers
independent that you've described in your testing in this case?

\end{description}

Presumably, this dialogue was as confusing to you as it was for the
jury.  Essentially, the jury was told that genetic markers in blood
found at the crime scene matched Simpson's.  Furthermore, they were
told that the probability that the markers would be found in a
randomly-selected person was at most 1 in 170 million.  This
astronomical figure was derived from statistics such as:
%
\begin{itemize}
\item 1 person in 100 has marker $A$.
\item 1 person in 50 marker $B$.
\item 1 person in 40 has marker $C$.
\item 1 person in 5 has marker $D$.
\item 1 person in 170 has marker $E$.
\end{itemize}
%
Then these numbers were multiplied to give the probability that a
randomly-selected person would have all five markers:
\begin{align*}
\pr{A \intersect B \intersect C \intersect D \intersect E}
    & = \pr{A} \cdot \pr{B} \cdot \pr{C} \cdot \pr{D} \cdot \pr{E}\\
    & = \frac{1}{100} \cdot \frac{1}{50} \cdot \frac{1}{40}
                     \cdot \frac{1}{5} \cdot \frac{1}{170}
     = \frac{1}{170{,}000{,}000}.
\end{align*}

\iffalse
\begin{align*}
\pr{A \intersect B \intersect C \intersect D \intersect E}
    & = \pr{A} \cdot \pr{B} \cdot \pr{C} \cdot \pr{D} \cdot \pr{E} \\[2pt]
    & = \frac{1}{100} \cdot \frac{1}{50} \cdot \frac{1}{40}
                      \cdot \frac{1}{5} \cdot \frac{1}{170} \\[2pt]
    & = \frac{1}{170{,}000{,}000}.
\end{align*}
\fi
%
The defense pointed out that this assumes that the markers appear
mutually independently.  Furthermore, all the statistics were based on
just a few hundred blood samples.  

After the trial, the jury was widely mocked for failing to
``understand'' the DNA evidence.  If you were a juror, would
\emph{you} accept the 1 in 170 million calculation?

\subsection{Pairwise Independence}

The definition of mutual independence seems awfully complicated
---there are so many subsets of events to consider!  Here's an example
that illustrates the subtlety of independence when more than two
events are involved.  Suppose that we flip three fair,
mutually-independent coins.  Define the following events:
%
\begin{itemize}
\item $A_1$ is the event that coin 1 matches coin 2.
\item $A_2$ is the event that coin 2 matches coin 3.
\item $A_3$ is the event that coin 3 matches coin 1.
\end{itemize}
%
Are $A_1$, $A_2$, $A_3$ mutually independent?

The sample space for this experiment is:
%
\[
    \set{HHH,\, HHT,\, HTH,\, HTT,\, THH,\, THT,\, TTH,\, TTT}.
\]
%
Every outcome has probability $(1/2)^3 = 1/8$ by our assumption that
the coins are mutually independent.

To see if events $A_1$, $A_2$, and $A_3$ are mutually independent, we
must check a sequence of equalities.  It will be helpful first to
compute the probability of each event $A_i$:
%
\begin{align*}
\pr{A_1} & = \pr{HHH} + \pr{HHT} + \pr{TTH} + \pr{TTT} \\[2pt]
         & = \frac{1}{8} + \frac{1}{8} + \frac{1}{8} + \frac{1}{8}%\\[2pt]
          = \frac{1}{2}.
\end{align*}
%
By symmetry, $\pr{A_2} = \pr{A_3} = 1/2$ as well.  Now we can begin
checking all the equalities required for mutual independence in
Definition~\ref{def:mutual_indep}:
\begin{align*}
\pr{A_1 \intersect A_2}
       & = \pr{HHH} + \pr{TTT}
         = \frac{1}{8} + \frac{1}{8}
         = \frac{1}{4}
         = \frac{1}{2} \cdot \frac{1}{2}\\
       & = \pr{A_1} \pr{A_2}.
\end{align*}

\iffalse
\begin{align*}
\pr{A_1 \intersect A_2}
	& = \pr{HHH} + \pr{TTT} \\[2pt]
        & = \frac{1}{8} + \frac{1}{8} \\[2pt]
        & = \frac{1}{4} \\[2pt]
        & = \frac{1}{2} \cdot \frac{1}{2}\\[2pt]
        & = \pr{A_1} \pr{A_2}.
\end{align*}\fi

By symmetry, $\pr{A_1 \intersect A_3} = \pr{A_1} \cdot \pr{A_3}$ and
$\pr{A_2 \intersect A_3} = \pr{A_2} \cdot \pr{A_3}$ must hold also.
Finally, we must check one last condition:

\begin{align*}
\pr{A_1 \intersect A_2 \intersect A_3}
        & = \pr{HHH} + \pr{TTT}
          = \frac{1}{8} + \frac{1}{8}
          = \frac{1}{4}\\
        & \textcolor{red}{\mathbf{\neq}} \frac{1}{8} = \pr{A_1} \pr{A_2} \pr{A_3}.
\end{align*}


\iffalse
\begin{align*}
\pr{A_1 \intersect A_2 \intersect A_3}      & = \pr{HHH} + \pr{TTT} \\[2pt]
                                & = \frac{1}{8} + \frac{1}{8} \\[2pt]
                                & = \frac{1}{4} \\[2pt]
                                & \neq \pr{A_1} \pr{A_2} \pr{A_3} = \frac{1}{8}.
\end{align*}
\fi
%
The three events $A_1$, $A_2$, and~$A_3$ are not mutually independent
even though any two of them are independent!  This not-quite mutual
independence seems weird at first, but it happens.  It even
generalizes:

\begin{definition}\label{kway_independent_events}
  A set $A_1$, $A_2$, \dots, of events is \term{$k$-way independent}
  iff every set of $k$ of these events is mutually independent.  The
  set is \term{pairwise independent} iff it is 2-way independent.
\end{definition}

So the sets $A_1$, $A_2$, $A_3$ above are pairwise independent, but
not mutually independent.  Pairwise independence is a much weaker
property than mutual independence.

For example, suppose that the prosecutors in the O.~J. Simpson trial
were wrong and markers $A$, $B$, $C$, $D$, and $E$ appear only
\emph{pairwise} independently.  Then the probability that a
randomly-selected person has all five markers is no more than:
%
\begin{align*}
\pr{A \intersect B \intersect C \intersect D \intersect E}
    & \leq \pr{A \intersect E} = \pr{A} \cdot \pr{E}\\
    & = \frac{1}{100} \cdot \frac{1}{170} = \frac{1}{17{,}000}.
\end{align*}
%
The first line uses the fact that $A \intersect B \intersect C \intersect
D \intersect E$ is a subset of $A \intersect E$.  (We picked out the $A$
and $E$ markers because they're the rarest.)  We use pairwise independence
on the second line.  Now the probability of a random match is 1 in
17,000---a far cry from 1 in 170 million!  And this is the strongest
conclusion we can reach assuming only pairwise independence.

On the other hand, the 1 in 17,000 bound that we get by assuming
pairwise independence is a lot better than the bound that we would
have if there were no independence at all.  For example, if the
markers are dependent, then it is possible that
\begin{quote}
everyone with marker~$E$ has marker~$A$,

everyone with marker~$A$ has marker~$B$,

everyone with marker~$B$ has marker~$C$, and

everyone with marker~$C$ has marker~$D$.
\end{quote}
In such a scenario, the probability of a match is
\begin{equation*}
    \pr{E} = \frac{1}{170}.
\end{equation*}

So a stronger independence assumption leads to a smaller bound on the
probability of a match.  The trick is to figure out what independence
assumption is reasonable.  Assuming that the markers are
\emph{mutually} independent may well \emph{not} be reasonable unless
you have examined hundreds of millions of blood samples.  Otherwise,
how would you know that marker~$D$ does not show up more frequently
whenever the other four markers are simultaneously present?

We will conclude our discussion of independence with a useful, and
somewhat famous, example known as the Birthday Principle.

\subsection{The Birthday Principle}\label{birthday_principle_sec}

There are 95 students in a class.  What is the probability that some
birthday is shared by two people?  Comparing 95 students to the 365
possible birthdays, you might guess the probability lies somewhere
around $1/4$---but you'd be wrong: the probability that there will be
two people in the class with matching birthdays is actually more than
$0.9999$.

To work this out, we'll assume that the probability that a randomly
chosen student has a given birthday is $1/d$, where $d= 365$ in this
case.  We'll also assume that a class is composed of $n$ randomly and
independently selected students, with $n=95$ in this case.  These
randomness assumptions are not really true, since more babies are born
at certain times of year, and students' class selections are typically
not independent of each other, but simplifying in this way gives us a
start on analyzing the problem.  More importantly, these assumptions
are justifiable in important computer science applications of birthday
matching.  For example, the birthday matching is a good model for
collisions between items randomly inserted into a hash table.  So we
won't worry about things like Spring procreation preferences that make
January birthdays more common, or about twins' preferences to take
classes together (or not).  \begin{editingnotes}
or that fact that a student
can't be selected twice in making up a class list.
\end{editingnotes}

Selecting a sequence of $n$ students for a class yields a sequence of
$n$ birthdays.  Under the assumptions above, the $d^n$ possible
birthday sequences are equally likely outcomes.  Let's examine the
consequences of this probability model by focussing on the $i$th and
$j$th elements in a birthday sequence, where $1 \leq i \neq j \leq n$.
It makes for a better story if we refer to the $i$th birthday as
``Alice's'' and the $j$th as ``Bob's.''

Now if Alice, Bob, Carol, and Don are four different people, then
whether Alice and Bob have matching birthdays is independent of
whether Carol and Don do.  What's more interesting is that whether
Alice and \emph{Carol} have the same birthday is independent of
whether Alice and Bob do.  This follows because Carol is as likely to
have the same birthday as Alice, independently of whatever birthdays
Alice and Bob happen to have; a formal proof of this claim appears in
Problem~\ref{PS_equal_birthdays}.  In short, the set of all events
that a couple has matching birthdays is \index{pairwise independent}
\emph{pairwise} independent, even for overlapping couples.  This will
be important in Chapter~\ref{deviation_chap} because pairwise
independence will be enough to justify some conclusions about the
expected number of matches.  However, these matching birthday events
are obviously \emph{not} even 3-way independent: if Alice and Bob
match, and also Alice and Carol match, then Bob and Carol will match.

\iffalse
We could justify all these assertions of independence using the four
step method, but it's pretty boring, and we'll skip it.
\fi

It turns out that as long as the number of students is noticeably
smaller than the number of possible birthdays, we can get a pretty
good estimate of the birthday matching probabilities by
\emph{pretending} that the matching events are mutually independent.
(An intuitive justification for this is that with only a small number
of matching pairs, it's likely that none of the pairs overlap.)  Then
the probability of \emph{no} matching birthdays would be the same as
the $r$th power of the probability that a couple does \emph{not} have
matching birthdays, where $r \eqdef \binom{n}{2}$ is the number of
couples.  That is, the probability of no matching birthdays would be
\begin{equation}\label{11dbinn2}
(1-1/d)^{\binom{n}{2}}.
\end{equation}
Using the fact that $1+x < e^x$ for all $x$,\footnote{This
  approximation is obtained by truncating the Taylor series $e^{-x} =
  1 - x + x^2/2!  - x^3/3! + \cdots$.  The approximation $e^{-x}
  \approx 1 - x$ is pretty accurate when $x$ is small.} we would conclude
that the probability of no matching birthdays is at most
\begin{equation}\label{bday-approx}
e^{-\binom{n}{2}/d}.
\end{equation}

The matching birthday problem fits in here so far as a nice example
illustrating pairwise and mutual independence, but it's actually not
hard to justify the bound~\eqref{bday-approx} without any pretence of
independence.  Namely, there are $d (d - 1) (d - 2) \cdots (d - (n -
1))$ length $n$ sequences of distinct birthdays.  So the probability
that everyone has a different birthday is:
\begin{align*}
\lefteqn{\frac{d (d - 1) (d - 2) \cdots (d - (n - 1))}{d^n}}\\
   & = \frac{d}{d} \cdot \frac{d-1}{d} \cdot \frac{d-2}{d} \cdots \frac{d - (n - 1)}{d}\\
   & = \paren{1 - \frac{0}{d}}
             \paren{1 - \frac{1}{d}}
             \paren{1 - \frac{2}{d}}
             \cdots
             \paren{1 - \frac{n - 1}{d}}\\
   & < e^0 \cdot e^{-1/d} \cdot e^{-2/d} \cdots e^{-(n-1)/d} 
             & \text{(since $1+x < e^x$)} \\
   & = e^{-\paren{\sum_{i=1}^{n-1} i/d}}\\
   & = e^{-\paren{n(n-1)/2d}}\\
   & = \text{the bound~\eqref{bday-approx}}.
\end{align*}

For $n=95$ and $d = 365$, the value of~\eqref{bday-approx} is less
than $1/200,000$, which means the probability of having some pair of
matching birthdays actually is more than $1 - 1/200,000 > 0.99999$.  So
it would be pretty astonishing if there were no pair of students in
the class with matching birthdays.

For $d \leq n^2/2$, the probability of no match turns out to be
asymptotically equal to the upper bound~\eqref{bday-approx}.  For $d =
n^2/2$ in particular, the probability of no match is asymptotically
equal to $1/e$.  This leads to a rule of thumb which is useful in many
contexts in computer science:

\textbox{
\begin{center}
\large The \index{birthday principle} Birthday Principle
\end{center}

If there are $d$ days in a year and $\sqrt{2d}$ people in a
room, then the probability that two share a birthday is about 
$1 - 1/e \approx 0.632$.
}

For example, the Birthday Principle says that if you have $\sqrt{2
  \cdot 365} \approx 27$ people in a room, then the probability that
two share a birthday is about $0.632$.  The actual probability is
about $0.626$, so the approximation is quite good.

Among other applications, it implies that to use a hash function that
maps $n$ items into a hash table of size $d$, you can expect many
collisions unless $n^2$ is a small fraction of $d$.  The Birthday
Principle also famously comes into play as the basis of ``birthday
attacks'' that crack certain cryptographic systems.


\begin{problems}
\practiceproblems
\pinput{TP_Binomial_Board_Breaking}
\pinput{TP_Practice_with_Bounds}
%\pinput{FP_random_sampling}

\examproblems
\pinput{FP_college_probability}
\pinput{FP_product_rule_and_independence}

\classproblems
\pinput{CP_mutual_independence}
\pinput{CP_three_fair_coins}

\homeworkproblems
\pinput{PS_bogus_discrimination_contradiction}
\pinput{FP_graph_logic_probability}

\end{problems}


\iffalse %ftl version

\subsection{The Birthday Paradox}\label{birthday_principle_sec}

Suppose that there are 100 students in a class.  What is the
probability that some birthday is shared by two people?  Comparing 100
students to the 365 possible birthdays, you might guess the
probability lies somewhere around~$1/3$---but you'd be wrong: the
probability that there will be two people in the class with matching
birthdays is actually~$0.999999692\dots$.  In other words, the
probability that all 100 birthdays are different is less than 1
in~3,000,000.

Why is this probability so small?  The answer involves a phenomenon
known as the \term{Birthday Paradox} (or the \term{Birthday
  Principle}), which is surprisingly important in computer science, as
we'll see later.

Before delving into the analysis, we'll need to make some modeling
assumptions:
\begin{itemize}

\item
For each student, all possible birthdays are equally likely.  The idea
underlying this assumption is that each student's birthday is
determined by a random process involving parents, fate, and, um, some
issues that we discussed earlier in the context of graph theory.
The assumption is not completely accurate, however; a disproportionate
number of babies are born in August and September, for example.

\item
Birthdays are mutually independent.  This isn't perfectly accurate
either.  For example, if there are twins in the class, then their
birthdays are surely not independent.

\end{itemize}
We'll stick with these assumptions, despite their limitations.  Part
of the reason is to simplify the analysis.  But the bigger reason is
that our conclusions will apply to many situations in computer science
where twins, leap days, and romantic holidays are not considerations.
After all, whether or not two items collide in a hash table really has
nothing to do with human reproductive preferences.  Also, in pursuit
of generality, let's switch from specific numbers to variables.  Let
$m$~be the number of people in the room, and let $N$~be the number of
days in a year.

We can solve this problem using the standard four-step method.
However, a tree diagram will be of little value because the sample
space is so enormous.  This time we'll have to proceed without the
visual aid!

\paragraph{Step 1: Find the Sample Space}

Let's number the people in the room from 1 to~$m$.  An outcome of the
experiment is a sequence $(b_1, \dots, b_m)$ where $b_i$~is the
birthday of the $i$th person.  The sample space is the set of all such
sequences:
\begin{equation*}
    \sspace = \{\, (b_1, \dots, b_m) \suchthat b_i \in \set{1, \dots
      N} \,\}.
\end{equation*}

\paragraph{Step 2: Define Events of Interest}

Our goal is to determine the probability of the event~$A$ in which
some pair of people have the same birthday.  This event is a little
awkward to study directly, however.  So we'll use a common trick,
which is to analyze the \term{complementary} event~$\setcomp{A}$, in
which all $m$~people have different birthdays:
\begin{equation*}
    \setcomp{A} = \set{\, (b_1, \dots, b_m) \in \sspace
                    \suchthat \text{all $b_i$ are distinct} \,}.
\end{equation*}
If we can compute $\pr{\setcomp{A}}$, then we can compute what
really want, $\pr{A}$, using the identity
\begin{equation*}
    \pr{A} + \pr{\setcomp{A}} = 1.
\end{equation*}

\paragraph{Step 3: Assign Outcome Probabilities}

We need to compute the probability that $m$~people have a particular
combination of birthdays ~$(b_1, \dots, b_m)$.  There are $N$~possible
birthdays and all of them are equally likely for each student.
Therefore, the probability that the $i$th person was born on day~$b_i$
is~$1/N$.  Since we're assuming that birthdays are mutually
independent, we can multiply probabilities.  Therefore, the
probability that the first person was born on day~$b_1$, the second
on~$b_2$, and so forth is~$(1/N)^m$.  This is the probability of every
outcome in the sample space, which means that the sample space is
uniform.  That's good news, because, as we have seen, it means that
the analysis will be simpler.

\paragraph{Step 4: Compute Event Probabilities}

We're interested in the probability of the event~$\setcomp{A}$ in
which everyone has a different birthday:
\begin{equation*}
    \setcomp{A} = \set{\, (b_1, \dots, b_n) \suchthat
                            \text{all $b_i$ are distinct} \,}.
\end{equation*}
This is a gigantic set.  In fact, there are $N$~choices for~$b_i$,
\ $N - 1$ choices for~$b_2$, and so forth.  Therefore, by the
Generalized Product Rule,
\begin{equation*}
\card{\setcomp{A}}
    = \frac{N!}{(N - m)!}
    = N (N - 1) (N - 2) \cdots (N - m + 1).
\end{equation*}
Since the sample space is uniform, we can conclude that
\begin{equation}\label{eqn:15E4}
\pr{\setcomp{A}}
    = \frac{\card{\setcomp{A}}}{N^m} \\
    = \frac{N!}{N^m (N - m)!}.
\end{equation}
We're done!

Or are we?  While correct, it would certainly be nicer to have a
closed-form expression for equation~\eqref{eqn:15E4}.  That means
finding an approximation for $N!$ and~$(N - m)!$.  But this is what we
learned how to do in Section~\ref{sec:closed_products}.  In fact, since
$N$ and~$N - m$ are each at least~100, we know from
Corollary~\ref{cor:9A2} that
\begin{equation*}
    \stirling{N} \quad \text{and} \quad \stirling*{N - m}
\end{equation*}
are excellent approximations (accurate to within~.09\%) of~$N!$ and~$(N
- m)!$, respectively.  Plugging these values into
equation~\eqref{eqn:15E4} means that (to within~.2\%)\footnote{If there
are two terms that can be off by~.09\%, then the ratio can be off by
at most a factor of~$(1.0009)^2 < 1.002$.}
\begingroup
\openup2\jot
\begin{align}
\prob{\setcomp{A}}
    &= \frac{ \stirling{N} }{ N^m \stirling*{N - m} } \notag\\
    &= \sqrt{\frac{N}{N - m}}
             \frac{ e^{N \ln(N) - N} }
                  { e^{m \ln(N)} e^{(N - m) \ln(N - m) - (N - m) } }
                  \notag\\
    &= \sqrt{\frac{N}{N - m}}
         e^{ (N - m)\ln(N) - (N - m) \ln(N - m) - m } \notag\\
    &= \sqrt{\frac{N}{N - m}}
         e^{ (N - m)\ln\paren{\frac{N}{N - m}} - m } \notag\\
    &= e^{ \paren{N - m + \frac{1}{2}}\ln\paren{\frac{N}{N - m}} - m }.
        \label{eqn:15E9}
\end{align}
\endgroup
We can now evaluate equation~\eqref{eqn:15E9} for $m = 100$ and $N =
365$ to find that the probability that all 100 birthdays are different
is\footnote{The possible .2\%~error is so small that
  it is lost in the \dots after 3.07.}
\begin{equation*}
    3.07\ldots \cdot 10^{-7}.
\end{equation*}

We can also plug in other values of~$m$ to find the number of people
so that the probability of a matching birthday will be about~$1/2$.
In particular, for $m = 23$ and $N = 365$, equation~\eqref{eqn:15E9}
reveals that the probability that all the birthdays differ is
0.49\dots.  So if you are in a room with 23 other people, the
probability that some pair of people share a birthday will be a little
better than~$1/2$.  It is because 23 seems like such a small number of
people for a match that the phenomenon is called the \term{Birthday
  Paradox}.

\subsection{Applications to Hashing}

Hashing is frequently used in computer science to map large strings of
data into short strings of data.  In a typical scenario, you have a
set of $m$~items and you would like to assign each item to a number
from 1 to~$N$ where no pair of items is assigned to the same number
and $N$~is as small as possible.  For example, the items might be
messages, addresses, or variables.  The numbers might represent
storage locations, devices, indices, or digital signatures.

If two items are assigned to the same number, then a \term{collision}
is said to occur.  Collisions are generally bad.  For example,
collisions can correspond to two variables being stored in the same
place or two messages being assigned the same digital signature.  Just
imagine if you were doing electronic banking and your digital
signature for a \$10~check were the same as your signature for a
\$10~million dollar check.  In fact, finding collisions is a common
technique in breaking cryptographic codes.\footnote{Such techniques
  are often referred to as \term{birthday attacks} because of the
  association of such attacks with the Birthday Paradox.}

In practice, the assignment of a number to an item is done using a
hash function
\begin{equation*}
    h: S \to [1, N],
\end{equation*}
where $S$~is the set of items and $m = \card{S}$.  Typically, the
values of~$h(S)$ are assigned randomly and are assumed to be equally
likely in~$[1, N]$ and mutually independent.

For efficiency purposes, it is generally desirable to make~$N$ as
small as necessary to accommodate the hashing of $m$~items without
collisions.  Ideally, $N$~would be only a little larger than~$m$.
Unfortunately, this is not possible for random hash functions.  To see
why, let's take a closer look at equation~\eqref{eqn:15E9}.

By Theorem~\ref{thm:stirling} and the derivation of
equation~\eqref{eqn:15E9}, we know that the probability that there are
no collisions for a random hash function is
\begin{equation}\label{eqn:16K}
    \sim e^{ \paren{N - m + \frac{1}{2}} \ln\paren{\frac{N}{N - m}} - m }.
\end{equation}
For any~$m$, we now need to find a value of~$N$ for which this
expression is at least~1/2.  That will tell us how big the hash table
needs to be in order to have at least a 50\%~chance of avoiding
collisions.  This means that we need to find a value of~$N$ for which
\begin{equation}\label{eqn:16P}
    \paren{N - m + \frac{1}{2}} \ln\paren{\frac{N}{N - m}} - m 
        \sim
    \ln\paren{\frac{1}{2}}.
\end{equation}

To simplify equation~\eqref{eqn:16P}, we need to get rid of the
$\ln\paren{\frac{N}{N - m}}$~term.  We can do this by using the Taylor
Series expansion for
\begin{equation*}
    \ln(1 - x) = -x - \frac{x^2}{2} - \frac{x^3}{3} - \cdots
\end{equation*}
to find that\footnote{This may not look like a simplification, but
  stick with us here.}
\begin{align*}
\ln\paren{\frac{N}{N - m}}
    &= - \ln \paren{\frac{N - m}{N}} \\
    &= - \ln \paren{1 - \frac{m}{N}} \\
    &= - \paren{ -\frac{m}{N} - \frac{m^2}{2N^2} - \frac{m^3}{3N^3} - \cdots }\\
    &= \frac{m}{N} + \frac{m^2}{2N^2} + \frac{m^3}{3N^3} + \cdots.
\end{align*}
Hence,
\begin{align}
\paren{N - m + \frac{1}{2}} \ln\paren{\frac{N}{N - m}} - m
    &= \paren{N - m + \frac{1}{2}}
        \paren{\frac{m}{N} + \frac{m^2}{2N^2} + \frac{m^3}{3N^3} + \cdots}
        - m \notag\\
    &= \paren{ m + \frac{m^2}{2N} + \frac{m^3}{3N^2} + \cdots }
            \notag\\
    &\phantom{=}\qquad - \paren{ \frac{m^2}{N} + \frac{m^3}{2N^2} +
          \frac{m^4}{3N^3} + \cdots }
            \notag\\
    &\phantom{=}\qquad + \frac{1}{2} \paren{\frac{m}{N} +
          \frac{m^2}{2N^2} + \frac{m^3}{3N^3} + \cdots} -m 
            \notag\\
    &= - \paren{ \frac{m^2}{2N} + \frac{m^3}{6N^2} + \frac{m^4}{12N^3}
            + \cdots} \notag\\
    &\phantom{=}\qquad
        + \frac{1}{2}\paren{ \frac{m}{N} + \frac{m^2}{2N^2} +
          \frac{m^3}{3N^3} + \cdots }.
    \label{eqn:16Q}
\end{align}

If $N$~grows faster than~$m^2$, then the value in
equation~\eqref{eqn:16Q} tends to~0 and equation~\eqref{eqn:16P}
cannot be satisfied.  If $N$~grows more slowly than~$m^2$, then the
value in equation~\eqref{eqn:16Q} diverges to negative infinity, and,
once again, equation~\eqref{eqn:16P} cannot be satisfied.  This suggests
that we should focus on the case where~$N = \Theta(m^2)$, when
equation~\eqref{eqn:16Q} simplifies to
\begin{equation*}
    \sim \frac{-m^2}{2N}
\end{equation*}
and equation~\eqref{eqn:16P} becomes
\begin{equation}\label{eqn:16R}
    \frac{-m^2}{2N} \sim \ln\paren{\frac{1}{2}}.
\end{equation}

equation~\eqref{eqn:16R} is satisfied when
\begin{equation}\label{eqn:16S}
    N \sim \frac{m^2}{2 \ln(2)}.
\end{equation}

In other words, $N$~needs to grow quadratically with~$m$ in order to
avoid collisions.  This unfortunate fact is known as the
\term{Birthday Principle} and it limits the efficiency of hashing in
practice---either $N$~is quadratic in the number of items being hashed
or you need to be able to deal with collisions.

\fi
