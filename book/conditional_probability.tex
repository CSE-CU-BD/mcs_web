\chapter{Conditional Probability}\label{chap:cond_prob} %\label{cond_prob_sec}

\begin{editingnotes}
\arm{DO MAJOR REVISION.} Replace MIT/Cambridge example by Monty Hall
examples of conditioning on wrong events from ARM slides.  Introduce
Strogatz article approach using natural frequencies to explain
diagnosis probabilities.  Include OJ/Dershowitz/Paulos discussion of
wife-battering.  Improve explanation of \emph{a posteriori}
probability.
\end{editingnotes}

Remember how we said that the Monty Hall problem confused even
professional mathematicians?  Based on the work we did with tree
diagrams, this may seem surprising—the conclusion we reached was
definitive.  So how could this problem be so confusing to so many
people?  

Well, one flawed argument goes as follows: let’s say the contestant
picks door A.  And suppose that Carol, Monty’s assistant, opens door B
and shows us a goat.  Now, let’s use what we know and the \href{}{tree
  diagram} from Section 16.  The contestant chooses door $A$, and
there’s a goat behind door $B$, in the following outcomes:
\[
(A, A, B),\ (A, A, C),\ (C, A, B).
\]

Now among those outcomes, switching doors wins on $(A, A, B)$ and $(A,
A, C)$.  Each has a probability of 1/18, which means that
\emph{together} they have the \emph{same} 1/9 probability as losing by
swithching in $(C, A, B)$.  So in this situation, the probability that
we win is the same as the probability that we lose.  Switching isn't
any better than sticking!

It’s hard to spot the hole in this reasoning, even though we know from
our earlier results that it can’t be correct.  But before we pick
apart this explanation, let's pin down the precise math behind it.
What we’re asking for is a definition of \idx{conditional
  probability}—the odds that one event (win by switching) happens,
given some other event (pick A, goat at B) happens.

\section{Behind the Curtain}

Like all probabilities, conditional probability is a ratio comparing
the likelihood of a desired set of outcomes to a sample
space. However, conditional probabilities do not concern themselves
with all possible outcomes.  When something is ``given'' as a
condition, that condition essentially limits our new sample space to a
subset of possible outcomes.  In this particular example, we're given
that the player chooses door A and that there is a goat behind B.  Our
new sample space therefore consists solely of the three outcomes in
which this occurs.  The conditional probability of winning by
switching, given that the player chose door A and the a goat is behind
door B, is the fraction of this new sample space that consists of
``win-by-switching''; that is to say, it can be found by putting the
probability of the two win-by-switching outcomes in our conditioning
event over the probability of the three outcomes $(A, A, B)$, $(A, A,
C)$, and $(C, A, B)$, in our conditioning event.  All we need to do is
look at the probabilities of these three outcomes and do some
arithmetic.

According to the tree diagram, $(A, A, B)$, $(A, A, C)$, and $(C, A,
B)$ have respective probabilities 1/18, 1/18, 1/9.  A Monty Hall game
has a
\[
\frac{1}{18} + \frac{1}{18} + \frac{1}{9} = \frac{2}{9}
\]
chance of ending in one of these outcomes.  This 2/9 fraction is all
we need; we can ignore all other outcomes on the tree.

As we mentioned earlier, the two winning outcomes have a total
probability of 1/9.  We find the conditional probability by figuring
out how that compares to the total probability of our condition:
\[ 
\frac{\pr{(C, A, B)}}{\pr{(C, A, B)}+\pr{(A, A, B)}+\pr{(A, A, C)}} =
\frac{1/9}{2/9}=\frac{1}{2}
\]
is our chance of winning, namely, 50\%.  Mathematically, at least, the
logic in the introduction is sound.  Curiouser and curiouser!

\section{Definition and Notation}

The expression $\prcond{A}{B}$ denotes the probability of event $A$,
given that event $B$ happens.  In the example above, event $A$ is the
event of winning on a switch, and event $B$ is the event that a goat
is behind door B and the contestant chose door A.

How did we compute $\prcond{A}{B}$?  Since we were \emph{given} that
the goat was behind B, we limit ourselves to the outcomes where this
occurs.  The conditional probability takes the form of the fraction of
the given subset of outcomes with event $B$ in which event $A$ also
occurs.  This motivates the definition of conditional probability:
\begin{definition}\label{LN12:prcond}
Let $A$ and $B$ be events where $B$ has nonzero probability.  Then
\[
\prcond{A}{B} \eqdef \frac{\pr{A \intersect B}}{\pr{B}}\,.
\]
\end{definition}

The conditional probability $\prcond{A}{B}$ is undefined when the
probability of event $B$ is zero.  Assertions about conditional
probability can get cluttered up with uninteresting hypotheses that
conditioning events like $B$ have nonzero probability.  To avoid the
clutter, we will \emph{implicitly} assume from now on that all such
events have nonzero probability.

Pure probability is often counterintuitive, but conditional
probability is even worse!  Conditioning can subtly alter
probabilities and produce unexpected results in randomized algorithms
and computer systems as well as in betting games.  In fact, the
solution to the above problem appears to contradict what we already
know about Monty Hall!  Yet, the mathematical definition of
conditional probability given above is very simple and should give you
no trouble---provided that you rely on mathematical reasoning and not
naive intuition.

\subsection{What went wrong}

So if everything at the beginning of this chapter was mathematically
sound, why does it seem to contradict the results that we established
in Chapter~\ref{probability_chap}?  The problem is a common one:
\emph{we chose the wrong thing to condition on}.  In our initial
description of the scenario, we learned the location of the goat when
Carol opened door B.  But when we defined our condition as ``the
contestant opens A and the goat is behind B,'' we included $(A, A,
C)$, an outcome in which Carol opens door C!  The correct conditional
probability should have been ``what are the odds of winning by
switching given the contestant chooses door A and Carol opens door
B.''  By misstating our condition, we inadvertently included
extraneous outcomes in our calculation.  With the correct
conditioning, we still win by switching 1/9 of the time in case $(C,
A, B)$, but the total probability is smaller:
\[
\pr{(A, A, B)} + \pr{(C, A, B)} = \frac{1}{18} + \frac{1}{9} = \frac{3}{18}.
\]
The conditional probability would then be:
\[
\frac{1/9}{3/18} = \frac{2}{3},
\]
which is exactly what we expected!


%\floating\
\textbox{
\textboxheader{O.J. Simpson} %\label{OJ_NY_Times}

In an opinion article\footnote{\arm{INSERT REF}} in the \emph{New York
  Times}, Steven Strogatz points to the O.J.  Simpson trial as an
example of poor choice of conditions.  O.J.  Simpson was a retired
football player who was accused, and later acquitted, of the murder of
his wife, Nicole Brown Simpson.  The trial was widely publicized and
called the ``trial of the century''; racial tensions, allegations of
police misconduct, and new-at-the-time DNA evidence captured the
public's attention.  But Strogatz, citing mathemetician and author
I.J. Good, focuses on a less well-known aspect of the case: whether
O.J.'s history of abuse towards his wife was admissible into evidence.

The prosecution argued that abuse is often a precursor to murder,
pointing to statistics indicating that an abuser was as much as ten
times more likely to commit murder than was a random indidual.  The
defense, however, countered with statistics indicating that the odds
of an abusive husband murdering his wife were ``infinitesimal,''
roughly 1 in 2500.  Based on those numbers, the actual relevance of a
history of abuse to a murder case would appear limited at best---and
therefore, according to the defense, introducing that history would
make the jury hate Simpson but would lack any probitive value, and its
discussion should be barred as prejudicial.

In other words, both the defense and the prosecution were arguing
conditional probability: the likelihood that a woman will be murdered
by her husband, given that her husband abuses her.  But both defense
and prosecution omitted a vital piece of data from their calculations:
Nicole Brown Simpson \emph{was} murdered.  Strogatz points out that
based on the defense's numbers and the crime statistics of the time,
the probability that a woman was murdered by her abuser, given that
she was abused \emph{and} murdered, is around 80\%!  The calculation
is spelled out in detail in Problem~\eqref{TBA}.

But the real point we wanted to make is that conditional probability
is used and misused all the time, and even experts subject to
widespread scrutiny can make mistakes.  }


\begin{editingnotes}

\arm{EM: This section is deprecated. It's possible that something similar
to thisshould be added to give an excuse to illustrate this section
with Venn diagrams, but it doesn't fit the currect chapter as written.}

Suppose that we pick a random person in the world.  Everyone has an
equal chance of being selected.  Let $A$ be the event that the person
is an MIT student, and let $B$ be the event that the person lives in
Cambridge.  What are the probabilities of these events?  Intuitively,
we're picking a random point in the big ellipse shown in
Figure~\ref{fig:15B1} and asking how likely that point is to fall into
region $A$ or $B$.

\begin{figure}[h]

\graphic{cambridge-conditional}

\caption{Selecting a random person.  $A$ is the event that the person
  is an MIT student.  $B$ is the event that the person lives in
  Cambridge.}

\label{fig:15B1}

\end{figure}

The vast majority of people in the world neither live in Cambridge nor
are MIT students, so events $A$ and $B$ both have low probability.
But what about the probability that a person is an MIT student,
\emph{given} that the person lives in Cambridge?  This should be
much greater---but what is it exactly?

What we're asking for is called a \term{conditional
  probability}---that is, the probability that one event happens,
given that some other event definitely happens.  Questions about
conditional probabilities come up all the time:
%
\begin{itemize}
\item What is the probability that it will rain this afternoon, given
that it is cloudy this morning?
\item What is the probability that two rolled dice sum to 10, given
that both are odd?
\item What is the probability that I'll get four-of-a-kind in Texas No
Limit Hold 'Em Poker, given that I'm initially dealt two queens?
\end{itemize}
\end{editingnotes}

\section{The Four-Step Method for Conditional Probability}
%: Best of Three Tournament}

\iffalse
The \emph{Halting Problem} was the first example of a property that
could not be tested by any program.  It was introduced by Alan Turing
in his seminal 1936 paper.  The problem is to determine whether a
Turing machine halts on a given \dots yadda yadda yadda \dots more
importantly, it was \emph{the name of the MIT EECS department's famed
  C-league hockey team}.
\fi

In a best-of-three tournament, the local C-league hockey team wins the
first game with probability $1/2$.  In subsequent games, their
probability of winning is determined by the outcome of the previous
game.  If the local team won the previous game, then they are
invigorated by victory and win the current game with probability
$2/3$.  If they lost the previous game, then they are demoralized by
defeat and win the current game with probability only $1/3$.  What is
the probability that the local team wins the tournament, given that
they win the first game?

This is a question about a conditional probability.  Let $A$ be the
event that the local team wins the tournament, and let $B$ be the
event that they win the first game.  Our goal is then to determine the
conditional probability $\prcond{A}{B}$.

We can tackle conditional probability questions just like ordinary
probability problems: using a tree diagram and the four step method.
A complete tree diagram is shown in Figure~\ref{fig:15B2}.

\begin{figure}[h]

\graphic{hockey}

\caption{The tree diagram for computing the probability that the
  local team wins two out of three games given that they won
  the first game.}

\label{fig:15B2}

\end{figure}

\paragraph{Step 1:  Find the Sample Space}

Each internal vertex in the tree diagram has two children, one
corresponding to a win for the local team (labeled~$W$) and one
corresponding to a loss (labeled~$L$).  The complete sample space is:
%
\[
    \sspace = \set{ WW, \, WLW,\, WLL,\, LWW,\, LWL,\, LL }.
\]

\paragraph{Step 2:  Define Events of Interest}

The event that the local team wins the whole tournament is:
%
\[
    T = \set{WW,\, WLW,\, LWW}.
\]
%
And the event that the local team wins the first game is:
%
\[
    F = \set{WW,\, WLW,\, WLL }.
\]
%
The outcomes in these events are indicated with check marks in the tree
diagram in Figure~\ref{fig:15B2}.

\paragraph{Step 3:  Determine Outcome Probabilities}

Next, we must assign a probability to each outcome.  We begin by
labeling edges as specified in the problem statement.  Specifically,
The local team has a $1/2$ chance of winning the first game, so
the two edges leaving the root are each assigned probability $1/2$.
Other edges are labeled $1/3$ or $2/3$ based on the outcome of the
preceding game.  We then find the probability of each outcome by
multiplying all probabilities along the corresponding root-to-leaf
path.  For example, the probability of outcome $WLL$ is:
%
\[
    \frac{1}{2} \cdot \frac{1}{3} \cdot \frac{2}{3} = \frac{1}{9}.
\]

\paragraph{Step 4: Compute Event Probabilities}

We can now compute the probability that the local team wins the
tournament, given that they win the first game:
%
\begingroup
\openup2pt
\begin{align*}
\prcond{A}{B}
    & = \frac{\pr{A \intersect B}}{\pr{B}} \\
    & = \frac{\pr{\set{WW, WLW}}}{\pr{\set{WW, WLW, WLL}}} \\
    & = \frac{1/3 + 1/18}{1/3 + 1/18 + 1/9} \\
    & = \frac{7}{9}.
\end{align*}
\endgroup
%
We're done!  If the local team wins the first game, then they win
the whole tournament with probability $7 / 9$.

\section{Why Tree Diagrams Work}\label{product_rule_subsec}

We've now settled into a routine of solving probability problems using
tree diagrams.  But we've left a big question unaddressed: what is the
mathematical justification behind those funny little pictures?  Why do
they work?

The answer involves conditional probabilities.  In fact, the
probabilities that we've been recording on the edges of tree diagrams
\emph{are} conditional probabilities.  For example, consider the
uppermost path in the tree diagram for the hockey team problem, which
corresponds to the outcome $WW$.  The first edge is labeled $1/2$,
which is the probability that the local team wins the first game.  The
second edge is labeled $2 / 3$, which is the probability that the
local team wins the second game, \emph{given} that they won the
first---that's a conditional probability!  More generally, on each
edge of a tree diagram, we record the probability that the experiment
proceeds along that path, given that it reaches the parent vertex.

So we've been using conditional probabilities all along.  But why can
we multiply edge probabilities to get outcome probabilities?  For
example, we concluded that:
%
\begin{equation*}
\pr{WW} = \frac{1}{2} \cdot \frac{2}{3}
	= \frac{1}{3}.
\end{equation*}
%
Why is this correct?

The answer goes back to Definition~\ref{LN12:prcond} of conditional probability
which could be written in a form called the \term{Product Rule} for
conditional probabilities:
%
\begin{rul*}[Conditional Probability Product Rule: 2 Events]
%If $\pr{E_1} \neq 0$, then:
%
\[
    \pr{E_1 \intersect E_2} = \pr{E_1} \cdot \prcond{E_2}{E_1}.
\]
\end{rul*}
Multiplying edge probabilities in a tree diagram amounts to evaluating
the right side of this equation.  For example:
\begin{align*}
\lefteqn{\pr{\text{win first game} \intersect \text{win second game}}}
		\hspace{0.5in} \\[2pt]
	& = \pr{\text{win first game}} \cdot
            \prcond{\text{win second game}}{\text{win first game}} \\[2pt]
	& = \frac{1}{2} \cdot \frac{2}{3}.
\end{align*}
So the Conditional Probability Product Rule is the formal
justification for multiplying edge probabilities to get outcome
probabilities.

To justify multiplying edge probabilities along a path of length three, we need
a rule for three events:
\begin{rul*}[Conditional Probability Product Rule: 3 Events]
%If $\pr{E_1} \neq 0$, then:
%
\[
    \pr{E_1 \intersect E_2 \intersect E_3} = \pr{E_1} \cdot
    \prcond{E_2}{E_1} \cdot
\prcond{E_3}{E_1 \intersect E_2}.
\]
\end{rul*}
Of course there is an $n$-event version of the Rule.  This is given in
Problem~\ref{PS_conditional_probability_product_rule}, but its form
should be clear from the three event version.

\subsection{Probability of Size-$k$ Subsets}
As a simple application of the product rule for conditional
probabilities, we can use the rule to calculate the number of size-$k$
subsets of the integers $[1,n]$.  Of course we already this number
isnn $\binom{n}{k}$, but now the rule will give us new derivation of
the formula for $\binom{n}{k}$.

So suppose we choose a size-$k$ subset at random, with all subsets of
$[1,n]$ equally likely to be chosen.  Now if $p$ is the probability of
randomly choosing a given size-$k$ subset, $S$, then $1/p$ equals the number of
such subsets.  So what's $p$?  Well, the probability
that the \emph{smallest} number in the random set is one of the $k$
numbers in $S$ is $k/n$.  Then, \emph{given} that the smallest number
in the random set is in $S$, the probability that the \emph{second}
smallest number in the random set is one of the remaining $k-1$
elements in $S$ is $(k-1)/(n-1)$.  So by the product rule, the
probability that the \emph{two} smallest numbers in the random set are
both in $S$ is
\[
\frac{k}{n} \cdot \frac{k-1}{n-1}\, .
\]
Next, given that the two smallest numbers in the random set are in
$S$, the probability that the third smallest number is one of the
$k-2$ remaining elements in $S$ is $(k-2)/(n-2)$.  So by the product
rule, the probability that the \emph{three} smallest numbers in the
random set are all in $S$ is
\[
\frac{k}{n} \cdot \frac{k-1}{n-1} \cdot \frac{k-2}{n-2}.
\]
Continuing in this way, it follows that the probability that
\emph{all} $k$ elements in the randomly chosen set are in $S$ is
\begin{align*}
p & = \frac{k}{n} \cdot \frac{k-1}{n-1} \cdot \frac{k-2}{n-2}
              \cdots \frac{k-(k-1)}{n-(k-1)}\\
  & = \frac{k \cdot (k-1) \cdot (k-1) \cdots 1}%
           {n \cdot (n-1) \cdot (n-2) \cdots (n-(k-1))}\\
  & = \frac{k!}{n!/(n-k)!}\\
  & = \frac{k!(n-k)!}{n!}.
\end{align*}
So we have again shown the number of size-$k$ subsets of $[1,n]$,
namely $1/p$, is
\[
\frac{n!}{k!(n-k)!}.
\]


\section{Medical Testing}\label{med_test-subsection}

Breast cancer is a deadly disease that claims thousands of lives every
year.  Early detection and accurate diagnosis are high priorities, and
routine mammograms are one of the first lines of defense.  They're not
very accurate as far as medical tests go, but they are correct between
90\% and 95\% of the time, which seems pretty good for a relatively
inexpensive non-invasive test.\footnote{For illustrative purposes, we
  are rounding or simplifying many of the statistics in this example.}
\begin{editingnotes}
  To solve this same problem
  with more accurate statistics, try Problem??
\end{editingnotes}
However, mammogram results are also an example of conditional
probabilities having counterintuitive consequences.  If the test was
positive for breast cancer in you or a loved one, and the test is
better than 90\% accurate, you'd naturally expect that to mean there
is better than 90\% chance that the disease was present. But a
mathematical analysis belies that gut instinct.  Let's start by
precisely defining how accurate a mammogram is:
\begin{itemize}

\item If you have the condition, there is a 10\% chance that the test
  will say you do not have it.  This is called a ``false negative.''

\item If you do not have the condition, there is a 5\% chance that the
  test will say you do.  This is a ``false positive.''n

\end{itemize}

\subsection{Four Steps Again}

Now suppose that we are testing middle-aged women with no family
history of cancer.  Among this cohort, incidence of breast cancer
rounds up to about 1\%.

\paragraph{Step 1: Find the Sample Space}

The sample space is found with the tree diagram in
Figure~\ref{fig:15C1}.
\begin{editingnotes}
need a new tree diagram here.
\end{editingnotes}
\begin{figure}[h]

\graphic{BO}

\caption{The tree diagram for a breast cancer test.}

\label{fig:15C1}

\end{figure}

Notice that the test gives the correct answer with probability $0.9405
+ 0.009 = 0.9495$.

\paragraph{Step 2: Define Events of Interest}

Let $A$ be the event that the person has breast cancer.  Let $B$ be the
event that the test was positive.  The outcomes in each event are marked
in the tree diagram.  We want to find $\prcond{A}{B}$, the probability
that a person has breast cancer, given that the test was positive.

\paragraph{Step 3: Find Outcome Probabilities}

First, we assign probabilities to edges.  These probabilities are
drawn directly from the problem statement.  By the Product Rule, the
probability of an outcome is the product of the probabilities on the
corresponding root-to-leaf path.  All probabilities are shown in
Figure~\ref{fig:15C1}.

\paragraph{Step 4: Compute Event Probabilities}

From Definition~\ref{LN12:prcond}, we have
\begin{equation*}
\prcond{A}{B}	= \frac{\pr{A \intersect B}}{\pr{B}} %\\[2pt]
		= \frac{0.009}{0.009 + 0.4905} %\\[2pt]
		\approx 18\%.
\end{equation*}
So, if the test is positive, then there is an 82\% chance that the
result is incorrect---even though the test is 90\% accurate!  In that
sense, it seems that even a 95\% accurate test doesn't tell us much.  Can we
do better?  Well, there is a simple way to make the test correct 99\%
of the time: always return a negative result!  This ``test'' gives the
right answer for all healthy people and the wrong answer only for the
1\% that actually have cancer.


\begin{editingnotes}
There is a similar ``paradox'' in weather forecasting.  During winter,
almost all days in Boston are wet and overcast.  Predicting miserable
weather every day may be more accurate than really trying to get it
right!
\end{editingnotes}


\begin{editingnotes}
\textbf{Eli:} This needs a ``this just goes to show'' lesson here, but I'm not
actually sure what that lesson should be.
\end{editingnotes}

\subsection{Natural Frequencies}

That 18\% result may initially be surprising, but it makes sense with
a little thought.  There are two ways you could test positive: first,
it could be that the patient has the condition and the test is
correct; second, it could be that the patient is healthy and the test
is incorrect.  But almost everyone is healthy!  The number of healthy
individuals is so large that even the mere 5\% with false positive
results overwhelm the number of genuinely positive results from the
truly ill.

Thinking in terms of these ``natural frequencies'' can be a useful
tool for interpreting some of the strange seeming results coming from
those formulas.  For example, let's take a closer look at the
mammogram example.

Imagine 10,000 women in our demographic.  Based on the frequency of
the disease, we'd expect 100 of them to have breast cancer.  Of those,
90 would have a positve result.  The remaining 9,900 woman are healthy,
but 5\% of them---500, give or take---will show a false positive on
the mammogram.  That gives us 90 real positives out of a little fewer
than 600 positives.  That 82\% error rate isn't so so surprising after all.

\begin{editingnotes}
Add a box or problem with an example where this test would be useful.
For example, if you have only enough medicine to treat half the
population, and you randomly choose people to treat, then the
probability that you treat a sick person will be 1/2.  But if you
first treat all the people who test positive, the probability that a
sick person will be treated will be nearly 0.9.
\end{editingnotes}


\section{\emph{A Posteriori} Probabilities}\label{aposteriori_subsec}

If you think about it too much, the medical testing problem we just
considered could start to trouble you.  You may wonder if a statement
like ``If you tested positive, then you have the condition with
probability~18\%'' makes sense, since at the time you take the test
you either have the condition or you don't.

\begin{editingnotes}
ARM: THIS IS A LITTLE SLOPPY; REVISE:
\end{editingnotes}

In fact, such a statement does make sense.  It means that 18\% of the
people who test positive actually have the condition.  It is true that
any particular person has it or they don't, but a \emph{randomly
  selected} person among those who test positive will have the
condition with probability~18\%.

But what does this~18\% probability tell you if you \emph{personally}
got a positive result?  Should you be relieved that there is less than
one chance on five that you have the disease?  Should you worry that
there is nearly one chance in five that you do have the disease?
Should you start treatment just in case?  Should you get more tests?

These are crucial practical questions, but it is important to
understand that they are not \emph{mathematical} questions.  Rather,
these are questions about statistical judgements and the philosophical
meaning of probability.  A thoughtful examination of such issues would
take us well beyond our topic of the basic mathematical theory of
probability.

\begin{editingnotes}
Cite a source that discusses the philosophy of statistics.
\end{editingnotes}

Anyway, if the medical testing example bothers you, you will
definitely be worried by the following examples, which go even further
down this path.

\subsection{The Hockey Team in Reverse}

Suppose that we turn the hockey question around: what is the
probability that the local C-league hockey team won their first game,
given that they won the series?

This seems like an absurd question!  After all, if the team won the
series, then the winner of the first game has already been determined.
Therefore, who won the first game is a question of fact, not a
question of probability.  However, our mathematical theory of
probability contains no notion of one event preceding another---there
is no notion of time at all.  Therefore, from a mathematical
perspective, this is a perfectly valid question.  And this is also a
meaningful question from a practical perspective.  Suppose that you're
told that the local team won the series, but not told the results
of individual games.  Then, from your perspective, it makes perfect
sense to wonder how likely it is that local team won the first game.

A conditional probability $\prcond{B}{A}$ is called  \term{a
posteriori} if event $B$ precedes event $A$ in time.  Here are some
other examples of a posteriori probabilities:
%
\begin{itemize}
\item The probability it was cloudy this morning, given that it rained
in the afternoon.
\item The probability that I was initially dealt two queens in Texas
No Limit Hold 'Em poker, given that I eventually got four-of-a-kind.
\end{itemize}
%
Mathematically, a posteriori probabilities are \emph{no different}
from ordinary probabilities; the distinction is only at a higher,
philosophical level.  Our only reason for drawing attention to them is
to say, ``Don't let them rattle you.''

Let's return to the original problem.  The probability that the
local team won their first game, given that they won the series
is $\prcond{B}{A}$.  We can compute this using the definition of
conditional probability and the tree diagram in Figure~\ref{fig:15B2}:
%
\begin{align*}
\prcond{B}{A}  = \frac{\pr{B \intersect A}}{\pr{A}} %\\[2pt]
               = \frac{1/3 + 1/18}{1/3 + 1/18 + 1/9} %\\[2pt]
               = \frac{7}{9}.
\end{align*}

This answer is suspicious!  In the preceding section, we showed that
$\prcond{A}{B}$ was also $7/9$.  Could it be true that $\prcond{A}{B}
= \prcond{B}{A}$ in general?  Some reflection suggests this is
unlikely.  For example, the probability that I feel uneasy, given that
I was abducted by aliens, is pretty large.  But the probability that I
was abducted by aliens, given that I feel uneasy, is rather small.

Let's work out the general conditions under which $\prcond{A}{B} =
\prcond{B}{A}$.  By the definition of conditional probability, this
equation holds if an only if:
%
\[
\frac{\pr{A \intersect B}}{\pr{B}} = \frac{\pr{A \intersect B}}{\pr{A}}
\]
%
This equation, in turn, holds only if the denominators are equal or
the numerator is~0; namely if
%
\[
\pr{B} = \pr{A}
\hspace{0.25in} \text{or} \hspace{0.25in}
\pr{A \intersect B} = 0.
\]
%
The former condition holds in the hockey example; the probability that
the local team wins the series (event~$A$) is equal to the
probability that it wins the first game (event~$B$) since both
probabilities are~$1/2$.

In general, such pairs of probabilities are related by \idx{Bayes'
  Rule}:
%
\begin{theorem}[Bayes' Rule]
%If $\pr{A}$ and $\pr{B}$ are nonzero, then:
%
\begin{equation}\label{bayesrule}
    \prcond{B}{A} = \frac{\prcond{A}{B} \cdot \pr{B}}{\pr{A}}
\end{equation}
\end{theorem}

\begin{proof}
%When $\pr{A}$ and $\pr{B}$ are nonzero,
We have
\[
\prcond{A}{B} \cdot \pr{B} = \prob{A \intersect B} = \prcond{B}{A} \cdot \pr{A}
\]
by definition of conditional probability.  Dividing by $\prob{A}$
gives~\eqref{bayesrule}.
\end{proof}

\iffalse

Next, let's look at a problem that even bothers us.

\subsection{A Coin Problem}

Suppose that someone hands you either a fair coin or a trick coin with
heads on both sides.  You flip the coin 100 times and see heads every
time.  What can you say about the probability that you flipped the
fair coin?  Remarkably, nothing!

In order to make sense out of this outrageous claim, let's formalize
the problem.  The sample space is worked out in the tree diagram shown
in Figure~\ref{fig:15C2}.  We do not know the probability~$p$ that you
were handed the fair coin initially---you were just given one coin or
the other.
%
\begin{figure}[h]

\graphic{trick-coin}

\caption{The tree diagram for the coin-flipping problem.}

\label{fig:15C2}

\end{figure}
%
Let $A$ be the event that you were handed the fair coin, and let $B$
be the event that you flipped 100 straight heads.  We're looking
for $\prcond{A}{B}$, the probability that you were handed the fair
coin, given that you flipped 100 heads.  The outcome probabilities are
worked out in Figure~\ref{fig:15C2}.  Plugging the results into the
definition of conditional probability gives:
%
\begin{align*}
\prcond{A}{B}	& = \frac{\pr{A \intersect B}}{\pr{B}} \\[2pt]
		& = \frac{p / 2^{100}}{1 - p + p / 2^{100}} \\[2pt]
		& = \frac{p}{2^{100} (1 - p) + p}.
\end{align*}
%
This expression is very small for moderate values of $p$ because of
the $2^{100}$ term in the denominator.  For example, if $p = 1/2$,
then the probability that you were given the fair coin is essentially
zero.

But we \emph{do not know} the probability $p$ that you were given
the fair coin.  And perhaps the value of $p$ is \emph{not} moderate;
in fact, maybe $p = 1 - 2^{-100}$.  Then there is nearly an even
chance that you have the fair coin, given that you flipped 100 heads.
In fact, maybe you were handed the fair coin with probability $p = 1$.
Then the probability that you were given the fair coin is, well,~1!

Of course, it is extremely unlikely that you would flip 100 straight
heads, but in this case, that is a given from the assumption of the
conditional probability.  And so if you really did see 100 straight
heads, it would be very tempting to also assume that $p$~is not close
to~1 and hence that you are very likely to have flipped the trick
coin.

We will encounter a very similar issue when we look at methods for
estimation by sampling in Section~\ref{sec:sampling}.
\fi

%\subsection{Conditional Identities}

\section{The Law of Total Probability}\label{sec:total_probability}

Breaking a probability calculation into cases simplifies many
problems.  The idea is to calculate the probability of an event $A$ by
splitting into two cases based on whether or not another event $E$
occurs.  That is, calculate the probability of $A\nobreak
\intersect\nobreak E$ and $A \intersect \setcomp{E}$.  By the Sum
Rule, the sum of these probabilities equals $\pr{A}$.  Expressing the
intersection probabilities as conditional probabilities yields:
\begin{rul}[Law of Total Probability, single event]\label{total_prob_Ebar}
%If $\prob{E}$ and $\prob{\setcomp{E}}$~are nonzero, then
\[
\pr{A} = \prcond{A}{E} \cdot \pr{E} +
         \prcond{A}{\setcomp{E}} \cdot \pr{\setcomp{E}}.
\]
\end{rul}

For example, suppose we conduct the following experiment.  First, we
flip a fair coin.  If heads comes up, then we roll one die and take the
result.  If tails comes up, then we roll two dice and take the sum of
the two results.  What is the probability that this process yields a
2?  Let $E$ be the event that the coin comes up heads, and let $A$ be
the event that we get a 2 overall.  Assuming that the coin is fair,
$\pr{E} = \pr{\setcomp{E}} = 1/2$.  There are now two cases. If we
flip heads, then we roll a 2 on a single die with probability
$\prcond{A}{E} = 1/6$.  On the other hand, if we flip tails, then we
get a sum of 2 on two dice with probability
$\prcond{A}{\setcomp{E}} = 1/36$.  Therefore, the probability that
the whole process yields a 2 is
\[
\pr{A} = \frac{1}{2} \cdot \frac{1}{6} + \frac{1}{2} \cdot \frac{1}{36} =
  \frac{7}{72}.
\]

\begin{editingnotes}
Illustrate with 3 events and put $n$ event version in a problem:
\end{editingnotes}

There is also a form of the rule to handle more than two cases.
\begin{rul}[Law of Total Probability]
If $E_1, \dots, E_n$ are disjoint events whose union is the whole
sample space, then:
\begin{equation}\label{multitotalprob}
\pr{A} = \sum_{i=1}^{n} \prcond{A}{E_i} \cdot \pr{E_i}.
\end{equation}
\end{rul}

\begin{editingnotes}
Again, illustrate with 3 events and put $n$ event version in a problem:
\end{editingnotes}

This Law is the basis for a multi-event version of Bayes' Rule which
we state for the record.  Here the conditional probability of event
$E_1$ given event $A$ is given in terms of the ``inverse'' conditional
probabilities of $A$ given events $E_1,E_2,\dots,E_n$:

\begin{theorem}[Multi-event Bayes' Rule]
If $E_1, \dots, E_n$ are disjoint events whose union is the whole
sample space, then
\begin{equation}\label{multibayesrule}
\prcond{E_1}{A} = \frac{\prcond{A}{E_1}\cdot \pr{E_1}}{\sum_{i=1}^{n}
  \prcond{A}{E_i} \cdot \pr{E_i}}
\end{equation}
\end{theorem}


\subsection{Conditioning on a Single Event}\label{cond_ident_subsec}

The probability rules that we derived in Section~\ref{sec:union_bound}
extend to probabilities conditioned on the same event.  For example,
the Inclusion-Exclusion formula for two sets holds when all
probabilities are conditioned on an event $C$:
\[
\prcond{A \union B}{C} = \prcond{A}{C} + \prcond{B}{C} - \prcond{A \intersect B}{C}.
\]
This is easy to verify by plugging in the Definition~\ref{LN12:prcond}
of conditional
probability.\footnote{Problem~\ref{PS_conditional_space} explains why
  this and similar conditional identities follow on general principles
  from the corresponding unconditional identities.}

\iffalse
%This follows from the fact that if $\pr{C} \neq 0$, then
Namely,
\begin{align*}
\prcond{A \union B}{C}
    &= \frac{\pr{(A \union B) \intersect C}}{\pr{C}} \\[3pt]
    &= \frac{\pr{(A \intersect C) \union (B \intersect C)}}{\pr{C}} \\[3pt]
    &= \frac{\pr{A \intersect C} + \pr{B \intersect C}
             - \pr{A \intersect B \intersect C}}
            {\pr{C}} \\[3pt]
    &= \prcond{A}{C} + \prcond{B}{C} - \prcond{A \intersect B}{C}.
\end{align*}
\fi

It is important not to mix up events before and after the conditioning
bar.  For example, the following is \emph{not} a valid identity:
%
\begin{falseclm*}
\begin{equation}\label{LN12:fc}
\prcond{A}{B \union C} = \prcond{A}{B} + \prcond{A}{C} - \prcond{A}{B \intersect C}.
\end{equation}
\end{falseclm*}

A simple counter-example is to let $B$ and $C$ be events over a
uniform space with most of their outcomes in $A$, but not overlapping.
This ensures that $\prcond{A}{B}$ and $\prcond{A}{C}$ are both close
to 1.  For example,
\begin{align*}
B & \eqdef [0,9],\\
C & \eqdef [10,18] \union \set{0},\\
A & \eqdef [1,18],
\end{align*}
so
\[
\prcond{A}{B} = \frac{9}{10} = \prcond{A}{C}.
\]
Also, since 0 is the only outcome in $B \intersect C$ and $0 \notin
A$, we have
\[
\prcond{A}{B \intersect C} = 0
\]
So the right hand side of~\eqref{LN12:fc} is 1.8, while the left hand
side is a probability which can be at most 1---actually, it is 18/19.

\iffalse

A counterexample is shown in Figure~\ref{fig:15D2}.  In this case,
$\prcond{A}{B} = 1/2$, $\prcond{A}{C} = 1/2$, $\prcond{A}{B \intersect
  C} = 1$, and $\prcond{A}{B \union C} = 1/3$.  However, since
$1/3 \ne 1/2 + 1/2 - 1$, equation~\eqref{LN12:fc} does not hold.
%
\begin{figure}

\graphic{cx19}

\caption{A counterexample to equation~\eqref{LN12:fc}.  Event~$A$ is
  the dark-bordered rectangle, event~$B$ is the rectangle with
  vertical stripes, and event~$C$ is the rectangle with horizontal
  stripes.  $B \intersect C$ lies entirely within~$A$ while $B - C$
  and $C - B$ are entirely outside of~$A$.}

\label{fig:15D2}

\end{figure}
\fi


\section{Discrimination Lawsuit}\label{discrimination_subsec}

In the early 70's, a famous university was sued for gender
discrimination.  The basis of this suit was evidence that, at first
glance, appeared definitive: in 1973, 44\% of male applicants to the
school's graduate programs were accepted, but only 35\% of female
applicants were admitted.

However, statisticians working for the defense made in interesting
discovery.  Analyzing the individual departments and programs at the
school, they not only found that few showed any signs of bias, but
also showed that among the few departments that \emph{did} show
statistical irregularities, most were slanted \emph{in favor of women}.
This suggests that if there was any sex discrimination, then it was
against men!  

Given the discrepancy in these findings, it feels like someone must be
doing bad math---intentionally or otherwise.  But it actually makes
perfect sense.  In fact, this statistical hiccup is common enough to
merit its own name: ``Simpson's Paradox.'' To explain, let's first
clarify the problem by expressing both arguments in terms of
conditional probabilities.  To simplify matters, suppose that there
are only two departments, EE and CS, and consider the experiment where
we pick a random candidate.  Define the following events:
%
\begin{itemize}
\item $A \eqdef$ the candidate is admitted to his or her program of choice,
\item $F_{EE} \eqdef$ the candidate is a woman applying to the EE department,
\item $F_{CS} \eqdef$ the candidate is a woman applying to the CS department,
\item $M_{EE} \eqdef$ the candidate is a man applying to the EE department,
\item $M_{CS} \eqdef$ the candidate is a man applying to the CS department.
\end{itemize}
Assume that all candidates are either men or women, and that no
candidate belongs to both departments.  That is, the events $F_{EE}$,
$F_{CS}$, $M_{EE}$, and $M_{CS}$ are all disjoint.

In these terms, the plaintiff is making the following argument:
\[
    \prcond{A}{M_{EE} \union M_{CS}} > \prcond{A}{F_{EE} \union F_{CS}}.
\]
That is, across the university, the total probability that a woman candidate
is admitted is less than the probability for a man.

The university retorts that \emph{in any given department}, a woman
candidate has chances \emph{equal to or greater} than those of a male
candidate; more formally, that
\begin{align*}
\prcond{A}{M_{EE}} & \leq \prcond{A}{F_{EE}} \quad\text{and}\\
\prcond{A}{M_{CS}} & \leq \prcond{A}{F_{CS}}.
\end{align*}

It is easy to believe that these two positions are contradictory.  But
there is no contradiction, and in fact, Table~\ref{fig:15D3} shows a
set of candidate statistics for which the assertions of both the
plaintiff and the university hold.  In this case, a higher percentage
of female applicants were admitted to each department, but overall a
higher percentage of males were accepted!  How do we make sense of
this?

\begin{table}

\begin{tabular}{crr}
CS & 2 men admitted out of 5 candidates      &   40\% \\
   & 50 women admitted out of 100 candidates     &  50\% \\
EE & 70 men admitted out of 100 candidates   &  70\% \\
   & 4 women admitted out of 5 candidates         & 80\% \\
\hline
Overall & 72 men granted tenure, 105 candidates & $\approx 69\%$ \\
        & 54 women granted tenure, 105 candidates   & $\approx 51\%$
\end{tabular}

\caption{A scenario in which men are overall more likely than women to
  be admitted to a school, despite being less likely to be admitted
  into any given program.}

\label{fig:15D3}

\end{table}

At first we, and the plaintiffs, assumed that the overall admissions
statistics for the university could only be explained by
discrimination.  However, the department-by-department breakdown shows
that the cause of the discrepancy has to do with the CS department
being both more exclusive and more attractive to women than the
permissive EE department\footnote{At the university in the actual
  lawsuit, the ``exclusive'' departments more popular among women were
  those that did not require a mathematical foundation, such as
  English and Education.  Women's choice of these careers was
  certainly influence by gender discrimination, but on the whole of society's
  part, not the university's in particular}.  This leads us strongly to the
conclusion that the admissions gap in not due to any systematic bias
on the school's part.

But suppose we replaced ``the candidate is a man/woman applying to the
EE department,'' by ``the candidate is a man/woman for whom an
admissions decision was made during an odd-numbered day of the
amonth,'' and likewise with CS and an even-numbered day of the month.
Since we don't think the parity of a date is a cause for the outcome
of an admission decision, we would most likely dismiss the
``coincidence'' that on both odd and even dates, women are more
frequently admitted.  Instead we would judge, based on the overall
data showing women less likely to be admitted, that gender bias
against women \emph{was} an issue in the university.

Bear in mind that it would be the \emph{same numerical data} that we
would be using to justify our different conclusions in the
department-by-department case and the even-day-odd-day case.  We
interpreted the same numbers differently based on our implicit causal
beliefs, namely, that departments mattered and date parity did not.
It is circular to claim that the data corroborated our beliefs that
there is or is not discrimination, since our interpretation of the
data correlation depends on our beliefs about the causes of
admission.\footnote{These issues are thoughtfully examined in
  \emph{Causality: Models, Reasoning and Inference}, Judea Pearl,
  Cambridge U. Press, 2001.}  This example highlights a basic
principle in statistics which people constantly ignore: \emph{never
  assume that correlation implies causation}.

\begin{editingnotes}
ALTERNATIVE MADE-UP VERSION:

Several years ago there was a sex discrimination lawsuit against a
famous university.  A woman math professor was denied tenure,
allegedly because she was a woman.  She argued that in every one of
the university's 22 departments, the percentage of men candidates
granted tenure was greater than the percentage of women candidates
granted tenure.  This sounds very suspicious!

However, the university's lawyers argued that across the university as
a whole, the percentage of male candidates granted tenure was actually
\emph{lower} than the percentage for women candidates.  This suggests
that if there was any sex discrimination, then it was against men!
Surely, at least one party in the dispute must be lying.

Let's clarify the problem by expressing both arguments in terms of
conditional probabilities.  To simplify matters, suppose that there
are only two departments, EE and CS, and consider the experiment where
we pick a random candidate.  Define the following events:
%
\begin{itemize}
\item $A \eqdef$ the candidate is granted tenure,
\item $F_{EE} \eqdef$ the candidate is a woman in the EE department,
\item $F_{CS} \eqdef$ the candidate is a woman in the CS department,
\item $M_{EE} \eqdef$ the candidate is a man in the EE department,
\item $M_{CS} \eqdef$ the candidate is a man in the CS department.
\end{itemize}

Assume that all candidates are either men or women, and that no
candidate belongs to both departments.  That is, the events $F_{EE}$,
$F_{CS}$, $M_{EE}$, and $M_{CS}$ are all disjoint.

In these terms, the plaintiff is making the following argument:
%
\begin{align*}
\prcond{A}{F_{EE}} & < \prcond{A}{M_{EE}} \quad\text{and}\\
\prcond{A}{F_{CS}} & < \prcond{A}{M_{CS}}.
\end{align*}
That is, in both departments, the probability that a woman candidate
is granted tenure is less than the probability for a man.

The university retorts that \emph{overall}, a woman candidate is
\emph{more} likely to be granted tenure than a man; namely that
\[
    \prcond{A}{F_{EE} \union F_{CS}} > \prcond{A}{M_{EE} \union M_{CS}}.
\]

It is easy to believe that these two positions are contradictory, and
the phenomenon illustrated here is widely referred to as ``Simpson's
Paradox.''  But there is no contradiction or paradox, and in fact,
Table~\ref{fig:15D3} shows a set of candidate statistics for which the
assertions of both the plaintiff and the university hold.  In this
case, a higher percentage of men candidates were granted tenure in
each department, but overall a higher percentage of women candidates
were granted tenure!  How do we make sense of this?

\begin{table}

\begin{tabular}{crr}
CS & 0 women granted tenure, 1 candidates      &   0\% \\
   & 50 men granted tenure, 100 candidates     &  50\% \\
EE & 70 women granted tenure, 100 candidates   &  70\% \\
   & 1 man granted tenure, 1 candidates         & 100\% \\
\hline
Overall & 70 women granted tenure, 101 candidates & $\approx 70\%$ \\
        & 51 men granted tenure, 101 candidates   & $\approx 51\%$
\end{tabular}

\caption{A scenario where women are less likely to be granted tenure
  than men in each department, but more likely to be granted tenure
  overall.}

\label{fig:15D3}

\end{table}

With data like this showing that at the department level, women
candidates were less likely to be granted tenure than men, university
administrators would likely see an indication of bias against women,
and the departments would be directed to reexamine their tenure
procedures.

But suppose we replaced ``the candidate is a man/woman in the EE
department,'' by ``the candidate is a man/woman for whom a tenure
decision was made during an odd-numbered day of the month,'' and
likewise with CS and an even-numbered day of the month.  Since we
don't think the parity of a date is a cause for the outcome of a
tenure decision, we would ignore the ``coincidence'' that on both odd
and even dates, men are more frequently granted tenure.  Instead, we
would judge, based on the overall data showing women more likely to be
granted tenure, that gender bias against women was \emph{not} an issue
in the university.

The point is that it's the \emph{same data} that we interpret
differently based on our implicit causal beliefs.  It would be
circular to claim that the gender correlation observed in the data
corroborates our belief that there is discrimination, since our
interpretation of the data correlation \emph{depends} on our beliefs
about the causes of tenure decisions.\footnote{These issues are
  thoughtfully examined in \emph{Causality: Models, Reasoning and
    Inference}, Judea Pearl, Cambridge U. Press, 2001.}  This
illustrates a basic principle in statistics which people constantly
ignore: \emph{never assume that correlation implies causation}.

\end{editingnotes}


%% Conditional Probability Problems %%%%%%%%%%%%%%%%%%%%%%%%%%%%%%%%%%%%%%%%%%%

\begin{problems}
\practiceproblems
\pinput{TP_six_shooter_probability}
\pinput{TP_bayes_proof}

\classproblems
\pinput{CP_missing_card_probability}
\pinput{PS_conditional_aces}
\pinput{CP_conditional_prob_says_so_bug}
\pinput{FP_skywalker_prob_lin_recur_gen_func}
\pinput{FP_directed_graphs_and_probability}

\homeworkproblems
\pinput{PS_levitating_LAs}
\pinput{PS_conditional_probability_problem_errors}
\pinput{PS_coin_flip_sequences}
\pinput{PS_13_card_hand}
\pinput{PS_conditional_probability_product_rule}
\pinput{PS_conditional_space}

\examproblems
\pinput{FP_monty_hall_variant}
\pinput{FP_conditional_prob_inequality}
\pinput{MQ_conditional_prob_inequality}
\pinput{FP_conditional_beaver_fever}
\pinput{FP_red_and_blue_goats}
\pinput{FP_neighborhood_census}
\pinput{MQ_voldemort_returns}
\end{problems}

\section{Independence}
Suppose that we flip two fair coins simultaneously on opposite sides
of a room.  Intuitively, the way one coin lands does not affect the
way the other coin lands.  The mathematical concept that captures
this intuition is called \term{independence}.
\begin{definition}\label{def:independence}
An event with probability 0 is defined to be independent of every
event (including itself).  If $\pr{B} \neq 0$, then
event $A$ is independent of event $B$ iff
\begin{equation}\label{eqn:independence}
    \prcond{A}{B} = \pr{A}.
\end{equation}
\end{definition}
In other words, $A$ and~$B$ are independent if knowing that $B$
happens does not alter the probability that $A$~happens, as is the
case with flipping two coins on opposite sides of a room.

\subsubsection*{Potential Pitfall}

Students sometimes get the idea that disjoint events are independent.
The \emph{opposite} is true: if $A \intersect B = \emptyset$, then
knowing that $A$ happens means you know that $B$ does not happen.  So
disjoint events are \emph{never} independent---unless one of them has
probability zero.

\subsection{Alternative Formulation}

Sometimes it is useful to express independence in an alternate form
which follows immediately from Definition~\ref{def:independence}:

\begin{theorem}\label{thm:16A1}
$A$ is independent of~$B$ if and only if
\begin{equation}\label{eqn:15D3}
    \pr{A \intersect B} = \pr{A} \cdot \pr{B}.
\end{equation}
\end{theorem}

Notice that Theorem~\ref{thm:16A1} makes apparent the symmetry between
$A$ being independent of $B$ and $B$ being independent of $A$:
\begin{corollary}
$A$ is independent of $B$ iff $B$ is independent of $A$.
\end{corollary}


\iffalse

\begin{proof}
There are two cases to consider depending on whether or not $\prob{B} =
0$.
\begin{description}

\item[Case 1 $(\prob{B} = 0)$:]
If $\prob{B} = 0$, $A$ and~$B$ are independent by
Definition~\ref{def:independence}.  In addition,
equation~\eqref{eqn:15D3} holds since both sides are~0.  Hence, the
theorem is true in this case.

\item[Case 2 $(\prob{B} > 0)$:]
By Definition~\ref{LN12:prcond},
\begin{equation*}
    \prob{A \cap B} = \prcond{A}{B} \prob{B}.
\end{equation*}
So equation~\eqref{eqn:15D3} holds if
\begin{equation*}
    \prcond{A}{B} = \prob{A},
\end{equation*}
which, by Definition~\ref{def:independence}, is true iff $A$ and~$B$
are independent.  Hence, the theorem is true in this case as well.
\qedhere
\end{description}
\end{proof}
\fi

\subsection{Independence Is an Assumption}

Generally, independence is something that you \emph{assume} in
modeling a phenomenon.  For example, consider the experiment of
flipping two fair coins.  Let $A$~be the event that the first coin
comes up heads, and let $B$~be the event that the second coin is
heads.  If we assume that $A$ and~$B$ are independent, then the
probability that both coins come up heads is:
%
\begin{equation*}
\pr{A \intersect B}  = \pr{A} \cdot \pr{B} %\\[2pt]
               = \frac{1}{2} \cdot \frac{1}{2} %\\[2pt]
               = \frac{1}{4}.
\end{equation*}

In this example, the assumption of independence is reasonable.  The
result of one coin toss should have negligible impact on the outcome
of the other coin toss.  And if we were to repeat the experiment many
times, we would be likely to have~$A \cap B$ about~1/4 of the time.

There are, of course, many examples of events where assuming
independence is \emph{not} justified.  For example, an hourly weather
forecast for a clear day might list a 10\% chance of rain every hour
from noon to midnight, meaning each hour has a 90\% chance of being
dry.  Does that mean the odds of a precipitation-free day are only
$0.9^{12} \approx 0.28$?  Of course not.  If it's still dry at 5pm,
the odds are higher that 90\% that 6pm will still be dry---and if it's
pouring at 5pm, the chances are much higher than 10\% that it will
still be raining an hour later.


\iffalse
let $C$~be the
event that tomorrow is cloudy and $R$ be the event that tomorrow is
rainy.  Perhaps $\pr{C} = 1/5$ and $\pr{R} = 1/10$ in Boston.  If
these events were independent, then we could conclude that the
probability of a rainy, cloudy day was quite small:
%
\begin{equation*}
\pr{R \intersect C} = \pr{R} \cdot \pr{C} % \\[2pt]
               = \frac{1}{5} \cdot \frac{1}{10} % \\[2pt]
               = \frac{1}{50}.
\end{equation*}
%
Unfortunately, these events are definitely not independent; in
particular, every rainy day is cloudy.  Thus, the probability of a
rainy, cloudy day is actually~$1/10$.
\fi

Deciding when to \emph{assume} that events are independent is a tricky
business.  In practice, there are strong motivations to assume
independence since many useful formulas (such as
equation~\eqref{eqn:15D3}) only hold if the events are independent.
But you need to be careful:
\iffalse
 lest you end up deriving false conclusions.
\fi
we'll describe several famous examples where (false) assumptions of
independence led to trouble.
\iffalse
 over the next several chapters
\fi
This problem gets even trickier when there are more than two events in
play.

\section{Mutual Independence}

%\subsection{Definition}

We have defined what it means for two events to be independent.  What
if there are more than two events?  For example, how can we say that
the flips of $n$~coins are all independent of one another?  A set of
events is said to be \term{mutually independent} if the probability of
each event in the set is the same no matter which of the other events
has occurred.  This is equivalent to saying that for any selection of
two or more of the events, the probability that all the selected
events occur equals the product of the probabilities of the selected
events.

For example, four events~$E_1, E_2, E_3, E_4$ are mutually
independent if and only if all of the following equations hold:
%
\begin{align*}
\pr{E_1 \intersect E_2}
    & = \pr{E_1} \cdot \pr{E_2}
\\
\pr{E_1 \intersect E_3}
    & = \pr{E_1} \cdot \pr{E_3}
\\
\pr{E_1 \intersect E_4}
    & = \pr{E_1} \cdot \pr{E_4}
\\
\pr{E_2 \intersect E_3}
    & = \pr{E_2} \cdot \pr{E_3}
\\
\pr{E_2 \intersect E_4}
    & = \pr{E_2} \cdot \pr{E_4}
\\
\pr{E_3 \intersect E_4}
    & = \pr{E_3} \cdot \pr{E_4}
 \\
\pr{E_1 \intersect E_2 \intersect E_3}
    & = \pr{E_1} \cdot \pr{E_2} \cdot \pr{E_3}
\\
\pr{E_1 \intersect E_2 \intersect E_4}
    & = \pr{E_1} \cdot \pr{E_2} \cdot \pr{E_4}
\\
\pr{E_1 \intersect E_3 \intersect E_4}
    & = \pr{E_1} \cdot \pr{E_3} \cdot \pr{E_4}
\\
\pr{E_2 \intersect E_3 \intersect E_4}
    & = \pr{E_2} \cdot \pr{E_3} \cdot \pr{E_4}
 \\
\pr{E_1 \intersect E_2 \intersect E_3 \intersect E_4} & = \pr{E_1} \cdot \pr{E_2} \cdot \pr{E_3} \cdot \pr{E_4}
\end{align*}

The generalization to mutual independence of $n$ events should now be clear.

\begin{editingnotes}
MAKE INTO A PROBLEM:

We could formalize this with conditional probabilities
as in Definition~\ref{def:independence}, but we'll jump directly to
the cleaner definition based on products of probabilities as in
Theorem~\ref{thm:16A1}:

\begin{definition}\label{def:mutual_independence}
A set of events~$E_1, E_2, \dots, E_n$, is \term{mutually independent}
if $\forall i \in [1, n]$ and $\forall S \subseteq [1, n] - \set{i}$,
either
\begin{equation*}
    \Prob{\bigcap_{j \in S} E_j} = 0
\quad
\text{or}
\quad
    \prob{E_i} = \prcond{E_i}{\bigcap_{j \in S} E_j}.
\end{equation*}
\end{definition}

\subsection{Alternative Formulation}

Just as Theorem~\ref{thm:16A1} provided an alternative definition of
independence for two events, there is an alternative definition for
mutual independence.

%\begin{theorem}\label{thm:16A2}

\begin{definition}\label{def:mutual_indep}
A set of events~$E_1, E_2, \dots, E_n$ is mutually independent iff
for all subsets $S \subseteq [1, n]$,
\begin{equation*}
    \Prob{\bigcap_{j \in S} E_j} = \prod_{j \in S} \prob{E_j}.
\end{equation*}
\end{definition}

%\end{theorem}
\iffalse
The proof of Theorem~\ref{thm:16A2} uses induction and reasoning
similar to the proof of Theorem~\ref{thm:16A1}.  We will not include
the details here.
\fi

Definition~\ref{def:mutual_indep} says that $E_1, E_2, \dots, E_n$~are
mutually independent if and only if all of the following equations
hold for all distinct $i$, $j$, $k$, and~$l$:
%
\begin{align*}
\pr{E_i \intersect E_j}
    & = \pr{E_i} \cdot \pr{E_j}
%    & \text{for all distinct $i$, $j$}
 \\
\pr{E_i \intersect E_j \intersect E_k}
    & = \pr{E_i} \cdot \pr{E_j} \cdot \pr{E_k}
%     & \text{for all distinct $i$, $j$, $k$}
 \\
\pr{E_i \intersect E_j \intersect E_k \intersect E_l}
    & = \pr{E_i} \cdot \pr{E_j} \cdot \pr{E_k} \cdot \pr{E_l}
%    & \text{for all distinct $i$, $j$, $k$, $l$}
 \\
    & \XasWideAsY{\vdots}{${}={}$} \\
\pr{E_1 \intersect \cdots \intersect E_n} & = \pr{E_1} \cdots \pr{E_n}.
\end{align*}

For example, if we toss $n$~fair coins, the tosses are mutually
independent iff for every subset of~$m$~coins, the probability that
every coin in the subset comes up heads is~$2^{-m}$.
\end{editingnotes}

\subsection{DNA Testing}

Assumptions about independence are routinely made in practice.
Frequently, such assumptions are quite reasonable.  Sometimes,
however, the reasonableness of an independence assumption is not so
clear, and the consequences of a faulty assumption can be severe.

Let's return to the O. J. Simpson murder trial.  The following expert
testimony was given on May 15, 1995:
\begin{description}

\item[Mr. Clarke:] When you make these estimations of frequency---and
I believe you touched a little bit on a concept called independence?

\item[Dr. Cotton:] Yes, I did.

\item[Mr. Clarke:] And what is that again?

\item[Dr. Cotton:] It means whether or not you inherit one allele that
you have is not---does not affect the second allele that you might
get.  That is, if you inherit a band at 5,000 base pairs, that doesn't
mean you'll automatically or with some probability inherit one at
6,000.  What you inherit from one parent is what you inherit from the
other.

\item[Mr. Clarke:] Why is that important?

\item[Dr. Cotton:] Mathematically that's important because if that
were not the case, it would be improper to multiply the frequencies
between the different genetic locations.

\item[Mr. Clarke:] How do you---well, first of all, are these markers
independent that you've described in your testing in this case?

\end{description}

Presumably, this dialogue was as confusing to you as it was for the
jury.  Essentially, the jury was told that genetic markers in blood
found at the crime scene matched Simpson's.  Furthermore, they were
told that the probability that the markers would be found in a
randomly-selected person was at most 1 in 170 million.  This
astronomical figure was derived from statistics such as:
%
\begin{itemize}
\item 1 person in 100 has marker $A$.
\item 1 person in 50 marker $B$.
\item 1 person in 40 has marker $C$.
\item 1 person in 5 has marker $D$.
\item 1 person in 170 has marker $E$.
\end{itemize}
%
Then these numbers were multiplied to give the probability that a
randomly-selected person would have all five markers:
\begin{align*}
\pr{A \intersect B \intersect C \intersect D \intersect E}
    & = \pr{A} \cdot \pr{B} \cdot \pr{C} \cdot \pr{D} \cdot \pr{E}\\
    & = \frac{1}{100} \cdot \frac{1}{50} \cdot \frac{1}{40}
                     \cdot \frac{1}{5} \cdot \frac{1}{170}
     = \frac{1}{170{,}000{,}000}.
\end{align*}

\iffalse
\begin{align*}
\pr{A \intersect B \intersect C \intersect D \intersect E}
    & = \pr{A} \cdot \pr{B} \cdot \pr{C} \cdot \pr{D} \cdot \pr{E} \\[2pt]
    & = \frac{1}{100} \cdot \frac{1}{50} \cdot \frac{1}{40}
                      \cdot \frac{1}{5} \cdot \frac{1}{170} \\[2pt]
    & = \frac{1}{170{,}000{,}000}.
\end{align*}
\fi
%
The defense pointed out that this assumes that the markers appear
mutually independently.  Furthermore, all the statistics were based on
just a few hundred blood samples.  

After the trial, the jury was widely mocked for failing to
``understand'' the DNA evidence.  If you were a juror, would
\emph{you} accept the 1 in 170 million calculation?


\subsection{Pairwise Independence}

The definition of mutual independence seems awfully
complicated---there are so many selections of events to consider!
Here's an example that illustrates the subtlety of independence when
more than two events are involved.  Suppose that we flip three fair,
mutually-independent coins.  Define the following events:
%
\begin{itemize}
\item $A_1$ is the event that coin 1 matches coin 2.
\item $A_2$ is the event that coin 2 matches coin 3.
\item $A_3$ is the event that coin 3 matches coin 1.
\end{itemize}
%
Are $A_1$, $A_2$, $A_3$ mutually independent?

The sample space for this experiment is:
%
\[
    \set{HHH,\, HHT,\, HTH,\, HTT,\, THH,\, THT,\, TTH,\, TTT}.
\]
%
Every outcome has probability $(1/2)^3 = 1/8$ by our assumption that
the coins are mutually independent.

To see if events $A_1$, $A_2$, and $A_3$ are mutually independent, we
must check a sequence of equalities.  It will be helpful first to
compute the probability of each event $A_i$:
%
\begin{align*}
\pr{A_1} & = \pr{HHH} + \pr{HHT} + \pr{TTH} + \pr{TTT} \\[2pt]
         & = \frac{1}{8} + \frac{1}{8} + \frac{1}{8} + \frac{1}{8}%\\[2pt]
          = \frac{1}{2}.
\end{align*}
%
By symmetry, $\pr{A_2} = \pr{A_3} = 1/2$ as well.  Now we can begin
checking all the equalities required for mutual independence:
%in Definition~\ref{def:mutual_indep}:
\begin{align*}
\pr{A_1 \intersect A_2}
       & = \pr{HHH} + \pr{TTT}
         = \frac{1}{8} + \frac{1}{8}
         = \frac{1}{4}
         = \frac{1}{2} \cdot \frac{1}{2}\\
       & = \pr{A_1} \pr{A_2}.
\end{align*}

\iffalse
\begin{align*}
\pr{A_1 \intersect A_2}
	& = \pr{HHH} + \pr{TTT} \\[2pt]
        & = \frac{1}{8} + \frac{1}{8} \\[2pt]
        & = \frac{1}{4} \\[2pt]
        & = \frac{1}{2} \cdot \frac{1}{2}\\[2pt]
        & = \pr{A_1} \pr{A_2}.
\end{align*}\fi

By symmetry, $\pr{A_1 \intersect A_3} = \pr{A_1} \cdot \pr{A_3}$ and
$\pr{A_2 \intersect A_3} = \pr{A_2} \cdot \pr{A_3}$ must hold also.
Finally, we must check one last condition:

\begin{align*}
\pr{A_1 \intersect A_2 \intersect A_3}
        & = \pr{HHH} + \pr{TTT}
          = \frac{1}{8} + \frac{1}{8}
          = \frac{1}{4}\\
        & \textcolor{red}{\mathbf{\neq}} \frac{1}{8} = \pr{A_1} \pr{A_2} \pr{A_3}.
\end{align*}


\iffalse
\begin{align*}
\pr{A_1 \intersect A_2 \intersect A_3}      & = \pr{HHH} + \pr{TTT} \\[2pt]
                                & = \frac{1}{8} + \frac{1}{8} \\[2pt]
                                & = \frac{1}{4} \\[2pt]
                                & \neq \pr{A_1} \pr{A_2} \pr{A_3} = \frac{1}{8}.
\end{align*}
\fi
%
The three events $A_1$, $A_2$, and~$A_3$ are not mutually independent
even though any two of them are independent!  This not-quite mutual
independence seems weird at first, but it happens.  It even
generalizes:

\begin{definition}\label{kway_independent_events}
  A set $A_1$, $A_2$, \dots, of events is \term{$k$-way independent}
  iff every set of $k$ of these events is mutually independent.  The
  set is \term{pairwise independent} iff it is 2-way independent.
\end{definition}

So the sets $A_1$, $A_2$, $A_3$ above are pairwise independent, but
not mutually independent.  Pairwise independence is a much weaker
property than mutual independence.

For example, suppose that the prosecutors in the O.~J. Simpson trial
were wrong and markers $A$, $B$, $C$, $D$, and $E$ appear only
\emph{pairwise} independently.  Then the probability that a
randomly-selected person has all five markers is no more than:
%
\begin{align*}
\pr{A \intersect B \intersect C \intersect D \intersect E}
    & \leq \pr{A \intersect E} = \pr{A} \cdot \pr{E}\\
    & = \frac{1}{100} \cdot \frac{1}{170} = \frac{1}{17{,}000}.
\end{align*}
%
The first line uses the fact that $A \intersect B \intersect C \intersect
D \intersect E$ is a subset of $A \intersect E$.  (We picked out the $A$
and $E$ markers because they're the rarest.)  We use pairwise independence
on the second line.  Now the probability of a random match is 1 in
17,000---a far cry from 1 in 170 million!  And this is the strongest
conclusion we can reach assuming only pairwise independence.

On the other hand, the 1 in 17,000 bound that we get by assuming
pairwise independence is a lot better than the bound that we would
have if there were no independence at all.  For example, if the
markers are dependent, then it is possible that
\begin{quote}
everyone with marker~$E$ has marker~$A$,

everyone with marker~$A$ has marker~$B$,

everyone with marker~$B$ has marker~$C$, and

everyone with marker~$C$ has marker~$D$.
\end{quote}
In such a scenario, the probability of a match is
\begin{equation*}
    \pr{E} = \frac{1}{170}.
\end{equation*}

So a stronger independence assumption leads to a smaller bound on the
probability of a match.  The trick is to figure out what independence
assumption is reasonable.  Assuming that the markers are
\emph{mutually} independent may well \emph{not} be reasonable unless
you have examined hundreds of millions of blood samples.  Otherwise,
how would you know that marker~$D$ does not show up more frequently
whenever the other four markers are simultaneously present?

\iffalse
%this section is moved to Probability and to Deviation

\section{The Birthday Principle}\label{birthday_principle_sec}

There are 95 students in a class.  What is the probability that some
birthday is shared by two people?  Comparing 95 students to the 365
possible birthdays, you might guess the probability lies somewhere
around $1/4$---but you'd be wrong: the probability that there will be
two people in the class with matching birthdays is actually more than
$0.9999$.

To work this out, we'll assume that the probability that a randomly
chosen student has a given birthday is $1/d$, where $d= 365$ in this
case.  We'll also assume that a class is composed of $n$ randomly and
independently selected students, with $n=95$ in this case.  These
randomness assumptions are not really true, since more babies are born
at certain times of year, and students' class selections are typically
not independent of each other, but simplifying in this way gives us a
start on analyzing the problem.  More importantly, these assumptions
are justifiable in important computer science applications of birthday
matching.  For example, the birthday matching is a good model for
collisions between items randomly inserted into a hash table.  So we
won't worry about things like Spring procreation preferences that make
January birthdays more common, or about twins' preferences to take
classes together (or not).  \begin{editingnotes}
or that fact that a student
can't be selected twice in making up a class list.
\end{editingnotes}

Selecting a sequence of $n$ students for a class yields a sequence of
$n$ birthdays.  Under the assumptions above, the $d^n$ possible
birthday sequences are equally likely outcomes.  Let's examine the
consequences of this probability model by focussing on the $i$th and
$j$th elements in a birthday sequence, where $1 \leq i \neq j \leq n$.
It makes for a better story if we refer to the $i$th birthday as
``Alice's'' and the $j$th as ``Bob's.''

\subsection{Pairwise Independent Matches}
Now if Alice, Bob, Carol, and Don are four different people, then
whether Alice and Bob have matching birthdays is independent of
whether Carol and Don do.  What's more interesting is that whether
Alice and \emph{Carol} have the same birthday is independent of
whether Alice and Bob do.  This follows because Carol is as likely to
have the same birthday as Alice, independently of whatever birthdays
Alice and Bob happen to have; a formal proof of this claim appears in
Problem~\ref{PS_equal_birthdays}.  In short, the set of all events
that a couple has matching birthdays is \index{pairwise independent}
\emph{pairwise} independent, even for overlapping couples.  This will
be important in Chapter~\ref{deviation_chap} because pairwise
independence will be enough to justify some conclusions about the
expected number of matches.  However, these matching birthday events
are obviously \emph{not} even 3-way independent: if Alice and Bob
match, and also Alice and Carol match, then Bob and Carol will match.

\iffalse
We could justify all these assertions of independence using the four
step method, but it's pretty boring, and we'll skip it.
\fi

But as long as the number of students is noticeably smaller than the
number of possible birthdays, we can get a pretty good estimate of the
birthday matching probabilities by \emph{pretending} that the matching
events are mutually independent.  (An intuitive justification for this
is that with only a small number of matching pairs, it's likely that
none of the pairs overlap.)  Then the probability of \emph{no}
matching birthdays would be the same as the $r$th power of the
probability that a couple does \emph{not} have matching birthdays,
where $r \eqdef \binom{n}{2}$ is the number of couples.  That is, the
probability of no matching birthdays would be
\begin{equation}\label{11dbinn2}
(1-1/d)^{\binom{n}{2}}.
\end{equation}
Using the fact that $1+x < e^x$ for all $x$,\footnote{This
  approximation is obtained by truncating the Taylor series $e^{-x} =
  1 - x + x^2/2!  - x^3/3! + \cdots$.  The approximation $e^{-x}
  \approx 1 - x$ is pretty accurate when $x$ is small.} we would conclude
that the probability of no matching birthdays is at most
\begin{equation}\label{bday-approx}
e^{-\binom{n}{2}/d}
\end{equation}
which actually is the same as the bound we get in the next section
using the exact formula for the probability.

\subsection{Exact Formula for Match Probability}
The matching birthday problem fits in here so far as a nice example
illustrating pairwise and mutual independence, but it's actually not
hard to justify the bound~\eqref{bday-approx} without any pretence of
independence.  Namely, there are $d (d - 1) (d - 2) \cdots (d - (n -
1))$ length $n$ sequences of distinct birthdays.  So the probability
that everyone has a different birthday is:
\begin{align*}
\lefteqn{\frac{d (d - 1) (d - 2) \cdots (d - (n - 1))}{d^n}}\\
   & = \frac{d}{d} \cdot \frac{d-1}{d} \cdot \frac{d-2}{d} \cdots \frac{d - (n - 1)}{d}\\
   & = \paren{1 - \frac{0}{d}}
             \paren{1 - \frac{1}{d}}
             \paren{1 - \frac{2}{d}}
             \cdots
             \paren{1 - \frac{n - 1}{d}}\\
   & < e^0 \cdot e^{-1/d} \cdot e^{-2/d} \cdots e^{-(n-1)/d} 
             & \text{(since $1+x < e^x$)} \\
   & = e^{-\paren{\sum_{i=1}^{n-1} i/d}}\\
   & = e^{-\paren{n(n-1)/2d}}\\
   & = \text{the bound~\eqref{bday-approx}}.
\end{align*}

For $n=95$ and $d = 365$, the value of~\eqref{bday-approx} is less
than $1/200,000$, which means the probability of having some pair of
matching birthdays actually is more than $1 - 1/200,000 > 0.99999$.  So
it would be pretty astonishing if there were no pair of students in
the class with matching birthdays.

For $d \leq n^2/2$, the probability of no match turns out to be
asymptotically equal to the upper bound~\eqref{bday-approx}.  For $d =
n^2/2$ in particular, the probability of no match is asymptotically
equal to $1/e$.  This leads to a rule of thumb which is useful in many
contexts in computer science:

\textbox{
\textboxtitle{The \index{birthday principle} Birthday Principle}

If there are $d$ days in a year and $\sqrt{2d}$ people in a
room, then the probability that two share a birthday is about 
$1 - 1/e \approx 0.632$.
}

For example, the Birthday Principle says that if you have $\sqrt{2
  \cdot 365} \approx 27$ people in a room, then the probability that
two share a birthday is about $0.632$.  The actual probability is
about $0.626$, so the approximation is quite good.

Among other applications, it implies that to use a hash function that
maps $n$ items into a hash table of size $d$, you can expect many
collisions unless $n^2$ is a small fraction of $d$.  The Birthday
Principle also famously comes into play as the basis of ``birthday
attacks'' that crack certain cryptographic systems.
\fi

\begin{problems}
%\practiceproblems
%\pinput{TP_Binomial_Board_Breaking}
%\pinput{TP_Practice_with_Bounds}
%\pinput{FP_random_sampling}

\examproblems
\pinput{FP_college_probability}
\pinput{FP_product_rule_and_independence}

\classproblems
\pinput{CP_mutual_independence}
\pinput{CP_three_fair_coins}

\homeworkproblems
\pinput{PS_bogus_discrimination_contradiction}
\pinput{FP_graph_logic_probability}

\end{problems}
