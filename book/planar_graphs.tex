\chapter{Planar Graphs}\label{planar_graphs_chap}

\section{Drawing Graphs in the Plane}

Suppose there are three dog houses and three human houses, as shown in
Figure~\ref{fig:5DP}.  Can you find a route from each dog house to
each human house such that no route crosses any other route?

\begin{figure}

\graphic{dog-houses}

\caption{Three dog houses and and three human houses.  Is there a
  route from each dog house to each human house so that no pair of
  routes cross each other?}
\label{fig:5DP}
\end{figure}

A similar question comes up about a little-known animal called a
\emph{quadrapus} that looks like an octopus with four stretchy arms
instead of eight.  If five quadrapi are resting on the sea floor, as shown
in Figure~\ref{fig:5DA}, can each quadrapus simultaneously shake hands
with every other in such a way that no arms cross?

\begin{figure}

\graphic{quadrapi}

\caption{Five quadrapi (4-armed creatures).}

\label{fig:5DA}

\end{figure}


\begin{editingnotes}
\textcolor{red}{rephrased by ARM 7/3/10:}
\end{editingnotes}

Both these puzzles can be understood as asking about drawing graphs in the
plane.  Replacing dogs and houses by nodes, the dog house puzzle can be
rephrased as asking whether there is a planar drawing of the graph with
six nodes and edges between each of the first three nodes and each of the
second three nodes.  This graph is called the \term{complete bipartite
  graph} \idx{$K_{3,3}$} and is shown in Figure~\ref{fig:nonplanar}.(a).
\begin{editingnotes}
Insert graphic and fix figure refs.
\end{editingnotes}
The quadrapi puzzle asks whether there is a planar drawing of the
\idx{complete graph} \idx{$K_5$} shown in
Figure~\ref{fig:nonplanar}.(b).

\begin{figure}

\subfloat[]{%
    \graphic{Fig_5_36a}
}
\qquad\qquad
\subfloat[]{%
    \graphic{Fig_5_36b}
}

\caption{$K_{3, 3}$ (a) and $K_5$ (b).  Can you redraw these graphs so
that no pairs of edges cross?}

\label{fig:nonplanar}

\end{figure}

In each case, the answer is, ``No ---but almost!''  In fact, if you
remove an edge from either of these graphs, then the resulting graph
\emph{can} be redrawn in the plane so that no edges cross, as shown in
Figure~\ref{fig:5DC}.


\begin{figure}

\subfloat[]{
    \graphic{Fig_5DC_a}
}
\qquad
\subfloat[]{
    \graphic{Fig_5DC_b}
}

\caption{Planar drawings of (a) $K_{3, 3}$ without $\edge{u}{v}$, and
  (b) $K_5$ without $\edge{u}{v}$.}
\label{fig:5DC}
\end{figure}

Planar drawings have applications in circuit layout and are helpful in
displaying graphical data such as program flow charts, organizational
charts, and scheduling conflicts.  For these applications, the goal is
to draw the graph in the plane with as few edge crossings as possible.
(See the box on the following page for one such example.)

\begin{figure}[p]\redrawntrue
\textbox{\textboxtitle{Steve Wozniak and a Planar Circuit Design}

\noindent
When wires are arranged on a surface, like a circuit board or
microchip, crossings require troublesome three-dimensional structures.
When Steve Wozniak designed the disk drive for the early Apple II
computer, he struggled mightily to achieve a nearly planar design
according to the following excerpt from \texttt{apple2history.org}
which in turn quotes \emph{Fire in the Valley} by Freiberger and
Swaine:

\begin{quotation}
\noindent For two weeks, he worked late each night to make a satisfactory
design.  When he was finished, he found that if he moved a connector he
could cut down on feedthroughs, making the board more reliable.  To make
that move, however, he had to start over in his design.  This time it only
took twenty hours. He then saw another feedthrough that could be
eliminated, and again started over on his design.  ``The final design was
generally recognized by computer engineers as brilliant and was by
engineering aesthetics beautiful.  Woz later said, 'It's something you can
only do if you're the engineer and the PC board layout person yourself.
That was an artistic layout.  The board has virtually no
feedthroughs.'
\end{quotation}
}
\end{figure}

\section{Definitions of Planar Graphs}\label{sec:recdef_planar}

We took the idea of a planar drawing for granted in the previous
section, but if we're going to \emph{prove} things about planar graphs, we
better have precise definitions.

\begin{definition}\label{def:planar_drawing}
A \term{drawing} of a graph assigns to each node a distinct point in
the plane and assigns to each edge a smooth curve in the plane whose
endpoints correspond to the nodes incident to the edge.  The drawing
is \index{planar drawing}\emph{planar} if none of the curves
cross themselves or other curves, namely, the only points that
appear more than once on any of the curves are the node points.  A
graph is \index{planar graph}\emph{planar} when it has a planar
drawing.
\end{definition}

Definition~\ref{def:planar_drawing} is precise but depends on
further concepts: ``smooth planar curves'' and ``points appearing more
than once'' on them.  We haven't defined these concepts ---we just
showed the simple picture in Figure~\ref{fig:5DC} and hoped you would
get the idea.

Pictures can be a great way to get a new idea across, but it is generally
not a good idea to use a picture to replace precise mathematics.  Relying
solely on pictures can sometimes lead to disaster ---or to bogus proofs,
anyway.  There is a long history of bogus proofs about planar graphs based
on misleading pictures.\iffalse
\footnote{The bogus proof of the
  4-Color Theorem for planar graphs is not the only example.  Mistakes
  creep in with statements like,
\begin{quote}
    As you can see from Figure~ABC, it must be that property~XYZ holds
    for all planar graphs.
\end{quote}}\fi

The bad news is that to prove things about planar graphs using the
planar drawings of Definition~\ref{def:planar_drawing}, we'd have to
take a chapter-long excursion into continuous mathematics just to
develop the needed concepts from plane geometry and point-set topology.
\begin{editingnotes} and this makes
  Definition~\ref{def:planar_drawing} troublesome.
\end{editingnotes}
The good news is that there is another way to define planar graphs that
uses only discrete mathematics.  In particular, we can define planar
graphs as a recursive data type.  In order to understand how it works, we
first need to understand the concept of a \emph{face} in a planar drawing.

\subsection{Faces}

The curves in a planar drawing divide up the plane into connected
regions called the \term{continuous faces}\footnote{Most texts drop
  the adjective \emph{continuous} from the definition of a face as a
  connected region.  We need the adjective to distinguish continuous
  faces from the \emph{discrete} faces we're about to define.} of the
drawing.  For example, the drawing in
Figure~\ref{fig:continuous-faces} has four continuous faces.  Face IV,
which extends off to infinity in all directions, is called the
\term{outside face}.

\begin{figure}

\graphic{continuous-faces}

\caption{A planar drawing with four continuous faces.}
\label{fig:continuous-faces}
\end{figure}

The vertices along the boundary of each continuous face in
Figure~\ref{fig:continuous-faces} form a cycle.  For example, labeling the
vertices as in Figure~\ref{fig:continuous-cycles}, the cycles for each of the face
boundaries can be described by the vertex sequences
\begin{equation}\label{eq:5DA}
abca \qquad abda \qquad bcdb \qquad acda.
\end{equation}
These four cycles correspond nicely to the four continuous faces in
Figure~\ref{fig:continuous-cycles} ---so nicely, in fact, that we can
identify each of the faces in Figure~\ref{fig:continuous-cycles} by
its cycle.  For example, the cycle $abca$ identifies
face~III\@.  The cycles in list~\ref{eq:5DA} are called the
\emph{discrete faces} of the graph in
Figure~\ref{fig:continuous-cycles}.  We use the term ``discrete''
since cycles in a graph are a discrete data type ---as opposed to a
region in the plane, which is a continuous data type.

\begin{figure}

\graphic{continuous-cycles}

\caption{The drawing with labeled vertices.}
\label{fig:continuous-cycles}
\end{figure}

Unfortunately, continuous faces in planar drawings are not always
bounded by cycles in the graph ---things can get a little more
complicated.  For example, the planar drawing in
Figure~\ref{fig:bridge} has what we will call a \emph{bridge}, namely,
a cut edge $\edge{c}{e}$.  The sequence of vertices along the boundary
of the outer region of the drawing is
\[
abcefgecda.
\]
This sequence defines a closed walk, but does not define a cycle since
the walk has two occurrences of the bridge $\edge{c}{e}$ and each of
its endpoints.

\begin{figure}

\graphic{edge-twice-same-face}

\caption{A planar drawing with a \emph{bridge}.}
\label{fig:bridge}
\end{figure}

The planar drawing in Figure~\ref{fig:dongle} illustrates another
complication.  This drawing has what we will call a \emph{dongle},
namely, the nodes $v$, $x$, $y$, and~$w$, and the edges incident to
them.  The sequence of vertices along the boundary
of the inner region is
\[
rstvxyxvwvtur.
\]
This sequence defines a closed walk, but once again does not define a
cycle because it has two occurrences of \emph{every} edge of the
dongle ---once ``coming'' and once ``going.''

\begin{figure}

\graphic{dongle-face}

\caption{A planar drawing with a \emph{dongle}.}
\label{fig:dongle}
\end{figure}

It turns out that bridges and dongles are the only complications, at
least for connected graphs.  In particular, every continuous face in a
planar drawing corresponds to a closed walk in the graph.  These
closed walks will be called the \emph{discrete faces} of the drawing,
and we'll define them next.

\subsection{A Recursive Definition for Planar Embeddings}

The association between the continuous faces of a planar drawing and
closed walks provides the discrete data type we can use instead of
continuous drawings.  We'll define a \term{planar embedding} of
\emph{connected} graph to be the set of closed walks that are its face
boundaries.  Since all we care about in a graph are the connections
between vertices ---not what a drawing of the graph actually looks
like ---planar embeddings are exactly what we need.

The question is how to define planar embeddings without appealing to
continuous drawings.  There is a simple way to do this based on the
idea that any continuous drawing can drawn step by step: 
\begin{itemize}
\item either draw a new point somewhere in the plane to represent a vertex,

\item or draw a curve between two vertex points that have already been
  laid down, making sure the new curve doesn't cross any of the
  previously drawn curves.
\end{itemize}

A new curve won't cross any other curves precisely when it stays
within one of the continuous faces.  Alternatively, a new curve won't
have to cross any other curves if it can go between the outer faces of
two different drawings.  So to be sure it's ok to draw a new curve, we
just need to check that its endpoints are on the boundary of the same
face, or that its endpoints are on the outer faces of different
drawings.  Of course drawing the new curve changes the faces slightly,
so the face boundaries will have to be updated once the new curve is
drawn.  This is the idea behind the following recursive definition.

\begin{definition}\label{def:embedding}%\label{embeddingdef}
A \term{planar embedding} of a \emph{connected} graph consists of a
nonempty set of closed walks of the graph called the \term{discrete
  faces} of the embedding.  Planar embeddings are defined recursively
as follows:

\inductioncase{Base case}: If $G$ is a graph consisting of a single
vertex, $v$, then a planar embedding of $G$ has one discrete face,
namely, the length zero closed walk, $v$.

\inductioncase{Constructor case} (split a face): Suppose $G$ is a
connected graph with a planar embedding, and suppose $a$ and $b$ are
distinct, nonadjacent vertices of $G$ that occur in some discrete
face, $\gamma$, of the planar embedding.  That is, $\gamma$ is a
closed walk of the form
\[
%a \dots b \cdots a.
%\catv{\alpha}{b}{\beta}
\gamma = \merge{\alpha}{\beta}
\]
where $\alpha$ is a walk from $a$ to $b$ and $\beta$ is a walk from
$b$ to $a$.  Then the graph obtained by adding the edge $\edge{a}{b}$
to the edges of $G$ has a planar embedding with the same discrete
faces as $G$, except that face $\gamma$ is replaced by the two
discrete faces\footnote{\label{C} There is a minor exception to this
  definition of embedding in the special case when $G$ is a line graph
  beginning with $a$ and ending with $b$.  In this case the cycles
  into which $\gamma$ splits are actually the same.  That's because
  adding edge $\edge{a}{b}$ creates a cycle that divides the plane
  into ``inner'' and ``outer'' continuous faces that are both bordered
  by this cycle.  In order to maintain the correspondence between
  continuous faces and discrete faces in this case, we define the two
  discrete faces of the embedding to be two ``copies'' of this same
  cycle.}
\begin{equation}\label{alphbetsplit}
%a\dots ba\quad \text{ and } \quad ab\cdots a,
\merge{\alpha}{\edge{b}{a}}
 \quad \text{ and } \quad \merge{\edge{a}{b}}{\beta}
\end{equation}
as illustrated in Figure~\ref{fig:face-splitting}.\footnote{Formally,
  merge is an operation on walks, not a walk and an edge, so
  in~\eqref{alphbetsplit}, we should have used a walk $(a\ \edge{a}{b}\ b)$
  instead of an edge $\edge{a}{b}$ and written
\[
%a\dots ab\cdots ba.
\merge{\alpha}{(b\ \edge{b}{a}\ a)}
 \quad \text{ and } \quad \merge{(a\ \edge{a}{b}\ b)}{\beta}
\]
}

\begin{figure}

\graphic{split-a-face}

\caption{The ``split a face'' case: $awxbyza$ splits into $awxba$ and $abyza$.}
\label{fig:face-splitting}
\end{figure}

\inductioncase{Constructor case} (add a bridge): Suppose $G$ and~$H$
are connected graphs with planar embeddings and disjoint sets of
vertices.  Let $\gamma$ be a discrete face of the embedding of $G$ and
suppose that $\gamma$ begins and ends at vertex $a$.
\iffalse
That is, $\gamma$ is of the form
\[
a\dots a.
\]
\fi

Similarly, let $\delta$ be a discrete face of the embedding of $H$
that begins and ends at vertex $b$.
\iffalse
So $\delta$ is of the form
\[
b\cdots b.
\]
\fi

Then the graph obtained by connecting $G$ and $H$ with a new edge,
$\edge{a}{b}$, has a planar embedding whose discrete faces are the
union of the discrete faces of $G$ and $H$, except that faces $\gamma$
and $\delta$ are replaced by one new face
\[
%a\dots ab\cdots ba.
\merge{\merge{\merge{\gamma}{\edge{a}{b}}}{\delta}}{\edge{b}{a}}.
\]

This is illustrated in Figure~\ref{fig:add-bridge}, where the vertex
sequences of the faces of $G$ and $H$ are:
\[
G: \set{ axyza,\; axya,\; ayza }
    \qquad H: \set{ btuvwb,\; btvwb,\; tuvt },
\]
and after adding the bridge $\edge{a}{b}$, there is a
single connected graph whose faces have the vertex sequences
\[
\set{ axyz{\color{blue}ab}tuvw{\color{blue}ba},\;
         axya,\; ayza,\; btvwb,\; tuvt }.
\]

\begin{figure}

\graphic{add-bridge}

\caption{The ``add a bridge'' case.}
\label{fig:add-bridge}
\end{figure}

\end{definition}

A \idx{bridge} is simply a cut edge, but in the context of planar
embeddings, the bridges are precisely the edges that occur \emph{twice
  on the same discrete face} ---as opposed to once on each of two
faces.  \idx{Dongles} are trees made of bridges; we only use dongles
in illustrations, so there's no need to define them more precisely.

\subsection{Does It Work?}

Yes!  In general, a graph is planar because it has a planar drawing
according to Definition~\ref{def:planar_drawing} if and only if each
of its connected components has a planar embedding as specified in
Definition~\ref{def:embedding}.  Of course we can't prove this without
an excursion into exactly the kind of continuous math that we're
trying to avoid.  But now that the recursive definition of planar
graphs is in place, we won't ever need to fall back on the continuous
stuff.  That's the good news.

The bad news is that Definition~\ref{def:embedding} is a lot more
technical than the intuitively simple notion of a drawing whose edges
don't cross.  In many cases it's easier to stick to the idea of planar
drawings and give proofs in those terms.  For example, it's obvious
that erasing edges from a planar drawing leaves a planar drawing.  On
the other hand, it's not at all obvious, though of course it is true,
that you can delete an edge from a planar embedding and still get a
planar embedding (see
Problem~\ref{PS_planar_graph_construction_order}).

In the hands of experts, and perhaps in your hands too with a little
more experience, proofs about planar graphs by appeal to drawings can
be convincing and reliable.  But given the long history of mistakes in
such proofs, it's safer to work from the precise definition of planar
embedding.  More generally, it's also important to see how the
abstract properties of curved drawings in the plane can be modelled
successfully using a discrete data type.

\subsection{Where Did the Outer Face Go?}

Every planar drawing has an immediately-recognizable outer face
---it's the one that goes to infinity in all directions.  But where is
the outer face in a planar embedding?

There isn't one!  That's because there really isn't any need to
distinguish one face from another.  In fact, a planar embedding could
be drawn with any given face on the outside.  An intuitive explanation
of this is to think of drawing the embedding on a \emph{sphere}
instead of the plane.  Then any face can be made the outside face by
``puncturing'' that face of the sphere, stretching the puncture hole
to a circle around the rest of the faces, and flattening the circular
drawing onto the plane.

So pictures that show different ``outside'' boundaries may actually be
illustrations of the same planar embedding.  For example, the two
embeddings shown in Figure~\ref{fig:5DE} are really the same ---check
it: they have the same boundary cycles.

\begin{figure}

\graphic{Fig_5DE}

\caption{Two illustrations of the same embedding.}
\label{fig:5DE}
\end{figure}

This is what justifies the ``add bridge'' case in
Definition~\ref{def:embedding}: whatever face is chosen in the
embeddings of each of the disjoint planar graphs, we can draw a
bridge between them without needing to cross any other edges in the
drawing, because we can assume the bridge connects two ``outer''
faces.

\begin{problems}
\practiceproblems
\pinput{TP_Faces_of_a_Planar_Embedding}

\end{problems}


\section{Euler's Formula}

The value of the recursive definition is that it provides a powerful
technique for proving properties of planar graphs, namely, structural
induction.  For example, we will now use
Definition~\ref{def:embedding} and structural induction to establish
one of the most basic properties of a connected planar graph, namely,
that the number of vertices and edges completely determines the number
of faces in every possible planar embedding of the graph.

\begin{theorem}[Euler's Formula\index{Euler!formula}]\label{thm:eulers_formula}
If a connected graph has a planar embedding, then
\begin{equation*}
    v - e + f = 2
\end{equation*}
where $v$ is the number of vertices, $e$ is the number of edges, and
$f$ is the number of faces.
\end{theorem}

For example, in Figure~\ref{fig:continuous-faces}, $v = 4$,
$e = 6$, and $f = 4$.  Sure enough, $4 - 6 + 4 = 2$, as Euler's
Formula claims.

\begin{proof}
The proof is by structural induction on the definition of planar
embeddings.  Let $P(\embed{E})$ be the proposition that $v - e + f = 2$ for an
embedding, $\embed{E}$.

\inductioncase{Base case} ($\embed{E}$ is the one-vertex planar
embedding): By definition, $v=1$, $e=0$, and $f=1$, and $1-0+1 =2$, so
$P(\embed{E})$ indeed holds.

\inductioncase{Constructor case} (split a face): Suppose $G$ is a
connected graph with a planar embedding, and suppose $a$ and $b$ are
distinct, nonadjacent vertices of $G$ that appear on some discrete
face, $\gamma= a \dots b \cdots a$, of the planar embedding.

Then the graph obtained by adding the edge $\edge{a}{b}$ to the edges of
$G$ has a planar embedding with one more face and one more edge than $G$.
So the quantity $v-e+f$ will remain the same for both graphs, and since by
structural induction this quantity is 2 for $G$'s embedding, it's also 2
for the embedding of $G$ with the added edge.  So $P$ holds for the
constructed embedding.

\inductioncase{Constructor case} (add bridge): Suppose $G$ and $H$ are
connected graphs with planar embeddings and disjoint sets of vertices.
Then connecting these two graphs with a bridge merges the two bridged
faces into a single face, and leaves all other faces unchanged.  So
the bridge operation yields a planar embedding of a connected graph
with $v_G +v_H$ vertices, $e_G + e_H +1$ edges, and $f_G + f_H - 1$
faces.  Since
\begin{align*}
\lefteqn{(v_G +v_H) - (e_G + e_H +1) + (f_G + f_H - 1)} \qquad\\
   & = (v_G  - e_G + f_G) + (v_H  - e_H  + f_H) -2\\
   & = (2)+(2)-2 \qquad \text{(by structural induction hypothesis)}\\
   & = 2,
\end{align*}
$v-e+f$ remains equal to~2 for the constructed embedding.  That is,
$P(\embed{E})$ also holds in this case.

This completes the proof of the constructor cases, and the theorem follows
by structural induction.
\end{proof}

\section{Bounding the Number of Edges in a Planar Graph}

Like Euler's formula, the following lemmas follow by structural induction
directly from Definition~\ref{def:embedding}.

\begin{lemma}\label{2e}
In a planar embedding of a connected graph, each edge occurs once in
each of two different faces, or occurs exactly twice in one face.
\end{lemma}

\begin{lemma}\label{3f}
  In a planar embedding of a connected graph with at least three vertices,
  each face is of length at least three.
\end{lemma}

Combining Lemmas~\ref{2e} and~\ref{3f} with Euler's Formula, we can
now prove that planar graphs have a limited number of edges:

\begin{theorem}\label{th:e3v}
  Suppose a connected planar graph has $v \geq 3$ vertices and $e$
  edges.  Then
\begin{equation}\label{eq:e3v}
    e \leq 3v-6.
\end{equation}
\end{theorem}

\begin{proof}
By definition, a connected graph is planar iff it has a planar
embedding.  So suppose a connected graph with $v$ vertices and $e$
edges has a planar embedding with $f$ faces.  By Lemma~\ref{2e}, every
edge has exactly two occurrences in the face boundaries.  So the sum
of the lengths of the face boundaries is exactly $2e$.  Also by
Lemma~\ref{3f}, when $v \geq 3$, each face boundary is of length at
least three, so this sum is at least $3f$.  This implies that
\begin{equation}\label{e3f}
3f \leq 2e.
\end{equation}
But $f = e-v+2$ by Euler's formula, and substituting into~\eqref{e3f} gives
\begin{align*}
3(e-v+2) & \leq 2e\\
e-3v + 6  & \leq 0\\
e & \leq 3v - 6 \qedhere
\end{align*}
\end{proof}

\section{Returning to $K_5$ and $K_{3,3}$}

Finally we have a simple way to answer the quadrapi question at the
beginning of this chapter: the five quadrapi can't all shake hands without
crossing.  The reason is that we know the quadrupi question is the same as
asking whether a complete graph $K_5$ is planar, and 
Theorem~\ref{th:e3v} has the immediate:
\begin{corollary}\label{k5not}
$K_5$ is not planar.
\end{corollary}
\begin{proof}
  $K_5$ is connected and has 5 vertices and 10 edges.  But since $10 > 3
  \cdot 5-6$, $K_5$ does not satisfy the inequality~\eqref{eq:e3v} that
  holds in all planar graphs.
\end{proof}

We can also use Euler's Formula to show that $K_{3, 3}$ is not
planar.  The proof is similar to that of Theorem~\ref{eq:e3v} except that
we use the additional fact that $K_{3, 3}$ is a bipartite graph.

\begin{editingnotes}
\textcolor{red}{CUT by FTL since proved in Theorem~\ref{thm:2-colorable-equiv}}

\begin{lemma*}\label{lem:5D5}
Every closed walk in a bipartite graph has even length.
\end{lemma*}

\begin{proof}
Any closed walk in a bipartite graph~$G$ must alternate between nodes
in~$\leftbi{G}$ and $\rightbi{G}$.  Since a closed walk ends on the
same node it started with, it must visit nodes in~$\leftbi{G}$
equally often as nodes in~$\rightbi{G}$.  Hence it must have even
length.
\end{proof}

\begin{corollary}\label{cor:5D6}
In a planar embedding of a connected \emph{bipartite} graph with at
least 3 vertices, each face has length at least~4.
\end{corollary}
\begin{proof}
  By Lemma~\ref{3f}, every face has length~3.  Since the graph is
  bipartite and since each face is a closed walk,
  Lemma~\ref{2color-iff-bip} and
  Theorem~\ref{thm:2-colorable-equiv}.\ref{has-odd-closed-walk} imply that
  no face can have length~3.  Hence, every face must actually have length
  at least~4.
\end{proof}
\end{editingnotes}

\begin{lemma}\label{lem:5D6}
In a planar embedding of a connected \idx{bipartite graph} with at
least 3 vertices, each face has length at least~4.
\end{lemma}

\begin{proof}
  By Lemma~\ref{3f}, every face of a planar embedding of the graph has
  length at least~3.  But by Lemma~\ref{2color-iff-bip} and
  Theorem~\ref{thm:2-colorable-equiv}.\ref{has-odd-closed-walk}, a
    bipartite graph can't have odd length closed walks.  Since the faces
    of a planar embedding are closed walks, there can't be any faces of
    length 3 in a bipartite embedding.  So every face must have length at
    least~4.
\end{proof}

\begin{theorem}\label{th:e2v}
Suppose a connected bipartite graph with $v \geq 3$ vertices and $e$ edges
is planar.  Then
\begin{equation}\label{eq:e2v}
    e \leq 2v-4.
\end{equation}
\end{theorem}

\begin{proof}
  Lemma~\ref{lem:5D6} implies that all the faces of an embedding of the
  graph have length at least 4.  Now arguing as in the proof of
  Theorem~\ref{th:e3v}, we find that the sum of the lengths of the face
  boundaries is exactly~$2e$ and at least~$4f$.  Hence,
\begin{equation}\label{4ele2e}
    4f \le 2e
\end{equation}
for any embedding of a planar bipartite graph.  By Euler's theorem,
$f=2-v+e$.  Substituting $2-v+e$ for $f$ in~\eqref{4ele2e}, we have
\[
4(2-v+e) \leq 2e,
\]
which simplies to~\eqref{eq:e2v}.
\end{proof}

\begin{corollary}\label{cor:K33-nonplanar} %\label{thm:K33-nonplanar}
$K_{3, 3}$ is not planar.
\end{corollary}

\begin{proof}
  $K_{3,3}$ is connected, bipartite and has 6 vertices and 9 edges.  But
  since $ 9 > 2 \cdot 6-4$, $K_{3,3}$ does not satisfy the
  inequality~\eqref{eq:e3v} that holds in all bipartite planar graphs.
\end{proof}

\begin{editingnotes}
\section{Planar Subgraphs}

If you draw a graph in the plane by repeatedly adding edges that don't
cross, you clearly could add the edges in any other order and still
wind up with the same drawing.  This is so basic that we might presume
that our recursively defined planar embeddings have this property.
But that wouldn't be fair: we really need to prove it.  After all, the
recursive definition of planar embedding was pretty technical ---maybe
we got it a little bit wrong, with the result that our embeddings
don't have this basic draw-in-any-order property.

Now any ordering of edges can be obtained just by repeatedly switching the
order of successive edges, and if you think about the recursive definition
of embedding for a minute, you should realize that you can switch
\emph{any} pair of successive edges if you can just switch the last two.
So it all comes down to the following lemma.

\begin{lemma}\label{switch-edges} Suppose that,
  starting from some embeddings of planar graphs with disjoint sets of
  vertices, it is possible by two successive applications of constructor
  operations to add edges $e$ and then $f$ to obtain a planar embedding,
  $\embed{F}$.  Then starting from the same embeddings, it is also
  possible to obtain $\embed{F}$ by adding $f$ and then $e$ with two
  successive applications of constructor operations.
\end{lemma}

We'll leave the proof of Lemma~\ref{switch-edges} to
Problem~\ref{PS_planar_graph_construction_order}.

\begin{corollary}\label{permute-edges} Suppose that, starting from some
  embeddings of planar graphs with disjoint sets of vertices, it is
  possible to add a sequence of edges $e_0,e_1,\dots,e_n$ by successive
  applications of constructor operations to obtain a planar embedding,
  $\embed{F}$.  Then starting from the same embeddings, it is also
  possible to obtain $\embed{F}$ by applications of constructor operations
  that successively add any permutation\footnote{If $\pi:\set{0,1,\dots,n} \to
    \set{0,1,\dots,n}$ is a bijection, then the sequence
    $e_{\pi(0)},e_{\pi(1)},\dots,e_{\pi(n)}$ is called a \term{permutation} of
    the sequence $e_0,e_1,\dots,e_n$.} of the edges $e_0,e_1,\dots,e_n$.
\end{corollary}

\begin{corollary}\label{delete-edge}
Deleting an edge from a planar graph leaves a planar graph.

\begin{proof}
  By Corollary~\ref{permute-edges}, we may assume the deleted edge was the
  last one added in constructing an embedding of the graph.  So the
  embedding to which this last edge was added must be an embedding of the
  graph without that edge.
\end{proof}

\end{corollary}

Since we can delete a vertex by deleting all its incident edges,
Corollary~\ref{delete-edge} immediately implies

\begin{corollary}%\label{delete-vertex}
Deleting a vertex from a planar graph, along with all its incident
edges of course, leaves another planar graph.
\end{corollary}

A \term{subgraph} of a graph, $G$, is any graph whose set of vertices is a
subset of the vertices of $G$ and whose set of edges is a subset of the
set of edges of $G$.  So we can summarize these Corollaries %~\ref{delete-edge}
%and %~\ref{delete-vertex}
and their consequences in a Theorem.

\begin{theorem}%\label{planar-subgraph}
  Any \index{planar subgraph}subgraph of a planar graph is planar.
\end{theorem}
\end{editingnotes}


\section{Coloring Planar Graphs}

We've covered a lot of ground with planar graphs, but not nearly
enough to prove the famous 4-color theorem.  But we can get awfully
close.  Indeed, we have done almost enough work to prove that every
planar graph can be colored using only 5 colors.

There are two familiar facts about planarity that we will need.

\begin{lemma}\label{planar-subgraph}
  Any \index{planar subgraph}subgraph of a planar graph is planar.
\end{lemma}

\begin{lemma}\label{mergelem}
Merging two adjacent vertices of a planar graph leaves another planar
graph.
\end{lemma}

\emph{Merging} two adjacent vertices, $n_1$ and~$n_2$ of a
  graph means deleting the two vertices and then replacing them by a
  new ``merged'' vertex, $m$, adjacent to all the vertices that were
  adjacent to either of~$n_1$ or~$n_2$, as illustrated in
  Figure~\ref{fig:merged}.

\begin{figure}

\graphic{vertex-merge-arrows}

\caption{Merging adjacent vertices $n_1$ and $n_2$ into new vertex, $m$.}
\label{fig:merged}
\end{figure}

Many authors take Lemmas~\ref{planar-subgraph} and~\ref{mergelem} for
granted for continuous drawings of planar graphs described by
Definition~\ref{def:planar_drawing}.  With the recursive
Definition~\ref{def:embedding} both Lemmas can actually be proved
using structural induction (see
Problem~\ref{PS_planar_graph_construction_order}).

\begin{editingnotes}
Problem~\ref{PS_planar_graph_construction_order} needs a solution,
maybe an extension too.
\end{editingnotes}

\begin{editingnotes}
\arm{CUT: this are special cases of Lemma~\ref{planar-subgraph}.  Only
  purpose in mentioning them is if we were doing the proof.}

\begin{lemma*}%\label{lem:deleting_planar_edge}
Deleting an edge from a planar graph leaves another planar graph.
\end{lemma*}

\begin{corollary*}%\label{delete-vertex}
Deleting a vertex from a planar graph, along with all its incident
edges, leaves another planar graph.
\end{corollary*}
\end{editingnotes}

We need only one more lemma:
\begin{lemma}\label{lem:pg5}
Every planar graph has a vertex of degree at most five.
\end{lemma}

\begin{proof}
Assuming to the contrary that every vertex of some planar graph had
degree at least~6, then the sum of the vertex degrees is at
least~$6v$.  But the sum of the vertex degrees equals~$2e$ by the
Handshake Lemma~\ref{sumedges}, so we have $e \ge 3v$ contradicting
the fact that $e \le 3v - 6 < 3v$ by Theorem~\ref{th:e3v}.
\end{proof}

\begin{theorem}
Every planar graph is five-colorable.
\end{theorem}

\begin{proof}
The proof will be by strong induction on the number, $v$, of vertices, with
induction hypothesis:
\begin{quote}
Every planar graph with $v$ vertices is five-colorable.
\end{quote}

\inductioncase{Base cases} ($v \leq 5$): immediate.

\inductioncase{Inductive case}: Suppose $G$ is a planar graph with
$v+1$ vertices.  We will describe a five-coloring of $G$.

First, choose a vertex, $g$, of $G$ with degree at most 5;
Lemma~\ref{lem:pg5} guarantees there will be such a vertex.
\begin{description}

\item[Case 1:] ($\degr{g}<5$): Deleting $g$ from $G$ leaves a graph,
$H$, that is planar by Lemma~\ref{planar-subgraph}, and, since $H$
has $v$ vertices, it is five-colorable by induction hypothesis.  Now
define a five coloring of $G$ as follows: use the five-coloring of $H$
for all the vertices besides $g$, and assign one of the five colors to
$g$ that is not the same as the color assigned to any of its
neighbors.  Since there are fewer than 5 neighbors, there will always
be such a color available for $g$.

\item[Case 2:] ($\degr{g}=5$): If the five neighbors of $g$ in $G$
  were all adjacent to each other, then these five vertices would form
  a nonplanar subgraph isomorphic to $K_5$, contradicting
  Lemma~\ref{planar-subgraph} (since $K_5$ is not planar).  So there
  must be two neighbors, $n_1$ and $n_2$, of $g$ that are not
  adjacent.  Now merge $n_1$ and $g$ into a new vertex,~$m$.  In this
  new graph, $n_2$ is adjacent to $m$, and the graph is planar by
  Lemma~\ref{mergelem}.  So we can then merge $m$ and $n_2$ into a
  another new vertex, $m'$, resulting in a new graph, $G'$, which by
  Lemma~\ref{mergelem} is also planar.  Since $G'$ has $v-1$
  vertices, it is five-colorable by the induction hypothesis.

  Now define a five coloring of $G$ as follows: use the five-coloring of $G'$
  for all the vertices besides $g$, $n_1$ and $n_2$.  Next assign the
  color of $m'$ in $G'$ to be the color of the neighbors $n_1$ and $n_2$.
  Since $n_1$ and $n_2$ are not adjacent in $G$, this defines a proper
  five-coloring of $G$ except for vertex $g$.  But since these two
  neighbors of $g$ have the same color, the neighbors of $g$ have been
  colored using fewer than five colors altogether.  So complete the
  five-coloring of $G$ by assigning one of the five colors to $g$ that is
  not the same as any of the colors assigned to its neighbors.
\end{description}

\end{proof}

\section{Classifying \idx{Polyhedra}}

The \idx{Pythagoreans} had two great mathematical secrets, the
irrationality of $\sqrt{2}$ and a geometric construct that we're about
to rediscover!

A \term{polyhedron} is a convex, three-dimensional region bounded by a
finite number of polygonal faces.  If the faces are identical regular
polygons and an equal number of polygons meet at each corner, then the
polyhedron is \index{regular polyhedron}\term*{regular}.  Three
examples of regular polyhedra are shown in Figure~\ref{fig:polyhedra}: the
tetrahedron, the cube, and the octahedron.

\begin{figure}

\subfloat[]{
    \graphic{Fig_5_47a}
}
\quad
\subfloat[]{
    \graphic{Fig_5_47b}
}
\quad
\subfloat[]{
    \graphic{Fig_5_47c}
}

\caption{The tetrahedron~(a), cube~(b), and octahedron~(c).}
v
\label{fig:polyhedra}
\end{figure}

We can determine how many more regular polyhedra there are by thinking
about planarity.  Suppose we took \emph{any} polyhedron and placed a
sphere inside it.  Then we could project the polyhedron face
boundaries onto the sphere, which would give an image that was a
planar graph embedded on the sphere, with the images of the corners of
the polyhedron corresponding to vertices of the graph.  We've already
observed that embeddings on a sphere are the same as embeddings on the
plane, so Euler's formula for planar graphs can help guide our search
for regular polyhedra.

For example, planar embeddings of the three polyhedra in
Figure~\ref{fig:5DP} are shown in Figure~\ref{fig:5DQ}.

\begin{figure}

\subfloat[]{
    \graphic{Fig_5_48a}
}
\quad
\subfloat[]{
    \graphic{Fig_5_48b}
}
\quad
\subfloat[]{
    \graphic{Fig_5_48c}
}

\caption{Planar embeddings of the tetrahedron~(a), cube~(b), and
  octahedron~(c).}

\label{fig:5DQ}

\end{figure}

Let $m$ be the number of faces that meet at each corner of a
polyhedron, and let $n$ be the number of edges on each face.  In the
corresponding planar graph, there are $m$ edges incident to each of
the $v$ vertices.  By the Handshake Lemma~\ref{sumedges}, we
know:
%
\[
m v = 2 e.
\]
%
Also, each face is bounded by $n$ edges.  Since each edge is on the
boundary of two faces, we have:
%
\[
n f = 2 e
\]
%
Solving for $v$ and $f$ in these equations and then substituting into
\idx{Euler's formula} gives:
\[
\frac{2e}{m} - e + \frac{2e}{n} = 2
\]
which simplifies to
\begin{equation}\label{1m1n}
\frac{1}{m} + \frac{1}{n} = \frac{1}{e} + \frac{1}{2}
\end{equation}
%
Equation~\ref{1m1n} places strong restrictions on the structure of a
polyhedron.  Every nondegenerate polygon has at least 3 sides, so $n
\geq 3$.  And at least 3 polygons must meet to form a corner, so $m
\geq 3$.  On the other hand, if either $n$ or $m$ were 6 or more, then
the left side of the equation could be at most $1/3 + 1/6 = 1/2$,
which is less than the right side.  Checking the finitely-many cases
that remain turns up only five solutions, as shown in
Figure~\ref{fig:5DR}.  For each valid combination of $n$ and $m$, we
can compute the associated number of vertices $v$, edges $e$, and
faces $f$.  And polyhedra with these properties do actually exist.
The largest polyhedron, the dodecahedron, was the other great
mathematical secret of the Pythagorean sect.

\begin{figure}\redrawntrue

\gnote{Add illustration of icosa- and dodecahedra?}

\[
\begin{array}{cc|ccc|l}
n & m & v  & e  &  f & \text{polyhedron} \\ \hline
3 & 3 & 4  & 6  &  4 & \text{tetrahedron} \\
4 & 3 & 8  & 12 &  6 & \text{cube} \\
3 & 4 & 6  & 12 &  8 & \text{octahedron} \\
3 & 5 & 12 & 30 & 20 & \text{icosahedron} \\
5 & 3 & 20 & 30 & 12 & \text{dodecahedron}
\end{array}
\]

\caption{The only possible regular polyhedra.}

\label{fig:5DR}

\end{figure}

The 5 polyhedra in Figure~\ref{fig:5DR} are the only possible regular
polyhedra.  So if you want to put more than 20 geocentric satellites
in orbit so that they \emph{uniformly} blanket the globe---tough luck!

\section{Another Characterization for Planar Graphs}

We did not pick ~$K_5$ and~$K_{3, 3}$ as examples because of their
application to dog houses or quadrapi shaking hands.  We really picked
them because they provide another, famous, discrete characterizarion
of planar graphs:
\begin{theorem}[Kuratowski]\label{thm:kuratowski}
A graph is not planar if and only if it contains $K_5$ or~$K_{3, 3}$
as a minor.
\end{theorem}

\begin{definition}
  A \term{minor} of a graph~$G$ is a graph that can be obtained by
  repeatedly\footnote{The three operations can each be performed any
    number of times in any order.} deleting vertices, deleting edges,
  and \index{merging vertices}merging \emph{adjacent} vertices
  of~$G$.
\end{definition}

For example, Figure~\ref{fig:5DL} illustrates why $C_3$ is a minor of
the graph in Figure~\ref{fig:5DL}(a).  In fact $C_3$ is a minor of a
connected graph~$G$ if and only if $G$ is not a tree.

\begin{figure}

\gnote{Tom: Should we label $v_1$, $v_2$ and $v_3$ in all 6 graphs?}

\graphic{Fig_5DL}

\caption{One method by which the graph in~(a) can be reduced
  to~$C_3$~(f), thereby showing that $C_3$ is a minor of the graph.
  The steps are: merging the nodes incident to~$e_1$~(b),
  deleting~$v_1$ and all edges incident to it~(c), deleting $v_2$~(d),
deleting~$e_2$, and deleting $v_3$~(f).}

\label{fig:5DL}
\end{figure}

The known proofs of Kuratowski's Theorem~\ref{thm:kuratowski} are a little
too long to include in an introductory text, so we won't give one.


%\section{Problems}
%\problemsection

%% Planar Graphs Problems %%%%%%%%%%%%%%%%%%%%%%%%%%%%%%%%%%%%%%%%%%%%%%%%%%%%%
\begin{problems}

\examproblems
\pinput{MQ_planar_isomorphism}
\pinput{MQ_planar_face_length}
\pinput{FP_planar_structural_induction}
\pinput{MQ_simple_graphs_short_answer}

\classproblems
\pinput{CP_planar_embedding_isomorphism}
%\pinput{CP_K33_not_planar}
\pinput{CP_planar_structural_induction}

\homeworkproblems
\pinput{PS_triangle_free_planar_graphs}
\pinput{PS_planar_graph_construction_order}

%\pinput{CP0506_}
\end{problems}

%% Conclusion %%%%%%%%%%%%%%%%%%%%%%%%%%%%%%%%%%%%%%%%%%%%%%%%%%%%%%%%%%%%%%%%%
%\TBA{Add conclusion here...}

\endinput
