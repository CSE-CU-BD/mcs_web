%Intro to Part I

\iffalse

\chapter{What is a Proof?}\label{proofs_chap}

\newtheorem{method}{Method}

\section{Introduction}

Proofs are the part of Math that Computer Science students

This is a Math text aimed at students of Computer Science, and the
topics we cover were selected because of their connection to Computer
Science.  
 for students to Computer Scientists need who mostly
chose not to be Math majors, even though they did just fine in High
School Math and introductory calculus.
\fi

\section*{Mathematical Proofs}

A proof is a method of establishing truth.  What constitutes a proof
differs among fields.

\begin{itemize}

\item \emph{Legal} truth is decided by a jury based on
allowable evidence presented at trial.

\item \emph{Authoritative} truth is specified by a trusted person or
organization.

\item \emph{Scientific} truth\footnote{Actually, only scientific
\emph{falsehood} can be demonstrated by an experiment ---when the experiment
fails to behave as predicted.  But no amount of experiment can confirm
that the \emph{next} experiment won't fail.  For this reason, scientists
rarely speak of truth, but rather of \emph{theories} that accurately
predict past, and anticipated future, experiments.} is confirmed by
experiment.

\item \emph{Probable} truth is established by statistical analysis of
sample data.

\item \emph{Philosophical} proof involves careful exposition and
  persuasion typically based on a series of small, plausible arguments.
  The best example begins with ``Cogito ergo sum,'' a Latin sentence that
  translates as ``I think, therefore I am.''  It comes from the beginning
  of a 17th century essay by the mathematician/philospher, Ren\'e
  Descartes, and it is one of the most famous quotes in the world: do a
  web search on the phrase and you will be flooded with hits.

  Deducing your existence from the fact that you're thinking about your
  existence is a pretty cool and persuasive-sounding first axiom.
  However, with just a few more lines of argument in this vein, Descartes
  \href{http://www.btinternet.com/~glynhughes/squashed/descartes.htm}{goes
    on} to conclude that there is an infinitely beneficent God.  Whether
  or not you believe in a beneficent God, you'll probably agree that any
  very short proof of God's existence is bound to be far-fetched.  So even
  in masterful hands, this approach is not reliable.
\end{itemize}

Mathematics also has a specific notion of ``proof.''

\begin{definition*}
A \term{formal proof} of a \term{proposition} is a chain of \term{logical
deductions} leading to the proposition from a base set of \term{axioms}.
\end{definition*}

The three key ideas in this definition are highlighted: proposition,
logical deduction, and axiom.  These three ideas are explained in
Chapter~\ref{prop_chap}, and Chapter~\ref{templates_chap} describes
some basic ways of organizing proofs.

\iffalse The last section contains some examples of complete proofs.\fi

\begin{problems}
%\practiceproblems
\classproblems
\pinput{CP_buggy_highschool_proofs}
\pinput{CP_false_arithmetic_mean_proof}
\pinput{CP_surprise_quiz_next_week}
%\homeworkproblems
\end{problems}

\endinput
