\part{Structures}
\label{part:structures}

\clearpage

\section*{Introduction}

Structure is fundamental in computer science.  Whether you are writing
code, solving an optimization problem, or designing a network, you
will be dealing with structure.  The better you can understand the
structure, the better your results will be.  And if you can reason
about structure, then you will be in a good position to convince
others (and yourself) that your results are worthy.

The most important structure in computer science is a \emph{graph},
also known as a \emph{network}).  Graphs provide an excellent
mechanism for modeling associations between pairs of objects, for
example, two exams that cannot be given at the same time, two people
that like each other, or two subroutines that can be run
independently.  In Chapter~\ref{chap:graph_theory}, we study graphs
that represent \emph{symmetric} relationships, like those just
mentioned.  In Chapter~\ref{chap:digraphs}, we consider graphs where
the relationship is \emph{one-way}, that is, a situation where you can
go from $x$ to~$y$ but not necessarily vice-versa.

In Chapter~\ref{chap:partial_orders}, we consider the more general
notion of a \emph{relation} and we examine important classes of
relations such as partially ordered sets.  Partially ordered sets
arise frequently in scheduling problems.

We conclude in Chapter~\ref{chap:state-machines} with a study of
\emph{state machines}.  State machines can be used to model a variety
of processes and are a fundamental tool in proving that an algorithm
terminates and that it produces the correct output.

\endinput
