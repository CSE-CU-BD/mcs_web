\part{Structures}
\label{part:structures}

\partintro

The properties of the set of integers are the subject of Number
Theory.  This part of the text starts with a chapter on this topic
because the integers are a very familiar mathematical structure that
have lots of easy-to-state and interesting-to-prove properties.  This
makes Number Theory a good place to start serious practice with the
methods of proof outlined in Part 1.  Moreover, Number Theory has
turned out to have multiple applications in computer science.  For
example, most modern data encryption methods are based on Number
theory.

We study numbers as a ``\term{structure}'' that has multiple parts of
different kinds.  One part is, of course, the set of all the integers.
A second part is the collection of basic integer operations:
addition, multiplication, exponentiation,\dots.  Other parts are the
important subsets of integers---like the prime numbers---out of which
all integers can be built using multiplication.  Structured objects
more generally are fundamental in computer science.  Whether you are
writing code, solving an optimization problem, or designing a network,
you will be dealing with structures.

\iffalse
The better you can understand the structure, the better your results will be.
And if you can reason about structure, then you will be in a good
position to convince others (and yourself) that your results are
worthy.\fi

\emph{Graphs}, also known as \emph{networks}, are a fundamental
structure in computer science.  Graphs can model associations between
pairs of objects; for example, two exams that cannot be given at the
same time, two people that like each other, or two subroutines that
can be run independently.  In Chapter~\ref{digraphs_chap}, we study
\emph{\idx{directed graphs}} which model \emph{one-way} relationships
such as being bigger than, loving (sadly, it's often not mutual), and
being a prerequisite for.  A highlight is the special case of acyclic
digraphs (\emph{\idx{DAG}s}) that correspond to a class of relations
called \emph{partial orders}.  Partial orders arise frequently in the
study of scheduling and concurrency.  Digraphs as models for data
communication and routing problems are the topic of
Chapter~\ref{comm_net_chap}.

In Chapter~\ref{simple_graphs_chap} we focus on \emph{\idx{simple
    graphs}} that represent mutual or \emph{\idx{symmetric}}
relationships, such as being in conflict, \iffalse being congruent
modulo 17\fi being compatible, being independent, being capable of
running in parallel.  Planar Graphs---simple graphs that can be drawn
in the plane---are examined in Chapter~\ref{planar_graphs_chap}, the
final chapter of Part~\ref{part:structures}.  The impossibility of
placing 50 geocentric satellites in orbit so that they
\emph{uniformly} blanket the globe will be one of the conclusions
reached in this chapter.

\iffalse
This part of the text concludes with Chapter 13
%~\ref{state_machine_chap}
which elaborates the use of the \emph{\idx{state machines}} in program
verification and modeling concurrent computation.
\fi

\endinput
