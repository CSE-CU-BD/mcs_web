\part{Structures}
\label{part:structures}

\partintro

Structure is fundamental in computer science.  Whether you are writing
code, solving an optimization problem, or designing a network, you
will be dealing with structure.  The better you can understand the
structure, the better your results will be.  And if you can reason
about structure, then you will be in a good position to convince
others (and yourself) that your results are worthy.

The most important structure in computer science is a \emph{graph},
also known as a \emph{network}).  Graphs provide an excellent
mechanism for modeling associations between pairs of objects; for
example, two exams that cannot be given at the same time, two people
that like each other, or two subroutines that can be run
independently.  In Chapter~\ref{digraphs_chap}, we study
\emph{\idx{directed graphs}} which model \emph{one-way} relationships
such as being bigger than, loving (sadly, it's often not mutual),
being a prerequisite for.  A highlight is the special case of acyclic
digraphs (\emph{\idx{DAG}s}) that correspond to a class of relations
called \emph{partial orders}.  Partial orders arise frequently in the
study of scheduling and concurrency.  Digraphs as models for data
communication and routing problems are the topic of
Chapter~\ref{comm_net_chap}.

In Chapter 11
%~\ref{simple_graphs_chap}
we focus on \emph{\idx{simple graphs}} that represent mutual or
\emph{\idx{symmetric}} relationships, such as being congruent modulo
17, being in conflict, being compatible, being independent, being
capable of running in parallel.

This part of the text concludes with Chapter 12
%~\ref{state_machine_chap}
which elaborates the use of the \emph{\idx{state machines}} in program
verification and modeling concurrent computation.

\endinput
