\part{Proofs}
\label{part:proofs}

\partintro

This text explains how to use mathematical models and methods to analyze
problems that arise in computer science.  Proofs play a central role in
this work because the authors share a belief with \arm{proofs \&
  understanding.  proofs in CS} most mathematicians that proofs are
essential for genuine understanding.  Proofs also play a growing role in
computer science; they are used to certify that software and hardware will
\emph{always} behave correctly, something that no amount of testing can
do.

Simply put, a proof is a method of establishing truth.  Like beauty,
``truth'' sometimes depends on the eye of the beholder, and
it should not be surprising that what constitutes a proof differs
among fields.  For example, in the judicial system, \emph{legal} truth
is decided by a jury based on the allowable evidence presented at
trial.  In the business world, \emph{authoritative} truth is specified
by a trusted person or organization, or maybe just your boss.  In
fields such as physics and biology, \emph{scientific}
truth\footnote{Actually, only scientific
\emph{falsehood} can be demonstrated by an experiment---when
the experiment fails to behave as predicted.  But no amount of
experiment can confirm that the \emph{next} experiment won't fail.
For this reason, scientists rarely speak of truth, but rather
of \emph{theories} that accurately predict past, and anticipated
future, experiments.} is confirmed by experiment.  In
statistics, \emph{probable} truth is established by statistical
analysis of sample data.

\emph{Philosophical} proof involves careful exposition and
persuasion typically based on a series of small, plausible arguments.
The best example begins with ``Cogito ergo sum,'' a Latin sentence
that translates as ``I think, therefore I am.''  It comes from the
beginning of a 17th century essay by the mathematician/philosopher,
Ren\'e Descartes, and it is one of the most famous quotes in the
world: do a web search on the phrase and you will be flooded with
hits.

Deducing your existence from the fact that you're thinking about your
existence is a pretty cool and persuasive-sounding idea.
However, with just a few more lines of argument in this vein, Descartes
\href{http://www.btinternet.com/~glynhughes/squashed/descartes.htm}{goes
  on} to conclude that there is an infinitely beneficent God.  Whether
or not you believe in a beneficent God, you'll probably agree that any
very short proof of God's existence is bound to be far-fetched.  So even
in masterful hands, this approach is not reliable.

Mathematics has its own specific notion of ``proof.''

\begin{definition*}
A \term{mathematical proof} of a \term{proposition} is a chain of \term{logical
deductions} leading to the proposition from a base set of \term{axioms}.
\end{definition*}

The three key ideas in this definition are highlighted:
\emph{proposition}, \emph{logical deduction}, and \emph{axiom}.
Chapter~\ref{prop_chap} explains how mathematicians work with \arm{forming
  new props} propositions, in particular, how propositions can be combined
to form new propositions.  Chapters~\ref{templates_chap}
and~\ref{induction_chap} provide principles for making logical deductions
from axioms---along with \emph{lots} of examples of proofs.  Throughout
the text there are also examples of \term{bogus proofs}---arguments that
look like proofs but aren't.  Sometimes a bogus proof can reach false
conclusions because of missteps or mistaken assumptions.  More subtle
bogus proofs reach correct conclusions, but in improper ways, for example
by circular reasoning, by leaping to unjustified conclusions, or by saying
that the hard part of ``the proof is left to the reader.''  Learning to
spot the flaws in improper proofs will hone your skills at seeing how each
proof step follows logically from prior steps.  It will also enable you to
spot flaws in your own proofs.

Creating a good proof is a lot like creating a beautiful work of art.  In
fact, mathematicians often refer to really good proofs as being
``elegant'' or ``beautiful.''  It takes a practice and experience to write
proofs that merit such praises, \arm{rephrased} but to get you started in
the right direction, we will provide templates for the most useful proof
techniques in Chapters~\ref{templates_chap} and~\ref{induction_chap}.
These techniques are then used in Chapter~\ref{number_theory_chap} to
establish some important facts about numbers---facts that form the
\arm{``some'' not ``one''} underpinning of some of the world's most
widely-used cryptosystems.


\endinput
