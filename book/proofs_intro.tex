\part{Proofs}
\label{part:proofs}

%\cleardoublepage

\partintro
%\cleardoublepage
\addcontentsline{toc}{chapter}{Introduction}

This text explains how to use mathematical models and methods to analyze
problems that arise in computer science.  Proofs play a central role in
this work because the authors share a belief with most mathematicians that
proofs are essential for genuine understanding.  Proofs also play a
growing role in computer science; they are used to certify that software
and hardware will \emph{always} behave correctly, something that no amount
of testing can do.

Simply put, a proof is a method of establishing truth.  Like beauty,
``truth'' sometimes depends on the eye of the beholder, and it should
not be surprising that what constitutes a proof differs among fields.
For example, in the judicial system, \emph{legal} truth is decided by
a jury based on the allowable evidence presented at trial.  In the
business world, \emph{authoritative} truth is specified by a trusted
person or organization, or maybe just your boss.  In fields such as
physics or biology, \emph{scientific} truth\footnote{Actually, only
  scientific \emph{falsehood} can be demonstrated by an
  experiment---when the experiment fails to behave as predicted.  But
  no amount of experiment can confirm that the \emph{next} experiment
  won't fail.  For this reason, scientists rarely speak of truth, but
  rather of \emph{theories} that accurately predict past, and
  anticipated future, experiments.
\begin{editingnotes}
Add comment re ``theory'' of evolution?
\end{editingnotes}
} is confirmed by experiment.  In statistics,
\emph{probable} truth is established by statistical analysis of sample
data.

\emph{Philosophical} proof involves careful exposition and persuasion
typically based on a series of small, plausible arguments.  The best
example begins with ``Cogito ergo sum,'' a Latin sentence that
translates as ``I think, therefore I am.''  This phrase comes from the
beginning of a 17th century essay by the mathematician/philosopher,
Ren\'e Descartes, and it is one of the most famous quotes in the
world: do a web search for it, and you will be flooded with
hits.

Deducing your existence from the fact that you're thinking about your
existence is a pretty cool and persuasive-sounding idea.  However,
with just a few more lines of argument in this vein, Descartes
\href{http://www.btinternet.com/~glynhughes/squashed/descartes.htm}{goes
  on} to conclude that there is an infinitely beneficent God.  Whether
or not you believe in an infinitely beneficent God, you'll probably
agree that any very short ``proof'' of God's infinite beneficence is
bound to be far-fetched.  So even in masterful hands, this approach is
not reliable.

Mathematics has its own specific notion of ``proof.''

\begin{definition*}
A \emph{mathematical proof} of a \term{proposition} is a chain of \emph{logical
deductions} leading to the proposition from a base set of%
\index{axiom}
\emph{axioms}.
\end{definition*}

The three key ideas in this definition are highlighted:
\emph{proposition}, \emph{logical deduction}, and \emph{axiom}.
Chapter~\ref{proofs_chap} examines these three ideas along with some
basic ways of organizing proofs.  Chapter~\ref{well_ordering_chap}
introduces the Well Ordering Principle, a basic method of proof; later,
Chapter~\ref{induction_chap} introduces the closely related proof
method of induction.

If you're going to prove a proposition, you'd better have a precise
understanding of what the proposition means.  To avoid ambiguity and
uncertain definitions in ordinary language, mathematicians use
language very precisely, and they often express propositions using
logical formulas; these are the subject of
Chapter~\ref{logicform_chap}.

The first three Chapters assume the reader is familiar with a few
mathematical concepts like sets and functions.
Chapters~\ref{data_chap} and~\ref{set_theory_chap} offer a more
careful look at such mathematical data types, examining in particular
properties and methods for proving things about infinite sets.
Chapter~\ref{recursive_data_chap} goes on to examine recursively
defined data types.

\iffalse
Number theory is the study of properties of the integers.  This part
of the text ends with Chapter~\ref{number_theory_chap} on Number
theory because there are lots of easy-to-state and
interesting-to-prove properties of numbers.  This subject was once
thought to have few, if any, practical applications, but it has turned
out to have multiple applications in Computer Science.  For example,
most modern data encryption methods are based on Number theory.
\fi


\section{References}
 \cite{Cupillari12},
 \cite{Velleman1994},
 \cite{AignerG99}     %Proofs from The Book

\endinput
