\chapter{Recurrences}\label{chap:recurrences}

\section{The \idx{Towers of Hanoi}}

\iffalse
According to legend, there is a temple in Hanoi with three posts and
64 gold disks of different sizes.  Each disk has a hole through the
center so that it fits on a post.  In the misty past, all the disks
were on the first post, with the largest on the bottom and the
smallest on top, as shown in Figure~\ref{fig:10A1}.

\begin{figure}

\graphic{Fig_10-1}

\caption{The initial configuration of the disks in the Towers of Hanoi
  problem.}

\label{fig:10A1}

\end{figure}

Monks in the temple have labored through the years since to move all
the disks to one of the other two posts according to the following
rules:
\begin{itemize}
\item The only permitted action is removing the top disk from one post
and dropping it onto another post.
\item A larger disk can never lie above a smaller disk on any post.
\end{itemize}
So, for example, picking up the whole stack of disks at once
and dropping them on another post is illegal.  That's good, because
the legend says that when the monks complete the puzzle, the world
will end!

To clarify the problem, suppose there were only 3 gold disks instead
of 64.  Then the puzzle could be solved in 7 steps as shown in
Figure~\ref{fig:10A2}.

\begin{figure}

\graphic{Fig_10-2}

\caption{The 7-step solution to the Towers of Hanoi problem when there
are $n = 3$~disks.}

\label{fig:10A2}

\end{figure}

The questions we must answer are, ``Given sufficient time, can the
monks succeed?''  If so, ``How long until the world ends?''  And, most
importantly, ``Will this happen before the final exam?''

\subsection{A Recursive Solution}

\begin{editingnotes}
Why subscript notation $T_n$ here, but function notation $T(n)$ in
later sections?
\end{editingnotes}

The Towers of Hanoi problem can be solved recursively.  As we describe
the procedure, we'll also analyze the running time.  To that end, let
$T_n$ be the minimum number of steps required to solve the $n$-disk
problem.  For example, some experimentation shows that $T_1 = 1$ and
$T_2$ = 3.  The procedure illustrated above shows that $T_3$ is at
most 7, though there might be a solution with fewer steps.

The recursive solution has three stages, which are described below and
illustrated in Figure~\ref{fig:10A3}.
For clarity, the largest disk is shaded in the figures.

\begin{figure}

\graphic{Fig_10-3}

\caption{A recursive solution to the Towers of Hanoi problem.}

\label{fig:10A3}

\end{figure}
\begin{description}

\item[Stage 1.]  Move the top $n-1$ disks from the first post to
  the second using the solution for $n - 1$ disks.  This can be done
  in $T_{n-1}$ steps.

\item[Stage 2.]  Move the largest disk from the first post to the
  third post.  This takes just 1 step.

\item[Stage 3.]  Move the $n-1$ disks from the second post to
  the third post, again using the solution for $n - 1$ disks.  This
  can also be done in $T_{n-1}$ steps.

\end{description}

This algorithm shows that $T_n$, the minimum number of steps required
to move $n$ disks to a different post, is at most $T_{n-1} + 1 +
T_{n-1} = 2 T_{n-1} + 1$.  We can use this fact to upper bound the
number of operations required to move towers of various heights:
\begin{align*}
T_3 & \leq 2 \cdot T_2 + 1 = 7 \\
T_4 & \leq 2 \cdot T_3 + 1 \leq 15
\end{align*}
Continuing in this way, we could eventually compute an upper bound on
$T_{64}$, the number of steps required to move 64 disks.  So this
algorithm answers our first question: given sufficient time, the monks
can finish their task and end the world.  This is a shame.  After all
that effort, they'd probably want to smack a few high-fives and go out
for burgers and ice cream, but nope---world's over.

\subsection{Finding a Recurrence}

We cannot yet compute the exact number of steps that the monks need
to move the 64 disks, only an upper bound.  Perhaps, having pondered
the problem since the beginning of time, the monks have devised a
better algorithm.

In fact, there is no better algorithm, and here is why.  At some step,
the monks must move the largest disk from the first post to a
different post.  For this to happen, the $n - 1$ smaller disks must all
be stacked out of the way on the only remaining post.  Arranging the $n
- 1$ smaller disks this way requires at least $T_{n-1}$ moves.  After
the largest disk is moved, at least another $T_{n-1}$ moves are
required to pile the $n - 1$ smaller disks on top.

This argument shows that the number of steps required is at least
$2T_{n-1} + 1$.  Since we gave an algorithm using exactly that number
of steps, we can now write an expression for $T_n$, the number of
moves required to complete the Towers of Hanoi problem with $n$ disks:
\begin{align*}
T_1 & = 1 \\
T_n & = 2T_{n-1} + 1 & \text{(for $n \geq 2$)}.
\end{align*}

This is a typical recurrence.  These two lines define a sequence of
values, $T_1, T_2, T_3, \dots$.  The first line says that the first
number in the sequence, $T_1$, is equal to 1.  The second line defines
every other number in the sequence in terms of its predecessor.  So we
can use this recurrence to compute any number of terms in the sequence:
\begin{align*}
T_1 & = 1 \\
T_2 & = 2 \cdot T_1 + 1 = 3 \\
T_3 & = 2 \cdot T_2 + 1 = 7 \\
T_4 & = 2 \cdot T_3 + 1 = 15 \\
T_5 & = 2 \cdot T_4 + 1 = 31 \\
T_6 & = 2 \cdot T_5 + 1 = 63.
\end{align*}

\subsection{Solving the Recurrence}

We could determine the number of steps to move a 64-disk tower by
computing $T_7$, $T_8$, and so on up to $T_{64}$.  But that would take
a lot of work.  It would be nice to have a closed-form expression for
$T_n$, so that we could quickly find the number of steps required for
any given number of disks.  (For example, we might want to know how
much sooner the world would end if the monks melted down one disk to
purchase burgers and ice cream \emph{before} the end of the world.)
\fi

There are several methods for solving recurrence equations.  The
simplest is to \emph{guess} the solution and then \emph{verify} that
the guess is correct with an induction proof.

For example, as a alternative to the generating function derivation in
Section~\ref{hanoi-subsec} of the value of the number, $T_n$, of moves
in the Tower of Hanoi problem with $n$ disks, we could have tried
guessing.  As a basis for a good guess, let's look for a pattern in
the values of $T_n$ computed above: 1, 3, 7, 15, 31, 63.  A natural
guess is $T_n = 2^n - 1$.  But whenever you guess a solution to a
recurrence, you should always verify it with a proof, typically by
induction.  After all, your guess might be wrong.  (But why bother to
verify in this case?  After all, if we're wrong, its not the end of
the\dots no, let's check.)

\begin{claim}
$T_n = 2^n - 1$ satisfies the recurrence:
\begin{align*}
T_1 & = 1 \\
T_n & = 2T_{n-1} + 1 & \text{(for $n \geq 2$)}.
\end{align*}
\end{claim}
\begin{proof}
  The proof is by induction on $n$.  The induction hypothesis is that
  $T_n = 2^n - 1$.  This is true for $n = 1$ because $T_1 = 1 = 2^1 -
  1$.  Now assume that $T_{n-1} = 2^{n-1} - 1$ in order to prove
  that $T_n = 2^n - 1$, where $n \geq 2$:
\begin{align*}
T_{n} & = 2 T_{n-1} + 1 \\
  & = 2 (2^{n-1} - 1) + 1 \\
  & = 2^n - 1.
\end{align*}
The first equality is the recurrence equation, the second follows from
the induction assumption, and the last step is simplification.
\end{proof}

Such verification proofs are especially tidy because recurrence
equations and induction proofs have analogous structures.  In
particular, the base case relies on the first line of the recurrence,
which defines $T_1$.  And the inductive step uses the second line of
the recurrence, which defines $T_n$ as a function of preceding terms.

Our guess is verified.  So we can now resolve our remaining questions
about the 64-disk puzzle.  Since $T_{64} = 2^{64} - 1$, the monks must
complete more than 18 billion billion steps before the world ends.
Better study for the final.

\subsection{The Upper Bound Trap}

When the solution to a recurrence is complicated, one might try to
prove that some simpler expression is an upper bound on the solution.
For example, the exact solution to the Towers of Hanoi recurrence is
$T_n = 2^n - 1$.  Let's try to prove the ``nicer'' upper bound $T_n
\leq 2^n$, proceeding exactly as before.

\begin{proof} (Failed attempt.)
  The proof is by induction on $n$.  The induction hypothesis is that
  $T_n \leq 2^n$.  This is true for $n = 1$ because $T_1 = 1 \leq 2^1$.
  Now assume that $T_{n-1} \leq 2^{n-1}$ in order to prove that $T_n
  \leq 2^n$, where $n \geq 2$:
\begin{align*}
T_n & = 2 T_{n-1} + 1 \\ & \leq 2 (2^{n-1}) + 1 \\ & \not\leq 2^n
\qquad \text{\textcolor{red}{$\QIMPLIES$ Uh-oh!}}
\end{align*}
The first equality is the recurrence relation, the second follows from
the induction hypothesis, and the third step is a flaming train wreck.
\end{proof}

The proof doesn't work!  As is so often the case with induction
proofs, the argument only goes through with a \emph{stronger}
hypothesis.  This isn't to say that upper bounding the solution to a
recurrence is hopeless, but this is a situation where induction and
recurrences do not mix well.

\subsection{Plug and Chug}

Guess-and-verify is a simple and general way to solve recurrence
equations.  But there is one big drawback: you have to \emph{guess
  right}.  That was not hard for the Towers of Hanoi example.  But
sometimes the solution to a recurrence has a strange form that is
quite difficult to guess.  Practice helps, of course, but so can some
other methods.

Plug-and-chug is another way to solve recurrences.  This is also
sometimes called ``expansion'' or ``iteration.''  As in
guess-and-verify, the key step is identifying a pattern.  But instead
of looking at a sequence of \emph{numbers}, you have to spot a pattern
in a sequence of \emph{expressions}, which is sometimes easier.  The
method consists of three steps, which are described below and
illustrated with the Towers of Hanoi example.

\subsubsection*{Step 1: Plug and Chug Until a Pattern Appears}

The first step is to expand the recurrence equation by alternately
``plugging'' (applying the recurrence) and ``chugging'' (simplifying the
result) until a pattern appears.  Be careful: too much simplification
can make a pattern harder to spot.  The rule to remember---indeed, a
rule applicable to the whole of college life---is \emph{chug in
  moderation}.
\begin{align*}
T_n & = 2T_{n-1} + 1 \\
  & = 2 (2 T_{n-2} + 1)  + 1 & \text{plug} \\
  & = 4 T_{n-2} + 2 + 1 & \text{chug} \\
  & = 4 (2 T_{n-3} + 1) + 2 + 1 & \text{plug} \\
  & = 8 T_{n-3} + 4 + 2 + 1 & \text{chug} \\
  & = 8 (2 T_{n-4} + 1) + 4 + 2 + 1 & \text{plug} \\
  & = 16 T_{n-4} + 8 + 4 + 2 + 1 & \text{chug}
\end{align*}
Above, we started with the recurrence equation.  Then we replaced
$T_{n-1}$ with $2 T_{n-2} + 1$, since the recurrence says the two are
equivalent.  In the third step, we simplified a little---but not too
much!  After several similar rounds of plugging and chugging, a
pattern is apparent.  The following formula seems to hold:
\begin{align*}
T_n & = 2^k T_{n - k} + 2^{k-1} + 2^{k-2} + \cdots + 2^2 + 2^1 + 2^0 \\
  & = 2^k T_{n-k} + 2^k - 1
\end{align*}
Once the pattern is clear, simplifying is safe and convenient.  In
particular, we've collapsed the geometric sum to a closed form on the
second line.

\subsubsection*{Step 2:  Verify the Pattern}

The next step is to verify the general formula with one more round of plug-and-chug.
\begin{align*}
T_n & = 2^k T_{n-k} + 2^k - 1 \\
  & = 2^k (2 T_{n-(k+1)} + 1) + 2^k - 1 & \text{plug} \\
  & = 2^{k+1} T_{n-(k+1)} + 2^{k+1} - 1 & \text{chug} \\
\end{align*}
The final expression on the right is the same as the expression on the
first line, except that $k$ is replaced by $k+1$.  Surprisingly, this
effectively \emph{proves} that the formula is correct for all $k$.
Here is why: we know the formula holds for $k = 1$, because that's the
original recurrence equation.  And we've just shown that if the
formula holds for some $k \geq 1$, then it also holds for $k + 1$.  So
the formula holds for all $k \geq 1$ by induction.

\subsubsection*{Step 3: Write $T_n$ Using Early Terms with Known Values}

The last step is to express $T_n$ as a function of early terms whose
values are known.  Here, choosing $k = n - 1$ expresses $T_n$ in terms
of $T_1$, which is equal to 1.  Simplifying gives a closed-form
expression for $T_n$:
\begin{align*}
T_n & = 2^{n - 1} T_1 + 2^{n-1} - 1 \\
  & = 2^{n-1} \cdot 1 + 2^{n-1} - 1 \\
  & = 2^n - 1.
\end{align*}
We're done!  This is the same answer we got from guess-and-verify.

\medskip
%%%
%\vspace{1ex}
%%%

Let's compare guess-and-verify with plug-and-chug.  In the
guess-and-verify method, we computed several terms at the beginning of
the sequence, $T_1$, $T_2$, $T_3$, etc., until a pattern appeared.  We
generalized to a formula for the $n$th term, $T_n$.  In contrast,
plug-and-chug works backward from the $n$th term.  Specifically, we
started with an expression for $T_n$ involving the preceding term,
$T_{n-1}$, and rewrote this using progressively earlier terms,
$T_{n-2}$, $T_{n-3}$, etc.  Eventually, we noticed a pattern, which
allowed us to express $T_n$ using the very first term, $T_1$, whose
value we knew.  Substituting this value gave a closed-form expression
for $T_n$.  So guess-and-verify and plug-and-chug tackle the problem
from opposite directions.

\section{Merge Sort}

Algorithms textbooks traditionally claim that sorting is an important,
fundamental problem in computer science.  Then they smack you with
sorting algorithms until life as a disk-stacking monk in Hanoi sounds
delightful.  Here, we'll cover just \emph{one} well-known sorting
algorithm, \term{Merge Sort}.  The analysis introduces another kind of
recurrence.

Here is how Merge Sort works.  The input is a list of $n$ numbers, and
the output is those same numbers in nondecreasing order.  There are
two cases:
\begin{itemize}
\item If the input is a single number, then the algorithm does nothing,
  because the list is already sorted.
\item Otherwise, the list contains two or more numbers.  The first
  half and the second half of the list are each sorted recursively.
  Then the two halves are merged to form a sorted list with all $n$
  numbers.
\end{itemize}

Let's work through an example.  Suppose we want to sort this list:
\begin{center}
10, 7, 23, 5, 2, 8, 6, 9.
\end{center}
Since there is more than one number, the first half (10, 7, 23, 5) and
the second half (2, 8, 6, 9) are sorted recursively.  The results are
5, 7, 10, 23 and 2, 6, 8, 9.  All that remains is to merge these two
lists.  This is done by repeatedly emitting the smaller of the two
leading terms.  When one list is empty, the whole other list is
emitted.  The example is worked out below.   In this table, underlined
numbers are about to be emitted.
\begin{center}
\begin{tabular}{lll}
First Half & Second Half & Output \\
5, 7, 10, 23 & \underline{2}, 6, 8, 9 & \\
\underline{5}, 7, 10, 23 & 6, 8, 9 & 2\\
7, 10, 23 & \underline{6}, 8, 9 & 2, 5 \\
\underline{7}, 10, 23 & 8, 9 & 2, 5, 6 \\
10, 23 & \underline{8}, 9 & 2, 5, 6, 7 \\
10, 23 & \underline{9} & 2, 5, 6, 7, 8 \\
\underline{10}, \underline{23} & & 2, 5, 6, 7, 8, 9 \\
& & 2, 5, 6, 7, 8, 9, 10, 23
\end{tabular}
\end{center}
The leading terms are initially 5 and 2.  So we output 2.  Then the
leading terms are 5 and 6, so we output 5.  Eventually, the second
list becomes empty.  At that point, we output the whole first list,
which consists of 10 and 23.  The complete output consists of all the
numbers in sorted order.

\subsection{Finding a Recurrence}
A traditional question about sorting algorithms is, ``What is the
maximum number of comparisons used in sorting $n$ items?''  This is
taken as an estimate of the running time.  In the case of Merge Sort,
we can express this quantity with a recurrence.  Let $T_n$ be the
maximum number of comparisons used while Merge Sorting a list of $n$
numbers.  For now, assume that $n$ is a power of 2.  This ensures that
the input can be divided in half at every stage of the recursion.
\begin{itemize}
\item If there is only one number in the list, then no comparisons are
  required, so $T_1 = 0$.
\item Otherwise, $T_n$ includes comparisons used in sorting the first
  half (at most $T_{n/2}$), in sorting the second half (also at most
  $T_{n/2}$), and in merging the two halves.  The number of
  comparisons in the merging step is at most $n - 1$.  This is because
  at least one number is emitted after each comparison and one more
  number is emitted at the end when one list becomes empty.  Since $n$
  items are emitted in all, there can be at most $n - 1$ comparisons.
\end{itemize}
Therefore, the maximum number of comparisons needed to Merge Sort $n$
items is given by this recurrence:
\begin{align*}
T_1 & = 0 \\
T_{n} & = 2 T_{n/2} + n - 1 & \text{(for $n \geq 2$ and a power of 2)}.
\end{align*}
This fully describes the number of comparisons, but not in a very
useful way; a closed-form expression would be much more helpful.  To
get that, we have to solve the recurrence.

\subsection{Solving the Recurrence}

Let's first try to solve the Merge Sort recurrence with the
guess-and-verify technique.  Here are the first few values:
\begin{align*}
T_1 & = 0 \\
T_2 & = 2 T_1 + 2 - 1 = 1 \\
T_4 & = 2 T_2 + 4 - 1 = 5 \\
T_8 & = 2 T_4 + 8 - 1 = 17 \\
T_{16} & = 2 T_8 + 16 - 1 = 49.
\end{align*}
We're in trouble!  Guessing the solution to this recurrence is hard
because there is no obvious pattern.  So let's try the plug-and-chug
method instead.

\subsubsection*{Step 1: Plug and Chug Until a Pattern Appears}

First, we expand the recurrence equation by alternately plugging and
chugging until a pattern appears.
\begin{align*}
T_n & = 2T_{n/2} + n - 1 \\
  & = 2 (2 T_{n/4} + n/2 - 1)  + (n - 1) & \text{plug} \\
  & = 4 T_{n/4} + (n - 2) + (n - 1) & \text{chug} \\
  & = 4 (2 T_{n/8} + n / 4 - 1) + (n - 2) + (n - 1) & \text{plug} \\
  & = 8 T_{n/8} + (n - 4) + (n - 2) + (n - 1) & \text{chug} \\
  & = 8 (2 T_{n/16} + n / 8 - 1) +  (n - 4) + (n - 2) + (n - 1) &
  \text{plug} \\
  & = 16 T_{n/16} + (n - 8) + (n - 4) + (n - 2) + (n - 1) & \text{chug}
\end{align*}
A pattern is emerging.  In particular, this formula seems holds:
\begin{align*}
T_n & = 2^k T_{n/2^k} + (n - 2^{k-1}) + (n - 2^{k -2}) + \cdots + (n -
2^0) \\
  & = 2^k T_{n/2^k} + k n - 2^{k-1}  - 2^{k-2} \cdots - 2^0 \\
  & = 2^k T_{n/2^k} + k n - 2^k + 1.
\end{align*}
On the second line, we grouped the $n$ terms and powers of $2$.  On
the third, we collapsed the geometric sum.

\subsubsection*{Step 2:  Verify the Pattern}

Next, we verify the pattern with one additional round of
plug-and-chug.  If we guessed the wrong pattern, then this is where
we'll discover the mistake.
\begin{align*}
T_n & = 2^k T_{n/2^k} + k n - 2^k + 1 \\
 & = 2^k (2 T_{n/2^{k+1}} + n/2^k - 1) + kn - 2^k + 1 & \text{plug} \\
 & = 2^{k+1} T_{n/2^{k+1}}  + (k+ 1) n - 2^{k+1} + 1 & \text{chug}
\end{align*}
The formula is unchanged except that $k$ is replaced by $k+1$.  This
amounts to the induction step in a proof that the formula holds for
all $k \geq 1$.

\subsubsection*{Step 3:  Write $T_n$ Using Early Terms with Known Values}

Finally, we express $T_n$ using early terms whose values are known.
Specifically, if we let $k = \log n$, then $T_{n/2^k} = T_1$, which we
know is~0:
\begin{align*}
T_n & = 2^k T_{n/2^k} + kn - 2^k + 1 \\
  & = 2^{\log n} T_{n/2^{\log n}} + n \log n - 2^{\log n} + 1 \\
  & = n T_1 + n \log n - n + 1 \\
  & = n \log n - n + 1.
\end{align*}
We're done!  We have a closed-form expression for the maximum number
of comparisons used in Merge Sorting a list of $n$ numbers.  In
retrospect, it is easy to see why guess-and-verify failed:  this
formula is fairly complicated.

As a check, we can confirm that this formula gives the same values
that we computed earlier:
\[
\begin{array}{c|c|c}
n & T_n & n \log n - n + 1 \\ \hline
1 & 0 & 1 \log 1 - 1 + 1 = 0 \\
2 & 1 & 2 \log 2 - 2 + 1 = 1 \\
4 & 5 & 4 \log 4 - 4 + 1 = 5 \\
8 & 17 & 8 \log 8 - 8 + 1 = 17 \\
16 & 49 & 16 \log 16 - 16 + 1 = 49
\end{array}
\]
As a double-check, we could write out an explicit induction proof.
This would be straightforward, because we already worked out the guts
of the proof in step 2 of the plug-and-chug procedure.

\section{Linear Recurrences}

So far we've solved recurrences with two techniques: guess-and-verify
and plug-and-chug.  These methods require spotting a pattern in a
sequence of numbers or expressions.  In this section and the next,
we'll give cookbook solutions for two large classes of recurrences.
These methods require no flash of insight; you just follow the recipe
and get the answer.

\subsection{Climbing Stairs}

How many different ways are there to climb $n$ stairs, if you can
either step up one stair or hop up two?  For example, there are five
different ways to climb four stairs:
\begin{enumerate}
\item step, step, step, step
\item hop, hop
\item hop, step, step
\item step, hop step
\item step, step, hop
\end{enumerate}

\noindent Working through this problem will demonstrate the major
features of our first cookbook method for solving recurrences.  We'll
fill in the details of the general solution afterward.

\subsubsection{Finding a Recurrence}

As special cases, there is 1 way to climb 0 stairs (do nothing) and 1
way to climb 1 stair (step up).  In general, an ascent of $n$ stairs
consists of either a step followed by an ascent of the remaining $n -
1$ stairs or a hop followed by an ascent of $n - 2$ stairs.  So the
total number of ways to climb $n$ stairs is equal to the number of
ways to climb $n-1$ plus the number of ways to climb $n-2$.  These
observations define a recurrence:
\begin{align*}
f(0) & = 1 \\
f(1) & = 1 \\
f(n) & = f(n-1) + f(n-2) &\text{for $n \geq 2$}.
\end{align*}
Here, $f(n)$ denotes the number of ways to climb $n$ stairs.  Also,
we've switched from subscript notation to functional notation, from
$T_n$ to $f_n$.  Here the change is cosmetic, but the expressiveness
of functions will be useful later.

This is the Fibonacci%
\index{Fibonnaci numbers} recurrence, the most famous of all
recurrence equations.  Fibonacci numbers arise in all sorts of
applications and in nature.  Fibonacci introduced the numbers in 1202
to study rabbit reproduction.  Fibonacci numbers also appear, oddly
enough, in the spiral patterns on the faces of sunflowers.  And the
input numbers that make Euclid's GCD algorithm require the greatest
number of steps are consecutive Fibonacci numbers.

\subsubsection{Solving the Recurrence}

The Fibonacci recurrence belongs to the class of linear recurrences,
which are essentially all solvable with a technique that you can learn
in an hour.  This is somewhat amazing, since the Fibonacci recurrence
remained unsolved for almost six centuries!

In general, a \term{homogeneous linear recurrence} has the form
\begin{align*}
f(n) = a_1 f(n-1) + a_2 f(n-2) + \cdots + a_d f(n - d)
\end{align*}
where $a_1, a_2, \dots, a_d$ and $d$ are constants.  The \term{order}
of the recurrence is $d$.  Commonly, the value of the function $f$ is
also specified at a few points; these are called \term{boundary
  conditions}.  For example, the Fibonacci recurrence has order $d =
2$ with coefficients $a_1 = a_2 = 1$ and $g(n) = 0$.  The boundary
conditions are $f(0) = 1$ and $f(1) = 1$.  The word ``homogeneous''
sounds scary, but effectively means ``the simpler kind.''  We'll
consider linear recurrences with a more complicated form later.

Let's try to solve the Fibonacci recurrence with the benefit centuries
of hindsight.  In general, linear recurrences tend to have exponential
solutions.  So let's guess that
\begin{align*}
  f(n) = x^n
\end{align*}
where $x$ is a parameter introduced to improve our odds of making a
correct guess.  We'll figure out the best value for $x$ later.  To
further improve our odds, let's neglect the boundary conditions, $f(0)
= 0$ and $f(1) = 1$, for now.  Plugging this guess into the recurrence
$f(n) = f(n - 1) + f(n - 2)$ gives
\[
x^n  = x^{n-1} + x^{n-2}.
\]
Dividing both sides by $x^{n-2}$ leaves a quadratic equation:
\[
x^2 = x + 1.
\]
Solving this equation gives \emph{two} plausible values for the
parameter $x$:
\[
x = \frac{1 \pm \sqrt{5}}{2}.
\]
This suggests that there are at least two different solutions to the
recurrence, neglecting the boundary conditions.
\[
  f(n) = \paren{\frac{1 + \sqrt{5}}{2}}^n \quad \text{or}\quad
  f(n) = \paren{\frac{1 - \sqrt{5}}{2}}^n
\]

A charming features of homogeneous linear recurrences is that any
linear combination of solutions is another solution.
\begin{theorem}
\label{th:recur-linearity}
  If $f(n)$ and $g(n)$ are both solutions to a homogeneous linear
  recurrence, then $h(n) = s f(n) + t g(n)$ is also a solution for all $s, t \in
  \mathbb{R}$.
\end{theorem}
\begin{proof}
\begin{align*}
h(n) & = sf(n) + tg(n) \\
  & = s \paren{a_1 f(n-1) + \cdots + a_d f(n - d)}
        + t \paren{a_1 g(n-1) + \cdots + a_d g(n - d)} \\
 & = a_1 (s f(n-1) + t g(n-1)) + \cdots + a_d (s f(n-d) + t g(n-d)) \\
  & =  a_1 h(n-1) + \cdots + a_d h(n-d)
\end{align*}
The first step uses the definition of the function $h$, and the second
uses the fact that $f$ and $g$ are solutions to the recurrence.  In
the last two steps, we rearrange terms and use the definition of $h$
again.  Since the first expression is equal to the last, $h$ is also a
solution to the recurrence.
\end{proof}

The phenomenon described in this theorem---a linear combination of
solutions is another solution---also holds for many differential
equations and physical systems.  In fact, linear recurrences are so
similar to linear differential equations that you can safely snooze
through that topic in some future math class.

Returning to the Fibonacci recurrence, this theorem implies that
\[
  f(n) = s \paren{\frac{1 + \sqrt{5}}{2}}^n + t \paren{\frac{1 - \sqrt{5}}{2}}^n
\]
is a solution for all real numbers $s$ and $t$.  The theorem expanded
two solutions to a whole spectrum of possibilities!  Now, given all
these options to choose from, we can find one solution that satisfies
the boundary conditions, $f(0) = 1$ and $f(1) = 1$.  Each boundary
condition puts some constraints on the parameters $s$ and $t$.  In
particular, the first boundary condition implies that
\[
f(0) = s \paren{\frac{1 + \sqrt{5}}{2}}^0 +
       t \paren{\frac{1 - \sqrt{5}}{2}}^0 = s + t = 1.
\]
Similarly, the second boundary condition implies that
\[
f(1) = s \paren{\frac{1 + \sqrt{5}}{2}}^1 + t \paren{\frac{1 - \sqrt{5}}{2}}^1 = 1.
\]
Now we have two linear equations in two unknowns.  The system is not
degenerate, so there is a unique solution:
\[
s = \frac{1}{\sqrt{5}} \cdot \frac{1 + \sqrt{5}}{2}\qquad
t = - \frac{1}{\sqrt{5}} \cdot \frac{1 - \sqrt{5}}{2}.
\]
These values of $s$ and $t$ identify a solution to the Fibonacci
recurrence that also satisfies the boundary conditions:
\begin{align*}
  f(n) & = \frac{1}{\sqrt{5}} \cdot \frac{1 + \sqrt{5}}{2} \left(\frac{1 + \sqrt{5}}{2}\right)^n-
  \frac{1}{\sqrt{5}} \cdot \frac{1 - \sqrt{5}}{2} \left(\frac{1 - \sqrt{5}}{2}\right)^n \\
  & = \frac{1}{\sqrt{5}} \left(\frac{1 + \sqrt{5}}{2}\right)^{n+1} -
  \frac{1}{\sqrt{5}} \left(\frac{1 - \sqrt{5}}{2}\right)^{n+1}.
\end{align*}
It is easy to see why no one stumbled across this solution for almost
six centuries.  All Fibonacci numbers are integers, but this expression
is full of square roots of five!  Amazingly, the square roots always
cancel out.  This expression really does give the Fibonacci numbers if
we plug in $n = 0, 1, 2$, etc.

This closed form for Fibonacci numbers is known as \idx{Binet's
  formula} and has some interesting corollaries.  The first term tends
to infinity because the base of the exponential, $(1+\sqrt{5})/2 =
1.618\dots$ is greater than one.  This value is often denoted $\phi$
and called the ``golden ratio.''  The second term tends to zero,
because $(1-\sqrt{5})/2 = -0.618033988\dots$ has absolute value less
than 1.  This implies that the $n$th Fibonacci number is:
\[
f(n) = \frac{\phi^{n+1}}{\sqrt{5}} + o(1).
\]
Remarkably, this expression involving irrational numbers is actually
very close to an integer for all large $n$---namely, a Fibonacci
number!  For example:
\[
\frac{\phi^{20}}{\sqrt{5}} = 6765.000029\dots \approx f(19).
\]
This also implies that the ratio of consecutive
Fibonacci numbers rapidly approaches the golden ratio.  For example:
\[
\frac{f(20)}{f(19)} = \frac{10946}{6765} = 1.618033998\dots.
\]

\subsection{Solving Homogeneous Linear Recurrences}

The method we used to solve the Fibonacci recurrence can be extended
to solve any homogeneous linear recurrence; that is, a recurrence of
the form
\[
f(n) = a_1 f(n-1) + a_2 f(n-2) + \cdots + a_d f(n - d)
\]
where $a_1, a_2, \dots, a_d$ and $d$ are constants.  Substituting the
guess $f(n) = x^n$, as with the Fibonacci recurrence, gives
\[
   x^n = a_1x^{n-1} +a_2x^{n-2} + \cdots +a_dx^{n-d}.
\]
Dividing by $x^{n-d}$ gives
\[
x^d = a_1x^{d-1} + a_2x^{d-2} + \cdots+a_{d-1}x+a_d.
\]
This is called the \term{characteristic equation} of the
recurrence.  The characteristic equation can be read off quickly
since the coefficients of the equation are the same as the
coefficients of the recurrence.

The solutions to a linear recurrence are defined by the roots of the
characteristic equation.  Neglecting boundary conditions for the
moment:
\begin{itemize}
\item If $r$ is a nonrepeated root of the characteristic equation,
  then $r^n$ is a solution to the recurrence.
\item If $r$ is a repeated root with multiplicity $k$ then $r^n$,
  $nr^n$, $n^2r^n$, \dots, $n^{k-1}r^n$ are all solutions to the
  recurrence.
\end{itemize}
Theorem~\ref{th:recur-linearity} implies that every linear
  combination of these solutions is also a solution.

For example, suppose that the characteristic equation of a
recurrence has roots $s$, $t$, and $u$ twice.  These four roots
imply four distinct solutions:
\[
f(n) = s^n \qquad f(n) = t^n \qquad f(n) = u^n \qquad f(n)
  = nu^n.
\]
Furthermore, every linear combination
\begin{equation}
f(n) = a \cdot s^n + b \cdot t^n + c \cdot u^n +d \cdot nu^n
\end{equation}
is also a solution.

All that remains is to select a solution consistent with the boundary
conditions by choosing the constants appropriately.  Each boundary
condition implies a linear equation involving these constants.  So we
can determine the constants by solving a system of linear
equations.  For example, suppose our boundary conditions were $f(0) =
0$, $f(1) = 1$, $f(2) = 4$, and $f(3) = 9$.  Then we would obtain four
equations in four unknowns:
\[
\begin{array}{r@{\quad\qimplies\quad}l}
f(0) = 0 & a \cdot s^0 + b \cdot t^0 + c \cdot u^0 + d \cdot 0u^0 = 0 \\
f(1) = 1 & a \cdot s^1 + b \cdot t^1 + c \cdot u^1 + d \cdot 1u^1 = 1 \\
f(2) = 4 & a \cdot s^2 + b \cdot t^2 + c \cdot u^2 + d \cdot 2u^2 = 4 \\
f(3) = 9 & a \cdot s^3 + b \cdot t^3 + c \cdot u^3 + d \cdot 3u^3 = 9
\end{array}
\]
This looks nasty, but remember that $s$, $t$, and $u$ are just
constants.  Solving this system gives values for $a$, $b$, $c$, and $d$
that define a solution to the recurrence consistent with the boundary
conditions.

\subsection{Solving General Linear Recurrences}

We can now solve all linear homogeneous recurrences, which have the
form
\begin{align*}
f(n) = a_1 f(n-1) + a_2 f(n-2) + \cdots + a_d f(n - d).
\end{align*}
Many recurrences that arise in practice do not quite fit this mold.
For example, the Towers of Hanoi problem led to this recurrence:
\begin{align*}
f(1) & = 1 \\
f(n) & = 2 f(n - 1) + 1 & \text{(for $n \geq 2$)}.
\end{align*}
The problem is the extra $+1$; that is not allowed in a homogeneous
linear recurrence.  In general, adding an extra function $g(n)$ to the
right side of a linear recurrence gives an \term{inhomogeneous linear
  recurrence}:
\begin{align*}
f(n) = a_1 f(n-1) + a_2 f(n-2) + \cdots + a_d f(n - d) + g(n).
\end{align*}

Solving inhomogeneous linear recurrences is neither very different nor
very difficult.  We can divide the whole job into five steps:

\begin{enumerate}

\item Replace $g(n)$ by 0, leaving a homogeneous recurrence.  As
  before, find roots of the characteristic equation.

\item Write down the solution to the homogeneous recurrence, but do
  not yet use the boundary conditions to determine coefficients.  This
  is called the \term{homogeneous solution}.

\item Now restore $g(n)$ and find a single solution to the recurrence,
  ignoring boundary conditions.  This is called a \term{particular
   solution}.  We'll explain how to find a particular solution
  shortly.

\item Add the homogeneous and particular solutions together to obtain
  the \emph{general solution}.

\item Now use the boundary conditions to determine constants by the
  usual method of generating and solving a system of linear equations.

\end{enumerate}

As an example, let's consider a variation of the Towers of Hanoi
problem.  Suppose that moving a disk takes time proportional to its
size.  Specifically, moving the smallest disk takes 1 second, the
next-smallest takes 2 seconds, and moving the $n$th disk then
requires $n$ seconds instead of 1.  So, in this variation, the time to
complete the job is given by a recurrence with a $+n$ term instead of
a $+1$:
\begin{align*}
f(1) & = 1 \\
f(n) & = 2 f(n - 1) + n & \text{for $n \geq 2$}.
\end{align*}
Clearly, this will take longer, but how much longer?  Let's solve the
recurrence with the method described above.

In Steps 1 and 2, dropping the $+n$ leaves the homogeneous recurrence
$f(n) = 2 f(n -1)$.  The characteristic equation is $x = 2$.  So the
homogeneous solution is $f(n) = c2^n$.

In Step 3, we must find a solution to the full recurrence $f(n) = 2
f(n - 1) + n$, without regard to the boundary condition.  Let's guess
that there is a solution of the form $f(n) = a n + b$ for some
constants $a$ and $b$.  Substituting this guess into the recurrence
gives
\begin{align*}
a n + b & = 2 (a (n - 1) + b) + n \\
0 & = (a + 1) n + (b - 2 a).
\end{align*}
The second equation is a simplification of the first.  The second
equation holds for all $n$ if both $a + 1 = 0$ (which implies $a =
-1$) and $b - 2a = 0$ (which implies that $b = -2$).  So $f(n) = an +
b = -n - 2$ is a particular solution.

In the Step 4, we add the homogeneous and particular solutions to
obtain the general solution
\begin{align*}
f(n) & = c 2^n - n - 2.
\end{align*}

Finally, in step~5, we use the boundary condition, $f(1) = 1$,
determine the value of the constant~$c$:
\begin{align*}
f(1) = 1 \quad & \QIMPLIES \quad c 2^1 - 1 - 2 = 1 \\
  & \QIMPLIES \quad c = 2.
\end{align*}
Therefore, the function $f(n) = 2 \cdot 2^n - n - 2$ solves this
variant of the Towers of Hanoi recurrence.  For comparison, the
solution to the original Towers of Hanoi problem was $2^n - 1$.  So if
moving disks takes time proportional to their size, then the monks
will need about twice as much time to solve the whole puzzle.

\subsection{How to Guess a Particular Solution}

Finding a particular solution can be the hardest part of solving
inhomogeneous recurrences.  This involves guessing, and you might
guess wrong.\footnote{Chapter~\ref{generating_function_chap} explains
  how to solve linear recurrences with generating functions---it's a
  little more complicated, but it does not require guessing.}
However, some rules of thumb make this job fairly easy most of the
time.

\begin{itemize}
\item Generally, look for a particular solution with the same form as
  the inhomogeneous term $g(n)$.
\item If $g(n)$ is a constant, then guess a particular solution $f(n)
  = c$.  If this doesn't work, try polynomials of progressively higher
  degree:  $f(n)=bn+c$, then $f(n)=an^2 +bn+c$, etc.
\item More generally, if $g(n)$ is a polynomial, try a polynomial of
  the same degree, then a polynomial of degree one higher, then two
  higher, etc.  For example, if $g(n) = 6n + 5$, then try $f(n)=bn+c$
  and then $f(n)=an^2 +bn+c$.
\item If $g(n)$ is an exponential, such as $3^n$, then first guess
  that $f(n) = c3^n$.  Failing that, try $f(n) = bn3^n + c3^n$ and then
  $an^23^n + bn3^n + c3^n$, etc.
\end{itemize}

The entire process is summarized on the following page.

\begin{figure}[p]\redrawntrue

\begin{pagesidebar}[to \textheight]
\textboxtitle{Short Guide to Solving Linear Recurrences}

A linear recurrence is an equation
\begin{equation*}
f(n) = \underbrace{a_1 f(n-1) + a_2 f(n-2) + \cdots + a_d f(n -
  d)}_{\text{homogeneous part}}
\underbrace{+\ g(n)}_{\text{inhomogeneous part}}
\end{equation*}
together with boundary conditions such as $f(0) = b_0$, $f(1) = b_1$,
etc.  Linear recurrences are solved as follows:

\begin{enumerate}
\item Find the roots of the characteristic equation
\begin{equation*}
x^n = a_1 x^{n-1} + a_2 x^{n-2} + \cdots + a_{k-1} x + a_k.
\end{equation*}
\item Write down the homogeneous solution.  Each root generates one
  term and the homogeneous solution is their sum.  A nonrepeated
  root~$r$ generates the term~$c r^n$, where $c$ is a constant to be 
  determined later.  A root $r$ with multiplicity $k$ generates the
  terms
\[
d_{1} r^n \qquad d_2 n r^n \qquad d_3 n^2 r^n \qquad \dots \qquad d_k n^{k-1} r^n
\]
where $d_1, \dots d_k$ are constants to be determined later.
\item Find a particular solution.  This is a solution to the full
  recurrence that need not be consistent with the boundary conditions.
  Use guess-and-verify.  If $g(n)$ is a constant or a polynomial, try
  a polynomial of the same degree, then of one higher degree, then two
  higher.  For example, if $g(n) = n$, then try $f(n) = bn + c$ and
  then $an^2 + bn + c$.  If $g(n)$ is an exponential, such as~$3^n$,
  then first guess $f(n) = c3^n$.  Failing that, try $f(n) = (bn + c) 3^n$
  and then $(an^2 + bn + c)3^n$, etc.
\item Form the general solution, which is the sum of the homogeneous
  solution and the particular solution.  Here is a typical general
  solution:
\begin{equation*}
f(n) = \underbrace{c2^n + d(-1)^n}_{\text{homogeneous solution}} +
\underbrace{3n + 1\rlap{.}}_{\text{inhomogeneous solution}}
\end{equation*}
\item Substitute the boundary conditions into the general solution.
  Each boundary condition gives a linear equation in the unknown
  constants.  For example, substituting $f(1) = 2$ into the general
  solution above gives
\begin{align*}
2 & = c\cdot2^1 + d \cdot(-1)^1 + 3 \cdot 1 + 1 \\
\QIMPLIES \quad -2 & = 2c - d.
\end{align*}
Determine the values of these constants by solving the resulting
system of linear equations.
\end{enumerate}
\end{pagesidebar}

\end{figure}

\section{Divide-and-Conquer Recurrences}

We now have a recipe for solving general linear recurrences.  But the
Merge Sort recurrence, which we encountered earlier, is not linear:
\begin{align*}
T(1) & = 0 \\
T(n) & = 2 T(n/2) + n - 1 & \text{(for $n \geq 2$)}.
\end{align*}
In particular, $T(n)$ is not a linear combination of a fixed number of
immediately preceding terms; rather, $T(n)$ is a function of $T(n/2)$,
a term halfway back in the sequence.

Merge Sort is an example of a divide-and-conquer algorithm: it divides
the input, ``conquers'' the pieces, and combines the results.
Analysis of such algorithms commonly leads to \term{divide-and-conquer}
recurrences, which have this form:
\[
T(n) =  \sum_{i=1}^k a_i T(b_i n) + g(n)
\]
Here $a_1, \dots a_k$ are positive constants, $b_1, \dots, b_k$ are
constants between 0 and 1, and $g(n)$ is a nonnegative function.  For
example, setting $a_1 = 2$, $b_1 = 1/2$, and $g(n) = n - 1$ gives the
Merge Sort recurrence.

\subsection{The Akra-Bazzi Formula}

The solution to virtually all divide and conquer solutions is given by
the amazing \term{Akra-Bazzi formula}.  Quite simply, the asymptotic
solution to the general divide-and-conquer recurrence
\[
T(n) = \sum_{i=1}^k a_i T(b_i n) + g(n)
\]
is
\begin{equation}\label{eqn:10C1}
T(n) = \Theta\paren{n^p \paren{1 + \int_1^n \frac{g(u)}{u^{p+1}}\ du}}
\end{equation}
where $p$ satisfies
\begin{equation}\label{eqn:10C2}
\sum_{i=1}^k {a_i b_i^p} = 1.
\end{equation}

A rarely-troublesome requirement is that the function $g(n)$ must not
grow or oscillate too quickly.  Specifically, $\card{g'(n)}$ must be
bounded by some polynomial.  So, for example, the Akra-Bazzi formula
is valid when $g(n) = x^2 \log n$, but not when $g(n) = 2^n$.

Let's solve the Merge Sort recurrence again, using the Akra-Bazzi
formula instead of plug-and-chug.  First, we find the value $p$
that satisfies
\[
   2 \cdot (1/2)^p = 1.
\]
Looks like $p = 1$ does the job.  Then we compute the integral:
\begin{align*}
T(n) & = \Theta\paren{n \paren{1 + \int_1^n \frac{u - 1}{u^2}\ du}} \\
  & = \Theta\paren{n \paren{1 + \brac{\log u + \frac{1}{u}}_1^n}}\\
  & = \Theta\paren{n \paren{\log n + \frac{1}{n}}}\\
  & = \Theta(n \log n).
\end{align*}
The first step is integration and the second is simplification.  We
can drop the $1/n$ term in the last step, because the $\log n$ term
dominates.  We're done!

Let's try a scary-looking recurrence:
\begin{align*}
   T(n) & = 2T(n/2) + (8/9)T(3n/4) + n^2.
\end{align*}
Here, $a_1 = 2$, $b_1 = 1/2$, $a_2 = 8/9$, and $b_2 = 3/4$.  So we
find the value $p$ that satisfies
\[
   2 \cdot (1/2)^p + (8/9) (3/4)^p = 1.
\]
Equations of this form don't always have closed-form solutions, so you
may need to approximate $p$ numerically sometimes.  But in this case
the solution is simple: $p = 2$.  Then we integrate:
\begin{align*}
T(n) & = \Theta\left(
n^2 \left(1 + \int_1^n \frac{u^2}{u^3}\ du \right) \right) \\
  & = \Theta\left(
n^2 (1 + \log n) \right) \\
  & = \Theta\left(n^2 \log n\right).
\end{align*}
That was easy!

\subsection{Two Technical Issues}

Until now, we've swept a couple issues related to divide-and-conquer
recurrences under the rug.  Let's address those issues now.

First, the Akra-Bazzi formula makes no use of boundary conditions.  To
see why, let's go back to Merge Sort.  During the plug-and-chug
analysis, we found that
\[
   T_n = n T_1 + n \log n - n + 1.
\]
This expresses the $n$th term as a function of the first term, whose
value is specified in a boundary condition.  But notice that $T_n =
\Theta(n \log n)$ for \emph{every} value of $T_1$.  The boundary
condition doesn't matter!

This is the typical situation: \emph{the asymptotic solution to a
  divide-and-conquer recurrence is independent of the boundary
  conditions}.  Intuitively, if the bottom-level operation in a
recursive algorithm takes, say, twice as long, then the overall
running time will at most double.  This matters in practice, but the
factor of 2 is concealed by asymptotic notation.  There are
corner-case exceptions.  For example, the solution to $T(n) = 2
T(n/2)$ is either $\Theta(n)$ or zero, depending on whether $T(1)$ is
zero.  These cases are of little practical interest, so we won't
consider them further.

There is a second nagging issue with divide-and-conquer recurrences
that does not arise with linear recurrences.  Specifically, dividing
a problem of size $n$ may create subproblems of non-integer size.  For
example, the Merge Sort recurrence contains the term $T(n/2)$.  So
what if $n$ is 15?  How long does it take to sort seven-and-a-half
items?  Previously, we dodged this issue by analyzing Merge Sort only
when the size of the input was a power of 2.  But then we don't know
what happens for an input of size, say, 100.

Of course, a practical implementation of Merge Sort would split the
input \emph{approximately} in half, sort the halves recursively, and
merge the results.  For example, a list of 15 numbers would be split
into lists of 7 and 8.  More generally, a list of $n$ numbers would be
split into approximate halves of size $\lceil n / 2 \rceil$ and
$\lfloor n / 2 \rfloor$.  So the maximum number of comparisons is
actually given by this recurrence:
\begin{align*}
T(1) & = 0 \\
T(n) & = T(\ceil{n/2}) + T(\floor{n/2})+ n - 1 & \text{(for $n \geq 2$)}.
\end{align*}
This may be rigorously correct, but the ceiling and floor operations
make the recurrence hard to solve exactly.

Fortunately, \emph{the asymptotic solution to a divide and conquer
  recurrence is unaffected by floors and ceilings}.  More precisely,
the solution is not changed by replacing a term $T(b_i n)$ with either
$T(ceil{b_i n})$ or $T(\floor{b_i n})$.  So leaving
floors and ceilings out of divide-and-conquer recurrences makes sense
in many contexts; those are complications that make no difference.

\subsection{The Akra-Bazzi Theorem}

The Akra-Bazzi formula together with our assertions about boundary
conditions and integrality all follow from the
\term{Akra-Bazzi Theorem}, which is stated below.

\begin{theorem}[Akra-Bazzi]
\label{th:akra-bazzi}
Suppose that the function $T: \reals \to \reals$ is nonnegative and
bounded for $0 \leq x \leq x_0$ and satisfies the recurrence
\begin{equation}\label{A-B_recurrence}
T(x) = \sum\limits_{i=1}^k a_i T(b_i x + h_i(x)) + g(x) \qquad \text{for $x > x_0$},
\end{equation}
where:
\begin{enumerate}
\item $x_0$ is large enough so that $T$ is well-defined,
\item $a_1, \dots, a_k$ are positive constants,
\item $b_1, \dots, b_k$ are constants between 0 and 1,
\item $g(x)$ is a nonnegative function such that $\card{g'(x)}$ is bounded
by a polynomial,
\item $\card{h_i(x)} = O(x / \log^2 x)$.
\end{enumerate}
Then
\[
T(x) = \Theta\paren{x^p \paren{1 + \int_1^x \frac{g(u)}{u^{p+1}}\ du}}
\]
where $p$ satisfies
\[
\sum_{i=1}^k {a_i b_i^p} = 1.
\]
\end{theorem}

The Akra-Bazzi theorem can be proved using a complicated induction
argument, though we won't do that here.  But let's at least go over
the statement of the theorem.

All the recurrences we've considered were defined over the integers, and
that is the common case.  But the Akra-Bazzi theorem applies more
generally to functions defined over the real numbers.

The Akra-Bazzi formula is lifted directed from the theorem statement,
except that the recurrence in the theorem includes extra functions,
$h_i$.  These functions extend the theorem to address floors,
ceilings, and other small adjustments to the sizes of subproblems.
The trick is illustrated by this combination of parameters
\begin{align*}
a_1 &= 1 & b_1 & = 1/2 & h_1(x) & =  \ceil{\frac{x}{2}} - \frac{x}{2}\\
a_2 &= 1 & b_2 & = 1/2 & h_2(x) & = \floor{\frac{x}{2}} - \frac{x}{2} \\
& & g(x) & = x - 1
\end{align*}
which corresponds the recurrence
\begin{align*}
T(x) & = 1 \cdot T\paren{\frac{x}{2} + \paren{ \ceil{\frac{x}{2}} -\frac{x}{2}}} + 
           \cdot T\paren{\frac{x}{2} + \paren{\floor{\frac{x}{2}} - \frac{x}{2}}} + x - 1 \\
 & = T\paren{\ceil{\frac{x}{2}}} +  T\paren{\floor{\frac{x}{2}}} + x - 1.
\end{align*}

This is the rigorously correct Merge Sort recurrence valid for all
input sizes, complete with floor and ceiling operators.  In this case,
the functions $h_1(x)$ and $h_2(x)$ are both at most 1, which is
easily $O(x / \log^2 x)$ as required by the theorem statement.  These
functions $h_i$ do not affect---or even appear in---the asymptotic
solution to the recurrence.  This justifies our earlier claim that
applying floor and ceiling operators to the size of a subproblem does
not alter the asymptotic solution to a divide-and-conquer recurrence.

\subsection{The Master Theorem}

There is a special case of the Akra-Bazzi formula known as the Master
Theorem that handles some of the recurrences that commonly arise in
computer science.  It is called the \emph{Master} Theorem because it
was proved long before Akra and Bazzi arrived on the scene and, for
many years, it was the final word on solving divide-and-conquer
recurrences.  We include the Master Theorem here because it is still
widely referenced in algorithms courses and you can use it without
having to know anything about integration.

\begin{theorem}[Master Theorem]\label{thm:master_theorem}
Let $T$ be a recurrence of the form
\[
   T(n) = a T\paren{\frac{n}{b}} + g(n).
\]
\begin{description}

\item[Case 1:]
If $g(n) = O\paren{n^{\log_b(a) - \epsilon}}$ for some constant
$\epsilon > 0$, then
\[
T(n) = \Theta\paren{n^{\log_b(a)}}.
\]

\item[Case 2:]
If $g(n) = \Theta\paren{n^{\log_b(a)} \log^k(n)}$ for some constant
$k \ge 0$, then
\[
T(n) = \Theta\paren{n^{\log_b(a)} \log^{k + 1}(n)}.
\]

\item[Case 3:]

If $g(n) = \Omega\paren{n^{\log_b(a) + \epsilon}}$ for some constant
$\epsilon > 0$ and $a g(n/b) < c g(n)$ for some constant~$c < 1$ and
sufficiently large~$n$, then
\[
T(n) = \Theta(g(n)).
\]

\end{description}
\end{theorem}

The Master Theorem can be proved by induction on~$n$ or, more easily,
as a corollary of Theorem~\ref{th:akra-bazzi}.  We will not include
the details here.

\begin{editingnotes}
\subsection*{Pitfalls with Asymptotic Notation and Induction}
MOVED to Asymptotics Chapter~\ref{chap:asymptotics} and commented out
because redundant. --ARM
\end{editingnotes}

\begin{problems}
\homeworkproblems
\pinput{PS_AkraB_fall04}
\pinput{PS_AkraB_multipart}

\classproblems
\pinput{CP_AkraB}

\examproblems
\pinput{FP_AkraB3}

\end{problems}

\section{A Feel for Recurrences}

We've guessed and verified, plugged and chugged, found roots, computed
integrals, and solved linear systems and exponential equations.  Now
let's step back and look for some rules of thumb.  What kinds of
recurrences have what sorts of solutions?

Here are some recurrences we solved earlier:
\[
\begin{array}{lll}
& \text{Recurrence} & \text{Solution} \\
\text{Towers of Hanoi} & T_n = 2 T_{n-1} + 1 & T_n \sim 2^n \\
\text{Merge Sort} & T_n = 2 T_{n/2} + n - 1 & T_n \sim n \log n \\
\text{Hanoi variation} & T_n = 2 T_{n-1} + n & T_n \sim 2 \cdot 2^n \\
\text{Fibonacci} & T_n = T_{n-1} + T_{n-2} & T_n \sim
(1.618\dots)^{n+1} / \sqrt{5}
\end{array}
\]
Notice that the recurrence equations for Towers of Hanoi and Merge
Sort are somewhat similar, but the solutions are radically different.
Merge Sorting $n = 64$ items takes a few hundred comparisons, while
moving $n = 64$ disks takes more than $10^{19}$ steps!

Each recurrence has one strength and one weakness.  In the Towers of
Hanoi, we broke a problem of size $n$ into two subproblem of size $n -
1$ (which is large), but needed only 1 additional step (which is
small).  In Merge Sort, we divided the problem of size $n$ into two
subproblems of size $n/2$ (which is small), but needed $(n - 1)$
additional steps (which is large).  Yet, Merge Sort is faster by a
mile!

This suggests that \emph{generating smaller subproblems is far more
  important to algorithmic speed than reducing the additional steps
  per recursive call}.  For example, shifting to the variation of
Towers of Hanoi increased the last term from $+1$ to $+n$, but the
solution only doubled.  And one of the two subproblems in the
Fibonacci recurrence is just \emph{slightly} smaller than in Towers of
Hanoi (size $n -2$ instead of $n-1$).  Yet the solution is
exponentially smaller!  More generally, linear recurrences (which have
big subproblems) typically have exponential solutions, while
divide-and-conquer recurrences (which have small subproblems) usually
have solutions bounded above by a polynomial.

All the examples listed above break a problem of size $n$ into two
smaller problems.  How does the number of subproblems affect the
solution?  For example, suppose we increased the number of subproblems
in Towers of Hanoi from 2 to 3, giving this recurrence:
\begin{equation*}
T_n = 3 T_{n-1} + 1
\end{equation*}
This increases the root of the characteristic equation from 2 to 3,
which raises the solution exponentially, from $\Theta(2^n)$ to
$\Theta(3^n)$.

Divide-and-conquer recurrences are also sensitive to the number of
subproblems.  For example, for this generalization of the Merge Sort
recurrence:
\begin{align*}
T_1 & = 0 \\
T_{n} & = a T_{n/2} + n - 1.
\end{align*}
the Akra-Bazzi formula gives:
\begin{equation*}
T_n = \begin{cases}
\Theta(n) & \text{for $a < 2$} \\
\Theta(n \log n) & \text{for $a = 2$} \\
\Theta(n^{\log a}) & \text{for $a > 2$}.
\end{cases}
\end{equation*}
So the solution takes on three completely different forms as $a$ goes
from 1.99 to~2.01!

How do boundary conditions affect the solution to a recurrence?  We've
seen that they are almost irrelevant for divide-and-conquer
recurrences.  For linear recurrences, the solution is usually
dominated by an exponential whose base is determined by the number and
size of subproblems.  Boundary conditions matter greatly only when
they give the dominant term a zero coefficient, which changes the
asymptotic solution.

So now we have a rule of thumb!  The performance of a recursive
procedure is usually dictated by the size and number of subproblems,
rather than the amount of work per recursive call or time spent at the
base of the recursion.  In particular, if subproblems are smaller than
the original by an additive factor, the solution is most often
exponential.   But if the subproblems are only a fraction the size of
the original, then the solution is typically bounded by a polynomial.

\begin{editingnotes}
\TBA{More problems}
\end{editingnotes}

\endinput
