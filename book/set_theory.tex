\chapter{Infinite Sets}\label{infinite_chap}\label{set_theory_chap}

This chapter is about infinite sets%
\index{set!infinite set}%
\index{infinity|see{countable, Mapping Rules, set, set theory}}  
and some challenges in proving things about them.

Wait a minute!  Why bring up infinity in a Mathematics for
\emph{Computer Science} text?  After all, any data set in a computer
is limited by the size of the computer's memory, and there is a bound
on the possible size of computer memory, for the simple reason that
the universe is (or at least appears to be) bounded.  So why not stick
with \emph{finite} sets of some large, but bounded, size?  This
is a good question, but let's see if we can persuade you that dealing
with infinite sets is inevitable.

\iffalse We've run into a lot of computer science students who wonder
why they should care about infinite sets.  They point out that
\fi

You may not have noticed, but up to now you've already accepted the
routine use of the integers, the rationals and irrationals, and
sequences of them.  These are all infinite sets.  Further, do you
really want Physics or the other sciences to give up the real numbers
on the grounds that only a bounded number of bounded measurements can
be made in a bounded universe?  It's pretty convincing---and a lot
simpler---to ignore such big and uncertain bounds (the universe seems
to be getting bigger all the time) and accept theories using real
numbers.

Likewise in computer science, it's implausible to think that writing a
program to add nonnegative integers with up to as many digits as, say,
the stars in the sky---trillions of galaxies each with billions of
stars---would be different from writing a program that would add
\emph{any} two integers, no matter how many digits they had.  The same
is true in designing a compiler: it's neither useful nor sensible to
make use of the fact that in a bounded universe, only a bounded number
of programs will ever be compiled.

\begin{editingnotes}
That's why basic programming data types like integers or
strings, for example, are defined without imposing any bound on the
sizes of data items.  For example, each datum of type
\idx{\texttt{string}} consists of characters from a finite alphabet
and has a finite length, but the data type definition does not require
that there be bound on the sizes of these finite numbers.  So we
accept the fact that conceptually there are an infinite number of item
of data type \texttt{string}---even though in any given
implementation, storage limits would impose overflow bounds.

When we then consider string procedures of type
\idx{\texttt{string-->string}}, not only are there an infinite number
of such procedures, but each procedure generally behaves differently
on different inputs, so that a single \texttt{string-->string}
procedure may embody an infinite number of behaviors.  In short, an
educated computer scientist can't get around having to cope with
infinite sets.
\end{editingnotes}

Infinite sets also provide a nice setting to practice proof methods,
because it's harder to sneak in unjustified steps under the guise of
intuition.  And there has been a truly astonishing outcome of studying
infinite sets.  Their study led to the discovery of fundamental, logical
limits on what computers can possibly do.  For example, in
Section~\ref{halting_sec}, we'll use reasoning developed for infinite
sets to prove that it's impossible to have a perfect type-checker for
a programming language.

So in this chapter, we ask you to bite the bullet and start learning to
cope with infinity.

\iffalse 
But as a warmup, we'll first examine some basic properties of
\emph{finite} sets.
\fi

\section{Infinite Cardinality}\label{infinite_sec}

In the late nineteenth century, the mathematician Georg Cantor%
\index{Cantor, Georg} 
was studying the convergence of Fourier series and found some series that
he wanted to say converged ``most of the time,'' even though there
were an infinite number of points where they didn't converge.  As a
result, Cantor needed a way to compare the size of infinite sets.  To
get a grip on this, he got the idea of extending the Mapping Rule
Theorem~\ref{maprul_thm}%
\index{Mapping Rules!for infinite sets}
to infinite sets: he regarded two infinite
sets as having the ``same size'' when there was a bijection%
\index{binary relation!bijection} 
between them.  Likewise, an infinite set $A$ should be considered ``as big
as'' a set $B$ when%
\index{binary relation!surjection} 
$A \surj B$.  So we could consider $A$ to be
``strictly smaller'' than $B$, which we abbreviate as $A \strict B$,%
\index{strict@$\strict$|textbf}
when $A$ is \emph{not} ``as big as'' $B$:
\begin{definition}\label{def:strict}
$\qquad A \strict B  \qiff \QNOT(A \surj B)$.
\end{definition}
On finite sets, this $\strict$ relation really does mean ``strictly
smaller.''  This follows immediately from the Mapping Rule Theorem~\ref{maprul_thm}.
\begin{corollary}\label{cor:strict}
For finite sets $A,B$,
\[
A \strict B  \qiff \card{A} < \card{B}. 
\]
\end{corollary}

\begin{proof}
\begin{align*}
A \strict B
  & \qiff \QNOT(A \surj B)
    & \text{(Def~\ref{def:strict})}\\
  & \qiff \QNOT(\card{A} \geq \card{B})
    & \text{(Theorem~\ref{maprul_thm}.\eqref{sur_ge_fincard})}\\
  & \qiff \card{A} < \card{B}.
\end{align*}
\end{proof}
Cantor got diverted from his study of Fourier series by his effort to
develop a theory of infinite sizes based on these ideas.  His theory
ultimately had profound consequences for the foundations of
mathematics and computer science.  But Cantor made a lot of enemies in
his own time because of his work: the general mathematical community
doubted the relevance of what they called ``Cantor's paradise''
of unheard-of infinite sizes.

A nice technical feature of Cantor's idea is that it avoids the need
for a definition of what the ``size'' of an infinite set might
be---all it does is compare ``sizes.''

\textbf{\textcolor{red}{Warning}}: We haven't, and won't, define what
the ``size'' of an infinite set is.  The definition of infinite
``sizes'' requires the definition of some infinite sets called
\emph{ordinals} with special well-ordering properties.  The theory of
ordinals requires getting deeper into technical set theory than we
need to go, and we can get by just fine without defining infinite
sizes.  All we need are the relations $\surj$ and $\bij$ which reflect
``as big as'' and ``same size'' relations between sets.

But there's something else to watch out for: $\surj$ is
\emph{metaphorically} an ``as big as'' relation, and likewise $\bij$
is \emph{metaphorically} the ``same size'' relation on sets.  As you
would expect, most of the ``as big as'' and ``same size'' properties
of $\surj$ and $\bij$ on finite sets do carry over to infinite sets,
but \emph{some important ones don't}---as we're about to show.  So you
have to be careful: don't assume that $\surj$, for example, has any
particular ``as big as'' property on \emph{infinite} sets until it's
been proven.

Let's begin with some familiar properties of the ``as big as'' and
``same size'' relations on finite sets that do carry over exactly to
infinite sets:
\begin{lemma}\label{surjinjbij_properties}
For any sets $A,B,C$,
\begin{enumerate}

\item \label{surjvsinj} $A \surj B$ iff $B \inj A$.

\item \label{bigtrans} If $A \surj B$ and $B \surj C$, then $A \surj
  C$.

\item \label{sametrans} If $A \bij B$ and $B \bij C$, then $A \bij C$.

\item\label{sameABA} $A \bij B$ iff $B \bij A$.
\end{enumerate}
\end{lemma}

Part~\ref{surjvsinj}.\ follows from the fact that $R$ has the $[\le
  1\ \text{out}, \ge 1\ \text{in}]$ surjective function property iff
$\inv{R}$ has the $[\ge 1\ \text{out}, \le 1\ \text{in}]$ total,
injective property.  Part~\ref{bigtrans}.\ follows from the fact that
compositions of surjections are surjections.
Parts~\ref{sametrans}.\ and~\ref{sameABA}.\ follow from the first two
parts because $R$ is a bijection iff $R$ and $\inv{R}$ are surjective
functions.  We'll leave verification of these facts to
Problem~\ref{CP_surj_relation}.

Another familiar property of finite sets carries over to infinite
sets, but this time some real ingenuity is needed to prove it:
\begin{theorem}\label{S-B_thm} \mbox{}
 [\idx{Schr\"oder-Bernstein}] For any sets $A,B$, if $A \surj B$ and
 $B \surj A$, then $A \bij B$.
\end{theorem}

That is, the Schr\"oder-Bernstein Theorem says that if $A$ is ``at least
as big as'' $B$ and conversely, $B$ is ``at least as big as'' $A$, then $A$
``is the same size as'' $B$.  Phrased this way, you might be tempted to
take this theorem for granted, but that would be a mistake.  For
infinite sets $A$ and $B$, the Schr\"oder-Bernstein Theorem is
actually pretty technical.  Just because there is a surjective
function $f:A\to B$---which need not be a bijection---and a surjective
function $g:B \to A$---which also need not be a bijection---it's not
at all clear that there must be a bijection $e:A \to B$.  The idea is
to construct $e$ from parts of both $f$ and $g$.  We'll leave the
actual construction to Problem~\ref{CP_Schroeder_Bernstein_theorem}.

Another familiar set property is that for any two sets, either the
first is at least as big as the second, or vice-versa.  For finite
sets this follows trivially from the Mapping Rule.  It's actually
still true for infinite sets, but assuming it is obvious would be
mistaken again.
  \begin{theorem}\label{surj-comparable}
    For \emph{all} sets $A,B$,
    \[
    A \surj B\quad \QOR\quad  B \surj A.
    \]
  \end{theorem}

Theorem~\ref{surj-comparable} lets us prove that another basic
property of finite sets carries over to infinite ones:
\begin{lemma}\label{strict-transitive}
\begin{equation}\label{AstrBstrC}
A \strict B\ \QAND\ B \strict C
\end{equation}
implies
\[
A \strict C
\]
for all sets $A,B,C$.
\end{lemma}

\begin{proof} (of Lemma~\ref{strict-transitive})

Suppose~\ref{AstrBstrC} holds, and assume for the sake of
contradiction that $\QNOT(A \strict C)$, which means that $A \surj C$.
Now since $B \strict C$, Theorem~\ref{surj-comparable} lets us
conclude that $C \surj B$.  So we have
\[
A \surj C\ \QAND\ C \surj B,
\]
and Lemma~\ref{surjinjbij_properties}.\ref{bigtrans} lets us conclude
that $A \surj B$, contradicting the fact that $A \strict B$.
\end{proof}

We're omitting a proof of Theorem~\ref{surj-comparable} because
proving it involves technical set theory---typically the theory of
\idx{ordinals} again---that we're not going to get into.  But since
proving Lemma~\ref{strict-transitive} is the only use we'll make of
Theorem~\ref{surj-comparable}, we hope you won't feel cheated not
seeing a proof.

\subsection{Infinity is different}

A basic property of finite sets that does \emph{not} carry over to
infinite sets is that adding something new makes a set bigger.  That
is, if $A$ is a finite set and $b \notin A$, then $\card{A \union
  \set{b}} = \card{A}+1$, and so $A$ and $A \union \set{b}$ are not
the same size.  But if $A$ is infinite, then these two sets \emph{are}
``the same size!''

\begin{lemma}\label{AUb}
  Let $A$ be a set and $b \notin A$.  Then $A$ is infinite iff $A \bij
  A \union \set{b}$.
\end{lemma}
\begin{proof}
Since $A$ is \emph{not} the same size as $A \union \set{b}$ when $A$
is finite, we only have to show that $A \union \set{b}$ \emph{is} the
same size as $A$ when $A$ is infinite.  That is, we have to find a
bijection between $A \union \set{b}$ and $A$ when $A$ is infinite.

Here's how: since $A$ is infinite, it certainly has at least one
element; call it $a_0$.  But since $A$ is infinite, it has at least
two elements, and one of them must not equal to $a_0$; call this new
element $a_1$.  But since $A$ is infinite, it has at least three
elements, one of which must be different from both $a_0$ and $a_1$; call this
new element $a_2$.  Continuing in this way, we conclude that there is
an infinite sequence $a_0,a_1,a_2,\dots,a_n,\dots$ of different
elements of $A$.  Now it's easy to define a bijection $e: A \union
\set{b} \to A$:
\begin{align*}
e(b) & \eqdef a_0,\\
e(a_n) & \eqdef a_{n+1}
          & \text{ for } n \in \nngint,\\
e(a) & \eqdef a
          & \text{ for } a \in A - \set{b,a_0,a_1,\dots}.
\end{align*}
\end{proof}

\subsection{Countable Sets}\label{countable_subsec}
A set $C$ is \term{countable} iff its elements can be listed in
order, that is, the elements in $C$ are precisely the elements in the
sequence
\[
c_0, c_1, \dots, c_n, \dots.
\]
Assuming no repeats in the list, saying that $C$ can be listed in this
way is formally the same as saying that the function, $f:\nngint \to
C$ defined by the rule that $f(i) \eqdef c_i$, is a bijection.

\begin{definition}\label{def_countable}
A set $C$ is \emph{countably infinite}%
\index{countable!countably infinite} iff\  $\nngint \bij C$.  A
set is \emph{countable} iff it is finite or countably infinite.
A set is \term{uncountable} iff it is not countable.
\end{definition}

We can also make an infinite list using just a finite set of elements
if we allow repeats.  For example, we can list the elements in the
three-element set $\set{2, 4, 6}$ as
\[
2,4,6,6,6,\dots.
\]
This simple observation leads to an alternative characterization of
countable sets that does not make separate cases of finite and
infinite sets.  Namely, a set $C$ is countable iff there is a list
\[
c_0, c_1, \dots, c_n, \dots
\]
of the elements of $C$, possibly with repeats.

\begin{lemma}\label{NsurjC}
A set $C$ is countable iff $\nngint \surj C$.  In fact, a nonempty
set $C$ is countable iff there is a \emph{total} surjective function
$g:\nngint \to C$.
\end{lemma}
The proof is left to Problem~\ref{CP_countable_from_surj}.

The most fundamental countably infinite set is the set $\nngint$
itself.  But the set $\integers$ of \emph{all} integers is also
countably infinite, because the integers can be listed in the order:
\begin{equation}\label{intlist}
0,-1,1,-2,2,-3,3,\dots.
\end{equation}
In this case, there is a simple formula for the $n$th element of the
list~\eqref{intlist}.  That is, the bijection $f:\nngint \to
\integers$ such that $f(n)$ is the $n$th element of the list can be
defined as:
\[
f(n) \eqdef \begin{cases} n/2 & \text{if $n$ is even},\\ -(n+1)/2 &
  \text{if $n$ is odd}.
           \end{cases} 
\]    
There is also a simple way to list all \emph{pairs} of nonnegative
integers, which shows that $(\nngint \cross \nngint)$ is also
countably infinite (Problem~\ref{MQ_product_of_countables}).  From
this, it's a small step to reach the conclusion that the set
$\rationals^{\ge 0}$ of nonnegative rational numbers is countable.
This may be a surprise---after all, the rationals densely fill up the
space between integers, and for any two, there's another in between.
So it might seem that you couldn't write out all the rationals in
a list, but Problem~\ref{CP_rationals_are_countable} illustrates how
to do it.  More generally, it is easy to show that countable sets are
closed under unions and products
(Problems~\ref{MQ_countable_union}
and~\ref{MQ_product_of_countables}) which implies the countability of
a bunch of familiar sets:
\begin{corollary}\label{countable_examples}
The following sets are countably infinite:
\[\integers^+, 
 \integers, \nngint \cross \nngint, \rationals^+, \integers \cross
 \integers, \rationals.
\]
\end{corollary}

A small modification of the proof of Lemma~\ref{AUb} shows that
countably infinite sets are the ``smallest'' infinite sets.  Namely,

\begin{lemma}\label{smallestinf}
If $A$ is an infinite set, and $B$ is countable, then $A \surj B$.
\end{lemma}
We leave the proof to Problem~\ref{CP_smallest_infinite_set}.

Also, since adding one new element to an infinite set doesn't change
its size, you can add any \emph{finite} number of elements without
changing the size by simply adding one element after another.
Something even stronger is true: you can add a \emph{countably}
infinite number of new elements to an infinite set and still wind up
with just a set of the same size
(Problem~\ref{PS_add_countable_elements}).

By the way, it's a common mistake to think that, because you can add
any finite number of elements to an infinite set and have a bijection
with the original set, that you can also throw in infinitely many new
elements.  In general it isn't true that just because it's OK to do
something any finite number of times, it's also OK to do it an infinite
number of times.  For example, starting from 3, you can increment by 1
any finite number of times, and the result will be some integer
greater than or equal to 3.  But if you increment an infinite number
of times, you don't get an integer at all.

\subsection{Power sets are strictly bigger}
\index{set!power set}

Cantor's astonishing discovery was that \emph{not all infinite sets
  are the same size}.  In particular, he proved that for any set $A$
the power set $\power(A)$ is ``strictly bigger'' than
$A$.  That is,
\begin{theorem}\label{powbig}[Cantor]\mbox{}
For any set $A$,
\[
A \strict \power(A).
\]
\end{theorem}
\begin{proof}
\iffalse
  First of all, $\power(A)$ is as big as $A$: for example, the partial
  function $f:\power(A) \to A$, where $f(\set{a}) \eqdef a$ for $a \in
  A$ and $f$ is only defined on one-element sets, is a surjection.
\fi
 
To show that $A$ is strictly smaller than $\power(A)$, we have to show
that if $g$ is a function from $A$ to $\power(A)$, then $g$ is
\emph{not} a surjection.  Since any partial function with nonempty
codomain can be extended to a total function with the same range (ask
yourself how), we can safely assume that $g$ is total.

To show that $g$ is not a surjection, we'll simply find a subset $A_g
\subseteq A$ that is not in the range of $g$.  The idea is, for each
element $a \in A$, to look at the set $g(a) \subseteq A$ and ask
whether or not $a$ happens to be in $g(a)$.  First, define \iffalse
mimicking Russell's Paradox,\fi
  \[
  A_g \eqdef \set{a \in A \suchthat a \notin g(a)}.
  \]
  $A_g$ is a well-defined subset of $A$, which means it is a
  member of $\power(A)$.  But $A_g$ can't be in the range of $g$,
  because it differs at $a$ from each set $g(a)$ in the range of $g$.

  To spell this out more, suppose to the contrary that $A_g$ was in
  the range of $g$, that is,
\[
A_g = g(a_0)
\]
for some $a_0 \in A$.  Now by definition of $A_g$,
\[
a \in g(a_0) \qiff a \in A_g \qiff a \notin g(a)
\]
for all $a \in A$.  Now letting $a = a_0$ yields the contradiction
\[
a_0 \in g(a_0) \qiff a_0 \notin g(a_0).
\]
So $g$ is not a surjection, because there is an element in the power
set of $A$, specifically the set $A_g$, that is not in the range of $g$.
\end{proof}

Cantor's Theorem immediately implies:
\begin{corollary}
$\power(\nngint)$ is uncountable.
\end{corollary}

\begin{proof}
By Lemma~\ref{NsurjC}, $U$ is uncountable iff $\nngint \strict U$.
\end{proof}

The bijection between subsets of an $n$-element set and the length $n$
bit-strings $\set{0,1}^n$ used to prove
Theorem~\ref{powset_fincard}, carries over to a bijection between
subsets of a countably infinite set and the infinite bit-strings,
$\binw$.  That is,
\begin{lemma}\label{pownnegbijbinw}
\power(\nngint) \bij \binw.
\end{lemma}
This immediately implies
\begin{corollary}
$\binw$ is uncountable.
\end{corollary}

\subsubsection{More Countable and Uncountable Sets}

Once we have a few sets we know are countable or uncountable, we can
get lots more examples using Lemma~\ref{surjinjbij_properties}.  In particular, 
we can appeal to the following immediate corollary of the Lemma:
\noindent \begin{corollary}\label{UinjAu}
\mbox{}
\begin{enumerate}[(a)]

\item\label{AsurjUA}
If $U$ is an uncountable set and $A \surj U$, then $A$ is uncountable.

\item\label{CsurjAc}
If $C$ is a countable set and $C \surj A$, then $A$ is countable.
\end{enumerate}
\end{corollary}

For example, now that we know that the set $\binw$ of infinite bit
strings is uncountable, it's a small step to conclude that
\begin{corollary}\label{uncountR}
The set $\reals$ of real numbers is uncountable.
\end{corollary}

To prove this, think about the infinite decimal expansion of a real
number:
\begin{align*}
\sqrt{2} & = 1.4142\dots,\\
5 & = 5.000\dots,\\
1/10 & = 0.1000\dots,\\
1/3 & = 0.333\dots,\\
%0 & = 0.000\dots,\\
1/9 & = 0.111\dots,\\
4\, \frac{1}{99} & = 4.010101\dots.
\end{align*}
Let's map any real number $r$ to the infinite bit string $b(r)$ equal
to the sequence of bits in the decimal expansion of $r$, starting at the decimal
point.  If the decimal expansion of $r$ happens to contain a digit
other than 0 or 1, leave $b(r)$ undefined.  For example, 
\begin{align*}
b(5) & = 000\dots,\\
b(1/10) & = 1000\dots,\\
b(1/9) & = 111\dots,\\
b(4\, \frac{1}{99}) & = 010101\dots\\
b(\sqrt{2}), b(1/3) &\text{ are undefined}.
\end{align*}
Now $b$ is a function from real numbers to infinite bit
strings.\footnote{Some rational numbers can be expanded in two
  ways---as an infinite sequence ending in all 0's or as an infinite
  sequence ending in all 9's.  For example,
\begin{align*}
5 & = 5.000\dots = 4.999\dots,\\
\frac{1}{10} & = 0.1000\dots = 0.0999\dots.
\end{align*}
In such cases, define $b(r)$ to be the sequence that ends with all
0's.}  It is not a total function, but it clearly is a surjection.
This shows that
\[
\reals \surj \binw,
\]
and the uncountability of the reals now follows by
Corollary~\ref{UinjAu}.\eqref{AsurjUA}.

For another example, let's prove
\begin{corollary}\label{countN*}
The set $\strings{(\integers^+)}$ of all finite sequences of positive
integers is countable.
\end{corollary}

To prove this, think about the prime factorization of a nonnegative
integer:
\begin{align*}
20 & = 2^2 \cdot 3^0 \cdot 5^1 \cdot 7^0 \cdot 11^0\cdot 13^0 \cdots,\\
6615 & = 2^0 \cdot 3^3 \cdot 5^1 \cdot 7^2 \cdot 11^0\cdot 13^0 \cdots.
\end{align*}
Let's map any nonnegative integer $n$ to the finite sequence $e(n)$ of
nonzero exponents in its prime factorization.  For example,
\begin{align*}
e(20) & = (2,1),\\
e(6615) & = (3,1,2),\\
e(5^{13} \cdot 11^{9} \cdot 47^{817} \cdot 103^{44}) & = (13, 9, 817, 44),\\
e(1) & = \emptystring, & \text{(the empty string)}\\
e(0) & \text{ is undefined}.
\end{align*}
Now $e$ is a function from $\nngint$ to $\strings{(\integers^{+})}$.  It
is defined on all positive integers, and it clearly is a surjection.  This
shows that
\[
\nngint \surj \strings{(\integers^{+})},
\]
and the countability of the finite strings of positive integers now
follows by Corollary~\ref{UinjAu}.\eqref{CsurjAc}.

\iffalse
To prove this, think about putting a decimal point in front of an
infinite bit string to get the decimal expansion of some real number.
That is, putting a decimal point at the beginning of a bit string
defines a function $p:\binw \to \reals$.  For example,
\[
p(000\dots) = .000\dots = 0, \qquad p(1000\dots) = .1000\dots = 1/10,\qquad
p(010101\dots) = .010101\dots = 1/99, \qquad p(11111\dots) = .1111\dots = 1/9\, .
\]
In fact, $p$ is a total function ranging from 0 to 1/9.  Moreover, $p$
is an injection: if two bit strings $s,t \binw$ differ anywhere in
their first $n$ bits, then $p(s)$ and $p(t)$ differ by at least
$8/(9\cdot 10^{n+1})$---a fact which it may amuse you to verify.  This
shows that
\[
\binw \inj \reals.
\]
\fi

\subsubsection{Larger Infinities}

There are lots of different sizes of infinite sets.  For example,
starting with the infinite set $\nngint$ of nonnegative integers,
we can build the infinite sequence of sets
\[
\nngint \strict \power(\nngint) \strict \power(\power(\nngint))
\strict \power(\power(\power(\nngint))) \strict \dots.
\]
By Cantor's Theorem~\ref{powbig}, each of these sets is strictly
bigger than all the preceding ones.  But that's not all: the union of
all the sets in the sequence is strictly bigger than each set in the
sequence (see Problem~\ref{CP_power_set_tower}).  In this way you can
keep going indefinitely, building ``bigger'' infinities all the way.

\subsection{Diagonal Argument}
Theorem~\ref{powbig} and similar proofs are collectively known as
``\index{diagonal argument|textbf}{diagonal arguments}'' because of a
more intuitive version of the proof described in terms of on an
infinite square array.  Namely, suppose there was a bijection between
$\nngint$ and $\binw$.  If such a relation existed, we would be able
to display it as a list of the infinite bit strings in some countable
order or another.  Once we'd found a viable way to organize this list,
any given string in $\binw$ would appear in a finite number of steps,
just as any integer you can name will show up a finite number of steps
from 0.  This hypothetical list would look something like the one
below, extending to infinity both vertically and horizontally:
\[\begin{array}{|rc c c c c c c l|}
\hline
A_0 &=  & 1 & 0 & 0 & 0 & 1 & 1 & \cdots\\
A_1 &=  & 0 & 1 & 1 & 1 & 0 & 1 & \cdots\\
A_2 &=  & 1 & 1 & 1 & 1 & 1 & 1 & \cdots\\
A_3 &=  & 0 & 1 & 0 & 0 & 1 & 0 & \cdots\\
A_4 &=  & 0 & 0 & 1 & 0 & 0 & 0 & \cdots\\
A_5 &=  & 1 & 0 & 0 & 1 & 1 & 1 & \cdots\\
\vdots & & \vdots & \vdots & \vdots & \vdots & \vdots &
         \vdots & \ddots\\
\hline
\end{array}\]
But now we can exhibit a sequence that's missing from our allegedly
complete list of all the sequences.  Look at the diagonal in our
sample list:
\[\begin{array}{|rc c c c c c c l|}
\hline
A_0 &=  & \textcolor{blue}{\mathbf{1}} & 0 & 0 & 0 & 1 & 1 & \cdots \\
A_1 &=  & 0 & \textcolor{blue}{\mathbf{1}} & 1 & 1 & 0 & 1 & \cdots \\
A_2 &=  & 1 & 1 & \textcolor{blue}{\mathbf{1}} & 1 & 1 & 1 & \cdots \\
A_3 &=  & 0 & 1 & 0 & \textcolor{blue}{\mathbf{0}} & 1 & 0 & \cdots \\
A_4 &=  & 0 & 0 & 1 & 0 & \textcolor{blue}{\mathbf{0}} & 0 & \cdots \\
A_5 &=  & 1 & 0 & 0 & 1 & 1 & \textcolor{blue}{\mathbf{1}} & \cdots \\
\vdots & & \vdots & \vdots & \vdots & \vdots & \vdots & \vdots & \ddots\\
\hline
\end{array}\]
Here is why the diagonal argument has its name: we can form a sequence
$\textcolor{blue}{D}$ consisting of the bits on the diagonal.
\[\begin{array}{l c c c c c c l}
\textcolor{blue}{D} = & \textcolor{blue}{1} & \textcolor{blue}{1} &
  \textcolor{blue}{1} & \textcolor{blue}{0} & \textcolor{blue}{0} &
  \textcolor{blue}{1} & \textcolor{blue}{\cdots},
\end{array}\]
Then, we can form another sequence by switching the
$\textcolor{blue}{\mathbf{1}}$'s and $\textcolor{blue}{\mathbf{0}}$'s
along the diagonal.  Call this sequence $\textcolor{red}{C}$:
\[\begin{array}{l c c c c c c l}
\textcolor{red}{C} = & \textcolor{red}{0} & \textcolor{red}{0} &
  \textcolor{red}{0} & \textcolor{red}{1} & \textcolor{red}{1} &
  \textcolor{red}{0} & \textcolor{red}{\cdots}.
\end{array}\]
Now if the $n$th term of $A_n$ is $\textcolor{blue}{\mathbf{1}}$ then the
$n$th term of $\textcolor{red}{C}$ is $\textcolor{red}{\mathbf{0}}$,
and \emph{vice versa}, which guarantees that $\textcolor{red}{C}$
differs from $A_n$.  In other words, $\textcolor{red}{C}$ has at least
one bit different from \emph{every} sequence on our list.  So
$\textcolor{red}{C}$ is an element of $\binw$ that does
not appear in our list---our list can't be complete!

This diagonal sequence $\textcolor{red}{C}$ corresponds to the set
$\set{a \in A \suchthat a \notin g(a)}$ in the proof of
Theorem~\ref{powbig}.  Both are defined in terms of a countable subset
of the uncountable infinity in a way that excludes them from that
subset, thereby proving that no countable subset can be as big as the
uncountable set.

\iffalse It shows that any function arranging the elements of~$\binw$
into a countable list will necessarily generate a list that is
incomplete.  Elements of the codomain will be omitted from the range
of the function, so~$\QNOT (\nngint \surj \binw$ which means
$\nngint \strict \binw$.\fi

\begin{problems}
\practiceproblems
%\pinput{TP_Inclusion_Exclusion}  used in counting chapter
%\pinput{TP_union_two_countable}
\pinput{TP_countable_strings}
\pinput{TP_diff_uncountable_example} %TP_uncountable_example
%\pinput{TP_cardinality_examples}
%\pinput{TP_cardinality_class}
\pinput{TP_countable_strict}
\pinput{TP_countable_injection}
\pinput{TP_uncountable_powerset}
\pinput{TP_majorizing}
\pinput{TP_majorizing_short}
\pinput{TP_finitely_discontinuous}
\pinput{TP_countable_subsets_integers}
\pinput{TP_countable_chain}

\classproblems
\pinput{CP_finite_strings_of_nonneg}
\pinput{CP_smallest_infinite_set}
\pinput{CP_rationals_are_countable}
\pinput{CP_Schroeder_Bernstein_theorem}
\pinput{CP_countable_from_surj}  %subsumes CP_countable_surjection
%TP_cardinality_examples
\pinput{CP_power_set_tower}
\pinput{FP_diagonalization_lonely_sequences}

\homeworkproblems
\pinput{PS_add_countable_elements}
\pinput{PS_unit_interval}
\pinput{PS_off_diagonal_arguments}
\pinput{PS_bogus_Cantor}
\pinput{PS_immune}
%\pinput{PS_immuneset} is an alternative, but needs more motivation

\examproblems
\pinput{MQ_set_cardinality}
\pinput{MQ_countable_union}
\pinput{MQ_product_of_countables}
\pinput{FP_uncountable_infinite_sequences_2ways}
%\pinput{MQ_uncountable_infinite_sequences}
%\pinput{FP_countable_sets}
%\pinput{FP_infinite_binary_sequences_S14}
%\pinput{FP_uncountable_ones}
\pinput{FP_uncountable_sparse1s}
\pinput{FP_diagonalization_lonely_subsets}
\pinput{FP_rapid_increasing_diag}
\pinput{FP_countable_quadratics}
\pinput{FP_card_bij}
\pinput{FP_countable_Nseqs}
\end{problems}

\section{The Halting Problem}\label{halting_sec}

Although towers of larger and larger infinite sets are at best a
romantic concern for a computer scientist, the \emph{reasoning} that
leads to these conclusions plays a critical role in the theory of
computation.  Diagonal arguments are used to show that lots of
problems can't be solved by computation, and there is no getting
around it.

This story begins with a reminder that having procedures operate on
programs is a basic part of computer science technology.  For example,
\emph{compilation} refers to taking any given program text
written in some ``high level'' programming language like Java, C++,
Python, \dots, and then generating a program of low-level instructions
that does the same thing but is targeted to run well on available
hardware.  Similarly, \emph{interpreters} or \emph{virtual
    machines} are procedures that take a program text designed to be
run on one kind of computer and simulate it on another kind of
computer.  Routine features of compilers involve
``type-checking''%
\index{type-checking|see{Halting Problem}} 
programs to ensure that certain kinds of
run-time errors won't happen, and ``optimizing'' the generated
programs so they run faster or use less memory.

The fundamental thing that just can't be done by computation is a
\emph{perfect} job of type-checking, optimizing, or any kind of
analysis of the overall run time behavior of programs.  In this
section, we'll illustrate this with a basic example known as the
\term{Halting Problem}.  Pick your favorite programming
language---Python, Java, C++,\dots---and assume that ``program''
refers to a program written in your language. 

Once a program is started, if its initial computation stops for some
reason---such as producing a final value, waiting for an input,
suffering an error interrupt, or simply freezing---the program is said
to \term{halt}.  So a program that does \emph{not} halt would run
forever using up cycles and energy (unless it was interrupted by an
external operating system command).  The Halting Problem is the
general problem of determining, given an arbitrary program, whether or
not the program halts.

There is a simple way to determine when an arbitrary program does
halt, at least in theory: just run it until it stops.  Well not quite.
Just running the program in the usual way would not detect when it
freezes without warning.  What really needs to be done is to simulate
the program using an interpreter or a virtual machine that will
recognize any kind of halting, including a freeze.  But interpreters
and virtual machines capable of simulating any program are familiar
technology.  So there is a general way to detect when a program halts.

The hard part is determining when a program does \emph{not} halt.  At
any point in simulating it, you generally won't know whether to
continue the simulation because the program will halt later, or to
abort the simulation because it wouldn't stop otherwise.

So is there some way besides simulation to detect when a program does
\emph{not} halt?  Could there be some program analysis tool that could
inspect any program and correctly report when the program does not
halt?  The answer is ``No.''  Using a standard diagonal argument, we
will prove that it is impossible to have such a non-halting analysis
tool.  Any method for detecting non-halting is bound to go wrong.  It
will falsely report that some halting program does not halt, or it
will fail to report anything about some program.  That is, there will
be a program on which the analyzer runs forever without halting.

To set up the diagonal argument, we will focus on \emph{string
  procedures}.  A string procedure is a procedure that takes a single
argument that is supposed to be a string over the 256 character
\asciibet\ alphabet.
As a simple example, you might think about how to write a string
procedure that halts precisely when it is applied to a \emph{double
  letter} string in \asciistr, namely, a string in which every
character occurs twice in a row.  For example, \texttt{aaCC33}, and
\texttt{zz++ccBB} are double letter strings, but \texttt{aa;bb},
\texttt{b33}, and \texttt{AAAAA} are not.

When the computation of a procedure applied to a string halts, the
procedure will be said to \emph{recognize} the string.  In this
context, a set of strings a commonly called a (formal)
\emph{language}.  We let $\rcg{P}$ to be the language recognized by
procedure $P$:
\[
\rcg{P} \eqdef \set{s \in \asciistr \suchthat P \text{ applied to $s$
  halts}}.
\]
A language is called \emph{recognizable} when it equals $\rcg{P}$ for
some string procedure $P$.  For example, we've just agreed that the
set of double letter strings is recognizable.

There is no harm in assuming that every program can be written as a
string in \asciistr---that's typically how we would enter them into a
computer in the first place.  When a string $s \in \asciistr$ is
actually the \asciibet\ description of some string procedure, we'll
refer to that string procedure as $P_s$.  You can think of $P_s$ as
the result of compiling $s$ into something executable.\footnote{The
  string $s \in \asciistr$ and the procedure $P_s$ have to be
  distinguished to avoid a type error: you can't apply a string to
  string.  For example, let $s$ be the string that you wrote as your
  program to recognize the double letter strings.  Applying $s$ to a
  string argument, say \texttt{aabbccdd}, should throw a type
  exception; what you need to do is compile $s$ to the procedure $P_s$
  and then apply $P_s$ to \texttt{aabbccdd}.} It's technically helpful
to treat \emph{every} string in \asciistr\ as a program for a string
procedure.  So when a string $s \in \asciistr$ doesn't parse as a
proper string procedure, we'll define $P_s$ to be some default string
procedure---say one that never halts on anything it is applied to.

Focusing just on string procedures, the general Halting Problem is to
decide, given strings $s$ and $t$, whether or not the procedure $P_s$
recognizes $t$.  Following the usual diagonal approach, we define the language \nohalt:
\begin{definition}\label{nohalt_def}
\begin{equation}\label{nohalt_eqdef}
\nohalt \eqdef \set{s \suchthat P_s\ \text{applied to $s$ does not halt}} = \set{s \notin \rcg{P_s}}.
\end{equation}
\end{definition}
So if $P_s$ is the recognizer for some language, then \nohalt\ differs
from that language on the string $s$.  This shows that \nohalt\ cannot
have a recognizer:
\begin{theorem}\label{nohalt_thm}
\nohalt\ is not recognizable.
\end{theorem}

Let's spell out the reasoning behind this theorem more fully.  By
definition, we have
\begin{equation}\label{snohaltsnfs}
s \in \nohalt \QIFF\ s \notin \rcg{P_s},
\end{equation}
for all strings $s \in \asciistr$.

Now suppose to the contrary that \nohalt\ was recognizable.  This
means there is some procedure $P_{s_0}$ that recognizes \nohalt, that is,
\[
\nohalt = \rcg{P_{s_0}}.
\]
Combined with~\eqref{snohaltsnfs}, we have
\begin{equation}\label{sfs0iff}
s \in \rcg{P_{s_0}} \qiff s \notin \rcg{P_s}
\end{equation}
for all $s \in \asciistr$.  Now letting $s = s_0$ in~\eqref{sfs0iff}
yields the immediate contradiction
\[
s_0 \in \rcg{P_{s_0}} \qiff s_0 \notin \rcg{P_{s_0}}.
\]

So that does it: the reasoning above applied to whatever favorite
programming language you picked.  It is \emph{logically impossible}
for a Java program to be a recognizer for non-halting Java programs,
or for a Python program to be a recognizer for non-halting Python
programs, and so forth for other favorite programming languages.

Now you might wonder if there was a loophole around this logical
limitation by having a recognizer the non-halting programs in one
language that was written in another language.  In other words, could
there be a C++ procedure that recognizes all the non-halting Java
programs?  After all, C++ does allow more intimate manipulation of
computer memory than Java does.  But there is no loophole here.  If
you've learned about programming language implementation, you will
realize that it's possible to write a simulator in Java for C++
programs.  This means that if there were a C++ procedure that
recognized non-halting Java programs, then a Java procedure could also
do it by simulating the C++ program, and that's impossible.  This
reasoning finally leads to a transcendent insight.  No procedure can
be written in \emph{any} programming language that recognizes
\nohalt\ for your favorite language.  Recognizing \nohalt\ is simply
beyond the capacity of computation.

But it's not just \nohalt.  If there was a perfect recognizer for
\emph{any} property that depends on the complete run time behavior
programs,\footnote{ The weasel words ``complete run time behavior''
  creep in here to rule out some run time properties that are easy to
  recognize because they depend only on part of the run time behavior.
  For example, the set of programs that halt after executing at most
  100 instructions is recognizable.} it could be altered slightly to
become a recognizer for \nohalt.  We'll take this claim for granted
now, giving its full justification in an assigned problem.

For example, most compilers do ``static'' type-checking at compile
time to ensure that programs won't make run-time type errors.  A
program that type-checks is guaranteed not to cause a run-time
type-error.  But given that it's impossible to recognize when the
complete run time behavior of a program does not lead to a type-error,
it follows that the type-checker must be rejecting programs that
really wouldn't cause a type-error.  The conclusion is that no
type-checker is perfect---you can always do better!

It's a different story if we think about the \emph{practical}
possibility of writing program analyzers.  The fact that it's
logically impossible to analyze utterly arbitrary programs does not
mean that you can't do a very good job analyzing interesting programs
that come up in practice.  In fact, these ``interesting'' programs are
commonly \emph{intended} to be analyzable in order to confirm that
they do what they're supposed to do.

In the end, it's not clear how much of a hurdle this theoretical
limitation implies in practice.  But the theory does provide some
perspective on claims about general analysis methods for programs.
The theory tells us that people who make such claims either
\begin{itemize}
\item are exaggerating the power (if any) of their methods, perhaps to make a
  sale or get a grant, or

\item are trying to keep things simple by not going into technical
  limitations they're aware of, or

\item perhaps most commonly, are so excited about some useful practical
    successes of their methods that they haven't bothered to think about
    the limitations which must be there.
\end{itemize}
So from now on, if you hear people making claims about having general
program analysis/verification/optimization methods, you'll know they
can't be telling the whole story.

\begin{editingnotes}
Highlight CS insight that lots of problems from other fields turn out
to be programming probs, and hence undecidable: integer
multi-var polys can be viewed as programs ala Hilbert's 10th; likewise
group theory equivalence; provability even in limited systems of pred
calc; Petri Net containment, ....
\end{editingnotes}

\begin{problems}

\classproblems
\pinput{CP_N_to_N_diagonal_argument}

\homeworkproblems
\pinput{PS_A_to_B_diagonal_argument}
\pinput{CP_computable_reducibility}
%\pinput{CP_indescribable_language}

\begin{editingnotes}
Add problem that the $4^n$ time-bounded halting problem requires time
$2^n$?
\end{editingnotes}

\end{problems}

\section{The Logic of Sets}\label{set_logic_sec}%\hyperdef{logic}{sets}

\subsection{\idx{Russell's Paradox}}

Reasoning naively about sets turns out to be risky.  In fact, one of the
earliest attempts to come up with precise axioms for sets in the late
nineteenth century by the logician \index{Frege, Gotlob} Gotlob Frege, was
shot down by a three line argument known as \emph{Russell's
  Paradox}\footnote{Bertrand Russell%
\index{Russel, Bertrand} 
was a mathematician/logician at Cambridge University at the turn of the Twentieth
 Century.  He reported that when he felt too old to do mathematics, he began to study
  and write about philosophy, and when he was no longer smart enough to do
  philosophy, he began writing about politics.  He was jailed as a
  conscientious objector during World War I.  For his extensive
  philosophical and political writing, he won a Nobel Prize for
  Literature.} which reasons in nearly the same way as the proof of
Cantor's Theorem~\ref{powbig}.  This was an astonishing blow to efforts to
provide an axiomatic foundation for mathematics:

\textbox{
\begin{center}
\large Russell's Paradox
\end{center}

\begin{quote}
Let $S$ be a variable ranging over all sets, and define
\[
W \eqdef \set{S \suchthat S \not\in S}.
\]
So by definition,
\[
S \in W  \mbox{  iff  } S \not\in S,
\] 
for every set $S$.  In particular, we can let $S$ be $W$, and obtain
the contradictory result that
\[
W \in W  \mbox{  iff  } W \not\in W.
\]
\end{quote}}

The simplest reasoning about sets crashes mathematics!  Russell and his
colleague Whitehead spent years trying to develop a set theory that was
not contradictory, but would still do the job of serving as a solid
logical foundation for all of mathematics.

Actually, a way out of the paradox was clear to Russell and others at
the time: \emph{it's unjustified to assume that $W$ is a set}.  The
step in the proof where we let $S$ be $W$ has no justification,
because $S$ ranges over sets, and $W$ might not be a set.  In fact, the
paradox implies that $W$ had better not be a set!

But denying that $W$ is a set means we must \emph{reject} the very
natural axiom that every mathematically well-defined collection of
sets is actually a set.  The problem faced by Frege, Russell and their
fellow logicians was how to specify \emph{which} well-defined
collections are sets.  Russell and his Cambridge University colleague
Whitehead immediately went to work on this problem.  They spent a
dozen years developing a huge new axiom system in an even huger
monograph called \emph{Principia Mathematica}, but for all intents and
purposes, their approach failed.  It was so cumbersome no one ever
used it, and it was subsumed by a much simpler, and now widely
accepted, axiomatization of set theory by the logicians Zermelo
and Fraenkel.

\subsection{The ZFC Axioms for Sets}\label{ZFC_sec}

It's generally agreed that essentially \emph{all of mathematics} can
be derived from a few formulas of set theory, called the Axioms of
\term{Zermelo-Fraenkel Set Theory} with Choice (ZFC), using a few
simple logical deduction rules.\index{ZFC|see{Zermello-Fraenkel Set
    Theory}}

These axioms are about ``pure'' sets that are literally built up from
nothing.  For example, we can start with the set $\emptyset$ with
nothing in it, and then build new sets from stuff we already have.  So
from $\emptyset$ we can build the set $\set{\emptyset}$ whose sole
member is the set we started with.  But now we have built two things,
so we can build two new sets
\[
\set{\set{\emptyset}},\quad \set{\emptyset, \set{\emptyset}}.
\]
Continuing like this, we could build up a simple sequence of sets
\[
\emptyset,\quad \set{\emptyset},\quad \set{\set{\emptyset}},\quad \set{\set{\set{\emptyset}}},\dots
\]
where each element is simply the set whose sole element is the
previous element.  This means we can now build an infinite set $N$
consisting of all the elements in the sequence.

Of course there are lots of types of mathematical objects---sets of
numbers like \nngint\ or \reals, sets of functions on numbers, sets of
sequences like \finbin\ or \binw, or sets of graphs, for example---that
we don't normally regard as pure sets.  But in fact pure sets are
enough to model all these other types of objects.

For example, the infinite set $N$ is like a copy of the nonnegative
integers \nngint, where we define the result of ``adding one'' to an
element $s \in N$ to be $\set{x}$.  If we go on and take all the
pairs\footnote{Doing this requires a way to represent an ordered pair
  $(a,b)$ of things in terms of sets built up from just $a$ and $b$.
  See Problem~\ref{CP_set_pairing} for a slightly ingenious, but still
  simple, way to represent the pair as a set.}  of elements $(a,b) \in
N \times (N-\set{\emptyset})$, we get a set that is like the
numerator/denominator pairs of all the fractions with nonnegative
numerators and positive denominators.  With a little fiddling of these
``fractions,'' we can get a set that acts like a copy of the rational
numbers $\rationals$.  Next from the rationals we can get a set that
acts like the real numbers by modelling a real number $r \in \reals$
by the set of rationals greater than $r$.  And so on.  The point is
that if we can reason correctly about pure sets, we get a solid basis
for reasoning about all mathematical objects.

So the domain of discourse of ZFC is the pure sets, and ZFC involves
using a few logical inference rules to prove properties of pure sets
starting from a few axioms.  The axioms themselves are written as
\emph{pure first-order formulas}\index{pure set theory} of set theory.
These are predicate formulas that only talk about membership in sets.
That is, a first-order formula of set theory is built using logical
connectives and quantifiers starting \emph{solely} from expressions of
the form ``$x\in y$'' which is interpreted to mean that $x$ and $y$
are pure sets, and $x$ is one of the elements in $y$.

Formulas of set theory are not even allowed to have the equality
symbol ``$=$.''  However, sets are equal iff they have the same
elements, so there is an easy way to express equality of sets purely
in terms of membership:
\begin{equation}\label{xsetequaly}
(x = y) \eqdef\  \forall z.\; (z \in x \QIFF z \in y).
\end{equation}

Similarly, the subset symbol ``$\subseteq$'' is not allowed in
formulas of set theory, but we can also express subset purely in terms
of membership:
\begin{equation}\label{xsubeqy}
(x \subseteq y) \eqdef\ \forall z.\; (z \in x\ \QIMPLIES\ z \in y).
\end{equation}

So formulas using symbols ``$=$,'' ``$\subseteq$,'' in addition to
``$\in$'' can be understood as \emph{abbreviations} for pure formulas
only using ``$\in$.''  We won't worry about this distinction between
formulas and abbreviations for formulas---we'll now just call them all
``formulas of set theory.''  For example,
\[
x= y\ \QIFF\ [x \subseteq y \QAND y \subseteq x]
\]
is a formula of set theory that explains a basic connection between
set equality and set containment.

%\index{Zermello-Fraenkel Set Theory|seealso{axiom}}

We're \emph{not} going to develop much of set theory in this text, but
we thought you might like to see the actual axioms of ZFC---and while
you're at it, get some more practice reading and writing quantified
formulas: \index{axiom!ZFC axioms|textbf}

\subsubsection{The Axioms}

\begin{description}

\item[\index{axiom!ZFC axioms!axiom of
    extensionality}\emph{Extensionality}.]  Two sets are equal iff
  they are members of the same sets:
\[
x = y\ \QIFF\ (\forall z.\; x \in z \QIFF y \in z).
\]

\item[\index{axiom!ZFC axioms!axiom of pairing}\emph{Pairing}.] For
  any two sets $x$ and $y$, there is a set $\set{x,y}$ with $x$ and
  $y$ as its only elements:
\[
\forall x,y \exists u \forall z.\;
[z \in u \QIFF (z = x \QOR z = y)]
\]

\item[\index{axiom!ZFC axioms!union axiom}\emph{Union}.]  The union
  $u$ of a collection $z$ of sets is also a set:
\[
\forall z \exists u \forall x.\;
  (x \in u)\ \QIFF\ (\exists y.\; x \in y \QAND y \in z) 
\]

\item[\index{axiom!ZFC axioms!infinity axiom}\emph{Infinity}.]  There
  is an infinite set.  Specifically, there is a nonempty set $x$
  such that for any set $y \in x$, the set $\set{y}$ is also a member
  of $x$.

\item[\index{axiom!ZFC axioms!subset
    axiom}\emph{Subset}.] Given any set~$x$ and any
  definable property of sets, there is a set~$y$ containing precisely
  those elements in~$x$ that have the property.
\[
\forall x \exists y \forall z.\; z \in y \QIFF [z \in x \QAND \phi(z)]
\]
where $\phi(z)$ is a formula of set theory.\footnote{This axiom is more commonly called
    the \term{Comprehension Axiom}.}

\item[\index{axiom!ZFC axioms!power set axiom}\emph{Power Set}.]  All the
  subsets of a set form another set:
\[
\forall x \exists p \forall u.\; u \subseteq x \QIFF u \in p.
\]

\item[\index{axiom!ZFC axioms!replacement axiom}\emph{Replacement}.]
  Suppose a formula $\phi$ of set theory defines the graph of a
  function on a set $s$, that is,
\iffalse
\[
\forall x \in s\, \exists y.\; \phi(x,y),
\]
and
\fi

\[
\forall x \in s\, \forall y, z.\; [\phi(x,y) \QAND \phi(x,z)] \QIMPLIES y = z.
\]
Then the image of $s$ under that function is also a set $t$.  Namely,
\[
\exists t \forall y.\; y \in t \QIFF [\exists x\in s.\; \phi(x,y)].
\]

\item[\index{axiom!ZFC axioms!foundation
    axiom}\emph{Foundation}.]  The aim is to forbid any infinite
  sequence of sets of the form
\[
\cdots \in x_n \in \cdots \in x_1 \in x_0
\]
in which each set is a member of the next one.  This can be
captured by saying every nonempty set has a ``member-minimal''
element.  Namely, define
\[
\text{member-minimal}(m, x) \eqdef [m \in x \QAND \forall y \in x.\; y \notin m].
\]
Then the Foundation Axiom\footnote{This axiom is also called the
  \term{Regularity Axiom}.} is
\[
\forall x.\ x \neq \emptyset\ \QIMPLIES\ \exists m.\; \text{member-minimal}(m, x).
\]

\item[\index{axiom!ZFC axioms!axiom of choice}\emph{Choice}.]  Let $s$
  be a set of nonempty, disjoint sets.  Then there is a set $c$
  consisting of exactly one element from each set in $s$.  The formula
  is given in Problem~\ref{CP_axiom_of_choice_formula}.

\end{description}

\subsection{Avoiding \idx{Russell's Paradox}}\label{avoidingrussellsparadox}

These modern ZFC axioms for set theory are much simpler than the
system Russell and Whitehead first came up with to avoid paradox.  In
fact, the ZFC axioms are as simple and intuitive as Frege's original
axioms, with one technical addition: the Foundation axiom.  Foundation
captures the intuitive idea that sets must be built up from
``simpler'' sets in certain standard ways.  And in particular,
Foundation avoids the ``paradox'' of Russell's set collection $W
\eqdef \set{S \suchthat S\notin S}$ because it implies that $W$ is not
a set.  Namely Foundation implies that $S \not \in S$ for \emph{every}
set $S$, so $W$ is the collection of \emph{all} sets.  Now if $W$ was
a set, we would have $W \in W$, violating Foundation.

The Foundation axiom also justifies proofs about sets using structural
induction.  We'll begin with a recursive definition of a class of sets
called \recset.  
\begin{definition}\label{recset_def}
The class of \emph{recursive sets} \recset\ is defined as follows:

\inductioncase{Base case}: The empty set $\emptyset$ is a \recset.

\inductioncase{Constructor step}: If $S$ is a nonempty set of
\recset's, then $S$ is a \recset.
\end{definition}

Since \recset\ is defined recursively, we know that we can use
structural induction to prove things about it.  This means that
we can actually use structural induction to prove things about all
sets, because the Foundation axiom implies that \emph{all} sets are
recursive sets:

\begin{theorem}\label{thm_recursive_sets}
The class \recset\ of recursive sets is the same as the class of all
sets.
\end{theorem}

\begin{proof}
Everything in \recset\ is a set by definition, so we need only show
that every set is in \recset.

Suppose to the contrary that a set $S$ is not in \recset.  So $S$ must
have an element $N_0$ that is \emph{not} in \recset, since otherwise
$S$ would by definition be in \recset.  Likewise, $N_0$ must have an
element $N_1$ that is \emph{not} in \recset, and $N_1$ must have an
element $N_2$ that is \emph{not} in \recset, and so on.  So we have an
infinite sequence
\[
S \ni N_0 \ni N_1 \ni N_2 \ni \dots.
\]
Therefore the set\footnote{A rigorous proof, which we are skipping,
  requires careful use of the Choice axiom to prove that $T$ is a
  set.}
\[
T \eqdef \set{S,N_0, N_1, N_2, \dots}
\]
has no member-minimal element, in violation of Foundation.
\end{proof}

\begin{problems}
\practiceproblems
\pinput{TP_not_member_of_self}
\pinput{TP_recset_circular}

\classproblems
\pinput{CP_set_pairing}
\pinput{CP_axiom_of_choice_formula}
\pinput{CP_foundation_axiom}

\homeworkproblems
\pinput{PS_size_n_set_formula}
\pinput{PS_recursive_set_data_type_alt}

%\pinput{PS_equality_axioms}  %DRAFT

\examproblems
\pinput{MQ_pair_predicate}
\end{problems}

\section{Does All This Really Work?}\label{setsreallywork}

So this is where mainstream mathematics stands today: there is a
handful of ZFC axioms from which virtually everything else in
mathematics can be derived logically.  This sounds like a rosy
situation, but there are several dark clouds, suggesting that the
essence of truth in mathematics is not completely resolved.

%
\begin{itemize}

\item The ZFC axioms weren't etched in stone by God.  Instead,
  they were mostly made up by Zermelo, who may have been a brilliant
  logician, but was also a fallible human being---probably some days
  he forgot his house keys.  So maybe Zermelo, just like Frege, didn't
  get his axioms right and will be shot down by some successor to
  Russell%
\index{Russell, Bertrand} 
who will use his axioms to prove a proposition $P$ and
  its negation $\bar{P}$.  Then math as we understand it would be
  broken---this may sound crazy, but it has happened before.

  In fact, while there is broad agreement that the ZFC axioms are
  capable of proving all of standard mathematics, the axioms have some
  further consequences that sound paradoxical.  For example, the
  \idx{Banach-Tarski Theorem} says that, as a consequence of the
  \index{axiom!ZFC axioms!axiom of choice}\emph{axiom of choice}, a solid
  ball can be divided into six pieces and then the pieces can be
  rigidly rearranged to give \emph{two} solid balls of the same size
  as the original!

\item Some basic questions about the nature of sets remain unresolved.
  For example, Cantor raised the question whether there is a set whose
  size is strictly between the smallest infinite set $\nngint$ (see
  Problem~\ref{CP_smallest_infinite_set}) and the strictly larger
  set $\power(\nngint)$?  Cantor guessed not:

  \textbf{Cantor's \index{Continuum Hypothesis}\emph{Contiuum
      Hypothesis}}: There is no set $A$ such that
  \[
  \nngint \strict A \strict \power(\nngint).
  \]

  The Continuum Hypothesis remains an open problem a century later.
  Its difficulty arises from one of the deepest results in modern Set
  Theory---discovered in part by G\"odel in the 1930's and Paul
  Cohen in the 1960's---namely, the ZFC axioms are not
  sufficient to settle the Continuum Hypothesis: there are two
  collections of sets, each obeying the laws of
  \index{Zermelo-Fraenkel Set Theory}ZFC, and in one collection the
  Continuum Hypothesis is true, and in the other it is false.  Until a
  mathematician with a deep understanding of sets can extend ZFC with
  persuasive new axioms, the Continuum Hypothesis will remain
  undecided.  \iffalse So settling the Continuum Hypothesis some new
  understanding of what Sets should be to arrive at persuasive new
  axioms that extend ZFC and are strong enough to determine the truth
  of the Continuum Hypothesis one way or the other.  \fi
\item But even if we use more or different axioms about sets, there
  are some unavoidable problems.  In the 1930's, G\"odel%
  \index{Godel@G\"odel, Kurt} proved that, assuming that an axiom
  system like ZFC is consistent---meaning you can't prove both $P$ and
  $\bar{P}$ for any proposition, $P$---then the very proposition that
  the system is consistent (which is not too hard to express as a
  logical formula) cannot be proved in the system.  In other words, no
  consistent system is strong enough to verify itself In particular,
  every axiom system is \term{incomplete}: it cannot prove all the
  truths of mathematics.\index{incompleteness theorem}
  
\end{itemize}

\subsection{Large Infinities in Computer Science}

If the romance of different-size infinities and continuum hypotheses
doesn't appeal to you, not knowing about them is not going to limit
you as a computer scientist.  These abstract issues about infinite
sets rarely come up in mainstream mathematics, and they don't come up
at all in computer science, where the focus is generally on
``\idx{countable},'' and often just finite, sets.  In practice, only
logicians and set theorists have to worry about collections that are
``too big'' to be sets.  That's part of the reason that the 19th
century mathematical community made jokes about ``Cantor's
  paradise'' of obscure infinities.  But the challenge of
reasoning correctly about this far-out stuff led directly to the
profound discoveries about the logical limits of computation described
in Section~\ref{halting_sec}, and that really is something every
computer scientist should understand.

\iffalse
\begin{problems}
\homeworkproblems
\pinput{PS_infinite_ordinals}
\end{problems}
\fi

\endinput

\iffalse When a string procedure applied to an ASCII string halts,
we'll say the procedure halts on the string.  If it runs forever, then
we'll say it does not halt on the string.\fi

\iffalse This application should result in a halting computation,
since \texttt{aabbccdd} is a double letter string.\fi

  \iffalse So if $s$ is an ill-formed string, $P_S$
will be a recognizer for the empty set of strings.  \fi
