\section{The Chernoff Bound}

Fussbook is a new social networking site oriented toward unpleasant
people.

Like all major web services, Fussbook has a load balancing problem.
Specifically, Fussbook receives 24,000 forum posts every 10 minutes.
Each post is assigned to one of $m$ computers for processing, and each
computer works sequentially through its assigned tasks.  Processing an
average post takes a computer $1/4$ second.  Some posts, such as
pointless grammar critiques and snide witticisms, are easier.  But the
most protracted harangues require 1 full second.

Balancing the work load across the $m$ computers is vital; if any
computer is assigned more than 10 minutes of work in a 10-minute
interval, then that computer is overloaded and system performance
suffers.  That would be bad, because Fussbook users are {\em not} a
tolerant bunch.

An early idea was to assign each computer an alphabetic range of forum
topics.  (``That oughta work!'', one programmer said.)  But after the
computer handling the ``{\em pr}ivacy'' and ``{\em pr}eferred text
editor'' threads melted, the drawback of an ad hoc approach was clear:
there are no guarantees.

If the length of every task were known in advance, then finding a
balanced distribution would be a kind of ``bin packing'' problem.
Such problems are hard to solve exactly, though approximation
algorithms can come close.  But in this case task lengths are not
known in advance, which is typical for workload problems.

So the load balancing problem seems sort of hopeless, because there is
no data available to guide decisions.  Heck, we might as well assign
tasks to computers at random!

As it turns out, random assignment not only balances load reasonably
well, but also permits provable performance guarantees in place of
``That oughta work!''  assertions.  In general, a randomized approach
to a problem is worth considering when a deterministic solution is
hard to compute or requires unavailable information.

Some arithmetic shows that Fussbook's traffic is sufficient to keep $m
= 10$ computers running at 100\% capacity with perfect load balancing.
Surely, more than 10 servers are needed to cope with random
fluctuations in task length and imperfect load balance.  But how many
is enough?  11?  15?  20?  We'll answer that question with a new
mathematical tool.

\subsection{The Chernoff Bound}

The Chernoff bound is a hammer that you can use to nail a great many
problems.  Roughly, the Chernoff bound says that certain random
variables are very unlikely to significantly exceed their expectation.
For example, if the expected load on a computer is just a bit below
its capacity, then that computer is unlikely to be overloaded,
provided the conditions of the Chernoff bound are satisfied.

More precisely, the Chernoff Bound says that \emph{the sum of lots of
  little, independent random variables is unlikely to significantly
  exceed the mean}.  The Markov and Chebychev bounds lead to the same
kind of conclusion but typically provide much weaker conclusions.

\begin{editingnotes}
In particular, the Markov and Chebychev bounds are polynomial, while
the Chernoff bound is exponential.
\end{editingnotes}

Here is the theorem.  The proof is at the end of the chapter.

\begin{theorem}[Chernoff Bound]
\label{chernoff}
Let $T_1, \dots T_n$ be mutually independent random variables such
that $0 \leq T_i \leq 1$ for all $i$.  Let $T = T_1 + \cdots + T_n$.
Then for all $c \geq 1$,
\begin{equation}\label{chernoff-leq}
\pr{T \geq c \expect{T}} \leq e^{-k \expect{T}} 
\end{equation}
where $k = c \ln c - c + 1$.
\end{theorem}

The Chernoff bound applies only to distributions of sums of
independent random variables that take on values in the interval $[0,
  1]$.  The binomial distribution is of course such a distribution,
but are lots of other distributions because the Chernoff bound allows
the variables in the sum to have differing, arbitrary, and even unknown
distributions over the range $[0, 1]$.  Furthermore, there is no
direct dependence on the number of random variables in the sum or
their expectations.  In short, the Chernoff bound gives strong results
for lots of problems based on little information ---no wonder it is
widely used!

\subsubsection{A Simple Example}

The Chernoff bound is pretty easy apply, though the details can be
daunting at first.  Let's walk through a simple example to get the
hang of it.

What are the odds that the number of heads that come up in 1000
independent tosses of a fair coin exceeds the expectation by 20\% or
more?  Let $T_i$ be an indicator variable for the event that the
$i$-th coin is heads.  Then the total number of heads is $T = T_1 +
\cdots + T_{1000}$.  The Chernoff bound requires that the random
variables $T_i$ be mututally independent and take on values in the
range $[0, 1]$.  Both conditions hold here.  In fact, this example is
similar to many applications of the Chernoff bound in that every $T_i$
is {\em either} 0 or 1, since they're indicators.

The goal is to bound the probability that the number of heads exceeds
its expectation by 20\% or more; that is, to bound $\pr{T \geq c
  \expect{T}}$ where c = $1.2$.  To that end, we compute $k$ as
defined in the theorem:
\[
k = c \ln c - c + 1 = 0.0187\dots
\]
Plugging this value into the Cheroff bound gives:
\begin{align*}
\pr{T \geq 1.2 \expect{T}} & \leq  e^{- k \expect{T}} \\
  & = e^{- (0.0187\dots) \cdot 500} \\
  & <  0.0000834
\end{align*}
So the probability of getting 20\% or more extra heads on 1000 coins is
less than 1 in 10,000.

The bound becomes much stronger as the number of coins increases,
because the expected number of heads appears in the exponent of the
upper bound.  For example, the probability of getting at least 20\%
extra heads on a million coins is at most
\[
e^{- (0.0187\dots) \cdot 500000} < e^{-9392}
\]
which is pretty darn small.

Alternatively, the bound also becomes stronger for larger deviations.
For example, suppose we're interested in the odds of getting 30\% or
more extra heads in 1000 tosses, rather than 20\%.  In that case,
$c= 1.3$ instead of $1.2$.  Consequently, the parameter $k$ rises from
$0.0187$ to about $0.0410$, which may seem insignificant.  But because
$k$ appears in the exponent of the upper bound, the final probability
decreases from around 1 in 10,000 to about 1 in a billion!

\subsubsection{Pick-4}

Pick-4 is a lottery game where you pick a 4-digit number between 0000
and 9999.  If your number comes up in a random drawing, then you win.
Your chance of winning is 1 in 10,000.  And if 10 million people play,
then the expected number of winners is 1000.  The lottery operator's
nightmare is that the number of winners is much greater; say, 2000 or
more.  What are the odds of that?

Let $T_i$ be an indicator for the event that the $i$-th player wins.
Then $T = T_1 + \cdots + T_n$ is the total number of winners.  If we
assume that the players' picks and the winning number are independent
and uniform, then the indicators $T_i$ are independent, as required by
the Chernoff bound.  

\begin{editingnotes}
Add comment about how unrealistic these assumptions are because people
frequently play a few favorite numbers.

The assumptions would be plausible for a version where people buy
tickets with randomly assigned numbers, so they can't pick their own
number.
\end{editingnotes}

Now, 2000 winners would be twice the expected
number.  So we choose $c = 2$, compute $k = c \ln c - c + 1 =
0.386\dots$, and plug these values into the Chernoff bound:
\begin{align*}
\pr{T \geq 2000} & = \pr{T \geq 2 \expect{T}} \\
  & \leq e^{-k \expect{T}} \\
  & = e^{- (0.386\dots) \cdot 1000} \\
  & < e^{-386}
\end{align*}
So there is almost no chance that the lottery operator pays out
double.  In fact, the number of winners won't even be 10\% higher than
expected very often.  To prove that, let $c = 1.1$, compute $k = c \ln
c - c + 1 = 0.00484\dots$, and plug in again:
\begin{align*}
\pr{T \geq 1.1 \expect{T}} & \leq e^{-k \expect{T}} \\
  & = e^- 0.00484\dots * 1000 \\
  & < 0.01
\end{align*}
So the Pick-4 lottery may be exciting for the players, but the lottery
operator has little doubt about the outcome!

\subsection{Randomized Load Balancing}

Now let's return to Fussbook and its load balancing problem.
Specifically, we need to determine how many machines suffice to ensure
that no server is overloaded; that is, assigned to do more than 10
minutes of work in a 10-minute interval.

To begin, let's find the probability that the first server is
overloaded.  Let $T_i$ be the number of seconds that the first server
spends on the $i$-th task.  So $T_i$ is zero if the task is assigned
to another machine, and otherwise $T_i$ is the length of the task.
Then $T = \sum T_i$ is the total length of tasks assigned to the
server.  We need to upper bound $\pr{T \geq 600}$; that is, the
probability that the first server is assigned more than 600 seconds
(or, equivalently, 10 minutes) of work.

The Chernoff bound is applicable only if the $T_i$ are mutually
independent and take on values in the range $[0, 1]$.  The first
condition is satisfied if we assume that tasks lengths and assignments
are independent.  And the second condition is satisfied because
processing even the most interminable harangue takes at most 1 second.

In all, there are 24,000 tasks each with an expected length of 1/4
second.  Since tasks are assigned to computers at random, the expected
load on the first server is:
\begin{align*}
\expect{T} & = \frac{24,000 \mbox{ tasks} \cdot 1/4 \mbox{ second per task}}
  {m \mbox{ machines}} \\
  & = 6000 / m \mbox{ seconds}
\end{align*}
For example, if there are $m = 10$ machines, then the expected load on
the first server is 600 seconds, which is 100\% of its capacity.

Now we can use the Chernoff bound to upper bound the probability that
the first server is overloaded:
\begin{align*}
\pr{T \geq 600} & = \pr{T \geq c \expect{T}} \\
  & \leq e^{-(c \ln c - c + 1) \cdot 6000 / m}
\end{align*}
Equality holds on the first line when $c = m / 10$, since $c \expect{T} =
(m / 10) \cdot (6000 / m) = 600$.  The probability that {\em some}
server is overloaded is at most $m$ times the probability that the
first server is overloaded:
\[
\pr{\mbox{some server is overloaded}} \leq m e^{-(c \ln c - c + 1) \cdot 6000 / m}
\]
Some values of this upper bound are tabulated below:
\[
\begin{array}{rcll}
m & = & 11: & 0.784\dots \\
m & = & 12: & 0.000999\dots \\
m & = & 13: & 0.0000000760\dots
\end{array}
\]
These values suggest that a system with $m = 11$ machines might suffer
immediate overload, $m = 12$ machines could fail in a few days, but $m
= 13$ should be fine for a century or two!

\subsection{Proof of the Chernoff Bound}

The proof of the Chernoff bound is somewhat involved.  Heck, even {\em
  Chernoff} didn't come up with it!  His friend, Herman Rubin, showed
him the argument.  Thinking the bound not very significant, Chernoff
did not credit Rubin in print.  He felt pretty bad when it became
famous!
\begin{editingnotes}
 References: ``A Conversation with Herman Chernoff" Statistical
 Science 1996, Vol 11, No 4, pp 335-350.

Here is the theorem again, for reference:

\begin{theorem}[Chernoff Bound]
Let $T_1, \dots T_n$ be mutually independent random variables such
that $0 \leq T_i \leq 1$ for all $i$.  Let $T = T_1 + \cdots + T_n$.
Then for all $c \geq 1$,
\[
\pr{T \geq c \expect{T}} \leq e^{- k \expect{T}}
\]
where $k = c \ln c - c + 1$.
\end{theorem}
\end{editingnotes}

For clarity, we'll go through the proof "top down"; that is, we'll use
facts that are proved immediately afterward.

\begin{proof} The key step is to exponentiate both sides of the
  inequality $T > c \expect{T}$ and then apply the Markov bound.
\begin{align*}
\pr{T \geq c \expect{T}} & = \pr{c^T \geq c^{c \expect{T}}} \\
  & \leq \frac{\expect{c^T}}{c^{c \expect{T}}} & \text{(by Markov)}\\
  & \leq \frac{e^{(c-1) \expect{T}}}{c^{c \expect{T}}} \\
  & = e^{- (c \ln c - c + 1) \expect{T}}
\end{align*}
In the third step, the numerator is rewritten using the inequality
\[
\expect{c^T} \leq e^{(c-1) \expect{T}}
\]
which is proved below in Lemma~\ref{chernoff-lemma1}.  The final step
is simplification.  (Recall that $c^c$ is equal to $e ^{c \ln c}$.)
\end{proof}

Algebra aside, there is a brilliant idea in this proof: in this
context, exponentiating somehow supercharges the Markov bound.  This
is not true in general!  One unfortunate side-effect is that we have
to bound some nasty expectations involving exponentials in order to
complete the proof.  This is done in the two lemmas below, where
variables take on values as in Theorem~\ref{chernoff}.

\begin{lemma}
\label{chernoff-lemma1}
\[
\expect{c^T} \leq e^{(c-1) \expect{T}}
\]
\end{lemma}

\begin{proof}
\begin{align*}
    \expect{c^T} & = \expect{c^{T_1 + \cdots + T_n}} \\
            & = \expect{c^{T_1} \cdots c^{T_n}} \\
            & = \expect{c^{T_1}}  \cdots \expect{c^{T_n}} \\
            & \leq e^{(c-1) \expect{T_1}} \cdots  e^{(c-1) \expect{T_n}} \\
            & = e^{(c-1) (\expect{T_1} + \cdots + \expect{T_n})} \\
            & = e^{(c-1) \expect{T_1 + \cdots + T_n}} \\
            & = e^{(c-1) \expect{T}}
\end{align*}
The first step uses the definition of $T$, and the second is just
algegra.  The third step uses the fact that the expectation of a
product of independent random variables is the product of the
expectations.  (This is where the requirement that the $T_i$ be
independent is used.)  Then we bound each term using the inquality
\[
    \expect{c^{T_i}} \leq e^{(c - 1) \expect{T_i}}
\]
which is proved in Lemma~\ref{chernoff-lemma2}.  The last steps are
simplifications using algebra and linearity of expectation.
\end{proof}

\begin{lemma}
\label{chernoff-lemma2}
\[
\expect{c^{T_i}} \leq e^{(c - 1) \expect{T_i}}
\]
\end{lemma}

\begin{proof}
All summations below range over values $v$ taken on by the random
variable $T_i$, which are all required to be in the interval $[0, 1]$.
\begin{align*}
\expect{c^{T_i}} & = \sum c^v \pr{T_i = v} \\
           & \leq \sum (1 + (c-1) v) \pr{T_i = v} \\
           & = \sum \pr{T_i = v} + (c-1) v \pr{T_i = v} \\
           & = \sum \pr{T_i = v} + \sum(c-1) v \pr{T_i = v} \\
           & = 1 + (c - 1) \sum v \pr{T_i = v} \\
           & = 1 + (c - 1) \expect{T_i} \\
           & \leq e^{(c - 1) \expect{T_i}}
\end{align*}
The first step uses the definition of expectation.  The second step
relies on the inequality $c^v \leq 1 + (c-1) v$, which holds for all v
in $[0,1]$ and $c \geq 1$.  This follows from the general principle that a
convex function ($c^v$) is less than a linear function ($1 + (c-1) v$)
between their points of intersection ($v = 0$ and $1$).  This
inequality is why the variables $T_i$ are restricted to the interval
$[0, 1]$.)  We then multiply out inside the summation and split into
two sums.  The first sum adds the probabilities of all possible
outcomes, so it is equal to 1.  After pulling the constant $c - 1$ out
of the second sum, we're left with the definition of $\expect{T_i}$.  The
final step uses the standard inequality $1 + z \leq e^z$, which holds
for all real $z$.
\end{proof}

\begin{editingnotes}
Add problems
\end{editingnotes}

\iffalse

\begin{problems}
\classproblems

\end{problems}
\fi

\endinput
