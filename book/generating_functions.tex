\chapter{Generating Functions}\label{generating_function_chap}

\idx{Generating Functions} are one of the most surprising and useful
inventions in Discrete Math.  Roughly speaking, generating functions
transform problems about \textit{sequences} into problems about
\textit{functions}.  This is great because we've got piles of mathematical
machinery for manipulating functions.  Thanks to generating functions, we
can apply all that machinery to problems about sequences.  In this way, we
can use generating functions to solve all sorts of counting problems.
There is a huge chunk of mathematics concerning generating functions, so
we will only get a taste of the subject.

In this chapter, we'll put sequences in angle brackets to more clearly
distinguish them from the many other mathematical expressions floating
around.

The \term{ordinary generating function} for $\ang{g_0, g_1, g_2, g_3
  \dots}$ is the power series:
\[
G(x) = g_0 + g_1 x + g_2 x^2 + g_3 x^3 + \cdots.
\]
There are a few other kinds of generating functions in common use,
but ordinary generating functions are enough to illustrate the power of
the idea, so we'll stick to them.  So from now on \emph{generating
  function} will mean the ordinary kind.

A generating function is a ``formal'' power series in the sense that we
usually regard $x$ as a placeholder rather than a number.  Only in rare
cases will we actually evaluate a generating function by letting $x$ take
a real number value, so we generally ignore the issue of \idx{convergence}.

Throughout this chapter, we'll indicate the correspondence between a
sequence and its generating function with a double-sided arrow as
follows:
%
\[
\ang{g_0, g_1, g_2, g_3, \dots}
    \corresp g_0 + g_1 x + g_2 x^2 + g_3 x^3 + \cdots
\]
%
For example, here are some sequences and their generating functions:
%
\begin{align*}
\ang{0, 0, 0, 0, \dots}
    & \corresp 0 + 0 x + 0 x^2 + 0 x^3 + \cdots = 0 \\
\ang{1, 0, 0, 0, \dots}
    & \corresp 1 + 0 x + 0 x^2 + 0 x^3 + \cdots = 1 \\
\ang{3, 2, 1, 0, \dots}
    & \corresp 3 + 2 x + 1 x^2 + 0 x^3 + \cdots = 3 + 2 x + x^2
\end{align*}
%
The pattern here is simple: the $i$th term in the sequence (indexing
from 0) is the coefficient of $x^i$ in the generating function.

Recall that the sum of an infinite \idx{geometric series} is:
%
\[
1 + z + z^2 + z^3 + \cdots = \frac{1}{1 - z}
\]
%
This equation does not hold when $\abs{z} \geq 1$, but as remarked, we
don't worry about convergence issues.  This formula gives \idx{closed
  form} generating functions for a whole range of sequences.  For example:
%
\[
\begin{array}{rll}
\ang{1, 1, 1, 1, \dots}
    & \corresp 1 + x + x^2 + x^3 + \cdots
    & = \dfrac{1}{1-x} \\
\\
\ang{1, -1, 1, -1, \dots}
    & \corresp 1 - x + x^2 - x^3 + x^4 - \cdots
    &  = \dfrac{1}{1+x} \\
\\
\ang{1, a, a^2, a^3, \dots}
    & \corresp 1 + a x + a^2 x^2 + a^3 x^3 + \cdots
    &  = \dfrac{1}{1-ax} \\
\\
\ang{1, 0, 1, 0, 1, 0, \dots}
    & \corresp 1 + x^2 + x^4 + x^6 + \cdots
    & = \dfrac{1}{1 - x^2}
\end{array}
\]

\section{Operations on Generating Functions}

The magic of generating functions is that we can carry out all sorts
of manipulations on sequences by performing mathematical operations on
their associated generating functions.  Let's experiment with various
operations and characterize their effects in terms of sequences.

\subsection{Scaling}

Multiplying a generating function by a constant scales every term in
the associated sequence by the same constant.  For example, we noted
above that:
%
\[
\ang{1, 0, 1, 0, 1, 0, \dots}
    \corresp 1 + x^2 + x^4 + x^6 + \cdots = \frac{1}{1 - x^2}
\]
%
Multiplying the generating function by 2 gives
%
\[
\frac{2}{1 - x^2} = 2 + 2 x^2 + 2 x^4 + 2 x^6 + \cdots
\]
%
which generates the sequence:
%
\[
\ang{2, 0, 2, 0, 2, 0, \dots}
\]

\begin{mathrule}[\idx{Scaling Rule}]
\label{rule:scaling}
If
\[\ang{f_0, f_1, f_2, \dots} \corresp F(x),\]
then
%
\[
\ang{c f_0,\ c f_1,\ c f_2,\ \dots} \corresp c \cdot F(x).
\]
\end{mathrule}

The idea behind this rule is that:
\begin{eqnarray*}
\ang{c f_0, c f_1, c f_2, \dots}
    & \corresp & c f_0 + c f_1 x + c f_2 x^2 + \cdots \\
    & = & c \cdot (f_0 + f_1 x + f_2 x^2 + \cdots) \\
    & = & c F(x)
\end{eqnarray*}

\subsection{Addition}

Adding generating functions corresponds to adding the two sequences
term by term.  For example, adding two of our earlier examples gives:
%
\[
\begin{array}{cccccccccccl}
  & \langle & 1, & 1, & 1, & 1, & 1, & 1, & \dots & \rangle &
    \corresp & \dfrac{1}{1-x} \\
\\
+ & \langle & 1, & -1, & 1, & -1, & 1, & -1, & \dots & \rangle &
    \corresp & \dfrac{1}{1+x} \\
\\
\hline
\\
& \langle & 2, & 0, & 2, & 0, & 2, & 0, & \dots & \rangle &
    \corresp & \dfrac{1}{1-x} + \dfrac{1}{1+x}
\end{array}
\]
%
We've now derived two different expressions that both generate the
sequence $\ang{2, 0, 2, 0, \dots}$.  They are, of course, equal:
%
\[
\frac{1}{1-x} + \frac{1}{1+x} = \frac{(1 + x) + (1 - x)}{(1-x)(1+x)} = \frac{2}{1-x^2}
\]

\begin{mathrule}[\index{Addition Rule (for generating functions)}Addition Rule]
\label{rule:addition}
If
%
\begin{align*}
\ang{f_0, f_1, f_2, \dots} & \corresp F(x), & \text{and} \\
\ang{g_0, g_1, g_2, \dots} & \corresp G(x),
\end{align*}
%
then
%
\[
\ang{f_0 + g_0,\ f_1 + g_1,\ f_2 + g_2,\ \dots}
    \corresp F(x) + G(x).
\]
\end{mathrule}

The idea behind this rule is that:
\begin{eqnarray*}
\ang{f_0 + g_0,\ f_1 + g_1,\ f_2 + g_2,\ \dots}
    & \corresp & \sum_{n=0}^\infty (f_n + g_n) x^n \\
    & = & \paren{\sum_{n=0}^\infty f_n x^n} +
        \paren{\sum_{n=0}^\infty g_n x^n} \\
    & = & F(x) + G(x)
\end{eqnarray*}

\subsection{Right Shifting}

Let's start over again with a simple sequence and its generating
function:
%
\[
\ang{1, 1, 1, 1, \dots} \corresp \frac{1}{1-x}
\]
%
Now let's \term{right-shift} the sequence by adding $k$ leading
zeros:
%
\begin{eqnarray*}
\langle\underbrace{0, 0, \dots, 0}_{\text{$k$ zeroes}}, 1, 1, 1, \dots\rangle
        & \corresp & x^k + x^{k+1} + x^{k+2} + x^{k+3} + \cdots \\
        & = & x^k \cdot (1  + x + x^2 + x^3 + \cdots) \\
        & = &\frac{x^k}{1-x}
\end{eqnarray*}
%
Evidently, adding $k$ leading zeros to the sequence corresponds to
multiplying the generating function by $x^k$.  This holds true in
general.

\begin{mathrule}[\idx{Right-Shift Rule}]
\label{rule:shift}
If $\ang{f_0, f_1, f_2, \dots} \corresp F(x)$, then:
%
\[
\langle \underbrace{0, 0, \dots, 0}_{\text{$k$ zeroes}}, f_0, f_1, f_2, \dots\rangle \corresp x^k \cdot F(x)
\]
\end{mathrule}

The idea behind this rule is that:
\begin{eqnarray*}
\langle\overbrace{0, 0, \dots, 0}^{\text{$k$ zeroes}}, f_0, f_1, f_2, \dots\rangle
 & \corresp & f_0 x^k + f_1 x^{k+1} + f_2 x^{k+2} + \cdots \\
 & = & x^k \cdot (f_0 + f_1 x + f_2 x^2 + f_3 x^3 + \cdots) \\
 & = & x^k \cdot F(x)
\end{eqnarray*}

\subsection{Differentiation}

What happens if we take the \term{derivative} of a generating
function?  As an example, let's differentiate the now-familiar
generating function for an infinite sequence of 1's.
%
\begin{eqnarray}
\frac{d}{dx} \ (1 + x + x^2 + x^3 + x^4 + \cdots)
        & = & \frac{d}{dx} \paren{\frac{1}{1-x}} \notag\\
1 + 2x + 3x^2 + 4x^3 + \cdots & = & \frac{1}{(1-x)^2} \label{sumixi-1}\\
\ang{1, 2, 3, 4, \dots}  & \corresp & \frac{1}{(1-x)^2}\notag
\end{eqnarray}
%
We found a generating function for the sequence $\ang{1, 2, 3, 4,
\dots}$ of positive integers!

In general, differentiating a generating function has two effects on
the corresponding sequence: each term is multiplied by its index and
the entire sequence is shifted left one place.

\begin{mathrule}[\idx{Derivative Rule}]
\label{rule:derivative}
If
\[
\ang{f_0, f_1, f_2, f_3, \dots} \corresp F(x),
\]
then
%
\[
\ang{f_1, 2f_2, 3f_3, \dots} \corresp F'(x).
\]
\end{mathrule}

The idea behind this rule is that:
\begin{align*}
\ang{f_1, 2f_2, 3f_3, \dots}
    & \corresp f_1 + 2 f_2 x + 3 f_3 x^2 + \cdots \\
    & = \frac{d}{dx} \ (f_0 + f_1 x + f_2 x^2 + f_3 x^3 + \cdots) \\
    & = \frac{d}{dx} \ F(x)
\end{align*}

The Derivative Rule is very useful.  In fact, there is frequent,
independent need for each of differentiation's two effects,
multiplying terms by their index and left-shifting one place.
Typically, we want just one effect and must somehow cancel out the
other.  For example, let's try to find the generating function for the
sequence of squares, $\ang{0, 1, 4, 9, 16, \dots}$.  If we could
start with the sequence $\ang{1, 1, 1, 1, \dots}$ and multiply each term by
its index two times, then we'd have the desired result:
%
\[
\ang{0 \cdot 0,\ 1 \cdot 1,\ 2 \cdot 2,\ 3 \cdot 3,\ \dots}
= 
\ang{0, 1, 4, 9, \dots}
\]
%
A challenge is that differentiation not only multiplies each term by
its index, but also shifts the whole sequence left one place.
However, the Right-Shift Rule~\ref{rule:shift} tells how to cancel out
this unwanted left-shift: multiply the generating function by $x$.

Our procedure, therefore, is to begin with the generating function for
$\ang{1, 1, 1, 1, \dots}$, differentiate, multiply by $x$, and then
differentiate and multiply by $x$ once more.
%
\begin{align*}
\ang{1, 1, 1, 1, \dots}  & \corresp \frac{1}{1-x} \\
\ang{1, 2, 3, 4, \dots}  & \corresp \frac{d}{dx} \ \frac{1}{1-x}
                                          = \frac{1}{(1-x)^2} \\
\ang{0, 1, 2, 3, \dots}  & \corresp x \cdot \frac{1}{(1-x)^2}
                                          = \frac{x}{(1-x)^2} \\
\ang{1, 4, 9, 16, \dots} & \corresp \frac{d}{dx} \ \frac{x}{(1-x)^2}
                                          = \frac{1+x}{(1-x)^3} \\
\ang{0, 1, 4, 9, \dots}  & \corresp x \cdot \frac{1+x}{(1-x)^3}
                                          = \frac{x(1+x)}{(1-x)^3}
\end{align*}
%
Thus, the generating function for squares is:
%
\begin{equation}\label{squares_gen_func}
\frac{x(1+x)}{(1-x)^3}
\end{equation}

\subsection{Products}

\begin{mathrule}[\index{Product Rule (for generating functions)} Product Rule]
\label{rule:product}
If
%
\[
\ang{a_0, a_1, a_2, \dots} \corresp A(x),\quad \text{and}\quad
\ang{b_0, b_1, b_2, \dots} \corresp B(x),
\]
%
then
%
\[
\ang{c_0,\ c_1 ,\ c_2,\ \dots} \corresp A(x) \cdot B(x),
\]
\end{mathrule}
where
\[
c_n \eqdef a_0 b_n + a_1 b_{n-1} + a_2 b_{n-2} + \cdots + a_n b_0.
\]

To understand this rule, let
\[
C(x) \eqdef A(x) \cdot B(x) = \sum_{n=0}^{\infty} c_n x^n.
\]

We can evaluate the product $A(x) \cdot B(x)$ by using a table to identify
all the cross-terms from the product of the sums:
%
\[
\begin{array}{c|@{\quad}c@{\qquad}c@{\qquad}c@{\qquad}c@{\qquad}c}
        & b_0 x^0 & b_1 x^1 & b_2 x^2 & b_3 x^3 & \dots \\
\hline
\\
a_0 x^0 & a_0 b_0 x^0 & a_0 b_1 x^1 & a_0 b_2 x^2 & a_0 b_3 x^3 & \dots \\
\\
a_1 x^1 & a_1 b_0 x^1 & a_1 b_1 x^2 & a_1 b_2 x^3 & \dots \\
\\
a_2 x^2 & a_2 b_0 x^2 & a_2 b_1 x^3 & \dots \\
\\
a_3 x^3 & a_3 b_0 x^3 & \dots \\
\\
\vdots & \dots\\
\end{array}
\]
%
Notice that all terms involving the same power of $x$ lie on a
/-sloped diagonal.  Collecting these terms together, we find that the
coefficient of $x^n$ in the product is the sum of all the terms on the
$(n+1)$st diagonal, namely,
\begin{equation}\label{conv}
a_0 b_n + a_1 b_{n-1} + a_2 b_{n-2} + \cdots + a_n b_0.
\end{equation}
This expression~\eqref{conv} may be familiar from a signal processing
course; the sequence $\ang{c_0, c_1, c_2, \dots}$ is called the
\term{convolution} of sequences $\ang{a_0, a_1, a_2, \dots}$ and
$\ang{b_0, b_1, b_2, \dots}$.

\hyperdef{recur}{rences}{\section{The Fibonacci Sequence}}

Sometimes we can find nice generating functions for more complicated
sequences.  For example, here is a generating function for the
\idx{Fibonacci} numbers:
%
\[
\ang{0, 1, 1, 2, 3, 5, 8, 13, 21, \dots} \corresp \frac{x}{1-x-x^2}
\]
%
The Fibonacci numbers may seem like a fairly nasty bunch, but the
generating function is simple!

We're going to derive this generating function and then use it to find a
closed form for the $n$th Fibonacci number.  The techniques we'll use are
applicable to a large class of recurrence equations.

\subsection{Finding a Generating Function}

Let's begin by recalling the definition of the Fibonacci numbers:
%
\begin{align*}
f_0 & = 0 \\
f_1 & = 1 \\
f_n & = f_{n-1} + f_{n-2} \qquad \text{(for $n \geq 2$)}
\end{align*}
%
We can expand the final clause into an infinite sequence of equations.
Thus, the Fibonacci numbers are defined by:
%
\begin{align*}
f_0 = & 0 \\
f_1 = & 1 \\
f_2 = & f_1 + f_0 \\
f_3 = & f_2 + f_1 \\
f_4 = & f_3 + f_2 \\
      & \vdots
\end{align*}

Now the overall plan is to \textit{define} a function $F(x)$ that
generates the sequence on the left side of the equality symbols, which
are the Fibonacci numbers.  Then we \textit{derive} a function that
generates the sequence on the right side.  Finally, we equate the two
and solve for $F(x)$.  Let's try this.  First, we define:
%
\[
F(x) = f_0 + f_1 x + f_2 x^2 + f_3 x^3 + f_4 x^4 + \cdots
\]
%
Now we need to derive a generating function for the sequence:
%
\[
\ang{0,\ 1,\ f_1 + f_0,\ f_2 + f_1,\ f_3 + f_2,\ \dots}
\]
%
One approach is to break this into a sum of three sequences for which
we know generating functions and then apply the Addition Rule:
%
\[
\begin{array}{ccccccccccl}
  & \langle & 0, & 1, & 0, & 0, & 0, & \dots & \rangle
    & \corresp & x \\
  & \langle & 0, & f_0, & f_1, & f_2, & f_3, & \dots & \rangle
    & \corresp & x F(x) \\
+ & \langle & 0, & 0, & f_0, & f_1, & f_2, & \dots & \rangle
    & \corresp & x^2 F(x) \\ \hline
  & \langle & 0, & 1 + f_0, & f_1 + f_0, & f_2 + f_1, & f_3 + f_2, & \dots & \rangle
    & \corresp & x + x F(x) + x^2 F(x) \\
\end{array}
\]
%
This sequence is almost identical to the right sides of the Fibonacci
equations.  The one blemish is that the second term is $1 + f_0$
instead of simply 1.  However, this amounts to nothing, since $f_0 =
0$ anyway.

Now if we equate $F(x)$ with the new function $x + x F(x) + x^2 F(x)$,
then we're implicitly writing down \textit{all} of the equations that
define the Fibonacci numbers in one fell swoop:
%
\[
\begin{array}{ccc@{\ }c@{\ }c@{\ }c@{\ }c@{\ }c@{\ }c@{\ }c@{\ }c@{\ }c@{\ }c@{\ }c@{\ }c}
F(x)
    & = & f_0 & + & f_1 & x & + & f_2 & x^2 & + & f_3 & x^3 & + \cdots \\
\shortparallel && \shortparallel && \shortparallel &&& \shortparallel &&& \shortparallel\\
x + x F(x) + x^2 F(x)
    & = & 0 & + & (1 + f_0) & x & + & (f_1 + f_0) & x^2 & + & (f_2 + f_1) & x^3 & + \cdots
\end{array}
\]
%
Solving for $F(x)$ gives the generating function for the Fibonacci
sequence:
%
\[
F(x)  = x + x F(x) + x^2 F(x)
\]
so
\[
F(x) = \frac{x}{1 - x - x^2}.
\]
%
Sure enough, this is the simple generating function we claimed at the
outset.

\subsection{Finding a Closed Form}

Why should one care about the generating function for a sequence?  There
are several answers, but here is one: if we can find a generating function
for a sequence, then we can often find a closed form for the $n$th
coefficient--- which can be pretty useful!  For example, a \idx{closed
  form} for the coefficient of $x^n$ in the power series for $x / (1 - x -
x^2)$ would be an explicit formula for the $n$th Fibonacci number.

So our next task is to extract coefficients from a generating function.
There are several approaches.  For a generating function that is a ratio
of polynomials, we can use the method of \term{partial fractions}, which
you learned in calculus.  Just as the terms in a partial fraction
expansion are easier to integrate, the coefficients of those terms are
easy to compute.

Let's try this approach with the generating function for Fibonacci
numbers.  First, we factor the denominator:
%
\[
1 - x - x^2 = (1 - \alpha_1 x) (1 - \alpha_2 x)
\]
%
where $\alpha_1 = \frac{1}{2}(1 + \sqrt{5})$ and $\alpha_2 =
\frac{1}{2}(1 - \sqrt{5})$.  Next, we find $A_1$ and $A_2$ which
satisfy:
%
\[
\frac{x}{1 - x - x^2} =
        \frac{A_1}{1 - \alpha_1 x} + \frac{A_2}{1 - \alpha_2 x}
\]
%
We do this by plugging in various values of $x$ to generate linear
equations in $A_1$ and $A_2$.  We can then find $A_1$ and $A_2$ by
solving a linear system.  This gives:
%
\begin{gather*}
A_1 = \frac{1}{\alpha_1 - \alpha_2} = \frac{1}{\sqrt{5}} \\
A_2 = \frac{-1}{\alpha_1 - \alpha_2} = -\frac{1}{\sqrt{5}}
\end{gather*}

Substituting into the equation above gives the partial fractions
expansion of $F(x)$:
%
\[
\frac{x}{1 - x - x^2} =
        \frac{1}{\sqrt{5}}
        \paren{\frac{1}{1 - \alpha_1 x} - \frac{1}{1 - \alpha_2 x}}
\]
%
Each term in the partial fractions expansion has a simple power series
given by the geometric sum formula:
%
\begin{align*}
\frac{1}{1 - \alpha_1 x} & = 1 + \alpha_1 x + \alpha_1^2 x^2 + \cdots \\
\frac{1}{1 - \alpha_2 x} & = 1 + \alpha_2 x + \alpha_2^2 x^2 + \cdots
\end{align*}
%
Substituting in these series gives a power series for the generating
function:
%
\begin{align*}
F(x) & =
  \frac{1}{\sqrt{5}}
  \paren{\frac{1}{1 - \alpha_1 x} -
        \frac{1}{1 - \alpha_2 x}}\\
    & = \frac{1}{\sqrt{5}}
        \paren{(1 + \alpha_1 x + \alpha_1^2 x^2 + \cdots) -
        (1 + \alpha_2 x + \alpha_2^2 x^2 + \cdots)},
\end{align*}
so
\begin{align*}
 f_n & = \frac{\alpha_1^n - \alpha_2^n}{\sqrt{5}} \\
                & = \frac{1}{\sqrt{5}}
        \paren{
         \paren{\frac{1 + \sqrt{5}}{2}}^n -
         \paren{\frac{1 - \sqrt{5}}{2}}^n}
\end{align*}

This formula may be scary and astonishing ---it's not even obvious that
its value is an integer ---but it's very useful.  For example, it provides
(via the repeated squaring method) a much more efficient way to compute
Fibonacci numbers than crunching through the recurrence, and it also
clearly reveals the exponential growth of these numbers.

\begin{problems}
\classproblems
\pinput{CP_Fibonacci_and_bunnies}
\pinput{CP_towers_of_Sheboygan}

\homeworkproblems
\pinput{PS_gen_fcn_quotient_polynomials}
\pinput{PS_Catalan_numbers_meyer_version}

\examproblems
\pinput{MQ_gen_2}

\end{problems}

\hyperdef{gen}{count}{\section{Counting with Generating Functions}}

Generating functions are particularly useful for solving counting
problems.  In particular, problems involving choosing items from a set
often lead to nice generating functions by letting the coefficient of
$x^n$ be the number of ways to choose $n$ items.

\subsection{Choosing Distinct Items from a Set}

The generating function for binomial coefficients follows directly
from the \idx{Binomial Theorem}:
%
\begin{eqnarray*}
\ang{ \binom{k}{0}, \binom{k}{1}, \binom{k}{2}, \dots, \binom{k}{k},
        0, 0, 0, \dots }
    & \corresp & \binom{k}{0} + \binom{k}{1} x + \binom{k}{2} x^2 + \cdots + \binom{k}{k} x^k \\
    & = & (1 + x)^k
\end{eqnarray*}

Thus, the coefficient of $x^n$ in $(1 + x)^k$ is $\binom{k}{n}$, the
number of ways to choose $n$ distinct items from a set of size $k$.  For
example, the coefficient of $x^2$ is $\binom{k}{2}$, the number of ways to
choose 2 items from a set with $k$ elements.  Similarly, the coefficient
of $x^{k+1}$ is the number of ways to choose $k+1$ items from a size $k$
set, which is zero.  (Watch out for this reversal of the roles that $k$
and $n$ played in earlier examples; we're led to this reversal because
we've been using $n$ to refer to the power of $x$ in a power series.)

\subsection{Building Generating Functions that Count}

Often we can translate the description of a counting problem directly
into a generating function for the solution.  For example, we could
figure out that $(1 + x)^k$ generates the number of ways to select $n$
distinct items from a $k$-element set without resorting to the
Binomial Theorem or even fussing with binomial coefficients!

Here is how.  First, consider a single-element set $\set{a_1}$.  The
generating function for the number of ways to select $n$ elements from
this set is simply $1 + x$: we have 1 way to select zero elements, 1
way to select one element, and 0 ways to select more than one element.
Similarly, the number of ways to select $n$ elements from the set
$\set{a_2}$ is also given by the generating function $1 + x$.  The
fact that the elements differ in the two cases is irrelevant.

Now here is the main trick: \textit{the generating function for
choosing elements from a union of disjoint sets is the product of the
generating functions for choosing from each set.}  We'll justify this
in a moment, but let's first look at an example.  According to this
principle, the generating function for the number of ways to select
$n$ elements from the $\set{a_1, a_2}$ is:
%
\[
\underbrace{(1 + x)}_{
\begin{array}{cc}
\text{gen func for} \\
\text{selecting an $a_1$}
\end{array}}
\cdot
\underbrace{(1 + x)}_{
\begin{array}{cc}
\text{gen func for} \\
\text{selecting an $a_2$}
\end{array}}
=
\underbrace{(1 + x)^2}_{
\begin{array}{cc}
\text{gen func for} \\
\text{selecting from} \\
\set{a_1, a_2}
\end{array}}
= 1 + 2x + x^2
\]
%
Sure enough, for the set $\set{a_1, a_2}$, we have 1 way to select
zero elements, 2 ways to select one element, 1 way to select two
elements, and 0 ways to select more than two elements.

Repeated application of this rule gives the generating function for
selecting $n$ items from a $k$-element set $\set{a_1, a_2, \dots,
a_k}$:
%
\[
\underbrace{(1 + x)}_{
\begin{array}{cc}
\mbox{gen func for} \\
\text{selecting an $a_1$}
\end{array}}
\cdot
\underbrace{(1 + x)}_{
\begin{array}{cc}
\mbox{gen func for} \\
\text{selecting an $a_2$}
\end{array}}
\cdots
\underbrace{(1 + x)}_{
\begin{array}{cc}
\mbox{gen func for} \\
\text{selecting an $a_k$}
\end{array}}
=
\underbrace{(1 + x)^k}_{
\begin{array}{cc}
\mbox{gen func for} \\
\text{selecting from} \\
\set{a_1, a_2, \dots, a_k}
\end{array}}
\]
%
This is the same generating function that we obtained by using the
Binomial Theorem.  But this time around we translated directly from
the counting problem to the generating function.

We can extend these ideas to a general principle:

% This is hand wavy and could use improvement.  I hope they get the idea.

\begin{mathrule}[\idx{Convolution Rule}]
Let $A(x)$ be the generating function for selecting items from set
${\cal A}$, and let $B(x)$ be the generating function for selecting
items from set ${\cal B}$.  If ${\cal A}$ and ${\cal B}$ are disjoint,
then the generating function for selecting items from the union ${\cal
A} \cup {\cal B}$ is the product $A(x) \cdot B(x)$.
\end{mathrule}

This rule is rather ambiguous: what exactly are the rules governing
the selection of items from a set?  Remarkably, the Convolution Rule
remains valid under \textit{many} interpretations of selection.  For
example, we could insist that distinct items be selected or we might
allow the same item to be picked a limited number of times or any
number of times.  Informally, the only restrictions are that (1) the
order in which items are selected is disregarded and (2) restrictions
on the selection of items from sets ${\cal A}$ and ${\cal B}$ also
apply in selecting items from ${\cal A} \cup {\cal B}$.  (Formally,
there must be a bijection between $n$-element selections from ${\cal
A} \cup {\cal B}$ and ordered pairs of selections from ${\cal A}$ and
${\cal B}$ containing a total of $n$ elements.)

To count the number of ways to select $n$ items from ${\cal A} \cup {\cal
B}$, we observe that we can select $n$ items by choosing $j$ items from
${\cal A}$ and $n - j$ items from ${\cal B}$, where $j$ is any number from
0 to $n$.  This can be done in $a_j b_{n-j}$ ways.  Summing over all the
possible values of $j$ gives a total of
\[
a_0 b_n + a_1 b_{n-1} + a_2 b_{n-2} + \cdots + a_n b_0
\]
ways to select $n$ items from ${\cal A} \cup {\cal B}$.  By the Product
Rule, this is precisely the coefficient of $x^n$ in the series for
$A(x)B(x)$.


\subsection{Choosing Items with Repetition}
\label{sec:rep}

The first counting problem we considered was the number of ways to select
a dozen doughnuts when five flavors were available.  We can generalize
this question as follows: in how many ways can we select $n$ items from a
$k$-element set if we're allowed to pick the same item multiple times?
In these terms, the doughnut problem asks in how many ways we can select
$n=12$ doughnuts from the set of $k=5$ flavors
\[
\set{\text{chocolate}, \text{lemon-filled}, \text{sugar}, \text{glazed}, \text{plain}}
\]
%
where, of course, we're allowed to pick several doughnuts of the same
flavor.   Let's approach this question from a generating functions
perspective.

Suppose we make $n$ choices (with repetition allowed) of items from a set
containing a single item.  Then there is one way to choose zero items, one
way to choose one item, one way to choose two items, etc.  Thus, the
generating function for choosing $n$ elements with repetition from a
1-element set is:
%
\begin{eqnarray*}
\ang{1, 1, 1, 1, \dots}
     & \corresp & 1 + x + x^2 + x^3 + \cdots \\
     & = & \frac{1}{1-x}
\end{eqnarray*}

The Convolution Rule says that the generating function for selecting
items from a union of disjoint sets is the product of the generating
functions for selecting items from each set:
%
\[
\underbrace{\frac{1}{1-x}}_{
\begin{array}{cc}
\mbox{gen func for} \\
\text{choosing $a_1$'s}
\end{array}}
\cdot
\underbrace{\frac{1}{1-x}}_{
\begin{array}{cc}
\mbox{gen func for} \\
\text{choosing $a_2$'s}
\end{array}}
\cdots
\underbrace{\frac{1}{1-x}}_{
\begin{array}{cc}
\mbox{gen func for} \\
\text{choosing $a_k$'s}
\end{array}}
=
\underbrace{\frac{1}{(1-x)^k}}_{
\begin{array}{cc}
\mbox{gen func for} \\
\text{repeated choice from} \\
\set{a_1, a_2, \dots, a_k}
\end{array}}
\]
%
Therefore, the generating function for choosing items from a
$k$-element set with repetition allowed is $1 / (1 - x)^k$.

Now the Bookkeeper Rule tells us that the number of ways to choose $n$
items with repetition from an $k$ element set is
\[
\binom{n+k-1}{n},
\]
so this is the coefficient of $x^n$ in the series expansion of $1 / (1 -
x)^k$.

On the other hand, it's instructive to derive this coefficient
algebraically, which we can do using Taylor's Theorem:

\begin{theorem}[Taylor's Theorem]
\[
f(x) = f(0) + f'(0) x + \frac{f''(0)}{2!} x^2 + \frac{f'''(0)}{3!} x^3 + \cdots
+ \frac{f^{(n)}(0)}{n!} x^n + \cdots.
\]
\end{theorem}

This theorem says that the $n$th coefficient of $1 / (1 - x)^k$ is
equal to its $n$th derivative evaluated at 0 and divided by $n!$.
Computing the $n$th derivative turns out not to be very difficult
(Problem~\ref{CP_nth_derivative_of_A}).
\begin{editingnotes}

Let
%
\[
G(x) \eqdef \frac{1}{(1-x)^k} = (1-x)^{-k}.
\]
%
Then we have:
%
\begin{align*}
G'(x) & = k (1-x)^{-(k+1)} \\
G''(x) & = k (k+1) (1-x)^{-(k+2)} \\
G'''(x) & = k (k+1) (k+2) (1-x)^{-(k+3)} \\
G^{(n)}(x) & = k (k+1) \cdots (k + n - 1)(1-x)^{-(k+n)}
\end{align*}
%
Thus, the coefficient of $x^n$ in the generating function is:
%
\begin{align*}
G^{(n)}(0) / n! & = \frac{k (k+1) \cdots (k + n - 1)}{n!} \\
                & = \frac{(k + n - 1)!}{(k - 1)! \ n!} \\
                & = \binom{n + k - 1}{n}.
\end{align*}

\end{editingnotes}

\begin{problems}
\practiceproblems
\pinput{MQ_bouquet}

\classproblems
\pinput{CP_nth_derivative_of_A}
\pinput{CP_bag_of_donuts}
\pinput{CP_gen_func_sum_of_squares}

\homeworkproblems
\pinput{PS_gen_fcns_pennies_nickels_etc}

\end{problems}

\examproblems
\pinput{MQ_sick_days}
\pinput{MQ_gen_2}
\pinput{MQ_binomial_coef}

\subsection{An ``Impossible'' Counting Problem}

\begin{editingnotes}
Not so impossible.  From Rebecca Freund, F09:

Note that the fruits can be divided into two groups, the apples-and-pears
and the bananas-and-oranges. Once you know how many are apples-and-pears,
there's only one way to distribute them: Use a pear if the number is odd,
otherwise don't. Make the rest apples. Similarly, once you've decided on
the number of bananas-and-oranges, you have to throw in
the-greatest-multiple-of-five-less-than-or-equal-to-that bananas and add
oranges as needed. So the number of apples-and-pears exactly determines
the arrangement. You can have 0-n apples-and-pears, so there are n+1
possibilities.
\end{editingnotes}

So far everything we've done with generating functions we could have
done another way.  But here is an absurd counting problem ---really
over the top!  In how many ways can we fill a bag with $n$ fruits
subject to the following constraints?

\begin{itemize}
\item The number of apples must be even.
\item The number of bananas must be a multiple of 5.
\item There can be at most four oranges.
\item There can be at most one pear.
\end{itemize}

For example, there are 7 ways to form a bag with 6 fruits:
%
\[
\begin{array}{c|ccccccc}
\text{Apples}  & 6 & 4 & 4 & 2 & 2 & 0 & 0 \\
\text{Bananas} & 0 & 0 & 0 & 0 & 0 & 5 & 5 \\
\text{Oranges} & 0 & 2 & 1 & 4 & 3 & 1 & 0 \\
\text{Pears}   & 0 & 0 & 1 & 0 & 1 & 0 & 1
\end{array}
\]
%
These constraints are so complicated that the problem seems hopeless!
But let's see what generating functions reveal.

Let's first construct a generating function for choosing apples.  We
can choose a set of 0 apples in one way, a set of 1 apple in zero
ways (since the number of apples must be even), a set of 2 apples in
one way, a set of 3 apples in zero ways, and so forth.  So we have:
%
\[
A(x) = 1 + x^2 + x^4 + x^6 + \cdots = \frac{1}{1 - x^2}
\]
%
Similarly, the generating function for choosing bananas is:
%
\[
B(x) = 1 + x^5 + x^{10} + x^{15} + \cdots = \frac{1}{1 - x^5}
\]
Now, we can choose a set of 0 oranges in one way, a set of 1 orange in
one way, and so on.  However, we can not choose more than four
oranges, so we have the generating function:
%
\[
O(x) = 1 + x + x^2 + x^3 + x^4 = \frac{1-x^5}{1-x}
\]
%
Here we're using the geometric sum formula.  Finally, we can choose
only zero or one pear, so we have:
%
\[
P(x) = 1 + x
\]

The Convolution Rule says that the generating function for choosing
from among all four kinds of fruit is:
%
\begin{align*}
A(x) B(x) O(x) P(x)
    & = \frac{1}{1-x^2} \frac{1}{1-x^5} \frac{1-x^5}{1-x} (1 + x) \\
    & = \frac{1}{(1-x)^2} \\
    & = 1 + 2x + 3x^2 + 4 x^3 + \cdots
\end{align*}
%
Almost everything cancels!  We're left with $1 / (1-x)^2$, which we
found a power series for earlier: the coefficient of $x^n$ is simply
$n+1$.  Thus, the number of ways to form a bag of $n$ fruits is just
$n+1$.  This is consistent with the example we worked out, since there
were 7 different fruit bags containing 6 fruits.  \textit{Amazing!}
\begin{problems}
\practiceproblems
\pinput{TP_Counting_with_Generating_Functions}

\homeworkproblems
\pinput{PS_crazy_pet_lady}

%\pinput{PS_gen_fcns_with_donuts}

\pinput{PS_Catalan_numbers_meyer_version}

\examproblems
\pinput{FP_boat_trip}

\end{problems}

\endinput
