%NOT IN BOOK

\def\vl{\text{value}}


\chapter{Well-Founded Partial Orderings}

\subsection{A Long-jumping Robot}

\iffalse Begin by defining the trivial ``pick how long'' game: P1 picks $n
\in \naturals$, the P2 and P1 alternate making forced moves.  The game
ends after $n$ forced moves; the last person to move wins.  So P1 strategy
is ``pick and even number.''  Insert here the discussion of ``terminates,
but no bound on number of steps...'' used below.

May also tell the ``guess a bigger number game''joke.
\fi

Suppose we had a robot that moved to points in the plane with natural
number coordinates, that is, at an integer lattice-point in the Northeast
quadrant of the plane.  At every second the robot must move a unit
distance South or West until it can no longer move without leaving the
quadrant.  It may also jump \emph{any} integer distance East, but at every
point in its travels, the number of jumps East is not allowed to be more
than twice the number of previous moves South.

For example, suppose the robot starts at the position (9,8).  It can now
move South to (9,7) or West to (8,8); it can't jump East because there
haven't been any previous South moves.

The robot's moves might continue along the following trajectory: South to
(9,7), East to (23,7), South to (23,6), East to (399,6), West to (398,6),
East to (511,6), West to (510,6), and East to $(10^5,6)$.  At this point
it has moved South twice and East four times, so it can't jump East again
until it makes another move South.

\begin{claim}\label{robotcl}
The robot will always get stuck at the origin.
\end{claim}

If we think of the robot as a nondeterministic state machine, then
Claim~\ref{robotcl} is a termination assertion.  The Claim may seem
obvious, but it really has a different character than the termination
results for the algorithms we've considered so far.  That's because, even
knowing that the starting position was (9,8), for example, there is no
way to bound the total number of moves the robot can make before it gets
stuck.  So we will not be able to prove termination using the
natural-number-valued decreasing variable method of Theorem~\ref{th:decr}.
The robot can delay getting stuck at the origin for as many seconds as it
wants; nevertheless, it can't avoid getting stuck eventually.

Does Claim~\ref{robotcl} still seem obvious?  Before reading further, it's
worth thinking how you might prove it.

\iffalse
First give simple proof using Least Num Principle, first on number of
South moves to show robot reaches row 0, then number of East moves in row
0, then number of West moves in row 0.  Then segue into well-founded
decreasing variables.
\fi

We will prove that the robot always gets stuck at the origin by
generalizing the decreasing variable method, but with decreasing values
that are more general than nonnegative integers.  Namely, the traveling robot
can be modeled with a state machine with states of the form $((x,y),s,e)$
where
\begin{itemize}
\item $(x,y) \in \naturals^2$ is the robot's position,
\item $s$ is the number of moves South the robot took to get to this
position, and
\item $e \le 2s$ is the number of moves East the robot took to get to this
position. 
\end{itemize}

Now we define a derived variable $\vl:\text{States}\to \naturals^3$:
\[
\vl(((x,y),s,e)) \ \eqdef\quad (y,2s-e,x),
\]
and we order the values of states with the \emph{lexicographic} order,
$\lexle$, on $\naturals^3$:
\begin{equation}\label{lex3}
(k,l,m) \lexle (k',l',m') \ \eqdef\quad k < k' \text{ or } (k=k' \text{
and } l < l') \text{ or } (k=k' \text{ and } l = l' \text{ and } m \le m')
\end{equation}

Let's check that values are lexicographically decreasing.  Suppose the
robot is in state $((x,y),s,e)$.
\begin{itemize}
\item If the robot moves West it enters state $((x-1,y),s,e)$, and
\[
\vl(((x-1,y),s,e)) = (y,2s-e,x-1) \lex< (y,2s-e,x) = \vl(((x,y),s,e)),
\]
as required.


\item If the robot jumps East it enters a state $((z,y),s,e+1)$ for some
$z>x$.  Now
\[
\vl(((z,y),s,e+1)) = (y,2s-(e+1),z) = (y,2s-e-1,z),
\]
but since $2s-e-1 < 2s-e$, the rule~(\ref{lex3}) implies that
\[
\vl(((z,y),s,e+1)) = (y,2s-e-1,z)  \lex< (y,2s-e,x) = \vl(((x,y),s,e)),
\]
as required.

\item If the robot moves South it enters state $((x,y-1),s+1,e)$, and
\[
\vl(((x,y-1),s+1,e)) = (y-1,2(s+1)-e,x) \lex< (y,2s-e,x) = \vl(((x,y),s,e)),
\]
as required.

\end{itemize}

So indeed state-value is a decreasing variable under lexicographic order.
We'll show in the next section that it is impossible for a
lexicographically-ordered value to be decreased an infinite number of
times.  That's just what we need to finish verifying Claim~\ref{robotcl}.

\subsection{Well-founded Partial Orders}

\begin{definition}\label{wfpo}
  A partial order is \emph{well-founded} iff every nonempty subset of its
  domain has a minim\emph{al} element.
\end{definition}

\iffalse
That is, $m \in S$ is minimal if $m' \preceq m$ implies that $m' =
m$ for all $m' \in S$.\fi

So the Well Ordering Principle is equivalent to the assertion that the
nonnegative integers are well-founded under $\leq$.  But well-foundedness
makes sense much more generally than for just nonnegative integers.
\iffalse

For example, by Lemma~\ref{finmin}, every partially ordered finite set has
a minimal element, so all finite partial orders are automatically well
founded.  Note that in this case we can't expect to find a minim\emph{um}
element, since even in a finite partial order, there often isn't any
minimum.


\textbf{Quick Exercise:} Give an example of a partial order with a
\emph{unique} minim\emph{al} element but no minim\emph{um} element.

\hint It can't be a finite partial order.

\solution{The set of all real numbers totally ordered by $\leq$ has no
minimal element.  We can regard $\le$ as a partial order on $\reals \union
\set{i}$, where $i$ is the complex number $\sqrt{-1}$, by agreeing that $i$
is not comparable to every real number.  This makes $i$ the unique minimal
element -- it's also the unique maximal element -- but there is no minimum
element.}
\fi

There is another helpful way to characterize well-founded partial orders:
\hyperdef{infinite}{decreasing}
{\begin{lemma}\label{d-chain}
A partial order is well-founded iff it has no infinite decreasing chain.
\end{lemma}}

Saying that the partial order $\preceq$ has no infinite decreasing chain
means there is no infinite sequence $p_1,p_2,\dots,p_n\dots$ of elements
in $P$ such that
\[
p_1 \succ p_2 \succ \dots \succ p_n \dots.
\]
Here we're using the notation ``$p \succ q$'' to mean [$q \preceq p$ and
  $q \neq p$].  That's so we can read the decreasing chain left to right,
as usual in English.

\begin{proof}
(\textbf{left to right}) (By contradiction) If there was such an infinite
decreasing sequence, then the set of elements in the sequence itself would
be a nonempty subset without a minimal element.

(\textbf{right to left}) (By contradiction) Suppose $\preceq$ was not
well-founded.  So there is some subset $S \subseteq \domain{\preceq}P$,
such that $S$ has at least one element, $s_1$, but $S$ has no minimal
element.  In particular, since $s_1$ is not minimal, there must be
\emph{another} element $s_2 \in S$ such that $s_1 \succ s_2$.  Similarly,
since $s_2$ is not minimal, there must be still another element $s_3 \in S$
such that $s_2 \succ s_3$.  Continuing in this way, we can construct an
infinite decreasing chain $s_1 \succ s_2 \succ s_3 \cdots$ in $S$.
\end{proof}

It now follows immediately from Lemma~\ref{d-chain} that every finite
partial order is well-founded ---without needing the counting argument
used to prove this in Lemma~\ref{finmin}.

\begin{problem}
Let $D$ be the usual \emph{dictionary order} on finite sequences of letters
of the alphabet.  Show that neither $D$ nor $D^{-1}$ is well-founded.

\solution{\[
a\ D\ aa\ D\ aaa D\dots
\]
is an infinite $D$-increasing sequence, and so is an infinite
$D^{-1}$-decreasing sequence, proving that $D^{-1}$ is not well-founded.
To show $D$ is not well-founded, we describe 
an infinite $D^{-1}$-increasing sequence:
\[
b\ D^{-1}\ ab\ D^{-1}\ aab\  D^{-1}\ aaab \dots.
\]
}
\end{problem}

\section{Termination}

We can formulate our general termination method as follows:

\begin{definition}
  Let $\prec$ be a strict partial order on a set, $A$.  A derived variable
  $f : \text{states } \to A$ is \emph{strictly decreasing} iff
\[
q \movesto q' \text{  implies  } f(q') \prec f(q).
\]
\end{definition}

We confirmed termination of the GCD and Extended GCD procedures by finding
derived variables, $y$ and \texttt{Y}, respectively, that were nonnegative
integer-valued and strictly decreasing.  We can summarize this approach to
proving termination as follows:
\begin{theorem}
\label{th:decr}
If $f$ is a strictly decreasing $\naturals$-valued derived variable of a
state machine, then the length of any execution starting at state $q$ is
at most $f(q)$.
\end{theorem}

Of course we could prove Theorem~\ref{th:decr} by induction on the value
of $f(q)$, but think about what it says: ``If you start counting down at
some nonnegative integer $f(q)$, then you can't count down more than
$f(q)$ times.''  Put this way, it's obvious.

There are cases where it's easier to prove termination based on more
general partial orders than ``less-than'' on $\naturals$.  Termination is
guaranteed whenever there is a derived variable that strictly decreases with
respect to any well-founded partial order.

\iffalse
We now define some other useful flavors of derived variables taking values
over partial ordered sets.  We'll use the notational convention that when
$\prec$ denotes a strict partial order on some set, then $\preceq$ is the
corresponding \emph{weak} partial order
\[
a\preceq a' \ \eqdef\quad a \prec a' \lor a = a'.
\]

A relation like $\prec$ is called a \emph{strict} partial order.  It is
transitive, antisymmetric, and but \emph{non}reflexive in the strongest
sense: $a \not\prec a$ for every $a \in A$.\footnote{In other words, if $a
\prec b$, then it is not the case that $b \prec a$.  This property is also
called \emph{a}symmetry.}\fi

\begin{definition}
Let $\prec$ be a strict partial order on a set, $A$.  A derived variable
$f : Q \to A$ is \emph{strictly decreasing} with respect to $\prec$ iff
\[
q \movesto q' \text{ implies } f(q') \prec f(q).
\]
Also, $f$ is \emph{weakly decreasing} iff
\[
q \movesto q' \text{  implies  } f(q') \preceq f(q).
\]
where $\preceq$ is the weak partial order corresponding to $\prec$,
namely,
\[
[a_1 \preceq a_2] \eqdef [(a_1 \prec a_2) \text{ or } (a_1=a_2)].
\]

\emph{Strictly increasing} and \emph{weakly increasing} derived variables
are defined similarly.\footnote{Weakly increasing variables are often also
called \emph{nondecreasing}.  We will avoid this terminology to prevent
confusion between nondecreasing variables and variables with the much
weaker property of \emph{not} being a decreasing variable.}
\end{definition}

\begin{theorem}\label{well-founded-decreasing}
  If there exists a derived variable for a state machine that is strictly
  decreasing with respect to some well-founded partial order, then every
  execution terminates.
\end{theorem}

Theorem~\ref{well-founded-decreasing} follows immediately from
Lemma~\ref{d-chain}.

Note that the existence of a nonnegative integer-valued \emph{weakly}
decreasing derived variable does not guarantee that every execution
terminates.  That's because an infinite execution could proceed through
states in which a weakly decreasing variable remained constant.

\subsubsection{A Southeast Jumping Robot}

\iffalse Begin by defining the trivial ``pick how long'' game: P1 picks $n
\in \naturals$, the P2 and P1 alternate making forced moves.  The game
ends after $n$ forced moves; the last person to move wins.  So P1 strategy
is ``pick and even number.''  Insert here the discussion of ``terminates,
but no bound on number of steps...'' used below.

May also tell the ``guess a bigger number game''joke.
\fi

Here's a contrived but simple example of proving termination based on a
variable that is strictly decreasing over a well-founded order.  Let's
think about a robot positioned at an integer lattice-point in the
Northeast quadrant of the plane, that is, at $(x,y) \in \naturals^2$.

At every second when it is away from the origin, $(0,0)$, the robot must
make a move, which may be
\begin{itemize}

\item a unit distance West when it is not at the boundary of the Northeast
  quadrant (that is, $(x,y) \movesto (x-1,y)$ for $x>0$), or

\item a unit distance South combined with an arbitrary jump East (that is,
     $(x,y) \movesto (z,y-1)$ for $z\geq x$).

\end{itemize}
\begin{claim}\label{robotcl}
The robot will always get stuck at the origin.
\end{claim}

If we think of the robot as a nondeterministic state machine, then
Claim~\ref{robotcl} is a termination assertion.  The Claim may seem
obvious, but it really has a different character than termination based on
nonnegative integer-valued variables.  That's because, even knowing that
the robot is at position $(0,1)$, for example, there is no way to bound
the time it takes for the robot to get stuck.  It can delay getting stuck
for as many seconds as it wants by making its next move to a distant point
in the Far East.  This rules out proving termination using
Theorem~\ref{th:decr}.

So does Claim~\ref{robotcl} still seem obvious?

Well it is if you see the trick: if we reverse the coordinates, then every
robot move goes to a position that is smaller under lexicographic order.
More precisely, let $f:\naturals^2 \to \naturals^2$ be the derived variable
mapping a robot state ---its position $(x,y)$ ---to $(y,x) \in
\naturals^2$.  Now $(x,y)\movesto (x',y')$ is a legitimate robot move iff
$f((x',y')) \lex< f((x,y))$.  In particular, $f$ is a strictly
$\lex<$-decreasing derived variable, so
Theorem~\ref{well-founded-decreasing} proves that the robot always get
stuck as claimed.
\fi


\TBA{A REPEAT:?}

We will prove that the robot always gets stuck at the origin by
generalizing the decreasing variable method, but with decreasing values
that are more general than nonnegative integers.  Namely, the traveling robot
can be modeled with a state machine with states of the form $((x,y),s,e)$
where
\begin{itemize}
\item $(x,y) \in \naturals^2$ is the robot's position,
\item $s$ is the number of moves South the robot took to get to this
position, and
\item $e \le 2s$ is the number of moves East the robot took to get to this
position. 
\end{itemize}

Now we define a derived variable $\vl:\text{States}\to \naturals^3$:
\[
\vl(((x,y),s,e)) \ \eqdef\quad (y,2s-e,x),
\]
and we order the values of states with the \emph{lexicographic} order,
$\lexle$, on $\naturals^3$:
\begin{equation}\label{lex3}
(k,l,m) \lexle (k',l',m') \ \eqdef\quad k < k' \text{ or } (k=k' \text{
and } l < l') \text{ or } (k=k' \text{ and } l = l' \text{ and } m \le m')
\end{equation}

Let's check that values are lexicographically decreasing.  Suppose the
robot is in state $((x,y),s,e)$.
\begin{itemize}
\item If the robot moves West it enters state $((x-1,y),s,e)$, and
\[
\vl(((x-1,y),s,e)) = (y,2s-e,x-1) \lex< (y,2s-e,x) = \vl(((x,y),s,e)),
\]
as required.


\item If the robot jumps East it enters a state $((z,y),s,e+1)$ for some
$z>x$.  Now
\[
\vl(((z,y),s,e+1)) = (y,2s-(e+1),z) = (y,2s-e-1,z),
\]
but since $2s-e-1 < 2s-e$, the rule~(\ref{lex3}) implies that
\[
\vl(((z,y),s,e+1)) = (y,2s-e-1,z)  \lex< (y,2s-e,x) = \vl(((x,y),s,e)),
\]
as required.

\item If the robot moves South it enters state $((x,y-1),s+1,e)$, and
\[
\vl(((x,y-1),s+1,e)) = (y-1,2(s+1)-e,x) \lex< (y,2s-e,x) = \vl(((x,y),s,e)),
\]
as required.

\end{itemize}

So indeed state-value is a decreasing variable under lexicographic order.
But since lexicographic order is well-founded, it is impossible for a
lexicographically-ordered value to be decreased an infinite number of
times.  That's just what we need to finish verifying Claim~\ref{robotcl}.

\iffalse
Now notice that the lexicographic order on $\naturals^3$ defined in the
previous section (by the condition~(\ref{lex3})) is exactly the same as
$\lexle$ on $\naturals^3$ according to Definition~\ref{lexn}.

But we already proved that the value of the robot's state decreases at
every step.  And we have just proved that the order on these values is
well-founded.  So Lemma~\ref{d-chain} implies that the values cannot keep
decreasing forever.  That means the robot cannot keep moving forever: it
must always terminate.
\fi



\subsection{Product Partial Orders}

An easy way to construct well-founded partial orders is by taking products
of well-founded partial orders.  For example, the nonnegative integers are
well-founded under $\leq$, so the product partial order $(\leq \cross
\leq)$ on pairs of nonnegative integers is going to be well-founded.

\iffalse

But why stop at $n$ when we can be much more general without being any more
complicated?

The idea is that a Cartesion product, $A_1 \cross A_2 \cross \dots \cross
A_n$ of sets $A_1,A_2, \dots, A_n$ consists of all the $n$-tuples
$\ang{a_1,a_2,\dots,a_n}$ such that $a_i \in A_i$ for $1 \le i \le n$.  But
any such $n$-tuple can be represented by the function, $f$, where $f(i)
\eqdef a_i$.

GO ON TO DEFINE: a \emph{dependent-type} is a set-valued total function.
For example, for $r \in \reals$, define the dependent-type, $F_0$, by the
rule:
\[
F_0(r) \eqdef\ \ \set{t \in \reals \suchthat t \neq r}.
\]

If $F$ is a dependent-type, then an \emph{element of type, $F$,} is
defined to be a total function, $f$, with the same domain as $F$, and such
that
\[
\forall a \in \domain(F)\ f(a) \in F(a).
\]

AND SO ON...
\fi

To prove this, we first generalize partial orders on pairs of integers
to partial orders pairs of elements from \emph{any} partial orders.

\begin{definition}\label{lex-coord-def}
Let $\preceq_1$ and $\preceq_2$ be partial orders with domains $A_1$
and $A_2$.

The \hyperdef{coordinatewise}{po}{\term{coordinatewise partial
    order}}, $\coordle$, is defined to be the product relation,
$\paren{\preceq_1 \cross \preceq_2}$, which we observed in
Section~\ref{prodsec} really is a partial order when $\preceq_1$ and
$\preceq_2$ are.

The \emph{lexicographic partial order}, $\lexle$, for $\preceq_1$ and
$\preceq_2$ is defined by the conditions:
\begin{eqnarray*}
%\label{lexc2}
\domain{\lexle} &\eqdef& A_1 \cross A_2\\
(a_1,a_2) \lexle (b_1,b_2) & \qiff & a_1 \prec_1 b_1\text{ or } (a_1=b_1 \text{
and } a_2 \preceq_2 b_2).
\end{eqnarray*}

\end{definition}

It's easy to check that $\lexle$ is also a partial order, so we're
justified in calling it that.  But not only are coordinatewise and
lexicographic partial orders really partial orders, but they will also be
well-founded providing $\preceq_1$ and $\preceq_2$ are well-founded.
Namely,

\hyperdef{lex}{order}{\begin{theorem}}\label{lex-coord-thm} For
well-founded partial orders, $\preceq_1$ and $\preceq_2$, let $\coordle$
be their coordinatewise partial order and $\lexle$ be their lexicographic
partial order.  Then
\begin{enumerate}

\iffalse
\item\label{Lpo} $\lexle$ and $\coordle$ are partial orders.  Moreover,
\fi

\item\label{Cwf} $\coordle$ is well-founded,

\item\label{Lexwf} $\lexle$ is well-founded,

\item\label{Lto} if $\preceq_1$ and $\preceq_2$ are both total orders,
then so is $\lexle$.
\end{enumerate}
\end{theorem}

The proof of part~\ref{Cwf} is easy: a coordinatewise minimal element of
any nonempty subset, $S$, of $A_1 \cross A_2$ is simply a pair $(m_1,m_2)$
consisting of a minimal element, $m_1 \in A_1$, among the first
coordinates of pairs in $S$, and a minimal element, $m_2 \in A_2$, among
the second coordinates of pairs in $S$.

A refinement of this idea works for lexicographic order.  To find a
minimal element of $S$ under $\lexle$, let $m_1 \in A_1$, be minimal among
the first coordinates of pairs in $S$, as before.  But now find a minimal
element $n_2 \in A_2$ among the second coordinates pairs in $S$
\emph{whose first coordinate is $m_1$}.  It's not hard to see that the
element $(m_1,n_2) \in S$ must be a lexicographically minimal element of
$S$.

\textcolor{red}{Watch out!}  When we, or any other author, says ``It's not
hard to see,'' a savvy reader should quickly check if it really is easy,
and if it doesn't seem to be, they should suspect that the author is being
lazy or possibly wrong.

In this case, we're not being lazy ---we're just omitting an almost
automatic (and uninformative) series of simple proof steps, which it would
be more educational for you work out yourself than to see written out by
us.  For the same reason, we'll let you verify the last Part~\ref{Lto}.\
that lexicographic order determined by two total orders is also a total
order.

\iffalse
But to build your faith in us ---we hope you'll be ready to buy a used car
from us at the end of the term ---and to show we're not lazy, we present
this step by step proof below.  We hope that not many readers will need to
take the time to plow through it.

\begin{proof} (of part~\ref{Lexwf}.)

  Suppose $S$ is a nonempty subset of $A_1 \cross A_2$.  We must show that
  $S$ has a $\lexle$-minimal element.

  We begin by noting that
  \[
  S_1 \eqdef\ \ \set{a_1 \in A_1 \suchthat (a_1,a_2) \in S \mbox{ for some
    } a_2 \in A_2}
  \]
  is a nonempty subset of $A_1$, and so has a $\preceq_1$-minimal element,
  $m_1$.  This means the set
  \[
  S_{12} \eqdef\ \ \set{a_2 \in A_2 \suchthat (m_1,a_2) \in S}
  \]
  is a nonempty subset of $A_2$ and so has a $\preceq_2$-minimal element,
  $m_2$.  We claim that $(m_1,m_2)$ is a minimal element of $S$ under
  $\lexle$.

To check this, note first that $(m_1,m_2) \in S$ by definition.  So to show
it is minimal, we need only show that if
\begin{eqnarray}
(a_1,a_2) & \in & S,\text{ and }\label{aaS}\\
(a_1,a_2) & \preceq_{\text{lex}} & (m_1,m_2)\label{mmaa}
\end{eqnarray}
then
\begin{eqnarray}\label{aamm}
(a_1,a_2) = (m_1,m_2).
\end{eqnarray}
But
\begin{align}
a_1 & \in S_1
       & \text{by~(\ref{aaS}) and def.\ of $S_1$},\label{e0}\\
a_1 & \preceq_1 m_1
       & \text{by~(\ref{mmaa}) and def.\ of $\lexle$},\label{e1}\\
a_1 & =_1 m_1
      &  \text{by~(\ref{e0}),~(\ref{e1}),
               and minimality of $m_1$ in $S_1$},\label{ln4.e2}\\
a_2 & \in S_{12}
        &  \text{by~(\ref{aaS}),~(\ref{ln4.e2}) and def.\ of $S_{12}$},\label{e3}\\
a_2 & \preceq_2 m_2
      & \text{by~(\ref{mmaa}),~(\ref{ln4.e2}), and def.\ of $\lexle$},\label{e4}\\
a_2 & = m_2
      & \text{by~(\ref{e3}),~(\ref{e4}), and minimality of $m_1$ in
       $S_{12}$}. \label{e5}
\end{align}

Now~(\ref{aamm}) follows from~(\ref{ln4.e2}) and~(\ref{e5}), completing
the proof of part~\ref{Lexwf}.
\end{proof}
\fi

This completes the proof of Theorem~\ref{lex-coord-thm}.


Of course \iffalse the values of states for the robot in the previous
section were triples not pairs, but\fi
we can easily define the
lexicographic partial order on $n$-tuples for any $n\ge 1$.  Namely,
\begin{definition}\label{lexn}
  Suppose $\preceq_1,\dots, \preceq_n$ are partial orders with domains
  $A_1, \dots, A_n$, and define the partial order, $\lexle$, with domain,
  $A_1 \cross \cdots \cross A_n$, recursively in $n$:
\begin{eqnarray*}
\ang{a_1} \lexle \ang{b_1} &\qiff& a_1 \preceq_1 b_1\\
\ang{a_1,\dots,a_n,a_{n+1}} \lexle \ang{b_1,\dots,b_n,b_{n+1}} &\qiff&
  a_1 \preceq_1 b_1\ \QAND\
    \ang{a_2,\dots,a_{n+1}} \lexle \ang{b_2,\dots,b_{n+1}}
\end{eqnarray*}
\end{definition}
Theorem~\ref{lex-coord-thm} now generalizes straightforwardly to
$n$-tuples.  In particular, we conclude that since $\le$ is a well-founded
total order on $\naturals$, lexicographic order is a well-founded total
order on $\naturals^n$ for all $n \ge 1$.


\subsection{Well founded recursive definitions}

Ackermann's function is an extremely fast-growing function of two
nonegative arguments, which means its inverse is extremely slow-growing.
For example, its inverse grows slower than $\log n$, $\log \log n$, $\log
\log \log n$, \dots, but it does grow unboundly.  It turns out that this
extremely slow growing function actually comes up in the study of
algorithms.  It is the coefficient of $n$ in a formula for the average
number of steps used by a \emph{Union-Find algorithm} to process $n$
queries and declarations of equivalence between pairs of elements of a
set.  Since the coefficient of $n$ grows unboundly, the Union-Find
algorithm theoretically uses a nonlinear number of operations.  But it may
as well be linear, since the coefficient is less than 5 for any $n$ that
could conceivably be achieved.

You will learn about Union-Find if you take the Algorithms course, 6.046.
We're mentioning this story to motivate an examination of the somewhat
unusual recursive definition of Ackermann's function, $A(m,n)$.

Ackermann's function can be defined recursively as the function, $f$,
given by the following rules:
\begin{align}
f(m,n) &=  2n, &&\text{if $m=0$ or $n \le 1$},\label{Am0}\\ 
f(m,n) &=  f(m-1,f(m,n-1)), &&\text{otherwise}.\label{AA}
\end{align}

Now these rules are unusual because the definition of $f(m,n)$ involves an
evaluation of $f$ at arguments that may be a lot bigger than $m$ and $n$.
Definitions of values at small arguments in terms of larger arguments can
easily lead to divergent (non terminating) evaluations, so how can we be
sure this one is ok?  The simple answer is that the definition of $f(m,n)$
actually doesn't involve evaluation at bigger arguments than $(m,n)$ if we
think of the right way to order the arguments.  In this case,
lexicographic order does the job.

Being an ok definition means that equations~(\ref{Am0}) and~(\ref{AA})
define a \emph{unique}, \emph{total} function, $f$.  We'll focus on
proving uniqueness.  Namely, if $g$ satisfies the same equations, that is,
\begin{align}
g(m,n) &=  2n, &&\text{if $m=0$ or $n \le 1$},\label{gm0}\\ 
g(m,n) &=  g(m-1,g(m,n-1)), &&\text{otherwise},\label{gg}
\end{align}
then $f=g$.

We'll give a simple proof of this using the well foundedness of
lexicographic order, $\lexle$, on $\naturals^2$.  Namely, assume for the
sake of contradiction that $f \neq g$.  Then the set, $S$, of pairs
$(m,n)$ such that $f(m,n) \ne g(m,n)$ is not empty, so by well-foundedness
of lexicographic order, there is a $\lex<$-\emph{least} element $(m_0,n_0)
\in S$.  In other words,

\begin{equation}\label{m0n0}
f(m_0,n_0) \neq g(m_0,n_0),
\end{equation}
but
\begin{eqnarray}\label{<m0n0}
f(m,n) = g(m,n) \text{ for } (m,n) \lex< (m_0,n_0).
\end{eqnarray}

Now if $m_0 = 0$ or $n_0 \le 1$, then $f(m_0,n_0)$ and $g(m_0,n_0)$ both
equal $2n_0$ by~\eqref{Am0} and~\eqref{gm0}, which
contradicts~\eqref{m0n0}.  So it must be that $m_0 >0$ and $n_0 > 1$.
This implies
\begin{equation}\label{fm0f}
f(m_0,n_0) = f(m_0-1,f(m_0,n_0-1)),
\end{equation}
and
\begin{equation}\label{gm0g}
g(m_0,n_0) = g(m_0-1,g(m_0,n_0-1)),
\end{equation}
by~\eqref{AA} and~\eqref{gg}.

Next, by definition of lexicographic order, $(m_0,n_0-1) \lex< (m_0,n_0)$,
so~\eqref{<m0n0} implies that
\[f(m_0,n_0-1)) \text{ and } g(m_0,n_0-1))\text{ have the same value, } v.
\]

Similarly, $(m_0-1,v)\lex< (m_0,n_0)$ by definition of $\lex<$,
so~\eqref{<m0n0} also implies that
\begin{equation}
f(m_0-1,v) = g(m_0-1,v)
\end{equation}
which by~\eqref{fm0f} and~\eqref{gm0g}, implies that $f(m_0,n_0) =
g(m_0,n_0)$.  But this contradicts~\eqref{m0n0}, which completes the
proof.


\endinput
