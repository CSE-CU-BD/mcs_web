\documentclass[handout]{mcs}

\begin{document}

\renewcommand{\reading}{
\begin{itemize}
\item Chapter 7.1 - 7.4\bref{recursive_def}{.\ \emph{Recursive Definitions}}

\item Chapter 8.1\bref{inf_sets}{.\ \emph{Infinite Sets}}

\item Chapter 8.3-8.4\bref{set_theory}{.\ \emph{Set Theory}}}

\end{itemize}

\problemset{4}

\medskip

\textbf{Reminder}:

\begin{itemize}

\item

  \href{https://courses.csail.mit.edu/6.042/spring18/pset_instructions}
       {Instructions for PSet submission} are on the class
       \href{https://stellar.mit.edu/S/course/6/sp18/6.042/}{Stellar
         page}.  Remember that each problem should prefaced with a
       \href{http://courses.csail.mit.edu/6.042/spring18/pset_instructions.shtml#collab-state}
            {\emph{collaboration statement}}.

\item The class has a
  \href{https://piazza.com/mit/spring2018/6042/home} {Piazza
    forum}.  With Piazza you may post questions---both administrative
  and content related---to the entire class or to just the staff.  You
  are likely to get faster response through Piazza than from direct
  email to staff.

  You should post a question or comment to Piazza at least once by the
  end of the second week of the class; after that Piazza use is
  optional.
\end{itemize}


%%%%%%%%%%%%%%%%%%%%%%%%%%%%%%%%%%%%%%%%%%%%%%%%%%%%%%%%%%%%%%%%%%%%%
% Problems start here
%%%%%%%%%%%%%%%%%%%%%%%%%%%%%%%%%%%%%%%%%%%%%%%%%%%%%%%%%%%%%%%%%%%%%

\pinput{CP_DAG_to_WPO.tex}
\pinput{MQ_chains_and_antichains_sched}
\pinput{MQ_task_parallel_scheduling_v5}
\pinput{PS_top_sort_for_closure_of_DAG}
\pinput{PS_A_to_B_diagonal_argument}
\pinput{PS_recursive_set_data_type}
\pinput{PS_equality_axioms}
\pinput{PS_bogus_Cantor}

\end{document}

