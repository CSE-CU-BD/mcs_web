\documentclass[handout]{mcs}

\begin{document}

\inclassproblems{4, Wed.}

%%%%%%%%%%%%%%%%%%%%%%%%%%%%%%%%%%%%%%%%%%%%%%%%%%%%%%%%%%%%%%%%%%%%%
% Problems start here
%%%%%%%%%%%%%%%%%%%%%%%%%%%%%%%%%%%%%%%%%%%%%%%%%%%%%%%%%%%%%%%%%%%%%

\pinput{CP_3_and_7_cent_stamps_by_induction}
\pinput{PS_bogus_prime_divides_integer_product}
\begin{staffnotes}
Students tend to neglect base cases in induction proofs. Emphasize that base cases should always be explicitly written and thoroughly proved.
\end{staffnotes}

\pinput{CP_triangle_tiling}
\pinput{CP_box_unstacking}
\begin{staffnotes}
Discuss Figure 3 with students to give them better idea how the game works.

Unlike previous cases where at each step only one statement was proved here two statements need to be inductively proved, those are, $p(A)\ge p(B)$ and score of the sequence $=p(A)-p(B)$.
\end{staffnotes}


\pinput{CP_bogus_induction_nth_power}

\examspace

\begin{figure}\redrawntrue
\[
\begin{array}{cccccccccccl}
\multicolumn{10}{c}{\textbf{Stack Heights}} & \quad & \textbf{Score} \\
\underline{10}&&&&&&&&& && \\
5&\underline{5}&&&&&&&& && 25 \text{ points} \\
\underline{5}&3&2&&&&&&& && 6 \\
\underline{4}&3&2&1&&&&&& && 4 \\
2&\underline{3}&2&1&2&&&&& && 4 \\
\underline{2}&2&2&1&2&1&&&& && 2 \\
1&\underline{2}&2&1&2&1&1&&& && 1 \\
1&1&\underline{2}&1&2&1&1&1&& && 1 \\
1&1&1&1&\underline{2}&1&1&1&1& && 1 \\
1&1&1&1&1&1&1&1&1&1 && 1 \\ \hline
\multicolumn{10}{r}{\textbf{Total Score}} & = & 45 \text{ points}
\end{array}
\]
\caption{An example of the stacking game with $n = 10$ boxes.  On each
line, the underlined stack is divided in the next step.}
\label{fig:stacking-10}
\end{figure}


%%%%%%%%%%%%%%%%%%%%%%%%%%%%%%%%%%%%%%%%%%%%%%%%%%%%%%%%%%%%%%%%%%%%%
% Problems end here
%%%%%%%%%%%%%%%%%%%%%%%%%%%%%%%%%%%%%%%%%%%%%%%%%%%%%%%%%%%%%%%%%%%%%
\end{document}
