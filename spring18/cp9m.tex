\documentclass[handout]{mcs}

\begin{document}

\inclassproblems{9, Mon.}

%%%%%%%%%%%%%%%%%%%%%%%%%%%%%%%%%%%%%%%%%%%%%%%%%%%%%%%%%%%%%%%%%%%%%
% Problems start here
%%%%%%%%%%%%%%%%%%%%%%%%%%%%%%%%%%%%%%%%%%%%%%%%%%%%%%%%%%%%%%%%%%%%%

\begin{staffnotes}
Simple Graphs: Degree \& Isomorphism

Spring17: half the teams finished in 45 min.  Asked them to work out
the 4 possible isomorphisms for 2(b), and gave a sermon about
``buildup error'' re problem 3.  Then asked for comments on midterm 3.
Most still dismissed after one hour.

For Fall17 using CP\_bipartite\_sex instead of TP\_handshake\_average.

For Spring18, TP\_handshake\_average is added before CP\_bipartite\_sex and PS\_choose\_isomorphic\_graphs is added at the end.
\end{staffnotes}

\pinput{TP_preserved_under_isomorphism}

\pinput{CP_isomorphic_graphs}

\pinput{PS_bogus_graph_two_ends}

\pinput{TP_handshake_average.tex}

\pinput{CP_bipartite_sex}

%suggested new problem
\pinput{PS_choose_isomorphic_graphs}
\begin{staffnotes}
Let the students discuss which properties are preserved under isomorphism. If they cannot think of many such properties ask them about some properties that you can think of.
\end{staffnotes}

%%%%%%%%%%%%%%%%%%%%%%%%%%%%%%%%%%%%%%%%%%%%%%%%%%%%%%%%%%%%%%%%%%%%%
% Problems end here
%%%%%%%%%%%%%%%%%%%%%%%%%%%%%%%%%%%%%%%%%%%%%%%%%%%%%%%%%%%%%%%%%%%%%

\end{document}

\endinput


