\documentclass[handout]{mcs}

\begin{document}

\inclassproblems{6, Fri.}

\begin{staffnotes}
Set Theory and Russell's Paradox
\end{staffnotes}

%%%%%%%%%%%%%%%%%%%%%%%%%%%%%%%%%%%%%%%%%%%%%%%%%%%%%%%%%%%%%%%%%%%%%
% Problems start here
%%%%%%%%%%%%%%%%%%%%%%%%%%%%%%%%%%%%%%%%%%%%%%%%%%%%%%%%%%%%%%%%%%%%%

% \begin{staffnotes}
% This first problem is meant as an easy review of set/power-set
% properties.  If part(a) takes more than 10 min, end there, and in any
% case, not more than 15 min on this problem, at which point just show
% the soln to part(a).
% \end{staffnotes}

% \pinput{CP_powerset_union}


\begin{staffnotes}
  Please don't spend more than 10 or 15 minutes on this opening problem. If it takes more than a few minutes to break into the problem, consider pointing them to Russell's Paradox.
\end{staffnotes}
\pinput{TP_not_member_of_self}

\pinput{CP_foundation_axiom}

\pinput{CP_Russells_and_cardinality}

%\pinput{PS_infinite_ordinals}



\begin{staffnotes}
This next problem is just a rephrasing solely in terms of sets of the
discussion of Nim games and the Fundamental Thm of Win-lose Games of
Section~\bref{FundThm_Games}.  It is not necessary to review this
material to do the problem, as long as you are \textbf{comfortable
  with this kind of structural induction with infinite constructors.}
However, a review of Section~\bref{FundThm_Games} and
Problem~\bref{TP_nimset} is useful to motivate this problem.
\end{staffnotes}

\pinput{PS_recursive_set_data_type_alt}



% \begin{center}
% Supplemental Problems
% \end{center}

% %\pinput{PS_size_n_set_formula}






%%%%%%%%%%%%%%%%%%%%%%%%%%%%%%%%%%%%%%%%%%%%%%%%%%%%%%%%%%%%%%%%%%%%%
% Problems end here
%%%%%%%%%%%%%%%%%%%%%%%%%%%%%%%%%%%%%%%%%%%%%%%%%%%%%%%%%%%%%%%%%%%%%

\end{document}
