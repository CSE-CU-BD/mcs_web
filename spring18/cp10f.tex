\documentclass[handout]{mcs}

\begin{document}

\inclassproblems{7, Wed.}

%%%%%%%%%%%%%%%%%%%%%%%%%%%%%%%%%%%%%%%%%%%%%%%%%%%%%%%%%%%%%%%%%%%%%
% Problems start here
%%%%%%%%%%%%%%%%%%%%%%%%%%%%%%%%%%%%%%%%%%%%%%%%%%%%%%%%%%%%%%%%%%%%%

\begin{staffnotes}
Chapter~\bref{Turing_sec}.\ \emph{Turing}
through~\bref{sec:inverse}.\ \emph{Cancelling $\pmod{n}$}
\end{staffnotes}

%\pinput{CP_fast_exponentiation}
%\pinput{CP_calculating_inverses_fermat} %PS_calculating_inverses (b)

\pinput{FP_inverse17mod29}

%\pinput{CP_7777}


%\pinput{CP_divisible_by_24}

\pinput{MQ_congruent_mod_product}  %lightweight intro to Chinese remainder

\pinput{CP_chinese_remainder} % Rearranged things since I'm guessing we want students to get to this one?

\textbf{Supplemental problems:}

\pinput{CP_multiples_of_9_and_11}

\pinput{CP_remainder_computation_practice}

%\pinput{PS_check_factor_by_digits}


%\pinput{CP_polynomials_produce_multiples} 

%\pinput{CP_pirate_treasure}

%%%%%%%%%%%%%%%%%%%%%%%%%%%%%%%%%55
% Problems end here
%%%%%%%%%%%%%%%%%%%%%%%%%%%%%%%%%%%%%%%%%%%%%%%%%%%%%%%%%%%%%%%%%%%%%

\end{document}

\endinput

