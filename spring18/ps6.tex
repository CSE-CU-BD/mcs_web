\documentclass[handout]{mcs}

\begin{document}

\renewcommand{\reading}{
  Chapter~\bref{number_theory_chap} through~\bref{Euler_sec}: \emph{GCDs, Congruences, and Euler's Theorem}
}

\problemset{6}

\begin{staffnotes}
\begin{verbatim}
Enter reading/topic here
\end{verbatim}
\end{staffnotes}

%%%%%%%%%%%%%%%%%%%%%%%%%%%%%%%%%%%%%%%%%%%%%%%%%%%%%%%%%%%%%%%%%%%%%
% Problems start here
%%%%%%%%%%%%%%%%%%%%%%%%%%%%%%%%%%%%%%%%%%%%%%%%%%%%%%%%%%%%%%%%%%%%%

% Simple Graphs: Coloring 		    12.6

% \pinput{CP_bipartite_coloring}
% \pinput{CP_coloring}
% \pinput{PS_coloring_induction}

% % Connectivity and Trees 		   12.7-12
% \pinput{CP_n_dim_hypercube_no_tree}
% \pinput{CP_graph_edge_mark}
% \pinput{PS_forestsize}

% Number Theory: GCDs                       9.1-4
%\pinput{PS_fill_bucket_gcd}
\pinput{PS_linear_combination_game}
%\pinput{CP_GCD_algebra}

% Number Theory: Congruences                9.5-9
\pinput{CP_nonparallel_lines}
% \pinput{CP_pirate_treasure}

% Number Theory: Euler's Theorem            9.10
% yes
\pinput{PS_Euler_theorem_not_rel_prime}


\end{document}

